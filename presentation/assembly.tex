%
% x86-64 assembly
%
\subsection{x86-64 asssembly}
\frameSubsection{pictures/Nebuchadnezzar.png}{}

\newcommand\assemblyExample{%
}
\begin{frame}{x86-64 assembly}{Syntax and registers}
\begin{columns}[c]
\begin{column}{0.5\textwidth}
\begin{itemize}
\item x86-64 CPU have 64-bit registers and 64-bit word size
\item Assembly is shown with the AT\&T syntax
\begin{itemize}
\item Parameter order is: source first then destination
\item \inputminted{gas}{src/assembly/movq.S}
\item \text{``}q\text{''} suffix for QWORD operand (quad word) with $4 x 16\text{bit word} = 64\text{bit}$
\end{itemize}
\item I'll use \emph{word} for \emph{machine word} (64-bit)
\end{itemize}
\end{column}
\begin{column}{0.5\textwidth}
\begin{table}
\begin{tabular}{|c|c|}
\hline
Monikers & Description \\
\hline
\hline
RAX & \\
\hline
RBX & \\
\hline
RCX & \\
\hline
RDX & \\
\hline
RSI & \\
\hline
RDI & \\
\hline
RSP & Stack Pointer \\
\hline
RBP & \\
\hline
R8  & \\
\hline
R9  & \\
\hline
R10 & \\
\hline
R11 & \\
\hline
R12 & \\
\hline
R13 & \\
\hline
R14 & caml_domain_state* \\
\hline
R15 & young_ptr* \\
\hline
RIP & Instruction Pointer \\
\hline
\end{tabular}
\caption{General Purpose and Pointers Registers}
\end{table}
\end{column}
\end{columns}
\end{frame}
% https://en.wikipedia.org/wiki/X86_assembly_language#Syntax
% https://wiki.osdev.org/CPU_Registers_x86-64

\newcommand\listassembly[1][]{
\mintobjdump[firstline=4,lastline=10,#1]{../src/assembly/do\_asm.objdump}}
\begin{frame}{Instructions}
\begin{columns}[c]
\begin{column}{0.5\textwidth}
\onslide*<2->{
\begin{itemize}
\only<2>{\item\funcname{sub}: Substracts the source operand from the destination operand and stores the result in the destination operand}
\only<3-4>{\item \funcname{mov}: Copies the source operand to the destination operand}
\only<5>{\item \funcname{push}: Decrements the stack pointer and then stores the source operand on the top of the stack}
\only<6>{\item \funcname{pop}: Loads the value from the top of the stack to the location of the destination operand and then increments the stack pointer}
\only<7>{\item \funcname{call}: Saves RIP on the stack and branches to the procedure specified with the target operand}
\only<8>{\item \funcname{add}: Adds the source operand and the destination operand and stores the result in the destination operand}
\end{itemize}
\bigskip
}
\only<1>{\listc[firstline=8,lastline=19]{../src/assembly/assembly.c}{assembly/assembly.c}}%
\only<2>{\listassembly[highlightlines=4]{}}%
\only<3>{\listassembly[highlightlines=5]{}}%
\only<4>{\listassembly[highlightlines=6]{}}%
\only<5>{\listassembly[highlightlines=7]{}}%
\only<6>{\listassembly[highlightlines=8]{}}%
\only<7>{\listassembly[highlightlines=9]{}}%
\only<8>{\listassembly[highlightlines=10]{}}%
\end{column}
\begin{column}{0.5\textwidth}
\centering
\only<1>{\providecommand\step{1}\begin{tikzpicture}[font=\sffamily\tiny]
\input{../src/assembly/gdb.out}

\begin{stackDiagram}[yFactor=0.5,showAddress=true]{\doAsmBegin}{\sayHelloEnd}
\callStack

\ifthenelse{\step=1 \or \step=8}{
\begin{funFrame}[name=do\_asm,color=GreenYellow]{\doAsmBegin}{\doAsmSavedRbp}
}{}
\ifthenelse{\step>1 \and \step<5}{
\begin{funFrame}[name=do\_asm,color=GreenYellow]{\doAsmBegin}{\rspStepA}
}{}
\ifthenelse{\step=5}{
\begin{funFrame}[name=do\_asm,color=GreenYellow]{\doAsmBegin}{\rspStepB}
}{}
\ifthenelse{\step>5 \and \step<8}{
\begin{funFrame}[name=do\_asm,color=GreenYellow]{\doAsmBegin}{\rspStepA}
}{}
\funReturnAddr{\doAsmRetaddr}
\frameLocal{1}{saved RBP}
\ifthenelse{\step>2 \and \step<8}{
\frameLocal{3}{\doAsmLocalTwo}
}{}
\ifthenelse{\step>3 \and \step<8}{
\frameLocal{2}{\doAsmLocalOne}
}{}
\ifthenelse{\step=5}{
\frameLocal{4}{\doAsmLocalThree}
}{}
\end{funFrame}

\ifthenelse{\step=7}{
\begin{funFrame}[name=say\_hello,color=YellowGreen]{\sayHelloBegin}{\sayHelloEnd}
\funReturnAddr{\sayHelloRetaddr}
\frameLocal{1}{saved RBP}
\end{funFrame}
}{}



\ifthenelse{\step=1}{
\showRegister{RSP}{\doAsmSavedRbp}
}{}
\ifthenelse{\step>1 \and \step<5}{
\showRegister[color=Black!25]{RSP}{\doAsmSavedRbp}
\showRegister{RSP}{\rspStepA}
}{}
\ifthenelse{\step=5}{
\showRegister[color=Black!25]{RSP}{\rspStepA}
\showRegister{RSP}{\rspStepB}
}{}
\ifthenelse{\step=6}{
\showRegister[color=Black!25]{RSP}{\rspStepB}
\showRegister{RSP}{\rspStepC}
}{}
\ifthenelse{\step=7}{
\showRegister[color=Black!25]{RSP}{\rspStepC}
\showRegister{RSP}{\sayHelloEnd}
}{}
\ifthenelse{\step=8}{
\showRegister[color=Black!25]{RSP}{\rspStepC}
\showRegister{RSP}{\doAsmSavedRbp}
}{}


\end{stackDiagram}
\end{tikzpicture}
}%
\only<2>{\providecommand\step{2}\begin{tikzpicture}[font=\sffamily\tiny]
\input{../src/assembly/gdb.out}

\begin{stackDiagram}[yFactor=0.5,showAddress=true]{\doAsmBegin}{\sayHelloEnd}
\callStack

\ifthenelse{\step=1 \or \step=8}{
\begin{funFrame}[name=do\_asm,color=GreenYellow]{\doAsmBegin}{\doAsmSavedRbp}
}{}
\ifthenelse{\step>1 \and \step<5}{
\begin{funFrame}[name=do\_asm,color=GreenYellow]{\doAsmBegin}{\rspStepA}
}{}
\ifthenelse{\step=5}{
\begin{funFrame}[name=do\_asm,color=GreenYellow]{\doAsmBegin}{\rspStepB}
}{}
\ifthenelse{\step>5 \and \step<8}{
\begin{funFrame}[name=do\_asm,color=GreenYellow]{\doAsmBegin}{\rspStepA}
}{}
\funReturnAddr{\doAsmRetaddr}
\frameLocal{1}{saved RBP}
\ifthenelse{\step>2 \and \step<8}{
\frameLocal{3}{\doAsmLocalTwo}
}{}
\ifthenelse{\step>3 \and \step<8}{
\frameLocal{2}{\doAsmLocalOne}
}{}
\ifthenelse{\step=5}{
\frameLocal{4}{\doAsmLocalThree}
}{}
\end{funFrame}

\ifthenelse{\step=7}{
\begin{funFrame}[name=say\_hello,color=YellowGreen]{\sayHelloBegin}{\sayHelloEnd}
\funReturnAddr{\sayHelloRetaddr}
\frameLocal{1}{saved RBP}
\end{funFrame}
}{}



\ifthenelse{\step=1}{
\showRegister{RSP}{\doAsmSavedRbp}
}{}
\ifthenelse{\step>1 \and \step<5}{
\showRegister[color=Black!25]{RSP}{\doAsmSavedRbp}
\showRegister{RSP}{\rspStepA}
}{}
\ifthenelse{\step=5}{
\showRegister[color=Black!25]{RSP}{\rspStepA}
\showRegister{RSP}{\rspStepB}
}{}
\ifthenelse{\step=6}{
\showRegister[color=Black!25]{RSP}{\rspStepB}
\showRegister{RSP}{\rspStepC}
}{}
\ifthenelse{\step=7}{
\showRegister[color=Black!25]{RSP}{\rspStepC}
\showRegister{RSP}{\sayHelloEnd}
}{}
\ifthenelse{\step=8}{
\showRegister[color=Black!25]{RSP}{\rspStepC}
\showRegister{RSP}{\doAsmSavedRbp}
}{}


\end{stackDiagram}
\end{tikzpicture}
}%
\only<3>{\providecommand\step{3}\begin{tikzpicture}[font=\sffamily\tiny]
\input{../src/assembly/gdb.out}

\begin{stackDiagram}[yFactor=0.5,showAddress=true]{\doAsmBegin}{\sayHelloEnd}
\callStack

\ifthenelse{\step=1 \or \step=8}{
\begin{funFrame}[name=do\_asm,color=GreenYellow]{\doAsmBegin}{\doAsmSavedRbp}
}{}
\ifthenelse{\step>1 \and \step<5}{
\begin{funFrame}[name=do\_asm,color=GreenYellow]{\doAsmBegin}{\rspStepA}
}{}
\ifthenelse{\step=5}{
\begin{funFrame}[name=do\_asm,color=GreenYellow]{\doAsmBegin}{\rspStepB}
}{}
\ifthenelse{\step>5 \and \step<8}{
\begin{funFrame}[name=do\_asm,color=GreenYellow]{\doAsmBegin}{\rspStepA}
}{}
\funReturnAddr{\doAsmRetaddr}
\frameLocal{1}{saved RBP}
\ifthenelse{\step>2 \and \step<8}{
\frameLocal{3}{\doAsmLocalTwo}
}{}
\ifthenelse{\step>3 \and \step<8}{
\frameLocal{2}{\doAsmLocalOne}
}{}
\ifthenelse{\step=5}{
\frameLocal{4}{\doAsmLocalThree}
}{}
\end{funFrame}

\ifthenelse{\step=7}{
\begin{funFrame}[name=say\_hello,color=YellowGreen]{\sayHelloBegin}{\sayHelloEnd}
\funReturnAddr{\sayHelloRetaddr}
\frameLocal{1}{saved RBP}
\end{funFrame}
}{}



\ifthenelse{\step=1}{
\showRegister{RSP}{\doAsmSavedRbp}
}{}
\ifthenelse{\step>1 \and \step<5}{
\showRegister[color=Black!25]{RSP}{\doAsmSavedRbp}
\showRegister{RSP}{\rspStepA}
}{}
\ifthenelse{\step=5}{
\showRegister[color=Black!25]{RSP}{\rspStepA}
\showRegister{RSP}{\rspStepB}
}{}
\ifthenelse{\step=6}{
\showRegister[color=Black!25]{RSP}{\rspStepB}
\showRegister{RSP}{\rspStepC}
}{}
\ifthenelse{\step=7}{
\showRegister[color=Black!25]{RSP}{\rspStepC}
\showRegister{RSP}{\sayHelloEnd}
}{}
\ifthenelse{\step=8}{
\showRegister[color=Black!25]{RSP}{\rspStepC}
\showRegister{RSP}{\doAsmSavedRbp}
}{}


\end{stackDiagram}
\end{tikzpicture}
}%
\only<4>{\providecommand\step{4}\begin{tikzpicture}[font=\sffamily\tiny]
\input{../src/assembly/gdb.out}

\begin{stackDiagram}[yFactor=0.5,showAddress=true]{\doAsmBegin}{\sayHelloEnd}
\callStack

\ifthenelse{\step=1 \or \step=8}{
\begin{funFrame}[name=do\_asm,color=GreenYellow]{\doAsmBegin}{\doAsmSavedRbp}
}{}
\ifthenelse{\step>1 \and \step<5}{
\begin{funFrame}[name=do\_asm,color=GreenYellow]{\doAsmBegin}{\rspStepA}
}{}
\ifthenelse{\step=5}{
\begin{funFrame}[name=do\_asm,color=GreenYellow]{\doAsmBegin}{\rspStepB}
}{}
\ifthenelse{\step>5 \and \step<8}{
\begin{funFrame}[name=do\_asm,color=GreenYellow]{\doAsmBegin}{\rspStepA}
}{}
\funReturnAddr{\doAsmRetaddr}
\frameLocal{1}{saved RBP}
\ifthenelse{\step>2 \and \step<8}{
\frameLocal{3}{\doAsmLocalTwo}
}{}
\ifthenelse{\step>3 \and \step<8}{
\frameLocal{2}{\doAsmLocalOne}
}{}
\ifthenelse{\step=5}{
\frameLocal{4}{\doAsmLocalThree}
}{}
\end{funFrame}

\ifthenelse{\step=7}{
\begin{funFrame}[name=say\_hello,color=YellowGreen]{\sayHelloBegin}{\sayHelloEnd}
\funReturnAddr{\sayHelloRetaddr}
\frameLocal{1}{saved RBP}
\end{funFrame}
}{}



\ifthenelse{\step=1}{
\showRegister{RSP}{\doAsmSavedRbp}
}{}
\ifthenelse{\step>1 \and \step<5}{
\showRegister[color=Black!25]{RSP}{\doAsmSavedRbp}
\showRegister{RSP}{\rspStepA}
}{}
\ifthenelse{\step=5}{
\showRegister[color=Black!25]{RSP}{\rspStepA}
\showRegister{RSP}{\rspStepB}
}{}
\ifthenelse{\step=6}{
\showRegister[color=Black!25]{RSP}{\rspStepB}
\showRegister{RSP}{\rspStepC}
}{}
\ifthenelse{\step=7}{
\showRegister[color=Black!25]{RSP}{\rspStepC}
\showRegister{RSP}{\sayHelloEnd}
}{}
\ifthenelse{\step=8}{
\showRegister[color=Black!25]{RSP}{\rspStepC}
\showRegister{RSP}{\doAsmSavedRbp}
}{}


\end{stackDiagram}
\end{tikzpicture}
}%
\only<5>{\providecommand\step{5}\begin{tikzpicture}[font=\sffamily\tiny]
\input{../src/assembly/gdb.out}

\begin{stackDiagram}[yFactor=0.5,showAddress=true]{\doAsmBegin}{\sayHelloEnd}
\callStack

\ifthenelse{\step=1 \or \step=8}{
\begin{funFrame}[name=do\_asm,color=GreenYellow]{\doAsmBegin}{\doAsmSavedRbp}
}{}
\ifthenelse{\step>1 \and \step<5}{
\begin{funFrame}[name=do\_asm,color=GreenYellow]{\doAsmBegin}{\rspStepA}
}{}
\ifthenelse{\step=5}{
\begin{funFrame}[name=do\_asm,color=GreenYellow]{\doAsmBegin}{\rspStepB}
}{}
\ifthenelse{\step>5 \and \step<8}{
\begin{funFrame}[name=do\_asm,color=GreenYellow]{\doAsmBegin}{\rspStepA}
}{}
\funReturnAddr{\doAsmRetaddr}
\frameLocal{1}{saved RBP}
\ifthenelse{\step>2 \and \step<8}{
\frameLocal{3}{\doAsmLocalTwo}
}{}
\ifthenelse{\step>3 \and \step<8}{
\frameLocal{2}{\doAsmLocalOne}
}{}
\ifthenelse{\step=5}{
\frameLocal{4}{\doAsmLocalThree}
}{}
\end{funFrame}

\ifthenelse{\step=7}{
\begin{funFrame}[name=say\_hello,color=YellowGreen]{\sayHelloBegin}{\sayHelloEnd}
\funReturnAddr{\sayHelloRetaddr}
\frameLocal{1}{saved RBP}
\end{funFrame}
}{}



\ifthenelse{\step=1}{
\showRegister{RSP}{\doAsmSavedRbp}
}{}
\ifthenelse{\step>1 \and \step<5}{
\showRegister[color=Black!25]{RSP}{\doAsmSavedRbp}
\showRegister{RSP}{\rspStepA}
}{}
\ifthenelse{\step=5}{
\showRegister[color=Black!25]{RSP}{\rspStepA}
\showRegister{RSP}{\rspStepB}
}{}
\ifthenelse{\step=6}{
\showRegister[color=Black!25]{RSP}{\rspStepB}
\showRegister{RSP}{\rspStepC}
}{}
\ifthenelse{\step=7}{
\showRegister[color=Black!25]{RSP}{\rspStepC}
\showRegister{RSP}{\sayHelloEnd}
}{}
\ifthenelse{\step=8}{
\showRegister[color=Black!25]{RSP}{\rspStepC}
\showRegister{RSP}{\doAsmSavedRbp}
}{}


\end{stackDiagram}
\end{tikzpicture}
}%
\only<6>{\providecommand\step{6}\begin{tikzpicture}[font=\sffamily\tiny]
\input{../src/assembly/gdb.out}

\begin{stackDiagram}[yFactor=0.5,showAddress=true]{\doAsmBegin}{\sayHelloEnd}
\callStack

\ifthenelse{\step=1 \or \step=8}{
\begin{funFrame}[name=do\_asm,color=GreenYellow]{\doAsmBegin}{\doAsmSavedRbp}
}{}
\ifthenelse{\step>1 \and \step<5}{
\begin{funFrame}[name=do\_asm,color=GreenYellow]{\doAsmBegin}{\rspStepA}
}{}
\ifthenelse{\step=5}{
\begin{funFrame}[name=do\_asm,color=GreenYellow]{\doAsmBegin}{\rspStepB}
}{}
\ifthenelse{\step>5 \and \step<8}{
\begin{funFrame}[name=do\_asm,color=GreenYellow]{\doAsmBegin}{\rspStepA}
}{}
\funReturnAddr{\doAsmRetaddr}
\frameLocal{1}{saved RBP}
\ifthenelse{\step>2 \and \step<8}{
\frameLocal{3}{\doAsmLocalTwo}
}{}
\ifthenelse{\step>3 \and \step<8}{
\frameLocal{2}{\doAsmLocalOne}
}{}
\ifthenelse{\step=5}{
\frameLocal{4}{\doAsmLocalThree}
}{}
\end{funFrame}

\ifthenelse{\step=7}{
\begin{funFrame}[name=say\_hello,color=YellowGreen]{\sayHelloBegin}{\sayHelloEnd}
\funReturnAddr{\sayHelloRetaddr}
\frameLocal{1}{saved RBP}
\end{funFrame}
}{}



\ifthenelse{\step=1}{
\showRegister{RSP}{\doAsmSavedRbp}
}{}
\ifthenelse{\step>1 \and \step<5}{
\showRegister[color=Black!25]{RSP}{\doAsmSavedRbp}
\showRegister{RSP}{\rspStepA}
}{}
\ifthenelse{\step=5}{
\showRegister[color=Black!25]{RSP}{\rspStepA}
\showRegister{RSP}{\rspStepB}
}{}
\ifthenelse{\step=6}{
\showRegister[color=Black!25]{RSP}{\rspStepB}
\showRegister{RSP}{\rspStepC}
}{}
\ifthenelse{\step=7}{
\showRegister[color=Black!25]{RSP}{\rspStepC}
\showRegister{RSP}{\sayHelloEnd}
}{}
\ifthenelse{\step=8}{
\showRegister[color=Black!25]{RSP}{\rspStepC}
\showRegister{RSP}{\doAsmSavedRbp}
}{}


\end{stackDiagram}
\end{tikzpicture}
}%
\only<7>{\providecommand\step{7}\begin{tikzpicture}[font=\sffamily\tiny]
\input{../src/assembly/gdb.out}

\begin{stackDiagram}[yFactor=0.5,showAddress=true]{\doAsmBegin}{\sayHelloEnd}
\callStack

\ifthenelse{\step=1 \or \step=8}{
\begin{funFrame}[name=do\_asm,color=GreenYellow]{\doAsmBegin}{\doAsmSavedRbp}
}{}
\ifthenelse{\step>1 \and \step<5}{
\begin{funFrame}[name=do\_asm,color=GreenYellow]{\doAsmBegin}{\rspStepA}
}{}
\ifthenelse{\step=5}{
\begin{funFrame}[name=do\_asm,color=GreenYellow]{\doAsmBegin}{\rspStepB}
}{}
\ifthenelse{\step>5 \and \step<8}{
\begin{funFrame}[name=do\_asm,color=GreenYellow]{\doAsmBegin}{\rspStepA}
}{}
\funReturnAddr{\doAsmRetaddr}
\frameLocal{1}{saved RBP}
\ifthenelse{\step>2 \and \step<8}{
\frameLocal{3}{\doAsmLocalTwo}
}{}
\ifthenelse{\step>3 \and \step<8}{
\frameLocal{2}{\doAsmLocalOne}
}{}
\ifthenelse{\step=5}{
\frameLocal{4}{\doAsmLocalThree}
}{}
\end{funFrame}

\ifthenelse{\step=7}{
\begin{funFrame}[name=say\_hello,color=YellowGreen]{\sayHelloBegin}{\sayHelloEnd}
\funReturnAddr{\sayHelloRetaddr}
\frameLocal{1}{saved RBP}
\end{funFrame}
}{}



\ifthenelse{\step=1}{
\showRegister{RSP}{\doAsmSavedRbp}
}{}
\ifthenelse{\step>1 \and \step<5}{
\showRegister[color=Black!25]{RSP}{\doAsmSavedRbp}
\showRegister{RSP}{\rspStepA}
}{}
\ifthenelse{\step=5}{
\showRegister[color=Black!25]{RSP}{\rspStepA}
\showRegister{RSP}{\rspStepB}
}{}
\ifthenelse{\step=6}{
\showRegister[color=Black!25]{RSP}{\rspStepB}
\showRegister{RSP}{\rspStepC}
}{}
\ifthenelse{\step=7}{
\showRegister[color=Black!25]{RSP}{\rspStepC}
\showRegister{RSP}{\sayHelloEnd}
}{}
\ifthenelse{\step=8}{
\showRegister[color=Black!25]{RSP}{\rspStepC}
\showRegister{RSP}{\doAsmSavedRbp}
}{}


\end{stackDiagram}
\end{tikzpicture}
}%
\only<8>{\providecommand\step{8}\begin{tikzpicture}[font=\sffamily\tiny]
\input{../src/assembly/gdb.out}

\begin{stackDiagram}[yFactor=0.5,showAddress=true]{\doAsmBegin}{\sayHelloEnd}
\callStack

\ifthenelse{\step=1 \or \step=8}{
\begin{funFrame}[name=do\_asm,color=GreenYellow]{\doAsmBegin}{\doAsmSavedRbp}
}{}
\ifthenelse{\step>1 \and \step<5}{
\begin{funFrame}[name=do\_asm,color=GreenYellow]{\doAsmBegin}{\rspStepA}
}{}
\ifthenelse{\step=5}{
\begin{funFrame}[name=do\_asm,color=GreenYellow]{\doAsmBegin}{\rspStepB}
}{}
\ifthenelse{\step>5 \and \step<8}{
\begin{funFrame}[name=do\_asm,color=GreenYellow]{\doAsmBegin}{\rspStepA}
}{}
\funReturnAddr{\doAsmRetaddr}
\frameLocal{1}{saved RBP}
\ifthenelse{\step>2 \and \step<8}{
\frameLocal{3}{\doAsmLocalTwo}
}{}
\ifthenelse{\step>3 \and \step<8}{
\frameLocal{2}{\doAsmLocalOne}
}{}
\ifthenelse{\step=5}{
\frameLocal{4}{\doAsmLocalThree}
}{}
\end{funFrame}

\ifthenelse{\step=7}{
\begin{funFrame}[name=say\_hello,color=YellowGreen]{\sayHelloBegin}{\sayHelloEnd}
\funReturnAddr{\sayHelloRetaddr}
\frameLocal{1}{saved RBP}
\end{funFrame}
}{}



\ifthenelse{\step=1}{
\showRegister{RSP}{\doAsmSavedRbp}
}{}
\ifthenelse{\step>1 \and \step<5}{
\showRegister[color=Black!25]{RSP}{\doAsmSavedRbp}
\showRegister{RSP}{\rspStepA}
}{}
\ifthenelse{\step=5}{
\showRegister[color=Black!25]{RSP}{\rspStepA}
\showRegister{RSP}{\rspStepB}
}{}
\ifthenelse{\step=6}{
\showRegister[color=Black!25]{RSP}{\rspStepB}
\showRegister{RSP}{\rspStepC}
}{}
\ifthenelse{\step=7}{
\showRegister[color=Black!25]{RSP}{\rspStepC}
\showRegister{RSP}{\sayHelloEnd}
}{}
\ifthenelse{\step=8}{
\showRegister[color=Black!25]{RSP}{\rspStepC}
\showRegister{RSP}{\doAsmSavedRbp}
}{}


\end{stackDiagram}
\end{tikzpicture}
}%
\end{column}
\end{columns}
\end{frame}
% https://www.felixcloutier.com/x86/
