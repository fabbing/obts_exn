%
% Runtime's default exception handler
%
\subsection*{Runtime's default exception handler}
\frameSubsection{pictures/The\_Architect.png}{}


\begin{frame}{Default exception handler}
\begin{columns}
\begin{column}{0.5\textwidth}
\begin{itemize}
\item \localname{domain\_state->exn\_handler} points to a trap located before \funcname{entry}!
\item What installed this very first trap there?
\end{itemize}
\bigskip
\listocaml{../src/default/default.ml}{default/default.ml}
\bigskip
\inputminted[fontsize=\footnotesize]{text}{../src/default/default.out}
\end{column}
\begin{column}{0.5\textwidth}
\centering
\only<1>{\providecommand\step{0}%
% Runtime's default exception handler
%
\subsection*{Runtime's default exception handler}
\frameSubsection{pictures/The\_Architect.png}{}


\begin{frame}{Default exception handler}
\begin{columns}
\begin{column}{0.5\textwidth}
\begin{itemize}
\item \localname{domain\_state->exn\_handler} points to a trap located before \funcname{entry}!
\item What installed this very first trap there?
\end{itemize}
\bigskip
\listocaml{../src/default/default.ml}{default/default.ml}
\bigskip
\inputminted[fontsize=\footnotesize]{text}{../src/default/default.out}
\end{column}
\begin{column}{0.5\textwidth}
\centering
\only<1>{\providecommand\step{0}%
% Runtime's default exception handler
%
\subsection*{Runtime's default exception handler}
\frameSubsection{pictures/The\_Architect.png}{}


\begin{frame}{Default exception handler}
\begin{columns}
\begin{column}{0.5\textwidth}
\begin{itemize}
\item \localname{domain\_state->exn\_handler} points to a trap located before \funcname{entry}!
\item What installed this very first trap there?
\end{itemize}
\bigskip
\listocaml{../src/default/default.ml}{default/default.ml}
\bigskip
\inputminted[fontsize=\footnotesize]{text}{../src/default/default.out}
\end{column}
\begin{column}{0.5\textwidth}
\centering
\only<1>{\providecommand\step{0}%
% Runtime's default exception handler
%
\subsection*{Runtime's default exception handler}
\frameSubsection{pictures/The\_Architect.png}{}


\begin{frame}{Default exception handler}
\begin{columns}
\begin{column}{0.5\textwidth}
\begin{itemize}
\item \localname{domain\_state->exn\_handler} points to a trap located before \funcname{entry}!
\item What installed this very first trap there?
\end{itemize}
\bigskip
\listocaml{../src/default/default.ml}{default/default.ml}
\bigskip
\inputminted[fontsize=\footnotesize]{text}{../src/default/default.out}
\end{column}
\begin{column}{0.5\textwidth}
\centering
\only<1>{\providecommand\step{0}\input{diagrams/default/default.tex}}%
\only<2>{\providecommand\step{4}\input{diagrams/default/default.tex}}%
\end{column}
\end{columns}
\end{frame}


\begin{frame}{Startup}{How did we get there by the way?}
\begin{columns}
\begin{column}{0.45\textwidth}
\begin{itemize}
\item The entry point address of the binary is \funcname{main} (\funcname{\_start}) from the runtime
\item The program starts like a C program:
\begin{itemize}
\item \funcname{caml\_start\_program}
\item \funcname{caml\_startup\_common}
\item \funcname{caml\_startup\_exn}
\item \funcname{caml\_startup}
\item \funcname{caml\_main}
\item \funcname{main}
\end{itemize}
%\item \funcname{caml\_startup\_common}: initializes GC, domains \dots
% Also: codefrag, locale, custom opeartions, frame_descr, signals, stack
\item \funcname{caml\_start\_program} is the transition point between the C realm and the OCaml one
\end{itemize}
\bigskip
\listocaml{../src/default/default.ml}{default/default.ml}
\end{column}
\begin{column}{0.55\textwidth}
\listasm[lastline=24,highlightlines={22-24}]{src/ocaml/caml\_start\_program.s}{runtime/amd64.S:604}
\end{column}
\end{columns}
\end{frame}


\begin{frame}{Setup to jump into OCaml entry point}{\funcname{caml\_start\_program} initializes the main fiber}
\begin{columns}
\begin{column}{0.6\textwidth}
\only<1,3,5>{
\begin{itemize}
\item<1-> Stores information to allow switching from OCaml stack to the C stack
\item<3-> Constructs the default exception handler
\item<5-> Switches to the OCaml stack and calls \funcname{caml\_program}
\end{itemize}
\bigskip
\listocaml{../src/default/default.ml}{default/default.ml}
}%
\only<2>{
\listasm[firstline=22,lastline=41,highlightlines={25-27}]{src/ocaml/caml\_start\_program.s}{caml\_start\_program}
}%
\only<4>{
\mintasm[firstline=22,lastline=41,highlightlines={31-38}]{src/ocaml/caml\_start\_program.s}
\listasm[firstline=66,lastline=72]{src/ocaml/caml\_start\_program.s}{caml\_start\_program}
}%
\only<6>{
\listasm[firstline=22,lastline=41,highlightlines={39-41}]{src/ocaml/caml\_start\_program.s}{caml\_start\_program}
}%
\end{column}
\begin{column}{0.4\textwidth}
\centering
\only<1-2>{\providecommand\step{2}\input{diagrams/default/default.tex}}%
\only<3-5>{\providecommand\step{3}\input{diagrams/default/default.tex}}%
\onslide*<6->{\providecommand\step{4}\input{diagrams/default/default.tex}}%
\end{column}
\end{columns}
\end{frame}

\begin{frame}{Executing the OCaml program}{And raising an exception without handler}
\begin{columns}
\begin{column}{0.6\textwidth}
\only<1-3>{
\begin{itemize}
\item<1-> Execution continues with OCaml code
\begin{itemize}
\item \funcname{camlDefault\_\_main\_269}
\item \funcname{camlDefault\_\_entry}
\item \funcname{caml\_program}
\end{itemize}
\item<1-> \funcname{camlDefault\_\_main\_269} raises a \typename{Failure} exception
\item<1-> \typename{Failure} is caught by \funcname{caml\_start\_program}
\begin{itemize}
\item<1-> The handler set the 2nd LSB of the exception value to \funcarg{1}
\item<3-> And then forces \funcname{caml\_start\_program} to terminate
\end{itemize}
\end{itemize}
}
\bigskip
\only<1,3>{\listocaml[highlightlines=3]{../src/default/default.ml}{default/default.ml}}%
\only<2>{\listasm[firstline=66,lastline=72,highlightlines=69]{src/ocaml/caml\_start\_program.s}{caml\_start\_program}}%
\end{column}
\begin{column}{0.4\textwidth}
\centering
\providecommand\step{4}\input{diagrams/default/default.tex}
\end{column}
\end{columns}
\end{frame}
%\only<>{\listasm[firstline=47,lastline=65]{src/ocaml/caml\_start\_program.s}{caml\_start\_program}}


\begin{frame}{Back to C}{Clean up, complain and terminate}
\begin{columns}
\begin{column}{0.45\textwidth}
\begin{itemize}
\item Backtrace:
\begin{itemize}
\item \funcname{caml\_start\_program} $\rightarrow$ OCaml code
\item \funcname{caml\_startup\_common}
\item \funcname{caml\_startup\_exn}
\item \funcname{caml\_startup}
\item \funcname{caml\_main}
\item \funcname{main}
\end{itemize}
\smallskip
\item \funcname{caml\_startup} checks the return value of \funcname{caml\_startup\_exn}
\begin{itemize}
\item The return value has been specially marked by \funcname{caml\_start\_program} handler
\item Calls \funcname{caml\_fatal\_uncaught\_exception}
\end{itemize}
\item<2-> \funcname{caml\_fatal\_uncaught\_exception} either
\begin{itemize}
\item<2-> Calls the user custom uncaught exception handler, if set with \mintinline{ocaml}{Printexc.set_uncaught_exception_handler fn}\footnotemark
\item<2-> Or falls back to the default one
\item<2-> Which notifies about the uncaught exception
\end{itemize}
\item<4-> Terminates the program either with \funcname{abort} or with a non-zero exit code
\end{itemize}
\end{column}
\begin{column}{0.55\textwidth}
\only<1>{
\listc[firstline=84,lastline=89,highlightlines=88]{../ocaml/runtime/caml/mlvalues.h}{runtime/caml/mlvalues.h:84}
\listc[firstline=138,lastline=143,highlightlines={141-142}]{../ocaml/runtime/startup\_nat.c}{runtime/startup\_nat.c:138}
}%
\only<2>{
\listc[firstline=138,highlightlines={147,149}]{../ocaml/runtime/printexc.c}{runtime/printexc.c:138}
}%
\only<3>{
\listc[firstline=110,lastline=134,highlightlines=129]{../ocaml/runtime/printexc.c}{runtime/printexc.c:138}
}%
\only<4>{
\listc[firstline=138,highlightlines={152,154}]{../ocaml/runtime/printexc.c}{runtime/printexc.c:138}
}%
\end{column}
\end{columns}
\footnotetext[1]{\url{https://v2.ocaml.org/api/Printexc.html\#1\_Uncaughtexceptions}}
\end{frame}
}%
\only<2>{\providecommand\step{4}%
% Runtime's default exception handler
%
\subsection*{Runtime's default exception handler}
\frameSubsection{pictures/The\_Architect.png}{}


\begin{frame}{Default exception handler}
\begin{columns}
\begin{column}{0.5\textwidth}
\begin{itemize}
\item \localname{domain\_state->exn\_handler} points to a trap located before \funcname{entry}!
\item What installed this very first trap there?
\end{itemize}
\bigskip
\listocaml{../src/default/default.ml}{default/default.ml}
\bigskip
\inputminted[fontsize=\footnotesize]{text}{../src/default/default.out}
\end{column}
\begin{column}{0.5\textwidth}
\centering
\only<1>{\providecommand\step{0}\input{diagrams/default/default.tex}}%
\only<2>{\providecommand\step{4}\input{diagrams/default/default.tex}}%
\end{column}
\end{columns}
\end{frame}


\begin{frame}{Startup}{How did we get there by the way?}
\begin{columns}
\begin{column}{0.45\textwidth}
\begin{itemize}
\item The entry point address of the binary is \funcname{main} (\funcname{\_start}) from the runtime
\item The program starts like a C program:
\begin{itemize}
\item \funcname{caml\_start\_program}
\item \funcname{caml\_startup\_common}
\item \funcname{caml\_startup\_exn}
\item \funcname{caml\_startup}
\item \funcname{caml\_main}
\item \funcname{main}
\end{itemize}
%\item \funcname{caml\_startup\_common}: initializes GC, domains \dots
% Also: codefrag, locale, custom opeartions, frame_descr, signals, stack
\item \funcname{caml\_start\_program} is the transition point between the C realm and the OCaml one
\end{itemize}
\bigskip
\listocaml{../src/default/default.ml}{default/default.ml}
\end{column}
\begin{column}{0.55\textwidth}
\listasm[lastline=24,highlightlines={22-24}]{src/ocaml/caml\_start\_program.s}{runtime/amd64.S:604}
\end{column}
\end{columns}
\end{frame}


\begin{frame}{Setup to jump into OCaml entry point}{\funcname{caml\_start\_program} initializes the main fiber}
\begin{columns}
\begin{column}{0.6\textwidth}
\only<1,3,5>{
\begin{itemize}
\item<1-> Stores information to allow switching from OCaml stack to the C stack
\item<3-> Constructs the default exception handler
\item<5-> Switches to the OCaml stack and calls \funcname{caml\_program}
\end{itemize}
\bigskip
\listocaml{../src/default/default.ml}{default/default.ml}
}%
\only<2>{
\listasm[firstline=22,lastline=41,highlightlines={25-27}]{src/ocaml/caml\_start\_program.s}{caml\_start\_program}
}%
\only<4>{
\mintasm[firstline=22,lastline=41,highlightlines={31-38}]{src/ocaml/caml\_start\_program.s}
\listasm[firstline=66,lastline=72]{src/ocaml/caml\_start\_program.s}{caml\_start\_program}
}%
\only<6>{
\listasm[firstline=22,lastline=41,highlightlines={39-41}]{src/ocaml/caml\_start\_program.s}{caml\_start\_program}
}%
\end{column}
\begin{column}{0.4\textwidth}
\centering
\only<1-2>{\providecommand\step{2}\input{diagrams/default/default.tex}}%
\only<3-5>{\providecommand\step{3}\input{diagrams/default/default.tex}}%
\onslide*<6->{\providecommand\step{4}\input{diagrams/default/default.tex}}%
\end{column}
\end{columns}
\end{frame}

\begin{frame}{Executing the OCaml program}{And raising an exception without handler}
\begin{columns}
\begin{column}{0.6\textwidth}
\only<1-3>{
\begin{itemize}
\item<1-> Execution continues with OCaml code
\begin{itemize}
\item \funcname{camlDefault\_\_main\_269}
\item \funcname{camlDefault\_\_entry}
\item \funcname{caml\_program}
\end{itemize}
\item<1-> \funcname{camlDefault\_\_main\_269} raises a \typename{Failure} exception
\item<1-> \typename{Failure} is caught by \funcname{caml\_start\_program}
\begin{itemize}
\item<1-> The handler set the 2nd LSB of the exception value to \funcarg{1}
\item<3-> And then forces \funcname{caml\_start\_program} to terminate
\end{itemize}
\end{itemize}
}
\bigskip
\only<1,3>{\listocaml[highlightlines=3]{../src/default/default.ml}{default/default.ml}}%
\only<2>{\listasm[firstline=66,lastline=72,highlightlines=69]{src/ocaml/caml\_start\_program.s}{caml\_start\_program}}%
\end{column}
\begin{column}{0.4\textwidth}
\centering
\providecommand\step{4}\input{diagrams/default/default.tex}
\end{column}
\end{columns}
\end{frame}
%\only<>{\listasm[firstline=47,lastline=65]{src/ocaml/caml\_start\_program.s}{caml\_start\_program}}


\begin{frame}{Back to C}{Clean up, complain and terminate}
\begin{columns}
\begin{column}{0.45\textwidth}
\begin{itemize}
\item Backtrace:
\begin{itemize}
\item \funcname{caml\_start\_program} $\rightarrow$ OCaml code
\item \funcname{caml\_startup\_common}
\item \funcname{caml\_startup\_exn}
\item \funcname{caml\_startup}
\item \funcname{caml\_main}
\item \funcname{main}
\end{itemize}
\smallskip
\item \funcname{caml\_startup} checks the return value of \funcname{caml\_startup\_exn}
\begin{itemize}
\item The return value has been specially marked by \funcname{caml\_start\_program} handler
\item Calls \funcname{caml\_fatal\_uncaught\_exception}
\end{itemize}
\item<2-> \funcname{caml\_fatal\_uncaught\_exception} either
\begin{itemize}
\item<2-> Calls the user custom uncaught exception handler, if set with \mintinline{ocaml}{Printexc.set_uncaught_exception_handler fn}\footnotemark
\item<2-> Or falls back to the default one
\item<2-> Which notifies about the uncaught exception
\end{itemize}
\item<4-> Terminates the program either with \funcname{abort} or with a non-zero exit code
\end{itemize}
\end{column}
\begin{column}{0.55\textwidth}
\only<1>{
\listc[firstline=84,lastline=89,highlightlines=88]{../ocaml/runtime/caml/mlvalues.h}{runtime/caml/mlvalues.h:84}
\listc[firstline=138,lastline=143,highlightlines={141-142}]{../ocaml/runtime/startup\_nat.c}{runtime/startup\_nat.c:138}
}%
\only<2>{
\listc[firstline=138,highlightlines={147,149}]{../ocaml/runtime/printexc.c}{runtime/printexc.c:138}
}%
\only<3>{
\listc[firstline=110,lastline=134,highlightlines=129]{../ocaml/runtime/printexc.c}{runtime/printexc.c:138}
}%
\only<4>{
\listc[firstline=138,highlightlines={152,154}]{../ocaml/runtime/printexc.c}{runtime/printexc.c:138}
}%
\end{column}
\end{columns}
\footnotetext[1]{\url{https://v2.ocaml.org/api/Printexc.html\#1\_Uncaughtexceptions}}
\end{frame}
}%
\end{column}
\end{columns}
\end{frame}


\begin{frame}{Startup}{How did we get there by the way?}
\begin{columns}
\begin{column}{0.45\textwidth}
\begin{itemize}
\item The entry point address of the binary is \funcname{main} (\funcname{\_start}) from the runtime
\item The program starts like a C program:
\begin{itemize}
\item \funcname{caml\_start\_program}
\item \funcname{caml\_startup\_common}
\item \funcname{caml\_startup\_exn}
\item \funcname{caml\_startup}
\item \funcname{caml\_main}
\item \funcname{main}
\end{itemize}
%\item \funcname{caml\_startup\_common}: initializes GC, domains \dots
% Also: codefrag, locale, custom opeartions, frame_descr, signals, stack
\item \funcname{caml\_start\_program} is the transition point between the C realm and the OCaml one
\end{itemize}
\bigskip
\listocaml{../src/default/default.ml}{default/default.ml}
\end{column}
\begin{column}{0.55\textwidth}
\listasm[lastline=24,highlightlines={22-24}]{src/ocaml/caml\_start\_program.s}{runtime/amd64.S:604}
\end{column}
\end{columns}
\end{frame}


\begin{frame}{Setup to jump into OCaml entry point}{\funcname{caml\_start\_program} initializes the main fiber}
\begin{columns}
\begin{column}{0.6\textwidth}
\only<1,3,5>{
\begin{itemize}
\item<1-> Stores information to allow switching from OCaml stack to the C stack
\item<3-> Constructs the default exception handler
\item<5-> Switches to the OCaml stack and calls \funcname{caml\_program}
\end{itemize}
\bigskip
\listocaml{../src/default/default.ml}{default/default.ml}
}%
\only<2>{
\listasm[firstline=22,lastline=41,highlightlines={25-27}]{src/ocaml/caml\_start\_program.s}{caml\_start\_program}
}%
\only<4>{
\mintasm[firstline=22,lastline=41,highlightlines={31-38}]{src/ocaml/caml\_start\_program.s}
\listasm[firstline=66,lastline=72]{src/ocaml/caml\_start\_program.s}{caml\_start\_program}
}%
\only<6>{
\listasm[firstline=22,lastline=41,highlightlines={39-41}]{src/ocaml/caml\_start\_program.s}{caml\_start\_program}
}%
\end{column}
\begin{column}{0.4\textwidth}
\centering
\only<1-2>{\providecommand\step{2}%
% Runtime's default exception handler
%
\subsection*{Runtime's default exception handler}
\frameSubsection{pictures/The\_Architect.png}{}


\begin{frame}{Default exception handler}
\begin{columns}
\begin{column}{0.5\textwidth}
\begin{itemize}
\item \localname{domain\_state->exn\_handler} points to a trap located before \funcname{entry}!
\item What installed this very first trap there?
\end{itemize}
\bigskip
\listocaml{../src/default/default.ml}{default/default.ml}
\bigskip
\inputminted[fontsize=\footnotesize]{text}{../src/default/default.out}
\end{column}
\begin{column}{0.5\textwidth}
\centering
\only<1>{\providecommand\step{0}\input{diagrams/default/default.tex}}%
\only<2>{\providecommand\step{4}\input{diagrams/default/default.tex}}%
\end{column}
\end{columns}
\end{frame}


\begin{frame}{Startup}{How did we get there by the way?}
\begin{columns}
\begin{column}{0.45\textwidth}
\begin{itemize}
\item The entry point address of the binary is \funcname{main} (\funcname{\_start}) from the runtime
\item The program starts like a C program:
\begin{itemize}
\item \funcname{caml\_start\_program}
\item \funcname{caml\_startup\_common}
\item \funcname{caml\_startup\_exn}
\item \funcname{caml\_startup}
\item \funcname{caml\_main}
\item \funcname{main}
\end{itemize}
%\item \funcname{caml\_startup\_common}: initializes GC, domains \dots
% Also: codefrag, locale, custom opeartions, frame_descr, signals, stack
\item \funcname{caml\_start\_program} is the transition point between the C realm and the OCaml one
\end{itemize}
\bigskip
\listocaml{../src/default/default.ml}{default/default.ml}
\end{column}
\begin{column}{0.55\textwidth}
\listasm[lastline=24,highlightlines={22-24}]{src/ocaml/caml\_start\_program.s}{runtime/amd64.S:604}
\end{column}
\end{columns}
\end{frame}


\begin{frame}{Setup to jump into OCaml entry point}{\funcname{caml\_start\_program} initializes the main fiber}
\begin{columns}
\begin{column}{0.6\textwidth}
\only<1,3,5>{
\begin{itemize}
\item<1-> Stores information to allow switching from OCaml stack to the C stack
\item<3-> Constructs the default exception handler
\item<5-> Switches to the OCaml stack and calls \funcname{caml\_program}
\end{itemize}
\bigskip
\listocaml{../src/default/default.ml}{default/default.ml}
}%
\only<2>{
\listasm[firstline=22,lastline=41,highlightlines={25-27}]{src/ocaml/caml\_start\_program.s}{caml\_start\_program}
}%
\only<4>{
\mintasm[firstline=22,lastline=41,highlightlines={31-38}]{src/ocaml/caml\_start\_program.s}
\listasm[firstline=66,lastline=72]{src/ocaml/caml\_start\_program.s}{caml\_start\_program}
}%
\only<6>{
\listasm[firstline=22,lastline=41,highlightlines={39-41}]{src/ocaml/caml\_start\_program.s}{caml\_start\_program}
}%
\end{column}
\begin{column}{0.4\textwidth}
\centering
\only<1-2>{\providecommand\step{2}\input{diagrams/default/default.tex}}%
\only<3-5>{\providecommand\step{3}\input{diagrams/default/default.tex}}%
\onslide*<6->{\providecommand\step{4}\input{diagrams/default/default.tex}}%
\end{column}
\end{columns}
\end{frame}

\begin{frame}{Executing the OCaml program}{And raising an exception without handler}
\begin{columns}
\begin{column}{0.6\textwidth}
\only<1-3>{
\begin{itemize}
\item<1-> Execution continues with OCaml code
\begin{itemize}
\item \funcname{camlDefault\_\_main\_269}
\item \funcname{camlDefault\_\_entry}
\item \funcname{caml\_program}
\end{itemize}
\item<1-> \funcname{camlDefault\_\_main\_269} raises a \typename{Failure} exception
\item<1-> \typename{Failure} is caught by \funcname{caml\_start\_program}
\begin{itemize}
\item<1-> The handler set the 2nd LSB of the exception value to \funcarg{1}
\item<3-> And then forces \funcname{caml\_start\_program} to terminate
\end{itemize}
\end{itemize}
}
\bigskip
\only<1,3>{\listocaml[highlightlines=3]{../src/default/default.ml}{default/default.ml}}%
\only<2>{\listasm[firstline=66,lastline=72,highlightlines=69]{src/ocaml/caml\_start\_program.s}{caml\_start\_program}}%
\end{column}
\begin{column}{0.4\textwidth}
\centering
\providecommand\step{4}\input{diagrams/default/default.tex}
\end{column}
\end{columns}
\end{frame}
%\only<>{\listasm[firstline=47,lastline=65]{src/ocaml/caml\_start\_program.s}{caml\_start\_program}}


\begin{frame}{Back to C}{Clean up, complain and terminate}
\begin{columns}
\begin{column}{0.45\textwidth}
\begin{itemize}
\item Backtrace:
\begin{itemize}
\item \funcname{caml\_start\_program} $\rightarrow$ OCaml code
\item \funcname{caml\_startup\_common}
\item \funcname{caml\_startup\_exn}
\item \funcname{caml\_startup}
\item \funcname{caml\_main}
\item \funcname{main}
\end{itemize}
\smallskip
\item \funcname{caml\_startup} checks the return value of \funcname{caml\_startup\_exn}
\begin{itemize}
\item The return value has been specially marked by \funcname{caml\_start\_program} handler
\item Calls \funcname{caml\_fatal\_uncaught\_exception}
\end{itemize}
\item<2-> \funcname{caml\_fatal\_uncaught\_exception} either
\begin{itemize}
\item<2-> Calls the user custom uncaught exception handler, if set with \mintinline{ocaml}{Printexc.set_uncaught_exception_handler fn}\footnotemark
\item<2-> Or falls back to the default one
\item<2-> Which notifies about the uncaught exception
\end{itemize}
\item<4-> Terminates the program either with \funcname{abort} or with a non-zero exit code
\end{itemize}
\end{column}
\begin{column}{0.55\textwidth}
\only<1>{
\listc[firstline=84,lastline=89,highlightlines=88]{../ocaml/runtime/caml/mlvalues.h}{runtime/caml/mlvalues.h:84}
\listc[firstline=138,lastline=143,highlightlines={141-142}]{../ocaml/runtime/startup\_nat.c}{runtime/startup\_nat.c:138}
}%
\only<2>{
\listc[firstline=138,highlightlines={147,149}]{../ocaml/runtime/printexc.c}{runtime/printexc.c:138}
}%
\only<3>{
\listc[firstline=110,lastline=134,highlightlines=129]{../ocaml/runtime/printexc.c}{runtime/printexc.c:138}
}%
\only<4>{
\listc[firstline=138,highlightlines={152,154}]{../ocaml/runtime/printexc.c}{runtime/printexc.c:138}
}%
\end{column}
\end{columns}
\footnotetext[1]{\url{https://v2.ocaml.org/api/Printexc.html\#1\_Uncaughtexceptions}}
\end{frame}
}%
\only<3-5>{\providecommand\step{3}%
% Runtime's default exception handler
%
\subsection*{Runtime's default exception handler}
\frameSubsection{pictures/The\_Architect.png}{}


\begin{frame}{Default exception handler}
\begin{columns}
\begin{column}{0.5\textwidth}
\begin{itemize}
\item \localname{domain\_state->exn\_handler} points to a trap located before \funcname{entry}!
\item What installed this very first trap there?
\end{itemize}
\bigskip
\listocaml{../src/default/default.ml}{default/default.ml}
\bigskip
\inputminted[fontsize=\footnotesize]{text}{../src/default/default.out}
\end{column}
\begin{column}{0.5\textwidth}
\centering
\only<1>{\providecommand\step{0}\input{diagrams/default/default.tex}}%
\only<2>{\providecommand\step{4}\input{diagrams/default/default.tex}}%
\end{column}
\end{columns}
\end{frame}


\begin{frame}{Startup}{How did we get there by the way?}
\begin{columns}
\begin{column}{0.45\textwidth}
\begin{itemize}
\item The entry point address of the binary is \funcname{main} (\funcname{\_start}) from the runtime
\item The program starts like a C program:
\begin{itemize}
\item \funcname{caml\_start\_program}
\item \funcname{caml\_startup\_common}
\item \funcname{caml\_startup\_exn}
\item \funcname{caml\_startup}
\item \funcname{caml\_main}
\item \funcname{main}
\end{itemize}
%\item \funcname{caml\_startup\_common}: initializes GC, domains \dots
% Also: codefrag, locale, custom opeartions, frame_descr, signals, stack
\item \funcname{caml\_start\_program} is the transition point between the C realm and the OCaml one
\end{itemize}
\bigskip
\listocaml{../src/default/default.ml}{default/default.ml}
\end{column}
\begin{column}{0.55\textwidth}
\listasm[lastline=24,highlightlines={22-24}]{src/ocaml/caml\_start\_program.s}{runtime/amd64.S:604}
\end{column}
\end{columns}
\end{frame}


\begin{frame}{Setup to jump into OCaml entry point}{\funcname{caml\_start\_program} initializes the main fiber}
\begin{columns}
\begin{column}{0.6\textwidth}
\only<1,3,5>{
\begin{itemize}
\item<1-> Stores information to allow switching from OCaml stack to the C stack
\item<3-> Constructs the default exception handler
\item<5-> Switches to the OCaml stack and calls \funcname{caml\_program}
\end{itemize}
\bigskip
\listocaml{../src/default/default.ml}{default/default.ml}
}%
\only<2>{
\listasm[firstline=22,lastline=41,highlightlines={25-27}]{src/ocaml/caml\_start\_program.s}{caml\_start\_program}
}%
\only<4>{
\mintasm[firstline=22,lastline=41,highlightlines={31-38}]{src/ocaml/caml\_start\_program.s}
\listasm[firstline=66,lastline=72]{src/ocaml/caml\_start\_program.s}{caml\_start\_program}
}%
\only<6>{
\listasm[firstline=22,lastline=41,highlightlines={39-41}]{src/ocaml/caml\_start\_program.s}{caml\_start\_program}
}%
\end{column}
\begin{column}{0.4\textwidth}
\centering
\only<1-2>{\providecommand\step{2}\input{diagrams/default/default.tex}}%
\only<3-5>{\providecommand\step{3}\input{diagrams/default/default.tex}}%
\onslide*<6->{\providecommand\step{4}\input{diagrams/default/default.tex}}%
\end{column}
\end{columns}
\end{frame}

\begin{frame}{Executing the OCaml program}{And raising an exception without handler}
\begin{columns}
\begin{column}{0.6\textwidth}
\only<1-3>{
\begin{itemize}
\item<1-> Execution continues with OCaml code
\begin{itemize}
\item \funcname{camlDefault\_\_main\_269}
\item \funcname{camlDefault\_\_entry}
\item \funcname{caml\_program}
\end{itemize}
\item<1-> \funcname{camlDefault\_\_main\_269} raises a \typename{Failure} exception
\item<1-> \typename{Failure} is caught by \funcname{caml\_start\_program}
\begin{itemize}
\item<1-> The handler set the 2nd LSB of the exception value to \funcarg{1}
\item<3-> And then forces \funcname{caml\_start\_program} to terminate
\end{itemize}
\end{itemize}
}
\bigskip
\only<1,3>{\listocaml[highlightlines=3]{../src/default/default.ml}{default/default.ml}}%
\only<2>{\listasm[firstline=66,lastline=72,highlightlines=69]{src/ocaml/caml\_start\_program.s}{caml\_start\_program}}%
\end{column}
\begin{column}{0.4\textwidth}
\centering
\providecommand\step{4}\input{diagrams/default/default.tex}
\end{column}
\end{columns}
\end{frame}
%\only<>{\listasm[firstline=47,lastline=65]{src/ocaml/caml\_start\_program.s}{caml\_start\_program}}


\begin{frame}{Back to C}{Clean up, complain and terminate}
\begin{columns}
\begin{column}{0.45\textwidth}
\begin{itemize}
\item Backtrace:
\begin{itemize}
\item \funcname{caml\_start\_program} $\rightarrow$ OCaml code
\item \funcname{caml\_startup\_common}
\item \funcname{caml\_startup\_exn}
\item \funcname{caml\_startup}
\item \funcname{caml\_main}
\item \funcname{main}
\end{itemize}
\smallskip
\item \funcname{caml\_startup} checks the return value of \funcname{caml\_startup\_exn}
\begin{itemize}
\item The return value has been specially marked by \funcname{caml\_start\_program} handler
\item Calls \funcname{caml\_fatal\_uncaught\_exception}
\end{itemize}
\item<2-> \funcname{caml\_fatal\_uncaught\_exception} either
\begin{itemize}
\item<2-> Calls the user custom uncaught exception handler, if set with \mintinline{ocaml}{Printexc.set_uncaught_exception_handler fn}\footnotemark
\item<2-> Or falls back to the default one
\item<2-> Which notifies about the uncaught exception
\end{itemize}
\item<4-> Terminates the program either with \funcname{abort} or with a non-zero exit code
\end{itemize}
\end{column}
\begin{column}{0.55\textwidth}
\only<1>{
\listc[firstline=84,lastline=89,highlightlines=88]{../ocaml/runtime/caml/mlvalues.h}{runtime/caml/mlvalues.h:84}
\listc[firstline=138,lastline=143,highlightlines={141-142}]{../ocaml/runtime/startup\_nat.c}{runtime/startup\_nat.c:138}
}%
\only<2>{
\listc[firstline=138,highlightlines={147,149}]{../ocaml/runtime/printexc.c}{runtime/printexc.c:138}
}%
\only<3>{
\listc[firstline=110,lastline=134,highlightlines=129]{../ocaml/runtime/printexc.c}{runtime/printexc.c:138}
}%
\only<4>{
\listc[firstline=138,highlightlines={152,154}]{../ocaml/runtime/printexc.c}{runtime/printexc.c:138}
}%
\end{column}
\end{columns}
\footnotetext[1]{\url{https://v2.ocaml.org/api/Printexc.html\#1\_Uncaughtexceptions}}
\end{frame}
}%
\onslide*<6->{\providecommand\step{4}%
% Runtime's default exception handler
%
\subsection*{Runtime's default exception handler}
\frameSubsection{pictures/The\_Architect.png}{}


\begin{frame}{Default exception handler}
\begin{columns}
\begin{column}{0.5\textwidth}
\begin{itemize}
\item \localname{domain\_state->exn\_handler} points to a trap located before \funcname{entry}!
\item What installed this very first trap there?
\end{itemize}
\bigskip
\listocaml{../src/default/default.ml}{default/default.ml}
\bigskip
\inputminted[fontsize=\footnotesize]{text}{../src/default/default.out}
\end{column}
\begin{column}{0.5\textwidth}
\centering
\only<1>{\providecommand\step{0}\input{diagrams/default/default.tex}}%
\only<2>{\providecommand\step{4}\input{diagrams/default/default.tex}}%
\end{column}
\end{columns}
\end{frame}


\begin{frame}{Startup}{How did we get there by the way?}
\begin{columns}
\begin{column}{0.45\textwidth}
\begin{itemize}
\item The entry point address of the binary is \funcname{main} (\funcname{\_start}) from the runtime
\item The program starts like a C program:
\begin{itemize}
\item \funcname{caml\_start\_program}
\item \funcname{caml\_startup\_common}
\item \funcname{caml\_startup\_exn}
\item \funcname{caml\_startup}
\item \funcname{caml\_main}
\item \funcname{main}
\end{itemize}
%\item \funcname{caml\_startup\_common}: initializes GC, domains \dots
% Also: codefrag, locale, custom opeartions, frame_descr, signals, stack
\item \funcname{caml\_start\_program} is the transition point between the C realm and the OCaml one
\end{itemize}
\bigskip
\listocaml{../src/default/default.ml}{default/default.ml}
\end{column}
\begin{column}{0.55\textwidth}
\listasm[lastline=24,highlightlines={22-24}]{src/ocaml/caml\_start\_program.s}{runtime/amd64.S:604}
\end{column}
\end{columns}
\end{frame}


\begin{frame}{Setup to jump into OCaml entry point}{\funcname{caml\_start\_program} initializes the main fiber}
\begin{columns}
\begin{column}{0.6\textwidth}
\only<1,3,5>{
\begin{itemize}
\item<1-> Stores information to allow switching from OCaml stack to the C stack
\item<3-> Constructs the default exception handler
\item<5-> Switches to the OCaml stack and calls \funcname{caml\_program}
\end{itemize}
\bigskip
\listocaml{../src/default/default.ml}{default/default.ml}
}%
\only<2>{
\listasm[firstline=22,lastline=41,highlightlines={25-27}]{src/ocaml/caml\_start\_program.s}{caml\_start\_program}
}%
\only<4>{
\mintasm[firstline=22,lastline=41,highlightlines={31-38}]{src/ocaml/caml\_start\_program.s}
\listasm[firstline=66,lastline=72]{src/ocaml/caml\_start\_program.s}{caml\_start\_program}
}%
\only<6>{
\listasm[firstline=22,lastline=41,highlightlines={39-41}]{src/ocaml/caml\_start\_program.s}{caml\_start\_program}
}%
\end{column}
\begin{column}{0.4\textwidth}
\centering
\only<1-2>{\providecommand\step{2}\input{diagrams/default/default.tex}}%
\only<3-5>{\providecommand\step{3}\input{diagrams/default/default.tex}}%
\onslide*<6->{\providecommand\step{4}\input{diagrams/default/default.tex}}%
\end{column}
\end{columns}
\end{frame}

\begin{frame}{Executing the OCaml program}{And raising an exception without handler}
\begin{columns}
\begin{column}{0.6\textwidth}
\only<1-3>{
\begin{itemize}
\item<1-> Execution continues with OCaml code
\begin{itemize}
\item \funcname{camlDefault\_\_main\_269}
\item \funcname{camlDefault\_\_entry}
\item \funcname{caml\_program}
\end{itemize}
\item<1-> \funcname{camlDefault\_\_main\_269} raises a \typename{Failure} exception
\item<1-> \typename{Failure} is caught by \funcname{caml\_start\_program}
\begin{itemize}
\item<1-> The handler set the 2nd LSB of the exception value to \funcarg{1}
\item<3-> And then forces \funcname{caml\_start\_program} to terminate
\end{itemize}
\end{itemize}
}
\bigskip
\only<1,3>{\listocaml[highlightlines=3]{../src/default/default.ml}{default/default.ml}}%
\only<2>{\listasm[firstline=66,lastline=72,highlightlines=69]{src/ocaml/caml\_start\_program.s}{caml\_start\_program}}%
\end{column}
\begin{column}{0.4\textwidth}
\centering
\providecommand\step{4}\input{diagrams/default/default.tex}
\end{column}
\end{columns}
\end{frame}
%\only<>{\listasm[firstline=47,lastline=65]{src/ocaml/caml\_start\_program.s}{caml\_start\_program}}


\begin{frame}{Back to C}{Clean up, complain and terminate}
\begin{columns}
\begin{column}{0.45\textwidth}
\begin{itemize}
\item Backtrace:
\begin{itemize}
\item \funcname{caml\_start\_program} $\rightarrow$ OCaml code
\item \funcname{caml\_startup\_common}
\item \funcname{caml\_startup\_exn}
\item \funcname{caml\_startup}
\item \funcname{caml\_main}
\item \funcname{main}
\end{itemize}
\smallskip
\item \funcname{caml\_startup} checks the return value of \funcname{caml\_startup\_exn}
\begin{itemize}
\item The return value has been specially marked by \funcname{caml\_start\_program} handler
\item Calls \funcname{caml\_fatal\_uncaught\_exception}
\end{itemize}
\item<2-> \funcname{caml\_fatal\_uncaught\_exception} either
\begin{itemize}
\item<2-> Calls the user custom uncaught exception handler, if set with \mintinline{ocaml}{Printexc.set_uncaught_exception_handler fn}\footnotemark
\item<2-> Or falls back to the default one
\item<2-> Which notifies about the uncaught exception
\end{itemize}
\item<4-> Terminates the program either with \funcname{abort} or with a non-zero exit code
\end{itemize}
\end{column}
\begin{column}{0.55\textwidth}
\only<1>{
\listc[firstline=84,lastline=89,highlightlines=88]{../ocaml/runtime/caml/mlvalues.h}{runtime/caml/mlvalues.h:84}
\listc[firstline=138,lastline=143,highlightlines={141-142}]{../ocaml/runtime/startup\_nat.c}{runtime/startup\_nat.c:138}
}%
\only<2>{
\listc[firstline=138,highlightlines={147,149}]{../ocaml/runtime/printexc.c}{runtime/printexc.c:138}
}%
\only<3>{
\listc[firstline=110,lastline=134,highlightlines=129]{../ocaml/runtime/printexc.c}{runtime/printexc.c:138}
}%
\only<4>{
\listc[firstline=138,highlightlines={152,154}]{../ocaml/runtime/printexc.c}{runtime/printexc.c:138}
}%
\end{column}
\end{columns}
\footnotetext[1]{\url{https://v2.ocaml.org/api/Printexc.html\#1\_Uncaughtexceptions}}
\end{frame}
}%
\end{column}
\end{columns}
\end{frame}

\begin{frame}{Executing the OCaml program}{And raising an exception without handler}
\begin{columns}
\begin{column}{0.6\textwidth}
\only<1-3>{
\begin{itemize}
\item<1-> Execution continues with OCaml code
\begin{itemize}
\item \funcname{camlDefault\_\_main\_269}
\item \funcname{camlDefault\_\_entry}
\item \funcname{caml\_program}
\end{itemize}
\item<1-> \funcname{camlDefault\_\_main\_269} raises a \typename{Failure} exception
\item<1-> \typename{Failure} is caught by \funcname{caml\_start\_program}
\begin{itemize}
\item<1-> The handler set the 2nd LSB of the exception value to \funcarg{1}
\item<3-> And then forces \funcname{caml\_start\_program} to terminate
\end{itemize}
\end{itemize}
}
\bigskip
\only<1,3>{\listocaml[highlightlines=3]{../src/default/default.ml}{default/default.ml}}%
\only<2>{\listasm[firstline=66,lastline=72,highlightlines=69]{src/ocaml/caml\_start\_program.s}{caml\_start\_program}}%
\end{column}
\begin{column}{0.4\textwidth}
\centering
\providecommand\step{4}%
% Runtime's default exception handler
%
\subsection*{Runtime's default exception handler}
\frameSubsection{pictures/The\_Architect.png}{}


\begin{frame}{Default exception handler}
\begin{columns}
\begin{column}{0.5\textwidth}
\begin{itemize}
\item \localname{domain\_state->exn\_handler} points to a trap located before \funcname{entry}!
\item What installed this very first trap there?
\end{itemize}
\bigskip
\listocaml{../src/default/default.ml}{default/default.ml}
\bigskip
\inputminted[fontsize=\footnotesize]{text}{../src/default/default.out}
\end{column}
\begin{column}{0.5\textwidth}
\centering
\only<1>{\providecommand\step{0}\input{diagrams/default/default.tex}}%
\only<2>{\providecommand\step{4}\input{diagrams/default/default.tex}}%
\end{column}
\end{columns}
\end{frame}


\begin{frame}{Startup}{How did we get there by the way?}
\begin{columns}
\begin{column}{0.45\textwidth}
\begin{itemize}
\item The entry point address of the binary is \funcname{main} (\funcname{\_start}) from the runtime
\item The program starts like a C program:
\begin{itemize}
\item \funcname{caml\_start\_program}
\item \funcname{caml\_startup\_common}
\item \funcname{caml\_startup\_exn}
\item \funcname{caml\_startup}
\item \funcname{caml\_main}
\item \funcname{main}
\end{itemize}
%\item \funcname{caml\_startup\_common}: initializes GC, domains \dots
% Also: codefrag, locale, custom opeartions, frame_descr, signals, stack
\item \funcname{caml\_start\_program} is the transition point between the C realm and the OCaml one
\end{itemize}
\bigskip
\listocaml{../src/default/default.ml}{default/default.ml}
\end{column}
\begin{column}{0.55\textwidth}
\listasm[lastline=24,highlightlines={22-24}]{src/ocaml/caml\_start\_program.s}{runtime/amd64.S:604}
\end{column}
\end{columns}
\end{frame}


\begin{frame}{Setup to jump into OCaml entry point}{\funcname{caml\_start\_program} initializes the main fiber}
\begin{columns}
\begin{column}{0.6\textwidth}
\only<1,3,5>{
\begin{itemize}
\item<1-> Stores information to allow switching from OCaml stack to the C stack
\item<3-> Constructs the default exception handler
\item<5-> Switches to the OCaml stack and calls \funcname{caml\_program}
\end{itemize}
\bigskip
\listocaml{../src/default/default.ml}{default/default.ml}
}%
\only<2>{
\listasm[firstline=22,lastline=41,highlightlines={25-27}]{src/ocaml/caml\_start\_program.s}{caml\_start\_program}
}%
\only<4>{
\mintasm[firstline=22,lastline=41,highlightlines={31-38}]{src/ocaml/caml\_start\_program.s}
\listasm[firstline=66,lastline=72]{src/ocaml/caml\_start\_program.s}{caml\_start\_program}
}%
\only<6>{
\listasm[firstline=22,lastline=41,highlightlines={39-41}]{src/ocaml/caml\_start\_program.s}{caml\_start\_program}
}%
\end{column}
\begin{column}{0.4\textwidth}
\centering
\only<1-2>{\providecommand\step{2}\input{diagrams/default/default.tex}}%
\only<3-5>{\providecommand\step{3}\input{diagrams/default/default.tex}}%
\onslide*<6->{\providecommand\step{4}\input{diagrams/default/default.tex}}%
\end{column}
\end{columns}
\end{frame}

\begin{frame}{Executing the OCaml program}{And raising an exception without handler}
\begin{columns}
\begin{column}{0.6\textwidth}
\only<1-3>{
\begin{itemize}
\item<1-> Execution continues with OCaml code
\begin{itemize}
\item \funcname{camlDefault\_\_main\_269}
\item \funcname{camlDefault\_\_entry}
\item \funcname{caml\_program}
\end{itemize}
\item<1-> \funcname{camlDefault\_\_main\_269} raises a \typename{Failure} exception
\item<1-> \typename{Failure} is caught by \funcname{caml\_start\_program}
\begin{itemize}
\item<1-> The handler set the 2nd LSB of the exception value to \funcarg{1}
\item<3-> And then forces \funcname{caml\_start\_program} to terminate
\end{itemize}
\end{itemize}
}
\bigskip
\only<1,3>{\listocaml[highlightlines=3]{../src/default/default.ml}{default/default.ml}}%
\only<2>{\listasm[firstline=66,lastline=72,highlightlines=69]{src/ocaml/caml\_start\_program.s}{caml\_start\_program}}%
\end{column}
\begin{column}{0.4\textwidth}
\centering
\providecommand\step{4}\input{diagrams/default/default.tex}
\end{column}
\end{columns}
\end{frame}
%\only<>{\listasm[firstline=47,lastline=65]{src/ocaml/caml\_start\_program.s}{caml\_start\_program}}


\begin{frame}{Back to C}{Clean up, complain and terminate}
\begin{columns}
\begin{column}{0.45\textwidth}
\begin{itemize}
\item Backtrace:
\begin{itemize}
\item \funcname{caml\_start\_program} $\rightarrow$ OCaml code
\item \funcname{caml\_startup\_common}
\item \funcname{caml\_startup\_exn}
\item \funcname{caml\_startup}
\item \funcname{caml\_main}
\item \funcname{main}
\end{itemize}
\smallskip
\item \funcname{caml\_startup} checks the return value of \funcname{caml\_startup\_exn}
\begin{itemize}
\item The return value has been specially marked by \funcname{caml\_start\_program} handler
\item Calls \funcname{caml\_fatal\_uncaught\_exception}
\end{itemize}
\item<2-> \funcname{caml\_fatal\_uncaught\_exception} either
\begin{itemize}
\item<2-> Calls the user custom uncaught exception handler, if set with \mintinline{ocaml}{Printexc.set_uncaught_exception_handler fn}\footnotemark
\item<2-> Or falls back to the default one
\item<2-> Which notifies about the uncaught exception
\end{itemize}
\item<4-> Terminates the program either with \funcname{abort} or with a non-zero exit code
\end{itemize}
\end{column}
\begin{column}{0.55\textwidth}
\only<1>{
\listc[firstline=84,lastline=89,highlightlines=88]{../ocaml/runtime/caml/mlvalues.h}{runtime/caml/mlvalues.h:84}
\listc[firstline=138,lastline=143,highlightlines={141-142}]{../ocaml/runtime/startup\_nat.c}{runtime/startup\_nat.c:138}
}%
\only<2>{
\listc[firstline=138,highlightlines={147,149}]{../ocaml/runtime/printexc.c}{runtime/printexc.c:138}
}%
\only<3>{
\listc[firstline=110,lastline=134,highlightlines=129]{../ocaml/runtime/printexc.c}{runtime/printexc.c:138}
}%
\only<4>{
\listc[firstline=138,highlightlines={152,154}]{../ocaml/runtime/printexc.c}{runtime/printexc.c:138}
}%
\end{column}
\end{columns}
\footnotetext[1]{\url{https://v2.ocaml.org/api/Printexc.html\#1\_Uncaughtexceptions}}
\end{frame}

\end{column}
\end{columns}
\end{frame}
%\only<>{\listasm[firstline=47,lastline=65]{src/ocaml/caml\_start\_program.s}{caml\_start\_program}}


\begin{frame}{Back to C}{Clean up, complain and terminate}
\begin{columns}
\begin{column}{0.45\textwidth}
\begin{itemize}
\item Backtrace:
\begin{itemize}
\item \funcname{caml\_start\_program} $\rightarrow$ OCaml code
\item \funcname{caml\_startup\_common}
\item \funcname{caml\_startup\_exn}
\item \funcname{caml\_startup}
\item \funcname{caml\_main}
\item \funcname{main}
\end{itemize}
\smallskip
\item \funcname{caml\_startup} checks the return value of \funcname{caml\_startup\_exn}
\begin{itemize}
\item The return value has been specially marked by \funcname{caml\_start\_program} handler
\item Calls \funcname{caml\_fatal\_uncaught\_exception}
\end{itemize}
\item<2-> \funcname{caml\_fatal\_uncaught\_exception} either
\begin{itemize}
\item<2-> Calls the user custom uncaught exception handler, if set with \mintinline{ocaml}{Printexc.set_uncaught_exception_handler fn}\footnotemark
\item<2-> Or falls back to the default one
\item<2-> Which notifies about the uncaught exception
\end{itemize}
\item<4-> Terminates the program either with \funcname{abort} or with a non-zero exit code
\end{itemize}
\end{column}
\begin{column}{0.55\textwidth}
\only<1>{
\listc[firstline=84,lastline=89,highlightlines=88]{../ocaml/runtime/caml/mlvalues.h}{runtime/caml/mlvalues.h:84}
\listc[firstline=138,lastline=143,highlightlines={141-142}]{../ocaml/runtime/startup\_nat.c}{runtime/startup\_nat.c:138}
}%
\only<2>{
\listc[firstline=138,highlightlines={147,149}]{../ocaml/runtime/printexc.c}{runtime/printexc.c:138}
}%
\only<3>{
\listc[firstline=110,lastline=134,highlightlines=129]{../ocaml/runtime/printexc.c}{runtime/printexc.c:138}
}%
\only<4>{
\listc[firstline=138,highlightlines={152,154}]{../ocaml/runtime/printexc.c}{runtime/printexc.c:138}
}%
\end{column}
\end{columns}
\footnotetext[1]{\url{https://v2.ocaml.org/api/Printexc.html\#1\_Uncaughtexceptions}}
\end{frame}
}%
\only<2>{\providecommand\step{4}%
% Runtime's default exception handler
%
\subsection*{Runtime's default exception handler}
\frameSubsection{pictures/The\_Architect.png}{}


\begin{frame}{Default exception handler}
\begin{columns}
\begin{column}{0.5\textwidth}
\begin{itemize}
\item \localname{domain\_state->exn\_handler} points to a trap located before \funcname{entry}!
\item What installed this very first trap there?
\end{itemize}
\bigskip
\listocaml{../src/default/default.ml}{default/default.ml}
\bigskip
\inputminted[fontsize=\footnotesize]{text}{../src/default/default.out}
\end{column}
\begin{column}{0.5\textwidth}
\centering
\only<1>{\providecommand\step{0}%
% Runtime's default exception handler
%
\subsection*{Runtime's default exception handler}
\frameSubsection{pictures/The\_Architect.png}{}


\begin{frame}{Default exception handler}
\begin{columns}
\begin{column}{0.5\textwidth}
\begin{itemize}
\item \localname{domain\_state->exn\_handler} points to a trap located before \funcname{entry}!
\item What installed this very first trap there?
\end{itemize}
\bigskip
\listocaml{../src/default/default.ml}{default/default.ml}
\bigskip
\inputminted[fontsize=\footnotesize]{text}{../src/default/default.out}
\end{column}
\begin{column}{0.5\textwidth}
\centering
\only<1>{\providecommand\step{0}\input{diagrams/default/default.tex}}%
\only<2>{\providecommand\step{4}\input{diagrams/default/default.tex}}%
\end{column}
\end{columns}
\end{frame}


\begin{frame}{Startup}{How did we get there by the way?}
\begin{columns}
\begin{column}{0.45\textwidth}
\begin{itemize}
\item The entry point address of the binary is \funcname{main} (\funcname{\_start}) from the runtime
\item The program starts like a C program:
\begin{itemize}
\item \funcname{caml\_start\_program}
\item \funcname{caml\_startup\_common}
\item \funcname{caml\_startup\_exn}
\item \funcname{caml\_startup}
\item \funcname{caml\_main}
\item \funcname{main}
\end{itemize}
%\item \funcname{caml\_startup\_common}: initializes GC, domains \dots
% Also: codefrag, locale, custom opeartions, frame_descr, signals, stack
\item \funcname{caml\_start\_program} is the transition point between the C realm and the OCaml one
\end{itemize}
\bigskip
\listocaml{../src/default/default.ml}{default/default.ml}
\end{column}
\begin{column}{0.55\textwidth}
\listasm[lastline=24,highlightlines={22-24}]{src/ocaml/caml\_start\_program.s}{runtime/amd64.S:604}
\end{column}
\end{columns}
\end{frame}


\begin{frame}{Setup to jump into OCaml entry point}{\funcname{caml\_start\_program} initializes the main fiber}
\begin{columns}
\begin{column}{0.6\textwidth}
\only<1,3,5>{
\begin{itemize}
\item<1-> Stores information to allow switching from OCaml stack to the C stack
\item<3-> Constructs the default exception handler
\item<5-> Switches to the OCaml stack and calls \funcname{caml\_program}
\end{itemize}
\bigskip
\listocaml{../src/default/default.ml}{default/default.ml}
}%
\only<2>{
\listasm[firstline=22,lastline=41,highlightlines={25-27}]{src/ocaml/caml\_start\_program.s}{caml\_start\_program}
}%
\only<4>{
\mintasm[firstline=22,lastline=41,highlightlines={31-38}]{src/ocaml/caml\_start\_program.s}
\listasm[firstline=66,lastline=72]{src/ocaml/caml\_start\_program.s}{caml\_start\_program}
}%
\only<6>{
\listasm[firstline=22,lastline=41,highlightlines={39-41}]{src/ocaml/caml\_start\_program.s}{caml\_start\_program}
}%
\end{column}
\begin{column}{0.4\textwidth}
\centering
\only<1-2>{\providecommand\step{2}\input{diagrams/default/default.tex}}%
\only<3-5>{\providecommand\step{3}\input{diagrams/default/default.tex}}%
\onslide*<6->{\providecommand\step{4}\input{diagrams/default/default.tex}}%
\end{column}
\end{columns}
\end{frame}

\begin{frame}{Executing the OCaml program}{And raising an exception without handler}
\begin{columns}
\begin{column}{0.6\textwidth}
\only<1-3>{
\begin{itemize}
\item<1-> Execution continues with OCaml code
\begin{itemize}
\item \funcname{camlDefault\_\_main\_269}
\item \funcname{camlDefault\_\_entry}
\item \funcname{caml\_program}
\end{itemize}
\item<1-> \funcname{camlDefault\_\_main\_269} raises a \typename{Failure} exception
\item<1-> \typename{Failure} is caught by \funcname{caml\_start\_program}
\begin{itemize}
\item<1-> The handler set the 2nd LSB of the exception value to \funcarg{1}
\item<3-> And then forces \funcname{caml\_start\_program} to terminate
\end{itemize}
\end{itemize}
}
\bigskip
\only<1,3>{\listocaml[highlightlines=3]{../src/default/default.ml}{default/default.ml}}%
\only<2>{\listasm[firstline=66,lastline=72,highlightlines=69]{src/ocaml/caml\_start\_program.s}{caml\_start\_program}}%
\end{column}
\begin{column}{0.4\textwidth}
\centering
\providecommand\step{4}\input{diagrams/default/default.tex}
\end{column}
\end{columns}
\end{frame}
%\only<>{\listasm[firstline=47,lastline=65]{src/ocaml/caml\_start\_program.s}{caml\_start\_program}}


\begin{frame}{Back to C}{Clean up, complain and terminate}
\begin{columns}
\begin{column}{0.45\textwidth}
\begin{itemize}
\item Backtrace:
\begin{itemize}
\item \funcname{caml\_start\_program} $\rightarrow$ OCaml code
\item \funcname{caml\_startup\_common}
\item \funcname{caml\_startup\_exn}
\item \funcname{caml\_startup}
\item \funcname{caml\_main}
\item \funcname{main}
\end{itemize}
\smallskip
\item \funcname{caml\_startup} checks the return value of \funcname{caml\_startup\_exn}
\begin{itemize}
\item The return value has been specially marked by \funcname{caml\_start\_program} handler
\item Calls \funcname{caml\_fatal\_uncaught\_exception}
\end{itemize}
\item<2-> \funcname{caml\_fatal\_uncaught\_exception} either
\begin{itemize}
\item<2-> Calls the user custom uncaught exception handler, if set with \mintinline{ocaml}{Printexc.set_uncaught_exception_handler fn}\footnotemark
\item<2-> Or falls back to the default one
\item<2-> Which notifies about the uncaught exception
\end{itemize}
\item<4-> Terminates the program either with \funcname{abort} or with a non-zero exit code
\end{itemize}
\end{column}
\begin{column}{0.55\textwidth}
\only<1>{
\listc[firstline=84,lastline=89,highlightlines=88]{../ocaml/runtime/caml/mlvalues.h}{runtime/caml/mlvalues.h:84}
\listc[firstline=138,lastline=143,highlightlines={141-142}]{../ocaml/runtime/startup\_nat.c}{runtime/startup\_nat.c:138}
}%
\only<2>{
\listc[firstline=138,highlightlines={147,149}]{../ocaml/runtime/printexc.c}{runtime/printexc.c:138}
}%
\only<3>{
\listc[firstline=110,lastline=134,highlightlines=129]{../ocaml/runtime/printexc.c}{runtime/printexc.c:138}
}%
\only<4>{
\listc[firstline=138,highlightlines={152,154}]{../ocaml/runtime/printexc.c}{runtime/printexc.c:138}
}%
\end{column}
\end{columns}
\footnotetext[1]{\url{https://v2.ocaml.org/api/Printexc.html\#1\_Uncaughtexceptions}}
\end{frame}
}%
\only<2>{\providecommand\step{4}%
% Runtime's default exception handler
%
\subsection*{Runtime's default exception handler}
\frameSubsection{pictures/The\_Architect.png}{}


\begin{frame}{Default exception handler}
\begin{columns}
\begin{column}{0.5\textwidth}
\begin{itemize}
\item \localname{domain\_state->exn\_handler} points to a trap located before \funcname{entry}!
\item What installed this very first trap there?
\end{itemize}
\bigskip
\listocaml{../src/default/default.ml}{default/default.ml}
\bigskip
\inputminted[fontsize=\footnotesize]{text}{../src/default/default.out}
\end{column}
\begin{column}{0.5\textwidth}
\centering
\only<1>{\providecommand\step{0}\input{diagrams/default/default.tex}}%
\only<2>{\providecommand\step{4}\input{diagrams/default/default.tex}}%
\end{column}
\end{columns}
\end{frame}


\begin{frame}{Startup}{How did we get there by the way?}
\begin{columns}
\begin{column}{0.45\textwidth}
\begin{itemize}
\item The entry point address of the binary is \funcname{main} (\funcname{\_start}) from the runtime
\item The program starts like a C program:
\begin{itemize}
\item \funcname{caml\_start\_program}
\item \funcname{caml\_startup\_common}
\item \funcname{caml\_startup\_exn}
\item \funcname{caml\_startup}
\item \funcname{caml\_main}
\item \funcname{main}
\end{itemize}
%\item \funcname{caml\_startup\_common}: initializes GC, domains \dots
% Also: codefrag, locale, custom opeartions, frame_descr, signals, stack
\item \funcname{caml\_start\_program} is the transition point between the C realm and the OCaml one
\end{itemize}
\bigskip
\listocaml{../src/default/default.ml}{default/default.ml}
\end{column}
\begin{column}{0.55\textwidth}
\listasm[lastline=24,highlightlines={22-24}]{src/ocaml/caml\_start\_program.s}{runtime/amd64.S:604}
\end{column}
\end{columns}
\end{frame}


\begin{frame}{Setup to jump into OCaml entry point}{\funcname{caml\_start\_program} initializes the main fiber}
\begin{columns}
\begin{column}{0.6\textwidth}
\only<1,3,5>{
\begin{itemize}
\item<1-> Stores information to allow switching from OCaml stack to the C stack
\item<3-> Constructs the default exception handler
\item<5-> Switches to the OCaml stack and calls \funcname{caml\_program}
\end{itemize}
\bigskip
\listocaml{../src/default/default.ml}{default/default.ml}
}%
\only<2>{
\listasm[firstline=22,lastline=41,highlightlines={25-27}]{src/ocaml/caml\_start\_program.s}{caml\_start\_program}
}%
\only<4>{
\mintasm[firstline=22,lastline=41,highlightlines={31-38}]{src/ocaml/caml\_start\_program.s}
\listasm[firstline=66,lastline=72]{src/ocaml/caml\_start\_program.s}{caml\_start\_program}
}%
\only<6>{
\listasm[firstline=22,lastline=41,highlightlines={39-41}]{src/ocaml/caml\_start\_program.s}{caml\_start\_program}
}%
\end{column}
\begin{column}{0.4\textwidth}
\centering
\only<1-2>{\providecommand\step{2}\input{diagrams/default/default.tex}}%
\only<3-5>{\providecommand\step{3}\input{diagrams/default/default.tex}}%
\onslide*<6->{\providecommand\step{4}\input{diagrams/default/default.tex}}%
\end{column}
\end{columns}
\end{frame}

\begin{frame}{Executing the OCaml program}{And raising an exception without handler}
\begin{columns}
\begin{column}{0.6\textwidth}
\only<1-3>{
\begin{itemize}
\item<1-> Execution continues with OCaml code
\begin{itemize}
\item \funcname{camlDefault\_\_main\_269}
\item \funcname{camlDefault\_\_entry}
\item \funcname{caml\_program}
\end{itemize}
\item<1-> \funcname{camlDefault\_\_main\_269} raises a \typename{Failure} exception
\item<1-> \typename{Failure} is caught by \funcname{caml\_start\_program}
\begin{itemize}
\item<1-> The handler set the 2nd LSB of the exception value to \funcarg{1}
\item<3-> And then forces \funcname{caml\_start\_program} to terminate
\end{itemize}
\end{itemize}
}
\bigskip
\only<1,3>{\listocaml[highlightlines=3]{../src/default/default.ml}{default/default.ml}}%
\only<2>{\listasm[firstline=66,lastline=72,highlightlines=69]{src/ocaml/caml\_start\_program.s}{caml\_start\_program}}%
\end{column}
\begin{column}{0.4\textwidth}
\centering
\providecommand\step{4}\input{diagrams/default/default.tex}
\end{column}
\end{columns}
\end{frame}
%\only<>{\listasm[firstline=47,lastline=65]{src/ocaml/caml\_start\_program.s}{caml\_start\_program}}


\begin{frame}{Back to C}{Clean up, complain and terminate}
\begin{columns}
\begin{column}{0.45\textwidth}
\begin{itemize}
\item Backtrace:
\begin{itemize}
\item \funcname{caml\_start\_program} $\rightarrow$ OCaml code
\item \funcname{caml\_startup\_common}
\item \funcname{caml\_startup\_exn}
\item \funcname{caml\_startup}
\item \funcname{caml\_main}
\item \funcname{main}
\end{itemize}
\smallskip
\item \funcname{caml\_startup} checks the return value of \funcname{caml\_startup\_exn}
\begin{itemize}
\item The return value has been specially marked by \funcname{caml\_start\_program} handler
\item Calls \funcname{caml\_fatal\_uncaught\_exception}
\end{itemize}
\item<2-> \funcname{caml\_fatal\_uncaught\_exception} either
\begin{itemize}
\item<2-> Calls the user custom uncaught exception handler, if set with \mintinline{ocaml}{Printexc.set_uncaught_exception_handler fn}\footnotemark
\item<2-> Or falls back to the default one
\item<2-> Which notifies about the uncaught exception
\end{itemize}
\item<4-> Terminates the program either with \funcname{abort} or with a non-zero exit code
\end{itemize}
\end{column}
\begin{column}{0.55\textwidth}
\only<1>{
\listc[firstline=84,lastline=89,highlightlines=88]{../ocaml/runtime/caml/mlvalues.h}{runtime/caml/mlvalues.h:84}
\listc[firstline=138,lastline=143,highlightlines={141-142}]{../ocaml/runtime/startup\_nat.c}{runtime/startup\_nat.c:138}
}%
\only<2>{
\listc[firstline=138,highlightlines={147,149}]{../ocaml/runtime/printexc.c}{runtime/printexc.c:138}
}%
\only<3>{
\listc[firstline=110,lastline=134,highlightlines=129]{../ocaml/runtime/printexc.c}{runtime/printexc.c:138}
}%
\only<4>{
\listc[firstline=138,highlightlines={152,154}]{../ocaml/runtime/printexc.c}{runtime/printexc.c:138}
}%
\end{column}
\end{columns}
\footnotetext[1]{\url{https://v2.ocaml.org/api/Printexc.html\#1\_Uncaughtexceptions}}
\end{frame}
}%
\end{column}
\end{columns}
\end{frame}


\begin{frame}{Startup}{How did we get there by the way?}
\begin{columns}
\begin{column}{0.45\textwidth}
\begin{itemize}
\item The entry point address of the binary is \funcname{main} (\funcname{\_start}) from the runtime
\item The program starts like a C program:
\begin{itemize}
\item \funcname{caml\_start\_program}
\item \funcname{caml\_startup\_common}
\item \funcname{caml\_startup\_exn}
\item \funcname{caml\_startup}
\item \funcname{caml\_main}
\item \funcname{main}
\end{itemize}
%\item \funcname{caml\_startup\_common}: initializes GC, domains \dots
% Also: codefrag, locale, custom opeartions, frame_descr, signals, stack
\item \funcname{caml\_start\_program} is the transition point between the C realm and the OCaml one
\end{itemize}
\bigskip
\listocaml{../src/default/default.ml}{default/default.ml}
\end{column}
\begin{column}{0.55\textwidth}
\listasm[lastline=24,highlightlines={22-24}]{src/ocaml/caml\_start\_program.s}{runtime/amd64.S:604}
\end{column}
\end{columns}
\end{frame}


\begin{frame}{Setup to jump into OCaml entry point}{\funcname{caml\_start\_program} initializes the main fiber}
\begin{columns}
\begin{column}{0.6\textwidth}
\only<1,3,5>{
\begin{itemize}
\item<1-> Stores information to allow switching from OCaml stack to the C stack
\item<3-> Constructs the default exception handler
\item<5-> Switches to the OCaml stack and calls \funcname{caml\_program}
\end{itemize}
\bigskip
\listocaml{../src/default/default.ml}{default/default.ml}
}%
\only<2>{
\listasm[firstline=22,lastline=41,highlightlines={25-27}]{src/ocaml/caml\_start\_program.s}{caml\_start\_program}
}%
\only<4>{
\mintasm[firstline=22,lastline=41,highlightlines={31-38}]{src/ocaml/caml\_start\_program.s}
\listasm[firstline=66,lastline=72]{src/ocaml/caml\_start\_program.s}{caml\_start\_program}
}%
\only<6>{
\listasm[firstline=22,lastline=41,highlightlines={39-41}]{src/ocaml/caml\_start\_program.s}{caml\_start\_program}
}%
\end{column}
\begin{column}{0.4\textwidth}
\centering
\only<1-2>{\providecommand\step{2}%
% Runtime's default exception handler
%
\subsection*{Runtime's default exception handler}
\frameSubsection{pictures/The\_Architect.png}{}


\begin{frame}{Default exception handler}
\begin{columns}
\begin{column}{0.5\textwidth}
\begin{itemize}
\item \localname{domain\_state->exn\_handler} points to a trap located before \funcname{entry}!
\item What installed this very first trap there?
\end{itemize}
\bigskip
\listocaml{../src/default/default.ml}{default/default.ml}
\bigskip
\inputminted[fontsize=\footnotesize]{text}{../src/default/default.out}
\end{column}
\begin{column}{0.5\textwidth}
\centering
\only<1>{\providecommand\step{0}\input{diagrams/default/default.tex}}%
\only<2>{\providecommand\step{4}\input{diagrams/default/default.tex}}%
\end{column}
\end{columns}
\end{frame}


\begin{frame}{Startup}{How did we get there by the way?}
\begin{columns}
\begin{column}{0.45\textwidth}
\begin{itemize}
\item The entry point address of the binary is \funcname{main} (\funcname{\_start}) from the runtime
\item The program starts like a C program:
\begin{itemize}
\item \funcname{caml\_start\_program}
\item \funcname{caml\_startup\_common}
\item \funcname{caml\_startup\_exn}
\item \funcname{caml\_startup}
\item \funcname{caml\_main}
\item \funcname{main}
\end{itemize}
%\item \funcname{caml\_startup\_common}: initializes GC, domains \dots
% Also: codefrag, locale, custom opeartions, frame_descr, signals, stack
\item \funcname{caml\_start\_program} is the transition point between the C realm and the OCaml one
\end{itemize}
\bigskip
\listocaml{../src/default/default.ml}{default/default.ml}
\end{column}
\begin{column}{0.55\textwidth}
\listasm[lastline=24,highlightlines={22-24}]{src/ocaml/caml\_start\_program.s}{runtime/amd64.S:604}
\end{column}
\end{columns}
\end{frame}


\begin{frame}{Setup to jump into OCaml entry point}{\funcname{caml\_start\_program} initializes the main fiber}
\begin{columns}
\begin{column}{0.6\textwidth}
\only<1,3,5>{
\begin{itemize}
\item<1-> Stores information to allow switching from OCaml stack to the C stack
\item<3-> Constructs the default exception handler
\item<5-> Switches to the OCaml stack and calls \funcname{caml\_program}
\end{itemize}
\bigskip
\listocaml{../src/default/default.ml}{default/default.ml}
}%
\only<2>{
\listasm[firstline=22,lastline=41,highlightlines={25-27}]{src/ocaml/caml\_start\_program.s}{caml\_start\_program}
}%
\only<4>{
\mintasm[firstline=22,lastline=41,highlightlines={31-38}]{src/ocaml/caml\_start\_program.s}
\listasm[firstline=66,lastline=72]{src/ocaml/caml\_start\_program.s}{caml\_start\_program}
}%
\only<6>{
\listasm[firstline=22,lastline=41,highlightlines={39-41}]{src/ocaml/caml\_start\_program.s}{caml\_start\_program}
}%
\end{column}
\begin{column}{0.4\textwidth}
\centering
\only<1-2>{\providecommand\step{2}\input{diagrams/default/default.tex}}%
\only<3-5>{\providecommand\step{3}\input{diagrams/default/default.tex}}%
\onslide*<6->{\providecommand\step{4}\input{diagrams/default/default.tex}}%
\end{column}
\end{columns}
\end{frame}

\begin{frame}{Executing the OCaml program}{And raising an exception without handler}
\begin{columns}
\begin{column}{0.6\textwidth}
\only<1-3>{
\begin{itemize}
\item<1-> Execution continues with OCaml code
\begin{itemize}
\item \funcname{camlDefault\_\_main\_269}
\item \funcname{camlDefault\_\_entry}
\item \funcname{caml\_program}
\end{itemize}
\item<1-> \funcname{camlDefault\_\_main\_269} raises a \typename{Failure} exception
\item<1-> \typename{Failure} is caught by \funcname{caml\_start\_program}
\begin{itemize}
\item<1-> The handler set the 2nd LSB of the exception value to \funcarg{1}
\item<3-> And then forces \funcname{caml\_start\_program} to terminate
\end{itemize}
\end{itemize}
}
\bigskip
\only<1,3>{\listocaml[highlightlines=3]{../src/default/default.ml}{default/default.ml}}%
\only<2>{\listasm[firstline=66,lastline=72,highlightlines=69]{src/ocaml/caml\_start\_program.s}{caml\_start\_program}}%
\end{column}
\begin{column}{0.4\textwidth}
\centering
\providecommand\step{4}\input{diagrams/default/default.tex}
\end{column}
\end{columns}
\end{frame}
%\only<>{\listasm[firstline=47,lastline=65]{src/ocaml/caml\_start\_program.s}{caml\_start\_program}}


\begin{frame}{Back to C}{Clean up, complain and terminate}
\begin{columns}
\begin{column}{0.45\textwidth}
\begin{itemize}
\item Backtrace:
\begin{itemize}
\item \funcname{caml\_start\_program} $\rightarrow$ OCaml code
\item \funcname{caml\_startup\_common}
\item \funcname{caml\_startup\_exn}
\item \funcname{caml\_startup}
\item \funcname{caml\_main}
\item \funcname{main}
\end{itemize}
\smallskip
\item \funcname{caml\_startup} checks the return value of \funcname{caml\_startup\_exn}
\begin{itemize}
\item The return value has been specially marked by \funcname{caml\_start\_program} handler
\item Calls \funcname{caml\_fatal\_uncaught\_exception}
\end{itemize}
\item<2-> \funcname{caml\_fatal\_uncaught\_exception} either
\begin{itemize}
\item<2-> Calls the user custom uncaught exception handler, if set with \mintinline{ocaml}{Printexc.set_uncaught_exception_handler fn}\footnotemark
\item<2-> Or falls back to the default one
\item<2-> Which notifies about the uncaught exception
\end{itemize}
\item<4-> Terminates the program either with \funcname{abort} or with a non-zero exit code
\end{itemize}
\end{column}
\begin{column}{0.55\textwidth}
\only<1>{
\listc[firstline=84,lastline=89,highlightlines=88]{../ocaml/runtime/caml/mlvalues.h}{runtime/caml/mlvalues.h:84}
\listc[firstline=138,lastline=143,highlightlines={141-142}]{../ocaml/runtime/startup\_nat.c}{runtime/startup\_nat.c:138}
}%
\only<2>{
\listc[firstline=138,highlightlines={147,149}]{../ocaml/runtime/printexc.c}{runtime/printexc.c:138}
}%
\only<3>{
\listc[firstline=110,lastline=134,highlightlines=129]{../ocaml/runtime/printexc.c}{runtime/printexc.c:138}
}%
\only<4>{
\listc[firstline=138,highlightlines={152,154}]{../ocaml/runtime/printexc.c}{runtime/printexc.c:138}
}%
\end{column}
\end{columns}
\footnotetext[1]{\url{https://v2.ocaml.org/api/Printexc.html\#1\_Uncaughtexceptions}}
\end{frame}
}%
\only<3-5>{\providecommand\step{3}%
% Runtime's default exception handler
%
\subsection*{Runtime's default exception handler}
\frameSubsection{pictures/The\_Architect.png}{}


\begin{frame}{Default exception handler}
\begin{columns}
\begin{column}{0.5\textwidth}
\begin{itemize}
\item \localname{domain\_state->exn\_handler} points to a trap located before \funcname{entry}!
\item What installed this very first trap there?
\end{itemize}
\bigskip
\listocaml{../src/default/default.ml}{default/default.ml}
\bigskip
\inputminted[fontsize=\footnotesize]{text}{../src/default/default.out}
\end{column}
\begin{column}{0.5\textwidth}
\centering
\only<1>{\providecommand\step{0}\input{diagrams/default/default.tex}}%
\only<2>{\providecommand\step{4}\input{diagrams/default/default.tex}}%
\end{column}
\end{columns}
\end{frame}


\begin{frame}{Startup}{How did we get there by the way?}
\begin{columns}
\begin{column}{0.45\textwidth}
\begin{itemize}
\item The entry point address of the binary is \funcname{main} (\funcname{\_start}) from the runtime
\item The program starts like a C program:
\begin{itemize}
\item \funcname{caml\_start\_program}
\item \funcname{caml\_startup\_common}
\item \funcname{caml\_startup\_exn}
\item \funcname{caml\_startup}
\item \funcname{caml\_main}
\item \funcname{main}
\end{itemize}
%\item \funcname{caml\_startup\_common}: initializes GC, domains \dots
% Also: codefrag, locale, custom opeartions, frame_descr, signals, stack
\item \funcname{caml\_start\_program} is the transition point between the C realm and the OCaml one
\end{itemize}
\bigskip
\listocaml{../src/default/default.ml}{default/default.ml}
\end{column}
\begin{column}{0.55\textwidth}
\listasm[lastline=24,highlightlines={22-24}]{src/ocaml/caml\_start\_program.s}{runtime/amd64.S:604}
\end{column}
\end{columns}
\end{frame}


\begin{frame}{Setup to jump into OCaml entry point}{\funcname{caml\_start\_program} initializes the main fiber}
\begin{columns}
\begin{column}{0.6\textwidth}
\only<1,3,5>{
\begin{itemize}
\item<1-> Stores information to allow switching from OCaml stack to the C stack
\item<3-> Constructs the default exception handler
\item<5-> Switches to the OCaml stack and calls \funcname{caml\_program}
\end{itemize}
\bigskip
\listocaml{../src/default/default.ml}{default/default.ml}
}%
\only<2>{
\listasm[firstline=22,lastline=41,highlightlines={25-27}]{src/ocaml/caml\_start\_program.s}{caml\_start\_program}
}%
\only<4>{
\mintasm[firstline=22,lastline=41,highlightlines={31-38}]{src/ocaml/caml\_start\_program.s}
\listasm[firstline=66,lastline=72]{src/ocaml/caml\_start\_program.s}{caml\_start\_program}
}%
\only<6>{
\listasm[firstline=22,lastline=41,highlightlines={39-41}]{src/ocaml/caml\_start\_program.s}{caml\_start\_program}
}%
\end{column}
\begin{column}{0.4\textwidth}
\centering
\only<1-2>{\providecommand\step{2}\input{diagrams/default/default.tex}}%
\only<3-5>{\providecommand\step{3}\input{diagrams/default/default.tex}}%
\onslide*<6->{\providecommand\step{4}\input{diagrams/default/default.tex}}%
\end{column}
\end{columns}
\end{frame}

\begin{frame}{Executing the OCaml program}{And raising an exception without handler}
\begin{columns}
\begin{column}{0.6\textwidth}
\only<1-3>{
\begin{itemize}
\item<1-> Execution continues with OCaml code
\begin{itemize}
\item \funcname{camlDefault\_\_main\_269}
\item \funcname{camlDefault\_\_entry}
\item \funcname{caml\_program}
\end{itemize}
\item<1-> \funcname{camlDefault\_\_main\_269} raises a \typename{Failure} exception
\item<1-> \typename{Failure} is caught by \funcname{caml\_start\_program}
\begin{itemize}
\item<1-> The handler set the 2nd LSB of the exception value to \funcarg{1}
\item<3-> And then forces \funcname{caml\_start\_program} to terminate
\end{itemize}
\end{itemize}
}
\bigskip
\only<1,3>{\listocaml[highlightlines=3]{../src/default/default.ml}{default/default.ml}}%
\only<2>{\listasm[firstline=66,lastline=72,highlightlines=69]{src/ocaml/caml\_start\_program.s}{caml\_start\_program}}%
\end{column}
\begin{column}{0.4\textwidth}
\centering
\providecommand\step{4}\input{diagrams/default/default.tex}
\end{column}
\end{columns}
\end{frame}
%\only<>{\listasm[firstline=47,lastline=65]{src/ocaml/caml\_start\_program.s}{caml\_start\_program}}


\begin{frame}{Back to C}{Clean up, complain and terminate}
\begin{columns}
\begin{column}{0.45\textwidth}
\begin{itemize}
\item Backtrace:
\begin{itemize}
\item \funcname{caml\_start\_program} $\rightarrow$ OCaml code
\item \funcname{caml\_startup\_common}
\item \funcname{caml\_startup\_exn}
\item \funcname{caml\_startup}
\item \funcname{caml\_main}
\item \funcname{main}
\end{itemize}
\smallskip
\item \funcname{caml\_startup} checks the return value of \funcname{caml\_startup\_exn}
\begin{itemize}
\item The return value has been specially marked by \funcname{caml\_start\_program} handler
\item Calls \funcname{caml\_fatal\_uncaught\_exception}
\end{itemize}
\item<2-> \funcname{caml\_fatal\_uncaught\_exception} either
\begin{itemize}
\item<2-> Calls the user custom uncaught exception handler, if set with \mintinline{ocaml}{Printexc.set_uncaught_exception_handler fn}\footnotemark
\item<2-> Or falls back to the default one
\item<2-> Which notifies about the uncaught exception
\end{itemize}
\item<4-> Terminates the program either with \funcname{abort} or with a non-zero exit code
\end{itemize}
\end{column}
\begin{column}{0.55\textwidth}
\only<1>{
\listc[firstline=84,lastline=89,highlightlines=88]{../ocaml/runtime/caml/mlvalues.h}{runtime/caml/mlvalues.h:84}
\listc[firstline=138,lastline=143,highlightlines={141-142}]{../ocaml/runtime/startup\_nat.c}{runtime/startup\_nat.c:138}
}%
\only<2>{
\listc[firstline=138,highlightlines={147,149}]{../ocaml/runtime/printexc.c}{runtime/printexc.c:138}
}%
\only<3>{
\listc[firstline=110,lastline=134,highlightlines=129]{../ocaml/runtime/printexc.c}{runtime/printexc.c:138}
}%
\only<4>{
\listc[firstline=138,highlightlines={152,154}]{../ocaml/runtime/printexc.c}{runtime/printexc.c:138}
}%
\end{column}
\end{columns}
\footnotetext[1]{\url{https://v2.ocaml.org/api/Printexc.html\#1\_Uncaughtexceptions}}
\end{frame}
}%
\onslide*<6->{\providecommand\step{4}%
% Runtime's default exception handler
%
\subsection*{Runtime's default exception handler}
\frameSubsection{pictures/The\_Architect.png}{}


\begin{frame}{Default exception handler}
\begin{columns}
\begin{column}{0.5\textwidth}
\begin{itemize}
\item \localname{domain\_state->exn\_handler} points to a trap located before \funcname{entry}!
\item What installed this very first trap there?
\end{itemize}
\bigskip
\listocaml{../src/default/default.ml}{default/default.ml}
\bigskip
\inputminted[fontsize=\footnotesize]{text}{../src/default/default.out}
\end{column}
\begin{column}{0.5\textwidth}
\centering
\only<1>{\providecommand\step{0}\input{diagrams/default/default.tex}}%
\only<2>{\providecommand\step{4}\input{diagrams/default/default.tex}}%
\end{column}
\end{columns}
\end{frame}


\begin{frame}{Startup}{How did we get there by the way?}
\begin{columns}
\begin{column}{0.45\textwidth}
\begin{itemize}
\item The entry point address of the binary is \funcname{main} (\funcname{\_start}) from the runtime
\item The program starts like a C program:
\begin{itemize}
\item \funcname{caml\_start\_program}
\item \funcname{caml\_startup\_common}
\item \funcname{caml\_startup\_exn}
\item \funcname{caml\_startup}
\item \funcname{caml\_main}
\item \funcname{main}
\end{itemize}
%\item \funcname{caml\_startup\_common}: initializes GC, domains \dots
% Also: codefrag, locale, custom opeartions, frame_descr, signals, stack
\item \funcname{caml\_start\_program} is the transition point between the C realm and the OCaml one
\end{itemize}
\bigskip
\listocaml{../src/default/default.ml}{default/default.ml}
\end{column}
\begin{column}{0.55\textwidth}
\listasm[lastline=24,highlightlines={22-24}]{src/ocaml/caml\_start\_program.s}{runtime/amd64.S:604}
\end{column}
\end{columns}
\end{frame}


\begin{frame}{Setup to jump into OCaml entry point}{\funcname{caml\_start\_program} initializes the main fiber}
\begin{columns}
\begin{column}{0.6\textwidth}
\only<1,3,5>{
\begin{itemize}
\item<1-> Stores information to allow switching from OCaml stack to the C stack
\item<3-> Constructs the default exception handler
\item<5-> Switches to the OCaml stack and calls \funcname{caml\_program}
\end{itemize}
\bigskip
\listocaml{../src/default/default.ml}{default/default.ml}
}%
\only<2>{
\listasm[firstline=22,lastline=41,highlightlines={25-27}]{src/ocaml/caml\_start\_program.s}{caml\_start\_program}
}%
\only<4>{
\mintasm[firstline=22,lastline=41,highlightlines={31-38}]{src/ocaml/caml\_start\_program.s}
\listasm[firstline=66,lastline=72]{src/ocaml/caml\_start\_program.s}{caml\_start\_program}
}%
\only<6>{
\listasm[firstline=22,lastline=41,highlightlines={39-41}]{src/ocaml/caml\_start\_program.s}{caml\_start\_program}
}%
\end{column}
\begin{column}{0.4\textwidth}
\centering
\only<1-2>{\providecommand\step{2}\input{diagrams/default/default.tex}}%
\only<3-5>{\providecommand\step{3}\input{diagrams/default/default.tex}}%
\onslide*<6->{\providecommand\step{4}\input{diagrams/default/default.tex}}%
\end{column}
\end{columns}
\end{frame}

\begin{frame}{Executing the OCaml program}{And raising an exception without handler}
\begin{columns}
\begin{column}{0.6\textwidth}
\only<1-3>{
\begin{itemize}
\item<1-> Execution continues with OCaml code
\begin{itemize}
\item \funcname{camlDefault\_\_main\_269}
\item \funcname{camlDefault\_\_entry}
\item \funcname{caml\_program}
\end{itemize}
\item<1-> \funcname{camlDefault\_\_main\_269} raises a \typename{Failure} exception
\item<1-> \typename{Failure} is caught by \funcname{caml\_start\_program}
\begin{itemize}
\item<1-> The handler set the 2nd LSB of the exception value to \funcarg{1}
\item<3-> And then forces \funcname{caml\_start\_program} to terminate
\end{itemize}
\end{itemize}
}
\bigskip
\only<1,3>{\listocaml[highlightlines=3]{../src/default/default.ml}{default/default.ml}}%
\only<2>{\listasm[firstline=66,lastline=72,highlightlines=69]{src/ocaml/caml\_start\_program.s}{caml\_start\_program}}%
\end{column}
\begin{column}{0.4\textwidth}
\centering
\providecommand\step{4}\input{diagrams/default/default.tex}
\end{column}
\end{columns}
\end{frame}
%\only<>{\listasm[firstline=47,lastline=65]{src/ocaml/caml\_start\_program.s}{caml\_start\_program}}


\begin{frame}{Back to C}{Clean up, complain and terminate}
\begin{columns}
\begin{column}{0.45\textwidth}
\begin{itemize}
\item Backtrace:
\begin{itemize}
\item \funcname{caml\_start\_program} $\rightarrow$ OCaml code
\item \funcname{caml\_startup\_common}
\item \funcname{caml\_startup\_exn}
\item \funcname{caml\_startup}
\item \funcname{caml\_main}
\item \funcname{main}
\end{itemize}
\smallskip
\item \funcname{caml\_startup} checks the return value of \funcname{caml\_startup\_exn}
\begin{itemize}
\item The return value has been specially marked by \funcname{caml\_start\_program} handler
\item Calls \funcname{caml\_fatal\_uncaught\_exception}
\end{itemize}
\item<2-> \funcname{caml\_fatal\_uncaught\_exception} either
\begin{itemize}
\item<2-> Calls the user custom uncaught exception handler, if set with \mintinline{ocaml}{Printexc.set_uncaught_exception_handler fn}\footnotemark
\item<2-> Or falls back to the default one
\item<2-> Which notifies about the uncaught exception
\end{itemize}
\item<4-> Terminates the program either with \funcname{abort} or with a non-zero exit code
\end{itemize}
\end{column}
\begin{column}{0.55\textwidth}
\only<1>{
\listc[firstline=84,lastline=89,highlightlines=88]{../ocaml/runtime/caml/mlvalues.h}{runtime/caml/mlvalues.h:84}
\listc[firstline=138,lastline=143,highlightlines={141-142}]{../ocaml/runtime/startup\_nat.c}{runtime/startup\_nat.c:138}
}%
\only<2>{
\listc[firstline=138,highlightlines={147,149}]{../ocaml/runtime/printexc.c}{runtime/printexc.c:138}
}%
\only<3>{
\listc[firstline=110,lastline=134,highlightlines=129]{../ocaml/runtime/printexc.c}{runtime/printexc.c:138}
}%
\only<4>{
\listc[firstline=138,highlightlines={152,154}]{../ocaml/runtime/printexc.c}{runtime/printexc.c:138}
}%
\end{column}
\end{columns}
\footnotetext[1]{\url{https://v2.ocaml.org/api/Printexc.html\#1\_Uncaughtexceptions}}
\end{frame}
}%
\end{column}
\end{columns}
\end{frame}

\begin{frame}{Executing the OCaml program}{And raising an exception without handler}
\begin{columns}
\begin{column}{0.6\textwidth}
\only<1-3>{
\begin{itemize}
\item<1-> Execution continues with OCaml code
\begin{itemize}
\item \funcname{camlDefault\_\_main\_269}
\item \funcname{camlDefault\_\_entry}
\item \funcname{caml\_program}
\end{itemize}
\item<1-> \funcname{camlDefault\_\_main\_269} raises a \typename{Failure} exception
\item<1-> \typename{Failure} is caught by \funcname{caml\_start\_program}
\begin{itemize}
\item<1-> The handler set the 2nd LSB of the exception value to \funcarg{1}
\item<3-> And then forces \funcname{caml\_start\_program} to terminate
\end{itemize}
\end{itemize}
}
\bigskip
\only<1,3>{\listocaml[highlightlines=3]{../src/default/default.ml}{default/default.ml}}%
\only<2>{\listasm[firstline=66,lastline=72,highlightlines=69]{src/ocaml/caml\_start\_program.s}{caml\_start\_program}}%
\end{column}
\begin{column}{0.4\textwidth}
\centering
\providecommand\step{4}%
% Runtime's default exception handler
%
\subsection*{Runtime's default exception handler}
\frameSubsection{pictures/The\_Architect.png}{}


\begin{frame}{Default exception handler}
\begin{columns}
\begin{column}{0.5\textwidth}
\begin{itemize}
\item \localname{domain\_state->exn\_handler} points to a trap located before \funcname{entry}!
\item What installed this very first trap there?
\end{itemize}
\bigskip
\listocaml{../src/default/default.ml}{default/default.ml}
\bigskip
\inputminted[fontsize=\footnotesize]{text}{../src/default/default.out}
\end{column}
\begin{column}{0.5\textwidth}
\centering
\only<1>{\providecommand\step{0}\input{diagrams/default/default.tex}}%
\only<2>{\providecommand\step{4}\input{diagrams/default/default.tex}}%
\end{column}
\end{columns}
\end{frame}


\begin{frame}{Startup}{How did we get there by the way?}
\begin{columns}
\begin{column}{0.45\textwidth}
\begin{itemize}
\item The entry point address of the binary is \funcname{main} (\funcname{\_start}) from the runtime
\item The program starts like a C program:
\begin{itemize}
\item \funcname{caml\_start\_program}
\item \funcname{caml\_startup\_common}
\item \funcname{caml\_startup\_exn}
\item \funcname{caml\_startup}
\item \funcname{caml\_main}
\item \funcname{main}
\end{itemize}
%\item \funcname{caml\_startup\_common}: initializes GC, domains \dots
% Also: codefrag, locale, custom opeartions, frame_descr, signals, stack
\item \funcname{caml\_start\_program} is the transition point between the C realm and the OCaml one
\end{itemize}
\bigskip
\listocaml{../src/default/default.ml}{default/default.ml}
\end{column}
\begin{column}{0.55\textwidth}
\listasm[lastline=24,highlightlines={22-24}]{src/ocaml/caml\_start\_program.s}{runtime/amd64.S:604}
\end{column}
\end{columns}
\end{frame}


\begin{frame}{Setup to jump into OCaml entry point}{\funcname{caml\_start\_program} initializes the main fiber}
\begin{columns}
\begin{column}{0.6\textwidth}
\only<1,3,5>{
\begin{itemize}
\item<1-> Stores information to allow switching from OCaml stack to the C stack
\item<3-> Constructs the default exception handler
\item<5-> Switches to the OCaml stack and calls \funcname{caml\_program}
\end{itemize}
\bigskip
\listocaml{../src/default/default.ml}{default/default.ml}
}%
\only<2>{
\listasm[firstline=22,lastline=41,highlightlines={25-27}]{src/ocaml/caml\_start\_program.s}{caml\_start\_program}
}%
\only<4>{
\mintasm[firstline=22,lastline=41,highlightlines={31-38}]{src/ocaml/caml\_start\_program.s}
\listasm[firstline=66,lastline=72]{src/ocaml/caml\_start\_program.s}{caml\_start\_program}
}%
\only<6>{
\listasm[firstline=22,lastline=41,highlightlines={39-41}]{src/ocaml/caml\_start\_program.s}{caml\_start\_program}
}%
\end{column}
\begin{column}{0.4\textwidth}
\centering
\only<1-2>{\providecommand\step{2}\input{diagrams/default/default.tex}}%
\only<3-5>{\providecommand\step{3}\input{diagrams/default/default.tex}}%
\onslide*<6->{\providecommand\step{4}\input{diagrams/default/default.tex}}%
\end{column}
\end{columns}
\end{frame}

\begin{frame}{Executing the OCaml program}{And raising an exception without handler}
\begin{columns}
\begin{column}{0.6\textwidth}
\only<1-3>{
\begin{itemize}
\item<1-> Execution continues with OCaml code
\begin{itemize}
\item \funcname{camlDefault\_\_main\_269}
\item \funcname{camlDefault\_\_entry}
\item \funcname{caml\_program}
\end{itemize}
\item<1-> \funcname{camlDefault\_\_main\_269} raises a \typename{Failure} exception
\item<1-> \typename{Failure} is caught by \funcname{caml\_start\_program}
\begin{itemize}
\item<1-> The handler set the 2nd LSB of the exception value to \funcarg{1}
\item<3-> And then forces \funcname{caml\_start\_program} to terminate
\end{itemize}
\end{itemize}
}
\bigskip
\only<1,3>{\listocaml[highlightlines=3]{../src/default/default.ml}{default/default.ml}}%
\only<2>{\listasm[firstline=66,lastline=72,highlightlines=69]{src/ocaml/caml\_start\_program.s}{caml\_start\_program}}%
\end{column}
\begin{column}{0.4\textwidth}
\centering
\providecommand\step{4}\input{diagrams/default/default.tex}
\end{column}
\end{columns}
\end{frame}
%\only<>{\listasm[firstline=47,lastline=65]{src/ocaml/caml\_start\_program.s}{caml\_start\_program}}


\begin{frame}{Back to C}{Clean up, complain and terminate}
\begin{columns}
\begin{column}{0.45\textwidth}
\begin{itemize}
\item Backtrace:
\begin{itemize}
\item \funcname{caml\_start\_program} $\rightarrow$ OCaml code
\item \funcname{caml\_startup\_common}
\item \funcname{caml\_startup\_exn}
\item \funcname{caml\_startup}
\item \funcname{caml\_main}
\item \funcname{main}
\end{itemize}
\smallskip
\item \funcname{caml\_startup} checks the return value of \funcname{caml\_startup\_exn}
\begin{itemize}
\item The return value has been specially marked by \funcname{caml\_start\_program} handler
\item Calls \funcname{caml\_fatal\_uncaught\_exception}
\end{itemize}
\item<2-> \funcname{caml\_fatal\_uncaught\_exception} either
\begin{itemize}
\item<2-> Calls the user custom uncaught exception handler, if set with \mintinline{ocaml}{Printexc.set_uncaught_exception_handler fn}\footnotemark
\item<2-> Or falls back to the default one
\item<2-> Which notifies about the uncaught exception
\end{itemize}
\item<4-> Terminates the program either with \funcname{abort} or with a non-zero exit code
\end{itemize}
\end{column}
\begin{column}{0.55\textwidth}
\only<1>{
\listc[firstline=84,lastline=89,highlightlines=88]{../ocaml/runtime/caml/mlvalues.h}{runtime/caml/mlvalues.h:84}
\listc[firstline=138,lastline=143,highlightlines={141-142}]{../ocaml/runtime/startup\_nat.c}{runtime/startup\_nat.c:138}
}%
\only<2>{
\listc[firstline=138,highlightlines={147,149}]{../ocaml/runtime/printexc.c}{runtime/printexc.c:138}
}%
\only<3>{
\listc[firstline=110,lastline=134,highlightlines=129]{../ocaml/runtime/printexc.c}{runtime/printexc.c:138}
}%
\only<4>{
\listc[firstline=138,highlightlines={152,154}]{../ocaml/runtime/printexc.c}{runtime/printexc.c:138}
}%
\end{column}
\end{columns}
\footnotetext[1]{\url{https://v2.ocaml.org/api/Printexc.html\#1\_Uncaughtexceptions}}
\end{frame}

\end{column}
\end{columns}
\end{frame}
%\only<>{\listasm[firstline=47,lastline=65]{src/ocaml/caml\_start\_program.s}{caml\_start\_program}}


\begin{frame}{Back to C}{Clean up, complain and terminate}
\begin{columns}
\begin{column}{0.45\textwidth}
\begin{itemize}
\item Backtrace:
\begin{itemize}
\item \funcname{caml\_start\_program} $\rightarrow$ OCaml code
\item \funcname{caml\_startup\_common}
\item \funcname{caml\_startup\_exn}
\item \funcname{caml\_startup}
\item \funcname{caml\_main}
\item \funcname{main}
\end{itemize}
\smallskip
\item \funcname{caml\_startup} checks the return value of \funcname{caml\_startup\_exn}
\begin{itemize}
\item The return value has been specially marked by \funcname{caml\_start\_program} handler
\item Calls \funcname{caml\_fatal\_uncaught\_exception}
\end{itemize}
\item<2-> \funcname{caml\_fatal\_uncaught\_exception} either
\begin{itemize}
\item<2-> Calls the user custom uncaught exception handler, if set with \mintinline{ocaml}{Printexc.set_uncaught_exception_handler fn}\footnotemark
\item<2-> Or falls back to the default one
\item<2-> Which notifies about the uncaught exception
\end{itemize}
\item<4-> Terminates the program either with \funcname{abort} or with a non-zero exit code
\end{itemize}
\end{column}
\begin{column}{0.55\textwidth}
\only<1>{
\listc[firstline=84,lastline=89,highlightlines=88]{../ocaml/runtime/caml/mlvalues.h}{runtime/caml/mlvalues.h:84}
\listc[firstline=138,lastline=143,highlightlines={141-142}]{../ocaml/runtime/startup\_nat.c}{runtime/startup\_nat.c:138}
}%
\only<2>{
\listc[firstline=138,highlightlines={147,149}]{../ocaml/runtime/printexc.c}{runtime/printexc.c:138}
}%
\only<3>{
\listc[firstline=110,lastline=134,highlightlines=129]{../ocaml/runtime/printexc.c}{runtime/printexc.c:138}
}%
\only<4>{
\listc[firstline=138,highlightlines={152,154}]{../ocaml/runtime/printexc.c}{runtime/printexc.c:138}
}%
\end{column}
\end{columns}
\footnotetext[1]{\url{https://v2.ocaml.org/api/Printexc.html\#1\_Uncaughtexceptions}}
\end{frame}
}%
\end{column}
\end{columns}
\end{frame}


\begin{frame}{Startup}{How did we get there by the way?}
\begin{columns}
\begin{column}{0.45\textwidth}
\begin{itemize}
\item The entry point address of the binary is \funcname{main} (\funcname{\_start}) from the runtime
\item The program starts like a C program:
\begin{itemize}
\item \funcname{caml\_start\_program}
\item \funcname{caml\_startup\_common}
\item \funcname{caml\_startup\_exn}
\item \funcname{caml\_startup}
\item \funcname{caml\_main}
\item \funcname{main}
\end{itemize}
%\item \funcname{caml\_startup\_common}: initializes GC, domains \dots
% Also: codefrag, locale, custom opeartions, frame_descr, signals, stack
\item \funcname{caml\_start\_program} is the transition point between the C realm and the OCaml one
\end{itemize}
\bigskip
\listocaml{../src/default/default.ml}{default/default.ml}
\end{column}
\begin{column}{0.55\textwidth}
\listasm[lastline=24,highlightlines={22-24}]{src/ocaml/caml\_start\_program.s}{runtime/amd64.S:604}
\end{column}
\end{columns}
\end{frame}


\begin{frame}{Setup to jump into OCaml entry point}{\funcname{caml\_start\_program} initializes the main fiber}
\begin{columns}
\begin{column}{0.6\textwidth}
\only<1,3,5>{
\begin{itemize}
\item<1-> Stores information to allow switching from OCaml stack to the C stack
\item<3-> Constructs the default exception handler
\item<5-> Switches to the OCaml stack and calls \funcname{caml\_program}
\end{itemize}
\bigskip
\listocaml{../src/default/default.ml}{default/default.ml}
}%
\only<2>{
\listasm[firstline=22,lastline=41,highlightlines={25-27}]{src/ocaml/caml\_start\_program.s}{caml\_start\_program}
}%
\only<4>{
\mintasm[firstline=22,lastline=41,highlightlines={31-38}]{src/ocaml/caml\_start\_program.s}
\listasm[firstline=66,lastline=72]{src/ocaml/caml\_start\_program.s}{caml\_start\_program}
}%
\only<6>{
\listasm[firstline=22,lastline=41,highlightlines={39-41}]{src/ocaml/caml\_start\_program.s}{caml\_start\_program}
}%
\end{column}
\begin{column}{0.4\textwidth}
\centering
\only<1-2>{\providecommand\step{2}%
% Runtime's default exception handler
%
\subsection*{Runtime's default exception handler}
\frameSubsection{pictures/The\_Architect.png}{}


\begin{frame}{Default exception handler}
\begin{columns}
\begin{column}{0.5\textwidth}
\begin{itemize}
\item \localname{domain\_state->exn\_handler} points to a trap located before \funcname{entry}!
\item What installed this very first trap there?
\end{itemize}
\bigskip
\listocaml{../src/default/default.ml}{default/default.ml}
\bigskip
\inputminted[fontsize=\footnotesize]{text}{../src/default/default.out}
\end{column}
\begin{column}{0.5\textwidth}
\centering
\only<1>{\providecommand\step{0}%
% Runtime's default exception handler
%
\subsection*{Runtime's default exception handler}
\frameSubsection{pictures/The\_Architect.png}{}


\begin{frame}{Default exception handler}
\begin{columns}
\begin{column}{0.5\textwidth}
\begin{itemize}
\item \localname{domain\_state->exn\_handler} points to a trap located before \funcname{entry}!
\item What installed this very first trap there?
\end{itemize}
\bigskip
\listocaml{../src/default/default.ml}{default/default.ml}
\bigskip
\inputminted[fontsize=\footnotesize]{text}{../src/default/default.out}
\end{column}
\begin{column}{0.5\textwidth}
\centering
\only<1>{\providecommand\step{0}\input{diagrams/default/default.tex}}%
\only<2>{\providecommand\step{4}\input{diagrams/default/default.tex}}%
\end{column}
\end{columns}
\end{frame}


\begin{frame}{Startup}{How did we get there by the way?}
\begin{columns}
\begin{column}{0.45\textwidth}
\begin{itemize}
\item The entry point address of the binary is \funcname{main} (\funcname{\_start}) from the runtime
\item The program starts like a C program:
\begin{itemize}
\item \funcname{caml\_start\_program}
\item \funcname{caml\_startup\_common}
\item \funcname{caml\_startup\_exn}
\item \funcname{caml\_startup}
\item \funcname{caml\_main}
\item \funcname{main}
\end{itemize}
%\item \funcname{caml\_startup\_common}: initializes GC, domains \dots
% Also: codefrag, locale, custom opeartions, frame_descr, signals, stack
\item \funcname{caml\_start\_program} is the transition point between the C realm and the OCaml one
\end{itemize}
\bigskip
\listocaml{../src/default/default.ml}{default/default.ml}
\end{column}
\begin{column}{0.55\textwidth}
\listasm[lastline=24,highlightlines={22-24}]{src/ocaml/caml\_start\_program.s}{runtime/amd64.S:604}
\end{column}
\end{columns}
\end{frame}


\begin{frame}{Setup to jump into OCaml entry point}{\funcname{caml\_start\_program} initializes the main fiber}
\begin{columns}
\begin{column}{0.6\textwidth}
\only<1,3,5>{
\begin{itemize}
\item<1-> Stores information to allow switching from OCaml stack to the C stack
\item<3-> Constructs the default exception handler
\item<5-> Switches to the OCaml stack and calls \funcname{caml\_program}
\end{itemize}
\bigskip
\listocaml{../src/default/default.ml}{default/default.ml}
}%
\only<2>{
\listasm[firstline=22,lastline=41,highlightlines={25-27}]{src/ocaml/caml\_start\_program.s}{caml\_start\_program}
}%
\only<4>{
\mintasm[firstline=22,lastline=41,highlightlines={31-38}]{src/ocaml/caml\_start\_program.s}
\listasm[firstline=66,lastline=72]{src/ocaml/caml\_start\_program.s}{caml\_start\_program}
}%
\only<6>{
\listasm[firstline=22,lastline=41,highlightlines={39-41}]{src/ocaml/caml\_start\_program.s}{caml\_start\_program}
}%
\end{column}
\begin{column}{0.4\textwidth}
\centering
\only<1-2>{\providecommand\step{2}\input{diagrams/default/default.tex}}%
\only<3-5>{\providecommand\step{3}\input{diagrams/default/default.tex}}%
\onslide*<6->{\providecommand\step{4}\input{diagrams/default/default.tex}}%
\end{column}
\end{columns}
\end{frame}

\begin{frame}{Executing the OCaml program}{And raising an exception without handler}
\begin{columns}
\begin{column}{0.6\textwidth}
\only<1-3>{
\begin{itemize}
\item<1-> Execution continues with OCaml code
\begin{itemize}
\item \funcname{camlDefault\_\_main\_269}
\item \funcname{camlDefault\_\_entry}
\item \funcname{caml\_program}
\end{itemize}
\item<1-> \funcname{camlDefault\_\_main\_269} raises a \typename{Failure} exception
\item<1-> \typename{Failure} is caught by \funcname{caml\_start\_program}
\begin{itemize}
\item<1-> The handler set the 2nd LSB of the exception value to \funcarg{1}
\item<3-> And then forces \funcname{caml\_start\_program} to terminate
\end{itemize}
\end{itemize}
}
\bigskip
\only<1,3>{\listocaml[highlightlines=3]{../src/default/default.ml}{default/default.ml}}%
\only<2>{\listasm[firstline=66,lastline=72,highlightlines=69]{src/ocaml/caml\_start\_program.s}{caml\_start\_program}}%
\end{column}
\begin{column}{0.4\textwidth}
\centering
\providecommand\step{4}\input{diagrams/default/default.tex}
\end{column}
\end{columns}
\end{frame}
%\only<>{\listasm[firstline=47,lastline=65]{src/ocaml/caml\_start\_program.s}{caml\_start\_program}}


\begin{frame}{Back to C}{Clean up, complain and terminate}
\begin{columns}
\begin{column}{0.45\textwidth}
\begin{itemize}
\item Backtrace:
\begin{itemize}
\item \funcname{caml\_start\_program} $\rightarrow$ OCaml code
\item \funcname{caml\_startup\_common}
\item \funcname{caml\_startup\_exn}
\item \funcname{caml\_startup}
\item \funcname{caml\_main}
\item \funcname{main}
\end{itemize}
\smallskip
\item \funcname{caml\_startup} checks the return value of \funcname{caml\_startup\_exn}
\begin{itemize}
\item The return value has been specially marked by \funcname{caml\_start\_program} handler
\item Calls \funcname{caml\_fatal\_uncaught\_exception}
\end{itemize}
\item<2-> \funcname{caml\_fatal\_uncaught\_exception} either
\begin{itemize}
\item<2-> Calls the user custom uncaught exception handler, if set with \mintinline{ocaml}{Printexc.set_uncaught_exception_handler fn}\footnotemark
\item<2-> Or falls back to the default one
\item<2-> Which notifies about the uncaught exception
\end{itemize}
\item<4-> Terminates the program either with \funcname{abort} or with a non-zero exit code
\end{itemize}
\end{column}
\begin{column}{0.55\textwidth}
\only<1>{
\listc[firstline=84,lastline=89,highlightlines=88]{../ocaml/runtime/caml/mlvalues.h}{runtime/caml/mlvalues.h:84}
\listc[firstline=138,lastline=143,highlightlines={141-142}]{../ocaml/runtime/startup\_nat.c}{runtime/startup\_nat.c:138}
}%
\only<2>{
\listc[firstline=138,highlightlines={147,149}]{../ocaml/runtime/printexc.c}{runtime/printexc.c:138}
}%
\only<3>{
\listc[firstline=110,lastline=134,highlightlines=129]{../ocaml/runtime/printexc.c}{runtime/printexc.c:138}
}%
\only<4>{
\listc[firstline=138,highlightlines={152,154}]{../ocaml/runtime/printexc.c}{runtime/printexc.c:138}
}%
\end{column}
\end{columns}
\footnotetext[1]{\url{https://v2.ocaml.org/api/Printexc.html\#1\_Uncaughtexceptions}}
\end{frame}
}%
\only<2>{\providecommand\step{4}%
% Runtime's default exception handler
%
\subsection*{Runtime's default exception handler}
\frameSubsection{pictures/The\_Architect.png}{}


\begin{frame}{Default exception handler}
\begin{columns}
\begin{column}{0.5\textwidth}
\begin{itemize}
\item \localname{domain\_state->exn\_handler} points to a trap located before \funcname{entry}!
\item What installed this very first trap there?
\end{itemize}
\bigskip
\listocaml{../src/default/default.ml}{default/default.ml}
\bigskip
\inputminted[fontsize=\footnotesize]{text}{../src/default/default.out}
\end{column}
\begin{column}{0.5\textwidth}
\centering
\only<1>{\providecommand\step{0}\input{diagrams/default/default.tex}}%
\only<2>{\providecommand\step{4}\input{diagrams/default/default.tex}}%
\end{column}
\end{columns}
\end{frame}


\begin{frame}{Startup}{How did we get there by the way?}
\begin{columns}
\begin{column}{0.45\textwidth}
\begin{itemize}
\item The entry point address of the binary is \funcname{main} (\funcname{\_start}) from the runtime
\item The program starts like a C program:
\begin{itemize}
\item \funcname{caml\_start\_program}
\item \funcname{caml\_startup\_common}
\item \funcname{caml\_startup\_exn}
\item \funcname{caml\_startup}
\item \funcname{caml\_main}
\item \funcname{main}
\end{itemize}
%\item \funcname{caml\_startup\_common}: initializes GC, domains \dots
% Also: codefrag, locale, custom opeartions, frame_descr, signals, stack
\item \funcname{caml\_start\_program} is the transition point between the C realm and the OCaml one
\end{itemize}
\bigskip
\listocaml{../src/default/default.ml}{default/default.ml}
\end{column}
\begin{column}{0.55\textwidth}
\listasm[lastline=24,highlightlines={22-24}]{src/ocaml/caml\_start\_program.s}{runtime/amd64.S:604}
\end{column}
\end{columns}
\end{frame}


\begin{frame}{Setup to jump into OCaml entry point}{\funcname{caml\_start\_program} initializes the main fiber}
\begin{columns}
\begin{column}{0.6\textwidth}
\only<1,3,5>{
\begin{itemize}
\item<1-> Stores information to allow switching from OCaml stack to the C stack
\item<3-> Constructs the default exception handler
\item<5-> Switches to the OCaml stack and calls \funcname{caml\_program}
\end{itemize}
\bigskip
\listocaml{../src/default/default.ml}{default/default.ml}
}%
\only<2>{
\listasm[firstline=22,lastline=41,highlightlines={25-27}]{src/ocaml/caml\_start\_program.s}{caml\_start\_program}
}%
\only<4>{
\mintasm[firstline=22,lastline=41,highlightlines={31-38}]{src/ocaml/caml\_start\_program.s}
\listasm[firstline=66,lastline=72]{src/ocaml/caml\_start\_program.s}{caml\_start\_program}
}%
\only<6>{
\listasm[firstline=22,lastline=41,highlightlines={39-41}]{src/ocaml/caml\_start\_program.s}{caml\_start\_program}
}%
\end{column}
\begin{column}{0.4\textwidth}
\centering
\only<1-2>{\providecommand\step{2}\input{diagrams/default/default.tex}}%
\only<3-5>{\providecommand\step{3}\input{diagrams/default/default.tex}}%
\onslide*<6->{\providecommand\step{4}\input{diagrams/default/default.tex}}%
\end{column}
\end{columns}
\end{frame}

\begin{frame}{Executing the OCaml program}{And raising an exception without handler}
\begin{columns}
\begin{column}{0.6\textwidth}
\only<1-3>{
\begin{itemize}
\item<1-> Execution continues with OCaml code
\begin{itemize}
\item \funcname{camlDefault\_\_main\_269}
\item \funcname{camlDefault\_\_entry}
\item \funcname{caml\_program}
\end{itemize}
\item<1-> \funcname{camlDefault\_\_main\_269} raises a \typename{Failure} exception
\item<1-> \typename{Failure} is caught by \funcname{caml\_start\_program}
\begin{itemize}
\item<1-> The handler set the 2nd LSB of the exception value to \funcarg{1}
\item<3-> And then forces \funcname{caml\_start\_program} to terminate
\end{itemize}
\end{itemize}
}
\bigskip
\only<1,3>{\listocaml[highlightlines=3]{../src/default/default.ml}{default/default.ml}}%
\only<2>{\listasm[firstline=66,lastline=72,highlightlines=69]{src/ocaml/caml\_start\_program.s}{caml\_start\_program}}%
\end{column}
\begin{column}{0.4\textwidth}
\centering
\providecommand\step{4}\input{diagrams/default/default.tex}
\end{column}
\end{columns}
\end{frame}
%\only<>{\listasm[firstline=47,lastline=65]{src/ocaml/caml\_start\_program.s}{caml\_start\_program}}


\begin{frame}{Back to C}{Clean up, complain and terminate}
\begin{columns}
\begin{column}{0.45\textwidth}
\begin{itemize}
\item Backtrace:
\begin{itemize}
\item \funcname{caml\_start\_program} $\rightarrow$ OCaml code
\item \funcname{caml\_startup\_common}
\item \funcname{caml\_startup\_exn}
\item \funcname{caml\_startup}
\item \funcname{caml\_main}
\item \funcname{main}
\end{itemize}
\smallskip
\item \funcname{caml\_startup} checks the return value of \funcname{caml\_startup\_exn}
\begin{itemize}
\item The return value has been specially marked by \funcname{caml\_start\_program} handler
\item Calls \funcname{caml\_fatal\_uncaught\_exception}
\end{itemize}
\item<2-> \funcname{caml\_fatal\_uncaught\_exception} either
\begin{itemize}
\item<2-> Calls the user custom uncaught exception handler, if set with \mintinline{ocaml}{Printexc.set_uncaught_exception_handler fn}\footnotemark
\item<2-> Or falls back to the default one
\item<2-> Which notifies about the uncaught exception
\end{itemize}
\item<4-> Terminates the program either with \funcname{abort} or with a non-zero exit code
\end{itemize}
\end{column}
\begin{column}{0.55\textwidth}
\only<1>{
\listc[firstline=84,lastline=89,highlightlines=88]{../ocaml/runtime/caml/mlvalues.h}{runtime/caml/mlvalues.h:84}
\listc[firstline=138,lastline=143,highlightlines={141-142}]{../ocaml/runtime/startup\_nat.c}{runtime/startup\_nat.c:138}
}%
\only<2>{
\listc[firstline=138,highlightlines={147,149}]{../ocaml/runtime/printexc.c}{runtime/printexc.c:138}
}%
\only<3>{
\listc[firstline=110,lastline=134,highlightlines=129]{../ocaml/runtime/printexc.c}{runtime/printexc.c:138}
}%
\only<4>{
\listc[firstline=138,highlightlines={152,154}]{../ocaml/runtime/printexc.c}{runtime/printexc.c:138}
}%
\end{column}
\end{columns}
\footnotetext[1]{\url{https://v2.ocaml.org/api/Printexc.html\#1\_Uncaughtexceptions}}
\end{frame}
}%
\end{column}
\end{columns}
\end{frame}


\begin{frame}{Startup}{How did we get there by the way?}
\begin{columns}
\begin{column}{0.45\textwidth}
\begin{itemize}
\item The entry point address of the binary is \funcname{main} (\funcname{\_start}) from the runtime
\item The program starts like a C program:
\begin{itemize}
\item \funcname{caml\_start\_program}
\item \funcname{caml\_startup\_common}
\item \funcname{caml\_startup\_exn}
\item \funcname{caml\_startup}
\item \funcname{caml\_main}
\item \funcname{main}
\end{itemize}
%\item \funcname{caml\_startup\_common}: initializes GC, domains \dots
% Also: codefrag, locale, custom opeartions, frame_descr, signals, stack
\item \funcname{caml\_start\_program} is the transition point between the C realm and the OCaml one
\end{itemize}
\bigskip
\listocaml{../src/default/default.ml}{default/default.ml}
\end{column}
\begin{column}{0.55\textwidth}
\listasm[lastline=24,highlightlines={22-24}]{src/ocaml/caml\_start\_program.s}{runtime/amd64.S:604}
\end{column}
\end{columns}
\end{frame}


\begin{frame}{Setup to jump into OCaml entry point}{\funcname{caml\_start\_program} initializes the main fiber}
\begin{columns}
\begin{column}{0.6\textwidth}
\only<1,3,5>{
\begin{itemize}
\item<1-> Stores information to allow switching from OCaml stack to the C stack
\item<3-> Constructs the default exception handler
\item<5-> Switches to the OCaml stack and calls \funcname{caml\_program}
\end{itemize}
\bigskip
\listocaml{../src/default/default.ml}{default/default.ml}
}%
\only<2>{
\listasm[firstline=22,lastline=41,highlightlines={25-27}]{src/ocaml/caml\_start\_program.s}{caml\_start\_program}
}%
\only<4>{
\mintasm[firstline=22,lastline=41,highlightlines={31-38}]{src/ocaml/caml\_start\_program.s}
\listasm[firstline=66,lastline=72]{src/ocaml/caml\_start\_program.s}{caml\_start\_program}
}%
\only<6>{
\listasm[firstline=22,lastline=41,highlightlines={39-41}]{src/ocaml/caml\_start\_program.s}{caml\_start\_program}
}%
\end{column}
\begin{column}{0.4\textwidth}
\centering
\only<1-2>{\providecommand\step{2}%
% Runtime's default exception handler
%
\subsection*{Runtime's default exception handler}
\frameSubsection{pictures/The\_Architect.png}{}


\begin{frame}{Default exception handler}
\begin{columns}
\begin{column}{0.5\textwidth}
\begin{itemize}
\item \localname{domain\_state->exn\_handler} points to a trap located before \funcname{entry}!
\item What installed this very first trap there?
\end{itemize}
\bigskip
\listocaml{../src/default/default.ml}{default/default.ml}
\bigskip
\inputminted[fontsize=\footnotesize]{text}{../src/default/default.out}
\end{column}
\begin{column}{0.5\textwidth}
\centering
\only<1>{\providecommand\step{0}\input{diagrams/default/default.tex}}%
\only<2>{\providecommand\step{4}\input{diagrams/default/default.tex}}%
\end{column}
\end{columns}
\end{frame}


\begin{frame}{Startup}{How did we get there by the way?}
\begin{columns}
\begin{column}{0.45\textwidth}
\begin{itemize}
\item The entry point address of the binary is \funcname{main} (\funcname{\_start}) from the runtime
\item The program starts like a C program:
\begin{itemize}
\item \funcname{caml\_start\_program}
\item \funcname{caml\_startup\_common}
\item \funcname{caml\_startup\_exn}
\item \funcname{caml\_startup}
\item \funcname{caml\_main}
\item \funcname{main}
\end{itemize}
%\item \funcname{caml\_startup\_common}: initializes GC, domains \dots
% Also: codefrag, locale, custom opeartions, frame_descr, signals, stack
\item \funcname{caml\_start\_program} is the transition point between the C realm and the OCaml one
\end{itemize}
\bigskip
\listocaml{../src/default/default.ml}{default/default.ml}
\end{column}
\begin{column}{0.55\textwidth}
\listasm[lastline=24,highlightlines={22-24}]{src/ocaml/caml\_start\_program.s}{runtime/amd64.S:604}
\end{column}
\end{columns}
\end{frame}


\begin{frame}{Setup to jump into OCaml entry point}{\funcname{caml\_start\_program} initializes the main fiber}
\begin{columns}
\begin{column}{0.6\textwidth}
\only<1,3,5>{
\begin{itemize}
\item<1-> Stores information to allow switching from OCaml stack to the C stack
\item<3-> Constructs the default exception handler
\item<5-> Switches to the OCaml stack and calls \funcname{caml\_program}
\end{itemize}
\bigskip
\listocaml{../src/default/default.ml}{default/default.ml}
}%
\only<2>{
\listasm[firstline=22,lastline=41,highlightlines={25-27}]{src/ocaml/caml\_start\_program.s}{caml\_start\_program}
}%
\only<4>{
\mintasm[firstline=22,lastline=41,highlightlines={31-38}]{src/ocaml/caml\_start\_program.s}
\listasm[firstline=66,lastline=72]{src/ocaml/caml\_start\_program.s}{caml\_start\_program}
}%
\only<6>{
\listasm[firstline=22,lastline=41,highlightlines={39-41}]{src/ocaml/caml\_start\_program.s}{caml\_start\_program}
}%
\end{column}
\begin{column}{0.4\textwidth}
\centering
\only<1-2>{\providecommand\step{2}\input{diagrams/default/default.tex}}%
\only<3-5>{\providecommand\step{3}\input{diagrams/default/default.tex}}%
\onslide*<6->{\providecommand\step{4}\input{diagrams/default/default.tex}}%
\end{column}
\end{columns}
\end{frame}

\begin{frame}{Executing the OCaml program}{And raising an exception without handler}
\begin{columns}
\begin{column}{0.6\textwidth}
\only<1-3>{
\begin{itemize}
\item<1-> Execution continues with OCaml code
\begin{itemize}
\item \funcname{camlDefault\_\_main\_269}
\item \funcname{camlDefault\_\_entry}
\item \funcname{caml\_program}
\end{itemize}
\item<1-> \funcname{camlDefault\_\_main\_269} raises a \typename{Failure} exception
\item<1-> \typename{Failure} is caught by \funcname{caml\_start\_program}
\begin{itemize}
\item<1-> The handler set the 2nd LSB of the exception value to \funcarg{1}
\item<3-> And then forces \funcname{caml\_start\_program} to terminate
\end{itemize}
\end{itemize}
}
\bigskip
\only<1,3>{\listocaml[highlightlines=3]{../src/default/default.ml}{default/default.ml}}%
\only<2>{\listasm[firstline=66,lastline=72,highlightlines=69]{src/ocaml/caml\_start\_program.s}{caml\_start\_program}}%
\end{column}
\begin{column}{0.4\textwidth}
\centering
\providecommand\step{4}\input{diagrams/default/default.tex}
\end{column}
\end{columns}
\end{frame}
%\only<>{\listasm[firstline=47,lastline=65]{src/ocaml/caml\_start\_program.s}{caml\_start\_program}}


\begin{frame}{Back to C}{Clean up, complain and terminate}
\begin{columns}
\begin{column}{0.45\textwidth}
\begin{itemize}
\item Backtrace:
\begin{itemize}
\item \funcname{caml\_start\_program} $\rightarrow$ OCaml code
\item \funcname{caml\_startup\_common}
\item \funcname{caml\_startup\_exn}
\item \funcname{caml\_startup}
\item \funcname{caml\_main}
\item \funcname{main}
\end{itemize}
\smallskip
\item \funcname{caml\_startup} checks the return value of \funcname{caml\_startup\_exn}
\begin{itemize}
\item The return value has been specially marked by \funcname{caml\_start\_program} handler
\item Calls \funcname{caml\_fatal\_uncaught\_exception}
\end{itemize}
\item<2-> \funcname{caml\_fatal\_uncaught\_exception} either
\begin{itemize}
\item<2-> Calls the user custom uncaught exception handler, if set with \mintinline{ocaml}{Printexc.set_uncaught_exception_handler fn}\footnotemark
\item<2-> Or falls back to the default one
\item<2-> Which notifies about the uncaught exception
\end{itemize}
\item<4-> Terminates the program either with \funcname{abort} or with a non-zero exit code
\end{itemize}
\end{column}
\begin{column}{0.55\textwidth}
\only<1>{
\listc[firstline=84,lastline=89,highlightlines=88]{../ocaml/runtime/caml/mlvalues.h}{runtime/caml/mlvalues.h:84}
\listc[firstline=138,lastline=143,highlightlines={141-142}]{../ocaml/runtime/startup\_nat.c}{runtime/startup\_nat.c:138}
}%
\only<2>{
\listc[firstline=138,highlightlines={147,149}]{../ocaml/runtime/printexc.c}{runtime/printexc.c:138}
}%
\only<3>{
\listc[firstline=110,lastline=134,highlightlines=129]{../ocaml/runtime/printexc.c}{runtime/printexc.c:138}
}%
\only<4>{
\listc[firstline=138,highlightlines={152,154}]{../ocaml/runtime/printexc.c}{runtime/printexc.c:138}
}%
\end{column}
\end{columns}
\footnotetext[1]{\url{https://v2.ocaml.org/api/Printexc.html\#1\_Uncaughtexceptions}}
\end{frame}
}%
\only<3-5>{\providecommand\step{3}%
% Runtime's default exception handler
%
\subsection*{Runtime's default exception handler}
\frameSubsection{pictures/The\_Architect.png}{}


\begin{frame}{Default exception handler}
\begin{columns}
\begin{column}{0.5\textwidth}
\begin{itemize}
\item \localname{domain\_state->exn\_handler} points to a trap located before \funcname{entry}!
\item What installed this very first trap there?
\end{itemize}
\bigskip
\listocaml{../src/default/default.ml}{default/default.ml}
\bigskip
\inputminted[fontsize=\footnotesize]{text}{../src/default/default.out}
\end{column}
\begin{column}{0.5\textwidth}
\centering
\only<1>{\providecommand\step{0}\input{diagrams/default/default.tex}}%
\only<2>{\providecommand\step{4}\input{diagrams/default/default.tex}}%
\end{column}
\end{columns}
\end{frame}


\begin{frame}{Startup}{How did we get there by the way?}
\begin{columns}
\begin{column}{0.45\textwidth}
\begin{itemize}
\item The entry point address of the binary is \funcname{main} (\funcname{\_start}) from the runtime
\item The program starts like a C program:
\begin{itemize}
\item \funcname{caml\_start\_program}
\item \funcname{caml\_startup\_common}
\item \funcname{caml\_startup\_exn}
\item \funcname{caml\_startup}
\item \funcname{caml\_main}
\item \funcname{main}
\end{itemize}
%\item \funcname{caml\_startup\_common}: initializes GC, domains \dots
% Also: codefrag, locale, custom opeartions, frame_descr, signals, stack
\item \funcname{caml\_start\_program} is the transition point between the C realm and the OCaml one
\end{itemize}
\bigskip
\listocaml{../src/default/default.ml}{default/default.ml}
\end{column}
\begin{column}{0.55\textwidth}
\listasm[lastline=24,highlightlines={22-24}]{src/ocaml/caml\_start\_program.s}{runtime/amd64.S:604}
\end{column}
\end{columns}
\end{frame}


\begin{frame}{Setup to jump into OCaml entry point}{\funcname{caml\_start\_program} initializes the main fiber}
\begin{columns}
\begin{column}{0.6\textwidth}
\only<1,3,5>{
\begin{itemize}
\item<1-> Stores information to allow switching from OCaml stack to the C stack
\item<3-> Constructs the default exception handler
\item<5-> Switches to the OCaml stack and calls \funcname{caml\_program}
\end{itemize}
\bigskip
\listocaml{../src/default/default.ml}{default/default.ml}
}%
\only<2>{
\listasm[firstline=22,lastline=41,highlightlines={25-27}]{src/ocaml/caml\_start\_program.s}{caml\_start\_program}
}%
\only<4>{
\mintasm[firstline=22,lastline=41,highlightlines={31-38}]{src/ocaml/caml\_start\_program.s}
\listasm[firstline=66,lastline=72]{src/ocaml/caml\_start\_program.s}{caml\_start\_program}
}%
\only<6>{
\listasm[firstline=22,lastline=41,highlightlines={39-41}]{src/ocaml/caml\_start\_program.s}{caml\_start\_program}
}%
\end{column}
\begin{column}{0.4\textwidth}
\centering
\only<1-2>{\providecommand\step{2}\input{diagrams/default/default.tex}}%
\only<3-5>{\providecommand\step{3}\input{diagrams/default/default.tex}}%
\onslide*<6->{\providecommand\step{4}\input{diagrams/default/default.tex}}%
\end{column}
\end{columns}
\end{frame}

\begin{frame}{Executing the OCaml program}{And raising an exception without handler}
\begin{columns}
\begin{column}{0.6\textwidth}
\only<1-3>{
\begin{itemize}
\item<1-> Execution continues with OCaml code
\begin{itemize}
\item \funcname{camlDefault\_\_main\_269}
\item \funcname{camlDefault\_\_entry}
\item \funcname{caml\_program}
\end{itemize}
\item<1-> \funcname{camlDefault\_\_main\_269} raises a \typename{Failure} exception
\item<1-> \typename{Failure} is caught by \funcname{caml\_start\_program}
\begin{itemize}
\item<1-> The handler set the 2nd LSB of the exception value to \funcarg{1}
\item<3-> And then forces \funcname{caml\_start\_program} to terminate
\end{itemize}
\end{itemize}
}
\bigskip
\only<1,3>{\listocaml[highlightlines=3]{../src/default/default.ml}{default/default.ml}}%
\only<2>{\listasm[firstline=66,lastline=72,highlightlines=69]{src/ocaml/caml\_start\_program.s}{caml\_start\_program}}%
\end{column}
\begin{column}{0.4\textwidth}
\centering
\providecommand\step{4}\input{diagrams/default/default.tex}
\end{column}
\end{columns}
\end{frame}
%\only<>{\listasm[firstline=47,lastline=65]{src/ocaml/caml\_start\_program.s}{caml\_start\_program}}


\begin{frame}{Back to C}{Clean up, complain and terminate}
\begin{columns}
\begin{column}{0.45\textwidth}
\begin{itemize}
\item Backtrace:
\begin{itemize}
\item \funcname{caml\_start\_program} $\rightarrow$ OCaml code
\item \funcname{caml\_startup\_common}
\item \funcname{caml\_startup\_exn}
\item \funcname{caml\_startup}
\item \funcname{caml\_main}
\item \funcname{main}
\end{itemize}
\smallskip
\item \funcname{caml\_startup} checks the return value of \funcname{caml\_startup\_exn}
\begin{itemize}
\item The return value has been specially marked by \funcname{caml\_start\_program} handler
\item Calls \funcname{caml\_fatal\_uncaught\_exception}
\end{itemize}
\item<2-> \funcname{caml\_fatal\_uncaught\_exception} either
\begin{itemize}
\item<2-> Calls the user custom uncaught exception handler, if set with \mintinline{ocaml}{Printexc.set_uncaught_exception_handler fn}\footnotemark
\item<2-> Or falls back to the default one
\item<2-> Which notifies about the uncaught exception
\end{itemize}
\item<4-> Terminates the program either with \funcname{abort} or with a non-zero exit code
\end{itemize}
\end{column}
\begin{column}{0.55\textwidth}
\only<1>{
\listc[firstline=84,lastline=89,highlightlines=88]{../ocaml/runtime/caml/mlvalues.h}{runtime/caml/mlvalues.h:84}
\listc[firstline=138,lastline=143,highlightlines={141-142}]{../ocaml/runtime/startup\_nat.c}{runtime/startup\_nat.c:138}
}%
\only<2>{
\listc[firstline=138,highlightlines={147,149}]{../ocaml/runtime/printexc.c}{runtime/printexc.c:138}
}%
\only<3>{
\listc[firstline=110,lastline=134,highlightlines=129]{../ocaml/runtime/printexc.c}{runtime/printexc.c:138}
}%
\only<4>{
\listc[firstline=138,highlightlines={152,154}]{../ocaml/runtime/printexc.c}{runtime/printexc.c:138}
}%
\end{column}
\end{columns}
\footnotetext[1]{\url{https://v2.ocaml.org/api/Printexc.html\#1\_Uncaughtexceptions}}
\end{frame}
}%
\onslide*<6->{\providecommand\step{4}%
% Runtime's default exception handler
%
\subsection*{Runtime's default exception handler}
\frameSubsection{pictures/The\_Architect.png}{}


\begin{frame}{Default exception handler}
\begin{columns}
\begin{column}{0.5\textwidth}
\begin{itemize}
\item \localname{domain\_state->exn\_handler} points to a trap located before \funcname{entry}!
\item What installed this very first trap there?
\end{itemize}
\bigskip
\listocaml{../src/default/default.ml}{default/default.ml}
\bigskip
\inputminted[fontsize=\footnotesize]{text}{../src/default/default.out}
\end{column}
\begin{column}{0.5\textwidth}
\centering
\only<1>{\providecommand\step{0}\input{diagrams/default/default.tex}}%
\only<2>{\providecommand\step{4}\input{diagrams/default/default.tex}}%
\end{column}
\end{columns}
\end{frame}


\begin{frame}{Startup}{How did we get there by the way?}
\begin{columns}
\begin{column}{0.45\textwidth}
\begin{itemize}
\item The entry point address of the binary is \funcname{main} (\funcname{\_start}) from the runtime
\item The program starts like a C program:
\begin{itemize}
\item \funcname{caml\_start\_program}
\item \funcname{caml\_startup\_common}
\item \funcname{caml\_startup\_exn}
\item \funcname{caml\_startup}
\item \funcname{caml\_main}
\item \funcname{main}
\end{itemize}
%\item \funcname{caml\_startup\_common}: initializes GC, domains \dots
% Also: codefrag, locale, custom opeartions, frame_descr, signals, stack
\item \funcname{caml\_start\_program} is the transition point between the C realm and the OCaml one
\end{itemize}
\bigskip
\listocaml{../src/default/default.ml}{default/default.ml}
\end{column}
\begin{column}{0.55\textwidth}
\listasm[lastline=24,highlightlines={22-24}]{src/ocaml/caml\_start\_program.s}{runtime/amd64.S:604}
\end{column}
\end{columns}
\end{frame}


\begin{frame}{Setup to jump into OCaml entry point}{\funcname{caml\_start\_program} initializes the main fiber}
\begin{columns}
\begin{column}{0.6\textwidth}
\only<1,3,5>{
\begin{itemize}
\item<1-> Stores information to allow switching from OCaml stack to the C stack
\item<3-> Constructs the default exception handler
\item<5-> Switches to the OCaml stack and calls \funcname{caml\_program}
\end{itemize}
\bigskip
\listocaml{../src/default/default.ml}{default/default.ml}
}%
\only<2>{
\listasm[firstline=22,lastline=41,highlightlines={25-27}]{src/ocaml/caml\_start\_program.s}{caml\_start\_program}
}%
\only<4>{
\mintasm[firstline=22,lastline=41,highlightlines={31-38}]{src/ocaml/caml\_start\_program.s}
\listasm[firstline=66,lastline=72]{src/ocaml/caml\_start\_program.s}{caml\_start\_program}
}%
\only<6>{
\listasm[firstline=22,lastline=41,highlightlines={39-41}]{src/ocaml/caml\_start\_program.s}{caml\_start\_program}
}%
\end{column}
\begin{column}{0.4\textwidth}
\centering
\only<1-2>{\providecommand\step{2}\input{diagrams/default/default.tex}}%
\only<3-5>{\providecommand\step{3}\input{diagrams/default/default.tex}}%
\onslide*<6->{\providecommand\step{4}\input{diagrams/default/default.tex}}%
\end{column}
\end{columns}
\end{frame}

\begin{frame}{Executing the OCaml program}{And raising an exception without handler}
\begin{columns}
\begin{column}{0.6\textwidth}
\only<1-3>{
\begin{itemize}
\item<1-> Execution continues with OCaml code
\begin{itemize}
\item \funcname{camlDefault\_\_main\_269}
\item \funcname{camlDefault\_\_entry}
\item \funcname{caml\_program}
\end{itemize}
\item<1-> \funcname{camlDefault\_\_main\_269} raises a \typename{Failure} exception
\item<1-> \typename{Failure} is caught by \funcname{caml\_start\_program}
\begin{itemize}
\item<1-> The handler set the 2nd LSB of the exception value to \funcarg{1}
\item<3-> And then forces \funcname{caml\_start\_program} to terminate
\end{itemize}
\end{itemize}
}
\bigskip
\only<1,3>{\listocaml[highlightlines=3]{../src/default/default.ml}{default/default.ml}}%
\only<2>{\listasm[firstline=66,lastline=72,highlightlines=69]{src/ocaml/caml\_start\_program.s}{caml\_start\_program}}%
\end{column}
\begin{column}{0.4\textwidth}
\centering
\providecommand\step{4}\input{diagrams/default/default.tex}
\end{column}
\end{columns}
\end{frame}
%\only<>{\listasm[firstline=47,lastline=65]{src/ocaml/caml\_start\_program.s}{caml\_start\_program}}


\begin{frame}{Back to C}{Clean up, complain and terminate}
\begin{columns}
\begin{column}{0.45\textwidth}
\begin{itemize}
\item Backtrace:
\begin{itemize}
\item \funcname{caml\_start\_program} $\rightarrow$ OCaml code
\item \funcname{caml\_startup\_common}
\item \funcname{caml\_startup\_exn}
\item \funcname{caml\_startup}
\item \funcname{caml\_main}
\item \funcname{main}
\end{itemize}
\smallskip
\item \funcname{caml\_startup} checks the return value of \funcname{caml\_startup\_exn}
\begin{itemize}
\item The return value has been specially marked by \funcname{caml\_start\_program} handler
\item Calls \funcname{caml\_fatal\_uncaught\_exception}
\end{itemize}
\item<2-> \funcname{caml\_fatal\_uncaught\_exception} either
\begin{itemize}
\item<2-> Calls the user custom uncaught exception handler, if set with \mintinline{ocaml}{Printexc.set_uncaught_exception_handler fn}\footnotemark
\item<2-> Or falls back to the default one
\item<2-> Which notifies about the uncaught exception
\end{itemize}
\item<4-> Terminates the program either with \funcname{abort} or with a non-zero exit code
\end{itemize}
\end{column}
\begin{column}{0.55\textwidth}
\only<1>{
\listc[firstline=84,lastline=89,highlightlines=88]{../ocaml/runtime/caml/mlvalues.h}{runtime/caml/mlvalues.h:84}
\listc[firstline=138,lastline=143,highlightlines={141-142}]{../ocaml/runtime/startup\_nat.c}{runtime/startup\_nat.c:138}
}%
\only<2>{
\listc[firstline=138,highlightlines={147,149}]{../ocaml/runtime/printexc.c}{runtime/printexc.c:138}
}%
\only<3>{
\listc[firstline=110,lastline=134,highlightlines=129]{../ocaml/runtime/printexc.c}{runtime/printexc.c:138}
}%
\only<4>{
\listc[firstline=138,highlightlines={152,154}]{../ocaml/runtime/printexc.c}{runtime/printexc.c:138}
}%
\end{column}
\end{columns}
\footnotetext[1]{\url{https://v2.ocaml.org/api/Printexc.html\#1\_Uncaughtexceptions}}
\end{frame}
}%
\end{column}
\end{columns}
\end{frame}

\begin{frame}{Executing the OCaml program}{And raising an exception without handler}
\begin{columns}
\begin{column}{0.6\textwidth}
\only<1-3>{
\begin{itemize}
\item<1-> Execution continues with OCaml code
\begin{itemize}
\item \funcname{camlDefault\_\_main\_269}
\item \funcname{camlDefault\_\_entry}
\item \funcname{caml\_program}
\end{itemize}
\item<1-> \funcname{camlDefault\_\_main\_269} raises a \typename{Failure} exception
\item<1-> \typename{Failure} is caught by \funcname{caml\_start\_program}
\begin{itemize}
\item<1-> The handler set the 2nd LSB of the exception value to \funcarg{1}
\item<3-> And then forces \funcname{caml\_start\_program} to terminate
\end{itemize}
\end{itemize}
}
\bigskip
\only<1,3>{\listocaml[highlightlines=3]{../src/default/default.ml}{default/default.ml}}%
\only<2>{\listasm[firstline=66,lastline=72,highlightlines=69]{src/ocaml/caml\_start\_program.s}{caml\_start\_program}}%
\end{column}
\begin{column}{0.4\textwidth}
\centering
\providecommand\step{4}%
% Runtime's default exception handler
%
\subsection*{Runtime's default exception handler}
\frameSubsection{pictures/The\_Architect.png}{}


\begin{frame}{Default exception handler}
\begin{columns}
\begin{column}{0.5\textwidth}
\begin{itemize}
\item \localname{domain\_state->exn\_handler} points to a trap located before \funcname{entry}!
\item What installed this very first trap there?
\end{itemize}
\bigskip
\listocaml{../src/default/default.ml}{default/default.ml}
\bigskip
\inputminted[fontsize=\footnotesize]{text}{../src/default/default.out}
\end{column}
\begin{column}{0.5\textwidth}
\centering
\only<1>{\providecommand\step{0}\input{diagrams/default/default.tex}}%
\only<2>{\providecommand\step{4}\input{diagrams/default/default.tex}}%
\end{column}
\end{columns}
\end{frame}


\begin{frame}{Startup}{How did we get there by the way?}
\begin{columns}
\begin{column}{0.45\textwidth}
\begin{itemize}
\item The entry point address of the binary is \funcname{main} (\funcname{\_start}) from the runtime
\item The program starts like a C program:
\begin{itemize}
\item \funcname{caml\_start\_program}
\item \funcname{caml\_startup\_common}
\item \funcname{caml\_startup\_exn}
\item \funcname{caml\_startup}
\item \funcname{caml\_main}
\item \funcname{main}
\end{itemize}
%\item \funcname{caml\_startup\_common}: initializes GC, domains \dots
% Also: codefrag, locale, custom opeartions, frame_descr, signals, stack
\item \funcname{caml\_start\_program} is the transition point between the C realm and the OCaml one
\end{itemize}
\bigskip
\listocaml{../src/default/default.ml}{default/default.ml}
\end{column}
\begin{column}{0.55\textwidth}
\listasm[lastline=24,highlightlines={22-24}]{src/ocaml/caml\_start\_program.s}{runtime/amd64.S:604}
\end{column}
\end{columns}
\end{frame}


\begin{frame}{Setup to jump into OCaml entry point}{\funcname{caml\_start\_program} initializes the main fiber}
\begin{columns}
\begin{column}{0.6\textwidth}
\only<1,3,5>{
\begin{itemize}
\item<1-> Stores information to allow switching from OCaml stack to the C stack
\item<3-> Constructs the default exception handler
\item<5-> Switches to the OCaml stack and calls \funcname{caml\_program}
\end{itemize}
\bigskip
\listocaml{../src/default/default.ml}{default/default.ml}
}%
\only<2>{
\listasm[firstline=22,lastline=41,highlightlines={25-27}]{src/ocaml/caml\_start\_program.s}{caml\_start\_program}
}%
\only<4>{
\mintasm[firstline=22,lastline=41,highlightlines={31-38}]{src/ocaml/caml\_start\_program.s}
\listasm[firstline=66,lastline=72]{src/ocaml/caml\_start\_program.s}{caml\_start\_program}
}%
\only<6>{
\listasm[firstline=22,lastline=41,highlightlines={39-41}]{src/ocaml/caml\_start\_program.s}{caml\_start\_program}
}%
\end{column}
\begin{column}{0.4\textwidth}
\centering
\only<1-2>{\providecommand\step{2}\input{diagrams/default/default.tex}}%
\only<3-5>{\providecommand\step{3}\input{diagrams/default/default.tex}}%
\onslide*<6->{\providecommand\step{4}\input{diagrams/default/default.tex}}%
\end{column}
\end{columns}
\end{frame}

\begin{frame}{Executing the OCaml program}{And raising an exception without handler}
\begin{columns}
\begin{column}{0.6\textwidth}
\only<1-3>{
\begin{itemize}
\item<1-> Execution continues with OCaml code
\begin{itemize}
\item \funcname{camlDefault\_\_main\_269}
\item \funcname{camlDefault\_\_entry}
\item \funcname{caml\_program}
\end{itemize}
\item<1-> \funcname{camlDefault\_\_main\_269} raises a \typename{Failure} exception
\item<1-> \typename{Failure} is caught by \funcname{caml\_start\_program}
\begin{itemize}
\item<1-> The handler set the 2nd LSB of the exception value to \funcarg{1}
\item<3-> And then forces \funcname{caml\_start\_program} to terminate
\end{itemize}
\end{itemize}
}
\bigskip
\only<1,3>{\listocaml[highlightlines=3]{../src/default/default.ml}{default/default.ml}}%
\only<2>{\listasm[firstline=66,lastline=72,highlightlines=69]{src/ocaml/caml\_start\_program.s}{caml\_start\_program}}%
\end{column}
\begin{column}{0.4\textwidth}
\centering
\providecommand\step{4}\input{diagrams/default/default.tex}
\end{column}
\end{columns}
\end{frame}
%\only<>{\listasm[firstline=47,lastline=65]{src/ocaml/caml\_start\_program.s}{caml\_start\_program}}


\begin{frame}{Back to C}{Clean up, complain and terminate}
\begin{columns}
\begin{column}{0.45\textwidth}
\begin{itemize}
\item Backtrace:
\begin{itemize}
\item \funcname{caml\_start\_program} $\rightarrow$ OCaml code
\item \funcname{caml\_startup\_common}
\item \funcname{caml\_startup\_exn}
\item \funcname{caml\_startup}
\item \funcname{caml\_main}
\item \funcname{main}
\end{itemize}
\smallskip
\item \funcname{caml\_startup} checks the return value of \funcname{caml\_startup\_exn}
\begin{itemize}
\item The return value has been specially marked by \funcname{caml\_start\_program} handler
\item Calls \funcname{caml\_fatal\_uncaught\_exception}
\end{itemize}
\item<2-> \funcname{caml\_fatal\_uncaught\_exception} either
\begin{itemize}
\item<2-> Calls the user custom uncaught exception handler, if set with \mintinline{ocaml}{Printexc.set_uncaught_exception_handler fn}\footnotemark
\item<2-> Or falls back to the default one
\item<2-> Which notifies about the uncaught exception
\end{itemize}
\item<4-> Terminates the program either with \funcname{abort} or with a non-zero exit code
\end{itemize}
\end{column}
\begin{column}{0.55\textwidth}
\only<1>{
\listc[firstline=84,lastline=89,highlightlines=88]{../ocaml/runtime/caml/mlvalues.h}{runtime/caml/mlvalues.h:84}
\listc[firstline=138,lastline=143,highlightlines={141-142}]{../ocaml/runtime/startup\_nat.c}{runtime/startup\_nat.c:138}
}%
\only<2>{
\listc[firstline=138,highlightlines={147,149}]{../ocaml/runtime/printexc.c}{runtime/printexc.c:138}
}%
\only<3>{
\listc[firstline=110,lastline=134,highlightlines=129]{../ocaml/runtime/printexc.c}{runtime/printexc.c:138}
}%
\only<4>{
\listc[firstline=138,highlightlines={152,154}]{../ocaml/runtime/printexc.c}{runtime/printexc.c:138}
}%
\end{column}
\end{columns}
\footnotetext[1]{\url{https://v2.ocaml.org/api/Printexc.html\#1\_Uncaughtexceptions}}
\end{frame}

\end{column}
\end{columns}
\end{frame}
%\only<>{\listasm[firstline=47,lastline=65]{src/ocaml/caml\_start\_program.s}{caml\_start\_program}}


\begin{frame}{Back to C}{Clean up, complain and terminate}
\begin{columns}
\begin{column}{0.45\textwidth}
\begin{itemize}
\item Backtrace:
\begin{itemize}
\item \funcname{caml\_start\_program} $\rightarrow$ OCaml code
\item \funcname{caml\_startup\_common}
\item \funcname{caml\_startup\_exn}
\item \funcname{caml\_startup}
\item \funcname{caml\_main}
\item \funcname{main}
\end{itemize}
\smallskip
\item \funcname{caml\_startup} checks the return value of \funcname{caml\_startup\_exn}
\begin{itemize}
\item The return value has been specially marked by \funcname{caml\_start\_program} handler
\item Calls \funcname{caml\_fatal\_uncaught\_exception}
\end{itemize}
\item<2-> \funcname{caml\_fatal\_uncaught\_exception} either
\begin{itemize}
\item<2-> Calls the user custom uncaught exception handler, if set with \mintinline{ocaml}{Printexc.set_uncaught_exception_handler fn}\footnotemark
\item<2-> Or falls back to the default one
\item<2-> Which notifies about the uncaught exception
\end{itemize}
\item<4-> Terminates the program either with \funcname{abort} or with a non-zero exit code
\end{itemize}
\end{column}
\begin{column}{0.55\textwidth}
\only<1>{
\listc[firstline=84,lastline=89,highlightlines=88]{../ocaml/runtime/caml/mlvalues.h}{runtime/caml/mlvalues.h:84}
\listc[firstline=138,lastline=143,highlightlines={141-142}]{../ocaml/runtime/startup\_nat.c}{runtime/startup\_nat.c:138}
}%
\only<2>{
\listc[firstline=138,highlightlines={147,149}]{../ocaml/runtime/printexc.c}{runtime/printexc.c:138}
}%
\only<3>{
\listc[firstline=110,lastline=134,highlightlines=129]{../ocaml/runtime/printexc.c}{runtime/printexc.c:138}
}%
\only<4>{
\listc[firstline=138,highlightlines={152,154}]{../ocaml/runtime/printexc.c}{runtime/printexc.c:138}
}%
\end{column}
\end{columns}
\footnotetext[1]{\url{https://v2.ocaml.org/api/Printexc.html\#1\_Uncaughtexceptions}}
\end{frame}
}%
\only<3-5>{\providecommand\step{3}%
% Runtime's default exception handler
%
\subsection*{Runtime's default exception handler}
\frameSubsection{pictures/The\_Architect.png}{}


\begin{frame}{Default exception handler}
\begin{columns}
\begin{column}{0.5\textwidth}
\begin{itemize}
\item \localname{domain\_state->exn\_handler} points to a trap located before \funcname{entry}!
\item What installed this very first trap there?
\end{itemize}
\bigskip
\listocaml{../src/default/default.ml}{default/default.ml}
\bigskip
\inputminted[fontsize=\footnotesize]{text}{../src/default/default.out}
\end{column}
\begin{column}{0.5\textwidth}
\centering
\only<1>{\providecommand\step{0}%
% Runtime's default exception handler
%
\subsection*{Runtime's default exception handler}
\frameSubsection{pictures/The\_Architect.png}{}


\begin{frame}{Default exception handler}
\begin{columns}
\begin{column}{0.5\textwidth}
\begin{itemize}
\item \localname{domain\_state->exn\_handler} points to a trap located before \funcname{entry}!
\item What installed this very first trap there?
\end{itemize}
\bigskip
\listocaml{../src/default/default.ml}{default/default.ml}
\bigskip
\inputminted[fontsize=\footnotesize]{text}{../src/default/default.out}
\end{column}
\begin{column}{0.5\textwidth}
\centering
\only<1>{\providecommand\step{0}\input{diagrams/default/default.tex}}%
\only<2>{\providecommand\step{4}\input{diagrams/default/default.tex}}%
\end{column}
\end{columns}
\end{frame}


\begin{frame}{Startup}{How did we get there by the way?}
\begin{columns}
\begin{column}{0.45\textwidth}
\begin{itemize}
\item The entry point address of the binary is \funcname{main} (\funcname{\_start}) from the runtime
\item The program starts like a C program:
\begin{itemize}
\item \funcname{caml\_start\_program}
\item \funcname{caml\_startup\_common}
\item \funcname{caml\_startup\_exn}
\item \funcname{caml\_startup}
\item \funcname{caml\_main}
\item \funcname{main}
\end{itemize}
%\item \funcname{caml\_startup\_common}: initializes GC, domains \dots
% Also: codefrag, locale, custom opeartions, frame_descr, signals, stack
\item \funcname{caml\_start\_program} is the transition point between the C realm and the OCaml one
\end{itemize}
\bigskip
\listocaml{../src/default/default.ml}{default/default.ml}
\end{column}
\begin{column}{0.55\textwidth}
\listasm[lastline=24,highlightlines={22-24}]{src/ocaml/caml\_start\_program.s}{runtime/amd64.S:604}
\end{column}
\end{columns}
\end{frame}


\begin{frame}{Setup to jump into OCaml entry point}{\funcname{caml\_start\_program} initializes the main fiber}
\begin{columns}
\begin{column}{0.6\textwidth}
\only<1,3,5>{
\begin{itemize}
\item<1-> Stores information to allow switching from OCaml stack to the C stack
\item<3-> Constructs the default exception handler
\item<5-> Switches to the OCaml stack and calls \funcname{caml\_program}
\end{itemize}
\bigskip
\listocaml{../src/default/default.ml}{default/default.ml}
}%
\only<2>{
\listasm[firstline=22,lastline=41,highlightlines={25-27}]{src/ocaml/caml\_start\_program.s}{caml\_start\_program}
}%
\only<4>{
\mintasm[firstline=22,lastline=41,highlightlines={31-38}]{src/ocaml/caml\_start\_program.s}
\listasm[firstline=66,lastline=72]{src/ocaml/caml\_start\_program.s}{caml\_start\_program}
}%
\only<6>{
\listasm[firstline=22,lastline=41,highlightlines={39-41}]{src/ocaml/caml\_start\_program.s}{caml\_start\_program}
}%
\end{column}
\begin{column}{0.4\textwidth}
\centering
\only<1-2>{\providecommand\step{2}\input{diagrams/default/default.tex}}%
\only<3-5>{\providecommand\step{3}\input{diagrams/default/default.tex}}%
\onslide*<6->{\providecommand\step{4}\input{diagrams/default/default.tex}}%
\end{column}
\end{columns}
\end{frame}

\begin{frame}{Executing the OCaml program}{And raising an exception without handler}
\begin{columns}
\begin{column}{0.6\textwidth}
\only<1-3>{
\begin{itemize}
\item<1-> Execution continues with OCaml code
\begin{itemize}
\item \funcname{camlDefault\_\_main\_269}
\item \funcname{camlDefault\_\_entry}
\item \funcname{caml\_program}
\end{itemize}
\item<1-> \funcname{camlDefault\_\_main\_269} raises a \typename{Failure} exception
\item<1-> \typename{Failure} is caught by \funcname{caml\_start\_program}
\begin{itemize}
\item<1-> The handler set the 2nd LSB of the exception value to \funcarg{1}
\item<3-> And then forces \funcname{caml\_start\_program} to terminate
\end{itemize}
\end{itemize}
}
\bigskip
\only<1,3>{\listocaml[highlightlines=3]{../src/default/default.ml}{default/default.ml}}%
\only<2>{\listasm[firstline=66,lastline=72,highlightlines=69]{src/ocaml/caml\_start\_program.s}{caml\_start\_program}}%
\end{column}
\begin{column}{0.4\textwidth}
\centering
\providecommand\step{4}\input{diagrams/default/default.tex}
\end{column}
\end{columns}
\end{frame}
%\only<>{\listasm[firstline=47,lastline=65]{src/ocaml/caml\_start\_program.s}{caml\_start\_program}}


\begin{frame}{Back to C}{Clean up, complain and terminate}
\begin{columns}
\begin{column}{0.45\textwidth}
\begin{itemize}
\item Backtrace:
\begin{itemize}
\item \funcname{caml\_start\_program} $\rightarrow$ OCaml code
\item \funcname{caml\_startup\_common}
\item \funcname{caml\_startup\_exn}
\item \funcname{caml\_startup}
\item \funcname{caml\_main}
\item \funcname{main}
\end{itemize}
\smallskip
\item \funcname{caml\_startup} checks the return value of \funcname{caml\_startup\_exn}
\begin{itemize}
\item The return value has been specially marked by \funcname{caml\_start\_program} handler
\item Calls \funcname{caml\_fatal\_uncaught\_exception}
\end{itemize}
\item<2-> \funcname{caml\_fatal\_uncaught\_exception} either
\begin{itemize}
\item<2-> Calls the user custom uncaught exception handler, if set with \mintinline{ocaml}{Printexc.set_uncaught_exception_handler fn}\footnotemark
\item<2-> Or falls back to the default one
\item<2-> Which notifies about the uncaught exception
\end{itemize}
\item<4-> Terminates the program either with \funcname{abort} or with a non-zero exit code
\end{itemize}
\end{column}
\begin{column}{0.55\textwidth}
\only<1>{
\listc[firstline=84,lastline=89,highlightlines=88]{../ocaml/runtime/caml/mlvalues.h}{runtime/caml/mlvalues.h:84}
\listc[firstline=138,lastline=143,highlightlines={141-142}]{../ocaml/runtime/startup\_nat.c}{runtime/startup\_nat.c:138}
}%
\only<2>{
\listc[firstline=138,highlightlines={147,149}]{../ocaml/runtime/printexc.c}{runtime/printexc.c:138}
}%
\only<3>{
\listc[firstline=110,lastline=134,highlightlines=129]{../ocaml/runtime/printexc.c}{runtime/printexc.c:138}
}%
\only<4>{
\listc[firstline=138,highlightlines={152,154}]{../ocaml/runtime/printexc.c}{runtime/printexc.c:138}
}%
\end{column}
\end{columns}
\footnotetext[1]{\url{https://v2.ocaml.org/api/Printexc.html\#1\_Uncaughtexceptions}}
\end{frame}
}%
\only<2>{\providecommand\step{4}%
% Runtime's default exception handler
%
\subsection*{Runtime's default exception handler}
\frameSubsection{pictures/The\_Architect.png}{}


\begin{frame}{Default exception handler}
\begin{columns}
\begin{column}{0.5\textwidth}
\begin{itemize}
\item \localname{domain\_state->exn\_handler} points to a trap located before \funcname{entry}!
\item What installed this very first trap there?
\end{itemize}
\bigskip
\listocaml{../src/default/default.ml}{default/default.ml}
\bigskip
\inputminted[fontsize=\footnotesize]{text}{../src/default/default.out}
\end{column}
\begin{column}{0.5\textwidth}
\centering
\only<1>{\providecommand\step{0}\input{diagrams/default/default.tex}}%
\only<2>{\providecommand\step{4}\input{diagrams/default/default.tex}}%
\end{column}
\end{columns}
\end{frame}


\begin{frame}{Startup}{How did we get there by the way?}
\begin{columns}
\begin{column}{0.45\textwidth}
\begin{itemize}
\item The entry point address of the binary is \funcname{main} (\funcname{\_start}) from the runtime
\item The program starts like a C program:
\begin{itemize}
\item \funcname{caml\_start\_program}
\item \funcname{caml\_startup\_common}
\item \funcname{caml\_startup\_exn}
\item \funcname{caml\_startup}
\item \funcname{caml\_main}
\item \funcname{main}
\end{itemize}
%\item \funcname{caml\_startup\_common}: initializes GC, domains \dots
% Also: codefrag, locale, custom opeartions, frame_descr, signals, stack
\item \funcname{caml\_start\_program} is the transition point between the C realm and the OCaml one
\end{itemize}
\bigskip
\listocaml{../src/default/default.ml}{default/default.ml}
\end{column}
\begin{column}{0.55\textwidth}
\listasm[lastline=24,highlightlines={22-24}]{src/ocaml/caml\_start\_program.s}{runtime/amd64.S:604}
\end{column}
\end{columns}
\end{frame}


\begin{frame}{Setup to jump into OCaml entry point}{\funcname{caml\_start\_program} initializes the main fiber}
\begin{columns}
\begin{column}{0.6\textwidth}
\only<1,3,5>{
\begin{itemize}
\item<1-> Stores information to allow switching from OCaml stack to the C stack
\item<3-> Constructs the default exception handler
\item<5-> Switches to the OCaml stack and calls \funcname{caml\_program}
\end{itemize}
\bigskip
\listocaml{../src/default/default.ml}{default/default.ml}
}%
\only<2>{
\listasm[firstline=22,lastline=41,highlightlines={25-27}]{src/ocaml/caml\_start\_program.s}{caml\_start\_program}
}%
\only<4>{
\mintasm[firstline=22,lastline=41,highlightlines={31-38}]{src/ocaml/caml\_start\_program.s}
\listasm[firstline=66,lastline=72]{src/ocaml/caml\_start\_program.s}{caml\_start\_program}
}%
\only<6>{
\listasm[firstline=22,lastline=41,highlightlines={39-41}]{src/ocaml/caml\_start\_program.s}{caml\_start\_program}
}%
\end{column}
\begin{column}{0.4\textwidth}
\centering
\only<1-2>{\providecommand\step{2}\input{diagrams/default/default.tex}}%
\only<3-5>{\providecommand\step{3}\input{diagrams/default/default.tex}}%
\onslide*<6->{\providecommand\step{4}\input{diagrams/default/default.tex}}%
\end{column}
\end{columns}
\end{frame}

\begin{frame}{Executing the OCaml program}{And raising an exception without handler}
\begin{columns}
\begin{column}{0.6\textwidth}
\only<1-3>{
\begin{itemize}
\item<1-> Execution continues with OCaml code
\begin{itemize}
\item \funcname{camlDefault\_\_main\_269}
\item \funcname{camlDefault\_\_entry}
\item \funcname{caml\_program}
\end{itemize}
\item<1-> \funcname{camlDefault\_\_main\_269} raises a \typename{Failure} exception
\item<1-> \typename{Failure} is caught by \funcname{caml\_start\_program}
\begin{itemize}
\item<1-> The handler set the 2nd LSB of the exception value to \funcarg{1}
\item<3-> And then forces \funcname{caml\_start\_program} to terminate
\end{itemize}
\end{itemize}
}
\bigskip
\only<1,3>{\listocaml[highlightlines=3]{../src/default/default.ml}{default/default.ml}}%
\only<2>{\listasm[firstline=66,lastline=72,highlightlines=69]{src/ocaml/caml\_start\_program.s}{caml\_start\_program}}%
\end{column}
\begin{column}{0.4\textwidth}
\centering
\providecommand\step{4}\input{diagrams/default/default.tex}
\end{column}
\end{columns}
\end{frame}
%\only<>{\listasm[firstline=47,lastline=65]{src/ocaml/caml\_start\_program.s}{caml\_start\_program}}


\begin{frame}{Back to C}{Clean up, complain and terminate}
\begin{columns}
\begin{column}{0.45\textwidth}
\begin{itemize}
\item Backtrace:
\begin{itemize}
\item \funcname{caml\_start\_program} $\rightarrow$ OCaml code
\item \funcname{caml\_startup\_common}
\item \funcname{caml\_startup\_exn}
\item \funcname{caml\_startup}
\item \funcname{caml\_main}
\item \funcname{main}
\end{itemize}
\smallskip
\item \funcname{caml\_startup} checks the return value of \funcname{caml\_startup\_exn}
\begin{itemize}
\item The return value has been specially marked by \funcname{caml\_start\_program} handler
\item Calls \funcname{caml\_fatal\_uncaught\_exception}
\end{itemize}
\item<2-> \funcname{caml\_fatal\_uncaught\_exception} either
\begin{itemize}
\item<2-> Calls the user custom uncaught exception handler, if set with \mintinline{ocaml}{Printexc.set_uncaught_exception_handler fn}\footnotemark
\item<2-> Or falls back to the default one
\item<2-> Which notifies about the uncaught exception
\end{itemize}
\item<4-> Terminates the program either with \funcname{abort} or with a non-zero exit code
\end{itemize}
\end{column}
\begin{column}{0.55\textwidth}
\only<1>{
\listc[firstline=84,lastline=89,highlightlines=88]{../ocaml/runtime/caml/mlvalues.h}{runtime/caml/mlvalues.h:84}
\listc[firstline=138,lastline=143,highlightlines={141-142}]{../ocaml/runtime/startup\_nat.c}{runtime/startup\_nat.c:138}
}%
\only<2>{
\listc[firstline=138,highlightlines={147,149}]{../ocaml/runtime/printexc.c}{runtime/printexc.c:138}
}%
\only<3>{
\listc[firstline=110,lastline=134,highlightlines=129]{../ocaml/runtime/printexc.c}{runtime/printexc.c:138}
}%
\only<4>{
\listc[firstline=138,highlightlines={152,154}]{../ocaml/runtime/printexc.c}{runtime/printexc.c:138}
}%
\end{column}
\end{columns}
\footnotetext[1]{\url{https://v2.ocaml.org/api/Printexc.html\#1\_Uncaughtexceptions}}
\end{frame}
}%
\end{column}
\end{columns}
\end{frame}


\begin{frame}{Startup}{How did we get there by the way?}
\begin{columns}
\begin{column}{0.45\textwidth}
\begin{itemize}
\item The entry point address of the binary is \funcname{main} (\funcname{\_start}) from the runtime
\item The program starts like a C program:
\begin{itemize}
\item \funcname{caml\_start\_program}
\item \funcname{caml\_startup\_common}
\item \funcname{caml\_startup\_exn}
\item \funcname{caml\_startup}
\item \funcname{caml\_main}
\item \funcname{main}
\end{itemize}
%\item \funcname{caml\_startup\_common}: initializes GC, domains \dots
% Also: codefrag, locale, custom opeartions, frame_descr, signals, stack
\item \funcname{caml\_start\_program} is the transition point between the C realm and the OCaml one
\end{itemize}
\bigskip
\listocaml{../src/default/default.ml}{default/default.ml}
\end{column}
\begin{column}{0.55\textwidth}
\listasm[lastline=24,highlightlines={22-24}]{src/ocaml/caml\_start\_program.s}{runtime/amd64.S:604}
\end{column}
\end{columns}
\end{frame}


\begin{frame}{Setup to jump into OCaml entry point}{\funcname{caml\_start\_program} initializes the main fiber}
\begin{columns}
\begin{column}{0.6\textwidth}
\only<1,3,5>{
\begin{itemize}
\item<1-> Stores information to allow switching from OCaml stack to the C stack
\item<3-> Constructs the default exception handler
\item<5-> Switches to the OCaml stack and calls \funcname{caml\_program}
\end{itemize}
\bigskip
\listocaml{../src/default/default.ml}{default/default.ml}
}%
\only<2>{
\listasm[firstline=22,lastline=41,highlightlines={25-27}]{src/ocaml/caml\_start\_program.s}{caml\_start\_program}
}%
\only<4>{
\mintasm[firstline=22,lastline=41,highlightlines={31-38}]{src/ocaml/caml\_start\_program.s}
\listasm[firstline=66,lastline=72]{src/ocaml/caml\_start\_program.s}{caml\_start\_program}
}%
\only<6>{
\listasm[firstline=22,lastline=41,highlightlines={39-41}]{src/ocaml/caml\_start\_program.s}{caml\_start\_program}
}%
\end{column}
\begin{column}{0.4\textwidth}
\centering
\only<1-2>{\providecommand\step{2}%
% Runtime's default exception handler
%
\subsection*{Runtime's default exception handler}
\frameSubsection{pictures/The\_Architect.png}{}


\begin{frame}{Default exception handler}
\begin{columns}
\begin{column}{0.5\textwidth}
\begin{itemize}
\item \localname{domain\_state->exn\_handler} points to a trap located before \funcname{entry}!
\item What installed this very first trap there?
\end{itemize}
\bigskip
\listocaml{../src/default/default.ml}{default/default.ml}
\bigskip
\inputminted[fontsize=\footnotesize]{text}{../src/default/default.out}
\end{column}
\begin{column}{0.5\textwidth}
\centering
\only<1>{\providecommand\step{0}\input{diagrams/default/default.tex}}%
\only<2>{\providecommand\step{4}\input{diagrams/default/default.tex}}%
\end{column}
\end{columns}
\end{frame}


\begin{frame}{Startup}{How did we get there by the way?}
\begin{columns}
\begin{column}{0.45\textwidth}
\begin{itemize}
\item The entry point address of the binary is \funcname{main} (\funcname{\_start}) from the runtime
\item The program starts like a C program:
\begin{itemize}
\item \funcname{caml\_start\_program}
\item \funcname{caml\_startup\_common}
\item \funcname{caml\_startup\_exn}
\item \funcname{caml\_startup}
\item \funcname{caml\_main}
\item \funcname{main}
\end{itemize}
%\item \funcname{caml\_startup\_common}: initializes GC, domains \dots
% Also: codefrag, locale, custom opeartions, frame_descr, signals, stack
\item \funcname{caml\_start\_program} is the transition point between the C realm and the OCaml one
\end{itemize}
\bigskip
\listocaml{../src/default/default.ml}{default/default.ml}
\end{column}
\begin{column}{0.55\textwidth}
\listasm[lastline=24,highlightlines={22-24}]{src/ocaml/caml\_start\_program.s}{runtime/amd64.S:604}
\end{column}
\end{columns}
\end{frame}


\begin{frame}{Setup to jump into OCaml entry point}{\funcname{caml\_start\_program} initializes the main fiber}
\begin{columns}
\begin{column}{0.6\textwidth}
\only<1,3,5>{
\begin{itemize}
\item<1-> Stores information to allow switching from OCaml stack to the C stack
\item<3-> Constructs the default exception handler
\item<5-> Switches to the OCaml stack and calls \funcname{caml\_program}
\end{itemize}
\bigskip
\listocaml{../src/default/default.ml}{default/default.ml}
}%
\only<2>{
\listasm[firstline=22,lastline=41,highlightlines={25-27}]{src/ocaml/caml\_start\_program.s}{caml\_start\_program}
}%
\only<4>{
\mintasm[firstline=22,lastline=41,highlightlines={31-38}]{src/ocaml/caml\_start\_program.s}
\listasm[firstline=66,lastline=72]{src/ocaml/caml\_start\_program.s}{caml\_start\_program}
}%
\only<6>{
\listasm[firstline=22,lastline=41,highlightlines={39-41}]{src/ocaml/caml\_start\_program.s}{caml\_start\_program}
}%
\end{column}
\begin{column}{0.4\textwidth}
\centering
\only<1-2>{\providecommand\step{2}\input{diagrams/default/default.tex}}%
\only<3-5>{\providecommand\step{3}\input{diagrams/default/default.tex}}%
\onslide*<6->{\providecommand\step{4}\input{diagrams/default/default.tex}}%
\end{column}
\end{columns}
\end{frame}

\begin{frame}{Executing the OCaml program}{And raising an exception without handler}
\begin{columns}
\begin{column}{0.6\textwidth}
\only<1-3>{
\begin{itemize}
\item<1-> Execution continues with OCaml code
\begin{itemize}
\item \funcname{camlDefault\_\_main\_269}
\item \funcname{camlDefault\_\_entry}
\item \funcname{caml\_program}
\end{itemize}
\item<1-> \funcname{camlDefault\_\_main\_269} raises a \typename{Failure} exception
\item<1-> \typename{Failure} is caught by \funcname{caml\_start\_program}
\begin{itemize}
\item<1-> The handler set the 2nd LSB of the exception value to \funcarg{1}
\item<3-> And then forces \funcname{caml\_start\_program} to terminate
\end{itemize}
\end{itemize}
}
\bigskip
\only<1,3>{\listocaml[highlightlines=3]{../src/default/default.ml}{default/default.ml}}%
\only<2>{\listasm[firstline=66,lastline=72,highlightlines=69]{src/ocaml/caml\_start\_program.s}{caml\_start\_program}}%
\end{column}
\begin{column}{0.4\textwidth}
\centering
\providecommand\step{4}\input{diagrams/default/default.tex}
\end{column}
\end{columns}
\end{frame}
%\only<>{\listasm[firstline=47,lastline=65]{src/ocaml/caml\_start\_program.s}{caml\_start\_program}}


\begin{frame}{Back to C}{Clean up, complain and terminate}
\begin{columns}
\begin{column}{0.45\textwidth}
\begin{itemize}
\item Backtrace:
\begin{itemize}
\item \funcname{caml\_start\_program} $\rightarrow$ OCaml code
\item \funcname{caml\_startup\_common}
\item \funcname{caml\_startup\_exn}
\item \funcname{caml\_startup}
\item \funcname{caml\_main}
\item \funcname{main}
\end{itemize}
\smallskip
\item \funcname{caml\_startup} checks the return value of \funcname{caml\_startup\_exn}
\begin{itemize}
\item The return value has been specially marked by \funcname{caml\_start\_program} handler
\item Calls \funcname{caml\_fatal\_uncaught\_exception}
\end{itemize}
\item<2-> \funcname{caml\_fatal\_uncaught\_exception} either
\begin{itemize}
\item<2-> Calls the user custom uncaught exception handler, if set with \mintinline{ocaml}{Printexc.set_uncaught_exception_handler fn}\footnotemark
\item<2-> Or falls back to the default one
\item<2-> Which notifies about the uncaught exception
\end{itemize}
\item<4-> Terminates the program either with \funcname{abort} or with a non-zero exit code
\end{itemize}
\end{column}
\begin{column}{0.55\textwidth}
\only<1>{
\listc[firstline=84,lastline=89,highlightlines=88]{../ocaml/runtime/caml/mlvalues.h}{runtime/caml/mlvalues.h:84}
\listc[firstline=138,lastline=143,highlightlines={141-142}]{../ocaml/runtime/startup\_nat.c}{runtime/startup\_nat.c:138}
}%
\only<2>{
\listc[firstline=138,highlightlines={147,149}]{../ocaml/runtime/printexc.c}{runtime/printexc.c:138}
}%
\only<3>{
\listc[firstline=110,lastline=134,highlightlines=129]{../ocaml/runtime/printexc.c}{runtime/printexc.c:138}
}%
\only<4>{
\listc[firstline=138,highlightlines={152,154}]{../ocaml/runtime/printexc.c}{runtime/printexc.c:138}
}%
\end{column}
\end{columns}
\footnotetext[1]{\url{https://v2.ocaml.org/api/Printexc.html\#1\_Uncaughtexceptions}}
\end{frame}
}%
\only<3-5>{\providecommand\step{3}%
% Runtime's default exception handler
%
\subsection*{Runtime's default exception handler}
\frameSubsection{pictures/The\_Architect.png}{}


\begin{frame}{Default exception handler}
\begin{columns}
\begin{column}{0.5\textwidth}
\begin{itemize}
\item \localname{domain\_state->exn\_handler} points to a trap located before \funcname{entry}!
\item What installed this very first trap there?
\end{itemize}
\bigskip
\listocaml{../src/default/default.ml}{default/default.ml}
\bigskip
\inputminted[fontsize=\footnotesize]{text}{../src/default/default.out}
\end{column}
\begin{column}{0.5\textwidth}
\centering
\only<1>{\providecommand\step{0}\input{diagrams/default/default.tex}}%
\only<2>{\providecommand\step{4}\input{diagrams/default/default.tex}}%
\end{column}
\end{columns}
\end{frame}


\begin{frame}{Startup}{How did we get there by the way?}
\begin{columns}
\begin{column}{0.45\textwidth}
\begin{itemize}
\item The entry point address of the binary is \funcname{main} (\funcname{\_start}) from the runtime
\item The program starts like a C program:
\begin{itemize}
\item \funcname{caml\_start\_program}
\item \funcname{caml\_startup\_common}
\item \funcname{caml\_startup\_exn}
\item \funcname{caml\_startup}
\item \funcname{caml\_main}
\item \funcname{main}
\end{itemize}
%\item \funcname{caml\_startup\_common}: initializes GC, domains \dots
% Also: codefrag, locale, custom opeartions, frame_descr, signals, stack
\item \funcname{caml\_start\_program} is the transition point between the C realm and the OCaml one
\end{itemize}
\bigskip
\listocaml{../src/default/default.ml}{default/default.ml}
\end{column}
\begin{column}{0.55\textwidth}
\listasm[lastline=24,highlightlines={22-24}]{src/ocaml/caml\_start\_program.s}{runtime/amd64.S:604}
\end{column}
\end{columns}
\end{frame}


\begin{frame}{Setup to jump into OCaml entry point}{\funcname{caml\_start\_program} initializes the main fiber}
\begin{columns}
\begin{column}{0.6\textwidth}
\only<1,3,5>{
\begin{itemize}
\item<1-> Stores information to allow switching from OCaml stack to the C stack
\item<3-> Constructs the default exception handler
\item<5-> Switches to the OCaml stack and calls \funcname{caml\_program}
\end{itemize}
\bigskip
\listocaml{../src/default/default.ml}{default/default.ml}
}%
\only<2>{
\listasm[firstline=22,lastline=41,highlightlines={25-27}]{src/ocaml/caml\_start\_program.s}{caml\_start\_program}
}%
\only<4>{
\mintasm[firstline=22,lastline=41,highlightlines={31-38}]{src/ocaml/caml\_start\_program.s}
\listasm[firstline=66,lastline=72]{src/ocaml/caml\_start\_program.s}{caml\_start\_program}
}%
\only<6>{
\listasm[firstline=22,lastline=41,highlightlines={39-41}]{src/ocaml/caml\_start\_program.s}{caml\_start\_program}
}%
\end{column}
\begin{column}{0.4\textwidth}
\centering
\only<1-2>{\providecommand\step{2}\input{diagrams/default/default.tex}}%
\only<3-5>{\providecommand\step{3}\input{diagrams/default/default.tex}}%
\onslide*<6->{\providecommand\step{4}\input{diagrams/default/default.tex}}%
\end{column}
\end{columns}
\end{frame}

\begin{frame}{Executing the OCaml program}{And raising an exception without handler}
\begin{columns}
\begin{column}{0.6\textwidth}
\only<1-3>{
\begin{itemize}
\item<1-> Execution continues with OCaml code
\begin{itemize}
\item \funcname{camlDefault\_\_main\_269}
\item \funcname{camlDefault\_\_entry}
\item \funcname{caml\_program}
\end{itemize}
\item<1-> \funcname{camlDefault\_\_main\_269} raises a \typename{Failure} exception
\item<1-> \typename{Failure} is caught by \funcname{caml\_start\_program}
\begin{itemize}
\item<1-> The handler set the 2nd LSB of the exception value to \funcarg{1}
\item<3-> And then forces \funcname{caml\_start\_program} to terminate
\end{itemize}
\end{itemize}
}
\bigskip
\only<1,3>{\listocaml[highlightlines=3]{../src/default/default.ml}{default/default.ml}}%
\only<2>{\listasm[firstline=66,lastline=72,highlightlines=69]{src/ocaml/caml\_start\_program.s}{caml\_start\_program}}%
\end{column}
\begin{column}{0.4\textwidth}
\centering
\providecommand\step{4}\input{diagrams/default/default.tex}
\end{column}
\end{columns}
\end{frame}
%\only<>{\listasm[firstline=47,lastline=65]{src/ocaml/caml\_start\_program.s}{caml\_start\_program}}


\begin{frame}{Back to C}{Clean up, complain and terminate}
\begin{columns}
\begin{column}{0.45\textwidth}
\begin{itemize}
\item Backtrace:
\begin{itemize}
\item \funcname{caml\_start\_program} $\rightarrow$ OCaml code
\item \funcname{caml\_startup\_common}
\item \funcname{caml\_startup\_exn}
\item \funcname{caml\_startup}
\item \funcname{caml\_main}
\item \funcname{main}
\end{itemize}
\smallskip
\item \funcname{caml\_startup} checks the return value of \funcname{caml\_startup\_exn}
\begin{itemize}
\item The return value has been specially marked by \funcname{caml\_start\_program} handler
\item Calls \funcname{caml\_fatal\_uncaught\_exception}
\end{itemize}
\item<2-> \funcname{caml\_fatal\_uncaught\_exception} either
\begin{itemize}
\item<2-> Calls the user custom uncaught exception handler, if set with \mintinline{ocaml}{Printexc.set_uncaught_exception_handler fn}\footnotemark
\item<2-> Or falls back to the default one
\item<2-> Which notifies about the uncaught exception
\end{itemize}
\item<4-> Terminates the program either with \funcname{abort} or with a non-zero exit code
\end{itemize}
\end{column}
\begin{column}{0.55\textwidth}
\only<1>{
\listc[firstline=84,lastline=89,highlightlines=88]{../ocaml/runtime/caml/mlvalues.h}{runtime/caml/mlvalues.h:84}
\listc[firstline=138,lastline=143,highlightlines={141-142}]{../ocaml/runtime/startup\_nat.c}{runtime/startup\_nat.c:138}
}%
\only<2>{
\listc[firstline=138,highlightlines={147,149}]{../ocaml/runtime/printexc.c}{runtime/printexc.c:138}
}%
\only<3>{
\listc[firstline=110,lastline=134,highlightlines=129]{../ocaml/runtime/printexc.c}{runtime/printexc.c:138}
}%
\only<4>{
\listc[firstline=138,highlightlines={152,154}]{../ocaml/runtime/printexc.c}{runtime/printexc.c:138}
}%
\end{column}
\end{columns}
\footnotetext[1]{\url{https://v2.ocaml.org/api/Printexc.html\#1\_Uncaughtexceptions}}
\end{frame}
}%
\onslide*<6->{\providecommand\step{4}%
% Runtime's default exception handler
%
\subsection*{Runtime's default exception handler}
\frameSubsection{pictures/The\_Architect.png}{}


\begin{frame}{Default exception handler}
\begin{columns}
\begin{column}{0.5\textwidth}
\begin{itemize}
\item \localname{domain\_state->exn\_handler} points to a trap located before \funcname{entry}!
\item What installed this very first trap there?
\end{itemize}
\bigskip
\listocaml{../src/default/default.ml}{default/default.ml}
\bigskip
\inputminted[fontsize=\footnotesize]{text}{../src/default/default.out}
\end{column}
\begin{column}{0.5\textwidth}
\centering
\only<1>{\providecommand\step{0}\input{diagrams/default/default.tex}}%
\only<2>{\providecommand\step{4}\input{diagrams/default/default.tex}}%
\end{column}
\end{columns}
\end{frame}


\begin{frame}{Startup}{How did we get there by the way?}
\begin{columns}
\begin{column}{0.45\textwidth}
\begin{itemize}
\item The entry point address of the binary is \funcname{main} (\funcname{\_start}) from the runtime
\item The program starts like a C program:
\begin{itemize}
\item \funcname{caml\_start\_program}
\item \funcname{caml\_startup\_common}
\item \funcname{caml\_startup\_exn}
\item \funcname{caml\_startup}
\item \funcname{caml\_main}
\item \funcname{main}
\end{itemize}
%\item \funcname{caml\_startup\_common}: initializes GC, domains \dots
% Also: codefrag, locale, custom opeartions, frame_descr, signals, stack
\item \funcname{caml\_start\_program} is the transition point between the C realm and the OCaml one
\end{itemize}
\bigskip
\listocaml{../src/default/default.ml}{default/default.ml}
\end{column}
\begin{column}{0.55\textwidth}
\listasm[lastline=24,highlightlines={22-24}]{src/ocaml/caml\_start\_program.s}{runtime/amd64.S:604}
\end{column}
\end{columns}
\end{frame}


\begin{frame}{Setup to jump into OCaml entry point}{\funcname{caml\_start\_program} initializes the main fiber}
\begin{columns}
\begin{column}{0.6\textwidth}
\only<1,3,5>{
\begin{itemize}
\item<1-> Stores information to allow switching from OCaml stack to the C stack
\item<3-> Constructs the default exception handler
\item<5-> Switches to the OCaml stack and calls \funcname{caml\_program}
\end{itemize}
\bigskip
\listocaml{../src/default/default.ml}{default/default.ml}
}%
\only<2>{
\listasm[firstline=22,lastline=41,highlightlines={25-27}]{src/ocaml/caml\_start\_program.s}{caml\_start\_program}
}%
\only<4>{
\mintasm[firstline=22,lastline=41,highlightlines={31-38}]{src/ocaml/caml\_start\_program.s}
\listasm[firstline=66,lastline=72]{src/ocaml/caml\_start\_program.s}{caml\_start\_program}
}%
\only<6>{
\listasm[firstline=22,lastline=41,highlightlines={39-41}]{src/ocaml/caml\_start\_program.s}{caml\_start\_program}
}%
\end{column}
\begin{column}{0.4\textwidth}
\centering
\only<1-2>{\providecommand\step{2}\input{diagrams/default/default.tex}}%
\only<3-5>{\providecommand\step{3}\input{diagrams/default/default.tex}}%
\onslide*<6->{\providecommand\step{4}\input{diagrams/default/default.tex}}%
\end{column}
\end{columns}
\end{frame}

\begin{frame}{Executing the OCaml program}{And raising an exception without handler}
\begin{columns}
\begin{column}{0.6\textwidth}
\only<1-3>{
\begin{itemize}
\item<1-> Execution continues with OCaml code
\begin{itemize}
\item \funcname{camlDefault\_\_main\_269}
\item \funcname{camlDefault\_\_entry}
\item \funcname{caml\_program}
\end{itemize}
\item<1-> \funcname{camlDefault\_\_main\_269} raises a \typename{Failure} exception
\item<1-> \typename{Failure} is caught by \funcname{caml\_start\_program}
\begin{itemize}
\item<1-> The handler set the 2nd LSB of the exception value to \funcarg{1}
\item<3-> And then forces \funcname{caml\_start\_program} to terminate
\end{itemize}
\end{itemize}
}
\bigskip
\only<1,3>{\listocaml[highlightlines=3]{../src/default/default.ml}{default/default.ml}}%
\only<2>{\listasm[firstline=66,lastline=72,highlightlines=69]{src/ocaml/caml\_start\_program.s}{caml\_start\_program}}%
\end{column}
\begin{column}{0.4\textwidth}
\centering
\providecommand\step{4}\input{diagrams/default/default.tex}
\end{column}
\end{columns}
\end{frame}
%\only<>{\listasm[firstline=47,lastline=65]{src/ocaml/caml\_start\_program.s}{caml\_start\_program}}


\begin{frame}{Back to C}{Clean up, complain and terminate}
\begin{columns}
\begin{column}{0.45\textwidth}
\begin{itemize}
\item Backtrace:
\begin{itemize}
\item \funcname{caml\_start\_program} $\rightarrow$ OCaml code
\item \funcname{caml\_startup\_common}
\item \funcname{caml\_startup\_exn}
\item \funcname{caml\_startup}
\item \funcname{caml\_main}
\item \funcname{main}
\end{itemize}
\smallskip
\item \funcname{caml\_startup} checks the return value of \funcname{caml\_startup\_exn}
\begin{itemize}
\item The return value has been specially marked by \funcname{caml\_start\_program} handler
\item Calls \funcname{caml\_fatal\_uncaught\_exception}
\end{itemize}
\item<2-> \funcname{caml\_fatal\_uncaught\_exception} either
\begin{itemize}
\item<2-> Calls the user custom uncaught exception handler, if set with \mintinline{ocaml}{Printexc.set_uncaught_exception_handler fn}\footnotemark
\item<2-> Or falls back to the default one
\item<2-> Which notifies about the uncaught exception
\end{itemize}
\item<4-> Terminates the program either with \funcname{abort} or with a non-zero exit code
\end{itemize}
\end{column}
\begin{column}{0.55\textwidth}
\only<1>{
\listc[firstline=84,lastline=89,highlightlines=88]{../ocaml/runtime/caml/mlvalues.h}{runtime/caml/mlvalues.h:84}
\listc[firstline=138,lastline=143,highlightlines={141-142}]{../ocaml/runtime/startup\_nat.c}{runtime/startup\_nat.c:138}
}%
\only<2>{
\listc[firstline=138,highlightlines={147,149}]{../ocaml/runtime/printexc.c}{runtime/printexc.c:138}
}%
\only<3>{
\listc[firstline=110,lastline=134,highlightlines=129]{../ocaml/runtime/printexc.c}{runtime/printexc.c:138}
}%
\only<4>{
\listc[firstline=138,highlightlines={152,154}]{../ocaml/runtime/printexc.c}{runtime/printexc.c:138}
}%
\end{column}
\end{columns}
\footnotetext[1]{\url{https://v2.ocaml.org/api/Printexc.html\#1\_Uncaughtexceptions}}
\end{frame}
}%
\end{column}
\end{columns}
\end{frame}

\begin{frame}{Executing the OCaml program}{And raising an exception without handler}
\begin{columns}
\begin{column}{0.6\textwidth}
\only<1-3>{
\begin{itemize}
\item<1-> Execution continues with OCaml code
\begin{itemize}
\item \funcname{camlDefault\_\_main\_269}
\item \funcname{camlDefault\_\_entry}
\item \funcname{caml\_program}
\end{itemize}
\item<1-> \funcname{camlDefault\_\_main\_269} raises a \typename{Failure} exception
\item<1-> \typename{Failure} is caught by \funcname{caml\_start\_program}
\begin{itemize}
\item<1-> The handler set the 2nd LSB of the exception value to \funcarg{1}
\item<3-> And then forces \funcname{caml\_start\_program} to terminate
\end{itemize}
\end{itemize}
}
\bigskip
\only<1,3>{\listocaml[highlightlines=3]{../src/default/default.ml}{default/default.ml}}%
\only<2>{\listasm[firstline=66,lastline=72,highlightlines=69]{src/ocaml/caml\_start\_program.s}{caml\_start\_program}}%
\end{column}
\begin{column}{0.4\textwidth}
\centering
\providecommand\step{4}%
% Runtime's default exception handler
%
\subsection*{Runtime's default exception handler}
\frameSubsection{pictures/The\_Architect.png}{}


\begin{frame}{Default exception handler}
\begin{columns}
\begin{column}{0.5\textwidth}
\begin{itemize}
\item \localname{domain\_state->exn\_handler} points to a trap located before \funcname{entry}!
\item What installed this very first trap there?
\end{itemize}
\bigskip
\listocaml{../src/default/default.ml}{default/default.ml}
\bigskip
\inputminted[fontsize=\footnotesize]{text}{../src/default/default.out}
\end{column}
\begin{column}{0.5\textwidth}
\centering
\only<1>{\providecommand\step{0}\input{diagrams/default/default.tex}}%
\only<2>{\providecommand\step{4}\input{diagrams/default/default.tex}}%
\end{column}
\end{columns}
\end{frame}


\begin{frame}{Startup}{How did we get there by the way?}
\begin{columns}
\begin{column}{0.45\textwidth}
\begin{itemize}
\item The entry point address of the binary is \funcname{main} (\funcname{\_start}) from the runtime
\item The program starts like a C program:
\begin{itemize}
\item \funcname{caml\_start\_program}
\item \funcname{caml\_startup\_common}
\item \funcname{caml\_startup\_exn}
\item \funcname{caml\_startup}
\item \funcname{caml\_main}
\item \funcname{main}
\end{itemize}
%\item \funcname{caml\_startup\_common}: initializes GC, domains \dots
% Also: codefrag, locale, custom opeartions, frame_descr, signals, stack
\item \funcname{caml\_start\_program} is the transition point between the C realm and the OCaml one
\end{itemize}
\bigskip
\listocaml{../src/default/default.ml}{default/default.ml}
\end{column}
\begin{column}{0.55\textwidth}
\listasm[lastline=24,highlightlines={22-24}]{src/ocaml/caml\_start\_program.s}{runtime/amd64.S:604}
\end{column}
\end{columns}
\end{frame}


\begin{frame}{Setup to jump into OCaml entry point}{\funcname{caml\_start\_program} initializes the main fiber}
\begin{columns}
\begin{column}{0.6\textwidth}
\only<1,3,5>{
\begin{itemize}
\item<1-> Stores information to allow switching from OCaml stack to the C stack
\item<3-> Constructs the default exception handler
\item<5-> Switches to the OCaml stack and calls \funcname{caml\_program}
\end{itemize}
\bigskip
\listocaml{../src/default/default.ml}{default/default.ml}
}%
\only<2>{
\listasm[firstline=22,lastline=41,highlightlines={25-27}]{src/ocaml/caml\_start\_program.s}{caml\_start\_program}
}%
\only<4>{
\mintasm[firstline=22,lastline=41,highlightlines={31-38}]{src/ocaml/caml\_start\_program.s}
\listasm[firstline=66,lastline=72]{src/ocaml/caml\_start\_program.s}{caml\_start\_program}
}%
\only<6>{
\listasm[firstline=22,lastline=41,highlightlines={39-41}]{src/ocaml/caml\_start\_program.s}{caml\_start\_program}
}%
\end{column}
\begin{column}{0.4\textwidth}
\centering
\only<1-2>{\providecommand\step{2}\input{diagrams/default/default.tex}}%
\only<3-5>{\providecommand\step{3}\input{diagrams/default/default.tex}}%
\onslide*<6->{\providecommand\step{4}\input{diagrams/default/default.tex}}%
\end{column}
\end{columns}
\end{frame}

\begin{frame}{Executing the OCaml program}{And raising an exception without handler}
\begin{columns}
\begin{column}{0.6\textwidth}
\only<1-3>{
\begin{itemize}
\item<1-> Execution continues with OCaml code
\begin{itemize}
\item \funcname{camlDefault\_\_main\_269}
\item \funcname{camlDefault\_\_entry}
\item \funcname{caml\_program}
\end{itemize}
\item<1-> \funcname{camlDefault\_\_main\_269} raises a \typename{Failure} exception
\item<1-> \typename{Failure} is caught by \funcname{caml\_start\_program}
\begin{itemize}
\item<1-> The handler set the 2nd LSB of the exception value to \funcarg{1}
\item<3-> And then forces \funcname{caml\_start\_program} to terminate
\end{itemize}
\end{itemize}
}
\bigskip
\only<1,3>{\listocaml[highlightlines=3]{../src/default/default.ml}{default/default.ml}}%
\only<2>{\listasm[firstline=66,lastline=72,highlightlines=69]{src/ocaml/caml\_start\_program.s}{caml\_start\_program}}%
\end{column}
\begin{column}{0.4\textwidth}
\centering
\providecommand\step{4}\input{diagrams/default/default.tex}
\end{column}
\end{columns}
\end{frame}
%\only<>{\listasm[firstline=47,lastline=65]{src/ocaml/caml\_start\_program.s}{caml\_start\_program}}


\begin{frame}{Back to C}{Clean up, complain and terminate}
\begin{columns}
\begin{column}{0.45\textwidth}
\begin{itemize}
\item Backtrace:
\begin{itemize}
\item \funcname{caml\_start\_program} $\rightarrow$ OCaml code
\item \funcname{caml\_startup\_common}
\item \funcname{caml\_startup\_exn}
\item \funcname{caml\_startup}
\item \funcname{caml\_main}
\item \funcname{main}
\end{itemize}
\smallskip
\item \funcname{caml\_startup} checks the return value of \funcname{caml\_startup\_exn}
\begin{itemize}
\item The return value has been specially marked by \funcname{caml\_start\_program} handler
\item Calls \funcname{caml\_fatal\_uncaught\_exception}
\end{itemize}
\item<2-> \funcname{caml\_fatal\_uncaught\_exception} either
\begin{itemize}
\item<2-> Calls the user custom uncaught exception handler, if set with \mintinline{ocaml}{Printexc.set_uncaught_exception_handler fn}\footnotemark
\item<2-> Or falls back to the default one
\item<2-> Which notifies about the uncaught exception
\end{itemize}
\item<4-> Terminates the program either with \funcname{abort} or with a non-zero exit code
\end{itemize}
\end{column}
\begin{column}{0.55\textwidth}
\only<1>{
\listc[firstline=84,lastline=89,highlightlines=88]{../ocaml/runtime/caml/mlvalues.h}{runtime/caml/mlvalues.h:84}
\listc[firstline=138,lastline=143,highlightlines={141-142}]{../ocaml/runtime/startup\_nat.c}{runtime/startup\_nat.c:138}
}%
\only<2>{
\listc[firstline=138,highlightlines={147,149}]{../ocaml/runtime/printexc.c}{runtime/printexc.c:138}
}%
\only<3>{
\listc[firstline=110,lastline=134,highlightlines=129]{../ocaml/runtime/printexc.c}{runtime/printexc.c:138}
}%
\only<4>{
\listc[firstline=138,highlightlines={152,154}]{../ocaml/runtime/printexc.c}{runtime/printexc.c:138}
}%
\end{column}
\end{columns}
\footnotetext[1]{\url{https://v2.ocaml.org/api/Printexc.html\#1\_Uncaughtexceptions}}
\end{frame}

\end{column}
\end{columns}
\end{frame}
%\only<>{\listasm[firstline=47,lastline=65]{src/ocaml/caml\_start\_program.s}{caml\_start\_program}}


\begin{frame}{Back to C}{Clean up, complain and terminate}
\begin{columns}
\begin{column}{0.45\textwidth}
\begin{itemize}
\item Backtrace:
\begin{itemize}
\item \funcname{caml\_start\_program} $\rightarrow$ OCaml code
\item \funcname{caml\_startup\_common}
\item \funcname{caml\_startup\_exn}
\item \funcname{caml\_startup}
\item \funcname{caml\_main}
\item \funcname{main}
\end{itemize}
\smallskip
\item \funcname{caml\_startup} checks the return value of \funcname{caml\_startup\_exn}
\begin{itemize}
\item The return value has been specially marked by \funcname{caml\_start\_program} handler
\item Calls \funcname{caml\_fatal\_uncaught\_exception}
\end{itemize}
\item<2-> \funcname{caml\_fatal\_uncaught\_exception} either
\begin{itemize}
\item<2-> Calls the user custom uncaught exception handler, if set with \mintinline{ocaml}{Printexc.set_uncaught_exception_handler fn}\footnotemark
\item<2-> Or falls back to the default one
\item<2-> Which notifies about the uncaught exception
\end{itemize}
\item<4-> Terminates the program either with \funcname{abort} or with a non-zero exit code
\end{itemize}
\end{column}
\begin{column}{0.55\textwidth}
\only<1>{
\listc[firstline=84,lastline=89,highlightlines=88]{../ocaml/runtime/caml/mlvalues.h}{runtime/caml/mlvalues.h:84}
\listc[firstline=138,lastline=143,highlightlines={141-142}]{../ocaml/runtime/startup\_nat.c}{runtime/startup\_nat.c:138}
}%
\only<2>{
\listc[firstline=138,highlightlines={147,149}]{../ocaml/runtime/printexc.c}{runtime/printexc.c:138}
}%
\only<3>{
\listc[firstline=110,lastline=134,highlightlines=129]{../ocaml/runtime/printexc.c}{runtime/printexc.c:138}
}%
\only<4>{
\listc[firstline=138,highlightlines={152,154}]{../ocaml/runtime/printexc.c}{runtime/printexc.c:138}
}%
\end{column}
\end{columns}
\footnotetext[1]{\url{https://v2.ocaml.org/api/Printexc.html\#1\_Uncaughtexceptions}}
\end{frame}
}%
\onslide*<6->{\providecommand\step{4}%
% Runtime's default exception handler
%
\subsection*{Runtime's default exception handler}
\frameSubsection{pictures/The\_Architect.png}{}


\begin{frame}{Default exception handler}
\begin{columns}
\begin{column}{0.5\textwidth}
\begin{itemize}
\item \localname{domain\_state->exn\_handler} points to a trap located before \funcname{entry}!
\item What installed this very first trap there?
\end{itemize}
\bigskip
\listocaml{../src/default/default.ml}{default/default.ml}
\bigskip
\inputminted[fontsize=\footnotesize]{text}{../src/default/default.out}
\end{column}
\begin{column}{0.5\textwidth}
\centering
\only<1>{\providecommand\step{0}%
% Runtime's default exception handler
%
\subsection*{Runtime's default exception handler}
\frameSubsection{pictures/The\_Architect.png}{}


\begin{frame}{Default exception handler}
\begin{columns}
\begin{column}{0.5\textwidth}
\begin{itemize}
\item \localname{domain\_state->exn\_handler} points to a trap located before \funcname{entry}!
\item What installed this very first trap there?
\end{itemize}
\bigskip
\listocaml{../src/default/default.ml}{default/default.ml}
\bigskip
\inputminted[fontsize=\footnotesize]{text}{../src/default/default.out}
\end{column}
\begin{column}{0.5\textwidth}
\centering
\only<1>{\providecommand\step{0}\input{diagrams/default/default.tex}}%
\only<2>{\providecommand\step{4}\input{diagrams/default/default.tex}}%
\end{column}
\end{columns}
\end{frame}


\begin{frame}{Startup}{How did we get there by the way?}
\begin{columns}
\begin{column}{0.45\textwidth}
\begin{itemize}
\item The entry point address of the binary is \funcname{main} (\funcname{\_start}) from the runtime
\item The program starts like a C program:
\begin{itemize}
\item \funcname{caml\_start\_program}
\item \funcname{caml\_startup\_common}
\item \funcname{caml\_startup\_exn}
\item \funcname{caml\_startup}
\item \funcname{caml\_main}
\item \funcname{main}
\end{itemize}
%\item \funcname{caml\_startup\_common}: initializes GC, domains \dots
% Also: codefrag, locale, custom opeartions, frame_descr, signals, stack
\item \funcname{caml\_start\_program} is the transition point between the C realm and the OCaml one
\end{itemize}
\bigskip
\listocaml{../src/default/default.ml}{default/default.ml}
\end{column}
\begin{column}{0.55\textwidth}
\listasm[lastline=24,highlightlines={22-24}]{src/ocaml/caml\_start\_program.s}{runtime/amd64.S:604}
\end{column}
\end{columns}
\end{frame}


\begin{frame}{Setup to jump into OCaml entry point}{\funcname{caml\_start\_program} initializes the main fiber}
\begin{columns}
\begin{column}{0.6\textwidth}
\only<1,3,5>{
\begin{itemize}
\item<1-> Stores information to allow switching from OCaml stack to the C stack
\item<3-> Constructs the default exception handler
\item<5-> Switches to the OCaml stack and calls \funcname{caml\_program}
\end{itemize}
\bigskip
\listocaml{../src/default/default.ml}{default/default.ml}
}%
\only<2>{
\listasm[firstline=22,lastline=41,highlightlines={25-27}]{src/ocaml/caml\_start\_program.s}{caml\_start\_program}
}%
\only<4>{
\mintasm[firstline=22,lastline=41,highlightlines={31-38}]{src/ocaml/caml\_start\_program.s}
\listasm[firstline=66,lastline=72]{src/ocaml/caml\_start\_program.s}{caml\_start\_program}
}%
\only<6>{
\listasm[firstline=22,lastline=41,highlightlines={39-41}]{src/ocaml/caml\_start\_program.s}{caml\_start\_program}
}%
\end{column}
\begin{column}{0.4\textwidth}
\centering
\only<1-2>{\providecommand\step{2}\input{diagrams/default/default.tex}}%
\only<3-5>{\providecommand\step{3}\input{diagrams/default/default.tex}}%
\onslide*<6->{\providecommand\step{4}\input{diagrams/default/default.tex}}%
\end{column}
\end{columns}
\end{frame}

\begin{frame}{Executing the OCaml program}{And raising an exception without handler}
\begin{columns}
\begin{column}{0.6\textwidth}
\only<1-3>{
\begin{itemize}
\item<1-> Execution continues with OCaml code
\begin{itemize}
\item \funcname{camlDefault\_\_main\_269}
\item \funcname{camlDefault\_\_entry}
\item \funcname{caml\_program}
\end{itemize}
\item<1-> \funcname{camlDefault\_\_main\_269} raises a \typename{Failure} exception
\item<1-> \typename{Failure} is caught by \funcname{caml\_start\_program}
\begin{itemize}
\item<1-> The handler set the 2nd LSB of the exception value to \funcarg{1}
\item<3-> And then forces \funcname{caml\_start\_program} to terminate
\end{itemize}
\end{itemize}
}
\bigskip
\only<1,3>{\listocaml[highlightlines=3]{../src/default/default.ml}{default/default.ml}}%
\only<2>{\listasm[firstline=66,lastline=72,highlightlines=69]{src/ocaml/caml\_start\_program.s}{caml\_start\_program}}%
\end{column}
\begin{column}{0.4\textwidth}
\centering
\providecommand\step{4}\input{diagrams/default/default.tex}
\end{column}
\end{columns}
\end{frame}
%\only<>{\listasm[firstline=47,lastline=65]{src/ocaml/caml\_start\_program.s}{caml\_start\_program}}


\begin{frame}{Back to C}{Clean up, complain and terminate}
\begin{columns}
\begin{column}{0.45\textwidth}
\begin{itemize}
\item Backtrace:
\begin{itemize}
\item \funcname{caml\_start\_program} $\rightarrow$ OCaml code
\item \funcname{caml\_startup\_common}
\item \funcname{caml\_startup\_exn}
\item \funcname{caml\_startup}
\item \funcname{caml\_main}
\item \funcname{main}
\end{itemize}
\smallskip
\item \funcname{caml\_startup} checks the return value of \funcname{caml\_startup\_exn}
\begin{itemize}
\item The return value has been specially marked by \funcname{caml\_start\_program} handler
\item Calls \funcname{caml\_fatal\_uncaught\_exception}
\end{itemize}
\item<2-> \funcname{caml\_fatal\_uncaught\_exception} either
\begin{itemize}
\item<2-> Calls the user custom uncaught exception handler, if set with \mintinline{ocaml}{Printexc.set_uncaught_exception_handler fn}\footnotemark
\item<2-> Or falls back to the default one
\item<2-> Which notifies about the uncaught exception
\end{itemize}
\item<4-> Terminates the program either with \funcname{abort} or with a non-zero exit code
\end{itemize}
\end{column}
\begin{column}{0.55\textwidth}
\only<1>{
\listc[firstline=84,lastline=89,highlightlines=88]{../ocaml/runtime/caml/mlvalues.h}{runtime/caml/mlvalues.h:84}
\listc[firstline=138,lastline=143,highlightlines={141-142}]{../ocaml/runtime/startup\_nat.c}{runtime/startup\_nat.c:138}
}%
\only<2>{
\listc[firstline=138,highlightlines={147,149}]{../ocaml/runtime/printexc.c}{runtime/printexc.c:138}
}%
\only<3>{
\listc[firstline=110,lastline=134,highlightlines=129]{../ocaml/runtime/printexc.c}{runtime/printexc.c:138}
}%
\only<4>{
\listc[firstline=138,highlightlines={152,154}]{../ocaml/runtime/printexc.c}{runtime/printexc.c:138}
}%
\end{column}
\end{columns}
\footnotetext[1]{\url{https://v2.ocaml.org/api/Printexc.html\#1\_Uncaughtexceptions}}
\end{frame}
}%
\only<2>{\providecommand\step{4}%
% Runtime's default exception handler
%
\subsection*{Runtime's default exception handler}
\frameSubsection{pictures/The\_Architect.png}{}


\begin{frame}{Default exception handler}
\begin{columns}
\begin{column}{0.5\textwidth}
\begin{itemize}
\item \localname{domain\_state->exn\_handler} points to a trap located before \funcname{entry}!
\item What installed this very first trap there?
\end{itemize}
\bigskip
\listocaml{../src/default/default.ml}{default/default.ml}
\bigskip
\inputminted[fontsize=\footnotesize]{text}{../src/default/default.out}
\end{column}
\begin{column}{0.5\textwidth}
\centering
\only<1>{\providecommand\step{0}\input{diagrams/default/default.tex}}%
\only<2>{\providecommand\step{4}\input{diagrams/default/default.tex}}%
\end{column}
\end{columns}
\end{frame}


\begin{frame}{Startup}{How did we get there by the way?}
\begin{columns}
\begin{column}{0.45\textwidth}
\begin{itemize}
\item The entry point address of the binary is \funcname{main} (\funcname{\_start}) from the runtime
\item The program starts like a C program:
\begin{itemize}
\item \funcname{caml\_start\_program}
\item \funcname{caml\_startup\_common}
\item \funcname{caml\_startup\_exn}
\item \funcname{caml\_startup}
\item \funcname{caml\_main}
\item \funcname{main}
\end{itemize}
%\item \funcname{caml\_startup\_common}: initializes GC, domains \dots
% Also: codefrag, locale, custom opeartions, frame_descr, signals, stack
\item \funcname{caml\_start\_program} is the transition point between the C realm and the OCaml one
\end{itemize}
\bigskip
\listocaml{../src/default/default.ml}{default/default.ml}
\end{column}
\begin{column}{0.55\textwidth}
\listasm[lastline=24,highlightlines={22-24}]{src/ocaml/caml\_start\_program.s}{runtime/amd64.S:604}
\end{column}
\end{columns}
\end{frame}


\begin{frame}{Setup to jump into OCaml entry point}{\funcname{caml\_start\_program} initializes the main fiber}
\begin{columns}
\begin{column}{0.6\textwidth}
\only<1,3,5>{
\begin{itemize}
\item<1-> Stores information to allow switching from OCaml stack to the C stack
\item<3-> Constructs the default exception handler
\item<5-> Switches to the OCaml stack and calls \funcname{caml\_program}
\end{itemize}
\bigskip
\listocaml{../src/default/default.ml}{default/default.ml}
}%
\only<2>{
\listasm[firstline=22,lastline=41,highlightlines={25-27}]{src/ocaml/caml\_start\_program.s}{caml\_start\_program}
}%
\only<4>{
\mintasm[firstline=22,lastline=41,highlightlines={31-38}]{src/ocaml/caml\_start\_program.s}
\listasm[firstline=66,lastline=72]{src/ocaml/caml\_start\_program.s}{caml\_start\_program}
}%
\only<6>{
\listasm[firstline=22,lastline=41,highlightlines={39-41}]{src/ocaml/caml\_start\_program.s}{caml\_start\_program}
}%
\end{column}
\begin{column}{0.4\textwidth}
\centering
\only<1-2>{\providecommand\step{2}\input{diagrams/default/default.tex}}%
\only<3-5>{\providecommand\step{3}\input{diagrams/default/default.tex}}%
\onslide*<6->{\providecommand\step{4}\input{diagrams/default/default.tex}}%
\end{column}
\end{columns}
\end{frame}

\begin{frame}{Executing the OCaml program}{And raising an exception without handler}
\begin{columns}
\begin{column}{0.6\textwidth}
\only<1-3>{
\begin{itemize}
\item<1-> Execution continues with OCaml code
\begin{itemize}
\item \funcname{camlDefault\_\_main\_269}
\item \funcname{camlDefault\_\_entry}
\item \funcname{caml\_program}
\end{itemize}
\item<1-> \funcname{camlDefault\_\_main\_269} raises a \typename{Failure} exception
\item<1-> \typename{Failure} is caught by \funcname{caml\_start\_program}
\begin{itemize}
\item<1-> The handler set the 2nd LSB of the exception value to \funcarg{1}
\item<3-> And then forces \funcname{caml\_start\_program} to terminate
\end{itemize}
\end{itemize}
}
\bigskip
\only<1,3>{\listocaml[highlightlines=3]{../src/default/default.ml}{default/default.ml}}%
\only<2>{\listasm[firstline=66,lastline=72,highlightlines=69]{src/ocaml/caml\_start\_program.s}{caml\_start\_program}}%
\end{column}
\begin{column}{0.4\textwidth}
\centering
\providecommand\step{4}\input{diagrams/default/default.tex}
\end{column}
\end{columns}
\end{frame}
%\only<>{\listasm[firstline=47,lastline=65]{src/ocaml/caml\_start\_program.s}{caml\_start\_program}}


\begin{frame}{Back to C}{Clean up, complain and terminate}
\begin{columns}
\begin{column}{0.45\textwidth}
\begin{itemize}
\item Backtrace:
\begin{itemize}
\item \funcname{caml\_start\_program} $\rightarrow$ OCaml code
\item \funcname{caml\_startup\_common}
\item \funcname{caml\_startup\_exn}
\item \funcname{caml\_startup}
\item \funcname{caml\_main}
\item \funcname{main}
\end{itemize}
\smallskip
\item \funcname{caml\_startup} checks the return value of \funcname{caml\_startup\_exn}
\begin{itemize}
\item The return value has been specially marked by \funcname{caml\_start\_program} handler
\item Calls \funcname{caml\_fatal\_uncaught\_exception}
\end{itemize}
\item<2-> \funcname{caml\_fatal\_uncaught\_exception} either
\begin{itemize}
\item<2-> Calls the user custom uncaught exception handler, if set with \mintinline{ocaml}{Printexc.set_uncaught_exception_handler fn}\footnotemark
\item<2-> Or falls back to the default one
\item<2-> Which notifies about the uncaught exception
\end{itemize}
\item<4-> Terminates the program either with \funcname{abort} or with a non-zero exit code
\end{itemize}
\end{column}
\begin{column}{0.55\textwidth}
\only<1>{
\listc[firstline=84,lastline=89,highlightlines=88]{../ocaml/runtime/caml/mlvalues.h}{runtime/caml/mlvalues.h:84}
\listc[firstline=138,lastline=143,highlightlines={141-142}]{../ocaml/runtime/startup\_nat.c}{runtime/startup\_nat.c:138}
}%
\only<2>{
\listc[firstline=138,highlightlines={147,149}]{../ocaml/runtime/printexc.c}{runtime/printexc.c:138}
}%
\only<3>{
\listc[firstline=110,lastline=134,highlightlines=129]{../ocaml/runtime/printexc.c}{runtime/printexc.c:138}
}%
\only<4>{
\listc[firstline=138,highlightlines={152,154}]{../ocaml/runtime/printexc.c}{runtime/printexc.c:138}
}%
\end{column}
\end{columns}
\footnotetext[1]{\url{https://v2.ocaml.org/api/Printexc.html\#1\_Uncaughtexceptions}}
\end{frame}
}%
\end{column}
\end{columns}
\end{frame}


\begin{frame}{Startup}{How did we get there by the way?}
\begin{columns}
\begin{column}{0.45\textwidth}
\begin{itemize}
\item The entry point address of the binary is \funcname{main} (\funcname{\_start}) from the runtime
\item The program starts like a C program:
\begin{itemize}
\item \funcname{caml\_start\_program}
\item \funcname{caml\_startup\_common}
\item \funcname{caml\_startup\_exn}
\item \funcname{caml\_startup}
\item \funcname{caml\_main}
\item \funcname{main}
\end{itemize}
%\item \funcname{caml\_startup\_common}: initializes GC, domains \dots
% Also: codefrag, locale, custom opeartions, frame_descr, signals, stack
\item \funcname{caml\_start\_program} is the transition point between the C realm and the OCaml one
\end{itemize}
\bigskip
\listocaml{../src/default/default.ml}{default/default.ml}
\end{column}
\begin{column}{0.55\textwidth}
\listasm[lastline=24,highlightlines={22-24}]{src/ocaml/caml\_start\_program.s}{runtime/amd64.S:604}
\end{column}
\end{columns}
\end{frame}


\begin{frame}{Setup to jump into OCaml entry point}{\funcname{caml\_start\_program} initializes the main fiber}
\begin{columns}
\begin{column}{0.6\textwidth}
\only<1,3,5>{
\begin{itemize}
\item<1-> Stores information to allow switching from OCaml stack to the C stack
\item<3-> Constructs the default exception handler
\item<5-> Switches to the OCaml stack and calls \funcname{caml\_program}
\end{itemize}
\bigskip
\listocaml{../src/default/default.ml}{default/default.ml}
}%
\only<2>{
\listasm[firstline=22,lastline=41,highlightlines={25-27}]{src/ocaml/caml\_start\_program.s}{caml\_start\_program}
}%
\only<4>{
\mintasm[firstline=22,lastline=41,highlightlines={31-38}]{src/ocaml/caml\_start\_program.s}
\listasm[firstline=66,lastline=72]{src/ocaml/caml\_start\_program.s}{caml\_start\_program}
}%
\only<6>{
\listasm[firstline=22,lastline=41,highlightlines={39-41}]{src/ocaml/caml\_start\_program.s}{caml\_start\_program}
}%
\end{column}
\begin{column}{0.4\textwidth}
\centering
\only<1-2>{\providecommand\step{2}%
% Runtime's default exception handler
%
\subsection*{Runtime's default exception handler}
\frameSubsection{pictures/The\_Architect.png}{}


\begin{frame}{Default exception handler}
\begin{columns}
\begin{column}{0.5\textwidth}
\begin{itemize}
\item \localname{domain\_state->exn\_handler} points to a trap located before \funcname{entry}!
\item What installed this very first trap there?
\end{itemize}
\bigskip
\listocaml{../src/default/default.ml}{default/default.ml}
\bigskip
\inputminted[fontsize=\footnotesize]{text}{../src/default/default.out}
\end{column}
\begin{column}{0.5\textwidth}
\centering
\only<1>{\providecommand\step{0}\input{diagrams/default/default.tex}}%
\only<2>{\providecommand\step{4}\input{diagrams/default/default.tex}}%
\end{column}
\end{columns}
\end{frame}


\begin{frame}{Startup}{How did we get there by the way?}
\begin{columns}
\begin{column}{0.45\textwidth}
\begin{itemize}
\item The entry point address of the binary is \funcname{main} (\funcname{\_start}) from the runtime
\item The program starts like a C program:
\begin{itemize}
\item \funcname{caml\_start\_program}
\item \funcname{caml\_startup\_common}
\item \funcname{caml\_startup\_exn}
\item \funcname{caml\_startup}
\item \funcname{caml\_main}
\item \funcname{main}
\end{itemize}
%\item \funcname{caml\_startup\_common}: initializes GC, domains \dots
% Also: codefrag, locale, custom opeartions, frame_descr, signals, stack
\item \funcname{caml\_start\_program} is the transition point between the C realm and the OCaml one
\end{itemize}
\bigskip
\listocaml{../src/default/default.ml}{default/default.ml}
\end{column}
\begin{column}{0.55\textwidth}
\listasm[lastline=24,highlightlines={22-24}]{src/ocaml/caml\_start\_program.s}{runtime/amd64.S:604}
\end{column}
\end{columns}
\end{frame}


\begin{frame}{Setup to jump into OCaml entry point}{\funcname{caml\_start\_program} initializes the main fiber}
\begin{columns}
\begin{column}{0.6\textwidth}
\only<1,3,5>{
\begin{itemize}
\item<1-> Stores information to allow switching from OCaml stack to the C stack
\item<3-> Constructs the default exception handler
\item<5-> Switches to the OCaml stack and calls \funcname{caml\_program}
\end{itemize}
\bigskip
\listocaml{../src/default/default.ml}{default/default.ml}
}%
\only<2>{
\listasm[firstline=22,lastline=41,highlightlines={25-27}]{src/ocaml/caml\_start\_program.s}{caml\_start\_program}
}%
\only<4>{
\mintasm[firstline=22,lastline=41,highlightlines={31-38}]{src/ocaml/caml\_start\_program.s}
\listasm[firstline=66,lastline=72]{src/ocaml/caml\_start\_program.s}{caml\_start\_program}
}%
\only<6>{
\listasm[firstline=22,lastline=41,highlightlines={39-41}]{src/ocaml/caml\_start\_program.s}{caml\_start\_program}
}%
\end{column}
\begin{column}{0.4\textwidth}
\centering
\only<1-2>{\providecommand\step{2}\input{diagrams/default/default.tex}}%
\only<3-5>{\providecommand\step{3}\input{diagrams/default/default.tex}}%
\onslide*<6->{\providecommand\step{4}\input{diagrams/default/default.tex}}%
\end{column}
\end{columns}
\end{frame}

\begin{frame}{Executing the OCaml program}{And raising an exception without handler}
\begin{columns}
\begin{column}{0.6\textwidth}
\only<1-3>{
\begin{itemize}
\item<1-> Execution continues with OCaml code
\begin{itemize}
\item \funcname{camlDefault\_\_main\_269}
\item \funcname{camlDefault\_\_entry}
\item \funcname{caml\_program}
\end{itemize}
\item<1-> \funcname{camlDefault\_\_main\_269} raises a \typename{Failure} exception
\item<1-> \typename{Failure} is caught by \funcname{caml\_start\_program}
\begin{itemize}
\item<1-> The handler set the 2nd LSB of the exception value to \funcarg{1}
\item<3-> And then forces \funcname{caml\_start\_program} to terminate
\end{itemize}
\end{itemize}
}
\bigskip
\only<1,3>{\listocaml[highlightlines=3]{../src/default/default.ml}{default/default.ml}}%
\only<2>{\listasm[firstline=66,lastline=72,highlightlines=69]{src/ocaml/caml\_start\_program.s}{caml\_start\_program}}%
\end{column}
\begin{column}{0.4\textwidth}
\centering
\providecommand\step{4}\input{diagrams/default/default.tex}
\end{column}
\end{columns}
\end{frame}
%\only<>{\listasm[firstline=47,lastline=65]{src/ocaml/caml\_start\_program.s}{caml\_start\_program}}


\begin{frame}{Back to C}{Clean up, complain and terminate}
\begin{columns}
\begin{column}{0.45\textwidth}
\begin{itemize}
\item Backtrace:
\begin{itemize}
\item \funcname{caml\_start\_program} $\rightarrow$ OCaml code
\item \funcname{caml\_startup\_common}
\item \funcname{caml\_startup\_exn}
\item \funcname{caml\_startup}
\item \funcname{caml\_main}
\item \funcname{main}
\end{itemize}
\smallskip
\item \funcname{caml\_startup} checks the return value of \funcname{caml\_startup\_exn}
\begin{itemize}
\item The return value has been specially marked by \funcname{caml\_start\_program} handler
\item Calls \funcname{caml\_fatal\_uncaught\_exception}
\end{itemize}
\item<2-> \funcname{caml\_fatal\_uncaught\_exception} either
\begin{itemize}
\item<2-> Calls the user custom uncaught exception handler, if set with \mintinline{ocaml}{Printexc.set_uncaught_exception_handler fn}\footnotemark
\item<2-> Or falls back to the default one
\item<2-> Which notifies about the uncaught exception
\end{itemize}
\item<4-> Terminates the program either with \funcname{abort} or with a non-zero exit code
\end{itemize}
\end{column}
\begin{column}{0.55\textwidth}
\only<1>{
\listc[firstline=84,lastline=89,highlightlines=88]{../ocaml/runtime/caml/mlvalues.h}{runtime/caml/mlvalues.h:84}
\listc[firstline=138,lastline=143,highlightlines={141-142}]{../ocaml/runtime/startup\_nat.c}{runtime/startup\_nat.c:138}
}%
\only<2>{
\listc[firstline=138,highlightlines={147,149}]{../ocaml/runtime/printexc.c}{runtime/printexc.c:138}
}%
\only<3>{
\listc[firstline=110,lastline=134,highlightlines=129]{../ocaml/runtime/printexc.c}{runtime/printexc.c:138}
}%
\only<4>{
\listc[firstline=138,highlightlines={152,154}]{../ocaml/runtime/printexc.c}{runtime/printexc.c:138}
}%
\end{column}
\end{columns}
\footnotetext[1]{\url{https://v2.ocaml.org/api/Printexc.html\#1\_Uncaughtexceptions}}
\end{frame}
}%
\only<3-5>{\providecommand\step{3}%
% Runtime's default exception handler
%
\subsection*{Runtime's default exception handler}
\frameSubsection{pictures/The\_Architect.png}{}


\begin{frame}{Default exception handler}
\begin{columns}
\begin{column}{0.5\textwidth}
\begin{itemize}
\item \localname{domain\_state->exn\_handler} points to a trap located before \funcname{entry}!
\item What installed this very first trap there?
\end{itemize}
\bigskip
\listocaml{../src/default/default.ml}{default/default.ml}
\bigskip
\inputminted[fontsize=\footnotesize]{text}{../src/default/default.out}
\end{column}
\begin{column}{0.5\textwidth}
\centering
\only<1>{\providecommand\step{0}\input{diagrams/default/default.tex}}%
\only<2>{\providecommand\step{4}\input{diagrams/default/default.tex}}%
\end{column}
\end{columns}
\end{frame}


\begin{frame}{Startup}{How did we get there by the way?}
\begin{columns}
\begin{column}{0.45\textwidth}
\begin{itemize}
\item The entry point address of the binary is \funcname{main} (\funcname{\_start}) from the runtime
\item The program starts like a C program:
\begin{itemize}
\item \funcname{caml\_start\_program}
\item \funcname{caml\_startup\_common}
\item \funcname{caml\_startup\_exn}
\item \funcname{caml\_startup}
\item \funcname{caml\_main}
\item \funcname{main}
\end{itemize}
%\item \funcname{caml\_startup\_common}: initializes GC, domains \dots
% Also: codefrag, locale, custom opeartions, frame_descr, signals, stack
\item \funcname{caml\_start\_program} is the transition point between the C realm and the OCaml one
\end{itemize}
\bigskip
\listocaml{../src/default/default.ml}{default/default.ml}
\end{column}
\begin{column}{0.55\textwidth}
\listasm[lastline=24,highlightlines={22-24}]{src/ocaml/caml\_start\_program.s}{runtime/amd64.S:604}
\end{column}
\end{columns}
\end{frame}


\begin{frame}{Setup to jump into OCaml entry point}{\funcname{caml\_start\_program} initializes the main fiber}
\begin{columns}
\begin{column}{0.6\textwidth}
\only<1,3,5>{
\begin{itemize}
\item<1-> Stores information to allow switching from OCaml stack to the C stack
\item<3-> Constructs the default exception handler
\item<5-> Switches to the OCaml stack and calls \funcname{caml\_program}
\end{itemize}
\bigskip
\listocaml{../src/default/default.ml}{default/default.ml}
}%
\only<2>{
\listasm[firstline=22,lastline=41,highlightlines={25-27}]{src/ocaml/caml\_start\_program.s}{caml\_start\_program}
}%
\only<4>{
\mintasm[firstline=22,lastline=41,highlightlines={31-38}]{src/ocaml/caml\_start\_program.s}
\listasm[firstline=66,lastline=72]{src/ocaml/caml\_start\_program.s}{caml\_start\_program}
}%
\only<6>{
\listasm[firstline=22,lastline=41,highlightlines={39-41}]{src/ocaml/caml\_start\_program.s}{caml\_start\_program}
}%
\end{column}
\begin{column}{0.4\textwidth}
\centering
\only<1-2>{\providecommand\step{2}\input{diagrams/default/default.tex}}%
\only<3-5>{\providecommand\step{3}\input{diagrams/default/default.tex}}%
\onslide*<6->{\providecommand\step{4}\input{diagrams/default/default.tex}}%
\end{column}
\end{columns}
\end{frame}

\begin{frame}{Executing the OCaml program}{And raising an exception without handler}
\begin{columns}
\begin{column}{0.6\textwidth}
\only<1-3>{
\begin{itemize}
\item<1-> Execution continues with OCaml code
\begin{itemize}
\item \funcname{camlDefault\_\_main\_269}
\item \funcname{camlDefault\_\_entry}
\item \funcname{caml\_program}
\end{itemize}
\item<1-> \funcname{camlDefault\_\_main\_269} raises a \typename{Failure} exception
\item<1-> \typename{Failure} is caught by \funcname{caml\_start\_program}
\begin{itemize}
\item<1-> The handler set the 2nd LSB of the exception value to \funcarg{1}
\item<3-> And then forces \funcname{caml\_start\_program} to terminate
\end{itemize}
\end{itemize}
}
\bigskip
\only<1,3>{\listocaml[highlightlines=3]{../src/default/default.ml}{default/default.ml}}%
\only<2>{\listasm[firstline=66,lastline=72,highlightlines=69]{src/ocaml/caml\_start\_program.s}{caml\_start\_program}}%
\end{column}
\begin{column}{0.4\textwidth}
\centering
\providecommand\step{4}\input{diagrams/default/default.tex}
\end{column}
\end{columns}
\end{frame}
%\only<>{\listasm[firstline=47,lastline=65]{src/ocaml/caml\_start\_program.s}{caml\_start\_program}}


\begin{frame}{Back to C}{Clean up, complain and terminate}
\begin{columns}
\begin{column}{0.45\textwidth}
\begin{itemize}
\item Backtrace:
\begin{itemize}
\item \funcname{caml\_start\_program} $\rightarrow$ OCaml code
\item \funcname{caml\_startup\_common}
\item \funcname{caml\_startup\_exn}
\item \funcname{caml\_startup}
\item \funcname{caml\_main}
\item \funcname{main}
\end{itemize}
\smallskip
\item \funcname{caml\_startup} checks the return value of \funcname{caml\_startup\_exn}
\begin{itemize}
\item The return value has been specially marked by \funcname{caml\_start\_program} handler
\item Calls \funcname{caml\_fatal\_uncaught\_exception}
\end{itemize}
\item<2-> \funcname{caml\_fatal\_uncaught\_exception} either
\begin{itemize}
\item<2-> Calls the user custom uncaught exception handler, if set with \mintinline{ocaml}{Printexc.set_uncaught_exception_handler fn}\footnotemark
\item<2-> Or falls back to the default one
\item<2-> Which notifies about the uncaught exception
\end{itemize}
\item<4-> Terminates the program either with \funcname{abort} or with a non-zero exit code
\end{itemize}
\end{column}
\begin{column}{0.55\textwidth}
\only<1>{
\listc[firstline=84,lastline=89,highlightlines=88]{../ocaml/runtime/caml/mlvalues.h}{runtime/caml/mlvalues.h:84}
\listc[firstline=138,lastline=143,highlightlines={141-142}]{../ocaml/runtime/startup\_nat.c}{runtime/startup\_nat.c:138}
}%
\only<2>{
\listc[firstline=138,highlightlines={147,149}]{../ocaml/runtime/printexc.c}{runtime/printexc.c:138}
}%
\only<3>{
\listc[firstline=110,lastline=134,highlightlines=129]{../ocaml/runtime/printexc.c}{runtime/printexc.c:138}
}%
\only<4>{
\listc[firstline=138,highlightlines={152,154}]{../ocaml/runtime/printexc.c}{runtime/printexc.c:138}
}%
\end{column}
\end{columns}
\footnotetext[1]{\url{https://v2.ocaml.org/api/Printexc.html\#1\_Uncaughtexceptions}}
\end{frame}
}%
\onslide*<6->{\providecommand\step{4}%
% Runtime's default exception handler
%
\subsection*{Runtime's default exception handler}
\frameSubsection{pictures/The\_Architect.png}{}


\begin{frame}{Default exception handler}
\begin{columns}
\begin{column}{0.5\textwidth}
\begin{itemize}
\item \localname{domain\_state->exn\_handler} points to a trap located before \funcname{entry}!
\item What installed this very first trap there?
\end{itemize}
\bigskip
\listocaml{../src/default/default.ml}{default/default.ml}
\bigskip
\inputminted[fontsize=\footnotesize]{text}{../src/default/default.out}
\end{column}
\begin{column}{0.5\textwidth}
\centering
\only<1>{\providecommand\step{0}\input{diagrams/default/default.tex}}%
\only<2>{\providecommand\step{4}\input{diagrams/default/default.tex}}%
\end{column}
\end{columns}
\end{frame}


\begin{frame}{Startup}{How did we get there by the way?}
\begin{columns}
\begin{column}{0.45\textwidth}
\begin{itemize}
\item The entry point address of the binary is \funcname{main} (\funcname{\_start}) from the runtime
\item The program starts like a C program:
\begin{itemize}
\item \funcname{caml\_start\_program}
\item \funcname{caml\_startup\_common}
\item \funcname{caml\_startup\_exn}
\item \funcname{caml\_startup}
\item \funcname{caml\_main}
\item \funcname{main}
\end{itemize}
%\item \funcname{caml\_startup\_common}: initializes GC, domains \dots
% Also: codefrag, locale, custom opeartions, frame_descr, signals, stack
\item \funcname{caml\_start\_program} is the transition point between the C realm and the OCaml one
\end{itemize}
\bigskip
\listocaml{../src/default/default.ml}{default/default.ml}
\end{column}
\begin{column}{0.55\textwidth}
\listasm[lastline=24,highlightlines={22-24}]{src/ocaml/caml\_start\_program.s}{runtime/amd64.S:604}
\end{column}
\end{columns}
\end{frame}


\begin{frame}{Setup to jump into OCaml entry point}{\funcname{caml\_start\_program} initializes the main fiber}
\begin{columns}
\begin{column}{0.6\textwidth}
\only<1,3,5>{
\begin{itemize}
\item<1-> Stores information to allow switching from OCaml stack to the C stack
\item<3-> Constructs the default exception handler
\item<5-> Switches to the OCaml stack and calls \funcname{caml\_program}
\end{itemize}
\bigskip
\listocaml{../src/default/default.ml}{default/default.ml}
}%
\only<2>{
\listasm[firstline=22,lastline=41,highlightlines={25-27}]{src/ocaml/caml\_start\_program.s}{caml\_start\_program}
}%
\only<4>{
\mintasm[firstline=22,lastline=41,highlightlines={31-38}]{src/ocaml/caml\_start\_program.s}
\listasm[firstline=66,lastline=72]{src/ocaml/caml\_start\_program.s}{caml\_start\_program}
}%
\only<6>{
\listasm[firstline=22,lastline=41,highlightlines={39-41}]{src/ocaml/caml\_start\_program.s}{caml\_start\_program}
}%
\end{column}
\begin{column}{0.4\textwidth}
\centering
\only<1-2>{\providecommand\step{2}\input{diagrams/default/default.tex}}%
\only<3-5>{\providecommand\step{3}\input{diagrams/default/default.tex}}%
\onslide*<6->{\providecommand\step{4}\input{diagrams/default/default.tex}}%
\end{column}
\end{columns}
\end{frame}

\begin{frame}{Executing the OCaml program}{And raising an exception without handler}
\begin{columns}
\begin{column}{0.6\textwidth}
\only<1-3>{
\begin{itemize}
\item<1-> Execution continues with OCaml code
\begin{itemize}
\item \funcname{camlDefault\_\_main\_269}
\item \funcname{camlDefault\_\_entry}
\item \funcname{caml\_program}
\end{itemize}
\item<1-> \funcname{camlDefault\_\_main\_269} raises a \typename{Failure} exception
\item<1-> \typename{Failure} is caught by \funcname{caml\_start\_program}
\begin{itemize}
\item<1-> The handler set the 2nd LSB of the exception value to \funcarg{1}
\item<3-> And then forces \funcname{caml\_start\_program} to terminate
\end{itemize}
\end{itemize}
}
\bigskip
\only<1,3>{\listocaml[highlightlines=3]{../src/default/default.ml}{default/default.ml}}%
\only<2>{\listasm[firstline=66,lastline=72,highlightlines=69]{src/ocaml/caml\_start\_program.s}{caml\_start\_program}}%
\end{column}
\begin{column}{0.4\textwidth}
\centering
\providecommand\step{4}\input{diagrams/default/default.tex}
\end{column}
\end{columns}
\end{frame}
%\only<>{\listasm[firstline=47,lastline=65]{src/ocaml/caml\_start\_program.s}{caml\_start\_program}}


\begin{frame}{Back to C}{Clean up, complain and terminate}
\begin{columns}
\begin{column}{0.45\textwidth}
\begin{itemize}
\item Backtrace:
\begin{itemize}
\item \funcname{caml\_start\_program} $\rightarrow$ OCaml code
\item \funcname{caml\_startup\_common}
\item \funcname{caml\_startup\_exn}
\item \funcname{caml\_startup}
\item \funcname{caml\_main}
\item \funcname{main}
\end{itemize}
\smallskip
\item \funcname{caml\_startup} checks the return value of \funcname{caml\_startup\_exn}
\begin{itemize}
\item The return value has been specially marked by \funcname{caml\_start\_program} handler
\item Calls \funcname{caml\_fatal\_uncaught\_exception}
\end{itemize}
\item<2-> \funcname{caml\_fatal\_uncaught\_exception} either
\begin{itemize}
\item<2-> Calls the user custom uncaught exception handler, if set with \mintinline{ocaml}{Printexc.set_uncaught_exception_handler fn}\footnotemark
\item<2-> Or falls back to the default one
\item<2-> Which notifies about the uncaught exception
\end{itemize}
\item<4-> Terminates the program either with \funcname{abort} or with a non-zero exit code
\end{itemize}
\end{column}
\begin{column}{0.55\textwidth}
\only<1>{
\listc[firstline=84,lastline=89,highlightlines=88]{../ocaml/runtime/caml/mlvalues.h}{runtime/caml/mlvalues.h:84}
\listc[firstline=138,lastline=143,highlightlines={141-142}]{../ocaml/runtime/startup\_nat.c}{runtime/startup\_nat.c:138}
}%
\only<2>{
\listc[firstline=138,highlightlines={147,149}]{../ocaml/runtime/printexc.c}{runtime/printexc.c:138}
}%
\only<3>{
\listc[firstline=110,lastline=134,highlightlines=129]{../ocaml/runtime/printexc.c}{runtime/printexc.c:138}
}%
\only<4>{
\listc[firstline=138,highlightlines={152,154}]{../ocaml/runtime/printexc.c}{runtime/printexc.c:138}
}%
\end{column}
\end{columns}
\footnotetext[1]{\url{https://v2.ocaml.org/api/Printexc.html\#1\_Uncaughtexceptions}}
\end{frame}
}%
\end{column}
\end{columns}
\end{frame}

\begin{frame}{Executing the OCaml program}{And raising an exception without handler}
\begin{columns}
\begin{column}{0.6\textwidth}
\only<1-3>{
\begin{itemize}
\item<1-> Execution continues with OCaml code
\begin{itemize}
\item \funcname{camlDefault\_\_main\_269}
\item \funcname{camlDefault\_\_entry}
\item \funcname{caml\_program}
\end{itemize}
\item<1-> \funcname{camlDefault\_\_main\_269} raises a \typename{Failure} exception
\item<1-> \typename{Failure} is caught by \funcname{caml\_start\_program}
\begin{itemize}
\item<1-> The handler set the 2nd LSB of the exception value to \funcarg{1}
\item<3-> And then forces \funcname{caml\_start\_program} to terminate
\end{itemize}
\end{itemize}
}
\bigskip
\only<1,3>{\listocaml[highlightlines=3]{../src/default/default.ml}{default/default.ml}}%
\only<2>{\listasm[firstline=66,lastline=72,highlightlines=69]{src/ocaml/caml\_start\_program.s}{caml\_start\_program}}%
\end{column}
\begin{column}{0.4\textwidth}
\centering
\providecommand\step{4}%
% Runtime's default exception handler
%
\subsection*{Runtime's default exception handler}
\frameSubsection{pictures/The\_Architect.png}{}


\begin{frame}{Default exception handler}
\begin{columns}
\begin{column}{0.5\textwidth}
\begin{itemize}
\item \localname{domain\_state->exn\_handler} points to a trap located before \funcname{entry}!
\item What installed this very first trap there?
\end{itemize}
\bigskip
\listocaml{../src/default/default.ml}{default/default.ml}
\bigskip
\inputminted[fontsize=\footnotesize]{text}{../src/default/default.out}
\end{column}
\begin{column}{0.5\textwidth}
\centering
\only<1>{\providecommand\step{0}\input{diagrams/default/default.tex}}%
\only<2>{\providecommand\step{4}\input{diagrams/default/default.tex}}%
\end{column}
\end{columns}
\end{frame}


\begin{frame}{Startup}{How did we get there by the way?}
\begin{columns}
\begin{column}{0.45\textwidth}
\begin{itemize}
\item The entry point address of the binary is \funcname{main} (\funcname{\_start}) from the runtime
\item The program starts like a C program:
\begin{itemize}
\item \funcname{caml\_start\_program}
\item \funcname{caml\_startup\_common}
\item \funcname{caml\_startup\_exn}
\item \funcname{caml\_startup}
\item \funcname{caml\_main}
\item \funcname{main}
\end{itemize}
%\item \funcname{caml\_startup\_common}: initializes GC, domains \dots
% Also: codefrag, locale, custom opeartions, frame_descr, signals, stack
\item \funcname{caml\_start\_program} is the transition point between the C realm and the OCaml one
\end{itemize}
\bigskip
\listocaml{../src/default/default.ml}{default/default.ml}
\end{column}
\begin{column}{0.55\textwidth}
\listasm[lastline=24,highlightlines={22-24}]{src/ocaml/caml\_start\_program.s}{runtime/amd64.S:604}
\end{column}
\end{columns}
\end{frame}


\begin{frame}{Setup to jump into OCaml entry point}{\funcname{caml\_start\_program} initializes the main fiber}
\begin{columns}
\begin{column}{0.6\textwidth}
\only<1,3,5>{
\begin{itemize}
\item<1-> Stores information to allow switching from OCaml stack to the C stack
\item<3-> Constructs the default exception handler
\item<5-> Switches to the OCaml stack and calls \funcname{caml\_program}
\end{itemize}
\bigskip
\listocaml{../src/default/default.ml}{default/default.ml}
}%
\only<2>{
\listasm[firstline=22,lastline=41,highlightlines={25-27}]{src/ocaml/caml\_start\_program.s}{caml\_start\_program}
}%
\only<4>{
\mintasm[firstline=22,lastline=41,highlightlines={31-38}]{src/ocaml/caml\_start\_program.s}
\listasm[firstline=66,lastline=72]{src/ocaml/caml\_start\_program.s}{caml\_start\_program}
}%
\only<6>{
\listasm[firstline=22,lastline=41,highlightlines={39-41}]{src/ocaml/caml\_start\_program.s}{caml\_start\_program}
}%
\end{column}
\begin{column}{0.4\textwidth}
\centering
\only<1-2>{\providecommand\step{2}\input{diagrams/default/default.tex}}%
\only<3-5>{\providecommand\step{3}\input{diagrams/default/default.tex}}%
\onslide*<6->{\providecommand\step{4}\input{diagrams/default/default.tex}}%
\end{column}
\end{columns}
\end{frame}

\begin{frame}{Executing the OCaml program}{And raising an exception without handler}
\begin{columns}
\begin{column}{0.6\textwidth}
\only<1-3>{
\begin{itemize}
\item<1-> Execution continues with OCaml code
\begin{itemize}
\item \funcname{camlDefault\_\_main\_269}
\item \funcname{camlDefault\_\_entry}
\item \funcname{caml\_program}
\end{itemize}
\item<1-> \funcname{camlDefault\_\_main\_269} raises a \typename{Failure} exception
\item<1-> \typename{Failure} is caught by \funcname{caml\_start\_program}
\begin{itemize}
\item<1-> The handler set the 2nd LSB of the exception value to \funcarg{1}
\item<3-> And then forces \funcname{caml\_start\_program} to terminate
\end{itemize}
\end{itemize}
}
\bigskip
\only<1,3>{\listocaml[highlightlines=3]{../src/default/default.ml}{default/default.ml}}%
\only<2>{\listasm[firstline=66,lastline=72,highlightlines=69]{src/ocaml/caml\_start\_program.s}{caml\_start\_program}}%
\end{column}
\begin{column}{0.4\textwidth}
\centering
\providecommand\step{4}\input{diagrams/default/default.tex}
\end{column}
\end{columns}
\end{frame}
%\only<>{\listasm[firstline=47,lastline=65]{src/ocaml/caml\_start\_program.s}{caml\_start\_program}}


\begin{frame}{Back to C}{Clean up, complain and terminate}
\begin{columns}
\begin{column}{0.45\textwidth}
\begin{itemize}
\item Backtrace:
\begin{itemize}
\item \funcname{caml\_start\_program} $\rightarrow$ OCaml code
\item \funcname{caml\_startup\_common}
\item \funcname{caml\_startup\_exn}
\item \funcname{caml\_startup}
\item \funcname{caml\_main}
\item \funcname{main}
\end{itemize}
\smallskip
\item \funcname{caml\_startup} checks the return value of \funcname{caml\_startup\_exn}
\begin{itemize}
\item The return value has been specially marked by \funcname{caml\_start\_program} handler
\item Calls \funcname{caml\_fatal\_uncaught\_exception}
\end{itemize}
\item<2-> \funcname{caml\_fatal\_uncaught\_exception} either
\begin{itemize}
\item<2-> Calls the user custom uncaught exception handler, if set with \mintinline{ocaml}{Printexc.set_uncaught_exception_handler fn}\footnotemark
\item<2-> Or falls back to the default one
\item<2-> Which notifies about the uncaught exception
\end{itemize}
\item<4-> Terminates the program either with \funcname{abort} or with a non-zero exit code
\end{itemize}
\end{column}
\begin{column}{0.55\textwidth}
\only<1>{
\listc[firstline=84,lastline=89,highlightlines=88]{../ocaml/runtime/caml/mlvalues.h}{runtime/caml/mlvalues.h:84}
\listc[firstline=138,lastline=143,highlightlines={141-142}]{../ocaml/runtime/startup\_nat.c}{runtime/startup\_nat.c:138}
}%
\only<2>{
\listc[firstline=138,highlightlines={147,149}]{../ocaml/runtime/printexc.c}{runtime/printexc.c:138}
}%
\only<3>{
\listc[firstline=110,lastline=134,highlightlines=129]{../ocaml/runtime/printexc.c}{runtime/printexc.c:138}
}%
\only<4>{
\listc[firstline=138,highlightlines={152,154}]{../ocaml/runtime/printexc.c}{runtime/printexc.c:138}
}%
\end{column}
\end{columns}
\footnotetext[1]{\url{https://v2.ocaml.org/api/Printexc.html\#1\_Uncaughtexceptions}}
\end{frame}

\end{column}
\end{columns}
\end{frame}
%\only<>{\listasm[firstline=47,lastline=65]{src/ocaml/caml\_start\_program.s}{caml\_start\_program}}


\begin{frame}{Back to C}{Clean up, complain and terminate}
\begin{columns}
\begin{column}{0.45\textwidth}
\begin{itemize}
\item Backtrace:
\begin{itemize}
\item \funcname{caml\_start\_program} $\rightarrow$ OCaml code
\item \funcname{caml\_startup\_common}
\item \funcname{caml\_startup\_exn}
\item \funcname{caml\_startup}
\item \funcname{caml\_main}
\item \funcname{main}
\end{itemize}
\smallskip
\item \funcname{caml\_startup} checks the return value of \funcname{caml\_startup\_exn}
\begin{itemize}
\item The return value has been specially marked by \funcname{caml\_start\_program} handler
\item Calls \funcname{caml\_fatal\_uncaught\_exception}
\end{itemize}
\item<2-> \funcname{caml\_fatal\_uncaught\_exception} either
\begin{itemize}
\item<2-> Calls the user custom uncaught exception handler, if set with \mintinline{ocaml}{Printexc.set_uncaught_exception_handler fn}\footnotemark
\item<2-> Or falls back to the default one
\item<2-> Which notifies about the uncaught exception
\end{itemize}
\item<4-> Terminates the program either with \funcname{abort} or with a non-zero exit code
\end{itemize}
\end{column}
\begin{column}{0.55\textwidth}
\only<1>{
\listc[firstline=84,lastline=89,highlightlines=88]{../ocaml/runtime/caml/mlvalues.h}{runtime/caml/mlvalues.h:84}
\listc[firstline=138,lastline=143,highlightlines={141-142}]{../ocaml/runtime/startup\_nat.c}{runtime/startup\_nat.c:138}
}%
\only<2>{
\listc[firstline=138,highlightlines={147,149}]{../ocaml/runtime/printexc.c}{runtime/printexc.c:138}
}%
\only<3>{
\listc[firstline=110,lastline=134,highlightlines=129]{../ocaml/runtime/printexc.c}{runtime/printexc.c:138}
}%
\only<4>{
\listc[firstline=138,highlightlines={152,154}]{../ocaml/runtime/printexc.c}{runtime/printexc.c:138}
}%
\end{column}
\end{columns}
\footnotetext[1]{\url{https://v2.ocaml.org/api/Printexc.html\#1\_Uncaughtexceptions}}
\end{frame}
}%
\end{column}
\end{columns}
\end{frame}

\begin{frame}{Executing the OCaml program}{And raising an exception without handler}
\begin{columns}
\begin{column}{0.6\textwidth}
\only<1-3>{
\begin{itemize}
\item<1-> Execution continues with OCaml code
\begin{itemize}
\item \funcname{camlDefault\_\_main\_269}
\item \funcname{camlDefault\_\_entry}
\item \funcname{caml\_program}
\end{itemize}
\item<1-> \funcname{camlDefault\_\_main\_269} raises a \typename{Failure} exception
\item<1-> \typename{Failure} is caught by \funcname{caml\_start\_program}
\begin{itemize}
\item<1-> The handler set the 2nd LSB of the exception value to \funcarg{1}
\item<3-> And then forces \funcname{caml\_start\_program} to terminate
\end{itemize}
\end{itemize}
}
\bigskip
\only<1,3>{\listocaml[highlightlines=3]{../src/default/default.ml}{default/default.ml}}%
\only<2>{\listasm[firstline=66,lastline=72,highlightlines=69]{src/ocaml/caml\_start\_program.s}{caml\_start\_program}}%
\end{column}
\begin{column}{0.4\textwidth}
\centering
\providecommand\step{4}%
% Runtime's default exception handler
%
\subsection*{Runtime's default exception handler}
\frameSubsection{pictures/The\_Architect.png}{}


\begin{frame}{Default exception handler}
\begin{columns}
\begin{column}{0.5\textwidth}
\begin{itemize}
\item \localname{domain\_state->exn\_handler} points to a trap located before \funcname{entry}!
\item What installed this very first trap there?
\end{itemize}
\bigskip
\listocaml{../src/default/default.ml}{default/default.ml}
\bigskip
\inputminted[fontsize=\footnotesize]{text}{../src/default/default.out}
\end{column}
\begin{column}{0.5\textwidth}
\centering
\only<1>{\providecommand\step{0}%
% Runtime's default exception handler
%
\subsection*{Runtime's default exception handler}
\frameSubsection{pictures/The\_Architect.png}{}


\begin{frame}{Default exception handler}
\begin{columns}
\begin{column}{0.5\textwidth}
\begin{itemize}
\item \localname{domain\_state->exn\_handler} points to a trap located before \funcname{entry}!
\item What installed this very first trap there?
\end{itemize}
\bigskip
\listocaml{../src/default/default.ml}{default/default.ml}
\bigskip
\inputminted[fontsize=\footnotesize]{text}{../src/default/default.out}
\end{column}
\begin{column}{0.5\textwidth}
\centering
\only<1>{\providecommand\step{0}\input{diagrams/default/default.tex}}%
\only<2>{\providecommand\step{4}\input{diagrams/default/default.tex}}%
\end{column}
\end{columns}
\end{frame}


\begin{frame}{Startup}{How did we get there by the way?}
\begin{columns}
\begin{column}{0.45\textwidth}
\begin{itemize}
\item The entry point address of the binary is \funcname{main} (\funcname{\_start}) from the runtime
\item The program starts like a C program:
\begin{itemize}
\item \funcname{caml\_start\_program}
\item \funcname{caml\_startup\_common}
\item \funcname{caml\_startup\_exn}
\item \funcname{caml\_startup}
\item \funcname{caml\_main}
\item \funcname{main}
\end{itemize}
%\item \funcname{caml\_startup\_common}: initializes GC, domains \dots
% Also: codefrag, locale, custom opeartions, frame_descr, signals, stack
\item \funcname{caml\_start\_program} is the transition point between the C realm and the OCaml one
\end{itemize}
\bigskip
\listocaml{../src/default/default.ml}{default/default.ml}
\end{column}
\begin{column}{0.55\textwidth}
\listasm[lastline=24,highlightlines={22-24}]{src/ocaml/caml\_start\_program.s}{runtime/amd64.S:604}
\end{column}
\end{columns}
\end{frame}


\begin{frame}{Setup to jump into OCaml entry point}{\funcname{caml\_start\_program} initializes the main fiber}
\begin{columns}
\begin{column}{0.6\textwidth}
\only<1,3,5>{
\begin{itemize}
\item<1-> Stores information to allow switching from OCaml stack to the C stack
\item<3-> Constructs the default exception handler
\item<5-> Switches to the OCaml stack and calls \funcname{caml\_program}
\end{itemize}
\bigskip
\listocaml{../src/default/default.ml}{default/default.ml}
}%
\only<2>{
\listasm[firstline=22,lastline=41,highlightlines={25-27}]{src/ocaml/caml\_start\_program.s}{caml\_start\_program}
}%
\only<4>{
\mintasm[firstline=22,lastline=41,highlightlines={31-38}]{src/ocaml/caml\_start\_program.s}
\listasm[firstline=66,lastline=72]{src/ocaml/caml\_start\_program.s}{caml\_start\_program}
}%
\only<6>{
\listasm[firstline=22,lastline=41,highlightlines={39-41}]{src/ocaml/caml\_start\_program.s}{caml\_start\_program}
}%
\end{column}
\begin{column}{0.4\textwidth}
\centering
\only<1-2>{\providecommand\step{2}\input{diagrams/default/default.tex}}%
\only<3-5>{\providecommand\step{3}\input{diagrams/default/default.tex}}%
\onslide*<6->{\providecommand\step{4}\input{diagrams/default/default.tex}}%
\end{column}
\end{columns}
\end{frame}

\begin{frame}{Executing the OCaml program}{And raising an exception without handler}
\begin{columns}
\begin{column}{0.6\textwidth}
\only<1-3>{
\begin{itemize}
\item<1-> Execution continues with OCaml code
\begin{itemize}
\item \funcname{camlDefault\_\_main\_269}
\item \funcname{camlDefault\_\_entry}
\item \funcname{caml\_program}
\end{itemize}
\item<1-> \funcname{camlDefault\_\_main\_269} raises a \typename{Failure} exception
\item<1-> \typename{Failure} is caught by \funcname{caml\_start\_program}
\begin{itemize}
\item<1-> The handler set the 2nd LSB of the exception value to \funcarg{1}
\item<3-> And then forces \funcname{caml\_start\_program} to terminate
\end{itemize}
\end{itemize}
}
\bigskip
\only<1,3>{\listocaml[highlightlines=3]{../src/default/default.ml}{default/default.ml}}%
\only<2>{\listasm[firstline=66,lastline=72,highlightlines=69]{src/ocaml/caml\_start\_program.s}{caml\_start\_program}}%
\end{column}
\begin{column}{0.4\textwidth}
\centering
\providecommand\step{4}\input{diagrams/default/default.tex}
\end{column}
\end{columns}
\end{frame}
%\only<>{\listasm[firstline=47,lastline=65]{src/ocaml/caml\_start\_program.s}{caml\_start\_program}}


\begin{frame}{Back to C}{Clean up, complain and terminate}
\begin{columns}
\begin{column}{0.45\textwidth}
\begin{itemize}
\item Backtrace:
\begin{itemize}
\item \funcname{caml\_start\_program} $\rightarrow$ OCaml code
\item \funcname{caml\_startup\_common}
\item \funcname{caml\_startup\_exn}
\item \funcname{caml\_startup}
\item \funcname{caml\_main}
\item \funcname{main}
\end{itemize}
\smallskip
\item \funcname{caml\_startup} checks the return value of \funcname{caml\_startup\_exn}
\begin{itemize}
\item The return value has been specially marked by \funcname{caml\_start\_program} handler
\item Calls \funcname{caml\_fatal\_uncaught\_exception}
\end{itemize}
\item<2-> \funcname{caml\_fatal\_uncaught\_exception} either
\begin{itemize}
\item<2-> Calls the user custom uncaught exception handler, if set with \mintinline{ocaml}{Printexc.set_uncaught_exception_handler fn}\footnotemark
\item<2-> Or falls back to the default one
\item<2-> Which notifies about the uncaught exception
\end{itemize}
\item<4-> Terminates the program either with \funcname{abort} or with a non-zero exit code
\end{itemize}
\end{column}
\begin{column}{0.55\textwidth}
\only<1>{
\listc[firstline=84,lastline=89,highlightlines=88]{../ocaml/runtime/caml/mlvalues.h}{runtime/caml/mlvalues.h:84}
\listc[firstline=138,lastline=143,highlightlines={141-142}]{../ocaml/runtime/startup\_nat.c}{runtime/startup\_nat.c:138}
}%
\only<2>{
\listc[firstline=138,highlightlines={147,149}]{../ocaml/runtime/printexc.c}{runtime/printexc.c:138}
}%
\only<3>{
\listc[firstline=110,lastline=134,highlightlines=129]{../ocaml/runtime/printexc.c}{runtime/printexc.c:138}
}%
\only<4>{
\listc[firstline=138,highlightlines={152,154}]{../ocaml/runtime/printexc.c}{runtime/printexc.c:138}
}%
\end{column}
\end{columns}
\footnotetext[1]{\url{https://v2.ocaml.org/api/Printexc.html\#1\_Uncaughtexceptions}}
\end{frame}
}%
\only<2>{\providecommand\step{4}%
% Runtime's default exception handler
%
\subsection*{Runtime's default exception handler}
\frameSubsection{pictures/The\_Architect.png}{}


\begin{frame}{Default exception handler}
\begin{columns}
\begin{column}{0.5\textwidth}
\begin{itemize}
\item \localname{domain\_state->exn\_handler} points to a trap located before \funcname{entry}!
\item What installed this very first trap there?
\end{itemize}
\bigskip
\listocaml{../src/default/default.ml}{default/default.ml}
\bigskip
\inputminted[fontsize=\footnotesize]{text}{../src/default/default.out}
\end{column}
\begin{column}{0.5\textwidth}
\centering
\only<1>{\providecommand\step{0}\input{diagrams/default/default.tex}}%
\only<2>{\providecommand\step{4}\input{diagrams/default/default.tex}}%
\end{column}
\end{columns}
\end{frame}


\begin{frame}{Startup}{How did we get there by the way?}
\begin{columns}
\begin{column}{0.45\textwidth}
\begin{itemize}
\item The entry point address of the binary is \funcname{main} (\funcname{\_start}) from the runtime
\item The program starts like a C program:
\begin{itemize}
\item \funcname{caml\_start\_program}
\item \funcname{caml\_startup\_common}
\item \funcname{caml\_startup\_exn}
\item \funcname{caml\_startup}
\item \funcname{caml\_main}
\item \funcname{main}
\end{itemize}
%\item \funcname{caml\_startup\_common}: initializes GC, domains \dots
% Also: codefrag, locale, custom opeartions, frame_descr, signals, stack
\item \funcname{caml\_start\_program} is the transition point between the C realm and the OCaml one
\end{itemize}
\bigskip
\listocaml{../src/default/default.ml}{default/default.ml}
\end{column}
\begin{column}{0.55\textwidth}
\listasm[lastline=24,highlightlines={22-24}]{src/ocaml/caml\_start\_program.s}{runtime/amd64.S:604}
\end{column}
\end{columns}
\end{frame}


\begin{frame}{Setup to jump into OCaml entry point}{\funcname{caml\_start\_program} initializes the main fiber}
\begin{columns}
\begin{column}{0.6\textwidth}
\only<1,3,5>{
\begin{itemize}
\item<1-> Stores information to allow switching from OCaml stack to the C stack
\item<3-> Constructs the default exception handler
\item<5-> Switches to the OCaml stack and calls \funcname{caml\_program}
\end{itemize}
\bigskip
\listocaml{../src/default/default.ml}{default/default.ml}
}%
\only<2>{
\listasm[firstline=22,lastline=41,highlightlines={25-27}]{src/ocaml/caml\_start\_program.s}{caml\_start\_program}
}%
\only<4>{
\mintasm[firstline=22,lastline=41,highlightlines={31-38}]{src/ocaml/caml\_start\_program.s}
\listasm[firstline=66,lastline=72]{src/ocaml/caml\_start\_program.s}{caml\_start\_program}
}%
\only<6>{
\listasm[firstline=22,lastline=41,highlightlines={39-41}]{src/ocaml/caml\_start\_program.s}{caml\_start\_program}
}%
\end{column}
\begin{column}{0.4\textwidth}
\centering
\only<1-2>{\providecommand\step{2}\input{diagrams/default/default.tex}}%
\only<3-5>{\providecommand\step{3}\input{diagrams/default/default.tex}}%
\onslide*<6->{\providecommand\step{4}\input{diagrams/default/default.tex}}%
\end{column}
\end{columns}
\end{frame}

\begin{frame}{Executing the OCaml program}{And raising an exception without handler}
\begin{columns}
\begin{column}{0.6\textwidth}
\only<1-3>{
\begin{itemize}
\item<1-> Execution continues with OCaml code
\begin{itemize}
\item \funcname{camlDefault\_\_main\_269}
\item \funcname{camlDefault\_\_entry}
\item \funcname{caml\_program}
\end{itemize}
\item<1-> \funcname{camlDefault\_\_main\_269} raises a \typename{Failure} exception
\item<1-> \typename{Failure} is caught by \funcname{caml\_start\_program}
\begin{itemize}
\item<1-> The handler set the 2nd LSB of the exception value to \funcarg{1}
\item<3-> And then forces \funcname{caml\_start\_program} to terminate
\end{itemize}
\end{itemize}
}
\bigskip
\only<1,3>{\listocaml[highlightlines=3]{../src/default/default.ml}{default/default.ml}}%
\only<2>{\listasm[firstline=66,lastline=72,highlightlines=69]{src/ocaml/caml\_start\_program.s}{caml\_start\_program}}%
\end{column}
\begin{column}{0.4\textwidth}
\centering
\providecommand\step{4}\input{diagrams/default/default.tex}
\end{column}
\end{columns}
\end{frame}
%\only<>{\listasm[firstline=47,lastline=65]{src/ocaml/caml\_start\_program.s}{caml\_start\_program}}


\begin{frame}{Back to C}{Clean up, complain and terminate}
\begin{columns}
\begin{column}{0.45\textwidth}
\begin{itemize}
\item Backtrace:
\begin{itemize}
\item \funcname{caml\_start\_program} $\rightarrow$ OCaml code
\item \funcname{caml\_startup\_common}
\item \funcname{caml\_startup\_exn}
\item \funcname{caml\_startup}
\item \funcname{caml\_main}
\item \funcname{main}
\end{itemize}
\smallskip
\item \funcname{caml\_startup} checks the return value of \funcname{caml\_startup\_exn}
\begin{itemize}
\item The return value has been specially marked by \funcname{caml\_start\_program} handler
\item Calls \funcname{caml\_fatal\_uncaught\_exception}
\end{itemize}
\item<2-> \funcname{caml\_fatal\_uncaught\_exception} either
\begin{itemize}
\item<2-> Calls the user custom uncaught exception handler, if set with \mintinline{ocaml}{Printexc.set_uncaught_exception_handler fn}\footnotemark
\item<2-> Or falls back to the default one
\item<2-> Which notifies about the uncaught exception
\end{itemize}
\item<4-> Terminates the program either with \funcname{abort} or with a non-zero exit code
\end{itemize}
\end{column}
\begin{column}{0.55\textwidth}
\only<1>{
\listc[firstline=84,lastline=89,highlightlines=88]{../ocaml/runtime/caml/mlvalues.h}{runtime/caml/mlvalues.h:84}
\listc[firstline=138,lastline=143,highlightlines={141-142}]{../ocaml/runtime/startup\_nat.c}{runtime/startup\_nat.c:138}
}%
\only<2>{
\listc[firstline=138,highlightlines={147,149}]{../ocaml/runtime/printexc.c}{runtime/printexc.c:138}
}%
\only<3>{
\listc[firstline=110,lastline=134,highlightlines=129]{../ocaml/runtime/printexc.c}{runtime/printexc.c:138}
}%
\only<4>{
\listc[firstline=138,highlightlines={152,154}]{../ocaml/runtime/printexc.c}{runtime/printexc.c:138}
}%
\end{column}
\end{columns}
\footnotetext[1]{\url{https://v2.ocaml.org/api/Printexc.html\#1\_Uncaughtexceptions}}
\end{frame}
}%
\end{column}
\end{columns}
\end{frame}


\begin{frame}{Startup}{How did we get there by the way?}
\begin{columns}
\begin{column}{0.45\textwidth}
\begin{itemize}
\item The entry point address of the binary is \funcname{main} (\funcname{\_start}) from the runtime
\item The program starts like a C program:
\begin{itemize}
\item \funcname{caml\_start\_program}
\item \funcname{caml\_startup\_common}
\item \funcname{caml\_startup\_exn}
\item \funcname{caml\_startup}
\item \funcname{caml\_main}
\item \funcname{main}
\end{itemize}
%\item \funcname{caml\_startup\_common}: initializes GC, domains \dots
% Also: codefrag, locale, custom opeartions, frame_descr, signals, stack
\item \funcname{caml\_start\_program} is the transition point between the C realm and the OCaml one
\end{itemize}
\bigskip
\listocaml{../src/default/default.ml}{default/default.ml}
\end{column}
\begin{column}{0.55\textwidth}
\listasm[lastline=24,highlightlines={22-24}]{src/ocaml/caml\_start\_program.s}{runtime/amd64.S:604}
\end{column}
\end{columns}
\end{frame}


\begin{frame}{Setup to jump into OCaml entry point}{\funcname{caml\_start\_program} initializes the main fiber}
\begin{columns}
\begin{column}{0.6\textwidth}
\only<1,3,5>{
\begin{itemize}
\item<1-> Stores information to allow switching from OCaml stack to the C stack
\item<3-> Constructs the default exception handler
\item<5-> Switches to the OCaml stack and calls \funcname{caml\_program}
\end{itemize}
\bigskip
\listocaml{../src/default/default.ml}{default/default.ml}
}%
\only<2>{
\listasm[firstline=22,lastline=41,highlightlines={25-27}]{src/ocaml/caml\_start\_program.s}{caml\_start\_program}
}%
\only<4>{
\mintasm[firstline=22,lastline=41,highlightlines={31-38}]{src/ocaml/caml\_start\_program.s}
\listasm[firstline=66,lastline=72]{src/ocaml/caml\_start\_program.s}{caml\_start\_program}
}%
\only<6>{
\listasm[firstline=22,lastline=41,highlightlines={39-41}]{src/ocaml/caml\_start\_program.s}{caml\_start\_program}
}%
\end{column}
\begin{column}{0.4\textwidth}
\centering
\only<1-2>{\providecommand\step{2}%
% Runtime's default exception handler
%
\subsection*{Runtime's default exception handler}
\frameSubsection{pictures/The\_Architect.png}{}


\begin{frame}{Default exception handler}
\begin{columns}
\begin{column}{0.5\textwidth}
\begin{itemize}
\item \localname{domain\_state->exn\_handler} points to a trap located before \funcname{entry}!
\item What installed this very first trap there?
\end{itemize}
\bigskip
\listocaml{../src/default/default.ml}{default/default.ml}
\bigskip
\inputminted[fontsize=\footnotesize]{text}{../src/default/default.out}
\end{column}
\begin{column}{0.5\textwidth}
\centering
\only<1>{\providecommand\step{0}\input{diagrams/default/default.tex}}%
\only<2>{\providecommand\step{4}\input{diagrams/default/default.tex}}%
\end{column}
\end{columns}
\end{frame}


\begin{frame}{Startup}{How did we get there by the way?}
\begin{columns}
\begin{column}{0.45\textwidth}
\begin{itemize}
\item The entry point address of the binary is \funcname{main} (\funcname{\_start}) from the runtime
\item The program starts like a C program:
\begin{itemize}
\item \funcname{caml\_start\_program}
\item \funcname{caml\_startup\_common}
\item \funcname{caml\_startup\_exn}
\item \funcname{caml\_startup}
\item \funcname{caml\_main}
\item \funcname{main}
\end{itemize}
%\item \funcname{caml\_startup\_common}: initializes GC, domains \dots
% Also: codefrag, locale, custom opeartions, frame_descr, signals, stack
\item \funcname{caml\_start\_program} is the transition point between the C realm and the OCaml one
\end{itemize}
\bigskip
\listocaml{../src/default/default.ml}{default/default.ml}
\end{column}
\begin{column}{0.55\textwidth}
\listasm[lastline=24,highlightlines={22-24}]{src/ocaml/caml\_start\_program.s}{runtime/amd64.S:604}
\end{column}
\end{columns}
\end{frame}


\begin{frame}{Setup to jump into OCaml entry point}{\funcname{caml\_start\_program} initializes the main fiber}
\begin{columns}
\begin{column}{0.6\textwidth}
\only<1,3,5>{
\begin{itemize}
\item<1-> Stores information to allow switching from OCaml stack to the C stack
\item<3-> Constructs the default exception handler
\item<5-> Switches to the OCaml stack and calls \funcname{caml\_program}
\end{itemize}
\bigskip
\listocaml{../src/default/default.ml}{default/default.ml}
}%
\only<2>{
\listasm[firstline=22,lastline=41,highlightlines={25-27}]{src/ocaml/caml\_start\_program.s}{caml\_start\_program}
}%
\only<4>{
\mintasm[firstline=22,lastline=41,highlightlines={31-38}]{src/ocaml/caml\_start\_program.s}
\listasm[firstline=66,lastline=72]{src/ocaml/caml\_start\_program.s}{caml\_start\_program}
}%
\only<6>{
\listasm[firstline=22,lastline=41,highlightlines={39-41}]{src/ocaml/caml\_start\_program.s}{caml\_start\_program}
}%
\end{column}
\begin{column}{0.4\textwidth}
\centering
\only<1-2>{\providecommand\step{2}\input{diagrams/default/default.tex}}%
\only<3-5>{\providecommand\step{3}\input{diagrams/default/default.tex}}%
\onslide*<6->{\providecommand\step{4}\input{diagrams/default/default.tex}}%
\end{column}
\end{columns}
\end{frame}

\begin{frame}{Executing the OCaml program}{And raising an exception without handler}
\begin{columns}
\begin{column}{0.6\textwidth}
\only<1-3>{
\begin{itemize}
\item<1-> Execution continues with OCaml code
\begin{itemize}
\item \funcname{camlDefault\_\_main\_269}
\item \funcname{camlDefault\_\_entry}
\item \funcname{caml\_program}
\end{itemize}
\item<1-> \funcname{camlDefault\_\_main\_269} raises a \typename{Failure} exception
\item<1-> \typename{Failure} is caught by \funcname{caml\_start\_program}
\begin{itemize}
\item<1-> The handler set the 2nd LSB of the exception value to \funcarg{1}
\item<3-> And then forces \funcname{caml\_start\_program} to terminate
\end{itemize}
\end{itemize}
}
\bigskip
\only<1,3>{\listocaml[highlightlines=3]{../src/default/default.ml}{default/default.ml}}%
\only<2>{\listasm[firstline=66,lastline=72,highlightlines=69]{src/ocaml/caml\_start\_program.s}{caml\_start\_program}}%
\end{column}
\begin{column}{0.4\textwidth}
\centering
\providecommand\step{4}\input{diagrams/default/default.tex}
\end{column}
\end{columns}
\end{frame}
%\only<>{\listasm[firstline=47,lastline=65]{src/ocaml/caml\_start\_program.s}{caml\_start\_program}}


\begin{frame}{Back to C}{Clean up, complain and terminate}
\begin{columns}
\begin{column}{0.45\textwidth}
\begin{itemize}
\item Backtrace:
\begin{itemize}
\item \funcname{caml\_start\_program} $\rightarrow$ OCaml code
\item \funcname{caml\_startup\_common}
\item \funcname{caml\_startup\_exn}
\item \funcname{caml\_startup}
\item \funcname{caml\_main}
\item \funcname{main}
\end{itemize}
\smallskip
\item \funcname{caml\_startup} checks the return value of \funcname{caml\_startup\_exn}
\begin{itemize}
\item The return value has been specially marked by \funcname{caml\_start\_program} handler
\item Calls \funcname{caml\_fatal\_uncaught\_exception}
\end{itemize}
\item<2-> \funcname{caml\_fatal\_uncaught\_exception} either
\begin{itemize}
\item<2-> Calls the user custom uncaught exception handler, if set with \mintinline{ocaml}{Printexc.set_uncaught_exception_handler fn}\footnotemark
\item<2-> Or falls back to the default one
\item<2-> Which notifies about the uncaught exception
\end{itemize}
\item<4-> Terminates the program either with \funcname{abort} or with a non-zero exit code
\end{itemize}
\end{column}
\begin{column}{0.55\textwidth}
\only<1>{
\listc[firstline=84,lastline=89,highlightlines=88]{../ocaml/runtime/caml/mlvalues.h}{runtime/caml/mlvalues.h:84}
\listc[firstline=138,lastline=143,highlightlines={141-142}]{../ocaml/runtime/startup\_nat.c}{runtime/startup\_nat.c:138}
}%
\only<2>{
\listc[firstline=138,highlightlines={147,149}]{../ocaml/runtime/printexc.c}{runtime/printexc.c:138}
}%
\only<3>{
\listc[firstline=110,lastline=134,highlightlines=129]{../ocaml/runtime/printexc.c}{runtime/printexc.c:138}
}%
\only<4>{
\listc[firstline=138,highlightlines={152,154}]{../ocaml/runtime/printexc.c}{runtime/printexc.c:138}
}%
\end{column}
\end{columns}
\footnotetext[1]{\url{https://v2.ocaml.org/api/Printexc.html\#1\_Uncaughtexceptions}}
\end{frame}
}%
\only<3-5>{\providecommand\step{3}%
% Runtime's default exception handler
%
\subsection*{Runtime's default exception handler}
\frameSubsection{pictures/The\_Architect.png}{}


\begin{frame}{Default exception handler}
\begin{columns}
\begin{column}{0.5\textwidth}
\begin{itemize}
\item \localname{domain\_state->exn\_handler} points to a trap located before \funcname{entry}!
\item What installed this very first trap there?
\end{itemize}
\bigskip
\listocaml{../src/default/default.ml}{default/default.ml}
\bigskip
\inputminted[fontsize=\footnotesize]{text}{../src/default/default.out}
\end{column}
\begin{column}{0.5\textwidth}
\centering
\only<1>{\providecommand\step{0}\input{diagrams/default/default.tex}}%
\only<2>{\providecommand\step{4}\input{diagrams/default/default.tex}}%
\end{column}
\end{columns}
\end{frame}


\begin{frame}{Startup}{How did we get there by the way?}
\begin{columns}
\begin{column}{0.45\textwidth}
\begin{itemize}
\item The entry point address of the binary is \funcname{main} (\funcname{\_start}) from the runtime
\item The program starts like a C program:
\begin{itemize}
\item \funcname{caml\_start\_program}
\item \funcname{caml\_startup\_common}
\item \funcname{caml\_startup\_exn}
\item \funcname{caml\_startup}
\item \funcname{caml\_main}
\item \funcname{main}
\end{itemize}
%\item \funcname{caml\_startup\_common}: initializes GC, domains \dots
% Also: codefrag, locale, custom opeartions, frame_descr, signals, stack
\item \funcname{caml\_start\_program} is the transition point between the C realm and the OCaml one
\end{itemize}
\bigskip
\listocaml{../src/default/default.ml}{default/default.ml}
\end{column}
\begin{column}{0.55\textwidth}
\listasm[lastline=24,highlightlines={22-24}]{src/ocaml/caml\_start\_program.s}{runtime/amd64.S:604}
\end{column}
\end{columns}
\end{frame}


\begin{frame}{Setup to jump into OCaml entry point}{\funcname{caml\_start\_program} initializes the main fiber}
\begin{columns}
\begin{column}{0.6\textwidth}
\only<1,3,5>{
\begin{itemize}
\item<1-> Stores information to allow switching from OCaml stack to the C stack
\item<3-> Constructs the default exception handler
\item<5-> Switches to the OCaml stack and calls \funcname{caml\_program}
\end{itemize}
\bigskip
\listocaml{../src/default/default.ml}{default/default.ml}
}%
\only<2>{
\listasm[firstline=22,lastline=41,highlightlines={25-27}]{src/ocaml/caml\_start\_program.s}{caml\_start\_program}
}%
\only<4>{
\mintasm[firstline=22,lastline=41,highlightlines={31-38}]{src/ocaml/caml\_start\_program.s}
\listasm[firstline=66,lastline=72]{src/ocaml/caml\_start\_program.s}{caml\_start\_program}
}%
\only<6>{
\listasm[firstline=22,lastline=41,highlightlines={39-41}]{src/ocaml/caml\_start\_program.s}{caml\_start\_program}
}%
\end{column}
\begin{column}{0.4\textwidth}
\centering
\only<1-2>{\providecommand\step{2}\input{diagrams/default/default.tex}}%
\only<3-5>{\providecommand\step{3}\input{diagrams/default/default.tex}}%
\onslide*<6->{\providecommand\step{4}\input{diagrams/default/default.tex}}%
\end{column}
\end{columns}
\end{frame}

\begin{frame}{Executing the OCaml program}{And raising an exception without handler}
\begin{columns}
\begin{column}{0.6\textwidth}
\only<1-3>{
\begin{itemize}
\item<1-> Execution continues with OCaml code
\begin{itemize}
\item \funcname{camlDefault\_\_main\_269}
\item \funcname{camlDefault\_\_entry}
\item \funcname{caml\_program}
\end{itemize}
\item<1-> \funcname{camlDefault\_\_main\_269} raises a \typename{Failure} exception
\item<1-> \typename{Failure} is caught by \funcname{caml\_start\_program}
\begin{itemize}
\item<1-> The handler set the 2nd LSB of the exception value to \funcarg{1}
\item<3-> And then forces \funcname{caml\_start\_program} to terminate
\end{itemize}
\end{itemize}
}
\bigskip
\only<1,3>{\listocaml[highlightlines=3]{../src/default/default.ml}{default/default.ml}}%
\only<2>{\listasm[firstline=66,lastline=72,highlightlines=69]{src/ocaml/caml\_start\_program.s}{caml\_start\_program}}%
\end{column}
\begin{column}{0.4\textwidth}
\centering
\providecommand\step{4}\input{diagrams/default/default.tex}
\end{column}
\end{columns}
\end{frame}
%\only<>{\listasm[firstline=47,lastline=65]{src/ocaml/caml\_start\_program.s}{caml\_start\_program}}


\begin{frame}{Back to C}{Clean up, complain and terminate}
\begin{columns}
\begin{column}{0.45\textwidth}
\begin{itemize}
\item Backtrace:
\begin{itemize}
\item \funcname{caml\_start\_program} $\rightarrow$ OCaml code
\item \funcname{caml\_startup\_common}
\item \funcname{caml\_startup\_exn}
\item \funcname{caml\_startup}
\item \funcname{caml\_main}
\item \funcname{main}
\end{itemize}
\smallskip
\item \funcname{caml\_startup} checks the return value of \funcname{caml\_startup\_exn}
\begin{itemize}
\item The return value has been specially marked by \funcname{caml\_start\_program} handler
\item Calls \funcname{caml\_fatal\_uncaught\_exception}
\end{itemize}
\item<2-> \funcname{caml\_fatal\_uncaught\_exception} either
\begin{itemize}
\item<2-> Calls the user custom uncaught exception handler, if set with \mintinline{ocaml}{Printexc.set_uncaught_exception_handler fn}\footnotemark
\item<2-> Or falls back to the default one
\item<2-> Which notifies about the uncaught exception
\end{itemize}
\item<4-> Terminates the program either with \funcname{abort} or with a non-zero exit code
\end{itemize}
\end{column}
\begin{column}{0.55\textwidth}
\only<1>{
\listc[firstline=84,lastline=89,highlightlines=88]{../ocaml/runtime/caml/mlvalues.h}{runtime/caml/mlvalues.h:84}
\listc[firstline=138,lastline=143,highlightlines={141-142}]{../ocaml/runtime/startup\_nat.c}{runtime/startup\_nat.c:138}
}%
\only<2>{
\listc[firstline=138,highlightlines={147,149}]{../ocaml/runtime/printexc.c}{runtime/printexc.c:138}
}%
\only<3>{
\listc[firstline=110,lastline=134,highlightlines=129]{../ocaml/runtime/printexc.c}{runtime/printexc.c:138}
}%
\only<4>{
\listc[firstline=138,highlightlines={152,154}]{../ocaml/runtime/printexc.c}{runtime/printexc.c:138}
}%
\end{column}
\end{columns}
\footnotetext[1]{\url{https://v2.ocaml.org/api/Printexc.html\#1\_Uncaughtexceptions}}
\end{frame}
}%
\onslide*<6->{\providecommand\step{4}%
% Runtime's default exception handler
%
\subsection*{Runtime's default exception handler}
\frameSubsection{pictures/The\_Architect.png}{}


\begin{frame}{Default exception handler}
\begin{columns}
\begin{column}{0.5\textwidth}
\begin{itemize}
\item \localname{domain\_state->exn\_handler} points to a trap located before \funcname{entry}!
\item What installed this very first trap there?
\end{itemize}
\bigskip
\listocaml{../src/default/default.ml}{default/default.ml}
\bigskip
\inputminted[fontsize=\footnotesize]{text}{../src/default/default.out}
\end{column}
\begin{column}{0.5\textwidth}
\centering
\only<1>{\providecommand\step{0}\input{diagrams/default/default.tex}}%
\only<2>{\providecommand\step{4}\input{diagrams/default/default.tex}}%
\end{column}
\end{columns}
\end{frame}


\begin{frame}{Startup}{How did we get there by the way?}
\begin{columns}
\begin{column}{0.45\textwidth}
\begin{itemize}
\item The entry point address of the binary is \funcname{main} (\funcname{\_start}) from the runtime
\item The program starts like a C program:
\begin{itemize}
\item \funcname{caml\_start\_program}
\item \funcname{caml\_startup\_common}
\item \funcname{caml\_startup\_exn}
\item \funcname{caml\_startup}
\item \funcname{caml\_main}
\item \funcname{main}
\end{itemize}
%\item \funcname{caml\_startup\_common}: initializes GC, domains \dots
% Also: codefrag, locale, custom opeartions, frame_descr, signals, stack
\item \funcname{caml\_start\_program} is the transition point between the C realm and the OCaml one
\end{itemize}
\bigskip
\listocaml{../src/default/default.ml}{default/default.ml}
\end{column}
\begin{column}{0.55\textwidth}
\listasm[lastline=24,highlightlines={22-24}]{src/ocaml/caml\_start\_program.s}{runtime/amd64.S:604}
\end{column}
\end{columns}
\end{frame}


\begin{frame}{Setup to jump into OCaml entry point}{\funcname{caml\_start\_program} initializes the main fiber}
\begin{columns}
\begin{column}{0.6\textwidth}
\only<1,3,5>{
\begin{itemize}
\item<1-> Stores information to allow switching from OCaml stack to the C stack
\item<3-> Constructs the default exception handler
\item<5-> Switches to the OCaml stack and calls \funcname{caml\_program}
\end{itemize}
\bigskip
\listocaml{../src/default/default.ml}{default/default.ml}
}%
\only<2>{
\listasm[firstline=22,lastline=41,highlightlines={25-27}]{src/ocaml/caml\_start\_program.s}{caml\_start\_program}
}%
\only<4>{
\mintasm[firstline=22,lastline=41,highlightlines={31-38}]{src/ocaml/caml\_start\_program.s}
\listasm[firstline=66,lastline=72]{src/ocaml/caml\_start\_program.s}{caml\_start\_program}
}%
\only<6>{
\listasm[firstline=22,lastline=41,highlightlines={39-41}]{src/ocaml/caml\_start\_program.s}{caml\_start\_program}
}%
\end{column}
\begin{column}{0.4\textwidth}
\centering
\only<1-2>{\providecommand\step{2}\input{diagrams/default/default.tex}}%
\only<3-5>{\providecommand\step{3}\input{diagrams/default/default.tex}}%
\onslide*<6->{\providecommand\step{4}\input{diagrams/default/default.tex}}%
\end{column}
\end{columns}
\end{frame}

\begin{frame}{Executing the OCaml program}{And raising an exception without handler}
\begin{columns}
\begin{column}{0.6\textwidth}
\only<1-3>{
\begin{itemize}
\item<1-> Execution continues with OCaml code
\begin{itemize}
\item \funcname{camlDefault\_\_main\_269}
\item \funcname{camlDefault\_\_entry}
\item \funcname{caml\_program}
\end{itemize}
\item<1-> \funcname{camlDefault\_\_main\_269} raises a \typename{Failure} exception
\item<1-> \typename{Failure} is caught by \funcname{caml\_start\_program}
\begin{itemize}
\item<1-> The handler set the 2nd LSB of the exception value to \funcarg{1}
\item<3-> And then forces \funcname{caml\_start\_program} to terminate
\end{itemize}
\end{itemize}
}
\bigskip
\only<1,3>{\listocaml[highlightlines=3]{../src/default/default.ml}{default/default.ml}}%
\only<2>{\listasm[firstline=66,lastline=72,highlightlines=69]{src/ocaml/caml\_start\_program.s}{caml\_start\_program}}%
\end{column}
\begin{column}{0.4\textwidth}
\centering
\providecommand\step{4}\input{diagrams/default/default.tex}
\end{column}
\end{columns}
\end{frame}
%\only<>{\listasm[firstline=47,lastline=65]{src/ocaml/caml\_start\_program.s}{caml\_start\_program}}


\begin{frame}{Back to C}{Clean up, complain and terminate}
\begin{columns}
\begin{column}{0.45\textwidth}
\begin{itemize}
\item Backtrace:
\begin{itemize}
\item \funcname{caml\_start\_program} $\rightarrow$ OCaml code
\item \funcname{caml\_startup\_common}
\item \funcname{caml\_startup\_exn}
\item \funcname{caml\_startup}
\item \funcname{caml\_main}
\item \funcname{main}
\end{itemize}
\smallskip
\item \funcname{caml\_startup} checks the return value of \funcname{caml\_startup\_exn}
\begin{itemize}
\item The return value has been specially marked by \funcname{caml\_start\_program} handler
\item Calls \funcname{caml\_fatal\_uncaught\_exception}
\end{itemize}
\item<2-> \funcname{caml\_fatal\_uncaught\_exception} either
\begin{itemize}
\item<2-> Calls the user custom uncaught exception handler, if set with \mintinline{ocaml}{Printexc.set_uncaught_exception_handler fn}\footnotemark
\item<2-> Or falls back to the default one
\item<2-> Which notifies about the uncaught exception
\end{itemize}
\item<4-> Terminates the program either with \funcname{abort} or with a non-zero exit code
\end{itemize}
\end{column}
\begin{column}{0.55\textwidth}
\only<1>{
\listc[firstline=84,lastline=89,highlightlines=88]{../ocaml/runtime/caml/mlvalues.h}{runtime/caml/mlvalues.h:84}
\listc[firstline=138,lastline=143,highlightlines={141-142}]{../ocaml/runtime/startup\_nat.c}{runtime/startup\_nat.c:138}
}%
\only<2>{
\listc[firstline=138,highlightlines={147,149}]{../ocaml/runtime/printexc.c}{runtime/printexc.c:138}
}%
\only<3>{
\listc[firstline=110,lastline=134,highlightlines=129]{../ocaml/runtime/printexc.c}{runtime/printexc.c:138}
}%
\only<4>{
\listc[firstline=138,highlightlines={152,154}]{../ocaml/runtime/printexc.c}{runtime/printexc.c:138}
}%
\end{column}
\end{columns}
\footnotetext[1]{\url{https://v2.ocaml.org/api/Printexc.html\#1\_Uncaughtexceptions}}
\end{frame}
}%
\end{column}
\end{columns}
\end{frame}

\begin{frame}{Executing the OCaml program}{And raising an exception without handler}
\begin{columns}
\begin{column}{0.6\textwidth}
\only<1-3>{
\begin{itemize}
\item<1-> Execution continues with OCaml code
\begin{itemize}
\item \funcname{camlDefault\_\_main\_269}
\item \funcname{camlDefault\_\_entry}
\item \funcname{caml\_program}
\end{itemize}
\item<1-> \funcname{camlDefault\_\_main\_269} raises a \typename{Failure} exception
\item<1-> \typename{Failure} is caught by \funcname{caml\_start\_program}
\begin{itemize}
\item<1-> The handler set the 2nd LSB of the exception value to \funcarg{1}
\item<3-> And then forces \funcname{caml\_start\_program} to terminate
\end{itemize}
\end{itemize}
}
\bigskip
\only<1,3>{\listocaml[highlightlines=3]{../src/default/default.ml}{default/default.ml}}%
\only<2>{\listasm[firstline=66,lastline=72,highlightlines=69]{src/ocaml/caml\_start\_program.s}{caml\_start\_program}}%
\end{column}
\begin{column}{0.4\textwidth}
\centering
\providecommand\step{4}%
% Runtime's default exception handler
%
\subsection*{Runtime's default exception handler}
\frameSubsection{pictures/The\_Architect.png}{}


\begin{frame}{Default exception handler}
\begin{columns}
\begin{column}{0.5\textwidth}
\begin{itemize}
\item \localname{domain\_state->exn\_handler} points to a trap located before \funcname{entry}!
\item What installed this very first trap there?
\end{itemize}
\bigskip
\listocaml{../src/default/default.ml}{default/default.ml}
\bigskip
\inputminted[fontsize=\footnotesize]{text}{../src/default/default.out}
\end{column}
\begin{column}{0.5\textwidth}
\centering
\only<1>{\providecommand\step{0}\input{diagrams/default/default.tex}}%
\only<2>{\providecommand\step{4}\input{diagrams/default/default.tex}}%
\end{column}
\end{columns}
\end{frame}


\begin{frame}{Startup}{How did we get there by the way?}
\begin{columns}
\begin{column}{0.45\textwidth}
\begin{itemize}
\item The entry point address of the binary is \funcname{main} (\funcname{\_start}) from the runtime
\item The program starts like a C program:
\begin{itemize}
\item \funcname{caml\_start\_program}
\item \funcname{caml\_startup\_common}
\item \funcname{caml\_startup\_exn}
\item \funcname{caml\_startup}
\item \funcname{caml\_main}
\item \funcname{main}
\end{itemize}
%\item \funcname{caml\_startup\_common}: initializes GC, domains \dots
% Also: codefrag, locale, custom opeartions, frame_descr, signals, stack
\item \funcname{caml\_start\_program} is the transition point between the C realm and the OCaml one
\end{itemize}
\bigskip
\listocaml{../src/default/default.ml}{default/default.ml}
\end{column}
\begin{column}{0.55\textwidth}
\listasm[lastline=24,highlightlines={22-24}]{src/ocaml/caml\_start\_program.s}{runtime/amd64.S:604}
\end{column}
\end{columns}
\end{frame}


\begin{frame}{Setup to jump into OCaml entry point}{\funcname{caml\_start\_program} initializes the main fiber}
\begin{columns}
\begin{column}{0.6\textwidth}
\only<1,3,5>{
\begin{itemize}
\item<1-> Stores information to allow switching from OCaml stack to the C stack
\item<3-> Constructs the default exception handler
\item<5-> Switches to the OCaml stack and calls \funcname{caml\_program}
\end{itemize}
\bigskip
\listocaml{../src/default/default.ml}{default/default.ml}
}%
\only<2>{
\listasm[firstline=22,lastline=41,highlightlines={25-27}]{src/ocaml/caml\_start\_program.s}{caml\_start\_program}
}%
\only<4>{
\mintasm[firstline=22,lastline=41,highlightlines={31-38}]{src/ocaml/caml\_start\_program.s}
\listasm[firstline=66,lastline=72]{src/ocaml/caml\_start\_program.s}{caml\_start\_program}
}%
\only<6>{
\listasm[firstline=22,lastline=41,highlightlines={39-41}]{src/ocaml/caml\_start\_program.s}{caml\_start\_program}
}%
\end{column}
\begin{column}{0.4\textwidth}
\centering
\only<1-2>{\providecommand\step{2}\input{diagrams/default/default.tex}}%
\only<3-5>{\providecommand\step{3}\input{diagrams/default/default.tex}}%
\onslide*<6->{\providecommand\step{4}\input{diagrams/default/default.tex}}%
\end{column}
\end{columns}
\end{frame}

\begin{frame}{Executing the OCaml program}{And raising an exception without handler}
\begin{columns}
\begin{column}{0.6\textwidth}
\only<1-3>{
\begin{itemize}
\item<1-> Execution continues with OCaml code
\begin{itemize}
\item \funcname{camlDefault\_\_main\_269}
\item \funcname{camlDefault\_\_entry}
\item \funcname{caml\_program}
\end{itemize}
\item<1-> \funcname{camlDefault\_\_main\_269} raises a \typename{Failure} exception
\item<1-> \typename{Failure} is caught by \funcname{caml\_start\_program}
\begin{itemize}
\item<1-> The handler set the 2nd LSB of the exception value to \funcarg{1}
\item<3-> And then forces \funcname{caml\_start\_program} to terminate
\end{itemize}
\end{itemize}
}
\bigskip
\only<1,3>{\listocaml[highlightlines=3]{../src/default/default.ml}{default/default.ml}}%
\only<2>{\listasm[firstline=66,lastline=72,highlightlines=69]{src/ocaml/caml\_start\_program.s}{caml\_start\_program}}%
\end{column}
\begin{column}{0.4\textwidth}
\centering
\providecommand\step{4}\input{diagrams/default/default.tex}
\end{column}
\end{columns}
\end{frame}
%\only<>{\listasm[firstline=47,lastline=65]{src/ocaml/caml\_start\_program.s}{caml\_start\_program}}


\begin{frame}{Back to C}{Clean up, complain and terminate}
\begin{columns}
\begin{column}{0.45\textwidth}
\begin{itemize}
\item Backtrace:
\begin{itemize}
\item \funcname{caml\_start\_program} $\rightarrow$ OCaml code
\item \funcname{caml\_startup\_common}
\item \funcname{caml\_startup\_exn}
\item \funcname{caml\_startup}
\item \funcname{caml\_main}
\item \funcname{main}
\end{itemize}
\smallskip
\item \funcname{caml\_startup} checks the return value of \funcname{caml\_startup\_exn}
\begin{itemize}
\item The return value has been specially marked by \funcname{caml\_start\_program} handler
\item Calls \funcname{caml\_fatal\_uncaught\_exception}
\end{itemize}
\item<2-> \funcname{caml\_fatal\_uncaught\_exception} either
\begin{itemize}
\item<2-> Calls the user custom uncaught exception handler, if set with \mintinline{ocaml}{Printexc.set_uncaught_exception_handler fn}\footnotemark
\item<2-> Or falls back to the default one
\item<2-> Which notifies about the uncaught exception
\end{itemize}
\item<4-> Terminates the program either with \funcname{abort} or with a non-zero exit code
\end{itemize}
\end{column}
\begin{column}{0.55\textwidth}
\only<1>{
\listc[firstline=84,lastline=89,highlightlines=88]{../ocaml/runtime/caml/mlvalues.h}{runtime/caml/mlvalues.h:84}
\listc[firstline=138,lastline=143,highlightlines={141-142}]{../ocaml/runtime/startup\_nat.c}{runtime/startup\_nat.c:138}
}%
\only<2>{
\listc[firstline=138,highlightlines={147,149}]{../ocaml/runtime/printexc.c}{runtime/printexc.c:138}
}%
\only<3>{
\listc[firstline=110,lastline=134,highlightlines=129]{../ocaml/runtime/printexc.c}{runtime/printexc.c:138}
}%
\only<4>{
\listc[firstline=138,highlightlines={152,154}]{../ocaml/runtime/printexc.c}{runtime/printexc.c:138}
}%
\end{column}
\end{columns}
\footnotetext[1]{\url{https://v2.ocaml.org/api/Printexc.html\#1\_Uncaughtexceptions}}
\end{frame}

\end{column}
\end{columns}
\end{frame}
%\only<>{\listasm[firstline=47,lastline=65]{src/ocaml/caml\_start\_program.s}{caml\_start\_program}}


\begin{frame}{Back to C}{Clean up, complain and terminate}
\begin{columns}
\begin{column}{0.45\textwidth}
\begin{itemize}
\item Backtrace:
\begin{itemize}
\item \funcname{caml\_start\_program} $\rightarrow$ OCaml code
\item \funcname{caml\_startup\_common}
\item \funcname{caml\_startup\_exn}
\item \funcname{caml\_startup}
\item \funcname{caml\_main}
\item \funcname{main}
\end{itemize}
\smallskip
\item \funcname{caml\_startup} checks the return value of \funcname{caml\_startup\_exn}
\begin{itemize}
\item The return value has been specially marked by \funcname{caml\_start\_program} handler
\item Calls \funcname{caml\_fatal\_uncaught\_exception}
\end{itemize}
\item<2-> \funcname{caml\_fatal\_uncaught\_exception} either
\begin{itemize}
\item<2-> Calls the user custom uncaught exception handler, if set with \mintinline{ocaml}{Printexc.set_uncaught_exception_handler fn}\footnotemark
\item<2-> Or falls back to the default one
\item<2-> Which notifies about the uncaught exception
\end{itemize}
\item<4-> Terminates the program either with \funcname{abort} or with a non-zero exit code
\end{itemize}
\end{column}
\begin{column}{0.55\textwidth}
\only<1>{
\listc[firstline=84,lastline=89,highlightlines=88]{../ocaml/runtime/caml/mlvalues.h}{runtime/caml/mlvalues.h:84}
\listc[firstline=138,lastline=143,highlightlines={141-142}]{../ocaml/runtime/startup\_nat.c}{runtime/startup\_nat.c:138}
}%
\only<2>{
\listc[firstline=138,highlightlines={147,149}]{../ocaml/runtime/printexc.c}{runtime/printexc.c:138}
}%
\only<3>{
\listc[firstline=110,lastline=134,highlightlines=129]{../ocaml/runtime/printexc.c}{runtime/printexc.c:138}
}%
\only<4>{
\listc[firstline=138,highlightlines={152,154}]{../ocaml/runtime/printexc.c}{runtime/printexc.c:138}
}%
\end{column}
\end{columns}
\footnotetext[1]{\url{https://v2.ocaml.org/api/Printexc.html\#1\_Uncaughtexceptions}}
\end{frame}

\end{column}
\end{columns}
\end{frame}
%\only<>{\listasm[firstline=47,lastline=65]{src/ocaml/caml\_start\_program.s}{caml\_start\_program}}


\begin{frame}{Back to C}{Clean up, complain and terminate}
\begin{columns}
\begin{column}{0.45\textwidth}
\begin{itemize}
\item Backtrace:
\begin{itemize}
\item \funcname{caml\_start\_program} $\rightarrow$ OCaml code
\item \funcname{caml\_startup\_common}
\item \funcname{caml\_startup\_exn}
\item \funcname{caml\_startup}
\item \funcname{caml\_main}
\item \funcname{main}
\end{itemize}
\smallskip
\item \funcname{caml\_startup} checks the return value of \funcname{caml\_startup\_exn}
\begin{itemize}
\item The return value has been specially marked by \funcname{caml\_start\_program} handler
\item Calls \funcname{caml\_fatal\_uncaught\_exception}
\end{itemize}
\item<2-> \funcname{caml\_fatal\_uncaught\_exception} either
\begin{itemize}
\item<2-> Calls the user custom uncaught exception handler, if set with \mintinline{ocaml}{Printexc.set_uncaught_exception_handler fn}\footnotemark
\item<2-> Or falls back to the default one
\item<2-> Which notifies about the uncaught exception
\end{itemize}
\item<4-> Terminates the program either with \funcname{abort} or with a non-zero exit code
\end{itemize}
\end{column}
\begin{column}{0.55\textwidth}
\only<1>{
\listc[firstline=84,lastline=89,highlightlines=88]{../ocaml/runtime/caml/mlvalues.h}{runtime/caml/mlvalues.h:84}
\listc[firstline=138,lastline=143,highlightlines={141-142}]{../ocaml/runtime/startup\_nat.c}{runtime/startup\_nat.c:138}
}%
\only<2>{
\listc[firstline=138,highlightlines={147,149}]{../ocaml/runtime/printexc.c}{runtime/printexc.c:138}
}%
\only<3>{
\listc[firstline=110,lastline=134,highlightlines=129]{../ocaml/runtime/printexc.c}{runtime/printexc.c:138}
}%
\only<4>{
\listc[firstline=138,highlightlines={152,154}]{../ocaml/runtime/printexc.c}{runtime/printexc.c:138}
}%
\end{column}
\end{columns}
\footnotetext[1]{\url{https://v2.ocaml.org/api/Printexc.html\#1\_Uncaughtexceptions}}
\end{frame}
}%
\only<2>{\providecommand\step{4}%
% Runtime's default exception handler
%
\subsection*{Runtime's default exception handler}
\frameSubsection{pictures/The\_Architect.png}{}


\begin{frame}{Default exception handler}
\begin{columns}
\begin{column}{0.5\textwidth}
\begin{itemize}
\item \localname{domain\_state->exn\_handler} points to a trap located before \funcname{entry}!
\item What installed this very first trap there?
\end{itemize}
\bigskip
\listocaml{../src/default/default.ml}{default/default.ml}
\bigskip
\inputminted[fontsize=\footnotesize]{text}{../src/default/default.out}
\end{column}
\begin{column}{0.5\textwidth}
\centering
\only<1>{\providecommand\step{0}%
% Runtime's default exception handler
%
\subsection*{Runtime's default exception handler}
\frameSubsection{pictures/The\_Architect.png}{}


\begin{frame}{Default exception handler}
\begin{columns}
\begin{column}{0.5\textwidth}
\begin{itemize}
\item \localname{domain\_state->exn\_handler} points to a trap located before \funcname{entry}!
\item What installed this very first trap there?
\end{itemize}
\bigskip
\listocaml{../src/default/default.ml}{default/default.ml}
\bigskip
\inputminted[fontsize=\footnotesize]{text}{../src/default/default.out}
\end{column}
\begin{column}{0.5\textwidth}
\centering
\only<1>{\providecommand\step{0}%
% Runtime's default exception handler
%
\subsection*{Runtime's default exception handler}
\frameSubsection{pictures/The\_Architect.png}{}


\begin{frame}{Default exception handler}
\begin{columns}
\begin{column}{0.5\textwidth}
\begin{itemize}
\item \localname{domain\_state->exn\_handler} points to a trap located before \funcname{entry}!
\item What installed this very first trap there?
\end{itemize}
\bigskip
\listocaml{../src/default/default.ml}{default/default.ml}
\bigskip
\inputminted[fontsize=\footnotesize]{text}{../src/default/default.out}
\end{column}
\begin{column}{0.5\textwidth}
\centering
\only<1>{\providecommand\step{0}\input{diagrams/default/default.tex}}%
\only<2>{\providecommand\step{4}\input{diagrams/default/default.tex}}%
\end{column}
\end{columns}
\end{frame}


\begin{frame}{Startup}{How did we get there by the way?}
\begin{columns}
\begin{column}{0.45\textwidth}
\begin{itemize}
\item The entry point address of the binary is \funcname{main} (\funcname{\_start}) from the runtime
\item The program starts like a C program:
\begin{itemize}
\item \funcname{caml\_start\_program}
\item \funcname{caml\_startup\_common}
\item \funcname{caml\_startup\_exn}
\item \funcname{caml\_startup}
\item \funcname{caml\_main}
\item \funcname{main}
\end{itemize}
%\item \funcname{caml\_startup\_common}: initializes GC, domains \dots
% Also: codefrag, locale, custom opeartions, frame_descr, signals, stack
\item \funcname{caml\_start\_program} is the transition point between the C realm and the OCaml one
\end{itemize}
\bigskip
\listocaml{../src/default/default.ml}{default/default.ml}
\end{column}
\begin{column}{0.55\textwidth}
\listasm[lastline=24,highlightlines={22-24}]{src/ocaml/caml\_start\_program.s}{runtime/amd64.S:604}
\end{column}
\end{columns}
\end{frame}


\begin{frame}{Setup to jump into OCaml entry point}{\funcname{caml\_start\_program} initializes the main fiber}
\begin{columns}
\begin{column}{0.6\textwidth}
\only<1,3,5>{
\begin{itemize}
\item<1-> Stores information to allow switching from OCaml stack to the C stack
\item<3-> Constructs the default exception handler
\item<5-> Switches to the OCaml stack and calls \funcname{caml\_program}
\end{itemize}
\bigskip
\listocaml{../src/default/default.ml}{default/default.ml}
}%
\only<2>{
\listasm[firstline=22,lastline=41,highlightlines={25-27}]{src/ocaml/caml\_start\_program.s}{caml\_start\_program}
}%
\only<4>{
\mintasm[firstline=22,lastline=41,highlightlines={31-38}]{src/ocaml/caml\_start\_program.s}
\listasm[firstline=66,lastline=72]{src/ocaml/caml\_start\_program.s}{caml\_start\_program}
}%
\only<6>{
\listasm[firstline=22,lastline=41,highlightlines={39-41}]{src/ocaml/caml\_start\_program.s}{caml\_start\_program}
}%
\end{column}
\begin{column}{0.4\textwidth}
\centering
\only<1-2>{\providecommand\step{2}\input{diagrams/default/default.tex}}%
\only<3-5>{\providecommand\step{3}\input{diagrams/default/default.tex}}%
\onslide*<6->{\providecommand\step{4}\input{diagrams/default/default.tex}}%
\end{column}
\end{columns}
\end{frame}

\begin{frame}{Executing the OCaml program}{And raising an exception without handler}
\begin{columns}
\begin{column}{0.6\textwidth}
\only<1-3>{
\begin{itemize}
\item<1-> Execution continues with OCaml code
\begin{itemize}
\item \funcname{camlDefault\_\_main\_269}
\item \funcname{camlDefault\_\_entry}
\item \funcname{caml\_program}
\end{itemize}
\item<1-> \funcname{camlDefault\_\_main\_269} raises a \typename{Failure} exception
\item<1-> \typename{Failure} is caught by \funcname{caml\_start\_program}
\begin{itemize}
\item<1-> The handler set the 2nd LSB of the exception value to \funcarg{1}
\item<3-> And then forces \funcname{caml\_start\_program} to terminate
\end{itemize}
\end{itemize}
}
\bigskip
\only<1,3>{\listocaml[highlightlines=3]{../src/default/default.ml}{default/default.ml}}%
\only<2>{\listasm[firstline=66,lastline=72,highlightlines=69]{src/ocaml/caml\_start\_program.s}{caml\_start\_program}}%
\end{column}
\begin{column}{0.4\textwidth}
\centering
\providecommand\step{4}\input{diagrams/default/default.tex}
\end{column}
\end{columns}
\end{frame}
%\only<>{\listasm[firstline=47,lastline=65]{src/ocaml/caml\_start\_program.s}{caml\_start\_program}}


\begin{frame}{Back to C}{Clean up, complain and terminate}
\begin{columns}
\begin{column}{0.45\textwidth}
\begin{itemize}
\item Backtrace:
\begin{itemize}
\item \funcname{caml\_start\_program} $\rightarrow$ OCaml code
\item \funcname{caml\_startup\_common}
\item \funcname{caml\_startup\_exn}
\item \funcname{caml\_startup}
\item \funcname{caml\_main}
\item \funcname{main}
\end{itemize}
\smallskip
\item \funcname{caml\_startup} checks the return value of \funcname{caml\_startup\_exn}
\begin{itemize}
\item The return value has been specially marked by \funcname{caml\_start\_program} handler
\item Calls \funcname{caml\_fatal\_uncaught\_exception}
\end{itemize}
\item<2-> \funcname{caml\_fatal\_uncaught\_exception} either
\begin{itemize}
\item<2-> Calls the user custom uncaught exception handler, if set with \mintinline{ocaml}{Printexc.set_uncaught_exception_handler fn}\footnotemark
\item<2-> Or falls back to the default one
\item<2-> Which notifies about the uncaught exception
\end{itemize}
\item<4-> Terminates the program either with \funcname{abort} or with a non-zero exit code
\end{itemize}
\end{column}
\begin{column}{0.55\textwidth}
\only<1>{
\listc[firstline=84,lastline=89,highlightlines=88]{../ocaml/runtime/caml/mlvalues.h}{runtime/caml/mlvalues.h:84}
\listc[firstline=138,lastline=143,highlightlines={141-142}]{../ocaml/runtime/startup\_nat.c}{runtime/startup\_nat.c:138}
}%
\only<2>{
\listc[firstline=138,highlightlines={147,149}]{../ocaml/runtime/printexc.c}{runtime/printexc.c:138}
}%
\only<3>{
\listc[firstline=110,lastline=134,highlightlines=129]{../ocaml/runtime/printexc.c}{runtime/printexc.c:138}
}%
\only<4>{
\listc[firstline=138,highlightlines={152,154}]{../ocaml/runtime/printexc.c}{runtime/printexc.c:138}
}%
\end{column}
\end{columns}
\footnotetext[1]{\url{https://v2.ocaml.org/api/Printexc.html\#1\_Uncaughtexceptions}}
\end{frame}
}%
\only<2>{\providecommand\step{4}%
% Runtime's default exception handler
%
\subsection*{Runtime's default exception handler}
\frameSubsection{pictures/The\_Architect.png}{}


\begin{frame}{Default exception handler}
\begin{columns}
\begin{column}{0.5\textwidth}
\begin{itemize}
\item \localname{domain\_state->exn\_handler} points to a trap located before \funcname{entry}!
\item What installed this very first trap there?
\end{itemize}
\bigskip
\listocaml{../src/default/default.ml}{default/default.ml}
\bigskip
\inputminted[fontsize=\footnotesize]{text}{../src/default/default.out}
\end{column}
\begin{column}{0.5\textwidth}
\centering
\only<1>{\providecommand\step{0}\input{diagrams/default/default.tex}}%
\only<2>{\providecommand\step{4}\input{diagrams/default/default.tex}}%
\end{column}
\end{columns}
\end{frame}


\begin{frame}{Startup}{How did we get there by the way?}
\begin{columns}
\begin{column}{0.45\textwidth}
\begin{itemize}
\item The entry point address of the binary is \funcname{main} (\funcname{\_start}) from the runtime
\item The program starts like a C program:
\begin{itemize}
\item \funcname{caml\_start\_program}
\item \funcname{caml\_startup\_common}
\item \funcname{caml\_startup\_exn}
\item \funcname{caml\_startup}
\item \funcname{caml\_main}
\item \funcname{main}
\end{itemize}
%\item \funcname{caml\_startup\_common}: initializes GC, domains \dots
% Also: codefrag, locale, custom opeartions, frame_descr, signals, stack
\item \funcname{caml\_start\_program} is the transition point between the C realm and the OCaml one
\end{itemize}
\bigskip
\listocaml{../src/default/default.ml}{default/default.ml}
\end{column}
\begin{column}{0.55\textwidth}
\listasm[lastline=24,highlightlines={22-24}]{src/ocaml/caml\_start\_program.s}{runtime/amd64.S:604}
\end{column}
\end{columns}
\end{frame}


\begin{frame}{Setup to jump into OCaml entry point}{\funcname{caml\_start\_program} initializes the main fiber}
\begin{columns}
\begin{column}{0.6\textwidth}
\only<1,3,5>{
\begin{itemize}
\item<1-> Stores information to allow switching from OCaml stack to the C stack
\item<3-> Constructs the default exception handler
\item<5-> Switches to the OCaml stack and calls \funcname{caml\_program}
\end{itemize}
\bigskip
\listocaml{../src/default/default.ml}{default/default.ml}
}%
\only<2>{
\listasm[firstline=22,lastline=41,highlightlines={25-27}]{src/ocaml/caml\_start\_program.s}{caml\_start\_program}
}%
\only<4>{
\mintasm[firstline=22,lastline=41,highlightlines={31-38}]{src/ocaml/caml\_start\_program.s}
\listasm[firstline=66,lastline=72]{src/ocaml/caml\_start\_program.s}{caml\_start\_program}
}%
\only<6>{
\listasm[firstline=22,lastline=41,highlightlines={39-41}]{src/ocaml/caml\_start\_program.s}{caml\_start\_program}
}%
\end{column}
\begin{column}{0.4\textwidth}
\centering
\only<1-2>{\providecommand\step{2}\input{diagrams/default/default.tex}}%
\only<3-5>{\providecommand\step{3}\input{diagrams/default/default.tex}}%
\onslide*<6->{\providecommand\step{4}\input{diagrams/default/default.tex}}%
\end{column}
\end{columns}
\end{frame}

\begin{frame}{Executing the OCaml program}{And raising an exception without handler}
\begin{columns}
\begin{column}{0.6\textwidth}
\only<1-3>{
\begin{itemize}
\item<1-> Execution continues with OCaml code
\begin{itemize}
\item \funcname{camlDefault\_\_main\_269}
\item \funcname{camlDefault\_\_entry}
\item \funcname{caml\_program}
\end{itemize}
\item<1-> \funcname{camlDefault\_\_main\_269} raises a \typename{Failure} exception
\item<1-> \typename{Failure} is caught by \funcname{caml\_start\_program}
\begin{itemize}
\item<1-> The handler set the 2nd LSB of the exception value to \funcarg{1}
\item<3-> And then forces \funcname{caml\_start\_program} to terminate
\end{itemize}
\end{itemize}
}
\bigskip
\only<1,3>{\listocaml[highlightlines=3]{../src/default/default.ml}{default/default.ml}}%
\only<2>{\listasm[firstline=66,lastline=72,highlightlines=69]{src/ocaml/caml\_start\_program.s}{caml\_start\_program}}%
\end{column}
\begin{column}{0.4\textwidth}
\centering
\providecommand\step{4}\input{diagrams/default/default.tex}
\end{column}
\end{columns}
\end{frame}
%\only<>{\listasm[firstline=47,lastline=65]{src/ocaml/caml\_start\_program.s}{caml\_start\_program}}


\begin{frame}{Back to C}{Clean up, complain and terminate}
\begin{columns}
\begin{column}{0.45\textwidth}
\begin{itemize}
\item Backtrace:
\begin{itemize}
\item \funcname{caml\_start\_program} $\rightarrow$ OCaml code
\item \funcname{caml\_startup\_common}
\item \funcname{caml\_startup\_exn}
\item \funcname{caml\_startup}
\item \funcname{caml\_main}
\item \funcname{main}
\end{itemize}
\smallskip
\item \funcname{caml\_startup} checks the return value of \funcname{caml\_startup\_exn}
\begin{itemize}
\item The return value has been specially marked by \funcname{caml\_start\_program} handler
\item Calls \funcname{caml\_fatal\_uncaught\_exception}
\end{itemize}
\item<2-> \funcname{caml\_fatal\_uncaught\_exception} either
\begin{itemize}
\item<2-> Calls the user custom uncaught exception handler, if set with \mintinline{ocaml}{Printexc.set_uncaught_exception_handler fn}\footnotemark
\item<2-> Or falls back to the default one
\item<2-> Which notifies about the uncaught exception
\end{itemize}
\item<4-> Terminates the program either with \funcname{abort} or with a non-zero exit code
\end{itemize}
\end{column}
\begin{column}{0.55\textwidth}
\only<1>{
\listc[firstline=84,lastline=89,highlightlines=88]{../ocaml/runtime/caml/mlvalues.h}{runtime/caml/mlvalues.h:84}
\listc[firstline=138,lastline=143,highlightlines={141-142}]{../ocaml/runtime/startup\_nat.c}{runtime/startup\_nat.c:138}
}%
\only<2>{
\listc[firstline=138,highlightlines={147,149}]{../ocaml/runtime/printexc.c}{runtime/printexc.c:138}
}%
\only<3>{
\listc[firstline=110,lastline=134,highlightlines=129]{../ocaml/runtime/printexc.c}{runtime/printexc.c:138}
}%
\only<4>{
\listc[firstline=138,highlightlines={152,154}]{../ocaml/runtime/printexc.c}{runtime/printexc.c:138}
}%
\end{column}
\end{columns}
\footnotetext[1]{\url{https://v2.ocaml.org/api/Printexc.html\#1\_Uncaughtexceptions}}
\end{frame}
}%
\end{column}
\end{columns}
\end{frame}


\begin{frame}{Startup}{How did we get there by the way?}
\begin{columns}
\begin{column}{0.45\textwidth}
\begin{itemize}
\item The entry point address of the binary is \funcname{main} (\funcname{\_start}) from the runtime
\item The program starts like a C program:
\begin{itemize}
\item \funcname{caml\_start\_program}
\item \funcname{caml\_startup\_common}
\item \funcname{caml\_startup\_exn}
\item \funcname{caml\_startup}
\item \funcname{caml\_main}
\item \funcname{main}
\end{itemize}
%\item \funcname{caml\_startup\_common}: initializes GC, domains \dots
% Also: codefrag, locale, custom opeartions, frame_descr, signals, stack
\item \funcname{caml\_start\_program} is the transition point between the C realm and the OCaml one
\end{itemize}
\bigskip
\listocaml{../src/default/default.ml}{default/default.ml}
\end{column}
\begin{column}{0.55\textwidth}
\listasm[lastline=24,highlightlines={22-24}]{src/ocaml/caml\_start\_program.s}{runtime/amd64.S:604}
\end{column}
\end{columns}
\end{frame}


\begin{frame}{Setup to jump into OCaml entry point}{\funcname{caml\_start\_program} initializes the main fiber}
\begin{columns}
\begin{column}{0.6\textwidth}
\only<1,3,5>{
\begin{itemize}
\item<1-> Stores information to allow switching from OCaml stack to the C stack
\item<3-> Constructs the default exception handler
\item<5-> Switches to the OCaml stack and calls \funcname{caml\_program}
\end{itemize}
\bigskip
\listocaml{../src/default/default.ml}{default/default.ml}
}%
\only<2>{
\listasm[firstline=22,lastline=41,highlightlines={25-27}]{src/ocaml/caml\_start\_program.s}{caml\_start\_program}
}%
\only<4>{
\mintasm[firstline=22,lastline=41,highlightlines={31-38}]{src/ocaml/caml\_start\_program.s}
\listasm[firstline=66,lastline=72]{src/ocaml/caml\_start\_program.s}{caml\_start\_program}
}%
\only<6>{
\listasm[firstline=22,lastline=41,highlightlines={39-41}]{src/ocaml/caml\_start\_program.s}{caml\_start\_program}
}%
\end{column}
\begin{column}{0.4\textwidth}
\centering
\only<1-2>{\providecommand\step{2}%
% Runtime's default exception handler
%
\subsection*{Runtime's default exception handler}
\frameSubsection{pictures/The\_Architect.png}{}


\begin{frame}{Default exception handler}
\begin{columns}
\begin{column}{0.5\textwidth}
\begin{itemize}
\item \localname{domain\_state->exn\_handler} points to a trap located before \funcname{entry}!
\item What installed this very first trap there?
\end{itemize}
\bigskip
\listocaml{../src/default/default.ml}{default/default.ml}
\bigskip
\inputminted[fontsize=\footnotesize]{text}{../src/default/default.out}
\end{column}
\begin{column}{0.5\textwidth}
\centering
\only<1>{\providecommand\step{0}\input{diagrams/default/default.tex}}%
\only<2>{\providecommand\step{4}\input{diagrams/default/default.tex}}%
\end{column}
\end{columns}
\end{frame}


\begin{frame}{Startup}{How did we get there by the way?}
\begin{columns}
\begin{column}{0.45\textwidth}
\begin{itemize}
\item The entry point address of the binary is \funcname{main} (\funcname{\_start}) from the runtime
\item The program starts like a C program:
\begin{itemize}
\item \funcname{caml\_start\_program}
\item \funcname{caml\_startup\_common}
\item \funcname{caml\_startup\_exn}
\item \funcname{caml\_startup}
\item \funcname{caml\_main}
\item \funcname{main}
\end{itemize}
%\item \funcname{caml\_startup\_common}: initializes GC, domains \dots
% Also: codefrag, locale, custom opeartions, frame_descr, signals, stack
\item \funcname{caml\_start\_program} is the transition point between the C realm and the OCaml one
\end{itemize}
\bigskip
\listocaml{../src/default/default.ml}{default/default.ml}
\end{column}
\begin{column}{0.55\textwidth}
\listasm[lastline=24,highlightlines={22-24}]{src/ocaml/caml\_start\_program.s}{runtime/amd64.S:604}
\end{column}
\end{columns}
\end{frame}


\begin{frame}{Setup to jump into OCaml entry point}{\funcname{caml\_start\_program} initializes the main fiber}
\begin{columns}
\begin{column}{0.6\textwidth}
\only<1,3,5>{
\begin{itemize}
\item<1-> Stores information to allow switching from OCaml stack to the C stack
\item<3-> Constructs the default exception handler
\item<5-> Switches to the OCaml stack and calls \funcname{caml\_program}
\end{itemize}
\bigskip
\listocaml{../src/default/default.ml}{default/default.ml}
}%
\only<2>{
\listasm[firstline=22,lastline=41,highlightlines={25-27}]{src/ocaml/caml\_start\_program.s}{caml\_start\_program}
}%
\only<4>{
\mintasm[firstline=22,lastline=41,highlightlines={31-38}]{src/ocaml/caml\_start\_program.s}
\listasm[firstline=66,lastline=72]{src/ocaml/caml\_start\_program.s}{caml\_start\_program}
}%
\only<6>{
\listasm[firstline=22,lastline=41,highlightlines={39-41}]{src/ocaml/caml\_start\_program.s}{caml\_start\_program}
}%
\end{column}
\begin{column}{0.4\textwidth}
\centering
\only<1-2>{\providecommand\step{2}\input{diagrams/default/default.tex}}%
\only<3-5>{\providecommand\step{3}\input{diagrams/default/default.tex}}%
\onslide*<6->{\providecommand\step{4}\input{diagrams/default/default.tex}}%
\end{column}
\end{columns}
\end{frame}

\begin{frame}{Executing the OCaml program}{And raising an exception without handler}
\begin{columns}
\begin{column}{0.6\textwidth}
\only<1-3>{
\begin{itemize}
\item<1-> Execution continues with OCaml code
\begin{itemize}
\item \funcname{camlDefault\_\_main\_269}
\item \funcname{camlDefault\_\_entry}
\item \funcname{caml\_program}
\end{itemize}
\item<1-> \funcname{camlDefault\_\_main\_269} raises a \typename{Failure} exception
\item<1-> \typename{Failure} is caught by \funcname{caml\_start\_program}
\begin{itemize}
\item<1-> The handler set the 2nd LSB of the exception value to \funcarg{1}
\item<3-> And then forces \funcname{caml\_start\_program} to terminate
\end{itemize}
\end{itemize}
}
\bigskip
\only<1,3>{\listocaml[highlightlines=3]{../src/default/default.ml}{default/default.ml}}%
\only<2>{\listasm[firstline=66,lastline=72,highlightlines=69]{src/ocaml/caml\_start\_program.s}{caml\_start\_program}}%
\end{column}
\begin{column}{0.4\textwidth}
\centering
\providecommand\step{4}\input{diagrams/default/default.tex}
\end{column}
\end{columns}
\end{frame}
%\only<>{\listasm[firstline=47,lastline=65]{src/ocaml/caml\_start\_program.s}{caml\_start\_program}}


\begin{frame}{Back to C}{Clean up, complain and terminate}
\begin{columns}
\begin{column}{0.45\textwidth}
\begin{itemize}
\item Backtrace:
\begin{itemize}
\item \funcname{caml\_start\_program} $\rightarrow$ OCaml code
\item \funcname{caml\_startup\_common}
\item \funcname{caml\_startup\_exn}
\item \funcname{caml\_startup}
\item \funcname{caml\_main}
\item \funcname{main}
\end{itemize}
\smallskip
\item \funcname{caml\_startup} checks the return value of \funcname{caml\_startup\_exn}
\begin{itemize}
\item The return value has been specially marked by \funcname{caml\_start\_program} handler
\item Calls \funcname{caml\_fatal\_uncaught\_exception}
\end{itemize}
\item<2-> \funcname{caml\_fatal\_uncaught\_exception} either
\begin{itemize}
\item<2-> Calls the user custom uncaught exception handler, if set with \mintinline{ocaml}{Printexc.set_uncaught_exception_handler fn}\footnotemark
\item<2-> Or falls back to the default one
\item<2-> Which notifies about the uncaught exception
\end{itemize}
\item<4-> Terminates the program either with \funcname{abort} or with a non-zero exit code
\end{itemize}
\end{column}
\begin{column}{0.55\textwidth}
\only<1>{
\listc[firstline=84,lastline=89,highlightlines=88]{../ocaml/runtime/caml/mlvalues.h}{runtime/caml/mlvalues.h:84}
\listc[firstline=138,lastline=143,highlightlines={141-142}]{../ocaml/runtime/startup\_nat.c}{runtime/startup\_nat.c:138}
}%
\only<2>{
\listc[firstline=138,highlightlines={147,149}]{../ocaml/runtime/printexc.c}{runtime/printexc.c:138}
}%
\only<3>{
\listc[firstline=110,lastline=134,highlightlines=129]{../ocaml/runtime/printexc.c}{runtime/printexc.c:138}
}%
\only<4>{
\listc[firstline=138,highlightlines={152,154}]{../ocaml/runtime/printexc.c}{runtime/printexc.c:138}
}%
\end{column}
\end{columns}
\footnotetext[1]{\url{https://v2.ocaml.org/api/Printexc.html\#1\_Uncaughtexceptions}}
\end{frame}
}%
\only<3-5>{\providecommand\step{3}%
% Runtime's default exception handler
%
\subsection*{Runtime's default exception handler}
\frameSubsection{pictures/The\_Architect.png}{}


\begin{frame}{Default exception handler}
\begin{columns}
\begin{column}{0.5\textwidth}
\begin{itemize}
\item \localname{domain\_state->exn\_handler} points to a trap located before \funcname{entry}!
\item What installed this very first trap there?
\end{itemize}
\bigskip
\listocaml{../src/default/default.ml}{default/default.ml}
\bigskip
\inputminted[fontsize=\footnotesize]{text}{../src/default/default.out}
\end{column}
\begin{column}{0.5\textwidth}
\centering
\only<1>{\providecommand\step{0}\input{diagrams/default/default.tex}}%
\only<2>{\providecommand\step{4}\input{diagrams/default/default.tex}}%
\end{column}
\end{columns}
\end{frame}


\begin{frame}{Startup}{How did we get there by the way?}
\begin{columns}
\begin{column}{0.45\textwidth}
\begin{itemize}
\item The entry point address of the binary is \funcname{main} (\funcname{\_start}) from the runtime
\item The program starts like a C program:
\begin{itemize}
\item \funcname{caml\_start\_program}
\item \funcname{caml\_startup\_common}
\item \funcname{caml\_startup\_exn}
\item \funcname{caml\_startup}
\item \funcname{caml\_main}
\item \funcname{main}
\end{itemize}
%\item \funcname{caml\_startup\_common}: initializes GC, domains \dots
% Also: codefrag, locale, custom opeartions, frame_descr, signals, stack
\item \funcname{caml\_start\_program} is the transition point between the C realm and the OCaml one
\end{itemize}
\bigskip
\listocaml{../src/default/default.ml}{default/default.ml}
\end{column}
\begin{column}{0.55\textwidth}
\listasm[lastline=24,highlightlines={22-24}]{src/ocaml/caml\_start\_program.s}{runtime/amd64.S:604}
\end{column}
\end{columns}
\end{frame}


\begin{frame}{Setup to jump into OCaml entry point}{\funcname{caml\_start\_program} initializes the main fiber}
\begin{columns}
\begin{column}{0.6\textwidth}
\only<1,3,5>{
\begin{itemize}
\item<1-> Stores information to allow switching from OCaml stack to the C stack
\item<3-> Constructs the default exception handler
\item<5-> Switches to the OCaml stack and calls \funcname{caml\_program}
\end{itemize}
\bigskip
\listocaml{../src/default/default.ml}{default/default.ml}
}%
\only<2>{
\listasm[firstline=22,lastline=41,highlightlines={25-27}]{src/ocaml/caml\_start\_program.s}{caml\_start\_program}
}%
\only<4>{
\mintasm[firstline=22,lastline=41,highlightlines={31-38}]{src/ocaml/caml\_start\_program.s}
\listasm[firstline=66,lastline=72]{src/ocaml/caml\_start\_program.s}{caml\_start\_program}
}%
\only<6>{
\listasm[firstline=22,lastline=41,highlightlines={39-41}]{src/ocaml/caml\_start\_program.s}{caml\_start\_program}
}%
\end{column}
\begin{column}{0.4\textwidth}
\centering
\only<1-2>{\providecommand\step{2}\input{diagrams/default/default.tex}}%
\only<3-5>{\providecommand\step{3}\input{diagrams/default/default.tex}}%
\onslide*<6->{\providecommand\step{4}\input{diagrams/default/default.tex}}%
\end{column}
\end{columns}
\end{frame}

\begin{frame}{Executing the OCaml program}{And raising an exception without handler}
\begin{columns}
\begin{column}{0.6\textwidth}
\only<1-3>{
\begin{itemize}
\item<1-> Execution continues with OCaml code
\begin{itemize}
\item \funcname{camlDefault\_\_main\_269}
\item \funcname{camlDefault\_\_entry}
\item \funcname{caml\_program}
\end{itemize}
\item<1-> \funcname{camlDefault\_\_main\_269} raises a \typename{Failure} exception
\item<1-> \typename{Failure} is caught by \funcname{caml\_start\_program}
\begin{itemize}
\item<1-> The handler set the 2nd LSB of the exception value to \funcarg{1}
\item<3-> And then forces \funcname{caml\_start\_program} to terminate
\end{itemize}
\end{itemize}
}
\bigskip
\only<1,3>{\listocaml[highlightlines=3]{../src/default/default.ml}{default/default.ml}}%
\only<2>{\listasm[firstline=66,lastline=72,highlightlines=69]{src/ocaml/caml\_start\_program.s}{caml\_start\_program}}%
\end{column}
\begin{column}{0.4\textwidth}
\centering
\providecommand\step{4}\input{diagrams/default/default.tex}
\end{column}
\end{columns}
\end{frame}
%\only<>{\listasm[firstline=47,lastline=65]{src/ocaml/caml\_start\_program.s}{caml\_start\_program}}


\begin{frame}{Back to C}{Clean up, complain and terminate}
\begin{columns}
\begin{column}{0.45\textwidth}
\begin{itemize}
\item Backtrace:
\begin{itemize}
\item \funcname{caml\_start\_program} $\rightarrow$ OCaml code
\item \funcname{caml\_startup\_common}
\item \funcname{caml\_startup\_exn}
\item \funcname{caml\_startup}
\item \funcname{caml\_main}
\item \funcname{main}
\end{itemize}
\smallskip
\item \funcname{caml\_startup} checks the return value of \funcname{caml\_startup\_exn}
\begin{itemize}
\item The return value has been specially marked by \funcname{caml\_start\_program} handler
\item Calls \funcname{caml\_fatal\_uncaught\_exception}
\end{itemize}
\item<2-> \funcname{caml\_fatal\_uncaught\_exception} either
\begin{itemize}
\item<2-> Calls the user custom uncaught exception handler, if set with \mintinline{ocaml}{Printexc.set_uncaught_exception_handler fn}\footnotemark
\item<2-> Or falls back to the default one
\item<2-> Which notifies about the uncaught exception
\end{itemize}
\item<4-> Terminates the program either with \funcname{abort} or with a non-zero exit code
\end{itemize}
\end{column}
\begin{column}{0.55\textwidth}
\only<1>{
\listc[firstline=84,lastline=89,highlightlines=88]{../ocaml/runtime/caml/mlvalues.h}{runtime/caml/mlvalues.h:84}
\listc[firstline=138,lastline=143,highlightlines={141-142}]{../ocaml/runtime/startup\_nat.c}{runtime/startup\_nat.c:138}
}%
\only<2>{
\listc[firstline=138,highlightlines={147,149}]{../ocaml/runtime/printexc.c}{runtime/printexc.c:138}
}%
\only<3>{
\listc[firstline=110,lastline=134,highlightlines=129]{../ocaml/runtime/printexc.c}{runtime/printexc.c:138}
}%
\only<4>{
\listc[firstline=138,highlightlines={152,154}]{../ocaml/runtime/printexc.c}{runtime/printexc.c:138}
}%
\end{column}
\end{columns}
\footnotetext[1]{\url{https://v2.ocaml.org/api/Printexc.html\#1\_Uncaughtexceptions}}
\end{frame}
}%
\onslide*<6->{\providecommand\step{4}%
% Runtime's default exception handler
%
\subsection*{Runtime's default exception handler}
\frameSubsection{pictures/The\_Architect.png}{}


\begin{frame}{Default exception handler}
\begin{columns}
\begin{column}{0.5\textwidth}
\begin{itemize}
\item \localname{domain\_state->exn\_handler} points to a trap located before \funcname{entry}!
\item What installed this very first trap there?
\end{itemize}
\bigskip
\listocaml{../src/default/default.ml}{default/default.ml}
\bigskip
\inputminted[fontsize=\footnotesize]{text}{../src/default/default.out}
\end{column}
\begin{column}{0.5\textwidth}
\centering
\only<1>{\providecommand\step{0}\input{diagrams/default/default.tex}}%
\only<2>{\providecommand\step{4}\input{diagrams/default/default.tex}}%
\end{column}
\end{columns}
\end{frame}


\begin{frame}{Startup}{How did we get there by the way?}
\begin{columns}
\begin{column}{0.45\textwidth}
\begin{itemize}
\item The entry point address of the binary is \funcname{main} (\funcname{\_start}) from the runtime
\item The program starts like a C program:
\begin{itemize}
\item \funcname{caml\_start\_program}
\item \funcname{caml\_startup\_common}
\item \funcname{caml\_startup\_exn}
\item \funcname{caml\_startup}
\item \funcname{caml\_main}
\item \funcname{main}
\end{itemize}
%\item \funcname{caml\_startup\_common}: initializes GC, domains \dots
% Also: codefrag, locale, custom opeartions, frame_descr, signals, stack
\item \funcname{caml\_start\_program} is the transition point between the C realm and the OCaml one
\end{itemize}
\bigskip
\listocaml{../src/default/default.ml}{default/default.ml}
\end{column}
\begin{column}{0.55\textwidth}
\listasm[lastline=24,highlightlines={22-24}]{src/ocaml/caml\_start\_program.s}{runtime/amd64.S:604}
\end{column}
\end{columns}
\end{frame}


\begin{frame}{Setup to jump into OCaml entry point}{\funcname{caml\_start\_program} initializes the main fiber}
\begin{columns}
\begin{column}{0.6\textwidth}
\only<1,3,5>{
\begin{itemize}
\item<1-> Stores information to allow switching from OCaml stack to the C stack
\item<3-> Constructs the default exception handler
\item<5-> Switches to the OCaml stack and calls \funcname{caml\_program}
\end{itemize}
\bigskip
\listocaml{../src/default/default.ml}{default/default.ml}
}%
\only<2>{
\listasm[firstline=22,lastline=41,highlightlines={25-27}]{src/ocaml/caml\_start\_program.s}{caml\_start\_program}
}%
\only<4>{
\mintasm[firstline=22,lastline=41,highlightlines={31-38}]{src/ocaml/caml\_start\_program.s}
\listasm[firstline=66,lastline=72]{src/ocaml/caml\_start\_program.s}{caml\_start\_program}
}%
\only<6>{
\listasm[firstline=22,lastline=41,highlightlines={39-41}]{src/ocaml/caml\_start\_program.s}{caml\_start\_program}
}%
\end{column}
\begin{column}{0.4\textwidth}
\centering
\only<1-2>{\providecommand\step{2}\input{diagrams/default/default.tex}}%
\only<3-5>{\providecommand\step{3}\input{diagrams/default/default.tex}}%
\onslide*<6->{\providecommand\step{4}\input{diagrams/default/default.tex}}%
\end{column}
\end{columns}
\end{frame}

\begin{frame}{Executing the OCaml program}{And raising an exception without handler}
\begin{columns}
\begin{column}{0.6\textwidth}
\only<1-3>{
\begin{itemize}
\item<1-> Execution continues with OCaml code
\begin{itemize}
\item \funcname{camlDefault\_\_main\_269}
\item \funcname{camlDefault\_\_entry}
\item \funcname{caml\_program}
\end{itemize}
\item<1-> \funcname{camlDefault\_\_main\_269} raises a \typename{Failure} exception
\item<1-> \typename{Failure} is caught by \funcname{caml\_start\_program}
\begin{itemize}
\item<1-> The handler set the 2nd LSB of the exception value to \funcarg{1}
\item<3-> And then forces \funcname{caml\_start\_program} to terminate
\end{itemize}
\end{itemize}
}
\bigskip
\only<1,3>{\listocaml[highlightlines=3]{../src/default/default.ml}{default/default.ml}}%
\only<2>{\listasm[firstline=66,lastline=72,highlightlines=69]{src/ocaml/caml\_start\_program.s}{caml\_start\_program}}%
\end{column}
\begin{column}{0.4\textwidth}
\centering
\providecommand\step{4}\input{diagrams/default/default.tex}
\end{column}
\end{columns}
\end{frame}
%\only<>{\listasm[firstline=47,lastline=65]{src/ocaml/caml\_start\_program.s}{caml\_start\_program}}


\begin{frame}{Back to C}{Clean up, complain and terminate}
\begin{columns}
\begin{column}{0.45\textwidth}
\begin{itemize}
\item Backtrace:
\begin{itemize}
\item \funcname{caml\_start\_program} $\rightarrow$ OCaml code
\item \funcname{caml\_startup\_common}
\item \funcname{caml\_startup\_exn}
\item \funcname{caml\_startup}
\item \funcname{caml\_main}
\item \funcname{main}
\end{itemize}
\smallskip
\item \funcname{caml\_startup} checks the return value of \funcname{caml\_startup\_exn}
\begin{itemize}
\item The return value has been specially marked by \funcname{caml\_start\_program} handler
\item Calls \funcname{caml\_fatal\_uncaught\_exception}
\end{itemize}
\item<2-> \funcname{caml\_fatal\_uncaught\_exception} either
\begin{itemize}
\item<2-> Calls the user custom uncaught exception handler, if set with \mintinline{ocaml}{Printexc.set_uncaught_exception_handler fn}\footnotemark
\item<2-> Or falls back to the default one
\item<2-> Which notifies about the uncaught exception
\end{itemize}
\item<4-> Terminates the program either with \funcname{abort} or with a non-zero exit code
\end{itemize}
\end{column}
\begin{column}{0.55\textwidth}
\only<1>{
\listc[firstline=84,lastline=89,highlightlines=88]{../ocaml/runtime/caml/mlvalues.h}{runtime/caml/mlvalues.h:84}
\listc[firstline=138,lastline=143,highlightlines={141-142}]{../ocaml/runtime/startup\_nat.c}{runtime/startup\_nat.c:138}
}%
\only<2>{
\listc[firstline=138,highlightlines={147,149}]{../ocaml/runtime/printexc.c}{runtime/printexc.c:138}
}%
\only<3>{
\listc[firstline=110,lastline=134,highlightlines=129]{../ocaml/runtime/printexc.c}{runtime/printexc.c:138}
}%
\only<4>{
\listc[firstline=138,highlightlines={152,154}]{../ocaml/runtime/printexc.c}{runtime/printexc.c:138}
}%
\end{column}
\end{columns}
\footnotetext[1]{\url{https://v2.ocaml.org/api/Printexc.html\#1\_Uncaughtexceptions}}
\end{frame}
}%
\end{column}
\end{columns}
\end{frame}

\begin{frame}{Executing the OCaml program}{And raising an exception without handler}
\begin{columns}
\begin{column}{0.6\textwidth}
\only<1-3>{
\begin{itemize}
\item<1-> Execution continues with OCaml code
\begin{itemize}
\item \funcname{camlDefault\_\_main\_269}
\item \funcname{camlDefault\_\_entry}
\item \funcname{caml\_program}
\end{itemize}
\item<1-> \funcname{camlDefault\_\_main\_269} raises a \typename{Failure} exception
\item<1-> \typename{Failure} is caught by \funcname{caml\_start\_program}
\begin{itemize}
\item<1-> The handler set the 2nd LSB of the exception value to \funcarg{1}
\item<3-> And then forces \funcname{caml\_start\_program} to terminate
\end{itemize}
\end{itemize}
}
\bigskip
\only<1,3>{\listocaml[highlightlines=3]{../src/default/default.ml}{default/default.ml}}%
\only<2>{\listasm[firstline=66,lastline=72,highlightlines=69]{src/ocaml/caml\_start\_program.s}{caml\_start\_program}}%
\end{column}
\begin{column}{0.4\textwidth}
\centering
\providecommand\step{4}%
% Runtime's default exception handler
%
\subsection*{Runtime's default exception handler}
\frameSubsection{pictures/The\_Architect.png}{}


\begin{frame}{Default exception handler}
\begin{columns}
\begin{column}{0.5\textwidth}
\begin{itemize}
\item \localname{domain\_state->exn\_handler} points to a trap located before \funcname{entry}!
\item What installed this very first trap there?
\end{itemize}
\bigskip
\listocaml{../src/default/default.ml}{default/default.ml}
\bigskip
\inputminted[fontsize=\footnotesize]{text}{../src/default/default.out}
\end{column}
\begin{column}{0.5\textwidth}
\centering
\only<1>{\providecommand\step{0}\input{diagrams/default/default.tex}}%
\only<2>{\providecommand\step{4}\input{diagrams/default/default.tex}}%
\end{column}
\end{columns}
\end{frame}


\begin{frame}{Startup}{How did we get there by the way?}
\begin{columns}
\begin{column}{0.45\textwidth}
\begin{itemize}
\item The entry point address of the binary is \funcname{main} (\funcname{\_start}) from the runtime
\item The program starts like a C program:
\begin{itemize}
\item \funcname{caml\_start\_program}
\item \funcname{caml\_startup\_common}
\item \funcname{caml\_startup\_exn}
\item \funcname{caml\_startup}
\item \funcname{caml\_main}
\item \funcname{main}
\end{itemize}
%\item \funcname{caml\_startup\_common}: initializes GC, domains \dots
% Also: codefrag, locale, custom opeartions, frame_descr, signals, stack
\item \funcname{caml\_start\_program} is the transition point between the C realm and the OCaml one
\end{itemize}
\bigskip
\listocaml{../src/default/default.ml}{default/default.ml}
\end{column}
\begin{column}{0.55\textwidth}
\listasm[lastline=24,highlightlines={22-24}]{src/ocaml/caml\_start\_program.s}{runtime/amd64.S:604}
\end{column}
\end{columns}
\end{frame}


\begin{frame}{Setup to jump into OCaml entry point}{\funcname{caml\_start\_program} initializes the main fiber}
\begin{columns}
\begin{column}{0.6\textwidth}
\only<1,3,5>{
\begin{itemize}
\item<1-> Stores information to allow switching from OCaml stack to the C stack
\item<3-> Constructs the default exception handler
\item<5-> Switches to the OCaml stack and calls \funcname{caml\_program}
\end{itemize}
\bigskip
\listocaml{../src/default/default.ml}{default/default.ml}
}%
\only<2>{
\listasm[firstline=22,lastline=41,highlightlines={25-27}]{src/ocaml/caml\_start\_program.s}{caml\_start\_program}
}%
\only<4>{
\mintasm[firstline=22,lastline=41,highlightlines={31-38}]{src/ocaml/caml\_start\_program.s}
\listasm[firstline=66,lastline=72]{src/ocaml/caml\_start\_program.s}{caml\_start\_program}
}%
\only<6>{
\listasm[firstline=22,lastline=41,highlightlines={39-41}]{src/ocaml/caml\_start\_program.s}{caml\_start\_program}
}%
\end{column}
\begin{column}{0.4\textwidth}
\centering
\only<1-2>{\providecommand\step{2}\input{diagrams/default/default.tex}}%
\only<3-5>{\providecommand\step{3}\input{diagrams/default/default.tex}}%
\onslide*<6->{\providecommand\step{4}\input{diagrams/default/default.tex}}%
\end{column}
\end{columns}
\end{frame}

\begin{frame}{Executing the OCaml program}{And raising an exception without handler}
\begin{columns}
\begin{column}{0.6\textwidth}
\only<1-3>{
\begin{itemize}
\item<1-> Execution continues with OCaml code
\begin{itemize}
\item \funcname{camlDefault\_\_main\_269}
\item \funcname{camlDefault\_\_entry}
\item \funcname{caml\_program}
\end{itemize}
\item<1-> \funcname{camlDefault\_\_main\_269} raises a \typename{Failure} exception
\item<1-> \typename{Failure} is caught by \funcname{caml\_start\_program}
\begin{itemize}
\item<1-> The handler set the 2nd LSB of the exception value to \funcarg{1}
\item<3-> And then forces \funcname{caml\_start\_program} to terminate
\end{itemize}
\end{itemize}
}
\bigskip
\only<1,3>{\listocaml[highlightlines=3]{../src/default/default.ml}{default/default.ml}}%
\only<2>{\listasm[firstline=66,lastline=72,highlightlines=69]{src/ocaml/caml\_start\_program.s}{caml\_start\_program}}%
\end{column}
\begin{column}{0.4\textwidth}
\centering
\providecommand\step{4}\input{diagrams/default/default.tex}
\end{column}
\end{columns}
\end{frame}
%\only<>{\listasm[firstline=47,lastline=65]{src/ocaml/caml\_start\_program.s}{caml\_start\_program}}


\begin{frame}{Back to C}{Clean up, complain and terminate}
\begin{columns}
\begin{column}{0.45\textwidth}
\begin{itemize}
\item Backtrace:
\begin{itemize}
\item \funcname{caml\_start\_program} $\rightarrow$ OCaml code
\item \funcname{caml\_startup\_common}
\item \funcname{caml\_startup\_exn}
\item \funcname{caml\_startup}
\item \funcname{caml\_main}
\item \funcname{main}
\end{itemize}
\smallskip
\item \funcname{caml\_startup} checks the return value of \funcname{caml\_startup\_exn}
\begin{itemize}
\item The return value has been specially marked by \funcname{caml\_start\_program} handler
\item Calls \funcname{caml\_fatal\_uncaught\_exception}
\end{itemize}
\item<2-> \funcname{caml\_fatal\_uncaught\_exception} either
\begin{itemize}
\item<2-> Calls the user custom uncaught exception handler, if set with \mintinline{ocaml}{Printexc.set_uncaught_exception_handler fn}\footnotemark
\item<2-> Or falls back to the default one
\item<2-> Which notifies about the uncaught exception
\end{itemize}
\item<4-> Terminates the program either with \funcname{abort} or with a non-zero exit code
\end{itemize}
\end{column}
\begin{column}{0.55\textwidth}
\only<1>{
\listc[firstline=84,lastline=89,highlightlines=88]{../ocaml/runtime/caml/mlvalues.h}{runtime/caml/mlvalues.h:84}
\listc[firstline=138,lastline=143,highlightlines={141-142}]{../ocaml/runtime/startup\_nat.c}{runtime/startup\_nat.c:138}
}%
\only<2>{
\listc[firstline=138,highlightlines={147,149}]{../ocaml/runtime/printexc.c}{runtime/printexc.c:138}
}%
\only<3>{
\listc[firstline=110,lastline=134,highlightlines=129]{../ocaml/runtime/printexc.c}{runtime/printexc.c:138}
}%
\only<4>{
\listc[firstline=138,highlightlines={152,154}]{../ocaml/runtime/printexc.c}{runtime/printexc.c:138}
}%
\end{column}
\end{columns}
\footnotetext[1]{\url{https://v2.ocaml.org/api/Printexc.html\#1\_Uncaughtexceptions}}
\end{frame}

\end{column}
\end{columns}
\end{frame}
%\only<>{\listasm[firstline=47,lastline=65]{src/ocaml/caml\_start\_program.s}{caml\_start\_program}}


\begin{frame}{Back to C}{Clean up, complain and terminate}
\begin{columns}
\begin{column}{0.45\textwidth}
\begin{itemize}
\item Backtrace:
\begin{itemize}
\item \funcname{caml\_start\_program} $\rightarrow$ OCaml code
\item \funcname{caml\_startup\_common}
\item \funcname{caml\_startup\_exn}
\item \funcname{caml\_startup}
\item \funcname{caml\_main}
\item \funcname{main}
\end{itemize}
\smallskip
\item \funcname{caml\_startup} checks the return value of \funcname{caml\_startup\_exn}
\begin{itemize}
\item The return value has been specially marked by \funcname{caml\_start\_program} handler
\item Calls \funcname{caml\_fatal\_uncaught\_exception}
\end{itemize}
\item<2-> \funcname{caml\_fatal\_uncaught\_exception} either
\begin{itemize}
\item<2-> Calls the user custom uncaught exception handler, if set with \mintinline{ocaml}{Printexc.set_uncaught_exception_handler fn}\footnotemark
\item<2-> Or falls back to the default one
\item<2-> Which notifies about the uncaught exception
\end{itemize}
\item<4-> Terminates the program either with \funcname{abort} or with a non-zero exit code
\end{itemize}
\end{column}
\begin{column}{0.55\textwidth}
\only<1>{
\listc[firstline=84,lastline=89,highlightlines=88]{../ocaml/runtime/caml/mlvalues.h}{runtime/caml/mlvalues.h:84}
\listc[firstline=138,lastline=143,highlightlines={141-142}]{../ocaml/runtime/startup\_nat.c}{runtime/startup\_nat.c:138}
}%
\only<2>{
\listc[firstline=138,highlightlines={147,149}]{../ocaml/runtime/printexc.c}{runtime/printexc.c:138}
}%
\only<3>{
\listc[firstline=110,lastline=134,highlightlines=129]{../ocaml/runtime/printexc.c}{runtime/printexc.c:138}
}%
\only<4>{
\listc[firstline=138,highlightlines={152,154}]{../ocaml/runtime/printexc.c}{runtime/printexc.c:138}
}%
\end{column}
\end{columns}
\footnotetext[1]{\url{https://v2.ocaml.org/api/Printexc.html\#1\_Uncaughtexceptions}}
\end{frame}
}%
\only<2>{\providecommand\step{4}%
% Runtime's default exception handler
%
\subsection*{Runtime's default exception handler}
\frameSubsection{pictures/The\_Architect.png}{}


\begin{frame}{Default exception handler}
\begin{columns}
\begin{column}{0.5\textwidth}
\begin{itemize}
\item \localname{domain\_state->exn\_handler} points to a trap located before \funcname{entry}!
\item What installed this very first trap there?
\end{itemize}
\bigskip
\listocaml{../src/default/default.ml}{default/default.ml}
\bigskip
\inputminted[fontsize=\footnotesize]{text}{../src/default/default.out}
\end{column}
\begin{column}{0.5\textwidth}
\centering
\only<1>{\providecommand\step{0}%
% Runtime's default exception handler
%
\subsection*{Runtime's default exception handler}
\frameSubsection{pictures/The\_Architect.png}{}


\begin{frame}{Default exception handler}
\begin{columns}
\begin{column}{0.5\textwidth}
\begin{itemize}
\item \localname{domain\_state->exn\_handler} points to a trap located before \funcname{entry}!
\item What installed this very first trap there?
\end{itemize}
\bigskip
\listocaml{../src/default/default.ml}{default/default.ml}
\bigskip
\inputminted[fontsize=\footnotesize]{text}{../src/default/default.out}
\end{column}
\begin{column}{0.5\textwidth}
\centering
\only<1>{\providecommand\step{0}\input{diagrams/default/default.tex}}%
\only<2>{\providecommand\step{4}\input{diagrams/default/default.tex}}%
\end{column}
\end{columns}
\end{frame}


\begin{frame}{Startup}{How did we get there by the way?}
\begin{columns}
\begin{column}{0.45\textwidth}
\begin{itemize}
\item The entry point address of the binary is \funcname{main} (\funcname{\_start}) from the runtime
\item The program starts like a C program:
\begin{itemize}
\item \funcname{caml\_start\_program}
\item \funcname{caml\_startup\_common}
\item \funcname{caml\_startup\_exn}
\item \funcname{caml\_startup}
\item \funcname{caml\_main}
\item \funcname{main}
\end{itemize}
%\item \funcname{caml\_startup\_common}: initializes GC, domains \dots
% Also: codefrag, locale, custom opeartions, frame_descr, signals, stack
\item \funcname{caml\_start\_program} is the transition point between the C realm and the OCaml one
\end{itemize}
\bigskip
\listocaml{../src/default/default.ml}{default/default.ml}
\end{column}
\begin{column}{0.55\textwidth}
\listasm[lastline=24,highlightlines={22-24}]{src/ocaml/caml\_start\_program.s}{runtime/amd64.S:604}
\end{column}
\end{columns}
\end{frame}


\begin{frame}{Setup to jump into OCaml entry point}{\funcname{caml\_start\_program} initializes the main fiber}
\begin{columns}
\begin{column}{0.6\textwidth}
\only<1,3,5>{
\begin{itemize}
\item<1-> Stores information to allow switching from OCaml stack to the C stack
\item<3-> Constructs the default exception handler
\item<5-> Switches to the OCaml stack and calls \funcname{caml\_program}
\end{itemize}
\bigskip
\listocaml{../src/default/default.ml}{default/default.ml}
}%
\only<2>{
\listasm[firstline=22,lastline=41,highlightlines={25-27}]{src/ocaml/caml\_start\_program.s}{caml\_start\_program}
}%
\only<4>{
\mintasm[firstline=22,lastline=41,highlightlines={31-38}]{src/ocaml/caml\_start\_program.s}
\listasm[firstline=66,lastline=72]{src/ocaml/caml\_start\_program.s}{caml\_start\_program}
}%
\only<6>{
\listasm[firstline=22,lastline=41,highlightlines={39-41}]{src/ocaml/caml\_start\_program.s}{caml\_start\_program}
}%
\end{column}
\begin{column}{0.4\textwidth}
\centering
\only<1-2>{\providecommand\step{2}\input{diagrams/default/default.tex}}%
\only<3-5>{\providecommand\step{3}\input{diagrams/default/default.tex}}%
\onslide*<6->{\providecommand\step{4}\input{diagrams/default/default.tex}}%
\end{column}
\end{columns}
\end{frame}

\begin{frame}{Executing the OCaml program}{And raising an exception without handler}
\begin{columns}
\begin{column}{0.6\textwidth}
\only<1-3>{
\begin{itemize}
\item<1-> Execution continues with OCaml code
\begin{itemize}
\item \funcname{camlDefault\_\_main\_269}
\item \funcname{camlDefault\_\_entry}
\item \funcname{caml\_program}
\end{itemize}
\item<1-> \funcname{camlDefault\_\_main\_269} raises a \typename{Failure} exception
\item<1-> \typename{Failure} is caught by \funcname{caml\_start\_program}
\begin{itemize}
\item<1-> The handler set the 2nd LSB of the exception value to \funcarg{1}
\item<3-> And then forces \funcname{caml\_start\_program} to terminate
\end{itemize}
\end{itemize}
}
\bigskip
\only<1,3>{\listocaml[highlightlines=3]{../src/default/default.ml}{default/default.ml}}%
\only<2>{\listasm[firstline=66,lastline=72,highlightlines=69]{src/ocaml/caml\_start\_program.s}{caml\_start\_program}}%
\end{column}
\begin{column}{0.4\textwidth}
\centering
\providecommand\step{4}\input{diagrams/default/default.tex}
\end{column}
\end{columns}
\end{frame}
%\only<>{\listasm[firstline=47,lastline=65]{src/ocaml/caml\_start\_program.s}{caml\_start\_program}}


\begin{frame}{Back to C}{Clean up, complain and terminate}
\begin{columns}
\begin{column}{0.45\textwidth}
\begin{itemize}
\item Backtrace:
\begin{itemize}
\item \funcname{caml\_start\_program} $\rightarrow$ OCaml code
\item \funcname{caml\_startup\_common}
\item \funcname{caml\_startup\_exn}
\item \funcname{caml\_startup}
\item \funcname{caml\_main}
\item \funcname{main}
\end{itemize}
\smallskip
\item \funcname{caml\_startup} checks the return value of \funcname{caml\_startup\_exn}
\begin{itemize}
\item The return value has been specially marked by \funcname{caml\_start\_program} handler
\item Calls \funcname{caml\_fatal\_uncaught\_exception}
\end{itemize}
\item<2-> \funcname{caml\_fatal\_uncaught\_exception} either
\begin{itemize}
\item<2-> Calls the user custom uncaught exception handler, if set with \mintinline{ocaml}{Printexc.set_uncaught_exception_handler fn}\footnotemark
\item<2-> Or falls back to the default one
\item<2-> Which notifies about the uncaught exception
\end{itemize}
\item<4-> Terminates the program either with \funcname{abort} or with a non-zero exit code
\end{itemize}
\end{column}
\begin{column}{0.55\textwidth}
\only<1>{
\listc[firstline=84,lastline=89,highlightlines=88]{../ocaml/runtime/caml/mlvalues.h}{runtime/caml/mlvalues.h:84}
\listc[firstline=138,lastline=143,highlightlines={141-142}]{../ocaml/runtime/startup\_nat.c}{runtime/startup\_nat.c:138}
}%
\only<2>{
\listc[firstline=138,highlightlines={147,149}]{../ocaml/runtime/printexc.c}{runtime/printexc.c:138}
}%
\only<3>{
\listc[firstline=110,lastline=134,highlightlines=129]{../ocaml/runtime/printexc.c}{runtime/printexc.c:138}
}%
\only<4>{
\listc[firstline=138,highlightlines={152,154}]{../ocaml/runtime/printexc.c}{runtime/printexc.c:138}
}%
\end{column}
\end{columns}
\footnotetext[1]{\url{https://v2.ocaml.org/api/Printexc.html\#1\_Uncaughtexceptions}}
\end{frame}
}%
\only<2>{\providecommand\step{4}%
% Runtime's default exception handler
%
\subsection*{Runtime's default exception handler}
\frameSubsection{pictures/The\_Architect.png}{}


\begin{frame}{Default exception handler}
\begin{columns}
\begin{column}{0.5\textwidth}
\begin{itemize}
\item \localname{domain\_state->exn\_handler} points to a trap located before \funcname{entry}!
\item What installed this very first trap there?
\end{itemize}
\bigskip
\listocaml{../src/default/default.ml}{default/default.ml}
\bigskip
\inputminted[fontsize=\footnotesize]{text}{../src/default/default.out}
\end{column}
\begin{column}{0.5\textwidth}
\centering
\only<1>{\providecommand\step{0}\input{diagrams/default/default.tex}}%
\only<2>{\providecommand\step{4}\input{diagrams/default/default.tex}}%
\end{column}
\end{columns}
\end{frame}


\begin{frame}{Startup}{How did we get there by the way?}
\begin{columns}
\begin{column}{0.45\textwidth}
\begin{itemize}
\item The entry point address of the binary is \funcname{main} (\funcname{\_start}) from the runtime
\item The program starts like a C program:
\begin{itemize}
\item \funcname{caml\_start\_program}
\item \funcname{caml\_startup\_common}
\item \funcname{caml\_startup\_exn}
\item \funcname{caml\_startup}
\item \funcname{caml\_main}
\item \funcname{main}
\end{itemize}
%\item \funcname{caml\_startup\_common}: initializes GC, domains \dots
% Also: codefrag, locale, custom opeartions, frame_descr, signals, stack
\item \funcname{caml\_start\_program} is the transition point between the C realm and the OCaml one
\end{itemize}
\bigskip
\listocaml{../src/default/default.ml}{default/default.ml}
\end{column}
\begin{column}{0.55\textwidth}
\listasm[lastline=24,highlightlines={22-24}]{src/ocaml/caml\_start\_program.s}{runtime/amd64.S:604}
\end{column}
\end{columns}
\end{frame}


\begin{frame}{Setup to jump into OCaml entry point}{\funcname{caml\_start\_program} initializes the main fiber}
\begin{columns}
\begin{column}{0.6\textwidth}
\only<1,3,5>{
\begin{itemize}
\item<1-> Stores information to allow switching from OCaml stack to the C stack
\item<3-> Constructs the default exception handler
\item<5-> Switches to the OCaml stack and calls \funcname{caml\_program}
\end{itemize}
\bigskip
\listocaml{../src/default/default.ml}{default/default.ml}
}%
\only<2>{
\listasm[firstline=22,lastline=41,highlightlines={25-27}]{src/ocaml/caml\_start\_program.s}{caml\_start\_program}
}%
\only<4>{
\mintasm[firstline=22,lastline=41,highlightlines={31-38}]{src/ocaml/caml\_start\_program.s}
\listasm[firstline=66,lastline=72]{src/ocaml/caml\_start\_program.s}{caml\_start\_program}
}%
\only<6>{
\listasm[firstline=22,lastline=41,highlightlines={39-41}]{src/ocaml/caml\_start\_program.s}{caml\_start\_program}
}%
\end{column}
\begin{column}{0.4\textwidth}
\centering
\only<1-2>{\providecommand\step{2}\input{diagrams/default/default.tex}}%
\only<3-5>{\providecommand\step{3}\input{diagrams/default/default.tex}}%
\onslide*<6->{\providecommand\step{4}\input{diagrams/default/default.tex}}%
\end{column}
\end{columns}
\end{frame}

\begin{frame}{Executing the OCaml program}{And raising an exception without handler}
\begin{columns}
\begin{column}{0.6\textwidth}
\only<1-3>{
\begin{itemize}
\item<1-> Execution continues with OCaml code
\begin{itemize}
\item \funcname{camlDefault\_\_main\_269}
\item \funcname{camlDefault\_\_entry}
\item \funcname{caml\_program}
\end{itemize}
\item<1-> \funcname{camlDefault\_\_main\_269} raises a \typename{Failure} exception
\item<1-> \typename{Failure} is caught by \funcname{caml\_start\_program}
\begin{itemize}
\item<1-> The handler set the 2nd LSB of the exception value to \funcarg{1}
\item<3-> And then forces \funcname{caml\_start\_program} to terminate
\end{itemize}
\end{itemize}
}
\bigskip
\only<1,3>{\listocaml[highlightlines=3]{../src/default/default.ml}{default/default.ml}}%
\only<2>{\listasm[firstline=66,lastline=72,highlightlines=69]{src/ocaml/caml\_start\_program.s}{caml\_start\_program}}%
\end{column}
\begin{column}{0.4\textwidth}
\centering
\providecommand\step{4}\input{diagrams/default/default.tex}
\end{column}
\end{columns}
\end{frame}
%\only<>{\listasm[firstline=47,lastline=65]{src/ocaml/caml\_start\_program.s}{caml\_start\_program}}


\begin{frame}{Back to C}{Clean up, complain and terminate}
\begin{columns}
\begin{column}{0.45\textwidth}
\begin{itemize}
\item Backtrace:
\begin{itemize}
\item \funcname{caml\_start\_program} $\rightarrow$ OCaml code
\item \funcname{caml\_startup\_common}
\item \funcname{caml\_startup\_exn}
\item \funcname{caml\_startup}
\item \funcname{caml\_main}
\item \funcname{main}
\end{itemize}
\smallskip
\item \funcname{caml\_startup} checks the return value of \funcname{caml\_startup\_exn}
\begin{itemize}
\item The return value has been specially marked by \funcname{caml\_start\_program} handler
\item Calls \funcname{caml\_fatal\_uncaught\_exception}
\end{itemize}
\item<2-> \funcname{caml\_fatal\_uncaught\_exception} either
\begin{itemize}
\item<2-> Calls the user custom uncaught exception handler, if set with \mintinline{ocaml}{Printexc.set_uncaught_exception_handler fn}\footnotemark
\item<2-> Or falls back to the default one
\item<2-> Which notifies about the uncaught exception
\end{itemize}
\item<4-> Terminates the program either with \funcname{abort} or with a non-zero exit code
\end{itemize}
\end{column}
\begin{column}{0.55\textwidth}
\only<1>{
\listc[firstline=84,lastline=89,highlightlines=88]{../ocaml/runtime/caml/mlvalues.h}{runtime/caml/mlvalues.h:84}
\listc[firstline=138,lastline=143,highlightlines={141-142}]{../ocaml/runtime/startup\_nat.c}{runtime/startup\_nat.c:138}
}%
\only<2>{
\listc[firstline=138,highlightlines={147,149}]{../ocaml/runtime/printexc.c}{runtime/printexc.c:138}
}%
\only<3>{
\listc[firstline=110,lastline=134,highlightlines=129]{../ocaml/runtime/printexc.c}{runtime/printexc.c:138}
}%
\only<4>{
\listc[firstline=138,highlightlines={152,154}]{../ocaml/runtime/printexc.c}{runtime/printexc.c:138}
}%
\end{column}
\end{columns}
\footnotetext[1]{\url{https://v2.ocaml.org/api/Printexc.html\#1\_Uncaughtexceptions}}
\end{frame}
}%
\end{column}
\end{columns}
\end{frame}


\begin{frame}{Startup}{How did we get there by the way?}
\begin{columns}
\begin{column}{0.45\textwidth}
\begin{itemize}
\item The entry point address of the binary is \funcname{main} (\funcname{\_start}) from the runtime
\item The program starts like a C program:
\begin{itemize}
\item \funcname{caml\_start\_program}
\item \funcname{caml\_startup\_common}
\item \funcname{caml\_startup\_exn}
\item \funcname{caml\_startup}
\item \funcname{caml\_main}
\item \funcname{main}
\end{itemize}
%\item \funcname{caml\_startup\_common}: initializes GC, domains \dots
% Also: codefrag, locale, custom opeartions, frame_descr, signals, stack
\item \funcname{caml\_start\_program} is the transition point between the C realm and the OCaml one
\end{itemize}
\bigskip
\listocaml{../src/default/default.ml}{default/default.ml}
\end{column}
\begin{column}{0.55\textwidth}
\listasm[lastline=24,highlightlines={22-24}]{src/ocaml/caml\_start\_program.s}{runtime/amd64.S:604}
\end{column}
\end{columns}
\end{frame}


\begin{frame}{Setup to jump into OCaml entry point}{\funcname{caml\_start\_program} initializes the main fiber}
\begin{columns}
\begin{column}{0.6\textwidth}
\only<1,3,5>{
\begin{itemize}
\item<1-> Stores information to allow switching from OCaml stack to the C stack
\item<3-> Constructs the default exception handler
\item<5-> Switches to the OCaml stack and calls \funcname{caml\_program}
\end{itemize}
\bigskip
\listocaml{../src/default/default.ml}{default/default.ml}
}%
\only<2>{
\listasm[firstline=22,lastline=41,highlightlines={25-27}]{src/ocaml/caml\_start\_program.s}{caml\_start\_program}
}%
\only<4>{
\mintasm[firstline=22,lastline=41,highlightlines={31-38}]{src/ocaml/caml\_start\_program.s}
\listasm[firstline=66,lastline=72]{src/ocaml/caml\_start\_program.s}{caml\_start\_program}
}%
\only<6>{
\listasm[firstline=22,lastline=41,highlightlines={39-41}]{src/ocaml/caml\_start\_program.s}{caml\_start\_program}
}%
\end{column}
\begin{column}{0.4\textwidth}
\centering
\only<1-2>{\providecommand\step{2}%
% Runtime's default exception handler
%
\subsection*{Runtime's default exception handler}
\frameSubsection{pictures/The\_Architect.png}{}


\begin{frame}{Default exception handler}
\begin{columns}
\begin{column}{0.5\textwidth}
\begin{itemize}
\item \localname{domain\_state->exn\_handler} points to a trap located before \funcname{entry}!
\item What installed this very first trap there?
\end{itemize}
\bigskip
\listocaml{../src/default/default.ml}{default/default.ml}
\bigskip
\inputminted[fontsize=\footnotesize]{text}{../src/default/default.out}
\end{column}
\begin{column}{0.5\textwidth}
\centering
\only<1>{\providecommand\step{0}\input{diagrams/default/default.tex}}%
\only<2>{\providecommand\step{4}\input{diagrams/default/default.tex}}%
\end{column}
\end{columns}
\end{frame}


\begin{frame}{Startup}{How did we get there by the way?}
\begin{columns}
\begin{column}{0.45\textwidth}
\begin{itemize}
\item The entry point address of the binary is \funcname{main} (\funcname{\_start}) from the runtime
\item The program starts like a C program:
\begin{itemize}
\item \funcname{caml\_start\_program}
\item \funcname{caml\_startup\_common}
\item \funcname{caml\_startup\_exn}
\item \funcname{caml\_startup}
\item \funcname{caml\_main}
\item \funcname{main}
\end{itemize}
%\item \funcname{caml\_startup\_common}: initializes GC, domains \dots
% Also: codefrag, locale, custom opeartions, frame_descr, signals, stack
\item \funcname{caml\_start\_program} is the transition point between the C realm and the OCaml one
\end{itemize}
\bigskip
\listocaml{../src/default/default.ml}{default/default.ml}
\end{column}
\begin{column}{0.55\textwidth}
\listasm[lastline=24,highlightlines={22-24}]{src/ocaml/caml\_start\_program.s}{runtime/amd64.S:604}
\end{column}
\end{columns}
\end{frame}


\begin{frame}{Setup to jump into OCaml entry point}{\funcname{caml\_start\_program} initializes the main fiber}
\begin{columns}
\begin{column}{0.6\textwidth}
\only<1,3,5>{
\begin{itemize}
\item<1-> Stores information to allow switching from OCaml stack to the C stack
\item<3-> Constructs the default exception handler
\item<5-> Switches to the OCaml stack and calls \funcname{caml\_program}
\end{itemize}
\bigskip
\listocaml{../src/default/default.ml}{default/default.ml}
}%
\only<2>{
\listasm[firstline=22,lastline=41,highlightlines={25-27}]{src/ocaml/caml\_start\_program.s}{caml\_start\_program}
}%
\only<4>{
\mintasm[firstline=22,lastline=41,highlightlines={31-38}]{src/ocaml/caml\_start\_program.s}
\listasm[firstline=66,lastline=72]{src/ocaml/caml\_start\_program.s}{caml\_start\_program}
}%
\only<6>{
\listasm[firstline=22,lastline=41,highlightlines={39-41}]{src/ocaml/caml\_start\_program.s}{caml\_start\_program}
}%
\end{column}
\begin{column}{0.4\textwidth}
\centering
\only<1-2>{\providecommand\step{2}\input{diagrams/default/default.tex}}%
\only<3-5>{\providecommand\step{3}\input{diagrams/default/default.tex}}%
\onslide*<6->{\providecommand\step{4}\input{diagrams/default/default.tex}}%
\end{column}
\end{columns}
\end{frame}

\begin{frame}{Executing the OCaml program}{And raising an exception without handler}
\begin{columns}
\begin{column}{0.6\textwidth}
\only<1-3>{
\begin{itemize}
\item<1-> Execution continues with OCaml code
\begin{itemize}
\item \funcname{camlDefault\_\_main\_269}
\item \funcname{camlDefault\_\_entry}
\item \funcname{caml\_program}
\end{itemize}
\item<1-> \funcname{camlDefault\_\_main\_269} raises a \typename{Failure} exception
\item<1-> \typename{Failure} is caught by \funcname{caml\_start\_program}
\begin{itemize}
\item<1-> The handler set the 2nd LSB of the exception value to \funcarg{1}
\item<3-> And then forces \funcname{caml\_start\_program} to terminate
\end{itemize}
\end{itemize}
}
\bigskip
\only<1,3>{\listocaml[highlightlines=3]{../src/default/default.ml}{default/default.ml}}%
\only<2>{\listasm[firstline=66,lastline=72,highlightlines=69]{src/ocaml/caml\_start\_program.s}{caml\_start\_program}}%
\end{column}
\begin{column}{0.4\textwidth}
\centering
\providecommand\step{4}\input{diagrams/default/default.tex}
\end{column}
\end{columns}
\end{frame}
%\only<>{\listasm[firstline=47,lastline=65]{src/ocaml/caml\_start\_program.s}{caml\_start\_program}}


\begin{frame}{Back to C}{Clean up, complain and terminate}
\begin{columns}
\begin{column}{0.45\textwidth}
\begin{itemize}
\item Backtrace:
\begin{itemize}
\item \funcname{caml\_start\_program} $\rightarrow$ OCaml code
\item \funcname{caml\_startup\_common}
\item \funcname{caml\_startup\_exn}
\item \funcname{caml\_startup}
\item \funcname{caml\_main}
\item \funcname{main}
\end{itemize}
\smallskip
\item \funcname{caml\_startup} checks the return value of \funcname{caml\_startup\_exn}
\begin{itemize}
\item The return value has been specially marked by \funcname{caml\_start\_program} handler
\item Calls \funcname{caml\_fatal\_uncaught\_exception}
\end{itemize}
\item<2-> \funcname{caml\_fatal\_uncaught\_exception} either
\begin{itemize}
\item<2-> Calls the user custom uncaught exception handler, if set with \mintinline{ocaml}{Printexc.set_uncaught_exception_handler fn}\footnotemark
\item<2-> Or falls back to the default one
\item<2-> Which notifies about the uncaught exception
\end{itemize}
\item<4-> Terminates the program either with \funcname{abort} or with a non-zero exit code
\end{itemize}
\end{column}
\begin{column}{0.55\textwidth}
\only<1>{
\listc[firstline=84,lastline=89,highlightlines=88]{../ocaml/runtime/caml/mlvalues.h}{runtime/caml/mlvalues.h:84}
\listc[firstline=138,lastline=143,highlightlines={141-142}]{../ocaml/runtime/startup\_nat.c}{runtime/startup\_nat.c:138}
}%
\only<2>{
\listc[firstline=138,highlightlines={147,149}]{../ocaml/runtime/printexc.c}{runtime/printexc.c:138}
}%
\only<3>{
\listc[firstline=110,lastline=134,highlightlines=129]{../ocaml/runtime/printexc.c}{runtime/printexc.c:138}
}%
\only<4>{
\listc[firstline=138,highlightlines={152,154}]{../ocaml/runtime/printexc.c}{runtime/printexc.c:138}
}%
\end{column}
\end{columns}
\footnotetext[1]{\url{https://v2.ocaml.org/api/Printexc.html\#1\_Uncaughtexceptions}}
\end{frame}
}%
\only<3-5>{\providecommand\step{3}%
% Runtime's default exception handler
%
\subsection*{Runtime's default exception handler}
\frameSubsection{pictures/The\_Architect.png}{}


\begin{frame}{Default exception handler}
\begin{columns}
\begin{column}{0.5\textwidth}
\begin{itemize}
\item \localname{domain\_state->exn\_handler} points to a trap located before \funcname{entry}!
\item What installed this very first trap there?
\end{itemize}
\bigskip
\listocaml{../src/default/default.ml}{default/default.ml}
\bigskip
\inputminted[fontsize=\footnotesize]{text}{../src/default/default.out}
\end{column}
\begin{column}{0.5\textwidth}
\centering
\only<1>{\providecommand\step{0}\input{diagrams/default/default.tex}}%
\only<2>{\providecommand\step{4}\input{diagrams/default/default.tex}}%
\end{column}
\end{columns}
\end{frame}


\begin{frame}{Startup}{How did we get there by the way?}
\begin{columns}
\begin{column}{0.45\textwidth}
\begin{itemize}
\item The entry point address of the binary is \funcname{main} (\funcname{\_start}) from the runtime
\item The program starts like a C program:
\begin{itemize}
\item \funcname{caml\_start\_program}
\item \funcname{caml\_startup\_common}
\item \funcname{caml\_startup\_exn}
\item \funcname{caml\_startup}
\item \funcname{caml\_main}
\item \funcname{main}
\end{itemize}
%\item \funcname{caml\_startup\_common}: initializes GC, domains \dots
% Also: codefrag, locale, custom opeartions, frame_descr, signals, stack
\item \funcname{caml\_start\_program} is the transition point between the C realm and the OCaml one
\end{itemize}
\bigskip
\listocaml{../src/default/default.ml}{default/default.ml}
\end{column}
\begin{column}{0.55\textwidth}
\listasm[lastline=24,highlightlines={22-24}]{src/ocaml/caml\_start\_program.s}{runtime/amd64.S:604}
\end{column}
\end{columns}
\end{frame}


\begin{frame}{Setup to jump into OCaml entry point}{\funcname{caml\_start\_program} initializes the main fiber}
\begin{columns}
\begin{column}{0.6\textwidth}
\only<1,3,5>{
\begin{itemize}
\item<1-> Stores information to allow switching from OCaml stack to the C stack
\item<3-> Constructs the default exception handler
\item<5-> Switches to the OCaml stack and calls \funcname{caml\_program}
\end{itemize}
\bigskip
\listocaml{../src/default/default.ml}{default/default.ml}
}%
\only<2>{
\listasm[firstline=22,lastline=41,highlightlines={25-27}]{src/ocaml/caml\_start\_program.s}{caml\_start\_program}
}%
\only<4>{
\mintasm[firstline=22,lastline=41,highlightlines={31-38}]{src/ocaml/caml\_start\_program.s}
\listasm[firstline=66,lastline=72]{src/ocaml/caml\_start\_program.s}{caml\_start\_program}
}%
\only<6>{
\listasm[firstline=22,lastline=41,highlightlines={39-41}]{src/ocaml/caml\_start\_program.s}{caml\_start\_program}
}%
\end{column}
\begin{column}{0.4\textwidth}
\centering
\only<1-2>{\providecommand\step{2}\input{diagrams/default/default.tex}}%
\only<3-5>{\providecommand\step{3}\input{diagrams/default/default.tex}}%
\onslide*<6->{\providecommand\step{4}\input{diagrams/default/default.tex}}%
\end{column}
\end{columns}
\end{frame}

\begin{frame}{Executing the OCaml program}{And raising an exception without handler}
\begin{columns}
\begin{column}{0.6\textwidth}
\only<1-3>{
\begin{itemize}
\item<1-> Execution continues with OCaml code
\begin{itemize}
\item \funcname{camlDefault\_\_main\_269}
\item \funcname{camlDefault\_\_entry}
\item \funcname{caml\_program}
\end{itemize}
\item<1-> \funcname{camlDefault\_\_main\_269} raises a \typename{Failure} exception
\item<1-> \typename{Failure} is caught by \funcname{caml\_start\_program}
\begin{itemize}
\item<1-> The handler set the 2nd LSB of the exception value to \funcarg{1}
\item<3-> And then forces \funcname{caml\_start\_program} to terminate
\end{itemize}
\end{itemize}
}
\bigskip
\only<1,3>{\listocaml[highlightlines=3]{../src/default/default.ml}{default/default.ml}}%
\only<2>{\listasm[firstline=66,lastline=72,highlightlines=69]{src/ocaml/caml\_start\_program.s}{caml\_start\_program}}%
\end{column}
\begin{column}{0.4\textwidth}
\centering
\providecommand\step{4}\input{diagrams/default/default.tex}
\end{column}
\end{columns}
\end{frame}
%\only<>{\listasm[firstline=47,lastline=65]{src/ocaml/caml\_start\_program.s}{caml\_start\_program}}


\begin{frame}{Back to C}{Clean up, complain and terminate}
\begin{columns}
\begin{column}{0.45\textwidth}
\begin{itemize}
\item Backtrace:
\begin{itemize}
\item \funcname{caml\_start\_program} $\rightarrow$ OCaml code
\item \funcname{caml\_startup\_common}
\item \funcname{caml\_startup\_exn}
\item \funcname{caml\_startup}
\item \funcname{caml\_main}
\item \funcname{main}
\end{itemize}
\smallskip
\item \funcname{caml\_startup} checks the return value of \funcname{caml\_startup\_exn}
\begin{itemize}
\item The return value has been specially marked by \funcname{caml\_start\_program} handler
\item Calls \funcname{caml\_fatal\_uncaught\_exception}
\end{itemize}
\item<2-> \funcname{caml\_fatal\_uncaught\_exception} either
\begin{itemize}
\item<2-> Calls the user custom uncaught exception handler, if set with \mintinline{ocaml}{Printexc.set_uncaught_exception_handler fn}\footnotemark
\item<2-> Or falls back to the default one
\item<2-> Which notifies about the uncaught exception
\end{itemize}
\item<4-> Terminates the program either with \funcname{abort} or with a non-zero exit code
\end{itemize}
\end{column}
\begin{column}{0.55\textwidth}
\only<1>{
\listc[firstline=84,lastline=89,highlightlines=88]{../ocaml/runtime/caml/mlvalues.h}{runtime/caml/mlvalues.h:84}
\listc[firstline=138,lastline=143,highlightlines={141-142}]{../ocaml/runtime/startup\_nat.c}{runtime/startup\_nat.c:138}
}%
\only<2>{
\listc[firstline=138,highlightlines={147,149}]{../ocaml/runtime/printexc.c}{runtime/printexc.c:138}
}%
\only<3>{
\listc[firstline=110,lastline=134,highlightlines=129]{../ocaml/runtime/printexc.c}{runtime/printexc.c:138}
}%
\only<4>{
\listc[firstline=138,highlightlines={152,154}]{../ocaml/runtime/printexc.c}{runtime/printexc.c:138}
}%
\end{column}
\end{columns}
\footnotetext[1]{\url{https://v2.ocaml.org/api/Printexc.html\#1\_Uncaughtexceptions}}
\end{frame}
}%
\onslide*<6->{\providecommand\step{4}%
% Runtime's default exception handler
%
\subsection*{Runtime's default exception handler}
\frameSubsection{pictures/The\_Architect.png}{}


\begin{frame}{Default exception handler}
\begin{columns}
\begin{column}{0.5\textwidth}
\begin{itemize}
\item \localname{domain\_state->exn\_handler} points to a trap located before \funcname{entry}!
\item What installed this very first trap there?
\end{itemize}
\bigskip
\listocaml{../src/default/default.ml}{default/default.ml}
\bigskip
\inputminted[fontsize=\footnotesize]{text}{../src/default/default.out}
\end{column}
\begin{column}{0.5\textwidth}
\centering
\only<1>{\providecommand\step{0}\input{diagrams/default/default.tex}}%
\only<2>{\providecommand\step{4}\input{diagrams/default/default.tex}}%
\end{column}
\end{columns}
\end{frame}


\begin{frame}{Startup}{How did we get there by the way?}
\begin{columns}
\begin{column}{0.45\textwidth}
\begin{itemize}
\item The entry point address of the binary is \funcname{main} (\funcname{\_start}) from the runtime
\item The program starts like a C program:
\begin{itemize}
\item \funcname{caml\_start\_program}
\item \funcname{caml\_startup\_common}
\item \funcname{caml\_startup\_exn}
\item \funcname{caml\_startup}
\item \funcname{caml\_main}
\item \funcname{main}
\end{itemize}
%\item \funcname{caml\_startup\_common}: initializes GC, domains \dots
% Also: codefrag, locale, custom opeartions, frame_descr, signals, stack
\item \funcname{caml\_start\_program} is the transition point between the C realm and the OCaml one
\end{itemize}
\bigskip
\listocaml{../src/default/default.ml}{default/default.ml}
\end{column}
\begin{column}{0.55\textwidth}
\listasm[lastline=24,highlightlines={22-24}]{src/ocaml/caml\_start\_program.s}{runtime/amd64.S:604}
\end{column}
\end{columns}
\end{frame}


\begin{frame}{Setup to jump into OCaml entry point}{\funcname{caml\_start\_program} initializes the main fiber}
\begin{columns}
\begin{column}{0.6\textwidth}
\only<1,3,5>{
\begin{itemize}
\item<1-> Stores information to allow switching from OCaml stack to the C stack
\item<3-> Constructs the default exception handler
\item<5-> Switches to the OCaml stack and calls \funcname{caml\_program}
\end{itemize}
\bigskip
\listocaml{../src/default/default.ml}{default/default.ml}
}%
\only<2>{
\listasm[firstline=22,lastline=41,highlightlines={25-27}]{src/ocaml/caml\_start\_program.s}{caml\_start\_program}
}%
\only<4>{
\mintasm[firstline=22,lastline=41,highlightlines={31-38}]{src/ocaml/caml\_start\_program.s}
\listasm[firstline=66,lastline=72]{src/ocaml/caml\_start\_program.s}{caml\_start\_program}
}%
\only<6>{
\listasm[firstline=22,lastline=41,highlightlines={39-41}]{src/ocaml/caml\_start\_program.s}{caml\_start\_program}
}%
\end{column}
\begin{column}{0.4\textwidth}
\centering
\only<1-2>{\providecommand\step{2}\input{diagrams/default/default.tex}}%
\only<3-5>{\providecommand\step{3}\input{diagrams/default/default.tex}}%
\onslide*<6->{\providecommand\step{4}\input{diagrams/default/default.tex}}%
\end{column}
\end{columns}
\end{frame}

\begin{frame}{Executing the OCaml program}{And raising an exception without handler}
\begin{columns}
\begin{column}{0.6\textwidth}
\only<1-3>{
\begin{itemize}
\item<1-> Execution continues with OCaml code
\begin{itemize}
\item \funcname{camlDefault\_\_main\_269}
\item \funcname{camlDefault\_\_entry}
\item \funcname{caml\_program}
\end{itemize}
\item<1-> \funcname{camlDefault\_\_main\_269} raises a \typename{Failure} exception
\item<1-> \typename{Failure} is caught by \funcname{caml\_start\_program}
\begin{itemize}
\item<1-> The handler set the 2nd LSB of the exception value to \funcarg{1}
\item<3-> And then forces \funcname{caml\_start\_program} to terminate
\end{itemize}
\end{itemize}
}
\bigskip
\only<1,3>{\listocaml[highlightlines=3]{../src/default/default.ml}{default/default.ml}}%
\only<2>{\listasm[firstline=66,lastline=72,highlightlines=69]{src/ocaml/caml\_start\_program.s}{caml\_start\_program}}%
\end{column}
\begin{column}{0.4\textwidth}
\centering
\providecommand\step{4}\input{diagrams/default/default.tex}
\end{column}
\end{columns}
\end{frame}
%\only<>{\listasm[firstline=47,lastline=65]{src/ocaml/caml\_start\_program.s}{caml\_start\_program}}


\begin{frame}{Back to C}{Clean up, complain and terminate}
\begin{columns}
\begin{column}{0.45\textwidth}
\begin{itemize}
\item Backtrace:
\begin{itemize}
\item \funcname{caml\_start\_program} $\rightarrow$ OCaml code
\item \funcname{caml\_startup\_common}
\item \funcname{caml\_startup\_exn}
\item \funcname{caml\_startup}
\item \funcname{caml\_main}
\item \funcname{main}
\end{itemize}
\smallskip
\item \funcname{caml\_startup} checks the return value of \funcname{caml\_startup\_exn}
\begin{itemize}
\item The return value has been specially marked by \funcname{caml\_start\_program} handler
\item Calls \funcname{caml\_fatal\_uncaught\_exception}
\end{itemize}
\item<2-> \funcname{caml\_fatal\_uncaught\_exception} either
\begin{itemize}
\item<2-> Calls the user custom uncaught exception handler, if set with \mintinline{ocaml}{Printexc.set_uncaught_exception_handler fn}\footnotemark
\item<2-> Or falls back to the default one
\item<2-> Which notifies about the uncaught exception
\end{itemize}
\item<4-> Terminates the program either with \funcname{abort} or with a non-zero exit code
\end{itemize}
\end{column}
\begin{column}{0.55\textwidth}
\only<1>{
\listc[firstline=84,lastline=89,highlightlines=88]{../ocaml/runtime/caml/mlvalues.h}{runtime/caml/mlvalues.h:84}
\listc[firstline=138,lastline=143,highlightlines={141-142}]{../ocaml/runtime/startup\_nat.c}{runtime/startup\_nat.c:138}
}%
\only<2>{
\listc[firstline=138,highlightlines={147,149}]{../ocaml/runtime/printexc.c}{runtime/printexc.c:138}
}%
\only<3>{
\listc[firstline=110,lastline=134,highlightlines=129]{../ocaml/runtime/printexc.c}{runtime/printexc.c:138}
}%
\only<4>{
\listc[firstline=138,highlightlines={152,154}]{../ocaml/runtime/printexc.c}{runtime/printexc.c:138}
}%
\end{column}
\end{columns}
\footnotetext[1]{\url{https://v2.ocaml.org/api/Printexc.html\#1\_Uncaughtexceptions}}
\end{frame}
}%
\end{column}
\end{columns}
\end{frame}

\begin{frame}{Executing the OCaml program}{And raising an exception without handler}
\begin{columns}
\begin{column}{0.6\textwidth}
\only<1-3>{
\begin{itemize}
\item<1-> Execution continues with OCaml code
\begin{itemize}
\item \funcname{camlDefault\_\_main\_269}
\item \funcname{camlDefault\_\_entry}
\item \funcname{caml\_program}
\end{itemize}
\item<1-> \funcname{camlDefault\_\_main\_269} raises a \typename{Failure} exception
\item<1-> \typename{Failure} is caught by \funcname{caml\_start\_program}
\begin{itemize}
\item<1-> The handler set the 2nd LSB of the exception value to \funcarg{1}
\item<3-> And then forces \funcname{caml\_start\_program} to terminate
\end{itemize}
\end{itemize}
}
\bigskip
\only<1,3>{\listocaml[highlightlines=3]{../src/default/default.ml}{default/default.ml}}%
\only<2>{\listasm[firstline=66,lastline=72,highlightlines=69]{src/ocaml/caml\_start\_program.s}{caml\_start\_program}}%
\end{column}
\begin{column}{0.4\textwidth}
\centering
\providecommand\step{4}%
% Runtime's default exception handler
%
\subsection*{Runtime's default exception handler}
\frameSubsection{pictures/The\_Architect.png}{}


\begin{frame}{Default exception handler}
\begin{columns}
\begin{column}{0.5\textwidth}
\begin{itemize}
\item \localname{domain\_state->exn\_handler} points to a trap located before \funcname{entry}!
\item What installed this very first trap there?
\end{itemize}
\bigskip
\listocaml{../src/default/default.ml}{default/default.ml}
\bigskip
\inputminted[fontsize=\footnotesize]{text}{../src/default/default.out}
\end{column}
\begin{column}{0.5\textwidth}
\centering
\only<1>{\providecommand\step{0}\input{diagrams/default/default.tex}}%
\only<2>{\providecommand\step{4}\input{diagrams/default/default.tex}}%
\end{column}
\end{columns}
\end{frame}


\begin{frame}{Startup}{How did we get there by the way?}
\begin{columns}
\begin{column}{0.45\textwidth}
\begin{itemize}
\item The entry point address of the binary is \funcname{main} (\funcname{\_start}) from the runtime
\item The program starts like a C program:
\begin{itemize}
\item \funcname{caml\_start\_program}
\item \funcname{caml\_startup\_common}
\item \funcname{caml\_startup\_exn}
\item \funcname{caml\_startup}
\item \funcname{caml\_main}
\item \funcname{main}
\end{itemize}
%\item \funcname{caml\_startup\_common}: initializes GC, domains \dots
% Also: codefrag, locale, custom opeartions, frame_descr, signals, stack
\item \funcname{caml\_start\_program} is the transition point between the C realm and the OCaml one
\end{itemize}
\bigskip
\listocaml{../src/default/default.ml}{default/default.ml}
\end{column}
\begin{column}{0.55\textwidth}
\listasm[lastline=24,highlightlines={22-24}]{src/ocaml/caml\_start\_program.s}{runtime/amd64.S:604}
\end{column}
\end{columns}
\end{frame}


\begin{frame}{Setup to jump into OCaml entry point}{\funcname{caml\_start\_program} initializes the main fiber}
\begin{columns}
\begin{column}{0.6\textwidth}
\only<1,3,5>{
\begin{itemize}
\item<1-> Stores information to allow switching from OCaml stack to the C stack
\item<3-> Constructs the default exception handler
\item<5-> Switches to the OCaml stack and calls \funcname{caml\_program}
\end{itemize}
\bigskip
\listocaml{../src/default/default.ml}{default/default.ml}
}%
\only<2>{
\listasm[firstline=22,lastline=41,highlightlines={25-27}]{src/ocaml/caml\_start\_program.s}{caml\_start\_program}
}%
\only<4>{
\mintasm[firstline=22,lastline=41,highlightlines={31-38}]{src/ocaml/caml\_start\_program.s}
\listasm[firstline=66,lastline=72]{src/ocaml/caml\_start\_program.s}{caml\_start\_program}
}%
\only<6>{
\listasm[firstline=22,lastline=41,highlightlines={39-41}]{src/ocaml/caml\_start\_program.s}{caml\_start\_program}
}%
\end{column}
\begin{column}{0.4\textwidth}
\centering
\only<1-2>{\providecommand\step{2}\input{diagrams/default/default.tex}}%
\only<3-5>{\providecommand\step{3}\input{diagrams/default/default.tex}}%
\onslide*<6->{\providecommand\step{4}\input{diagrams/default/default.tex}}%
\end{column}
\end{columns}
\end{frame}

\begin{frame}{Executing the OCaml program}{And raising an exception without handler}
\begin{columns}
\begin{column}{0.6\textwidth}
\only<1-3>{
\begin{itemize}
\item<1-> Execution continues with OCaml code
\begin{itemize}
\item \funcname{camlDefault\_\_main\_269}
\item \funcname{camlDefault\_\_entry}
\item \funcname{caml\_program}
\end{itemize}
\item<1-> \funcname{camlDefault\_\_main\_269} raises a \typename{Failure} exception
\item<1-> \typename{Failure} is caught by \funcname{caml\_start\_program}
\begin{itemize}
\item<1-> The handler set the 2nd LSB of the exception value to \funcarg{1}
\item<3-> And then forces \funcname{caml\_start\_program} to terminate
\end{itemize}
\end{itemize}
}
\bigskip
\only<1,3>{\listocaml[highlightlines=3]{../src/default/default.ml}{default/default.ml}}%
\only<2>{\listasm[firstline=66,lastline=72,highlightlines=69]{src/ocaml/caml\_start\_program.s}{caml\_start\_program}}%
\end{column}
\begin{column}{0.4\textwidth}
\centering
\providecommand\step{4}\input{diagrams/default/default.tex}
\end{column}
\end{columns}
\end{frame}
%\only<>{\listasm[firstline=47,lastline=65]{src/ocaml/caml\_start\_program.s}{caml\_start\_program}}


\begin{frame}{Back to C}{Clean up, complain and terminate}
\begin{columns}
\begin{column}{0.45\textwidth}
\begin{itemize}
\item Backtrace:
\begin{itemize}
\item \funcname{caml\_start\_program} $\rightarrow$ OCaml code
\item \funcname{caml\_startup\_common}
\item \funcname{caml\_startup\_exn}
\item \funcname{caml\_startup}
\item \funcname{caml\_main}
\item \funcname{main}
\end{itemize}
\smallskip
\item \funcname{caml\_startup} checks the return value of \funcname{caml\_startup\_exn}
\begin{itemize}
\item The return value has been specially marked by \funcname{caml\_start\_program} handler
\item Calls \funcname{caml\_fatal\_uncaught\_exception}
\end{itemize}
\item<2-> \funcname{caml\_fatal\_uncaught\_exception} either
\begin{itemize}
\item<2-> Calls the user custom uncaught exception handler, if set with \mintinline{ocaml}{Printexc.set_uncaught_exception_handler fn}\footnotemark
\item<2-> Or falls back to the default one
\item<2-> Which notifies about the uncaught exception
\end{itemize}
\item<4-> Terminates the program either with \funcname{abort} or with a non-zero exit code
\end{itemize}
\end{column}
\begin{column}{0.55\textwidth}
\only<1>{
\listc[firstline=84,lastline=89,highlightlines=88]{../ocaml/runtime/caml/mlvalues.h}{runtime/caml/mlvalues.h:84}
\listc[firstline=138,lastline=143,highlightlines={141-142}]{../ocaml/runtime/startup\_nat.c}{runtime/startup\_nat.c:138}
}%
\only<2>{
\listc[firstline=138,highlightlines={147,149}]{../ocaml/runtime/printexc.c}{runtime/printexc.c:138}
}%
\only<3>{
\listc[firstline=110,lastline=134,highlightlines=129]{../ocaml/runtime/printexc.c}{runtime/printexc.c:138}
}%
\only<4>{
\listc[firstline=138,highlightlines={152,154}]{../ocaml/runtime/printexc.c}{runtime/printexc.c:138}
}%
\end{column}
\end{columns}
\footnotetext[1]{\url{https://v2.ocaml.org/api/Printexc.html\#1\_Uncaughtexceptions}}
\end{frame}

\end{column}
\end{columns}
\end{frame}
%\only<>{\listasm[firstline=47,lastline=65]{src/ocaml/caml\_start\_program.s}{caml\_start\_program}}


\begin{frame}{Back to C}{Clean up, complain and terminate}
\begin{columns}
\begin{column}{0.45\textwidth}
\begin{itemize}
\item Backtrace:
\begin{itemize}
\item \funcname{caml\_start\_program} $\rightarrow$ OCaml code
\item \funcname{caml\_startup\_common}
\item \funcname{caml\_startup\_exn}
\item \funcname{caml\_startup}
\item \funcname{caml\_main}
\item \funcname{main}
\end{itemize}
\smallskip
\item \funcname{caml\_startup} checks the return value of \funcname{caml\_startup\_exn}
\begin{itemize}
\item The return value has been specially marked by \funcname{caml\_start\_program} handler
\item Calls \funcname{caml\_fatal\_uncaught\_exception}
\end{itemize}
\item<2-> \funcname{caml\_fatal\_uncaught\_exception} either
\begin{itemize}
\item<2-> Calls the user custom uncaught exception handler, if set with \mintinline{ocaml}{Printexc.set_uncaught_exception_handler fn}\footnotemark
\item<2-> Or falls back to the default one
\item<2-> Which notifies about the uncaught exception
\end{itemize}
\item<4-> Terminates the program either with \funcname{abort} or with a non-zero exit code
\end{itemize}
\end{column}
\begin{column}{0.55\textwidth}
\only<1>{
\listc[firstline=84,lastline=89,highlightlines=88]{../ocaml/runtime/caml/mlvalues.h}{runtime/caml/mlvalues.h:84}
\listc[firstline=138,lastline=143,highlightlines={141-142}]{../ocaml/runtime/startup\_nat.c}{runtime/startup\_nat.c:138}
}%
\only<2>{
\listc[firstline=138,highlightlines={147,149}]{../ocaml/runtime/printexc.c}{runtime/printexc.c:138}
}%
\only<3>{
\listc[firstline=110,lastline=134,highlightlines=129]{../ocaml/runtime/printexc.c}{runtime/printexc.c:138}
}%
\only<4>{
\listc[firstline=138,highlightlines={152,154}]{../ocaml/runtime/printexc.c}{runtime/printexc.c:138}
}%
\end{column}
\end{columns}
\footnotetext[1]{\url{https://v2.ocaml.org/api/Printexc.html\#1\_Uncaughtexceptions}}
\end{frame}
}%
\end{column}
\end{columns}
\end{frame}


\begin{frame}{Startup}{How did we get there by the way?}
\begin{columns}
\begin{column}{0.45\textwidth}
\begin{itemize}
\item The entry point address of the binary is \funcname{main} (\funcname{\_start}) from the runtime
\item The program starts like a C program:
\begin{itemize}
\item \funcname{caml\_start\_program}
\item \funcname{caml\_startup\_common}
\item \funcname{caml\_startup\_exn}
\item \funcname{caml\_startup}
\item \funcname{caml\_main}
\item \funcname{main}
\end{itemize}
%\item \funcname{caml\_startup\_common}: initializes GC, domains \dots
% Also: codefrag, locale, custom opeartions, frame_descr, signals, stack
\item \funcname{caml\_start\_program} is the transition point between the C realm and the OCaml one
\end{itemize}
\bigskip
\listocaml{../src/default/default.ml}{default/default.ml}
\end{column}
\begin{column}{0.55\textwidth}
\listasm[lastline=24,highlightlines={22-24}]{src/ocaml/caml\_start\_program.s}{runtime/amd64.S:604}
\end{column}
\end{columns}
\end{frame}


\begin{frame}{Setup to jump into OCaml entry point}{\funcname{caml\_start\_program} initializes the main fiber}
\begin{columns}
\begin{column}{0.6\textwidth}
\only<1,3,5>{
\begin{itemize}
\item<1-> Stores information to allow switching from OCaml stack to the C stack
\item<3-> Constructs the default exception handler
\item<5-> Switches to the OCaml stack and calls \funcname{caml\_program}
\end{itemize}
\bigskip
\listocaml{../src/default/default.ml}{default/default.ml}
}%
\only<2>{
\listasm[firstline=22,lastline=41,highlightlines={25-27}]{src/ocaml/caml\_start\_program.s}{caml\_start\_program}
}%
\only<4>{
\mintasm[firstline=22,lastline=41,highlightlines={31-38}]{src/ocaml/caml\_start\_program.s}
\listasm[firstline=66,lastline=72]{src/ocaml/caml\_start\_program.s}{caml\_start\_program}
}%
\only<6>{
\listasm[firstline=22,lastline=41,highlightlines={39-41}]{src/ocaml/caml\_start\_program.s}{caml\_start\_program}
}%
\end{column}
\begin{column}{0.4\textwidth}
\centering
\only<1-2>{\providecommand\step{2}%
% Runtime's default exception handler
%
\subsection*{Runtime's default exception handler}
\frameSubsection{pictures/The\_Architect.png}{}


\begin{frame}{Default exception handler}
\begin{columns}
\begin{column}{0.5\textwidth}
\begin{itemize}
\item \localname{domain\_state->exn\_handler} points to a trap located before \funcname{entry}!
\item What installed this very first trap there?
\end{itemize}
\bigskip
\listocaml{../src/default/default.ml}{default/default.ml}
\bigskip
\inputminted[fontsize=\footnotesize]{text}{../src/default/default.out}
\end{column}
\begin{column}{0.5\textwidth}
\centering
\only<1>{\providecommand\step{0}%
% Runtime's default exception handler
%
\subsection*{Runtime's default exception handler}
\frameSubsection{pictures/The\_Architect.png}{}


\begin{frame}{Default exception handler}
\begin{columns}
\begin{column}{0.5\textwidth}
\begin{itemize}
\item \localname{domain\_state->exn\_handler} points to a trap located before \funcname{entry}!
\item What installed this very first trap there?
\end{itemize}
\bigskip
\listocaml{../src/default/default.ml}{default/default.ml}
\bigskip
\inputminted[fontsize=\footnotesize]{text}{../src/default/default.out}
\end{column}
\begin{column}{0.5\textwidth}
\centering
\only<1>{\providecommand\step{0}\input{diagrams/default/default.tex}}%
\only<2>{\providecommand\step{4}\input{diagrams/default/default.tex}}%
\end{column}
\end{columns}
\end{frame}


\begin{frame}{Startup}{How did we get there by the way?}
\begin{columns}
\begin{column}{0.45\textwidth}
\begin{itemize}
\item The entry point address of the binary is \funcname{main} (\funcname{\_start}) from the runtime
\item The program starts like a C program:
\begin{itemize}
\item \funcname{caml\_start\_program}
\item \funcname{caml\_startup\_common}
\item \funcname{caml\_startup\_exn}
\item \funcname{caml\_startup}
\item \funcname{caml\_main}
\item \funcname{main}
\end{itemize}
%\item \funcname{caml\_startup\_common}: initializes GC, domains \dots
% Also: codefrag, locale, custom opeartions, frame_descr, signals, stack
\item \funcname{caml\_start\_program} is the transition point between the C realm and the OCaml one
\end{itemize}
\bigskip
\listocaml{../src/default/default.ml}{default/default.ml}
\end{column}
\begin{column}{0.55\textwidth}
\listasm[lastline=24,highlightlines={22-24}]{src/ocaml/caml\_start\_program.s}{runtime/amd64.S:604}
\end{column}
\end{columns}
\end{frame}


\begin{frame}{Setup to jump into OCaml entry point}{\funcname{caml\_start\_program} initializes the main fiber}
\begin{columns}
\begin{column}{0.6\textwidth}
\only<1,3,5>{
\begin{itemize}
\item<1-> Stores information to allow switching from OCaml stack to the C stack
\item<3-> Constructs the default exception handler
\item<5-> Switches to the OCaml stack and calls \funcname{caml\_program}
\end{itemize}
\bigskip
\listocaml{../src/default/default.ml}{default/default.ml}
}%
\only<2>{
\listasm[firstline=22,lastline=41,highlightlines={25-27}]{src/ocaml/caml\_start\_program.s}{caml\_start\_program}
}%
\only<4>{
\mintasm[firstline=22,lastline=41,highlightlines={31-38}]{src/ocaml/caml\_start\_program.s}
\listasm[firstline=66,lastline=72]{src/ocaml/caml\_start\_program.s}{caml\_start\_program}
}%
\only<6>{
\listasm[firstline=22,lastline=41,highlightlines={39-41}]{src/ocaml/caml\_start\_program.s}{caml\_start\_program}
}%
\end{column}
\begin{column}{0.4\textwidth}
\centering
\only<1-2>{\providecommand\step{2}\input{diagrams/default/default.tex}}%
\only<3-5>{\providecommand\step{3}\input{diagrams/default/default.tex}}%
\onslide*<6->{\providecommand\step{4}\input{diagrams/default/default.tex}}%
\end{column}
\end{columns}
\end{frame}

\begin{frame}{Executing the OCaml program}{And raising an exception without handler}
\begin{columns}
\begin{column}{0.6\textwidth}
\only<1-3>{
\begin{itemize}
\item<1-> Execution continues with OCaml code
\begin{itemize}
\item \funcname{camlDefault\_\_main\_269}
\item \funcname{camlDefault\_\_entry}
\item \funcname{caml\_program}
\end{itemize}
\item<1-> \funcname{camlDefault\_\_main\_269} raises a \typename{Failure} exception
\item<1-> \typename{Failure} is caught by \funcname{caml\_start\_program}
\begin{itemize}
\item<1-> The handler set the 2nd LSB of the exception value to \funcarg{1}
\item<3-> And then forces \funcname{caml\_start\_program} to terminate
\end{itemize}
\end{itemize}
}
\bigskip
\only<1,3>{\listocaml[highlightlines=3]{../src/default/default.ml}{default/default.ml}}%
\only<2>{\listasm[firstline=66,lastline=72,highlightlines=69]{src/ocaml/caml\_start\_program.s}{caml\_start\_program}}%
\end{column}
\begin{column}{0.4\textwidth}
\centering
\providecommand\step{4}\input{diagrams/default/default.tex}
\end{column}
\end{columns}
\end{frame}
%\only<>{\listasm[firstline=47,lastline=65]{src/ocaml/caml\_start\_program.s}{caml\_start\_program}}


\begin{frame}{Back to C}{Clean up, complain and terminate}
\begin{columns}
\begin{column}{0.45\textwidth}
\begin{itemize}
\item Backtrace:
\begin{itemize}
\item \funcname{caml\_start\_program} $\rightarrow$ OCaml code
\item \funcname{caml\_startup\_common}
\item \funcname{caml\_startup\_exn}
\item \funcname{caml\_startup}
\item \funcname{caml\_main}
\item \funcname{main}
\end{itemize}
\smallskip
\item \funcname{caml\_startup} checks the return value of \funcname{caml\_startup\_exn}
\begin{itemize}
\item The return value has been specially marked by \funcname{caml\_start\_program} handler
\item Calls \funcname{caml\_fatal\_uncaught\_exception}
\end{itemize}
\item<2-> \funcname{caml\_fatal\_uncaught\_exception} either
\begin{itemize}
\item<2-> Calls the user custom uncaught exception handler, if set with \mintinline{ocaml}{Printexc.set_uncaught_exception_handler fn}\footnotemark
\item<2-> Or falls back to the default one
\item<2-> Which notifies about the uncaught exception
\end{itemize}
\item<4-> Terminates the program either with \funcname{abort} or with a non-zero exit code
\end{itemize}
\end{column}
\begin{column}{0.55\textwidth}
\only<1>{
\listc[firstline=84,lastline=89,highlightlines=88]{../ocaml/runtime/caml/mlvalues.h}{runtime/caml/mlvalues.h:84}
\listc[firstline=138,lastline=143,highlightlines={141-142}]{../ocaml/runtime/startup\_nat.c}{runtime/startup\_nat.c:138}
}%
\only<2>{
\listc[firstline=138,highlightlines={147,149}]{../ocaml/runtime/printexc.c}{runtime/printexc.c:138}
}%
\only<3>{
\listc[firstline=110,lastline=134,highlightlines=129]{../ocaml/runtime/printexc.c}{runtime/printexc.c:138}
}%
\only<4>{
\listc[firstline=138,highlightlines={152,154}]{../ocaml/runtime/printexc.c}{runtime/printexc.c:138}
}%
\end{column}
\end{columns}
\footnotetext[1]{\url{https://v2.ocaml.org/api/Printexc.html\#1\_Uncaughtexceptions}}
\end{frame}
}%
\only<2>{\providecommand\step{4}%
% Runtime's default exception handler
%
\subsection*{Runtime's default exception handler}
\frameSubsection{pictures/The\_Architect.png}{}


\begin{frame}{Default exception handler}
\begin{columns}
\begin{column}{0.5\textwidth}
\begin{itemize}
\item \localname{domain\_state->exn\_handler} points to a trap located before \funcname{entry}!
\item What installed this very first trap there?
\end{itemize}
\bigskip
\listocaml{../src/default/default.ml}{default/default.ml}
\bigskip
\inputminted[fontsize=\footnotesize]{text}{../src/default/default.out}
\end{column}
\begin{column}{0.5\textwidth}
\centering
\only<1>{\providecommand\step{0}\input{diagrams/default/default.tex}}%
\only<2>{\providecommand\step{4}\input{diagrams/default/default.tex}}%
\end{column}
\end{columns}
\end{frame}


\begin{frame}{Startup}{How did we get there by the way?}
\begin{columns}
\begin{column}{0.45\textwidth}
\begin{itemize}
\item The entry point address of the binary is \funcname{main} (\funcname{\_start}) from the runtime
\item The program starts like a C program:
\begin{itemize}
\item \funcname{caml\_start\_program}
\item \funcname{caml\_startup\_common}
\item \funcname{caml\_startup\_exn}
\item \funcname{caml\_startup}
\item \funcname{caml\_main}
\item \funcname{main}
\end{itemize}
%\item \funcname{caml\_startup\_common}: initializes GC, domains \dots
% Also: codefrag, locale, custom opeartions, frame_descr, signals, stack
\item \funcname{caml\_start\_program} is the transition point between the C realm and the OCaml one
\end{itemize}
\bigskip
\listocaml{../src/default/default.ml}{default/default.ml}
\end{column}
\begin{column}{0.55\textwidth}
\listasm[lastline=24,highlightlines={22-24}]{src/ocaml/caml\_start\_program.s}{runtime/amd64.S:604}
\end{column}
\end{columns}
\end{frame}


\begin{frame}{Setup to jump into OCaml entry point}{\funcname{caml\_start\_program} initializes the main fiber}
\begin{columns}
\begin{column}{0.6\textwidth}
\only<1,3,5>{
\begin{itemize}
\item<1-> Stores information to allow switching from OCaml stack to the C stack
\item<3-> Constructs the default exception handler
\item<5-> Switches to the OCaml stack and calls \funcname{caml\_program}
\end{itemize}
\bigskip
\listocaml{../src/default/default.ml}{default/default.ml}
}%
\only<2>{
\listasm[firstline=22,lastline=41,highlightlines={25-27}]{src/ocaml/caml\_start\_program.s}{caml\_start\_program}
}%
\only<4>{
\mintasm[firstline=22,lastline=41,highlightlines={31-38}]{src/ocaml/caml\_start\_program.s}
\listasm[firstline=66,lastline=72]{src/ocaml/caml\_start\_program.s}{caml\_start\_program}
}%
\only<6>{
\listasm[firstline=22,lastline=41,highlightlines={39-41}]{src/ocaml/caml\_start\_program.s}{caml\_start\_program}
}%
\end{column}
\begin{column}{0.4\textwidth}
\centering
\only<1-2>{\providecommand\step{2}\input{diagrams/default/default.tex}}%
\only<3-5>{\providecommand\step{3}\input{diagrams/default/default.tex}}%
\onslide*<6->{\providecommand\step{4}\input{diagrams/default/default.tex}}%
\end{column}
\end{columns}
\end{frame}

\begin{frame}{Executing the OCaml program}{And raising an exception without handler}
\begin{columns}
\begin{column}{0.6\textwidth}
\only<1-3>{
\begin{itemize}
\item<1-> Execution continues with OCaml code
\begin{itemize}
\item \funcname{camlDefault\_\_main\_269}
\item \funcname{camlDefault\_\_entry}
\item \funcname{caml\_program}
\end{itemize}
\item<1-> \funcname{camlDefault\_\_main\_269} raises a \typename{Failure} exception
\item<1-> \typename{Failure} is caught by \funcname{caml\_start\_program}
\begin{itemize}
\item<1-> The handler set the 2nd LSB of the exception value to \funcarg{1}
\item<3-> And then forces \funcname{caml\_start\_program} to terminate
\end{itemize}
\end{itemize}
}
\bigskip
\only<1,3>{\listocaml[highlightlines=3]{../src/default/default.ml}{default/default.ml}}%
\only<2>{\listasm[firstline=66,lastline=72,highlightlines=69]{src/ocaml/caml\_start\_program.s}{caml\_start\_program}}%
\end{column}
\begin{column}{0.4\textwidth}
\centering
\providecommand\step{4}\input{diagrams/default/default.tex}
\end{column}
\end{columns}
\end{frame}
%\only<>{\listasm[firstline=47,lastline=65]{src/ocaml/caml\_start\_program.s}{caml\_start\_program}}


\begin{frame}{Back to C}{Clean up, complain and terminate}
\begin{columns}
\begin{column}{0.45\textwidth}
\begin{itemize}
\item Backtrace:
\begin{itemize}
\item \funcname{caml\_start\_program} $\rightarrow$ OCaml code
\item \funcname{caml\_startup\_common}
\item \funcname{caml\_startup\_exn}
\item \funcname{caml\_startup}
\item \funcname{caml\_main}
\item \funcname{main}
\end{itemize}
\smallskip
\item \funcname{caml\_startup} checks the return value of \funcname{caml\_startup\_exn}
\begin{itemize}
\item The return value has been specially marked by \funcname{caml\_start\_program} handler
\item Calls \funcname{caml\_fatal\_uncaught\_exception}
\end{itemize}
\item<2-> \funcname{caml\_fatal\_uncaught\_exception} either
\begin{itemize}
\item<2-> Calls the user custom uncaught exception handler, if set with \mintinline{ocaml}{Printexc.set_uncaught_exception_handler fn}\footnotemark
\item<2-> Or falls back to the default one
\item<2-> Which notifies about the uncaught exception
\end{itemize}
\item<4-> Terminates the program either with \funcname{abort} or with a non-zero exit code
\end{itemize}
\end{column}
\begin{column}{0.55\textwidth}
\only<1>{
\listc[firstline=84,lastline=89,highlightlines=88]{../ocaml/runtime/caml/mlvalues.h}{runtime/caml/mlvalues.h:84}
\listc[firstline=138,lastline=143,highlightlines={141-142}]{../ocaml/runtime/startup\_nat.c}{runtime/startup\_nat.c:138}
}%
\only<2>{
\listc[firstline=138,highlightlines={147,149}]{../ocaml/runtime/printexc.c}{runtime/printexc.c:138}
}%
\only<3>{
\listc[firstline=110,lastline=134,highlightlines=129]{../ocaml/runtime/printexc.c}{runtime/printexc.c:138}
}%
\only<4>{
\listc[firstline=138,highlightlines={152,154}]{../ocaml/runtime/printexc.c}{runtime/printexc.c:138}
}%
\end{column}
\end{columns}
\footnotetext[1]{\url{https://v2.ocaml.org/api/Printexc.html\#1\_Uncaughtexceptions}}
\end{frame}
}%
\end{column}
\end{columns}
\end{frame}


\begin{frame}{Startup}{How did we get there by the way?}
\begin{columns}
\begin{column}{0.45\textwidth}
\begin{itemize}
\item The entry point address of the binary is \funcname{main} (\funcname{\_start}) from the runtime
\item The program starts like a C program:
\begin{itemize}
\item \funcname{caml\_start\_program}
\item \funcname{caml\_startup\_common}
\item \funcname{caml\_startup\_exn}
\item \funcname{caml\_startup}
\item \funcname{caml\_main}
\item \funcname{main}
\end{itemize}
%\item \funcname{caml\_startup\_common}: initializes GC, domains \dots
% Also: codefrag, locale, custom opeartions, frame_descr, signals, stack
\item \funcname{caml\_start\_program} is the transition point between the C realm and the OCaml one
\end{itemize}
\bigskip
\listocaml{../src/default/default.ml}{default/default.ml}
\end{column}
\begin{column}{0.55\textwidth}
\listasm[lastline=24,highlightlines={22-24}]{src/ocaml/caml\_start\_program.s}{runtime/amd64.S:604}
\end{column}
\end{columns}
\end{frame}


\begin{frame}{Setup to jump into OCaml entry point}{\funcname{caml\_start\_program} initializes the main fiber}
\begin{columns}
\begin{column}{0.6\textwidth}
\only<1,3,5>{
\begin{itemize}
\item<1-> Stores information to allow switching from OCaml stack to the C stack
\item<3-> Constructs the default exception handler
\item<5-> Switches to the OCaml stack and calls \funcname{caml\_program}
\end{itemize}
\bigskip
\listocaml{../src/default/default.ml}{default/default.ml}
}%
\only<2>{
\listasm[firstline=22,lastline=41,highlightlines={25-27}]{src/ocaml/caml\_start\_program.s}{caml\_start\_program}
}%
\only<4>{
\mintasm[firstline=22,lastline=41,highlightlines={31-38}]{src/ocaml/caml\_start\_program.s}
\listasm[firstline=66,lastline=72]{src/ocaml/caml\_start\_program.s}{caml\_start\_program}
}%
\only<6>{
\listasm[firstline=22,lastline=41,highlightlines={39-41}]{src/ocaml/caml\_start\_program.s}{caml\_start\_program}
}%
\end{column}
\begin{column}{0.4\textwidth}
\centering
\only<1-2>{\providecommand\step{2}%
% Runtime's default exception handler
%
\subsection*{Runtime's default exception handler}
\frameSubsection{pictures/The\_Architect.png}{}


\begin{frame}{Default exception handler}
\begin{columns}
\begin{column}{0.5\textwidth}
\begin{itemize}
\item \localname{domain\_state->exn\_handler} points to a trap located before \funcname{entry}!
\item What installed this very first trap there?
\end{itemize}
\bigskip
\listocaml{../src/default/default.ml}{default/default.ml}
\bigskip
\inputminted[fontsize=\footnotesize]{text}{../src/default/default.out}
\end{column}
\begin{column}{0.5\textwidth}
\centering
\only<1>{\providecommand\step{0}\input{diagrams/default/default.tex}}%
\only<2>{\providecommand\step{4}\input{diagrams/default/default.tex}}%
\end{column}
\end{columns}
\end{frame}


\begin{frame}{Startup}{How did we get there by the way?}
\begin{columns}
\begin{column}{0.45\textwidth}
\begin{itemize}
\item The entry point address of the binary is \funcname{main} (\funcname{\_start}) from the runtime
\item The program starts like a C program:
\begin{itemize}
\item \funcname{caml\_start\_program}
\item \funcname{caml\_startup\_common}
\item \funcname{caml\_startup\_exn}
\item \funcname{caml\_startup}
\item \funcname{caml\_main}
\item \funcname{main}
\end{itemize}
%\item \funcname{caml\_startup\_common}: initializes GC, domains \dots
% Also: codefrag, locale, custom opeartions, frame_descr, signals, stack
\item \funcname{caml\_start\_program} is the transition point between the C realm and the OCaml one
\end{itemize}
\bigskip
\listocaml{../src/default/default.ml}{default/default.ml}
\end{column}
\begin{column}{0.55\textwidth}
\listasm[lastline=24,highlightlines={22-24}]{src/ocaml/caml\_start\_program.s}{runtime/amd64.S:604}
\end{column}
\end{columns}
\end{frame}


\begin{frame}{Setup to jump into OCaml entry point}{\funcname{caml\_start\_program} initializes the main fiber}
\begin{columns}
\begin{column}{0.6\textwidth}
\only<1,3,5>{
\begin{itemize}
\item<1-> Stores information to allow switching from OCaml stack to the C stack
\item<3-> Constructs the default exception handler
\item<5-> Switches to the OCaml stack and calls \funcname{caml\_program}
\end{itemize}
\bigskip
\listocaml{../src/default/default.ml}{default/default.ml}
}%
\only<2>{
\listasm[firstline=22,lastline=41,highlightlines={25-27}]{src/ocaml/caml\_start\_program.s}{caml\_start\_program}
}%
\only<4>{
\mintasm[firstline=22,lastline=41,highlightlines={31-38}]{src/ocaml/caml\_start\_program.s}
\listasm[firstline=66,lastline=72]{src/ocaml/caml\_start\_program.s}{caml\_start\_program}
}%
\only<6>{
\listasm[firstline=22,lastline=41,highlightlines={39-41}]{src/ocaml/caml\_start\_program.s}{caml\_start\_program}
}%
\end{column}
\begin{column}{0.4\textwidth}
\centering
\only<1-2>{\providecommand\step{2}\input{diagrams/default/default.tex}}%
\only<3-5>{\providecommand\step{3}\input{diagrams/default/default.tex}}%
\onslide*<6->{\providecommand\step{4}\input{diagrams/default/default.tex}}%
\end{column}
\end{columns}
\end{frame}

\begin{frame}{Executing the OCaml program}{And raising an exception without handler}
\begin{columns}
\begin{column}{0.6\textwidth}
\only<1-3>{
\begin{itemize}
\item<1-> Execution continues with OCaml code
\begin{itemize}
\item \funcname{camlDefault\_\_main\_269}
\item \funcname{camlDefault\_\_entry}
\item \funcname{caml\_program}
\end{itemize}
\item<1-> \funcname{camlDefault\_\_main\_269} raises a \typename{Failure} exception
\item<1-> \typename{Failure} is caught by \funcname{caml\_start\_program}
\begin{itemize}
\item<1-> The handler set the 2nd LSB of the exception value to \funcarg{1}
\item<3-> And then forces \funcname{caml\_start\_program} to terminate
\end{itemize}
\end{itemize}
}
\bigskip
\only<1,3>{\listocaml[highlightlines=3]{../src/default/default.ml}{default/default.ml}}%
\only<2>{\listasm[firstline=66,lastline=72,highlightlines=69]{src/ocaml/caml\_start\_program.s}{caml\_start\_program}}%
\end{column}
\begin{column}{0.4\textwidth}
\centering
\providecommand\step{4}\input{diagrams/default/default.tex}
\end{column}
\end{columns}
\end{frame}
%\only<>{\listasm[firstline=47,lastline=65]{src/ocaml/caml\_start\_program.s}{caml\_start\_program}}


\begin{frame}{Back to C}{Clean up, complain and terminate}
\begin{columns}
\begin{column}{0.45\textwidth}
\begin{itemize}
\item Backtrace:
\begin{itemize}
\item \funcname{caml\_start\_program} $\rightarrow$ OCaml code
\item \funcname{caml\_startup\_common}
\item \funcname{caml\_startup\_exn}
\item \funcname{caml\_startup}
\item \funcname{caml\_main}
\item \funcname{main}
\end{itemize}
\smallskip
\item \funcname{caml\_startup} checks the return value of \funcname{caml\_startup\_exn}
\begin{itemize}
\item The return value has been specially marked by \funcname{caml\_start\_program} handler
\item Calls \funcname{caml\_fatal\_uncaught\_exception}
\end{itemize}
\item<2-> \funcname{caml\_fatal\_uncaught\_exception} either
\begin{itemize}
\item<2-> Calls the user custom uncaught exception handler, if set with \mintinline{ocaml}{Printexc.set_uncaught_exception_handler fn}\footnotemark
\item<2-> Or falls back to the default one
\item<2-> Which notifies about the uncaught exception
\end{itemize}
\item<4-> Terminates the program either with \funcname{abort} or with a non-zero exit code
\end{itemize}
\end{column}
\begin{column}{0.55\textwidth}
\only<1>{
\listc[firstline=84,lastline=89,highlightlines=88]{../ocaml/runtime/caml/mlvalues.h}{runtime/caml/mlvalues.h:84}
\listc[firstline=138,lastline=143,highlightlines={141-142}]{../ocaml/runtime/startup\_nat.c}{runtime/startup\_nat.c:138}
}%
\only<2>{
\listc[firstline=138,highlightlines={147,149}]{../ocaml/runtime/printexc.c}{runtime/printexc.c:138}
}%
\only<3>{
\listc[firstline=110,lastline=134,highlightlines=129]{../ocaml/runtime/printexc.c}{runtime/printexc.c:138}
}%
\only<4>{
\listc[firstline=138,highlightlines={152,154}]{../ocaml/runtime/printexc.c}{runtime/printexc.c:138}
}%
\end{column}
\end{columns}
\footnotetext[1]{\url{https://v2.ocaml.org/api/Printexc.html\#1\_Uncaughtexceptions}}
\end{frame}
}%
\only<3-5>{\providecommand\step{3}%
% Runtime's default exception handler
%
\subsection*{Runtime's default exception handler}
\frameSubsection{pictures/The\_Architect.png}{}


\begin{frame}{Default exception handler}
\begin{columns}
\begin{column}{0.5\textwidth}
\begin{itemize}
\item \localname{domain\_state->exn\_handler} points to a trap located before \funcname{entry}!
\item What installed this very first trap there?
\end{itemize}
\bigskip
\listocaml{../src/default/default.ml}{default/default.ml}
\bigskip
\inputminted[fontsize=\footnotesize]{text}{../src/default/default.out}
\end{column}
\begin{column}{0.5\textwidth}
\centering
\only<1>{\providecommand\step{0}\input{diagrams/default/default.tex}}%
\only<2>{\providecommand\step{4}\input{diagrams/default/default.tex}}%
\end{column}
\end{columns}
\end{frame}


\begin{frame}{Startup}{How did we get there by the way?}
\begin{columns}
\begin{column}{0.45\textwidth}
\begin{itemize}
\item The entry point address of the binary is \funcname{main} (\funcname{\_start}) from the runtime
\item The program starts like a C program:
\begin{itemize}
\item \funcname{caml\_start\_program}
\item \funcname{caml\_startup\_common}
\item \funcname{caml\_startup\_exn}
\item \funcname{caml\_startup}
\item \funcname{caml\_main}
\item \funcname{main}
\end{itemize}
%\item \funcname{caml\_startup\_common}: initializes GC, domains \dots
% Also: codefrag, locale, custom opeartions, frame_descr, signals, stack
\item \funcname{caml\_start\_program} is the transition point between the C realm and the OCaml one
\end{itemize}
\bigskip
\listocaml{../src/default/default.ml}{default/default.ml}
\end{column}
\begin{column}{0.55\textwidth}
\listasm[lastline=24,highlightlines={22-24}]{src/ocaml/caml\_start\_program.s}{runtime/amd64.S:604}
\end{column}
\end{columns}
\end{frame}


\begin{frame}{Setup to jump into OCaml entry point}{\funcname{caml\_start\_program} initializes the main fiber}
\begin{columns}
\begin{column}{0.6\textwidth}
\only<1,3,5>{
\begin{itemize}
\item<1-> Stores information to allow switching from OCaml stack to the C stack
\item<3-> Constructs the default exception handler
\item<5-> Switches to the OCaml stack and calls \funcname{caml\_program}
\end{itemize}
\bigskip
\listocaml{../src/default/default.ml}{default/default.ml}
}%
\only<2>{
\listasm[firstline=22,lastline=41,highlightlines={25-27}]{src/ocaml/caml\_start\_program.s}{caml\_start\_program}
}%
\only<4>{
\mintasm[firstline=22,lastline=41,highlightlines={31-38}]{src/ocaml/caml\_start\_program.s}
\listasm[firstline=66,lastline=72]{src/ocaml/caml\_start\_program.s}{caml\_start\_program}
}%
\only<6>{
\listasm[firstline=22,lastline=41,highlightlines={39-41}]{src/ocaml/caml\_start\_program.s}{caml\_start\_program}
}%
\end{column}
\begin{column}{0.4\textwidth}
\centering
\only<1-2>{\providecommand\step{2}\input{diagrams/default/default.tex}}%
\only<3-5>{\providecommand\step{3}\input{diagrams/default/default.tex}}%
\onslide*<6->{\providecommand\step{4}\input{diagrams/default/default.tex}}%
\end{column}
\end{columns}
\end{frame}

\begin{frame}{Executing the OCaml program}{And raising an exception without handler}
\begin{columns}
\begin{column}{0.6\textwidth}
\only<1-3>{
\begin{itemize}
\item<1-> Execution continues with OCaml code
\begin{itemize}
\item \funcname{camlDefault\_\_main\_269}
\item \funcname{camlDefault\_\_entry}
\item \funcname{caml\_program}
\end{itemize}
\item<1-> \funcname{camlDefault\_\_main\_269} raises a \typename{Failure} exception
\item<1-> \typename{Failure} is caught by \funcname{caml\_start\_program}
\begin{itemize}
\item<1-> The handler set the 2nd LSB of the exception value to \funcarg{1}
\item<3-> And then forces \funcname{caml\_start\_program} to terminate
\end{itemize}
\end{itemize}
}
\bigskip
\only<1,3>{\listocaml[highlightlines=3]{../src/default/default.ml}{default/default.ml}}%
\only<2>{\listasm[firstline=66,lastline=72,highlightlines=69]{src/ocaml/caml\_start\_program.s}{caml\_start\_program}}%
\end{column}
\begin{column}{0.4\textwidth}
\centering
\providecommand\step{4}\input{diagrams/default/default.tex}
\end{column}
\end{columns}
\end{frame}
%\only<>{\listasm[firstline=47,lastline=65]{src/ocaml/caml\_start\_program.s}{caml\_start\_program}}


\begin{frame}{Back to C}{Clean up, complain and terminate}
\begin{columns}
\begin{column}{0.45\textwidth}
\begin{itemize}
\item Backtrace:
\begin{itemize}
\item \funcname{caml\_start\_program} $\rightarrow$ OCaml code
\item \funcname{caml\_startup\_common}
\item \funcname{caml\_startup\_exn}
\item \funcname{caml\_startup}
\item \funcname{caml\_main}
\item \funcname{main}
\end{itemize}
\smallskip
\item \funcname{caml\_startup} checks the return value of \funcname{caml\_startup\_exn}
\begin{itemize}
\item The return value has been specially marked by \funcname{caml\_start\_program} handler
\item Calls \funcname{caml\_fatal\_uncaught\_exception}
\end{itemize}
\item<2-> \funcname{caml\_fatal\_uncaught\_exception} either
\begin{itemize}
\item<2-> Calls the user custom uncaught exception handler, if set with \mintinline{ocaml}{Printexc.set_uncaught_exception_handler fn}\footnotemark
\item<2-> Or falls back to the default one
\item<2-> Which notifies about the uncaught exception
\end{itemize}
\item<4-> Terminates the program either with \funcname{abort} or with a non-zero exit code
\end{itemize}
\end{column}
\begin{column}{0.55\textwidth}
\only<1>{
\listc[firstline=84,lastline=89,highlightlines=88]{../ocaml/runtime/caml/mlvalues.h}{runtime/caml/mlvalues.h:84}
\listc[firstline=138,lastline=143,highlightlines={141-142}]{../ocaml/runtime/startup\_nat.c}{runtime/startup\_nat.c:138}
}%
\only<2>{
\listc[firstline=138,highlightlines={147,149}]{../ocaml/runtime/printexc.c}{runtime/printexc.c:138}
}%
\only<3>{
\listc[firstline=110,lastline=134,highlightlines=129]{../ocaml/runtime/printexc.c}{runtime/printexc.c:138}
}%
\only<4>{
\listc[firstline=138,highlightlines={152,154}]{../ocaml/runtime/printexc.c}{runtime/printexc.c:138}
}%
\end{column}
\end{columns}
\footnotetext[1]{\url{https://v2.ocaml.org/api/Printexc.html\#1\_Uncaughtexceptions}}
\end{frame}
}%
\onslide*<6->{\providecommand\step{4}%
% Runtime's default exception handler
%
\subsection*{Runtime's default exception handler}
\frameSubsection{pictures/The\_Architect.png}{}


\begin{frame}{Default exception handler}
\begin{columns}
\begin{column}{0.5\textwidth}
\begin{itemize}
\item \localname{domain\_state->exn\_handler} points to a trap located before \funcname{entry}!
\item What installed this very first trap there?
\end{itemize}
\bigskip
\listocaml{../src/default/default.ml}{default/default.ml}
\bigskip
\inputminted[fontsize=\footnotesize]{text}{../src/default/default.out}
\end{column}
\begin{column}{0.5\textwidth}
\centering
\only<1>{\providecommand\step{0}\input{diagrams/default/default.tex}}%
\only<2>{\providecommand\step{4}\input{diagrams/default/default.tex}}%
\end{column}
\end{columns}
\end{frame}


\begin{frame}{Startup}{How did we get there by the way?}
\begin{columns}
\begin{column}{0.45\textwidth}
\begin{itemize}
\item The entry point address of the binary is \funcname{main} (\funcname{\_start}) from the runtime
\item The program starts like a C program:
\begin{itemize}
\item \funcname{caml\_start\_program}
\item \funcname{caml\_startup\_common}
\item \funcname{caml\_startup\_exn}
\item \funcname{caml\_startup}
\item \funcname{caml\_main}
\item \funcname{main}
\end{itemize}
%\item \funcname{caml\_startup\_common}: initializes GC, domains \dots
% Also: codefrag, locale, custom opeartions, frame_descr, signals, stack
\item \funcname{caml\_start\_program} is the transition point between the C realm and the OCaml one
\end{itemize}
\bigskip
\listocaml{../src/default/default.ml}{default/default.ml}
\end{column}
\begin{column}{0.55\textwidth}
\listasm[lastline=24,highlightlines={22-24}]{src/ocaml/caml\_start\_program.s}{runtime/amd64.S:604}
\end{column}
\end{columns}
\end{frame}


\begin{frame}{Setup to jump into OCaml entry point}{\funcname{caml\_start\_program} initializes the main fiber}
\begin{columns}
\begin{column}{0.6\textwidth}
\only<1,3,5>{
\begin{itemize}
\item<1-> Stores information to allow switching from OCaml stack to the C stack
\item<3-> Constructs the default exception handler
\item<5-> Switches to the OCaml stack and calls \funcname{caml\_program}
\end{itemize}
\bigskip
\listocaml{../src/default/default.ml}{default/default.ml}
}%
\only<2>{
\listasm[firstline=22,lastline=41,highlightlines={25-27}]{src/ocaml/caml\_start\_program.s}{caml\_start\_program}
}%
\only<4>{
\mintasm[firstline=22,lastline=41,highlightlines={31-38}]{src/ocaml/caml\_start\_program.s}
\listasm[firstline=66,lastline=72]{src/ocaml/caml\_start\_program.s}{caml\_start\_program}
}%
\only<6>{
\listasm[firstline=22,lastline=41,highlightlines={39-41}]{src/ocaml/caml\_start\_program.s}{caml\_start\_program}
}%
\end{column}
\begin{column}{0.4\textwidth}
\centering
\only<1-2>{\providecommand\step{2}\input{diagrams/default/default.tex}}%
\only<3-5>{\providecommand\step{3}\input{diagrams/default/default.tex}}%
\onslide*<6->{\providecommand\step{4}\input{diagrams/default/default.tex}}%
\end{column}
\end{columns}
\end{frame}

\begin{frame}{Executing the OCaml program}{And raising an exception without handler}
\begin{columns}
\begin{column}{0.6\textwidth}
\only<1-3>{
\begin{itemize}
\item<1-> Execution continues with OCaml code
\begin{itemize}
\item \funcname{camlDefault\_\_main\_269}
\item \funcname{camlDefault\_\_entry}
\item \funcname{caml\_program}
\end{itemize}
\item<1-> \funcname{camlDefault\_\_main\_269} raises a \typename{Failure} exception
\item<1-> \typename{Failure} is caught by \funcname{caml\_start\_program}
\begin{itemize}
\item<1-> The handler set the 2nd LSB of the exception value to \funcarg{1}
\item<3-> And then forces \funcname{caml\_start\_program} to terminate
\end{itemize}
\end{itemize}
}
\bigskip
\only<1,3>{\listocaml[highlightlines=3]{../src/default/default.ml}{default/default.ml}}%
\only<2>{\listasm[firstline=66,lastline=72,highlightlines=69]{src/ocaml/caml\_start\_program.s}{caml\_start\_program}}%
\end{column}
\begin{column}{0.4\textwidth}
\centering
\providecommand\step{4}\input{diagrams/default/default.tex}
\end{column}
\end{columns}
\end{frame}
%\only<>{\listasm[firstline=47,lastline=65]{src/ocaml/caml\_start\_program.s}{caml\_start\_program}}


\begin{frame}{Back to C}{Clean up, complain and terminate}
\begin{columns}
\begin{column}{0.45\textwidth}
\begin{itemize}
\item Backtrace:
\begin{itemize}
\item \funcname{caml\_start\_program} $\rightarrow$ OCaml code
\item \funcname{caml\_startup\_common}
\item \funcname{caml\_startup\_exn}
\item \funcname{caml\_startup}
\item \funcname{caml\_main}
\item \funcname{main}
\end{itemize}
\smallskip
\item \funcname{caml\_startup} checks the return value of \funcname{caml\_startup\_exn}
\begin{itemize}
\item The return value has been specially marked by \funcname{caml\_start\_program} handler
\item Calls \funcname{caml\_fatal\_uncaught\_exception}
\end{itemize}
\item<2-> \funcname{caml\_fatal\_uncaught\_exception} either
\begin{itemize}
\item<2-> Calls the user custom uncaught exception handler, if set with \mintinline{ocaml}{Printexc.set_uncaught_exception_handler fn}\footnotemark
\item<2-> Or falls back to the default one
\item<2-> Which notifies about the uncaught exception
\end{itemize}
\item<4-> Terminates the program either with \funcname{abort} or with a non-zero exit code
\end{itemize}
\end{column}
\begin{column}{0.55\textwidth}
\only<1>{
\listc[firstline=84,lastline=89,highlightlines=88]{../ocaml/runtime/caml/mlvalues.h}{runtime/caml/mlvalues.h:84}
\listc[firstline=138,lastline=143,highlightlines={141-142}]{../ocaml/runtime/startup\_nat.c}{runtime/startup\_nat.c:138}
}%
\only<2>{
\listc[firstline=138,highlightlines={147,149}]{../ocaml/runtime/printexc.c}{runtime/printexc.c:138}
}%
\only<3>{
\listc[firstline=110,lastline=134,highlightlines=129]{../ocaml/runtime/printexc.c}{runtime/printexc.c:138}
}%
\only<4>{
\listc[firstline=138,highlightlines={152,154}]{../ocaml/runtime/printexc.c}{runtime/printexc.c:138}
}%
\end{column}
\end{columns}
\footnotetext[1]{\url{https://v2.ocaml.org/api/Printexc.html\#1\_Uncaughtexceptions}}
\end{frame}
}%
\end{column}
\end{columns}
\end{frame}

\begin{frame}{Executing the OCaml program}{And raising an exception without handler}
\begin{columns}
\begin{column}{0.6\textwidth}
\only<1-3>{
\begin{itemize}
\item<1-> Execution continues with OCaml code
\begin{itemize}
\item \funcname{camlDefault\_\_main\_269}
\item \funcname{camlDefault\_\_entry}
\item \funcname{caml\_program}
\end{itemize}
\item<1-> \funcname{camlDefault\_\_main\_269} raises a \typename{Failure} exception
\item<1-> \typename{Failure} is caught by \funcname{caml\_start\_program}
\begin{itemize}
\item<1-> The handler set the 2nd LSB of the exception value to \funcarg{1}
\item<3-> And then forces \funcname{caml\_start\_program} to terminate
\end{itemize}
\end{itemize}
}
\bigskip
\only<1,3>{\listocaml[highlightlines=3]{../src/default/default.ml}{default/default.ml}}%
\only<2>{\listasm[firstline=66,lastline=72,highlightlines=69]{src/ocaml/caml\_start\_program.s}{caml\_start\_program}}%
\end{column}
\begin{column}{0.4\textwidth}
\centering
\providecommand\step{4}%
% Runtime's default exception handler
%
\subsection*{Runtime's default exception handler}
\frameSubsection{pictures/The\_Architect.png}{}


\begin{frame}{Default exception handler}
\begin{columns}
\begin{column}{0.5\textwidth}
\begin{itemize}
\item \localname{domain\_state->exn\_handler} points to a trap located before \funcname{entry}!
\item What installed this very first trap there?
\end{itemize}
\bigskip
\listocaml{../src/default/default.ml}{default/default.ml}
\bigskip
\inputminted[fontsize=\footnotesize]{text}{../src/default/default.out}
\end{column}
\begin{column}{0.5\textwidth}
\centering
\only<1>{\providecommand\step{0}\input{diagrams/default/default.tex}}%
\only<2>{\providecommand\step{4}\input{diagrams/default/default.tex}}%
\end{column}
\end{columns}
\end{frame}


\begin{frame}{Startup}{How did we get there by the way?}
\begin{columns}
\begin{column}{0.45\textwidth}
\begin{itemize}
\item The entry point address of the binary is \funcname{main} (\funcname{\_start}) from the runtime
\item The program starts like a C program:
\begin{itemize}
\item \funcname{caml\_start\_program}
\item \funcname{caml\_startup\_common}
\item \funcname{caml\_startup\_exn}
\item \funcname{caml\_startup}
\item \funcname{caml\_main}
\item \funcname{main}
\end{itemize}
%\item \funcname{caml\_startup\_common}: initializes GC, domains \dots
% Also: codefrag, locale, custom opeartions, frame_descr, signals, stack
\item \funcname{caml\_start\_program} is the transition point between the C realm and the OCaml one
\end{itemize}
\bigskip
\listocaml{../src/default/default.ml}{default/default.ml}
\end{column}
\begin{column}{0.55\textwidth}
\listasm[lastline=24,highlightlines={22-24}]{src/ocaml/caml\_start\_program.s}{runtime/amd64.S:604}
\end{column}
\end{columns}
\end{frame}


\begin{frame}{Setup to jump into OCaml entry point}{\funcname{caml\_start\_program} initializes the main fiber}
\begin{columns}
\begin{column}{0.6\textwidth}
\only<1,3,5>{
\begin{itemize}
\item<1-> Stores information to allow switching from OCaml stack to the C stack
\item<3-> Constructs the default exception handler
\item<5-> Switches to the OCaml stack and calls \funcname{caml\_program}
\end{itemize}
\bigskip
\listocaml{../src/default/default.ml}{default/default.ml}
}%
\only<2>{
\listasm[firstline=22,lastline=41,highlightlines={25-27}]{src/ocaml/caml\_start\_program.s}{caml\_start\_program}
}%
\only<4>{
\mintasm[firstline=22,lastline=41,highlightlines={31-38}]{src/ocaml/caml\_start\_program.s}
\listasm[firstline=66,lastline=72]{src/ocaml/caml\_start\_program.s}{caml\_start\_program}
}%
\only<6>{
\listasm[firstline=22,lastline=41,highlightlines={39-41}]{src/ocaml/caml\_start\_program.s}{caml\_start\_program}
}%
\end{column}
\begin{column}{0.4\textwidth}
\centering
\only<1-2>{\providecommand\step{2}\input{diagrams/default/default.tex}}%
\only<3-5>{\providecommand\step{3}\input{diagrams/default/default.tex}}%
\onslide*<6->{\providecommand\step{4}\input{diagrams/default/default.tex}}%
\end{column}
\end{columns}
\end{frame}

\begin{frame}{Executing the OCaml program}{And raising an exception without handler}
\begin{columns}
\begin{column}{0.6\textwidth}
\only<1-3>{
\begin{itemize}
\item<1-> Execution continues with OCaml code
\begin{itemize}
\item \funcname{camlDefault\_\_main\_269}
\item \funcname{camlDefault\_\_entry}
\item \funcname{caml\_program}
\end{itemize}
\item<1-> \funcname{camlDefault\_\_main\_269} raises a \typename{Failure} exception
\item<1-> \typename{Failure} is caught by \funcname{caml\_start\_program}
\begin{itemize}
\item<1-> The handler set the 2nd LSB of the exception value to \funcarg{1}
\item<3-> And then forces \funcname{caml\_start\_program} to terminate
\end{itemize}
\end{itemize}
}
\bigskip
\only<1,3>{\listocaml[highlightlines=3]{../src/default/default.ml}{default/default.ml}}%
\only<2>{\listasm[firstline=66,lastline=72,highlightlines=69]{src/ocaml/caml\_start\_program.s}{caml\_start\_program}}%
\end{column}
\begin{column}{0.4\textwidth}
\centering
\providecommand\step{4}\input{diagrams/default/default.tex}
\end{column}
\end{columns}
\end{frame}
%\only<>{\listasm[firstline=47,lastline=65]{src/ocaml/caml\_start\_program.s}{caml\_start\_program}}


\begin{frame}{Back to C}{Clean up, complain and terminate}
\begin{columns}
\begin{column}{0.45\textwidth}
\begin{itemize}
\item Backtrace:
\begin{itemize}
\item \funcname{caml\_start\_program} $\rightarrow$ OCaml code
\item \funcname{caml\_startup\_common}
\item \funcname{caml\_startup\_exn}
\item \funcname{caml\_startup}
\item \funcname{caml\_main}
\item \funcname{main}
\end{itemize}
\smallskip
\item \funcname{caml\_startup} checks the return value of \funcname{caml\_startup\_exn}
\begin{itemize}
\item The return value has been specially marked by \funcname{caml\_start\_program} handler
\item Calls \funcname{caml\_fatal\_uncaught\_exception}
\end{itemize}
\item<2-> \funcname{caml\_fatal\_uncaught\_exception} either
\begin{itemize}
\item<2-> Calls the user custom uncaught exception handler, if set with \mintinline{ocaml}{Printexc.set_uncaught_exception_handler fn}\footnotemark
\item<2-> Or falls back to the default one
\item<2-> Which notifies about the uncaught exception
\end{itemize}
\item<4-> Terminates the program either with \funcname{abort} or with a non-zero exit code
\end{itemize}
\end{column}
\begin{column}{0.55\textwidth}
\only<1>{
\listc[firstline=84,lastline=89,highlightlines=88]{../ocaml/runtime/caml/mlvalues.h}{runtime/caml/mlvalues.h:84}
\listc[firstline=138,lastline=143,highlightlines={141-142}]{../ocaml/runtime/startup\_nat.c}{runtime/startup\_nat.c:138}
}%
\only<2>{
\listc[firstline=138,highlightlines={147,149}]{../ocaml/runtime/printexc.c}{runtime/printexc.c:138}
}%
\only<3>{
\listc[firstline=110,lastline=134,highlightlines=129]{../ocaml/runtime/printexc.c}{runtime/printexc.c:138}
}%
\only<4>{
\listc[firstline=138,highlightlines={152,154}]{../ocaml/runtime/printexc.c}{runtime/printexc.c:138}
}%
\end{column}
\end{columns}
\footnotetext[1]{\url{https://v2.ocaml.org/api/Printexc.html\#1\_Uncaughtexceptions}}
\end{frame}

\end{column}
\end{columns}
\end{frame}
%\only<>{\listasm[firstline=47,lastline=65]{src/ocaml/caml\_start\_program.s}{caml\_start\_program}}


\begin{frame}{Back to C}{Clean up, complain and terminate}
\begin{columns}
\begin{column}{0.45\textwidth}
\begin{itemize}
\item Backtrace:
\begin{itemize}
\item \funcname{caml\_start\_program} $\rightarrow$ OCaml code
\item \funcname{caml\_startup\_common}
\item \funcname{caml\_startup\_exn}
\item \funcname{caml\_startup}
\item \funcname{caml\_main}
\item \funcname{main}
\end{itemize}
\smallskip
\item \funcname{caml\_startup} checks the return value of \funcname{caml\_startup\_exn}
\begin{itemize}
\item The return value has been specially marked by \funcname{caml\_start\_program} handler
\item Calls \funcname{caml\_fatal\_uncaught\_exception}
\end{itemize}
\item<2-> \funcname{caml\_fatal\_uncaught\_exception} either
\begin{itemize}
\item<2-> Calls the user custom uncaught exception handler, if set with \mintinline{ocaml}{Printexc.set_uncaught_exception_handler fn}\footnotemark
\item<2-> Or falls back to the default one
\item<2-> Which notifies about the uncaught exception
\end{itemize}
\item<4-> Terminates the program either with \funcname{abort} or with a non-zero exit code
\end{itemize}
\end{column}
\begin{column}{0.55\textwidth}
\only<1>{
\listc[firstline=84,lastline=89,highlightlines=88]{../ocaml/runtime/caml/mlvalues.h}{runtime/caml/mlvalues.h:84}
\listc[firstline=138,lastline=143,highlightlines={141-142}]{../ocaml/runtime/startup\_nat.c}{runtime/startup\_nat.c:138}
}%
\only<2>{
\listc[firstline=138,highlightlines={147,149}]{../ocaml/runtime/printexc.c}{runtime/printexc.c:138}
}%
\only<3>{
\listc[firstline=110,lastline=134,highlightlines=129]{../ocaml/runtime/printexc.c}{runtime/printexc.c:138}
}%
\only<4>{
\listc[firstline=138,highlightlines={152,154}]{../ocaml/runtime/printexc.c}{runtime/printexc.c:138}
}%
\end{column}
\end{columns}
\footnotetext[1]{\url{https://v2.ocaml.org/api/Printexc.html\#1\_Uncaughtexceptions}}
\end{frame}
}%
\only<3-5>{\providecommand\step{3}%
% Runtime's default exception handler
%
\subsection*{Runtime's default exception handler}
\frameSubsection{pictures/The\_Architect.png}{}


\begin{frame}{Default exception handler}
\begin{columns}
\begin{column}{0.5\textwidth}
\begin{itemize}
\item \localname{domain\_state->exn\_handler} points to a trap located before \funcname{entry}!
\item What installed this very first trap there?
\end{itemize}
\bigskip
\listocaml{../src/default/default.ml}{default/default.ml}
\bigskip
\inputminted[fontsize=\footnotesize]{text}{../src/default/default.out}
\end{column}
\begin{column}{0.5\textwidth}
\centering
\only<1>{\providecommand\step{0}%
% Runtime's default exception handler
%
\subsection*{Runtime's default exception handler}
\frameSubsection{pictures/The\_Architect.png}{}


\begin{frame}{Default exception handler}
\begin{columns}
\begin{column}{0.5\textwidth}
\begin{itemize}
\item \localname{domain\_state->exn\_handler} points to a trap located before \funcname{entry}!
\item What installed this very first trap there?
\end{itemize}
\bigskip
\listocaml{../src/default/default.ml}{default/default.ml}
\bigskip
\inputminted[fontsize=\footnotesize]{text}{../src/default/default.out}
\end{column}
\begin{column}{0.5\textwidth}
\centering
\only<1>{\providecommand\step{0}\input{diagrams/default/default.tex}}%
\only<2>{\providecommand\step{4}\input{diagrams/default/default.tex}}%
\end{column}
\end{columns}
\end{frame}


\begin{frame}{Startup}{How did we get there by the way?}
\begin{columns}
\begin{column}{0.45\textwidth}
\begin{itemize}
\item The entry point address of the binary is \funcname{main} (\funcname{\_start}) from the runtime
\item The program starts like a C program:
\begin{itemize}
\item \funcname{caml\_start\_program}
\item \funcname{caml\_startup\_common}
\item \funcname{caml\_startup\_exn}
\item \funcname{caml\_startup}
\item \funcname{caml\_main}
\item \funcname{main}
\end{itemize}
%\item \funcname{caml\_startup\_common}: initializes GC, domains \dots
% Also: codefrag, locale, custom opeartions, frame_descr, signals, stack
\item \funcname{caml\_start\_program} is the transition point between the C realm and the OCaml one
\end{itemize}
\bigskip
\listocaml{../src/default/default.ml}{default/default.ml}
\end{column}
\begin{column}{0.55\textwidth}
\listasm[lastline=24,highlightlines={22-24}]{src/ocaml/caml\_start\_program.s}{runtime/amd64.S:604}
\end{column}
\end{columns}
\end{frame}


\begin{frame}{Setup to jump into OCaml entry point}{\funcname{caml\_start\_program} initializes the main fiber}
\begin{columns}
\begin{column}{0.6\textwidth}
\only<1,3,5>{
\begin{itemize}
\item<1-> Stores information to allow switching from OCaml stack to the C stack
\item<3-> Constructs the default exception handler
\item<5-> Switches to the OCaml stack and calls \funcname{caml\_program}
\end{itemize}
\bigskip
\listocaml{../src/default/default.ml}{default/default.ml}
}%
\only<2>{
\listasm[firstline=22,lastline=41,highlightlines={25-27}]{src/ocaml/caml\_start\_program.s}{caml\_start\_program}
}%
\only<4>{
\mintasm[firstline=22,lastline=41,highlightlines={31-38}]{src/ocaml/caml\_start\_program.s}
\listasm[firstline=66,lastline=72]{src/ocaml/caml\_start\_program.s}{caml\_start\_program}
}%
\only<6>{
\listasm[firstline=22,lastline=41,highlightlines={39-41}]{src/ocaml/caml\_start\_program.s}{caml\_start\_program}
}%
\end{column}
\begin{column}{0.4\textwidth}
\centering
\only<1-2>{\providecommand\step{2}\input{diagrams/default/default.tex}}%
\only<3-5>{\providecommand\step{3}\input{diagrams/default/default.tex}}%
\onslide*<6->{\providecommand\step{4}\input{diagrams/default/default.tex}}%
\end{column}
\end{columns}
\end{frame}

\begin{frame}{Executing the OCaml program}{And raising an exception without handler}
\begin{columns}
\begin{column}{0.6\textwidth}
\only<1-3>{
\begin{itemize}
\item<1-> Execution continues with OCaml code
\begin{itemize}
\item \funcname{camlDefault\_\_main\_269}
\item \funcname{camlDefault\_\_entry}
\item \funcname{caml\_program}
\end{itemize}
\item<1-> \funcname{camlDefault\_\_main\_269} raises a \typename{Failure} exception
\item<1-> \typename{Failure} is caught by \funcname{caml\_start\_program}
\begin{itemize}
\item<1-> The handler set the 2nd LSB of the exception value to \funcarg{1}
\item<3-> And then forces \funcname{caml\_start\_program} to terminate
\end{itemize}
\end{itemize}
}
\bigskip
\only<1,3>{\listocaml[highlightlines=3]{../src/default/default.ml}{default/default.ml}}%
\only<2>{\listasm[firstline=66,lastline=72,highlightlines=69]{src/ocaml/caml\_start\_program.s}{caml\_start\_program}}%
\end{column}
\begin{column}{0.4\textwidth}
\centering
\providecommand\step{4}\input{diagrams/default/default.tex}
\end{column}
\end{columns}
\end{frame}
%\only<>{\listasm[firstline=47,lastline=65]{src/ocaml/caml\_start\_program.s}{caml\_start\_program}}


\begin{frame}{Back to C}{Clean up, complain and terminate}
\begin{columns}
\begin{column}{0.45\textwidth}
\begin{itemize}
\item Backtrace:
\begin{itemize}
\item \funcname{caml\_start\_program} $\rightarrow$ OCaml code
\item \funcname{caml\_startup\_common}
\item \funcname{caml\_startup\_exn}
\item \funcname{caml\_startup}
\item \funcname{caml\_main}
\item \funcname{main}
\end{itemize}
\smallskip
\item \funcname{caml\_startup} checks the return value of \funcname{caml\_startup\_exn}
\begin{itemize}
\item The return value has been specially marked by \funcname{caml\_start\_program} handler
\item Calls \funcname{caml\_fatal\_uncaught\_exception}
\end{itemize}
\item<2-> \funcname{caml\_fatal\_uncaught\_exception} either
\begin{itemize}
\item<2-> Calls the user custom uncaught exception handler, if set with \mintinline{ocaml}{Printexc.set_uncaught_exception_handler fn}\footnotemark
\item<2-> Or falls back to the default one
\item<2-> Which notifies about the uncaught exception
\end{itemize}
\item<4-> Terminates the program either with \funcname{abort} or with a non-zero exit code
\end{itemize}
\end{column}
\begin{column}{0.55\textwidth}
\only<1>{
\listc[firstline=84,lastline=89,highlightlines=88]{../ocaml/runtime/caml/mlvalues.h}{runtime/caml/mlvalues.h:84}
\listc[firstline=138,lastline=143,highlightlines={141-142}]{../ocaml/runtime/startup\_nat.c}{runtime/startup\_nat.c:138}
}%
\only<2>{
\listc[firstline=138,highlightlines={147,149}]{../ocaml/runtime/printexc.c}{runtime/printexc.c:138}
}%
\only<3>{
\listc[firstline=110,lastline=134,highlightlines=129]{../ocaml/runtime/printexc.c}{runtime/printexc.c:138}
}%
\only<4>{
\listc[firstline=138,highlightlines={152,154}]{../ocaml/runtime/printexc.c}{runtime/printexc.c:138}
}%
\end{column}
\end{columns}
\footnotetext[1]{\url{https://v2.ocaml.org/api/Printexc.html\#1\_Uncaughtexceptions}}
\end{frame}
}%
\only<2>{\providecommand\step{4}%
% Runtime's default exception handler
%
\subsection*{Runtime's default exception handler}
\frameSubsection{pictures/The\_Architect.png}{}


\begin{frame}{Default exception handler}
\begin{columns}
\begin{column}{0.5\textwidth}
\begin{itemize}
\item \localname{domain\_state->exn\_handler} points to a trap located before \funcname{entry}!
\item What installed this very first trap there?
\end{itemize}
\bigskip
\listocaml{../src/default/default.ml}{default/default.ml}
\bigskip
\inputminted[fontsize=\footnotesize]{text}{../src/default/default.out}
\end{column}
\begin{column}{0.5\textwidth}
\centering
\only<1>{\providecommand\step{0}\input{diagrams/default/default.tex}}%
\only<2>{\providecommand\step{4}\input{diagrams/default/default.tex}}%
\end{column}
\end{columns}
\end{frame}


\begin{frame}{Startup}{How did we get there by the way?}
\begin{columns}
\begin{column}{0.45\textwidth}
\begin{itemize}
\item The entry point address of the binary is \funcname{main} (\funcname{\_start}) from the runtime
\item The program starts like a C program:
\begin{itemize}
\item \funcname{caml\_start\_program}
\item \funcname{caml\_startup\_common}
\item \funcname{caml\_startup\_exn}
\item \funcname{caml\_startup}
\item \funcname{caml\_main}
\item \funcname{main}
\end{itemize}
%\item \funcname{caml\_startup\_common}: initializes GC, domains \dots
% Also: codefrag, locale, custom opeartions, frame_descr, signals, stack
\item \funcname{caml\_start\_program} is the transition point between the C realm and the OCaml one
\end{itemize}
\bigskip
\listocaml{../src/default/default.ml}{default/default.ml}
\end{column}
\begin{column}{0.55\textwidth}
\listasm[lastline=24,highlightlines={22-24}]{src/ocaml/caml\_start\_program.s}{runtime/amd64.S:604}
\end{column}
\end{columns}
\end{frame}


\begin{frame}{Setup to jump into OCaml entry point}{\funcname{caml\_start\_program} initializes the main fiber}
\begin{columns}
\begin{column}{0.6\textwidth}
\only<1,3,5>{
\begin{itemize}
\item<1-> Stores information to allow switching from OCaml stack to the C stack
\item<3-> Constructs the default exception handler
\item<5-> Switches to the OCaml stack and calls \funcname{caml\_program}
\end{itemize}
\bigskip
\listocaml{../src/default/default.ml}{default/default.ml}
}%
\only<2>{
\listasm[firstline=22,lastline=41,highlightlines={25-27}]{src/ocaml/caml\_start\_program.s}{caml\_start\_program}
}%
\only<4>{
\mintasm[firstline=22,lastline=41,highlightlines={31-38}]{src/ocaml/caml\_start\_program.s}
\listasm[firstline=66,lastline=72]{src/ocaml/caml\_start\_program.s}{caml\_start\_program}
}%
\only<6>{
\listasm[firstline=22,lastline=41,highlightlines={39-41}]{src/ocaml/caml\_start\_program.s}{caml\_start\_program}
}%
\end{column}
\begin{column}{0.4\textwidth}
\centering
\only<1-2>{\providecommand\step{2}\input{diagrams/default/default.tex}}%
\only<3-5>{\providecommand\step{3}\input{diagrams/default/default.tex}}%
\onslide*<6->{\providecommand\step{4}\input{diagrams/default/default.tex}}%
\end{column}
\end{columns}
\end{frame}

\begin{frame}{Executing the OCaml program}{And raising an exception without handler}
\begin{columns}
\begin{column}{0.6\textwidth}
\only<1-3>{
\begin{itemize}
\item<1-> Execution continues with OCaml code
\begin{itemize}
\item \funcname{camlDefault\_\_main\_269}
\item \funcname{camlDefault\_\_entry}
\item \funcname{caml\_program}
\end{itemize}
\item<1-> \funcname{camlDefault\_\_main\_269} raises a \typename{Failure} exception
\item<1-> \typename{Failure} is caught by \funcname{caml\_start\_program}
\begin{itemize}
\item<1-> The handler set the 2nd LSB of the exception value to \funcarg{1}
\item<3-> And then forces \funcname{caml\_start\_program} to terminate
\end{itemize}
\end{itemize}
}
\bigskip
\only<1,3>{\listocaml[highlightlines=3]{../src/default/default.ml}{default/default.ml}}%
\only<2>{\listasm[firstline=66,lastline=72,highlightlines=69]{src/ocaml/caml\_start\_program.s}{caml\_start\_program}}%
\end{column}
\begin{column}{0.4\textwidth}
\centering
\providecommand\step{4}\input{diagrams/default/default.tex}
\end{column}
\end{columns}
\end{frame}
%\only<>{\listasm[firstline=47,lastline=65]{src/ocaml/caml\_start\_program.s}{caml\_start\_program}}


\begin{frame}{Back to C}{Clean up, complain and terminate}
\begin{columns}
\begin{column}{0.45\textwidth}
\begin{itemize}
\item Backtrace:
\begin{itemize}
\item \funcname{caml\_start\_program} $\rightarrow$ OCaml code
\item \funcname{caml\_startup\_common}
\item \funcname{caml\_startup\_exn}
\item \funcname{caml\_startup}
\item \funcname{caml\_main}
\item \funcname{main}
\end{itemize}
\smallskip
\item \funcname{caml\_startup} checks the return value of \funcname{caml\_startup\_exn}
\begin{itemize}
\item The return value has been specially marked by \funcname{caml\_start\_program} handler
\item Calls \funcname{caml\_fatal\_uncaught\_exception}
\end{itemize}
\item<2-> \funcname{caml\_fatal\_uncaught\_exception} either
\begin{itemize}
\item<2-> Calls the user custom uncaught exception handler, if set with \mintinline{ocaml}{Printexc.set_uncaught_exception_handler fn}\footnotemark
\item<2-> Or falls back to the default one
\item<2-> Which notifies about the uncaught exception
\end{itemize}
\item<4-> Terminates the program either with \funcname{abort} or with a non-zero exit code
\end{itemize}
\end{column}
\begin{column}{0.55\textwidth}
\only<1>{
\listc[firstline=84,lastline=89,highlightlines=88]{../ocaml/runtime/caml/mlvalues.h}{runtime/caml/mlvalues.h:84}
\listc[firstline=138,lastline=143,highlightlines={141-142}]{../ocaml/runtime/startup\_nat.c}{runtime/startup\_nat.c:138}
}%
\only<2>{
\listc[firstline=138,highlightlines={147,149}]{../ocaml/runtime/printexc.c}{runtime/printexc.c:138}
}%
\only<3>{
\listc[firstline=110,lastline=134,highlightlines=129]{../ocaml/runtime/printexc.c}{runtime/printexc.c:138}
}%
\only<4>{
\listc[firstline=138,highlightlines={152,154}]{../ocaml/runtime/printexc.c}{runtime/printexc.c:138}
}%
\end{column}
\end{columns}
\footnotetext[1]{\url{https://v2.ocaml.org/api/Printexc.html\#1\_Uncaughtexceptions}}
\end{frame}
}%
\end{column}
\end{columns}
\end{frame}


\begin{frame}{Startup}{How did we get there by the way?}
\begin{columns}
\begin{column}{0.45\textwidth}
\begin{itemize}
\item The entry point address of the binary is \funcname{main} (\funcname{\_start}) from the runtime
\item The program starts like a C program:
\begin{itemize}
\item \funcname{caml\_start\_program}
\item \funcname{caml\_startup\_common}
\item \funcname{caml\_startup\_exn}
\item \funcname{caml\_startup}
\item \funcname{caml\_main}
\item \funcname{main}
\end{itemize}
%\item \funcname{caml\_startup\_common}: initializes GC, domains \dots
% Also: codefrag, locale, custom opeartions, frame_descr, signals, stack
\item \funcname{caml\_start\_program} is the transition point between the C realm and the OCaml one
\end{itemize}
\bigskip
\listocaml{../src/default/default.ml}{default/default.ml}
\end{column}
\begin{column}{0.55\textwidth}
\listasm[lastline=24,highlightlines={22-24}]{src/ocaml/caml\_start\_program.s}{runtime/amd64.S:604}
\end{column}
\end{columns}
\end{frame}


\begin{frame}{Setup to jump into OCaml entry point}{\funcname{caml\_start\_program} initializes the main fiber}
\begin{columns}
\begin{column}{0.6\textwidth}
\only<1,3,5>{
\begin{itemize}
\item<1-> Stores information to allow switching from OCaml stack to the C stack
\item<3-> Constructs the default exception handler
\item<5-> Switches to the OCaml stack and calls \funcname{caml\_program}
\end{itemize}
\bigskip
\listocaml{../src/default/default.ml}{default/default.ml}
}%
\only<2>{
\listasm[firstline=22,lastline=41,highlightlines={25-27}]{src/ocaml/caml\_start\_program.s}{caml\_start\_program}
}%
\only<4>{
\mintasm[firstline=22,lastline=41,highlightlines={31-38}]{src/ocaml/caml\_start\_program.s}
\listasm[firstline=66,lastline=72]{src/ocaml/caml\_start\_program.s}{caml\_start\_program}
}%
\only<6>{
\listasm[firstline=22,lastline=41,highlightlines={39-41}]{src/ocaml/caml\_start\_program.s}{caml\_start\_program}
}%
\end{column}
\begin{column}{0.4\textwidth}
\centering
\only<1-2>{\providecommand\step{2}%
% Runtime's default exception handler
%
\subsection*{Runtime's default exception handler}
\frameSubsection{pictures/The\_Architect.png}{}


\begin{frame}{Default exception handler}
\begin{columns}
\begin{column}{0.5\textwidth}
\begin{itemize}
\item \localname{domain\_state->exn\_handler} points to a trap located before \funcname{entry}!
\item What installed this very first trap there?
\end{itemize}
\bigskip
\listocaml{../src/default/default.ml}{default/default.ml}
\bigskip
\inputminted[fontsize=\footnotesize]{text}{../src/default/default.out}
\end{column}
\begin{column}{0.5\textwidth}
\centering
\only<1>{\providecommand\step{0}\input{diagrams/default/default.tex}}%
\only<2>{\providecommand\step{4}\input{diagrams/default/default.tex}}%
\end{column}
\end{columns}
\end{frame}


\begin{frame}{Startup}{How did we get there by the way?}
\begin{columns}
\begin{column}{0.45\textwidth}
\begin{itemize}
\item The entry point address of the binary is \funcname{main} (\funcname{\_start}) from the runtime
\item The program starts like a C program:
\begin{itemize}
\item \funcname{caml\_start\_program}
\item \funcname{caml\_startup\_common}
\item \funcname{caml\_startup\_exn}
\item \funcname{caml\_startup}
\item \funcname{caml\_main}
\item \funcname{main}
\end{itemize}
%\item \funcname{caml\_startup\_common}: initializes GC, domains \dots
% Also: codefrag, locale, custom opeartions, frame_descr, signals, stack
\item \funcname{caml\_start\_program} is the transition point between the C realm and the OCaml one
\end{itemize}
\bigskip
\listocaml{../src/default/default.ml}{default/default.ml}
\end{column}
\begin{column}{0.55\textwidth}
\listasm[lastline=24,highlightlines={22-24}]{src/ocaml/caml\_start\_program.s}{runtime/amd64.S:604}
\end{column}
\end{columns}
\end{frame}


\begin{frame}{Setup to jump into OCaml entry point}{\funcname{caml\_start\_program} initializes the main fiber}
\begin{columns}
\begin{column}{0.6\textwidth}
\only<1,3,5>{
\begin{itemize}
\item<1-> Stores information to allow switching from OCaml stack to the C stack
\item<3-> Constructs the default exception handler
\item<5-> Switches to the OCaml stack and calls \funcname{caml\_program}
\end{itemize}
\bigskip
\listocaml{../src/default/default.ml}{default/default.ml}
}%
\only<2>{
\listasm[firstline=22,lastline=41,highlightlines={25-27}]{src/ocaml/caml\_start\_program.s}{caml\_start\_program}
}%
\only<4>{
\mintasm[firstline=22,lastline=41,highlightlines={31-38}]{src/ocaml/caml\_start\_program.s}
\listasm[firstline=66,lastline=72]{src/ocaml/caml\_start\_program.s}{caml\_start\_program}
}%
\only<6>{
\listasm[firstline=22,lastline=41,highlightlines={39-41}]{src/ocaml/caml\_start\_program.s}{caml\_start\_program}
}%
\end{column}
\begin{column}{0.4\textwidth}
\centering
\only<1-2>{\providecommand\step{2}\input{diagrams/default/default.tex}}%
\only<3-5>{\providecommand\step{3}\input{diagrams/default/default.tex}}%
\onslide*<6->{\providecommand\step{4}\input{diagrams/default/default.tex}}%
\end{column}
\end{columns}
\end{frame}

\begin{frame}{Executing the OCaml program}{And raising an exception without handler}
\begin{columns}
\begin{column}{0.6\textwidth}
\only<1-3>{
\begin{itemize}
\item<1-> Execution continues with OCaml code
\begin{itemize}
\item \funcname{camlDefault\_\_main\_269}
\item \funcname{camlDefault\_\_entry}
\item \funcname{caml\_program}
\end{itemize}
\item<1-> \funcname{camlDefault\_\_main\_269} raises a \typename{Failure} exception
\item<1-> \typename{Failure} is caught by \funcname{caml\_start\_program}
\begin{itemize}
\item<1-> The handler set the 2nd LSB of the exception value to \funcarg{1}
\item<3-> And then forces \funcname{caml\_start\_program} to terminate
\end{itemize}
\end{itemize}
}
\bigskip
\only<1,3>{\listocaml[highlightlines=3]{../src/default/default.ml}{default/default.ml}}%
\only<2>{\listasm[firstline=66,lastline=72,highlightlines=69]{src/ocaml/caml\_start\_program.s}{caml\_start\_program}}%
\end{column}
\begin{column}{0.4\textwidth}
\centering
\providecommand\step{4}\input{diagrams/default/default.tex}
\end{column}
\end{columns}
\end{frame}
%\only<>{\listasm[firstline=47,lastline=65]{src/ocaml/caml\_start\_program.s}{caml\_start\_program}}


\begin{frame}{Back to C}{Clean up, complain and terminate}
\begin{columns}
\begin{column}{0.45\textwidth}
\begin{itemize}
\item Backtrace:
\begin{itemize}
\item \funcname{caml\_start\_program} $\rightarrow$ OCaml code
\item \funcname{caml\_startup\_common}
\item \funcname{caml\_startup\_exn}
\item \funcname{caml\_startup}
\item \funcname{caml\_main}
\item \funcname{main}
\end{itemize}
\smallskip
\item \funcname{caml\_startup} checks the return value of \funcname{caml\_startup\_exn}
\begin{itemize}
\item The return value has been specially marked by \funcname{caml\_start\_program} handler
\item Calls \funcname{caml\_fatal\_uncaught\_exception}
\end{itemize}
\item<2-> \funcname{caml\_fatal\_uncaught\_exception} either
\begin{itemize}
\item<2-> Calls the user custom uncaught exception handler, if set with \mintinline{ocaml}{Printexc.set_uncaught_exception_handler fn}\footnotemark
\item<2-> Or falls back to the default one
\item<2-> Which notifies about the uncaught exception
\end{itemize}
\item<4-> Terminates the program either with \funcname{abort} or with a non-zero exit code
\end{itemize}
\end{column}
\begin{column}{0.55\textwidth}
\only<1>{
\listc[firstline=84,lastline=89,highlightlines=88]{../ocaml/runtime/caml/mlvalues.h}{runtime/caml/mlvalues.h:84}
\listc[firstline=138,lastline=143,highlightlines={141-142}]{../ocaml/runtime/startup\_nat.c}{runtime/startup\_nat.c:138}
}%
\only<2>{
\listc[firstline=138,highlightlines={147,149}]{../ocaml/runtime/printexc.c}{runtime/printexc.c:138}
}%
\only<3>{
\listc[firstline=110,lastline=134,highlightlines=129]{../ocaml/runtime/printexc.c}{runtime/printexc.c:138}
}%
\only<4>{
\listc[firstline=138,highlightlines={152,154}]{../ocaml/runtime/printexc.c}{runtime/printexc.c:138}
}%
\end{column}
\end{columns}
\footnotetext[1]{\url{https://v2.ocaml.org/api/Printexc.html\#1\_Uncaughtexceptions}}
\end{frame}
}%
\only<3-5>{\providecommand\step{3}%
% Runtime's default exception handler
%
\subsection*{Runtime's default exception handler}
\frameSubsection{pictures/The\_Architect.png}{}


\begin{frame}{Default exception handler}
\begin{columns}
\begin{column}{0.5\textwidth}
\begin{itemize}
\item \localname{domain\_state->exn\_handler} points to a trap located before \funcname{entry}!
\item What installed this very first trap there?
\end{itemize}
\bigskip
\listocaml{../src/default/default.ml}{default/default.ml}
\bigskip
\inputminted[fontsize=\footnotesize]{text}{../src/default/default.out}
\end{column}
\begin{column}{0.5\textwidth}
\centering
\only<1>{\providecommand\step{0}\input{diagrams/default/default.tex}}%
\only<2>{\providecommand\step{4}\input{diagrams/default/default.tex}}%
\end{column}
\end{columns}
\end{frame}


\begin{frame}{Startup}{How did we get there by the way?}
\begin{columns}
\begin{column}{0.45\textwidth}
\begin{itemize}
\item The entry point address of the binary is \funcname{main} (\funcname{\_start}) from the runtime
\item The program starts like a C program:
\begin{itemize}
\item \funcname{caml\_start\_program}
\item \funcname{caml\_startup\_common}
\item \funcname{caml\_startup\_exn}
\item \funcname{caml\_startup}
\item \funcname{caml\_main}
\item \funcname{main}
\end{itemize}
%\item \funcname{caml\_startup\_common}: initializes GC, domains \dots
% Also: codefrag, locale, custom opeartions, frame_descr, signals, stack
\item \funcname{caml\_start\_program} is the transition point between the C realm and the OCaml one
\end{itemize}
\bigskip
\listocaml{../src/default/default.ml}{default/default.ml}
\end{column}
\begin{column}{0.55\textwidth}
\listasm[lastline=24,highlightlines={22-24}]{src/ocaml/caml\_start\_program.s}{runtime/amd64.S:604}
\end{column}
\end{columns}
\end{frame}


\begin{frame}{Setup to jump into OCaml entry point}{\funcname{caml\_start\_program} initializes the main fiber}
\begin{columns}
\begin{column}{0.6\textwidth}
\only<1,3,5>{
\begin{itemize}
\item<1-> Stores information to allow switching from OCaml stack to the C stack
\item<3-> Constructs the default exception handler
\item<5-> Switches to the OCaml stack and calls \funcname{caml\_program}
\end{itemize}
\bigskip
\listocaml{../src/default/default.ml}{default/default.ml}
}%
\only<2>{
\listasm[firstline=22,lastline=41,highlightlines={25-27}]{src/ocaml/caml\_start\_program.s}{caml\_start\_program}
}%
\only<4>{
\mintasm[firstline=22,lastline=41,highlightlines={31-38}]{src/ocaml/caml\_start\_program.s}
\listasm[firstline=66,lastline=72]{src/ocaml/caml\_start\_program.s}{caml\_start\_program}
}%
\only<6>{
\listasm[firstline=22,lastline=41,highlightlines={39-41}]{src/ocaml/caml\_start\_program.s}{caml\_start\_program}
}%
\end{column}
\begin{column}{0.4\textwidth}
\centering
\only<1-2>{\providecommand\step{2}\input{diagrams/default/default.tex}}%
\only<3-5>{\providecommand\step{3}\input{diagrams/default/default.tex}}%
\onslide*<6->{\providecommand\step{4}\input{diagrams/default/default.tex}}%
\end{column}
\end{columns}
\end{frame}

\begin{frame}{Executing the OCaml program}{And raising an exception without handler}
\begin{columns}
\begin{column}{0.6\textwidth}
\only<1-3>{
\begin{itemize}
\item<1-> Execution continues with OCaml code
\begin{itemize}
\item \funcname{camlDefault\_\_main\_269}
\item \funcname{camlDefault\_\_entry}
\item \funcname{caml\_program}
\end{itemize}
\item<1-> \funcname{camlDefault\_\_main\_269} raises a \typename{Failure} exception
\item<1-> \typename{Failure} is caught by \funcname{caml\_start\_program}
\begin{itemize}
\item<1-> The handler set the 2nd LSB of the exception value to \funcarg{1}
\item<3-> And then forces \funcname{caml\_start\_program} to terminate
\end{itemize}
\end{itemize}
}
\bigskip
\only<1,3>{\listocaml[highlightlines=3]{../src/default/default.ml}{default/default.ml}}%
\only<2>{\listasm[firstline=66,lastline=72,highlightlines=69]{src/ocaml/caml\_start\_program.s}{caml\_start\_program}}%
\end{column}
\begin{column}{0.4\textwidth}
\centering
\providecommand\step{4}\input{diagrams/default/default.tex}
\end{column}
\end{columns}
\end{frame}
%\only<>{\listasm[firstline=47,lastline=65]{src/ocaml/caml\_start\_program.s}{caml\_start\_program}}


\begin{frame}{Back to C}{Clean up, complain and terminate}
\begin{columns}
\begin{column}{0.45\textwidth}
\begin{itemize}
\item Backtrace:
\begin{itemize}
\item \funcname{caml\_start\_program} $\rightarrow$ OCaml code
\item \funcname{caml\_startup\_common}
\item \funcname{caml\_startup\_exn}
\item \funcname{caml\_startup}
\item \funcname{caml\_main}
\item \funcname{main}
\end{itemize}
\smallskip
\item \funcname{caml\_startup} checks the return value of \funcname{caml\_startup\_exn}
\begin{itemize}
\item The return value has been specially marked by \funcname{caml\_start\_program} handler
\item Calls \funcname{caml\_fatal\_uncaught\_exception}
\end{itemize}
\item<2-> \funcname{caml\_fatal\_uncaught\_exception} either
\begin{itemize}
\item<2-> Calls the user custom uncaught exception handler, if set with \mintinline{ocaml}{Printexc.set_uncaught_exception_handler fn}\footnotemark
\item<2-> Or falls back to the default one
\item<2-> Which notifies about the uncaught exception
\end{itemize}
\item<4-> Terminates the program either with \funcname{abort} or with a non-zero exit code
\end{itemize}
\end{column}
\begin{column}{0.55\textwidth}
\only<1>{
\listc[firstline=84,lastline=89,highlightlines=88]{../ocaml/runtime/caml/mlvalues.h}{runtime/caml/mlvalues.h:84}
\listc[firstline=138,lastline=143,highlightlines={141-142}]{../ocaml/runtime/startup\_nat.c}{runtime/startup\_nat.c:138}
}%
\only<2>{
\listc[firstline=138,highlightlines={147,149}]{../ocaml/runtime/printexc.c}{runtime/printexc.c:138}
}%
\only<3>{
\listc[firstline=110,lastline=134,highlightlines=129]{../ocaml/runtime/printexc.c}{runtime/printexc.c:138}
}%
\only<4>{
\listc[firstline=138,highlightlines={152,154}]{../ocaml/runtime/printexc.c}{runtime/printexc.c:138}
}%
\end{column}
\end{columns}
\footnotetext[1]{\url{https://v2.ocaml.org/api/Printexc.html\#1\_Uncaughtexceptions}}
\end{frame}
}%
\onslide*<6->{\providecommand\step{4}%
% Runtime's default exception handler
%
\subsection*{Runtime's default exception handler}
\frameSubsection{pictures/The\_Architect.png}{}


\begin{frame}{Default exception handler}
\begin{columns}
\begin{column}{0.5\textwidth}
\begin{itemize}
\item \localname{domain\_state->exn\_handler} points to a trap located before \funcname{entry}!
\item What installed this very first trap there?
\end{itemize}
\bigskip
\listocaml{../src/default/default.ml}{default/default.ml}
\bigskip
\inputminted[fontsize=\footnotesize]{text}{../src/default/default.out}
\end{column}
\begin{column}{0.5\textwidth}
\centering
\only<1>{\providecommand\step{0}\input{diagrams/default/default.tex}}%
\only<2>{\providecommand\step{4}\input{diagrams/default/default.tex}}%
\end{column}
\end{columns}
\end{frame}


\begin{frame}{Startup}{How did we get there by the way?}
\begin{columns}
\begin{column}{0.45\textwidth}
\begin{itemize}
\item The entry point address of the binary is \funcname{main} (\funcname{\_start}) from the runtime
\item The program starts like a C program:
\begin{itemize}
\item \funcname{caml\_start\_program}
\item \funcname{caml\_startup\_common}
\item \funcname{caml\_startup\_exn}
\item \funcname{caml\_startup}
\item \funcname{caml\_main}
\item \funcname{main}
\end{itemize}
%\item \funcname{caml\_startup\_common}: initializes GC, domains \dots
% Also: codefrag, locale, custom opeartions, frame_descr, signals, stack
\item \funcname{caml\_start\_program} is the transition point between the C realm and the OCaml one
\end{itemize}
\bigskip
\listocaml{../src/default/default.ml}{default/default.ml}
\end{column}
\begin{column}{0.55\textwidth}
\listasm[lastline=24,highlightlines={22-24}]{src/ocaml/caml\_start\_program.s}{runtime/amd64.S:604}
\end{column}
\end{columns}
\end{frame}


\begin{frame}{Setup to jump into OCaml entry point}{\funcname{caml\_start\_program} initializes the main fiber}
\begin{columns}
\begin{column}{0.6\textwidth}
\only<1,3,5>{
\begin{itemize}
\item<1-> Stores information to allow switching from OCaml stack to the C stack
\item<3-> Constructs the default exception handler
\item<5-> Switches to the OCaml stack and calls \funcname{caml\_program}
\end{itemize}
\bigskip
\listocaml{../src/default/default.ml}{default/default.ml}
}%
\only<2>{
\listasm[firstline=22,lastline=41,highlightlines={25-27}]{src/ocaml/caml\_start\_program.s}{caml\_start\_program}
}%
\only<4>{
\mintasm[firstline=22,lastline=41,highlightlines={31-38}]{src/ocaml/caml\_start\_program.s}
\listasm[firstline=66,lastline=72]{src/ocaml/caml\_start\_program.s}{caml\_start\_program}
}%
\only<6>{
\listasm[firstline=22,lastline=41,highlightlines={39-41}]{src/ocaml/caml\_start\_program.s}{caml\_start\_program}
}%
\end{column}
\begin{column}{0.4\textwidth}
\centering
\only<1-2>{\providecommand\step{2}\input{diagrams/default/default.tex}}%
\only<3-5>{\providecommand\step{3}\input{diagrams/default/default.tex}}%
\onslide*<6->{\providecommand\step{4}\input{diagrams/default/default.tex}}%
\end{column}
\end{columns}
\end{frame}

\begin{frame}{Executing the OCaml program}{And raising an exception without handler}
\begin{columns}
\begin{column}{0.6\textwidth}
\only<1-3>{
\begin{itemize}
\item<1-> Execution continues with OCaml code
\begin{itemize}
\item \funcname{camlDefault\_\_main\_269}
\item \funcname{camlDefault\_\_entry}
\item \funcname{caml\_program}
\end{itemize}
\item<1-> \funcname{camlDefault\_\_main\_269} raises a \typename{Failure} exception
\item<1-> \typename{Failure} is caught by \funcname{caml\_start\_program}
\begin{itemize}
\item<1-> The handler set the 2nd LSB of the exception value to \funcarg{1}
\item<3-> And then forces \funcname{caml\_start\_program} to terminate
\end{itemize}
\end{itemize}
}
\bigskip
\only<1,3>{\listocaml[highlightlines=3]{../src/default/default.ml}{default/default.ml}}%
\only<2>{\listasm[firstline=66,lastline=72,highlightlines=69]{src/ocaml/caml\_start\_program.s}{caml\_start\_program}}%
\end{column}
\begin{column}{0.4\textwidth}
\centering
\providecommand\step{4}\input{diagrams/default/default.tex}
\end{column}
\end{columns}
\end{frame}
%\only<>{\listasm[firstline=47,lastline=65]{src/ocaml/caml\_start\_program.s}{caml\_start\_program}}


\begin{frame}{Back to C}{Clean up, complain and terminate}
\begin{columns}
\begin{column}{0.45\textwidth}
\begin{itemize}
\item Backtrace:
\begin{itemize}
\item \funcname{caml\_start\_program} $\rightarrow$ OCaml code
\item \funcname{caml\_startup\_common}
\item \funcname{caml\_startup\_exn}
\item \funcname{caml\_startup}
\item \funcname{caml\_main}
\item \funcname{main}
\end{itemize}
\smallskip
\item \funcname{caml\_startup} checks the return value of \funcname{caml\_startup\_exn}
\begin{itemize}
\item The return value has been specially marked by \funcname{caml\_start\_program} handler
\item Calls \funcname{caml\_fatal\_uncaught\_exception}
\end{itemize}
\item<2-> \funcname{caml\_fatal\_uncaught\_exception} either
\begin{itemize}
\item<2-> Calls the user custom uncaught exception handler, if set with \mintinline{ocaml}{Printexc.set_uncaught_exception_handler fn}\footnotemark
\item<2-> Or falls back to the default one
\item<2-> Which notifies about the uncaught exception
\end{itemize}
\item<4-> Terminates the program either with \funcname{abort} or with a non-zero exit code
\end{itemize}
\end{column}
\begin{column}{0.55\textwidth}
\only<1>{
\listc[firstline=84,lastline=89,highlightlines=88]{../ocaml/runtime/caml/mlvalues.h}{runtime/caml/mlvalues.h:84}
\listc[firstline=138,lastline=143,highlightlines={141-142}]{../ocaml/runtime/startup\_nat.c}{runtime/startup\_nat.c:138}
}%
\only<2>{
\listc[firstline=138,highlightlines={147,149}]{../ocaml/runtime/printexc.c}{runtime/printexc.c:138}
}%
\only<3>{
\listc[firstline=110,lastline=134,highlightlines=129]{../ocaml/runtime/printexc.c}{runtime/printexc.c:138}
}%
\only<4>{
\listc[firstline=138,highlightlines={152,154}]{../ocaml/runtime/printexc.c}{runtime/printexc.c:138}
}%
\end{column}
\end{columns}
\footnotetext[1]{\url{https://v2.ocaml.org/api/Printexc.html\#1\_Uncaughtexceptions}}
\end{frame}
}%
\end{column}
\end{columns}
\end{frame}

\begin{frame}{Executing the OCaml program}{And raising an exception without handler}
\begin{columns}
\begin{column}{0.6\textwidth}
\only<1-3>{
\begin{itemize}
\item<1-> Execution continues with OCaml code
\begin{itemize}
\item \funcname{camlDefault\_\_main\_269}
\item \funcname{camlDefault\_\_entry}
\item \funcname{caml\_program}
\end{itemize}
\item<1-> \funcname{camlDefault\_\_main\_269} raises a \typename{Failure} exception
\item<1-> \typename{Failure} is caught by \funcname{caml\_start\_program}
\begin{itemize}
\item<1-> The handler set the 2nd LSB of the exception value to \funcarg{1}
\item<3-> And then forces \funcname{caml\_start\_program} to terminate
\end{itemize}
\end{itemize}
}
\bigskip
\only<1,3>{\listocaml[highlightlines=3]{../src/default/default.ml}{default/default.ml}}%
\only<2>{\listasm[firstline=66,lastline=72,highlightlines=69]{src/ocaml/caml\_start\_program.s}{caml\_start\_program}}%
\end{column}
\begin{column}{0.4\textwidth}
\centering
\providecommand\step{4}%
% Runtime's default exception handler
%
\subsection*{Runtime's default exception handler}
\frameSubsection{pictures/The\_Architect.png}{}


\begin{frame}{Default exception handler}
\begin{columns}
\begin{column}{0.5\textwidth}
\begin{itemize}
\item \localname{domain\_state->exn\_handler} points to a trap located before \funcname{entry}!
\item What installed this very first trap there?
\end{itemize}
\bigskip
\listocaml{../src/default/default.ml}{default/default.ml}
\bigskip
\inputminted[fontsize=\footnotesize]{text}{../src/default/default.out}
\end{column}
\begin{column}{0.5\textwidth}
\centering
\only<1>{\providecommand\step{0}\input{diagrams/default/default.tex}}%
\only<2>{\providecommand\step{4}\input{diagrams/default/default.tex}}%
\end{column}
\end{columns}
\end{frame}


\begin{frame}{Startup}{How did we get there by the way?}
\begin{columns}
\begin{column}{0.45\textwidth}
\begin{itemize}
\item The entry point address of the binary is \funcname{main} (\funcname{\_start}) from the runtime
\item The program starts like a C program:
\begin{itemize}
\item \funcname{caml\_start\_program}
\item \funcname{caml\_startup\_common}
\item \funcname{caml\_startup\_exn}
\item \funcname{caml\_startup}
\item \funcname{caml\_main}
\item \funcname{main}
\end{itemize}
%\item \funcname{caml\_startup\_common}: initializes GC, domains \dots
% Also: codefrag, locale, custom opeartions, frame_descr, signals, stack
\item \funcname{caml\_start\_program} is the transition point between the C realm and the OCaml one
\end{itemize}
\bigskip
\listocaml{../src/default/default.ml}{default/default.ml}
\end{column}
\begin{column}{0.55\textwidth}
\listasm[lastline=24,highlightlines={22-24}]{src/ocaml/caml\_start\_program.s}{runtime/amd64.S:604}
\end{column}
\end{columns}
\end{frame}


\begin{frame}{Setup to jump into OCaml entry point}{\funcname{caml\_start\_program} initializes the main fiber}
\begin{columns}
\begin{column}{0.6\textwidth}
\only<1,3,5>{
\begin{itemize}
\item<1-> Stores information to allow switching from OCaml stack to the C stack
\item<3-> Constructs the default exception handler
\item<5-> Switches to the OCaml stack and calls \funcname{caml\_program}
\end{itemize}
\bigskip
\listocaml{../src/default/default.ml}{default/default.ml}
}%
\only<2>{
\listasm[firstline=22,lastline=41,highlightlines={25-27}]{src/ocaml/caml\_start\_program.s}{caml\_start\_program}
}%
\only<4>{
\mintasm[firstline=22,lastline=41,highlightlines={31-38}]{src/ocaml/caml\_start\_program.s}
\listasm[firstline=66,lastline=72]{src/ocaml/caml\_start\_program.s}{caml\_start\_program}
}%
\only<6>{
\listasm[firstline=22,lastline=41,highlightlines={39-41}]{src/ocaml/caml\_start\_program.s}{caml\_start\_program}
}%
\end{column}
\begin{column}{0.4\textwidth}
\centering
\only<1-2>{\providecommand\step{2}\input{diagrams/default/default.tex}}%
\only<3-5>{\providecommand\step{3}\input{diagrams/default/default.tex}}%
\onslide*<6->{\providecommand\step{4}\input{diagrams/default/default.tex}}%
\end{column}
\end{columns}
\end{frame}

\begin{frame}{Executing the OCaml program}{And raising an exception without handler}
\begin{columns}
\begin{column}{0.6\textwidth}
\only<1-3>{
\begin{itemize}
\item<1-> Execution continues with OCaml code
\begin{itemize}
\item \funcname{camlDefault\_\_main\_269}
\item \funcname{camlDefault\_\_entry}
\item \funcname{caml\_program}
\end{itemize}
\item<1-> \funcname{camlDefault\_\_main\_269} raises a \typename{Failure} exception
\item<1-> \typename{Failure} is caught by \funcname{caml\_start\_program}
\begin{itemize}
\item<1-> The handler set the 2nd LSB of the exception value to \funcarg{1}
\item<3-> And then forces \funcname{caml\_start\_program} to terminate
\end{itemize}
\end{itemize}
}
\bigskip
\only<1,3>{\listocaml[highlightlines=3]{../src/default/default.ml}{default/default.ml}}%
\only<2>{\listasm[firstline=66,lastline=72,highlightlines=69]{src/ocaml/caml\_start\_program.s}{caml\_start\_program}}%
\end{column}
\begin{column}{0.4\textwidth}
\centering
\providecommand\step{4}\input{diagrams/default/default.tex}
\end{column}
\end{columns}
\end{frame}
%\only<>{\listasm[firstline=47,lastline=65]{src/ocaml/caml\_start\_program.s}{caml\_start\_program}}


\begin{frame}{Back to C}{Clean up, complain and terminate}
\begin{columns}
\begin{column}{0.45\textwidth}
\begin{itemize}
\item Backtrace:
\begin{itemize}
\item \funcname{caml\_start\_program} $\rightarrow$ OCaml code
\item \funcname{caml\_startup\_common}
\item \funcname{caml\_startup\_exn}
\item \funcname{caml\_startup}
\item \funcname{caml\_main}
\item \funcname{main}
\end{itemize}
\smallskip
\item \funcname{caml\_startup} checks the return value of \funcname{caml\_startup\_exn}
\begin{itemize}
\item The return value has been specially marked by \funcname{caml\_start\_program} handler
\item Calls \funcname{caml\_fatal\_uncaught\_exception}
\end{itemize}
\item<2-> \funcname{caml\_fatal\_uncaught\_exception} either
\begin{itemize}
\item<2-> Calls the user custom uncaught exception handler, if set with \mintinline{ocaml}{Printexc.set_uncaught_exception_handler fn}\footnotemark
\item<2-> Or falls back to the default one
\item<2-> Which notifies about the uncaught exception
\end{itemize}
\item<4-> Terminates the program either with \funcname{abort} or with a non-zero exit code
\end{itemize}
\end{column}
\begin{column}{0.55\textwidth}
\only<1>{
\listc[firstline=84,lastline=89,highlightlines=88]{../ocaml/runtime/caml/mlvalues.h}{runtime/caml/mlvalues.h:84}
\listc[firstline=138,lastline=143,highlightlines={141-142}]{../ocaml/runtime/startup\_nat.c}{runtime/startup\_nat.c:138}
}%
\only<2>{
\listc[firstline=138,highlightlines={147,149}]{../ocaml/runtime/printexc.c}{runtime/printexc.c:138}
}%
\only<3>{
\listc[firstline=110,lastline=134,highlightlines=129]{../ocaml/runtime/printexc.c}{runtime/printexc.c:138}
}%
\only<4>{
\listc[firstline=138,highlightlines={152,154}]{../ocaml/runtime/printexc.c}{runtime/printexc.c:138}
}%
\end{column}
\end{columns}
\footnotetext[1]{\url{https://v2.ocaml.org/api/Printexc.html\#1\_Uncaughtexceptions}}
\end{frame}

\end{column}
\end{columns}
\end{frame}
%\only<>{\listasm[firstline=47,lastline=65]{src/ocaml/caml\_start\_program.s}{caml\_start\_program}}


\begin{frame}{Back to C}{Clean up, complain and terminate}
\begin{columns}
\begin{column}{0.45\textwidth}
\begin{itemize}
\item Backtrace:
\begin{itemize}
\item \funcname{caml\_start\_program} $\rightarrow$ OCaml code
\item \funcname{caml\_startup\_common}
\item \funcname{caml\_startup\_exn}
\item \funcname{caml\_startup}
\item \funcname{caml\_main}
\item \funcname{main}
\end{itemize}
\smallskip
\item \funcname{caml\_startup} checks the return value of \funcname{caml\_startup\_exn}
\begin{itemize}
\item The return value has been specially marked by \funcname{caml\_start\_program} handler
\item Calls \funcname{caml\_fatal\_uncaught\_exception}
\end{itemize}
\item<2-> \funcname{caml\_fatal\_uncaught\_exception} either
\begin{itemize}
\item<2-> Calls the user custom uncaught exception handler, if set with \mintinline{ocaml}{Printexc.set_uncaught_exception_handler fn}\footnotemark
\item<2-> Or falls back to the default one
\item<2-> Which notifies about the uncaught exception
\end{itemize}
\item<4-> Terminates the program either with \funcname{abort} or with a non-zero exit code
\end{itemize}
\end{column}
\begin{column}{0.55\textwidth}
\only<1>{
\listc[firstline=84,lastline=89,highlightlines=88]{../ocaml/runtime/caml/mlvalues.h}{runtime/caml/mlvalues.h:84}
\listc[firstline=138,lastline=143,highlightlines={141-142}]{../ocaml/runtime/startup\_nat.c}{runtime/startup\_nat.c:138}
}%
\only<2>{
\listc[firstline=138,highlightlines={147,149}]{../ocaml/runtime/printexc.c}{runtime/printexc.c:138}
}%
\only<3>{
\listc[firstline=110,lastline=134,highlightlines=129]{../ocaml/runtime/printexc.c}{runtime/printexc.c:138}
}%
\only<4>{
\listc[firstline=138,highlightlines={152,154}]{../ocaml/runtime/printexc.c}{runtime/printexc.c:138}
}%
\end{column}
\end{columns}
\footnotetext[1]{\url{https://v2.ocaml.org/api/Printexc.html\#1\_Uncaughtexceptions}}
\end{frame}
}%
\onslide*<6->{\providecommand\step{4}%
% Runtime's default exception handler
%
\subsection*{Runtime's default exception handler}
\frameSubsection{pictures/The\_Architect.png}{}


\begin{frame}{Default exception handler}
\begin{columns}
\begin{column}{0.5\textwidth}
\begin{itemize}
\item \localname{domain\_state->exn\_handler} points to a trap located before \funcname{entry}!
\item What installed this very first trap there?
\end{itemize}
\bigskip
\listocaml{../src/default/default.ml}{default/default.ml}
\bigskip
\inputminted[fontsize=\footnotesize]{text}{../src/default/default.out}
\end{column}
\begin{column}{0.5\textwidth}
\centering
\only<1>{\providecommand\step{0}%
% Runtime's default exception handler
%
\subsection*{Runtime's default exception handler}
\frameSubsection{pictures/The\_Architect.png}{}


\begin{frame}{Default exception handler}
\begin{columns}
\begin{column}{0.5\textwidth}
\begin{itemize}
\item \localname{domain\_state->exn\_handler} points to a trap located before \funcname{entry}!
\item What installed this very first trap there?
\end{itemize}
\bigskip
\listocaml{../src/default/default.ml}{default/default.ml}
\bigskip
\inputminted[fontsize=\footnotesize]{text}{../src/default/default.out}
\end{column}
\begin{column}{0.5\textwidth}
\centering
\only<1>{\providecommand\step{0}\input{diagrams/default/default.tex}}%
\only<2>{\providecommand\step{4}\input{diagrams/default/default.tex}}%
\end{column}
\end{columns}
\end{frame}


\begin{frame}{Startup}{How did we get there by the way?}
\begin{columns}
\begin{column}{0.45\textwidth}
\begin{itemize}
\item The entry point address of the binary is \funcname{main} (\funcname{\_start}) from the runtime
\item The program starts like a C program:
\begin{itemize}
\item \funcname{caml\_start\_program}
\item \funcname{caml\_startup\_common}
\item \funcname{caml\_startup\_exn}
\item \funcname{caml\_startup}
\item \funcname{caml\_main}
\item \funcname{main}
\end{itemize}
%\item \funcname{caml\_startup\_common}: initializes GC, domains \dots
% Also: codefrag, locale, custom opeartions, frame_descr, signals, stack
\item \funcname{caml\_start\_program} is the transition point between the C realm and the OCaml one
\end{itemize}
\bigskip
\listocaml{../src/default/default.ml}{default/default.ml}
\end{column}
\begin{column}{0.55\textwidth}
\listasm[lastline=24,highlightlines={22-24}]{src/ocaml/caml\_start\_program.s}{runtime/amd64.S:604}
\end{column}
\end{columns}
\end{frame}


\begin{frame}{Setup to jump into OCaml entry point}{\funcname{caml\_start\_program} initializes the main fiber}
\begin{columns}
\begin{column}{0.6\textwidth}
\only<1,3,5>{
\begin{itemize}
\item<1-> Stores information to allow switching from OCaml stack to the C stack
\item<3-> Constructs the default exception handler
\item<5-> Switches to the OCaml stack and calls \funcname{caml\_program}
\end{itemize}
\bigskip
\listocaml{../src/default/default.ml}{default/default.ml}
}%
\only<2>{
\listasm[firstline=22,lastline=41,highlightlines={25-27}]{src/ocaml/caml\_start\_program.s}{caml\_start\_program}
}%
\only<4>{
\mintasm[firstline=22,lastline=41,highlightlines={31-38}]{src/ocaml/caml\_start\_program.s}
\listasm[firstline=66,lastline=72]{src/ocaml/caml\_start\_program.s}{caml\_start\_program}
}%
\only<6>{
\listasm[firstline=22,lastline=41,highlightlines={39-41}]{src/ocaml/caml\_start\_program.s}{caml\_start\_program}
}%
\end{column}
\begin{column}{0.4\textwidth}
\centering
\only<1-2>{\providecommand\step{2}\input{diagrams/default/default.tex}}%
\only<3-5>{\providecommand\step{3}\input{diagrams/default/default.tex}}%
\onslide*<6->{\providecommand\step{4}\input{diagrams/default/default.tex}}%
\end{column}
\end{columns}
\end{frame}

\begin{frame}{Executing the OCaml program}{And raising an exception without handler}
\begin{columns}
\begin{column}{0.6\textwidth}
\only<1-3>{
\begin{itemize}
\item<1-> Execution continues with OCaml code
\begin{itemize}
\item \funcname{camlDefault\_\_main\_269}
\item \funcname{camlDefault\_\_entry}
\item \funcname{caml\_program}
\end{itemize}
\item<1-> \funcname{camlDefault\_\_main\_269} raises a \typename{Failure} exception
\item<1-> \typename{Failure} is caught by \funcname{caml\_start\_program}
\begin{itemize}
\item<1-> The handler set the 2nd LSB of the exception value to \funcarg{1}
\item<3-> And then forces \funcname{caml\_start\_program} to terminate
\end{itemize}
\end{itemize}
}
\bigskip
\only<1,3>{\listocaml[highlightlines=3]{../src/default/default.ml}{default/default.ml}}%
\only<2>{\listasm[firstline=66,lastline=72,highlightlines=69]{src/ocaml/caml\_start\_program.s}{caml\_start\_program}}%
\end{column}
\begin{column}{0.4\textwidth}
\centering
\providecommand\step{4}\input{diagrams/default/default.tex}
\end{column}
\end{columns}
\end{frame}
%\only<>{\listasm[firstline=47,lastline=65]{src/ocaml/caml\_start\_program.s}{caml\_start\_program}}


\begin{frame}{Back to C}{Clean up, complain and terminate}
\begin{columns}
\begin{column}{0.45\textwidth}
\begin{itemize}
\item Backtrace:
\begin{itemize}
\item \funcname{caml\_start\_program} $\rightarrow$ OCaml code
\item \funcname{caml\_startup\_common}
\item \funcname{caml\_startup\_exn}
\item \funcname{caml\_startup}
\item \funcname{caml\_main}
\item \funcname{main}
\end{itemize}
\smallskip
\item \funcname{caml\_startup} checks the return value of \funcname{caml\_startup\_exn}
\begin{itemize}
\item The return value has been specially marked by \funcname{caml\_start\_program} handler
\item Calls \funcname{caml\_fatal\_uncaught\_exception}
\end{itemize}
\item<2-> \funcname{caml\_fatal\_uncaught\_exception} either
\begin{itemize}
\item<2-> Calls the user custom uncaught exception handler, if set with \mintinline{ocaml}{Printexc.set_uncaught_exception_handler fn}\footnotemark
\item<2-> Or falls back to the default one
\item<2-> Which notifies about the uncaught exception
\end{itemize}
\item<4-> Terminates the program either with \funcname{abort} or with a non-zero exit code
\end{itemize}
\end{column}
\begin{column}{0.55\textwidth}
\only<1>{
\listc[firstline=84,lastline=89,highlightlines=88]{../ocaml/runtime/caml/mlvalues.h}{runtime/caml/mlvalues.h:84}
\listc[firstline=138,lastline=143,highlightlines={141-142}]{../ocaml/runtime/startup\_nat.c}{runtime/startup\_nat.c:138}
}%
\only<2>{
\listc[firstline=138,highlightlines={147,149}]{../ocaml/runtime/printexc.c}{runtime/printexc.c:138}
}%
\only<3>{
\listc[firstline=110,lastline=134,highlightlines=129]{../ocaml/runtime/printexc.c}{runtime/printexc.c:138}
}%
\only<4>{
\listc[firstline=138,highlightlines={152,154}]{../ocaml/runtime/printexc.c}{runtime/printexc.c:138}
}%
\end{column}
\end{columns}
\footnotetext[1]{\url{https://v2.ocaml.org/api/Printexc.html\#1\_Uncaughtexceptions}}
\end{frame}
}%
\only<2>{\providecommand\step{4}%
% Runtime's default exception handler
%
\subsection*{Runtime's default exception handler}
\frameSubsection{pictures/The\_Architect.png}{}


\begin{frame}{Default exception handler}
\begin{columns}
\begin{column}{0.5\textwidth}
\begin{itemize}
\item \localname{domain\_state->exn\_handler} points to a trap located before \funcname{entry}!
\item What installed this very first trap there?
\end{itemize}
\bigskip
\listocaml{../src/default/default.ml}{default/default.ml}
\bigskip
\inputminted[fontsize=\footnotesize]{text}{../src/default/default.out}
\end{column}
\begin{column}{0.5\textwidth}
\centering
\only<1>{\providecommand\step{0}\input{diagrams/default/default.tex}}%
\only<2>{\providecommand\step{4}\input{diagrams/default/default.tex}}%
\end{column}
\end{columns}
\end{frame}


\begin{frame}{Startup}{How did we get there by the way?}
\begin{columns}
\begin{column}{0.45\textwidth}
\begin{itemize}
\item The entry point address of the binary is \funcname{main} (\funcname{\_start}) from the runtime
\item The program starts like a C program:
\begin{itemize}
\item \funcname{caml\_start\_program}
\item \funcname{caml\_startup\_common}
\item \funcname{caml\_startup\_exn}
\item \funcname{caml\_startup}
\item \funcname{caml\_main}
\item \funcname{main}
\end{itemize}
%\item \funcname{caml\_startup\_common}: initializes GC, domains \dots
% Also: codefrag, locale, custom opeartions, frame_descr, signals, stack
\item \funcname{caml\_start\_program} is the transition point between the C realm and the OCaml one
\end{itemize}
\bigskip
\listocaml{../src/default/default.ml}{default/default.ml}
\end{column}
\begin{column}{0.55\textwidth}
\listasm[lastline=24,highlightlines={22-24}]{src/ocaml/caml\_start\_program.s}{runtime/amd64.S:604}
\end{column}
\end{columns}
\end{frame}


\begin{frame}{Setup to jump into OCaml entry point}{\funcname{caml\_start\_program} initializes the main fiber}
\begin{columns}
\begin{column}{0.6\textwidth}
\only<1,3,5>{
\begin{itemize}
\item<1-> Stores information to allow switching from OCaml stack to the C stack
\item<3-> Constructs the default exception handler
\item<5-> Switches to the OCaml stack and calls \funcname{caml\_program}
\end{itemize}
\bigskip
\listocaml{../src/default/default.ml}{default/default.ml}
}%
\only<2>{
\listasm[firstline=22,lastline=41,highlightlines={25-27}]{src/ocaml/caml\_start\_program.s}{caml\_start\_program}
}%
\only<4>{
\mintasm[firstline=22,lastline=41,highlightlines={31-38}]{src/ocaml/caml\_start\_program.s}
\listasm[firstline=66,lastline=72]{src/ocaml/caml\_start\_program.s}{caml\_start\_program}
}%
\only<6>{
\listasm[firstline=22,lastline=41,highlightlines={39-41}]{src/ocaml/caml\_start\_program.s}{caml\_start\_program}
}%
\end{column}
\begin{column}{0.4\textwidth}
\centering
\only<1-2>{\providecommand\step{2}\input{diagrams/default/default.tex}}%
\only<3-5>{\providecommand\step{3}\input{diagrams/default/default.tex}}%
\onslide*<6->{\providecommand\step{4}\input{diagrams/default/default.tex}}%
\end{column}
\end{columns}
\end{frame}

\begin{frame}{Executing the OCaml program}{And raising an exception without handler}
\begin{columns}
\begin{column}{0.6\textwidth}
\only<1-3>{
\begin{itemize}
\item<1-> Execution continues with OCaml code
\begin{itemize}
\item \funcname{camlDefault\_\_main\_269}
\item \funcname{camlDefault\_\_entry}
\item \funcname{caml\_program}
\end{itemize}
\item<1-> \funcname{camlDefault\_\_main\_269} raises a \typename{Failure} exception
\item<1-> \typename{Failure} is caught by \funcname{caml\_start\_program}
\begin{itemize}
\item<1-> The handler set the 2nd LSB of the exception value to \funcarg{1}
\item<3-> And then forces \funcname{caml\_start\_program} to terminate
\end{itemize}
\end{itemize}
}
\bigskip
\only<1,3>{\listocaml[highlightlines=3]{../src/default/default.ml}{default/default.ml}}%
\only<2>{\listasm[firstline=66,lastline=72,highlightlines=69]{src/ocaml/caml\_start\_program.s}{caml\_start\_program}}%
\end{column}
\begin{column}{0.4\textwidth}
\centering
\providecommand\step{4}\input{diagrams/default/default.tex}
\end{column}
\end{columns}
\end{frame}
%\only<>{\listasm[firstline=47,lastline=65]{src/ocaml/caml\_start\_program.s}{caml\_start\_program}}


\begin{frame}{Back to C}{Clean up, complain and terminate}
\begin{columns}
\begin{column}{0.45\textwidth}
\begin{itemize}
\item Backtrace:
\begin{itemize}
\item \funcname{caml\_start\_program} $\rightarrow$ OCaml code
\item \funcname{caml\_startup\_common}
\item \funcname{caml\_startup\_exn}
\item \funcname{caml\_startup}
\item \funcname{caml\_main}
\item \funcname{main}
\end{itemize}
\smallskip
\item \funcname{caml\_startup} checks the return value of \funcname{caml\_startup\_exn}
\begin{itemize}
\item The return value has been specially marked by \funcname{caml\_start\_program} handler
\item Calls \funcname{caml\_fatal\_uncaught\_exception}
\end{itemize}
\item<2-> \funcname{caml\_fatal\_uncaught\_exception} either
\begin{itemize}
\item<2-> Calls the user custom uncaught exception handler, if set with \mintinline{ocaml}{Printexc.set_uncaught_exception_handler fn}\footnotemark
\item<2-> Or falls back to the default one
\item<2-> Which notifies about the uncaught exception
\end{itemize}
\item<4-> Terminates the program either with \funcname{abort} or with a non-zero exit code
\end{itemize}
\end{column}
\begin{column}{0.55\textwidth}
\only<1>{
\listc[firstline=84,lastline=89,highlightlines=88]{../ocaml/runtime/caml/mlvalues.h}{runtime/caml/mlvalues.h:84}
\listc[firstline=138,lastline=143,highlightlines={141-142}]{../ocaml/runtime/startup\_nat.c}{runtime/startup\_nat.c:138}
}%
\only<2>{
\listc[firstline=138,highlightlines={147,149}]{../ocaml/runtime/printexc.c}{runtime/printexc.c:138}
}%
\only<3>{
\listc[firstline=110,lastline=134,highlightlines=129]{../ocaml/runtime/printexc.c}{runtime/printexc.c:138}
}%
\only<4>{
\listc[firstline=138,highlightlines={152,154}]{../ocaml/runtime/printexc.c}{runtime/printexc.c:138}
}%
\end{column}
\end{columns}
\footnotetext[1]{\url{https://v2.ocaml.org/api/Printexc.html\#1\_Uncaughtexceptions}}
\end{frame}
}%
\end{column}
\end{columns}
\end{frame}


\begin{frame}{Startup}{How did we get there by the way?}
\begin{columns}
\begin{column}{0.45\textwidth}
\begin{itemize}
\item The entry point address of the binary is \funcname{main} (\funcname{\_start}) from the runtime
\item The program starts like a C program:
\begin{itemize}
\item \funcname{caml\_start\_program}
\item \funcname{caml\_startup\_common}
\item \funcname{caml\_startup\_exn}
\item \funcname{caml\_startup}
\item \funcname{caml\_main}
\item \funcname{main}
\end{itemize}
%\item \funcname{caml\_startup\_common}: initializes GC, domains \dots
% Also: codefrag, locale, custom opeartions, frame_descr, signals, stack
\item \funcname{caml\_start\_program} is the transition point between the C realm and the OCaml one
\end{itemize}
\bigskip
\listocaml{../src/default/default.ml}{default/default.ml}
\end{column}
\begin{column}{0.55\textwidth}
\listasm[lastline=24,highlightlines={22-24}]{src/ocaml/caml\_start\_program.s}{runtime/amd64.S:604}
\end{column}
\end{columns}
\end{frame}


\begin{frame}{Setup to jump into OCaml entry point}{\funcname{caml\_start\_program} initializes the main fiber}
\begin{columns}
\begin{column}{0.6\textwidth}
\only<1,3,5>{
\begin{itemize}
\item<1-> Stores information to allow switching from OCaml stack to the C stack
\item<3-> Constructs the default exception handler
\item<5-> Switches to the OCaml stack and calls \funcname{caml\_program}
\end{itemize}
\bigskip
\listocaml{../src/default/default.ml}{default/default.ml}
}%
\only<2>{
\listasm[firstline=22,lastline=41,highlightlines={25-27}]{src/ocaml/caml\_start\_program.s}{caml\_start\_program}
}%
\only<4>{
\mintasm[firstline=22,lastline=41,highlightlines={31-38}]{src/ocaml/caml\_start\_program.s}
\listasm[firstline=66,lastline=72]{src/ocaml/caml\_start\_program.s}{caml\_start\_program}
}%
\only<6>{
\listasm[firstline=22,lastline=41,highlightlines={39-41}]{src/ocaml/caml\_start\_program.s}{caml\_start\_program}
}%
\end{column}
\begin{column}{0.4\textwidth}
\centering
\only<1-2>{\providecommand\step{2}%
% Runtime's default exception handler
%
\subsection*{Runtime's default exception handler}
\frameSubsection{pictures/The\_Architect.png}{}


\begin{frame}{Default exception handler}
\begin{columns}
\begin{column}{0.5\textwidth}
\begin{itemize}
\item \localname{domain\_state->exn\_handler} points to a trap located before \funcname{entry}!
\item What installed this very first trap there?
\end{itemize}
\bigskip
\listocaml{../src/default/default.ml}{default/default.ml}
\bigskip
\inputminted[fontsize=\footnotesize]{text}{../src/default/default.out}
\end{column}
\begin{column}{0.5\textwidth}
\centering
\only<1>{\providecommand\step{0}\input{diagrams/default/default.tex}}%
\only<2>{\providecommand\step{4}\input{diagrams/default/default.tex}}%
\end{column}
\end{columns}
\end{frame}


\begin{frame}{Startup}{How did we get there by the way?}
\begin{columns}
\begin{column}{0.45\textwidth}
\begin{itemize}
\item The entry point address of the binary is \funcname{main} (\funcname{\_start}) from the runtime
\item The program starts like a C program:
\begin{itemize}
\item \funcname{caml\_start\_program}
\item \funcname{caml\_startup\_common}
\item \funcname{caml\_startup\_exn}
\item \funcname{caml\_startup}
\item \funcname{caml\_main}
\item \funcname{main}
\end{itemize}
%\item \funcname{caml\_startup\_common}: initializes GC, domains \dots
% Also: codefrag, locale, custom opeartions, frame_descr, signals, stack
\item \funcname{caml\_start\_program} is the transition point between the C realm and the OCaml one
\end{itemize}
\bigskip
\listocaml{../src/default/default.ml}{default/default.ml}
\end{column}
\begin{column}{0.55\textwidth}
\listasm[lastline=24,highlightlines={22-24}]{src/ocaml/caml\_start\_program.s}{runtime/amd64.S:604}
\end{column}
\end{columns}
\end{frame}


\begin{frame}{Setup to jump into OCaml entry point}{\funcname{caml\_start\_program} initializes the main fiber}
\begin{columns}
\begin{column}{0.6\textwidth}
\only<1,3,5>{
\begin{itemize}
\item<1-> Stores information to allow switching from OCaml stack to the C stack
\item<3-> Constructs the default exception handler
\item<5-> Switches to the OCaml stack and calls \funcname{caml\_program}
\end{itemize}
\bigskip
\listocaml{../src/default/default.ml}{default/default.ml}
}%
\only<2>{
\listasm[firstline=22,lastline=41,highlightlines={25-27}]{src/ocaml/caml\_start\_program.s}{caml\_start\_program}
}%
\only<4>{
\mintasm[firstline=22,lastline=41,highlightlines={31-38}]{src/ocaml/caml\_start\_program.s}
\listasm[firstline=66,lastline=72]{src/ocaml/caml\_start\_program.s}{caml\_start\_program}
}%
\only<6>{
\listasm[firstline=22,lastline=41,highlightlines={39-41}]{src/ocaml/caml\_start\_program.s}{caml\_start\_program}
}%
\end{column}
\begin{column}{0.4\textwidth}
\centering
\only<1-2>{\providecommand\step{2}\input{diagrams/default/default.tex}}%
\only<3-5>{\providecommand\step{3}\input{diagrams/default/default.tex}}%
\onslide*<6->{\providecommand\step{4}\input{diagrams/default/default.tex}}%
\end{column}
\end{columns}
\end{frame}

\begin{frame}{Executing the OCaml program}{And raising an exception without handler}
\begin{columns}
\begin{column}{0.6\textwidth}
\only<1-3>{
\begin{itemize}
\item<1-> Execution continues with OCaml code
\begin{itemize}
\item \funcname{camlDefault\_\_main\_269}
\item \funcname{camlDefault\_\_entry}
\item \funcname{caml\_program}
\end{itemize}
\item<1-> \funcname{camlDefault\_\_main\_269} raises a \typename{Failure} exception
\item<1-> \typename{Failure} is caught by \funcname{caml\_start\_program}
\begin{itemize}
\item<1-> The handler set the 2nd LSB of the exception value to \funcarg{1}
\item<3-> And then forces \funcname{caml\_start\_program} to terminate
\end{itemize}
\end{itemize}
}
\bigskip
\only<1,3>{\listocaml[highlightlines=3]{../src/default/default.ml}{default/default.ml}}%
\only<2>{\listasm[firstline=66,lastline=72,highlightlines=69]{src/ocaml/caml\_start\_program.s}{caml\_start\_program}}%
\end{column}
\begin{column}{0.4\textwidth}
\centering
\providecommand\step{4}\input{diagrams/default/default.tex}
\end{column}
\end{columns}
\end{frame}
%\only<>{\listasm[firstline=47,lastline=65]{src/ocaml/caml\_start\_program.s}{caml\_start\_program}}


\begin{frame}{Back to C}{Clean up, complain and terminate}
\begin{columns}
\begin{column}{0.45\textwidth}
\begin{itemize}
\item Backtrace:
\begin{itemize}
\item \funcname{caml\_start\_program} $\rightarrow$ OCaml code
\item \funcname{caml\_startup\_common}
\item \funcname{caml\_startup\_exn}
\item \funcname{caml\_startup}
\item \funcname{caml\_main}
\item \funcname{main}
\end{itemize}
\smallskip
\item \funcname{caml\_startup} checks the return value of \funcname{caml\_startup\_exn}
\begin{itemize}
\item The return value has been specially marked by \funcname{caml\_start\_program} handler
\item Calls \funcname{caml\_fatal\_uncaught\_exception}
\end{itemize}
\item<2-> \funcname{caml\_fatal\_uncaught\_exception} either
\begin{itemize}
\item<2-> Calls the user custom uncaught exception handler, if set with \mintinline{ocaml}{Printexc.set_uncaught_exception_handler fn}\footnotemark
\item<2-> Or falls back to the default one
\item<2-> Which notifies about the uncaught exception
\end{itemize}
\item<4-> Terminates the program either with \funcname{abort} or with a non-zero exit code
\end{itemize}
\end{column}
\begin{column}{0.55\textwidth}
\only<1>{
\listc[firstline=84,lastline=89,highlightlines=88]{../ocaml/runtime/caml/mlvalues.h}{runtime/caml/mlvalues.h:84}
\listc[firstline=138,lastline=143,highlightlines={141-142}]{../ocaml/runtime/startup\_nat.c}{runtime/startup\_nat.c:138}
}%
\only<2>{
\listc[firstline=138,highlightlines={147,149}]{../ocaml/runtime/printexc.c}{runtime/printexc.c:138}
}%
\only<3>{
\listc[firstline=110,lastline=134,highlightlines=129]{../ocaml/runtime/printexc.c}{runtime/printexc.c:138}
}%
\only<4>{
\listc[firstline=138,highlightlines={152,154}]{../ocaml/runtime/printexc.c}{runtime/printexc.c:138}
}%
\end{column}
\end{columns}
\footnotetext[1]{\url{https://v2.ocaml.org/api/Printexc.html\#1\_Uncaughtexceptions}}
\end{frame}
}%
\only<3-5>{\providecommand\step{3}%
% Runtime's default exception handler
%
\subsection*{Runtime's default exception handler}
\frameSubsection{pictures/The\_Architect.png}{}


\begin{frame}{Default exception handler}
\begin{columns}
\begin{column}{0.5\textwidth}
\begin{itemize}
\item \localname{domain\_state->exn\_handler} points to a trap located before \funcname{entry}!
\item What installed this very first trap there?
\end{itemize}
\bigskip
\listocaml{../src/default/default.ml}{default/default.ml}
\bigskip
\inputminted[fontsize=\footnotesize]{text}{../src/default/default.out}
\end{column}
\begin{column}{0.5\textwidth}
\centering
\only<1>{\providecommand\step{0}\input{diagrams/default/default.tex}}%
\only<2>{\providecommand\step{4}\input{diagrams/default/default.tex}}%
\end{column}
\end{columns}
\end{frame}


\begin{frame}{Startup}{How did we get there by the way?}
\begin{columns}
\begin{column}{0.45\textwidth}
\begin{itemize}
\item The entry point address of the binary is \funcname{main} (\funcname{\_start}) from the runtime
\item The program starts like a C program:
\begin{itemize}
\item \funcname{caml\_start\_program}
\item \funcname{caml\_startup\_common}
\item \funcname{caml\_startup\_exn}
\item \funcname{caml\_startup}
\item \funcname{caml\_main}
\item \funcname{main}
\end{itemize}
%\item \funcname{caml\_startup\_common}: initializes GC, domains \dots
% Also: codefrag, locale, custom opeartions, frame_descr, signals, stack
\item \funcname{caml\_start\_program} is the transition point between the C realm and the OCaml one
\end{itemize}
\bigskip
\listocaml{../src/default/default.ml}{default/default.ml}
\end{column}
\begin{column}{0.55\textwidth}
\listasm[lastline=24,highlightlines={22-24}]{src/ocaml/caml\_start\_program.s}{runtime/amd64.S:604}
\end{column}
\end{columns}
\end{frame}


\begin{frame}{Setup to jump into OCaml entry point}{\funcname{caml\_start\_program} initializes the main fiber}
\begin{columns}
\begin{column}{0.6\textwidth}
\only<1,3,5>{
\begin{itemize}
\item<1-> Stores information to allow switching from OCaml stack to the C stack
\item<3-> Constructs the default exception handler
\item<5-> Switches to the OCaml stack and calls \funcname{caml\_program}
\end{itemize}
\bigskip
\listocaml{../src/default/default.ml}{default/default.ml}
}%
\only<2>{
\listasm[firstline=22,lastline=41,highlightlines={25-27}]{src/ocaml/caml\_start\_program.s}{caml\_start\_program}
}%
\only<4>{
\mintasm[firstline=22,lastline=41,highlightlines={31-38}]{src/ocaml/caml\_start\_program.s}
\listasm[firstline=66,lastline=72]{src/ocaml/caml\_start\_program.s}{caml\_start\_program}
}%
\only<6>{
\listasm[firstline=22,lastline=41,highlightlines={39-41}]{src/ocaml/caml\_start\_program.s}{caml\_start\_program}
}%
\end{column}
\begin{column}{0.4\textwidth}
\centering
\only<1-2>{\providecommand\step{2}\input{diagrams/default/default.tex}}%
\only<3-5>{\providecommand\step{3}\input{diagrams/default/default.tex}}%
\onslide*<6->{\providecommand\step{4}\input{diagrams/default/default.tex}}%
\end{column}
\end{columns}
\end{frame}

\begin{frame}{Executing the OCaml program}{And raising an exception without handler}
\begin{columns}
\begin{column}{0.6\textwidth}
\only<1-3>{
\begin{itemize}
\item<1-> Execution continues with OCaml code
\begin{itemize}
\item \funcname{camlDefault\_\_main\_269}
\item \funcname{camlDefault\_\_entry}
\item \funcname{caml\_program}
\end{itemize}
\item<1-> \funcname{camlDefault\_\_main\_269} raises a \typename{Failure} exception
\item<1-> \typename{Failure} is caught by \funcname{caml\_start\_program}
\begin{itemize}
\item<1-> The handler set the 2nd LSB of the exception value to \funcarg{1}
\item<3-> And then forces \funcname{caml\_start\_program} to terminate
\end{itemize}
\end{itemize}
}
\bigskip
\only<1,3>{\listocaml[highlightlines=3]{../src/default/default.ml}{default/default.ml}}%
\only<2>{\listasm[firstline=66,lastline=72,highlightlines=69]{src/ocaml/caml\_start\_program.s}{caml\_start\_program}}%
\end{column}
\begin{column}{0.4\textwidth}
\centering
\providecommand\step{4}\input{diagrams/default/default.tex}
\end{column}
\end{columns}
\end{frame}
%\only<>{\listasm[firstline=47,lastline=65]{src/ocaml/caml\_start\_program.s}{caml\_start\_program}}


\begin{frame}{Back to C}{Clean up, complain and terminate}
\begin{columns}
\begin{column}{0.45\textwidth}
\begin{itemize}
\item Backtrace:
\begin{itemize}
\item \funcname{caml\_start\_program} $\rightarrow$ OCaml code
\item \funcname{caml\_startup\_common}
\item \funcname{caml\_startup\_exn}
\item \funcname{caml\_startup}
\item \funcname{caml\_main}
\item \funcname{main}
\end{itemize}
\smallskip
\item \funcname{caml\_startup} checks the return value of \funcname{caml\_startup\_exn}
\begin{itemize}
\item The return value has been specially marked by \funcname{caml\_start\_program} handler
\item Calls \funcname{caml\_fatal\_uncaught\_exception}
\end{itemize}
\item<2-> \funcname{caml\_fatal\_uncaught\_exception} either
\begin{itemize}
\item<2-> Calls the user custom uncaught exception handler, if set with \mintinline{ocaml}{Printexc.set_uncaught_exception_handler fn}\footnotemark
\item<2-> Or falls back to the default one
\item<2-> Which notifies about the uncaught exception
\end{itemize}
\item<4-> Terminates the program either with \funcname{abort} or with a non-zero exit code
\end{itemize}
\end{column}
\begin{column}{0.55\textwidth}
\only<1>{
\listc[firstline=84,lastline=89,highlightlines=88]{../ocaml/runtime/caml/mlvalues.h}{runtime/caml/mlvalues.h:84}
\listc[firstline=138,lastline=143,highlightlines={141-142}]{../ocaml/runtime/startup\_nat.c}{runtime/startup\_nat.c:138}
}%
\only<2>{
\listc[firstline=138,highlightlines={147,149}]{../ocaml/runtime/printexc.c}{runtime/printexc.c:138}
}%
\only<3>{
\listc[firstline=110,lastline=134,highlightlines=129]{../ocaml/runtime/printexc.c}{runtime/printexc.c:138}
}%
\only<4>{
\listc[firstline=138,highlightlines={152,154}]{../ocaml/runtime/printexc.c}{runtime/printexc.c:138}
}%
\end{column}
\end{columns}
\footnotetext[1]{\url{https://v2.ocaml.org/api/Printexc.html\#1\_Uncaughtexceptions}}
\end{frame}
}%
\onslide*<6->{\providecommand\step{4}%
% Runtime's default exception handler
%
\subsection*{Runtime's default exception handler}
\frameSubsection{pictures/The\_Architect.png}{}


\begin{frame}{Default exception handler}
\begin{columns}
\begin{column}{0.5\textwidth}
\begin{itemize}
\item \localname{domain\_state->exn\_handler} points to a trap located before \funcname{entry}!
\item What installed this very first trap there?
\end{itemize}
\bigskip
\listocaml{../src/default/default.ml}{default/default.ml}
\bigskip
\inputminted[fontsize=\footnotesize]{text}{../src/default/default.out}
\end{column}
\begin{column}{0.5\textwidth}
\centering
\only<1>{\providecommand\step{0}\input{diagrams/default/default.tex}}%
\only<2>{\providecommand\step{4}\input{diagrams/default/default.tex}}%
\end{column}
\end{columns}
\end{frame}


\begin{frame}{Startup}{How did we get there by the way?}
\begin{columns}
\begin{column}{0.45\textwidth}
\begin{itemize}
\item The entry point address of the binary is \funcname{main} (\funcname{\_start}) from the runtime
\item The program starts like a C program:
\begin{itemize}
\item \funcname{caml\_start\_program}
\item \funcname{caml\_startup\_common}
\item \funcname{caml\_startup\_exn}
\item \funcname{caml\_startup}
\item \funcname{caml\_main}
\item \funcname{main}
\end{itemize}
%\item \funcname{caml\_startup\_common}: initializes GC, domains \dots
% Also: codefrag, locale, custom opeartions, frame_descr, signals, stack
\item \funcname{caml\_start\_program} is the transition point between the C realm and the OCaml one
\end{itemize}
\bigskip
\listocaml{../src/default/default.ml}{default/default.ml}
\end{column}
\begin{column}{0.55\textwidth}
\listasm[lastline=24,highlightlines={22-24}]{src/ocaml/caml\_start\_program.s}{runtime/amd64.S:604}
\end{column}
\end{columns}
\end{frame}


\begin{frame}{Setup to jump into OCaml entry point}{\funcname{caml\_start\_program} initializes the main fiber}
\begin{columns}
\begin{column}{0.6\textwidth}
\only<1,3,5>{
\begin{itemize}
\item<1-> Stores information to allow switching from OCaml stack to the C stack
\item<3-> Constructs the default exception handler
\item<5-> Switches to the OCaml stack and calls \funcname{caml\_program}
\end{itemize}
\bigskip
\listocaml{../src/default/default.ml}{default/default.ml}
}%
\only<2>{
\listasm[firstline=22,lastline=41,highlightlines={25-27}]{src/ocaml/caml\_start\_program.s}{caml\_start\_program}
}%
\only<4>{
\mintasm[firstline=22,lastline=41,highlightlines={31-38}]{src/ocaml/caml\_start\_program.s}
\listasm[firstline=66,lastline=72]{src/ocaml/caml\_start\_program.s}{caml\_start\_program}
}%
\only<6>{
\listasm[firstline=22,lastline=41,highlightlines={39-41}]{src/ocaml/caml\_start\_program.s}{caml\_start\_program}
}%
\end{column}
\begin{column}{0.4\textwidth}
\centering
\only<1-2>{\providecommand\step{2}\input{diagrams/default/default.tex}}%
\only<3-5>{\providecommand\step{3}\input{diagrams/default/default.tex}}%
\onslide*<6->{\providecommand\step{4}\input{diagrams/default/default.tex}}%
\end{column}
\end{columns}
\end{frame}

\begin{frame}{Executing the OCaml program}{And raising an exception without handler}
\begin{columns}
\begin{column}{0.6\textwidth}
\only<1-3>{
\begin{itemize}
\item<1-> Execution continues with OCaml code
\begin{itemize}
\item \funcname{camlDefault\_\_main\_269}
\item \funcname{camlDefault\_\_entry}
\item \funcname{caml\_program}
\end{itemize}
\item<1-> \funcname{camlDefault\_\_main\_269} raises a \typename{Failure} exception
\item<1-> \typename{Failure} is caught by \funcname{caml\_start\_program}
\begin{itemize}
\item<1-> The handler set the 2nd LSB of the exception value to \funcarg{1}
\item<3-> And then forces \funcname{caml\_start\_program} to terminate
\end{itemize}
\end{itemize}
}
\bigskip
\only<1,3>{\listocaml[highlightlines=3]{../src/default/default.ml}{default/default.ml}}%
\only<2>{\listasm[firstline=66,lastline=72,highlightlines=69]{src/ocaml/caml\_start\_program.s}{caml\_start\_program}}%
\end{column}
\begin{column}{0.4\textwidth}
\centering
\providecommand\step{4}\input{diagrams/default/default.tex}
\end{column}
\end{columns}
\end{frame}
%\only<>{\listasm[firstline=47,lastline=65]{src/ocaml/caml\_start\_program.s}{caml\_start\_program}}


\begin{frame}{Back to C}{Clean up, complain and terminate}
\begin{columns}
\begin{column}{0.45\textwidth}
\begin{itemize}
\item Backtrace:
\begin{itemize}
\item \funcname{caml\_start\_program} $\rightarrow$ OCaml code
\item \funcname{caml\_startup\_common}
\item \funcname{caml\_startup\_exn}
\item \funcname{caml\_startup}
\item \funcname{caml\_main}
\item \funcname{main}
\end{itemize}
\smallskip
\item \funcname{caml\_startup} checks the return value of \funcname{caml\_startup\_exn}
\begin{itemize}
\item The return value has been specially marked by \funcname{caml\_start\_program} handler
\item Calls \funcname{caml\_fatal\_uncaught\_exception}
\end{itemize}
\item<2-> \funcname{caml\_fatal\_uncaught\_exception} either
\begin{itemize}
\item<2-> Calls the user custom uncaught exception handler, if set with \mintinline{ocaml}{Printexc.set_uncaught_exception_handler fn}\footnotemark
\item<2-> Or falls back to the default one
\item<2-> Which notifies about the uncaught exception
\end{itemize}
\item<4-> Terminates the program either with \funcname{abort} or with a non-zero exit code
\end{itemize}
\end{column}
\begin{column}{0.55\textwidth}
\only<1>{
\listc[firstline=84,lastline=89,highlightlines=88]{../ocaml/runtime/caml/mlvalues.h}{runtime/caml/mlvalues.h:84}
\listc[firstline=138,lastline=143,highlightlines={141-142}]{../ocaml/runtime/startup\_nat.c}{runtime/startup\_nat.c:138}
}%
\only<2>{
\listc[firstline=138,highlightlines={147,149}]{../ocaml/runtime/printexc.c}{runtime/printexc.c:138}
}%
\only<3>{
\listc[firstline=110,lastline=134,highlightlines=129]{../ocaml/runtime/printexc.c}{runtime/printexc.c:138}
}%
\only<4>{
\listc[firstline=138,highlightlines={152,154}]{../ocaml/runtime/printexc.c}{runtime/printexc.c:138}
}%
\end{column}
\end{columns}
\footnotetext[1]{\url{https://v2.ocaml.org/api/Printexc.html\#1\_Uncaughtexceptions}}
\end{frame}
}%
\end{column}
\end{columns}
\end{frame}

\begin{frame}{Executing the OCaml program}{And raising an exception without handler}
\begin{columns}
\begin{column}{0.6\textwidth}
\only<1-3>{
\begin{itemize}
\item<1-> Execution continues with OCaml code
\begin{itemize}
\item \funcname{camlDefault\_\_main\_269}
\item \funcname{camlDefault\_\_entry}
\item \funcname{caml\_program}
\end{itemize}
\item<1-> \funcname{camlDefault\_\_main\_269} raises a \typename{Failure} exception
\item<1-> \typename{Failure} is caught by \funcname{caml\_start\_program}
\begin{itemize}
\item<1-> The handler set the 2nd LSB of the exception value to \funcarg{1}
\item<3-> And then forces \funcname{caml\_start\_program} to terminate
\end{itemize}
\end{itemize}
}
\bigskip
\only<1,3>{\listocaml[highlightlines=3]{../src/default/default.ml}{default/default.ml}}%
\only<2>{\listasm[firstline=66,lastline=72,highlightlines=69]{src/ocaml/caml\_start\_program.s}{caml\_start\_program}}%
\end{column}
\begin{column}{0.4\textwidth}
\centering
\providecommand\step{4}%
% Runtime's default exception handler
%
\subsection*{Runtime's default exception handler}
\frameSubsection{pictures/The\_Architect.png}{}


\begin{frame}{Default exception handler}
\begin{columns}
\begin{column}{0.5\textwidth}
\begin{itemize}
\item \localname{domain\_state->exn\_handler} points to a trap located before \funcname{entry}!
\item What installed this very first trap there?
\end{itemize}
\bigskip
\listocaml{../src/default/default.ml}{default/default.ml}
\bigskip
\inputminted[fontsize=\footnotesize]{text}{../src/default/default.out}
\end{column}
\begin{column}{0.5\textwidth}
\centering
\only<1>{\providecommand\step{0}\input{diagrams/default/default.tex}}%
\only<2>{\providecommand\step{4}\input{diagrams/default/default.tex}}%
\end{column}
\end{columns}
\end{frame}


\begin{frame}{Startup}{How did we get there by the way?}
\begin{columns}
\begin{column}{0.45\textwidth}
\begin{itemize}
\item The entry point address of the binary is \funcname{main} (\funcname{\_start}) from the runtime
\item The program starts like a C program:
\begin{itemize}
\item \funcname{caml\_start\_program}
\item \funcname{caml\_startup\_common}
\item \funcname{caml\_startup\_exn}
\item \funcname{caml\_startup}
\item \funcname{caml\_main}
\item \funcname{main}
\end{itemize}
%\item \funcname{caml\_startup\_common}: initializes GC, domains \dots
% Also: codefrag, locale, custom opeartions, frame_descr, signals, stack
\item \funcname{caml\_start\_program} is the transition point between the C realm and the OCaml one
\end{itemize}
\bigskip
\listocaml{../src/default/default.ml}{default/default.ml}
\end{column}
\begin{column}{0.55\textwidth}
\listasm[lastline=24,highlightlines={22-24}]{src/ocaml/caml\_start\_program.s}{runtime/amd64.S:604}
\end{column}
\end{columns}
\end{frame}


\begin{frame}{Setup to jump into OCaml entry point}{\funcname{caml\_start\_program} initializes the main fiber}
\begin{columns}
\begin{column}{0.6\textwidth}
\only<1,3,5>{
\begin{itemize}
\item<1-> Stores information to allow switching from OCaml stack to the C stack
\item<3-> Constructs the default exception handler
\item<5-> Switches to the OCaml stack and calls \funcname{caml\_program}
\end{itemize}
\bigskip
\listocaml{../src/default/default.ml}{default/default.ml}
}%
\only<2>{
\listasm[firstline=22,lastline=41,highlightlines={25-27}]{src/ocaml/caml\_start\_program.s}{caml\_start\_program}
}%
\only<4>{
\mintasm[firstline=22,lastline=41,highlightlines={31-38}]{src/ocaml/caml\_start\_program.s}
\listasm[firstline=66,lastline=72]{src/ocaml/caml\_start\_program.s}{caml\_start\_program}
}%
\only<6>{
\listasm[firstline=22,lastline=41,highlightlines={39-41}]{src/ocaml/caml\_start\_program.s}{caml\_start\_program}
}%
\end{column}
\begin{column}{0.4\textwidth}
\centering
\only<1-2>{\providecommand\step{2}\input{diagrams/default/default.tex}}%
\only<3-5>{\providecommand\step{3}\input{diagrams/default/default.tex}}%
\onslide*<6->{\providecommand\step{4}\input{diagrams/default/default.tex}}%
\end{column}
\end{columns}
\end{frame}

\begin{frame}{Executing the OCaml program}{And raising an exception without handler}
\begin{columns}
\begin{column}{0.6\textwidth}
\only<1-3>{
\begin{itemize}
\item<1-> Execution continues with OCaml code
\begin{itemize}
\item \funcname{camlDefault\_\_main\_269}
\item \funcname{camlDefault\_\_entry}
\item \funcname{caml\_program}
\end{itemize}
\item<1-> \funcname{camlDefault\_\_main\_269} raises a \typename{Failure} exception
\item<1-> \typename{Failure} is caught by \funcname{caml\_start\_program}
\begin{itemize}
\item<1-> The handler set the 2nd LSB of the exception value to \funcarg{1}
\item<3-> And then forces \funcname{caml\_start\_program} to terminate
\end{itemize}
\end{itemize}
}
\bigskip
\only<1,3>{\listocaml[highlightlines=3]{../src/default/default.ml}{default/default.ml}}%
\only<2>{\listasm[firstline=66,lastline=72,highlightlines=69]{src/ocaml/caml\_start\_program.s}{caml\_start\_program}}%
\end{column}
\begin{column}{0.4\textwidth}
\centering
\providecommand\step{4}\input{diagrams/default/default.tex}
\end{column}
\end{columns}
\end{frame}
%\only<>{\listasm[firstline=47,lastline=65]{src/ocaml/caml\_start\_program.s}{caml\_start\_program}}


\begin{frame}{Back to C}{Clean up, complain and terminate}
\begin{columns}
\begin{column}{0.45\textwidth}
\begin{itemize}
\item Backtrace:
\begin{itemize}
\item \funcname{caml\_start\_program} $\rightarrow$ OCaml code
\item \funcname{caml\_startup\_common}
\item \funcname{caml\_startup\_exn}
\item \funcname{caml\_startup}
\item \funcname{caml\_main}
\item \funcname{main}
\end{itemize}
\smallskip
\item \funcname{caml\_startup} checks the return value of \funcname{caml\_startup\_exn}
\begin{itemize}
\item The return value has been specially marked by \funcname{caml\_start\_program} handler
\item Calls \funcname{caml\_fatal\_uncaught\_exception}
\end{itemize}
\item<2-> \funcname{caml\_fatal\_uncaught\_exception} either
\begin{itemize}
\item<2-> Calls the user custom uncaught exception handler, if set with \mintinline{ocaml}{Printexc.set_uncaught_exception_handler fn}\footnotemark
\item<2-> Or falls back to the default one
\item<2-> Which notifies about the uncaught exception
\end{itemize}
\item<4-> Terminates the program either with \funcname{abort} or with a non-zero exit code
\end{itemize}
\end{column}
\begin{column}{0.55\textwidth}
\only<1>{
\listc[firstline=84,lastline=89,highlightlines=88]{../ocaml/runtime/caml/mlvalues.h}{runtime/caml/mlvalues.h:84}
\listc[firstline=138,lastline=143,highlightlines={141-142}]{../ocaml/runtime/startup\_nat.c}{runtime/startup\_nat.c:138}
}%
\only<2>{
\listc[firstline=138,highlightlines={147,149}]{../ocaml/runtime/printexc.c}{runtime/printexc.c:138}
}%
\only<3>{
\listc[firstline=110,lastline=134,highlightlines=129]{../ocaml/runtime/printexc.c}{runtime/printexc.c:138}
}%
\only<4>{
\listc[firstline=138,highlightlines={152,154}]{../ocaml/runtime/printexc.c}{runtime/printexc.c:138}
}%
\end{column}
\end{columns}
\footnotetext[1]{\url{https://v2.ocaml.org/api/Printexc.html\#1\_Uncaughtexceptions}}
\end{frame}

\end{column}
\end{columns}
\end{frame}
%\only<>{\listasm[firstline=47,lastline=65]{src/ocaml/caml\_start\_program.s}{caml\_start\_program}}


\begin{frame}{Back to C}{Clean up, complain and terminate}
\begin{columns}
\begin{column}{0.45\textwidth}
\begin{itemize}
\item Backtrace:
\begin{itemize}
\item \funcname{caml\_start\_program} $\rightarrow$ OCaml code
\item \funcname{caml\_startup\_common}
\item \funcname{caml\_startup\_exn}
\item \funcname{caml\_startup}
\item \funcname{caml\_main}
\item \funcname{main}
\end{itemize}
\smallskip
\item \funcname{caml\_startup} checks the return value of \funcname{caml\_startup\_exn}
\begin{itemize}
\item The return value has been specially marked by \funcname{caml\_start\_program} handler
\item Calls \funcname{caml\_fatal\_uncaught\_exception}
\end{itemize}
\item<2-> \funcname{caml\_fatal\_uncaught\_exception} either
\begin{itemize}
\item<2-> Calls the user custom uncaught exception handler, if set with \mintinline{ocaml}{Printexc.set_uncaught_exception_handler fn}\footnotemark
\item<2-> Or falls back to the default one
\item<2-> Which notifies about the uncaught exception
\end{itemize}
\item<4-> Terminates the program either with \funcname{abort} or with a non-zero exit code
\end{itemize}
\end{column}
\begin{column}{0.55\textwidth}
\only<1>{
\listc[firstline=84,lastline=89,highlightlines=88]{../ocaml/runtime/caml/mlvalues.h}{runtime/caml/mlvalues.h:84}
\listc[firstline=138,lastline=143,highlightlines={141-142}]{../ocaml/runtime/startup\_nat.c}{runtime/startup\_nat.c:138}
}%
\only<2>{
\listc[firstline=138,highlightlines={147,149}]{../ocaml/runtime/printexc.c}{runtime/printexc.c:138}
}%
\only<3>{
\listc[firstline=110,lastline=134,highlightlines=129]{../ocaml/runtime/printexc.c}{runtime/printexc.c:138}
}%
\only<4>{
\listc[firstline=138,highlightlines={152,154}]{../ocaml/runtime/printexc.c}{runtime/printexc.c:138}
}%
\end{column}
\end{columns}
\footnotetext[1]{\url{https://v2.ocaml.org/api/Printexc.html\#1\_Uncaughtexceptions}}
\end{frame}
}%
\end{column}
\end{columns}
\end{frame}

\begin{frame}{Executing the OCaml program}{And raising an exception without handler}
\begin{columns}
\begin{column}{0.6\textwidth}
\only<1-3>{
\begin{itemize}
\item<1-> Execution continues with OCaml code
\begin{itemize}
\item \funcname{camlDefault\_\_main\_269}
\item \funcname{camlDefault\_\_entry}
\item \funcname{caml\_program}
\end{itemize}
\item<1-> \funcname{camlDefault\_\_main\_269} raises a \typename{Failure} exception
\item<1-> \typename{Failure} is caught by \funcname{caml\_start\_program}
\begin{itemize}
\item<1-> The handler set the 2nd LSB of the exception value to \funcarg{1}
\item<3-> And then forces \funcname{caml\_start\_program} to terminate
\end{itemize}
\end{itemize}
}
\bigskip
\only<1,3>{\listocaml[highlightlines=3]{../src/default/default.ml}{default/default.ml}}%
\only<2>{\listasm[firstline=66,lastline=72,highlightlines=69]{src/ocaml/caml\_start\_program.s}{caml\_start\_program}}%
\end{column}
\begin{column}{0.4\textwidth}
\centering
\providecommand\step{4}%
% Runtime's default exception handler
%
\subsection*{Runtime's default exception handler}
\frameSubsection{pictures/The\_Architect.png}{}


\begin{frame}{Default exception handler}
\begin{columns}
\begin{column}{0.5\textwidth}
\begin{itemize}
\item \localname{domain\_state->exn\_handler} points to a trap located before \funcname{entry}!
\item What installed this very first trap there?
\end{itemize}
\bigskip
\listocaml{../src/default/default.ml}{default/default.ml}
\bigskip
\inputminted[fontsize=\footnotesize]{text}{../src/default/default.out}
\end{column}
\begin{column}{0.5\textwidth}
\centering
\only<1>{\providecommand\step{0}%
% Runtime's default exception handler
%
\subsection*{Runtime's default exception handler}
\frameSubsection{pictures/The\_Architect.png}{}


\begin{frame}{Default exception handler}
\begin{columns}
\begin{column}{0.5\textwidth}
\begin{itemize}
\item \localname{domain\_state->exn\_handler} points to a trap located before \funcname{entry}!
\item What installed this very first trap there?
\end{itemize}
\bigskip
\listocaml{../src/default/default.ml}{default/default.ml}
\bigskip
\inputminted[fontsize=\footnotesize]{text}{../src/default/default.out}
\end{column}
\begin{column}{0.5\textwidth}
\centering
\only<1>{\providecommand\step{0}\input{diagrams/default/default.tex}}%
\only<2>{\providecommand\step{4}\input{diagrams/default/default.tex}}%
\end{column}
\end{columns}
\end{frame}


\begin{frame}{Startup}{How did we get there by the way?}
\begin{columns}
\begin{column}{0.45\textwidth}
\begin{itemize}
\item The entry point address of the binary is \funcname{main} (\funcname{\_start}) from the runtime
\item The program starts like a C program:
\begin{itemize}
\item \funcname{caml\_start\_program}
\item \funcname{caml\_startup\_common}
\item \funcname{caml\_startup\_exn}
\item \funcname{caml\_startup}
\item \funcname{caml\_main}
\item \funcname{main}
\end{itemize}
%\item \funcname{caml\_startup\_common}: initializes GC, domains \dots
% Also: codefrag, locale, custom opeartions, frame_descr, signals, stack
\item \funcname{caml\_start\_program} is the transition point between the C realm and the OCaml one
\end{itemize}
\bigskip
\listocaml{../src/default/default.ml}{default/default.ml}
\end{column}
\begin{column}{0.55\textwidth}
\listasm[lastline=24,highlightlines={22-24}]{src/ocaml/caml\_start\_program.s}{runtime/amd64.S:604}
\end{column}
\end{columns}
\end{frame}


\begin{frame}{Setup to jump into OCaml entry point}{\funcname{caml\_start\_program} initializes the main fiber}
\begin{columns}
\begin{column}{0.6\textwidth}
\only<1,3,5>{
\begin{itemize}
\item<1-> Stores information to allow switching from OCaml stack to the C stack
\item<3-> Constructs the default exception handler
\item<5-> Switches to the OCaml stack and calls \funcname{caml\_program}
\end{itemize}
\bigskip
\listocaml{../src/default/default.ml}{default/default.ml}
}%
\only<2>{
\listasm[firstline=22,lastline=41,highlightlines={25-27}]{src/ocaml/caml\_start\_program.s}{caml\_start\_program}
}%
\only<4>{
\mintasm[firstline=22,lastline=41,highlightlines={31-38}]{src/ocaml/caml\_start\_program.s}
\listasm[firstline=66,lastline=72]{src/ocaml/caml\_start\_program.s}{caml\_start\_program}
}%
\only<6>{
\listasm[firstline=22,lastline=41,highlightlines={39-41}]{src/ocaml/caml\_start\_program.s}{caml\_start\_program}
}%
\end{column}
\begin{column}{0.4\textwidth}
\centering
\only<1-2>{\providecommand\step{2}\input{diagrams/default/default.tex}}%
\only<3-5>{\providecommand\step{3}\input{diagrams/default/default.tex}}%
\onslide*<6->{\providecommand\step{4}\input{diagrams/default/default.tex}}%
\end{column}
\end{columns}
\end{frame}

\begin{frame}{Executing the OCaml program}{And raising an exception without handler}
\begin{columns}
\begin{column}{0.6\textwidth}
\only<1-3>{
\begin{itemize}
\item<1-> Execution continues with OCaml code
\begin{itemize}
\item \funcname{camlDefault\_\_main\_269}
\item \funcname{camlDefault\_\_entry}
\item \funcname{caml\_program}
\end{itemize}
\item<1-> \funcname{camlDefault\_\_main\_269} raises a \typename{Failure} exception
\item<1-> \typename{Failure} is caught by \funcname{caml\_start\_program}
\begin{itemize}
\item<1-> The handler set the 2nd LSB of the exception value to \funcarg{1}
\item<3-> And then forces \funcname{caml\_start\_program} to terminate
\end{itemize}
\end{itemize}
}
\bigskip
\only<1,3>{\listocaml[highlightlines=3]{../src/default/default.ml}{default/default.ml}}%
\only<2>{\listasm[firstline=66,lastline=72,highlightlines=69]{src/ocaml/caml\_start\_program.s}{caml\_start\_program}}%
\end{column}
\begin{column}{0.4\textwidth}
\centering
\providecommand\step{4}\input{diagrams/default/default.tex}
\end{column}
\end{columns}
\end{frame}
%\only<>{\listasm[firstline=47,lastline=65]{src/ocaml/caml\_start\_program.s}{caml\_start\_program}}


\begin{frame}{Back to C}{Clean up, complain and terminate}
\begin{columns}
\begin{column}{0.45\textwidth}
\begin{itemize}
\item Backtrace:
\begin{itemize}
\item \funcname{caml\_start\_program} $\rightarrow$ OCaml code
\item \funcname{caml\_startup\_common}
\item \funcname{caml\_startup\_exn}
\item \funcname{caml\_startup}
\item \funcname{caml\_main}
\item \funcname{main}
\end{itemize}
\smallskip
\item \funcname{caml\_startup} checks the return value of \funcname{caml\_startup\_exn}
\begin{itemize}
\item The return value has been specially marked by \funcname{caml\_start\_program} handler
\item Calls \funcname{caml\_fatal\_uncaught\_exception}
\end{itemize}
\item<2-> \funcname{caml\_fatal\_uncaught\_exception} either
\begin{itemize}
\item<2-> Calls the user custom uncaught exception handler, if set with \mintinline{ocaml}{Printexc.set_uncaught_exception_handler fn}\footnotemark
\item<2-> Or falls back to the default one
\item<2-> Which notifies about the uncaught exception
\end{itemize}
\item<4-> Terminates the program either with \funcname{abort} or with a non-zero exit code
\end{itemize}
\end{column}
\begin{column}{0.55\textwidth}
\only<1>{
\listc[firstline=84,lastline=89,highlightlines=88]{../ocaml/runtime/caml/mlvalues.h}{runtime/caml/mlvalues.h:84}
\listc[firstline=138,lastline=143,highlightlines={141-142}]{../ocaml/runtime/startup\_nat.c}{runtime/startup\_nat.c:138}
}%
\only<2>{
\listc[firstline=138,highlightlines={147,149}]{../ocaml/runtime/printexc.c}{runtime/printexc.c:138}
}%
\only<3>{
\listc[firstline=110,lastline=134,highlightlines=129]{../ocaml/runtime/printexc.c}{runtime/printexc.c:138}
}%
\only<4>{
\listc[firstline=138,highlightlines={152,154}]{../ocaml/runtime/printexc.c}{runtime/printexc.c:138}
}%
\end{column}
\end{columns}
\footnotetext[1]{\url{https://v2.ocaml.org/api/Printexc.html\#1\_Uncaughtexceptions}}
\end{frame}
}%
\only<2>{\providecommand\step{4}%
% Runtime's default exception handler
%
\subsection*{Runtime's default exception handler}
\frameSubsection{pictures/The\_Architect.png}{}


\begin{frame}{Default exception handler}
\begin{columns}
\begin{column}{0.5\textwidth}
\begin{itemize}
\item \localname{domain\_state->exn\_handler} points to a trap located before \funcname{entry}!
\item What installed this very first trap there?
\end{itemize}
\bigskip
\listocaml{../src/default/default.ml}{default/default.ml}
\bigskip
\inputminted[fontsize=\footnotesize]{text}{../src/default/default.out}
\end{column}
\begin{column}{0.5\textwidth}
\centering
\only<1>{\providecommand\step{0}\input{diagrams/default/default.tex}}%
\only<2>{\providecommand\step{4}\input{diagrams/default/default.tex}}%
\end{column}
\end{columns}
\end{frame}


\begin{frame}{Startup}{How did we get there by the way?}
\begin{columns}
\begin{column}{0.45\textwidth}
\begin{itemize}
\item The entry point address of the binary is \funcname{main} (\funcname{\_start}) from the runtime
\item The program starts like a C program:
\begin{itemize}
\item \funcname{caml\_start\_program}
\item \funcname{caml\_startup\_common}
\item \funcname{caml\_startup\_exn}
\item \funcname{caml\_startup}
\item \funcname{caml\_main}
\item \funcname{main}
\end{itemize}
%\item \funcname{caml\_startup\_common}: initializes GC, domains \dots
% Also: codefrag, locale, custom opeartions, frame_descr, signals, stack
\item \funcname{caml\_start\_program} is the transition point between the C realm and the OCaml one
\end{itemize}
\bigskip
\listocaml{../src/default/default.ml}{default/default.ml}
\end{column}
\begin{column}{0.55\textwidth}
\listasm[lastline=24,highlightlines={22-24}]{src/ocaml/caml\_start\_program.s}{runtime/amd64.S:604}
\end{column}
\end{columns}
\end{frame}


\begin{frame}{Setup to jump into OCaml entry point}{\funcname{caml\_start\_program} initializes the main fiber}
\begin{columns}
\begin{column}{0.6\textwidth}
\only<1,3,5>{
\begin{itemize}
\item<1-> Stores information to allow switching from OCaml stack to the C stack
\item<3-> Constructs the default exception handler
\item<5-> Switches to the OCaml stack and calls \funcname{caml\_program}
\end{itemize}
\bigskip
\listocaml{../src/default/default.ml}{default/default.ml}
}%
\only<2>{
\listasm[firstline=22,lastline=41,highlightlines={25-27}]{src/ocaml/caml\_start\_program.s}{caml\_start\_program}
}%
\only<4>{
\mintasm[firstline=22,lastline=41,highlightlines={31-38}]{src/ocaml/caml\_start\_program.s}
\listasm[firstline=66,lastline=72]{src/ocaml/caml\_start\_program.s}{caml\_start\_program}
}%
\only<6>{
\listasm[firstline=22,lastline=41,highlightlines={39-41}]{src/ocaml/caml\_start\_program.s}{caml\_start\_program}
}%
\end{column}
\begin{column}{0.4\textwidth}
\centering
\only<1-2>{\providecommand\step{2}\input{diagrams/default/default.tex}}%
\only<3-5>{\providecommand\step{3}\input{diagrams/default/default.tex}}%
\onslide*<6->{\providecommand\step{4}\input{diagrams/default/default.tex}}%
\end{column}
\end{columns}
\end{frame}

\begin{frame}{Executing the OCaml program}{And raising an exception without handler}
\begin{columns}
\begin{column}{0.6\textwidth}
\only<1-3>{
\begin{itemize}
\item<1-> Execution continues with OCaml code
\begin{itemize}
\item \funcname{camlDefault\_\_main\_269}
\item \funcname{camlDefault\_\_entry}
\item \funcname{caml\_program}
\end{itemize}
\item<1-> \funcname{camlDefault\_\_main\_269} raises a \typename{Failure} exception
\item<1-> \typename{Failure} is caught by \funcname{caml\_start\_program}
\begin{itemize}
\item<1-> The handler set the 2nd LSB of the exception value to \funcarg{1}
\item<3-> And then forces \funcname{caml\_start\_program} to terminate
\end{itemize}
\end{itemize}
}
\bigskip
\only<1,3>{\listocaml[highlightlines=3]{../src/default/default.ml}{default/default.ml}}%
\only<2>{\listasm[firstline=66,lastline=72,highlightlines=69]{src/ocaml/caml\_start\_program.s}{caml\_start\_program}}%
\end{column}
\begin{column}{0.4\textwidth}
\centering
\providecommand\step{4}\input{diagrams/default/default.tex}
\end{column}
\end{columns}
\end{frame}
%\only<>{\listasm[firstline=47,lastline=65]{src/ocaml/caml\_start\_program.s}{caml\_start\_program}}


\begin{frame}{Back to C}{Clean up, complain and terminate}
\begin{columns}
\begin{column}{0.45\textwidth}
\begin{itemize}
\item Backtrace:
\begin{itemize}
\item \funcname{caml\_start\_program} $\rightarrow$ OCaml code
\item \funcname{caml\_startup\_common}
\item \funcname{caml\_startup\_exn}
\item \funcname{caml\_startup}
\item \funcname{caml\_main}
\item \funcname{main}
\end{itemize}
\smallskip
\item \funcname{caml\_startup} checks the return value of \funcname{caml\_startup\_exn}
\begin{itemize}
\item The return value has been specially marked by \funcname{caml\_start\_program} handler
\item Calls \funcname{caml\_fatal\_uncaught\_exception}
\end{itemize}
\item<2-> \funcname{caml\_fatal\_uncaught\_exception} either
\begin{itemize}
\item<2-> Calls the user custom uncaught exception handler, if set with \mintinline{ocaml}{Printexc.set_uncaught_exception_handler fn}\footnotemark
\item<2-> Or falls back to the default one
\item<2-> Which notifies about the uncaught exception
\end{itemize}
\item<4-> Terminates the program either with \funcname{abort} or with a non-zero exit code
\end{itemize}
\end{column}
\begin{column}{0.55\textwidth}
\only<1>{
\listc[firstline=84,lastline=89,highlightlines=88]{../ocaml/runtime/caml/mlvalues.h}{runtime/caml/mlvalues.h:84}
\listc[firstline=138,lastline=143,highlightlines={141-142}]{../ocaml/runtime/startup\_nat.c}{runtime/startup\_nat.c:138}
}%
\only<2>{
\listc[firstline=138,highlightlines={147,149}]{../ocaml/runtime/printexc.c}{runtime/printexc.c:138}
}%
\only<3>{
\listc[firstline=110,lastline=134,highlightlines=129]{../ocaml/runtime/printexc.c}{runtime/printexc.c:138}
}%
\only<4>{
\listc[firstline=138,highlightlines={152,154}]{../ocaml/runtime/printexc.c}{runtime/printexc.c:138}
}%
\end{column}
\end{columns}
\footnotetext[1]{\url{https://v2.ocaml.org/api/Printexc.html\#1\_Uncaughtexceptions}}
\end{frame}
}%
\end{column}
\end{columns}
\end{frame}


\begin{frame}{Startup}{How did we get there by the way?}
\begin{columns}
\begin{column}{0.45\textwidth}
\begin{itemize}
\item The entry point address of the binary is \funcname{main} (\funcname{\_start}) from the runtime
\item The program starts like a C program:
\begin{itemize}
\item \funcname{caml\_start\_program}
\item \funcname{caml\_startup\_common}
\item \funcname{caml\_startup\_exn}
\item \funcname{caml\_startup}
\item \funcname{caml\_main}
\item \funcname{main}
\end{itemize}
%\item \funcname{caml\_startup\_common}: initializes GC, domains \dots
% Also: codefrag, locale, custom opeartions, frame_descr, signals, stack
\item \funcname{caml\_start\_program} is the transition point between the C realm and the OCaml one
\end{itemize}
\bigskip
\listocaml{../src/default/default.ml}{default/default.ml}
\end{column}
\begin{column}{0.55\textwidth}
\listasm[lastline=24,highlightlines={22-24}]{src/ocaml/caml\_start\_program.s}{runtime/amd64.S:604}
\end{column}
\end{columns}
\end{frame}


\begin{frame}{Setup to jump into OCaml entry point}{\funcname{caml\_start\_program} initializes the main fiber}
\begin{columns}
\begin{column}{0.6\textwidth}
\only<1,3,5>{
\begin{itemize}
\item<1-> Stores information to allow switching from OCaml stack to the C stack
\item<3-> Constructs the default exception handler
\item<5-> Switches to the OCaml stack and calls \funcname{caml\_program}
\end{itemize}
\bigskip
\listocaml{../src/default/default.ml}{default/default.ml}
}%
\only<2>{
\listasm[firstline=22,lastline=41,highlightlines={25-27}]{src/ocaml/caml\_start\_program.s}{caml\_start\_program}
}%
\only<4>{
\mintasm[firstline=22,lastline=41,highlightlines={31-38}]{src/ocaml/caml\_start\_program.s}
\listasm[firstline=66,lastline=72]{src/ocaml/caml\_start\_program.s}{caml\_start\_program}
}%
\only<6>{
\listasm[firstline=22,lastline=41,highlightlines={39-41}]{src/ocaml/caml\_start\_program.s}{caml\_start\_program}
}%
\end{column}
\begin{column}{0.4\textwidth}
\centering
\only<1-2>{\providecommand\step{2}%
% Runtime's default exception handler
%
\subsection*{Runtime's default exception handler}
\frameSubsection{pictures/The\_Architect.png}{}


\begin{frame}{Default exception handler}
\begin{columns}
\begin{column}{0.5\textwidth}
\begin{itemize}
\item \localname{domain\_state->exn\_handler} points to a trap located before \funcname{entry}!
\item What installed this very first trap there?
\end{itemize}
\bigskip
\listocaml{../src/default/default.ml}{default/default.ml}
\bigskip
\inputminted[fontsize=\footnotesize]{text}{../src/default/default.out}
\end{column}
\begin{column}{0.5\textwidth}
\centering
\only<1>{\providecommand\step{0}\input{diagrams/default/default.tex}}%
\only<2>{\providecommand\step{4}\input{diagrams/default/default.tex}}%
\end{column}
\end{columns}
\end{frame}


\begin{frame}{Startup}{How did we get there by the way?}
\begin{columns}
\begin{column}{0.45\textwidth}
\begin{itemize}
\item The entry point address of the binary is \funcname{main} (\funcname{\_start}) from the runtime
\item The program starts like a C program:
\begin{itemize}
\item \funcname{caml\_start\_program}
\item \funcname{caml\_startup\_common}
\item \funcname{caml\_startup\_exn}
\item \funcname{caml\_startup}
\item \funcname{caml\_main}
\item \funcname{main}
\end{itemize}
%\item \funcname{caml\_startup\_common}: initializes GC, domains \dots
% Also: codefrag, locale, custom opeartions, frame_descr, signals, stack
\item \funcname{caml\_start\_program} is the transition point between the C realm and the OCaml one
\end{itemize}
\bigskip
\listocaml{../src/default/default.ml}{default/default.ml}
\end{column}
\begin{column}{0.55\textwidth}
\listasm[lastline=24,highlightlines={22-24}]{src/ocaml/caml\_start\_program.s}{runtime/amd64.S:604}
\end{column}
\end{columns}
\end{frame}


\begin{frame}{Setup to jump into OCaml entry point}{\funcname{caml\_start\_program} initializes the main fiber}
\begin{columns}
\begin{column}{0.6\textwidth}
\only<1,3,5>{
\begin{itemize}
\item<1-> Stores information to allow switching from OCaml stack to the C stack
\item<3-> Constructs the default exception handler
\item<5-> Switches to the OCaml stack and calls \funcname{caml\_program}
\end{itemize}
\bigskip
\listocaml{../src/default/default.ml}{default/default.ml}
}%
\only<2>{
\listasm[firstline=22,lastline=41,highlightlines={25-27}]{src/ocaml/caml\_start\_program.s}{caml\_start\_program}
}%
\only<4>{
\mintasm[firstline=22,lastline=41,highlightlines={31-38}]{src/ocaml/caml\_start\_program.s}
\listasm[firstline=66,lastline=72]{src/ocaml/caml\_start\_program.s}{caml\_start\_program}
}%
\only<6>{
\listasm[firstline=22,lastline=41,highlightlines={39-41}]{src/ocaml/caml\_start\_program.s}{caml\_start\_program}
}%
\end{column}
\begin{column}{0.4\textwidth}
\centering
\only<1-2>{\providecommand\step{2}\input{diagrams/default/default.tex}}%
\only<3-5>{\providecommand\step{3}\input{diagrams/default/default.tex}}%
\onslide*<6->{\providecommand\step{4}\input{diagrams/default/default.tex}}%
\end{column}
\end{columns}
\end{frame}

\begin{frame}{Executing the OCaml program}{And raising an exception without handler}
\begin{columns}
\begin{column}{0.6\textwidth}
\only<1-3>{
\begin{itemize}
\item<1-> Execution continues with OCaml code
\begin{itemize}
\item \funcname{camlDefault\_\_main\_269}
\item \funcname{camlDefault\_\_entry}
\item \funcname{caml\_program}
\end{itemize}
\item<1-> \funcname{camlDefault\_\_main\_269} raises a \typename{Failure} exception
\item<1-> \typename{Failure} is caught by \funcname{caml\_start\_program}
\begin{itemize}
\item<1-> The handler set the 2nd LSB of the exception value to \funcarg{1}
\item<3-> And then forces \funcname{caml\_start\_program} to terminate
\end{itemize}
\end{itemize}
}
\bigskip
\only<1,3>{\listocaml[highlightlines=3]{../src/default/default.ml}{default/default.ml}}%
\only<2>{\listasm[firstline=66,lastline=72,highlightlines=69]{src/ocaml/caml\_start\_program.s}{caml\_start\_program}}%
\end{column}
\begin{column}{0.4\textwidth}
\centering
\providecommand\step{4}\input{diagrams/default/default.tex}
\end{column}
\end{columns}
\end{frame}
%\only<>{\listasm[firstline=47,lastline=65]{src/ocaml/caml\_start\_program.s}{caml\_start\_program}}


\begin{frame}{Back to C}{Clean up, complain and terminate}
\begin{columns}
\begin{column}{0.45\textwidth}
\begin{itemize}
\item Backtrace:
\begin{itemize}
\item \funcname{caml\_start\_program} $\rightarrow$ OCaml code
\item \funcname{caml\_startup\_common}
\item \funcname{caml\_startup\_exn}
\item \funcname{caml\_startup}
\item \funcname{caml\_main}
\item \funcname{main}
\end{itemize}
\smallskip
\item \funcname{caml\_startup} checks the return value of \funcname{caml\_startup\_exn}
\begin{itemize}
\item The return value has been specially marked by \funcname{caml\_start\_program} handler
\item Calls \funcname{caml\_fatal\_uncaught\_exception}
\end{itemize}
\item<2-> \funcname{caml\_fatal\_uncaught\_exception} either
\begin{itemize}
\item<2-> Calls the user custom uncaught exception handler, if set with \mintinline{ocaml}{Printexc.set_uncaught_exception_handler fn}\footnotemark
\item<2-> Or falls back to the default one
\item<2-> Which notifies about the uncaught exception
\end{itemize}
\item<4-> Terminates the program either with \funcname{abort} or with a non-zero exit code
\end{itemize}
\end{column}
\begin{column}{0.55\textwidth}
\only<1>{
\listc[firstline=84,lastline=89,highlightlines=88]{../ocaml/runtime/caml/mlvalues.h}{runtime/caml/mlvalues.h:84}
\listc[firstline=138,lastline=143,highlightlines={141-142}]{../ocaml/runtime/startup\_nat.c}{runtime/startup\_nat.c:138}
}%
\only<2>{
\listc[firstline=138,highlightlines={147,149}]{../ocaml/runtime/printexc.c}{runtime/printexc.c:138}
}%
\only<3>{
\listc[firstline=110,lastline=134,highlightlines=129]{../ocaml/runtime/printexc.c}{runtime/printexc.c:138}
}%
\only<4>{
\listc[firstline=138,highlightlines={152,154}]{../ocaml/runtime/printexc.c}{runtime/printexc.c:138}
}%
\end{column}
\end{columns}
\footnotetext[1]{\url{https://v2.ocaml.org/api/Printexc.html\#1\_Uncaughtexceptions}}
\end{frame}
}%
\only<3-5>{\providecommand\step{3}%
% Runtime's default exception handler
%
\subsection*{Runtime's default exception handler}
\frameSubsection{pictures/The\_Architect.png}{}


\begin{frame}{Default exception handler}
\begin{columns}
\begin{column}{0.5\textwidth}
\begin{itemize}
\item \localname{domain\_state->exn\_handler} points to a trap located before \funcname{entry}!
\item What installed this very first trap there?
\end{itemize}
\bigskip
\listocaml{../src/default/default.ml}{default/default.ml}
\bigskip
\inputminted[fontsize=\footnotesize]{text}{../src/default/default.out}
\end{column}
\begin{column}{0.5\textwidth}
\centering
\only<1>{\providecommand\step{0}\input{diagrams/default/default.tex}}%
\only<2>{\providecommand\step{4}\input{diagrams/default/default.tex}}%
\end{column}
\end{columns}
\end{frame}


\begin{frame}{Startup}{How did we get there by the way?}
\begin{columns}
\begin{column}{0.45\textwidth}
\begin{itemize}
\item The entry point address of the binary is \funcname{main} (\funcname{\_start}) from the runtime
\item The program starts like a C program:
\begin{itemize}
\item \funcname{caml\_start\_program}
\item \funcname{caml\_startup\_common}
\item \funcname{caml\_startup\_exn}
\item \funcname{caml\_startup}
\item \funcname{caml\_main}
\item \funcname{main}
\end{itemize}
%\item \funcname{caml\_startup\_common}: initializes GC, domains \dots
% Also: codefrag, locale, custom opeartions, frame_descr, signals, stack
\item \funcname{caml\_start\_program} is the transition point between the C realm and the OCaml one
\end{itemize}
\bigskip
\listocaml{../src/default/default.ml}{default/default.ml}
\end{column}
\begin{column}{0.55\textwidth}
\listasm[lastline=24,highlightlines={22-24}]{src/ocaml/caml\_start\_program.s}{runtime/amd64.S:604}
\end{column}
\end{columns}
\end{frame}


\begin{frame}{Setup to jump into OCaml entry point}{\funcname{caml\_start\_program} initializes the main fiber}
\begin{columns}
\begin{column}{0.6\textwidth}
\only<1,3,5>{
\begin{itemize}
\item<1-> Stores information to allow switching from OCaml stack to the C stack
\item<3-> Constructs the default exception handler
\item<5-> Switches to the OCaml stack and calls \funcname{caml\_program}
\end{itemize}
\bigskip
\listocaml{../src/default/default.ml}{default/default.ml}
}%
\only<2>{
\listasm[firstline=22,lastline=41,highlightlines={25-27}]{src/ocaml/caml\_start\_program.s}{caml\_start\_program}
}%
\only<4>{
\mintasm[firstline=22,lastline=41,highlightlines={31-38}]{src/ocaml/caml\_start\_program.s}
\listasm[firstline=66,lastline=72]{src/ocaml/caml\_start\_program.s}{caml\_start\_program}
}%
\only<6>{
\listasm[firstline=22,lastline=41,highlightlines={39-41}]{src/ocaml/caml\_start\_program.s}{caml\_start\_program}
}%
\end{column}
\begin{column}{0.4\textwidth}
\centering
\only<1-2>{\providecommand\step{2}\input{diagrams/default/default.tex}}%
\only<3-5>{\providecommand\step{3}\input{diagrams/default/default.tex}}%
\onslide*<6->{\providecommand\step{4}\input{diagrams/default/default.tex}}%
\end{column}
\end{columns}
\end{frame}

\begin{frame}{Executing the OCaml program}{And raising an exception without handler}
\begin{columns}
\begin{column}{0.6\textwidth}
\only<1-3>{
\begin{itemize}
\item<1-> Execution continues with OCaml code
\begin{itemize}
\item \funcname{camlDefault\_\_main\_269}
\item \funcname{camlDefault\_\_entry}
\item \funcname{caml\_program}
\end{itemize}
\item<1-> \funcname{camlDefault\_\_main\_269} raises a \typename{Failure} exception
\item<1-> \typename{Failure} is caught by \funcname{caml\_start\_program}
\begin{itemize}
\item<1-> The handler set the 2nd LSB of the exception value to \funcarg{1}
\item<3-> And then forces \funcname{caml\_start\_program} to terminate
\end{itemize}
\end{itemize}
}
\bigskip
\only<1,3>{\listocaml[highlightlines=3]{../src/default/default.ml}{default/default.ml}}%
\only<2>{\listasm[firstline=66,lastline=72,highlightlines=69]{src/ocaml/caml\_start\_program.s}{caml\_start\_program}}%
\end{column}
\begin{column}{0.4\textwidth}
\centering
\providecommand\step{4}\input{diagrams/default/default.tex}
\end{column}
\end{columns}
\end{frame}
%\only<>{\listasm[firstline=47,lastline=65]{src/ocaml/caml\_start\_program.s}{caml\_start\_program}}


\begin{frame}{Back to C}{Clean up, complain and terminate}
\begin{columns}
\begin{column}{0.45\textwidth}
\begin{itemize}
\item Backtrace:
\begin{itemize}
\item \funcname{caml\_start\_program} $\rightarrow$ OCaml code
\item \funcname{caml\_startup\_common}
\item \funcname{caml\_startup\_exn}
\item \funcname{caml\_startup}
\item \funcname{caml\_main}
\item \funcname{main}
\end{itemize}
\smallskip
\item \funcname{caml\_startup} checks the return value of \funcname{caml\_startup\_exn}
\begin{itemize}
\item The return value has been specially marked by \funcname{caml\_start\_program} handler
\item Calls \funcname{caml\_fatal\_uncaught\_exception}
\end{itemize}
\item<2-> \funcname{caml\_fatal\_uncaught\_exception} either
\begin{itemize}
\item<2-> Calls the user custom uncaught exception handler, if set with \mintinline{ocaml}{Printexc.set_uncaught_exception_handler fn}\footnotemark
\item<2-> Or falls back to the default one
\item<2-> Which notifies about the uncaught exception
\end{itemize}
\item<4-> Terminates the program either with \funcname{abort} or with a non-zero exit code
\end{itemize}
\end{column}
\begin{column}{0.55\textwidth}
\only<1>{
\listc[firstline=84,lastline=89,highlightlines=88]{../ocaml/runtime/caml/mlvalues.h}{runtime/caml/mlvalues.h:84}
\listc[firstline=138,lastline=143,highlightlines={141-142}]{../ocaml/runtime/startup\_nat.c}{runtime/startup\_nat.c:138}
}%
\only<2>{
\listc[firstline=138,highlightlines={147,149}]{../ocaml/runtime/printexc.c}{runtime/printexc.c:138}
}%
\only<3>{
\listc[firstline=110,lastline=134,highlightlines=129]{../ocaml/runtime/printexc.c}{runtime/printexc.c:138}
}%
\only<4>{
\listc[firstline=138,highlightlines={152,154}]{../ocaml/runtime/printexc.c}{runtime/printexc.c:138}
}%
\end{column}
\end{columns}
\footnotetext[1]{\url{https://v2.ocaml.org/api/Printexc.html\#1\_Uncaughtexceptions}}
\end{frame}
}%
\onslide*<6->{\providecommand\step{4}%
% Runtime's default exception handler
%
\subsection*{Runtime's default exception handler}
\frameSubsection{pictures/The\_Architect.png}{}


\begin{frame}{Default exception handler}
\begin{columns}
\begin{column}{0.5\textwidth}
\begin{itemize}
\item \localname{domain\_state->exn\_handler} points to a trap located before \funcname{entry}!
\item What installed this very first trap there?
\end{itemize}
\bigskip
\listocaml{../src/default/default.ml}{default/default.ml}
\bigskip
\inputminted[fontsize=\footnotesize]{text}{../src/default/default.out}
\end{column}
\begin{column}{0.5\textwidth}
\centering
\only<1>{\providecommand\step{0}\input{diagrams/default/default.tex}}%
\only<2>{\providecommand\step{4}\input{diagrams/default/default.tex}}%
\end{column}
\end{columns}
\end{frame}


\begin{frame}{Startup}{How did we get there by the way?}
\begin{columns}
\begin{column}{0.45\textwidth}
\begin{itemize}
\item The entry point address of the binary is \funcname{main} (\funcname{\_start}) from the runtime
\item The program starts like a C program:
\begin{itemize}
\item \funcname{caml\_start\_program}
\item \funcname{caml\_startup\_common}
\item \funcname{caml\_startup\_exn}
\item \funcname{caml\_startup}
\item \funcname{caml\_main}
\item \funcname{main}
\end{itemize}
%\item \funcname{caml\_startup\_common}: initializes GC, domains \dots
% Also: codefrag, locale, custom opeartions, frame_descr, signals, stack
\item \funcname{caml\_start\_program} is the transition point between the C realm and the OCaml one
\end{itemize}
\bigskip
\listocaml{../src/default/default.ml}{default/default.ml}
\end{column}
\begin{column}{0.55\textwidth}
\listasm[lastline=24,highlightlines={22-24}]{src/ocaml/caml\_start\_program.s}{runtime/amd64.S:604}
\end{column}
\end{columns}
\end{frame}


\begin{frame}{Setup to jump into OCaml entry point}{\funcname{caml\_start\_program} initializes the main fiber}
\begin{columns}
\begin{column}{0.6\textwidth}
\only<1,3,5>{
\begin{itemize}
\item<1-> Stores information to allow switching from OCaml stack to the C stack
\item<3-> Constructs the default exception handler
\item<5-> Switches to the OCaml stack and calls \funcname{caml\_program}
\end{itemize}
\bigskip
\listocaml{../src/default/default.ml}{default/default.ml}
}%
\only<2>{
\listasm[firstline=22,lastline=41,highlightlines={25-27}]{src/ocaml/caml\_start\_program.s}{caml\_start\_program}
}%
\only<4>{
\mintasm[firstline=22,lastline=41,highlightlines={31-38}]{src/ocaml/caml\_start\_program.s}
\listasm[firstline=66,lastline=72]{src/ocaml/caml\_start\_program.s}{caml\_start\_program}
}%
\only<6>{
\listasm[firstline=22,lastline=41,highlightlines={39-41}]{src/ocaml/caml\_start\_program.s}{caml\_start\_program}
}%
\end{column}
\begin{column}{0.4\textwidth}
\centering
\only<1-2>{\providecommand\step{2}\input{diagrams/default/default.tex}}%
\only<3-5>{\providecommand\step{3}\input{diagrams/default/default.tex}}%
\onslide*<6->{\providecommand\step{4}\input{diagrams/default/default.tex}}%
\end{column}
\end{columns}
\end{frame}

\begin{frame}{Executing the OCaml program}{And raising an exception without handler}
\begin{columns}
\begin{column}{0.6\textwidth}
\only<1-3>{
\begin{itemize}
\item<1-> Execution continues with OCaml code
\begin{itemize}
\item \funcname{camlDefault\_\_main\_269}
\item \funcname{camlDefault\_\_entry}
\item \funcname{caml\_program}
\end{itemize}
\item<1-> \funcname{camlDefault\_\_main\_269} raises a \typename{Failure} exception
\item<1-> \typename{Failure} is caught by \funcname{caml\_start\_program}
\begin{itemize}
\item<1-> The handler set the 2nd LSB of the exception value to \funcarg{1}
\item<3-> And then forces \funcname{caml\_start\_program} to terminate
\end{itemize}
\end{itemize}
}
\bigskip
\only<1,3>{\listocaml[highlightlines=3]{../src/default/default.ml}{default/default.ml}}%
\only<2>{\listasm[firstline=66,lastline=72,highlightlines=69]{src/ocaml/caml\_start\_program.s}{caml\_start\_program}}%
\end{column}
\begin{column}{0.4\textwidth}
\centering
\providecommand\step{4}\input{diagrams/default/default.tex}
\end{column}
\end{columns}
\end{frame}
%\only<>{\listasm[firstline=47,lastline=65]{src/ocaml/caml\_start\_program.s}{caml\_start\_program}}


\begin{frame}{Back to C}{Clean up, complain and terminate}
\begin{columns}
\begin{column}{0.45\textwidth}
\begin{itemize}
\item Backtrace:
\begin{itemize}
\item \funcname{caml\_start\_program} $\rightarrow$ OCaml code
\item \funcname{caml\_startup\_common}
\item \funcname{caml\_startup\_exn}
\item \funcname{caml\_startup}
\item \funcname{caml\_main}
\item \funcname{main}
\end{itemize}
\smallskip
\item \funcname{caml\_startup} checks the return value of \funcname{caml\_startup\_exn}
\begin{itemize}
\item The return value has been specially marked by \funcname{caml\_start\_program} handler
\item Calls \funcname{caml\_fatal\_uncaught\_exception}
\end{itemize}
\item<2-> \funcname{caml\_fatal\_uncaught\_exception} either
\begin{itemize}
\item<2-> Calls the user custom uncaught exception handler, if set with \mintinline{ocaml}{Printexc.set_uncaught_exception_handler fn}\footnotemark
\item<2-> Or falls back to the default one
\item<2-> Which notifies about the uncaught exception
\end{itemize}
\item<4-> Terminates the program either with \funcname{abort} or with a non-zero exit code
\end{itemize}
\end{column}
\begin{column}{0.55\textwidth}
\only<1>{
\listc[firstline=84,lastline=89,highlightlines=88]{../ocaml/runtime/caml/mlvalues.h}{runtime/caml/mlvalues.h:84}
\listc[firstline=138,lastline=143,highlightlines={141-142}]{../ocaml/runtime/startup\_nat.c}{runtime/startup\_nat.c:138}
}%
\only<2>{
\listc[firstline=138,highlightlines={147,149}]{../ocaml/runtime/printexc.c}{runtime/printexc.c:138}
}%
\only<3>{
\listc[firstline=110,lastline=134,highlightlines=129]{../ocaml/runtime/printexc.c}{runtime/printexc.c:138}
}%
\only<4>{
\listc[firstline=138,highlightlines={152,154}]{../ocaml/runtime/printexc.c}{runtime/printexc.c:138}
}%
\end{column}
\end{columns}
\footnotetext[1]{\url{https://v2.ocaml.org/api/Printexc.html\#1\_Uncaughtexceptions}}
\end{frame}
}%
\end{column}
\end{columns}
\end{frame}

\begin{frame}{Executing the OCaml program}{And raising an exception without handler}
\begin{columns}
\begin{column}{0.6\textwidth}
\only<1-3>{
\begin{itemize}
\item<1-> Execution continues with OCaml code
\begin{itemize}
\item \funcname{camlDefault\_\_main\_269}
\item \funcname{camlDefault\_\_entry}
\item \funcname{caml\_program}
\end{itemize}
\item<1-> \funcname{camlDefault\_\_main\_269} raises a \typename{Failure} exception
\item<1-> \typename{Failure} is caught by \funcname{caml\_start\_program}
\begin{itemize}
\item<1-> The handler set the 2nd LSB of the exception value to \funcarg{1}
\item<3-> And then forces \funcname{caml\_start\_program} to terminate
\end{itemize}
\end{itemize}
}
\bigskip
\only<1,3>{\listocaml[highlightlines=3]{../src/default/default.ml}{default/default.ml}}%
\only<2>{\listasm[firstline=66,lastline=72,highlightlines=69]{src/ocaml/caml\_start\_program.s}{caml\_start\_program}}%
\end{column}
\begin{column}{0.4\textwidth}
\centering
\providecommand\step{4}%
% Runtime's default exception handler
%
\subsection*{Runtime's default exception handler}
\frameSubsection{pictures/The\_Architect.png}{}


\begin{frame}{Default exception handler}
\begin{columns}
\begin{column}{0.5\textwidth}
\begin{itemize}
\item \localname{domain\_state->exn\_handler} points to a trap located before \funcname{entry}!
\item What installed this very first trap there?
\end{itemize}
\bigskip
\listocaml{../src/default/default.ml}{default/default.ml}
\bigskip
\inputminted[fontsize=\footnotesize]{text}{../src/default/default.out}
\end{column}
\begin{column}{0.5\textwidth}
\centering
\only<1>{\providecommand\step{0}\input{diagrams/default/default.tex}}%
\only<2>{\providecommand\step{4}\input{diagrams/default/default.tex}}%
\end{column}
\end{columns}
\end{frame}


\begin{frame}{Startup}{How did we get there by the way?}
\begin{columns}
\begin{column}{0.45\textwidth}
\begin{itemize}
\item The entry point address of the binary is \funcname{main} (\funcname{\_start}) from the runtime
\item The program starts like a C program:
\begin{itemize}
\item \funcname{caml\_start\_program}
\item \funcname{caml\_startup\_common}
\item \funcname{caml\_startup\_exn}
\item \funcname{caml\_startup}
\item \funcname{caml\_main}
\item \funcname{main}
\end{itemize}
%\item \funcname{caml\_startup\_common}: initializes GC, domains \dots
% Also: codefrag, locale, custom opeartions, frame_descr, signals, stack
\item \funcname{caml\_start\_program} is the transition point between the C realm and the OCaml one
\end{itemize}
\bigskip
\listocaml{../src/default/default.ml}{default/default.ml}
\end{column}
\begin{column}{0.55\textwidth}
\listasm[lastline=24,highlightlines={22-24}]{src/ocaml/caml\_start\_program.s}{runtime/amd64.S:604}
\end{column}
\end{columns}
\end{frame}


\begin{frame}{Setup to jump into OCaml entry point}{\funcname{caml\_start\_program} initializes the main fiber}
\begin{columns}
\begin{column}{0.6\textwidth}
\only<1,3,5>{
\begin{itemize}
\item<1-> Stores information to allow switching from OCaml stack to the C stack
\item<3-> Constructs the default exception handler
\item<5-> Switches to the OCaml stack and calls \funcname{caml\_program}
\end{itemize}
\bigskip
\listocaml{../src/default/default.ml}{default/default.ml}
}%
\only<2>{
\listasm[firstline=22,lastline=41,highlightlines={25-27}]{src/ocaml/caml\_start\_program.s}{caml\_start\_program}
}%
\only<4>{
\mintasm[firstline=22,lastline=41,highlightlines={31-38}]{src/ocaml/caml\_start\_program.s}
\listasm[firstline=66,lastline=72]{src/ocaml/caml\_start\_program.s}{caml\_start\_program}
}%
\only<6>{
\listasm[firstline=22,lastline=41,highlightlines={39-41}]{src/ocaml/caml\_start\_program.s}{caml\_start\_program}
}%
\end{column}
\begin{column}{0.4\textwidth}
\centering
\only<1-2>{\providecommand\step{2}\input{diagrams/default/default.tex}}%
\only<3-5>{\providecommand\step{3}\input{diagrams/default/default.tex}}%
\onslide*<6->{\providecommand\step{4}\input{diagrams/default/default.tex}}%
\end{column}
\end{columns}
\end{frame}

\begin{frame}{Executing the OCaml program}{And raising an exception without handler}
\begin{columns}
\begin{column}{0.6\textwidth}
\only<1-3>{
\begin{itemize}
\item<1-> Execution continues with OCaml code
\begin{itemize}
\item \funcname{camlDefault\_\_main\_269}
\item \funcname{camlDefault\_\_entry}
\item \funcname{caml\_program}
\end{itemize}
\item<1-> \funcname{camlDefault\_\_main\_269} raises a \typename{Failure} exception
\item<1-> \typename{Failure} is caught by \funcname{caml\_start\_program}
\begin{itemize}
\item<1-> The handler set the 2nd LSB of the exception value to \funcarg{1}
\item<3-> And then forces \funcname{caml\_start\_program} to terminate
\end{itemize}
\end{itemize}
}
\bigskip
\only<1,3>{\listocaml[highlightlines=3]{../src/default/default.ml}{default/default.ml}}%
\only<2>{\listasm[firstline=66,lastline=72,highlightlines=69]{src/ocaml/caml\_start\_program.s}{caml\_start\_program}}%
\end{column}
\begin{column}{0.4\textwidth}
\centering
\providecommand\step{4}\input{diagrams/default/default.tex}
\end{column}
\end{columns}
\end{frame}
%\only<>{\listasm[firstline=47,lastline=65]{src/ocaml/caml\_start\_program.s}{caml\_start\_program}}


\begin{frame}{Back to C}{Clean up, complain and terminate}
\begin{columns}
\begin{column}{0.45\textwidth}
\begin{itemize}
\item Backtrace:
\begin{itemize}
\item \funcname{caml\_start\_program} $\rightarrow$ OCaml code
\item \funcname{caml\_startup\_common}
\item \funcname{caml\_startup\_exn}
\item \funcname{caml\_startup}
\item \funcname{caml\_main}
\item \funcname{main}
\end{itemize}
\smallskip
\item \funcname{caml\_startup} checks the return value of \funcname{caml\_startup\_exn}
\begin{itemize}
\item The return value has been specially marked by \funcname{caml\_start\_program} handler
\item Calls \funcname{caml\_fatal\_uncaught\_exception}
\end{itemize}
\item<2-> \funcname{caml\_fatal\_uncaught\_exception} either
\begin{itemize}
\item<2-> Calls the user custom uncaught exception handler, if set with \mintinline{ocaml}{Printexc.set_uncaught_exception_handler fn}\footnotemark
\item<2-> Or falls back to the default one
\item<2-> Which notifies about the uncaught exception
\end{itemize}
\item<4-> Terminates the program either with \funcname{abort} or with a non-zero exit code
\end{itemize}
\end{column}
\begin{column}{0.55\textwidth}
\only<1>{
\listc[firstline=84,lastline=89,highlightlines=88]{../ocaml/runtime/caml/mlvalues.h}{runtime/caml/mlvalues.h:84}
\listc[firstline=138,lastline=143,highlightlines={141-142}]{../ocaml/runtime/startup\_nat.c}{runtime/startup\_nat.c:138}
}%
\only<2>{
\listc[firstline=138,highlightlines={147,149}]{../ocaml/runtime/printexc.c}{runtime/printexc.c:138}
}%
\only<3>{
\listc[firstline=110,lastline=134,highlightlines=129]{../ocaml/runtime/printexc.c}{runtime/printexc.c:138}
}%
\only<4>{
\listc[firstline=138,highlightlines={152,154}]{../ocaml/runtime/printexc.c}{runtime/printexc.c:138}
}%
\end{column}
\end{columns}
\footnotetext[1]{\url{https://v2.ocaml.org/api/Printexc.html\#1\_Uncaughtexceptions}}
\end{frame}

\end{column}
\end{columns}
\end{frame}
%\only<>{\listasm[firstline=47,lastline=65]{src/ocaml/caml\_start\_program.s}{caml\_start\_program}}


\begin{frame}{Back to C}{Clean up, complain and terminate}
\begin{columns}
\begin{column}{0.45\textwidth}
\begin{itemize}
\item Backtrace:
\begin{itemize}
\item \funcname{caml\_start\_program} $\rightarrow$ OCaml code
\item \funcname{caml\_startup\_common}
\item \funcname{caml\_startup\_exn}
\item \funcname{caml\_startup}
\item \funcname{caml\_main}
\item \funcname{main}
\end{itemize}
\smallskip
\item \funcname{caml\_startup} checks the return value of \funcname{caml\_startup\_exn}
\begin{itemize}
\item The return value has been specially marked by \funcname{caml\_start\_program} handler
\item Calls \funcname{caml\_fatal\_uncaught\_exception}
\end{itemize}
\item<2-> \funcname{caml\_fatal\_uncaught\_exception} either
\begin{itemize}
\item<2-> Calls the user custom uncaught exception handler, if set with \mintinline{ocaml}{Printexc.set_uncaught_exception_handler fn}\footnotemark
\item<2-> Or falls back to the default one
\item<2-> Which notifies about the uncaught exception
\end{itemize}
\item<4-> Terminates the program either with \funcname{abort} or with a non-zero exit code
\end{itemize}
\end{column}
\begin{column}{0.55\textwidth}
\only<1>{
\listc[firstline=84,lastline=89,highlightlines=88]{../ocaml/runtime/caml/mlvalues.h}{runtime/caml/mlvalues.h:84}
\listc[firstline=138,lastline=143,highlightlines={141-142}]{../ocaml/runtime/startup\_nat.c}{runtime/startup\_nat.c:138}
}%
\only<2>{
\listc[firstline=138,highlightlines={147,149}]{../ocaml/runtime/printexc.c}{runtime/printexc.c:138}
}%
\only<3>{
\listc[firstline=110,lastline=134,highlightlines=129]{../ocaml/runtime/printexc.c}{runtime/printexc.c:138}
}%
\only<4>{
\listc[firstline=138,highlightlines={152,154}]{../ocaml/runtime/printexc.c}{runtime/printexc.c:138}
}%
\end{column}
\end{columns}
\footnotetext[1]{\url{https://v2.ocaml.org/api/Printexc.html\#1\_Uncaughtexceptions}}
\end{frame}

\end{column}
\end{columns}
\end{frame}
%\only<>{\listasm[firstline=47,lastline=65]{src/ocaml/caml\_start\_program.s}{caml\_start\_program}}


\begin{frame}{Back to C}{Clean up, complain and terminate}
\begin{columns}
\begin{column}{0.45\textwidth}
\begin{itemize}
\item Backtrace:
\begin{itemize}
\item \funcname{caml\_start\_program} $\rightarrow$ OCaml code
\item \funcname{caml\_startup\_common}
\item \funcname{caml\_startup\_exn}
\item \funcname{caml\_startup}
\item \funcname{caml\_main}
\item \funcname{main}
\end{itemize}
\smallskip
\item \funcname{caml\_startup} checks the return value of \funcname{caml\_startup\_exn}
\begin{itemize}
\item The return value has been specially marked by \funcname{caml\_start\_program} handler
\item Calls \funcname{caml\_fatal\_uncaught\_exception}
\end{itemize}
\item<2-> \funcname{caml\_fatal\_uncaught\_exception} either
\begin{itemize}
\item<2-> Calls the user custom uncaught exception handler, if set with \mintinline{ocaml}{Printexc.set_uncaught_exception_handler fn}\footnotemark
\item<2-> Or falls back to the default one
\item<2-> Which notifies about the uncaught exception
\end{itemize}
\item<4-> Terminates the program either with \funcname{abort} or with a non-zero exit code
\end{itemize}
\end{column}
\begin{column}{0.55\textwidth}
\only<1>{
\listc[firstline=84,lastline=89,highlightlines=88]{../ocaml/runtime/caml/mlvalues.h}{runtime/caml/mlvalues.h:84}
\listc[firstline=138,lastline=143,highlightlines={141-142}]{../ocaml/runtime/startup\_nat.c}{runtime/startup\_nat.c:138}
}%
\only<2>{
\listc[firstline=138,highlightlines={147,149}]{../ocaml/runtime/printexc.c}{runtime/printexc.c:138}
}%
\only<3>{
\listc[firstline=110,lastline=134,highlightlines=129]{../ocaml/runtime/printexc.c}{runtime/printexc.c:138}
}%
\only<4>{
\listc[firstline=138,highlightlines={152,154}]{../ocaml/runtime/printexc.c}{runtime/printexc.c:138}
}%
\end{column}
\end{columns}
\footnotetext[1]{\url{https://v2.ocaml.org/api/Printexc.html\#1\_Uncaughtexceptions}}
\end{frame}
}%
\end{column}
\end{columns}
\end{frame}


\begin{frame}{Startup}{How did we get there by the way?}
\begin{columns}
\begin{column}{0.45\textwidth}
\begin{itemize}
\item The entry point address of the binary is \funcname{main} (\funcname{\_start}) from the runtime
\item The program starts like a C program:
\begin{itemize}
\item \funcname{caml\_start\_program}
\item \funcname{caml\_startup\_common}
\item \funcname{caml\_startup\_exn}
\item \funcname{caml\_startup}
\item \funcname{caml\_main}
\item \funcname{main}
\end{itemize}
%\item \funcname{caml\_startup\_common}: initializes GC, domains \dots
% Also: codefrag, locale, custom opeartions, frame_descr, signals, stack
\item \funcname{caml\_start\_program} is the transition point between the C realm and the OCaml one
\end{itemize}
\bigskip
\listocaml{../src/default/default.ml}{default/default.ml}
\end{column}
\begin{column}{0.55\textwidth}
\listasm[lastline=24,highlightlines={22-24}]{src/ocaml/caml\_start\_program.s}{runtime/amd64.S:604}
\end{column}
\end{columns}
\end{frame}


\begin{frame}{Setup to jump into OCaml entry point}{\funcname{caml\_start\_program} initializes the main fiber}
\begin{columns}
\begin{column}{0.6\textwidth}
\only<1,3,5>{
\begin{itemize}
\item<1-> Stores information to allow switching from OCaml stack to the C stack
\item<3-> Constructs the default exception handler
\item<5-> Switches to the OCaml stack and calls \funcname{caml\_program}
\end{itemize}
\bigskip
\listocaml{../src/default/default.ml}{default/default.ml}
}%
\only<2>{
\listasm[firstline=22,lastline=41,highlightlines={25-27}]{src/ocaml/caml\_start\_program.s}{caml\_start\_program}
}%
\only<4>{
\mintasm[firstline=22,lastline=41,highlightlines={31-38}]{src/ocaml/caml\_start\_program.s}
\listasm[firstline=66,lastline=72]{src/ocaml/caml\_start\_program.s}{caml\_start\_program}
}%
\only<6>{
\listasm[firstline=22,lastline=41,highlightlines={39-41}]{src/ocaml/caml\_start\_program.s}{caml\_start\_program}
}%
\end{column}
\begin{column}{0.4\textwidth}
\centering
\only<1-2>{\providecommand\step{2}%
% Runtime's default exception handler
%
\subsection*{Runtime's default exception handler}
\frameSubsection{pictures/The\_Architect.png}{}


\begin{frame}{Default exception handler}
\begin{columns}
\begin{column}{0.5\textwidth}
\begin{itemize}
\item \localname{domain\_state->exn\_handler} points to a trap located before \funcname{entry}!
\item What installed this very first trap there?
\end{itemize}
\bigskip
\listocaml{../src/default/default.ml}{default/default.ml}
\bigskip
\inputminted[fontsize=\footnotesize]{text}{../src/default/default.out}
\end{column}
\begin{column}{0.5\textwidth}
\centering
\only<1>{\providecommand\step{0}%
% Runtime's default exception handler
%
\subsection*{Runtime's default exception handler}
\frameSubsection{pictures/The\_Architect.png}{}


\begin{frame}{Default exception handler}
\begin{columns}
\begin{column}{0.5\textwidth}
\begin{itemize}
\item \localname{domain\_state->exn\_handler} points to a trap located before \funcname{entry}!
\item What installed this very first trap there?
\end{itemize}
\bigskip
\listocaml{../src/default/default.ml}{default/default.ml}
\bigskip
\inputminted[fontsize=\footnotesize]{text}{../src/default/default.out}
\end{column}
\begin{column}{0.5\textwidth}
\centering
\only<1>{\providecommand\step{0}%
% Runtime's default exception handler
%
\subsection*{Runtime's default exception handler}
\frameSubsection{pictures/The\_Architect.png}{}


\begin{frame}{Default exception handler}
\begin{columns}
\begin{column}{0.5\textwidth}
\begin{itemize}
\item \localname{domain\_state->exn\_handler} points to a trap located before \funcname{entry}!
\item What installed this very first trap there?
\end{itemize}
\bigskip
\listocaml{../src/default/default.ml}{default/default.ml}
\bigskip
\inputminted[fontsize=\footnotesize]{text}{../src/default/default.out}
\end{column}
\begin{column}{0.5\textwidth}
\centering
\only<1>{\providecommand\step{0}\input{diagrams/default/default.tex}}%
\only<2>{\providecommand\step{4}\input{diagrams/default/default.tex}}%
\end{column}
\end{columns}
\end{frame}


\begin{frame}{Startup}{How did we get there by the way?}
\begin{columns}
\begin{column}{0.45\textwidth}
\begin{itemize}
\item The entry point address of the binary is \funcname{main} (\funcname{\_start}) from the runtime
\item The program starts like a C program:
\begin{itemize}
\item \funcname{caml\_start\_program}
\item \funcname{caml\_startup\_common}
\item \funcname{caml\_startup\_exn}
\item \funcname{caml\_startup}
\item \funcname{caml\_main}
\item \funcname{main}
\end{itemize}
%\item \funcname{caml\_startup\_common}: initializes GC, domains \dots
% Also: codefrag, locale, custom opeartions, frame_descr, signals, stack
\item \funcname{caml\_start\_program} is the transition point between the C realm and the OCaml one
\end{itemize}
\bigskip
\listocaml{../src/default/default.ml}{default/default.ml}
\end{column}
\begin{column}{0.55\textwidth}
\listasm[lastline=24,highlightlines={22-24}]{src/ocaml/caml\_start\_program.s}{runtime/amd64.S:604}
\end{column}
\end{columns}
\end{frame}


\begin{frame}{Setup to jump into OCaml entry point}{\funcname{caml\_start\_program} initializes the main fiber}
\begin{columns}
\begin{column}{0.6\textwidth}
\only<1,3,5>{
\begin{itemize}
\item<1-> Stores information to allow switching from OCaml stack to the C stack
\item<3-> Constructs the default exception handler
\item<5-> Switches to the OCaml stack and calls \funcname{caml\_program}
\end{itemize}
\bigskip
\listocaml{../src/default/default.ml}{default/default.ml}
}%
\only<2>{
\listasm[firstline=22,lastline=41,highlightlines={25-27}]{src/ocaml/caml\_start\_program.s}{caml\_start\_program}
}%
\only<4>{
\mintasm[firstline=22,lastline=41,highlightlines={31-38}]{src/ocaml/caml\_start\_program.s}
\listasm[firstline=66,lastline=72]{src/ocaml/caml\_start\_program.s}{caml\_start\_program}
}%
\only<6>{
\listasm[firstline=22,lastline=41,highlightlines={39-41}]{src/ocaml/caml\_start\_program.s}{caml\_start\_program}
}%
\end{column}
\begin{column}{0.4\textwidth}
\centering
\only<1-2>{\providecommand\step{2}\input{diagrams/default/default.tex}}%
\only<3-5>{\providecommand\step{3}\input{diagrams/default/default.tex}}%
\onslide*<6->{\providecommand\step{4}\input{diagrams/default/default.tex}}%
\end{column}
\end{columns}
\end{frame}

\begin{frame}{Executing the OCaml program}{And raising an exception without handler}
\begin{columns}
\begin{column}{0.6\textwidth}
\only<1-3>{
\begin{itemize}
\item<1-> Execution continues with OCaml code
\begin{itemize}
\item \funcname{camlDefault\_\_main\_269}
\item \funcname{camlDefault\_\_entry}
\item \funcname{caml\_program}
\end{itemize}
\item<1-> \funcname{camlDefault\_\_main\_269} raises a \typename{Failure} exception
\item<1-> \typename{Failure} is caught by \funcname{caml\_start\_program}
\begin{itemize}
\item<1-> The handler set the 2nd LSB of the exception value to \funcarg{1}
\item<3-> And then forces \funcname{caml\_start\_program} to terminate
\end{itemize}
\end{itemize}
}
\bigskip
\only<1,3>{\listocaml[highlightlines=3]{../src/default/default.ml}{default/default.ml}}%
\only<2>{\listasm[firstline=66,lastline=72,highlightlines=69]{src/ocaml/caml\_start\_program.s}{caml\_start\_program}}%
\end{column}
\begin{column}{0.4\textwidth}
\centering
\providecommand\step{4}\input{diagrams/default/default.tex}
\end{column}
\end{columns}
\end{frame}
%\only<>{\listasm[firstline=47,lastline=65]{src/ocaml/caml\_start\_program.s}{caml\_start\_program}}


\begin{frame}{Back to C}{Clean up, complain and terminate}
\begin{columns}
\begin{column}{0.45\textwidth}
\begin{itemize}
\item Backtrace:
\begin{itemize}
\item \funcname{caml\_start\_program} $\rightarrow$ OCaml code
\item \funcname{caml\_startup\_common}
\item \funcname{caml\_startup\_exn}
\item \funcname{caml\_startup}
\item \funcname{caml\_main}
\item \funcname{main}
\end{itemize}
\smallskip
\item \funcname{caml\_startup} checks the return value of \funcname{caml\_startup\_exn}
\begin{itemize}
\item The return value has been specially marked by \funcname{caml\_start\_program} handler
\item Calls \funcname{caml\_fatal\_uncaught\_exception}
\end{itemize}
\item<2-> \funcname{caml\_fatal\_uncaught\_exception} either
\begin{itemize}
\item<2-> Calls the user custom uncaught exception handler, if set with \mintinline{ocaml}{Printexc.set_uncaught_exception_handler fn}\footnotemark
\item<2-> Or falls back to the default one
\item<2-> Which notifies about the uncaught exception
\end{itemize}
\item<4-> Terminates the program either with \funcname{abort} or with a non-zero exit code
\end{itemize}
\end{column}
\begin{column}{0.55\textwidth}
\only<1>{
\listc[firstline=84,lastline=89,highlightlines=88]{../ocaml/runtime/caml/mlvalues.h}{runtime/caml/mlvalues.h:84}
\listc[firstline=138,lastline=143,highlightlines={141-142}]{../ocaml/runtime/startup\_nat.c}{runtime/startup\_nat.c:138}
}%
\only<2>{
\listc[firstline=138,highlightlines={147,149}]{../ocaml/runtime/printexc.c}{runtime/printexc.c:138}
}%
\only<3>{
\listc[firstline=110,lastline=134,highlightlines=129]{../ocaml/runtime/printexc.c}{runtime/printexc.c:138}
}%
\only<4>{
\listc[firstline=138,highlightlines={152,154}]{../ocaml/runtime/printexc.c}{runtime/printexc.c:138}
}%
\end{column}
\end{columns}
\footnotetext[1]{\url{https://v2.ocaml.org/api/Printexc.html\#1\_Uncaughtexceptions}}
\end{frame}
}%
\only<2>{\providecommand\step{4}%
% Runtime's default exception handler
%
\subsection*{Runtime's default exception handler}
\frameSubsection{pictures/The\_Architect.png}{}


\begin{frame}{Default exception handler}
\begin{columns}
\begin{column}{0.5\textwidth}
\begin{itemize}
\item \localname{domain\_state->exn\_handler} points to a trap located before \funcname{entry}!
\item What installed this very first trap there?
\end{itemize}
\bigskip
\listocaml{../src/default/default.ml}{default/default.ml}
\bigskip
\inputminted[fontsize=\footnotesize]{text}{../src/default/default.out}
\end{column}
\begin{column}{0.5\textwidth}
\centering
\only<1>{\providecommand\step{0}\input{diagrams/default/default.tex}}%
\only<2>{\providecommand\step{4}\input{diagrams/default/default.tex}}%
\end{column}
\end{columns}
\end{frame}


\begin{frame}{Startup}{How did we get there by the way?}
\begin{columns}
\begin{column}{0.45\textwidth}
\begin{itemize}
\item The entry point address of the binary is \funcname{main} (\funcname{\_start}) from the runtime
\item The program starts like a C program:
\begin{itemize}
\item \funcname{caml\_start\_program}
\item \funcname{caml\_startup\_common}
\item \funcname{caml\_startup\_exn}
\item \funcname{caml\_startup}
\item \funcname{caml\_main}
\item \funcname{main}
\end{itemize}
%\item \funcname{caml\_startup\_common}: initializes GC, domains \dots
% Also: codefrag, locale, custom opeartions, frame_descr, signals, stack
\item \funcname{caml\_start\_program} is the transition point between the C realm and the OCaml one
\end{itemize}
\bigskip
\listocaml{../src/default/default.ml}{default/default.ml}
\end{column}
\begin{column}{0.55\textwidth}
\listasm[lastline=24,highlightlines={22-24}]{src/ocaml/caml\_start\_program.s}{runtime/amd64.S:604}
\end{column}
\end{columns}
\end{frame}


\begin{frame}{Setup to jump into OCaml entry point}{\funcname{caml\_start\_program} initializes the main fiber}
\begin{columns}
\begin{column}{0.6\textwidth}
\only<1,3,5>{
\begin{itemize}
\item<1-> Stores information to allow switching from OCaml stack to the C stack
\item<3-> Constructs the default exception handler
\item<5-> Switches to the OCaml stack and calls \funcname{caml\_program}
\end{itemize}
\bigskip
\listocaml{../src/default/default.ml}{default/default.ml}
}%
\only<2>{
\listasm[firstline=22,lastline=41,highlightlines={25-27}]{src/ocaml/caml\_start\_program.s}{caml\_start\_program}
}%
\only<4>{
\mintasm[firstline=22,lastline=41,highlightlines={31-38}]{src/ocaml/caml\_start\_program.s}
\listasm[firstline=66,lastline=72]{src/ocaml/caml\_start\_program.s}{caml\_start\_program}
}%
\only<6>{
\listasm[firstline=22,lastline=41,highlightlines={39-41}]{src/ocaml/caml\_start\_program.s}{caml\_start\_program}
}%
\end{column}
\begin{column}{0.4\textwidth}
\centering
\only<1-2>{\providecommand\step{2}\input{diagrams/default/default.tex}}%
\only<3-5>{\providecommand\step{3}\input{diagrams/default/default.tex}}%
\onslide*<6->{\providecommand\step{4}\input{diagrams/default/default.tex}}%
\end{column}
\end{columns}
\end{frame}

\begin{frame}{Executing the OCaml program}{And raising an exception without handler}
\begin{columns}
\begin{column}{0.6\textwidth}
\only<1-3>{
\begin{itemize}
\item<1-> Execution continues with OCaml code
\begin{itemize}
\item \funcname{camlDefault\_\_main\_269}
\item \funcname{camlDefault\_\_entry}
\item \funcname{caml\_program}
\end{itemize}
\item<1-> \funcname{camlDefault\_\_main\_269} raises a \typename{Failure} exception
\item<1-> \typename{Failure} is caught by \funcname{caml\_start\_program}
\begin{itemize}
\item<1-> The handler set the 2nd LSB of the exception value to \funcarg{1}
\item<3-> And then forces \funcname{caml\_start\_program} to terminate
\end{itemize}
\end{itemize}
}
\bigskip
\only<1,3>{\listocaml[highlightlines=3]{../src/default/default.ml}{default/default.ml}}%
\only<2>{\listasm[firstline=66,lastline=72,highlightlines=69]{src/ocaml/caml\_start\_program.s}{caml\_start\_program}}%
\end{column}
\begin{column}{0.4\textwidth}
\centering
\providecommand\step{4}\input{diagrams/default/default.tex}
\end{column}
\end{columns}
\end{frame}
%\only<>{\listasm[firstline=47,lastline=65]{src/ocaml/caml\_start\_program.s}{caml\_start\_program}}


\begin{frame}{Back to C}{Clean up, complain and terminate}
\begin{columns}
\begin{column}{0.45\textwidth}
\begin{itemize}
\item Backtrace:
\begin{itemize}
\item \funcname{caml\_start\_program} $\rightarrow$ OCaml code
\item \funcname{caml\_startup\_common}
\item \funcname{caml\_startup\_exn}
\item \funcname{caml\_startup}
\item \funcname{caml\_main}
\item \funcname{main}
\end{itemize}
\smallskip
\item \funcname{caml\_startup} checks the return value of \funcname{caml\_startup\_exn}
\begin{itemize}
\item The return value has been specially marked by \funcname{caml\_start\_program} handler
\item Calls \funcname{caml\_fatal\_uncaught\_exception}
\end{itemize}
\item<2-> \funcname{caml\_fatal\_uncaught\_exception} either
\begin{itemize}
\item<2-> Calls the user custom uncaught exception handler, if set with \mintinline{ocaml}{Printexc.set_uncaught_exception_handler fn}\footnotemark
\item<2-> Or falls back to the default one
\item<2-> Which notifies about the uncaught exception
\end{itemize}
\item<4-> Terminates the program either with \funcname{abort} or with a non-zero exit code
\end{itemize}
\end{column}
\begin{column}{0.55\textwidth}
\only<1>{
\listc[firstline=84,lastline=89,highlightlines=88]{../ocaml/runtime/caml/mlvalues.h}{runtime/caml/mlvalues.h:84}
\listc[firstline=138,lastline=143,highlightlines={141-142}]{../ocaml/runtime/startup\_nat.c}{runtime/startup\_nat.c:138}
}%
\only<2>{
\listc[firstline=138,highlightlines={147,149}]{../ocaml/runtime/printexc.c}{runtime/printexc.c:138}
}%
\only<3>{
\listc[firstline=110,lastline=134,highlightlines=129]{../ocaml/runtime/printexc.c}{runtime/printexc.c:138}
}%
\only<4>{
\listc[firstline=138,highlightlines={152,154}]{../ocaml/runtime/printexc.c}{runtime/printexc.c:138}
}%
\end{column}
\end{columns}
\footnotetext[1]{\url{https://v2.ocaml.org/api/Printexc.html\#1\_Uncaughtexceptions}}
\end{frame}
}%
\end{column}
\end{columns}
\end{frame}


\begin{frame}{Startup}{How did we get there by the way?}
\begin{columns}
\begin{column}{0.45\textwidth}
\begin{itemize}
\item The entry point address of the binary is \funcname{main} (\funcname{\_start}) from the runtime
\item The program starts like a C program:
\begin{itemize}
\item \funcname{caml\_start\_program}
\item \funcname{caml\_startup\_common}
\item \funcname{caml\_startup\_exn}
\item \funcname{caml\_startup}
\item \funcname{caml\_main}
\item \funcname{main}
\end{itemize}
%\item \funcname{caml\_startup\_common}: initializes GC, domains \dots
% Also: codefrag, locale, custom opeartions, frame_descr, signals, stack
\item \funcname{caml\_start\_program} is the transition point between the C realm and the OCaml one
\end{itemize}
\bigskip
\listocaml{../src/default/default.ml}{default/default.ml}
\end{column}
\begin{column}{0.55\textwidth}
\listasm[lastline=24,highlightlines={22-24}]{src/ocaml/caml\_start\_program.s}{runtime/amd64.S:604}
\end{column}
\end{columns}
\end{frame}


\begin{frame}{Setup to jump into OCaml entry point}{\funcname{caml\_start\_program} initializes the main fiber}
\begin{columns}
\begin{column}{0.6\textwidth}
\only<1,3,5>{
\begin{itemize}
\item<1-> Stores information to allow switching from OCaml stack to the C stack
\item<3-> Constructs the default exception handler
\item<5-> Switches to the OCaml stack and calls \funcname{caml\_program}
\end{itemize}
\bigskip
\listocaml{../src/default/default.ml}{default/default.ml}
}%
\only<2>{
\listasm[firstline=22,lastline=41,highlightlines={25-27}]{src/ocaml/caml\_start\_program.s}{caml\_start\_program}
}%
\only<4>{
\mintasm[firstline=22,lastline=41,highlightlines={31-38}]{src/ocaml/caml\_start\_program.s}
\listasm[firstline=66,lastline=72]{src/ocaml/caml\_start\_program.s}{caml\_start\_program}
}%
\only<6>{
\listasm[firstline=22,lastline=41,highlightlines={39-41}]{src/ocaml/caml\_start\_program.s}{caml\_start\_program}
}%
\end{column}
\begin{column}{0.4\textwidth}
\centering
\only<1-2>{\providecommand\step{2}%
% Runtime's default exception handler
%
\subsection*{Runtime's default exception handler}
\frameSubsection{pictures/The\_Architect.png}{}


\begin{frame}{Default exception handler}
\begin{columns}
\begin{column}{0.5\textwidth}
\begin{itemize}
\item \localname{domain\_state->exn\_handler} points to a trap located before \funcname{entry}!
\item What installed this very first trap there?
\end{itemize}
\bigskip
\listocaml{../src/default/default.ml}{default/default.ml}
\bigskip
\inputminted[fontsize=\footnotesize]{text}{../src/default/default.out}
\end{column}
\begin{column}{0.5\textwidth}
\centering
\only<1>{\providecommand\step{0}\input{diagrams/default/default.tex}}%
\only<2>{\providecommand\step{4}\input{diagrams/default/default.tex}}%
\end{column}
\end{columns}
\end{frame}


\begin{frame}{Startup}{How did we get there by the way?}
\begin{columns}
\begin{column}{0.45\textwidth}
\begin{itemize}
\item The entry point address of the binary is \funcname{main} (\funcname{\_start}) from the runtime
\item The program starts like a C program:
\begin{itemize}
\item \funcname{caml\_start\_program}
\item \funcname{caml\_startup\_common}
\item \funcname{caml\_startup\_exn}
\item \funcname{caml\_startup}
\item \funcname{caml\_main}
\item \funcname{main}
\end{itemize}
%\item \funcname{caml\_startup\_common}: initializes GC, domains \dots
% Also: codefrag, locale, custom opeartions, frame_descr, signals, stack
\item \funcname{caml\_start\_program} is the transition point between the C realm and the OCaml one
\end{itemize}
\bigskip
\listocaml{../src/default/default.ml}{default/default.ml}
\end{column}
\begin{column}{0.55\textwidth}
\listasm[lastline=24,highlightlines={22-24}]{src/ocaml/caml\_start\_program.s}{runtime/amd64.S:604}
\end{column}
\end{columns}
\end{frame}


\begin{frame}{Setup to jump into OCaml entry point}{\funcname{caml\_start\_program} initializes the main fiber}
\begin{columns}
\begin{column}{0.6\textwidth}
\only<1,3,5>{
\begin{itemize}
\item<1-> Stores information to allow switching from OCaml stack to the C stack
\item<3-> Constructs the default exception handler
\item<5-> Switches to the OCaml stack and calls \funcname{caml\_program}
\end{itemize}
\bigskip
\listocaml{../src/default/default.ml}{default/default.ml}
}%
\only<2>{
\listasm[firstline=22,lastline=41,highlightlines={25-27}]{src/ocaml/caml\_start\_program.s}{caml\_start\_program}
}%
\only<4>{
\mintasm[firstline=22,lastline=41,highlightlines={31-38}]{src/ocaml/caml\_start\_program.s}
\listasm[firstline=66,lastline=72]{src/ocaml/caml\_start\_program.s}{caml\_start\_program}
}%
\only<6>{
\listasm[firstline=22,lastline=41,highlightlines={39-41}]{src/ocaml/caml\_start\_program.s}{caml\_start\_program}
}%
\end{column}
\begin{column}{0.4\textwidth}
\centering
\only<1-2>{\providecommand\step{2}\input{diagrams/default/default.tex}}%
\only<3-5>{\providecommand\step{3}\input{diagrams/default/default.tex}}%
\onslide*<6->{\providecommand\step{4}\input{diagrams/default/default.tex}}%
\end{column}
\end{columns}
\end{frame}

\begin{frame}{Executing the OCaml program}{And raising an exception without handler}
\begin{columns}
\begin{column}{0.6\textwidth}
\only<1-3>{
\begin{itemize}
\item<1-> Execution continues with OCaml code
\begin{itemize}
\item \funcname{camlDefault\_\_main\_269}
\item \funcname{camlDefault\_\_entry}
\item \funcname{caml\_program}
\end{itemize}
\item<1-> \funcname{camlDefault\_\_main\_269} raises a \typename{Failure} exception
\item<1-> \typename{Failure} is caught by \funcname{caml\_start\_program}
\begin{itemize}
\item<1-> The handler set the 2nd LSB of the exception value to \funcarg{1}
\item<3-> And then forces \funcname{caml\_start\_program} to terminate
\end{itemize}
\end{itemize}
}
\bigskip
\only<1,3>{\listocaml[highlightlines=3]{../src/default/default.ml}{default/default.ml}}%
\only<2>{\listasm[firstline=66,lastline=72,highlightlines=69]{src/ocaml/caml\_start\_program.s}{caml\_start\_program}}%
\end{column}
\begin{column}{0.4\textwidth}
\centering
\providecommand\step{4}\input{diagrams/default/default.tex}
\end{column}
\end{columns}
\end{frame}
%\only<>{\listasm[firstline=47,lastline=65]{src/ocaml/caml\_start\_program.s}{caml\_start\_program}}


\begin{frame}{Back to C}{Clean up, complain and terminate}
\begin{columns}
\begin{column}{0.45\textwidth}
\begin{itemize}
\item Backtrace:
\begin{itemize}
\item \funcname{caml\_start\_program} $\rightarrow$ OCaml code
\item \funcname{caml\_startup\_common}
\item \funcname{caml\_startup\_exn}
\item \funcname{caml\_startup}
\item \funcname{caml\_main}
\item \funcname{main}
\end{itemize}
\smallskip
\item \funcname{caml\_startup} checks the return value of \funcname{caml\_startup\_exn}
\begin{itemize}
\item The return value has been specially marked by \funcname{caml\_start\_program} handler
\item Calls \funcname{caml\_fatal\_uncaught\_exception}
\end{itemize}
\item<2-> \funcname{caml\_fatal\_uncaught\_exception} either
\begin{itemize}
\item<2-> Calls the user custom uncaught exception handler, if set with \mintinline{ocaml}{Printexc.set_uncaught_exception_handler fn}\footnotemark
\item<2-> Or falls back to the default one
\item<2-> Which notifies about the uncaught exception
\end{itemize}
\item<4-> Terminates the program either with \funcname{abort} or with a non-zero exit code
\end{itemize}
\end{column}
\begin{column}{0.55\textwidth}
\only<1>{
\listc[firstline=84,lastline=89,highlightlines=88]{../ocaml/runtime/caml/mlvalues.h}{runtime/caml/mlvalues.h:84}
\listc[firstline=138,lastline=143,highlightlines={141-142}]{../ocaml/runtime/startup\_nat.c}{runtime/startup\_nat.c:138}
}%
\only<2>{
\listc[firstline=138,highlightlines={147,149}]{../ocaml/runtime/printexc.c}{runtime/printexc.c:138}
}%
\only<3>{
\listc[firstline=110,lastline=134,highlightlines=129]{../ocaml/runtime/printexc.c}{runtime/printexc.c:138}
}%
\only<4>{
\listc[firstline=138,highlightlines={152,154}]{../ocaml/runtime/printexc.c}{runtime/printexc.c:138}
}%
\end{column}
\end{columns}
\footnotetext[1]{\url{https://v2.ocaml.org/api/Printexc.html\#1\_Uncaughtexceptions}}
\end{frame}
}%
\only<3-5>{\providecommand\step{3}%
% Runtime's default exception handler
%
\subsection*{Runtime's default exception handler}
\frameSubsection{pictures/The\_Architect.png}{}


\begin{frame}{Default exception handler}
\begin{columns}
\begin{column}{0.5\textwidth}
\begin{itemize}
\item \localname{domain\_state->exn\_handler} points to a trap located before \funcname{entry}!
\item What installed this very first trap there?
\end{itemize}
\bigskip
\listocaml{../src/default/default.ml}{default/default.ml}
\bigskip
\inputminted[fontsize=\footnotesize]{text}{../src/default/default.out}
\end{column}
\begin{column}{0.5\textwidth}
\centering
\only<1>{\providecommand\step{0}\input{diagrams/default/default.tex}}%
\only<2>{\providecommand\step{4}\input{diagrams/default/default.tex}}%
\end{column}
\end{columns}
\end{frame}


\begin{frame}{Startup}{How did we get there by the way?}
\begin{columns}
\begin{column}{0.45\textwidth}
\begin{itemize}
\item The entry point address of the binary is \funcname{main} (\funcname{\_start}) from the runtime
\item The program starts like a C program:
\begin{itemize}
\item \funcname{caml\_start\_program}
\item \funcname{caml\_startup\_common}
\item \funcname{caml\_startup\_exn}
\item \funcname{caml\_startup}
\item \funcname{caml\_main}
\item \funcname{main}
\end{itemize}
%\item \funcname{caml\_startup\_common}: initializes GC, domains \dots
% Also: codefrag, locale, custom opeartions, frame_descr, signals, stack
\item \funcname{caml\_start\_program} is the transition point between the C realm and the OCaml one
\end{itemize}
\bigskip
\listocaml{../src/default/default.ml}{default/default.ml}
\end{column}
\begin{column}{0.55\textwidth}
\listasm[lastline=24,highlightlines={22-24}]{src/ocaml/caml\_start\_program.s}{runtime/amd64.S:604}
\end{column}
\end{columns}
\end{frame}


\begin{frame}{Setup to jump into OCaml entry point}{\funcname{caml\_start\_program} initializes the main fiber}
\begin{columns}
\begin{column}{0.6\textwidth}
\only<1,3,5>{
\begin{itemize}
\item<1-> Stores information to allow switching from OCaml stack to the C stack
\item<3-> Constructs the default exception handler
\item<5-> Switches to the OCaml stack and calls \funcname{caml\_program}
\end{itemize}
\bigskip
\listocaml{../src/default/default.ml}{default/default.ml}
}%
\only<2>{
\listasm[firstline=22,lastline=41,highlightlines={25-27}]{src/ocaml/caml\_start\_program.s}{caml\_start\_program}
}%
\only<4>{
\mintasm[firstline=22,lastline=41,highlightlines={31-38}]{src/ocaml/caml\_start\_program.s}
\listasm[firstline=66,lastline=72]{src/ocaml/caml\_start\_program.s}{caml\_start\_program}
}%
\only<6>{
\listasm[firstline=22,lastline=41,highlightlines={39-41}]{src/ocaml/caml\_start\_program.s}{caml\_start\_program}
}%
\end{column}
\begin{column}{0.4\textwidth}
\centering
\only<1-2>{\providecommand\step{2}\input{diagrams/default/default.tex}}%
\only<3-5>{\providecommand\step{3}\input{diagrams/default/default.tex}}%
\onslide*<6->{\providecommand\step{4}\input{diagrams/default/default.tex}}%
\end{column}
\end{columns}
\end{frame}

\begin{frame}{Executing the OCaml program}{And raising an exception without handler}
\begin{columns}
\begin{column}{0.6\textwidth}
\only<1-3>{
\begin{itemize}
\item<1-> Execution continues with OCaml code
\begin{itemize}
\item \funcname{camlDefault\_\_main\_269}
\item \funcname{camlDefault\_\_entry}
\item \funcname{caml\_program}
\end{itemize}
\item<1-> \funcname{camlDefault\_\_main\_269} raises a \typename{Failure} exception
\item<1-> \typename{Failure} is caught by \funcname{caml\_start\_program}
\begin{itemize}
\item<1-> The handler set the 2nd LSB of the exception value to \funcarg{1}
\item<3-> And then forces \funcname{caml\_start\_program} to terminate
\end{itemize}
\end{itemize}
}
\bigskip
\only<1,3>{\listocaml[highlightlines=3]{../src/default/default.ml}{default/default.ml}}%
\only<2>{\listasm[firstline=66,lastline=72,highlightlines=69]{src/ocaml/caml\_start\_program.s}{caml\_start\_program}}%
\end{column}
\begin{column}{0.4\textwidth}
\centering
\providecommand\step{4}\input{diagrams/default/default.tex}
\end{column}
\end{columns}
\end{frame}
%\only<>{\listasm[firstline=47,lastline=65]{src/ocaml/caml\_start\_program.s}{caml\_start\_program}}


\begin{frame}{Back to C}{Clean up, complain and terminate}
\begin{columns}
\begin{column}{0.45\textwidth}
\begin{itemize}
\item Backtrace:
\begin{itemize}
\item \funcname{caml\_start\_program} $\rightarrow$ OCaml code
\item \funcname{caml\_startup\_common}
\item \funcname{caml\_startup\_exn}
\item \funcname{caml\_startup}
\item \funcname{caml\_main}
\item \funcname{main}
\end{itemize}
\smallskip
\item \funcname{caml\_startup} checks the return value of \funcname{caml\_startup\_exn}
\begin{itemize}
\item The return value has been specially marked by \funcname{caml\_start\_program} handler
\item Calls \funcname{caml\_fatal\_uncaught\_exception}
\end{itemize}
\item<2-> \funcname{caml\_fatal\_uncaught\_exception} either
\begin{itemize}
\item<2-> Calls the user custom uncaught exception handler, if set with \mintinline{ocaml}{Printexc.set_uncaught_exception_handler fn}\footnotemark
\item<2-> Or falls back to the default one
\item<2-> Which notifies about the uncaught exception
\end{itemize}
\item<4-> Terminates the program either with \funcname{abort} or with a non-zero exit code
\end{itemize}
\end{column}
\begin{column}{0.55\textwidth}
\only<1>{
\listc[firstline=84,lastline=89,highlightlines=88]{../ocaml/runtime/caml/mlvalues.h}{runtime/caml/mlvalues.h:84}
\listc[firstline=138,lastline=143,highlightlines={141-142}]{../ocaml/runtime/startup\_nat.c}{runtime/startup\_nat.c:138}
}%
\only<2>{
\listc[firstline=138,highlightlines={147,149}]{../ocaml/runtime/printexc.c}{runtime/printexc.c:138}
}%
\only<3>{
\listc[firstline=110,lastline=134,highlightlines=129]{../ocaml/runtime/printexc.c}{runtime/printexc.c:138}
}%
\only<4>{
\listc[firstline=138,highlightlines={152,154}]{../ocaml/runtime/printexc.c}{runtime/printexc.c:138}
}%
\end{column}
\end{columns}
\footnotetext[1]{\url{https://v2.ocaml.org/api/Printexc.html\#1\_Uncaughtexceptions}}
\end{frame}
}%
\onslide*<6->{\providecommand\step{4}%
% Runtime's default exception handler
%
\subsection*{Runtime's default exception handler}
\frameSubsection{pictures/The\_Architect.png}{}


\begin{frame}{Default exception handler}
\begin{columns}
\begin{column}{0.5\textwidth}
\begin{itemize}
\item \localname{domain\_state->exn\_handler} points to a trap located before \funcname{entry}!
\item What installed this very first trap there?
\end{itemize}
\bigskip
\listocaml{../src/default/default.ml}{default/default.ml}
\bigskip
\inputminted[fontsize=\footnotesize]{text}{../src/default/default.out}
\end{column}
\begin{column}{0.5\textwidth}
\centering
\only<1>{\providecommand\step{0}\input{diagrams/default/default.tex}}%
\only<2>{\providecommand\step{4}\input{diagrams/default/default.tex}}%
\end{column}
\end{columns}
\end{frame}


\begin{frame}{Startup}{How did we get there by the way?}
\begin{columns}
\begin{column}{0.45\textwidth}
\begin{itemize}
\item The entry point address of the binary is \funcname{main} (\funcname{\_start}) from the runtime
\item The program starts like a C program:
\begin{itemize}
\item \funcname{caml\_start\_program}
\item \funcname{caml\_startup\_common}
\item \funcname{caml\_startup\_exn}
\item \funcname{caml\_startup}
\item \funcname{caml\_main}
\item \funcname{main}
\end{itemize}
%\item \funcname{caml\_startup\_common}: initializes GC, domains \dots
% Also: codefrag, locale, custom opeartions, frame_descr, signals, stack
\item \funcname{caml\_start\_program} is the transition point between the C realm and the OCaml one
\end{itemize}
\bigskip
\listocaml{../src/default/default.ml}{default/default.ml}
\end{column}
\begin{column}{0.55\textwidth}
\listasm[lastline=24,highlightlines={22-24}]{src/ocaml/caml\_start\_program.s}{runtime/amd64.S:604}
\end{column}
\end{columns}
\end{frame}


\begin{frame}{Setup to jump into OCaml entry point}{\funcname{caml\_start\_program} initializes the main fiber}
\begin{columns}
\begin{column}{0.6\textwidth}
\only<1,3,5>{
\begin{itemize}
\item<1-> Stores information to allow switching from OCaml stack to the C stack
\item<3-> Constructs the default exception handler
\item<5-> Switches to the OCaml stack and calls \funcname{caml\_program}
\end{itemize}
\bigskip
\listocaml{../src/default/default.ml}{default/default.ml}
}%
\only<2>{
\listasm[firstline=22,lastline=41,highlightlines={25-27}]{src/ocaml/caml\_start\_program.s}{caml\_start\_program}
}%
\only<4>{
\mintasm[firstline=22,lastline=41,highlightlines={31-38}]{src/ocaml/caml\_start\_program.s}
\listasm[firstline=66,lastline=72]{src/ocaml/caml\_start\_program.s}{caml\_start\_program}
}%
\only<6>{
\listasm[firstline=22,lastline=41,highlightlines={39-41}]{src/ocaml/caml\_start\_program.s}{caml\_start\_program}
}%
\end{column}
\begin{column}{0.4\textwidth}
\centering
\only<1-2>{\providecommand\step{2}\input{diagrams/default/default.tex}}%
\only<3-5>{\providecommand\step{3}\input{diagrams/default/default.tex}}%
\onslide*<6->{\providecommand\step{4}\input{diagrams/default/default.tex}}%
\end{column}
\end{columns}
\end{frame}

\begin{frame}{Executing the OCaml program}{And raising an exception without handler}
\begin{columns}
\begin{column}{0.6\textwidth}
\only<1-3>{
\begin{itemize}
\item<1-> Execution continues with OCaml code
\begin{itemize}
\item \funcname{camlDefault\_\_main\_269}
\item \funcname{camlDefault\_\_entry}
\item \funcname{caml\_program}
\end{itemize}
\item<1-> \funcname{camlDefault\_\_main\_269} raises a \typename{Failure} exception
\item<1-> \typename{Failure} is caught by \funcname{caml\_start\_program}
\begin{itemize}
\item<1-> The handler set the 2nd LSB of the exception value to \funcarg{1}
\item<3-> And then forces \funcname{caml\_start\_program} to terminate
\end{itemize}
\end{itemize}
}
\bigskip
\only<1,3>{\listocaml[highlightlines=3]{../src/default/default.ml}{default/default.ml}}%
\only<2>{\listasm[firstline=66,lastline=72,highlightlines=69]{src/ocaml/caml\_start\_program.s}{caml\_start\_program}}%
\end{column}
\begin{column}{0.4\textwidth}
\centering
\providecommand\step{4}\input{diagrams/default/default.tex}
\end{column}
\end{columns}
\end{frame}
%\only<>{\listasm[firstline=47,lastline=65]{src/ocaml/caml\_start\_program.s}{caml\_start\_program}}


\begin{frame}{Back to C}{Clean up, complain and terminate}
\begin{columns}
\begin{column}{0.45\textwidth}
\begin{itemize}
\item Backtrace:
\begin{itemize}
\item \funcname{caml\_start\_program} $\rightarrow$ OCaml code
\item \funcname{caml\_startup\_common}
\item \funcname{caml\_startup\_exn}
\item \funcname{caml\_startup}
\item \funcname{caml\_main}
\item \funcname{main}
\end{itemize}
\smallskip
\item \funcname{caml\_startup} checks the return value of \funcname{caml\_startup\_exn}
\begin{itemize}
\item The return value has been specially marked by \funcname{caml\_start\_program} handler
\item Calls \funcname{caml\_fatal\_uncaught\_exception}
\end{itemize}
\item<2-> \funcname{caml\_fatal\_uncaught\_exception} either
\begin{itemize}
\item<2-> Calls the user custom uncaught exception handler, if set with \mintinline{ocaml}{Printexc.set_uncaught_exception_handler fn}\footnotemark
\item<2-> Or falls back to the default one
\item<2-> Which notifies about the uncaught exception
\end{itemize}
\item<4-> Terminates the program either with \funcname{abort} or with a non-zero exit code
\end{itemize}
\end{column}
\begin{column}{0.55\textwidth}
\only<1>{
\listc[firstline=84,lastline=89,highlightlines=88]{../ocaml/runtime/caml/mlvalues.h}{runtime/caml/mlvalues.h:84}
\listc[firstline=138,lastline=143,highlightlines={141-142}]{../ocaml/runtime/startup\_nat.c}{runtime/startup\_nat.c:138}
}%
\only<2>{
\listc[firstline=138,highlightlines={147,149}]{../ocaml/runtime/printexc.c}{runtime/printexc.c:138}
}%
\only<3>{
\listc[firstline=110,lastline=134,highlightlines=129]{../ocaml/runtime/printexc.c}{runtime/printexc.c:138}
}%
\only<4>{
\listc[firstline=138,highlightlines={152,154}]{../ocaml/runtime/printexc.c}{runtime/printexc.c:138}
}%
\end{column}
\end{columns}
\footnotetext[1]{\url{https://v2.ocaml.org/api/Printexc.html\#1\_Uncaughtexceptions}}
\end{frame}
}%
\end{column}
\end{columns}
\end{frame}

\begin{frame}{Executing the OCaml program}{And raising an exception without handler}
\begin{columns}
\begin{column}{0.6\textwidth}
\only<1-3>{
\begin{itemize}
\item<1-> Execution continues with OCaml code
\begin{itemize}
\item \funcname{camlDefault\_\_main\_269}
\item \funcname{camlDefault\_\_entry}
\item \funcname{caml\_program}
\end{itemize}
\item<1-> \funcname{camlDefault\_\_main\_269} raises a \typename{Failure} exception
\item<1-> \typename{Failure} is caught by \funcname{caml\_start\_program}
\begin{itemize}
\item<1-> The handler set the 2nd LSB of the exception value to \funcarg{1}
\item<3-> And then forces \funcname{caml\_start\_program} to terminate
\end{itemize}
\end{itemize}
}
\bigskip
\only<1,3>{\listocaml[highlightlines=3]{../src/default/default.ml}{default/default.ml}}%
\only<2>{\listasm[firstline=66,lastline=72,highlightlines=69]{src/ocaml/caml\_start\_program.s}{caml\_start\_program}}%
\end{column}
\begin{column}{0.4\textwidth}
\centering
\providecommand\step{4}%
% Runtime's default exception handler
%
\subsection*{Runtime's default exception handler}
\frameSubsection{pictures/The\_Architect.png}{}


\begin{frame}{Default exception handler}
\begin{columns}
\begin{column}{0.5\textwidth}
\begin{itemize}
\item \localname{domain\_state->exn\_handler} points to a trap located before \funcname{entry}!
\item What installed this very first trap there?
\end{itemize}
\bigskip
\listocaml{../src/default/default.ml}{default/default.ml}
\bigskip
\inputminted[fontsize=\footnotesize]{text}{../src/default/default.out}
\end{column}
\begin{column}{0.5\textwidth}
\centering
\only<1>{\providecommand\step{0}\input{diagrams/default/default.tex}}%
\only<2>{\providecommand\step{4}\input{diagrams/default/default.tex}}%
\end{column}
\end{columns}
\end{frame}


\begin{frame}{Startup}{How did we get there by the way?}
\begin{columns}
\begin{column}{0.45\textwidth}
\begin{itemize}
\item The entry point address of the binary is \funcname{main} (\funcname{\_start}) from the runtime
\item The program starts like a C program:
\begin{itemize}
\item \funcname{caml\_start\_program}
\item \funcname{caml\_startup\_common}
\item \funcname{caml\_startup\_exn}
\item \funcname{caml\_startup}
\item \funcname{caml\_main}
\item \funcname{main}
\end{itemize}
%\item \funcname{caml\_startup\_common}: initializes GC, domains \dots
% Also: codefrag, locale, custom opeartions, frame_descr, signals, stack
\item \funcname{caml\_start\_program} is the transition point between the C realm and the OCaml one
\end{itemize}
\bigskip
\listocaml{../src/default/default.ml}{default/default.ml}
\end{column}
\begin{column}{0.55\textwidth}
\listasm[lastline=24,highlightlines={22-24}]{src/ocaml/caml\_start\_program.s}{runtime/amd64.S:604}
\end{column}
\end{columns}
\end{frame}


\begin{frame}{Setup to jump into OCaml entry point}{\funcname{caml\_start\_program} initializes the main fiber}
\begin{columns}
\begin{column}{0.6\textwidth}
\only<1,3,5>{
\begin{itemize}
\item<1-> Stores information to allow switching from OCaml stack to the C stack
\item<3-> Constructs the default exception handler
\item<5-> Switches to the OCaml stack and calls \funcname{caml\_program}
\end{itemize}
\bigskip
\listocaml{../src/default/default.ml}{default/default.ml}
}%
\only<2>{
\listasm[firstline=22,lastline=41,highlightlines={25-27}]{src/ocaml/caml\_start\_program.s}{caml\_start\_program}
}%
\only<4>{
\mintasm[firstline=22,lastline=41,highlightlines={31-38}]{src/ocaml/caml\_start\_program.s}
\listasm[firstline=66,lastline=72]{src/ocaml/caml\_start\_program.s}{caml\_start\_program}
}%
\only<6>{
\listasm[firstline=22,lastline=41,highlightlines={39-41}]{src/ocaml/caml\_start\_program.s}{caml\_start\_program}
}%
\end{column}
\begin{column}{0.4\textwidth}
\centering
\only<1-2>{\providecommand\step{2}\input{diagrams/default/default.tex}}%
\only<3-5>{\providecommand\step{3}\input{diagrams/default/default.tex}}%
\onslide*<6->{\providecommand\step{4}\input{diagrams/default/default.tex}}%
\end{column}
\end{columns}
\end{frame}

\begin{frame}{Executing the OCaml program}{And raising an exception without handler}
\begin{columns}
\begin{column}{0.6\textwidth}
\only<1-3>{
\begin{itemize}
\item<1-> Execution continues with OCaml code
\begin{itemize}
\item \funcname{camlDefault\_\_main\_269}
\item \funcname{camlDefault\_\_entry}
\item \funcname{caml\_program}
\end{itemize}
\item<1-> \funcname{camlDefault\_\_main\_269} raises a \typename{Failure} exception
\item<1-> \typename{Failure} is caught by \funcname{caml\_start\_program}
\begin{itemize}
\item<1-> The handler set the 2nd LSB of the exception value to \funcarg{1}
\item<3-> And then forces \funcname{caml\_start\_program} to terminate
\end{itemize}
\end{itemize}
}
\bigskip
\only<1,3>{\listocaml[highlightlines=3]{../src/default/default.ml}{default/default.ml}}%
\only<2>{\listasm[firstline=66,lastline=72,highlightlines=69]{src/ocaml/caml\_start\_program.s}{caml\_start\_program}}%
\end{column}
\begin{column}{0.4\textwidth}
\centering
\providecommand\step{4}\input{diagrams/default/default.tex}
\end{column}
\end{columns}
\end{frame}
%\only<>{\listasm[firstline=47,lastline=65]{src/ocaml/caml\_start\_program.s}{caml\_start\_program}}


\begin{frame}{Back to C}{Clean up, complain and terminate}
\begin{columns}
\begin{column}{0.45\textwidth}
\begin{itemize}
\item Backtrace:
\begin{itemize}
\item \funcname{caml\_start\_program} $\rightarrow$ OCaml code
\item \funcname{caml\_startup\_common}
\item \funcname{caml\_startup\_exn}
\item \funcname{caml\_startup}
\item \funcname{caml\_main}
\item \funcname{main}
\end{itemize}
\smallskip
\item \funcname{caml\_startup} checks the return value of \funcname{caml\_startup\_exn}
\begin{itemize}
\item The return value has been specially marked by \funcname{caml\_start\_program} handler
\item Calls \funcname{caml\_fatal\_uncaught\_exception}
\end{itemize}
\item<2-> \funcname{caml\_fatal\_uncaught\_exception} either
\begin{itemize}
\item<2-> Calls the user custom uncaught exception handler, if set with \mintinline{ocaml}{Printexc.set_uncaught_exception_handler fn}\footnotemark
\item<2-> Or falls back to the default one
\item<2-> Which notifies about the uncaught exception
\end{itemize}
\item<4-> Terminates the program either with \funcname{abort} or with a non-zero exit code
\end{itemize}
\end{column}
\begin{column}{0.55\textwidth}
\only<1>{
\listc[firstline=84,lastline=89,highlightlines=88]{../ocaml/runtime/caml/mlvalues.h}{runtime/caml/mlvalues.h:84}
\listc[firstline=138,lastline=143,highlightlines={141-142}]{../ocaml/runtime/startup\_nat.c}{runtime/startup\_nat.c:138}
}%
\only<2>{
\listc[firstline=138,highlightlines={147,149}]{../ocaml/runtime/printexc.c}{runtime/printexc.c:138}
}%
\only<3>{
\listc[firstline=110,lastline=134,highlightlines=129]{../ocaml/runtime/printexc.c}{runtime/printexc.c:138}
}%
\only<4>{
\listc[firstline=138,highlightlines={152,154}]{../ocaml/runtime/printexc.c}{runtime/printexc.c:138}
}%
\end{column}
\end{columns}
\footnotetext[1]{\url{https://v2.ocaml.org/api/Printexc.html\#1\_Uncaughtexceptions}}
\end{frame}

\end{column}
\end{columns}
\end{frame}
%\only<>{\listasm[firstline=47,lastline=65]{src/ocaml/caml\_start\_program.s}{caml\_start\_program}}


\begin{frame}{Back to C}{Clean up, complain and terminate}
\begin{columns}
\begin{column}{0.45\textwidth}
\begin{itemize}
\item Backtrace:
\begin{itemize}
\item \funcname{caml\_start\_program} $\rightarrow$ OCaml code
\item \funcname{caml\_startup\_common}
\item \funcname{caml\_startup\_exn}
\item \funcname{caml\_startup}
\item \funcname{caml\_main}
\item \funcname{main}
\end{itemize}
\smallskip
\item \funcname{caml\_startup} checks the return value of \funcname{caml\_startup\_exn}
\begin{itemize}
\item The return value has been specially marked by \funcname{caml\_start\_program} handler
\item Calls \funcname{caml\_fatal\_uncaught\_exception}
\end{itemize}
\item<2-> \funcname{caml\_fatal\_uncaught\_exception} either
\begin{itemize}
\item<2-> Calls the user custom uncaught exception handler, if set with \mintinline{ocaml}{Printexc.set_uncaught_exception_handler fn}\footnotemark
\item<2-> Or falls back to the default one
\item<2-> Which notifies about the uncaught exception
\end{itemize}
\item<4-> Terminates the program either with \funcname{abort} or with a non-zero exit code
\end{itemize}
\end{column}
\begin{column}{0.55\textwidth}
\only<1>{
\listc[firstline=84,lastline=89,highlightlines=88]{../ocaml/runtime/caml/mlvalues.h}{runtime/caml/mlvalues.h:84}
\listc[firstline=138,lastline=143,highlightlines={141-142}]{../ocaml/runtime/startup\_nat.c}{runtime/startup\_nat.c:138}
}%
\only<2>{
\listc[firstline=138,highlightlines={147,149}]{../ocaml/runtime/printexc.c}{runtime/printexc.c:138}
}%
\only<3>{
\listc[firstline=110,lastline=134,highlightlines=129]{../ocaml/runtime/printexc.c}{runtime/printexc.c:138}
}%
\only<4>{
\listc[firstline=138,highlightlines={152,154}]{../ocaml/runtime/printexc.c}{runtime/printexc.c:138}
}%
\end{column}
\end{columns}
\footnotetext[1]{\url{https://v2.ocaml.org/api/Printexc.html\#1\_Uncaughtexceptions}}
\end{frame}
}%
\only<2>{\providecommand\step{4}%
% Runtime's default exception handler
%
\subsection*{Runtime's default exception handler}
\frameSubsection{pictures/The\_Architect.png}{}


\begin{frame}{Default exception handler}
\begin{columns}
\begin{column}{0.5\textwidth}
\begin{itemize}
\item \localname{domain\_state->exn\_handler} points to a trap located before \funcname{entry}!
\item What installed this very first trap there?
\end{itemize}
\bigskip
\listocaml{../src/default/default.ml}{default/default.ml}
\bigskip
\inputminted[fontsize=\footnotesize]{text}{../src/default/default.out}
\end{column}
\begin{column}{0.5\textwidth}
\centering
\only<1>{\providecommand\step{0}%
% Runtime's default exception handler
%
\subsection*{Runtime's default exception handler}
\frameSubsection{pictures/The\_Architect.png}{}


\begin{frame}{Default exception handler}
\begin{columns}
\begin{column}{0.5\textwidth}
\begin{itemize}
\item \localname{domain\_state->exn\_handler} points to a trap located before \funcname{entry}!
\item What installed this very first trap there?
\end{itemize}
\bigskip
\listocaml{../src/default/default.ml}{default/default.ml}
\bigskip
\inputminted[fontsize=\footnotesize]{text}{../src/default/default.out}
\end{column}
\begin{column}{0.5\textwidth}
\centering
\only<1>{\providecommand\step{0}\input{diagrams/default/default.tex}}%
\only<2>{\providecommand\step{4}\input{diagrams/default/default.tex}}%
\end{column}
\end{columns}
\end{frame}


\begin{frame}{Startup}{How did we get there by the way?}
\begin{columns}
\begin{column}{0.45\textwidth}
\begin{itemize}
\item The entry point address of the binary is \funcname{main} (\funcname{\_start}) from the runtime
\item The program starts like a C program:
\begin{itemize}
\item \funcname{caml\_start\_program}
\item \funcname{caml\_startup\_common}
\item \funcname{caml\_startup\_exn}
\item \funcname{caml\_startup}
\item \funcname{caml\_main}
\item \funcname{main}
\end{itemize}
%\item \funcname{caml\_startup\_common}: initializes GC, domains \dots
% Also: codefrag, locale, custom opeartions, frame_descr, signals, stack
\item \funcname{caml\_start\_program} is the transition point between the C realm and the OCaml one
\end{itemize}
\bigskip
\listocaml{../src/default/default.ml}{default/default.ml}
\end{column}
\begin{column}{0.55\textwidth}
\listasm[lastline=24,highlightlines={22-24}]{src/ocaml/caml\_start\_program.s}{runtime/amd64.S:604}
\end{column}
\end{columns}
\end{frame}


\begin{frame}{Setup to jump into OCaml entry point}{\funcname{caml\_start\_program} initializes the main fiber}
\begin{columns}
\begin{column}{0.6\textwidth}
\only<1,3,5>{
\begin{itemize}
\item<1-> Stores information to allow switching from OCaml stack to the C stack
\item<3-> Constructs the default exception handler
\item<5-> Switches to the OCaml stack and calls \funcname{caml\_program}
\end{itemize}
\bigskip
\listocaml{../src/default/default.ml}{default/default.ml}
}%
\only<2>{
\listasm[firstline=22,lastline=41,highlightlines={25-27}]{src/ocaml/caml\_start\_program.s}{caml\_start\_program}
}%
\only<4>{
\mintasm[firstline=22,lastline=41,highlightlines={31-38}]{src/ocaml/caml\_start\_program.s}
\listasm[firstline=66,lastline=72]{src/ocaml/caml\_start\_program.s}{caml\_start\_program}
}%
\only<6>{
\listasm[firstline=22,lastline=41,highlightlines={39-41}]{src/ocaml/caml\_start\_program.s}{caml\_start\_program}
}%
\end{column}
\begin{column}{0.4\textwidth}
\centering
\only<1-2>{\providecommand\step{2}\input{diagrams/default/default.tex}}%
\only<3-5>{\providecommand\step{3}\input{diagrams/default/default.tex}}%
\onslide*<6->{\providecommand\step{4}\input{diagrams/default/default.tex}}%
\end{column}
\end{columns}
\end{frame}

\begin{frame}{Executing the OCaml program}{And raising an exception without handler}
\begin{columns}
\begin{column}{0.6\textwidth}
\only<1-3>{
\begin{itemize}
\item<1-> Execution continues with OCaml code
\begin{itemize}
\item \funcname{camlDefault\_\_main\_269}
\item \funcname{camlDefault\_\_entry}
\item \funcname{caml\_program}
\end{itemize}
\item<1-> \funcname{camlDefault\_\_main\_269} raises a \typename{Failure} exception
\item<1-> \typename{Failure} is caught by \funcname{caml\_start\_program}
\begin{itemize}
\item<1-> The handler set the 2nd LSB of the exception value to \funcarg{1}
\item<3-> And then forces \funcname{caml\_start\_program} to terminate
\end{itemize}
\end{itemize}
}
\bigskip
\only<1,3>{\listocaml[highlightlines=3]{../src/default/default.ml}{default/default.ml}}%
\only<2>{\listasm[firstline=66,lastline=72,highlightlines=69]{src/ocaml/caml\_start\_program.s}{caml\_start\_program}}%
\end{column}
\begin{column}{0.4\textwidth}
\centering
\providecommand\step{4}\input{diagrams/default/default.tex}
\end{column}
\end{columns}
\end{frame}
%\only<>{\listasm[firstline=47,lastline=65]{src/ocaml/caml\_start\_program.s}{caml\_start\_program}}


\begin{frame}{Back to C}{Clean up, complain and terminate}
\begin{columns}
\begin{column}{0.45\textwidth}
\begin{itemize}
\item Backtrace:
\begin{itemize}
\item \funcname{caml\_start\_program} $\rightarrow$ OCaml code
\item \funcname{caml\_startup\_common}
\item \funcname{caml\_startup\_exn}
\item \funcname{caml\_startup}
\item \funcname{caml\_main}
\item \funcname{main}
\end{itemize}
\smallskip
\item \funcname{caml\_startup} checks the return value of \funcname{caml\_startup\_exn}
\begin{itemize}
\item The return value has been specially marked by \funcname{caml\_start\_program} handler
\item Calls \funcname{caml\_fatal\_uncaught\_exception}
\end{itemize}
\item<2-> \funcname{caml\_fatal\_uncaught\_exception} either
\begin{itemize}
\item<2-> Calls the user custom uncaught exception handler, if set with \mintinline{ocaml}{Printexc.set_uncaught_exception_handler fn}\footnotemark
\item<2-> Or falls back to the default one
\item<2-> Which notifies about the uncaught exception
\end{itemize}
\item<4-> Terminates the program either with \funcname{abort} or with a non-zero exit code
\end{itemize}
\end{column}
\begin{column}{0.55\textwidth}
\only<1>{
\listc[firstline=84,lastline=89,highlightlines=88]{../ocaml/runtime/caml/mlvalues.h}{runtime/caml/mlvalues.h:84}
\listc[firstline=138,lastline=143,highlightlines={141-142}]{../ocaml/runtime/startup\_nat.c}{runtime/startup\_nat.c:138}
}%
\only<2>{
\listc[firstline=138,highlightlines={147,149}]{../ocaml/runtime/printexc.c}{runtime/printexc.c:138}
}%
\only<3>{
\listc[firstline=110,lastline=134,highlightlines=129]{../ocaml/runtime/printexc.c}{runtime/printexc.c:138}
}%
\only<4>{
\listc[firstline=138,highlightlines={152,154}]{../ocaml/runtime/printexc.c}{runtime/printexc.c:138}
}%
\end{column}
\end{columns}
\footnotetext[1]{\url{https://v2.ocaml.org/api/Printexc.html\#1\_Uncaughtexceptions}}
\end{frame}
}%
\only<2>{\providecommand\step{4}%
% Runtime's default exception handler
%
\subsection*{Runtime's default exception handler}
\frameSubsection{pictures/The\_Architect.png}{}


\begin{frame}{Default exception handler}
\begin{columns}
\begin{column}{0.5\textwidth}
\begin{itemize}
\item \localname{domain\_state->exn\_handler} points to a trap located before \funcname{entry}!
\item What installed this very first trap there?
\end{itemize}
\bigskip
\listocaml{../src/default/default.ml}{default/default.ml}
\bigskip
\inputminted[fontsize=\footnotesize]{text}{../src/default/default.out}
\end{column}
\begin{column}{0.5\textwidth}
\centering
\only<1>{\providecommand\step{0}\input{diagrams/default/default.tex}}%
\only<2>{\providecommand\step{4}\input{diagrams/default/default.tex}}%
\end{column}
\end{columns}
\end{frame}


\begin{frame}{Startup}{How did we get there by the way?}
\begin{columns}
\begin{column}{0.45\textwidth}
\begin{itemize}
\item The entry point address of the binary is \funcname{main} (\funcname{\_start}) from the runtime
\item The program starts like a C program:
\begin{itemize}
\item \funcname{caml\_start\_program}
\item \funcname{caml\_startup\_common}
\item \funcname{caml\_startup\_exn}
\item \funcname{caml\_startup}
\item \funcname{caml\_main}
\item \funcname{main}
\end{itemize}
%\item \funcname{caml\_startup\_common}: initializes GC, domains \dots
% Also: codefrag, locale, custom opeartions, frame_descr, signals, stack
\item \funcname{caml\_start\_program} is the transition point between the C realm and the OCaml one
\end{itemize}
\bigskip
\listocaml{../src/default/default.ml}{default/default.ml}
\end{column}
\begin{column}{0.55\textwidth}
\listasm[lastline=24,highlightlines={22-24}]{src/ocaml/caml\_start\_program.s}{runtime/amd64.S:604}
\end{column}
\end{columns}
\end{frame}


\begin{frame}{Setup to jump into OCaml entry point}{\funcname{caml\_start\_program} initializes the main fiber}
\begin{columns}
\begin{column}{0.6\textwidth}
\only<1,3,5>{
\begin{itemize}
\item<1-> Stores information to allow switching from OCaml stack to the C stack
\item<3-> Constructs the default exception handler
\item<5-> Switches to the OCaml stack and calls \funcname{caml\_program}
\end{itemize}
\bigskip
\listocaml{../src/default/default.ml}{default/default.ml}
}%
\only<2>{
\listasm[firstline=22,lastline=41,highlightlines={25-27}]{src/ocaml/caml\_start\_program.s}{caml\_start\_program}
}%
\only<4>{
\mintasm[firstline=22,lastline=41,highlightlines={31-38}]{src/ocaml/caml\_start\_program.s}
\listasm[firstline=66,lastline=72]{src/ocaml/caml\_start\_program.s}{caml\_start\_program}
}%
\only<6>{
\listasm[firstline=22,lastline=41,highlightlines={39-41}]{src/ocaml/caml\_start\_program.s}{caml\_start\_program}
}%
\end{column}
\begin{column}{0.4\textwidth}
\centering
\only<1-2>{\providecommand\step{2}\input{diagrams/default/default.tex}}%
\only<3-5>{\providecommand\step{3}\input{diagrams/default/default.tex}}%
\onslide*<6->{\providecommand\step{4}\input{diagrams/default/default.tex}}%
\end{column}
\end{columns}
\end{frame}

\begin{frame}{Executing the OCaml program}{And raising an exception without handler}
\begin{columns}
\begin{column}{0.6\textwidth}
\only<1-3>{
\begin{itemize}
\item<1-> Execution continues with OCaml code
\begin{itemize}
\item \funcname{camlDefault\_\_main\_269}
\item \funcname{camlDefault\_\_entry}
\item \funcname{caml\_program}
\end{itemize}
\item<1-> \funcname{camlDefault\_\_main\_269} raises a \typename{Failure} exception
\item<1-> \typename{Failure} is caught by \funcname{caml\_start\_program}
\begin{itemize}
\item<1-> The handler set the 2nd LSB of the exception value to \funcarg{1}
\item<3-> And then forces \funcname{caml\_start\_program} to terminate
\end{itemize}
\end{itemize}
}
\bigskip
\only<1,3>{\listocaml[highlightlines=3]{../src/default/default.ml}{default/default.ml}}%
\only<2>{\listasm[firstline=66,lastline=72,highlightlines=69]{src/ocaml/caml\_start\_program.s}{caml\_start\_program}}%
\end{column}
\begin{column}{0.4\textwidth}
\centering
\providecommand\step{4}\input{diagrams/default/default.tex}
\end{column}
\end{columns}
\end{frame}
%\only<>{\listasm[firstline=47,lastline=65]{src/ocaml/caml\_start\_program.s}{caml\_start\_program}}


\begin{frame}{Back to C}{Clean up, complain and terminate}
\begin{columns}
\begin{column}{0.45\textwidth}
\begin{itemize}
\item Backtrace:
\begin{itemize}
\item \funcname{caml\_start\_program} $\rightarrow$ OCaml code
\item \funcname{caml\_startup\_common}
\item \funcname{caml\_startup\_exn}
\item \funcname{caml\_startup}
\item \funcname{caml\_main}
\item \funcname{main}
\end{itemize}
\smallskip
\item \funcname{caml\_startup} checks the return value of \funcname{caml\_startup\_exn}
\begin{itemize}
\item The return value has been specially marked by \funcname{caml\_start\_program} handler
\item Calls \funcname{caml\_fatal\_uncaught\_exception}
\end{itemize}
\item<2-> \funcname{caml\_fatal\_uncaught\_exception} either
\begin{itemize}
\item<2-> Calls the user custom uncaught exception handler, if set with \mintinline{ocaml}{Printexc.set_uncaught_exception_handler fn}\footnotemark
\item<2-> Or falls back to the default one
\item<2-> Which notifies about the uncaught exception
\end{itemize}
\item<4-> Terminates the program either with \funcname{abort} or with a non-zero exit code
\end{itemize}
\end{column}
\begin{column}{0.55\textwidth}
\only<1>{
\listc[firstline=84,lastline=89,highlightlines=88]{../ocaml/runtime/caml/mlvalues.h}{runtime/caml/mlvalues.h:84}
\listc[firstline=138,lastline=143,highlightlines={141-142}]{../ocaml/runtime/startup\_nat.c}{runtime/startup\_nat.c:138}
}%
\only<2>{
\listc[firstline=138,highlightlines={147,149}]{../ocaml/runtime/printexc.c}{runtime/printexc.c:138}
}%
\only<3>{
\listc[firstline=110,lastline=134,highlightlines=129]{../ocaml/runtime/printexc.c}{runtime/printexc.c:138}
}%
\only<4>{
\listc[firstline=138,highlightlines={152,154}]{../ocaml/runtime/printexc.c}{runtime/printexc.c:138}
}%
\end{column}
\end{columns}
\footnotetext[1]{\url{https://v2.ocaml.org/api/Printexc.html\#1\_Uncaughtexceptions}}
\end{frame}
}%
\end{column}
\end{columns}
\end{frame}


\begin{frame}{Startup}{How did we get there by the way?}
\begin{columns}
\begin{column}{0.45\textwidth}
\begin{itemize}
\item The entry point address of the binary is \funcname{main} (\funcname{\_start}) from the runtime
\item The program starts like a C program:
\begin{itemize}
\item \funcname{caml\_start\_program}
\item \funcname{caml\_startup\_common}
\item \funcname{caml\_startup\_exn}
\item \funcname{caml\_startup}
\item \funcname{caml\_main}
\item \funcname{main}
\end{itemize}
%\item \funcname{caml\_startup\_common}: initializes GC, domains \dots
% Also: codefrag, locale, custom opeartions, frame_descr, signals, stack
\item \funcname{caml\_start\_program} is the transition point between the C realm and the OCaml one
\end{itemize}
\bigskip
\listocaml{../src/default/default.ml}{default/default.ml}
\end{column}
\begin{column}{0.55\textwidth}
\listasm[lastline=24,highlightlines={22-24}]{src/ocaml/caml\_start\_program.s}{runtime/amd64.S:604}
\end{column}
\end{columns}
\end{frame}


\begin{frame}{Setup to jump into OCaml entry point}{\funcname{caml\_start\_program} initializes the main fiber}
\begin{columns}
\begin{column}{0.6\textwidth}
\only<1,3,5>{
\begin{itemize}
\item<1-> Stores information to allow switching from OCaml stack to the C stack
\item<3-> Constructs the default exception handler
\item<5-> Switches to the OCaml stack and calls \funcname{caml\_program}
\end{itemize}
\bigskip
\listocaml{../src/default/default.ml}{default/default.ml}
}%
\only<2>{
\listasm[firstline=22,lastline=41,highlightlines={25-27}]{src/ocaml/caml\_start\_program.s}{caml\_start\_program}
}%
\only<4>{
\mintasm[firstline=22,lastline=41,highlightlines={31-38}]{src/ocaml/caml\_start\_program.s}
\listasm[firstline=66,lastline=72]{src/ocaml/caml\_start\_program.s}{caml\_start\_program}
}%
\only<6>{
\listasm[firstline=22,lastline=41,highlightlines={39-41}]{src/ocaml/caml\_start\_program.s}{caml\_start\_program}
}%
\end{column}
\begin{column}{0.4\textwidth}
\centering
\only<1-2>{\providecommand\step{2}%
% Runtime's default exception handler
%
\subsection*{Runtime's default exception handler}
\frameSubsection{pictures/The\_Architect.png}{}


\begin{frame}{Default exception handler}
\begin{columns}
\begin{column}{0.5\textwidth}
\begin{itemize}
\item \localname{domain\_state->exn\_handler} points to a trap located before \funcname{entry}!
\item What installed this very first trap there?
\end{itemize}
\bigskip
\listocaml{../src/default/default.ml}{default/default.ml}
\bigskip
\inputminted[fontsize=\footnotesize]{text}{../src/default/default.out}
\end{column}
\begin{column}{0.5\textwidth}
\centering
\only<1>{\providecommand\step{0}\input{diagrams/default/default.tex}}%
\only<2>{\providecommand\step{4}\input{diagrams/default/default.tex}}%
\end{column}
\end{columns}
\end{frame}


\begin{frame}{Startup}{How did we get there by the way?}
\begin{columns}
\begin{column}{0.45\textwidth}
\begin{itemize}
\item The entry point address of the binary is \funcname{main} (\funcname{\_start}) from the runtime
\item The program starts like a C program:
\begin{itemize}
\item \funcname{caml\_start\_program}
\item \funcname{caml\_startup\_common}
\item \funcname{caml\_startup\_exn}
\item \funcname{caml\_startup}
\item \funcname{caml\_main}
\item \funcname{main}
\end{itemize}
%\item \funcname{caml\_startup\_common}: initializes GC, domains \dots
% Also: codefrag, locale, custom opeartions, frame_descr, signals, stack
\item \funcname{caml\_start\_program} is the transition point between the C realm and the OCaml one
\end{itemize}
\bigskip
\listocaml{../src/default/default.ml}{default/default.ml}
\end{column}
\begin{column}{0.55\textwidth}
\listasm[lastline=24,highlightlines={22-24}]{src/ocaml/caml\_start\_program.s}{runtime/amd64.S:604}
\end{column}
\end{columns}
\end{frame}


\begin{frame}{Setup to jump into OCaml entry point}{\funcname{caml\_start\_program} initializes the main fiber}
\begin{columns}
\begin{column}{0.6\textwidth}
\only<1,3,5>{
\begin{itemize}
\item<1-> Stores information to allow switching from OCaml stack to the C stack
\item<3-> Constructs the default exception handler
\item<5-> Switches to the OCaml stack and calls \funcname{caml\_program}
\end{itemize}
\bigskip
\listocaml{../src/default/default.ml}{default/default.ml}
}%
\only<2>{
\listasm[firstline=22,lastline=41,highlightlines={25-27}]{src/ocaml/caml\_start\_program.s}{caml\_start\_program}
}%
\only<4>{
\mintasm[firstline=22,lastline=41,highlightlines={31-38}]{src/ocaml/caml\_start\_program.s}
\listasm[firstline=66,lastline=72]{src/ocaml/caml\_start\_program.s}{caml\_start\_program}
}%
\only<6>{
\listasm[firstline=22,lastline=41,highlightlines={39-41}]{src/ocaml/caml\_start\_program.s}{caml\_start\_program}
}%
\end{column}
\begin{column}{0.4\textwidth}
\centering
\only<1-2>{\providecommand\step{2}\input{diagrams/default/default.tex}}%
\only<3-5>{\providecommand\step{3}\input{diagrams/default/default.tex}}%
\onslide*<6->{\providecommand\step{4}\input{diagrams/default/default.tex}}%
\end{column}
\end{columns}
\end{frame}

\begin{frame}{Executing the OCaml program}{And raising an exception without handler}
\begin{columns}
\begin{column}{0.6\textwidth}
\only<1-3>{
\begin{itemize}
\item<1-> Execution continues with OCaml code
\begin{itemize}
\item \funcname{camlDefault\_\_main\_269}
\item \funcname{camlDefault\_\_entry}
\item \funcname{caml\_program}
\end{itemize}
\item<1-> \funcname{camlDefault\_\_main\_269} raises a \typename{Failure} exception
\item<1-> \typename{Failure} is caught by \funcname{caml\_start\_program}
\begin{itemize}
\item<1-> The handler set the 2nd LSB of the exception value to \funcarg{1}
\item<3-> And then forces \funcname{caml\_start\_program} to terminate
\end{itemize}
\end{itemize}
}
\bigskip
\only<1,3>{\listocaml[highlightlines=3]{../src/default/default.ml}{default/default.ml}}%
\only<2>{\listasm[firstline=66,lastline=72,highlightlines=69]{src/ocaml/caml\_start\_program.s}{caml\_start\_program}}%
\end{column}
\begin{column}{0.4\textwidth}
\centering
\providecommand\step{4}\input{diagrams/default/default.tex}
\end{column}
\end{columns}
\end{frame}
%\only<>{\listasm[firstline=47,lastline=65]{src/ocaml/caml\_start\_program.s}{caml\_start\_program}}


\begin{frame}{Back to C}{Clean up, complain and terminate}
\begin{columns}
\begin{column}{0.45\textwidth}
\begin{itemize}
\item Backtrace:
\begin{itemize}
\item \funcname{caml\_start\_program} $\rightarrow$ OCaml code
\item \funcname{caml\_startup\_common}
\item \funcname{caml\_startup\_exn}
\item \funcname{caml\_startup}
\item \funcname{caml\_main}
\item \funcname{main}
\end{itemize}
\smallskip
\item \funcname{caml\_startup} checks the return value of \funcname{caml\_startup\_exn}
\begin{itemize}
\item The return value has been specially marked by \funcname{caml\_start\_program} handler
\item Calls \funcname{caml\_fatal\_uncaught\_exception}
\end{itemize}
\item<2-> \funcname{caml\_fatal\_uncaught\_exception} either
\begin{itemize}
\item<2-> Calls the user custom uncaught exception handler, if set with \mintinline{ocaml}{Printexc.set_uncaught_exception_handler fn}\footnotemark
\item<2-> Or falls back to the default one
\item<2-> Which notifies about the uncaught exception
\end{itemize}
\item<4-> Terminates the program either with \funcname{abort} or with a non-zero exit code
\end{itemize}
\end{column}
\begin{column}{0.55\textwidth}
\only<1>{
\listc[firstline=84,lastline=89,highlightlines=88]{../ocaml/runtime/caml/mlvalues.h}{runtime/caml/mlvalues.h:84}
\listc[firstline=138,lastline=143,highlightlines={141-142}]{../ocaml/runtime/startup\_nat.c}{runtime/startup\_nat.c:138}
}%
\only<2>{
\listc[firstline=138,highlightlines={147,149}]{../ocaml/runtime/printexc.c}{runtime/printexc.c:138}
}%
\only<3>{
\listc[firstline=110,lastline=134,highlightlines=129]{../ocaml/runtime/printexc.c}{runtime/printexc.c:138}
}%
\only<4>{
\listc[firstline=138,highlightlines={152,154}]{../ocaml/runtime/printexc.c}{runtime/printexc.c:138}
}%
\end{column}
\end{columns}
\footnotetext[1]{\url{https://v2.ocaml.org/api/Printexc.html\#1\_Uncaughtexceptions}}
\end{frame}
}%
\only<3-5>{\providecommand\step{3}%
% Runtime's default exception handler
%
\subsection*{Runtime's default exception handler}
\frameSubsection{pictures/The\_Architect.png}{}


\begin{frame}{Default exception handler}
\begin{columns}
\begin{column}{0.5\textwidth}
\begin{itemize}
\item \localname{domain\_state->exn\_handler} points to a trap located before \funcname{entry}!
\item What installed this very first trap there?
\end{itemize}
\bigskip
\listocaml{../src/default/default.ml}{default/default.ml}
\bigskip
\inputminted[fontsize=\footnotesize]{text}{../src/default/default.out}
\end{column}
\begin{column}{0.5\textwidth}
\centering
\only<1>{\providecommand\step{0}\input{diagrams/default/default.tex}}%
\only<2>{\providecommand\step{4}\input{diagrams/default/default.tex}}%
\end{column}
\end{columns}
\end{frame}


\begin{frame}{Startup}{How did we get there by the way?}
\begin{columns}
\begin{column}{0.45\textwidth}
\begin{itemize}
\item The entry point address of the binary is \funcname{main} (\funcname{\_start}) from the runtime
\item The program starts like a C program:
\begin{itemize}
\item \funcname{caml\_start\_program}
\item \funcname{caml\_startup\_common}
\item \funcname{caml\_startup\_exn}
\item \funcname{caml\_startup}
\item \funcname{caml\_main}
\item \funcname{main}
\end{itemize}
%\item \funcname{caml\_startup\_common}: initializes GC, domains \dots
% Also: codefrag, locale, custom opeartions, frame_descr, signals, stack
\item \funcname{caml\_start\_program} is the transition point between the C realm and the OCaml one
\end{itemize}
\bigskip
\listocaml{../src/default/default.ml}{default/default.ml}
\end{column}
\begin{column}{0.55\textwidth}
\listasm[lastline=24,highlightlines={22-24}]{src/ocaml/caml\_start\_program.s}{runtime/amd64.S:604}
\end{column}
\end{columns}
\end{frame}


\begin{frame}{Setup to jump into OCaml entry point}{\funcname{caml\_start\_program} initializes the main fiber}
\begin{columns}
\begin{column}{0.6\textwidth}
\only<1,3,5>{
\begin{itemize}
\item<1-> Stores information to allow switching from OCaml stack to the C stack
\item<3-> Constructs the default exception handler
\item<5-> Switches to the OCaml stack and calls \funcname{caml\_program}
\end{itemize}
\bigskip
\listocaml{../src/default/default.ml}{default/default.ml}
}%
\only<2>{
\listasm[firstline=22,lastline=41,highlightlines={25-27}]{src/ocaml/caml\_start\_program.s}{caml\_start\_program}
}%
\only<4>{
\mintasm[firstline=22,lastline=41,highlightlines={31-38}]{src/ocaml/caml\_start\_program.s}
\listasm[firstline=66,lastline=72]{src/ocaml/caml\_start\_program.s}{caml\_start\_program}
}%
\only<6>{
\listasm[firstline=22,lastline=41,highlightlines={39-41}]{src/ocaml/caml\_start\_program.s}{caml\_start\_program}
}%
\end{column}
\begin{column}{0.4\textwidth}
\centering
\only<1-2>{\providecommand\step{2}\input{diagrams/default/default.tex}}%
\only<3-5>{\providecommand\step{3}\input{diagrams/default/default.tex}}%
\onslide*<6->{\providecommand\step{4}\input{diagrams/default/default.tex}}%
\end{column}
\end{columns}
\end{frame}

\begin{frame}{Executing the OCaml program}{And raising an exception without handler}
\begin{columns}
\begin{column}{0.6\textwidth}
\only<1-3>{
\begin{itemize}
\item<1-> Execution continues with OCaml code
\begin{itemize}
\item \funcname{camlDefault\_\_main\_269}
\item \funcname{camlDefault\_\_entry}
\item \funcname{caml\_program}
\end{itemize}
\item<1-> \funcname{camlDefault\_\_main\_269} raises a \typename{Failure} exception
\item<1-> \typename{Failure} is caught by \funcname{caml\_start\_program}
\begin{itemize}
\item<1-> The handler set the 2nd LSB of the exception value to \funcarg{1}
\item<3-> And then forces \funcname{caml\_start\_program} to terminate
\end{itemize}
\end{itemize}
}
\bigskip
\only<1,3>{\listocaml[highlightlines=3]{../src/default/default.ml}{default/default.ml}}%
\only<2>{\listasm[firstline=66,lastline=72,highlightlines=69]{src/ocaml/caml\_start\_program.s}{caml\_start\_program}}%
\end{column}
\begin{column}{0.4\textwidth}
\centering
\providecommand\step{4}\input{diagrams/default/default.tex}
\end{column}
\end{columns}
\end{frame}
%\only<>{\listasm[firstline=47,lastline=65]{src/ocaml/caml\_start\_program.s}{caml\_start\_program}}


\begin{frame}{Back to C}{Clean up, complain and terminate}
\begin{columns}
\begin{column}{0.45\textwidth}
\begin{itemize}
\item Backtrace:
\begin{itemize}
\item \funcname{caml\_start\_program} $\rightarrow$ OCaml code
\item \funcname{caml\_startup\_common}
\item \funcname{caml\_startup\_exn}
\item \funcname{caml\_startup}
\item \funcname{caml\_main}
\item \funcname{main}
\end{itemize}
\smallskip
\item \funcname{caml\_startup} checks the return value of \funcname{caml\_startup\_exn}
\begin{itemize}
\item The return value has been specially marked by \funcname{caml\_start\_program} handler
\item Calls \funcname{caml\_fatal\_uncaught\_exception}
\end{itemize}
\item<2-> \funcname{caml\_fatal\_uncaught\_exception} either
\begin{itemize}
\item<2-> Calls the user custom uncaught exception handler, if set with \mintinline{ocaml}{Printexc.set_uncaught_exception_handler fn}\footnotemark
\item<2-> Or falls back to the default one
\item<2-> Which notifies about the uncaught exception
\end{itemize}
\item<4-> Terminates the program either with \funcname{abort} or with a non-zero exit code
\end{itemize}
\end{column}
\begin{column}{0.55\textwidth}
\only<1>{
\listc[firstline=84,lastline=89,highlightlines=88]{../ocaml/runtime/caml/mlvalues.h}{runtime/caml/mlvalues.h:84}
\listc[firstline=138,lastline=143,highlightlines={141-142}]{../ocaml/runtime/startup\_nat.c}{runtime/startup\_nat.c:138}
}%
\only<2>{
\listc[firstline=138,highlightlines={147,149}]{../ocaml/runtime/printexc.c}{runtime/printexc.c:138}
}%
\only<3>{
\listc[firstline=110,lastline=134,highlightlines=129]{../ocaml/runtime/printexc.c}{runtime/printexc.c:138}
}%
\only<4>{
\listc[firstline=138,highlightlines={152,154}]{../ocaml/runtime/printexc.c}{runtime/printexc.c:138}
}%
\end{column}
\end{columns}
\footnotetext[1]{\url{https://v2.ocaml.org/api/Printexc.html\#1\_Uncaughtexceptions}}
\end{frame}
}%
\onslide*<6->{\providecommand\step{4}%
% Runtime's default exception handler
%
\subsection*{Runtime's default exception handler}
\frameSubsection{pictures/The\_Architect.png}{}


\begin{frame}{Default exception handler}
\begin{columns}
\begin{column}{0.5\textwidth}
\begin{itemize}
\item \localname{domain\_state->exn\_handler} points to a trap located before \funcname{entry}!
\item What installed this very first trap there?
\end{itemize}
\bigskip
\listocaml{../src/default/default.ml}{default/default.ml}
\bigskip
\inputminted[fontsize=\footnotesize]{text}{../src/default/default.out}
\end{column}
\begin{column}{0.5\textwidth}
\centering
\only<1>{\providecommand\step{0}\input{diagrams/default/default.tex}}%
\only<2>{\providecommand\step{4}\input{diagrams/default/default.tex}}%
\end{column}
\end{columns}
\end{frame}


\begin{frame}{Startup}{How did we get there by the way?}
\begin{columns}
\begin{column}{0.45\textwidth}
\begin{itemize}
\item The entry point address of the binary is \funcname{main} (\funcname{\_start}) from the runtime
\item The program starts like a C program:
\begin{itemize}
\item \funcname{caml\_start\_program}
\item \funcname{caml\_startup\_common}
\item \funcname{caml\_startup\_exn}
\item \funcname{caml\_startup}
\item \funcname{caml\_main}
\item \funcname{main}
\end{itemize}
%\item \funcname{caml\_startup\_common}: initializes GC, domains \dots
% Also: codefrag, locale, custom opeartions, frame_descr, signals, stack
\item \funcname{caml\_start\_program} is the transition point between the C realm and the OCaml one
\end{itemize}
\bigskip
\listocaml{../src/default/default.ml}{default/default.ml}
\end{column}
\begin{column}{0.55\textwidth}
\listasm[lastline=24,highlightlines={22-24}]{src/ocaml/caml\_start\_program.s}{runtime/amd64.S:604}
\end{column}
\end{columns}
\end{frame}


\begin{frame}{Setup to jump into OCaml entry point}{\funcname{caml\_start\_program} initializes the main fiber}
\begin{columns}
\begin{column}{0.6\textwidth}
\only<1,3,5>{
\begin{itemize}
\item<1-> Stores information to allow switching from OCaml stack to the C stack
\item<3-> Constructs the default exception handler
\item<5-> Switches to the OCaml stack and calls \funcname{caml\_program}
\end{itemize}
\bigskip
\listocaml{../src/default/default.ml}{default/default.ml}
}%
\only<2>{
\listasm[firstline=22,lastline=41,highlightlines={25-27}]{src/ocaml/caml\_start\_program.s}{caml\_start\_program}
}%
\only<4>{
\mintasm[firstline=22,lastline=41,highlightlines={31-38}]{src/ocaml/caml\_start\_program.s}
\listasm[firstline=66,lastline=72]{src/ocaml/caml\_start\_program.s}{caml\_start\_program}
}%
\only<6>{
\listasm[firstline=22,lastline=41,highlightlines={39-41}]{src/ocaml/caml\_start\_program.s}{caml\_start\_program}
}%
\end{column}
\begin{column}{0.4\textwidth}
\centering
\only<1-2>{\providecommand\step{2}\input{diagrams/default/default.tex}}%
\only<3-5>{\providecommand\step{3}\input{diagrams/default/default.tex}}%
\onslide*<6->{\providecommand\step{4}\input{diagrams/default/default.tex}}%
\end{column}
\end{columns}
\end{frame}

\begin{frame}{Executing the OCaml program}{And raising an exception without handler}
\begin{columns}
\begin{column}{0.6\textwidth}
\only<1-3>{
\begin{itemize}
\item<1-> Execution continues with OCaml code
\begin{itemize}
\item \funcname{camlDefault\_\_main\_269}
\item \funcname{camlDefault\_\_entry}
\item \funcname{caml\_program}
\end{itemize}
\item<1-> \funcname{camlDefault\_\_main\_269} raises a \typename{Failure} exception
\item<1-> \typename{Failure} is caught by \funcname{caml\_start\_program}
\begin{itemize}
\item<1-> The handler set the 2nd LSB of the exception value to \funcarg{1}
\item<3-> And then forces \funcname{caml\_start\_program} to terminate
\end{itemize}
\end{itemize}
}
\bigskip
\only<1,3>{\listocaml[highlightlines=3]{../src/default/default.ml}{default/default.ml}}%
\only<2>{\listasm[firstline=66,lastline=72,highlightlines=69]{src/ocaml/caml\_start\_program.s}{caml\_start\_program}}%
\end{column}
\begin{column}{0.4\textwidth}
\centering
\providecommand\step{4}\input{diagrams/default/default.tex}
\end{column}
\end{columns}
\end{frame}
%\only<>{\listasm[firstline=47,lastline=65]{src/ocaml/caml\_start\_program.s}{caml\_start\_program}}


\begin{frame}{Back to C}{Clean up, complain and terminate}
\begin{columns}
\begin{column}{0.45\textwidth}
\begin{itemize}
\item Backtrace:
\begin{itemize}
\item \funcname{caml\_start\_program} $\rightarrow$ OCaml code
\item \funcname{caml\_startup\_common}
\item \funcname{caml\_startup\_exn}
\item \funcname{caml\_startup}
\item \funcname{caml\_main}
\item \funcname{main}
\end{itemize}
\smallskip
\item \funcname{caml\_startup} checks the return value of \funcname{caml\_startup\_exn}
\begin{itemize}
\item The return value has been specially marked by \funcname{caml\_start\_program} handler
\item Calls \funcname{caml\_fatal\_uncaught\_exception}
\end{itemize}
\item<2-> \funcname{caml\_fatal\_uncaught\_exception} either
\begin{itemize}
\item<2-> Calls the user custom uncaught exception handler, if set with \mintinline{ocaml}{Printexc.set_uncaught_exception_handler fn}\footnotemark
\item<2-> Or falls back to the default one
\item<2-> Which notifies about the uncaught exception
\end{itemize}
\item<4-> Terminates the program either with \funcname{abort} or with a non-zero exit code
\end{itemize}
\end{column}
\begin{column}{0.55\textwidth}
\only<1>{
\listc[firstline=84,lastline=89,highlightlines=88]{../ocaml/runtime/caml/mlvalues.h}{runtime/caml/mlvalues.h:84}
\listc[firstline=138,lastline=143,highlightlines={141-142}]{../ocaml/runtime/startup\_nat.c}{runtime/startup\_nat.c:138}
}%
\only<2>{
\listc[firstline=138,highlightlines={147,149}]{../ocaml/runtime/printexc.c}{runtime/printexc.c:138}
}%
\only<3>{
\listc[firstline=110,lastline=134,highlightlines=129]{../ocaml/runtime/printexc.c}{runtime/printexc.c:138}
}%
\only<4>{
\listc[firstline=138,highlightlines={152,154}]{../ocaml/runtime/printexc.c}{runtime/printexc.c:138}
}%
\end{column}
\end{columns}
\footnotetext[1]{\url{https://v2.ocaml.org/api/Printexc.html\#1\_Uncaughtexceptions}}
\end{frame}
}%
\end{column}
\end{columns}
\end{frame}

\begin{frame}{Executing the OCaml program}{And raising an exception without handler}
\begin{columns}
\begin{column}{0.6\textwidth}
\only<1-3>{
\begin{itemize}
\item<1-> Execution continues with OCaml code
\begin{itemize}
\item \funcname{camlDefault\_\_main\_269}
\item \funcname{camlDefault\_\_entry}
\item \funcname{caml\_program}
\end{itemize}
\item<1-> \funcname{camlDefault\_\_main\_269} raises a \typename{Failure} exception
\item<1-> \typename{Failure} is caught by \funcname{caml\_start\_program}
\begin{itemize}
\item<1-> The handler set the 2nd LSB of the exception value to \funcarg{1}
\item<3-> And then forces \funcname{caml\_start\_program} to terminate
\end{itemize}
\end{itemize}
}
\bigskip
\only<1,3>{\listocaml[highlightlines=3]{../src/default/default.ml}{default/default.ml}}%
\only<2>{\listasm[firstline=66,lastline=72,highlightlines=69]{src/ocaml/caml\_start\_program.s}{caml\_start\_program}}%
\end{column}
\begin{column}{0.4\textwidth}
\centering
\providecommand\step{4}%
% Runtime's default exception handler
%
\subsection*{Runtime's default exception handler}
\frameSubsection{pictures/The\_Architect.png}{}


\begin{frame}{Default exception handler}
\begin{columns}
\begin{column}{0.5\textwidth}
\begin{itemize}
\item \localname{domain\_state->exn\_handler} points to a trap located before \funcname{entry}!
\item What installed this very first trap there?
\end{itemize}
\bigskip
\listocaml{../src/default/default.ml}{default/default.ml}
\bigskip
\inputminted[fontsize=\footnotesize]{text}{../src/default/default.out}
\end{column}
\begin{column}{0.5\textwidth}
\centering
\only<1>{\providecommand\step{0}\input{diagrams/default/default.tex}}%
\only<2>{\providecommand\step{4}\input{diagrams/default/default.tex}}%
\end{column}
\end{columns}
\end{frame}


\begin{frame}{Startup}{How did we get there by the way?}
\begin{columns}
\begin{column}{0.45\textwidth}
\begin{itemize}
\item The entry point address of the binary is \funcname{main} (\funcname{\_start}) from the runtime
\item The program starts like a C program:
\begin{itemize}
\item \funcname{caml\_start\_program}
\item \funcname{caml\_startup\_common}
\item \funcname{caml\_startup\_exn}
\item \funcname{caml\_startup}
\item \funcname{caml\_main}
\item \funcname{main}
\end{itemize}
%\item \funcname{caml\_startup\_common}: initializes GC, domains \dots
% Also: codefrag, locale, custom opeartions, frame_descr, signals, stack
\item \funcname{caml\_start\_program} is the transition point between the C realm and the OCaml one
\end{itemize}
\bigskip
\listocaml{../src/default/default.ml}{default/default.ml}
\end{column}
\begin{column}{0.55\textwidth}
\listasm[lastline=24,highlightlines={22-24}]{src/ocaml/caml\_start\_program.s}{runtime/amd64.S:604}
\end{column}
\end{columns}
\end{frame}


\begin{frame}{Setup to jump into OCaml entry point}{\funcname{caml\_start\_program} initializes the main fiber}
\begin{columns}
\begin{column}{0.6\textwidth}
\only<1,3,5>{
\begin{itemize}
\item<1-> Stores information to allow switching from OCaml stack to the C stack
\item<3-> Constructs the default exception handler
\item<5-> Switches to the OCaml stack and calls \funcname{caml\_program}
\end{itemize}
\bigskip
\listocaml{../src/default/default.ml}{default/default.ml}
}%
\only<2>{
\listasm[firstline=22,lastline=41,highlightlines={25-27}]{src/ocaml/caml\_start\_program.s}{caml\_start\_program}
}%
\only<4>{
\mintasm[firstline=22,lastline=41,highlightlines={31-38}]{src/ocaml/caml\_start\_program.s}
\listasm[firstline=66,lastline=72]{src/ocaml/caml\_start\_program.s}{caml\_start\_program}
}%
\only<6>{
\listasm[firstline=22,lastline=41,highlightlines={39-41}]{src/ocaml/caml\_start\_program.s}{caml\_start\_program}
}%
\end{column}
\begin{column}{0.4\textwidth}
\centering
\only<1-2>{\providecommand\step{2}\input{diagrams/default/default.tex}}%
\only<3-5>{\providecommand\step{3}\input{diagrams/default/default.tex}}%
\onslide*<6->{\providecommand\step{4}\input{diagrams/default/default.tex}}%
\end{column}
\end{columns}
\end{frame}

\begin{frame}{Executing the OCaml program}{And raising an exception without handler}
\begin{columns}
\begin{column}{0.6\textwidth}
\only<1-3>{
\begin{itemize}
\item<1-> Execution continues with OCaml code
\begin{itemize}
\item \funcname{camlDefault\_\_main\_269}
\item \funcname{camlDefault\_\_entry}
\item \funcname{caml\_program}
\end{itemize}
\item<1-> \funcname{camlDefault\_\_main\_269} raises a \typename{Failure} exception
\item<1-> \typename{Failure} is caught by \funcname{caml\_start\_program}
\begin{itemize}
\item<1-> The handler set the 2nd LSB of the exception value to \funcarg{1}
\item<3-> And then forces \funcname{caml\_start\_program} to terminate
\end{itemize}
\end{itemize}
}
\bigskip
\only<1,3>{\listocaml[highlightlines=3]{../src/default/default.ml}{default/default.ml}}%
\only<2>{\listasm[firstline=66,lastline=72,highlightlines=69]{src/ocaml/caml\_start\_program.s}{caml\_start\_program}}%
\end{column}
\begin{column}{0.4\textwidth}
\centering
\providecommand\step{4}\input{diagrams/default/default.tex}
\end{column}
\end{columns}
\end{frame}
%\only<>{\listasm[firstline=47,lastline=65]{src/ocaml/caml\_start\_program.s}{caml\_start\_program}}


\begin{frame}{Back to C}{Clean up, complain and terminate}
\begin{columns}
\begin{column}{0.45\textwidth}
\begin{itemize}
\item Backtrace:
\begin{itemize}
\item \funcname{caml\_start\_program} $\rightarrow$ OCaml code
\item \funcname{caml\_startup\_common}
\item \funcname{caml\_startup\_exn}
\item \funcname{caml\_startup}
\item \funcname{caml\_main}
\item \funcname{main}
\end{itemize}
\smallskip
\item \funcname{caml\_startup} checks the return value of \funcname{caml\_startup\_exn}
\begin{itemize}
\item The return value has been specially marked by \funcname{caml\_start\_program} handler
\item Calls \funcname{caml\_fatal\_uncaught\_exception}
\end{itemize}
\item<2-> \funcname{caml\_fatal\_uncaught\_exception} either
\begin{itemize}
\item<2-> Calls the user custom uncaught exception handler, if set with \mintinline{ocaml}{Printexc.set_uncaught_exception_handler fn}\footnotemark
\item<2-> Or falls back to the default one
\item<2-> Which notifies about the uncaught exception
\end{itemize}
\item<4-> Terminates the program either with \funcname{abort} or with a non-zero exit code
\end{itemize}
\end{column}
\begin{column}{0.55\textwidth}
\only<1>{
\listc[firstline=84,lastline=89,highlightlines=88]{../ocaml/runtime/caml/mlvalues.h}{runtime/caml/mlvalues.h:84}
\listc[firstline=138,lastline=143,highlightlines={141-142}]{../ocaml/runtime/startup\_nat.c}{runtime/startup\_nat.c:138}
}%
\only<2>{
\listc[firstline=138,highlightlines={147,149}]{../ocaml/runtime/printexc.c}{runtime/printexc.c:138}
}%
\only<3>{
\listc[firstline=110,lastline=134,highlightlines=129]{../ocaml/runtime/printexc.c}{runtime/printexc.c:138}
}%
\only<4>{
\listc[firstline=138,highlightlines={152,154}]{../ocaml/runtime/printexc.c}{runtime/printexc.c:138}
}%
\end{column}
\end{columns}
\footnotetext[1]{\url{https://v2.ocaml.org/api/Printexc.html\#1\_Uncaughtexceptions}}
\end{frame}

\end{column}
\end{columns}
\end{frame}
%\only<>{\listasm[firstline=47,lastline=65]{src/ocaml/caml\_start\_program.s}{caml\_start\_program}}


\begin{frame}{Back to C}{Clean up, complain and terminate}
\begin{columns}
\begin{column}{0.45\textwidth}
\begin{itemize}
\item Backtrace:
\begin{itemize}
\item \funcname{caml\_start\_program} $\rightarrow$ OCaml code
\item \funcname{caml\_startup\_common}
\item \funcname{caml\_startup\_exn}
\item \funcname{caml\_startup}
\item \funcname{caml\_main}
\item \funcname{main}
\end{itemize}
\smallskip
\item \funcname{caml\_startup} checks the return value of \funcname{caml\_startup\_exn}
\begin{itemize}
\item The return value has been specially marked by \funcname{caml\_start\_program} handler
\item Calls \funcname{caml\_fatal\_uncaught\_exception}
\end{itemize}
\item<2-> \funcname{caml\_fatal\_uncaught\_exception} either
\begin{itemize}
\item<2-> Calls the user custom uncaught exception handler, if set with \mintinline{ocaml}{Printexc.set_uncaught_exception_handler fn}\footnotemark
\item<2-> Or falls back to the default one
\item<2-> Which notifies about the uncaught exception
\end{itemize}
\item<4-> Terminates the program either with \funcname{abort} or with a non-zero exit code
\end{itemize}
\end{column}
\begin{column}{0.55\textwidth}
\only<1>{
\listc[firstline=84,lastline=89,highlightlines=88]{../ocaml/runtime/caml/mlvalues.h}{runtime/caml/mlvalues.h:84}
\listc[firstline=138,lastline=143,highlightlines={141-142}]{../ocaml/runtime/startup\_nat.c}{runtime/startup\_nat.c:138}
}%
\only<2>{
\listc[firstline=138,highlightlines={147,149}]{../ocaml/runtime/printexc.c}{runtime/printexc.c:138}
}%
\only<3>{
\listc[firstline=110,lastline=134,highlightlines=129]{../ocaml/runtime/printexc.c}{runtime/printexc.c:138}
}%
\only<4>{
\listc[firstline=138,highlightlines={152,154}]{../ocaml/runtime/printexc.c}{runtime/printexc.c:138}
}%
\end{column}
\end{columns}
\footnotetext[1]{\url{https://v2.ocaml.org/api/Printexc.html\#1\_Uncaughtexceptions}}
\end{frame}
}%
\end{column}
\end{columns}
\end{frame}


\begin{frame}{Startup}{How did we get there by the way?}
\begin{columns}
\begin{column}{0.45\textwidth}
\begin{itemize}
\item The entry point address of the binary is \funcname{main} (\funcname{\_start}) from the runtime
\item The program starts like a C program:
\begin{itemize}
\item \funcname{caml\_start\_program}
\item \funcname{caml\_startup\_common}
\item \funcname{caml\_startup\_exn}
\item \funcname{caml\_startup}
\item \funcname{caml\_main}
\item \funcname{main}
\end{itemize}
%\item \funcname{caml\_startup\_common}: initializes GC, domains \dots
% Also: codefrag, locale, custom opeartions, frame_descr, signals, stack
\item \funcname{caml\_start\_program} is the transition point between the C realm and the OCaml one
\end{itemize}
\bigskip
\listocaml{../src/default/default.ml}{default/default.ml}
\end{column}
\begin{column}{0.55\textwidth}
\listasm[lastline=24,highlightlines={22-24}]{src/ocaml/caml\_start\_program.s}{runtime/amd64.S:604}
\end{column}
\end{columns}
\end{frame}


\begin{frame}{Setup to jump into OCaml entry point}{\funcname{caml\_start\_program} initializes the main fiber}
\begin{columns}
\begin{column}{0.6\textwidth}
\only<1,3,5>{
\begin{itemize}
\item<1-> Stores information to allow switching from OCaml stack to the C stack
\item<3-> Constructs the default exception handler
\item<5-> Switches to the OCaml stack and calls \funcname{caml\_program}
\end{itemize}
\bigskip
\listocaml{../src/default/default.ml}{default/default.ml}
}%
\only<2>{
\listasm[firstline=22,lastline=41,highlightlines={25-27}]{src/ocaml/caml\_start\_program.s}{caml\_start\_program}
}%
\only<4>{
\mintasm[firstline=22,lastline=41,highlightlines={31-38}]{src/ocaml/caml\_start\_program.s}
\listasm[firstline=66,lastline=72]{src/ocaml/caml\_start\_program.s}{caml\_start\_program}
}%
\only<6>{
\listasm[firstline=22,lastline=41,highlightlines={39-41}]{src/ocaml/caml\_start\_program.s}{caml\_start\_program}
}%
\end{column}
\begin{column}{0.4\textwidth}
\centering
\only<1-2>{\providecommand\step{2}%
% Runtime's default exception handler
%
\subsection*{Runtime's default exception handler}
\frameSubsection{pictures/The\_Architect.png}{}


\begin{frame}{Default exception handler}
\begin{columns}
\begin{column}{0.5\textwidth}
\begin{itemize}
\item \localname{domain\_state->exn\_handler} points to a trap located before \funcname{entry}!
\item What installed this very first trap there?
\end{itemize}
\bigskip
\listocaml{../src/default/default.ml}{default/default.ml}
\bigskip
\inputminted[fontsize=\footnotesize]{text}{../src/default/default.out}
\end{column}
\begin{column}{0.5\textwidth}
\centering
\only<1>{\providecommand\step{0}%
% Runtime's default exception handler
%
\subsection*{Runtime's default exception handler}
\frameSubsection{pictures/The\_Architect.png}{}


\begin{frame}{Default exception handler}
\begin{columns}
\begin{column}{0.5\textwidth}
\begin{itemize}
\item \localname{domain\_state->exn\_handler} points to a trap located before \funcname{entry}!
\item What installed this very first trap there?
\end{itemize}
\bigskip
\listocaml{../src/default/default.ml}{default/default.ml}
\bigskip
\inputminted[fontsize=\footnotesize]{text}{../src/default/default.out}
\end{column}
\begin{column}{0.5\textwidth}
\centering
\only<1>{\providecommand\step{0}\input{diagrams/default/default.tex}}%
\only<2>{\providecommand\step{4}\input{diagrams/default/default.tex}}%
\end{column}
\end{columns}
\end{frame}


\begin{frame}{Startup}{How did we get there by the way?}
\begin{columns}
\begin{column}{0.45\textwidth}
\begin{itemize}
\item The entry point address of the binary is \funcname{main} (\funcname{\_start}) from the runtime
\item The program starts like a C program:
\begin{itemize}
\item \funcname{caml\_start\_program}
\item \funcname{caml\_startup\_common}
\item \funcname{caml\_startup\_exn}
\item \funcname{caml\_startup}
\item \funcname{caml\_main}
\item \funcname{main}
\end{itemize}
%\item \funcname{caml\_startup\_common}: initializes GC, domains \dots
% Also: codefrag, locale, custom opeartions, frame_descr, signals, stack
\item \funcname{caml\_start\_program} is the transition point between the C realm and the OCaml one
\end{itemize}
\bigskip
\listocaml{../src/default/default.ml}{default/default.ml}
\end{column}
\begin{column}{0.55\textwidth}
\listasm[lastline=24,highlightlines={22-24}]{src/ocaml/caml\_start\_program.s}{runtime/amd64.S:604}
\end{column}
\end{columns}
\end{frame}


\begin{frame}{Setup to jump into OCaml entry point}{\funcname{caml\_start\_program} initializes the main fiber}
\begin{columns}
\begin{column}{0.6\textwidth}
\only<1,3,5>{
\begin{itemize}
\item<1-> Stores information to allow switching from OCaml stack to the C stack
\item<3-> Constructs the default exception handler
\item<5-> Switches to the OCaml stack and calls \funcname{caml\_program}
\end{itemize}
\bigskip
\listocaml{../src/default/default.ml}{default/default.ml}
}%
\only<2>{
\listasm[firstline=22,lastline=41,highlightlines={25-27}]{src/ocaml/caml\_start\_program.s}{caml\_start\_program}
}%
\only<4>{
\mintasm[firstline=22,lastline=41,highlightlines={31-38}]{src/ocaml/caml\_start\_program.s}
\listasm[firstline=66,lastline=72]{src/ocaml/caml\_start\_program.s}{caml\_start\_program}
}%
\only<6>{
\listasm[firstline=22,lastline=41,highlightlines={39-41}]{src/ocaml/caml\_start\_program.s}{caml\_start\_program}
}%
\end{column}
\begin{column}{0.4\textwidth}
\centering
\only<1-2>{\providecommand\step{2}\input{diagrams/default/default.tex}}%
\only<3-5>{\providecommand\step{3}\input{diagrams/default/default.tex}}%
\onslide*<6->{\providecommand\step{4}\input{diagrams/default/default.tex}}%
\end{column}
\end{columns}
\end{frame}

\begin{frame}{Executing the OCaml program}{And raising an exception without handler}
\begin{columns}
\begin{column}{0.6\textwidth}
\only<1-3>{
\begin{itemize}
\item<1-> Execution continues with OCaml code
\begin{itemize}
\item \funcname{camlDefault\_\_main\_269}
\item \funcname{camlDefault\_\_entry}
\item \funcname{caml\_program}
\end{itemize}
\item<1-> \funcname{camlDefault\_\_main\_269} raises a \typename{Failure} exception
\item<1-> \typename{Failure} is caught by \funcname{caml\_start\_program}
\begin{itemize}
\item<1-> The handler set the 2nd LSB of the exception value to \funcarg{1}
\item<3-> And then forces \funcname{caml\_start\_program} to terminate
\end{itemize}
\end{itemize}
}
\bigskip
\only<1,3>{\listocaml[highlightlines=3]{../src/default/default.ml}{default/default.ml}}%
\only<2>{\listasm[firstline=66,lastline=72,highlightlines=69]{src/ocaml/caml\_start\_program.s}{caml\_start\_program}}%
\end{column}
\begin{column}{0.4\textwidth}
\centering
\providecommand\step{4}\input{diagrams/default/default.tex}
\end{column}
\end{columns}
\end{frame}
%\only<>{\listasm[firstline=47,lastline=65]{src/ocaml/caml\_start\_program.s}{caml\_start\_program}}


\begin{frame}{Back to C}{Clean up, complain and terminate}
\begin{columns}
\begin{column}{0.45\textwidth}
\begin{itemize}
\item Backtrace:
\begin{itemize}
\item \funcname{caml\_start\_program} $\rightarrow$ OCaml code
\item \funcname{caml\_startup\_common}
\item \funcname{caml\_startup\_exn}
\item \funcname{caml\_startup}
\item \funcname{caml\_main}
\item \funcname{main}
\end{itemize}
\smallskip
\item \funcname{caml\_startup} checks the return value of \funcname{caml\_startup\_exn}
\begin{itemize}
\item The return value has been specially marked by \funcname{caml\_start\_program} handler
\item Calls \funcname{caml\_fatal\_uncaught\_exception}
\end{itemize}
\item<2-> \funcname{caml\_fatal\_uncaught\_exception} either
\begin{itemize}
\item<2-> Calls the user custom uncaught exception handler, if set with \mintinline{ocaml}{Printexc.set_uncaught_exception_handler fn}\footnotemark
\item<2-> Or falls back to the default one
\item<2-> Which notifies about the uncaught exception
\end{itemize}
\item<4-> Terminates the program either with \funcname{abort} or with a non-zero exit code
\end{itemize}
\end{column}
\begin{column}{0.55\textwidth}
\only<1>{
\listc[firstline=84,lastline=89,highlightlines=88]{../ocaml/runtime/caml/mlvalues.h}{runtime/caml/mlvalues.h:84}
\listc[firstline=138,lastline=143,highlightlines={141-142}]{../ocaml/runtime/startup\_nat.c}{runtime/startup\_nat.c:138}
}%
\only<2>{
\listc[firstline=138,highlightlines={147,149}]{../ocaml/runtime/printexc.c}{runtime/printexc.c:138}
}%
\only<3>{
\listc[firstline=110,lastline=134,highlightlines=129]{../ocaml/runtime/printexc.c}{runtime/printexc.c:138}
}%
\only<4>{
\listc[firstline=138,highlightlines={152,154}]{../ocaml/runtime/printexc.c}{runtime/printexc.c:138}
}%
\end{column}
\end{columns}
\footnotetext[1]{\url{https://v2.ocaml.org/api/Printexc.html\#1\_Uncaughtexceptions}}
\end{frame}
}%
\only<2>{\providecommand\step{4}%
% Runtime's default exception handler
%
\subsection*{Runtime's default exception handler}
\frameSubsection{pictures/The\_Architect.png}{}


\begin{frame}{Default exception handler}
\begin{columns}
\begin{column}{0.5\textwidth}
\begin{itemize}
\item \localname{domain\_state->exn\_handler} points to a trap located before \funcname{entry}!
\item What installed this very first trap there?
\end{itemize}
\bigskip
\listocaml{../src/default/default.ml}{default/default.ml}
\bigskip
\inputminted[fontsize=\footnotesize]{text}{../src/default/default.out}
\end{column}
\begin{column}{0.5\textwidth}
\centering
\only<1>{\providecommand\step{0}\input{diagrams/default/default.tex}}%
\only<2>{\providecommand\step{4}\input{diagrams/default/default.tex}}%
\end{column}
\end{columns}
\end{frame}


\begin{frame}{Startup}{How did we get there by the way?}
\begin{columns}
\begin{column}{0.45\textwidth}
\begin{itemize}
\item The entry point address of the binary is \funcname{main} (\funcname{\_start}) from the runtime
\item The program starts like a C program:
\begin{itemize}
\item \funcname{caml\_start\_program}
\item \funcname{caml\_startup\_common}
\item \funcname{caml\_startup\_exn}
\item \funcname{caml\_startup}
\item \funcname{caml\_main}
\item \funcname{main}
\end{itemize}
%\item \funcname{caml\_startup\_common}: initializes GC, domains \dots
% Also: codefrag, locale, custom opeartions, frame_descr, signals, stack
\item \funcname{caml\_start\_program} is the transition point between the C realm and the OCaml one
\end{itemize}
\bigskip
\listocaml{../src/default/default.ml}{default/default.ml}
\end{column}
\begin{column}{0.55\textwidth}
\listasm[lastline=24,highlightlines={22-24}]{src/ocaml/caml\_start\_program.s}{runtime/amd64.S:604}
\end{column}
\end{columns}
\end{frame}


\begin{frame}{Setup to jump into OCaml entry point}{\funcname{caml\_start\_program} initializes the main fiber}
\begin{columns}
\begin{column}{0.6\textwidth}
\only<1,3,5>{
\begin{itemize}
\item<1-> Stores information to allow switching from OCaml stack to the C stack
\item<3-> Constructs the default exception handler
\item<5-> Switches to the OCaml stack and calls \funcname{caml\_program}
\end{itemize}
\bigskip
\listocaml{../src/default/default.ml}{default/default.ml}
}%
\only<2>{
\listasm[firstline=22,lastline=41,highlightlines={25-27}]{src/ocaml/caml\_start\_program.s}{caml\_start\_program}
}%
\only<4>{
\mintasm[firstline=22,lastline=41,highlightlines={31-38}]{src/ocaml/caml\_start\_program.s}
\listasm[firstline=66,lastline=72]{src/ocaml/caml\_start\_program.s}{caml\_start\_program}
}%
\only<6>{
\listasm[firstline=22,lastline=41,highlightlines={39-41}]{src/ocaml/caml\_start\_program.s}{caml\_start\_program}
}%
\end{column}
\begin{column}{0.4\textwidth}
\centering
\only<1-2>{\providecommand\step{2}\input{diagrams/default/default.tex}}%
\only<3-5>{\providecommand\step{3}\input{diagrams/default/default.tex}}%
\onslide*<6->{\providecommand\step{4}\input{diagrams/default/default.tex}}%
\end{column}
\end{columns}
\end{frame}

\begin{frame}{Executing the OCaml program}{And raising an exception without handler}
\begin{columns}
\begin{column}{0.6\textwidth}
\only<1-3>{
\begin{itemize}
\item<1-> Execution continues with OCaml code
\begin{itemize}
\item \funcname{camlDefault\_\_main\_269}
\item \funcname{camlDefault\_\_entry}
\item \funcname{caml\_program}
\end{itemize}
\item<1-> \funcname{camlDefault\_\_main\_269} raises a \typename{Failure} exception
\item<1-> \typename{Failure} is caught by \funcname{caml\_start\_program}
\begin{itemize}
\item<1-> The handler set the 2nd LSB of the exception value to \funcarg{1}
\item<3-> And then forces \funcname{caml\_start\_program} to terminate
\end{itemize}
\end{itemize}
}
\bigskip
\only<1,3>{\listocaml[highlightlines=3]{../src/default/default.ml}{default/default.ml}}%
\only<2>{\listasm[firstline=66,lastline=72,highlightlines=69]{src/ocaml/caml\_start\_program.s}{caml\_start\_program}}%
\end{column}
\begin{column}{0.4\textwidth}
\centering
\providecommand\step{4}\input{diagrams/default/default.tex}
\end{column}
\end{columns}
\end{frame}
%\only<>{\listasm[firstline=47,lastline=65]{src/ocaml/caml\_start\_program.s}{caml\_start\_program}}


\begin{frame}{Back to C}{Clean up, complain and terminate}
\begin{columns}
\begin{column}{0.45\textwidth}
\begin{itemize}
\item Backtrace:
\begin{itemize}
\item \funcname{caml\_start\_program} $\rightarrow$ OCaml code
\item \funcname{caml\_startup\_common}
\item \funcname{caml\_startup\_exn}
\item \funcname{caml\_startup}
\item \funcname{caml\_main}
\item \funcname{main}
\end{itemize}
\smallskip
\item \funcname{caml\_startup} checks the return value of \funcname{caml\_startup\_exn}
\begin{itemize}
\item The return value has been specially marked by \funcname{caml\_start\_program} handler
\item Calls \funcname{caml\_fatal\_uncaught\_exception}
\end{itemize}
\item<2-> \funcname{caml\_fatal\_uncaught\_exception} either
\begin{itemize}
\item<2-> Calls the user custom uncaught exception handler, if set with \mintinline{ocaml}{Printexc.set_uncaught_exception_handler fn}\footnotemark
\item<2-> Or falls back to the default one
\item<2-> Which notifies about the uncaught exception
\end{itemize}
\item<4-> Terminates the program either with \funcname{abort} or with a non-zero exit code
\end{itemize}
\end{column}
\begin{column}{0.55\textwidth}
\only<1>{
\listc[firstline=84,lastline=89,highlightlines=88]{../ocaml/runtime/caml/mlvalues.h}{runtime/caml/mlvalues.h:84}
\listc[firstline=138,lastline=143,highlightlines={141-142}]{../ocaml/runtime/startup\_nat.c}{runtime/startup\_nat.c:138}
}%
\only<2>{
\listc[firstline=138,highlightlines={147,149}]{../ocaml/runtime/printexc.c}{runtime/printexc.c:138}
}%
\only<3>{
\listc[firstline=110,lastline=134,highlightlines=129]{../ocaml/runtime/printexc.c}{runtime/printexc.c:138}
}%
\only<4>{
\listc[firstline=138,highlightlines={152,154}]{../ocaml/runtime/printexc.c}{runtime/printexc.c:138}
}%
\end{column}
\end{columns}
\footnotetext[1]{\url{https://v2.ocaml.org/api/Printexc.html\#1\_Uncaughtexceptions}}
\end{frame}
}%
\end{column}
\end{columns}
\end{frame}


\begin{frame}{Startup}{How did we get there by the way?}
\begin{columns}
\begin{column}{0.45\textwidth}
\begin{itemize}
\item The entry point address of the binary is \funcname{main} (\funcname{\_start}) from the runtime
\item The program starts like a C program:
\begin{itemize}
\item \funcname{caml\_start\_program}
\item \funcname{caml\_startup\_common}
\item \funcname{caml\_startup\_exn}
\item \funcname{caml\_startup}
\item \funcname{caml\_main}
\item \funcname{main}
\end{itemize}
%\item \funcname{caml\_startup\_common}: initializes GC, domains \dots
% Also: codefrag, locale, custom opeartions, frame_descr, signals, stack
\item \funcname{caml\_start\_program} is the transition point between the C realm and the OCaml one
\end{itemize}
\bigskip
\listocaml{../src/default/default.ml}{default/default.ml}
\end{column}
\begin{column}{0.55\textwidth}
\listasm[lastline=24,highlightlines={22-24}]{src/ocaml/caml\_start\_program.s}{runtime/amd64.S:604}
\end{column}
\end{columns}
\end{frame}


\begin{frame}{Setup to jump into OCaml entry point}{\funcname{caml\_start\_program} initializes the main fiber}
\begin{columns}
\begin{column}{0.6\textwidth}
\only<1,3,5>{
\begin{itemize}
\item<1-> Stores information to allow switching from OCaml stack to the C stack
\item<3-> Constructs the default exception handler
\item<5-> Switches to the OCaml stack and calls \funcname{caml\_program}
\end{itemize}
\bigskip
\listocaml{../src/default/default.ml}{default/default.ml}
}%
\only<2>{
\listasm[firstline=22,lastline=41,highlightlines={25-27}]{src/ocaml/caml\_start\_program.s}{caml\_start\_program}
}%
\only<4>{
\mintasm[firstline=22,lastline=41,highlightlines={31-38}]{src/ocaml/caml\_start\_program.s}
\listasm[firstline=66,lastline=72]{src/ocaml/caml\_start\_program.s}{caml\_start\_program}
}%
\only<6>{
\listasm[firstline=22,lastline=41,highlightlines={39-41}]{src/ocaml/caml\_start\_program.s}{caml\_start\_program}
}%
\end{column}
\begin{column}{0.4\textwidth}
\centering
\only<1-2>{\providecommand\step{2}%
% Runtime's default exception handler
%
\subsection*{Runtime's default exception handler}
\frameSubsection{pictures/The\_Architect.png}{}


\begin{frame}{Default exception handler}
\begin{columns}
\begin{column}{0.5\textwidth}
\begin{itemize}
\item \localname{domain\_state->exn\_handler} points to a trap located before \funcname{entry}!
\item What installed this very first trap there?
\end{itemize}
\bigskip
\listocaml{../src/default/default.ml}{default/default.ml}
\bigskip
\inputminted[fontsize=\footnotesize]{text}{../src/default/default.out}
\end{column}
\begin{column}{0.5\textwidth}
\centering
\only<1>{\providecommand\step{0}\input{diagrams/default/default.tex}}%
\only<2>{\providecommand\step{4}\input{diagrams/default/default.tex}}%
\end{column}
\end{columns}
\end{frame}


\begin{frame}{Startup}{How did we get there by the way?}
\begin{columns}
\begin{column}{0.45\textwidth}
\begin{itemize}
\item The entry point address of the binary is \funcname{main} (\funcname{\_start}) from the runtime
\item The program starts like a C program:
\begin{itemize}
\item \funcname{caml\_start\_program}
\item \funcname{caml\_startup\_common}
\item \funcname{caml\_startup\_exn}
\item \funcname{caml\_startup}
\item \funcname{caml\_main}
\item \funcname{main}
\end{itemize}
%\item \funcname{caml\_startup\_common}: initializes GC, domains \dots
% Also: codefrag, locale, custom opeartions, frame_descr, signals, stack
\item \funcname{caml\_start\_program} is the transition point between the C realm and the OCaml one
\end{itemize}
\bigskip
\listocaml{../src/default/default.ml}{default/default.ml}
\end{column}
\begin{column}{0.55\textwidth}
\listasm[lastline=24,highlightlines={22-24}]{src/ocaml/caml\_start\_program.s}{runtime/amd64.S:604}
\end{column}
\end{columns}
\end{frame}


\begin{frame}{Setup to jump into OCaml entry point}{\funcname{caml\_start\_program} initializes the main fiber}
\begin{columns}
\begin{column}{0.6\textwidth}
\only<1,3,5>{
\begin{itemize}
\item<1-> Stores information to allow switching from OCaml stack to the C stack
\item<3-> Constructs the default exception handler
\item<5-> Switches to the OCaml stack and calls \funcname{caml\_program}
\end{itemize}
\bigskip
\listocaml{../src/default/default.ml}{default/default.ml}
}%
\only<2>{
\listasm[firstline=22,lastline=41,highlightlines={25-27}]{src/ocaml/caml\_start\_program.s}{caml\_start\_program}
}%
\only<4>{
\mintasm[firstline=22,lastline=41,highlightlines={31-38}]{src/ocaml/caml\_start\_program.s}
\listasm[firstline=66,lastline=72]{src/ocaml/caml\_start\_program.s}{caml\_start\_program}
}%
\only<6>{
\listasm[firstline=22,lastline=41,highlightlines={39-41}]{src/ocaml/caml\_start\_program.s}{caml\_start\_program}
}%
\end{column}
\begin{column}{0.4\textwidth}
\centering
\only<1-2>{\providecommand\step{2}\input{diagrams/default/default.tex}}%
\only<3-5>{\providecommand\step{3}\input{diagrams/default/default.tex}}%
\onslide*<6->{\providecommand\step{4}\input{diagrams/default/default.tex}}%
\end{column}
\end{columns}
\end{frame}

\begin{frame}{Executing the OCaml program}{And raising an exception without handler}
\begin{columns}
\begin{column}{0.6\textwidth}
\only<1-3>{
\begin{itemize}
\item<1-> Execution continues with OCaml code
\begin{itemize}
\item \funcname{camlDefault\_\_main\_269}
\item \funcname{camlDefault\_\_entry}
\item \funcname{caml\_program}
\end{itemize}
\item<1-> \funcname{camlDefault\_\_main\_269} raises a \typename{Failure} exception
\item<1-> \typename{Failure} is caught by \funcname{caml\_start\_program}
\begin{itemize}
\item<1-> The handler set the 2nd LSB of the exception value to \funcarg{1}
\item<3-> And then forces \funcname{caml\_start\_program} to terminate
\end{itemize}
\end{itemize}
}
\bigskip
\only<1,3>{\listocaml[highlightlines=3]{../src/default/default.ml}{default/default.ml}}%
\only<2>{\listasm[firstline=66,lastline=72,highlightlines=69]{src/ocaml/caml\_start\_program.s}{caml\_start\_program}}%
\end{column}
\begin{column}{0.4\textwidth}
\centering
\providecommand\step{4}\input{diagrams/default/default.tex}
\end{column}
\end{columns}
\end{frame}
%\only<>{\listasm[firstline=47,lastline=65]{src/ocaml/caml\_start\_program.s}{caml\_start\_program}}


\begin{frame}{Back to C}{Clean up, complain and terminate}
\begin{columns}
\begin{column}{0.45\textwidth}
\begin{itemize}
\item Backtrace:
\begin{itemize}
\item \funcname{caml\_start\_program} $\rightarrow$ OCaml code
\item \funcname{caml\_startup\_common}
\item \funcname{caml\_startup\_exn}
\item \funcname{caml\_startup}
\item \funcname{caml\_main}
\item \funcname{main}
\end{itemize}
\smallskip
\item \funcname{caml\_startup} checks the return value of \funcname{caml\_startup\_exn}
\begin{itemize}
\item The return value has been specially marked by \funcname{caml\_start\_program} handler
\item Calls \funcname{caml\_fatal\_uncaught\_exception}
\end{itemize}
\item<2-> \funcname{caml\_fatal\_uncaught\_exception} either
\begin{itemize}
\item<2-> Calls the user custom uncaught exception handler, if set with \mintinline{ocaml}{Printexc.set_uncaught_exception_handler fn}\footnotemark
\item<2-> Or falls back to the default one
\item<2-> Which notifies about the uncaught exception
\end{itemize}
\item<4-> Terminates the program either with \funcname{abort} or with a non-zero exit code
\end{itemize}
\end{column}
\begin{column}{0.55\textwidth}
\only<1>{
\listc[firstline=84,lastline=89,highlightlines=88]{../ocaml/runtime/caml/mlvalues.h}{runtime/caml/mlvalues.h:84}
\listc[firstline=138,lastline=143,highlightlines={141-142}]{../ocaml/runtime/startup\_nat.c}{runtime/startup\_nat.c:138}
}%
\only<2>{
\listc[firstline=138,highlightlines={147,149}]{../ocaml/runtime/printexc.c}{runtime/printexc.c:138}
}%
\only<3>{
\listc[firstline=110,lastline=134,highlightlines=129]{../ocaml/runtime/printexc.c}{runtime/printexc.c:138}
}%
\only<4>{
\listc[firstline=138,highlightlines={152,154}]{../ocaml/runtime/printexc.c}{runtime/printexc.c:138}
}%
\end{column}
\end{columns}
\footnotetext[1]{\url{https://v2.ocaml.org/api/Printexc.html\#1\_Uncaughtexceptions}}
\end{frame}
}%
\only<3-5>{\providecommand\step{3}%
% Runtime's default exception handler
%
\subsection*{Runtime's default exception handler}
\frameSubsection{pictures/The\_Architect.png}{}


\begin{frame}{Default exception handler}
\begin{columns}
\begin{column}{0.5\textwidth}
\begin{itemize}
\item \localname{domain\_state->exn\_handler} points to a trap located before \funcname{entry}!
\item What installed this very first trap there?
\end{itemize}
\bigskip
\listocaml{../src/default/default.ml}{default/default.ml}
\bigskip
\inputminted[fontsize=\footnotesize]{text}{../src/default/default.out}
\end{column}
\begin{column}{0.5\textwidth}
\centering
\only<1>{\providecommand\step{0}\input{diagrams/default/default.tex}}%
\only<2>{\providecommand\step{4}\input{diagrams/default/default.tex}}%
\end{column}
\end{columns}
\end{frame}


\begin{frame}{Startup}{How did we get there by the way?}
\begin{columns}
\begin{column}{0.45\textwidth}
\begin{itemize}
\item The entry point address of the binary is \funcname{main} (\funcname{\_start}) from the runtime
\item The program starts like a C program:
\begin{itemize}
\item \funcname{caml\_start\_program}
\item \funcname{caml\_startup\_common}
\item \funcname{caml\_startup\_exn}
\item \funcname{caml\_startup}
\item \funcname{caml\_main}
\item \funcname{main}
\end{itemize}
%\item \funcname{caml\_startup\_common}: initializes GC, domains \dots
% Also: codefrag, locale, custom opeartions, frame_descr, signals, stack
\item \funcname{caml\_start\_program} is the transition point between the C realm and the OCaml one
\end{itemize}
\bigskip
\listocaml{../src/default/default.ml}{default/default.ml}
\end{column}
\begin{column}{0.55\textwidth}
\listasm[lastline=24,highlightlines={22-24}]{src/ocaml/caml\_start\_program.s}{runtime/amd64.S:604}
\end{column}
\end{columns}
\end{frame}


\begin{frame}{Setup to jump into OCaml entry point}{\funcname{caml\_start\_program} initializes the main fiber}
\begin{columns}
\begin{column}{0.6\textwidth}
\only<1,3,5>{
\begin{itemize}
\item<1-> Stores information to allow switching from OCaml stack to the C stack
\item<3-> Constructs the default exception handler
\item<5-> Switches to the OCaml stack and calls \funcname{caml\_program}
\end{itemize}
\bigskip
\listocaml{../src/default/default.ml}{default/default.ml}
}%
\only<2>{
\listasm[firstline=22,lastline=41,highlightlines={25-27}]{src/ocaml/caml\_start\_program.s}{caml\_start\_program}
}%
\only<4>{
\mintasm[firstline=22,lastline=41,highlightlines={31-38}]{src/ocaml/caml\_start\_program.s}
\listasm[firstline=66,lastline=72]{src/ocaml/caml\_start\_program.s}{caml\_start\_program}
}%
\only<6>{
\listasm[firstline=22,lastline=41,highlightlines={39-41}]{src/ocaml/caml\_start\_program.s}{caml\_start\_program}
}%
\end{column}
\begin{column}{0.4\textwidth}
\centering
\only<1-2>{\providecommand\step{2}\input{diagrams/default/default.tex}}%
\only<3-5>{\providecommand\step{3}\input{diagrams/default/default.tex}}%
\onslide*<6->{\providecommand\step{4}\input{diagrams/default/default.tex}}%
\end{column}
\end{columns}
\end{frame}

\begin{frame}{Executing the OCaml program}{And raising an exception without handler}
\begin{columns}
\begin{column}{0.6\textwidth}
\only<1-3>{
\begin{itemize}
\item<1-> Execution continues with OCaml code
\begin{itemize}
\item \funcname{camlDefault\_\_main\_269}
\item \funcname{camlDefault\_\_entry}
\item \funcname{caml\_program}
\end{itemize}
\item<1-> \funcname{camlDefault\_\_main\_269} raises a \typename{Failure} exception
\item<1-> \typename{Failure} is caught by \funcname{caml\_start\_program}
\begin{itemize}
\item<1-> The handler set the 2nd LSB of the exception value to \funcarg{1}
\item<3-> And then forces \funcname{caml\_start\_program} to terminate
\end{itemize}
\end{itemize}
}
\bigskip
\only<1,3>{\listocaml[highlightlines=3]{../src/default/default.ml}{default/default.ml}}%
\only<2>{\listasm[firstline=66,lastline=72,highlightlines=69]{src/ocaml/caml\_start\_program.s}{caml\_start\_program}}%
\end{column}
\begin{column}{0.4\textwidth}
\centering
\providecommand\step{4}\input{diagrams/default/default.tex}
\end{column}
\end{columns}
\end{frame}
%\only<>{\listasm[firstline=47,lastline=65]{src/ocaml/caml\_start\_program.s}{caml\_start\_program}}


\begin{frame}{Back to C}{Clean up, complain and terminate}
\begin{columns}
\begin{column}{0.45\textwidth}
\begin{itemize}
\item Backtrace:
\begin{itemize}
\item \funcname{caml\_start\_program} $\rightarrow$ OCaml code
\item \funcname{caml\_startup\_common}
\item \funcname{caml\_startup\_exn}
\item \funcname{caml\_startup}
\item \funcname{caml\_main}
\item \funcname{main}
\end{itemize}
\smallskip
\item \funcname{caml\_startup} checks the return value of \funcname{caml\_startup\_exn}
\begin{itemize}
\item The return value has been specially marked by \funcname{caml\_start\_program} handler
\item Calls \funcname{caml\_fatal\_uncaught\_exception}
\end{itemize}
\item<2-> \funcname{caml\_fatal\_uncaught\_exception} either
\begin{itemize}
\item<2-> Calls the user custom uncaught exception handler, if set with \mintinline{ocaml}{Printexc.set_uncaught_exception_handler fn}\footnotemark
\item<2-> Or falls back to the default one
\item<2-> Which notifies about the uncaught exception
\end{itemize}
\item<4-> Terminates the program either with \funcname{abort} or with a non-zero exit code
\end{itemize}
\end{column}
\begin{column}{0.55\textwidth}
\only<1>{
\listc[firstline=84,lastline=89,highlightlines=88]{../ocaml/runtime/caml/mlvalues.h}{runtime/caml/mlvalues.h:84}
\listc[firstline=138,lastline=143,highlightlines={141-142}]{../ocaml/runtime/startup\_nat.c}{runtime/startup\_nat.c:138}
}%
\only<2>{
\listc[firstline=138,highlightlines={147,149}]{../ocaml/runtime/printexc.c}{runtime/printexc.c:138}
}%
\only<3>{
\listc[firstline=110,lastline=134,highlightlines=129]{../ocaml/runtime/printexc.c}{runtime/printexc.c:138}
}%
\only<4>{
\listc[firstline=138,highlightlines={152,154}]{../ocaml/runtime/printexc.c}{runtime/printexc.c:138}
}%
\end{column}
\end{columns}
\footnotetext[1]{\url{https://v2.ocaml.org/api/Printexc.html\#1\_Uncaughtexceptions}}
\end{frame}
}%
\onslide*<6->{\providecommand\step{4}%
% Runtime's default exception handler
%
\subsection*{Runtime's default exception handler}
\frameSubsection{pictures/The\_Architect.png}{}


\begin{frame}{Default exception handler}
\begin{columns}
\begin{column}{0.5\textwidth}
\begin{itemize}
\item \localname{domain\_state->exn\_handler} points to a trap located before \funcname{entry}!
\item What installed this very first trap there?
\end{itemize}
\bigskip
\listocaml{../src/default/default.ml}{default/default.ml}
\bigskip
\inputminted[fontsize=\footnotesize]{text}{../src/default/default.out}
\end{column}
\begin{column}{0.5\textwidth}
\centering
\only<1>{\providecommand\step{0}\input{diagrams/default/default.tex}}%
\only<2>{\providecommand\step{4}\input{diagrams/default/default.tex}}%
\end{column}
\end{columns}
\end{frame}


\begin{frame}{Startup}{How did we get there by the way?}
\begin{columns}
\begin{column}{0.45\textwidth}
\begin{itemize}
\item The entry point address of the binary is \funcname{main} (\funcname{\_start}) from the runtime
\item The program starts like a C program:
\begin{itemize}
\item \funcname{caml\_start\_program}
\item \funcname{caml\_startup\_common}
\item \funcname{caml\_startup\_exn}
\item \funcname{caml\_startup}
\item \funcname{caml\_main}
\item \funcname{main}
\end{itemize}
%\item \funcname{caml\_startup\_common}: initializes GC, domains \dots
% Also: codefrag, locale, custom opeartions, frame_descr, signals, stack
\item \funcname{caml\_start\_program} is the transition point between the C realm and the OCaml one
\end{itemize}
\bigskip
\listocaml{../src/default/default.ml}{default/default.ml}
\end{column}
\begin{column}{0.55\textwidth}
\listasm[lastline=24,highlightlines={22-24}]{src/ocaml/caml\_start\_program.s}{runtime/amd64.S:604}
\end{column}
\end{columns}
\end{frame}


\begin{frame}{Setup to jump into OCaml entry point}{\funcname{caml\_start\_program} initializes the main fiber}
\begin{columns}
\begin{column}{0.6\textwidth}
\only<1,3,5>{
\begin{itemize}
\item<1-> Stores information to allow switching from OCaml stack to the C stack
\item<3-> Constructs the default exception handler
\item<5-> Switches to the OCaml stack and calls \funcname{caml\_program}
\end{itemize}
\bigskip
\listocaml{../src/default/default.ml}{default/default.ml}
}%
\only<2>{
\listasm[firstline=22,lastline=41,highlightlines={25-27}]{src/ocaml/caml\_start\_program.s}{caml\_start\_program}
}%
\only<4>{
\mintasm[firstline=22,lastline=41,highlightlines={31-38}]{src/ocaml/caml\_start\_program.s}
\listasm[firstline=66,lastline=72]{src/ocaml/caml\_start\_program.s}{caml\_start\_program}
}%
\only<6>{
\listasm[firstline=22,lastline=41,highlightlines={39-41}]{src/ocaml/caml\_start\_program.s}{caml\_start\_program}
}%
\end{column}
\begin{column}{0.4\textwidth}
\centering
\only<1-2>{\providecommand\step{2}\input{diagrams/default/default.tex}}%
\only<3-5>{\providecommand\step{3}\input{diagrams/default/default.tex}}%
\onslide*<6->{\providecommand\step{4}\input{diagrams/default/default.tex}}%
\end{column}
\end{columns}
\end{frame}

\begin{frame}{Executing the OCaml program}{And raising an exception without handler}
\begin{columns}
\begin{column}{0.6\textwidth}
\only<1-3>{
\begin{itemize}
\item<1-> Execution continues with OCaml code
\begin{itemize}
\item \funcname{camlDefault\_\_main\_269}
\item \funcname{camlDefault\_\_entry}
\item \funcname{caml\_program}
\end{itemize}
\item<1-> \funcname{camlDefault\_\_main\_269} raises a \typename{Failure} exception
\item<1-> \typename{Failure} is caught by \funcname{caml\_start\_program}
\begin{itemize}
\item<1-> The handler set the 2nd LSB of the exception value to \funcarg{1}
\item<3-> And then forces \funcname{caml\_start\_program} to terminate
\end{itemize}
\end{itemize}
}
\bigskip
\only<1,3>{\listocaml[highlightlines=3]{../src/default/default.ml}{default/default.ml}}%
\only<2>{\listasm[firstline=66,lastline=72,highlightlines=69]{src/ocaml/caml\_start\_program.s}{caml\_start\_program}}%
\end{column}
\begin{column}{0.4\textwidth}
\centering
\providecommand\step{4}\input{diagrams/default/default.tex}
\end{column}
\end{columns}
\end{frame}
%\only<>{\listasm[firstline=47,lastline=65]{src/ocaml/caml\_start\_program.s}{caml\_start\_program}}


\begin{frame}{Back to C}{Clean up, complain and terminate}
\begin{columns}
\begin{column}{0.45\textwidth}
\begin{itemize}
\item Backtrace:
\begin{itemize}
\item \funcname{caml\_start\_program} $\rightarrow$ OCaml code
\item \funcname{caml\_startup\_common}
\item \funcname{caml\_startup\_exn}
\item \funcname{caml\_startup}
\item \funcname{caml\_main}
\item \funcname{main}
\end{itemize}
\smallskip
\item \funcname{caml\_startup} checks the return value of \funcname{caml\_startup\_exn}
\begin{itemize}
\item The return value has been specially marked by \funcname{caml\_start\_program} handler
\item Calls \funcname{caml\_fatal\_uncaught\_exception}
\end{itemize}
\item<2-> \funcname{caml\_fatal\_uncaught\_exception} either
\begin{itemize}
\item<2-> Calls the user custom uncaught exception handler, if set with \mintinline{ocaml}{Printexc.set_uncaught_exception_handler fn}\footnotemark
\item<2-> Or falls back to the default one
\item<2-> Which notifies about the uncaught exception
\end{itemize}
\item<4-> Terminates the program either with \funcname{abort} or with a non-zero exit code
\end{itemize}
\end{column}
\begin{column}{0.55\textwidth}
\only<1>{
\listc[firstline=84,lastline=89,highlightlines=88]{../ocaml/runtime/caml/mlvalues.h}{runtime/caml/mlvalues.h:84}
\listc[firstline=138,lastline=143,highlightlines={141-142}]{../ocaml/runtime/startup\_nat.c}{runtime/startup\_nat.c:138}
}%
\only<2>{
\listc[firstline=138,highlightlines={147,149}]{../ocaml/runtime/printexc.c}{runtime/printexc.c:138}
}%
\only<3>{
\listc[firstline=110,lastline=134,highlightlines=129]{../ocaml/runtime/printexc.c}{runtime/printexc.c:138}
}%
\only<4>{
\listc[firstline=138,highlightlines={152,154}]{../ocaml/runtime/printexc.c}{runtime/printexc.c:138}
}%
\end{column}
\end{columns}
\footnotetext[1]{\url{https://v2.ocaml.org/api/Printexc.html\#1\_Uncaughtexceptions}}
\end{frame}
}%
\end{column}
\end{columns}
\end{frame}

\begin{frame}{Executing the OCaml program}{And raising an exception without handler}
\begin{columns}
\begin{column}{0.6\textwidth}
\only<1-3>{
\begin{itemize}
\item<1-> Execution continues with OCaml code
\begin{itemize}
\item \funcname{camlDefault\_\_main\_269}
\item \funcname{camlDefault\_\_entry}
\item \funcname{caml\_program}
\end{itemize}
\item<1-> \funcname{camlDefault\_\_main\_269} raises a \typename{Failure} exception
\item<1-> \typename{Failure} is caught by \funcname{caml\_start\_program}
\begin{itemize}
\item<1-> The handler set the 2nd LSB of the exception value to \funcarg{1}
\item<3-> And then forces \funcname{caml\_start\_program} to terminate
\end{itemize}
\end{itemize}
}
\bigskip
\only<1,3>{\listocaml[highlightlines=3]{../src/default/default.ml}{default/default.ml}}%
\only<2>{\listasm[firstline=66,lastline=72,highlightlines=69]{src/ocaml/caml\_start\_program.s}{caml\_start\_program}}%
\end{column}
\begin{column}{0.4\textwidth}
\centering
\providecommand\step{4}%
% Runtime's default exception handler
%
\subsection*{Runtime's default exception handler}
\frameSubsection{pictures/The\_Architect.png}{}


\begin{frame}{Default exception handler}
\begin{columns}
\begin{column}{0.5\textwidth}
\begin{itemize}
\item \localname{domain\_state->exn\_handler} points to a trap located before \funcname{entry}!
\item What installed this very first trap there?
\end{itemize}
\bigskip
\listocaml{../src/default/default.ml}{default/default.ml}
\bigskip
\inputminted[fontsize=\footnotesize]{text}{../src/default/default.out}
\end{column}
\begin{column}{0.5\textwidth}
\centering
\only<1>{\providecommand\step{0}\input{diagrams/default/default.tex}}%
\only<2>{\providecommand\step{4}\input{diagrams/default/default.tex}}%
\end{column}
\end{columns}
\end{frame}


\begin{frame}{Startup}{How did we get there by the way?}
\begin{columns}
\begin{column}{0.45\textwidth}
\begin{itemize}
\item The entry point address of the binary is \funcname{main} (\funcname{\_start}) from the runtime
\item The program starts like a C program:
\begin{itemize}
\item \funcname{caml\_start\_program}
\item \funcname{caml\_startup\_common}
\item \funcname{caml\_startup\_exn}
\item \funcname{caml\_startup}
\item \funcname{caml\_main}
\item \funcname{main}
\end{itemize}
%\item \funcname{caml\_startup\_common}: initializes GC, domains \dots
% Also: codefrag, locale, custom opeartions, frame_descr, signals, stack
\item \funcname{caml\_start\_program} is the transition point between the C realm and the OCaml one
\end{itemize}
\bigskip
\listocaml{../src/default/default.ml}{default/default.ml}
\end{column}
\begin{column}{0.55\textwidth}
\listasm[lastline=24,highlightlines={22-24}]{src/ocaml/caml\_start\_program.s}{runtime/amd64.S:604}
\end{column}
\end{columns}
\end{frame}


\begin{frame}{Setup to jump into OCaml entry point}{\funcname{caml\_start\_program} initializes the main fiber}
\begin{columns}
\begin{column}{0.6\textwidth}
\only<1,3,5>{
\begin{itemize}
\item<1-> Stores information to allow switching from OCaml stack to the C stack
\item<3-> Constructs the default exception handler
\item<5-> Switches to the OCaml stack and calls \funcname{caml\_program}
\end{itemize}
\bigskip
\listocaml{../src/default/default.ml}{default/default.ml}
}%
\only<2>{
\listasm[firstline=22,lastline=41,highlightlines={25-27}]{src/ocaml/caml\_start\_program.s}{caml\_start\_program}
}%
\only<4>{
\mintasm[firstline=22,lastline=41,highlightlines={31-38}]{src/ocaml/caml\_start\_program.s}
\listasm[firstline=66,lastline=72]{src/ocaml/caml\_start\_program.s}{caml\_start\_program}
}%
\only<6>{
\listasm[firstline=22,lastline=41,highlightlines={39-41}]{src/ocaml/caml\_start\_program.s}{caml\_start\_program}
}%
\end{column}
\begin{column}{0.4\textwidth}
\centering
\only<1-2>{\providecommand\step{2}\input{diagrams/default/default.tex}}%
\only<3-5>{\providecommand\step{3}\input{diagrams/default/default.tex}}%
\onslide*<6->{\providecommand\step{4}\input{diagrams/default/default.tex}}%
\end{column}
\end{columns}
\end{frame}

\begin{frame}{Executing the OCaml program}{And raising an exception without handler}
\begin{columns}
\begin{column}{0.6\textwidth}
\only<1-3>{
\begin{itemize}
\item<1-> Execution continues with OCaml code
\begin{itemize}
\item \funcname{camlDefault\_\_main\_269}
\item \funcname{camlDefault\_\_entry}
\item \funcname{caml\_program}
\end{itemize}
\item<1-> \funcname{camlDefault\_\_main\_269} raises a \typename{Failure} exception
\item<1-> \typename{Failure} is caught by \funcname{caml\_start\_program}
\begin{itemize}
\item<1-> The handler set the 2nd LSB of the exception value to \funcarg{1}
\item<3-> And then forces \funcname{caml\_start\_program} to terminate
\end{itemize}
\end{itemize}
}
\bigskip
\only<1,3>{\listocaml[highlightlines=3]{../src/default/default.ml}{default/default.ml}}%
\only<2>{\listasm[firstline=66,lastline=72,highlightlines=69]{src/ocaml/caml\_start\_program.s}{caml\_start\_program}}%
\end{column}
\begin{column}{0.4\textwidth}
\centering
\providecommand\step{4}\input{diagrams/default/default.tex}
\end{column}
\end{columns}
\end{frame}
%\only<>{\listasm[firstline=47,lastline=65]{src/ocaml/caml\_start\_program.s}{caml\_start\_program}}


\begin{frame}{Back to C}{Clean up, complain and terminate}
\begin{columns}
\begin{column}{0.45\textwidth}
\begin{itemize}
\item Backtrace:
\begin{itemize}
\item \funcname{caml\_start\_program} $\rightarrow$ OCaml code
\item \funcname{caml\_startup\_common}
\item \funcname{caml\_startup\_exn}
\item \funcname{caml\_startup}
\item \funcname{caml\_main}
\item \funcname{main}
\end{itemize}
\smallskip
\item \funcname{caml\_startup} checks the return value of \funcname{caml\_startup\_exn}
\begin{itemize}
\item The return value has been specially marked by \funcname{caml\_start\_program} handler
\item Calls \funcname{caml\_fatal\_uncaught\_exception}
\end{itemize}
\item<2-> \funcname{caml\_fatal\_uncaught\_exception} either
\begin{itemize}
\item<2-> Calls the user custom uncaught exception handler, if set with \mintinline{ocaml}{Printexc.set_uncaught_exception_handler fn}\footnotemark
\item<2-> Or falls back to the default one
\item<2-> Which notifies about the uncaught exception
\end{itemize}
\item<4-> Terminates the program either with \funcname{abort} or with a non-zero exit code
\end{itemize}
\end{column}
\begin{column}{0.55\textwidth}
\only<1>{
\listc[firstline=84,lastline=89,highlightlines=88]{../ocaml/runtime/caml/mlvalues.h}{runtime/caml/mlvalues.h:84}
\listc[firstline=138,lastline=143,highlightlines={141-142}]{../ocaml/runtime/startup\_nat.c}{runtime/startup\_nat.c:138}
}%
\only<2>{
\listc[firstline=138,highlightlines={147,149}]{../ocaml/runtime/printexc.c}{runtime/printexc.c:138}
}%
\only<3>{
\listc[firstline=110,lastline=134,highlightlines=129]{../ocaml/runtime/printexc.c}{runtime/printexc.c:138}
}%
\only<4>{
\listc[firstline=138,highlightlines={152,154}]{../ocaml/runtime/printexc.c}{runtime/printexc.c:138}
}%
\end{column}
\end{columns}
\footnotetext[1]{\url{https://v2.ocaml.org/api/Printexc.html\#1\_Uncaughtexceptions}}
\end{frame}

\end{column}
\end{columns}
\end{frame}
%\only<>{\listasm[firstline=47,lastline=65]{src/ocaml/caml\_start\_program.s}{caml\_start\_program}}


\begin{frame}{Back to C}{Clean up, complain and terminate}
\begin{columns}
\begin{column}{0.45\textwidth}
\begin{itemize}
\item Backtrace:
\begin{itemize}
\item \funcname{caml\_start\_program} $\rightarrow$ OCaml code
\item \funcname{caml\_startup\_common}
\item \funcname{caml\_startup\_exn}
\item \funcname{caml\_startup}
\item \funcname{caml\_main}
\item \funcname{main}
\end{itemize}
\smallskip
\item \funcname{caml\_startup} checks the return value of \funcname{caml\_startup\_exn}
\begin{itemize}
\item The return value has been specially marked by \funcname{caml\_start\_program} handler
\item Calls \funcname{caml\_fatal\_uncaught\_exception}
\end{itemize}
\item<2-> \funcname{caml\_fatal\_uncaught\_exception} either
\begin{itemize}
\item<2-> Calls the user custom uncaught exception handler, if set with \mintinline{ocaml}{Printexc.set_uncaught_exception_handler fn}\footnotemark
\item<2-> Or falls back to the default one
\item<2-> Which notifies about the uncaught exception
\end{itemize}
\item<4-> Terminates the program either with \funcname{abort} or with a non-zero exit code
\end{itemize}
\end{column}
\begin{column}{0.55\textwidth}
\only<1>{
\listc[firstline=84,lastline=89,highlightlines=88]{../ocaml/runtime/caml/mlvalues.h}{runtime/caml/mlvalues.h:84}
\listc[firstline=138,lastline=143,highlightlines={141-142}]{../ocaml/runtime/startup\_nat.c}{runtime/startup\_nat.c:138}
}%
\only<2>{
\listc[firstline=138,highlightlines={147,149}]{../ocaml/runtime/printexc.c}{runtime/printexc.c:138}
}%
\only<3>{
\listc[firstline=110,lastline=134,highlightlines=129]{../ocaml/runtime/printexc.c}{runtime/printexc.c:138}
}%
\only<4>{
\listc[firstline=138,highlightlines={152,154}]{../ocaml/runtime/printexc.c}{runtime/printexc.c:138}
}%
\end{column}
\end{columns}
\footnotetext[1]{\url{https://v2.ocaml.org/api/Printexc.html\#1\_Uncaughtexceptions}}
\end{frame}
}%
\only<3-5>{\providecommand\step{3}%
% Runtime's default exception handler
%
\subsection*{Runtime's default exception handler}
\frameSubsection{pictures/The\_Architect.png}{}


\begin{frame}{Default exception handler}
\begin{columns}
\begin{column}{0.5\textwidth}
\begin{itemize}
\item \localname{domain\_state->exn\_handler} points to a trap located before \funcname{entry}!
\item What installed this very first trap there?
\end{itemize}
\bigskip
\listocaml{../src/default/default.ml}{default/default.ml}
\bigskip
\inputminted[fontsize=\footnotesize]{text}{../src/default/default.out}
\end{column}
\begin{column}{0.5\textwidth}
\centering
\only<1>{\providecommand\step{0}%
% Runtime's default exception handler
%
\subsection*{Runtime's default exception handler}
\frameSubsection{pictures/The\_Architect.png}{}


\begin{frame}{Default exception handler}
\begin{columns}
\begin{column}{0.5\textwidth}
\begin{itemize}
\item \localname{domain\_state->exn\_handler} points to a trap located before \funcname{entry}!
\item What installed this very first trap there?
\end{itemize}
\bigskip
\listocaml{../src/default/default.ml}{default/default.ml}
\bigskip
\inputminted[fontsize=\footnotesize]{text}{../src/default/default.out}
\end{column}
\begin{column}{0.5\textwidth}
\centering
\only<1>{\providecommand\step{0}\input{diagrams/default/default.tex}}%
\only<2>{\providecommand\step{4}\input{diagrams/default/default.tex}}%
\end{column}
\end{columns}
\end{frame}


\begin{frame}{Startup}{How did we get there by the way?}
\begin{columns}
\begin{column}{0.45\textwidth}
\begin{itemize}
\item The entry point address of the binary is \funcname{main} (\funcname{\_start}) from the runtime
\item The program starts like a C program:
\begin{itemize}
\item \funcname{caml\_start\_program}
\item \funcname{caml\_startup\_common}
\item \funcname{caml\_startup\_exn}
\item \funcname{caml\_startup}
\item \funcname{caml\_main}
\item \funcname{main}
\end{itemize}
%\item \funcname{caml\_startup\_common}: initializes GC, domains \dots
% Also: codefrag, locale, custom opeartions, frame_descr, signals, stack
\item \funcname{caml\_start\_program} is the transition point between the C realm and the OCaml one
\end{itemize}
\bigskip
\listocaml{../src/default/default.ml}{default/default.ml}
\end{column}
\begin{column}{0.55\textwidth}
\listasm[lastline=24,highlightlines={22-24}]{src/ocaml/caml\_start\_program.s}{runtime/amd64.S:604}
\end{column}
\end{columns}
\end{frame}


\begin{frame}{Setup to jump into OCaml entry point}{\funcname{caml\_start\_program} initializes the main fiber}
\begin{columns}
\begin{column}{0.6\textwidth}
\only<1,3,5>{
\begin{itemize}
\item<1-> Stores information to allow switching from OCaml stack to the C stack
\item<3-> Constructs the default exception handler
\item<5-> Switches to the OCaml stack and calls \funcname{caml\_program}
\end{itemize}
\bigskip
\listocaml{../src/default/default.ml}{default/default.ml}
}%
\only<2>{
\listasm[firstline=22,lastline=41,highlightlines={25-27}]{src/ocaml/caml\_start\_program.s}{caml\_start\_program}
}%
\only<4>{
\mintasm[firstline=22,lastline=41,highlightlines={31-38}]{src/ocaml/caml\_start\_program.s}
\listasm[firstline=66,lastline=72]{src/ocaml/caml\_start\_program.s}{caml\_start\_program}
}%
\only<6>{
\listasm[firstline=22,lastline=41,highlightlines={39-41}]{src/ocaml/caml\_start\_program.s}{caml\_start\_program}
}%
\end{column}
\begin{column}{0.4\textwidth}
\centering
\only<1-2>{\providecommand\step{2}\input{diagrams/default/default.tex}}%
\only<3-5>{\providecommand\step{3}\input{diagrams/default/default.tex}}%
\onslide*<6->{\providecommand\step{4}\input{diagrams/default/default.tex}}%
\end{column}
\end{columns}
\end{frame}

\begin{frame}{Executing the OCaml program}{And raising an exception without handler}
\begin{columns}
\begin{column}{0.6\textwidth}
\only<1-3>{
\begin{itemize}
\item<1-> Execution continues with OCaml code
\begin{itemize}
\item \funcname{camlDefault\_\_main\_269}
\item \funcname{camlDefault\_\_entry}
\item \funcname{caml\_program}
\end{itemize}
\item<1-> \funcname{camlDefault\_\_main\_269} raises a \typename{Failure} exception
\item<1-> \typename{Failure} is caught by \funcname{caml\_start\_program}
\begin{itemize}
\item<1-> The handler set the 2nd LSB of the exception value to \funcarg{1}
\item<3-> And then forces \funcname{caml\_start\_program} to terminate
\end{itemize}
\end{itemize}
}
\bigskip
\only<1,3>{\listocaml[highlightlines=3]{../src/default/default.ml}{default/default.ml}}%
\only<2>{\listasm[firstline=66,lastline=72,highlightlines=69]{src/ocaml/caml\_start\_program.s}{caml\_start\_program}}%
\end{column}
\begin{column}{0.4\textwidth}
\centering
\providecommand\step{4}\input{diagrams/default/default.tex}
\end{column}
\end{columns}
\end{frame}
%\only<>{\listasm[firstline=47,lastline=65]{src/ocaml/caml\_start\_program.s}{caml\_start\_program}}


\begin{frame}{Back to C}{Clean up, complain and terminate}
\begin{columns}
\begin{column}{0.45\textwidth}
\begin{itemize}
\item Backtrace:
\begin{itemize}
\item \funcname{caml\_start\_program} $\rightarrow$ OCaml code
\item \funcname{caml\_startup\_common}
\item \funcname{caml\_startup\_exn}
\item \funcname{caml\_startup}
\item \funcname{caml\_main}
\item \funcname{main}
\end{itemize}
\smallskip
\item \funcname{caml\_startup} checks the return value of \funcname{caml\_startup\_exn}
\begin{itemize}
\item The return value has been specially marked by \funcname{caml\_start\_program} handler
\item Calls \funcname{caml\_fatal\_uncaught\_exception}
\end{itemize}
\item<2-> \funcname{caml\_fatal\_uncaught\_exception} either
\begin{itemize}
\item<2-> Calls the user custom uncaught exception handler, if set with \mintinline{ocaml}{Printexc.set_uncaught_exception_handler fn}\footnotemark
\item<2-> Or falls back to the default one
\item<2-> Which notifies about the uncaught exception
\end{itemize}
\item<4-> Terminates the program either with \funcname{abort} or with a non-zero exit code
\end{itemize}
\end{column}
\begin{column}{0.55\textwidth}
\only<1>{
\listc[firstline=84,lastline=89,highlightlines=88]{../ocaml/runtime/caml/mlvalues.h}{runtime/caml/mlvalues.h:84}
\listc[firstline=138,lastline=143,highlightlines={141-142}]{../ocaml/runtime/startup\_nat.c}{runtime/startup\_nat.c:138}
}%
\only<2>{
\listc[firstline=138,highlightlines={147,149}]{../ocaml/runtime/printexc.c}{runtime/printexc.c:138}
}%
\only<3>{
\listc[firstline=110,lastline=134,highlightlines=129]{../ocaml/runtime/printexc.c}{runtime/printexc.c:138}
}%
\only<4>{
\listc[firstline=138,highlightlines={152,154}]{../ocaml/runtime/printexc.c}{runtime/printexc.c:138}
}%
\end{column}
\end{columns}
\footnotetext[1]{\url{https://v2.ocaml.org/api/Printexc.html\#1\_Uncaughtexceptions}}
\end{frame}
}%
\only<2>{\providecommand\step{4}%
% Runtime's default exception handler
%
\subsection*{Runtime's default exception handler}
\frameSubsection{pictures/The\_Architect.png}{}


\begin{frame}{Default exception handler}
\begin{columns}
\begin{column}{0.5\textwidth}
\begin{itemize}
\item \localname{domain\_state->exn\_handler} points to a trap located before \funcname{entry}!
\item What installed this very first trap there?
\end{itemize}
\bigskip
\listocaml{../src/default/default.ml}{default/default.ml}
\bigskip
\inputminted[fontsize=\footnotesize]{text}{../src/default/default.out}
\end{column}
\begin{column}{0.5\textwidth}
\centering
\only<1>{\providecommand\step{0}\input{diagrams/default/default.tex}}%
\only<2>{\providecommand\step{4}\input{diagrams/default/default.tex}}%
\end{column}
\end{columns}
\end{frame}


\begin{frame}{Startup}{How did we get there by the way?}
\begin{columns}
\begin{column}{0.45\textwidth}
\begin{itemize}
\item The entry point address of the binary is \funcname{main} (\funcname{\_start}) from the runtime
\item The program starts like a C program:
\begin{itemize}
\item \funcname{caml\_start\_program}
\item \funcname{caml\_startup\_common}
\item \funcname{caml\_startup\_exn}
\item \funcname{caml\_startup}
\item \funcname{caml\_main}
\item \funcname{main}
\end{itemize}
%\item \funcname{caml\_startup\_common}: initializes GC, domains \dots
% Also: codefrag, locale, custom opeartions, frame_descr, signals, stack
\item \funcname{caml\_start\_program} is the transition point between the C realm and the OCaml one
\end{itemize}
\bigskip
\listocaml{../src/default/default.ml}{default/default.ml}
\end{column}
\begin{column}{0.55\textwidth}
\listasm[lastline=24,highlightlines={22-24}]{src/ocaml/caml\_start\_program.s}{runtime/amd64.S:604}
\end{column}
\end{columns}
\end{frame}


\begin{frame}{Setup to jump into OCaml entry point}{\funcname{caml\_start\_program} initializes the main fiber}
\begin{columns}
\begin{column}{0.6\textwidth}
\only<1,3,5>{
\begin{itemize}
\item<1-> Stores information to allow switching from OCaml stack to the C stack
\item<3-> Constructs the default exception handler
\item<5-> Switches to the OCaml stack and calls \funcname{caml\_program}
\end{itemize}
\bigskip
\listocaml{../src/default/default.ml}{default/default.ml}
}%
\only<2>{
\listasm[firstline=22,lastline=41,highlightlines={25-27}]{src/ocaml/caml\_start\_program.s}{caml\_start\_program}
}%
\only<4>{
\mintasm[firstline=22,lastline=41,highlightlines={31-38}]{src/ocaml/caml\_start\_program.s}
\listasm[firstline=66,lastline=72]{src/ocaml/caml\_start\_program.s}{caml\_start\_program}
}%
\only<6>{
\listasm[firstline=22,lastline=41,highlightlines={39-41}]{src/ocaml/caml\_start\_program.s}{caml\_start\_program}
}%
\end{column}
\begin{column}{0.4\textwidth}
\centering
\only<1-2>{\providecommand\step{2}\input{diagrams/default/default.tex}}%
\only<3-5>{\providecommand\step{3}\input{diagrams/default/default.tex}}%
\onslide*<6->{\providecommand\step{4}\input{diagrams/default/default.tex}}%
\end{column}
\end{columns}
\end{frame}

\begin{frame}{Executing the OCaml program}{And raising an exception without handler}
\begin{columns}
\begin{column}{0.6\textwidth}
\only<1-3>{
\begin{itemize}
\item<1-> Execution continues with OCaml code
\begin{itemize}
\item \funcname{camlDefault\_\_main\_269}
\item \funcname{camlDefault\_\_entry}
\item \funcname{caml\_program}
\end{itemize}
\item<1-> \funcname{camlDefault\_\_main\_269} raises a \typename{Failure} exception
\item<1-> \typename{Failure} is caught by \funcname{caml\_start\_program}
\begin{itemize}
\item<1-> The handler set the 2nd LSB of the exception value to \funcarg{1}
\item<3-> And then forces \funcname{caml\_start\_program} to terminate
\end{itemize}
\end{itemize}
}
\bigskip
\only<1,3>{\listocaml[highlightlines=3]{../src/default/default.ml}{default/default.ml}}%
\only<2>{\listasm[firstline=66,lastline=72,highlightlines=69]{src/ocaml/caml\_start\_program.s}{caml\_start\_program}}%
\end{column}
\begin{column}{0.4\textwidth}
\centering
\providecommand\step{4}\input{diagrams/default/default.tex}
\end{column}
\end{columns}
\end{frame}
%\only<>{\listasm[firstline=47,lastline=65]{src/ocaml/caml\_start\_program.s}{caml\_start\_program}}


\begin{frame}{Back to C}{Clean up, complain and terminate}
\begin{columns}
\begin{column}{0.45\textwidth}
\begin{itemize}
\item Backtrace:
\begin{itemize}
\item \funcname{caml\_start\_program} $\rightarrow$ OCaml code
\item \funcname{caml\_startup\_common}
\item \funcname{caml\_startup\_exn}
\item \funcname{caml\_startup}
\item \funcname{caml\_main}
\item \funcname{main}
\end{itemize}
\smallskip
\item \funcname{caml\_startup} checks the return value of \funcname{caml\_startup\_exn}
\begin{itemize}
\item The return value has been specially marked by \funcname{caml\_start\_program} handler
\item Calls \funcname{caml\_fatal\_uncaught\_exception}
\end{itemize}
\item<2-> \funcname{caml\_fatal\_uncaught\_exception} either
\begin{itemize}
\item<2-> Calls the user custom uncaught exception handler, if set with \mintinline{ocaml}{Printexc.set_uncaught_exception_handler fn}\footnotemark
\item<2-> Or falls back to the default one
\item<2-> Which notifies about the uncaught exception
\end{itemize}
\item<4-> Terminates the program either with \funcname{abort} or with a non-zero exit code
\end{itemize}
\end{column}
\begin{column}{0.55\textwidth}
\only<1>{
\listc[firstline=84,lastline=89,highlightlines=88]{../ocaml/runtime/caml/mlvalues.h}{runtime/caml/mlvalues.h:84}
\listc[firstline=138,lastline=143,highlightlines={141-142}]{../ocaml/runtime/startup\_nat.c}{runtime/startup\_nat.c:138}
}%
\only<2>{
\listc[firstline=138,highlightlines={147,149}]{../ocaml/runtime/printexc.c}{runtime/printexc.c:138}
}%
\only<3>{
\listc[firstline=110,lastline=134,highlightlines=129]{../ocaml/runtime/printexc.c}{runtime/printexc.c:138}
}%
\only<4>{
\listc[firstline=138,highlightlines={152,154}]{../ocaml/runtime/printexc.c}{runtime/printexc.c:138}
}%
\end{column}
\end{columns}
\footnotetext[1]{\url{https://v2.ocaml.org/api/Printexc.html\#1\_Uncaughtexceptions}}
\end{frame}
}%
\end{column}
\end{columns}
\end{frame}


\begin{frame}{Startup}{How did we get there by the way?}
\begin{columns}
\begin{column}{0.45\textwidth}
\begin{itemize}
\item The entry point address of the binary is \funcname{main} (\funcname{\_start}) from the runtime
\item The program starts like a C program:
\begin{itemize}
\item \funcname{caml\_start\_program}
\item \funcname{caml\_startup\_common}
\item \funcname{caml\_startup\_exn}
\item \funcname{caml\_startup}
\item \funcname{caml\_main}
\item \funcname{main}
\end{itemize}
%\item \funcname{caml\_startup\_common}: initializes GC, domains \dots
% Also: codefrag, locale, custom opeartions, frame_descr, signals, stack
\item \funcname{caml\_start\_program} is the transition point between the C realm and the OCaml one
\end{itemize}
\bigskip
\listocaml{../src/default/default.ml}{default/default.ml}
\end{column}
\begin{column}{0.55\textwidth}
\listasm[lastline=24,highlightlines={22-24}]{src/ocaml/caml\_start\_program.s}{runtime/amd64.S:604}
\end{column}
\end{columns}
\end{frame}


\begin{frame}{Setup to jump into OCaml entry point}{\funcname{caml\_start\_program} initializes the main fiber}
\begin{columns}
\begin{column}{0.6\textwidth}
\only<1,3,5>{
\begin{itemize}
\item<1-> Stores information to allow switching from OCaml stack to the C stack
\item<3-> Constructs the default exception handler
\item<5-> Switches to the OCaml stack and calls \funcname{caml\_program}
\end{itemize}
\bigskip
\listocaml{../src/default/default.ml}{default/default.ml}
}%
\only<2>{
\listasm[firstline=22,lastline=41,highlightlines={25-27}]{src/ocaml/caml\_start\_program.s}{caml\_start\_program}
}%
\only<4>{
\mintasm[firstline=22,lastline=41,highlightlines={31-38}]{src/ocaml/caml\_start\_program.s}
\listasm[firstline=66,lastline=72]{src/ocaml/caml\_start\_program.s}{caml\_start\_program}
}%
\only<6>{
\listasm[firstline=22,lastline=41,highlightlines={39-41}]{src/ocaml/caml\_start\_program.s}{caml\_start\_program}
}%
\end{column}
\begin{column}{0.4\textwidth}
\centering
\only<1-2>{\providecommand\step{2}%
% Runtime's default exception handler
%
\subsection*{Runtime's default exception handler}
\frameSubsection{pictures/The\_Architect.png}{}


\begin{frame}{Default exception handler}
\begin{columns}
\begin{column}{0.5\textwidth}
\begin{itemize}
\item \localname{domain\_state->exn\_handler} points to a trap located before \funcname{entry}!
\item What installed this very first trap there?
\end{itemize}
\bigskip
\listocaml{../src/default/default.ml}{default/default.ml}
\bigskip
\inputminted[fontsize=\footnotesize]{text}{../src/default/default.out}
\end{column}
\begin{column}{0.5\textwidth}
\centering
\only<1>{\providecommand\step{0}\input{diagrams/default/default.tex}}%
\only<2>{\providecommand\step{4}\input{diagrams/default/default.tex}}%
\end{column}
\end{columns}
\end{frame}


\begin{frame}{Startup}{How did we get there by the way?}
\begin{columns}
\begin{column}{0.45\textwidth}
\begin{itemize}
\item The entry point address of the binary is \funcname{main} (\funcname{\_start}) from the runtime
\item The program starts like a C program:
\begin{itemize}
\item \funcname{caml\_start\_program}
\item \funcname{caml\_startup\_common}
\item \funcname{caml\_startup\_exn}
\item \funcname{caml\_startup}
\item \funcname{caml\_main}
\item \funcname{main}
\end{itemize}
%\item \funcname{caml\_startup\_common}: initializes GC, domains \dots
% Also: codefrag, locale, custom opeartions, frame_descr, signals, stack
\item \funcname{caml\_start\_program} is the transition point between the C realm and the OCaml one
\end{itemize}
\bigskip
\listocaml{../src/default/default.ml}{default/default.ml}
\end{column}
\begin{column}{0.55\textwidth}
\listasm[lastline=24,highlightlines={22-24}]{src/ocaml/caml\_start\_program.s}{runtime/amd64.S:604}
\end{column}
\end{columns}
\end{frame}


\begin{frame}{Setup to jump into OCaml entry point}{\funcname{caml\_start\_program} initializes the main fiber}
\begin{columns}
\begin{column}{0.6\textwidth}
\only<1,3,5>{
\begin{itemize}
\item<1-> Stores information to allow switching from OCaml stack to the C stack
\item<3-> Constructs the default exception handler
\item<5-> Switches to the OCaml stack and calls \funcname{caml\_program}
\end{itemize}
\bigskip
\listocaml{../src/default/default.ml}{default/default.ml}
}%
\only<2>{
\listasm[firstline=22,lastline=41,highlightlines={25-27}]{src/ocaml/caml\_start\_program.s}{caml\_start\_program}
}%
\only<4>{
\mintasm[firstline=22,lastline=41,highlightlines={31-38}]{src/ocaml/caml\_start\_program.s}
\listasm[firstline=66,lastline=72]{src/ocaml/caml\_start\_program.s}{caml\_start\_program}
}%
\only<6>{
\listasm[firstline=22,lastline=41,highlightlines={39-41}]{src/ocaml/caml\_start\_program.s}{caml\_start\_program}
}%
\end{column}
\begin{column}{0.4\textwidth}
\centering
\only<1-2>{\providecommand\step{2}\input{diagrams/default/default.tex}}%
\only<3-5>{\providecommand\step{3}\input{diagrams/default/default.tex}}%
\onslide*<6->{\providecommand\step{4}\input{diagrams/default/default.tex}}%
\end{column}
\end{columns}
\end{frame}

\begin{frame}{Executing the OCaml program}{And raising an exception without handler}
\begin{columns}
\begin{column}{0.6\textwidth}
\only<1-3>{
\begin{itemize}
\item<1-> Execution continues with OCaml code
\begin{itemize}
\item \funcname{camlDefault\_\_main\_269}
\item \funcname{camlDefault\_\_entry}
\item \funcname{caml\_program}
\end{itemize}
\item<1-> \funcname{camlDefault\_\_main\_269} raises a \typename{Failure} exception
\item<1-> \typename{Failure} is caught by \funcname{caml\_start\_program}
\begin{itemize}
\item<1-> The handler set the 2nd LSB of the exception value to \funcarg{1}
\item<3-> And then forces \funcname{caml\_start\_program} to terminate
\end{itemize}
\end{itemize}
}
\bigskip
\only<1,3>{\listocaml[highlightlines=3]{../src/default/default.ml}{default/default.ml}}%
\only<2>{\listasm[firstline=66,lastline=72,highlightlines=69]{src/ocaml/caml\_start\_program.s}{caml\_start\_program}}%
\end{column}
\begin{column}{0.4\textwidth}
\centering
\providecommand\step{4}\input{diagrams/default/default.tex}
\end{column}
\end{columns}
\end{frame}
%\only<>{\listasm[firstline=47,lastline=65]{src/ocaml/caml\_start\_program.s}{caml\_start\_program}}


\begin{frame}{Back to C}{Clean up, complain and terminate}
\begin{columns}
\begin{column}{0.45\textwidth}
\begin{itemize}
\item Backtrace:
\begin{itemize}
\item \funcname{caml\_start\_program} $\rightarrow$ OCaml code
\item \funcname{caml\_startup\_common}
\item \funcname{caml\_startup\_exn}
\item \funcname{caml\_startup}
\item \funcname{caml\_main}
\item \funcname{main}
\end{itemize}
\smallskip
\item \funcname{caml\_startup} checks the return value of \funcname{caml\_startup\_exn}
\begin{itemize}
\item The return value has been specially marked by \funcname{caml\_start\_program} handler
\item Calls \funcname{caml\_fatal\_uncaught\_exception}
\end{itemize}
\item<2-> \funcname{caml\_fatal\_uncaught\_exception} either
\begin{itemize}
\item<2-> Calls the user custom uncaught exception handler, if set with \mintinline{ocaml}{Printexc.set_uncaught_exception_handler fn}\footnotemark
\item<2-> Or falls back to the default one
\item<2-> Which notifies about the uncaught exception
\end{itemize}
\item<4-> Terminates the program either with \funcname{abort} or with a non-zero exit code
\end{itemize}
\end{column}
\begin{column}{0.55\textwidth}
\only<1>{
\listc[firstline=84,lastline=89,highlightlines=88]{../ocaml/runtime/caml/mlvalues.h}{runtime/caml/mlvalues.h:84}
\listc[firstline=138,lastline=143,highlightlines={141-142}]{../ocaml/runtime/startup\_nat.c}{runtime/startup\_nat.c:138}
}%
\only<2>{
\listc[firstline=138,highlightlines={147,149}]{../ocaml/runtime/printexc.c}{runtime/printexc.c:138}
}%
\only<3>{
\listc[firstline=110,lastline=134,highlightlines=129]{../ocaml/runtime/printexc.c}{runtime/printexc.c:138}
}%
\only<4>{
\listc[firstline=138,highlightlines={152,154}]{../ocaml/runtime/printexc.c}{runtime/printexc.c:138}
}%
\end{column}
\end{columns}
\footnotetext[1]{\url{https://v2.ocaml.org/api/Printexc.html\#1\_Uncaughtexceptions}}
\end{frame}
}%
\only<3-5>{\providecommand\step{3}%
% Runtime's default exception handler
%
\subsection*{Runtime's default exception handler}
\frameSubsection{pictures/The\_Architect.png}{}


\begin{frame}{Default exception handler}
\begin{columns}
\begin{column}{0.5\textwidth}
\begin{itemize}
\item \localname{domain\_state->exn\_handler} points to a trap located before \funcname{entry}!
\item What installed this very first trap there?
\end{itemize}
\bigskip
\listocaml{../src/default/default.ml}{default/default.ml}
\bigskip
\inputminted[fontsize=\footnotesize]{text}{../src/default/default.out}
\end{column}
\begin{column}{0.5\textwidth}
\centering
\only<1>{\providecommand\step{0}\input{diagrams/default/default.tex}}%
\only<2>{\providecommand\step{4}\input{diagrams/default/default.tex}}%
\end{column}
\end{columns}
\end{frame}


\begin{frame}{Startup}{How did we get there by the way?}
\begin{columns}
\begin{column}{0.45\textwidth}
\begin{itemize}
\item The entry point address of the binary is \funcname{main} (\funcname{\_start}) from the runtime
\item The program starts like a C program:
\begin{itemize}
\item \funcname{caml\_start\_program}
\item \funcname{caml\_startup\_common}
\item \funcname{caml\_startup\_exn}
\item \funcname{caml\_startup}
\item \funcname{caml\_main}
\item \funcname{main}
\end{itemize}
%\item \funcname{caml\_startup\_common}: initializes GC, domains \dots
% Also: codefrag, locale, custom opeartions, frame_descr, signals, stack
\item \funcname{caml\_start\_program} is the transition point between the C realm and the OCaml one
\end{itemize}
\bigskip
\listocaml{../src/default/default.ml}{default/default.ml}
\end{column}
\begin{column}{0.55\textwidth}
\listasm[lastline=24,highlightlines={22-24}]{src/ocaml/caml\_start\_program.s}{runtime/amd64.S:604}
\end{column}
\end{columns}
\end{frame}


\begin{frame}{Setup to jump into OCaml entry point}{\funcname{caml\_start\_program} initializes the main fiber}
\begin{columns}
\begin{column}{0.6\textwidth}
\only<1,3,5>{
\begin{itemize}
\item<1-> Stores information to allow switching from OCaml stack to the C stack
\item<3-> Constructs the default exception handler
\item<5-> Switches to the OCaml stack and calls \funcname{caml\_program}
\end{itemize}
\bigskip
\listocaml{../src/default/default.ml}{default/default.ml}
}%
\only<2>{
\listasm[firstline=22,lastline=41,highlightlines={25-27}]{src/ocaml/caml\_start\_program.s}{caml\_start\_program}
}%
\only<4>{
\mintasm[firstline=22,lastline=41,highlightlines={31-38}]{src/ocaml/caml\_start\_program.s}
\listasm[firstline=66,lastline=72]{src/ocaml/caml\_start\_program.s}{caml\_start\_program}
}%
\only<6>{
\listasm[firstline=22,lastline=41,highlightlines={39-41}]{src/ocaml/caml\_start\_program.s}{caml\_start\_program}
}%
\end{column}
\begin{column}{0.4\textwidth}
\centering
\only<1-2>{\providecommand\step{2}\input{diagrams/default/default.tex}}%
\only<3-5>{\providecommand\step{3}\input{diagrams/default/default.tex}}%
\onslide*<6->{\providecommand\step{4}\input{diagrams/default/default.tex}}%
\end{column}
\end{columns}
\end{frame}

\begin{frame}{Executing the OCaml program}{And raising an exception without handler}
\begin{columns}
\begin{column}{0.6\textwidth}
\only<1-3>{
\begin{itemize}
\item<1-> Execution continues with OCaml code
\begin{itemize}
\item \funcname{camlDefault\_\_main\_269}
\item \funcname{camlDefault\_\_entry}
\item \funcname{caml\_program}
\end{itemize}
\item<1-> \funcname{camlDefault\_\_main\_269} raises a \typename{Failure} exception
\item<1-> \typename{Failure} is caught by \funcname{caml\_start\_program}
\begin{itemize}
\item<1-> The handler set the 2nd LSB of the exception value to \funcarg{1}
\item<3-> And then forces \funcname{caml\_start\_program} to terminate
\end{itemize}
\end{itemize}
}
\bigskip
\only<1,3>{\listocaml[highlightlines=3]{../src/default/default.ml}{default/default.ml}}%
\only<2>{\listasm[firstline=66,lastline=72,highlightlines=69]{src/ocaml/caml\_start\_program.s}{caml\_start\_program}}%
\end{column}
\begin{column}{0.4\textwidth}
\centering
\providecommand\step{4}\input{diagrams/default/default.tex}
\end{column}
\end{columns}
\end{frame}
%\only<>{\listasm[firstline=47,lastline=65]{src/ocaml/caml\_start\_program.s}{caml\_start\_program}}


\begin{frame}{Back to C}{Clean up, complain and terminate}
\begin{columns}
\begin{column}{0.45\textwidth}
\begin{itemize}
\item Backtrace:
\begin{itemize}
\item \funcname{caml\_start\_program} $\rightarrow$ OCaml code
\item \funcname{caml\_startup\_common}
\item \funcname{caml\_startup\_exn}
\item \funcname{caml\_startup}
\item \funcname{caml\_main}
\item \funcname{main}
\end{itemize}
\smallskip
\item \funcname{caml\_startup} checks the return value of \funcname{caml\_startup\_exn}
\begin{itemize}
\item The return value has been specially marked by \funcname{caml\_start\_program} handler
\item Calls \funcname{caml\_fatal\_uncaught\_exception}
\end{itemize}
\item<2-> \funcname{caml\_fatal\_uncaught\_exception} either
\begin{itemize}
\item<2-> Calls the user custom uncaught exception handler, if set with \mintinline{ocaml}{Printexc.set_uncaught_exception_handler fn}\footnotemark
\item<2-> Or falls back to the default one
\item<2-> Which notifies about the uncaught exception
\end{itemize}
\item<4-> Terminates the program either with \funcname{abort} or with a non-zero exit code
\end{itemize}
\end{column}
\begin{column}{0.55\textwidth}
\only<1>{
\listc[firstline=84,lastline=89,highlightlines=88]{../ocaml/runtime/caml/mlvalues.h}{runtime/caml/mlvalues.h:84}
\listc[firstline=138,lastline=143,highlightlines={141-142}]{../ocaml/runtime/startup\_nat.c}{runtime/startup\_nat.c:138}
}%
\only<2>{
\listc[firstline=138,highlightlines={147,149}]{../ocaml/runtime/printexc.c}{runtime/printexc.c:138}
}%
\only<3>{
\listc[firstline=110,lastline=134,highlightlines=129]{../ocaml/runtime/printexc.c}{runtime/printexc.c:138}
}%
\only<4>{
\listc[firstline=138,highlightlines={152,154}]{../ocaml/runtime/printexc.c}{runtime/printexc.c:138}
}%
\end{column}
\end{columns}
\footnotetext[1]{\url{https://v2.ocaml.org/api/Printexc.html\#1\_Uncaughtexceptions}}
\end{frame}
}%
\onslide*<6->{\providecommand\step{4}%
% Runtime's default exception handler
%
\subsection*{Runtime's default exception handler}
\frameSubsection{pictures/The\_Architect.png}{}


\begin{frame}{Default exception handler}
\begin{columns}
\begin{column}{0.5\textwidth}
\begin{itemize}
\item \localname{domain\_state->exn\_handler} points to a trap located before \funcname{entry}!
\item What installed this very first trap there?
\end{itemize}
\bigskip
\listocaml{../src/default/default.ml}{default/default.ml}
\bigskip
\inputminted[fontsize=\footnotesize]{text}{../src/default/default.out}
\end{column}
\begin{column}{0.5\textwidth}
\centering
\only<1>{\providecommand\step{0}\input{diagrams/default/default.tex}}%
\only<2>{\providecommand\step{4}\input{diagrams/default/default.tex}}%
\end{column}
\end{columns}
\end{frame}


\begin{frame}{Startup}{How did we get there by the way?}
\begin{columns}
\begin{column}{0.45\textwidth}
\begin{itemize}
\item The entry point address of the binary is \funcname{main} (\funcname{\_start}) from the runtime
\item The program starts like a C program:
\begin{itemize}
\item \funcname{caml\_start\_program}
\item \funcname{caml\_startup\_common}
\item \funcname{caml\_startup\_exn}
\item \funcname{caml\_startup}
\item \funcname{caml\_main}
\item \funcname{main}
\end{itemize}
%\item \funcname{caml\_startup\_common}: initializes GC, domains \dots
% Also: codefrag, locale, custom opeartions, frame_descr, signals, stack
\item \funcname{caml\_start\_program} is the transition point between the C realm and the OCaml one
\end{itemize}
\bigskip
\listocaml{../src/default/default.ml}{default/default.ml}
\end{column}
\begin{column}{0.55\textwidth}
\listasm[lastline=24,highlightlines={22-24}]{src/ocaml/caml\_start\_program.s}{runtime/amd64.S:604}
\end{column}
\end{columns}
\end{frame}


\begin{frame}{Setup to jump into OCaml entry point}{\funcname{caml\_start\_program} initializes the main fiber}
\begin{columns}
\begin{column}{0.6\textwidth}
\only<1,3,5>{
\begin{itemize}
\item<1-> Stores information to allow switching from OCaml stack to the C stack
\item<3-> Constructs the default exception handler
\item<5-> Switches to the OCaml stack and calls \funcname{caml\_program}
\end{itemize}
\bigskip
\listocaml{../src/default/default.ml}{default/default.ml}
}%
\only<2>{
\listasm[firstline=22,lastline=41,highlightlines={25-27}]{src/ocaml/caml\_start\_program.s}{caml\_start\_program}
}%
\only<4>{
\mintasm[firstline=22,lastline=41,highlightlines={31-38}]{src/ocaml/caml\_start\_program.s}
\listasm[firstline=66,lastline=72]{src/ocaml/caml\_start\_program.s}{caml\_start\_program}
}%
\only<6>{
\listasm[firstline=22,lastline=41,highlightlines={39-41}]{src/ocaml/caml\_start\_program.s}{caml\_start\_program}
}%
\end{column}
\begin{column}{0.4\textwidth}
\centering
\only<1-2>{\providecommand\step{2}\input{diagrams/default/default.tex}}%
\only<3-5>{\providecommand\step{3}\input{diagrams/default/default.tex}}%
\onslide*<6->{\providecommand\step{4}\input{diagrams/default/default.tex}}%
\end{column}
\end{columns}
\end{frame}

\begin{frame}{Executing the OCaml program}{And raising an exception without handler}
\begin{columns}
\begin{column}{0.6\textwidth}
\only<1-3>{
\begin{itemize}
\item<1-> Execution continues with OCaml code
\begin{itemize}
\item \funcname{camlDefault\_\_main\_269}
\item \funcname{camlDefault\_\_entry}
\item \funcname{caml\_program}
\end{itemize}
\item<1-> \funcname{camlDefault\_\_main\_269} raises a \typename{Failure} exception
\item<1-> \typename{Failure} is caught by \funcname{caml\_start\_program}
\begin{itemize}
\item<1-> The handler set the 2nd LSB of the exception value to \funcarg{1}
\item<3-> And then forces \funcname{caml\_start\_program} to terminate
\end{itemize}
\end{itemize}
}
\bigskip
\only<1,3>{\listocaml[highlightlines=3]{../src/default/default.ml}{default/default.ml}}%
\only<2>{\listasm[firstline=66,lastline=72,highlightlines=69]{src/ocaml/caml\_start\_program.s}{caml\_start\_program}}%
\end{column}
\begin{column}{0.4\textwidth}
\centering
\providecommand\step{4}\input{diagrams/default/default.tex}
\end{column}
\end{columns}
\end{frame}
%\only<>{\listasm[firstline=47,lastline=65]{src/ocaml/caml\_start\_program.s}{caml\_start\_program}}


\begin{frame}{Back to C}{Clean up, complain and terminate}
\begin{columns}
\begin{column}{0.45\textwidth}
\begin{itemize}
\item Backtrace:
\begin{itemize}
\item \funcname{caml\_start\_program} $\rightarrow$ OCaml code
\item \funcname{caml\_startup\_common}
\item \funcname{caml\_startup\_exn}
\item \funcname{caml\_startup}
\item \funcname{caml\_main}
\item \funcname{main}
\end{itemize}
\smallskip
\item \funcname{caml\_startup} checks the return value of \funcname{caml\_startup\_exn}
\begin{itemize}
\item The return value has been specially marked by \funcname{caml\_start\_program} handler
\item Calls \funcname{caml\_fatal\_uncaught\_exception}
\end{itemize}
\item<2-> \funcname{caml\_fatal\_uncaught\_exception} either
\begin{itemize}
\item<2-> Calls the user custom uncaught exception handler, if set with \mintinline{ocaml}{Printexc.set_uncaught_exception_handler fn}\footnotemark
\item<2-> Or falls back to the default one
\item<2-> Which notifies about the uncaught exception
\end{itemize}
\item<4-> Terminates the program either with \funcname{abort} or with a non-zero exit code
\end{itemize}
\end{column}
\begin{column}{0.55\textwidth}
\only<1>{
\listc[firstline=84,lastline=89,highlightlines=88]{../ocaml/runtime/caml/mlvalues.h}{runtime/caml/mlvalues.h:84}
\listc[firstline=138,lastline=143,highlightlines={141-142}]{../ocaml/runtime/startup\_nat.c}{runtime/startup\_nat.c:138}
}%
\only<2>{
\listc[firstline=138,highlightlines={147,149}]{../ocaml/runtime/printexc.c}{runtime/printexc.c:138}
}%
\only<3>{
\listc[firstline=110,lastline=134,highlightlines=129]{../ocaml/runtime/printexc.c}{runtime/printexc.c:138}
}%
\only<4>{
\listc[firstline=138,highlightlines={152,154}]{../ocaml/runtime/printexc.c}{runtime/printexc.c:138}
}%
\end{column}
\end{columns}
\footnotetext[1]{\url{https://v2.ocaml.org/api/Printexc.html\#1\_Uncaughtexceptions}}
\end{frame}
}%
\end{column}
\end{columns}
\end{frame}

\begin{frame}{Executing the OCaml program}{And raising an exception without handler}
\begin{columns}
\begin{column}{0.6\textwidth}
\only<1-3>{
\begin{itemize}
\item<1-> Execution continues with OCaml code
\begin{itemize}
\item \funcname{camlDefault\_\_main\_269}
\item \funcname{camlDefault\_\_entry}
\item \funcname{caml\_program}
\end{itemize}
\item<1-> \funcname{camlDefault\_\_main\_269} raises a \typename{Failure} exception
\item<1-> \typename{Failure} is caught by \funcname{caml\_start\_program}
\begin{itemize}
\item<1-> The handler set the 2nd LSB of the exception value to \funcarg{1}
\item<3-> And then forces \funcname{caml\_start\_program} to terminate
\end{itemize}
\end{itemize}
}
\bigskip
\only<1,3>{\listocaml[highlightlines=3]{../src/default/default.ml}{default/default.ml}}%
\only<2>{\listasm[firstline=66,lastline=72,highlightlines=69]{src/ocaml/caml\_start\_program.s}{caml\_start\_program}}%
\end{column}
\begin{column}{0.4\textwidth}
\centering
\providecommand\step{4}%
% Runtime's default exception handler
%
\subsection*{Runtime's default exception handler}
\frameSubsection{pictures/The\_Architect.png}{}


\begin{frame}{Default exception handler}
\begin{columns}
\begin{column}{0.5\textwidth}
\begin{itemize}
\item \localname{domain\_state->exn\_handler} points to a trap located before \funcname{entry}!
\item What installed this very first trap there?
\end{itemize}
\bigskip
\listocaml{../src/default/default.ml}{default/default.ml}
\bigskip
\inputminted[fontsize=\footnotesize]{text}{../src/default/default.out}
\end{column}
\begin{column}{0.5\textwidth}
\centering
\only<1>{\providecommand\step{0}\input{diagrams/default/default.tex}}%
\only<2>{\providecommand\step{4}\input{diagrams/default/default.tex}}%
\end{column}
\end{columns}
\end{frame}


\begin{frame}{Startup}{How did we get there by the way?}
\begin{columns}
\begin{column}{0.45\textwidth}
\begin{itemize}
\item The entry point address of the binary is \funcname{main} (\funcname{\_start}) from the runtime
\item The program starts like a C program:
\begin{itemize}
\item \funcname{caml\_start\_program}
\item \funcname{caml\_startup\_common}
\item \funcname{caml\_startup\_exn}
\item \funcname{caml\_startup}
\item \funcname{caml\_main}
\item \funcname{main}
\end{itemize}
%\item \funcname{caml\_startup\_common}: initializes GC, domains \dots
% Also: codefrag, locale, custom opeartions, frame_descr, signals, stack
\item \funcname{caml\_start\_program} is the transition point between the C realm and the OCaml one
\end{itemize}
\bigskip
\listocaml{../src/default/default.ml}{default/default.ml}
\end{column}
\begin{column}{0.55\textwidth}
\listasm[lastline=24,highlightlines={22-24}]{src/ocaml/caml\_start\_program.s}{runtime/amd64.S:604}
\end{column}
\end{columns}
\end{frame}


\begin{frame}{Setup to jump into OCaml entry point}{\funcname{caml\_start\_program} initializes the main fiber}
\begin{columns}
\begin{column}{0.6\textwidth}
\only<1,3,5>{
\begin{itemize}
\item<1-> Stores information to allow switching from OCaml stack to the C stack
\item<3-> Constructs the default exception handler
\item<5-> Switches to the OCaml stack and calls \funcname{caml\_program}
\end{itemize}
\bigskip
\listocaml{../src/default/default.ml}{default/default.ml}
}%
\only<2>{
\listasm[firstline=22,lastline=41,highlightlines={25-27}]{src/ocaml/caml\_start\_program.s}{caml\_start\_program}
}%
\only<4>{
\mintasm[firstline=22,lastline=41,highlightlines={31-38}]{src/ocaml/caml\_start\_program.s}
\listasm[firstline=66,lastline=72]{src/ocaml/caml\_start\_program.s}{caml\_start\_program}
}%
\only<6>{
\listasm[firstline=22,lastline=41,highlightlines={39-41}]{src/ocaml/caml\_start\_program.s}{caml\_start\_program}
}%
\end{column}
\begin{column}{0.4\textwidth}
\centering
\only<1-2>{\providecommand\step{2}\input{diagrams/default/default.tex}}%
\only<3-5>{\providecommand\step{3}\input{diagrams/default/default.tex}}%
\onslide*<6->{\providecommand\step{4}\input{diagrams/default/default.tex}}%
\end{column}
\end{columns}
\end{frame}

\begin{frame}{Executing the OCaml program}{And raising an exception without handler}
\begin{columns}
\begin{column}{0.6\textwidth}
\only<1-3>{
\begin{itemize}
\item<1-> Execution continues with OCaml code
\begin{itemize}
\item \funcname{camlDefault\_\_main\_269}
\item \funcname{camlDefault\_\_entry}
\item \funcname{caml\_program}
\end{itemize}
\item<1-> \funcname{camlDefault\_\_main\_269} raises a \typename{Failure} exception
\item<1-> \typename{Failure} is caught by \funcname{caml\_start\_program}
\begin{itemize}
\item<1-> The handler set the 2nd LSB of the exception value to \funcarg{1}
\item<3-> And then forces \funcname{caml\_start\_program} to terminate
\end{itemize}
\end{itemize}
}
\bigskip
\only<1,3>{\listocaml[highlightlines=3]{../src/default/default.ml}{default/default.ml}}%
\only<2>{\listasm[firstline=66,lastline=72,highlightlines=69]{src/ocaml/caml\_start\_program.s}{caml\_start\_program}}%
\end{column}
\begin{column}{0.4\textwidth}
\centering
\providecommand\step{4}\input{diagrams/default/default.tex}
\end{column}
\end{columns}
\end{frame}
%\only<>{\listasm[firstline=47,lastline=65]{src/ocaml/caml\_start\_program.s}{caml\_start\_program}}


\begin{frame}{Back to C}{Clean up, complain and terminate}
\begin{columns}
\begin{column}{0.45\textwidth}
\begin{itemize}
\item Backtrace:
\begin{itemize}
\item \funcname{caml\_start\_program} $\rightarrow$ OCaml code
\item \funcname{caml\_startup\_common}
\item \funcname{caml\_startup\_exn}
\item \funcname{caml\_startup}
\item \funcname{caml\_main}
\item \funcname{main}
\end{itemize}
\smallskip
\item \funcname{caml\_startup} checks the return value of \funcname{caml\_startup\_exn}
\begin{itemize}
\item The return value has been specially marked by \funcname{caml\_start\_program} handler
\item Calls \funcname{caml\_fatal\_uncaught\_exception}
\end{itemize}
\item<2-> \funcname{caml\_fatal\_uncaught\_exception} either
\begin{itemize}
\item<2-> Calls the user custom uncaught exception handler, if set with \mintinline{ocaml}{Printexc.set_uncaught_exception_handler fn}\footnotemark
\item<2-> Or falls back to the default one
\item<2-> Which notifies about the uncaught exception
\end{itemize}
\item<4-> Terminates the program either with \funcname{abort} or with a non-zero exit code
\end{itemize}
\end{column}
\begin{column}{0.55\textwidth}
\only<1>{
\listc[firstline=84,lastline=89,highlightlines=88]{../ocaml/runtime/caml/mlvalues.h}{runtime/caml/mlvalues.h:84}
\listc[firstline=138,lastline=143,highlightlines={141-142}]{../ocaml/runtime/startup\_nat.c}{runtime/startup\_nat.c:138}
}%
\only<2>{
\listc[firstline=138,highlightlines={147,149}]{../ocaml/runtime/printexc.c}{runtime/printexc.c:138}
}%
\only<3>{
\listc[firstline=110,lastline=134,highlightlines=129]{../ocaml/runtime/printexc.c}{runtime/printexc.c:138}
}%
\only<4>{
\listc[firstline=138,highlightlines={152,154}]{../ocaml/runtime/printexc.c}{runtime/printexc.c:138}
}%
\end{column}
\end{columns}
\footnotetext[1]{\url{https://v2.ocaml.org/api/Printexc.html\#1\_Uncaughtexceptions}}
\end{frame}

\end{column}
\end{columns}
\end{frame}
%\only<>{\listasm[firstline=47,lastline=65]{src/ocaml/caml\_start\_program.s}{caml\_start\_program}}


\begin{frame}{Back to C}{Clean up, complain and terminate}
\begin{columns}
\begin{column}{0.45\textwidth}
\begin{itemize}
\item Backtrace:
\begin{itemize}
\item \funcname{caml\_start\_program} $\rightarrow$ OCaml code
\item \funcname{caml\_startup\_common}
\item \funcname{caml\_startup\_exn}
\item \funcname{caml\_startup}
\item \funcname{caml\_main}
\item \funcname{main}
\end{itemize}
\smallskip
\item \funcname{caml\_startup} checks the return value of \funcname{caml\_startup\_exn}
\begin{itemize}
\item The return value has been specially marked by \funcname{caml\_start\_program} handler
\item Calls \funcname{caml\_fatal\_uncaught\_exception}
\end{itemize}
\item<2-> \funcname{caml\_fatal\_uncaught\_exception} either
\begin{itemize}
\item<2-> Calls the user custom uncaught exception handler, if set with \mintinline{ocaml}{Printexc.set_uncaught_exception_handler fn}\footnotemark
\item<2-> Or falls back to the default one
\item<2-> Which notifies about the uncaught exception
\end{itemize}
\item<4-> Terminates the program either with \funcname{abort} or with a non-zero exit code
\end{itemize}
\end{column}
\begin{column}{0.55\textwidth}
\only<1>{
\listc[firstline=84,lastline=89,highlightlines=88]{../ocaml/runtime/caml/mlvalues.h}{runtime/caml/mlvalues.h:84}
\listc[firstline=138,lastline=143,highlightlines={141-142}]{../ocaml/runtime/startup\_nat.c}{runtime/startup\_nat.c:138}
}%
\only<2>{
\listc[firstline=138,highlightlines={147,149}]{../ocaml/runtime/printexc.c}{runtime/printexc.c:138}
}%
\only<3>{
\listc[firstline=110,lastline=134,highlightlines=129]{../ocaml/runtime/printexc.c}{runtime/printexc.c:138}
}%
\only<4>{
\listc[firstline=138,highlightlines={152,154}]{../ocaml/runtime/printexc.c}{runtime/printexc.c:138}
}%
\end{column}
\end{columns}
\footnotetext[1]{\url{https://v2.ocaml.org/api/Printexc.html\#1\_Uncaughtexceptions}}
\end{frame}
}%
\onslide*<6->{\providecommand\step{4}%
% Runtime's default exception handler
%
\subsection*{Runtime's default exception handler}
\frameSubsection{pictures/The\_Architect.png}{}


\begin{frame}{Default exception handler}
\begin{columns}
\begin{column}{0.5\textwidth}
\begin{itemize}
\item \localname{domain\_state->exn\_handler} points to a trap located before \funcname{entry}!
\item What installed this very first trap there?
\end{itemize}
\bigskip
\listocaml{../src/default/default.ml}{default/default.ml}
\bigskip
\inputminted[fontsize=\footnotesize]{text}{../src/default/default.out}
\end{column}
\begin{column}{0.5\textwidth}
\centering
\only<1>{\providecommand\step{0}%
% Runtime's default exception handler
%
\subsection*{Runtime's default exception handler}
\frameSubsection{pictures/The\_Architect.png}{}


\begin{frame}{Default exception handler}
\begin{columns}
\begin{column}{0.5\textwidth}
\begin{itemize}
\item \localname{domain\_state->exn\_handler} points to a trap located before \funcname{entry}!
\item What installed this very first trap there?
\end{itemize}
\bigskip
\listocaml{../src/default/default.ml}{default/default.ml}
\bigskip
\inputminted[fontsize=\footnotesize]{text}{../src/default/default.out}
\end{column}
\begin{column}{0.5\textwidth}
\centering
\only<1>{\providecommand\step{0}\input{diagrams/default/default.tex}}%
\only<2>{\providecommand\step{4}\input{diagrams/default/default.tex}}%
\end{column}
\end{columns}
\end{frame}


\begin{frame}{Startup}{How did we get there by the way?}
\begin{columns}
\begin{column}{0.45\textwidth}
\begin{itemize}
\item The entry point address of the binary is \funcname{main} (\funcname{\_start}) from the runtime
\item The program starts like a C program:
\begin{itemize}
\item \funcname{caml\_start\_program}
\item \funcname{caml\_startup\_common}
\item \funcname{caml\_startup\_exn}
\item \funcname{caml\_startup}
\item \funcname{caml\_main}
\item \funcname{main}
\end{itemize}
%\item \funcname{caml\_startup\_common}: initializes GC, domains \dots
% Also: codefrag, locale, custom opeartions, frame_descr, signals, stack
\item \funcname{caml\_start\_program} is the transition point between the C realm and the OCaml one
\end{itemize}
\bigskip
\listocaml{../src/default/default.ml}{default/default.ml}
\end{column}
\begin{column}{0.55\textwidth}
\listasm[lastline=24,highlightlines={22-24}]{src/ocaml/caml\_start\_program.s}{runtime/amd64.S:604}
\end{column}
\end{columns}
\end{frame}


\begin{frame}{Setup to jump into OCaml entry point}{\funcname{caml\_start\_program} initializes the main fiber}
\begin{columns}
\begin{column}{0.6\textwidth}
\only<1,3,5>{
\begin{itemize}
\item<1-> Stores information to allow switching from OCaml stack to the C stack
\item<3-> Constructs the default exception handler
\item<5-> Switches to the OCaml stack and calls \funcname{caml\_program}
\end{itemize}
\bigskip
\listocaml{../src/default/default.ml}{default/default.ml}
}%
\only<2>{
\listasm[firstline=22,lastline=41,highlightlines={25-27}]{src/ocaml/caml\_start\_program.s}{caml\_start\_program}
}%
\only<4>{
\mintasm[firstline=22,lastline=41,highlightlines={31-38}]{src/ocaml/caml\_start\_program.s}
\listasm[firstline=66,lastline=72]{src/ocaml/caml\_start\_program.s}{caml\_start\_program}
}%
\only<6>{
\listasm[firstline=22,lastline=41,highlightlines={39-41}]{src/ocaml/caml\_start\_program.s}{caml\_start\_program}
}%
\end{column}
\begin{column}{0.4\textwidth}
\centering
\only<1-2>{\providecommand\step{2}\input{diagrams/default/default.tex}}%
\only<3-5>{\providecommand\step{3}\input{diagrams/default/default.tex}}%
\onslide*<6->{\providecommand\step{4}\input{diagrams/default/default.tex}}%
\end{column}
\end{columns}
\end{frame}

\begin{frame}{Executing the OCaml program}{And raising an exception without handler}
\begin{columns}
\begin{column}{0.6\textwidth}
\only<1-3>{
\begin{itemize}
\item<1-> Execution continues with OCaml code
\begin{itemize}
\item \funcname{camlDefault\_\_main\_269}
\item \funcname{camlDefault\_\_entry}
\item \funcname{caml\_program}
\end{itemize}
\item<1-> \funcname{camlDefault\_\_main\_269} raises a \typename{Failure} exception
\item<1-> \typename{Failure} is caught by \funcname{caml\_start\_program}
\begin{itemize}
\item<1-> The handler set the 2nd LSB of the exception value to \funcarg{1}
\item<3-> And then forces \funcname{caml\_start\_program} to terminate
\end{itemize}
\end{itemize}
}
\bigskip
\only<1,3>{\listocaml[highlightlines=3]{../src/default/default.ml}{default/default.ml}}%
\only<2>{\listasm[firstline=66,lastline=72,highlightlines=69]{src/ocaml/caml\_start\_program.s}{caml\_start\_program}}%
\end{column}
\begin{column}{0.4\textwidth}
\centering
\providecommand\step{4}\input{diagrams/default/default.tex}
\end{column}
\end{columns}
\end{frame}
%\only<>{\listasm[firstline=47,lastline=65]{src/ocaml/caml\_start\_program.s}{caml\_start\_program}}


\begin{frame}{Back to C}{Clean up, complain and terminate}
\begin{columns}
\begin{column}{0.45\textwidth}
\begin{itemize}
\item Backtrace:
\begin{itemize}
\item \funcname{caml\_start\_program} $\rightarrow$ OCaml code
\item \funcname{caml\_startup\_common}
\item \funcname{caml\_startup\_exn}
\item \funcname{caml\_startup}
\item \funcname{caml\_main}
\item \funcname{main}
\end{itemize}
\smallskip
\item \funcname{caml\_startup} checks the return value of \funcname{caml\_startup\_exn}
\begin{itemize}
\item The return value has been specially marked by \funcname{caml\_start\_program} handler
\item Calls \funcname{caml\_fatal\_uncaught\_exception}
\end{itemize}
\item<2-> \funcname{caml\_fatal\_uncaught\_exception} either
\begin{itemize}
\item<2-> Calls the user custom uncaught exception handler, if set with \mintinline{ocaml}{Printexc.set_uncaught_exception_handler fn}\footnotemark
\item<2-> Or falls back to the default one
\item<2-> Which notifies about the uncaught exception
\end{itemize}
\item<4-> Terminates the program either with \funcname{abort} or with a non-zero exit code
\end{itemize}
\end{column}
\begin{column}{0.55\textwidth}
\only<1>{
\listc[firstline=84,lastline=89,highlightlines=88]{../ocaml/runtime/caml/mlvalues.h}{runtime/caml/mlvalues.h:84}
\listc[firstline=138,lastline=143,highlightlines={141-142}]{../ocaml/runtime/startup\_nat.c}{runtime/startup\_nat.c:138}
}%
\only<2>{
\listc[firstline=138,highlightlines={147,149}]{../ocaml/runtime/printexc.c}{runtime/printexc.c:138}
}%
\only<3>{
\listc[firstline=110,lastline=134,highlightlines=129]{../ocaml/runtime/printexc.c}{runtime/printexc.c:138}
}%
\only<4>{
\listc[firstline=138,highlightlines={152,154}]{../ocaml/runtime/printexc.c}{runtime/printexc.c:138}
}%
\end{column}
\end{columns}
\footnotetext[1]{\url{https://v2.ocaml.org/api/Printexc.html\#1\_Uncaughtexceptions}}
\end{frame}
}%
\only<2>{\providecommand\step{4}%
% Runtime's default exception handler
%
\subsection*{Runtime's default exception handler}
\frameSubsection{pictures/The\_Architect.png}{}


\begin{frame}{Default exception handler}
\begin{columns}
\begin{column}{0.5\textwidth}
\begin{itemize}
\item \localname{domain\_state->exn\_handler} points to a trap located before \funcname{entry}!
\item What installed this very first trap there?
\end{itemize}
\bigskip
\listocaml{../src/default/default.ml}{default/default.ml}
\bigskip
\inputminted[fontsize=\footnotesize]{text}{../src/default/default.out}
\end{column}
\begin{column}{0.5\textwidth}
\centering
\only<1>{\providecommand\step{0}\input{diagrams/default/default.tex}}%
\only<2>{\providecommand\step{4}\input{diagrams/default/default.tex}}%
\end{column}
\end{columns}
\end{frame}


\begin{frame}{Startup}{How did we get there by the way?}
\begin{columns}
\begin{column}{0.45\textwidth}
\begin{itemize}
\item The entry point address of the binary is \funcname{main} (\funcname{\_start}) from the runtime
\item The program starts like a C program:
\begin{itemize}
\item \funcname{caml\_start\_program}
\item \funcname{caml\_startup\_common}
\item \funcname{caml\_startup\_exn}
\item \funcname{caml\_startup}
\item \funcname{caml\_main}
\item \funcname{main}
\end{itemize}
%\item \funcname{caml\_startup\_common}: initializes GC, domains \dots
% Also: codefrag, locale, custom opeartions, frame_descr, signals, stack
\item \funcname{caml\_start\_program} is the transition point between the C realm and the OCaml one
\end{itemize}
\bigskip
\listocaml{../src/default/default.ml}{default/default.ml}
\end{column}
\begin{column}{0.55\textwidth}
\listasm[lastline=24,highlightlines={22-24}]{src/ocaml/caml\_start\_program.s}{runtime/amd64.S:604}
\end{column}
\end{columns}
\end{frame}


\begin{frame}{Setup to jump into OCaml entry point}{\funcname{caml\_start\_program} initializes the main fiber}
\begin{columns}
\begin{column}{0.6\textwidth}
\only<1,3,5>{
\begin{itemize}
\item<1-> Stores information to allow switching from OCaml stack to the C stack
\item<3-> Constructs the default exception handler
\item<5-> Switches to the OCaml stack and calls \funcname{caml\_program}
\end{itemize}
\bigskip
\listocaml{../src/default/default.ml}{default/default.ml}
}%
\only<2>{
\listasm[firstline=22,lastline=41,highlightlines={25-27}]{src/ocaml/caml\_start\_program.s}{caml\_start\_program}
}%
\only<4>{
\mintasm[firstline=22,lastline=41,highlightlines={31-38}]{src/ocaml/caml\_start\_program.s}
\listasm[firstline=66,lastline=72]{src/ocaml/caml\_start\_program.s}{caml\_start\_program}
}%
\only<6>{
\listasm[firstline=22,lastline=41,highlightlines={39-41}]{src/ocaml/caml\_start\_program.s}{caml\_start\_program}
}%
\end{column}
\begin{column}{0.4\textwidth}
\centering
\only<1-2>{\providecommand\step{2}\input{diagrams/default/default.tex}}%
\only<3-5>{\providecommand\step{3}\input{diagrams/default/default.tex}}%
\onslide*<6->{\providecommand\step{4}\input{diagrams/default/default.tex}}%
\end{column}
\end{columns}
\end{frame}

\begin{frame}{Executing the OCaml program}{And raising an exception without handler}
\begin{columns}
\begin{column}{0.6\textwidth}
\only<1-3>{
\begin{itemize}
\item<1-> Execution continues with OCaml code
\begin{itemize}
\item \funcname{camlDefault\_\_main\_269}
\item \funcname{camlDefault\_\_entry}
\item \funcname{caml\_program}
\end{itemize}
\item<1-> \funcname{camlDefault\_\_main\_269} raises a \typename{Failure} exception
\item<1-> \typename{Failure} is caught by \funcname{caml\_start\_program}
\begin{itemize}
\item<1-> The handler set the 2nd LSB of the exception value to \funcarg{1}
\item<3-> And then forces \funcname{caml\_start\_program} to terminate
\end{itemize}
\end{itemize}
}
\bigskip
\only<1,3>{\listocaml[highlightlines=3]{../src/default/default.ml}{default/default.ml}}%
\only<2>{\listasm[firstline=66,lastline=72,highlightlines=69]{src/ocaml/caml\_start\_program.s}{caml\_start\_program}}%
\end{column}
\begin{column}{0.4\textwidth}
\centering
\providecommand\step{4}\input{diagrams/default/default.tex}
\end{column}
\end{columns}
\end{frame}
%\only<>{\listasm[firstline=47,lastline=65]{src/ocaml/caml\_start\_program.s}{caml\_start\_program}}


\begin{frame}{Back to C}{Clean up, complain and terminate}
\begin{columns}
\begin{column}{0.45\textwidth}
\begin{itemize}
\item Backtrace:
\begin{itemize}
\item \funcname{caml\_start\_program} $\rightarrow$ OCaml code
\item \funcname{caml\_startup\_common}
\item \funcname{caml\_startup\_exn}
\item \funcname{caml\_startup}
\item \funcname{caml\_main}
\item \funcname{main}
\end{itemize}
\smallskip
\item \funcname{caml\_startup} checks the return value of \funcname{caml\_startup\_exn}
\begin{itemize}
\item The return value has been specially marked by \funcname{caml\_start\_program} handler
\item Calls \funcname{caml\_fatal\_uncaught\_exception}
\end{itemize}
\item<2-> \funcname{caml\_fatal\_uncaught\_exception} either
\begin{itemize}
\item<2-> Calls the user custom uncaught exception handler, if set with \mintinline{ocaml}{Printexc.set_uncaught_exception_handler fn}\footnotemark
\item<2-> Or falls back to the default one
\item<2-> Which notifies about the uncaught exception
\end{itemize}
\item<4-> Terminates the program either with \funcname{abort} or with a non-zero exit code
\end{itemize}
\end{column}
\begin{column}{0.55\textwidth}
\only<1>{
\listc[firstline=84,lastline=89,highlightlines=88]{../ocaml/runtime/caml/mlvalues.h}{runtime/caml/mlvalues.h:84}
\listc[firstline=138,lastline=143,highlightlines={141-142}]{../ocaml/runtime/startup\_nat.c}{runtime/startup\_nat.c:138}
}%
\only<2>{
\listc[firstline=138,highlightlines={147,149}]{../ocaml/runtime/printexc.c}{runtime/printexc.c:138}
}%
\only<3>{
\listc[firstline=110,lastline=134,highlightlines=129]{../ocaml/runtime/printexc.c}{runtime/printexc.c:138}
}%
\only<4>{
\listc[firstline=138,highlightlines={152,154}]{../ocaml/runtime/printexc.c}{runtime/printexc.c:138}
}%
\end{column}
\end{columns}
\footnotetext[1]{\url{https://v2.ocaml.org/api/Printexc.html\#1\_Uncaughtexceptions}}
\end{frame}
}%
\end{column}
\end{columns}
\end{frame}


\begin{frame}{Startup}{How did we get there by the way?}
\begin{columns}
\begin{column}{0.45\textwidth}
\begin{itemize}
\item The entry point address of the binary is \funcname{main} (\funcname{\_start}) from the runtime
\item The program starts like a C program:
\begin{itemize}
\item \funcname{caml\_start\_program}
\item \funcname{caml\_startup\_common}
\item \funcname{caml\_startup\_exn}
\item \funcname{caml\_startup}
\item \funcname{caml\_main}
\item \funcname{main}
\end{itemize}
%\item \funcname{caml\_startup\_common}: initializes GC, domains \dots
% Also: codefrag, locale, custom opeartions, frame_descr, signals, stack
\item \funcname{caml\_start\_program} is the transition point between the C realm and the OCaml one
\end{itemize}
\bigskip
\listocaml{../src/default/default.ml}{default/default.ml}
\end{column}
\begin{column}{0.55\textwidth}
\listasm[lastline=24,highlightlines={22-24}]{src/ocaml/caml\_start\_program.s}{runtime/amd64.S:604}
\end{column}
\end{columns}
\end{frame}


\begin{frame}{Setup to jump into OCaml entry point}{\funcname{caml\_start\_program} initializes the main fiber}
\begin{columns}
\begin{column}{0.6\textwidth}
\only<1,3,5>{
\begin{itemize}
\item<1-> Stores information to allow switching from OCaml stack to the C stack
\item<3-> Constructs the default exception handler
\item<5-> Switches to the OCaml stack and calls \funcname{caml\_program}
\end{itemize}
\bigskip
\listocaml{../src/default/default.ml}{default/default.ml}
}%
\only<2>{
\listasm[firstline=22,lastline=41,highlightlines={25-27}]{src/ocaml/caml\_start\_program.s}{caml\_start\_program}
}%
\only<4>{
\mintasm[firstline=22,lastline=41,highlightlines={31-38}]{src/ocaml/caml\_start\_program.s}
\listasm[firstline=66,lastline=72]{src/ocaml/caml\_start\_program.s}{caml\_start\_program}
}%
\only<6>{
\listasm[firstline=22,lastline=41,highlightlines={39-41}]{src/ocaml/caml\_start\_program.s}{caml\_start\_program}
}%
\end{column}
\begin{column}{0.4\textwidth}
\centering
\only<1-2>{\providecommand\step{2}%
% Runtime's default exception handler
%
\subsection*{Runtime's default exception handler}
\frameSubsection{pictures/The\_Architect.png}{}


\begin{frame}{Default exception handler}
\begin{columns}
\begin{column}{0.5\textwidth}
\begin{itemize}
\item \localname{domain\_state->exn\_handler} points to a trap located before \funcname{entry}!
\item What installed this very first trap there?
\end{itemize}
\bigskip
\listocaml{../src/default/default.ml}{default/default.ml}
\bigskip
\inputminted[fontsize=\footnotesize]{text}{../src/default/default.out}
\end{column}
\begin{column}{0.5\textwidth}
\centering
\only<1>{\providecommand\step{0}\input{diagrams/default/default.tex}}%
\only<2>{\providecommand\step{4}\input{diagrams/default/default.tex}}%
\end{column}
\end{columns}
\end{frame}


\begin{frame}{Startup}{How did we get there by the way?}
\begin{columns}
\begin{column}{0.45\textwidth}
\begin{itemize}
\item The entry point address of the binary is \funcname{main} (\funcname{\_start}) from the runtime
\item The program starts like a C program:
\begin{itemize}
\item \funcname{caml\_start\_program}
\item \funcname{caml\_startup\_common}
\item \funcname{caml\_startup\_exn}
\item \funcname{caml\_startup}
\item \funcname{caml\_main}
\item \funcname{main}
\end{itemize}
%\item \funcname{caml\_startup\_common}: initializes GC, domains \dots
% Also: codefrag, locale, custom opeartions, frame_descr, signals, stack
\item \funcname{caml\_start\_program} is the transition point between the C realm and the OCaml one
\end{itemize}
\bigskip
\listocaml{../src/default/default.ml}{default/default.ml}
\end{column}
\begin{column}{0.55\textwidth}
\listasm[lastline=24,highlightlines={22-24}]{src/ocaml/caml\_start\_program.s}{runtime/amd64.S:604}
\end{column}
\end{columns}
\end{frame}


\begin{frame}{Setup to jump into OCaml entry point}{\funcname{caml\_start\_program} initializes the main fiber}
\begin{columns}
\begin{column}{0.6\textwidth}
\only<1,3,5>{
\begin{itemize}
\item<1-> Stores information to allow switching from OCaml stack to the C stack
\item<3-> Constructs the default exception handler
\item<5-> Switches to the OCaml stack and calls \funcname{caml\_program}
\end{itemize}
\bigskip
\listocaml{../src/default/default.ml}{default/default.ml}
}%
\only<2>{
\listasm[firstline=22,lastline=41,highlightlines={25-27}]{src/ocaml/caml\_start\_program.s}{caml\_start\_program}
}%
\only<4>{
\mintasm[firstline=22,lastline=41,highlightlines={31-38}]{src/ocaml/caml\_start\_program.s}
\listasm[firstline=66,lastline=72]{src/ocaml/caml\_start\_program.s}{caml\_start\_program}
}%
\only<6>{
\listasm[firstline=22,lastline=41,highlightlines={39-41}]{src/ocaml/caml\_start\_program.s}{caml\_start\_program}
}%
\end{column}
\begin{column}{0.4\textwidth}
\centering
\only<1-2>{\providecommand\step{2}\input{diagrams/default/default.tex}}%
\only<3-5>{\providecommand\step{3}\input{diagrams/default/default.tex}}%
\onslide*<6->{\providecommand\step{4}\input{diagrams/default/default.tex}}%
\end{column}
\end{columns}
\end{frame}

\begin{frame}{Executing the OCaml program}{And raising an exception without handler}
\begin{columns}
\begin{column}{0.6\textwidth}
\only<1-3>{
\begin{itemize}
\item<1-> Execution continues with OCaml code
\begin{itemize}
\item \funcname{camlDefault\_\_main\_269}
\item \funcname{camlDefault\_\_entry}
\item \funcname{caml\_program}
\end{itemize}
\item<1-> \funcname{camlDefault\_\_main\_269} raises a \typename{Failure} exception
\item<1-> \typename{Failure} is caught by \funcname{caml\_start\_program}
\begin{itemize}
\item<1-> The handler set the 2nd LSB of the exception value to \funcarg{1}
\item<3-> And then forces \funcname{caml\_start\_program} to terminate
\end{itemize}
\end{itemize}
}
\bigskip
\only<1,3>{\listocaml[highlightlines=3]{../src/default/default.ml}{default/default.ml}}%
\only<2>{\listasm[firstline=66,lastline=72,highlightlines=69]{src/ocaml/caml\_start\_program.s}{caml\_start\_program}}%
\end{column}
\begin{column}{0.4\textwidth}
\centering
\providecommand\step{4}\input{diagrams/default/default.tex}
\end{column}
\end{columns}
\end{frame}
%\only<>{\listasm[firstline=47,lastline=65]{src/ocaml/caml\_start\_program.s}{caml\_start\_program}}


\begin{frame}{Back to C}{Clean up, complain and terminate}
\begin{columns}
\begin{column}{0.45\textwidth}
\begin{itemize}
\item Backtrace:
\begin{itemize}
\item \funcname{caml\_start\_program} $\rightarrow$ OCaml code
\item \funcname{caml\_startup\_common}
\item \funcname{caml\_startup\_exn}
\item \funcname{caml\_startup}
\item \funcname{caml\_main}
\item \funcname{main}
\end{itemize}
\smallskip
\item \funcname{caml\_startup} checks the return value of \funcname{caml\_startup\_exn}
\begin{itemize}
\item The return value has been specially marked by \funcname{caml\_start\_program} handler
\item Calls \funcname{caml\_fatal\_uncaught\_exception}
\end{itemize}
\item<2-> \funcname{caml\_fatal\_uncaught\_exception} either
\begin{itemize}
\item<2-> Calls the user custom uncaught exception handler, if set with \mintinline{ocaml}{Printexc.set_uncaught_exception_handler fn}\footnotemark
\item<2-> Or falls back to the default one
\item<2-> Which notifies about the uncaught exception
\end{itemize}
\item<4-> Terminates the program either with \funcname{abort} or with a non-zero exit code
\end{itemize}
\end{column}
\begin{column}{0.55\textwidth}
\only<1>{
\listc[firstline=84,lastline=89,highlightlines=88]{../ocaml/runtime/caml/mlvalues.h}{runtime/caml/mlvalues.h:84}
\listc[firstline=138,lastline=143,highlightlines={141-142}]{../ocaml/runtime/startup\_nat.c}{runtime/startup\_nat.c:138}
}%
\only<2>{
\listc[firstline=138,highlightlines={147,149}]{../ocaml/runtime/printexc.c}{runtime/printexc.c:138}
}%
\only<3>{
\listc[firstline=110,lastline=134,highlightlines=129]{../ocaml/runtime/printexc.c}{runtime/printexc.c:138}
}%
\only<4>{
\listc[firstline=138,highlightlines={152,154}]{../ocaml/runtime/printexc.c}{runtime/printexc.c:138}
}%
\end{column}
\end{columns}
\footnotetext[1]{\url{https://v2.ocaml.org/api/Printexc.html\#1\_Uncaughtexceptions}}
\end{frame}
}%
\only<3-5>{\providecommand\step{3}%
% Runtime's default exception handler
%
\subsection*{Runtime's default exception handler}
\frameSubsection{pictures/The\_Architect.png}{}


\begin{frame}{Default exception handler}
\begin{columns}
\begin{column}{0.5\textwidth}
\begin{itemize}
\item \localname{domain\_state->exn\_handler} points to a trap located before \funcname{entry}!
\item What installed this very first trap there?
\end{itemize}
\bigskip
\listocaml{../src/default/default.ml}{default/default.ml}
\bigskip
\inputminted[fontsize=\footnotesize]{text}{../src/default/default.out}
\end{column}
\begin{column}{0.5\textwidth}
\centering
\only<1>{\providecommand\step{0}\input{diagrams/default/default.tex}}%
\only<2>{\providecommand\step{4}\input{diagrams/default/default.tex}}%
\end{column}
\end{columns}
\end{frame}


\begin{frame}{Startup}{How did we get there by the way?}
\begin{columns}
\begin{column}{0.45\textwidth}
\begin{itemize}
\item The entry point address of the binary is \funcname{main} (\funcname{\_start}) from the runtime
\item The program starts like a C program:
\begin{itemize}
\item \funcname{caml\_start\_program}
\item \funcname{caml\_startup\_common}
\item \funcname{caml\_startup\_exn}
\item \funcname{caml\_startup}
\item \funcname{caml\_main}
\item \funcname{main}
\end{itemize}
%\item \funcname{caml\_startup\_common}: initializes GC, domains \dots
% Also: codefrag, locale, custom opeartions, frame_descr, signals, stack
\item \funcname{caml\_start\_program} is the transition point between the C realm and the OCaml one
\end{itemize}
\bigskip
\listocaml{../src/default/default.ml}{default/default.ml}
\end{column}
\begin{column}{0.55\textwidth}
\listasm[lastline=24,highlightlines={22-24}]{src/ocaml/caml\_start\_program.s}{runtime/amd64.S:604}
\end{column}
\end{columns}
\end{frame}


\begin{frame}{Setup to jump into OCaml entry point}{\funcname{caml\_start\_program} initializes the main fiber}
\begin{columns}
\begin{column}{0.6\textwidth}
\only<1,3,5>{
\begin{itemize}
\item<1-> Stores information to allow switching from OCaml stack to the C stack
\item<3-> Constructs the default exception handler
\item<5-> Switches to the OCaml stack and calls \funcname{caml\_program}
\end{itemize}
\bigskip
\listocaml{../src/default/default.ml}{default/default.ml}
}%
\only<2>{
\listasm[firstline=22,lastline=41,highlightlines={25-27}]{src/ocaml/caml\_start\_program.s}{caml\_start\_program}
}%
\only<4>{
\mintasm[firstline=22,lastline=41,highlightlines={31-38}]{src/ocaml/caml\_start\_program.s}
\listasm[firstline=66,lastline=72]{src/ocaml/caml\_start\_program.s}{caml\_start\_program}
}%
\only<6>{
\listasm[firstline=22,lastline=41,highlightlines={39-41}]{src/ocaml/caml\_start\_program.s}{caml\_start\_program}
}%
\end{column}
\begin{column}{0.4\textwidth}
\centering
\only<1-2>{\providecommand\step{2}\input{diagrams/default/default.tex}}%
\only<3-5>{\providecommand\step{3}\input{diagrams/default/default.tex}}%
\onslide*<6->{\providecommand\step{4}\input{diagrams/default/default.tex}}%
\end{column}
\end{columns}
\end{frame}

\begin{frame}{Executing the OCaml program}{And raising an exception without handler}
\begin{columns}
\begin{column}{0.6\textwidth}
\only<1-3>{
\begin{itemize}
\item<1-> Execution continues with OCaml code
\begin{itemize}
\item \funcname{camlDefault\_\_main\_269}
\item \funcname{camlDefault\_\_entry}
\item \funcname{caml\_program}
\end{itemize}
\item<1-> \funcname{camlDefault\_\_main\_269} raises a \typename{Failure} exception
\item<1-> \typename{Failure} is caught by \funcname{caml\_start\_program}
\begin{itemize}
\item<1-> The handler set the 2nd LSB of the exception value to \funcarg{1}
\item<3-> And then forces \funcname{caml\_start\_program} to terminate
\end{itemize}
\end{itemize}
}
\bigskip
\only<1,3>{\listocaml[highlightlines=3]{../src/default/default.ml}{default/default.ml}}%
\only<2>{\listasm[firstline=66,lastline=72,highlightlines=69]{src/ocaml/caml\_start\_program.s}{caml\_start\_program}}%
\end{column}
\begin{column}{0.4\textwidth}
\centering
\providecommand\step{4}\input{diagrams/default/default.tex}
\end{column}
\end{columns}
\end{frame}
%\only<>{\listasm[firstline=47,lastline=65]{src/ocaml/caml\_start\_program.s}{caml\_start\_program}}


\begin{frame}{Back to C}{Clean up, complain and terminate}
\begin{columns}
\begin{column}{0.45\textwidth}
\begin{itemize}
\item Backtrace:
\begin{itemize}
\item \funcname{caml\_start\_program} $\rightarrow$ OCaml code
\item \funcname{caml\_startup\_common}
\item \funcname{caml\_startup\_exn}
\item \funcname{caml\_startup}
\item \funcname{caml\_main}
\item \funcname{main}
\end{itemize}
\smallskip
\item \funcname{caml\_startup} checks the return value of \funcname{caml\_startup\_exn}
\begin{itemize}
\item The return value has been specially marked by \funcname{caml\_start\_program} handler
\item Calls \funcname{caml\_fatal\_uncaught\_exception}
\end{itemize}
\item<2-> \funcname{caml\_fatal\_uncaught\_exception} either
\begin{itemize}
\item<2-> Calls the user custom uncaught exception handler, if set with \mintinline{ocaml}{Printexc.set_uncaught_exception_handler fn}\footnotemark
\item<2-> Or falls back to the default one
\item<2-> Which notifies about the uncaught exception
\end{itemize}
\item<4-> Terminates the program either with \funcname{abort} or with a non-zero exit code
\end{itemize}
\end{column}
\begin{column}{0.55\textwidth}
\only<1>{
\listc[firstline=84,lastline=89,highlightlines=88]{../ocaml/runtime/caml/mlvalues.h}{runtime/caml/mlvalues.h:84}
\listc[firstline=138,lastline=143,highlightlines={141-142}]{../ocaml/runtime/startup\_nat.c}{runtime/startup\_nat.c:138}
}%
\only<2>{
\listc[firstline=138,highlightlines={147,149}]{../ocaml/runtime/printexc.c}{runtime/printexc.c:138}
}%
\only<3>{
\listc[firstline=110,lastline=134,highlightlines=129]{../ocaml/runtime/printexc.c}{runtime/printexc.c:138}
}%
\only<4>{
\listc[firstline=138,highlightlines={152,154}]{../ocaml/runtime/printexc.c}{runtime/printexc.c:138}
}%
\end{column}
\end{columns}
\footnotetext[1]{\url{https://v2.ocaml.org/api/Printexc.html\#1\_Uncaughtexceptions}}
\end{frame}
}%
\onslide*<6->{\providecommand\step{4}%
% Runtime's default exception handler
%
\subsection*{Runtime's default exception handler}
\frameSubsection{pictures/The\_Architect.png}{}


\begin{frame}{Default exception handler}
\begin{columns}
\begin{column}{0.5\textwidth}
\begin{itemize}
\item \localname{domain\_state->exn\_handler} points to a trap located before \funcname{entry}!
\item What installed this very first trap there?
\end{itemize}
\bigskip
\listocaml{../src/default/default.ml}{default/default.ml}
\bigskip
\inputminted[fontsize=\footnotesize]{text}{../src/default/default.out}
\end{column}
\begin{column}{0.5\textwidth}
\centering
\only<1>{\providecommand\step{0}\input{diagrams/default/default.tex}}%
\only<2>{\providecommand\step{4}\input{diagrams/default/default.tex}}%
\end{column}
\end{columns}
\end{frame}


\begin{frame}{Startup}{How did we get there by the way?}
\begin{columns}
\begin{column}{0.45\textwidth}
\begin{itemize}
\item The entry point address of the binary is \funcname{main} (\funcname{\_start}) from the runtime
\item The program starts like a C program:
\begin{itemize}
\item \funcname{caml\_start\_program}
\item \funcname{caml\_startup\_common}
\item \funcname{caml\_startup\_exn}
\item \funcname{caml\_startup}
\item \funcname{caml\_main}
\item \funcname{main}
\end{itemize}
%\item \funcname{caml\_startup\_common}: initializes GC, domains \dots
% Also: codefrag, locale, custom opeartions, frame_descr, signals, stack
\item \funcname{caml\_start\_program} is the transition point between the C realm and the OCaml one
\end{itemize}
\bigskip
\listocaml{../src/default/default.ml}{default/default.ml}
\end{column}
\begin{column}{0.55\textwidth}
\listasm[lastline=24,highlightlines={22-24}]{src/ocaml/caml\_start\_program.s}{runtime/amd64.S:604}
\end{column}
\end{columns}
\end{frame}


\begin{frame}{Setup to jump into OCaml entry point}{\funcname{caml\_start\_program} initializes the main fiber}
\begin{columns}
\begin{column}{0.6\textwidth}
\only<1,3,5>{
\begin{itemize}
\item<1-> Stores information to allow switching from OCaml stack to the C stack
\item<3-> Constructs the default exception handler
\item<5-> Switches to the OCaml stack and calls \funcname{caml\_program}
\end{itemize}
\bigskip
\listocaml{../src/default/default.ml}{default/default.ml}
}%
\only<2>{
\listasm[firstline=22,lastline=41,highlightlines={25-27}]{src/ocaml/caml\_start\_program.s}{caml\_start\_program}
}%
\only<4>{
\mintasm[firstline=22,lastline=41,highlightlines={31-38}]{src/ocaml/caml\_start\_program.s}
\listasm[firstline=66,lastline=72]{src/ocaml/caml\_start\_program.s}{caml\_start\_program}
}%
\only<6>{
\listasm[firstline=22,lastline=41,highlightlines={39-41}]{src/ocaml/caml\_start\_program.s}{caml\_start\_program}
}%
\end{column}
\begin{column}{0.4\textwidth}
\centering
\only<1-2>{\providecommand\step{2}\input{diagrams/default/default.tex}}%
\only<3-5>{\providecommand\step{3}\input{diagrams/default/default.tex}}%
\onslide*<6->{\providecommand\step{4}\input{diagrams/default/default.tex}}%
\end{column}
\end{columns}
\end{frame}

\begin{frame}{Executing the OCaml program}{And raising an exception without handler}
\begin{columns}
\begin{column}{0.6\textwidth}
\only<1-3>{
\begin{itemize}
\item<1-> Execution continues with OCaml code
\begin{itemize}
\item \funcname{camlDefault\_\_main\_269}
\item \funcname{camlDefault\_\_entry}
\item \funcname{caml\_program}
\end{itemize}
\item<1-> \funcname{camlDefault\_\_main\_269} raises a \typename{Failure} exception
\item<1-> \typename{Failure} is caught by \funcname{caml\_start\_program}
\begin{itemize}
\item<1-> The handler set the 2nd LSB of the exception value to \funcarg{1}
\item<3-> And then forces \funcname{caml\_start\_program} to terminate
\end{itemize}
\end{itemize}
}
\bigskip
\only<1,3>{\listocaml[highlightlines=3]{../src/default/default.ml}{default/default.ml}}%
\only<2>{\listasm[firstline=66,lastline=72,highlightlines=69]{src/ocaml/caml\_start\_program.s}{caml\_start\_program}}%
\end{column}
\begin{column}{0.4\textwidth}
\centering
\providecommand\step{4}\input{diagrams/default/default.tex}
\end{column}
\end{columns}
\end{frame}
%\only<>{\listasm[firstline=47,lastline=65]{src/ocaml/caml\_start\_program.s}{caml\_start\_program}}


\begin{frame}{Back to C}{Clean up, complain and terminate}
\begin{columns}
\begin{column}{0.45\textwidth}
\begin{itemize}
\item Backtrace:
\begin{itemize}
\item \funcname{caml\_start\_program} $\rightarrow$ OCaml code
\item \funcname{caml\_startup\_common}
\item \funcname{caml\_startup\_exn}
\item \funcname{caml\_startup}
\item \funcname{caml\_main}
\item \funcname{main}
\end{itemize}
\smallskip
\item \funcname{caml\_startup} checks the return value of \funcname{caml\_startup\_exn}
\begin{itemize}
\item The return value has been specially marked by \funcname{caml\_start\_program} handler
\item Calls \funcname{caml\_fatal\_uncaught\_exception}
\end{itemize}
\item<2-> \funcname{caml\_fatal\_uncaught\_exception} either
\begin{itemize}
\item<2-> Calls the user custom uncaught exception handler, if set with \mintinline{ocaml}{Printexc.set_uncaught_exception_handler fn}\footnotemark
\item<2-> Or falls back to the default one
\item<2-> Which notifies about the uncaught exception
\end{itemize}
\item<4-> Terminates the program either with \funcname{abort} or with a non-zero exit code
\end{itemize}
\end{column}
\begin{column}{0.55\textwidth}
\only<1>{
\listc[firstline=84,lastline=89,highlightlines=88]{../ocaml/runtime/caml/mlvalues.h}{runtime/caml/mlvalues.h:84}
\listc[firstline=138,lastline=143,highlightlines={141-142}]{../ocaml/runtime/startup\_nat.c}{runtime/startup\_nat.c:138}
}%
\only<2>{
\listc[firstline=138,highlightlines={147,149}]{../ocaml/runtime/printexc.c}{runtime/printexc.c:138}
}%
\only<3>{
\listc[firstline=110,lastline=134,highlightlines=129]{../ocaml/runtime/printexc.c}{runtime/printexc.c:138}
}%
\only<4>{
\listc[firstline=138,highlightlines={152,154}]{../ocaml/runtime/printexc.c}{runtime/printexc.c:138}
}%
\end{column}
\end{columns}
\footnotetext[1]{\url{https://v2.ocaml.org/api/Printexc.html\#1\_Uncaughtexceptions}}
\end{frame}
}%
\end{column}
\end{columns}
\end{frame}

\begin{frame}{Executing the OCaml program}{And raising an exception without handler}
\begin{columns}
\begin{column}{0.6\textwidth}
\only<1-3>{
\begin{itemize}
\item<1-> Execution continues with OCaml code
\begin{itemize}
\item \funcname{camlDefault\_\_main\_269}
\item \funcname{camlDefault\_\_entry}
\item \funcname{caml\_program}
\end{itemize}
\item<1-> \funcname{camlDefault\_\_main\_269} raises a \typename{Failure} exception
\item<1-> \typename{Failure} is caught by \funcname{caml\_start\_program}
\begin{itemize}
\item<1-> The handler set the 2nd LSB of the exception value to \funcarg{1}
\item<3-> And then forces \funcname{caml\_start\_program} to terminate
\end{itemize}
\end{itemize}
}
\bigskip
\only<1,3>{\listocaml[highlightlines=3]{../src/default/default.ml}{default/default.ml}}%
\only<2>{\listasm[firstline=66,lastline=72,highlightlines=69]{src/ocaml/caml\_start\_program.s}{caml\_start\_program}}%
\end{column}
\begin{column}{0.4\textwidth}
\centering
\providecommand\step{4}%
% Runtime's default exception handler
%
\subsection*{Runtime's default exception handler}
\frameSubsection{pictures/The\_Architect.png}{}


\begin{frame}{Default exception handler}
\begin{columns}
\begin{column}{0.5\textwidth}
\begin{itemize}
\item \localname{domain\_state->exn\_handler} points to a trap located before \funcname{entry}!
\item What installed this very first trap there?
\end{itemize}
\bigskip
\listocaml{../src/default/default.ml}{default/default.ml}
\bigskip
\inputminted[fontsize=\footnotesize]{text}{../src/default/default.out}
\end{column}
\begin{column}{0.5\textwidth}
\centering
\only<1>{\providecommand\step{0}\input{diagrams/default/default.tex}}%
\only<2>{\providecommand\step{4}\input{diagrams/default/default.tex}}%
\end{column}
\end{columns}
\end{frame}


\begin{frame}{Startup}{How did we get there by the way?}
\begin{columns}
\begin{column}{0.45\textwidth}
\begin{itemize}
\item The entry point address of the binary is \funcname{main} (\funcname{\_start}) from the runtime
\item The program starts like a C program:
\begin{itemize}
\item \funcname{caml\_start\_program}
\item \funcname{caml\_startup\_common}
\item \funcname{caml\_startup\_exn}
\item \funcname{caml\_startup}
\item \funcname{caml\_main}
\item \funcname{main}
\end{itemize}
%\item \funcname{caml\_startup\_common}: initializes GC, domains \dots
% Also: codefrag, locale, custom opeartions, frame_descr, signals, stack
\item \funcname{caml\_start\_program} is the transition point between the C realm and the OCaml one
\end{itemize}
\bigskip
\listocaml{../src/default/default.ml}{default/default.ml}
\end{column}
\begin{column}{0.55\textwidth}
\listasm[lastline=24,highlightlines={22-24}]{src/ocaml/caml\_start\_program.s}{runtime/amd64.S:604}
\end{column}
\end{columns}
\end{frame}


\begin{frame}{Setup to jump into OCaml entry point}{\funcname{caml\_start\_program} initializes the main fiber}
\begin{columns}
\begin{column}{0.6\textwidth}
\only<1,3,5>{
\begin{itemize}
\item<1-> Stores information to allow switching from OCaml stack to the C stack
\item<3-> Constructs the default exception handler
\item<5-> Switches to the OCaml stack and calls \funcname{caml\_program}
\end{itemize}
\bigskip
\listocaml{../src/default/default.ml}{default/default.ml}
}%
\only<2>{
\listasm[firstline=22,lastline=41,highlightlines={25-27}]{src/ocaml/caml\_start\_program.s}{caml\_start\_program}
}%
\only<4>{
\mintasm[firstline=22,lastline=41,highlightlines={31-38}]{src/ocaml/caml\_start\_program.s}
\listasm[firstline=66,lastline=72]{src/ocaml/caml\_start\_program.s}{caml\_start\_program}
}%
\only<6>{
\listasm[firstline=22,lastline=41,highlightlines={39-41}]{src/ocaml/caml\_start\_program.s}{caml\_start\_program}
}%
\end{column}
\begin{column}{0.4\textwidth}
\centering
\only<1-2>{\providecommand\step{2}\input{diagrams/default/default.tex}}%
\only<3-5>{\providecommand\step{3}\input{diagrams/default/default.tex}}%
\onslide*<6->{\providecommand\step{4}\input{diagrams/default/default.tex}}%
\end{column}
\end{columns}
\end{frame}

\begin{frame}{Executing the OCaml program}{And raising an exception without handler}
\begin{columns}
\begin{column}{0.6\textwidth}
\only<1-3>{
\begin{itemize}
\item<1-> Execution continues with OCaml code
\begin{itemize}
\item \funcname{camlDefault\_\_main\_269}
\item \funcname{camlDefault\_\_entry}
\item \funcname{caml\_program}
\end{itemize}
\item<1-> \funcname{camlDefault\_\_main\_269} raises a \typename{Failure} exception
\item<1-> \typename{Failure} is caught by \funcname{caml\_start\_program}
\begin{itemize}
\item<1-> The handler set the 2nd LSB of the exception value to \funcarg{1}
\item<3-> And then forces \funcname{caml\_start\_program} to terminate
\end{itemize}
\end{itemize}
}
\bigskip
\only<1,3>{\listocaml[highlightlines=3]{../src/default/default.ml}{default/default.ml}}%
\only<2>{\listasm[firstline=66,lastline=72,highlightlines=69]{src/ocaml/caml\_start\_program.s}{caml\_start\_program}}%
\end{column}
\begin{column}{0.4\textwidth}
\centering
\providecommand\step{4}\input{diagrams/default/default.tex}
\end{column}
\end{columns}
\end{frame}
%\only<>{\listasm[firstline=47,lastline=65]{src/ocaml/caml\_start\_program.s}{caml\_start\_program}}


\begin{frame}{Back to C}{Clean up, complain and terminate}
\begin{columns}
\begin{column}{0.45\textwidth}
\begin{itemize}
\item Backtrace:
\begin{itemize}
\item \funcname{caml\_start\_program} $\rightarrow$ OCaml code
\item \funcname{caml\_startup\_common}
\item \funcname{caml\_startup\_exn}
\item \funcname{caml\_startup}
\item \funcname{caml\_main}
\item \funcname{main}
\end{itemize}
\smallskip
\item \funcname{caml\_startup} checks the return value of \funcname{caml\_startup\_exn}
\begin{itemize}
\item The return value has been specially marked by \funcname{caml\_start\_program} handler
\item Calls \funcname{caml\_fatal\_uncaught\_exception}
\end{itemize}
\item<2-> \funcname{caml\_fatal\_uncaught\_exception} either
\begin{itemize}
\item<2-> Calls the user custom uncaught exception handler, if set with \mintinline{ocaml}{Printexc.set_uncaught_exception_handler fn}\footnotemark
\item<2-> Or falls back to the default one
\item<2-> Which notifies about the uncaught exception
\end{itemize}
\item<4-> Terminates the program either with \funcname{abort} or with a non-zero exit code
\end{itemize}
\end{column}
\begin{column}{0.55\textwidth}
\only<1>{
\listc[firstline=84,lastline=89,highlightlines=88]{../ocaml/runtime/caml/mlvalues.h}{runtime/caml/mlvalues.h:84}
\listc[firstline=138,lastline=143,highlightlines={141-142}]{../ocaml/runtime/startup\_nat.c}{runtime/startup\_nat.c:138}
}%
\only<2>{
\listc[firstline=138,highlightlines={147,149}]{../ocaml/runtime/printexc.c}{runtime/printexc.c:138}
}%
\only<3>{
\listc[firstline=110,lastline=134,highlightlines=129]{../ocaml/runtime/printexc.c}{runtime/printexc.c:138}
}%
\only<4>{
\listc[firstline=138,highlightlines={152,154}]{../ocaml/runtime/printexc.c}{runtime/printexc.c:138}
}%
\end{column}
\end{columns}
\footnotetext[1]{\url{https://v2.ocaml.org/api/Printexc.html\#1\_Uncaughtexceptions}}
\end{frame}

\end{column}
\end{columns}
\end{frame}
%\only<>{\listasm[firstline=47,lastline=65]{src/ocaml/caml\_start\_program.s}{caml\_start\_program}}


\begin{frame}{Back to C}{Clean up, complain and terminate}
\begin{columns}
\begin{column}{0.45\textwidth}
\begin{itemize}
\item Backtrace:
\begin{itemize}
\item \funcname{caml\_start\_program} $\rightarrow$ OCaml code
\item \funcname{caml\_startup\_common}
\item \funcname{caml\_startup\_exn}
\item \funcname{caml\_startup}
\item \funcname{caml\_main}
\item \funcname{main}
\end{itemize}
\smallskip
\item \funcname{caml\_startup} checks the return value of \funcname{caml\_startup\_exn}
\begin{itemize}
\item The return value has been specially marked by \funcname{caml\_start\_program} handler
\item Calls \funcname{caml\_fatal\_uncaught\_exception}
\end{itemize}
\item<2-> \funcname{caml\_fatal\_uncaught\_exception} either
\begin{itemize}
\item<2-> Calls the user custom uncaught exception handler, if set with \mintinline{ocaml}{Printexc.set_uncaught_exception_handler fn}\footnotemark
\item<2-> Or falls back to the default one
\item<2-> Which notifies about the uncaught exception
\end{itemize}
\item<4-> Terminates the program either with \funcname{abort} or with a non-zero exit code
\end{itemize}
\end{column}
\begin{column}{0.55\textwidth}
\only<1>{
\listc[firstline=84,lastline=89,highlightlines=88]{../ocaml/runtime/caml/mlvalues.h}{runtime/caml/mlvalues.h:84}
\listc[firstline=138,lastline=143,highlightlines={141-142}]{../ocaml/runtime/startup\_nat.c}{runtime/startup\_nat.c:138}
}%
\only<2>{
\listc[firstline=138,highlightlines={147,149}]{../ocaml/runtime/printexc.c}{runtime/printexc.c:138}
}%
\only<3>{
\listc[firstline=110,lastline=134,highlightlines=129]{../ocaml/runtime/printexc.c}{runtime/printexc.c:138}
}%
\only<4>{
\listc[firstline=138,highlightlines={152,154}]{../ocaml/runtime/printexc.c}{runtime/printexc.c:138}
}%
\end{column}
\end{columns}
\footnotetext[1]{\url{https://v2.ocaml.org/api/Printexc.html\#1\_Uncaughtexceptions}}
\end{frame}
}%
\end{column}
\end{columns}
\end{frame}

\begin{frame}{Executing the OCaml program}{And raising an exception without handler}
\begin{columns}
\begin{column}{0.6\textwidth}
\only<1-3>{
\begin{itemize}
\item<1-> Execution continues with OCaml code
\begin{itemize}
\item \funcname{camlDefault\_\_main\_269}
\item \funcname{camlDefault\_\_entry}
\item \funcname{caml\_program}
\end{itemize}
\item<1-> \funcname{camlDefault\_\_main\_269} raises a \typename{Failure} exception
\item<1-> \typename{Failure} is caught by \funcname{caml\_start\_program}
\begin{itemize}
\item<1-> The handler set the 2nd LSB of the exception value to \funcarg{1}
\item<3-> And then forces \funcname{caml\_start\_program} to terminate
\end{itemize}
\end{itemize}
}
\bigskip
\only<1,3>{\listocaml[highlightlines=3]{../src/default/default.ml}{default/default.ml}}%
\only<2>{\listasm[firstline=66,lastline=72,highlightlines=69]{src/ocaml/caml\_start\_program.s}{caml\_start\_program}}%
\end{column}
\begin{column}{0.4\textwidth}
\centering
\providecommand\step{4}%
% Runtime's default exception handler
%
\subsection*{Runtime's default exception handler}
\frameSubsection{pictures/The\_Architect.png}{}


\begin{frame}{Default exception handler}
\begin{columns}
\begin{column}{0.5\textwidth}
\begin{itemize}
\item \localname{domain\_state->exn\_handler} points to a trap located before \funcname{entry}!
\item What installed this very first trap there?
\end{itemize}
\bigskip
\listocaml{../src/default/default.ml}{default/default.ml}
\bigskip
\inputminted[fontsize=\footnotesize]{text}{../src/default/default.out}
\end{column}
\begin{column}{0.5\textwidth}
\centering
\only<1>{\providecommand\step{0}%
% Runtime's default exception handler
%
\subsection*{Runtime's default exception handler}
\frameSubsection{pictures/The\_Architect.png}{}


\begin{frame}{Default exception handler}
\begin{columns}
\begin{column}{0.5\textwidth}
\begin{itemize}
\item \localname{domain\_state->exn\_handler} points to a trap located before \funcname{entry}!
\item What installed this very first trap there?
\end{itemize}
\bigskip
\listocaml{../src/default/default.ml}{default/default.ml}
\bigskip
\inputminted[fontsize=\footnotesize]{text}{../src/default/default.out}
\end{column}
\begin{column}{0.5\textwidth}
\centering
\only<1>{\providecommand\step{0}\input{diagrams/default/default.tex}}%
\only<2>{\providecommand\step{4}\input{diagrams/default/default.tex}}%
\end{column}
\end{columns}
\end{frame}


\begin{frame}{Startup}{How did we get there by the way?}
\begin{columns}
\begin{column}{0.45\textwidth}
\begin{itemize}
\item The entry point address of the binary is \funcname{main} (\funcname{\_start}) from the runtime
\item The program starts like a C program:
\begin{itemize}
\item \funcname{caml\_start\_program}
\item \funcname{caml\_startup\_common}
\item \funcname{caml\_startup\_exn}
\item \funcname{caml\_startup}
\item \funcname{caml\_main}
\item \funcname{main}
\end{itemize}
%\item \funcname{caml\_startup\_common}: initializes GC, domains \dots
% Also: codefrag, locale, custom opeartions, frame_descr, signals, stack
\item \funcname{caml\_start\_program} is the transition point between the C realm and the OCaml one
\end{itemize}
\bigskip
\listocaml{../src/default/default.ml}{default/default.ml}
\end{column}
\begin{column}{0.55\textwidth}
\listasm[lastline=24,highlightlines={22-24}]{src/ocaml/caml\_start\_program.s}{runtime/amd64.S:604}
\end{column}
\end{columns}
\end{frame}


\begin{frame}{Setup to jump into OCaml entry point}{\funcname{caml\_start\_program} initializes the main fiber}
\begin{columns}
\begin{column}{0.6\textwidth}
\only<1,3,5>{
\begin{itemize}
\item<1-> Stores information to allow switching from OCaml stack to the C stack
\item<3-> Constructs the default exception handler
\item<5-> Switches to the OCaml stack and calls \funcname{caml\_program}
\end{itemize}
\bigskip
\listocaml{../src/default/default.ml}{default/default.ml}
}%
\only<2>{
\listasm[firstline=22,lastline=41,highlightlines={25-27}]{src/ocaml/caml\_start\_program.s}{caml\_start\_program}
}%
\only<4>{
\mintasm[firstline=22,lastline=41,highlightlines={31-38}]{src/ocaml/caml\_start\_program.s}
\listasm[firstline=66,lastline=72]{src/ocaml/caml\_start\_program.s}{caml\_start\_program}
}%
\only<6>{
\listasm[firstline=22,lastline=41,highlightlines={39-41}]{src/ocaml/caml\_start\_program.s}{caml\_start\_program}
}%
\end{column}
\begin{column}{0.4\textwidth}
\centering
\only<1-2>{\providecommand\step{2}\input{diagrams/default/default.tex}}%
\only<3-5>{\providecommand\step{3}\input{diagrams/default/default.tex}}%
\onslide*<6->{\providecommand\step{4}\input{diagrams/default/default.tex}}%
\end{column}
\end{columns}
\end{frame}

\begin{frame}{Executing the OCaml program}{And raising an exception without handler}
\begin{columns}
\begin{column}{0.6\textwidth}
\only<1-3>{
\begin{itemize}
\item<1-> Execution continues with OCaml code
\begin{itemize}
\item \funcname{camlDefault\_\_main\_269}
\item \funcname{camlDefault\_\_entry}
\item \funcname{caml\_program}
\end{itemize}
\item<1-> \funcname{camlDefault\_\_main\_269} raises a \typename{Failure} exception
\item<1-> \typename{Failure} is caught by \funcname{caml\_start\_program}
\begin{itemize}
\item<1-> The handler set the 2nd LSB of the exception value to \funcarg{1}
\item<3-> And then forces \funcname{caml\_start\_program} to terminate
\end{itemize}
\end{itemize}
}
\bigskip
\only<1,3>{\listocaml[highlightlines=3]{../src/default/default.ml}{default/default.ml}}%
\only<2>{\listasm[firstline=66,lastline=72,highlightlines=69]{src/ocaml/caml\_start\_program.s}{caml\_start\_program}}%
\end{column}
\begin{column}{0.4\textwidth}
\centering
\providecommand\step{4}\input{diagrams/default/default.tex}
\end{column}
\end{columns}
\end{frame}
%\only<>{\listasm[firstline=47,lastline=65]{src/ocaml/caml\_start\_program.s}{caml\_start\_program}}


\begin{frame}{Back to C}{Clean up, complain and terminate}
\begin{columns}
\begin{column}{0.45\textwidth}
\begin{itemize}
\item Backtrace:
\begin{itemize}
\item \funcname{caml\_start\_program} $\rightarrow$ OCaml code
\item \funcname{caml\_startup\_common}
\item \funcname{caml\_startup\_exn}
\item \funcname{caml\_startup}
\item \funcname{caml\_main}
\item \funcname{main}
\end{itemize}
\smallskip
\item \funcname{caml\_startup} checks the return value of \funcname{caml\_startup\_exn}
\begin{itemize}
\item The return value has been specially marked by \funcname{caml\_start\_program} handler
\item Calls \funcname{caml\_fatal\_uncaught\_exception}
\end{itemize}
\item<2-> \funcname{caml\_fatal\_uncaught\_exception} either
\begin{itemize}
\item<2-> Calls the user custom uncaught exception handler, if set with \mintinline{ocaml}{Printexc.set_uncaught_exception_handler fn}\footnotemark
\item<2-> Or falls back to the default one
\item<2-> Which notifies about the uncaught exception
\end{itemize}
\item<4-> Terminates the program either with \funcname{abort} or with a non-zero exit code
\end{itemize}
\end{column}
\begin{column}{0.55\textwidth}
\only<1>{
\listc[firstline=84,lastline=89,highlightlines=88]{../ocaml/runtime/caml/mlvalues.h}{runtime/caml/mlvalues.h:84}
\listc[firstline=138,lastline=143,highlightlines={141-142}]{../ocaml/runtime/startup\_nat.c}{runtime/startup\_nat.c:138}
}%
\only<2>{
\listc[firstline=138,highlightlines={147,149}]{../ocaml/runtime/printexc.c}{runtime/printexc.c:138}
}%
\only<3>{
\listc[firstline=110,lastline=134,highlightlines=129]{../ocaml/runtime/printexc.c}{runtime/printexc.c:138}
}%
\only<4>{
\listc[firstline=138,highlightlines={152,154}]{../ocaml/runtime/printexc.c}{runtime/printexc.c:138}
}%
\end{column}
\end{columns}
\footnotetext[1]{\url{https://v2.ocaml.org/api/Printexc.html\#1\_Uncaughtexceptions}}
\end{frame}
}%
\only<2>{\providecommand\step{4}%
% Runtime's default exception handler
%
\subsection*{Runtime's default exception handler}
\frameSubsection{pictures/The\_Architect.png}{}


\begin{frame}{Default exception handler}
\begin{columns}
\begin{column}{0.5\textwidth}
\begin{itemize}
\item \localname{domain\_state->exn\_handler} points to a trap located before \funcname{entry}!
\item What installed this very first trap there?
\end{itemize}
\bigskip
\listocaml{../src/default/default.ml}{default/default.ml}
\bigskip
\inputminted[fontsize=\footnotesize]{text}{../src/default/default.out}
\end{column}
\begin{column}{0.5\textwidth}
\centering
\only<1>{\providecommand\step{0}\input{diagrams/default/default.tex}}%
\only<2>{\providecommand\step{4}\input{diagrams/default/default.tex}}%
\end{column}
\end{columns}
\end{frame}


\begin{frame}{Startup}{How did we get there by the way?}
\begin{columns}
\begin{column}{0.45\textwidth}
\begin{itemize}
\item The entry point address of the binary is \funcname{main} (\funcname{\_start}) from the runtime
\item The program starts like a C program:
\begin{itemize}
\item \funcname{caml\_start\_program}
\item \funcname{caml\_startup\_common}
\item \funcname{caml\_startup\_exn}
\item \funcname{caml\_startup}
\item \funcname{caml\_main}
\item \funcname{main}
\end{itemize}
%\item \funcname{caml\_startup\_common}: initializes GC, domains \dots
% Also: codefrag, locale, custom opeartions, frame_descr, signals, stack
\item \funcname{caml\_start\_program} is the transition point between the C realm and the OCaml one
\end{itemize}
\bigskip
\listocaml{../src/default/default.ml}{default/default.ml}
\end{column}
\begin{column}{0.55\textwidth}
\listasm[lastline=24,highlightlines={22-24}]{src/ocaml/caml\_start\_program.s}{runtime/amd64.S:604}
\end{column}
\end{columns}
\end{frame}


\begin{frame}{Setup to jump into OCaml entry point}{\funcname{caml\_start\_program} initializes the main fiber}
\begin{columns}
\begin{column}{0.6\textwidth}
\only<1,3,5>{
\begin{itemize}
\item<1-> Stores information to allow switching from OCaml stack to the C stack
\item<3-> Constructs the default exception handler
\item<5-> Switches to the OCaml stack and calls \funcname{caml\_program}
\end{itemize}
\bigskip
\listocaml{../src/default/default.ml}{default/default.ml}
}%
\only<2>{
\listasm[firstline=22,lastline=41,highlightlines={25-27}]{src/ocaml/caml\_start\_program.s}{caml\_start\_program}
}%
\only<4>{
\mintasm[firstline=22,lastline=41,highlightlines={31-38}]{src/ocaml/caml\_start\_program.s}
\listasm[firstline=66,lastline=72]{src/ocaml/caml\_start\_program.s}{caml\_start\_program}
}%
\only<6>{
\listasm[firstline=22,lastline=41,highlightlines={39-41}]{src/ocaml/caml\_start\_program.s}{caml\_start\_program}
}%
\end{column}
\begin{column}{0.4\textwidth}
\centering
\only<1-2>{\providecommand\step{2}\input{diagrams/default/default.tex}}%
\only<3-5>{\providecommand\step{3}\input{diagrams/default/default.tex}}%
\onslide*<6->{\providecommand\step{4}\input{diagrams/default/default.tex}}%
\end{column}
\end{columns}
\end{frame}

\begin{frame}{Executing the OCaml program}{And raising an exception without handler}
\begin{columns}
\begin{column}{0.6\textwidth}
\only<1-3>{
\begin{itemize}
\item<1-> Execution continues with OCaml code
\begin{itemize}
\item \funcname{camlDefault\_\_main\_269}
\item \funcname{camlDefault\_\_entry}
\item \funcname{caml\_program}
\end{itemize}
\item<1-> \funcname{camlDefault\_\_main\_269} raises a \typename{Failure} exception
\item<1-> \typename{Failure} is caught by \funcname{caml\_start\_program}
\begin{itemize}
\item<1-> The handler set the 2nd LSB of the exception value to \funcarg{1}
\item<3-> And then forces \funcname{caml\_start\_program} to terminate
\end{itemize}
\end{itemize}
}
\bigskip
\only<1,3>{\listocaml[highlightlines=3]{../src/default/default.ml}{default/default.ml}}%
\only<2>{\listasm[firstline=66,lastline=72,highlightlines=69]{src/ocaml/caml\_start\_program.s}{caml\_start\_program}}%
\end{column}
\begin{column}{0.4\textwidth}
\centering
\providecommand\step{4}\input{diagrams/default/default.tex}
\end{column}
\end{columns}
\end{frame}
%\only<>{\listasm[firstline=47,lastline=65]{src/ocaml/caml\_start\_program.s}{caml\_start\_program}}


\begin{frame}{Back to C}{Clean up, complain and terminate}
\begin{columns}
\begin{column}{0.45\textwidth}
\begin{itemize}
\item Backtrace:
\begin{itemize}
\item \funcname{caml\_start\_program} $\rightarrow$ OCaml code
\item \funcname{caml\_startup\_common}
\item \funcname{caml\_startup\_exn}
\item \funcname{caml\_startup}
\item \funcname{caml\_main}
\item \funcname{main}
\end{itemize}
\smallskip
\item \funcname{caml\_startup} checks the return value of \funcname{caml\_startup\_exn}
\begin{itemize}
\item The return value has been specially marked by \funcname{caml\_start\_program} handler
\item Calls \funcname{caml\_fatal\_uncaught\_exception}
\end{itemize}
\item<2-> \funcname{caml\_fatal\_uncaught\_exception} either
\begin{itemize}
\item<2-> Calls the user custom uncaught exception handler, if set with \mintinline{ocaml}{Printexc.set_uncaught_exception_handler fn}\footnotemark
\item<2-> Or falls back to the default one
\item<2-> Which notifies about the uncaught exception
\end{itemize}
\item<4-> Terminates the program either with \funcname{abort} or with a non-zero exit code
\end{itemize}
\end{column}
\begin{column}{0.55\textwidth}
\only<1>{
\listc[firstline=84,lastline=89,highlightlines=88]{../ocaml/runtime/caml/mlvalues.h}{runtime/caml/mlvalues.h:84}
\listc[firstline=138,lastline=143,highlightlines={141-142}]{../ocaml/runtime/startup\_nat.c}{runtime/startup\_nat.c:138}
}%
\only<2>{
\listc[firstline=138,highlightlines={147,149}]{../ocaml/runtime/printexc.c}{runtime/printexc.c:138}
}%
\only<3>{
\listc[firstline=110,lastline=134,highlightlines=129]{../ocaml/runtime/printexc.c}{runtime/printexc.c:138}
}%
\only<4>{
\listc[firstline=138,highlightlines={152,154}]{../ocaml/runtime/printexc.c}{runtime/printexc.c:138}
}%
\end{column}
\end{columns}
\footnotetext[1]{\url{https://v2.ocaml.org/api/Printexc.html\#1\_Uncaughtexceptions}}
\end{frame}
}%
\end{column}
\end{columns}
\end{frame}


\begin{frame}{Startup}{How did we get there by the way?}
\begin{columns}
\begin{column}{0.45\textwidth}
\begin{itemize}
\item The entry point address of the binary is \funcname{main} (\funcname{\_start}) from the runtime
\item The program starts like a C program:
\begin{itemize}
\item \funcname{caml\_start\_program}
\item \funcname{caml\_startup\_common}
\item \funcname{caml\_startup\_exn}
\item \funcname{caml\_startup}
\item \funcname{caml\_main}
\item \funcname{main}
\end{itemize}
%\item \funcname{caml\_startup\_common}: initializes GC, domains \dots
% Also: codefrag, locale, custom opeartions, frame_descr, signals, stack
\item \funcname{caml\_start\_program} is the transition point between the C realm and the OCaml one
\end{itemize}
\bigskip
\listocaml{../src/default/default.ml}{default/default.ml}
\end{column}
\begin{column}{0.55\textwidth}
\listasm[lastline=24,highlightlines={22-24}]{src/ocaml/caml\_start\_program.s}{runtime/amd64.S:604}
\end{column}
\end{columns}
\end{frame}


\begin{frame}{Setup to jump into OCaml entry point}{\funcname{caml\_start\_program} initializes the main fiber}
\begin{columns}
\begin{column}{0.6\textwidth}
\only<1,3,5>{
\begin{itemize}
\item<1-> Stores information to allow switching from OCaml stack to the C stack
\item<3-> Constructs the default exception handler
\item<5-> Switches to the OCaml stack and calls \funcname{caml\_program}
\end{itemize}
\bigskip
\listocaml{../src/default/default.ml}{default/default.ml}
}%
\only<2>{
\listasm[firstline=22,lastline=41,highlightlines={25-27}]{src/ocaml/caml\_start\_program.s}{caml\_start\_program}
}%
\only<4>{
\mintasm[firstline=22,lastline=41,highlightlines={31-38}]{src/ocaml/caml\_start\_program.s}
\listasm[firstline=66,lastline=72]{src/ocaml/caml\_start\_program.s}{caml\_start\_program}
}%
\only<6>{
\listasm[firstline=22,lastline=41,highlightlines={39-41}]{src/ocaml/caml\_start\_program.s}{caml\_start\_program}
}%
\end{column}
\begin{column}{0.4\textwidth}
\centering
\only<1-2>{\providecommand\step{2}%
% Runtime's default exception handler
%
\subsection*{Runtime's default exception handler}
\frameSubsection{pictures/The\_Architect.png}{}


\begin{frame}{Default exception handler}
\begin{columns}
\begin{column}{0.5\textwidth}
\begin{itemize}
\item \localname{domain\_state->exn\_handler} points to a trap located before \funcname{entry}!
\item What installed this very first trap there?
\end{itemize}
\bigskip
\listocaml{../src/default/default.ml}{default/default.ml}
\bigskip
\inputminted[fontsize=\footnotesize]{text}{../src/default/default.out}
\end{column}
\begin{column}{0.5\textwidth}
\centering
\only<1>{\providecommand\step{0}\input{diagrams/default/default.tex}}%
\only<2>{\providecommand\step{4}\input{diagrams/default/default.tex}}%
\end{column}
\end{columns}
\end{frame}


\begin{frame}{Startup}{How did we get there by the way?}
\begin{columns}
\begin{column}{0.45\textwidth}
\begin{itemize}
\item The entry point address of the binary is \funcname{main} (\funcname{\_start}) from the runtime
\item The program starts like a C program:
\begin{itemize}
\item \funcname{caml\_start\_program}
\item \funcname{caml\_startup\_common}
\item \funcname{caml\_startup\_exn}
\item \funcname{caml\_startup}
\item \funcname{caml\_main}
\item \funcname{main}
\end{itemize}
%\item \funcname{caml\_startup\_common}: initializes GC, domains \dots
% Also: codefrag, locale, custom opeartions, frame_descr, signals, stack
\item \funcname{caml\_start\_program} is the transition point between the C realm and the OCaml one
\end{itemize}
\bigskip
\listocaml{../src/default/default.ml}{default/default.ml}
\end{column}
\begin{column}{0.55\textwidth}
\listasm[lastline=24,highlightlines={22-24}]{src/ocaml/caml\_start\_program.s}{runtime/amd64.S:604}
\end{column}
\end{columns}
\end{frame}


\begin{frame}{Setup to jump into OCaml entry point}{\funcname{caml\_start\_program} initializes the main fiber}
\begin{columns}
\begin{column}{0.6\textwidth}
\only<1,3,5>{
\begin{itemize}
\item<1-> Stores information to allow switching from OCaml stack to the C stack
\item<3-> Constructs the default exception handler
\item<5-> Switches to the OCaml stack and calls \funcname{caml\_program}
\end{itemize}
\bigskip
\listocaml{../src/default/default.ml}{default/default.ml}
}%
\only<2>{
\listasm[firstline=22,lastline=41,highlightlines={25-27}]{src/ocaml/caml\_start\_program.s}{caml\_start\_program}
}%
\only<4>{
\mintasm[firstline=22,lastline=41,highlightlines={31-38}]{src/ocaml/caml\_start\_program.s}
\listasm[firstline=66,lastline=72]{src/ocaml/caml\_start\_program.s}{caml\_start\_program}
}%
\only<6>{
\listasm[firstline=22,lastline=41,highlightlines={39-41}]{src/ocaml/caml\_start\_program.s}{caml\_start\_program}
}%
\end{column}
\begin{column}{0.4\textwidth}
\centering
\only<1-2>{\providecommand\step{2}\input{diagrams/default/default.tex}}%
\only<3-5>{\providecommand\step{3}\input{diagrams/default/default.tex}}%
\onslide*<6->{\providecommand\step{4}\input{diagrams/default/default.tex}}%
\end{column}
\end{columns}
\end{frame}

\begin{frame}{Executing the OCaml program}{And raising an exception without handler}
\begin{columns}
\begin{column}{0.6\textwidth}
\only<1-3>{
\begin{itemize}
\item<1-> Execution continues with OCaml code
\begin{itemize}
\item \funcname{camlDefault\_\_main\_269}
\item \funcname{camlDefault\_\_entry}
\item \funcname{caml\_program}
\end{itemize}
\item<1-> \funcname{camlDefault\_\_main\_269} raises a \typename{Failure} exception
\item<1-> \typename{Failure} is caught by \funcname{caml\_start\_program}
\begin{itemize}
\item<1-> The handler set the 2nd LSB of the exception value to \funcarg{1}
\item<3-> And then forces \funcname{caml\_start\_program} to terminate
\end{itemize}
\end{itemize}
}
\bigskip
\only<1,3>{\listocaml[highlightlines=3]{../src/default/default.ml}{default/default.ml}}%
\only<2>{\listasm[firstline=66,lastline=72,highlightlines=69]{src/ocaml/caml\_start\_program.s}{caml\_start\_program}}%
\end{column}
\begin{column}{0.4\textwidth}
\centering
\providecommand\step{4}\input{diagrams/default/default.tex}
\end{column}
\end{columns}
\end{frame}
%\only<>{\listasm[firstline=47,lastline=65]{src/ocaml/caml\_start\_program.s}{caml\_start\_program}}


\begin{frame}{Back to C}{Clean up, complain and terminate}
\begin{columns}
\begin{column}{0.45\textwidth}
\begin{itemize}
\item Backtrace:
\begin{itemize}
\item \funcname{caml\_start\_program} $\rightarrow$ OCaml code
\item \funcname{caml\_startup\_common}
\item \funcname{caml\_startup\_exn}
\item \funcname{caml\_startup}
\item \funcname{caml\_main}
\item \funcname{main}
\end{itemize}
\smallskip
\item \funcname{caml\_startup} checks the return value of \funcname{caml\_startup\_exn}
\begin{itemize}
\item The return value has been specially marked by \funcname{caml\_start\_program} handler
\item Calls \funcname{caml\_fatal\_uncaught\_exception}
\end{itemize}
\item<2-> \funcname{caml\_fatal\_uncaught\_exception} either
\begin{itemize}
\item<2-> Calls the user custom uncaught exception handler, if set with \mintinline{ocaml}{Printexc.set_uncaught_exception_handler fn}\footnotemark
\item<2-> Or falls back to the default one
\item<2-> Which notifies about the uncaught exception
\end{itemize}
\item<4-> Terminates the program either with \funcname{abort} or with a non-zero exit code
\end{itemize}
\end{column}
\begin{column}{0.55\textwidth}
\only<1>{
\listc[firstline=84,lastline=89,highlightlines=88]{../ocaml/runtime/caml/mlvalues.h}{runtime/caml/mlvalues.h:84}
\listc[firstline=138,lastline=143,highlightlines={141-142}]{../ocaml/runtime/startup\_nat.c}{runtime/startup\_nat.c:138}
}%
\only<2>{
\listc[firstline=138,highlightlines={147,149}]{../ocaml/runtime/printexc.c}{runtime/printexc.c:138}
}%
\only<3>{
\listc[firstline=110,lastline=134,highlightlines=129]{../ocaml/runtime/printexc.c}{runtime/printexc.c:138}
}%
\only<4>{
\listc[firstline=138,highlightlines={152,154}]{../ocaml/runtime/printexc.c}{runtime/printexc.c:138}
}%
\end{column}
\end{columns}
\footnotetext[1]{\url{https://v2.ocaml.org/api/Printexc.html\#1\_Uncaughtexceptions}}
\end{frame}
}%
\only<3-5>{\providecommand\step{3}%
% Runtime's default exception handler
%
\subsection*{Runtime's default exception handler}
\frameSubsection{pictures/The\_Architect.png}{}


\begin{frame}{Default exception handler}
\begin{columns}
\begin{column}{0.5\textwidth}
\begin{itemize}
\item \localname{domain\_state->exn\_handler} points to a trap located before \funcname{entry}!
\item What installed this very first trap there?
\end{itemize}
\bigskip
\listocaml{../src/default/default.ml}{default/default.ml}
\bigskip
\inputminted[fontsize=\footnotesize]{text}{../src/default/default.out}
\end{column}
\begin{column}{0.5\textwidth}
\centering
\only<1>{\providecommand\step{0}\input{diagrams/default/default.tex}}%
\only<2>{\providecommand\step{4}\input{diagrams/default/default.tex}}%
\end{column}
\end{columns}
\end{frame}


\begin{frame}{Startup}{How did we get there by the way?}
\begin{columns}
\begin{column}{0.45\textwidth}
\begin{itemize}
\item The entry point address of the binary is \funcname{main} (\funcname{\_start}) from the runtime
\item The program starts like a C program:
\begin{itemize}
\item \funcname{caml\_start\_program}
\item \funcname{caml\_startup\_common}
\item \funcname{caml\_startup\_exn}
\item \funcname{caml\_startup}
\item \funcname{caml\_main}
\item \funcname{main}
\end{itemize}
%\item \funcname{caml\_startup\_common}: initializes GC, domains \dots
% Also: codefrag, locale, custom opeartions, frame_descr, signals, stack
\item \funcname{caml\_start\_program} is the transition point between the C realm and the OCaml one
\end{itemize}
\bigskip
\listocaml{../src/default/default.ml}{default/default.ml}
\end{column}
\begin{column}{0.55\textwidth}
\listasm[lastline=24,highlightlines={22-24}]{src/ocaml/caml\_start\_program.s}{runtime/amd64.S:604}
\end{column}
\end{columns}
\end{frame}


\begin{frame}{Setup to jump into OCaml entry point}{\funcname{caml\_start\_program} initializes the main fiber}
\begin{columns}
\begin{column}{0.6\textwidth}
\only<1,3,5>{
\begin{itemize}
\item<1-> Stores information to allow switching from OCaml stack to the C stack
\item<3-> Constructs the default exception handler
\item<5-> Switches to the OCaml stack and calls \funcname{caml\_program}
\end{itemize}
\bigskip
\listocaml{../src/default/default.ml}{default/default.ml}
}%
\only<2>{
\listasm[firstline=22,lastline=41,highlightlines={25-27}]{src/ocaml/caml\_start\_program.s}{caml\_start\_program}
}%
\only<4>{
\mintasm[firstline=22,lastline=41,highlightlines={31-38}]{src/ocaml/caml\_start\_program.s}
\listasm[firstline=66,lastline=72]{src/ocaml/caml\_start\_program.s}{caml\_start\_program}
}%
\only<6>{
\listasm[firstline=22,lastline=41,highlightlines={39-41}]{src/ocaml/caml\_start\_program.s}{caml\_start\_program}
}%
\end{column}
\begin{column}{0.4\textwidth}
\centering
\only<1-2>{\providecommand\step{2}\input{diagrams/default/default.tex}}%
\only<3-5>{\providecommand\step{3}\input{diagrams/default/default.tex}}%
\onslide*<6->{\providecommand\step{4}\input{diagrams/default/default.tex}}%
\end{column}
\end{columns}
\end{frame}

\begin{frame}{Executing the OCaml program}{And raising an exception without handler}
\begin{columns}
\begin{column}{0.6\textwidth}
\only<1-3>{
\begin{itemize}
\item<1-> Execution continues with OCaml code
\begin{itemize}
\item \funcname{camlDefault\_\_main\_269}
\item \funcname{camlDefault\_\_entry}
\item \funcname{caml\_program}
\end{itemize}
\item<1-> \funcname{camlDefault\_\_main\_269} raises a \typename{Failure} exception
\item<1-> \typename{Failure} is caught by \funcname{caml\_start\_program}
\begin{itemize}
\item<1-> The handler set the 2nd LSB of the exception value to \funcarg{1}
\item<3-> And then forces \funcname{caml\_start\_program} to terminate
\end{itemize}
\end{itemize}
}
\bigskip
\only<1,3>{\listocaml[highlightlines=3]{../src/default/default.ml}{default/default.ml}}%
\only<2>{\listasm[firstline=66,lastline=72,highlightlines=69]{src/ocaml/caml\_start\_program.s}{caml\_start\_program}}%
\end{column}
\begin{column}{0.4\textwidth}
\centering
\providecommand\step{4}\input{diagrams/default/default.tex}
\end{column}
\end{columns}
\end{frame}
%\only<>{\listasm[firstline=47,lastline=65]{src/ocaml/caml\_start\_program.s}{caml\_start\_program}}


\begin{frame}{Back to C}{Clean up, complain and terminate}
\begin{columns}
\begin{column}{0.45\textwidth}
\begin{itemize}
\item Backtrace:
\begin{itemize}
\item \funcname{caml\_start\_program} $\rightarrow$ OCaml code
\item \funcname{caml\_startup\_common}
\item \funcname{caml\_startup\_exn}
\item \funcname{caml\_startup}
\item \funcname{caml\_main}
\item \funcname{main}
\end{itemize}
\smallskip
\item \funcname{caml\_startup} checks the return value of \funcname{caml\_startup\_exn}
\begin{itemize}
\item The return value has been specially marked by \funcname{caml\_start\_program} handler
\item Calls \funcname{caml\_fatal\_uncaught\_exception}
\end{itemize}
\item<2-> \funcname{caml\_fatal\_uncaught\_exception} either
\begin{itemize}
\item<2-> Calls the user custom uncaught exception handler, if set with \mintinline{ocaml}{Printexc.set_uncaught_exception_handler fn}\footnotemark
\item<2-> Or falls back to the default one
\item<2-> Which notifies about the uncaught exception
\end{itemize}
\item<4-> Terminates the program either with \funcname{abort} or with a non-zero exit code
\end{itemize}
\end{column}
\begin{column}{0.55\textwidth}
\only<1>{
\listc[firstline=84,lastline=89,highlightlines=88]{../ocaml/runtime/caml/mlvalues.h}{runtime/caml/mlvalues.h:84}
\listc[firstline=138,lastline=143,highlightlines={141-142}]{../ocaml/runtime/startup\_nat.c}{runtime/startup\_nat.c:138}
}%
\only<2>{
\listc[firstline=138,highlightlines={147,149}]{../ocaml/runtime/printexc.c}{runtime/printexc.c:138}
}%
\only<3>{
\listc[firstline=110,lastline=134,highlightlines=129]{../ocaml/runtime/printexc.c}{runtime/printexc.c:138}
}%
\only<4>{
\listc[firstline=138,highlightlines={152,154}]{../ocaml/runtime/printexc.c}{runtime/printexc.c:138}
}%
\end{column}
\end{columns}
\footnotetext[1]{\url{https://v2.ocaml.org/api/Printexc.html\#1\_Uncaughtexceptions}}
\end{frame}
}%
\onslide*<6->{\providecommand\step{4}%
% Runtime's default exception handler
%
\subsection*{Runtime's default exception handler}
\frameSubsection{pictures/The\_Architect.png}{}


\begin{frame}{Default exception handler}
\begin{columns}
\begin{column}{0.5\textwidth}
\begin{itemize}
\item \localname{domain\_state->exn\_handler} points to a trap located before \funcname{entry}!
\item What installed this very first trap there?
\end{itemize}
\bigskip
\listocaml{../src/default/default.ml}{default/default.ml}
\bigskip
\inputminted[fontsize=\footnotesize]{text}{../src/default/default.out}
\end{column}
\begin{column}{0.5\textwidth}
\centering
\only<1>{\providecommand\step{0}\input{diagrams/default/default.tex}}%
\only<2>{\providecommand\step{4}\input{diagrams/default/default.tex}}%
\end{column}
\end{columns}
\end{frame}


\begin{frame}{Startup}{How did we get there by the way?}
\begin{columns}
\begin{column}{0.45\textwidth}
\begin{itemize}
\item The entry point address of the binary is \funcname{main} (\funcname{\_start}) from the runtime
\item The program starts like a C program:
\begin{itemize}
\item \funcname{caml\_start\_program}
\item \funcname{caml\_startup\_common}
\item \funcname{caml\_startup\_exn}
\item \funcname{caml\_startup}
\item \funcname{caml\_main}
\item \funcname{main}
\end{itemize}
%\item \funcname{caml\_startup\_common}: initializes GC, domains \dots
% Also: codefrag, locale, custom opeartions, frame_descr, signals, stack
\item \funcname{caml\_start\_program} is the transition point between the C realm and the OCaml one
\end{itemize}
\bigskip
\listocaml{../src/default/default.ml}{default/default.ml}
\end{column}
\begin{column}{0.55\textwidth}
\listasm[lastline=24,highlightlines={22-24}]{src/ocaml/caml\_start\_program.s}{runtime/amd64.S:604}
\end{column}
\end{columns}
\end{frame}


\begin{frame}{Setup to jump into OCaml entry point}{\funcname{caml\_start\_program} initializes the main fiber}
\begin{columns}
\begin{column}{0.6\textwidth}
\only<1,3,5>{
\begin{itemize}
\item<1-> Stores information to allow switching from OCaml stack to the C stack
\item<3-> Constructs the default exception handler
\item<5-> Switches to the OCaml stack and calls \funcname{caml\_program}
\end{itemize}
\bigskip
\listocaml{../src/default/default.ml}{default/default.ml}
}%
\only<2>{
\listasm[firstline=22,lastline=41,highlightlines={25-27}]{src/ocaml/caml\_start\_program.s}{caml\_start\_program}
}%
\only<4>{
\mintasm[firstline=22,lastline=41,highlightlines={31-38}]{src/ocaml/caml\_start\_program.s}
\listasm[firstline=66,lastline=72]{src/ocaml/caml\_start\_program.s}{caml\_start\_program}
}%
\only<6>{
\listasm[firstline=22,lastline=41,highlightlines={39-41}]{src/ocaml/caml\_start\_program.s}{caml\_start\_program}
}%
\end{column}
\begin{column}{0.4\textwidth}
\centering
\only<1-2>{\providecommand\step{2}\input{diagrams/default/default.tex}}%
\only<3-5>{\providecommand\step{3}\input{diagrams/default/default.tex}}%
\onslide*<6->{\providecommand\step{4}\input{diagrams/default/default.tex}}%
\end{column}
\end{columns}
\end{frame}

\begin{frame}{Executing the OCaml program}{And raising an exception without handler}
\begin{columns}
\begin{column}{0.6\textwidth}
\only<1-3>{
\begin{itemize}
\item<1-> Execution continues with OCaml code
\begin{itemize}
\item \funcname{camlDefault\_\_main\_269}
\item \funcname{camlDefault\_\_entry}
\item \funcname{caml\_program}
\end{itemize}
\item<1-> \funcname{camlDefault\_\_main\_269} raises a \typename{Failure} exception
\item<1-> \typename{Failure} is caught by \funcname{caml\_start\_program}
\begin{itemize}
\item<1-> The handler set the 2nd LSB of the exception value to \funcarg{1}
\item<3-> And then forces \funcname{caml\_start\_program} to terminate
\end{itemize}
\end{itemize}
}
\bigskip
\only<1,3>{\listocaml[highlightlines=3]{../src/default/default.ml}{default/default.ml}}%
\only<2>{\listasm[firstline=66,lastline=72,highlightlines=69]{src/ocaml/caml\_start\_program.s}{caml\_start\_program}}%
\end{column}
\begin{column}{0.4\textwidth}
\centering
\providecommand\step{4}\input{diagrams/default/default.tex}
\end{column}
\end{columns}
\end{frame}
%\only<>{\listasm[firstline=47,lastline=65]{src/ocaml/caml\_start\_program.s}{caml\_start\_program}}


\begin{frame}{Back to C}{Clean up, complain and terminate}
\begin{columns}
\begin{column}{0.45\textwidth}
\begin{itemize}
\item Backtrace:
\begin{itemize}
\item \funcname{caml\_start\_program} $\rightarrow$ OCaml code
\item \funcname{caml\_startup\_common}
\item \funcname{caml\_startup\_exn}
\item \funcname{caml\_startup}
\item \funcname{caml\_main}
\item \funcname{main}
\end{itemize}
\smallskip
\item \funcname{caml\_startup} checks the return value of \funcname{caml\_startup\_exn}
\begin{itemize}
\item The return value has been specially marked by \funcname{caml\_start\_program} handler
\item Calls \funcname{caml\_fatal\_uncaught\_exception}
\end{itemize}
\item<2-> \funcname{caml\_fatal\_uncaught\_exception} either
\begin{itemize}
\item<2-> Calls the user custom uncaught exception handler, if set with \mintinline{ocaml}{Printexc.set_uncaught_exception_handler fn}\footnotemark
\item<2-> Or falls back to the default one
\item<2-> Which notifies about the uncaught exception
\end{itemize}
\item<4-> Terminates the program either with \funcname{abort} or with a non-zero exit code
\end{itemize}
\end{column}
\begin{column}{0.55\textwidth}
\only<1>{
\listc[firstline=84,lastline=89,highlightlines=88]{../ocaml/runtime/caml/mlvalues.h}{runtime/caml/mlvalues.h:84}
\listc[firstline=138,lastline=143,highlightlines={141-142}]{../ocaml/runtime/startup\_nat.c}{runtime/startup\_nat.c:138}
}%
\only<2>{
\listc[firstline=138,highlightlines={147,149}]{../ocaml/runtime/printexc.c}{runtime/printexc.c:138}
}%
\only<3>{
\listc[firstline=110,lastline=134,highlightlines=129]{../ocaml/runtime/printexc.c}{runtime/printexc.c:138}
}%
\only<4>{
\listc[firstline=138,highlightlines={152,154}]{../ocaml/runtime/printexc.c}{runtime/printexc.c:138}
}%
\end{column}
\end{columns}
\footnotetext[1]{\url{https://v2.ocaml.org/api/Printexc.html\#1\_Uncaughtexceptions}}
\end{frame}
}%
\end{column}
\end{columns}
\end{frame}

\begin{frame}{Executing the OCaml program}{And raising an exception without handler}
\begin{columns}
\begin{column}{0.6\textwidth}
\only<1-3>{
\begin{itemize}
\item<1-> Execution continues with OCaml code
\begin{itemize}
\item \funcname{camlDefault\_\_main\_269}
\item \funcname{camlDefault\_\_entry}
\item \funcname{caml\_program}
\end{itemize}
\item<1-> \funcname{camlDefault\_\_main\_269} raises a \typename{Failure} exception
\item<1-> \typename{Failure} is caught by \funcname{caml\_start\_program}
\begin{itemize}
\item<1-> The handler set the 2nd LSB of the exception value to \funcarg{1}
\item<3-> And then forces \funcname{caml\_start\_program} to terminate
\end{itemize}
\end{itemize}
}
\bigskip
\only<1,3>{\listocaml[highlightlines=3]{../src/default/default.ml}{default/default.ml}}%
\only<2>{\listasm[firstline=66,lastline=72,highlightlines=69]{src/ocaml/caml\_start\_program.s}{caml\_start\_program}}%
\end{column}
\begin{column}{0.4\textwidth}
\centering
\providecommand\step{4}%
% Runtime's default exception handler
%
\subsection*{Runtime's default exception handler}
\frameSubsection{pictures/The\_Architect.png}{}


\begin{frame}{Default exception handler}
\begin{columns}
\begin{column}{0.5\textwidth}
\begin{itemize}
\item \localname{domain\_state->exn\_handler} points to a trap located before \funcname{entry}!
\item What installed this very first trap there?
\end{itemize}
\bigskip
\listocaml{../src/default/default.ml}{default/default.ml}
\bigskip
\inputminted[fontsize=\footnotesize]{text}{../src/default/default.out}
\end{column}
\begin{column}{0.5\textwidth}
\centering
\only<1>{\providecommand\step{0}\input{diagrams/default/default.tex}}%
\only<2>{\providecommand\step{4}\input{diagrams/default/default.tex}}%
\end{column}
\end{columns}
\end{frame}


\begin{frame}{Startup}{How did we get there by the way?}
\begin{columns}
\begin{column}{0.45\textwidth}
\begin{itemize}
\item The entry point address of the binary is \funcname{main} (\funcname{\_start}) from the runtime
\item The program starts like a C program:
\begin{itemize}
\item \funcname{caml\_start\_program}
\item \funcname{caml\_startup\_common}
\item \funcname{caml\_startup\_exn}
\item \funcname{caml\_startup}
\item \funcname{caml\_main}
\item \funcname{main}
\end{itemize}
%\item \funcname{caml\_startup\_common}: initializes GC, domains \dots
% Also: codefrag, locale, custom opeartions, frame_descr, signals, stack
\item \funcname{caml\_start\_program} is the transition point between the C realm and the OCaml one
\end{itemize}
\bigskip
\listocaml{../src/default/default.ml}{default/default.ml}
\end{column}
\begin{column}{0.55\textwidth}
\listasm[lastline=24,highlightlines={22-24}]{src/ocaml/caml\_start\_program.s}{runtime/amd64.S:604}
\end{column}
\end{columns}
\end{frame}


\begin{frame}{Setup to jump into OCaml entry point}{\funcname{caml\_start\_program} initializes the main fiber}
\begin{columns}
\begin{column}{0.6\textwidth}
\only<1,3,5>{
\begin{itemize}
\item<1-> Stores information to allow switching from OCaml stack to the C stack
\item<3-> Constructs the default exception handler
\item<5-> Switches to the OCaml stack and calls \funcname{caml\_program}
\end{itemize}
\bigskip
\listocaml{../src/default/default.ml}{default/default.ml}
}%
\only<2>{
\listasm[firstline=22,lastline=41,highlightlines={25-27}]{src/ocaml/caml\_start\_program.s}{caml\_start\_program}
}%
\only<4>{
\mintasm[firstline=22,lastline=41,highlightlines={31-38}]{src/ocaml/caml\_start\_program.s}
\listasm[firstline=66,lastline=72]{src/ocaml/caml\_start\_program.s}{caml\_start\_program}
}%
\only<6>{
\listasm[firstline=22,lastline=41,highlightlines={39-41}]{src/ocaml/caml\_start\_program.s}{caml\_start\_program}
}%
\end{column}
\begin{column}{0.4\textwidth}
\centering
\only<1-2>{\providecommand\step{2}\input{diagrams/default/default.tex}}%
\only<3-5>{\providecommand\step{3}\input{diagrams/default/default.tex}}%
\onslide*<6->{\providecommand\step{4}\input{diagrams/default/default.tex}}%
\end{column}
\end{columns}
\end{frame}

\begin{frame}{Executing the OCaml program}{And raising an exception without handler}
\begin{columns}
\begin{column}{0.6\textwidth}
\only<1-3>{
\begin{itemize}
\item<1-> Execution continues with OCaml code
\begin{itemize}
\item \funcname{camlDefault\_\_main\_269}
\item \funcname{camlDefault\_\_entry}
\item \funcname{caml\_program}
\end{itemize}
\item<1-> \funcname{camlDefault\_\_main\_269} raises a \typename{Failure} exception
\item<1-> \typename{Failure} is caught by \funcname{caml\_start\_program}
\begin{itemize}
\item<1-> The handler set the 2nd LSB of the exception value to \funcarg{1}
\item<3-> And then forces \funcname{caml\_start\_program} to terminate
\end{itemize}
\end{itemize}
}
\bigskip
\only<1,3>{\listocaml[highlightlines=3]{../src/default/default.ml}{default/default.ml}}%
\only<2>{\listasm[firstline=66,lastline=72,highlightlines=69]{src/ocaml/caml\_start\_program.s}{caml\_start\_program}}%
\end{column}
\begin{column}{0.4\textwidth}
\centering
\providecommand\step{4}\input{diagrams/default/default.tex}
\end{column}
\end{columns}
\end{frame}
%\only<>{\listasm[firstline=47,lastline=65]{src/ocaml/caml\_start\_program.s}{caml\_start\_program}}


\begin{frame}{Back to C}{Clean up, complain and terminate}
\begin{columns}
\begin{column}{0.45\textwidth}
\begin{itemize}
\item Backtrace:
\begin{itemize}
\item \funcname{caml\_start\_program} $\rightarrow$ OCaml code
\item \funcname{caml\_startup\_common}
\item \funcname{caml\_startup\_exn}
\item \funcname{caml\_startup}
\item \funcname{caml\_main}
\item \funcname{main}
\end{itemize}
\smallskip
\item \funcname{caml\_startup} checks the return value of \funcname{caml\_startup\_exn}
\begin{itemize}
\item The return value has been specially marked by \funcname{caml\_start\_program} handler
\item Calls \funcname{caml\_fatal\_uncaught\_exception}
\end{itemize}
\item<2-> \funcname{caml\_fatal\_uncaught\_exception} either
\begin{itemize}
\item<2-> Calls the user custom uncaught exception handler, if set with \mintinline{ocaml}{Printexc.set_uncaught_exception_handler fn}\footnotemark
\item<2-> Or falls back to the default one
\item<2-> Which notifies about the uncaught exception
\end{itemize}
\item<4-> Terminates the program either with \funcname{abort} or with a non-zero exit code
\end{itemize}
\end{column}
\begin{column}{0.55\textwidth}
\only<1>{
\listc[firstline=84,lastline=89,highlightlines=88]{../ocaml/runtime/caml/mlvalues.h}{runtime/caml/mlvalues.h:84}
\listc[firstline=138,lastline=143,highlightlines={141-142}]{../ocaml/runtime/startup\_nat.c}{runtime/startup\_nat.c:138}
}%
\only<2>{
\listc[firstline=138,highlightlines={147,149}]{../ocaml/runtime/printexc.c}{runtime/printexc.c:138}
}%
\only<3>{
\listc[firstline=110,lastline=134,highlightlines=129]{../ocaml/runtime/printexc.c}{runtime/printexc.c:138}
}%
\only<4>{
\listc[firstline=138,highlightlines={152,154}]{../ocaml/runtime/printexc.c}{runtime/printexc.c:138}
}%
\end{column}
\end{columns}
\footnotetext[1]{\url{https://v2.ocaml.org/api/Printexc.html\#1\_Uncaughtexceptions}}
\end{frame}

\end{column}
\end{columns}
\end{frame}
%\only<>{\listasm[firstline=47,lastline=65]{src/ocaml/caml\_start\_program.s}{caml\_start\_program}}


\begin{frame}{Back to C}{Clean up, complain and terminate}
\begin{columns}
\begin{column}{0.45\textwidth}
\begin{itemize}
\item Backtrace:
\begin{itemize}
\item \funcname{caml\_start\_program} $\rightarrow$ OCaml code
\item \funcname{caml\_startup\_common}
\item \funcname{caml\_startup\_exn}
\item \funcname{caml\_startup}
\item \funcname{caml\_main}
\item \funcname{main}
\end{itemize}
\smallskip
\item \funcname{caml\_startup} checks the return value of \funcname{caml\_startup\_exn}
\begin{itemize}
\item The return value has been specially marked by \funcname{caml\_start\_program} handler
\item Calls \funcname{caml\_fatal\_uncaught\_exception}
\end{itemize}
\item<2-> \funcname{caml\_fatal\_uncaught\_exception} either
\begin{itemize}
\item<2-> Calls the user custom uncaught exception handler, if set with \mintinline{ocaml}{Printexc.set_uncaught_exception_handler fn}\footnotemark
\item<2-> Or falls back to the default one
\item<2-> Which notifies about the uncaught exception
\end{itemize}
\item<4-> Terminates the program either with \funcname{abort} or with a non-zero exit code
\end{itemize}
\end{column}
\begin{column}{0.55\textwidth}
\only<1>{
\listc[firstline=84,lastline=89,highlightlines=88]{../ocaml/runtime/caml/mlvalues.h}{runtime/caml/mlvalues.h:84}
\listc[firstline=138,lastline=143,highlightlines={141-142}]{../ocaml/runtime/startup\_nat.c}{runtime/startup\_nat.c:138}
}%
\only<2>{
\listc[firstline=138,highlightlines={147,149}]{../ocaml/runtime/printexc.c}{runtime/printexc.c:138}
}%
\only<3>{
\listc[firstline=110,lastline=134,highlightlines=129]{../ocaml/runtime/printexc.c}{runtime/printexc.c:138}
}%
\only<4>{
\listc[firstline=138,highlightlines={152,154}]{../ocaml/runtime/printexc.c}{runtime/printexc.c:138}
}%
\end{column}
\end{columns}
\footnotetext[1]{\url{https://v2.ocaml.org/api/Printexc.html\#1\_Uncaughtexceptions}}
\end{frame}

\end{column}
\end{columns}
\end{frame}
%\only<>{\listasm[firstline=47,lastline=65]{src/ocaml/caml\_start\_program.s}{caml\_start\_program}}


\begin{frame}{Back to C}{Clean up, complain and terminate}
\begin{columns}
\begin{column}{0.45\textwidth}
\begin{itemize}
\item Backtrace:
\begin{itemize}
\item \funcname{caml\_start\_program} $\rightarrow$ OCaml code
\item \funcname{caml\_startup\_common}
\item \funcname{caml\_startup\_exn}
\item \funcname{caml\_startup}
\item \funcname{caml\_main}
\item \funcname{main}
\end{itemize}
\smallskip
\item \funcname{caml\_startup} checks the return value of \funcname{caml\_startup\_exn}
\begin{itemize}
\item The return value has been specially marked by \funcname{caml\_start\_program} handler
\item Calls \funcname{caml\_fatal\_uncaught\_exception}
\end{itemize}
\item<2-> \funcname{caml\_fatal\_uncaught\_exception} either
\begin{itemize}
\item<2-> Calls the user custom uncaught exception handler, if set with \mintinline{ocaml}{Printexc.set_uncaught_exception_handler fn}\footnotemark
\item<2-> Or falls back to the default one
\item<2-> Which notifies about the uncaught exception
\end{itemize}
\item<4-> Terminates the program either with \funcname{abort} or with a non-zero exit code
\end{itemize}
\end{column}
\begin{column}{0.55\textwidth}
\only<1>{
\listc[firstline=84,lastline=89,highlightlines=88]{../ocaml/runtime/caml/mlvalues.h}{runtime/caml/mlvalues.h:84}
\listc[firstline=138,lastline=143,highlightlines={141-142}]{../ocaml/runtime/startup\_nat.c}{runtime/startup\_nat.c:138}
}%
\only<2>{
\listc[firstline=138,highlightlines={147,149}]{../ocaml/runtime/printexc.c}{runtime/printexc.c:138}
}%
\only<3>{
\listc[firstline=110,lastline=134,highlightlines=129]{../ocaml/runtime/printexc.c}{runtime/printexc.c:138}
}%
\only<4>{
\listc[firstline=138,highlightlines={152,154}]{../ocaml/runtime/printexc.c}{runtime/printexc.c:138}
}%
\end{column}
\end{columns}
\footnotetext[1]{\url{https://v2.ocaml.org/api/Printexc.html\#1\_Uncaughtexceptions}}
\end{frame}
}%
\only<3-5>{\providecommand\step{3}%
% Runtime's default exception handler
%
\subsection*{Runtime's default exception handler}
\frameSubsection{pictures/The\_Architect.png}{}


\begin{frame}{Default exception handler}
\begin{columns}
\begin{column}{0.5\textwidth}
\begin{itemize}
\item \localname{domain\_state->exn\_handler} points to a trap located before \funcname{entry}!
\item What installed this very first trap there?
\end{itemize}
\bigskip
\listocaml{../src/default/default.ml}{default/default.ml}
\bigskip
\inputminted[fontsize=\footnotesize]{text}{../src/default/default.out}
\end{column}
\begin{column}{0.5\textwidth}
\centering
\only<1>{\providecommand\step{0}%
% Runtime's default exception handler
%
\subsection*{Runtime's default exception handler}
\frameSubsection{pictures/The\_Architect.png}{}


\begin{frame}{Default exception handler}
\begin{columns}
\begin{column}{0.5\textwidth}
\begin{itemize}
\item \localname{domain\_state->exn\_handler} points to a trap located before \funcname{entry}!
\item What installed this very first trap there?
\end{itemize}
\bigskip
\listocaml{../src/default/default.ml}{default/default.ml}
\bigskip
\inputminted[fontsize=\footnotesize]{text}{../src/default/default.out}
\end{column}
\begin{column}{0.5\textwidth}
\centering
\only<1>{\providecommand\step{0}%
% Runtime's default exception handler
%
\subsection*{Runtime's default exception handler}
\frameSubsection{pictures/The\_Architect.png}{}


\begin{frame}{Default exception handler}
\begin{columns}
\begin{column}{0.5\textwidth}
\begin{itemize}
\item \localname{domain\_state->exn\_handler} points to a trap located before \funcname{entry}!
\item What installed this very first trap there?
\end{itemize}
\bigskip
\listocaml{../src/default/default.ml}{default/default.ml}
\bigskip
\inputminted[fontsize=\footnotesize]{text}{../src/default/default.out}
\end{column}
\begin{column}{0.5\textwidth}
\centering
\only<1>{\providecommand\step{0}\input{diagrams/default/default.tex}}%
\only<2>{\providecommand\step{4}\input{diagrams/default/default.tex}}%
\end{column}
\end{columns}
\end{frame}


\begin{frame}{Startup}{How did we get there by the way?}
\begin{columns}
\begin{column}{0.45\textwidth}
\begin{itemize}
\item The entry point address of the binary is \funcname{main} (\funcname{\_start}) from the runtime
\item The program starts like a C program:
\begin{itemize}
\item \funcname{caml\_start\_program}
\item \funcname{caml\_startup\_common}
\item \funcname{caml\_startup\_exn}
\item \funcname{caml\_startup}
\item \funcname{caml\_main}
\item \funcname{main}
\end{itemize}
%\item \funcname{caml\_startup\_common}: initializes GC, domains \dots
% Also: codefrag, locale, custom opeartions, frame_descr, signals, stack
\item \funcname{caml\_start\_program} is the transition point between the C realm and the OCaml one
\end{itemize}
\bigskip
\listocaml{../src/default/default.ml}{default/default.ml}
\end{column}
\begin{column}{0.55\textwidth}
\listasm[lastline=24,highlightlines={22-24}]{src/ocaml/caml\_start\_program.s}{runtime/amd64.S:604}
\end{column}
\end{columns}
\end{frame}


\begin{frame}{Setup to jump into OCaml entry point}{\funcname{caml\_start\_program} initializes the main fiber}
\begin{columns}
\begin{column}{0.6\textwidth}
\only<1,3,5>{
\begin{itemize}
\item<1-> Stores information to allow switching from OCaml stack to the C stack
\item<3-> Constructs the default exception handler
\item<5-> Switches to the OCaml stack and calls \funcname{caml\_program}
\end{itemize}
\bigskip
\listocaml{../src/default/default.ml}{default/default.ml}
}%
\only<2>{
\listasm[firstline=22,lastline=41,highlightlines={25-27}]{src/ocaml/caml\_start\_program.s}{caml\_start\_program}
}%
\only<4>{
\mintasm[firstline=22,lastline=41,highlightlines={31-38}]{src/ocaml/caml\_start\_program.s}
\listasm[firstline=66,lastline=72]{src/ocaml/caml\_start\_program.s}{caml\_start\_program}
}%
\only<6>{
\listasm[firstline=22,lastline=41,highlightlines={39-41}]{src/ocaml/caml\_start\_program.s}{caml\_start\_program}
}%
\end{column}
\begin{column}{0.4\textwidth}
\centering
\only<1-2>{\providecommand\step{2}\input{diagrams/default/default.tex}}%
\only<3-5>{\providecommand\step{3}\input{diagrams/default/default.tex}}%
\onslide*<6->{\providecommand\step{4}\input{diagrams/default/default.tex}}%
\end{column}
\end{columns}
\end{frame}

\begin{frame}{Executing the OCaml program}{And raising an exception without handler}
\begin{columns}
\begin{column}{0.6\textwidth}
\only<1-3>{
\begin{itemize}
\item<1-> Execution continues with OCaml code
\begin{itemize}
\item \funcname{camlDefault\_\_main\_269}
\item \funcname{camlDefault\_\_entry}
\item \funcname{caml\_program}
\end{itemize}
\item<1-> \funcname{camlDefault\_\_main\_269} raises a \typename{Failure} exception
\item<1-> \typename{Failure} is caught by \funcname{caml\_start\_program}
\begin{itemize}
\item<1-> The handler set the 2nd LSB of the exception value to \funcarg{1}
\item<3-> And then forces \funcname{caml\_start\_program} to terminate
\end{itemize}
\end{itemize}
}
\bigskip
\only<1,3>{\listocaml[highlightlines=3]{../src/default/default.ml}{default/default.ml}}%
\only<2>{\listasm[firstline=66,lastline=72,highlightlines=69]{src/ocaml/caml\_start\_program.s}{caml\_start\_program}}%
\end{column}
\begin{column}{0.4\textwidth}
\centering
\providecommand\step{4}\input{diagrams/default/default.tex}
\end{column}
\end{columns}
\end{frame}
%\only<>{\listasm[firstline=47,lastline=65]{src/ocaml/caml\_start\_program.s}{caml\_start\_program}}


\begin{frame}{Back to C}{Clean up, complain and terminate}
\begin{columns}
\begin{column}{0.45\textwidth}
\begin{itemize}
\item Backtrace:
\begin{itemize}
\item \funcname{caml\_start\_program} $\rightarrow$ OCaml code
\item \funcname{caml\_startup\_common}
\item \funcname{caml\_startup\_exn}
\item \funcname{caml\_startup}
\item \funcname{caml\_main}
\item \funcname{main}
\end{itemize}
\smallskip
\item \funcname{caml\_startup} checks the return value of \funcname{caml\_startup\_exn}
\begin{itemize}
\item The return value has been specially marked by \funcname{caml\_start\_program} handler
\item Calls \funcname{caml\_fatal\_uncaught\_exception}
\end{itemize}
\item<2-> \funcname{caml\_fatal\_uncaught\_exception} either
\begin{itemize}
\item<2-> Calls the user custom uncaught exception handler, if set with \mintinline{ocaml}{Printexc.set_uncaught_exception_handler fn}\footnotemark
\item<2-> Or falls back to the default one
\item<2-> Which notifies about the uncaught exception
\end{itemize}
\item<4-> Terminates the program either with \funcname{abort} or with a non-zero exit code
\end{itemize}
\end{column}
\begin{column}{0.55\textwidth}
\only<1>{
\listc[firstline=84,lastline=89,highlightlines=88]{../ocaml/runtime/caml/mlvalues.h}{runtime/caml/mlvalues.h:84}
\listc[firstline=138,lastline=143,highlightlines={141-142}]{../ocaml/runtime/startup\_nat.c}{runtime/startup\_nat.c:138}
}%
\only<2>{
\listc[firstline=138,highlightlines={147,149}]{../ocaml/runtime/printexc.c}{runtime/printexc.c:138}
}%
\only<3>{
\listc[firstline=110,lastline=134,highlightlines=129]{../ocaml/runtime/printexc.c}{runtime/printexc.c:138}
}%
\only<4>{
\listc[firstline=138,highlightlines={152,154}]{../ocaml/runtime/printexc.c}{runtime/printexc.c:138}
}%
\end{column}
\end{columns}
\footnotetext[1]{\url{https://v2.ocaml.org/api/Printexc.html\#1\_Uncaughtexceptions}}
\end{frame}
}%
\only<2>{\providecommand\step{4}%
% Runtime's default exception handler
%
\subsection*{Runtime's default exception handler}
\frameSubsection{pictures/The\_Architect.png}{}


\begin{frame}{Default exception handler}
\begin{columns}
\begin{column}{0.5\textwidth}
\begin{itemize}
\item \localname{domain\_state->exn\_handler} points to a trap located before \funcname{entry}!
\item What installed this very first trap there?
\end{itemize}
\bigskip
\listocaml{../src/default/default.ml}{default/default.ml}
\bigskip
\inputminted[fontsize=\footnotesize]{text}{../src/default/default.out}
\end{column}
\begin{column}{0.5\textwidth}
\centering
\only<1>{\providecommand\step{0}\input{diagrams/default/default.tex}}%
\only<2>{\providecommand\step{4}\input{diagrams/default/default.tex}}%
\end{column}
\end{columns}
\end{frame}


\begin{frame}{Startup}{How did we get there by the way?}
\begin{columns}
\begin{column}{0.45\textwidth}
\begin{itemize}
\item The entry point address of the binary is \funcname{main} (\funcname{\_start}) from the runtime
\item The program starts like a C program:
\begin{itemize}
\item \funcname{caml\_start\_program}
\item \funcname{caml\_startup\_common}
\item \funcname{caml\_startup\_exn}
\item \funcname{caml\_startup}
\item \funcname{caml\_main}
\item \funcname{main}
\end{itemize}
%\item \funcname{caml\_startup\_common}: initializes GC, domains \dots
% Also: codefrag, locale, custom opeartions, frame_descr, signals, stack
\item \funcname{caml\_start\_program} is the transition point between the C realm and the OCaml one
\end{itemize}
\bigskip
\listocaml{../src/default/default.ml}{default/default.ml}
\end{column}
\begin{column}{0.55\textwidth}
\listasm[lastline=24,highlightlines={22-24}]{src/ocaml/caml\_start\_program.s}{runtime/amd64.S:604}
\end{column}
\end{columns}
\end{frame}


\begin{frame}{Setup to jump into OCaml entry point}{\funcname{caml\_start\_program} initializes the main fiber}
\begin{columns}
\begin{column}{0.6\textwidth}
\only<1,3,5>{
\begin{itemize}
\item<1-> Stores information to allow switching from OCaml stack to the C stack
\item<3-> Constructs the default exception handler
\item<5-> Switches to the OCaml stack and calls \funcname{caml\_program}
\end{itemize}
\bigskip
\listocaml{../src/default/default.ml}{default/default.ml}
}%
\only<2>{
\listasm[firstline=22,lastline=41,highlightlines={25-27}]{src/ocaml/caml\_start\_program.s}{caml\_start\_program}
}%
\only<4>{
\mintasm[firstline=22,lastline=41,highlightlines={31-38}]{src/ocaml/caml\_start\_program.s}
\listasm[firstline=66,lastline=72]{src/ocaml/caml\_start\_program.s}{caml\_start\_program}
}%
\only<6>{
\listasm[firstline=22,lastline=41,highlightlines={39-41}]{src/ocaml/caml\_start\_program.s}{caml\_start\_program}
}%
\end{column}
\begin{column}{0.4\textwidth}
\centering
\only<1-2>{\providecommand\step{2}\input{diagrams/default/default.tex}}%
\only<3-5>{\providecommand\step{3}\input{diagrams/default/default.tex}}%
\onslide*<6->{\providecommand\step{4}\input{diagrams/default/default.tex}}%
\end{column}
\end{columns}
\end{frame}

\begin{frame}{Executing the OCaml program}{And raising an exception without handler}
\begin{columns}
\begin{column}{0.6\textwidth}
\only<1-3>{
\begin{itemize}
\item<1-> Execution continues with OCaml code
\begin{itemize}
\item \funcname{camlDefault\_\_main\_269}
\item \funcname{camlDefault\_\_entry}
\item \funcname{caml\_program}
\end{itemize}
\item<1-> \funcname{camlDefault\_\_main\_269} raises a \typename{Failure} exception
\item<1-> \typename{Failure} is caught by \funcname{caml\_start\_program}
\begin{itemize}
\item<1-> The handler set the 2nd LSB of the exception value to \funcarg{1}
\item<3-> And then forces \funcname{caml\_start\_program} to terminate
\end{itemize}
\end{itemize}
}
\bigskip
\only<1,3>{\listocaml[highlightlines=3]{../src/default/default.ml}{default/default.ml}}%
\only<2>{\listasm[firstline=66,lastline=72,highlightlines=69]{src/ocaml/caml\_start\_program.s}{caml\_start\_program}}%
\end{column}
\begin{column}{0.4\textwidth}
\centering
\providecommand\step{4}\input{diagrams/default/default.tex}
\end{column}
\end{columns}
\end{frame}
%\only<>{\listasm[firstline=47,lastline=65]{src/ocaml/caml\_start\_program.s}{caml\_start\_program}}


\begin{frame}{Back to C}{Clean up, complain and terminate}
\begin{columns}
\begin{column}{0.45\textwidth}
\begin{itemize}
\item Backtrace:
\begin{itemize}
\item \funcname{caml\_start\_program} $\rightarrow$ OCaml code
\item \funcname{caml\_startup\_common}
\item \funcname{caml\_startup\_exn}
\item \funcname{caml\_startup}
\item \funcname{caml\_main}
\item \funcname{main}
\end{itemize}
\smallskip
\item \funcname{caml\_startup} checks the return value of \funcname{caml\_startup\_exn}
\begin{itemize}
\item The return value has been specially marked by \funcname{caml\_start\_program} handler
\item Calls \funcname{caml\_fatal\_uncaught\_exception}
\end{itemize}
\item<2-> \funcname{caml\_fatal\_uncaught\_exception} either
\begin{itemize}
\item<2-> Calls the user custom uncaught exception handler, if set with \mintinline{ocaml}{Printexc.set_uncaught_exception_handler fn}\footnotemark
\item<2-> Or falls back to the default one
\item<2-> Which notifies about the uncaught exception
\end{itemize}
\item<4-> Terminates the program either with \funcname{abort} or with a non-zero exit code
\end{itemize}
\end{column}
\begin{column}{0.55\textwidth}
\only<1>{
\listc[firstline=84,lastline=89,highlightlines=88]{../ocaml/runtime/caml/mlvalues.h}{runtime/caml/mlvalues.h:84}
\listc[firstline=138,lastline=143,highlightlines={141-142}]{../ocaml/runtime/startup\_nat.c}{runtime/startup\_nat.c:138}
}%
\only<2>{
\listc[firstline=138,highlightlines={147,149}]{../ocaml/runtime/printexc.c}{runtime/printexc.c:138}
}%
\only<3>{
\listc[firstline=110,lastline=134,highlightlines=129]{../ocaml/runtime/printexc.c}{runtime/printexc.c:138}
}%
\only<4>{
\listc[firstline=138,highlightlines={152,154}]{../ocaml/runtime/printexc.c}{runtime/printexc.c:138}
}%
\end{column}
\end{columns}
\footnotetext[1]{\url{https://v2.ocaml.org/api/Printexc.html\#1\_Uncaughtexceptions}}
\end{frame}
}%
\end{column}
\end{columns}
\end{frame}


\begin{frame}{Startup}{How did we get there by the way?}
\begin{columns}
\begin{column}{0.45\textwidth}
\begin{itemize}
\item The entry point address of the binary is \funcname{main} (\funcname{\_start}) from the runtime
\item The program starts like a C program:
\begin{itemize}
\item \funcname{caml\_start\_program}
\item \funcname{caml\_startup\_common}
\item \funcname{caml\_startup\_exn}
\item \funcname{caml\_startup}
\item \funcname{caml\_main}
\item \funcname{main}
\end{itemize}
%\item \funcname{caml\_startup\_common}: initializes GC, domains \dots
% Also: codefrag, locale, custom opeartions, frame_descr, signals, stack
\item \funcname{caml\_start\_program} is the transition point between the C realm and the OCaml one
\end{itemize}
\bigskip
\listocaml{../src/default/default.ml}{default/default.ml}
\end{column}
\begin{column}{0.55\textwidth}
\listasm[lastline=24,highlightlines={22-24}]{src/ocaml/caml\_start\_program.s}{runtime/amd64.S:604}
\end{column}
\end{columns}
\end{frame}


\begin{frame}{Setup to jump into OCaml entry point}{\funcname{caml\_start\_program} initializes the main fiber}
\begin{columns}
\begin{column}{0.6\textwidth}
\only<1,3,5>{
\begin{itemize}
\item<1-> Stores information to allow switching from OCaml stack to the C stack
\item<3-> Constructs the default exception handler
\item<5-> Switches to the OCaml stack and calls \funcname{caml\_program}
\end{itemize}
\bigskip
\listocaml{../src/default/default.ml}{default/default.ml}
}%
\only<2>{
\listasm[firstline=22,lastline=41,highlightlines={25-27}]{src/ocaml/caml\_start\_program.s}{caml\_start\_program}
}%
\only<4>{
\mintasm[firstline=22,lastline=41,highlightlines={31-38}]{src/ocaml/caml\_start\_program.s}
\listasm[firstline=66,lastline=72]{src/ocaml/caml\_start\_program.s}{caml\_start\_program}
}%
\only<6>{
\listasm[firstline=22,lastline=41,highlightlines={39-41}]{src/ocaml/caml\_start\_program.s}{caml\_start\_program}
}%
\end{column}
\begin{column}{0.4\textwidth}
\centering
\only<1-2>{\providecommand\step{2}%
% Runtime's default exception handler
%
\subsection*{Runtime's default exception handler}
\frameSubsection{pictures/The\_Architect.png}{}


\begin{frame}{Default exception handler}
\begin{columns}
\begin{column}{0.5\textwidth}
\begin{itemize}
\item \localname{domain\_state->exn\_handler} points to a trap located before \funcname{entry}!
\item What installed this very first trap there?
\end{itemize}
\bigskip
\listocaml{../src/default/default.ml}{default/default.ml}
\bigskip
\inputminted[fontsize=\footnotesize]{text}{../src/default/default.out}
\end{column}
\begin{column}{0.5\textwidth}
\centering
\only<1>{\providecommand\step{0}\input{diagrams/default/default.tex}}%
\only<2>{\providecommand\step{4}\input{diagrams/default/default.tex}}%
\end{column}
\end{columns}
\end{frame}


\begin{frame}{Startup}{How did we get there by the way?}
\begin{columns}
\begin{column}{0.45\textwidth}
\begin{itemize}
\item The entry point address of the binary is \funcname{main} (\funcname{\_start}) from the runtime
\item The program starts like a C program:
\begin{itemize}
\item \funcname{caml\_start\_program}
\item \funcname{caml\_startup\_common}
\item \funcname{caml\_startup\_exn}
\item \funcname{caml\_startup}
\item \funcname{caml\_main}
\item \funcname{main}
\end{itemize}
%\item \funcname{caml\_startup\_common}: initializes GC, domains \dots
% Also: codefrag, locale, custom opeartions, frame_descr, signals, stack
\item \funcname{caml\_start\_program} is the transition point between the C realm and the OCaml one
\end{itemize}
\bigskip
\listocaml{../src/default/default.ml}{default/default.ml}
\end{column}
\begin{column}{0.55\textwidth}
\listasm[lastline=24,highlightlines={22-24}]{src/ocaml/caml\_start\_program.s}{runtime/amd64.S:604}
\end{column}
\end{columns}
\end{frame}


\begin{frame}{Setup to jump into OCaml entry point}{\funcname{caml\_start\_program} initializes the main fiber}
\begin{columns}
\begin{column}{0.6\textwidth}
\only<1,3,5>{
\begin{itemize}
\item<1-> Stores information to allow switching from OCaml stack to the C stack
\item<3-> Constructs the default exception handler
\item<5-> Switches to the OCaml stack and calls \funcname{caml\_program}
\end{itemize}
\bigskip
\listocaml{../src/default/default.ml}{default/default.ml}
}%
\only<2>{
\listasm[firstline=22,lastline=41,highlightlines={25-27}]{src/ocaml/caml\_start\_program.s}{caml\_start\_program}
}%
\only<4>{
\mintasm[firstline=22,lastline=41,highlightlines={31-38}]{src/ocaml/caml\_start\_program.s}
\listasm[firstline=66,lastline=72]{src/ocaml/caml\_start\_program.s}{caml\_start\_program}
}%
\only<6>{
\listasm[firstline=22,lastline=41,highlightlines={39-41}]{src/ocaml/caml\_start\_program.s}{caml\_start\_program}
}%
\end{column}
\begin{column}{0.4\textwidth}
\centering
\only<1-2>{\providecommand\step{2}\input{diagrams/default/default.tex}}%
\only<3-5>{\providecommand\step{3}\input{diagrams/default/default.tex}}%
\onslide*<6->{\providecommand\step{4}\input{diagrams/default/default.tex}}%
\end{column}
\end{columns}
\end{frame}

\begin{frame}{Executing the OCaml program}{And raising an exception without handler}
\begin{columns}
\begin{column}{0.6\textwidth}
\only<1-3>{
\begin{itemize}
\item<1-> Execution continues with OCaml code
\begin{itemize}
\item \funcname{camlDefault\_\_main\_269}
\item \funcname{camlDefault\_\_entry}
\item \funcname{caml\_program}
\end{itemize}
\item<1-> \funcname{camlDefault\_\_main\_269} raises a \typename{Failure} exception
\item<1-> \typename{Failure} is caught by \funcname{caml\_start\_program}
\begin{itemize}
\item<1-> The handler set the 2nd LSB of the exception value to \funcarg{1}
\item<3-> And then forces \funcname{caml\_start\_program} to terminate
\end{itemize}
\end{itemize}
}
\bigskip
\only<1,3>{\listocaml[highlightlines=3]{../src/default/default.ml}{default/default.ml}}%
\only<2>{\listasm[firstline=66,lastline=72,highlightlines=69]{src/ocaml/caml\_start\_program.s}{caml\_start\_program}}%
\end{column}
\begin{column}{0.4\textwidth}
\centering
\providecommand\step{4}\input{diagrams/default/default.tex}
\end{column}
\end{columns}
\end{frame}
%\only<>{\listasm[firstline=47,lastline=65]{src/ocaml/caml\_start\_program.s}{caml\_start\_program}}


\begin{frame}{Back to C}{Clean up, complain and terminate}
\begin{columns}
\begin{column}{0.45\textwidth}
\begin{itemize}
\item Backtrace:
\begin{itemize}
\item \funcname{caml\_start\_program} $\rightarrow$ OCaml code
\item \funcname{caml\_startup\_common}
\item \funcname{caml\_startup\_exn}
\item \funcname{caml\_startup}
\item \funcname{caml\_main}
\item \funcname{main}
\end{itemize}
\smallskip
\item \funcname{caml\_startup} checks the return value of \funcname{caml\_startup\_exn}
\begin{itemize}
\item The return value has been specially marked by \funcname{caml\_start\_program} handler
\item Calls \funcname{caml\_fatal\_uncaught\_exception}
\end{itemize}
\item<2-> \funcname{caml\_fatal\_uncaught\_exception} either
\begin{itemize}
\item<2-> Calls the user custom uncaught exception handler, if set with \mintinline{ocaml}{Printexc.set_uncaught_exception_handler fn}\footnotemark
\item<2-> Or falls back to the default one
\item<2-> Which notifies about the uncaught exception
\end{itemize}
\item<4-> Terminates the program either with \funcname{abort} or with a non-zero exit code
\end{itemize}
\end{column}
\begin{column}{0.55\textwidth}
\only<1>{
\listc[firstline=84,lastline=89,highlightlines=88]{../ocaml/runtime/caml/mlvalues.h}{runtime/caml/mlvalues.h:84}
\listc[firstline=138,lastline=143,highlightlines={141-142}]{../ocaml/runtime/startup\_nat.c}{runtime/startup\_nat.c:138}
}%
\only<2>{
\listc[firstline=138,highlightlines={147,149}]{../ocaml/runtime/printexc.c}{runtime/printexc.c:138}
}%
\only<3>{
\listc[firstline=110,lastline=134,highlightlines=129]{../ocaml/runtime/printexc.c}{runtime/printexc.c:138}
}%
\only<4>{
\listc[firstline=138,highlightlines={152,154}]{../ocaml/runtime/printexc.c}{runtime/printexc.c:138}
}%
\end{column}
\end{columns}
\footnotetext[1]{\url{https://v2.ocaml.org/api/Printexc.html\#1\_Uncaughtexceptions}}
\end{frame}
}%
\only<3-5>{\providecommand\step{3}%
% Runtime's default exception handler
%
\subsection*{Runtime's default exception handler}
\frameSubsection{pictures/The\_Architect.png}{}


\begin{frame}{Default exception handler}
\begin{columns}
\begin{column}{0.5\textwidth}
\begin{itemize}
\item \localname{domain\_state->exn\_handler} points to a trap located before \funcname{entry}!
\item What installed this very first trap there?
\end{itemize}
\bigskip
\listocaml{../src/default/default.ml}{default/default.ml}
\bigskip
\inputminted[fontsize=\footnotesize]{text}{../src/default/default.out}
\end{column}
\begin{column}{0.5\textwidth}
\centering
\only<1>{\providecommand\step{0}\input{diagrams/default/default.tex}}%
\only<2>{\providecommand\step{4}\input{diagrams/default/default.tex}}%
\end{column}
\end{columns}
\end{frame}


\begin{frame}{Startup}{How did we get there by the way?}
\begin{columns}
\begin{column}{0.45\textwidth}
\begin{itemize}
\item The entry point address of the binary is \funcname{main} (\funcname{\_start}) from the runtime
\item The program starts like a C program:
\begin{itemize}
\item \funcname{caml\_start\_program}
\item \funcname{caml\_startup\_common}
\item \funcname{caml\_startup\_exn}
\item \funcname{caml\_startup}
\item \funcname{caml\_main}
\item \funcname{main}
\end{itemize}
%\item \funcname{caml\_startup\_common}: initializes GC, domains \dots
% Also: codefrag, locale, custom opeartions, frame_descr, signals, stack
\item \funcname{caml\_start\_program} is the transition point between the C realm and the OCaml one
\end{itemize}
\bigskip
\listocaml{../src/default/default.ml}{default/default.ml}
\end{column}
\begin{column}{0.55\textwidth}
\listasm[lastline=24,highlightlines={22-24}]{src/ocaml/caml\_start\_program.s}{runtime/amd64.S:604}
\end{column}
\end{columns}
\end{frame}


\begin{frame}{Setup to jump into OCaml entry point}{\funcname{caml\_start\_program} initializes the main fiber}
\begin{columns}
\begin{column}{0.6\textwidth}
\only<1,3,5>{
\begin{itemize}
\item<1-> Stores information to allow switching from OCaml stack to the C stack
\item<3-> Constructs the default exception handler
\item<5-> Switches to the OCaml stack and calls \funcname{caml\_program}
\end{itemize}
\bigskip
\listocaml{../src/default/default.ml}{default/default.ml}
}%
\only<2>{
\listasm[firstline=22,lastline=41,highlightlines={25-27}]{src/ocaml/caml\_start\_program.s}{caml\_start\_program}
}%
\only<4>{
\mintasm[firstline=22,lastline=41,highlightlines={31-38}]{src/ocaml/caml\_start\_program.s}
\listasm[firstline=66,lastline=72]{src/ocaml/caml\_start\_program.s}{caml\_start\_program}
}%
\only<6>{
\listasm[firstline=22,lastline=41,highlightlines={39-41}]{src/ocaml/caml\_start\_program.s}{caml\_start\_program}
}%
\end{column}
\begin{column}{0.4\textwidth}
\centering
\only<1-2>{\providecommand\step{2}\input{diagrams/default/default.tex}}%
\only<3-5>{\providecommand\step{3}\input{diagrams/default/default.tex}}%
\onslide*<6->{\providecommand\step{4}\input{diagrams/default/default.tex}}%
\end{column}
\end{columns}
\end{frame}

\begin{frame}{Executing the OCaml program}{And raising an exception without handler}
\begin{columns}
\begin{column}{0.6\textwidth}
\only<1-3>{
\begin{itemize}
\item<1-> Execution continues with OCaml code
\begin{itemize}
\item \funcname{camlDefault\_\_main\_269}
\item \funcname{camlDefault\_\_entry}
\item \funcname{caml\_program}
\end{itemize}
\item<1-> \funcname{camlDefault\_\_main\_269} raises a \typename{Failure} exception
\item<1-> \typename{Failure} is caught by \funcname{caml\_start\_program}
\begin{itemize}
\item<1-> The handler set the 2nd LSB of the exception value to \funcarg{1}
\item<3-> And then forces \funcname{caml\_start\_program} to terminate
\end{itemize}
\end{itemize}
}
\bigskip
\only<1,3>{\listocaml[highlightlines=3]{../src/default/default.ml}{default/default.ml}}%
\only<2>{\listasm[firstline=66,lastline=72,highlightlines=69]{src/ocaml/caml\_start\_program.s}{caml\_start\_program}}%
\end{column}
\begin{column}{0.4\textwidth}
\centering
\providecommand\step{4}\input{diagrams/default/default.tex}
\end{column}
\end{columns}
\end{frame}
%\only<>{\listasm[firstline=47,lastline=65]{src/ocaml/caml\_start\_program.s}{caml\_start\_program}}


\begin{frame}{Back to C}{Clean up, complain and terminate}
\begin{columns}
\begin{column}{0.45\textwidth}
\begin{itemize}
\item Backtrace:
\begin{itemize}
\item \funcname{caml\_start\_program} $\rightarrow$ OCaml code
\item \funcname{caml\_startup\_common}
\item \funcname{caml\_startup\_exn}
\item \funcname{caml\_startup}
\item \funcname{caml\_main}
\item \funcname{main}
\end{itemize}
\smallskip
\item \funcname{caml\_startup} checks the return value of \funcname{caml\_startup\_exn}
\begin{itemize}
\item The return value has been specially marked by \funcname{caml\_start\_program} handler
\item Calls \funcname{caml\_fatal\_uncaught\_exception}
\end{itemize}
\item<2-> \funcname{caml\_fatal\_uncaught\_exception} either
\begin{itemize}
\item<2-> Calls the user custom uncaught exception handler, if set with \mintinline{ocaml}{Printexc.set_uncaught_exception_handler fn}\footnotemark
\item<2-> Or falls back to the default one
\item<2-> Which notifies about the uncaught exception
\end{itemize}
\item<4-> Terminates the program either with \funcname{abort} or with a non-zero exit code
\end{itemize}
\end{column}
\begin{column}{0.55\textwidth}
\only<1>{
\listc[firstline=84,lastline=89,highlightlines=88]{../ocaml/runtime/caml/mlvalues.h}{runtime/caml/mlvalues.h:84}
\listc[firstline=138,lastline=143,highlightlines={141-142}]{../ocaml/runtime/startup\_nat.c}{runtime/startup\_nat.c:138}
}%
\only<2>{
\listc[firstline=138,highlightlines={147,149}]{../ocaml/runtime/printexc.c}{runtime/printexc.c:138}
}%
\only<3>{
\listc[firstline=110,lastline=134,highlightlines=129]{../ocaml/runtime/printexc.c}{runtime/printexc.c:138}
}%
\only<4>{
\listc[firstline=138,highlightlines={152,154}]{../ocaml/runtime/printexc.c}{runtime/printexc.c:138}
}%
\end{column}
\end{columns}
\footnotetext[1]{\url{https://v2.ocaml.org/api/Printexc.html\#1\_Uncaughtexceptions}}
\end{frame}
}%
\onslide*<6->{\providecommand\step{4}%
% Runtime's default exception handler
%
\subsection*{Runtime's default exception handler}
\frameSubsection{pictures/The\_Architect.png}{}


\begin{frame}{Default exception handler}
\begin{columns}
\begin{column}{0.5\textwidth}
\begin{itemize}
\item \localname{domain\_state->exn\_handler} points to a trap located before \funcname{entry}!
\item What installed this very first trap there?
\end{itemize}
\bigskip
\listocaml{../src/default/default.ml}{default/default.ml}
\bigskip
\inputminted[fontsize=\footnotesize]{text}{../src/default/default.out}
\end{column}
\begin{column}{0.5\textwidth}
\centering
\only<1>{\providecommand\step{0}\input{diagrams/default/default.tex}}%
\only<2>{\providecommand\step{4}\input{diagrams/default/default.tex}}%
\end{column}
\end{columns}
\end{frame}


\begin{frame}{Startup}{How did we get there by the way?}
\begin{columns}
\begin{column}{0.45\textwidth}
\begin{itemize}
\item The entry point address of the binary is \funcname{main} (\funcname{\_start}) from the runtime
\item The program starts like a C program:
\begin{itemize}
\item \funcname{caml\_start\_program}
\item \funcname{caml\_startup\_common}
\item \funcname{caml\_startup\_exn}
\item \funcname{caml\_startup}
\item \funcname{caml\_main}
\item \funcname{main}
\end{itemize}
%\item \funcname{caml\_startup\_common}: initializes GC, domains \dots
% Also: codefrag, locale, custom opeartions, frame_descr, signals, stack
\item \funcname{caml\_start\_program} is the transition point between the C realm and the OCaml one
\end{itemize}
\bigskip
\listocaml{../src/default/default.ml}{default/default.ml}
\end{column}
\begin{column}{0.55\textwidth}
\listasm[lastline=24,highlightlines={22-24}]{src/ocaml/caml\_start\_program.s}{runtime/amd64.S:604}
\end{column}
\end{columns}
\end{frame}


\begin{frame}{Setup to jump into OCaml entry point}{\funcname{caml\_start\_program} initializes the main fiber}
\begin{columns}
\begin{column}{0.6\textwidth}
\only<1,3,5>{
\begin{itemize}
\item<1-> Stores information to allow switching from OCaml stack to the C stack
\item<3-> Constructs the default exception handler
\item<5-> Switches to the OCaml stack and calls \funcname{caml\_program}
\end{itemize}
\bigskip
\listocaml{../src/default/default.ml}{default/default.ml}
}%
\only<2>{
\listasm[firstline=22,lastline=41,highlightlines={25-27}]{src/ocaml/caml\_start\_program.s}{caml\_start\_program}
}%
\only<4>{
\mintasm[firstline=22,lastline=41,highlightlines={31-38}]{src/ocaml/caml\_start\_program.s}
\listasm[firstline=66,lastline=72]{src/ocaml/caml\_start\_program.s}{caml\_start\_program}
}%
\only<6>{
\listasm[firstline=22,lastline=41,highlightlines={39-41}]{src/ocaml/caml\_start\_program.s}{caml\_start\_program}
}%
\end{column}
\begin{column}{0.4\textwidth}
\centering
\only<1-2>{\providecommand\step{2}\input{diagrams/default/default.tex}}%
\only<3-5>{\providecommand\step{3}\input{diagrams/default/default.tex}}%
\onslide*<6->{\providecommand\step{4}\input{diagrams/default/default.tex}}%
\end{column}
\end{columns}
\end{frame}

\begin{frame}{Executing the OCaml program}{And raising an exception without handler}
\begin{columns}
\begin{column}{0.6\textwidth}
\only<1-3>{
\begin{itemize}
\item<1-> Execution continues with OCaml code
\begin{itemize}
\item \funcname{camlDefault\_\_main\_269}
\item \funcname{camlDefault\_\_entry}
\item \funcname{caml\_program}
\end{itemize}
\item<1-> \funcname{camlDefault\_\_main\_269} raises a \typename{Failure} exception
\item<1-> \typename{Failure} is caught by \funcname{caml\_start\_program}
\begin{itemize}
\item<1-> The handler set the 2nd LSB of the exception value to \funcarg{1}
\item<3-> And then forces \funcname{caml\_start\_program} to terminate
\end{itemize}
\end{itemize}
}
\bigskip
\only<1,3>{\listocaml[highlightlines=3]{../src/default/default.ml}{default/default.ml}}%
\only<2>{\listasm[firstline=66,lastline=72,highlightlines=69]{src/ocaml/caml\_start\_program.s}{caml\_start\_program}}%
\end{column}
\begin{column}{0.4\textwidth}
\centering
\providecommand\step{4}\input{diagrams/default/default.tex}
\end{column}
\end{columns}
\end{frame}
%\only<>{\listasm[firstline=47,lastline=65]{src/ocaml/caml\_start\_program.s}{caml\_start\_program}}


\begin{frame}{Back to C}{Clean up, complain and terminate}
\begin{columns}
\begin{column}{0.45\textwidth}
\begin{itemize}
\item Backtrace:
\begin{itemize}
\item \funcname{caml\_start\_program} $\rightarrow$ OCaml code
\item \funcname{caml\_startup\_common}
\item \funcname{caml\_startup\_exn}
\item \funcname{caml\_startup}
\item \funcname{caml\_main}
\item \funcname{main}
\end{itemize}
\smallskip
\item \funcname{caml\_startup} checks the return value of \funcname{caml\_startup\_exn}
\begin{itemize}
\item The return value has been specially marked by \funcname{caml\_start\_program} handler
\item Calls \funcname{caml\_fatal\_uncaught\_exception}
\end{itemize}
\item<2-> \funcname{caml\_fatal\_uncaught\_exception} either
\begin{itemize}
\item<2-> Calls the user custom uncaught exception handler, if set with \mintinline{ocaml}{Printexc.set_uncaught_exception_handler fn}\footnotemark
\item<2-> Or falls back to the default one
\item<2-> Which notifies about the uncaught exception
\end{itemize}
\item<4-> Terminates the program either with \funcname{abort} or with a non-zero exit code
\end{itemize}
\end{column}
\begin{column}{0.55\textwidth}
\only<1>{
\listc[firstline=84,lastline=89,highlightlines=88]{../ocaml/runtime/caml/mlvalues.h}{runtime/caml/mlvalues.h:84}
\listc[firstline=138,lastline=143,highlightlines={141-142}]{../ocaml/runtime/startup\_nat.c}{runtime/startup\_nat.c:138}
}%
\only<2>{
\listc[firstline=138,highlightlines={147,149}]{../ocaml/runtime/printexc.c}{runtime/printexc.c:138}
}%
\only<3>{
\listc[firstline=110,lastline=134,highlightlines=129]{../ocaml/runtime/printexc.c}{runtime/printexc.c:138}
}%
\only<4>{
\listc[firstline=138,highlightlines={152,154}]{../ocaml/runtime/printexc.c}{runtime/printexc.c:138}
}%
\end{column}
\end{columns}
\footnotetext[1]{\url{https://v2.ocaml.org/api/Printexc.html\#1\_Uncaughtexceptions}}
\end{frame}
}%
\end{column}
\end{columns}
\end{frame}

\begin{frame}{Executing the OCaml program}{And raising an exception without handler}
\begin{columns}
\begin{column}{0.6\textwidth}
\only<1-3>{
\begin{itemize}
\item<1-> Execution continues with OCaml code
\begin{itemize}
\item \funcname{camlDefault\_\_main\_269}
\item \funcname{camlDefault\_\_entry}
\item \funcname{caml\_program}
\end{itemize}
\item<1-> \funcname{camlDefault\_\_main\_269} raises a \typename{Failure} exception
\item<1-> \typename{Failure} is caught by \funcname{caml\_start\_program}
\begin{itemize}
\item<1-> The handler set the 2nd LSB of the exception value to \funcarg{1}
\item<3-> And then forces \funcname{caml\_start\_program} to terminate
\end{itemize}
\end{itemize}
}
\bigskip
\only<1,3>{\listocaml[highlightlines=3]{../src/default/default.ml}{default/default.ml}}%
\only<2>{\listasm[firstline=66,lastline=72,highlightlines=69]{src/ocaml/caml\_start\_program.s}{caml\_start\_program}}%
\end{column}
\begin{column}{0.4\textwidth}
\centering
\providecommand\step{4}%
% Runtime's default exception handler
%
\subsection*{Runtime's default exception handler}
\frameSubsection{pictures/The\_Architect.png}{}


\begin{frame}{Default exception handler}
\begin{columns}
\begin{column}{0.5\textwidth}
\begin{itemize}
\item \localname{domain\_state->exn\_handler} points to a trap located before \funcname{entry}!
\item What installed this very first trap there?
\end{itemize}
\bigskip
\listocaml{../src/default/default.ml}{default/default.ml}
\bigskip
\inputminted[fontsize=\footnotesize]{text}{../src/default/default.out}
\end{column}
\begin{column}{0.5\textwidth}
\centering
\only<1>{\providecommand\step{0}\input{diagrams/default/default.tex}}%
\only<2>{\providecommand\step{4}\input{diagrams/default/default.tex}}%
\end{column}
\end{columns}
\end{frame}


\begin{frame}{Startup}{How did we get there by the way?}
\begin{columns}
\begin{column}{0.45\textwidth}
\begin{itemize}
\item The entry point address of the binary is \funcname{main} (\funcname{\_start}) from the runtime
\item The program starts like a C program:
\begin{itemize}
\item \funcname{caml\_start\_program}
\item \funcname{caml\_startup\_common}
\item \funcname{caml\_startup\_exn}
\item \funcname{caml\_startup}
\item \funcname{caml\_main}
\item \funcname{main}
\end{itemize}
%\item \funcname{caml\_startup\_common}: initializes GC, domains \dots
% Also: codefrag, locale, custom opeartions, frame_descr, signals, stack
\item \funcname{caml\_start\_program} is the transition point between the C realm and the OCaml one
\end{itemize}
\bigskip
\listocaml{../src/default/default.ml}{default/default.ml}
\end{column}
\begin{column}{0.55\textwidth}
\listasm[lastline=24,highlightlines={22-24}]{src/ocaml/caml\_start\_program.s}{runtime/amd64.S:604}
\end{column}
\end{columns}
\end{frame}


\begin{frame}{Setup to jump into OCaml entry point}{\funcname{caml\_start\_program} initializes the main fiber}
\begin{columns}
\begin{column}{0.6\textwidth}
\only<1,3,5>{
\begin{itemize}
\item<1-> Stores information to allow switching from OCaml stack to the C stack
\item<3-> Constructs the default exception handler
\item<5-> Switches to the OCaml stack and calls \funcname{caml\_program}
\end{itemize}
\bigskip
\listocaml{../src/default/default.ml}{default/default.ml}
}%
\only<2>{
\listasm[firstline=22,lastline=41,highlightlines={25-27}]{src/ocaml/caml\_start\_program.s}{caml\_start\_program}
}%
\only<4>{
\mintasm[firstline=22,lastline=41,highlightlines={31-38}]{src/ocaml/caml\_start\_program.s}
\listasm[firstline=66,lastline=72]{src/ocaml/caml\_start\_program.s}{caml\_start\_program}
}%
\only<6>{
\listasm[firstline=22,lastline=41,highlightlines={39-41}]{src/ocaml/caml\_start\_program.s}{caml\_start\_program}
}%
\end{column}
\begin{column}{0.4\textwidth}
\centering
\only<1-2>{\providecommand\step{2}\input{diagrams/default/default.tex}}%
\only<3-5>{\providecommand\step{3}\input{diagrams/default/default.tex}}%
\onslide*<6->{\providecommand\step{4}\input{diagrams/default/default.tex}}%
\end{column}
\end{columns}
\end{frame}

\begin{frame}{Executing the OCaml program}{And raising an exception without handler}
\begin{columns}
\begin{column}{0.6\textwidth}
\only<1-3>{
\begin{itemize}
\item<1-> Execution continues with OCaml code
\begin{itemize}
\item \funcname{camlDefault\_\_main\_269}
\item \funcname{camlDefault\_\_entry}
\item \funcname{caml\_program}
\end{itemize}
\item<1-> \funcname{camlDefault\_\_main\_269} raises a \typename{Failure} exception
\item<1-> \typename{Failure} is caught by \funcname{caml\_start\_program}
\begin{itemize}
\item<1-> The handler set the 2nd LSB of the exception value to \funcarg{1}
\item<3-> And then forces \funcname{caml\_start\_program} to terminate
\end{itemize}
\end{itemize}
}
\bigskip
\only<1,3>{\listocaml[highlightlines=3]{../src/default/default.ml}{default/default.ml}}%
\only<2>{\listasm[firstline=66,lastline=72,highlightlines=69]{src/ocaml/caml\_start\_program.s}{caml\_start\_program}}%
\end{column}
\begin{column}{0.4\textwidth}
\centering
\providecommand\step{4}\input{diagrams/default/default.tex}
\end{column}
\end{columns}
\end{frame}
%\only<>{\listasm[firstline=47,lastline=65]{src/ocaml/caml\_start\_program.s}{caml\_start\_program}}


\begin{frame}{Back to C}{Clean up, complain and terminate}
\begin{columns}
\begin{column}{0.45\textwidth}
\begin{itemize}
\item Backtrace:
\begin{itemize}
\item \funcname{caml\_start\_program} $\rightarrow$ OCaml code
\item \funcname{caml\_startup\_common}
\item \funcname{caml\_startup\_exn}
\item \funcname{caml\_startup}
\item \funcname{caml\_main}
\item \funcname{main}
\end{itemize}
\smallskip
\item \funcname{caml\_startup} checks the return value of \funcname{caml\_startup\_exn}
\begin{itemize}
\item The return value has been specially marked by \funcname{caml\_start\_program} handler
\item Calls \funcname{caml\_fatal\_uncaught\_exception}
\end{itemize}
\item<2-> \funcname{caml\_fatal\_uncaught\_exception} either
\begin{itemize}
\item<2-> Calls the user custom uncaught exception handler, if set with \mintinline{ocaml}{Printexc.set_uncaught_exception_handler fn}\footnotemark
\item<2-> Or falls back to the default one
\item<2-> Which notifies about the uncaught exception
\end{itemize}
\item<4-> Terminates the program either with \funcname{abort} or with a non-zero exit code
\end{itemize}
\end{column}
\begin{column}{0.55\textwidth}
\only<1>{
\listc[firstline=84,lastline=89,highlightlines=88]{../ocaml/runtime/caml/mlvalues.h}{runtime/caml/mlvalues.h:84}
\listc[firstline=138,lastline=143,highlightlines={141-142}]{../ocaml/runtime/startup\_nat.c}{runtime/startup\_nat.c:138}
}%
\only<2>{
\listc[firstline=138,highlightlines={147,149}]{../ocaml/runtime/printexc.c}{runtime/printexc.c:138}
}%
\only<3>{
\listc[firstline=110,lastline=134,highlightlines=129]{../ocaml/runtime/printexc.c}{runtime/printexc.c:138}
}%
\only<4>{
\listc[firstline=138,highlightlines={152,154}]{../ocaml/runtime/printexc.c}{runtime/printexc.c:138}
}%
\end{column}
\end{columns}
\footnotetext[1]{\url{https://v2.ocaml.org/api/Printexc.html\#1\_Uncaughtexceptions}}
\end{frame}

\end{column}
\end{columns}
\end{frame}
%\only<>{\listasm[firstline=47,lastline=65]{src/ocaml/caml\_start\_program.s}{caml\_start\_program}}


\begin{frame}{Back to C}{Clean up, complain and terminate}
\begin{columns}
\begin{column}{0.45\textwidth}
\begin{itemize}
\item Backtrace:
\begin{itemize}
\item \funcname{caml\_start\_program} $\rightarrow$ OCaml code
\item \funcname{caml\_startup\_common}
\item \funcname{caml\_startup\_exn}
\item \funcname{caml\_startup}
\item \funcname{caml\_main}
\item \funcname{main}
\end{itemize}
\smallskip
\item \funcname{caml\_startup} checks the return value of \funcname{caml\_startup\_exn}
\begin{itemize}
\item The return value has been specially marked by \funcname{caml\_start\_program} handler
\item Calls \funcname{caml\_fatal\_uncaught\_exception}
\end{itemize}
\item<2-> \funcname{caml\_fatal\_uncaught\_exception} either
\begin{itemize}
\item<2-> Calls the user custom uncaught exception handler, if set with \mintinline{ocaml}{Printexc.set_uncaught_exception_handler fn}\footnotemark
\item<2-> Or falls back to the default one
\item<2-> Which notifies about the uncaught exception
\end{itemize}
\item<4-> Terminates the program either with \funcname{abort} or with a non-zero exit code
\end{itemize}
\end{column}
\begin{column}{0.55\textwidth}
\only<1>{
\listc[firstline=84,lastline=89,highlightlines=88]{../ocaml/runtime/caml/mlvalues.h}{runtime/caml/mlvalues.h:84}
\listc[firstline=138,lastline=143,highlightlines={141-142}]{../ocaml/runtime/startup\_nat.c}{runtime/startup\_nat.c:138}
}%
\only<2>{
\listc[firstline=138,highlightlines={147,149}]{../ocaml/runtime/printexc.c}{runtime/printexc.c:138}
}%
\only<3>{
\listc[firstline=110,lastline=134,highlightlines=129]{../ocaml/runtime/printexc.c}{runtime/printexc.c:138}
}%
\only<4>{
\listc[firstline=138,highlightlines={152,154}]{../ocaml/runtime/printexc.c}{runtime/printexc.c:138}
}%
\end{column}
\end{columns}
\footnotetext[1]{\url{https://v2.ocaml.org/api/Printexc.html\#1\_Uncaughtexceptions}}
\end{frame}
}%
\only<2>{\providecommand\step{4}%
% Runtime's default exception handler
%
\subsection*{Runtime's default exception handler}
\frameSubsection{pictures/The\_Architect.png}{}


\begin{frame}{Default exception handler}
\begin{columns}
\begin{column}{0.5\textwidth}
\begin{itemize}
\item \localname{domain\_state->exn\_handler} points to a trap located before \funcname{entry}!
\item What installed this very first trap there?
\end{itemize}
\bigskip
\listocaml{../src/default/default.ml}{default/default.ml}
\bigskip
\inputminted[fontsize=\footnotesize]{text}{../src/default/default.out}
\end{column}
\begin{column}{0.5\textwidth}
\centering
\only<1>{\providecommand\step{0}%
% Runtime's default exception handler
%
\subsection*{Runtime's default exception handler}
\frameSubsection{pictures/The\_Architect.png}{}


\begin{frame}{Default exception handler}
\begin{columns}
\begin{column}{0.5\textwidth}
\begin{itemize}
\item \localname{domain\_state->exn\_handler} points to a trap located before \funcname{entry}!
\item What installed this very first trap there?
\end{itemize}
\bigskip
\listocaml{../src/default/default.ml}{default/default.ml}
\bigskip
\inputminted[fontsize=\footnotesize]{text}{../src/default/default.out}
\end{column}
\begin{column}{0.5\textwidth}
\centering
\only<1>{\providecommand\step{0}\input{diagrams/default/default.tex}}%
\only<2>{\providecommand\step{4}\input{diagrams/default/default.tex}}%
\end{column}
\end{columns}
\end{frame}


\begin{frame}{Startup}{How did we get there by the way?}
\begin{columns}
\begin{column}{0.45\textwidth}
\begin{itemize}
\item The entry point address of the binary is \funcname{main} (\funcname{\_start}) from the runtime
\item The program starts like a C program:
\begin{itemize}
\item \funcname{caml\_start\_program}
\item \funcname{caml\_startup\_common}
\item \funcname{caml\_startup\_exn}
\item \funcname{caml\_startup}
\item \funcname{caml\_main}
\item \funcname{main}
\end{itemize}
%\item \funcname{caml\_startup\_common}: initializes GC, domains \dots
% Also: codefrag, locale, custom opeartions, frame_descr, signals, stack
\item \funcname{caml\_start\_program} is the transition point between the C realm and the OCaml one
\end{itemize}
\bigskip
\listocaml{../src/default/default.ml}{default/default.ml}
\end{column}
\begin{column}{0.55\textwidth}
\listasm[lastline=24,highlightlines={22-24}]{src/ocaml/caml\_start\_program.s}{runtime/amd64.S:604}
\end{column}
\end{columns}
\end{frame}


\begin{frame}{Setup to jump into OCaml entry point}{\funcname{caml\_start\_program} initializes the main fiber}
\begin{columns}
\begin{column}{0.6\textwidth}
\only<1,3,5>{
\begin{itemize}
\item<1-> Stores information to allow switching from OCaml stack to the C stack
\item<3-> Constructs the default exception handler
\item<5-> Switches to the OCaml stack and calls \funcname{caml\_program}
\end{itemize}
\bigskip
\listocaml{../src/default/default.ml}{default/default.ml}
}%
\only<2>{
\listasm[firstline=22,lastline=41,highlightlines={25-27}]{src/ocaml/caml\_start\_program.s}{caml\_start\_program}
}%
\only<4>{
\mintasm[firstline=22,lastline=41,highlightlines={31-38}]{src/ocaml/caml\_start\_program.s}
\listasm[firstline=66,lastline=72]{src/ocaml/caml\_start\_program.s}{caml\_start\_program}
}%
\only<6>{
\listasm[firstline=22,lastline=41,highlightlines={39-41}]{src/ocaml/caml\_start\_program.s}{caml\_start\_program}
}%
\end{column}
\begin{column}{0.4\textwidth}
\centering
\only<1-2>{\providecommand\step{2}\input{diagrams/default/default.tex}}%
\only<3-5>{\providecommand\step{3}\input{diagrams/default/default.tex}}%
\onslide*<6->{\providecommand\step{4}\input{diagrams/default/default.tex}}%
\end{column}
\end{columns}
\end{frame}

\begin{frame}{Executing the OCaml program}{And raising an exception without handler}
\begin{columns}
\begin{column}{0.6\textwidth}
\only<1-3>{
\begin{itemize}
\item<1-> Execution continues with OCaml code
\begin{itemize}
\item \funcname{camlDefault\_\_main\_269}
\item \funcname{camlDefault\_\_entry}
\item \funcname{caml\_program}
\end{itemize}
\item<1-> \funcname{camlDefault\_\_main\_269} raises a \typename{Failure} exception
\item<1-> \typename{Failure} is caught by \funcname{caml\_start\_program}
\begin{itemize}
\item<1-> The handler set the 2nd LSB of the exception value to \funcarg{1}
\item<3-> And then forces \funcname{caml\_start\_program} to terminate
\end{itemize}
\end{itemize}
}
\bigskip
\only<1,3>{\listocaml[highlightlines=3]{../src/default/default.ml}{default/default.ml}}%
\only<2>{\listasm[firstline=66,lastline=72,highlightlines=69]{src/ocaml/caml\_start\_program.s}{caml\_start\_program}}%
\end{column}
\begin{column}{0.4\textwidth}
\centering
\providecommand\step{4}\input{diagrams/default/default.tex}
\end{column}
\end{columns}
\end{frame}
%\only<>{\listasm[firstline=47,lastline=65]{src/ocaml/caml\_start\_program.s}{caml\_start\_program}}


\begin{frame}{Back to C}{Clean up, complain and terminate}
\begin{columns}
\begin{column}{0.45\textwidth}
\begin{itemize}
\item Backtrace:
\begin{itemize}
\item \funcname{caml\_start\_program} $\rightarrow$ OCaml code
\item \funcname{caml\_startup\_common}
\item \funcname{caml\_startup\_exn}
\item \funcname{caml\_startup}
\item \funcname{caml\_main}
\item \funcname{main}
\end{itemize}
\smallskip
\item \funcname{caml\_startup} checks the return value of \funcname{caml\_startup\_exn}
\begin{itemize}
\item The return value has been specially marked by \funcname{caml\_start\_program} handler
\item Calls \funcname{caml\_fatal\_uncaught\_exception}
\end{itemize}
\item<2-> \funcname{caml\_fatal\_uncaught\_exception} either
\begin{itemize}
\item<2-> Calls the user custom uncaught exception handler, if set with \mintinline{ocaml}{Printexc.set_uncaught_exception_handler fn}\footnotemark
\item<2-> Or falls back to the default one
\item<2-> Which notifies about the uncaught exception
\end{itemize}
\item<4-> Terminates the program either with \funcname{abort} or with a non-zero exit code
\end{itemize}
\end{column}
\begin{column}{0.55\textwidth}
\only<1>{
\listc[firstline=84,lastline=89,highlightlines=88]{../ocaml/runtime/caml/mlvalues.h}{runtime/caml/mlvalues.h:84}
\listc[firstline=138,lastline=143,highlightlines={141-142}]{../ocaml/runtime/startup\_nat.c}{runtime/startup\_nat.c:138}
}%
\only<2>{
\listc[firstline=138,highlightlines={147,149}]{../ocaml/runtime/printexc.c}{runtime/printexc.c:138}
}%
\only<3>{
\listc[firstline=110,lastline=134,highlightlines=129]{../ocaml/runtime/printexc.c}{runtime/printexc.c:138}
}%
\only<4>{
\listc[firstline=138,highlightlines={152,154}]{../ocaml/runtime/printexc.c}{runtime/printexc.c:138}
}%
\end{column}
\end{columns}
\footnotetext[1]{\url{https://v2.ocaml.org/api/Printexc.html\#1\_Uncaughtexceptions}}
\end{frame}
}%
\only<2>{\providecommand\step{4}%
% Runtime's default exception handler
%
\subsection*{Runtime's default exception handler}
\frameSubsection{pictures/The\_Architect.png}{}


\begin{frame}{Default exception handler}
\begin{columns}
\begin{column}{0.5\textwidth}
\begin{itemize}
\item \localname{domain\_state->exn\_handler} points to a trap located before \funcname{entry}!
\item What installed this very first trap there?
\end{itemize}
\bigskip
\listocaml{../src/default/default.ml}{default/default.ml}
\bigskip
\inputminted[fontsize=\footnotesize]{text}{../src/default/default.out}
\end{column}
\begin{column}{0.5\textwidth}
\centering
\only<1>{\providecommand\step{0}\input{diagrams/default/default.tex}}%
\only<2>{\providecommand\step{4}\input{diagrams/default/default.tex}}%
\end{column}
\end{columns}
\end{frame}


\begin{frame}{Startup}{How did we get there by the way?}
\begin{columns}
\begin{column}{0.45\textwidth}
\begin{itemize}
\item The entry point address of the binary is \funcname{main} (\funcname{\_start}) from the runtime
\item The program starts like a C program:
\begin{itemize}
\item \funcname{caml\_start\_program}
\item \funcname{caml\_startup\_common}
\item \funcname{caml\_startup\_exn}
\item \funcname{caml\_startup}
\item \funcname{caml\_main}
\item \funcname{main}
\end{itemize}
%\item \funcname{caml\_startup\_common}: initializes GC, domains \dots
% Also: codefrag, locale, custom opeartions, frame_descr, signals, stack
\item \funcname{caml\_start\_program} is the transition point between the C realm and the OCaml one
\end{itemize}
\bigskip
\listocaml{../src/default/default.ml}{default/default.ml}
\end{column}
\begin{column}{0.55\textwidth}
\listasm[lastline=24,highlightlines={22-24}]{src/ocaml/caml\_start\_program.s}{runtime/amd64.S:604}
\end{column}
\end{columns}
\end{frame}


\begin{frame}{Setup to jump into OCaml entry point}{\funcname{caml\_start\_program} initializes the main fiber}
\begin{columns}
\begin{column}{0.6\textwidth}
\only<1,3,5>{
\begin{itemize}
\item<1-> Stores information to allow switching from OCaml stack to the C stack
\item<3-> Constructs the default exception handler
\item<5-> Switches to the OCaml stack and calls \funcname{caml\_program}
\end{itemize}
\bigskip
\listocaml{../src/default/default.ml}{default/default.ml}
}%
\only<2>{
\listasm[firstline=22,lastline=41,highlightlines={25-27}]{src/ocaml/caml\_start\_program.s}{caml\_start\_program}
}%
\only<4>{
\mintasm[firstline=22,lastline=41,highlightlines={31-38}]{src/ocaml/caml\_start\_program.s}
\listasm[firstline=66,lastline=72]{src/ocaml/caml\_start\_program.s}{caml\_start\_program}
}%
\only<6>{
\listasm[firstline=22,lastline=41,highlightlines={39-41}]{src/ocaml/caml\_start\_program.s}{caml\_start\_program}
}%
\end{column}
\begin{column}{0.4\textwidth}
\centering
\only<1-2>{\providecommand\step{2}\input{diagrams/default/default.tex}}%
\only<3-5>{\providecommand\step{3}\input{diagrams/default/default.tex}}%
\onslide*<6->{\providecommand\step{4}\input{diagrams/default/default.tex}}%
\end{column}
\end{columns}
\end{frame}

\begin{frame}{Executing the OCaml program}{And raising an exception without handler}
\begin{columns}
\begin{column}{0.6\textwidth}
\only<1-3>{
\begin{itemize}
\item<1-> Execution continues with OCaml code
\begin{itemize}
\item \funcname{camlDefault\_\_main\_269}
\item \funcname{camlDefault\_\_entry}
\item \funcname{caml\_program}
\end{itemize}
\item<1-> \funcname{camlDefault\_\_main\_269} raises a \typename{Failure} exception
\item<1-> \typename{Failure} is caught by \funcname{caml\_start\_program}
\begin{itemize}
\item<1-> The handler set the 2nd LSB of the exception value to \funcarg{1}
\item<3-> And then forces \funcname{caml\_start\_program} to terminate
\end{itemize}
\end{itemize}
}
\bigskip
\only<1,3>{\listocaml[highlightlines=3]{../src/default/default.ml}{default/default.ml}}%
\only<2>{\listasm[firstline=66,lastline=72,highlightlines=69]{src/ocaml/caml\_start\_program.s}{caml\_start\_program}}%
\end{column}
\begin{column}{0.4\textwidth}
\centering
\providecommand\step{4}\input{diagrams/default/default.tex}
\end{column}
\end{columns}
\end{frame}
%\only<>{\listasm[firstline=47,lastline=65]{src/ocaml/caml\_start\_program.s}{caml\_start\_program}}


\begin{frame}{Back to C}{Clean up, complain and terminate}
\begin{columns}
\begin{column}{0.45\textwidth}
\begin{itemize}
\item Backtrace:
\begin{itemize}
\item \funcname{caml\_start\_program} $\rightarrow$ OCaml code
\item \funcname{caml\_startup\_common}
\item \funcname{caml\_startup\_exn}
\item \funcname{caml\_startup}
\item \funcname{caml\_main}
\item \funcname{main}
\end{itemize}
\smallskip
\item \funcname{caml\_startup} checks the return value of \funcname{caml\_startup\_exn}
\begin{itemize}
\item The return value has been specially marked by \funcname{caml\_start\_program} handler
\item Calls \funcname{caml\_fatal\_uncaught\_exception}
\end{itemize}
\item<2-> \funcname{caml\_fatal\_uncaught\_exception} either
\begin{itemize}
\item<2-> Calls the user custom uncaught exception handler, if set with \mintinline{ocaml}{Printexc.set_uncaught_exception_handler fn}\footnotemark
\item<2-> Or falls back to the default one
\item<2-> Which notifies about the uncaught exception
\end{itemize}
\item<4-> Terminates the program either with \funcname{abort} or with a non-zero exit code
\end{itemize}
\end{column}
\begin{column}{0.55\textwidth}
\only<1>{
\listc[firstline=84,lastline=89,highlightlines=88]{../ocaml/runtime/caml/mlvalues.h}{runtime/caml/mlvalues.h:84}
\listc[firstline=138,lastline=143,highlightlines={141-142}]{../ocaml/runtime/startup\_nat.c}{runtime/startup\_nat.c:138}
}%
\only<2>{
\listc[firstline=138,highlightlines={147,149}]{../ocaml/runtime/printexc.c}{runtime/printexc.c:138}
}%
\only<3>{
\listc[firstline=110,lastline=134,highlightlines=129]{../ocaml/runtime/printexc.c}{runtime/printexc.c:138}
}%
\only<4>{
\listc[firstline=138,highlightlines={152,154}]{../ocaml/runtime/printexc.c}{runtime/printexc.c:138}
}%
\end{column}
\end{columns}
\footnotetext[1]{\url{https://v2.ocaml.org/api/Printexc.html\#1\_Uncaughtexceptions}}
\end{frame}
}%
\end{column}
\end{columns}
\end{frame}


\begin{frame}{Startup}{How did we get there by the way?}
\begin{columns}
\begin{column}{0.45\textwidth}
\begin{itemize}
\item The entry point address of the binary is \funcname{main} (\funcname{\_start}) from the runtime
\item The program starts like a C program:
\begin{itemize}
\item \funcname{caml\_start\_program}
\item \funcname{caml\_startup\_common}
\item \funcname{caml\_startup\_exn}
\item \funcname{caml\_startup}
\item \funcname{caml\_main}
\item \funcname{main}
\end{itemize}
%\item \funcname{caml\_startup\_common}: initializes GC, domains \dots
% Also: codefrag, locale, custom opeartions, frame_descr, signals, stack
\item \funcname{caml\_start\_program} is the transition point between the C realm and the OCaml one
\end{itemize}
\bigskip
\listocaml{../src/default/default.ml}{default/default.ml}
\end{column}
\begin{column}{0.55\textwidth}
\listasm[lastline=24,highlightlines={22-24}]{src/ocaml/caml\_start\_program.s}{runtime/amd64.S:604}
\end{column}
\end{columns}
\end{frame}


\begin{frame}{Setup to jump into OCaml entry point}{\funcname{caml\_start\_program} initializes the main fiber}
\begin{columns}
\begin{column}{0.6\textwidth}
\only<1,3,5>{
\begin{itemize}
\item<1-> Stores information to allow switching from OCaml stack to the C stack
\item<3-> Constructs the default exception handler
\item<5-> Switches to the OCaml stack and calls \funcname{caml\_program}
\end{itemize}
\bigskip
\listocaml{../src/default/default.ml}{default/default.ml}
}%
\only<2>{
\listasm[firstline=22,lastline=41,highlightlines={25-27}]{src/ocaml/caml\_start\_program.s}{caml\_start\_program}
}%
\only<4>{
\mintasm[firstline=22,lastline=41,highlightlines={31-38}]{src/ocaml/caml\_start\_program.s}
\listasm[firstline=66,lastline=72]{src/ocaml/caml\_start\_program.s}{caml\_start\_program}
}%
\only<6>{
\listasm[firstline=22,lastline=41,highlightlines={39-41}]{src/ocaml/caml\_start\_program.s}{caml\_start\_program}
}%
\end{column}
\begin{column}{0.4\textwidth}
\centering
\only<1-2>{\providecommand\step{2}%
% Runtime's default exception handler
%
\subsection*{Runtime's default exception handler}
\frameSubsection{pictures/The\_Architect.png}{}


\begin{frame}{Default exception handler}
\begin{columns}
\begin{column}{0.5\textwidth}
\begin{itemize}
\item \localname{domain\_state->exn\_handler} points to a trap located before \funcname{entry}!
\item What installed this very first trap there?
\end{itemize}
\bigskip
\listocaml{../src/default/default.ml}{default/default.ml}
\bigskip
\inputminted[fontsize=\footnotesize]{text}{../src/default/default.out}
\end{column}
\begin{column}{0.5\textwidth}
\centering
\only<1>{\providecommand\step{0}\input{diagrams/default/default.tex}}%
\only<2>{\providecommand\step{4}\input{diagrams/default/default.tex}}%
\end{column}
\end{columns}
\end{frame}


\begin{frame}{Startup}{How did we get there by the way?}
\begin{columns}
\begin{column}{0.45\textwidth}
\begin{itemize}
\item The entry point address of the binary is \funcname{main} (\funcname{\_start}) from the runtime
\item The program starts like a C program:
\begin{itemize}
\item \funcname{caml\_start\_program}
\item \funcname{caml\_startup\_common}
\item \funcname{caml\_startup\_exn}
\item \funcname{caml\_startup}
\item \funcname{caml\_main}
\item \funcname{main}
\end{itemize}
%\item \funcname{caml\_startup\_common}: initializes GC, domains \dots
% Also: codefrag, locale, custom opeartions, frame_descr, signals, stack
\item \funcname{caml\_start\_program} is the transition point between the C realm and the OCaml one
\end{itemize}
\bigskip
\listocaml{../src/default/default.ml}{default/default.ml}
\end{column}
\begin{column}{0.55\textwidth}
\listasm[lastline=24,highlightlines={22-24}]{src/ocaml/caml\_start\_program.s}{runtime/amd64.S:604}
\end{column}
\end{columns}
\end{frame}


\begin{frame}{Setup to jump into OCaml entry point}{\funcname{caml\_start\_program} initializes the main fiber}
\begin{columns}
\begin{column}{0.6\textwidth}
\only<1,3,5>{
\begin{itemize}
\item<1-> Stores information to allow switching from OCaml stack to the C stack
\item<3-> Constructs the default exception handler
\item<5-> Switches to the OCaml stack and calls \funcname{caml\_program}
\end{itemize}
\bigskip
\listocaml{../src/default/default.ml}{default/default.ml}
}%
\only<2>{
\listasm[firstline=22,lastline=41,highlightlines={25-27}]{src/ocaml/caml\_start\_program.s}{caml\_start\_program}
}%
\only<4>{
\mintasm[firstline=22,lastline=41,highlightlines={31-38}]{src/ocaml/caml\_start\_program.s}
\listasm[firstline=66,lastline=72]{src/ocaml/caml\_start\_program.s}{caml\_start\_program}
}%
\only<6>{
\listasm[firstline=22,lastline=41,highlightlines={39-41}]{src/ocaml/caml\_start\_program.s}{caml\_start\_program}
}%
\end{column}
\begin{column}{0.4\textwidth}
\centering
\only<1-2>{\providecommand\step{2}\input{diagrams/default/default.tex}}%
\only<3-5>{\providecommand\step{3}\input{diagrams/default/default.tex}}%
\onslide*<6->{\providecommand\step{4}\input{diagrams/default/default.tex}}%
\end{column}
\end{columns}
\end{frame}

\begin{frame}{Executing the OCaml program}{And raising an exception without handler}
\begin{columns}
\begin{column}{0.6\textwidth}
\only<1-3>{
\begin{itemize}
\item<1-> Execution continues with OCaml code
\begin{itemize}
\item \funcname{camlDefault\_\_main\_269}
\item \funcname{camlDefault\_\_entry}
\item \funcname{caml\_program}
\end{itemize}
\item<1-> \funcname{camlDefault\_\_main\_269} raises a \typename{Failure} exception
\item<1-> \typename{Failure} is caught by \funcname{caml\_start\_program}
\begin{itemize}
\item<1-> The handler set the 2nd LSB of the exception value to \funcarg{1}
\item<3-> And then forces \funcname{caml\_start\_program} to terminate
\end{itemize}
\end{itemize}
}
\bigskip
\only<1,3>{\listocaml[highlightlines=3]{../src/default/default.ml}{default/default.ml}}%
\only<2>{\listasm[firstline=66,lastline=72,highlightlines=69]{src/ocaml/caml\_start\_program.s}{caml\_start\_program}}%
\end{column}
\begin{column}{0.4\textwidth}
\centering
\providecommand\step{4}\input{diagrams/default/default.tex}
\end{column}
\end{columns}
\end{frame}
%\only<>{\listasm[firstline=47,lastline=65]{src/ocaml/caml\_start\_program.s}{caml\_start\_program}}


\begin{frame}{Back to C}{Clean up, complain and terminate}
\begin{columns}
\begin{column}{0.45\textwidth}
\begin{itemize}
\item Backtrace:
\begin{itemize}
\item \funcname{caml\_start\_program} $\rightarrow$ OCaml code
\item \funcname{caml\_startup\_common}
\item \funcname{caml\_startup\_exn}
\item \funcname{caml\_startup}
\item \funcname{caml\_main}
\item \funcname{main}
\end{itemize}
\smallskip
\item \funcname{caml\_startup} checks the return value of \funcname{caml\_startup\_exn}
\begin{itemize}
\item The return value has been specially marked by \funcname{caml\_start\_program} handler
\item Calls \funcname{caml\_fatal\_uncaught\_exception}
\end{itemize}
\item<2-> \funcname{caml\_fatal\_uncaught\_exception} either
\begin{itemize}
\item<2-> Calls the user custom uncaught exception handler, if set with \mintinline{ocaml}{Printexc.set_uncaught_exception_handler fn}\footnotemark
\item<2-> Or falls back to the default one
\item<2-> Which notifies about the uncaught exception
\end{itemize}
\item<4-> Terminates the program either with \funcname{abort} or with a non-zero exit code
\end{itemize}
\end{column}
\begin{column}{0.55\textwidth}
\only<1>{
\listc[firstline=84,lastline=89,highlightlines=88]{../ocaml/runtime/caml/mlvalues.h}{runtime/caml/mlvalues.h:84}
\listc[firstline=138,lastline=143,highlightlines={141-142}]{../ocaml/runtime/startup\_nat.c}{runtime/startup\_nat.c:138}
}%
\only<2>{
\listc[firstline=138,highlightlines={147,149}]{../ocaml/runtime/printexc.c}{runtime/printexc.c:138}
}%
\only<3>{
\listc[firstline=110,lastline=134,highlightlines=129]{../ocaml/runtime/printexc.c}{runtime/printexc.c:138}
}%
\only<4>{
\listc[firstline=138,highlightlines={152,154}]{../ocaml/runtime/printexc.c}{runtime/printexc.c:138}
}%
\end{column}
\end{columns}
\footnotetext[1]{\url{https://v2.ocaml.org/api/Printexc.html\#1\_Uncaughtexceptions}}
\end{frame}
}%
\only<3-5>{\providecommand\step{3}%
% Runtime's default exception handler
%
\subsection*{Runtime's default exception handler}
\frameSubsection{pictures/The\_Architect.png}{}


\begin{frame}{Default exception handler}
\begin{columns}
\begin{column}{0.5\textwidth}
\begin{itemize}
\item \localname{domain\_state->exn\_handler} points to a trap located before \funcname{entry}!
\item What installed this very first trap there?
\end{itemize}
\bigskip
\listocaml{../src/default/default.ml}{default/default.ml}
\bigskip
\inputminted[fontsize=\footnotesize]{text}{../src/default/default.out}
\end{column}
\begin{column}{0.5\textwidth}
\centering
\only<1>{\providecommand\step{0}\input{diagrams/default/default.tex}}%
\only<2>{\providecommand\step{4}\input{diagrams/default/default.tex}}%
\end{column}
\end{columns}
\end{frame}


\begin{frame}{Startup}{How did we get there by the way?}
\begin{columns}
\begin{column}{0.45\textwidth}
\begin{itemize}
\item The entry point address of the binary is \funcname{main} (\funcname{\_start}) from the runtime
\item The program starts like a C program:
\begin{itemize}
\item \funcname{caml\_start\_program}
\item \funcname{caml\_startup\_common}
\item \funcname{caml\_startup\_exn}
\item \funcname{caml\_startup}
\item \funcname{caml\_main}
\item \funcname{main}
\end{itemize}
%\item \funcname{caml\_startup\_common}: initializes GC, domains \dots
% Also: codefrag, locale, custom opeartions, frame_descr, signals, stack
\item \funcname{caml\_start\_program} is the transition point between the C realm and the OCaml one
\end{itemize}
\bigskip
\listocaml{../src/default/default.ml}{default/default.ml}
\end{column}
\begin{column}{0.55\textwidth}
\listasm[lastline=24,highlightlines={22-24}]{src/ocaml/caml\_start\_program.s}{runtime/amd64.S:604}
\end{column}
\end{columns}
\end{frame}


\begin{frame}{Setup to jump into OCaml entry point}{\funcname{caml\_start\_program} initializes the main fiber}
\begin{columns}
\begin{column}{0.6\textwidth}
\only<1,3,5>{
\begin{itemize}
\item<1-> Stores information to allow switching from OCaml stack to the C stack
\item<3-> Constructs the default exception handler
\item<5-> Switches to the OCaml stack and calls \funcname{caml\_program}
\end{itemize}
\bigskip
\listocaml{../src/default/default.ml}{default/default.ml}
}%
\only<2>{
\listasm[firstline=22,lastline=41,highlightlines={25-27}]{src/ocaml/caml\_start\_program.s}{caml\_start\_program}
}%
\only<4>{
\mintasm[firstline=22,lastline=41,highlightlines={31-38}]{src/ocaml/caml\_start\_program.s}
\listasm[firstline=66,lastline=72]{src/ocaml/caml\_start\_program.s}{caml\_start\_program}
}%
\only<6>{
\listasm[firstline=22,lastline=41,highlightlines={39-41}]{src/ocaml/caml\_start\_program.s}{caml\_start\_program}
}%
\end{column}
\begin{column}{0.4\textwidth}
\centering
\only<1-2>{\providecommand\step{2}\input{diagrams/default/default.tex}}%
\only<3-5>{\providecommand\step{3}\input{diagrams/default/default.tex}}%
\onslide*<6->{\providecommand\step{4}\input{diagrams/default/default.tex}}%
\end{column}
\end{columns}
\end{frame}

\begin{frame}{Executing the OCaml program}{And raising an exception without handler}
\begin{columns}
\begin{column}{0.6\textwidth}
\only<1-3>{
\begin{itemize}
\item<1-> Execution continues with OCaml code
\begin{itemize}
\item \funcname{camlDefault\_\_main\_269}
\item \funcname{camlDefault\_\_entry}
\item \funcname{caml\_program}
\end{itemize}
\item<1-> \funcname{camlDefault\_\_main\_269} raises a \typename{Failure} exception
\item<1-> \typename{Failure} is caught by \funcname{caml\_start\_program}
\begin{itemize}
\item<1-> The handler set the 2nd LSB of the exception value to \funcarg{1}
\item<3-> And then forces \funcname{caml\_start\_program} to terminate
\end{itemize}
\end{itemize}
}
\bigskip
\only<1,3>{\listocaml[highlightlines=3]{../src/default/default.ml}{default/default.ml}}%
\only<2>{\listasm[firstline=66,lastline=72,highlightlines=69]{src/ocaml/caml\_start\_program.s}{caml\_start\_program}}%
\end{column}
\begin{column}{0.4\textwidth}
\centering
\providecommand\step{4}\input{diagrams/default/default.tex}
\end{column}
\end{columns}
\end{frame}
%\only<>{\listasm[firstline=47,lastline=65]{src/ocaml/caml\_start\_program.s}{caml\_start\_program}}


\begin{frame}{Back to C}{Clean up, complain and terminate}
\begin{columns}
\begin{column}{0.45\textwidth}
\begin{itemize}
\item Backtrace:
\begin{itemize}
\item \funcname{caml\_start\_program} $\rightarrow$ OCaml code
\item \funcname{caml\_startup\_common}
\item \funcname{caml\_startup\_exn}
\item \funcname{caml\_startup}
\item \funcname{caml\_main}
\item \funcname{main}
\end{itemize}
\smallskip
\item \funcname{caml\_startup} checks the return value of \funcname{caml\_startup\_exn}
\begin{itemize}
\item The return value has been specially marked by \funcname{caml\_start\_program} handler
\item Calls \funcname{caml\_fatal\_uncaught\_exception}
\end{itemize}
\item<2-> \funcname{caml\_fatal\_uncaught\_exception} either
\begin{itemize}
\item<2-> Calls the user custom uncaught exception handler, if set with \mintinline{ocaml}{Printexc.set_uncaught_exception_handler fn}\footnotemark
\item<2-> Or falls back to the default one
\item<2-> Which notifies about the uncaught exception
\end{itemize}
\item<4-> Terminates the program either with \funcname{abort} or with a non-zero exit code
\end{itemize}
\end{column}
\begin{column}{0.55\textwidth}
\only<1>{
\listc[firstline=84,lastline=89,highlightlines=88]{../ocaml/runtime/caml/mlvalues.h}{runtime/caml/mlvalues.h:84}
\listc[firstline=138,lastline=143,highlightlines={141-142}]{../ocaml/runtime/startup\_nat.c}{runtime/startup\_nat.c:138}
}%
\only<2>{
\listc[firstline=138,highlightlines={147,149}]{../ocaml/runtime/printexc.c}{runtime/printexc.c:138}
}%
\only<3>{
\listc[firstline=110,lastline=134,highlightlines=129]{../ocaml/runtime/printexc.c}{runtime/printexc.c:138}
}%
\only<4>{
\listc[firstline=138,highlightlines={152,154}]{../ocaml/runtime/printexc.c}{runtime/printexc.c:138}
}%
\end{column}
\end{columns}
\footnotetext[1]{\url{https://v2.ocaml.org/api/Printexc.html\#1\_Uncaughtexceptions}}
\end{frame}
}%
\onslide*<6->{\providecommand\step{4}%
% Runtime's default exception handler
%
\subsection*{Runtime's default exception handler}
\frameSubsection{pictures/The\_Architect.png}{}


\begin{frame}{Default exception handler}
\begin{columns}
\begin{column}{0.5\textwidth}
\begin{itemize}
\item \localname{domain\_state->exn\_handler} points to a trap located before \funcname{entry}!
\item What installed this very first trap there?
\end{itemize}
\bigskip
\listocaml{../src/default/default.ml}{default/default.ml}
\bigskip
\inputminted[fontsize=\footnotesize]{text}{../src/default/default.out}
\end{column}
\begin{column}{0.5\textwidth}
\centering
\only<1>{\providecommand\step{0}\input{diagrams/default/default.tex}}%
\only<2>{\providecommand\step{4}\input{diagrams/default/default.tex}}%
\end{column}
\end{columns}
\end{frame}


\begin{frame}{Startup}{How did we get there by the way?}
\begin{columns}
\begin{column}{0.45\textwidth}
\begin{itemize}
\item The entry point address of the binary is \funcname{main} (\funcname{\_start}) from the runtime
\item The program starts like a C program:
\begin{itemize}
\item \funcname{caml\_start\_program}
\item \funcname{caml\_startup\_common}
\item \funcname{caml\_startup\_exn}
\item \funcname{caml\_startup}
\item \funcname{caml\_main}
\item \funcname{main}
\end{itemize}
%\item \funcname{caml\_startup\_common}: initializes GC, domains \dots
% Also: codefrag, locale, custom opeartions, frame_descr, signals, stack
\item \funcname{caml\_start\_program} is the transition point between the C realm and the OCaml one
\end{itemize}
\bigskip
\listocaml{../src/default/default.ml}{default/default.ml}
\end{column}
\begin{column}{0.55\textwidth}
\listasm[lastline=24,highlightlines={22-24}]{src/ocaml/caml\_start\_program.s}{runtime/amd64.S:604}
\end{column}
\end{columns}
\end{frame}


\begin{frame}{Setup to jump into OCaml entry point}{\funcname{caml\_start\_program} initializes the main fiber}
\begin{columns}
\begin{column}{0.6\textwidth}
\only<1,3,5>{
\begin{itemize}
\item<1-> Stores information to allow switching from OCaml stack to the C stack
\item<3-> Constructs the default exception handler
\item<5-> Switches to the OCaml stack and calls \funcname{caml\_program}
\end{itemize}
\bigskip
\listocaml{../src/default/default.ml}{default/default.ml}
}%
\only<2>{
\listasm[firstline=22,lastline=41,highlightlines={25-27}]{src/ocaml/caml\_start\_program.s}{caml\_start\_program}
}%
\only<4>{
\mintasm[firstline=22,lastline=41,highlightlines={31-38}]{src/ocaml/caml\_start\_program.s}
\listasm[firstline=66,lastline=72]{src/ocaml/caml\_start\_program.s}{caml\_start\_program}
}%
\only<6>{
\listasm[firstline=22,lastline=41,highlightlines={39-41}]{src/ocaml/caml\_start\_program.s}{caml\_start\_program}
}%
\end{column}
\begin{column}{0.4\textwidth}
\centering
\only<1-2>{\providecommand\step{2}\input{diagrams/default/default.tex}}%
\only<3-5>{\providecommand\step{3}\input{diagrams/default/default.tex}}%
\onslide*<6->{\providecommand\step{4}\input{diagrams/default/default.tex}}%
\end{column}
\end{columns}
\end{frame}

\begin{frame}{Executing the OCaml program}{And raising an exception without handler}
\begin{columns}
\begin{column}{0.6\textwidth}
\only<1-3>{
\begin{itemize}
\item<1-> Execution continues with OCaml code
\begin{itemize}
\item \funcname{camlDefault\_\_main\_269}
\item \funcname{camlDefault\_\_entry}
\item \funcname{caml\_program}
\end{itemize}
\item<1-> \funcname{camlDefault\_\_main\_269} raises a \typename{Failure} exception
\item<1-> \typename{Failure} is caught by \funcname{caml\_start\_program}
\begin{itemize}
\item<1-> The handler set the 2nd LSB of the exception value to \funcarg{1}
\item<3-> And then forces \funcname{caml\_start\_program} to terminate
\end{itemize}
\end{itemize}
}
\bigskip
\only<1,3>{\listocaml[highlightlines=3]{../src/default/default.ml}{default/default.ml}}%
\only<2>{\listasm[firstline=66,lastline=72,highlightlines=69]{src/ocaml/caml\_start\_program.s}{caml\_start\_program}}%
\end{column}
\begin{column}{0.4\textwidth}
\centering
\providecommand\step{4}\input{diagrams/default/default.tex}
\end{column}
\end{columns}
\end{frame}
%\only<>{\listasm[firstline=47,lastline=65]{src/ocaml/caml\_start\_program.s}{caml\_start\_program}}


\begin{frame}{Back to C}{Clean up, complain and terminate}
\begin{columns}
\begin{column}{0.45\textwidth}
\begin{itemize}
\item Backtrace:
\begin{itemize}
\item \funcname{caml\_start\_program} $\rightarrow$ OCaml code
\item \funcname{caml\_startup\_common}
\item \funcname{caml\_startup\_exn}
\item \funcname{caml\_startup}
\item \funcname{caml\_main}
\item \funcname{main}
\end{itemize}
\smallskip
\item \funcname{caml\_startup} checks the return value of \funcname{caml\_startup\_exn}
\begin{itemize}
\item The return value has been specially marked by \funcname{caml\_start\_program} handler
\item Calls \funcname{caml\_fatal\_uncaught\_exception}
\end{itemize}
\item<2-> \funcname{caml\_fatal\_uncaught\_exception} either
\begin{itemize}
\item<2-> Calls the user custom uncaught exception handler, if set with \mintinline{ocaml}{Printexc.set_uncaught_exception_handler fn}\footnotemark
\item<2-> Or falls back to the default one
\item<2-> Which notifies about the uncaught exception
\end{itemize}
\item<4-> Terminates the program either with \funcname{abort} or with a non-zero exit code
\end{itemize}
\end{column}
\begin{column}{0.55\textwidth}
\only<1>{
\listc[firstline=84,lastline=89,highlightlines=88]{../ocaml/runtime/caml/mlvalues.h}{runtime/caml/mlvalues.h:84}
\listc[firstline=138,lastline=143,highlightlines={141-142}]{../ocaml/runtime/startup\_nat.c}{runtime/startup\_nat.c:138}
}%
\only<2>{
\listc[firstline=138,highlightlines={147,149}]{../ocaml/runtime/printexc.c}{runtime/printexc.c:138}
}%
\only<3>{
\listc[firstline=110,lastline=134,highlightlines=129]{../ocaml/runtime/printexc.c}{runtime/printexc.c:138}
}%
\only<4>{
\listc[firstline=138,highlightlines={152,154}]{../ocaml/runtime/printexc.c}{runtime/printexc.c:138}
}%
\end{column}
\end{columns}
\footnotetext[1]{\url{https://v2.ocaml.org/api/Printexc.html\#1\_Uncaughtexceptions}}
\end{frame}
}%
\end{column}
\end{columns}
\end{frame}

\begin{frame}{Executing the OCaml program}{And raising an exception without handler}
\begin{columns}
\begin{column}{0.6\textwidth}
\only<1-3>{
\begin{itemize}
\item<1-> Execution continues with OCaml code
\begin{itemize}
\item \funcname{camlDefault\_\_main\_269}
\item \funcname{camlDefault\_\_entry}
\item \funcname{caml\_program}
\end{itemize}
\item<1-> \funcname{camlDefault\_\_main\_269} raises a \typename{Failure} exception
\item<1-> \typename{Failure} is caught by \funcname{caml\_start\_program}
\begin{itemize}
\item<1-> The handler set the 2nd LSB of the exception value to \funcarg{1}
\item<3-> And then forces \funcname{caml\_start\_program} to terminate
\end{itemize}
\end{itemize}
}
\bigskip
\only<1,3>{\listocaml[highlightlines=3]{../src/default/default.ml}{default/default.ml}}%
\only<2>{\listasm[firstline=66,lastline=72,highlightlines=69]{src/ocaml/caml\_start\_program.s}{caml\_start\_program}}%
\end{column}
\begin{column}{0.4\textwidth}
\centering
\providecommand\step{4}%
% Runtime's default exception handler
%
\subsection*{Runtime's default exception handler}
\frameSubsection{pictures/The\_Architect.png}{}


\begin{frame}{Default exception handler}
\begin{columns}
\begin{column}{0.5\textwidth}
\begin{itemize}
\item \localname{domain\_state->exn\_handler} points to a trap located before \funcname{entry}!
\item What installed this very first trap there?
\end{itemize}
\bigskip
\listocaml{../src/default/default.ml}{default/default.ml}
\bigskip
\inputminted[fontsize=\footnotesize]{text}{../src/default/default.out}
\end{column}
\begin{column}{0.5\textwidth}
\centering
\only<1>{\providecommand\step{0}\input{diagrams/default/default.tex}}%
\only<2>{\providecommand\step{4}\input{diagrams/default/default.tex}}%
\end{column}
\end{columns}
\end{frame}


\begin{frame}{Startup}{How did we get there by the way?}
\begin{columns}
\begin{column}{0.45\textwidth}
\begin{itemize}
\item The entry point address of the binary is \funcname{main} (\funcname{\_start}) from the runtime
\item The program starts like a C program:
\begin{itemize}
\item \funcname{caml\_start\_program}
\item \funcname{caml\_startup\_common}
\item \funcname{caml\_startup\_exn}
\item \funcname{caml\_startup}
\item \funcname{caml\_main}
\item \funcname{main}
\end{itemize}
%\item \funcname{caml\_startup\_common}: initializes GC, domains \dots
% Also: codefrag, locale, custom opeartions, frame_descr, signals, stack
\item \funcname{caml\_start\_program} is the transition point between the C realm and the OCaml one
\end{itemize}
\bigskip
\listocaml{../src/default/default.ml}{default/default.ml}
\end{column}
\begin{column}{0.55\textwidth}
\listasm[lastline=24,highlightlines={22-24}]{src/ocaml/caml\_start\_program.s}{runtime/amd64.S:604}
\end{column}
\end{columns}
\end{frame}


\begin{frame}{Setup to jump into OCaml entry point}{\funcname{caml\_start\_program} initializes the main fiber}
\begin{columns}
\begin{column}{0.6\textwidth}
\only<1,3,5>{
\begin{itemize}
\item<1-> Stores information to allow switching from OCaml stack to the C stack
\item<3-> Constructs the default exception handler
\item<5-> Switches to the OCaml stack and calls \funcname{caml\_program}
\end{itemize}
\bigskip
\listocaml{../src/default/default.ml}{default/default.ml}
}%
\only<2>{
\listasm[firstline=22,lastline=41,highlightlines={25-27}]{src/ocaml/caml\_start\_program.s}{caml\_start\_program}
}%
\only<4>{
\mintasm[firstline=22,lastline=41,highlightlines={31-38}]{src/ocaml/caml\_start\_program.s}
\listasm[firstline=66,lastline=72]{src/ocaml/caml\_start\_program.s}{caml\_start\_program}
}%
\only<6>{
\listasm[firstline=22,lastline=41,highlightlines={39-41}]{src/ocaml/caml\_start\_program.s}{caml\_start\_program}
}%
\end{column}
\begin{column}{0.4\textwidth}
\centering
\only<1-2>{\providecommand\step{2}\input{diagrams/default/default.tex}}%
\only<3-5>{\providecommand\step{3}\input{diagrams/default/default.tex}}%
\onslide*<6->{\providecommand\step{4}\input{diagrams/default/default.tex}}%
\end{column}
\end{columns}
\end{frame}

\begin{frame}{Executing the OCaml program}{And raising an exception without handler}
\begin{columns}
\begin{column}{0.6\textwidth}
\only<1-3>{
\begin{itemize}
\item<1-> Execution continues with OCaml code
\begin{itemize}
\item \funcname{camlDefault\_\_main\_269}
\item \funcname{camlDefault\_\_entry}
\item \funcname{caml\_program}
\end{itemize}
\item<1-> \funcname{camlDefault\_\_main\_269} raises a \typename{Failure} exception
\item<1-> \typename{Failure} is caught by \funcname{caml\_start\_program}
\begin{itemize}
\item<1-> The handler set the 2nd LSB of the exception value to \funcarg{1}
\item<3-> And then forces \funcname{caml\_start\_program} to terminate
\end{itemize}
\end{itemize}
}
\bigskip
\only<1,3>{\listocaml[highlightlines=3]{../src/default/default.ml}{default/default.ml}}%
\only<2>{\listasm[firstline=66,lastline=72,highlightlines=69]{src/ocaml/caml\_start\_program.s}{caml\_start\_program}}%
\end{column}
\begin{column}{0.4\textwidth}
\centering
\providecommand\step{4}\input{diagrams/default/default.tex}
\end{column}
\end{columns}
\end{frame}
%\only<>{\listasm[firstline=47,lastline=65]{src/ocaml/caml\_start\_program.s}{caml\_start\_program}}


\begin{frame}{Back to C}{Clean up, complain and terminate}
\begin{columns}
\begin{column}{0.45\textwidth}
\begin{itemize}
\item Backtrace:
\begin{itemize}
\item \funcname{caml\_start\_program} $\rightarrow$ OCaml code
\item \funcname{caml\_startup\_common}
\item \funcname{caml\_startup\_exn}
\item \funcname{caml\_startup}
\item \funcname{caml\_main}
\item \funcname{main}
\end{itemize}
\smallskip
\item \funcname{caml\_startup} checks the return value of \funcname{caml\_startup\_exn}
\begin{itemize}
\item The return value has been specially marked by \funcname{caml\_start\_program} handler
\item Calls \funcname{caml\_fatal\_uncaught\_exception}
\end{itemize}
\item<2-> \funcname{caml\_fatal\_uncaught\_exception} either
\begin{itemize}
\item<2-> Calls the user custom uncaught exception handler, if set with \mintinline{ocaml}{Printexc.set_uncaught_exception_handler fn}\footnotemark
\item<2-> Or falls back to the default one
\item<2-> Which notifies about the uncaught exception
\end{itemize}
\item<4-> Terminates the program either with \funcname{abort} or with a non-zero exit code
\end{itemize}
\end{column}
\begin{column}{0.55\textwidth}
\only<1>{
\listc[firstline=84,lastline=89,highlightlines=88]{../ocaml/runtime/caml/mlvalues.h}{runtime/caml/mlvalues.h:84}
\listc[firstline=138,lastline=143,highlightlines={141-142}]{../ocaml/runtime/startup\_nat.c}{runtime/startup\_nat.c:138}
}%
\only<2>{
\listc[firstline=138,highlightlines={147,149}]{../ocaml/runtime/printexc.c}{runtime/printexc.c:138}
}%
\only<3>{
\listc[firstline=110,lastline=134,highlightlines=129]{../ocaml/runtime/printexc.c}{runtime/printexc.c:138}
}%
\only<4>{
\listc[firstline=138,highlightlines={152,154}]{../ocaml/runtime/printexc.c}{runtime/printexc.c:138}
}%
\end{column}
\end{columns}
\footnotetext[1]{\url{https://v2.ocaml.org/api/Printexc.html\#1\_Uncaughtexceptions}}
\end{frame}

\end{column}
\end{columns}
\end{frame}
%\only<>{\listasm[firstline=47,lastline=65]{src/ocaml/caml\_start\_program.s}{caml\_start\_program}}


\begin{frame}{Back to C}{Clean up, complain and terminate}
\begin{columns}
\begin{column}{0.45\textwidth}
\begin{itemize}
\item Backtrace:
\begin{itemize}
\item \funcname{caml\_start\_program} $\rightarrow$ OCaml code
\item \funcname{caml\_startup\_common}
\item \funcname{caml\_startup\_exn}
\item \funcname{caml\_startup}
\item \funcname{caml\_main}
\item \funcname{main}
\end{itemize}
\smallskip
\item \funcname{caml\_startup} checks the return value of \funcname{caml\_startup\_exn}
\begin{itemize}
\item The return value has been specially marked by \funcname{caml\_start\_program} handler
\item Calls \funcname{caml\_fatal\_uncaught\_exception}
\end{itemize}
\item<2-> \funcname{caml\_fatal\_uncaught\_exception} either
\begin{itemize}
\item<2-> Calls the user custom uncaught exception handler, if set with \mintinline{ocaml}{Printexc.set_uncaught_exception_handler fn}\footnotemark
\item<2-> Or falls back to the default one
\item<2-> Which notifies about the uncaught exception
\end{itemize}
\item<4-> Terminates the program either with \funcname{abort} or with a non-zero exit code
\end{itemize}
\end{column}
\begin{column}{0.55\textwidth}
\only<1>{
\listc[firstline=84,lastline=89,highlightlines=88]{../ocaml/runtime/caml/mlvalues.h}{runtime/caml/mlvalues.h:84}
\listc[firstline=138,lastline=143,highlightlines={141-142}]{../ocaml/runtime/startup\_nat.c}{runtime/startup\_nat.c:138}
}%
\only<2>{
\listc[firstline=138,highlightlines={147,149}]{../ocaml/runtime/printexc.c}{runtime/printexc.c:138}
}%
\only<3>{
\listc[firstline=110,lastline=134,highlightlines=129]{../ocaml/runtime/printexc.c}{runtime/printexc.c:138}
}%
\only<4>{
\listc[firstline=138,highlightlines={152,154}]{../ocaml/runtime/printexc.c}{runtime/printexc.c:138}
}%
\end{column}
\end{columns}
\footnotetext[1]{\url{https://v2.ocaml.org/api/Printexc.html\#1\_Uncaughtexceptions}}
\end{frame}
}%
\end{column}
\end{columns}
\end{frame}


\begin{frame}{Startup}{How did we get there by the way?}
\begin{columns}
\begin{column}{0.45\textwidth}
\begin{itemize}
\item The entry point address of the binary is \funcname{main} (\funcname{\_start}) from the runtime
\item The program starts like a C program:
\begin{itemize}
\item \funcname{caml\_start\_program}
\item \funcname{caml\_startup\_common}
\item \funcname{caml\_startup\_exn}
\item \funcname{caml\_startup}
\item \funcname{caml\_main}
\item \funcname{main}
\end{itemize}
%\item \funcname{caml\_startup\_common}: initializes GC, domains \dots
% Also: codefrag, locale, custom opeartions, frame_descr, signals, stack
\item \funcname{caml\_start\_program} is the transition point between the C realm and the OCaml one
\end{itemize}
\bigskip
\listocaml{../src/default/default.ml}{default/default.ml}
\end{column}
\begin{column}{0.55\textwidth}
\listasm[lastline=24,highlightlines={22-24}]{src/ocaml/caml\_start\_program.s}{runtime/amd64.S:604}
\end{column}
\end{columns}
\end{frame}


\begin{frame}{Setup to jump into OCaml entry point}{\funcname{caml\_start\_program} initializes the main fiber}
\begin{columns}
\begin{column}{0.6\textwidth}
\only<1,3,5>{
\begin{itemize}
\item<1-> Stores information to allow switching from OCaml stack to the C stack
\item<3-> Constructs the default exception handler
\item<5-> Switches to the OCaml stack and calls \funcname{caml\_program}
\end{itemize}
\bigskip
\listocaml{../src/default/default.ml}{default/default.ml}
}%
\only<2>{
\listasm[firstline=22,lastline=41,highlightlines={25-27}]{src/ocaml/caml\_start\_program.s}{caml\_start\_program}
}%
\only<4>{
\mintasm[firstline=22,lastline=41,highlightlines={31-38}]{src/ocaml/caml\_start\_program.s}
\listasm[firstline=66,lastline=72]{src/ocaml/caml\_start\_program.s}{caml\_start\_program}
}%
\only<6>{
\listasm[firstline=22,lastline=41,highlightlines={39-41}]{src/ocaml/caml\_start\_program.s}{caml\_start\_program}
}%
\end{column}
\begin{column}{0.4\textwidth}
\centering
\only<1-2>{\providecommand\step{2}%
% Runtime's default exception handler
%
\subsection*{Runtime's default exception handler}
\frameSubsection{pictures/The\_Architect.png}{}


\begin{frame}{Default exception handler}
\begin{columns}
\begin{column}{0.5\textwidth}
\begin{itemize}
\item \localname{domain\_state->exn\_handler} points to a trap located before \funcname{entry}!
\item What installed this very first trap there?
\end{itemize}
\bigskip
\listocaml{../src/default/default.ml}{default/default.ml}
\bigskip
\inputminted[fontsize=\footnotesize]{text}{../src/default/default.out}
\end{column}
\begin{column}{0.5\textwidth}
\centering
\only<1>{\providecommand\step{0}%
% Runtime's default exception handler
%
\subsection*{Runtime's default exception handler}
\frameSubsection{pictures/The\_Architect.png}{}


\begin{frame}{Default exception handler}
\begin{columns}
\begin{column}{0.5\textwidth}
\begin{itemize}
\item \localname{domain\_state->exn\_handler} points to a trap located before \funcname{entry}!
\item What installed this very first trap there?
\end{itemize}
\bigskip
\listocaml{../src/default/default.ml}{default/default.ml}
\bigskip
\inputminted[fontsize=\footnotesize]{text}{../src/default/default.out}
\end{column}
\begin{column}{0.5\textwidth}
\centering
\only<1>{\providecommand\step{0}\input{diagrams/default/default.tex}}%
\only<2>{\providecommand\step{4}\input{diagrams/default/default.tex}}%
\end{column}
\end{columns}
\end{frame}


\begin{frame}{Startup}{How did we get there by the way?}
\begin{columns}
\begin{column}{0.45\textwidth}
\begin{itemize}
\item The entry point address of the binary is \funcname{main} (\funcname{\_start}) from the runtime
\item The program starts like a C program:
\begin{itemize}
\item \funcname{caml\_start\_program}
\item \funcname{caml\_startup\_common}
\item \funcname{caml\_startup\_exn}
\item \funcname{caml\_startup}
\item \funcname{caml\_main}
\item \funcname{main}
\end{itemize}
%\item \funcname{caml\_startup\_common}: initializes GC, domains \dots
% Also: codefrag, locale, custom opeartions, frame_descr, signals, stack
\item \funcname{caml\_start\_program} is the transition point between the C realm and the OCaml one
\end{itemize}
\bigskip
\listocaml{../src/default/default.ml}{default/default.ml}
\end{column}
\begin{column}{0.55\textwidth}
\listasm[lastline=24,highlightlines={22-24}]{src/ocaml/caml\_start\_program.s}{runtime/amd64.S:604}
\end{column}
\end{columns}
\end{frame}


\begin{frame}{Setup to jump into OCaml entry point}{\funcname{caml\_start\_program} initializes the main fiber}
\begin{columns}
\begin{column}{0.6\textwidth}
\only<1,3,5>{
\begin{itemize}
\item<1-> Stores information to allow switching from OCaml stack to the C stack
\item<3-> Constructs the default exception handler
\item<5-> Switches to the OCaml stack and calls \funcname{caml\_program}
\end{itemize}
\bigskip
\listocaml{../src/default/default.ml}{default/default.ml}
}%
\only<2>{
\listasm[firstline=22,lastline=41,highlightlines={25-27}]{src/ocaml/caml\_start\_program.s}{caml\_start\_program}
}%
\only<4>{
\mintasm[firstline=22,lastline=41,highlightlines={31-38}]{src/ocaml/caml\_start\_program.s}
\listasm[firstline=66,lastline=72]{src/ocaml/caml\_start\_program.s}{caml\_start\_program}
}%
\only<6>{
\listasm[firstline=22,lastline=41,highlightlines={39-41}]{src/ocaml/caml\_start\_program.s}{caml\_start\_program}
}%
\end{column}
\begin{column}{0.4\textwidth}
\centering
\only<1-2>{\providecommand\step{2}\input{diagrams/default/default.tex}}%
\only<3-5>{\providecommand\step{3}\input{diagrams/default/default.tex}}%
\onslide*<6->{\providecommand\step{4}\input{diagrams/default/default.tex}}%
\end{column}
\end{columns}
\end{frame}

\begin{frame}{Executing the OCaml program}{And raising an exception without handler}
\begin{columns}
\begin{column}{0.6\textwidth}
\only<1-3>{
\begin{itemize}
\item<1-> Execution continues with OCaml code
\begin{itemize}
\item \funcname{camlDefault\_\_main\_269}
\item \funcname{camlDefault\_\_entry}
\item \funcname{caml\_program}
\end{itemize}
\item<1-> \funcname{camlDefault\_\_main\_269} raises a \typename{Failure} exception
\item<1-> \typename{Failure} is caught by \funcname{caml\_start\_program}
\begin{itemize}
\item<1-> The handler set the 2nd LSB of the exception value to \funcarg{1}
\item<3-> And then forces \funcname{caml\_start\_program} to terminate
\end{itemize}
\end{itemize}
}
\bigskip
\only<1,3>{\listocaml[highlightlines=3]{../src/default/default.ml}{default/default.ml}}%
\only<2>{\listasm[firstline=66,lastline=72,highlightlines=69]{src/ocaml/caml\_start\_program.s}{caml\_start\_program}}%
\end{column}
\begin{column}{0.4\textwidth}
\centering
\providecommand\step{4}\input{diagrams/default/default.tex}
\end{column}
\end{columns}
\end{frame}
%\only<>{\listasm[firstline=47,lastline=65]{src/ocaml/caml\_start\_program.s}{caml\_start\_program}}


\begin{frame}{Back to C}{Clean up, complain and terminate}
\begin{columns}
\begin{column}{0.45\textwidth}
\begin{itemize}
\item Backtrace:
\begin{itemize}
\item \funcname{caml\_start\_program} $\rightarrow$ OCaml code
\item \funcname{caml\_startup\_common}
\item \funcname{caml\_startup\_exn}
\item \funcname{caml\_startup}
\item \funcname{caml\_main}
\item \funcname{main}
\end{itemize}
\smallskip
\item \funcname{caml\_startup} checks the return value of \funcname{caml\_startup\_exn}
\begin{itemize}
\item The return value has been specially marked by \funcname{caml\_start\_program} handler
\item Calls \funcname{caml\_fatal\_uncaught\_exception}
\end{itemize}
\item<2-> \funcname{caml\_fatal\_uncaught\_exception} either
\begin{itemize}
\item<2-> Calls the user custom uncaught exception handler, if set with \mintinline{ocaml}{Printexc.set_uncaught_exception_handler fn}\footnotemark
\item<2-> Or falls back to the default one
\item<2-> Which notifies about the uncaught exception
\end{itemize}
\item<4-> Terminates the program either with \funcname{abort} or with a non-zero exit code
\end{itemize}
\end{column}
\begin{column}{0.55\textwidth}
\only<1>{
\listc[firstline=84,lastline=89,highlightlines=88]{../ocaml/runtime/caml/mlvalues.h}{runtime/caml/mlvalues.h:84}
\listc[firstline=138,lastline=143,highlightlines={141-142}]{../ocaml/runtime/startup\_nat.c}{runtime/startup\_nat.c:138}
}%
\only<2>{
\listc[firstline=138,highlightlines={147,149}]{../ocaml/runtime/printexc.c}{runtime/printexc.c:138}
}%
\only<3>{
\listc[firstline=110,lastline=134,highlightlines=129]{../ocaml/runtime/printexc.c}{runtime/printexc.c:138}
}%
\only<4>{
\listc[firstline=138,highlightlines={152,154}]{../ocaml/runtime/printexc.c}{runtime/printexc.c:138}
}%
\end{column}
\end{columns}
\footnotetext[1]{\url{https://v2.ocaml.org/api/Printexc.html\#1\_Uncaughtexceptions}}
\end{frame}
}%
\only<2>{\providecommand\step{4}%
% Runtime's default exception handler
%
\subsection*{Runtime's default exception handler}
\frameSubsection{pictures/The\_Architect.png}{}


\begin{frame}{Default exception handler}
\begin{columns}
\begin{column}{0.5\textwidth}
\begin{itemize}
\item \localname{domain\_state->exn\_handler} points to a trap located before \funcname{entry}!
\item What installed this very first trap there?
\end{itemize}
\bigskip
\listocaml{../src/default/default.ml}{default/default.ml}
\bigskip
\inputminted[fontsize=\footnotesize]{text}{../src/default/default.out}
\end{column}
\begin{column}{0.5\textwidth}
\centering
\only<1>{\providecommand\step{0}\input{diagrams/default/default.tex}}%
\only<2>{\providecommand\step{4}\input{diagrams/default/default.tex}}%
\end{column}
\end{columns}
\end{frame}


\begin{frame}{Startup}{How did we get there by the way?}
\begin{columns}
\begin{column}{0.45\textwidth}
\begin{itemize}
\item The entry point address of the binary is \funcname{main} (\funcname{\_start}) from the runtime
\item The program starts like a C program:
\begin{itemize}
\item \funcname{caml\_start\_program}
\item \funcname{caml\_startup\_common}
\item \funcname{caml\_startup\_exn}
\item \funcname{caml\_startup}
\item \funcname{caml\_main}
\item \funcname{main}
\end{itemize}
%\item \funcname{caml\_startup\_common}: initializes GC, domains \dots
% Also: codefrag, locale, custom opeartions, frame_descr, signals, stack
\item \funcname{caml\_start\_program} is the transition point between the C realm and the OCaml one
\end{itemize}
\bigskip
\listocaml{../src/default/default.ml}{default/default.ml}
\end{column}
\begin{column}{0.55\textwidth}
\listasm[lastline=24,highlightlines={22-24}]{src/ocaml/caml\_start\_program.s}{runtime/amd64.S:604}
\end{column}
\end{columns}
\end{frame}


\begin{frame}{Setup to jump into OCaml entry point}{\funcname{caml\_start\_program} initializes the main fiber}
\begin{columns}
\begin{column}{0.6\textwidth}
\only<1,3,5>{
\begin{itemize}
\item<1-> Stores information to allow switching from OCaml stack to the C stack
\item<3-> Constructs the default exception handler
\item<5-> Switches to the OCaml stack and calls \funcname{caml\_program}
\end{itemize}
\bigskip
\listocaml{../src/default/default.ml}{default/default.ml}
}%
\only<2>{
\listasm[firstline=22,lastline=41,highlightlines={25-27}]{src/ocaml/caml\_start\_program.s}{caml\_start\_program}
}%
\only<4>{
\mintasm[firstline=22,lastline=41,highlightlines={31-38}]{src/ocaml/caml\_start\_program.s}
\listasm[firstline=66,lastline=72]{src/ocaml/caml\_start\_program.s}{caml\_start\_program}
}%
\only<6>{
\listasm[firstline=22,lastline=41,highlightlines={39-41}]{src/ocaml/caml\_start\_program.s}{caml\_start\_program}
}%
\end{column}
\begin{column}{0.4\textwidth}
\centering
\only<1-2>{\providecommand\step{2}\input{diagrams/default/default.tex}}%
\only<3-5>{\providecommand\step{3}\input{diagrams/default/default.tex}}%
\onslide*<6->{\providecommand\step{4}\input{diagrams/default/default.tex}}%
\end{column}
\end{columns}
\end{frame}

\begin{frame}{Executing the OCaml program}{And raising an exception without handler}
\begin{columns}
\begin{column}{0.6\textwidth}
\only<1-3>{
\begin{itemize}
\item<1-> Execution continues with OCaml code
\begin{itemize}
\item \funcname{camlDefault\_\_main\_269}
\item \funcname{camlDefault\_\_entry}
\item \funcname{caml\_program}
\end{itemize}
\item<1-> \funcname{camlDefault\_\_main\_269} raises a \typename{Failure} exception
\item<1-> \typename{Failure} is caught by \funcname{caml\_start\_program}
\begin{itemize}
\item<1-> The handler set the 2nd LSB of the exception value to \funcarg{1}
\item<3-> And then forces \funcname{caml\_start\_program} to terminate
\end{itemize}
\end{itemize}
}
\bigskip
\only<1,3>{\listocaml[highlightlines=3]{../src/default/default.ml}{default/default.ml}}%
\only<2>{\listasm[firstline=66,lastline=72,highlightlines=69]{src/ocaml/caml\_start\_program.s}{caml\_start\_program}}%
\end{column}
\begin{column}{0.4\textwidth}
\centering
\providecommand\step{4}\input{diagrams/default/default.tex}
\end{column}
\end{columns}
\end{frame}
%\only<>{\listasm[firstline=47,lastline=65]{src/ocaml/caml\_start\_program.s}{caml\_start\_program}}


\begin{frame}{Back to C}{Clean up, complain and terminate}
\begin{columns}
\begin{column}{0.45\textwidth}
\begin{itemize}
\item Backtrace:
\begin{itemize}
\item \funcname{caml\_start\_program} $\rightarrow$ OCaml code
\item \funcname{caml\_startup\_common}
\item \funcname{caml\_startup\_exn}
\item \funcname{caml\_startup}
\item \funcname{caml\_main}
\item \funcname{main}
\end{itemize}
\smallskip
\item \funcname{caml\_startup} checks the return value of \funcname{caml\_startup\_exn}
\begin{itemize}
\item The return value has been specially marked by \funcname{caml\_start\_program} handler
\item Calls \funcname{caml\_fatal\_uncaught\_exception}
\end{itemize}
\item<2-> \funcname{caml\_fatal\_uncaught\_exception} either
\begin{itemize}
\item<2-> Calls the user custom uncaught exception handler, if set with \mintinline{ocaml}{Printexc.set_uncaught_exception_handler fn}\footnotemark
\item<2-> Or falls back to the default one
\item<2-> Which notifies about the uncaught exception
\end{itemize}
\item<4-> Terminates the program either with \funcname{abort} or with a non-zero exit code
\end{itemize}
\end{column}
\begin{column}{0.55\textwidth}
\only<1>{
\listc[firstline=84,lastline=89,highlightlines=88]{../ocaml/runtime/caml/mlvalues.h}{runtime/caml/mlvalues.h:84}
\listc[firstline=138,lastline=143,highlightlines={141-142}]{../ocaml/runtime/startup\_nat.c}{runtime/startup\_nat.c:138}
}%
\only<2>{
\listc[firstline=138,highlightlines={147,149}]{../ocaml/runtime/printexc.c}{runtime/printexc.c:138}
}%
\only<3>{
\listc[firstline=110,lastline=134,highlightlines=129]{../ocaml/runtime/printexc.c}{runtime/printexc.c:138}
}%
\only<4>{
\listc[firstline=138,highlightlines={152,154}]{../ocaml/runtime/printexc.c}{runtime/printexc.c:138}
}%
\end{column}
\end{columns}
\footnotetext[1]{\url{https://v2.ocaml.org/api/Printexc.html\#1\_Uncaughtexceptions}}
\end{frame}
}%
\end{column}
\end{columns}
\end{frame}


\begin{frame}{Startup}{How did we get there by the way?}
\begin{columns}
\begin{column}{0.45\textwidth}
\begin{itemize}
\item The entry point address of the binary is \funcname{main} (\funcname{\_start}) from the runtime
\item The program starts like a C program:
\begin{itemize}
\item \funcname{caml\_start\_program}
\item \funcname{caml\_startup\_common}
\item \funcname{caml\_startup\_exn}
\item \funcname{caml\_startup}
\item \funcname{caml\_main}
\item \funcname{main}
\end{itemize}
%\item \funcname{caml\_startup\_common}: initializes GC, domains \dots
% Also: codefrag, locale, custom opeartions, frame_descr, signals, stack
\item \funcname{caml\_start\_program} is the transition point between the C realm and the OCaml one
\end{itemize}
\bigskip
\listocaml{../src/default/default.ml}{default/default.ml}
\end{column}
\begin{column}{0.55\textwidth}
\listasm[lastline=24,highlightlines={22-24}]{src/ocaml/caml\_start\_program.s}{runtime/amd64.S:604}
\end{column}
\end{columns}
\end{frame}


\begin{frame}{Setup to jump into OCaml entry point}{\funcname{caml\_start\_program} initializes the main fiber}
\begin{columns}
\begin{column}{0.6\textwidth}
\only<1,3,5>{
\begin{itemize}
\item<1-> Stores information to allow switching from OCaml stack to the C stack
\item<3-> Constructs the default exception handler
\item<5-> Switches to the OCaml stack and calls \funcname{caml\_program}
\end{itemize}
\bigskip
\listocaml{../src/default/default.ml}{default/default.ml}
}%
\only<2>{
\listasm[firstline=22,lastline=41,highlightlines={25-27}]{src/ocaml/caml\_start\_program.s}{caml\_start\_program}
}%
\only<4>{
\mintasm[firstline=22,lastline=41,highlightlines={31-38}]{src/ocaml/caml\_start\_program.s}
\listasm[firstline=66,lastline=72]{src/ocaml/caml\_start\_program.s}{caml\_start\_program}
}%
\only<6>{
\listasm[firstline=22,lastline=41,highlightlines={39-41}]{src/ocaml/caml\_start\_program.s}{caml\_start\_program}
}%
\end{column}
\begin{column}{0.4\textwidth}
\centering
\only<1-2>{\providecommand\step{2}%
% Runtime's default exception handler
%
\subsection*{Runtime's default exception handler}
\frameSubsection{pictures/The\_Architect.png}{}


\begin{frame}{Default exception handler}
\begin{columns}
\begin{column}{0.5\textwidth}
\begin{itemize}
\item \localname{domain\_state->exn\_handler} points to a trap located before \funcname{entry}!
\item What installed this very first trap there?
\end{itemize}
\bigskip
\listocaml{../src/default/default.ml}{default/default.ml}
\bigskip
\inputminted[fontsize=\footnotesize]{text}{../src/default/default.out}
\end{column}
\begin{column}{0.5\textwidth}
\centering
\only<1>{\providecommand\step{0}\input{diagrams/default/default.tex}}%
\only<2>{\providecommand\step{4}\input{diagrams/default/default.tex}}%
\end{column}
\end{columns}
\end{frame}


\begin{frame}{Startup}{How did we get there by the way?}
\begin{columns}
\begin{column}{0.45\textwidth}
\begin{itemize}
\item The entry point address of the binary is \funcname{main} (\funcname{\_start}) from the runtime
\item The program starts like a C program:
\begin{itemize}
\item \funcname{caml\_start\_program}
\item \funcname{caml\_startup\_common}
\item \funcname{caml\_startup\_exn}
\item \funcname{caml\_startup}
\item \funcname{caml\_main}
\item \funcname{main}
\end{itemize}
%\item \funcname{caml\_startup\_common}: initializes GC, domains \dots
% Also: codefrag, locale, custom opeartions, frame_descr, signals, stack
\item \funcname{caml\_start\_program} is the transition point between the C realm and the OCaml one
\end{itemize}
\bigskip
\listocaml{../src/default/default.ml}{default/default.ml}
\end{column}
\begin{column}{0.55\textwidth}
\listasm[lastline=24,highlightlines={22-24}]{src/ocaml/caml\_start\_program.s}{runtime/amd64.S:604}
\end{column}
\end{columns}
\end{frame}


\begin{frame}{Setup to jump into OCaml entry point}{\funcname{caml\_start\_program} initializes the main fiber}
\begin{columns}
\begin{column}{0.6\textwidth}
\only<1,3,5>{
\begin{itemize}
\item<1-> Stores information to allow switching from OCaml stack to the C stack
\item<3-> Constructs the default exception handler
\item<5-> Switches to the OCaml stack and calls \funcname{caml\_program}
\end{itemize}
\bigskip
\listocaml{../src/default/default.ml}{default/default.ml}
}%
\only<2>{
\listasm[firstline=22,lastline=41,highlightlines={25-27}]{src/ocaml/caml\_start\_program.s}{caml\_start\_program}
}%
\only<4>{
\mintasm[firstline=22,lastline=41,highlightlines={31-38}]{src/ocaml/caml\_start\_program.s}
\listasm[firstline=66,lastline=72]{src/ocaml/caml\_start\_program.s}{caml\_start\_program}
}%
\only<6>{
\listasm[firstline=22,lastline=41,highlightlines={39-41}]{src/ocaml/caml\_start\_program.s}{caml\_start\_program}
}%
\end{column}
\begin{column}{0.4\textwidth}
\centering
\only<1-2>{\providecommand\step{2}\input{diagrams/default/default.tex}}%
\only<3-5>{\providecommand\step{3}\input{diagrams/default/default.tex}}%
\onslide*<6->{\providecommand\step{4}\input{diagrams/default/default.tex}}%
\end{column}
\end{columns}
\end{frame}

\begin{frame}{Executing the OCaml program}{And raising an exception without handler}
\begin{columns}
\begin{column}{0.6\textwidth}
\only<1-3>{
\begin{itemize}
\item<1-> Execution continues with OCaml code
\begin{itemize}
\item \funcname{camlDefault\_\_main\_269}
\item \funcname{camlDefault\_\_entry}
\item \funcname{caml\_program}
\end{itemize}
\item<1-> \funcname{camlDefault\_\_main\_269} raises a \typename{Failure} exception
\item<1-> \typename{Failure} is caught by \funcname{caml\_start\_program}
\begin{itemize}
\item<1-> The handler set the 2nd LSB of the exception value to \funcarg{1}
\item<3-> And then forces \funcname{caml\_start\_program} to terminate
\end{itemize}
\end{itemize}
}
\bigskip
\only<1,3>{\listocaml[highlightlines=3]{../src/default/default.ml}{default/default.ml}}%
\only<2>{\listasm[firstline=66,lastline=72,highlightlines=69]{src/ocaml/caml\_start\_program.s}{caml\_start\_program}}%
\end{column}
\begin{column}{0.4\textwidth}
\centering
\providecommand\step{4}\input{diagrams/default/default.tex}
\end{column}
\end{columns}
\end{frame}
%\only<>{\listasm[firstline=47,lastline=65]{src/ocaml/caml\_start\_program.s}{caml\_start\_program}}


\begin{frame}{Back to C}{Clean up, complain and terminate}
\begin{columns}
\begin{column}{0.45\textwidth}
\begin{itemize}
\item Backtrace:
\begin{itemize}
\item \funcname{caml\_start\_program} $\rightarrow$ OCaml code
\item \funcname{caml\_startup\_common}
\item \funcname{caml\_startup\_exn}
\item \funcname{caml\_startup}
\item \funcname{caml\_main}
\item \funcname{main}
\end{itemize}
\smallskip
\item \funcname{caml\_startup} checks the return value of \funcname{caml\_startup\_exn}
\begin{itemize}
\item The return value has been specially marked by \funcname{caml\_start\_program} handler
\item Calls \funcname{caml\_fatal\_uncaught\_exception}
\end{itemize}
\item<2-> \funcname{caml\_fatal\_uncaught\_exception} either
\begin{itemize}
\item<2-> Calls the user custom uncaught exception handler, if set with \mintinline{ocaml}{Printexc.set_uncaught_exception_handler fn}\footnotemark
\item<2-> Or falls back to the default one
\item<2-> Which notifies about the uncaught exception
\end{itemize}
\item<4-> Terminates the program either with \funcname{abort} or with a non-zero exit code
\end{itemize}
\end{column}
\begin{column}{0.55\textwidth}
\only<1>{
\listc[firstline=84,lastline=89,highlightlines=88]{../ocaml/runtime/caml/mlvalues.h}{runtime/caml/mlvalues.h:84}
\listc[firstline=138,lastline=143,highlightlines={141-142}]{../ocaml/runtime/startup\_nat.c}{runtime/startup\_nat.c:138}
}%
\only<2>{
\listc[firstline=138,highlightlines={147,149}]{../ocaml/runtime/printexc.c}{runtime/printexc.c:138}
}%
\only<3>{
\listc[firstline=110,lastline=134,highlightlines=129]{../ocaml/runtime/printexc.c}{runtime/printexc.c:138}
}%
\only<4>{
\listc[firstline=138,highlightlines={152,154}]{../ocaml/runtime/printexc.c}{runtime/printexc.c:138}
}%
\end{column}
\end{columns}
\footnotetext[1]{\url{https://v2.ocaml.org/api/Printexc.html\#1\_Uncaughtexceptions}}
\end{frame}
}%
\only<3-5>{\providecommand\step{3}%
% Runtime's default exception handler
%
\subsection*{Runtime's default exception handler}
\frameSubsection{pictures/The\_Architect.png}{}


\begin{frame}{Default exception handler}
\begin{columns}
\begin{column}{0.5\textwidth}
\begin{itemize}
\item \localname{domain\_state->exn\_handler} points to a trap located before \funcname{entry}!
\item What installed this very first trap there?
\end{itemize}
\bigskip
\listocaml{../src/default/default.ml}{default/default.ml}
\bigskip
\inputminted[fontsize=\footnotesize]{text}{../src/default/default.out}
\end{column}
\begin{column}{0.5\textwidth}
\centering
\only<1>{\providecommand\step{0}\input{diagrams/default/default.tex}}%
\only<2>{\providecommand\step{4}\input{diagrams/default/default.tex}}%
\end{column}
\end{columns}
\end{frame}


\begin{frame}{Startup}{How did we get there by the way?}
\begin{columns}
\begin{column}{0.45\textwidth}
\begin{itemize}
\item The entry point address of the binary is \funcname{main} (\funcname{\_start}) from the runtime
\item The program starts like a C program:
\begin{itemize}
\item \funcname{caml\_start\_program}
\item \funcname{caml\_startup\_common}
\item \funcname{caml\_startup\_exn}
\item \funcname{caml\_startup}
\item \funcname{caml\_main}
\item \funcname{main}
\end{itemize}
%\item \funcname{caml\_startup\_common}: initializes GC, domains \dots
% Also: codefrag, locale, custom opeartions, frame_descr, signals, stack
\item \funcname{caml\_start\_program} is the transition point between the C realm and the OCaml one
\end{itemize}
\bigskip
\listocaml{../src/default/default.ml}{default/default.ml}
\end{column}
\begin{column}{0.55\textwidth}
\listasm[lastline=24,highlightlines={22-24}]{src/ocaml/caml\_start\_program.s}{runtime/amd64.S:604}
\end{column}
\end{columns}
\end{frame}


\begin{frame}{Setup to jump into OCaml entry point}{\funcname{caml\_start\_program} initializes the main fiber}
\begin{columns}
\begin{column}{0.6\textwidth}
\only<1,3,5>{
\begin{itemize}
\item<1-> Stores information to allow switching from OCaml stack to the C stack
\item<3-> Constructs the default exception handler
\item<5-> Switches to the OCaml stack and calls \funcname{caml\_program}
\end{itemize}
\bigskip
\listocaml{../src/default/default.ml}{default/default.ml}
}%
\only<2>{
\listasm[firstline=22,lastline=41,highlightlines={25-27}]{src/ocaml/caml\_start\_program.s}{caml\_start\_program}
}%
\only<4>{
\mintasm[firstline=22,lastline=41,highlightlines={31-38}]{src/ocaml/caml\_start\_program.s}
\listasm[firstline=66,lastline=72]{src/ocaml/caml\_start\_program.s}{caml\_start\_program}
}%
\only<6>{
\listasm[firstline=22,lastline=41,highlightlines={39-41}]{src/ocaml/caml\_start\_program.s}{caml\_start\_program}
}%
\end{column}
\begin{column}{0.4\textwidth}
\centering
\only<1-2>{\providecommand\step{2}\input{diagrams/default/default.tex}}%
\only<3-5>{\providecommand\step{3}\input{diagrams/default/default.tex}}%
\onslide*<6->{\providecommand\step{4}\input{diagrams/default/default.tex}}%
\end{column}
\end{columns}
\end{frame}

\begin{frame}{Executing the OCaml program}{And raising an exception without handler}
\begin{columns}
\begin{column}{0.6\textwidth}
\only<1-3>{
\begin{itemize}
\item<1-> Execution continues with OCaml code
\begin{itemize}
\item \funcname{camlDefault\_\_main\_269}
\item \funcname{camlDefault\_\_entry}
\item \funcname{caml\_program}
\end{itemize}
\item<1-> \funcname{camlDefault\_\_main\_269} raises a \typename{Failure} exception
\item<1-> \typename{Failure} is caught by \funcname{caml\_start\_program}
\begin{itemize}
\item<1-> The handler set the 2nd LSB of the exception value to \funcarg{1}
\item<3-> And then forces \funcname{caml\_start\_program} to terminate
\end{itemize}
\end{itemize}
}
\bigskip
\only<1,3>{\listocaml[highlightlines=3]{../src/default/default.ml}{default/default.ml}}%
\only<2>{\listasm[firstline=66,lastline=72,highlightlines=69]{src/ocaml/caml\_start\_program.s}{caml\_start\_program}}%
\end{column}
\begin{column}{0.4\textwidth}
\centering
\providecommand\step{4}\input{diagrams/default/default.tex}
\end{column}
\end{columns}
\end{frame}
%\only<>{\listasm[firstline=47,lastline=65]{src/ocaml/caml\_start\_program.s}{caml\_start\_program}}


\begin{frame}{Back to C}{Clean up, complain and terminate}
\begin{columns}
\begin{column}{0.45\textwidth}
\begin{itemize}
\item Backtrace:
\begin{itemize}
\item \funcname{caml\_start\_program} $\rightarrow$ OCaml code
\item \funcname{caml\_startup\_common}
\item \funcname{caml\_startup\_exn}
\item \funcname{caml\_startup}
\item \funcname{caml\_main}
\item \funcname{main}
\end{itemize}
\smallskip
\item \funcname{caml\_startup} checks the return value of \funcname{caml\_startup\_exn}
\begin{itemize}
\item The return value has been specially marked by \funcname{caml\_start\_program} handler
\item Calls \funcname{caml\_fatal\_uncaught\_exception}
\end{itemize}
\item<2-> \funcname{caml\_fatal\_uncaught\_exception} either
\begin{itemize}
\item<2-> Calls the user custom uncaught exception handler, if set with \mintinline{ocaml}{Printexc.set_uncaught_exception_handler fn}\footnotemark
\item<2-> Or falls back to the default one
\item<2-> Which notifies about the uncaught exception
\end{itemize}
\item<4-> Terminates the program either with \funcname{abort} or with a non-zero exit code
\end{itemize}
\end{column}
\begin{column}{0.55\textwidth}
\only<1>{
\listc[firstline=84,lastline=89,highlightlines=88]{../ocaml/runtime/caml/mlvalues.h}{runtime/caml/mlvalues.h:84}
\listc[firstline=138,lastline=143,highlightlines={141-142}]{../ocaml/runtime/startup\_nat.c}{runtime/startup\_nat.c:138}
}%
\only<2>{
\listc[firstline=138,highlightlines={147,149}]{../ocaml/runtime/printexc.c}{runtime/printexc.c:138}
}%
\only<3>{
\listc[firstline=110,lastline=134,highlightlines=129]{../ocaml/runtime/printexc.c}{runtime/printexc.c:138}
}%
\only<4>{
\listc[firstline=138,highlightlines={152,154}]{../ocaml/runtime/printexc.c}{runtime/printexc.c:138}
}%
\end{column}
\end{columns}
\footnotetext[1]{\url{https://v2.ocaml.org/api/Printexc.html\#1\_Uncaughtexceptions}}
\end{frame}
}%
\onslide*<6->{\providecommand\step{4}%
% Runtime's default exception handler
%
\subsection*{Runtime's default exception handler}
\frameSubsection{pictures/The\_Architect.png}{}


\begin{frame}{Default exception handler}
\begin{columns}
\begin{column}{0.5\textwidth}
\begin{itemize}
\item \localname{domain\_state->exn\_handler} points to a trap located before \funcname{entry}!
\item What installed this very first trap there?
\end{itemize}
\bigskip
\listocaml{../src/default/default.ml}{default/default.ml}
\bigskip
\inputminted[fontsize=\footnotesize]{text}{../src/default/default.out}
\end{column}
\begin{column}{0.5\textwidth}
\centering
\only<1>{\providecommand\step{0}\input{diagrams/default/default.tex}}%
\only<2>{\providecommand\step{4}\input{diagrams/default/default.tex}}%
\end{column}
\end{columns}
\end{frame}


\begin{frame}{Startup}{How did we get there by the way?}
\begin{columns}
\begin{column}{0.45\textwidth}
\begin{itemize}
\item The entry point address of the binary is \funcname{main} (\funcname{\_start}) from the runtime
\item The program starts like a C program:
\begin{itemize}
\item \funcname{caml\_start\_program}
\item \funcname{caml\_startup\_common}
\item \funcname{caml\_startup\_exn}
\item \funcname{caml\_startup}
\item \funcname{caml\_main}
\item \funcname{main}
\end{itemize}
%\item \funcname{caml\_startup\_common}: initializes GC, domains \dots
% Also: codefrag, locale, custom opeartions, frame_descr, signals, stack
\item \funcname{caml\_start\_program} is the transition point between the C realm and the OCaml one
\end{itemize}
\bigskip
\listocaml{../src/default/default.ml}{default/default.ml}
\end{column}
\begin{column}{0.55\textwidth}
\listasm[lastline=24,highlightlines={22-24}]{src/ocaml/caml\_start\_program.s}{runtime/amd64.S:604}
\end{column}
\end{columns}
\end{frame}


\begin{frame}{Setup to jump into OCaml entry point}{\funcname{caml\_start\_program} initializes the main fiber}
\begin{columns}
\begin{column}{0.6\textwidth}
\only<1,3,5>{
\begin{itemize}
\item<1-> Stores information to allow switching from OCaml stack to the C stack
\item<3-> Constructs the default exception handler
\item<5-> Switches to the OCaml stack and calls \funcname{caml\_program}
\end{itemize}
\bigskip
\listocaml{../src/default/default.ml}{default/default.ml}
}%
\only<2>{
\listasm[firstline=22,lastline=41,highlightlines={25-27}]{src/ocaml/caml\_start\_program.s}{caml\_start\_program}
}%
\only<4>{
\mintasm[firstline=22,lastline=41,highlightlines={31-38}]{src/ocaml/caml\_start\_program.s}
\listasm[firstline=66,lastline=72]{src/ocaml/caml\_start\_program.s}{caml\_start\_program}
}%
\only<6>{
\listasm[firstline=22,lastline=41,highlightlines={39-41}]{src/ocaml/caml\_start\_program.s}{caml\_start\_program}
}%
\end{column}
\begin{column}{0.4\textwidth}
\centering
\only<1-2>{\providecommand\step{2}\input{diagrams/default/default.tex}}%
\only<3-5>{\providecommand\step{3}\input{diagrams/default/default.tex}}%
\onslide*<6->{\providecommand\step{4}\input{diagrams/default/default.tex}}%
\end{column}
\end{columns}
\end{frame}

\begin{frame}{Executing the OCaml program}{And raising an exception without handler}
\begin{columns}
\begin{column}{0.6\textwidth}
\only<1-3>{
\begin{itemize}
\item<1-> Execution continues with OCaml code
\begin{itemize}
\item \funcname{camlDefault\_\_main\_269}
\item \funcname{camlDefault\_\_entry}
\item \funcname{caml\_program}
\end{itemize}
\item<1-> \funcname{camlDefault\_\_main\_269} raises a \typename{Failure} exception
\item<1-> \typename{Failure} is caught by \funcname{caml\_start\_program}
\begin{itemize}
\item<1-> The handler set the 2nd LSB of the exception value to \funcarg{1}
\item<3-> And then forces \funcname{caml\_start\_program} to terminate
\end{itemize}
\end{itemize}
}
\bigskip
\only<1,3>{\listocaml[highlightlines=3]{../src/default/default.ml}{default/default.ml}}%
\only<2>{\listasm[firstline=66,lastline=72,highlightlines=69]{src/ocaml/caml\_start\_program.s}{caml\_start\_program}}%
\end{column}
\begin{column}{0.4\textwidth}
\centering
\providecommand\step{4}\input{diagrams/default/default.tex}
\end{column}
\end{columns}
\end{frame}
%\only<>{\listasm[firstline=47,lastline=65]{src/ocaml/caml\_start\_program.s}{caml\_start\_program}}


\begin{frame}{Back to C}{Clean up, complain and terminate}
\begin{columns}
\begin{column}{0.45\textwidth}
\begin{itemize}
\item Backtrace:
\begin{itemize}
\item \funcname{caml\_start\_program} $\rightarrow$ OCaml code
\item \funcname{caml\_startup\_common}
\item \funcname{caml\_startup\_exn}
\item \funcname{caml\_startup}
\item \funcname{caml\_main}
\item \funcname{main}
\end{itemize}
\smallskip
\item \funcname{caml\_startup} checks the return value of \funcname{caml\_startup\_exn}
\begin{itemize}
\item The return value has been specially marked by \funcname{caml\_start\_program} handler
\item Calls \funcname{caml\_fatal\_uncaught\_exception}
\end{itemize}
\item<2-> \funcname{caml\_fatal\_uncaught\_exception} either
\begin{itemize}
\item<2-> Calls the user custom uncaught exception handler, if set with \mintinline{ocaml}{Printexc.set_uncaught_exception_handler fn}\footnotemark
\item<2-> Or falls back to the default one
\item<2-> Which notifies about the uncaught exception
\end{itemize}
\item<4-> Terminates the program either with \funcname{abort} or with a non-zero exit code
\end{itemize}
\end{column}
\begin{column}{0.55\textwidth}
\only<1>{
\listc[firstline=84,lastline=89,highlightlines=88]{../ocaml/runtime/caml/mlvalues.h}{runtime/caml/mlvalues.h:84}
\listc[firstline=138,lastline=143,highlightlines={141-142}]{../ocaml/runtime/startup\_nat.c}{runtime/startup\_nat.c:138}
}%
\only<2>{
\listc[firstline=138,highlightlines={147,149}]{../ocaml/runtime/printexc.c}{runtime/printexc.c:138}
}%
\only<3>{
\listc[firstline=110,lastline=134,highlightlines=129]{../ocaml/runtime/printexc.c}{runtime/printexc.c:138}
}%
\only<4>{
\listc[firstline=138,highlightlines={152,154}]{../ocaml/runtime/printexc.c}{runtime/printexc.c:138}
}%
\end{column}
\end{columns}
\footnotetext[1]{\url{https://v2.ocaml.org/api/Printexc.html\#1\_Uncaughtexceptions}}
\end{frame}
}%
\end{column}
\end{columns}
\end{frame}

\begin{frame}{Executing the OCaml program}{And raising an exception without handler}
\begin{columns}
\begin{column}{0.6\textwidth}
\only<1-3>{
\begin{itemize}
\item<1-> Execution continues with OCaml code
\begin{itemize}
\item \funcname{camlDefault\_\_main\_269}
\item \funcname{camlDefault\_\_entry}
\item \funcname{caml\_program}
\end{itemize}
\item<1-> \funcname{camlDefault\_\_main\_269} raises a \typename{Failure} exception
\item<1-> \typename{Failure} is caught by \funcname{caml\_start\_program}
\begin{itemize}
\item<1-> The handler set the 2nd LSB of the exception value to \funcarg{1}
\item<3-> And then forces \funcname{caml\_start\_program} to terminate
\end{itemize}
\end{itemize}
}
\bigskip
\only<1,3>{\listocaml[highlightlines=3]{../src/default/default.ml}{default/default.ml}}%
\only<2>{\listasm[firstline=66,lastline=72,highlightlines=69]{src/ocaml/caml\_start\_program.s}{caml\_start\_program}}%
\end{column}
\begin{column}{0.4\textwidth}
\centering
\providecommand\step{4}%
% Runtime's default exception handler
%
\subsection*{Runtime's default exception handler}
\frameSubsection{pictures/The\_Architect.png}{}


\begin{frame}{Default exception handler}
\begin{columns}
\begin{column}{0.5\textwidth}
\begin{itemize}
\item \localname{domain\_state->exn\_handler} points to a trap located before \funcname{entry}!
\item What installed this very first trap there?
\end{itemize}
\bigskip
\listocaml{../src/default/default.ml}{default/default.ml}
\bigskip
\inputminted[fontsize=\footnotesize]{text}{../src/default/default.out}
\end{column}
\begin{column}{0.5\textwidth}
\centering
\only<1>{\providecommand\step{0}\input{diagrams/default/default.tex}}%
\only<2>{\providecommand\step{4}\input{diagrams/default/default.tex}}%
\end{column}
\end{columns}
\end{frame}


\begin{frame}{Startup}{How did we get there by the way?}
\begin{columns}
\begin{column}{0.45\textwidth}
\begin{itemize}
\item The entry point address of the binary is \funcname{main} (\funcname{\_start}) from the runtime
\item The program starts like a C program:
\begin{itemize}
\item \funcname{caml\_start\_program}
\item \funcname{caml\_startup\_common}
\item \funcname{caml\_startup\_exn}
\item \funcname{caml\_startup}
\item \funcname{caml\_main}
\item \funcname{main}
\end{itemize}
%\item \funcname{caml\_startup\_common}: initializes GC, domains \dots
% Also: codefrag, locale, custom opeartions, frame_descr, signals, stack
\item \funcname{caml\_start\_program} is the transition point between the C realm and the OCaml one
\end{itemize}
\bigskip
\listocaml{../src/default/default.ml}{default/default.ml}
\end{column}
\begin{column}{0.55\textwidth}
\listasm[lastline=24,highlightlines={22-24}]{src/ocaml/caml\_start\_program.s}{runtime/amd64.S:604}
\end{column}
\end{columns}
\end{frame}


\begin{frame}{Setup to jump into OCaml entry point}{\funcname{caml\_start\_program} initializes the main fiber}
\begin{columns}
\begin{column}{0.6\textwidth}
\only<1,3,5>{
\begin{itemize}
\item<1-> Stores information to allow switching from OCaml stack to the C stack
\item<3-> Constructs the default exception handler
\item<5-> Switches to the OCaml stack and calls \funcname{caml\_program}
\end{itemize}
\bigskip
\listocaml{../src/default/default.ml}{default/default.ml}
}%
\only<2>{
\listasm[firstline=22,lastline=41,highlightlines={25-27}]{src/ocaml/caml\_start\_program.s}{caml\_start\_program}
}%
\only<4>{
\mintasm[firstline=22,lastline=41,highlightlines={31-38}]{src/ocaml/caml\_start\_program.s}
\listasm[firstline=66,lastline=72]{src/ocaml/caml\_start\_program.s}{caml\_start\_program}
}%
\only<6>{
\listasm[firstline=22,lastline=41,highlightlines={39-41}]{src/ocaml/caml\_start\_program.s}{caml\_start\_program}
}%
\end{column}
\begin{column}{0.4\textwidth}
\centering
\only<1-2>{\providecommand\step{2}\input{diagrams/default/default.tex}}%
\only<3-5>{\providecommand\step{3}\input{diagrams/default/default.tex}}%
\onslide*<6->{\providecommand\step{4}\input{diagrams/default/default.tex}}%
\end{column}
\end{columns}
\end{frame}

\begin{frame}{Executing the OCaml program}{And raising an exception without handler}
\begin{columns}
\begin{column}{0.6\textwidth}
\only<1-3>{
\begin{itemize}
\item<1-> Execution continues with OCaml code
\begin{itemize}
\item \funcname{camlDefault\_\_main\_269}
\item \funcname{camlDefault\_\_entry}
\item \funcname{caml\_program}
\end{itemize}
\item<1-> \funcname{camlDefault\_\_main\_269} raises a \typename{Failure} exception
\item<1-> \typename{Failure} is caught by \funcname{caml\_start\_program}
\begin{itemize}
\item<1-> The handler set the 2nd LSB of the exception value to \funcarg{1}
\item<3-> And then forces \funcname{caml\_start\_program} to terminate
\end{itemize}
\end{itemize}
}
\bigskip
\only<1,3>{\listocaml[highlightlines=3]{../src/default/default.ml}{default/default.ml}}%
\only<2>{\listasm[firstline=66,lastline=72,highlightlines=69]{src/ocaml/caml\_start\_program.s}{caml\_start\_program}}%
\end{column}
\begin{column}{0.4\textwidth}
\centering
\providecommand\step{4}\input{diagrams/default/default.tex}
\end{column}
\end{columns}
\end{frame}
%\only<>{\listasm[firstline=47,lastline=65]{src/ocaml/caml\_start\_program.s}{caml\_start\_program}}


\begin{frame}{Back to C}{Clean up, complain and terminate}
\begin{columns}
\begin{column}{0.45\textwidth}
\begin{itemize}
\item Backtrace:
\begin{itemize}
\item \funcname{caml\_start\_program} $\rightarrow$ OCaml code
\item \funcname{caml\_startup\_common}
\item \funcname{caml\_startup\_exn}
\item \funcname{caml\_startup}
\item \funcname{caml\_main}
\item \funcname{main}
\end{itemize}
\smallskip
\item \funcname{caml\_startup} checks the return value of \funcname{caml\_startup\_exn}
\begin{itemize}
\item The return value has been specially marked by \funcname{caml\_start\_program} handler
\item Calls \funcname{caml\_fatal\_uncaught\_exception}
\end{itemize}
\item<2-> \funcname{caml\_fatal\_uncaught\_exception} either
\begin{itemize}
\item<2-> Calls the user custom uncaught exception handler, if set with \mintinline{ocaml}{Printexc.set_uncaught_exception_handler fn}\footnotemark
\item<2-> Or falls back to the default one
\item<2-> Which notifies about the uncaught exception
\end{itemize}
\item<4-> Terminates the program either with \funcname{abort} or with a non-zero exit code
\end{itemize}
\end{column}
\begin{column}{0.55\textwidth}
\only<1>{
\listc[firstline=84,lastline=89,highlightlines=88]{../ocaml/runtime/caml/mlvalues.h}{runtime/caml/mlvalues.h:84}
\listc[firstline=138,lastline=143,highlightlines={141-142}]{../ocaml/runtime/startup\_nat.c}{runtime/startup\_nat.c:138}
}%
\only<2>{
\listc[firstline=138,highlightlines={147,149}]{../ocaml/runtime/printexc.c}{runtime/printexc.c:138}
}%
\only<3>{
\listc[firstline=110,lastline=134,highlightlines=129]{../ocaml/runtime/printexc.c}{runtime/printexc.c:138}
}%
\only<4>{
\listc[firstline=138,highlightlines={152,154}]{../ocaml/runtime/printexc.c}{runtime/printexc.c:138}
}%
\end{column}
\end{columns}
\footnotetext[1]{\url{https://v2.ocaml.org/api/Printexc.html\#1\_Uncaughtexceptions}}
\end{frame}

\end{column}
\end{columns}
\end{frame}
%\only<>{\listasm[firstline=47,lastline=65]{src/ocaml/caml\_start\_program.s}{caml\_start\_program}}


\begin{frame}{Back to C}{Clean up, complain and terminate}
\begin{columns}
\begin{column}{0.45\textwidth}
\begin{itemize}
\item Backtrace:
\begin{itemize}
\item \funcname{caml\_start\_program} $\rightarrow$ OCaml code
\item \funcname{caml\_startup\_common}
\item \funcname{caml\_startup\_exn}
\item \funcname{caml\_startup}
\item \funcname{caml\_main}
\item \funcname{main}
\end{itemize}
\smallskip
\item \funcname{caml\_startup} checks the return value of \funcname{caml\_startup\_exn}
\begin{itemize}
\item The return value has been specially marked by \funcname{caml\_start\_program} handler
\item Calls \funcname{caml\_fatal\_uncaught\_exception}
\end{itemize}
\item<2-> \funcname{caml\_fatal\_uncaught\_exception} either
\begin{itemize}
\item<2-> Calls the user custom uncaught exception handler, if set with \mintinline{ocaml}{Printexc.set_uncaught_exception_handler fn}\footnotemark
\item<2-> Or falls back to the default one
\item<2-> Which notifies about the uncaught exception
\end{itemize}
\item<4-> Terminates the program either with \funcname{abort} or with a non-zero exit code
\end{itemize}
\end{column}
\begin{column}{0.55\textwidth}
\only<1>{
\listc[firstline=84,lastline=89,highlightlines=88]{../ocaml/runtime/caml/mlvalues.h}{runtime/caml/mlvalues.h:84}
\listc[firstline=138,lastline=143,highlightlines={141-142}]{../ocaml/runtime/startup\_nat.c}{runtime/startup\_nat.c:138}
}%
\only<2>{
\listc[firstline=138,highlightlines={147,149}]{../ocaml/runtime/printexc.c}{runtime/printexc.c:138}
}%
\only<3>{
\listc[firstline=110,lastline=134,highlightlines=129]{../ocaml/runtime/printexc.c}{runtime/printexc.c:138}
}%
\only<4>{
\listc[firstline=138,highlightlines={152,154}]{../ocaml/runtime/printexc.c}{runtime/printexc.c:138}
}%
\end{column}
\end{columns}
\footnotetext[1]{\url{https://v2.ocaml.org/api/Printexc.html\#1\_Uncaughtexceptions}}
\end{frame}
}%
\only<3-5>{\providecommand\step{3}%
% Runtime's default exception handler
%
\subsection*{Runtime's default exception handler}
\frameSubsection{pictures/The\_Architect.png}{}


\begin{frame}{Default exception handler}
\begin{columns}
\begin{column}{0.5\textwidth}
\begin{itemize}
\item \localname{domain\_state->exn\_handler} points to a trap located before \funcname{entry}!
\item What installed this very first trap there?
\end{itemize}
\bigskip
\listocaml{../src/default/default.ml}{default/default.ml}
\bigskip
\inputminted[fontsize=\footnotesize]{text}{../src/default/default.out}
\end{column}
\begin{column}{0.5\textwidth}
\centering
\only<1>{\providecommand\step{0}%
% Runtime's default exception handler
%
\subsection*{Runtime's default exception handler}
\frameSubsection{pictures/The\_Architect.png}{}


\begin{frame}{Default exception handler}
\begin{columns}
\begin{column}{0.5\textwidth}
\begin{itemize}
\item \localname{domain\_state->exn\_handler} points to a trap located before \funcname{entry}!
\item What installed this very first trap there?
\end{itemize}
\bigskip
\listocaml{../src/default/default.ml}{default/default.ml}
\bigskip
\inputminted[fontsize=\footnotesize]{text}{../src/default/default.out}
\end{column}
\begin{column}{0.5\textwidth}
\centering
\only<1>{\providecommand\step{0}\input{diagrams/default/default.tex}}%
\only<2>{\providecommand\step{4}\input{diagrams/default/default.tex}}%
\end{column}
\end{columns}
\end{frame}


\begin{frame}{Startup}{How did we get there by the way?}
\begin{columns}
\begin{column}{0.45\textwidth}
\begin{itemize}
\item The entry point address of the binary is \funcname{main} (\funcname{\_start}) from the runtime
\item The program starts like a C program:
\begin{itemize}
\item \funcname{caml\_start\_program}
\item \funcname{caml\_startup\_common}
\item \funcname{caml\_startup\_exn}
\item \funcname{caml\_startup}
\item \funcname{caml\_main}
\item \funcname{main}
\end{itemize}
%\item \funcname{caml\_startup\_common}: initializes GC, domains \dots
% Also: codefrag, locale, custom opeartions, frame_descr, signals, stack
\item \funcname{caml\_start\_program} is the transition point between the C realm and the OCaml one
\end{itemize}
\bigskip
\listocaml{../src/default/default.ml}{default/default.ml}
\end{column}
\begin{column}{0.55\textwidth}
\listasm[lastline=24,highlightlines={22-24}]{src/ocaml/caml\_start\_program.s}{runtime/amd64.S:604}
\end{column}
\end{columns}
\end{frame}


\begin{frame}{Setup to jump into OCaml entry point}{\funcname{caml\_start\_program} initializes the main fiber}
\begin{columns}
\begin{column}{0.6\textwidth}
\only<1,3,5>{
\begin{itemize}
\item<1-> Stores information to allow switching from OCaml stack to the C stack
\item<3-> Constructs the default exception handler
\item<5-> Switches to the OCaml stack and calls \funcname{caml\_program}
\end{itemize}
\bigskip
\listocaml{../src/default/default.ml}{default/default.ml}
}%
\only<2>{
\listasm[firstline=22,lastline=41,highlightlines={25-27}]{src/ocaml/caml\_start\_program.s}{caml\_start\_program}
}%
\only<4>{
\mintasm[firstline=22,lastline=41,highlightlines={31-38}]{src/ocaml/caml\_start\_program.s}
\listasm[firstline=66,lastline=72]{src/ocaml/caml\_start\_program.s}{caml\_start\_program}
}%
\only<6>{
\listasm[firstline=22,lastline=41,highlightlines={39-41}]{src/ocaml/caml\_start\_program.s}{caml\_start\_program}
}%
\end{column}
\begin{column}{0.4\textwidth}
\centering
\only<1-2>{\providecommand\step{2}\input{diagrams/default/default.tex}}%
\only<3-5>{\providecommand\step{3}\input{diagrams/default/default.tex}}%
\onslide*<6->{\providecommand\step{4}\input{diagrams/default/default.tex}}%
\end{column}
\end{columns}
\end{frame}

\begin{frame}{Executing the OCaml program}{And raising an exception without handler}
\begin{columns}
\begin{column}{0.6\textwidth}
\only<1-3>{
\begin{itemize}
\item<1-> Execution continues with OCaml code
\begin{itemize}
\item \funcname{camlDefault\_\_main\_269}
\item \funcname{camlDefault\_\_entry}
\item \funcname{caml\_program}
\end{itemize}
\item<1-> \funcname{camlDefault\_\_main\_269} raises a \typename{Failure} exception
\item<1-> \typename{Failure} is caught by \funcname{caml\_start\_program}
\begin{itemize}
\item<1-> The handler set the 2nd LSB of the exception value to \funcarg{1}
\item<3-> And then forces \funcname{caml\_start\_program} to terminate
\end{itemize}
\end{itemize}
}
\bigskip
\only<1,3>{\listocaml[highlightlines=3]{../src/default/default.ml}{default/default.ml}}%
\only<2>{\listasm[firstline=66,lastline=72,highlightlines=69]{src/ocaml/caml\_start\_program.s}{caml\_start\_program}}%
\end{column}
\begin{column}{0.4\textwidth}
\centering
\providecommand\step{4}\input{diagrams/default/default.tex}
\end{column}
\end{columns}
\end{frame}
%\only<>{\listasm[firstline=47,lastline=65]{src/ocaml/caml\_start\_program.s}{caml\_start\_program}}


\begin{frame}{Back to C}{Clean up, complain and terminate}
\begin{columns}
\begin{column}{0.45\textwidth}
\begin{itemize}
\item Backtrace:
\begin{itemize}
\item \funcname{caml\_start\_program} $\rightarrow$ OCaml code
\item \funcname{caml\_startup\_common}
\item \funcname{caml\_startup\_exn}
\item \funcname{caml\_startup}
\item \funcname{caml\_main}
\item \funcname{main}
\end{itemize}
\smallskip
\item \funcname{caml\_startup} checks the return value of \funcname{caml\_startup\_exn}
\begin{itemize}
\item The return value has been specially marked by \funcname{caml\_start\_program} handler
\item Calls \funcname{caml\_fatal\_uncaught\_exception}
\end{itemize}
\item<2-> \funcname{caml\_fatal\_uncaught\_exception} either
\begin{itemize}
\item<2-> Calls the user custom uncaught exception handler, if set with \mintinline{ocaml}{Printexc.set_uncaught_exception_handler fn}\footnotemark
\item<2-> Or falls back to the default one
\item<2-> Which notifies about the uncaught exception
\end{itemize}
\item<4-> Terminates the program either with \funcname{abort} or with a non-zero exit code
\end{itemize}
\end{column}
\begin{column}{0.55\textwidth}
\only<1>{
\listc[firstline=84,lastline=89,highlightlines=88]{../ocaml/runtime/caml/mlvalues.h}{runtime/caml/mlvalues.h:84}
\listc[firstline=138,lastline=143,highlightlines={141-142}]{../ocaml/runtime/startup\_nat.c}{runtime/startup\_nat.c:138}
}%
\only<2>{
\listc[firstline=138,highlightlines={147,149}]{../ocaml/runtime/printexc.c}{runtime/printexc.c:138}
}%
\only<3>{
\listc[firstline=110,lastline=134,highlightlines=129]{../ocaml/runtime/printexc.c}{runtime/printexc.c:138}
}%
\only<4>{
\listc[firstline=138,highlightlines={152,154}]{../ocaml/runtime/printexc.c}{runtime/printexc.c:138}
}%
\end{column}
\end{columns}
\footnotetext[1]{\url{https://v2.ocaml.org/api/Printexc.html\#1\_Uncaughtexceptions}}
\end{frame}
}%
\only<2>{\providecommand\step{4}%
% Runtime's default exception handler
%
\subsection*{Runtime's default exception handler}
\frameSubsection{pictures/The\_Architect.png}{}


\begin{frame}{Default exception handler}
\begin{columns}
\begin{column}{0.5\textwidth}
\begin{itemize}
\item \localname{domain\_state->exn\_handler} points to a trap located before \funcname{entry}!
\item What installed this very first trap there?
\end{itemize}
\bigskip
\listocaml{../src/default/default.ml}{default/default.ml}
\bigskip
\inputminted[fontsize=\footnotesize]{text}{../src/default/default.out}
\end{column}
\begin{column}{0.5\textwidth}
\centering
\only<1>{\providecommand\step{0}\input{diagrams/default/default.tex}}%
\only<2>{\providecommand\step{4}\input{diagrams/default/default.tex}}%
\end{column}
\end{columns}
\end{frame}


\begin{frame}{Startup}{How did we get there by the way?}
\begin{columns}
\begin{column}{0.45\textwidth}
\begin{itemize}
\item The entry point address of the binary is \funcname{main} (\funcname{\_start}) from the runtime
\item The program starts like a C program:
\begin{itemize}
\item \funcname{caml\_start\_program}
\item \funcname{caml\_startup\_common}
\item \funcname{caml\_startup\_exn}
\item \funcname{caml\_startup}
\item \funcname{caml\_main}
\item \funcname{main}
\end{itemize}
%\item \funcname{caml\_startup\_common}: initializes GC, domains \dots
% Also: codefrag, locale, custom opeartions, frame_descr, signals, stack
\item \funcname{caml\_start\_program} is the transition point between the C realm and the OCaml one
\end{itemize}
\bigskip
\listocaml{../src/default/default.ml}{default/default.ml}
\end{column}
\begin{column}{0.55\textwidth}
\listasm[lastline=24,highlightlines={22-24}]{src/ocaml/caml\_start\_program.s}{runtime/amd64.S:604}
\end{column}
\end{columns}
\end{frame}


\begin{frame}{Setup to jump into OCaml entry point}{\funcname{caml\_start\_program} initializes the main fiber}
\begin{columns}
\begin{column}{0.6\textwidth}
\only<1,3,5>{
\begin{itemize}
\item<1-> Stores information to allow switching from OCaml stack to the C stack
\item<3-> Constructs the default exception handler
\item<5-> Switches to the OCaml stack and calls \funcname{caml\_program}
\end{itemize}
\bigskip
\listocaml{../src/default/default.ml}{default/default.ml}
}%
\only<2>{
\listasm[firstline=22,lastline=41,highlightlines={25-27}]{src/ocaml/caml\_start\_program.s}{caml\_start\_program}
}%
\only<4>{
\mintasm[firstline=22,lastline=41,highlightlines={31-38}]{src/ocaml/caml\_start\_program.s}
\listasm[firstline=66,lastline=72]{src/ocaml/caml\_start\_program.s}{caml\_start\_program}
}%
\only<6>{
\listasm[firstline=22,lastline=41,highlightlines={39-41}]{src/ocaml/caml\_start\_program.s}{caml\_start\_program}
}%
\end{column}
\begin{column}{0.4\textwidth}
\centering
\only<1-2>{\providecommand\step{2}\input{diagrams/default/default.tex}}%
\only<3-5>{\providecommand\step{3}\input{diagrams/default/default.tex}}%
\onslide*<6->{\providecommand\step{4}\input{diagrams/default/default.tex}}%
\end{column}
\end{columns}
\end{frame}

\begin{frame}{Executing the OCaml program}{And raising an exception without handler}
\begin{columns}
\begin{column}{0.6\textwidth}
\only<1-3>{
\begin{itemize}
\item<1-> Execution continues with OCaml code
\begin{itemize}
\item \funcname{camlDefault\_\_main\_269}
\item \funcname{camlDefault\_\_entry}
\item \funcname{caml\_program}
\end{itemize}
\item<1-> \funcname{camlDefault\_\_main\_269} raises a \typename{Failure} exception
\item<1-> \typename{Failure} is caught by \funcname{caml\_start\_program}
\begin{itemize}
\item<1-> The handler set the 2nd LSB of the exception value to \funcarg{1}
\item<3-> And then forces \funcname{caml\_start\_program} to terminate
\end{itemize}
\end{itemize}
}
\bigskip
\only<1,3>{\listocaml[highlightlines=3]{../src/default/default.ml}{default/default.ml}}%
\only<2>{\listasm[firstline=66,lastline=72,highlightlines=69]{src/ocaml/caml\_start\_program.s}{caml\_start\_program}}%
\end{column}
\begin{column}{0.4\textwidth}
\centering
\providecommand\step{4}\input{diagrams/default/default.tex}
\end{column}
\end{columns}
\end{frame}
%\only<>{\listasm[firstline=47,lastline=65]{src/ocaml/caml\_start\_program.s}{caml\_start\_program}}


\begin{frame}{Back to C}{Clean up, complain and terminate}
\begin{columns}
\begin{column}{0.45\textwidth}
\begin{itemize}
\item Backtrace:
\begin{itemize}
\item \funcname{caml\_start\_program} $\rightarrow$ OCaml code
\item \funcname{caml\_startup\_common}
\item \funcname{caml\_startup\_exn}
\item \funcname{caml\_startup}
\item \funcname{caml\_main}
\item \funcname{main}
\end{itemize}
\smallskip
\item \funcname{caml\_startup} checks the return value of \funcname{caml\_startup\_exn}
\begin{itemize}
\item The return value has been specially marked by \funcname{caml\_start\_program} handler
\item Calls \funcname{caml\_fatal\_uncaught\_exception}
\end{itemize}
\item<2-> \funcname{caml\_fatal\_uncaught\_exception} either
\begin{itemize}
\item<2-> Calls the user custom uncaught exception handler, if set with \mintinline{ocaml}{Printexc.set_uncaught_exception_handler fn}\footnotemark
\item<2-> Or falls back to the default one
\item<2-> Which notifies about the uncaught exception
\end{itemize}
\item<4-> Terminates the program either with \funcname{abort} or with a non-zero exit code
\end{itemize}
\end{column}
\begin{column}{0.55\textwidth}
\only<1>{
\listc[firstline=84,lastline=89,highlightlines=88]{../ocaml/runtime/caml/mlvalues.h}{runtime/caml/mlvalues.h:84}
\listc[firstline=138,lastline=143,highlightlines={141-142}]{../ocaml/runtime/startup\_nat.c}{runtime/startup\_nat.c:138}
}%
\only<2>{
\listc[firstline=138,highlightlines={147,149}]{../ocaml/runtime/printexc.c}{runtime/printexc.c:138}
}%
\only<3>{
\listc[firstline=110,lastline=134,highlightlines=129]{../ocaml/runtime/printexc.c}{runtime/printexc.c:138}
}%
\only<4>{
\listc[firstline=138,highlightlines={152,154}]{../ocaml/runtime/printexc.c}{runtime/printexc.c:138}
}%
\end{column}
\end{columns}
\footnotetext[1]{\url{https://v2.ocaml.org/api/Printexc.html\#1\_Uncaughtexceptions}}
\end{frame}
}%
\end{column}
\end{columns}
\end{frame}


\begin{frame}{Startup}{How did we get there by the way?}
\begin{columns}
\begin{column}{0.45\textwidth}
\begin{itemize}
\item The entry point address of the binary is \funcname{main} (\funcname{\_start}) from the runtime
\item The program starts like a C program:
\begin{itemize}
\item \funcname{caml\_start\_program}
\item \funcname{caml\_startup\_common}
\item \funcname{caml\_startup\_exn}
\item \funcname{caml\_startup}
\item \funcname{caml\_main}
\item \funcname{main}
\end{itemize}
%\item \funcname{caml\_startup\_common}: initializes GC, domains \dots
% Also: codefrag, locale, custom opeartions, frame_descr, signals, stack
\item \funcname{caml\_start\_program} is the transition point between the C realm and the OCaml one
\end{itemize}
\bigskip
\listocaml{../src/default/default.ml}{default/default.ml}
\end{column}
\begin{column}{0.55\textwidth}
\listasm[lastline=24,highlightlines={22-24}]{src/ocaml/caml\_start\_program.s}{runtime/amd64.S:604}
\end{column}
\end{columns}
\end{frame}


\begin{frame}{Setup to jump into OCaml entry point}{\funcname{caml\_start\_program} initializes the main fiber}
\begin{columns}
\begin{column}{0.6\textwidth}
\only<1,3,5>{
\begin{itemize}
\item<1-> Stores information to allow switching from OCaml stack to the C stack
\item<3-> Constructs the default exception handler
\item<5-> Switches to the OCaml stack and calls \funcname{caml\_program}
\end{itemize}
\bigskip
\listocaml{../src/default/default.ml}{default/default.ml}
}%
\only<2>{
\listasm[firstline=22,lastline=41,highlightlines={25-27}]{src/ocaml/caml\_start\_program.s}{caml\_start\_program}
}%
\only<4>{
\mintasm[firstline=22,lastline=41,highlightlines={31-38}]{src/ocaml/caml\_start\_program.s}
\listasm[firstline=66,lastline=72]{src/ocaml/caml\_start\_program.s}{caml\_start\_program}
}%
\only<6>{
\listasm[firstline=22,lastline=41,highlightlines={39-41}]{src/ocaml/caml\_start\_program.s}{caml\_start\_program}
}%
\end{column}
\begin{column}{0.4\textwidth}
\centering
\only<1-2>{\providecommand\step{2}%
% Runtime's default exception handler
%
\subsection*{Runtime's default exception handler}
\frameSubsection{pictures/The\_Architect.png}{}


\begin{frame}{Default exception handler}
\begin{columns}
\begin{column}{0.5\textwidth}
\begin{itemize}
\item \localname{domain\_state->exn\_handler} points to a trap located before \funcname{entry}!
\item What installed this very first trap there?
\end{itemize}
\bigskip
\listocaml{../src/default/default.ml}{default/default.ml}
\bigskip
\inputminted[fontsize=\footnotesize]{text}{../src/default/default.out}
\end{column}
\begin{column}{0.5\textwidth}
\centering
\only<1>{\providecommand\step{0}\input{diagrams/default/default.tex}}%
\only<2>{\providecommand\step{4}\input{diagrams/default/default.tex}}%
\end{column}
\end{columns}
\end{frame}


\begin{frame}{Startup}{How did we get there by the way?}
\begin{columns}
\begin{column}{0.45\textwidth}
\begin{itemize}
\item The entry point address of the binary is \funcname{main} (\funcname{\_start}) from the runtime
\item The program starts like a C program:
\begin{itemize}
\item \funcname{caml\_start\_program}
\item \funcname{caml\_startup\_common}
\item \funcname{caml\_startup\_exn}
\item \funcname{caml\_startup}
\item \funcname{caml\_main}
\item \funcname{main}
\end{itemize}
%\item \funcname{caml\_startup\_common}: initializes GC, domains \dots
% Also: codefrag, locale, custom opeartions, frame_descr, signals, stack
\item \funcname{caml\_start\_program} is the transition point between the C realm and the OCaml one
\end{itemize}
\bigskip
\listocaml{../src/default/default.ml}{default/default.ml}
\end{column}
\begin{column}{0.55\textwidth}
\listasm[lastline=24,highlightlines={22-24}]{src/ocaml/caml\_start\_program.s}{runtime/amd64.S:604}
\end{column}
\end{columns}
\end{frame}


\begin{frame}{Setup to jump into OCaml entry point}{\funcname{caml\_start\_program} initializes the main fiber}
\begin{columns}
\begin{column}{0.6\textwidth}
\only<1,3,5>{
\begin{itemize}
\item<1-> Stores information to allow switching from OCaml stack to the C stack
\item<3-> Constructs the default exception handler
\item<5-> Switches to the OCaml stack and calls \funcname{caml\_program}
\end{itemize}
\bigskip
\listocaml{../src/default/default.ml}{default/default.ml}
}%
\only<2>{
\listasm[firstline=22,lastline=41,highlightlines={25-27}]{src/ocaml/caml\_start\_program.s}{caml\_start\_program}
}%
\only<4>{
\mintasm[firstline=22,lastline=41,highlightlines={31-38}]{src/ocaml/caml\_start\_program.s}
\listasm[firstline=66,lastline=72]{src/ocaml/caml\_start\_program.s}{caml\_start\_program}
}%
\only<6>{
\listasm[firstline=22,lastline=41,highlightlines={39-41}]{src/ocaml/caml\_start\_program.s}{caml\_start\_program}
}%
\end{column}
\begin{column}{0.4\textwidth}
\centering
\only<1-2>{\providecommand\step{2}\input{diagrams/default/default.tex}}%
\only<3-5>{\providecommand\step{3}\input{diagrams/default/default.tex}}%
\onslide*<6->{\providecommand\step{4}\input{diagrams/default/default.tex}}%
\end{column}
\end{columns}
\end{frame}

\begin{frame}{Executing the OCaml program}{And raising an exception without handler}
\begin{columns}
\begin{column}{0.6\textwidth}
\only<1-3>{
\begin{itemize}
\item<1-> Execution continues with OCaml code
\begin{itemize}
\item \funcname{camlDefault\_\_main\_269}
\item \funcname{camlDefault\_\_entry}
\item \funcname{caml\_program}
\end{itemize}
\item<1-> \funcname{camlDefault\_\_main\_269} raises a \typename{Failure} exception
\item<1-> \typename{Failure} is caught by \funcname{caml\_start\_program}
\begin{itemize}
\item<1-> The handler set the 2nd LSB of the exception value to \funcarg{1}
\item<3-> And then forces \funcname{caml\_start\_program} to terminate
\end{itemize}
\end{itemize}
}
\bigskip
\only<1,3>{\listocaml[highlightlines=3]{../src/default/default.ml}{default/default.ml}}%
\only<2>{\listasm[firstline=66,lastline=72,highlightlines=69]{src/ocaml/caml\_start\_program.s}{caml\_start\_program}}%
\end{column}
\begin{column}{0.4\textwidth}
\centering
\providecommand\step{4}\input{diagrams/default/default.tex}
\end{column}
\end{columns}
\end{frame}
%\only<>{\listasm[firstline=47,lastline=65]{src/ocaml/caml\_start\_program.s}{caml\_start\_program}}


\begin{frame}{Back to C}{Clean up, complain and terminate}
\begin{columns}
\begin{column}{0.45\textwidth}
\begin{itemize}
\item Backtrace:
\begin{itemize}
\item \funcname{caml\_start\_program} $\rightarrow$ OCaml code
\item \funcname{caml\_startup\_common}
\item \funcname{caml\_startup\_exn}
\item \funcname{caml\_startup}
\item \funcname{caml\_main}
\item \funcname{main}
\end{itemize}
\smallskip
\item \funcname{caml\_startup} checks the return value of \funcname{caml\_startup\_exn}
\begin{itemize}
\item The return value has been specially marked by \funcname{caml\_start\_program} handler
\item Calls \funcname{caml\_fatal\_uncaught\_exception}
\end{itemize}
\item<2-> \funcname{caml\_fatal\_uncaught\_exception} either
\begin{itemize}
\item<2-> Calls the user custom uncaught exception handler, if set with \mintinline{ocaml}{Printexc.set_uncaught_exception_handler fn}\footnotemark
\item<2-> Or falls back to the default one
\item<2-> Which notifies about the uncaught exception
\end{itemize}
\item<4-> Terminates the program either with \funcname{abort} or with a non-zero exit code
\end{itemize}
\end{column}
\begin{column}{0.55\textwidth}
\only<1>{
\listc[firstline=84,lastline=89,highlightlines=88]{../ocaml/runtime/caml/mlvalues.h}{runtime/caml/mlvalues.h:84}
\listc[firstline=138,lastline=143,highlightlines={141-142}]{../ocaml/runtime/startup\_nat.c}{runtime/startup\_nat.c:138}
}%
\only<2>{
\listc[firstline=138,highlightlines={147,149}]{../ocaml/runtime/printexc.c}{runtime/printexc.c:138}
}%
\only<3>{
\listc[firstline=110,lastline=134,highlightlines=129]{../ocaml/runtime/printexc.c}{runtime/printexc.c:138}
}%
\only<4>{
\listc[firstline=138,highlightlines={152,154}]{../ocaml/runtime/printexc.c}{runtime/printexc.c:138}
}%
\end{column}
\end{columns}
\footnotetext[1]{\url{https://v2.ocaml.org/api/Printexc.html\#1\_Uncaughtexceptions}}
\end{frame}
}%
\only<3-5>{\providecommand\step{3}%
% Runtime's default exception handler
%
\subsection*{Runtime's default exception handler}
\frameSubsection{pictures/The\_Architect.png}{}


\begin{frame}{Default exception handler}
\begin{columns}
\begin{column}{0.5\textwidth}
\begin{itemize}
\item \localname{domain\_state->exn\_handler} points to a trap located before \funcname{entry}!
\item What installed this very first trap there?
\end{itemize}
\bigskip
\listocaml{../src/default/default.ml}{default/default.ml}
\bigskip
\inputminted[fontsize=\footnotesize]{text}{../src/default/default.out}
\end{column}
\begin{column}{0.5\textwidth}
\centering
\only<1>{\providecommand\step{0}\input{diagrams/default/default.tex}}%
\only<2>{\providecommand\step{4}\input{diagrams/default/default.tex}}%
\end{column}
\end{columns}
\end{frame}


\begin{frame}{Startup}{How did we get there by the way?}
\begin{columns}
\begin{column}{0.45\textwidth}
\begin{itemize}
\item The entry point address of the binary is \funcname{main} (\funcname{\_start}) from the runtime
\item The program starts like a C program:
\begin{itemize}
\item \funcname{caml\_start\_program}
\item \funcname{caml\_startup\_common}
\item \funcname{caml\_startup\_exn}
\item \funcname{caml\_startup}
\item \funcname{caml\_main}
\item \funcname{main}
\end{itemize}
%\item \funcname{caml\_startup\_common}: initializes GC, domains \dots
% Also: codefrag, locale, custom opeartions, frame_descr, signals, stack
\item \funcname{caml\_start\_program} is the transition point between the C realm and the OCaml one
\end{itemize}
\bigskip
\listocaml{../src/default/default.ml}{default/default.ml}
\end{column}
\begin{column}{0.55\textwidth}
\listasm[lastline=24,highlightlines={22-24}]{src/ocaml/caml\_start\_program.s}{runtime/amd64.S:604}
\end{column}
\end{columns}
\end{frame}


\begin{frame}{Setup to jump into OCaml entry point}{\funcname{caml\_start\_program} initializes the main fiber}
\begin{columns}
\begin{column}{0.6\textwidth}
\only<1,3,5>{
\begin{itemize}
\item<1-> Stores information to allow switching from OCaml stack to the C stack
\item<3-> Constructs the default exception handler
\item<5-> Switches to the OCaml stack and calls \funcname{caml\_program}
\end{itemize}
\bigskip
\listocaml{../src/default/default.ml}{default/default.ml}
}%
\only<2>{
\listasm[firstline=22,lastline=41,highlightlines={25-27}]{src/ocaml/caml\_start\_program.s}{caml\_start\_program}
}%
\only<4>{
\mintasm[firstline=22,lastline=41,highlightlines={31-38}]{src/ocaml/caml\_start\_program.s}
\listasm[firstline=66,lastline=72]{src/ocaml/caml\_start\_program.s}{caml\_start\_program}
}%
\only<6>{
\listasm[firstline=22,lastline=41,highlightlines={39-41}]{src/ocaml/caml\_start\_program.s}{caml\_start\_program}
}%
\end{column}
\begin{column}{0.4\textwidth}
\centering
\only<1-2>{\providecommand\step{2}\input{diagrams/default/default.tex}}%
\only<3-5>{\providecommand\step{3}\input{diagrams/default/default.tex}}%
\onslide*<6->{\providecommand\step{4}\input{diagrams/default/default.tex}}%
\end{column}
\end{columns}
\end{frame}

\begin{frame}{Executing the OCaml program}{And raising an exception without handler}
\begin{columns}
\begin{column}{0.6\textwidth}
\only<1-3>{
\begin{itemize}
\item<1-> Execution continues with OCaml code
\begin{itemize}
\item \funcname{camlDefault\_\_main\_269}
\item \funcname{camlDefault\_\_entry}
\item \funcname{caml\_program}
\end{itemize}
\item<1-> \funcname{camlDefault\_\_main\_269} raises a \typename{Failure} exception
\item<1-> \typename{Failure} is caught by \funcname{caml\_start\_program}
\begin{itemize}
\item<1-> The handler set the 2nd LSB of the exception value to \funcarg{1}
\item<3-> And then forces \funcname{caml\_start\_program} to terminate
\end{itemize}
\end{itemize}
}
\bigskip
\only<1,3>{\listocaml[highlightlines=3]{../src/default/default.ml}{default/default.ml}}%
\only<2>{\listasm[firstline=66,lastline=72,highlightlines=69]{src/ocaml/caml\_start\_program.s}{caml\_start\_program}}%
\end{column}
\begin{column}{0.4\textwidth}
\centering
\providecommand\step{4}\input{diagrams/default/default.tex}
\end{column}
\end{columns}
\end{frame}
%\only<>{\listasm[firstline=47,lastline=65]{src/ocaml/caml\_start\_program.s}{caml\_start\_program}}


\begin{frame}{Back to C}{Clean up, complain and terminate}
\begin{columns}
\begin{column}{0.45\textwidth}
\begin{itemize}
\item Backtrace:
\begin{itemize}
\item \funcname{caml\_start\_program} $\rightarrow$ OCaml code
\item \funcname{caml\_startup\_common}
\item \funcname{caml\_startup\_exn}
\item \funcname{caml\_startup}
\item \funcname{caml\_main}
\item \funcname{main}
\end{itemize}
\smallskip
\item \funcname{caml\_startup} checks the return value of \funcname{caml\_startup\_exn}
\begin{itemize}
\item The return value has been specially marked by \funcname{caml\_start\_program} handler
\item Calls \funcname{caml\_fatal\_uncaught\_exception}
\end{itemize}
\item<2-> \funcname{caml\_fatal\_uncaught\_exception} either
\begin{itemize}
\item<2-> Calls the user custom uncaught exception handler, if set with \mintinline{ocaml}{Printexc.set_uncaught_exception_handler fn}\footnotemark
\item<2-> Or falls back to the default one
\item<2-> Which notifies about the uncaught exception
\end{itemize}
\item<4-> Terminates the program either with \funcname{abort} or with a non-zero exit code
\end{itemize}
\end{column}
\begin{column}{0.55\textwidth}
\only<1>{
\listc[firstline=84,lastline=89,highlightlines=88]{../ocaml/runtime/caml/mlvalues.h}{runtime/caml/mlvalues.h:84}
\listc[firstline=138,lastline=143,highlightlines={141-142}]{../ocaml/runtime/startup\_nat.c}{runtime/startup\_nat.c:138}
}%
\only<2>{
\listc[firstline=138,highlightlines={147,149}]{../ocaml/runtime/printexc.c}{runtime/printexc.c:138}
}%
\only<3>{
\listc[firstline=110,lastline=134,highlightlines=129]{../ocaml/runtime/printexc.c}{runtime/printexc.c:138}
}%
\only<4>{
\listc[firstline=138,highlightlines={152,154}]{../ocaml/runtime/printexc.c}{runtime/printexc.c:138}
}%
\end{column}
\end{columns}
\footnotetext[1]{\url{https://v2.ocaml.org/api/Printexc.html\#1\_Uncaughtexceptions}}
\end{frame}
}%
\onslide*<6->{\providecommand\step{4}%
% Runtime's default exception handler
%
\subsection*{Runtime's default exception handler}
\frameSubsection{pictures/The\_Architect.png}{}


\begin{frame}{Default exception handler}
\begin{columns}
\begin{column}{0.5\textwidth}
\begin{itemize}
\item \localname{domain\_state->exn\_handler} points to a trap located before \funcname{entry}!
\item What installed this very first trap there?
\end{itemize}
\bigskip
\listocaml{../src/default/default.ml}{default/default.ml}
\bigskip
\inputminted[fontsize=\footnotesize]{text}{../src/default/default.out}
\end{column}
\begin{column}{0.5\textwidth}
\centering
\only<1>{\providecommand\step{0}\input{diagrams/default/default.tex}}%
\only<2>{\providecommand\step{4}\input{diagrams/default/default.tex}}%
\end{column}
\end{columns}
\end{frame}


\begin{frame}{Startup}{How did we get there by the way?}
\begin{columns}
\begin{column}{0.45\textwidth}
\begin{itemize}
\item The entry point address of the binary is \funcname{main} (\funcname{\_start}) from the runtime
\item The program starts like a C program:
\begin{itemize}
\item \funcname{caml\_start\_program}
\item \funcname{caml\_startup\_common}
\item \funcname{caml\_startup\_exn}
\item \funcname{caml\_startup}
\item \funcname{caml\_main}
\item \funcname{main}
\end{itemize}
%\item \funcname{caml\_startup\_common}: initializes GC, domains \dots
% Also: codefrag, locale, custom opeartions, frame_descr, signals, stack
\item \funcname{caml\_start\_program} is the transition point between the C realm and the OCaml one
\end{itemize}
\bigskip
\listocaml{../src/default/default.ml}{default/default.ml}
\end{column}
\begin{column}{0.55\textwidth}
\listasm[lastline=24,highlightlines={22-24}]{src/ocaml/caml\_start\_program.s}{runtime/amd64.S:604}
\end{column}
\end{columns}
\end{frame}


\begin{frame}{Setup to jump into OCaml entry point}{\funcname{caml\_start\_program} initializes the main fiber}
\begin{columns}
\begin{column}{0.6\textwidth}
\only<1,3,5>{
\begin{itemize}
\item<1-> Stores information to allow switching from OCaml stack to the C stack
\item<3-> Constructs the default exception handler
\item<5-> Switches to the OCaml stack and calls \funcname{caml\_program}
\end{itemize}
\bigskip
\listocaml{../src/default/default.ml}{default/default.ml}
}%
\only<2>{
\listasm[firstline=22,lastline=41,highlightlines={25-27}]{src/ocaml/caml\_start\_program.s}{caml\_start\_program}
}%
\only<4>{
\mintasm[firstline=22,lastline=41,highlightlines={31-38}]{src/ocaml/caml\_start\_program.s}
\listasm[firstline=66,lastline=72]{src/ocaml/caml\_start\_program.s}{caml\_start\_program}
}%
\only<6>{
\listasm[firstline=22,lastline=41,highlightlines={39-41}]{src/ocaml/caml\_start\_program.s}{caml\_start\_program}
}%
\end{column}
\begin{column}{0.4\textwidth}
\centering
\only<1-2>{\providecommand\step{2}\input{diagrams/default/default.tex}}%
\only<3-5>{\providecommand\step{3}\input{diagrams/default/default.tex}}%
\onslide*<6->{\providecommand\step{4}\input{diagrams/default/default.tex}}%
\end{column}
\end{columns}
\end{frame}

\begin{frame}{Executing the OCaml program}{And raising an exception without handler}
\begin{columns}
\begin{column}{0.6\textwidth}
\only<1-3>{
\begin{itemize}
\item<1-> Execution continues with OCaml code
\begin{itemize}
\item \funcname{camlDefault\_\_main\_269}
\item \funcname{camlDefault\_\_entry}
\item \funcname{caml\_program}
\end{itemize}
\item<1-> \funcname{camlDefault\_\_main\_269} raises a \typename{Failure} exception
\item<1-> \typename{Failure} is caught by \funcname{caml\_start\_program}
\begin{itemize}
\item<1-> The handler set the 2nd LSB of the exception value to \funcarg{1}
\item<3-> And then forces \funcname{caml\_start\_program} to terminate
\end{itemize}
\end{itemize}
}
\bigskip
\only<1,3>{\listocaml[highlightlines=3]{../src/default/default.ml}{default/default.ml}}%
\only<2>{\listasm[firstline=66,lastline=72,highlightlines=69]{src/ocaml/caml\_start\_program.s}{caml\_start\_program}}%
\end{column}
\begin{column}{0.4\textwidth}
\centering
\providecommand\step{4}\input{diagrams/default/default.tex}
\end{column}
\end{columns}
\end{frame}
%\only<>{\listasm[firstline=47,lastline=65]{src/ocaml/caml\_start\_program.s}{caml\_start\_program}}


\begin{frame}{Back to C}{Clean up, complain and terminate}
\begin{columns}
\begin{column}{0.45\textwidth}
\begin{itemize}
\item Backtrace:
\begin{itemize}
\item \funcname{caml\_start\_program} $\rightarrow$ OCaml code
\item \funcname{caml\_startup\_common}
\item \funcname{caml\_startup\_exn}
\item \funcname{caml\_startup}
\item \funcname{caml\_main}
\item \funcname{main}
\end{itemize}
\smallskip
\item \funcname{caml\_startup} checks the return value of \funcname{caml\_startup\_exn}
\begin{itemize}
\item The return value has been specially marked by \funcname{caml\_start\_program} handler
\item Calls \funcname{caml\_fatal\_uncaught\_exception}
\end{itemize}
\item<2-> \funcname{caml\_fatal\_uncaught\_exception} either
\begin{itemize}
\item<2-> Calls the user custom uncaught exception handler, if set with \mintinline{ocaml}{Printexc.set_uncaught_exception_handler fn}\footnotemark
\item<2-> Or falls back to the default one
\item<2-> Which notifies about the uncaught exception
\end{itemize}
\item<4-> Terminates the program either with \funcname{abort} or with a non-zero exit code
\end{itemize}
\end{column}
\begin{column}{0.55\textwidth}
\only<1>{
\listc[firstline=84,lastline=89,highlightlines=88]{../ocaml/runtime/caml/mlvalues.h}{runtime/caml/mlvalues.h:84}
\listc[firstline=138,lastline=143,highlightlines={141-142}]{../ocaml/runtime/startup\_nat.c}{runtime/startup\_nat.c:138}
}%
\only<2>{
\listc[firstline=138,highlightlines={147,149}]{../ocaml/runtime/printexc.c}{runtime/printexc.c:138}
}%
\only<3>{
\listc[firstline=110,lastline=134,highlightlines=129]{../ocaml/runtime/printexc.c}{runtime/printexc.c:138}
}%
\only<4>{
\listc[firstline=138,highlightlines={152,154}]{../ocaml/runtime/printexc.c}{runtime/printexc.c:138}
}%
\end{column}
\end{columns}
\footnotetext[1]{\url{https://v2.ocaml.org/api/Printexc.html\#1\_Uncaughtexceptions}}
\end{frame}
}%
\end{column}
\end{columns}
\end{frame}

\begin{frame}{Executing the OCaml program}{And raising an exception without handler}
\begin{columns}
\begin{column}{0.6\textwidth}
\only<1-3>{
\begin{itemize}
\item<1-> Execution continues with OCaml code
\begin{itemize}
\item \funcname{camlDefault\_\_main\_269}
\item \funcname{camlDefault\_\_entry}
\item \funcname{caml\_program}
\end{itemize}
\item<1-> \funcname{camlDefault\_\_main\_269} raises a \typename{Failure} exception
\item<1-> \typename{Failure} is caught by \funcname{caml\_start\_program}
\begin{itemize}
\item<1-> The handler set the 2nd LSB of the exception value to \funcarg{1}
\item<3-> And then forces \funcname{caml\_start\_program} to terminate
\end{itemize}
\end{itemize}
}
\bigskip
\only<1,3>{\listocaml[highlightlines=3]{../src/default/default.ml}{default/default.ml}}%
\only<2>{\listasm[firstline=66,lastline=72,highlightlines=69]{src/ocaml/caml\_start\_program.s}{caml\_start\_program}}%
\end{column}
\begin{column}{0.4\textwidth}
\centering
\providecommand\step{4}%
% Runtime's default exception handler
%
\subsection*{Runtime's default exception handler}
\frameSubsection{pictures/The\_Architect.png}{}


\begin{frame}{Default exception handler}
\begin{columns}
\begin{column}{0.5\textwidth}
\begin{itemize}
\item \localname{domain\_state->exn\_handler} points to a trap located before \funcname{entry}!
\item What installed this very first trap there?
\end{itemize}
\bigskip
\listocaml{../src/default/default.ml}{default/default.ml}
\bigskip
\inputminted[fontsize=\footnotesize]{text}{../src/default/default.out}
\end{column}
\begin{column}{0.5\textwidth}
\centering
\only<1>{\providecommand\step{0}\input{diagrams/default/default.tex}}%
\only<2>{\providecommand\step{4}\input{diagrams/default/default.tex}}%
\end{column}
\end{columns}
\end{frame}


\begin{frame}{Startup}{How did we get there by the way?}
\begin{columns}
\begin{column}{0.45\textwidth}
\begin{itemize}
\item The entry point address of the binary is \funcname{main} (\funcname{\_start}) from the runtime
\item The program starts like a C program:
\begin{itemize}
\item \funcname{caml\_start\_program}
\item \funcname{caml\_startup\_common}
\item \funcname{caml\_startup\_exn}
\item \funcname{caml\_startup}
\item \funcname{caml\_main}
\item \funcname{main}
\end{itemize}
%\item \funcname{caml\_startup\_common}: initializes GC, domains \dots
% Also: codefrag, locale, custom opeartions, frame_descr, signals, stack
\item \funcname{caml\_start\_program} is the transition point between the C realm and the OCaml one
\end{itemize}
\bigskip
\listocaml{../src/default/default.ml}{default/default.ml}
\end{column}
\begin{column}{0.55\textwidth}
\listasm[lastline=24,highlightlines={22-24}]{src/ocaml/caml\_start\_program.s}{runtime/amd64.S:604}
\end{column}
\end{columns}
\end{frame}


\begin{frame}{Setup to jump into OCaml entry point}{\funcname{caml\_start\_program} initializes the main fiber}
\begin{columns}
\begin{column}{0.6\textwidth}
\only<1,3,5>{
\begin{itemize}
\item<1-> Stores information to allow switching from OCaml stack to the C stack
\item<3-> Constructs the default exception handler
\item<5-> Switches to the OCaml stack and calls \funcname{caml\_program}
\end{itemize}
\bigskip
\listocaml{../src/default/default.ml}{default/default.ml}
}%
\only<2>{
\listasm[firstline=22,lastline=41,highlightlines={25-27}]{src/ocaml/caml\_start\_program.s}{caml\_start\_program}
}%
\only<4>{
\mintasm[firstline=22,lastline=41,highlightlines={31-38}]{src/ocaml/caml\_start\_program.s}
\listasm[firstline=66,lastline=72]{src/ocaml/caml\_start\_program.s}{caml\_start\_program}
}%
\only<6>{
\listasm[firstline=22,lastline=41,highlightlines={39-41}]{src/ocaml/caml\_start\_program.s}{caml\_start\_program}
}%
\end{column}
\begin{column}{0.4\textwidth}
\centering
\only<1-2>{\providecommand\step{2}\input{diagrams/default/default.tex}}%
\only<3-5>{\providecommand\step{3}\input{diagrams/default/default.tex}}%
\onslide*<6->{\providecommand\step{4}\input{diagrams/default/default.tex}}%
\end{column}
\end{columns}
\end{frame}

\begin{frame}{Executing the OCaml program}{And raising an exception without handler}
\begin{columns}
\begin{column}{0.6\textwidth}
\only<1-3>{
\begin{itemize}
\item<1-> Execution continues with OCaml code
\begin{itemize}
\item \funcname{camlDefault\_\_main\_269}
\item \funcname{camlDefault\_\_entry}
\item \funcname{caml\_program}
\end{itemize}
\item<1-> \funcname{camlDefault\_\_main\_269} raises a \typename{Failure} exception
\item<1-> \typename{Failure} is caught by \funcname{caml\_start\_program}
\begin{itemize}
\item<1-> The handler set the 2nd LSB of the exception value to \funcarg{1}
\item<3-> And then forces \funcname{caml\_start\_program} to terminate
\end{itemize}
\end{itemize}
}
\bigskip
\only<1,3>{\listocaml[highlightlines=3]{../src/default/default.ml}{default/default.ml}}%
\only<2>{\listasm[firstline=66,lastline=72,highlightlines=69]{src/ocaml/caml\_start\_program.s}{caml\_start\_program}}%
\end{column}
\begin{column}{0.4\textwidth}
\centering
\providecommand\step{4}\input{diagrams/default/default.tex}
\end{column}
\end{columns}
\end{frame}
%\only<>{\listasm[firstline=47,lastline=65]{src/ocaml/caml\_start\_program.s}{caml\_start\_program}}


\begin{frame}{Back to C}{Clean up, complain and terminate}
\begin{columns}
\begin{column}{0.45\textwidth}
\begin{itemize}
\item Backtrace:
\begin{itemize}
\item \funcname{caml\_start\_program} $\rightarrow$ OCaml code
\item \funcname{caml\_startup\_common}
\item \funcname{caml\_startup\_exn}
\item \funcname{caml\_startup}
\item \funcname{caml\_main}
\item \funcname{main}
\end{itemize}
\smallskip
\item \funcname{caml\_startup} checks the return value of \funcname{caml\_startup\_exn}
\begin{itemize}
\item The return value has been specially marked by \funcname{caml\_start\_program} handler
\item Calls \funcname{caml\_fatal\_uncaught\_exception}
\end{itemize}
\item<2-> \funcname{caml\_fatal\_uncaught\_exception} either
\begin{itemize}
\item<2-> Calls the user custom uncaught exception handler, if set with \mintinline{ocaml}{Printexc.set_uncaught_exception_handler fn}\footnotemark
\item<2-> Or falls back to the default one
\item<2-> Which notifies about the uncaught exception
\end{itemize}
\item<4-> Terminates the program either with \funcname{abort} or with a non-zero exit code
\end{itemize}
\end{column}
\begin{column}{0.55\textwidth}
\only<1>{
\listc[firstline=84,lastline=89,highlightlines=88]{../ocaml/runtime/caml/mlvalues.h}{runtime/caml/mlvalues.h:84}
\listc[firstline=138,lastline=143,highlightlines={141-142}]{../ocaml/runtime/startup\_nat.c}{runtime/startup\_nat.c:138}
}%
\only<2>{
\listc[firstline=138,highlightlines={147,149}]{../ocaml/runtime/printexc.c}{runtime/printexc.c:138}
}%
\only<3>{
\listc[firstline=110,lastline=134,highlightlines=129]{../ocaml/runtime/printexc.c}{runtime/printexc.c:138}
}%
\only<4>{
\listc[firstline=138,highlightlines={152,154}]{../ocaml/runtime/printexc.c}{runtime/printexc.c:138}
}%
\end{column}
\end{columns}
\footnotetext[1]{\url{https://v2.ocaml.org/api/Printexc.html\#1\_Uncaughtexceptions}}
\end{frame}

\end{column}
\end{columns}
\end{frame}
%\only<>{\listasm[firstline=47,lastline=65]{src/ocaml/caml\_start\_program.s}{caml\_start\_program}}


\begin{frame}{Back to C}{Clean up, complain and terminate}
\begin{columns}
\begin{column}{0.45\textwidth}
\begin{itemize}
\item Backtrace:
\begin{itemize}
\item \funcname{caml\_start\_program} $\rightarrow$ OCaml code
\item \funcname{caml\_startup\_common}
\item \funcname{caml\_startup\_exn}
\item \funcname{caml\_startup}
\item \funcname{caml\_main}
\item \funcname{main}
\end{itemize}
\smallskip
\item \funcname{caml\_startup} checks the return value of \funcname{caml\_startup\_exn}
\begin{itemize}
\item The return value has been specially marked by \funcname{caml\_start\_program} handler
\item Calls \funcname{caml\_fatal\_uncaught\_exception}
\end{itemize}
\item<2-> \funcname{caml\_fatal\_uncaught\_exception} either
\begin{itemize}
\item<2-> Calls the user custom uncaught exception handler, if set with \mintinline{ocaml}{Printexc.set_uncaught_exception_handler fn}\footnotemark
\item<2-> Or falls back to the default one
\item<2-> Which notifies about the uncaught exception
\end{itemize}
\item<4-> Terminates the program either with \funcname{abort} or with a non-zero exit code
\end{itemize}
\end{column}
\begin{column}{0.55\textwidth}
\only<1>{
\listc[firstline=84,lastline=89,highlightlines=88]{../ocaml/runtime/caml/mlvalues.h}{runtime/caml/mlvalues.h:84}
\listc[firstline=138,lastline=143,highlightlines={141-142}]{../ocaml/runtime/startup\_nat.c}{runtime/startup\_nat.c:138}
}%
\only<2>{
\listc[firstline=138,highlightlines={147,149}]{../ocaml/runtime/printexc.c}{runtime/printexc.c:138}
}%
\only<3>{
\listc[firstline=110,lastline=134,highlightlines=129]{../ocaml/runtime/printexc.c}{runtime/printexc.c:138}
}%
\only<4>{
\listc[firstline=138,highlightlines={152,154}]{../ocaml/runtime/printexc.c}{runtime/printexc.c:138}
}%
\end{column}
\end{columns}
\footnotetext[1]{\url{https://v2.ocaml.org/api/Printexc.html\#1\_Uncaughtexceptions}}
\end{frame}
}%
\onslide*<6->{\providecommand\step{4}%
% Runtime's default exception handler
%
\subsection*{Runtime's default exception handler}
\frameSubsection{pictures/The\_Architect.png}{}


\begin{frame}{Default exception handler}
\begin{columns}
\begin{column}{0.5\textwidth}
\begin{itemize}
\item \localname{domain\_state->exn\_handler} points to a trap located before \funcname{entry}!
\item What installed this very first trap there?
\end{itemize}
\bigskip
\listocaml{../src/default/default.ml}{default/default.ml}
\bigskip
\inputminted[fontsize=\footnotesize]{text}{../src/default/default.out}
\end{column}
\begin{column}{0.5\textwidth}
\centering
\only<1>{\providecommand\step{0}%
% Runtime's default exception handler
%
\subsection*{Runtime's default exception handler}
\frameSubsection{pictures/The\_Architect.png}{}


\begin{frame}{Default exception handler}
\begin{columns}
\begin{column}{0.5\textwidth}
\begin{itemize}
\item \localname{domain\_state->exn\_handler} points to a trap located before \funcname{entry}!
\item What installed this very first trap there?
\end{itemize}
\bigskip
\listocaml{../src/default/default.ml}{default/default.ml}
\bigskip
\inputminted[fontsize=\footnotesize]{text}{../src/default/default.out}
\end{column}
\begin{column}{0.5\textwidth}
\centering
\only<1>{\providecommand\step{0}\input{diagrams/default/default.tex}}%
\only<2>{\providecommand\step{4}\input{diagrams/default/default.tex}}%
\end{column}
\end{columns}
\end{frame}


\begin{frame}{Startup}{How did we get there by the way?}
\begin{columns}
\begin{column}{0.45\textwidth}
\begin{itemize}
\item The entry point address of the binary is \funcname{main} (\funcname{\_start}) from the runtime
\item The program starts like a C program:
\begin{itemize}
\item \funcname{caml\_start\_program}
\item \funcname{caml\_startup\_common}
\item \funcname{caml\_startup\_exn}
\item \funcname{caml\_startup}
\item \funcname{caml\_main}
\item \funcname{main}
\end{itemize}
%\item \funcname{caml\_startup\_common}: initializes GC, domains \dots
% Also: codefrag, locale, custom opeartions, frame_descr, signals, stack
\item \funcname{caml\_start\_program} is the transition point between the C realm and the OCaml one
\end{itemize}
\bigskip
\listocaml{../src/default/default.ml}{default/default.ml}
\end{column}
\begin{column}{0.55\textwidth}
\listasm[lastline=24,highlightlines={22-24}]{src/ocaml/caml\_start\_program.s}{runtime/amd64.S:604}
\end{column}
\end{columns}
\end{frame}


\begin{frame}{Setup to jump into OCaml entry point}{\funcname{caml\_start\_program} initializes the main fiber}
\begin{columns}
\begin{column}{0.6\textwidth}
\only<1,3,5>{
\begin{itemize}
\item<1-> Stores information to allow switching from OCaml stack to the C stack
\item<3-> Constructs the default exception handler
\item<5-> Switches to the OCaml stack and calls \funcname{caml\_program}
\end{itemize}
\bigskip
\listocaml{../src/default/default.ml}{default/default.ml}
}%
\only<2>{
\listasm[firstline=22,lastline=41,highlightlines={25-27}]{src/ocaml/caml\_start\_program.s}{caml\_start\_program}
}%
\only<4>{
\mintasm[firstline=22,lastline=41,highlightlines={31-38}]{src/ocaml/caml\_start\_program.s}
\listasm[firstline=66,lastline=72]{src/ocaml/caml\_start\_program.s}{caml\_start\_program}
}%
\only<6>{
\listasm[firstline=22,lastline=41,highlightlines={39-41}]{src/ocaml/caml\_start\_program.s}{caml\_start\_program}
}%
\end{column}
\begin{column}{0.4\textwidth}
\centering
\only<1-2>{\providecommand\step{2}\input{diagrams/default/default.tex}}%
\only<3-5>{\providecommand\step{3}\input{diagrams/default/default.tex}}%
\onslide*<6->{\providecommand\step{4}\input{diagrams/default/default.tex}}%
\end{column}
\end{columns}
\end{frame}

\begin{frame}{Executing the OCaml program}{And raising an exception without handler}
\begin{columns}
\begin{column}{0.6\textwidth}
\only<1-3>{
\begin{itemize}
\item<1-> Execution continues with OCaml code
\begin{itemize}
\item \funcname{camlDefault\_\_main\_269}
\item \funcname{camlDefault\_\_entry}
\item \funcname{caml\_program}
\end{itemize}
\item<1-> \funcname{camlDefault\_\_main\_269} raises a \typename{Failure} exception
\item<1-> \typename{Failure} is caught by \funcname{caml\_start\_program}
\begin{itemize}
\item<1-> The handler set the 2nd LSB of the exception value to \funcarg{1}
\item<3-> And then forces \funcname{caml\_start\_program} to terminate
\end{itemize}
\end{itemize}
}
\bigskip
\only<1,3>{\listocaml[highlightlines=3]{../src/default/default.ml}{default/default.ml}}%
\only<2>{\listasm[firstline=66,lastline=72,highlightlines=69]{src/ocaml/caml\_start\_program.s}{caml\_start\_program}}%
\end{column}
\begin{column}{0.4\textwidth}
\centering
\providecommand\step{4}\input{diagrams/default/default.tex}
\end{column}
\end{columns}
\end{frame}
%\only<>{\listasm[firstline=47,lastline=65]{src/ocaml/caml\_start\_program.s}{caml\_start\_program}}


\begin{frame}{Back to C}{Clean up, complain and terminate}
\begin{columns}
\begin{column}{0.45\textwidth}
\begin{itemize}
\item Backtrace:
\begin{itemize}
\item \funcname{caml\_start\_program} $\rightarrow$ OCaml code
\item \funcname{caml\_startup\_common}
\item \funcname{caml\_startup\_exn}
\item \funcname{caml\_startup}
\item \funcname{caml\_main}
\item \funcname{main}
\end{itemize}
\smallskip
\item \funcname{caml\_startup} checks the return value of \funcname{caml\_startup\_exn}
\begin{itemize}
\item The return value has been specially marked by \funcname{caml\_start\_program} handler
\item Calls \funcname{caml\_fatal\_uncaught\_exception}
\end{itemize}
\item<2-> \funcname{caml\_fatal\_uncaught\_exception} either
\begin{itemize}
\item<2-> Calls the user custom uncaught exception handler, if set with \mintinline{ocaml}{Printexc.set_uncaught_exception_handler fn}\footnotemark
\item<2-> Or falls back to the default one
\item<2-> Which notifies about the uncaught exception
\end{itemize}
\item<4-> Terminates the program either with \funcname{abort} or with a non-zero exit code
\end{itemize}
\end{column}
\begin{column}{0.55\textwidth}
\only<1>{
\listc[firstline=84,lastline=89,highlightlines=88]{../ocaml/runtime/caml/mlvalues.h}{runtime/caml/mlvalues.h:84}
\listc[firstline=138,lastline=143,highlightlines={141-142}]{../ocaml/runtime/startup\_nat.c}{runtime/startup\_nat.c:138}
}%
\only<2>{
\listc[firstline=138,highlightlines={147,149}]{../ocaml/runtime/printexc.c}{runtime/printexc.c:138}
}%
\only<3>{
\listc[firstline=110,lastline=134,highlightlines=129]{../ocaml/runtime/printexc.c}{runtime/printexc.c:138}
}%
\only<4>{
\listc[firstline=138,highlightlines={152,154}]{../ocaml/runtime/printexc.c}{runtime/printexc.c:138}
}%
\end{column}
\end{columns}
\footnotetext[1]{\url{https://v2.ocaml.org/api/Printexc.html\#1\_Uncaughtexceptions}}
\end{frame}
}%
\only<2>{\providecommand\step{4}%
% Runtime's default exception handler
%
\subsection*{Runtime's default exception handler}
\frameSubsection{pictures/The\_Architect.png}{}


\begin{frame}{Default exception handler}
\begin{columns}
\begin{column}{0.5\textwidth}
\begin{itemize}
\item \localname{domain\_state->exn\_handler} points to a trap located before \funcname{entry}!
\item What installed this very first trap there?
\end{itemize}
\bigskip
\listocaml{../src/default/default.ml}{default/default.ml}
\bigskip
\inputminted[fontsize=\footnotesize]{text}{../src/default/default.out}
\end{column}
\begin{column}{0.5\textwidth}
\centering
\only<1>{\providecommand\step{0}\input{diagrams/default/default.tex}}%
\only<2>{\providecommand\step{4}\input{diagrams/default/default.tex}}%
\end{column}
\end{columns}
\end{frame}


\begin{frame}{Startup}{How did we get there by the way?}
\begin{columns}
\begin{column}{0.45\textwidth}
\begin{itemize}
\item The entry point address of the binary is \funcname{main} (\funcname{\_start}) from the runtime
\item The program starts like a C program:
\begin{itemize}
\item \funcname{caml\_start\_program}
\item \funcname{caml\_startup\_common}
\item \funcname{caml\_startup\_exn}
\item \funcname{caml\_startup}
\item \funcname{caml\_main}
\item \funcname{main}
\end{itemize}
%\item \funcname{caml\_startup\_common}: initializes GC, domains \dots
% Also: codefrag, locale, custom opeartions, frame_descr, signals, stack
\item \funcname{caml\_start\_program} is the transition point between the C realm and the OCaml one
\end{itemize}
\bigskip
\listocaml{../src/default/default.ml}{default/default.ml}
\end{column}
\begin{column}{0.55\textwidth}
\listasm[lastline=24,highlightlines={22-24}]{src/ocaml/caml\_start\_program.s}{runtime/amd64.S:604}
\end{column}
\end{columns}
\end{frame}


\begin{frame}{Setup to jump into OCaml entry point}{\funcname{caml\_start\_program} initializes the main fiber}
\begin{columns}
\begin{column}{0.6\textwidth}
\only<1,3,5>{
\begin{itemize}
\item<1-> Stores information to allow switching from OCaml stack to the C stack
\item<3-> Constructs the default exception handler
\item<5-> Switches to the OCaml stack and calls \funcname{caml\_program}
\end{itemize}
\bigskip
\listocaml{../src/default/default.ml}{default/default.ml}
}%
\only<2>{
\listasm[firstline=22,lastline=41,highlightlines={25-27}]{src/ocaml/caml\_start\_program.s}{caml\_start\_program}
}%
\only<4>{
\mintasm[firstline=22,lastline=41,highlightlines={31-38}]{src/ocaml/caml\_start\_program.s}
\listasm[firstline=66,lastline=72]{src/ocaml/caml\_start\_program.s}{caml\_start\_program}
}%
\only<6>{
\listasm[firstline=22,lastline=41,highlightlines={39-41}]{src/ocaml/caml\_start\_program.s}{caml\_start\_program}
}%
\end{column}
\begin{column}{0.4\textwidth}
\centering
\only<1-2>{\providecommand\step{2}\input{diagrams/default/default.tex}}%
\only<3-5>{\providecommand\step{3}\input{diagrams/default/default.tex}}%
\onslide*<6->{\providecommand\step{4}\input{diagrams/default/default.tex}}%
\end{column}
\end{columns}
\end{frame}

\begin{frame}{Executing the OCaml program}{And raising an exception without handler}
\begin{columns}
\begin{column}{0.6\textwidth}
\only<1-3>{
\begin{itemize}
\item<1-> Execution continues with OCaml code
\begin{itemize}
\item \funcname{camlDefault\_\_main\_269}
\item \funcname{camlDefault\_\_entry}
\item \funcname{caml\_program}
\end{itemize}
\item<1-> \funcname{camlDefault\_\_main\_269} raises a \typename{Failure} exception
\item<1-> \typename{Failure} is caught by \funcname{caml\_start\_program}
\begin{itemize}
\item<1-> The handler set the 2nd LSB of the exception value to \funcarg{1}
\item<3-> And then forces \funcname{caml\_start\_program} to terminate
\end{itemize}
\end{itemize}
}
\bigskip
\only<1,3>{\listocaml[highlightlines=3]{../src/default/default.ml}{default/default.ml}}%
\only<2>{\listasm[firstline=66,lastline=72,highlightlines=69]{src/ocaml/caml\_start\_program.s}{caml\_start\_program}}%
\end{column}
\begin{column}{0.4\textwidth}
\centering
\providecommand\step{4}\input{diagrams/default/default.tex}
\end{column}
\end{columns}
\end{frame}
%\only<>{\listasm[firstline=47,lastline=65]{src/ocaml/caml\_start\_program.s}{caml\_start\_program}}


\begin{frame}{Back to C}{Clean up, complain and terminate}
\begin{columns}
\begin{column}{0.45\textwidth}
\begin{itemize}
\item Backtrace:
\begin{itemize}
\item \funcname{caml\_start\_program} $\rightarrow$ OCaml code
\item \funcname{caml\_startup\_common}
\item \funcname{caml\_startup\_exn}
\item \funcname{caml\_startup}
\item \funcname{caml\_main}
\item \funcname{main}
\end{itemize}
\smallskip
\item \funcname{caml\_startup} checks the return value of \funcname{caml\_startup\_exn}
\begin{itemize}
\item The return value has been specially marked by \funcname{caml\_start\_program} handler
\item Calls \funcname{caml\_fatal\_uncaught\_exception}
\end{itemize}
\item<2-> \funcname{caml\_fatal\_uncaught\_exception} either
\begin{itemize}
\item<2-> Calls the user custom uncaught exception handler, if set with \mintinline{ocaml}{Printexc.set_uncaught_exception_handler fn}\footnotemark
\item<2-> Or falls back to the default one
\item<2-> Which notifies about the uncaught exception
\end{itemize}
\item<4-> Terminates the program either with \funcname{abort} or with a non-zero exit code
\end{itemize}
\end{column}
\begin{column}{0.55\textwidth}
\only<1>{
\listc[firstline=84,lastline=89,highlightlines=88]{../ocaml/runtime/caml/mlvalues.h}{runtime/caml/mlvalues.h:84}
\listc[firstline=138,lastline=143,highlightlines={141-142}]{../ocaml/runtime/startup\_nat.c}{runtime/startup\_nat.c:138}
}%
\only<2>{
\listc[firstline=138,highlightlines={147,149}]{../ocaml/runtime/printexc.c}{runtime/printexc.c:138}
}%
\only<3>{
\listc[firstline=110,lastline=134,highlightlines=129]{../ocaml/runtime/printexc.c}{runtime/printexc.c:138}
}%
\only<4>{
\listc[firstline=138,highlightlines={152,154}]{../ocaml/runtime/printexc.c}{runtime/printexc.c:138}
}%
\end{column}
\end{columns}
\footnotetext[1]{\url{https://v2.ocaml.org/api/Printexc.html\#1\_Uncaughtexceptions}}
\end{frame}
}%
\end{column}
\end{columns}
\end{frame}


\begin{frame}{Startup}{How did we get there by the way?}
\begin{columns}
\begin{column}{0.45\textwidth}
\begin{itemize}
\item The entry point address of the binary is \funcname{main} (\funcname{\_start}) from the runtime
\item The program starts like a C program:
\begin{itemize}
\item \funcname{caml\_start\_program}
\item \funcname{caml\_startup\_common}
\item \funcname{caml\_startup\_exn}
\item \funcname{caml\_startup}
\item \funcname{caml\_main}
\item \funcname{main}
\end{itemize}
%\item \funcname{caml\_startup\_common}: initializes GC, domains \dots
% Also: codefrag, locale, custom opeartions, frame_descr, signals, stack
\item \funcname{caml\_start\_program} is the transition point between the C realm and the OCaml one
\end{itemize}
\bigskip
\listocaml{../src/default/default.ml}{default/default.ml}
\end{column}
\begin{column}{0.55\textwidth}
\listasm[lastline=24,highlightlines={22-24}]{src/ocaml/caml\_start\_program.s}{runtime/amd64.S:604}
\end{column}
\end{columns}
\end{frame}


\begin{frame}{Setup to jump into OCaml entry point}{\funcname{caml\_start\_program} initializes the main fiber}
\begin{columns}
\begin{column}{0.6\textwidth}
\only<1,3,5>{
\begin{itemize}
\item<1-> Stores information to allow switching from OCaml stack to the C stack
\item<3-> Constructs the default exception handler
\item<5-> Switches to the OCaml stack and calls \funcname{caml\_program}
\end{itemize}
\bigskip
\listocaml{../src/default/default.ml}{default/default.ml}
}%
\only<2>{
\listasm[firstline=22,lastline=41,highlightlines={25-27}]{src/ocaml/caml\_start\_program.s}{caml\_start\_program}
}%
\only<4>{
\mintasm[firstline=22,lastline=41,highlightlines={31-38}]{src/ocaml/caml\_start\_program.s}
\listasm[firstline=66,lastline=72]{src/ocaml/caml\_start\_program.s}{caml\_start\_program}
}%
\only<6>{
\listasm[firstline=22,lastline=41,highlightlines={39-41}]{src/ocaml/caml\_start\_program.s}{caml\_start\_program}
}%
\end{column}
\begin{column}{0.4\textwidth}
\centering
\only<1-2>{\providecommand\step{2}%
% Runtime's default exception handler
%
\subsection*{Runtime's default exception handler}
\frameSubsection{pictures/The\_Architect.png}{}


\begin{frame}{Default exception handler}
\begin{columns}
\begin{column}{0.5\textwidth}
\begin{itemize}
\item \localname{domain\_state->exn\_handler} points to a trap located before \funcname{entry}!
\item What installed this very first trap there?
\end{itemize}
\bigskip
\listocaml{../src/default/default.ml}{default/default.ml}
\bigskip
\inputminted[fontsize=\footnotesize]{text}{../src/default/default.out}
\end{column}
\begin{column}{0.5\textwidth}
\centering
\only<1>{\providecommand\step{0}\input{diagrams/default/default.tex}}%
\only<2>{\providecommand\step{4}\input{diagrams/default/default.tex}}%
\end{column}
\end{columns}
\end{frame}


\begin{frame}{Startup}{How did we get there by the way?}
\begin{columns}
\begin{column}{0.45\textwidth}
\begin{itemize}
\item The entry point address of the binary is \funcname{main} (\funcname{\_start}) from the runtime
\item The program starts like a C program:
\begin{itemize}
\item \funcname{caml\_start\_program}
\item \funcname{caml\_startup\_common}
\item \funcname{caml\_startup\_exn}
\item \funcname{caml\_startup}
\item \funcname{caml\_main}
\item \funcname{main}
\end{itemize}
%\item \funcname{caml\_startup\_common}: initializes GC, domains \dots
% Also: codefrag, locale, custom opeartions, frame_descr, signals, stack
\item \funcname{caml\_start\_program} is the transition point between the C realm and the OCaml one
\end{itemize}
\bigskip
\listocaml{../src/default/default.ml}{default/default.ml}
\end{column}
\begin{column}{0.55\textwidth}
\listasm[lastline=24,highlightlines={22-24}]{src/ocaml/caml\_start\_program.s}{runtime/amd64.S:604}
\end{column}
\end{columns}
\end{frame}


\begin{frame}{Setup to jump into OCaml entry point}{\funcname{caml\_start\_program} initializes the main fiber}
\begin{columns}
\begin{column}{0.6\textwidth}
\only<1,3,5>{
\begin{itemize}
\item<1-> Stores information to allow switching from OCaml stack to the C stack
\item<3-> Constructs the default exception handler
\item<5-> Switches to the OCaml stack and calls \funcname{caml\_program}
\end{itemize}
\bigskip
\listocaml{../src/default/default.ml}{default/default.ml}
}%
\only<2>{
\listasm[firstline=22,lastline=41,highlightlines={25-27}]{src/ocaml/caml\_start\_program.s}{caml\_start\_program}
}%
\only<4>{
\mintasm[firstline=22,lastline=41,highlightlines={31-38}]{src/ocaml/caml\_start\_program.s}
\listasm[firstline=66,lastline=72]{src/ocaml/caml\_start\_program.s}{caml\_start\_program}
}%
\only<6>{
\listasm[firstline=22,lastline=41,highlightlines={39-41}]{src/ocaml/caml\_start\_program.s}{caml\_start\_program}
}%
\end{column}
\begin{column}{0.4\textwidth}
\centering
\only<1-2>{\providecommand\step{2}\input{diagrams/default/default.tex}}%
\only<3-5>{\providecommand\step{3}\input{diagrams/default/default.tex}}%
\onslide*<6->{\providecommand\step{4}\input{diagrams/default/default.tex}}%
\end{column}
\end{columns}
\end{frame}

\begin{frame}{Executing the OCaml program}{And raising an exception without handler}
\begin{columns}
\begin{column}{0.6\textwidth}
\only<1-3>{
\begin{itemize}
\item<1-> Execution continues with OCaml code
\begin{itemize}
\item \funcname{camlDefault\_\_main\_269}
\item \funcname{camlDefault\_\_entry}
\item \funcname{caml\_program}
\end{itemize}
\item<1-> \funcname{camlDefault\_\_main\_269} raises a \typename{Failure} exception
\item<1-> \typename{Failure} is caught by \funcname{caml\_start\_program}
\begin{itemize}
\item<1-> The handler set the 2nd LSB of the exception value to \funcarg{1}
\item<3-> And then forces \funcname{caml\_start\_program} to terminate
\end{itemize}
\end{itemize}
}
\bigskip
\only<1,3>{\listocaml[highlightlines=3]{../src/default/default.ml}{default/default.ml}}%
\only<2>{\listasm[firstline=66,lastline=72,highlightlines=69]{src/ocaml/caml\_start\_program.s}{caml\_start\_program}}%
\end{column}
\begin{column}{0.4\textwidth}
\centering
\providecommand\step{4}\input{diagrams/default/default.tex}
\end{column}
\end{columns}
\end{frame}
%\only<>{\listasm[firstline=47,lastline=65]{src/ocaml/caml\_start\_program.s}{caml\_start\_program}}


\begin{frame}{Back to C}{Clean up, complain and terminate}
\begin{columns}
\begin{column}{0.45\textwidth}
\begin{itemize}
\item Backtrace:
\begin{itemize}
\item \funcname{caml\_start\_program} $\rightarrow$ OCaml code
\item \funcname{caml\_startup\_common}
\item \funcname{caml\_startup\_exn}
\item \funcname{caml\_startup}
\item \funcname{caml\_main}
\item \funcname{main}
\end{itemize}
\smallskip
\item \funcname{caml\_startup} checks the return value of \funcname{caml\_startup\_exn}
\begin{itemize}
\item The return value has been specially marked by \funcname{caml\_start\_program} handler
\item Calls \funcname{caml\_fatal\_uncaught\_exception}
\end{itemize}
\item<2-> \funcname{caml\_fatal\_uncaught\_exception} either
\begin{itemize}
\item<2-> Calls the user custom uncaught exception handler, if set with \mintinline{ocaml}{Printexc.set_uncaught_exception_handler fn}\footnotemark
\item<2-> Or falls back to the default one
\item<2-> Which notifies about the uncaught exception
\end{itemize}
\item<4-> Terminates the program either with \funcname{abort} or with a non-zero exit code
\end{itemize}
\end{column}
\begin{column}{0.55\textwidth}
\only<1>{
\listc[firstline=84,lastline=89,highlightlines=88]{../ocaml/runtime/caml/mlvalues.h}{runtime/caml/mlvalues.h:84}
\listc[firstline=138,lastline=143,highlightlines={141-142}]{../ocaml/runtime/startup\_nat.c}{runtime/startup\_nat.c:138}
}%
\only<2>{
\listc[firstline=138,highlightlines={147,149}]{../ocaml/runtime/printexc.c}{runtime/printexc.c:138}
}%
\only<3>{
\listc[firstline=110,lastline=134,highlightlines=129]{../ocaml/runtime/printexc.c}{runtime/printexc.c:138}
}%
\only<4>{
\listc[firstline=138,highlightlines={152,154}]{../ocaml/runtime/printexc.c}{runtime/printexc.c:138}
}%
\end{column}
\end{columns}
\footnotetext[1]{\url{https://v2.ocaml.org/api/Printexc.html\#1\_Uncaughtexceptions}}
\end{frame}
}%
\only<3-5>{\providecommand\step{3}%
% Runtime's default exception handler
%
\subsection*{Runtime's default exception handler}
\frameSubsection{pictures/The\_Architect.png}{}


\begin{frame}{Default exception handler}
\begin{columns}
\begin{column}{0.5\textwidth}
\begin{itemize}
\item \localname{domain\_state->exn\_handler} points to a trap located before \funcname{entry}!
\item What installed this very first trap there?
\end{itemize}
\bigskip
\listocaml{../src/default/default.ml}{default/default.ml}
\bigskip
\inputminted[fontsize=\footnotesize]{text}{../src/default/default.out}
\end{column}
\begin{column}{0.5\textwidth}
\centering
\only<1>{\providecommand\step{0}\input{diagrams/default/default.tex}}%
\only<2>{\providecommand\step{4}\input{diagrams/default/default.tex}}%
\end{column}
\end{columns}
\end{frame}


\begin{frame}{Startup}{How did we get there by the way?}
\begin{columns}
\begin{column}{0.45\textwidth}
\begin{itemize}
\item The entry point address of the binary is \funcname{main} (\funcname{\_start}) from the runtime
\item The program starts like a C program:
\begin{itemize}
\item \funcname{caml\_start\_program}
\item \funcname{caml\_startup\_common}
\item \funcname{caml\_startup\_exn}
\item \funcname{caml\_startup}
\item \funcname{caml\_main}
\item \funcname{main}
\end{itemize}
%\item \funcname{caml\_startup\_common}: initializes GC, domains \dots
% Also: codefrag, locale, custom opeartions, frame_descr, signals, stack
\item \funcname{caml\_start\_program} is the transition point between the C realm and the OCaml one
\end{itemize}
\bigskip
\listocaml{../src/default/default.ml}{default/default.ml}
\end{column}
\begin{column}{0.55\textwidth}
\listasm[lastline=24,highlightlines={22-24}]{src/ocaml/caml\_start\_program.s}{runtime/amd64.S:604}
\end{column}
\end{columns}
\end{frame}


\begin{frame}{Setup to jump into OCaml entry point}{\funcname{caml\_start\_program} initializes the main fiber}
\begin{columns}
\begin{column}{0.6\textwidth}
\only<1,3,5>{
\begin{itemize}
\item<1-> Stores information to allow switching from OCaml stack to the C stack
\item<3-> Constructs the default exception handler
\item<5-> Switches to the OCaml stack and calls \funcname{caml\_program}
\end{itemize}
\bigskip
\listocaml{../src/default/default.ml}{default/default.ml}
}%
\only<2>{
\listasm[firstline=22,lastline=41,highlightlines={25-27}]{src/ocaml/caml\_start\_program.s}{caml\_start\_program}
}%
\only<4>{
\mintasm[firstline=22,lastline=41,highlightlines={31-38}]{src/ocaml/caml\_start\_program.s}
\listasm[firstline=66,lastline=72]{src/ocaml/caml\_start\_program.s}{caml\_start\_program}
}%
\only<6>{
\listasm[firstline=22,lastline=41,highlightlines={39-41}]{src/ocaml/caml\_start\_program.s}{caml\_start\_program}
}%
\end{column}
\begin{column}{0.4\textwidth}
\centering
\only<1-2>{\providecommand\step{2}\input{diagrams/default/default.tex}}%
\only<3-5>{\providecommand\step{3}\input{diagrams/default/default.tex}}%
\onslide*<6->{\providecommand\step{4}\input{diagrams/default/default.tex}}%
\end{column}
\end{columns}
\end{frame}

\begin{frame}{Executing the OCaml program}{And raising an exception without handler}
\begin{columns}
\begin{column}{0.6\textwidth}
\only<1-3>{
\begin{itemize}
\item<1-> Execution continues with OCaml code
\begin{itemize}
\item \funcname{camlDefault\_\_main\_269}
\item \funcname{camlDefault\_\_entry}
\item \funcname{caml\_program}
\end{itemize}
\item<1-> \funcname{camlDefault\_\_main\_269} raises a \typename{Failure} exception
\item<1-> \typename{Failure} is caught by \funcname{caml\_start\_program}
\begin{itemize}
\item<1-> The handler set the 2nd LSB of the exception value to \funcarg{1}
\item<3-> And then forces \funcname{caml\_start\_program} to terminate
\end{itemize}
\end{itemize}
}
\bigskip
\only<1,3>{\listocaml[highlightlines=3]{../src/default/default.ml}{default/default.ml}}%
\only<2>{\listasm[firstline=66,lastline=72,highlightlines=69]{src/ocaml/caml\_start\_program.s}{caml\_start\_program}}%
\end{column}
\begin{column}{0.4\textwidth}
\centering
\providecommand\step{4}\input{diagrams/default/default.tex}
\end{column}
\end{columns}
\end{frame}
%\only<>{\listasm[firstline=47,lastline=65]{src/ocaml/caml\_start\_program.s}{caml\_start\_program}}


\begin{frame}{Back to C}{Clean up, complain and terminate}
\begin{columns}
\begin{column}{0.45\textwidth}
\begin{itemize}
\item Backtrace:
\begin{itemize}
\item \funcname{caml\_start\_program} $\rightarrow$ OCaml code
\item \funcname{caml\_startup\_common}
\item \funcname{caml\_startup\_exn}
\item \funcname{caml\_startup}
\item \funcname{caml\_main}
\item \funcname{main}
\end{itemize}
\smallskip
\item \funcname{caml\_startup} checks the return value of \funcname{caml\_startup\_exn}
\begin{itemize}
\item The return value has been specially marked by \funcname{caml\_start\_program} handler
\item Calls \funcname{caml\_fatal\_uncaught\_exception}
\end{itemize}
\item<2-> \funcname{caml\_fatal\_uncaught\_exception} either
\begin{itemize}
\item<2-> Calls the user custom uncaught exception handler, if set with \mintinline{ocaml}{Printexc.set_uncaught_exception_handler fn}\footnotemark
\item<2-> Or falls back to the default one
\item<2-> Which notifies about the uncaught exception
\end{itemize}
\item<4-> Terminates the program either with \funcname{abort} or with a non-zero exit code
\end{itemize}
\end{column}
\begin{column}{0.55\textwidth}
\only<1>{
\listc[firstline=84,lastline=89,highlightlines=88]{../ocaml/runtime/caml/mlvalues.h}{runtime/caml/mlvalues.h:84}
\listc[firstline=138,lastline=143,highlightlines={141-142}]{../ocaml/runtime/startup\_nat.c}{runtime/startup\_nat.c:138}
}%
\only<2>{
\listc[firstline=138,highlightlines={147,149}]{../ocaml/runtime/printexc.c}{runtime/printexc.c:138}
}%
\only<3>{
\listc[firstline=110,lastline=134,highlightlines=129]{../ocaml/runtime/printexc.c}{runtime/printexc.c:138}
}%
\only<4>{
\listc[firstline=138,highlightlines={152,154}]{../ocaml/runtime/printexc.c}{runtime/printexc.c:138}
}%
\end{column}
\end{columns}
\footnotetext[1]{\url{https://v2.ocaml.org/api/Printexc.html\#1\_Uncaughtexceptions}}
\end{frame}
}%
\onslide*<6->{\providecommand\step{4}%
% Runtime's default exception handler
%
\subsection*{Runtime's default exception handler}
\frameSubsection{pictures/The\_Architect.png}{}


\begin{frame}{Default exception handler}
\begin{columns}
\begin{column}{0.5\textwidth}
\begin{itemize}
\item \localname{domain\_state->exn\_handler} points to a trap located before \funcname{entry}!
\item What installed this very first trap there?
\end{itemize}
\bigskip
\listocaml{../src/default/default.ml}{default/default.ml}
\bigskip
\inputminted[fontsize=\footnotesize]{text}{../src/default/default.out}
\end{column}
\begin{column}{0.5\textwidth}
\centering
\only<1>{\providecommand\step{0}\input{diagrams/default/default.tex}}%
\only<2>{\providecommand\step{4}\input{diagrams/default/default.tex}}%
\end{column}
\end{columns}
\end{frame}


\begin{frame}{Startup}{How did we get there by the way?}
\begin{columns}
\begin{column}{0.45\textwidth}
\begin{itemize}
\item The entry point address of the binary is \funcname{main} (\funcname{\_start}) from the runtime
\item The program starts like a C program:
\begin{itemize}
\item \funcname{caml\_start\_program}
\item \funcname{caml\_startup\_common}
\item \funcname{caml\_startup\_exn}
\item \funcname{caml\_startup}
\item \funcname{caml\_main}
\item \funcname{main}
\end{itemize}
%\item \funcname{caml\_startup\_common}: initializes GC, domains \dots
% Also: codefrag, locale, custom opeartions, frame_descr, signals, stack
\item \funcname{caml\_start\_program} is the transition point between the C realm and the OCaml one
\end{itemize}
\bigskip
\listocaml{../src/default/default.ml}{default/default.ml}
\end{column}
\begin{column}{0.55\textwidth}
\listasm[lastline=24,highlightlines={22-24}]{src/ocaml/caml\_start\_program.s}{runtime/amd64.S:604}
\end{column}
\end{columns}
\end{frame}


\begin{frame}{Setup to jump into OCaml entry point}{\funcname{caml\_start\_program} initializes the main fiber}
\begin{columns}
\begin{column}{0.6\textwidth}
\only<1,3,5>{
\begin{itemize}
\item<1-> Stores information to allow switching from OCaml stack to the C stack
\item<3-> Constructs the default exception handler
\item<5-> Switches to the OCaml stack and calls \funcname{caml\_program}
\end{itemize}
\bigskip
\listocaml{../src/default/default.ml}{default/default.ml}
}%
\only<2>{
\listasm[firstline=22,lastline=41,highlightlines={25-27}]{src/ocaml/caml\_start\_program.s}{caml\_start\_program}
}%
\only<4>{
\mintasm[firstline=22,lastline=41,highlightlines={31-38}]{src/ocaml/caml\_start\_program.s}
\listasm[firstline=66,lastline=72]{src/ocaml/caml\_start\_program.s}{caml\_start\_program}
}%
\only<6>{
\listasm[firstline=22,lastline=41,highlightlines={39-41}]{src/ocaml/caml\_start\_program.s}{caml\_start\_program}
}%
\end{column}
\begin{column}{0.4\textwidth}
\centering
\only<1-2>{\providecommand\step{2}\input{diagrams/default/default.tex}}%
\only<3-5>{\providecommand\step{3}\input{diagrams/default/default.tex}}%
\onslide*<6->{\providecommand\step{4}\input{diagrams/default/default.tex}}%
\end{column}
\end{columns}
\end{frame}

\begin{frame}{Executing the OCaml program}{And raising an exception without handler}
\begin{columns}
\begin{column}{0.6\textwidth}
\only<1-3>{
\begin{itemize}
\item<1-> Execution continues with OCaml code
\begin{itemize}
\item \funcname{camlDefault\_\_main\_269}
\item \funcname{camlDefault\_\_entry}
\item \funcname{caml\_program}
\end{itemize}
\item<1-> \funcname{camlDefault\_\_main\_269} raises a \typename{Failure} exception
\item<1-> \typename{Failure} is caught by \funcname{caml\_start\_program}
\begin{itemize}
\item<1-> The handler set the 2nd LSB of the exception value to \funcarg{1}
\item<3-> And then forces \funcname{caml\_start\_program} to terminate
\end{itemize}
\end{itemize}
}
\bigskip
\only<1,3>{\listocaml[highlightlines=3]{../src/default/default.ml}{default/default.ml}}%
\only<2>{\listasm[firstline=66,lastline=72,highlightlines=69]{src/ocaml/caml\_start\_program.s}{caml\_start\_program}}%
\end{column}
\begin{column}{0.4\textwidth}
\centering
\providecommand\step{4}\input{diagrams/default/default.tex}
\end{column}
\end{columns}
\end{frame}
%\only<>{\listasm[firstline=47,lastline=65]{src/ocaml/caml\_start\_program.s}{caml\_start\_program}}


\begin{frame}{Back to C}{Clean up, complain and terminate}
\begin{columns}
\begin{column}{0.45\textwidth}
\begin{itemize}
\item Backtrace:
\begin{itemize}
\item \funcname{caml\_start\_program} $\rightarrow$ OCaml code
\item \funcname{caml\_startup\_common}
\item \funcname{caml\_startup\_exn}
\item \funcname{caml\_startup}
\item \funcname{caml\_main}
\item \funcname{main}
\end{itemize}
\smallskip
\item \funcname{caml\_startup} checks the return value of \funcname{caml\_startup\_exn}
\begin{itemize}
\item The return value has been specially marked by \funcname{caml\_start\_program} handler
\item Calls \funcname{caml\_fatal\_uncaught\_exception}
\end{itemize}
\item<2-> \funcname{caml\_fatal\_uncaught\_exception} either
\begin{itemize}
\item<2-> Calls the user custom uncaught exception handler, if set with \mintinline{ocaml}{Printexc.set_uncaught_exception_handler fn}\footnotemark
\item<2-> Or falls back to the default one
\item<2-> Which notifies about the uncaught exception
\end{itemize}
\item<4-> Terminates the program either with \funcname{abort} or with a non-zero exit code
\end{itemize}
\end{column}
\begin{column}{0.55\textwidth}
\only<1>{
\listc[firstline=84,lastline=89,highlightlines=88]{../ocaml/runtime/caml/mlvalues.h}{runtime/caml/mlvalues.h:84}
\listc[firstline=138,lastline=143,highlightlines={141-142}]{../ocaml/runtime/startup\_nat.c}{runtime/startup\_nat.c:138}
}%
\only<2>{
\listc[firstline=138,highlightlines={147,149}]{../ocaml/runtime/printexc.c}{runtime/printexc.c:138}
}%
\only<3>{
\listc[firstline=110,lastline=134,highlightlines=129]{../ocaml/runtime/printexc.c}{runtime/printexc.c:138}
}%
\only<4>{
\listc[firstline=138,highlightlines={152,154}]{../ocaml/runtime/printexc.c}{runtime/printexc.c:138}
}%
\end{column}
\end{columns}
\footnotetext[1]{\url{https://v2.ocaml.org/api/Printexc.html\#1\_Uncaughtexceptions}}
\end{frame}
}%
\end{column}
\end{columns}
\end{frame}

\begin{frame}{Executing the OCaml program}{And raising an exception without handler}
\begin{columns}
\begin{column}{0.6\textwidth}
\only<1-3>{
\begin{itemize}
\item<1-> Execution continues with OCaml code
\begin{itemize}
\item \funcname{camlDefault\_\_main\_269}
\item \funcname{camlDefault\_\_entry}
\item \funcname{caml\_program}
\end{itemize}
\item<1-> \funcname{camlDefault\_\_main\_269} raises a \typename{Failure} exception
\item<1-> \typename{Failure} is caught by \funcname{caml\_start\_program}
\begin{itemize}
\item<1-> The handler set the 2nd LSB of the exception value to \funcarg{1}
\item<3-> And then forces \funcname{caml\_start\_program} to terminate
\end{itemize}
\end{itemize}
}
\bigskip
\only<1,3>{\listocaml[highlightlines=3]{../src/default/default.ml}{default/default.ml}}%
\only<2>{\listasm[firstline=66,lastline=72,highlightlines=69]{src/ocaml/caml\_start\_program.s}{caml\_start\_program}}%
\end{column}
\begin{column}{0.4\textwidth}
\centering
\providecommand\step{4}%
% Runtime's default exception handler
%
\subsection*{Runtime's default exception handler}
\frameSubsection{pictures/The\_Architect.png}{}


\begin{frame}{Default exception handler}
\begin{columns}
\begin{column}{0.5\textwidth}
\begin{itemize}
\item \localname{domain\_state->exn\_handler} points to a trap located before \funcname{entry}!
\item What installed this very first trap there?
\end{itemize}
\bigskip
\listocaml{../src/default/default.ml}{default/default.ml}
\bigskip
\inputminted[fontsize=\footnotesize]{text}{../src/default/default.out}
\end{column}
\begin{column}{0.5\textwidth}
\centering
\only<1>{\providecommand\step{0}\input{diagrams/default/default.tex}}%
\only<2>{\providecommand\step{4}\input{diagrams/default/default.tex}}%
\end{column}
\end{columns}
\end{frame}


\begin{frame}{Startup}{How did we get there by the way?}
\begin{columns}
\begin{column}{0.45\textwidth}
\begin{itemize}
\item The entry point address of the binary is \funcname{main} (\funcname{\_start}) from the runtime
\item The program starts like a C program:
\begin{itemize}
\item \funcname{caml\_start\_program}
\item \funcname{caml\_startup\_common}
\item \funcname{caml\_startup\_exn}
\item \funcname{caml\_startup}
\item \funcname{caml\_main}
\item \funcname{main}
\end{itemize}
%\item \funcname{caml\_startup\_common}: initializes GC, domains \dots
% Also: codefrag, locale, custom opeartions, frame_descr, signals, stack
\item \funcname{caml\_start\_program} is the transition point between the C realm and the OCaml one
\end{itemize}
\bigskip
\listocaml{../src/default/default.ml}{default/default.ml}
\end{column}
\begin{column}{0.55\textwidth}
\listasm[lastline=24,highlightlines={22-24}]{src/ocaml/caml\_start\_program.s}{runtime/amd64.S:604}
\end{column}
\end{columns}
\end{frame}


\begin{frame}{Setup to jump into OCaml entry point}{\funcname{caml\_start\_program} initializes the main fiber}
\begin{columns}
\begin{column}{0.6\textwidth}
\only<1,3,5>{
\begin{itemize}
\item<1-> Stores information to allow switching from OCaml stack to the C stack
\item<3-> Constructs the default exception handler
\item<5-> Switches to the OCaml stack and calls \funcname{caml\_program}
\end{itemize}
\bigskip
\listocaml{../src/default/default.ml}{default/default.ml}
}%
\only<2>{
\listasm[firstline=22,lastline=41,highlightlines={25-27}]{src/ocaml/caml\_start\_program.s}{caml\_start\_program}
}%
\only<4>{
\mintasm[firstline=22,lastline=41,highlightlines={31-38}]{src/ocaml/caml\_start\_program.s}
\listasm[firstline=66,lastline=72]{src/ocaml/caml\_start\_program.s}{caml\_start\_program}
}%
\only<6>{
\listasm[firstline=22,lastline=41,highlightlines={39-41}]{src/ocaml/caml\_start\_program.s}{caml\_start\_program}
}%
\end{column}
\begin{column}{0.4\textwidth}
\centering
\only<1-2>{\providecommand\step{2}\input{diagrams/default/default.tex}}%
\only<3-5>{\providecommand\step{3}\input{diagrams/default/default.tex}}%
\onslide*<6->{\providecommand\step{4}\input{diagrams/default/default.tex}}%
\end{column}
\end{columns}
\end{frame}

\begin{frame}{Executing the OCaml program}{And raising an exception without handler}
\begin{columns}
\begin{column}{0.6\textwidth}
\only<1-3>{
\begin{itemize}
\item<1-> Execution continues with OCaml code
\begin{itemize}
\item \funcname{camlDefault\_\_main\_269}
\item \funcname{camlDefault\_\_entry}
\item \funcname{caml\_program}
\end{itemize}
\item<1-> \funcname{camlDefault\_\_main\_269} raises a \typename{Failure} exception
\item<1-> \typename{Failure} is caught by \funcname{caml\_start\_program}
\begin{itemize}
\item<1-> The handler set the 2nd LSB of the exception value to \funcarg{1}
\item<3-> And then forces \funcname{caml\_start\_program} to terminate
\end{itemize}
\end{itemize}
}
\bigskip
\only<1,3>{\listocaml[highlightlines=3]{../src/default/default.ml}{default/default.ml}}%
\only<2>{\listasm[firstline=66,lastline=72,highlightlines=69]{src/ocaml/caml\_start\_program.s}{caml\_start\_program}}%
\end{column}
\begin{column}{0.4\textwidth}
\centering
\providecommand\step{4}\input{diagrams/default/default.tex}
\end{column}
\end{columns}
\end{frame}
%\only<>{\listasm[firstline=47,lastline=65]{src/ocaml/caml\_start\_program.s}{caml\_start\_program}}


\begin{frame}{Back to C}{Clean up, complain and terminate}
\begin{columns}
\begin{column}{0.45\textwidth}
\begin{itemize}
\item Backtrace:
\begin{itemize}
\item \funcname{caml\_start\_program} $\rightarrow$ OCaml code
\item \funcname{caml\_startup\_common}
\item \funcname{caml\_startup\_exn}
\item \funcname{caml\_startup}
\item \funcname{caml\_main}
\item \funcname{main}
\end{itemize}
\smallskip
\item \funcname{caml\_startup} checks the return value of \funcname{caml\_startup\_exn}
\begin{itemize}
\item The return value has been specially marked by \funcname{caml\_start\_program} handler
\item Calls \funcname{caml\_fatal\_uncaught\_exception}
\end{itemize}
\item<2-> \funcname{caml\_fatal\_uncaught\_exception} either
\begin{itemize}
\item<2-> Calls the user custom uncaught exception handler, if set with \mintinline{ocaml}{Printexc.set_uncaught_exception_handler fn}\footnotemark
\item<2-> Or falls back to the default one
\item<2-> Which notifies about the uncaught exception
\end{itemize}
\item<4-> Terminates the program either with \funcname{abort} or with a non-zero exit code
\end{itemize}
\end{column}
\begin{column}{0.55\textwidth}
\only<1>{
\listc[firstline=84,lastline=89,highlightlines=88]{../ocaml/runtime/caml/mlvalues.h}{runtime/caml/mlvalues.h:84}
\listc[firstline=138,lastline=143,highlightlines={141-142}]{../ocaml/runtime/startup\_nat.c}{runtime/startup\_nat.c:138}
}%
\only<2>{
\listc[firstline=138,highlightlines={147,149}]{../ocaml/runtime/printexc.c}{runtime/printexc.c:138}
}%
\only<3>{
\listc[firstline=110,lastline=134,highlightlines=129]{../ocaml/runtime/printexc.c}{runtime/printexc.c:138}
}%
\only<4>{
\listc[firstline=138,highlightlines={152,154}]{../ocaml/runtime/printexc.c}{runtime/printexc.c:138}
}%
\end{column}
\end{columns}
\footnotetext[1]{\url{https://v2.ocaml.org/api/Printexc.html\#1\_Uncaughtexceptions}}
\end{frame}

\end{column}
\end{columns}
\end{frame}
%\only<>{\listasm[firstline=47,lastline=65]{src/ocaml/caml\_start\_program.s}{caml\_start\_program}}


\begin{frame}{Back to C}{Clean up, complain and terminate}
\begin{columns}
\begin{column}{0.45\textwidth}
\begin{itemize}
\item Backtrace:
\begin{itemize}
\item \funcname{caml\_start\_program} $\rightarrow$ OCaml code
\item \funcname{caml\_startup\_common}
\item \funcname{caml\_startup\_exn}
\item \funcname{caml\_startup}
\item \funcname{caml\_main}
\item \funcname{main}
\end{itemize}
\smallskip
\item \funcname{caml\_startup} checks the return value of \funcname{caml\_startup\_exn}
\begin{itemize}
\item The return value has been specially marked by \funcname{caml\_start\_program} handler
\item Calls \funcname{caml\_fatal\_uncaught\_exception}
\end{itemize}
\item<2-> \funcname{caml\_fatal\_uncaught\_exception} either
\begin{itemize}
\item<2-> Calls the user custom uncaught exception handler, if set with \mintinline{ocaml}{Printexc.set_uncaught_exception_handler fn}\footnotemark
\item<2-> Or falls back to the default one
\item<2-> Which notifies about the uncaught exception
\end{itemize}
\item<4-> Terminates the program either with \funcname{abort} or with a non-zero exit code
\end{itemize}
\end{column}
\begin{column}{0.55\textwidth}
\only<1>{
\listc[firstline=84,lastline=89,highlightlines=88]{../ocaml/runtime/caml/mlvalues.h}{runtime/caml/mlvalues.h:84}
\listc[firstline=138,lastline=143,highlightlines={141-142}]{../ocaml/runtime/startup\_nat.c}{runtime/startup\_nat.c:138}
}%
\only<2>{
\listc[firstline=138,highlightlines={147,149}]{../ocaml/runtime/printexc.c}{runtime/printexc.c:138}
}%
\only<3>{
\listc[firstline=110,lastline=134,highlightlines=129]{../ocaml/runtime/printexc.c}{runtime/printexc.c:138}
}%
\only<4>{
\listc[firstline=138,highlightlines={152,154}]{../ocaml/runtime/printexc.c}{runtime/printexc.c:138}
}%
\end{column}
\end{columns}
\footnotetext[1]{\url{https://v2.ocaml.org/api/Printexc.html\#1\_Uncaughtexceptions}}
\end{frame}
}%
\end{column}
\end{columns}
\end{frame}

\begin{frame}{Executing the OCaml program}{And raising an exception without handler}
\begin{columns}
\begin{column}{0.6\textwidth}
\only<1-3>{
\begin{itemize}
\item<1-> Execution continues with OCaml code
\begin{itemize}
\item \funcname{camlDefault\_\_main\_269}
\item \funcname{camlDefault\_\_entry}
\item \funcname{caml\_program}
\end{itemize}
\item<1-> \funcname{camlDefault\_\_main\_269} raises a \typename{Failure} exception
\item<1-> \typename{Failure} is caught by \funcname{caml\_start\_program}
\begin{itemize}
\item<1-> The handler set the 2nd LSB of the exception value to \funcarg{1}
\item<3-> And then forces \funcname{caml\_start\_program} to terminate
\end{itemize}
\end{itemize}
}
\bigskip
\only<1,3>{\listocaml[highlightlines=3]{../src/default/default.ml}{default/default.ml}}%
\only<2>{\listasm[firstline=66,lastline=72,highlightlines=69]{src/ocaml/caml\_start\_program.s}{caml\_start\_program}}%
\end{column}
\begin{column}{0.4\textwidth}
\centering
\providecommand\step{4}%
% Runtime's default exception handler
%
\subsection*{Runtime's default exception handler}
\frameSubsection{pictures/The\_Architect.png}{}


\begin{frame}{Default exception handler}
\begin{columns}
\begin{column}{0.5\textwidth}
\begin{itemize}
\item \localname{domain\_state->exn\_handler} points to a trap located before \funcname{entry}!
\item What installed this very first trap there?
\end{itemize}
\bigskip
\listocaml{../src/default/default.ml}{default/default.ml}
\bigskip
\inputminted[fontsize=\footnotesize]{text}{../src/default/default.out}
\end{column}
\begin{column}{0.5\textwidth}
\centering
\only<1>{\providecommand\step{0}%
% Runtime's default exception handler
%
\subsection*{Runtime's default exception handler}
\frameSubsection{pictures/The\_Architect.png}{}


\begin{frame}{Default exception handler}
\begin{columns}
\begin{column}{0.5\textwidth}
\begin{itemize}
\item \localname{domain\_state->exn\_handler} points to a trap located before \funcname{entry}!
\item What installed this very first trap there?
\end{itemize}
\bigskip
\listocaml{../src/default/default.ml}{default/default.ml}
\bigskip
\inputminted[fontsize=\footnotesize]{text}{../src/default/default.out}
\end{column}
\begin{column}{0.5\textwidth}
\centering
\only<1>{\providecommand\step{0}\input{diagrams/default/default.tex}}%
\only<2>{\providecommand\step{4}\input{diagrams/default/default.tex}}%
\end{column}
\end{columns}
\end{frame}


\begin{frame}{Startup}{How did we get there by the way?}
\begin{columns}
\begin{column}{0.45\textwidth}
\begin{itemize}
\item The entry point address of the binary is \funcname{main} (\funcname{\_start}) from the runtime
\item The program starts like a C program:
\begin{itemize}
\item \funcname{caml\_start\_program}
\item \funcname{caml\_startup\_common}
\item \funcname{caml\_startup\_exn}
\item \funcname{caml\_startup}
\item \funcname{caml\_main}
\item \funcname{main}
\end{itemize}
%\item \funcname{caml\_startup\_common}: initializes GC, domains \dots
% Also: codefrag, locale, custom opeartions, frame_descr, signals, stack
\item \funcname{caml\_start\_program} is the transition point between the C realm and the OCaml one
\end{itemize}
\bigskip
\listocaml{../src/default/default.ml}{default/default.ml}
\end{column}
\begin{column}{0.55\textwidth}
\listasm[lastline=24,highlightlines={22-24}]{src/ocaml/caml\_start\_program.s}{runtime/amd64.S:604}
\end{column}
\end{columns}
\end{frame}


\begin{frame}{Setup to jump into OCaml entry point}{\funcname{caml\_start\_program} initializes the main fiber}
\begin{columns}
\begin{column}{0.6\textwidth}
\only<1,3,5>{
\begin{itemize}
\item<1-> Stores information to allow switching from OCaml stack to the C stack
\item<3-> Constructs the default exception handler
\item<5-> Switches to the OCaml stack and calls \funcname{caml\_program}
\end{itemize}
\bigskip
\listocaml{../src/default/default.ml}{default/default.ml}
}%
\only<2>{
\listasm[firstline=22,lastline=41,highlightlines={25-27}]{src/ocaml/caml\_start\_program.s}{caml\_start\_program}
}%
\only<4>{
\mintasm[firstline=22,lastline=41,highlightlines={31-38}]{src/ocaml/caml\_start\_program.s}
\listasm[firstline=66,lastline=72]{src/ocaml/caml\_start\_program.s}{caml\_start\_program}
}%
\only<6>{
\listasm[firstline=22,lastline=41,highlightlines={39-41}]{src/ocaml/caml\_start\_program.s}{caml\_start\_program}
}%
\end{column}
\begin{column}{0.4\textwidth}
\centering
\only<1-2>{\providecommand\step{2}\input{diagrams/default/default.tex}}%
\only<3-5>{\providecommand\step{3}\input{diagrams/default/default.tex}}%
\onslide*<6->{\providecommand\step{4}\input{diagrams/default/default.tex}}%
\end{column}
\end{columns}
\end{frame}

\begin{frame}{Executing the OCaml program}{And raising an exception without handler}
\begin{columns}
\begin{column}{0.6\textwidth}
\only<1-3>{
\begin{itemize}
\item<1-> Execution continues with OCaml code
\begin{itemize}
\item \funcname{camlDefault\_\_main\_269}
\item \funcname{camlDefault\_\_entry}
\item \funcname{caml\_program}
\end{itemize}
\item<1-> \funcname{camlDefault\_\_main\_269} raises a \typename{Failure} exception
\item<1-> \typename{Failure} is caught by \funcname{caml\_start\_program}
\begin{itemize}
\item<1-> The handler set the 2nd LSB of the exception value to \funcarg{1}
\item<3-> And then forces \funcname{caml\_start\_program} to terminate
\end{itemize}
\end{itemize}
}
\bigskip
\only<1,3>{\listocaml[highlightlines=3]{../src/default/default.ml}{default/default.ml}}%
\only<2>{\listasm[firstline=66,lastline=72,highlightlines=69]{src/ocaml/caml\_start\_program.s}{caml\_start\_program}}%
\end{column}
\begin{column}{0.4\textwidth}
\centering
\providecommand\step{4}\input{diagrams/default/default.tex}
\end{column}
\end{columns}
\end{frame}
%\only<>{\listasm[firstline=47,lastline=65]{src/ocaml/caml\_start\_program.s}{caml\_start\_program}}


\begin{frame}{Back to C}{Clean up, complain and terminate}
\begin{columns}
\begin{column}{0.45\textwidth}
\begin{itemize}
\item Backtrace:
\begin{itemize}
\item \funcname{caml\_start\_program} $\rightarrow$ OCaml code
\item \funcname{caml\_startup\_common}
\item \funcname{caml\_startup\_exn}
\item \funcname{caml\_startup}
\item \funcname{caml\_main}
\item \funcname{main}
\end{itemize}
\smallskip
\item \funcname{caml\_startup} checks the return value of \funcname{caml\_startup\_exn}
\begin{itemize}
\item The return value has been specially marked by \funcname{caml\_start\_program} handler
\item Calls \funcname{caml\_fatal\_uncaught\_exception}
\end{itemize}
\item<2-> \funcname{caml\_fatal\_uncaught\_exception} either
\begin{itemize}
\item<2-> Calls the user custom uncaught exception handler, if set with \mintinline{ocaml}{Printexc.set_uncaught_exception_handler fn}\footnotemark
\item<2-> Or falls back to the default one
\item<2-> Which notifies about the uncaught exception
\end{itemize}
\item<4-> Terminates the program either with \funcname{abort} or with a non-zero exit code
\end{itemize}
\end{column}
\begin{column}{0.55\textwidth}
\only<1>{
\listc[firstline=84,lastline=89,highlightlines=88]{../ocaml/runtime/caml/mlvalues.h}{runtime/caml/mlvalues.h:84}
\listc[firstline=138,lastline=143,highlightlines={141-142}]{../ocaml/runtime/startup\_nat.c}{runtime/startup\_nat.c:138}
}%
\only<2>{
\listc[firstline=138,highlightlines={147,149}]{../ocaml/runtime/printexc.c}{runtime/printexc.c:138}
}%
\only<3>{
\listc[firstline=110,lastline=134,highlightlines=129]{../ocaml/runtime/printexc.c}{runtime/printexc.c:138}
}%
\only<4>{
\listc[firstline=138,highlightlines={152,154}]{../ocaml/runtime/printexc.c}{runtime/printexc.c:138}
}%
\end{column}
\end{columns}
\footnotetext[1]{\url{https://v2.ocaml.org/api/Printexc.html\#1\_Uncaughtexceptions}}
\end{frame}
}%
\only<2>{\providecommand\step{4}%
% Runtime's default exception handler
%
\subsection*{Runtime's default exception handler}
\frameSubsection{pictures/The\_Architect.png}{}


\begin{frame}{Default exception handler}
\begin{columns}
\begin{column}{0.5\textwidth}
\begin{itemize}
\item \localname{domain\_state->exn\_handler} points to a trap located before \funcname{entry}!
\item What installed this very first trap there?
\end{itemize}
\bigskip
\listocaml{../src/default/default.ml}{default/default.ml}
\bigskip
\inputminted[fontsize=\footnotesize]{text}{../src/default/default.out}
\end{column}
\begin{column}{0.5\textwidth}
\centering
\only<1>{\providecommand\step{0}\input{diagrams/default/default.tex}}%
\only<2>{\providecommand\step{4}\input{diagrams/default/default.tex}}%
\end{column}
\end{columns}
\end{frame}


\begin{frame}{Startup}{How did we get there by the way?}
\begin{columns}
\begin{column}{0.45\textwidth}
\begin{itemize}
\item The entry point address of the binary is \funcname{main} (\funcname{\_start}) from the runtime
\item The program starts like a C program:
\begin{itemize}
\item \funcname{caml\_start\_program}
\item \funcname{caml\_startup\_common}
\item \funcname{caml\_startup\_exn}
\item \funcname{caml\_startup}
\item \funcname{caml\_main}
\item \funcname{main}
\end{itemize}
%\item \funcname{caml\_startup\_common}: initializes GC, domains \dots
% Also: codefrag, locale, custom opeartions, frame_descr, signals, stack
\item \funcname{caml\_start\_program} is the transition point between the C realm and the OCaml one
\end{itemize}
\bigskip
\listocaml{../src/default/default.ml}{default/default.ml}
\end{column}
\begin{column}{0.55\textwidth}
\listasm[lastline=24,highlightlines={22-24}]{src/ocaml/caml\_start\_program.s}{runtime/amd64.S:604}
\end{column}
\end{columns}
\end{frame}


\begin{frame}{Setup to jump into OCaml entry point}{\funcname{caml\_start\_program} initializes the main fiber}
\begin{columns}
\begin{column}{0.6\textwidth}
\only<1,3,5>{
\begin{itemize}
\item<1-> Stores information to allow switching from OCaml stack to the C stack
\item<3-> Constructs the default exception handler
\item<5-> Switches to the OCaml stack and calls \funcname{caml\_program}
\end{itemize}
\bigskip
\listocaml{../src/default/default.ml}{default/default.ml}
}%
\only<2>{
\listasm[firstline=22,lastline=41,highlightlines={25-27}]{src/ocaml/caml\_start\_program.s}{caml\_start\_program}
}%
\only<4>{
\mintasm[firstline=22,lastline=41,highlightlines={31-38}]{src/ocaml/caml\_start\_program.s}
\listasm[firstline=66,lastline=72]{src/ocaml/caml\_start\_program.s}{caml\_start\_program}
}%
\only<6>{
\listasm[firstline=22,lastline=41,highlightlines={39-41}]{src/ocaml/caml\_start\_program.s}{caml\_start\_program}
}%
\end{column}
\begin{column}{0.4\textwidth}
\centering
\only<1-2>{\providecommand\step{2}\input{diagrams/default/default.tex}}%
\only<3-5>{\providecommand\step{3}\input{diagrams/default/default.tex}}%
\onslide*<6->{\providecommand\step{4}\input{diagrams/default/default.tex}}%
\end{column}
\end{columns}
\end{frame}

\begin{frame}{Executing the OCaml program}{And raising an exception without handler}
\begin{columns}
\begin{column}{0.6\textwidth}
\only<1-3>{
\begin{itemize}
\item<1-> Execution continues with OCaml code
\begin{itemize}
\item \funcname{camlDefault\_\_main\_269}
\item \funcname{camlDefault\_\_entry}
\item \funcname{caml\_program}
\end{itemize}
\item<1-> \funcname{camlDefault\_\_main\_269} raises a \typename{Failure} exception
\item<1-> \typename{Failure} is caught by \funcname{caml\_start\_program}
\begin{itemize}
\item<1-> The handler set the 2nd LSB of the exception value to \funcarg{1}
\item<3-> And then forces \funcname{caml\_start\_program} to terminate
\end{itemize}
\end{itemize}
}
\bigskip
\only<1,3>{\listocaml[highlightlines=3]{../src/default/default.ml}{default/default.ml}}%
\only<2>{\listasm[firstline=66,lastline=72,highlightlines=69]{src/ocaml/caml\_start\_program.s}{caml\_start\_program}}%
\end{column}
\begin{column}{0.4\textwidth}
\centering
\providecommand\step{4}\input{diagrams/default/default.tex}
\end{column}
\end{columns}
\end{frame}
%\only<>{\listasm[firstline=47,lastline=65]{src/ocaml/caml\_start\_program.s}{caml\_start\_program}}


\begin{frame}{Back to C}{Clean up, complain and terminate}
\begin{columns}
\begin{column}{0.45\textwidth}
\begin{itemize}
\item Backtrace:
\begin{itemize}
\item \funcname{caml\_start\_program} $\rightarrow$ OCaml code
\item \funcname{caml\_startup\_common}
\item \funcname{caml\_startup\_exn}
\item \funcname{caml\_startup}
\item \funcname{caml\_main}
\item \funcname{main}
\end{itemize}
\smallskip
\item \funcname{caml\_startup} checks the return value of \funcname{caml\_startup\_exn}
\begin{itemize}
\item The return value has been specially marked by \funcname{caml\_start\_program} handler
\item Calls \funcname{caml\_fatal\_uncaught\_exception}
\end{itemize}
\item<2-> \funcname{caml\_fatal\_uncaught\_exception} either
\begin{itemize}
\item<2-> Calls the user custom uncaught exception handler, if set with \mintinline{ocaml}{Printexc.set_uncaught_exception_handler fn}\footnotemark
\item<2-> Or falls back to the default one
\item<2-> Which notifies about the uncaught exception
\end{itemize}
\item<4-> Terminates the program either with \funcname{abort} or with a non-zero exit code
\end{itemize}
\end{column}
\begin{column}{0.55\textwidth}
\only<1>{
\listc[firstline=84,lastline=89,highlightlines=88]{../ocaml/runtime/caml/mlvalues.h}{runtime/caml/mlvalues.h:84}
\listc[firstline=138,lastline=143,highlightlines={141-142}]{../ocaml/runtime/startup\_nat.c}{runtime/startup\_nat.c:138}
}%
\only<2>{
\listc[firstline=138,highlightlines={147,149}]{../ocaml/runtime/printexc.c}{runtime/printexc.c:138}
}%
\only<3>{
\listc[firstline=110,lastline=134,highlightlines=129]{../ocaml/runtime/printexc.c}{runtime/printexc.c:138}
}%
\only<4>{
\listc[firstline=138,highlightlines={152,154}]{../ocaml/runtime/printexc.c}{runtime/printexc.c:138}
}%
\end{column}
\end{columns}
\footnotetext[1]{\url{https://v2.ocaml.org/api/Printexc.html\#1\_Uncaughtexceptions}}
\end{frame}
}%
\end{column}
\end{columns}
\end{frame}


\begin{frame}{Startup}{How did we get there by the way?}
\begin{columns}
\begin{column}{0.45\textwidth}
\begin{itemize}
\item The entry point address of the binary is \funcname{main} (\funcname{\_start}) from the runtime
\item The program starts like a C program:
\begin{itemize}
\item \funcname{caml\_start\_program}
\item \funcname{caml\_startup\_common}
\item \funcname{caml\_startup\_exn}
\item \funcname{caml\_startup}
\item \funcname{caml\_main}
\item \funcname{main}
\end{itemize}
%\item \funcname{caml\_startup\_common}: initializes GC, domains \dots
% Also: codefrag, locale, custom opeartions, frame_descr, signals, stack
\item \funcname{caml\_start\_program} is the transition point between the C realm and the OCaml one
\end{itemize}
\bigskip
\listocaml{../src/default/default.ml}{default/default.ml}
\end{column}
\begin{column}{0.55\textwidth}
\listasm[lastline=24,highlightlines={22-24}]{src/ocaml/caml\_start\_program.s}{runtime/amd64.S:604}
\end{column}
\end{columns}
\end{frame}


\begin{frame}{Setup to jump into OCaml entry point}{\funcname{caml\_start\_program} initializes the main fiber}
\begin{columns}
\begin{column}{0.6\textwidth}
\only<1,3,5>{
\begin{itemize}
\item<1-> Stores information to allow switching from OCaml stack to the C stack
\item<3-> Constructs the default exception handler
\item<5-> Switches to the OCaml stack and calls \funcname{caml\_program}
\end{itemize}
\bigskip
\listocaml{../src/default/default.ml}{default/default.ml}
}%
\only<2>{
\listasm[firstline=22,lastline=41,highlightlines={25-27}]{src/ocaml/caml\_start\_program.s}{caml\_start\_program}
}%
\only<4>{
\mintasm[firstline=22,lastline=41,highlightlines={31-38}]{src/ocaml/caml\_start\_program.s}
\listasm[firstline=66,lastline=72]{src/ocaml/caml\_start\_program.s}{caml\_start\_program}
}%
\only<6>{
\listasm[firstline=22,lastline=41,highlightlines={39-41}]{src/ocaml/caml\_start\_program.s}{caml\_start\_program}
}%
\end{column}
\begin{column}{0.4\textwidth}
\centering
\only<1-2>{\providecommand\step{2}%
% Runtime's default exception handler
%
\subsection*{Runtime's default exception handler}
\frameSubsection{pictures/The\_Architect.png}{}


\begin{frame}{Default exception handler}
\begin{columns}
\begin{column}{0.5\textwidth}
\begin{itemize}
\item \localname{domain\_state->exn\_handler} points to a trap located before \funcname{entry}!
\item What installed this very first trap there?
\end{itemize}
\bigskip
\listocaml{../src/default/default.ml}{default/default.ml}
\bigskip
\inputminted[fontsize=\footnotesize]{text}{../src/default/default.out}
\end{column}
\begin{column}{0.5\textwidth}
\centering
\only<1>{\providecommand\step{0}\input{diagrams/default/default.tex}}%
\only<2>{\providecommand\step{4}\input{diagrams/default/default.tex}}%
\end{column}
\end{columns}
\end{frame}


\begin{frame}{Startup}{How did we get there by the way?}
\begin{columns}
\begin{column}{0.45\textwidth}
\begin{itemize}
\item The entry point address of the binary is \funcname{main} (\funcname{\_start}) from the runtime
\item The program starts like a C program:
\begin{itemize}
\item \funcname{caml\_start\_program}
\item \funcname{caml\_startup\_common}
\item \funcname{caml\_startup\_exn}
\item \funcname{caml\_startup}
\item \funcname{caml\_main}
\item \funcname{main}
\end{itemize}
%\item \funcname{caml\_startup\_common}: initializes GC, domains \dots
% Also: codefrag, locale, custom opeartions, frame_descr, signals, stack
\item \funcname{caml\_start\_program} is the transition point between the C realm and the OCaml one
\end{itemize}
\bigskip
\listocaml{../src/default/default.ml}{default/default.ml}
\end{column}
\begin{column}{0.55\textwidth}
\listasm[lastline=24,highlightlines={22-24}]{src/ocaml/caml\_start\_program.s}{runtime/amd64.S:604}
\end{column}
\end{columns}
\end{frame}


\begin{frame}{Setup to jump into OCaml entry point}{\funcname{caml\_start\_program} initializes the main fiber}
\begin{columns}
\begin{column}{0.6\textwidth}
\only<1,3,5>{
\begin{itemize}
\item<1-> Stores information to allow switching from OCaml stack to the C stack
\item<3-> Constructs the default exception handler
\item<5-> Switches to the OCaml stack and calls \funcname{caml\_program}
\end{itemize}
\bigskip
\listocaml{../src/default/default.ml}{default/default.ml}
}%
\only<2>{
\listasm[firstline=22,lastline=41,highlightlines={25-27}]{src/ocaml/caml\_start\_program.s}{caml\_start\_program}
}%
\only<4>{
\mintasm[firstline=22,lastline=41,highlightlines={31-38}]{src/ocaml/caml\_start\_program.s}
\listasm[firstline=66,lastline=72]{src/ocaml/caml\_start\_program.s}{caml\_start\_program}
}%
\only<6>{
\listasm[firstline=22,lastline=41,highlightlines={39-41}]{src/ocaml/caml\_start\_program.s}{caml\_start\_program}
}%
\end{column}
\begin{column}{0.4\textwidth}
\centering
\only<1-2>{\providecommand\step{2}\input{diagrams/default/default.tex}}%
\only<3-5>{\providecommand\step{3}\input{diagrams/default/default.tex}}%
\onslide*<6->{\providecommand\step{4}\input{diagrams/default/default.tex}}%
\end{column}
\end{columns}
\end{frame}

\begin{frame}{Executing the OCaml program}{And raising an exception without handler}
\begin{columns}
\begin{column}{0.6\textwidth}
\only<1-3>{
\begin{itemize}
\item<1-> Execution continues with OCaml code
\begin{itemize}
\item \funcname{camlDefault\_\_main\_269}
\item \funcname{camlDefault\_\_entry}
\item \funcname{caml\_program}
\end{itemize}
\item<1-> \funcname{camlDefault\_\_main\_269} raises a \typename{Failure} exception
\item<1-> \typename{Failure} is caught by \funcname{caml\_start\_program}
\begin{itemize}
\item<1-> The handler set the 2nd LSB of the exception value to \funcarg{1}
\item<3-> And then forces \funcname{caml\_start\_program} to terminate
\end{itemize}
\end{itemize}
}
\bigskip
\only<1,3>{\listocaml[highlightlines=3]{../src/default/default.ml}{default/default.ml}}%
\only<2>{\listasm[firstline=66,lastline=72,highlightlines=69]{src/ocaml/caml\_start\_program.s}{caml\_start\_program}}%
\end{column}
\begin{column}{0.4\textwidth}
\centering
\providecommand\step{4}\input{diagrams/default/default.tex}
\end{column}
\end{columns}
\end{frame}
%\only<>{\listasm[firstline=47,lastline=65]{src/ocaml/caml\_start\_program.s}{caml\_start\_program}}


\begin{frame}{Back to C}{Clean up, complain and terminate}
\begin{columns}
\begin{column}{0.45\textwidth}
\begin{itemize}
\item Backtrace:
\begin{itemize}
\item \funcname{caml\_start\_program} $\rightarrow$ OCaml code
\item \funcname{caml\_startup\_common}
\item \funcname{caml\_startup\_exn}
\item \funcname{caml\_startup}
\item \funcname{caml\_main}
\item \funcname{main}
\end{itemize}
\smallskip
\item \funcname{caml\_startup} checks the return value of \funcname{caml\_startup\_exn}
\begin{itemize}
\item The return value has been specially marked by \funcname{caml\_start\_program} handler
\item Calls \funcname{caml\_fatal\_uncaught\_exception}
\end{itemize}
\item<2-> \funcname{caml\_fatal\_uncaught\_exception} either
\begin{itemize}
\item<2-> Calls the user custom uncaught exception handler, if set with \mintinline{ocaml}{Printexc.set_uncaught_exception_handler fn}\footnotemark
\item<2-> Or falls back to the default one
\item<2-> Which notifies about the uncaught exception
\end{itemize}
\item<4-> Terminates the program either with \funcname{abort} or with a non-zero exit code
\end{itemize}
\end{column}
\begin{column}{0.55\textwidth}
\only<1>{
\listc[firstline=84,lastline=89,highlightlines=88]{../ocaml/runtime/caml/mlvalues.h}{runtime/caml/mlvalues.h:84}
\listc[firstline=138,lastline=143,highlightlines={141-142}]{../ocaml/runtime/startup\_nat.c}{runtime/startup\_nat.c:138}
}%
\only<2>{
\listc[firstline=138,highlightlines={147,149}]{../ocaml/runtime/printexc.c}{runtime/printexc.c:138}
}%
\only<3>{
\listc[firstline=110,lastline=134,highlightlines=129]{../ocaml/runtime/printexc.c}{runtime/printexc.c:138}
}%
\only<4>{
\listc[firstline=138,highlightlines={152,154}]{../ocaml/runtime/printexc.c}{runtime/printexc.c:138}
}%
\end{column}
\end{columns}
\footnotetext[1]{\url{https://v2.ocaml.org/api/Printexc.html\#1\_Uncaughtexceptions}}
\end{frame}
}%
\only<3-5>{\providecommand\step{3}%
% Runtime's default exception handler
%
\subsection*{Runtime's default exception handler}
\frameSubsection{pictures/The\_Architect.png}{}


\begin{frame}{Default exception handler}
\begin{columns}
\begin{column}{0.5\textwidth}
\begin{itemize}
\item \localname{domain\_state->exn\_handler} points to a trap located before \funcname{entry}!
\item What installed this very first trap there?
\end{itemize}
\bigskip
\listocaml{../src/default/default.ml}{default/default.ml}
\bigskip
\inputminted[fontsize=\footnotesize]{text}{../src/default/default.out}
\end{column}
\begin{column}{0.5\textwidth}
\centering
\only<1>{\providecommand\step{0}\input{diagrams/default/default.tex}}%
\only<2>{\providecommand\step{4}\input{diagrams/default/default.tex}}%
\end{column}
\end{columns}
\end{frame}


\begin{frame}{Startup}{How did we get there by the way?}
\begin{columns}
\begin{column}{0.45\textwidth}
\begin{itemize}
\item The entry point address of the binary is \funcname{main} (\funcname{\_start}) from the runtime
\item The program starts like a C program:
\begin{itemize}
\item \funcname{caml\_start\_program}
\item \funcname{caml\_startup\_common}
\item \funcname{caml\_startup\_exn}
\item \funcname{caml\_startup}
\item \funcname{caml\_main}
\item \funcname{main}
\end{itemize}
%\item \funcname{caml\_startup\_common}: initializes GC, domains \dots
% Also: codefrag, locale, custom opeartions, frame_descr, signals, stack
\item \funcname{caml\_start\_program} is the transition point between the C realm and the OCaml one
\end{itemize}
\bigskip
\listocaml{../src/default/default.ml}{default/default.ml}
\end{column}
\begin{column}{0.55\textwidth}
\listasm[lastline=24,highlightlines={22-24}]{src/ocaml/caml\_start\_program.s}{runtime/amd64.S:604}
\end{column}
\end{columns}
\end{frame}


\begin{frame}{Setup to jump into OCaml entry point}{\funcname{caml\_start\_program} initializes the main fiber}
\begin{columns}
\begin{column}{0.6\textwidth}
\only<1,3,5>{
\begin{itemize}
\item<1-> Stores information to allow switching from OCaml stack to the C stack
\item<3-> Constructs the default exception handler
\item<5-> Switches to the OCaml stack and calls \funcname{caml\_program}
\end{itemize}
\bigskip
\listocaml{../src/default/default.ml}{default/default.ml}
}%
\only<2>{
\listasm[firstline=22,lastline=41,highlightlines={25-27}]{src/ocaml/caml\_start\_program.s}{caml\_start\_program}
}%
\only<4>{
\mintasm[firstline=22,lastline=41,highlightlines={31-38}]{src/ocaml/caml\_start\_program.s}
\listasm[firstline=66,lastline=72]{src/ocaml/caml\_start\_program.s}{caml\_start\_program}
}%
\only<6>{
\listasm[firstline=22,lastline=41,highlightlines={39-41}]{src/ocaml/caml\_start\_program.s}{caml\_start\_program}
}%
\end{column}
\begin{column}{0.4\textwidth}
\centering
\only<1-2>{\providecommand\step{2}\input{diagrams/default/default.tex}}%
\only<3-5>{\providecommand\step{3}\input{diagrams/default/default.tex}}%
\onslide*<6->{\providecommand\step{4}\input{diagrams/default/default.tex}}%
\end{column}
\end{columns}
\end{frame}

\begin{frame}{Executing the OCaml program}{And raising an exception without handler}
\begin{columns}
\begin{column}{0.6\textwidth}
\only<1-3>{
\begin{itemize}
\item<1-> Execution continues with OCaml code
\begin{itemize}
\item \funcname{camlDefault\_\_main\_269}
\item \funcname{camlDefault\_\_entry}
\item \funcname{caml\_program}
\end{itemize}
\item<1-> \funcname{camlDefault\_\_main\_269} raises a \typename{Failure} exception
\item<1-> \typename{Failure} is caught by \funcname{caml\_start\_program}
\begin{itemize}
\item<1-> The handler set the 2nd LSB of the exception value to \funcarg{1}
\item<3-> And then forces \funcname{caml\_start\_program} to terminate
\end{itemize}
\end{itemize}
}
\bigskip
\only<1,3>{\listocaml[highlightlines=3]{../src/default/default.ml}{default/default.ml}}%
\only<2>{\listasm[firstline=66,lastline=72,highlightlines=69]{src/ocaml/caml\_start\_program.s}{caml\_start\_program}}%
\end{column}
\begin{column}{0.4\textwidth}
\centering
\providecommand\step{4}\input{diagrams/default/default.tex}
\end{column}
\end{columns}
\end{frame}
%\only<>{\listasm[firstline=47,lastline=65]{src/ocaml/caml\_start\_program.s}{caml\_start\_program}}


\begin{frame}{Back to C}{Clean up, complain and terminate}
\begin{columns}
\begin{column}{0.45\textwidth}
\begin{itemize}
\item Backtrace:
\begin{itemize}
\item \funcname{caml\_start\_program} $\rightarrow$ OCaml code
\item \funcname{caml\_startup\_common}
\item \funcname{caml\_startup\_exn}
\item \funcname{caml\_startup}
\item \funcname{caml\_main}
\item \funcname{main}
\end{itemize}
\smallskip
\item \funcname{caml\_startup} checks the return value of \funcname{caml\_startup\_exn}
\begin{itemize}
\item The return value has been specially marked by \funcname{caml\_start\_program} handler
\item Calls \funcname{caml\_fatal\_uncaught\_exception}
\end{itemize}
\item<2-> \funcname{caml\_fatal\_uncaught\_exception} either
\begin{itemize}
\item<2-> Calls the user custom uncaught exception handler, if set with \mintinline{ocaml}{Printexc.set_uncaught_exception_handler fn}\footnotemark
\item<2-> Or falls back to the default one
\item<2-> Which notifies about the uncaught exception
\end{itemize}
\item<4-> Terminates the program either with \funcname{abort} or with a non-zero exit code
\end{itemize}
\end{column}
\begin{column}{0.55\textwidth}
\only<1>{
\listc[firstline=84,lastline=89,highlightlines=88]{../ocaml/runtime/caml/mlvalues.h}{runtime/caml/mlvalues.h:84}
\listc[firstline=138,lastline=143,highlightlines={141-142}]{../ocaml/runtime/startup\_nat.c}{runtime/startup\_nat.c:138}
}%
\only<2>{
\listc[firstline=138,highlightlines={147,149}]{../ocaml/runtime/printexc.c}{runtime/printexc.c:138}
}%
\only<3>{
\listc[firstline=110,lastline=134,highlightlines=129]{../ocaml/runtime/printexc.c}{runtime/printexc.c:138}
}%
\only<4>{
\listc[firstline=138,highlightlines={152,154}]{../ocaml/runtime/printexc.c}{runtime/printexc.c:138}
}%
\end{column}
\end{columns}
\footnotetext[1]{\url{https://v2.ocaml.org/api/Printexc.html\#1\_Uncaughtexceptions}}
\end{frame}
}%
\onslide*<6->{\providecommand\step{4}%
% Runtime's default exception handler
%
\subsection*{Runtime's default exception handler}
\frameSubsection{pictures/The\_Architect.png}{}


\begin{frame}{Default exception handler}
\begin{columns}
\begin{column}{0.5\textwidth}
\begin{itemize}
\item \localname{domain\_state->exn\_handler} points to a trap located before \funcname{entry}!
\item What installed this very first trap there?
\end{itemize}
\bigskip
\listocaml{../src/default/default.ml}{default/default.ml}
\bigskip
\inputminted[fontsize=\footnotesize]{text}{../src/default/default.out}
\end{column}
\begin{column}{0.5\textwidth}
\centering
\only<1>{\providecommand\step{0}\input{diagrams/default/default.tex}}%
\only<2>{\providecommand\step{4}\input{diagrams/default/default.tex}}%
\end{column}
\end{columns}
\end{frame}


\begin{frame}{Startup}{How did we get there by the way?}
\begin{columns}
\begin{column}{0.45\textwidth}
\begin{itemize}
\item The entry point address of the binary is \funcname{main} (\funcname{\_start}) from the runtime
\item The program starts like a C program:
\begin{itemize}
\item \funcname{caml\_start\_program}
\item \funcname{caml\_startup\_common}
\item \funcname{caml\_startup\_exn}
\item \funcname{caml\_startup}
\item \funcname{caml\_main}
\item \funcname{main}
\end{itemize}
%\item \funcname{caml\_startup\_common}: initializes GC, domains \dots
% Also: codefrag, locale, custom opeartions, frame_descr, signals, stack
\item \funcname{caml\_start\_program} is the transition point between the C realm and the OCaml one
\end{itemize}
\bigskip
\listocaml{../src/default/default.ml}{default/default.ml}
\end{column}
\begin{column}{0.55\textwidth}
\listasm[lastline=24,highlightlines={22-24}]{src/ocaml/caml\_start\_program.s}{runtime/amd64.S:604}
\end{column}
\end{columns}
\end{frame}


\begin{frame}{Setup to jump into OCaml entry point}{\funcname{caml\_start\_program} initializes the main fiber}
\begin{columns}
\begin{column}{0.6\textwidth}
\only<1,3,5>{
\begin{itemize}
\item<1-> Stores information to allow switching from OCaml stack to the C stack
\item<3-> Constructs the default exception handler
\item<5-> Switches to the OCaml stack and calls \funcname{caml\_program}
\end{itemize}
\bigskip
\listocaml{../src/default/default.ml}{default/default.ml}
}%
\only<2>{
\listasm[firstline=22,lastline=41,highlightlines={25-27}]{src/ocaml/caml\_start\_program.s}{caml\_start\_program}
}%
\only<4>{
\mintasm[firstline=22,lastline=41,highlightlines={31-38}]{src/ocaml/caml\_start\_program.s}
\listasm[firstline=66,lastline=72]{src/ocaml/caml\_start\_program.s}{caml\_start\_program}
}%
\only<6>{
\listasm[firstline=22,lastline=41,highlightlines={39-41}]{src/ocaml/caml\_start\_program.s}{caml\_start\_program}
}%
\end{column}
\begin{column}{0.4\textwidth}
\centering
\only<1-2>{\providecommand\step{2}\input{diagrams/default/default.tex}}%
\only<3-5>{\providecommand\step{3}\input{diagrams/default/default.tex}}%
\onslide*<6->{\providecommand\step{4}\input{diagrams/default/default.tex}}%
\end{column}
\end{columns}
\end{frame}

\begin{frame}{Executing the OCaml program}{And raising an exception without handler}
\begin{columns}
\begin{column}{0.6\textwidth}
\only<1-3>{
\begin{itemize}
\item<1-> Execution continues with OCaml code
\begin{itemize}
\item \funcname{camlDefault\_\_main\_269}
\item \funcname{camlDefault\_\_entry}
\item \funcname{caml\_program}
\end{itemize}
\item<1-> \funcname{camlDefault\_\_main\_269} raises a \typename{Failure} exception
\item<1-> \typename{Failure} is caught by \funcname{caml\_start\_program}
\begin{itemize}
\item<1-> The handler set the 2nd LSB of the exception value to \funcarg{1}
\item<3-> And then forces \funcname{caml\_start\_program} to terminate
\end{itemize}
\end{itemize}
}
\bigskip
\only<1,3>{\listocaml[highlightlines=3]{../src/default/default.ml}{default/default.ml}}%
\only<2>{\listasm[firstline=66,lastline=72,highlightlines=69]{src/ocaml/caml\_start\_program.s}{caml\_start\_program}}%
\end{column}
\begin{column}{0.4\textwidth}
\centering
\providecommand\step{4}\input{diagrams/default/default.tex}
\end{column}
\end{columns}
\end{frame}
%\only<>{\listasm[firstline=47,lastline=65]{src/ocaml/caml\_start\_program.s}{caml\_start\_program}}


\begin{frame}{Back to C}{Clean up, complain and terminate}
\begin{columns}
\begin{column}{0.45\textwidth}
\begin{itemize}
\item Backtrace:
\begin{itemize}
\item \funcname{caml\_start\_program} $\rightarrow$ OCaml code
\item \funcname{caml\_startup\_common}
\item \funcname{caml\_startup\_exn}
\item \funcname{caml\_startup}
\item \funcname{caml\_main}
\item \funcname{main}
\end{itemize}
\smallskip
\item \funcname{caml\_startup} checks the return value of \funcname{caml\_startup\_exn}
\begin{itemize}
\item The return value has been specially marked by \funcname{caml\_start\_program} handler
\item Calls \funcname{caml\_fatal\_uncaught\_exception}
\end{itemize}
\item<2-> \funcname{caml\_fatal\_uncaught\_exception} either
\begin{itemize}
\item<2-> Calls the user custom uncaught exception handler, if set with \mintinline{ocaml}{Printexc.set_uncaught_exception_handler fn}\footnotemark
\item<2-> Or falls back to the default one
\item<2-> Which notifies about the uncaught exception
\end{itemize}
\item<4-> Terminates the program either with \funcname{abort} or with a non-zero exit code
\end{itemize}
\end{column}
\begin{column}{0.55\textwidth}
\only<1>{
\listc[firstline=84,lastline=89,highlightlines=88]{../ocaml/runtime/caml/mlvalues.h}{runtime/caml/mlvalues.h:84}
\listc[firstline=138,lastline=143,highlightlines={141-142}]{../ocaml/runtime/startup\_nat.c}{runtime/startup\_nat.c:138}
}%
\only<2>{
\listc[firstline=138,highlightlines={147,149}]{../ocaml/runtime/printexc.c}{runtime/printexc.c:138}
}%
\only<3>{
\listc[firstline=110,lastline=134,highlightlines=129]{../ocaml/runtime/printexc.c}{runtime/printexc.c:138}
}%
\only<4>{
\listc[firstline=138,highlightlines={152,154}]{../ocaml/runtime/printexc.c}{runtime/printexc.c:138}
}%
\end{column}
\end{columns}
\footnotetext[1]{\url{https://v2.ocaml.org/api/Printexc.html\#1\_Uncaughtexceptions}}
\end{frame}
}%
\end{column}
\end{columns}
\end{frame}

\begin{frame}{Executing the OCaml program}{And raising an exception without handler}
\begin{columns}
\begin{column}{0.6\textwidth}
\only<1-3>{
\begin{itemize}
\item<1-> Execution continues with OCaml code
\begin{itemize}
\item \funcname{camlDefault\_\_main\_269}
\item \funcname{camlDefault\_\_entry}
\item \funcname{caml\_program}
\end{itemize}
\item<1-> \funcname{camlDefault\_\_main\_269} raises a \typename{Failure} exception
\item<1-> \typename{Failure} is caught by \funcname{caml\_start\_program}
\begin{itemize}
\item<1-> The handler set the 2nd LSB of the exception value to \funcarg{1}
\item<3-> And then forces \funcname{caml\_start\_program} to terminate
\end{itemize}
\end{itemize}
}
\bigskip
\only<1,3>{\listocaml[highlightlines=3]{../src/default/default.ml}{default/default.ml}}%
\only<2>{\listasm[firstline=66,lastline=72,highlightlines=69]{src/ocaml/caml\_start\_program.s}{caml\_start\_program}}%
\end{column}
\begin{column}{0.4\textwidth}
\centering
\providecommand\step{4}%
% Runtime's default exception handler
%
\subsection*{Runtime's default exception handler}
\frameSubsection{pictures/The\_Architect.png}{}


\begin{frame}{Default exception handler}
\begin{columns}
\begin{column}{0.5\textwidth}
\begin{itemize}
\item \localname{domain\_state->exn\_handler} points to a trap located before \funcname{entry}!
\item What installed this very first trap there?
\end{itemize}
\bigskip
\listocaml{../src/default/default.ml}{default/default.ml}
\bigskip
\inputminted[fontsize=\footnotesize]{text}{../src/default/default.out}
\end{column}
\begin{column}{0.5\textwidth}
\centering
\only<1>{\providecommand\step{0}\input{diagrams/default/default.tex}}%
\only<2>{\providecommand\step{4}\input{diagrams/default/default.tex}}%
\end{column}
\end{columns}
\end{frame}


\begin{frame}{Startup}{How did we get there by the way?}
\begin{columns}
\begin{column}{0.45\textwidth}
\begin{itemize}
\item The entry point address of the binary is \funcname{main} (\funcname{\_start}) from the runtime
\item The program starts like a C program:
\begin{itemize}
\item \funcname{caml\_start\_program}
\item \funcname{caml\_startup\_common}
\item \funcname{caml\_startup\_exn}
\item \funcname{caml\_startup}
\item \funcname{caml\_main}
\item \funcname{main}
\end{itemize}
%\item \funcname{caml\_startup\_common}: initializes GC, domains \dots
% Also: codefrag, locale, custom opeartions, frame_descr, signals, stack
\item \funcname{caml\_start\_program} is the transition point between the C realm and the OCaml one
\end{itemize}
\bigskip
\listocaml{../src/default/default.ml}{default/default.ml}
\end{column}
\begin{column}{0.55\textwidth}
\listasm[lastline=24,highlightlines={22-24}]{src/ocaml/caml\_start\_program.s}{runtime/amd64.S:604}
\end{column}
\end{columns}
\end{frame}


\begin{frame}{Setup to jump into OCaml entry point}{\funcname{caml\_start\_program} initializes the main fiber}
\begin{columns}
\begin{column}{0.6\textwidth}
\only<1,3,5>{
\begin{itemize}
\item<1-> Stores information to allow switching from OCaml stack to the C stack
\item<3-> Constructs the default exception handler
\item<5-> Switches to the OCaml stack and calls \funcname{caml\_program}
\end{itemize}
\bigskip
\listocaml{../src/default/default.ml}{default/default.ml}
}%
\only<2>{
\listasm[firstline=22,lastline=41,highlightlines={25-27}]{src/ocaml/caml\_start\_program.s}{caml\_start\_program}
}%
\only<4>{
\mintasm[firstline=22,lastline=41,highlightlines={31-38}]{src/ocaml/caml\_start\_program.s}
\listasm[firstline=66,lastline=72]{src/ocaml/caml\_start\_program.s}{caml\_start\_program}
}%
\only<6>{
\listasm[firstline=22,lastline=41,highlightlines={39-41}]{src/ocaml/caml\_start\_program.s}{caml\_start\_program}
}%
\end{column}
\begin{column}{0.4\textwidth}
\centering
\only<1-2>{\providecommand\step{2}\input{diagrams/default/default.tex}}%
\only<3-5>{\providecommand\step{3}\input{diagrams/default/default.tex}}%
\onslide*<6->{\providecommand\step{4}\input{diagrams/default/default.tex}}%
\end{column}
\end{columns}
\end{frame}

\begin{frame}{Executing the OCaml program}{And raising an exception without handler}
\begin{columns}
\begin{column}{0.6\textwidth}
\only<1-3>{
\begin{itemize}
\item<1-> Execution continues with OCaml code
\begin{itemize}
\item \funcname{camlDefault\_\_main\_269}
\item \funcname{camlDefault\_\_entry}
\item \funcname{caml\_program}
\end{itemize}
\item<1-> \funcname{camlDefault\_\_main\_269} raises a \typename{Failure} exception
\item<1-> \typename{Failure} is caught by \funcname{caml\_start\_program}
\begin{itemize}
\item<1-> The handler set the 2nd LSB of the exception value to \funcarg{1}
\item<3-> And then forces \funcname{caml\_start\_program} to terminate
\end{itemize}
\end{itemize}
}
\bigskip
\only<1,3>{\listocaml[highlightlines=3]{../src/default/default.ml}{default/default.ml}}%
\only<2>{\listasm[firstline=66,lastline=72,highlightlines=69]{src/ocaml/caml\_start\_program.s}{caml\_start\_program}}%
\end{column}
\begin{column}{0.4\textwidth}
\centering
\providecommand\step{4}\input{diagrams/default/default.tex}
\end{column}
\end{columns}
\end{frame}
%\only<>{\listasm[firstline=47,lastline=65]{src/ocaml/caml\_start\_program.s}{caml\_start\_program}}


\begin{frame}{Back to C}{Clean up, complain and terminate}
\begin{columns}
\begin{column}{0.45\textwidth}
\begin{itemize}
\item Backtrace:
\begin{itemize}
\item \funcname{caml\_start\_program} $\rightarrow$ OCaml code
\item \funcname{caml\_startup\_common}
\item \funcname{caml\_startup\_exn}
\item \funcname{caml\_startup}
\item \funcname{caml\_main}
\item \funcname{main}
\end{itemize}
\smallskip
\item \funcname{caml\_startup} checks the return value of \funcname{caml\_startup\_exn}
\begin{itemize}
\item The return value has been specially marked by \funcname{caml\_start\_program} handler
\item Calls \funcname{caml\_fatal\_uncaught\_exception}
\end{itemize}
\item<2-> \funcname{caml\_fatal\_uncaught\_exception} either
\begin{itemize}
\item<2-> Calls the user custom uncaught exception handler, if set with \mintinline{ocaml}{Printexc.set_uncaught_exception_handler fn}\footnotemark
\item<2-> Or falls back to the default one
\item<2-> Which notifies about the uncaught exception
\end{itemize}
\item<4-> Terminates the program either with \funcname{abort} or with a non-zero exit code
\end{itemize}
\end{column}
\begin{column}{0.55\textwidth}
\only<1>{
\listc[firstline=84,lastline=89,highlightlines=88]{../ocaml/runtime/caml/mlvalues.h}{runtime/caml/mlvalues.h:84}
\listc[firstline=138,lastline=143,highlightlines={141-142}]{../ocaml/runtime/startup\_nat.c}{runtime/startup\_nat.c:138}
}%
\only<2>{
\listc[firstline=138,highlightlines={147,149}]{../ocaml/runtime/printexc.c}{runtime/printexc.c:138}
}%
\only<3>{
\listc[firstline=110,lastline=134,highlightlines=129]{../ocaml/runtime/printexc.c}{runtime/printexc.c:138}
}%
\only<4>{
\listc[firstline=138,highlightlines={152,154}]{../ocaml/runtime/printexc.c}{runtime/printexc.c:138}
}%
\end{column}
\end{columns}
\footnotetext[1]{\url{https://v2.ocaml.org/api/Printexc.html\#1\_Uncaughtexceptions}}
\end{frame}

\end{column}
\end{columns}
\end{frame}
%\only<>{\listasm[firstline=47,lastline=65]{src/ocaml/caml\_start\_program.s}{caml\_start\_program}}


\begin{frame}{Back to C}{Clean up, complain and terminate}
\begin{columns}
\begin{column}{0.45\textwidth}
\begin{itemize}
\item Backtrace:
\begin{itemize}
\item \funcname{caml\_start\_program} $\rightarrow$ OCaml code
\item \funcname{caml\_startup\_common}
\item \funcname{caml\_startup\_exn}
\item \funcname{caml\_startup}
\item \funcname{caml\_main}
\item \funcname{main}
\end{itemize}
\smallskip
\item \funcname{caml\_startup} checks the return value of \funcname{caml\_startup\_exn}
\begin{itemize}
\item The return value has been specially marked by \funcname{caml\_start\_program} handler
\item Calls \funcname{caml\_fatal\_uncaught\_exception}
\end{itemize}
\item<2-> \funcname{caml\_fatal\_uncaught\_exception} either
\begin{itemize}
\item<2-> Calls the user custom uncaught exception handler, if set with \mintinline{ocaml}{Printexc.set_uncaught_exception_handler fn}\footnotemark
\item<2-> Or falls back to the default one
\item<2-> Which notifies about the uncaught exception
\end{itemize}
\item<4-> Terminates the program either with \funcname{abort} or with a non-zero exit code
\end{itemize}
\end{column}
\begin{column}{0.55\textwidth}
\only<1>{
\listc[firstline=84,lastline=89,highlightlines=88]{../ocaml/runtime/caml/mlvalues.h}{runtime/caml/mlvalues.h:84}
\listc[firstline=138,lastline=143,highlightlines={141-142}]{../ocaml/runtime/startup\_nat.c}{runtime/startup\_nat.c:138}
}%
\only<2>{
\listc[firstline=138,highlightlines={147,149}]{../ocaml/runtime/printexc.c}{runtime/printexc.c:138}
}%
\only<3>{
\listc[firstline=110,lastline=134,highlightlines=129]{../ocaml/runtime/printexc.c}{runtime/printexc.c:138}
}%
\only<4>{
\listc[firstline=138,highlightlines={152,154}]{../ocaml/runtime/printexc.c}{runtime/printexc.c:138}
}%
\end{column}
\end{columns}
\footnotetext[1]{\url{https://v2.ocaml.org/api/Printexc.html\#1\_Uncaughtexceptions}}
\end{frame}

\end{column}
\end{columns}
\end{frame}
%\only<>{\listasm[firstline=47,lastline=65]{src/ocaml/caml\_start\_program.s}{caml\_start\_program}}


\begin{frame}{Back to C}{Clean up, complain and terminate}
\begin{columns}
\begin{column}{0.45\textwidth}
\begin{itemize}
\item Backtrace:
\begin{itemize}
\item \funcname{caml\_start\_program} $\rightarrow$ OCaml code
\item \funcname{caml\_startup\_common}
\item \funcname{caml\_startup\_exn}
\item \funcname{caml\_startup}
\item \funcname{caml\_main}
\item \funcname{main}
\end{itemize}
\smallskip
\item \funcname{caml\_startup} checks the return value of \funcname{caml\_startup\_exn}
\begin{itemize}
\item The return value has been specially marked by \funcname{caml\_start\_program} handler
\item Calls \funcname{caml\_fatal\_uncaught\_exception}
\end{itemize}
\item<2-> \funcname{caml\_fatal\_uncaught\_exception} either
\begin{itemize}
\item<2-> Calls the user custom uncaught exception handler, if set with \mintinline{ocaml}{Printexc.set_uncaught_exception_handler fn}\footnotemark
\item<2-> Or falls back to the default one
\item<2-> Which notifies about the uncaught exception
\end{itemize}
\item<4-> Terminates the program either with \funcname{abort} or with a non-zero exit code
\end{itemize}
\end{column}
\begin{column}{0.55\textwidth}
\only<1>{
\listc[firstline=84,lastline=89,highlightlines=88]{../ocaml/runtime/caml/mlvalues.h}{runtime/caml/mlvalues.h:84}
\listc[firstline=138,lastline=143,highlightlines={141-142}]{../ocaml/runtime/startup\_nat.c}{runtime/startup\_nat.c:138}
}%
\only<2>{
\listc[firstline=138,highlightlines={147,149}]{../ocaml/runtime/printexc.c}{runtime/printexc.c:138}
}%
\only<3>{
\listc[firstline=110,lastline=134,highlightlines=129]{../ocaml/runtime/printexc.c}{runtime/printexc.c:138}
}%
\only<4>{
\listc[firstline=138,highlightlines={152,154}]{../ocaml/runtime/printexc.c}{runtime/printexc.c:138}
}%
\end{column}
\end{columns}
\footnotetext[1]{\url{https://v2.ocaml.org/api/Printexc.html\#1\_Uncaughtexceptions}}
\end{frame}
}%
\onslide*<6->{\providecommand\step{4}%
% Runtime's default exception handler
%
\subsection*{Runtime's default exception handler}
\frameSubsection{pictures/The\_Architect.png}{}


\begin{frame}{Default exception handler}
\begin{columns}
\begin{column}{0.5\textwidth}
\begin{itemize}
\item \localname{domain\_state->exn\_handler} points to a trap located before \funcname{entry}!
\item What installed this very first trap there?
\end{itemize}
\bigskip
\listocaml{../src/default/default.ml}{default/default.ml}
\bigskip
\inputminted[fontsize=\footnotesize]{text}{../src/default/default.out}
\end{column}
\begin{column}{0.5\textwidth}
\centering
\only<1>{\providecommand\step{0}%
% Runtime's default exception handler
%
\subsection*{Runtime's default exception handler}
\frameSubsection{pictures/The\_Architect.png}{}


\begin{frame}{Default exception handler}
\begin{columns}
\begin{column}{0.5\textwidth}
\begin{itemize}
\item \localname{domain\_state->exn\_handler} points to a trap located before \funcname{entry}!
\item What installed this very first trap there?
\end{itemize}
\bigskip
\listocaml{../src/default/default.ml}{default/default.ml}
\bigskip
\inputminted[fontsize=\footnotesize]{text}{../src/default/default.out}
\end{column}
\begin{column}{0.5\textwidth}
\centering
\only<1>{\providecommand\step{0}%
% Runtime's default exception handler
%
\subsection*{Runtime's default exception handler}
\frameSubsection{pictures/The\_Architect.png}{}


\begin{frame}{Default exception handler}
\begin{columns}
\begin{column}{0.5\textwidth}
\begin{itemize}
\item \localname{domain\_state->exn\_handler} points to a trap located before \funcname{entry}!
\item What installed this very first trap there?
\end{itemize}
\bigskip
\listocaml{../src/default/default.ml}{default/default.ml}
\bigskip
\inputminted[fontsize=\footnotesize]{text}{../src/default/default.out}
\end{column}
\begin{column}{0.5\textwidth}
\centering
\only<1>{\providecommand\step{0}\input{diagrams/default/default.tex}}%
\only<2>{\providecommand\step{4}\input{diagrams/default/default.tex}}%
\end{column}
\end{columns}
\end{frame}


\begin{frame}{Startup}{How did we get there by the way?}
\begin{columns}
\begin{column}{0.45\textwidth}
\begin{itemize}
\item The entry point address of the binary is \funcname{main} (\funcname{\_start}) from the runtime
\item The program starts like a C program:
\begin{itemize}
\item \funcname{caml\_start\_program}
\item \funcname{caml\_startup\_common}
\item \funcname{caml\_startup\_exn}
\item \funcname{caml\_startup}
\item \funcname{caml\_main}
\item \funcname{main}
\end{itemize}
%\item \funcname{caml\_startup\_common}: initializes GC, domains \dots
% Also: codefrag, locale, custom opeartions, frame_descr, signals, stack
\item \funcname{caml\_start\_program} is the transition point between the C realm and the OCaml one
\end{itemize}
\bigskip
\listocaml{../src/default/default.ml}{default/default.ml}
\end{column}
\begin{column}{0.55\textwidth}
\listasm[lastline=24,highlightlines={22-24}]{src/ocaml/caml\_start\_program.s}{runtime/amd64.S:604}
\end{column}
\end{columns}
\end{frame}


\begin{frame}{Setup to jump into OCaml entry point}{\funcname{caml\_start\_program} initializes the main fiber}
\begin{columns}
\begin{column}{0.6\textwidth}
\only<1,3,5>{
\begin{itemize}
\item<1-> Stores information to allow switching from OCaml stack to the C stack
\item<3-> Constructs the default exception handler
\item<5-> Switches to the OCaml stack and calls \funcname{caml\_program}
\end{itemize}
\bigskip
\listocaml{../src/default/default.ml}{default/default.ml}
}%
\only<2>{
\listasm[firstline=22,lastline=41,highlightlines={25-27}]{src/ocaml/caml\_start\_program.s}{caml\_start\_program}
}%
\only<4>{
\mintasm[firstline=22,lastline=41,highlightlines={31-38}]{src/ocaml/caml\_start\_program.s}
\listasm[firstline=66,lastline=72]{src/ocaml/caml\_start\_program.s}{caml\_start\_program}
}%
\only<6>{
\listasm[firstline=22,lastline=41,highlightlines={39-41}]{src/ocaml/caml\_start\_program.s}{caml\_start\_program}
}%
\end{column}
\begin{column}{0.4\textwidth}
\centering
\only<1-2>{\providecommand\step{2}\input{diagrams/default/default.tex}}%
\only<3-5>{\providecommand\step{3}\input{diagrams/default/default.tex}}%
\onslide*<6->{\providecommand\step{4}\input{diagrams/default/default.tex}}%
\end{column}
\end{columns}
\end{frame}

\begin{frame}{Executing the OCaml program}{And raising an exception without handler}
\begin{columns}
\begin{column}{0.6\textwidth}
\only<1-3>{
\begin{itemize}
\item<1-> Execution continues with OCaml code
\begin{itemize}
\item \funcname{camlDefault\_\_main\_269}
\item \funcname{camlDefault\_\_entry}
\item \funcname{caml\_program}
\end{itemize}
\item<1-> \funcname{camlDefault\_\_main\_269} raises a \typename{Failure} exception
\item<1-> \typename{Failure} is caught by \funcname{caml\_start\_program}
\begin{itemize}
\item<1-> The handler set the 2nd LSB of the exception value to \funcarg{1}
\item<3-> And then forces \funcname{caml\_start\_program} to terminate
\end{itemize}
\end{itemize}
}
\bigskip
\only<1,3>{\listocaml[highlightlines=3]{../src/default/default.ml}{default/default.ml}}%
\only<2>{\listasm[firstline=66,lastline=72,highlightlines=69]{src/ocaml/caml\_start\_program.s}{caml\_start\_program}}%
\end{column}
\begin{column}{0.4\textwidth}
\centering
\providecommand\step{4}\input{diagrams/default/default.tex}
\end{column}
\end{columns}
\end{frame}
%\only<>{\listasm[firstline=47,lastline=65]{src/ocaml/caml\_start\_program.s}{caml\_start\_program}}


\begin{frame}{Back to C}{Clean up, complain and terminate}
\begin{columns}
\begin{column}{0.45\textwidth}
\begin{itemize}
\item Backtrace:
\begin{itemize}
\item \funcname{caml\_start\_program} $\rightarrow$ OCaml code
\item \funcname{caml\_startup\_common}
\item \funcname{caml\_startup\_exn}
\item \funcname{caml\_startup}
\item \funcname{caml\_main}
\item \funcname{main}
\end{itemize}
\smallskip
\item \funcname{caml\_startup} checks the return value of \funcname{caml\_startup\_exn}
\begin{itemize}
\item The return value has been specially marked by \funcname{caml\_start\_program} handler
\item Calls \funcname{caml\_fatal\_uncaught\_exception}
\end{itemize}
\item<2-> \funcname{caml\_fatal\_uncaught\_exception} either
\begin{itemize}
\item<2-> Calls the user custom uncaught exception handler, if set with \mintinline{ocaml}{Printexc.set_uncaught_exception_handler fn}\footnotemark
\item<2-> Or falls back to the default one
\item<2-> Which notifies about the uncaught exception
\end{itemize}
\item<4-> Terminates the program either with \funcname{abort} or with a non-zero exit code
\end{itemize}
\end{column}
\begin{column}{0.55\textwidth}
\only<1>{
\listc[firstline=84,lastline=89,highlightlines=88]{../ocaml/runtime/caml/mlvalues.h}{runtime/caml/mlvalues.h:84}
\listc[firstline=138,lastline=143,highlightlines={141-142}]{../ocaml/runtime/startup\_nat.c}{runtime/startup\_nat.c:138}
}%
\only<2>{
\listc[firstline=138,highlightlines={147,149}]{../ocaml/runtime/printexc.c}{runtime/printexc.c:138}
}%
\only<3>{
\listc[firstline=110,lastline=134,highlightlines=129]{../ocaml/runtime/printexc.c}{runtime/printexc.c:138}
}%
\only<4>{
\listc[firstline=138,highlightlines={152,154}]{../ocaml/runtime/printexc.c}{runtime/printexc.c:138}
}%
\end{column}
\end{columns}
\footnotetext[1]{\url{https://v2.ocaml.org/api/Printexc.html\#1\_Uncaughtexceptions}}
\end{frame}
}%
\only<2>{\providecommand\step{4}%
% Runtime's default exception handler
%
\subsection*{Runtime's default exception handler}
\frameSubsection{pictures/The\_Architect.png}{}


\begin{frame}{Default exception handler}
\begin{columns}
\begin{column}{0.5\textwidth}
\begin{itemize}
\item \localname{domain\_state->exn\_handler} points to a trap located before \funcname{entry}!
\item What installed this very first trap there?
\end{itemize}
\bigskip
\listocaml{../src/default/default.ml}{default/default.ml}
\bigskip
\inputminted[fontsize=\footnotesize]{text}{../src/default/default.out}
\end{column}
\begin{column}{0.5\textwidth}
\centering
\only<1>{\providecommand\step{0}\input{diagrams/default/default.tex}}%
\only<2>{\providecommand\step{4}\input{diagrams/default/default.tex}}%
\end{column}
\end{columns}
\end{frame}


\begin{frame}{Startup}{How did we get there by the way?}
\begin{columns}
\begin{column}{0.45\textwidth}
\begin{itemize}
\item The entry point address of the binary is \funcname{main} (\funcname{\_start}) from the runtime
\item The program starts like a C program:
\begin{itemize}
\item \funcname{caml\_start\_program}
\item \funcname{caml\_startup\_common}
\item \funcname{caml\_startup\_exn}
\item \funcname{caml\_startup}
\item \funcname{caml\_main}
\item \funcname{main}
\end{itemize}
%\item \funcname{caml\_startup\_common}: initializes GC, domains \dots
% Also: codefrag, locale, custom opeartions, frame_descr, signals, stack
\item \funcname{caml\_start\_program} is the transition point between the C realm and the OCaml one
\end{itemize}
\bigskip
\listocaml{../src/default/default.ml}{default/default.ml}
\end{column}
\begin{column}{0.55\textwidth}
\listasm[lastline=24,highlightlines={22-24}]{src/ocaml/caml\_start\_program.s}{runtime/amd64.S:604}
\end{column}
\end{columns}
\end{frame}


\begin{frame}{Setup to jump into OCaml entry point}{\funcname{caml\_start\_program} initializes the main fiber}
\begin{columns}
\begin{column}{0.6\textwidth}
\only<1,3,5>{
\begin{itemize}
\item<1-> Stores information to allow switching from OCaml stack to the C stack
\item<3-> Constructs the default exception handler
\item<5-> Switches to the OCaml stack and calls \funcname{caml\_program}
\end{itemize}
\bigskip
\listocaml{../src/default/default.ml}{default/default.ml}
}%
\only<2>{
\listasm[firstline=22,lastline=41,highlightlines={25-27}]{src/ocaml/caml\_start\_program.s}{caml\_start\_program}
}%
\only<4>{
\mintasm[firstline=22,lastline=41,highlightlines={31-38}]{src/ocaml/caml\_start\_program.s}
\listasm[firstline=66,lastline=72]{src/ocaml/caml\_start\_program.s}{caml\_start\_program}
}%
\only<6>{
\listasm[firstline=22,lastline=41,highlightlines={39-41}]{src/ocaml/caml\_start\_program.s}{caml\_start\_program}
}%
\end{column}
\begin{column}{0.4\textwidth}
\centering
\only<1-2>{\providecommand\step{2}\input{diagrams/default/default.tex}}%
\only<3-5>{\providecommand\step{3}\input{diagrams/default/default.tex}}%
\onslide*<6->{\providecommand\step{4}\input{diagrams/default/default.tex}}%
\end{column}
\end{columns}
\end{frame}

\begin{frame}{Executing the OCaml program}{And raising an exception without handler}
\begin{columns}
\begin{column}{0.6\textwidth}
\only<1-3>{
\begin{itemize}
\item<1-> Execution continues with OCaml code
\begin{itemize}
\item \funcname{camlDefault\_\_main\_269}
\item \funcname{camlDefault\_\_entry}
\item \funcname{caml\_program}
\end{itemize}
\item<1-> \funcname{camlDefault\_\_main\_269} raises a \typename{Failure} exception
\item<1-> \typename{Failure} is caught by \funcname{caml\_start\_program}
\begin{itemize}
\item<1-> The handler set the 2nd LSB of the exception value to \funcarg{1}
\item<3-> And then forces \funcname{caml\_start\_program} to terminate
\end{itemize}
\end{itemize}
}
\bigskip
\only<1,3>{\listocaml[highlightlines=3]{../src/default/default.ml}{default/default.ml}}%
\only<2>{\listasm[firstline=66,lastline=72,highlightlines=69]{src/ocaml/caml\_start\_program.s}{caml\_start\_program}}%
\end{column}
\begin{column}{0.4\textwidth}
\centering
\providecommand\step{4}\input{diagrams/default/default.tex}
\end{column}
\end{columns}
\end{frame}
%\only<>{\listasm[firstline=47,lastline=65]{src/ocaml/caml\_start\_program.s}{caml\_start\_program}}


\begin{frame}{Back to C}{Clean up, complain and terminate}
\begin{columns}
\begin{column}{0.45\textwidth}
\begin{itemize}
\item Backtrace:
\begin{itemize}
\item \funcname{caml\_start\_program} $\rightarrow$ OCaml code
\item \funcname{caml\_startup\_common}
\item \funcname{caml\_startup\_exn}
\item \funcname{caml\_startup}
\item \funcname{caml\_main}
\item \funcname{main}
\end{itemize}
\smallskip
\item \funcname{caml\_startup} checks the return value of \funcname{caml\_startup\_exn}
\begin{itemize}
\item The return value has been specially marked by \funcname{caml\_start\_program} handler
\item Calls \funcname{caml\_fatal\_uncaught\_exception}
\end{itemize}
\item<2-> \funcname{caml\_fatal\_uncaught\_exception} either
\begin{itemize}
\item<2-> Calls the user custom uncaught exception handler, if set with \mintinline{ocaml}{Printexc.set_uncaught_exception_handler fn}\footnotemark
\item<2-> Or falls back to the default one
\item<2-> Which notifies about the uncaught exception
\end{itemize}
\item<4-> Terminates the program either with \funcname{abort} or with a non-zero exit code
\end{itemize}
\end{column}
\begin{column}{0.55\textwidth}
\only<1>{
\listc[firstline=84,lastline=89,highlightlines=88]{../ocaml/runtime/caml/mlvalues.h}{runtime/caml/mlvalues.h:84}
\listc[firstline=138,lastline=143,highlightlines={141-142}]{../ocaml/runtime/startup\_nat.c}{runtime/startup\_nat.c:138}
}%
\only<2>{
\listc[firstline=138,highlightlines={147,149}]{../ocaml/runtime/printexc.c}{runtime/printexc.c:138}
}%
\only<3>{
\listc[firstline=110,lastline=134,highlightlines=129]{../ocaml/runtime/printexc.c}{runtime/printexc.c:138}
}%
\only<4>{
\listc[firstline=138,highlightlines={152,154}]{../ocaml/runtime/printexc.c}{runtime/printexc.c:138}
}%
\end{column}
\end{columns}
\footnotetext[1]{\url{https://v2.ocaml.org/api/Printexc.html\#1\_Uncaughtexceptions}}
\end{frame}
}%
\end{column}
\end{columns}
\end{frame}


\begin{frame}{Startup}{How did we get there by the way?}
\begin{columns}
\begin{column}{0.45\textwidth}
\begin{itemize}
\item The entry point address of the binary is \funcname{main} (\funcname{\_start}) from the runtime
\item The program starts like a C program:
\begin{itemize}
\item \funcname{caml\_start\_program}
\item \funcname{caml\_startup\_common}
\item \funcname{caml\_startup\_exn}
\item \funcname{caml\_startup}
\item \funcname{caml\_main}
\item \funcname{main}
\end{itemize}
%\item \funcname{caml\_startup\_common}: initializes GC, domains \dots
% Also: codefrag, locale, custom opeartions, frame_descr, signals, stack
\item \funcname{caml\_start\_program} is the transition point between the C realm and the OCaml one
\end{itemize}
\bigskip
\listocaml{../src/default/default.ml}{default/default.ml}
\end{column}
\begin{column}{0.55\textwidth}
\listasm[lastline=24,highlightlines={22-24}]{src/ocaml/caml\_start\_program.s}{runtime/amd64.S:604}
\end{column}
\end{columns}
\end{frame}


\begin{frame}{Setup to jump into OCaml entry point}{\funcname{caml\_start\_program} initializes the main fiber}
\begin{columns}
\begin{column}{0.6\textwidth}
\only<1,3,5>{
\begin{itemize}
\item<1-> Stores information to allow switching from OCaml stack to the C stack
\item<3-> Constructs the default exception handler
\item<5-> Switches to the OCaml stack and calls \funcname{caml\_program}
\end{itemize}
\bigskip
\listocaml{../src/default/default.ml}{default/default.ml}
}%
\only<2>{
\listasm[firstline=22,lastline=41,highlightlines={25-27}]{src/ocaml/caml\_start\_program.s}{caml\_start\_program}
}%
\only<4>{
\mintasm[firstline=22,lastline=41,highlightlines={31-38}]{src/ocaml/caml\_start\_program.s}
\listasm[firstline=66,lastline=72]{src/ocaml/caml\_start\_program.s}{caml\_start\_program}
}%
\only<6>{
\listasm[firstline=22,lastline=41,highlightlines={39-41}]{src/ocaml/caml\_start\_program.s}{caml\_start\_program}
}%
\end{column}
\begin{column}{0.4\textwidth}
\centering
\only<1-2>{\providecommand\step{2}%
% Runtime's default exception handler
%
\subsection*{Runtime's default exception handler}
\frameSubsection{pictures/The\_Architect.png}{}


\begin{frame}{Default exception handler}
\begin{columns}
\begin{column}{0.5\textwidth}
\begin{itemize}
\item \localname{domain\_state->exn\_handler} points to a trap located before \funcname{entry}!
\item What installed this very first trap there?
\end{itemize}
\bigskip
\listocaml{../src/default/default.ml}{default/default.ml}
\bigskip
\inputminted[fontsize=\footnotesize]{text}{../src/default/default.out}
\end{column}
\begin{column}{0.5\textwidth}
\centering
\only<1>{\providecommand\step{0}\input{diagrams/default/default.tex}}%
\only<2>{\providecommand\step{4}\input{diagrams/default/default.tex}}%
\end{column}
\end{columns}
\end{frame}


\begin{frame}{Startup}{How did we get there by the way?}
\begin{columns}
\begin{column}{0.45\textwidth}
\begin{itemize}
\item The entry point address of the binary is \funcname{main} (\funcname{\_start}) from the runtime
\item The program starts like a C program:
\begin{itemize}
\item \funcname{caml\_start\_program}
\item \funcname{caml\_startup\_common}
\item \funcname{caml\_startup\_exn}
\item \funcname{caml\_startup}
\item \funcname{caml\_main}
\item \funcname{main}
\end{itemize}
%\item \funcname{caml\_startup\_common}: initializes GC, domains \dots
% Also: codefrag, locale, custom opeartions, frame_descr, signals, stack
\item \funcname{caml\_start\_program} is the transition point between the C realm and the OCaml one
\end{itemize}
\bigskip
\listocaml{../src/default/default.ml}{default/default.ml}
\end{column}
\begin{column}{0.55\textwidth}
\listasm[lastline=24,highlightlines={22-24}]{src/ocaml/caml\_start\_program.s}{runtime/amd64.S:604}
\end{column}
\end{columns}
\end{frame}


\begin{frame}{Setup to jump into OCaml entry point}{\funcname{caml\_start\_program} initializes the main fiber}
\begin{columns}
\begin{column}{0.6\textwidth}
\only<1,3,5>{
\begin{itemize}
\item<1-> Stores information to allow switching from OCaml stack to the C stack
\item<3-> Constructs the default exception handler
\item<5-> Switches to the OCaml stack and calls \funcname{caml\_program}
\end{itemize}
\bigskip
\listocaml{../src/default/default.ml}{default/default.ml}
}%
\only<2>{
\listasm[firstline=22,lastline=41,highlightlines={25-27}]{src/ocaml/caml\_start\_program.s}{caml\_start\_program}
}%
\only<4>{
\mintasm[firstline=22,lastline=41,highlightlines={31-38}]{src/ocaml/caml\_start\_program.s}
\listasm[firstline=66,lastline=72]{src/ocaml/caml\_start\_program.s}{caml\_start\_program}
}%
\only<6>{
\listasm[firstline=22,lastline=41,highlightlines={39-41}]{src/ocaml/caml\_start\_program.s}{caml\_start\_program}
}%
\end{column}
\begin{column}{0.4\textwidth}
\centering
\only<1-2>{\providecommand\step{2}\input{diagrams/default/default.tex}}%
\only<3-5>{\providecommand\step{3}\input{diagrams/default/default.tex}}%
\onslide*<6->{\providecommand\step{4}\input{diagrams/default/default.tex}}%
\end{column}
\end{columns}
\end{frame}

\begin{frame}{Executing the OCaml program}{And raising an exception without handler}
\begin{columns}
\begin{column}{0.6\textwidth}
\only<1-3>{
\begin{itemize}
\item<1-> Execution continues with OCaml code
\begin{itemize}
\item \funcname{camlDefault\_\_main\_269}
\item \funcname{camlDefault\_\_entry}
\item \funcname{caml\_program}
\end{itemize}
\item<1-> \funcname{camlDefault\_\_main\_269} raises a \typename{Failure} exception
\item<1-> \typename{Failure} is caught by \funcname{caml\_start\_program}
\begin{itemize}
\item<1-> The handler set the 2nd LSB of the exception value to \funcarg{1}
\item<3-> And then forces \funcname{caml\_start\_program} to terminate
\end{itemize}
\end{itemize}
}
\bigskip
\only<1,3>{\listocaml[highlightlines=3]{../src/default/default.ml}{default/default.ml}}%
\only<2>{\listasm[firstline=66,lastline=72,highlightlines=69]{src/ocaml/caml\_start\_program.s}{caml\_start\_program}}%
\end{column}
\begin{column}{0.4\textwidth}
\centering
\providecommand\step{4}\input{diagrams/default/default.tex}
\end{column}
\end{columns}
\end{frame}
%\only<>{\listasm[firstline=47,lastline=65]{src/ocaml/caml\_start\_program.s}{caml\_start\_program}}


\begin{frame}{Back to C}{Clean up, complain and terminate}
\begin{columns}
\begin{column}{0.45\textwidth}
\begin{itemize}
\item Backtrace:
\begin{itemize}
\item \funcname{caml\_start\_program} $\rightarrow$ OCaml code
\item \funcname{caml\_startup\_common}
\item \funcname{caml\_startup\_exn}
\item \funcname{caml\_startup}
\item \funcname{caml\_main}
\item \funcname{main}
\end{itemize}
\smallskip
\item \funcname{caml\_startup} checks the return value of \funcname{caml\_startup\_exn}
\begin{itemize}
\item The return value has been specially marked by \funcname{caml\_start\_program} handler
\item Calls \funcname{caml\_fatal\_uncaught\_exception}
\end{itemize}
\item<2-> \funcname{caml\_fatal\_uncaught\_exception} either
\begin{itemize}
\item<2-> Calls the user custom uncaught exception handler, if set with \mintinline{ocaml}{Printexc.set_uncaught_exception_handler fn}\footnotemark
\item<2-> Or falls back to the default one
\item<2-> Which notifies about the uncaught exception
\end{itemize}
\item<4-> Terminates the program either with \funcname{abort} or with a non-zero exit code
\end{itemize}
\end{column}
\begin{column}{0.55\textwidth}
\only<1>{
\listc[firstline=84,lastline=89,highlightlines=88]{../ocaml/runtime/caml/mlvalues.h}{runtime/caml/mlvalues.h:84}
\listc[firstline=138,lastline=143,highlightlines={141-142}]{../ocaml/runtime/startup\_nat.c}{runtime/startup\_nat.c:138}
}%
\only<2>{
\listc[firstline=138,highlightlines={147,149}]{../ocaml/runtime/printexc.c}{runtime/printexc.c:138}
}%
\only<3>{
\listc[firstline=110,lastline=134,highlightlines=129]{../ocaml/runtime/printexc.c}{runtime/printexc.c:138}
}%
\only<4>{
\listc[firstline=138,highlightlines={152,154}]{../ocaml/runtime/printexc.c}{runtime/printexc.c:138}
}%
\end{column}
\end{columns}
\footnotetext[1]{\url{https://v2.ocaml.org/api/Printexc.html\#1\_Uncaughtexceptions}}
\end{frame}
}%
\only<3-5>{\providecommand\step{3}%
% Runtime's default exception handler
%
\subsection*{Runtime's default exception handler}
\frameSubsection{pictures/The\_Architect.png}{}


\begin{frame}{Default exception handler}
\begin{columns}
\begin{column}{0.5\textwidth}
\begin{itemize}
\item \localname{domain\_state->exn\_handler} points to a trap located before \funcname{entry}!
\item What installed this very first trap there?
\end{itemize}
\bigskip
\listocaml{../src/default/default.ml}{default/default.ml}
\bigskip
\inputminted[fontsize=\footnotesize]{text}{../src/default/default.out}
\end{column}
\begin{column}{0.5\textwidth}
\centering
\only<1>{\providecommand\step{0}\input{diagrams/default/default.tex}}%
\only<2>{\providecommand\step{4}\input{diagrams/default/default.tex}}%
\end{column}
\end{columns}
\end{frame}


\begin{frame}{Startup}{How did we get there by the way?}
\begin{columns}
\begin{column}{0.45\textwidth}
\begin{itemize}
\item The entry point address of the binary is \funcname{main} (\funcname{\_start}) from the runtime
\item The program starts like a C program:
\begin{itemize}
\item \funcname{caml\_start\_program}
\item \funcname{caml\_startup\_common}
\item \funcname{caml\_startup\_exn}
\item \funcname{caml\_startup}
\item \funcname{caml\_main}
\item \funcname{main}
\end{itemize}
%\item \funcname{caml\_startup\_common}: initializes GC, domains \dots
% Also: codefrag, locale, custom opeartions, frame_descr, signals, stack
\item \funcname{caml\_start\_program} is the transition point between the C realm and the OCaml one
\end{itemize}
\bigskip
\listocaml{../src/default/default.ml}{default/default.ml}
\end{column}
\begin{column}{0.55\textwidth}
\listasm[lastline=24,highlightlines={22-24}]{src/ocaml/caml\_start\_program.s}{runtime/amd64.S:604}
\end{column}
\end{columns}
\end{frame}


\begin{frame}{Setup to jump into OCaml entry point}{\funcname{caml\_start\_program} initializes the main fiber}
\begin{columns}
\begin{column}{0.6\textwidth}
\only<1,3,5>{
\begin{itemize}
\item<1-> Stores information to allow switching from OCaml stack to the C stack
\item<3-> Constructs the default exception handler
\item<5-> Switches to the OCaml stack and calls \funcname{caml\_program}
\end{itemize}
\bigskip
\listocaml{../src/default/default.ml}{default/default.ml}
}%
\only<2>{
\listasm[firstline=22,lastline=41,highlightlines={25-27}]{src/ocaml/caml\_start\_program.s}{caml\_start\_program}
}%
\only<4>{
\mintasm[firstline=22,lastline=41,highlightlines={31-38}]{src/ocaml/caml\_start\_program.s}
\listasm[firstline=66,lastline=72]{src/ocaml/caml\_start\_program.s}{caml\_start\_program}
}%
\only<6>{
\listasm[firstline=22,lastline=41,highlightlines={39-41}]{src/ocaml/caml\_start\_program.s}{caml\_start\_program}
}%
\end{column}
\begin{column}{0.4\textwidth}
\centering
\only<1-2>{\providecommand\step{2}\input{diagrams/default/default.tex}}%
\only<3-5>{\providecommand\step{3}\input{diagrams/default/default.tex}}%
\onslide*<6->{\providecommand\step{4}\input{diagrams/default/default.tex}}%
\end{column}
\end{columns}
\end{frame}

\begin{frame}{Executing the OCaml program}{And raising an exception without handler}
\begin{columns}
\begin{column}{0.6\textwidth}
\only<1-3>{
\begin{itemize}
\item<1-> Execution continues with OCaml code
\begin{itemize}
\item \funcname{camlDefault\_\_main\_269}
\item \funcname{camlDefault\_\_entry}
\item \funcname{caml\_program}
\end{itemize}
\item<1-> \funcname{camlDefault\_\_main\_269} raises a \typename{Failure} exception
\item<1-> \typename{Failure} is caught by \funcname{caml\_start\_program}
\begin{itemize}
\item<1-> The handler set the 2nd LSB of the exception value to \funcarg{1}
\item<3-> And then forces \funcname{caml\_start\_program} to terminate
\end{itemize}
\end{itemize}
}
\bigskip
\only<1,3>{\listocaml[highlightlines=3]{../src/default/default.ml}{default/default.ml}}%
\only<2>{\listasm[firstline=66,lastline=72,highlightlines=69]{src/ocaml/caml\_start\_program.s}{caml\_start\_program}}%
\end{column}
\begin{column}{0.4\textwidth}
\centering
\providecommand\step{4}\input{diagrams/default/default.tex}
\end{column}
\end{columns}
\end{frame}
%\only<>{\listasm[firstline=47,lastline=65]{src/ocaml/caml\_start\_program.s}{caml\_start\_program}}


\begin{frame}{Back to C}{Clean up, complain and terminate}
\begin{columns}
\begin{column}{0.45\textwidth}
\begin{itemize}
\item Backtrace:
\begin{itemize}
\item \funcname{caml\_start\_program} $\rightarrow$ OCaml code
\item \funcname{caml\_startup\_common}
\item \funcname{caml\_startup\_exn}
\item \funcname{caml\_startup}
\item \funcname{caml\_main}
\item \funcname{main}
\end{itemize}
\smallskip
\item \funcname{caml\_startup} checks the return value of \funcname{caml\_startup\_exn}
\begin{itemize}
\item The return value has been specially marked by \funcname{caml\_start\_program} handler
\item Calls \funcname{caml\_fatal\_uncaught\_exception}
\end{itemize}
\item<2-> \funcname{caml\_fatal\_uncaught\_exception} either
\begin{itemize}
\item<2-> Calls the user custom uncaught exception handler, if set with \mintinline{ocaml}{Printexc.set_uncaught_exception_handler fn}\footnotemark
\item<2-> Or falls back to the default one
\item<2-> Which notifies about the uncaught exception
\end{itemize}
\item<4-> Terminates the program either with \funcname{abort} or with a non-zero exit code
\end{itemize}
\end{column}
\begin{column}{0.55\textwidth}
\only<1>{
\listc[firstline=84,lastline=89,highlightlines=88]{../ocaml/runtime/caml/mlvalues.h}{runtime/caml/mlvalues.h:84}
\listc[firstline=138,lastline=143,highlightlines={141-142}]{../ocaml/runtime/startup\_nat.c}{runtime/startup\_nat.c:138}
}%
\only<2>{
\listc[firstline=138,highlightlines={147,149}]{../ocaml/runtime/printexc.c}{runtime/printexc.c:138}
}%
\only<3>{
\listc[firstline=110,lastline=134,highlightlines=129]{../ocaml/runtime/printexc.c}{runtime/printexc.c:138}
}%
\only<4>{
\listc[firstline=138,highlightlines={152,154}]{../ocaml/runtime/printexc.c}{runtime/printexc.c:138}
}%
\end{column}
\end{columns}
\footnotetext[1]{\url{https://v2.ocaml.org/api/Printexc.html\#1\_Uncaughtexceptions}}
\end{frame}
}%
\onslide*<6->{\providecommand\step{4}%
% Runtime's default exception handler
%
\subsection*{Runtime's default exception handler}
\frameSubsection{pictures/The\_Architect.png}{}


\begin{frame}{Default exception handler}
\begin{columns}
\begin{column}{0.5\textwidth}
\begin{itemize}
\item \localname{domain\_state->exn\_handler} points to a trap located before \funcname{entry}!
\item What installed this very first trap there?
\end{itemize}
\bigskip
\listocaml{../src/default/default.ml}{default/default.ml}
\bigskip
\inputminted[fontsize=\footnotesize]{text}{../src/default/default.out}
\end{column}
\begin{column}{0.5\textwidth}
\centering
\only<1>{\providecommand\step{0}\input{diagrams/default/default.tex}}%
\only<2>{\providecommand\step{4}\input{diagrams/default/default.tex}}%
\end{column}
\end{columns}
\end{frame}


\begin{frame}{Startup}{How did we get there by the way?}
\begin{columns}
\begin{column}{0.45\textwidth}
\begin{itemize}
\item The entry point address of the binary is \funcname{main} (\funcname{\_start}) from the runtime
\item The program starts like a C program:
\begin{itemize}
\item \funcname{caml\_start\_program}
\item \funcname{caml\_startup\_common}
\item \funcname{caml\_startup\_exn}
\item \funcname{caml\_startup}
\item \funcname{caml\_main}
\item \funcname{main}
\end{itemize}
%\item \funcname{caml\_startup\_common}: initializes GC, domains \dots
% Also: codefrag, locale, custom opeartions, frame_descr, signals, stack
\item \funcname{caml\_start\_program} is the transition point between the C realm and the OCaml one
\end{itemize}
\bigskip
\listocaml{../src/default/default.ml}{default/default.ml}
\end{column}
\begin{column}{0.55\textwidth}
\listasm[lastline=24,highlightlines={22-24}]{src/ocaml/caml\_start\_program.s}{runtime/amd64.S:604}
\end{column}
\end{columns}
\end{frame}


\begin{frame}{Setup to jump into OCaml entry point}{\funcname{caml\_start\_program} initializes the main fiber}
\begin{columns}
\begin{column}{0.6\textwidth}
\only<1,3,5>{
\begin{itemize}
\item<1-> Stores information to allow switching from OCaml stack to the C stack
\item<3-> Constructs the default exception handler
\item<5-> Switches to the OCaml stack and calls \funcname{caml\_program}
\end{itemize}
\bigskip
\listocaml{../src/default/default.ml}{default/default.ml}
}%
\only<2>{
\listasm[firstline=22,lastline=41,highlightlines={25-27}]{src/ocaml/caml\_start\_program.s}{caml\_start\_program}
}%
\only<4>{
\mintasm[firstline=22,lastline=41,highlightlines={31-38}]{src/ocaml/caml\_start\_program.s}
\listasm[firstline=66,lastline=72]{src/ocaml/caml\_start\_program.s}{caml\_start\_program}
}%
\only<6>{
\listasm[firstline=22,lastline=41,highlightlines={39-41}]{src/ocaml/caml\_start\_program.s}{caml\_start\_program}
}%
\end{column}
\begin{column}{0.4\textwidth}
\centering
\only<1-2>{\providecommand\step{2}\input{diagrams/default/default.tex}}%
\only<3-5>{\providecommand\step{3}\input{diagrams/default/default.tex}}%
\onslide*<6->{\providecommand\step{4}\input{diagrams/default/default.tex}}%
\end{column}
\end{columns}
\end{frame}

\begin{frame}{Executing the OCaml program}{And raising an exception without handler}
\begin{columns}
\begin{column}{0.6\textwidth}
\only<1-3>{
\begin{itemize}
\item<1-> Execution continues with OCaml code
\begin{itemize}
\item \funcname{camlDefault\_\_main\_269}
\item \funcname{camlDefault\_\_entry}
\item \funcname{caml\_program}
\end{itemize}
\item<1-> \funcname{camlDefault\_\_main\_269} raises a \typename{Failure} exception
\item<1-> \typename{Failure} is caught by \funcname{caml\_start\_program}
\begin{itemize}
\item<1-> The handler set the 2nd LSB of the exception value to \funcarg{1}
\item<3-> And then forces \funcname{caml\_start\_program} to terminate
\end{itemize}
\end{itemize}
}
\bigskip
\only<1,3>{\listocaml[highlightlines=3]{../src/default/default.ml}{default/default.ml}}%
\only<2>{\listasm[firstline=66,lastline=72,highlightlines=69]{src/ocaml/caml\_start\_program.s}{caml\_start\_program}}%
\end{column}
\begin{column}{0.4\textwidth}
\centering
\providecommand\step{4}\input{diagrams/default/default.tex}
\end{column}
\end{columns}
\end{frame}
%\only<>{\listasm[firstline=47,lastline=65]{src/ocaml/caml\_start\_program.s}{caml\_start\_program}}


\begin{frame}{Back to C}{Clean up, complain and terminate}
\begin{columns}
\begin{column}{0.45\textwidth}
\begin{itemize}
\item Backtrace:
\begin{itemize}
\item \funcname{caml\_start\_program} $\rightarrow$ OCaml code
\item \funcname{caml\_startup\_common}
\item \funcname{caml\_startup\_exn}
\item \funcname{caml\_startup}
\item \funcname{caml\_main}
\item \funcname{main}
\end{itemize}
\smallskip
\item \funcname{caml\_startup} checks the return value of \funcname{caml\_startup\_exn}
\begin{itemize}
\item The return value has been specially marked by \funcname{caml\_start\_program} handler
\item Calls \funcname{caml\_fatal\_uncaught\_exception}
\end{itemize}
\item<2-> \funcname{caml\_fatal\_uncaught\_exception} either
\begin{itemize}
\item<2-> Calls the user custom uncaught exception handler, if set with \mintinline{ocaml}{Printexc.set_uncaught_exception_handler fn}\footnotemark
\item<2-> Or falls back to the default one
\item<2-> Which notifies about the uncaught exception
\end{itemize}
\item<4-> Terminates the program either with \funcname{abort} or with a non-zero exit code
\end{itemize}
\end{column}
\begin{column}{0.55\textwidth}
\only<1>{
\listc[firstline=84,lastline=89,highlightlines=88]{../ocaml/runtime/caml/mlvalues.h}{runtime/caml/mlvalues.h:84}
\listc[firstline=138,lastline=143,highlightlines={141-142}]{../ocaml/runtime/startup\_nat.c}{runtime/startup\_nat.c:138}
}%
\only<2>{
\listc[firstline=138,highlightlines={147,149}]{../ocaml/runtime/printexc.c}{runtime/printexc.c:138}
}%
\only<3>{
\listc[firstline=110,lastline=134,highlightlines=129]{../ocaml/runtime/printexc.c}{runtime/printexc.c:138}
}%
\only<4>{
\listc[firstline=138,highlightlines={152,154}]{../ocaml/runtime/printexc.c}{runtime/printexc.c:138}
}%
\end{column}
\end{columns}
\footnotetext[1]{\url{https://v2.ocaml.org/api/Printexc.html\#1\_Uncaughtexceptions}}
\end{frame}
}%
\end{column}
\end{columns}
\end{frame}

\begin{frame}{Executing the OCaml program}{And raising an exception without handler}
\begin{columns}
\begin{column}{0.6\textwidth}
\only<1-3>{
\begin{itemize}
\item<1-> Execution continues with OCaml code
\begin{itemize}
\item \funcname{camlDefault\_\_main\_269}
\item \funcname{camlDefault\_\_entry}
\item \funcname{caml\_program}
\end{itemize}
\item<1-> \funcname{camlDefault\_\_main\_269} raises a \typename{Failure} exception
\item<1-> \typename{Failure} is caught by \funcname{caml\_start\_program}
\begin{itemize}
\item<1-> The handler set the 2nd LSB of the exception value to \funcarg{1}
\item<3-> And then forces \funcname{caml\_start\_program} to terminate
\end{itemize}
\end{itemize}
}
\bigskip
\only<1,3>{\listocaml[highlightlines=3]{../src/default/default.ml}{default/default.ml}}%
\only<2>{\listasm[firstline=66,lastline=72,highlightlines=69]{src/ocaml/caml\_start\_program.s}{caml\_start\_program}}%
\end{column}
\begin{column}{0.4\textwidth}
\centering
\providecommand\step{4}%
% Runtime's default exception handler
%
\subsection*{Runtime's default exception handler}
\frameSubsection{pictures/The\_Architect.png}{}


\begin{frame}{Default exception handler}
\begin{columns}
\begin{column}{0.5\textwidth}
\begin{itemize}
\item \localname{domain\_state->exn\_handler} points to a trap located before \funcname{entry}!
\item What installed this very first trap there?
\end{itemize}
\bigskip
\listocaml{../src/default/default.ml}{default/default.ml}
\bigskip
\inputminted[fontsize=\footnotesize]{text}{../src/default/default.out}
\end{column}
\begin{column}{0.5\textwidth}
\centering
\only<1>{\providecommand\step{0}\input{diagrams/default/default.tex}}%
\only<2>{\providecommand\step{4}\input{diagrams/default/default.tex}}%
\end{column}
\end{columns}
\end{frame}


\begin{frame}{Startup}{How did we get there by the way?}
\begin{columns}
\begin{column}{0.45\textwidth}
\begin{itemize}
\item The entry point address of the binary is \funcname{main} (\funcname{\_start}) from the runtime
\item The program starts like a C program:
\begin{itemize}
\item \funcname{caml\_start\_program}
\item \funcname{caml\_startup\_common}
\item \funcname{caml\_startup\_exn}
\item \funcname{caml\_startup}
\item \funcname{caml\_main}
\item \funcname{main}
\end{itemize}
%\item \funcname{caml\_startup\_common}: initializes GC, domains \dots
% Also: codefrag, locale, custom opeartions, frame_descr, signals, stack
\item \funcname{caml\_start\_program} is the transition point between the C realm and the OCaml one
\end{itemize}
\bigskip
\listocaml{../src/default/default.ml}{default/default.ml}
\end{column}
\begin{column}{0.55\textwidth}
\listasm[lastline=24,highlightlines={22-24}]{src/ocaml/caml\_start\_program.s}{runtime/amd64.S:604}
\end{column}
\end{columns}
\end{frame}


\begin{frame}{Setup to jump into OCaml entry point}{\funcname{caml\_start\_program} initializes the main fiber}
\begin{columns}
\begin{column}{0.6\textwidth}
\only<1,3,5>{
\begin{itemize}
\item<1-> Stores information to allow switching from OCaml stack to the C stack
\item<3-> Constructs the default exception handler
\item<5-> Switches to the OCaml stack and calls \funcname{caml\_program}
\end{itemize}
\bigskip
\listocaml{../src/default/default.ml}{default/default.ml}
}%
\only<2>{
\listasm[firstline=22,lastline=41,highlightlines={25-27}]{src/ocaml/caml\_start\_program.s}{caml\_start\_program}
}%
\only<4>{
\mintasm[firstline=22,lastline=41,highlightlines={31-38}]{src/ocaml/caml\_start\_program.s}
\listasm[firstline=66,lastline=72]{src/ocaml/caml\_start\_program.s}{caml\_start\_program}
}%
\only<6>{
\listasm[firstline=22,lastline=41,highlightlines={39-41}]{src/ocaml/caml\_start\_program.s}{caml\_start\_program}
}%
\end{column}
\begin{column}{0.4\textwidth}
\centering
\only<1-2>{\providecommand\step{2}\input{diagrams/default/default.tex}}%
\only<3-5>{\providecommand\step{3}\input{diagrams/default/default.tex}}%
\onslide*<6->{\providecommand\step{4}\input{diagrams/default/default.tex}}%
\end{column}
\end{columns}
\end{frame}

\begin{frame}{Executing the OCaml program}{And raising an exception without handler}
\begin{columns}
\begin{column}{0.6\textwidth}
\only<1-3>{
\begin{itemize}
\item<1-> Execution continues with OCaml code
\begin{itemize}
\item \funcname{camlDefault\_\_main\_269}
\item \funcname{camlDefault\_\_entry}
\item \funcname{caml\_program}
\end{itemize}
\item<1-> \funcname{camlDefault\_\_main\_269} raises a \typename{Failure} exception
\item<1-> \typename{Failure} is caught by \funcname{caml\_start\_program}
\begin{itemize}
\item<1-> The handler set the 2nd LSB of the exception value to \funcarg{1}
\item<3-> And then forces \funcname{caml\_start\_program} to terminate
\end{itemize}
\end{itemize}
}
\bigskip
\only<1,3>{\listocaml[highlightlines=3]{../src/default/default.ml}{default/default.ml}}%
\only<2>{\listasm[firstline=66,lastline=72,highlightlines=69]{src/ocaml/caml\_start\_program.s}{caml\_start\_program}}%
\end{column}
\begin{column}{0.4\textwidth}
\centering
\providecommand\step{4}\input{diagrams/default/default.tex}
\end{column}
\end{columns}
\end{frame}
%\only<>{\listasm[firstline=47,lastline=65]{src/ocaml/caml\_start\_program.s}{caml\_start\_program}}


\begin{frame}{Back to C}{Clean up, complain and terminate}
\begin{columns}
\begin{column}{0.45\textwidth}
\begin{itemize}
\item Backtrace:
\begin{itemize}
\item \funcname{caml\_start\_program} $\rightarrow$ OCaml code
\item \funcname{caml\_startup\_common}
\item \funcname{caml\_startup\_exn}
\item \funcname{caml\_startup}
\item \funcname{caml\_main}
\item \funcname{main}
\end{itemize}
\smallskip
\item \funcname{caml\_startup} checks the return value of \funcname{caml\_startup\_exn}
\begin{itemize}
\item The return value has been specially marked by \funcname{caml\_start\_program} handler
\item Calls \funcname{caml\_fatal\_uncaught\_exception}
\end{itemize}
\item<2-> \funcname{caml\_fatal\_uncaught\_exception} either
\begin{itemize}
\item<2-> Calls the user custom uncaught exception handler, if set with \mintinline{ocaml}{Printexc.set_uncaught_exception_handler fn}\footnotemark
\item<2-> Or falls back to the default one
\item<2-> Which notifies about the uncaught exception
\end{itemize}
\item<4-> Terminates the program either with \funcname{abort} or with a non-zero exit code
\end{itemize}
\end{column}
\begin{column}{0.55\textwidth}
\only<1>{
\listc[firstline=84,lastline=89,highlightlines=88]{../ocaml/runtime/caml/mlvalues.h}{runtime/caml/mlvalues.h:84}
\listc[firstline=138,lastline=143,highlightlines={141-142}]{../ocaml/runtime/startup\_nat.c}{runtime/startup\_nat.c:138}
}%
\only<2>{
\listc[firstline=138,highlightlines={147,149}]{../ocaml/runtime/printexc.c}{runtime/printexc.c:138}
}%
\only<3>{
\listc[firstline=110,lastline=134,highlightlines=129]{../ocaml/runtime/printexc.c}{runtime/printexc.c:138}
}%
\only<4>{
\listc[firstline=138,highlightlines={152,154}]{../ocaml/runtime/printexc.c}{runtime/printexc.c:138}
}%
\end{column}
\end{columns}
\footnotetext[1]{\url{https://v2.ocaml.org/api/Printexc.html\#1\_Uncaughtexceptions}}
\end{frame}

\end{column}
\end{columns}
\end{frame}
%\only<>{\listasm[firstline=47,lastline=65]{src/ocaml/caml\_start\_program.s}{caml\_start\_program}}


\begin{frame}{Back to C}{Clean up, complain and terminate}
\begin{columns}
\begin{column}{0.45\textwidth}
\begin{itemize}
\item Backtrace:
\begin{itemize}
\item \funcname{caml\_start\_program} $\rightarrow$ OCaml code
\item \funcname{caml\_startup\_common}
\item \funcname{caml\_startup\_exn}
\item \funcname{caml\_startup}
\item \funcname{caml\_main}
\item \funcname{main}
\end{itemize}
\smallskip
\item \funcname{caml\_startup} checks the return value of \funcname{caml\_startup\_exn}
\begin{itemize}
\item The return value has been specially marked by \funcname{caml\_start\_program} handler
\item Calls \funcname{caml\_fatal\_uncaught\_exception}
\end{itemize}
\item<2-> \funcname{caml\_fatal\_uncaught\_exception} either
\begin{itemize}
\item<2-> Calls the user custom uncaught exception handler, if set with \mintinline{ocaml}{Printexc.set_uncaught_exception_handler fn}\footnotemark
\item<2-> Or falls back to the default one
\item<2-> Which notifies about the uncaught exception
\end{itemize}
\item<4-> Terminates the program either with \funcname{abort} or with a non-zero exit code
\end{itemize}
\end{column}
\begin{column}{0.55\textwidth}
\only<1>{
\listc[firstline=84,lastline=89,highlightlines=88]{../ocaml/runtime/caml/mlvalues.h}{runtime/caml/mlvalues.h:84}
\listc[firstline=138,lastline=143,highlightlines={141-142}]{../ocaml/runtime/startup\_nat.c}{runtime/startup\_nat.c:138}
}%
\only<2>{
\listc[firstline=138,highlightlines={147,149}]{../ocaml/runtime/printexc.c}{runtime/printexc.c:138}
}%
\only<3>{
\listc[firstline=110,lastline=134,highlightlines=129]{../ocaml/runtime/printexc.c}{runtime/printexc.c:138}
}%
\only<4>{
\listc[firstline=138,highlightlines={152,154}]{../ocaml/runtime/printexc.c}{runtime/printexc.c:138}
}%
\end{column}
\end{columns}
\footnotetext[1]{\url{https://v2.ocaml.org/api/Printexc.html\#1\_Uncaughtexceptions}}
\end{frame}
}%
\only<2>{\providecommand\step{4}%
% Runtime's default exception handler
%
\subsection*{Runtime's default exception handler}
\frameSubsection{pictures/The\_Architect.png}{}


\begin{frame}{Default exception handler}
\begin{columns}
\begin{column}{0.5\textwidth}
\begin{itemize}
\item \localname{domain\_state->exn\_handler} points to a trap located before \funcname{entry}!
\item What installed this very first trap there?
\end{itemize}
\bigskip
\listocaml{../src/default/default.ml}{default/default.ml}
\bigskip
\inputminted[fontsize=\footnotesize]{text}{../src/default/default.out}
\end{column}
\begin{column}{0.5\textwidth}
\centering
\only<1>{\providecommand\step{0}%
% Runtime's default exception handler
%
\subsection*{Runtime's default exception handler}
\frameSubsection{pictures/The\_Architect.png}{}


\begin{frame}{Default exception handler}
\begin{columns}
\begin{column}{0.5\textwidth}
\begin{itemize}
\item \localname{domain\_state->exn\_handler} points to a trap located before \funcname{entry}!
\item What installed this very first trap there?
\end{itemize}
\bigskip
\listocaml{../src/default/default.ml}{default/default.ml}
\bigskip
\inputminted[fontsize=\footnotesize]{text}{../src/default/default.out}
\end{column}
\begin{column}{0.5\textwidth}
\centering
\only<1>{\providecommand\step{0}\input{diagrams/default/default.tex}}%
\only<2>{\providecommand\step{4}\input{diagrams/default/default.tex}}%
\end{column}
\end{columns}
\end{frame}


\begin{frame}{Startup}{How did we get there by the way?}
\begin{columns}
\begin{column}{0.45\textwidth}
\begin{itemize}
\item The entry point address of the binary is \funcname{main} (\funcname{\_start}) from the runtime
\item The program starts like a C program:
\begin{itemize}
\item \funcname{caml\_start\_program}
\item \funcname{caml\_startup\_common}
\item \funcname{caml\_startup\_exn}
\item \funcname{caml\_startup}
\item \funcname{caml\_main}
\item \funcname{main}
\end{itemize}
%\item \funcname{caml\_startup\_common}: initializes GC, domains \dots
% Also: codefrag, locale, custom opeartions, frame_descr, signals, stack
\item \funcname{caml\_start\_program} is the transition point between the C realm and the OCaml one
\end{itemize}
\bigskip
\listocaml{../src/default/default.ml}{default/default.ml}
\end{column}
\begin{column}{0.55\textwidth}
\listasm[lastline=24,highlightlines={22-24}]{src/ocaml/caml\_start\_program.s}{runtime/amd64.S:604}
\end{column}
\end{columns}
\end{frame}


\begin{frame}{Setup to jump into OCaml entry point}{\funcname{caml\_start\_program} initializes the main fiber}
\begin{columns}
\begin{column}{0.6\textwidth}
\only<1,3,5>{
\begin{itemize}
\item<1-> Stores information to allow switching from OCaml stack to the C stack
\item<3-> Constructs the default exception handler
\item<5-> Switches to the OCaml stack and calls \funcname{caml\_program}
\end{itemize}
\bigskip
\listocaml{../src/default/default.ml}{default/default.ml}
}%
\only<2>{
\listasm[firstline=22,lastline=41,highlightlines={25-27}]{src/ocaml/caml\_start\_program.s}{caml\_start\_program}
}%
\only<4>{
\mintasm[firstline=22,lastline=41,highlightlines={31-38}]{src/ocaml/caml\_start\_program.s}
\listasm[firstline=66,lastline=72]{src/ocaml/caml\_start\_program.s}{caml\_start\_program}
}%
\only<6>{
\listasm[firstline=22,lastline=41,highlightlines={39-41}]{src/ocaml/caml\_start\_program.s}{caml\_start\_program}
}%
\end{column}
\begin{column}{0.4\textwidth}
\centering
\only<1-2>{\providecommand\step{2}\input{diagrams/default/default.tex}}%
\only<3-5>{\providecommand\step{3}\input{diagrams/default/default.tex}}%
\onslide*<6->{\providecommand\step{4}\input{diagrams/default/default.tex}}%
\end{column}
\end{columns}
\end{frame}

\begin{frame}{Executing the OCaml program}{And raising an exception without handler}
\begin{columns}
\begin{column}{0.6\textwidth}
\only<1-3>{
\begin{itemize}
\item<1-> Execution continues with OCaml code
\begin{itemize}
\item \funcname{camlDefault\_\_main\_269}
\item \funcname{camlDefault\_\_entry}
\item \funcname{caml\_program}
\end{itemize}
\item<1-> \funcname{camlDefault\_\_main\_269} raises a \typename{Failure} exception
\item<1-> \typename{Failure} is caught by \funcname{caml\_start\_program}
\begin{itemize}
\item<1-> The handler set the 2nd LSB of the exception value to \funcarg{1}
\item<3-> And then forces \funcname{caml\_start\_program} to terminate
\end{itemize}
\end{itemize}
}
\bigskip
\only<1,3>{\listocaml[highlightlines=3]{../src/default/default.ml}{default/default.ml}}%
\only<2>{\listasm[firstline=66,lastline=72,highlightlines=69]{src/ocaml/caml\_start\_program.s}{caml\_start\_program}}%
\end{column}
\begin{column}{0.4\textwidth}
\centering
\providecommand\step{4}\input{diagrams/default/default.tex}
\end{column}
\end{columns}
\end{frame}
%\only<>{\listasm[firstline=47,lastline=65]{src/ocaml/caml\_start\_program.s}{caml\_start\_program}}


\begin{frame}{Back to C}{Clean up, complain and terminate}
\begin{columns}
\begin{column}{0.45\textwidth}
\begin{itemize}
\item Backtrace:
\begin{itemize}
\item \funcname{caml\_start\_program} $\rightarrow$ OCaml code
\item \funcname{caml\_startup\_common}
\item \funcname{caml\_startup\_exn}
\item \funcname{caml\_startup}
\item \funcname{caml\_main}
\item \funcname{main}
\end{itemize}
\smallskip
\item \funcname{caml\_startup} checks the return value of \funcname{caml\_startup\_exn}
\begin{itemize}
\item The return value has been specially marked by \funcname{caml\_start\_program} handler
\item Calls \funcname{caml\_fatal\_uncaught\_exception}
\end{itemize}
\item<2-> \funcname{caml\_fatal\_uncaught\_exception} either
\begin{itemize}
\item<2-> Calls the user custom uncaught exception handler, if set with \mintinline{ocaml}{Printexc.set_uncaught_exception_handler fn}\footnotemark
\item<2-> Or falls back to the default one
\item<2-> Which notifies about the uncaught exception
\end{itemize}
\item<4-> Terminates the program either with \funcname{abort} or with a non-zero exit code
\end{itemize}
\end{column}
\begin{column}{0.55\textwidth}
\only<1>{
\listc[firstline=84,lastline=89,highlightlines=88]{../ocaml/runtime/caml/mlvalues.h}{runtime/caml/mlvalues.h:84}
\listc[firstline=138,lastline=143,highlightlines={141-142}]{../ocaml/runtime/startup\_nat.c}{runtime/startup\_nat.c:138}
}%
\only<2>{
\listc[firstline=138,highlightlines={147,149}]{../ocaml/runtime/printexc.c}{runtime/printexc.c:138}
}%
\only<3>{
\listc[firstline=110,lastline=134,highlightlines=129]{../ocaml/runtime/printexc.c}{runtime/printexc.c:138}
}%
\only<4>{
\listc[firstline=138,highlightlines={152,154}]{../ocaml/runtime/printexc.c}{runtime/printexc.c:138}
}%
\end{column}
\end{columns}
\footnotetext[1]{\url{https://v2.ocaml.org/api/Printexc.html\#1\_Uncaughtexceptions}}
\end{frame}
}%
\only<2>{\providecommand\step{4}%
% Runtime's default exception handler
%
\subsection*{Runtime's default exception handler}
\frameSubsection{pictures/The\_Architect.png}{}


\begin{frame}{Default exception handler}
\begin{columns}
\begin{column}{0.5\textwidth}
\begin{itemize}
\item \localname{domain\_state->exn\_handler} points to a trap located before \funcname{entry}!
\item What installed this very first trap there?
\end{itemize}
\bigskip
\listocaml{../src/default/default.ml}{default/default.ml}
\bigskip
\inputminted[fontsize=\footnotesize]{text}{../src/default/default.out}
\end{column}
\begin{column}{0.5\textwidth}
\centering
\only<1>{\providecommand\step{0}\input{diagrams/default/default.tex}}%
\only<2>{\providecommand\step{4}\input{diagrams/default/default.tex}}%
\end{column}
\end{columns}
\end{frame}


\begin{frame}{Startup}{How did we get there by the way?}
\begin{columns}
\begin{column}{0.45\textwidth}
\begin{itemize}
\item The entry point address of the binary is \funcname{main} (\funcname{\_start}) from the runtime
\item The program starts like a C program:
\begin{itemize}
\item \funcname{caml\_start\_program}
\item \funcname{caml\_startup\_common}
\item \funcname{caml\_startup\_exn}
\item \funcname{caml\_startup}
\item \funcname{caml\_main}
\item \funcname{main}
\end{itemize}
%\item \funcname{caml\_startup\_common}: initializes GC, domains \dots
% Also: codefrag, locale, custom opeartions, frame_descr, signals, stack
\item \funcname{caml\_start\_program} is the transition point between the C realm and the OCaml one
\end{itemize}
\bigskip
\listocaml{../src/default/default.ml}{default/default.ml}
\end{column}
\begin{column}{0.55\textwidth}
\listasm[lastline=24,highlightlines={22-24}]{src/ocaml/caml\_start\_program.s}{runtime/amd64.S:604}
\end{column}
\end{columns}
\end{frame}


\begin{frame}{Setup to jump into OCaml entry point}{\funcname{caml\_start\_program} initializes the main fiber}
\begin{columns}
\begin{column}{0.6\textwidth}
\only<1,3,5>{
\begin{itemize}
\item<1-> Stores information to allow switching from OCaml stack to the C stack
\item<3-> Constructs the default exception handler
\item<5-> Switches to the OCaml stack and calls \funcname{caml\_program}
\end{itemize}
\bigskip
\listocaml{../src/default/default.ml}{default/default.ml}
}%
\only<2>{
\listasm[firstline=22,lastline=41,highlightlines={25-27}]{src/ocaml/caml\_start\_program.s}{caml\_start\_program}
}%
\only<4>{
\mintasm[firstline=22,lastline=41,highlightlines={31-38}]{src/ocaml/caml\_start\_program.s}
\listasm[firstline=66,lastline=72]{src/ocaml/caml\_start\_program.s}{caml\_start\_program}
}%
\only<6>{
\listasm[firstline=22,lastline=41,highlightlines={39-41}]{src/ocaml/caml\_start\_program.s}{caml\_start\_program}
}%
\end{column}
\begin{column}{0.4\textwidth}
\centering
\only<1-2>{\providecommand\step{2}\input{diagrams/default/default.tex}}%
\only<3-5>{\providecommand\step{3}\input{diagrams/default/default.tex}}%
\onslide*<6->{\providecommand\step{4}\input{diagrams/default/default.tex}}%
\end{column}
\end{columns}
\end{frame}

\begin{frame}{Executing the OCaml program}{And raising an exception without handler}
\begin{columns}
\begin{column}{0.6\textwidth}
\only<1-3>{
\begin{itemize}
\item<1-> Execution continues with OCaml code
\begin{itemize}
\item \funcname{camlDefault\_\_main\_269}
\item \funcname{camlDefault\_\_entry}
\item \funcname{caml\_program}
\end{itemize}
\item<1-> \funcname{camlDefault\_\_main\_269} raises a \typename{Failure} exception
\item<1-> \typename{Failure} is caught by \funcname{caml\_start\_program}
\begin{itemize}
\item<1-> The handler set the 2nd LSB of the exception value to \funcarg{1}
\item<3-> And then forces \funcname{caml\_start\_program} to terminate
\end{itemize}
\end{itemize}
}
\bigskip
\only<1,3>{\listocaml[highlightlines=3]{../src/default/default.ml}{default/default.ml}}%
\only<2>{\listasm[firstline=66,lastline=72,highlightlines=69]{src/ocaml/caml\_start\_program.s}{caml\_start\_program}}%
\end{column}
\begin{column}{0.4\textwidth}
\centering
\providecommand\step{4}\input{diagrams/default/default.tex}
\end{column}
\end{columns}
\end{frame}
%\only<>{\listasm[firstline=47,lastline=65]{src/ocaml/caml\_start\_program.s}{caml\_start\_program}}


\begin{frame}{Back to C}{Clean up, complain and terminate}
\begin{columns}
\begin{column}{0.45\textwidth}
\begin{itemize}
\item Backtrace:
\begin{itemize}
\item \funcname{caml\_start\_program} $\rightarrow$ OCaml code
\item \funcname{caml\_startup\_common}
\item \funcname{caml\_startup\_exn}
\item \funcname{caml\_startup}
\item \funcname{caml\_main}
\item \funcname{main}
\end{itemize}
\smallskip
\item \funcname{caml\_startup} checks the return value of \funcname{caml\_startup\_exn}
\begin{itemize}
\item The return value has been specially marked by \funcname{caml\_start\_program} handler
\item Calls \funcname{caml\_fatal\_uncaught\_exception}
\end{itemize}
\item<2-> \funcname{caml\_fatal\_uncaught\_exception} either
\begin{itemize}
\item<2-> Calls the user custom uncaught exception handler, if set with \mintinline{ocaml}{Printexc.set_uncaught_exception_handler fn}\footnotemark
\item<2-> Or falls back to the default one
\item<2-> Which notifies about the uncaught exception
\end{itemize}
\item<4-> Terminates the program either with \funcname{abort} or with a non-zero exit code
\end{itemize}
\end{column}
\begin{column}{0.55\textwidth}
\only<1>{
\listc[firstline=84,lastline=89,highlightlines=88]{../ocaml/runtime/caml/mlvalues.h}{runtime/caml/mlvalues.h:84}
\listc[firstline=138,lastline=143,highlightlines={141-142}]{../ocaml/runtime/startup\_nat.c}{runtime/startup\_nat.c:138}
}%
\only<2>{
\listc[firstline=138,highlightlines={147,149}]{../ocaml/runtime/printexc.c}{runtime/printexc.c:138}
}%
\only<3>{
\listc[firstline=110,lastline=134,highlightlines=129]{../ocaml/runtime/printexc.c}{runtime/printexc.c:138}
}%
\only<4>{
\listc[firstline=138,highlightlines={152,154}]{../ocaml/runtime/printexc.c}{runtime/printexc.c:138}
}%
\end{column}
\end{columns}
\footnotetext[1]{\url{https://v2.ocaml.org/api/Printexc.html\#1\_Uncaughtexceptions}}
\end{frame}
}%
\end{column}
\end{columns}
\end{frame}


\begin{frame}{Startup}{How did we get there by the way?}
\begin{columns}
\begin{column}{0.45\textwidth}
\begin{itemize}
\item The entry point address of the binary is \funcname{main} (\funcname{\_start}) from the runtime
\item The program starts like a C program:
\begin{itemize}
\item \funcname{caml\_start\_program}
\item \funcname{caml\_startup\_common}
\item \funcname{caml\_startup\_exn}
\item \funcname{caml\_startup}
\item \funcname{caml\_main}
\item \funcname{main}
\end{itemize}
%\item \funcname{caml\_startup\_common}: initializes GC, domains \dots
% Also: codefrag, locale, custom opeartions, frame_descr, signals, stack
\item \funcname{caml\_start\_program} is the transition point between the C realm and the OCaml one
\end{itemize}
\bigskip
\listocaml{../src/default/default.ml}{default/default.ml}
\end{column}
\begin{column}{0.55\textwidth}
\listasm[lastline=24,highlightlines={22-24}]{src/ocaml/caml\_start\_program.s}{runtime/amd64.S:604}
\end{column}
\end{columns}
\end{frame}


\begin{frame}{Setup to jump into OCaml entry point}{\funcname{caml\_start\_program} initializes the main fiber}
\begin{columns}
\begin{column}{0.6\textwidth}
\only<1,3,5>{
\begin{itemize}
\item<1-> Stores information to allow switching from OCaml stack to the C stack
\item<3-> Constructs the default exception handler
\item<5-> Switches to the OCaml stack and calls \funcname{caml\_program}
\end{itemize}
\bigskip
\listocaml{../src/default/default.ml}{default/default.ml}
}%
\only<2>{
\listasm[firstline=22,lastline=41,highlightlines={25-27}]{src/ocaml/caml\_start\_program.s}{caml\_start\_program}
}%
\only<4>{
\mintasm[firstline=22,lastline=41,highlightlines={31-38}]{src/ocaml/caml\_start\_program.s}
\listasm[firstline=66,lastline=72]{src/ocaml/caml\_start\_program.s}{caml\_start\_program}
}%
\only<6>{
\listasm[firstline=22,lastline=41,highlightlines={39-41}]{src/ocaml/caml\_start\_program.s}{caml\_start\_program}
}%
\end{column}
\begin{column}{0.4\textwidth}
\centering
\only<1-2>{\providecommand\step{2}%
% Runtime's default exception handler
%
\subsection*{Runtime's default exception handler}
\frameSubsection{pictures/The\_Architect.png}{}


\begin{frame}{Default exception handler}
\begin{columns}
\begin{column}{0.5\textwidth}
\begin{itemize}
\item \localname{domain\_state->exn\_handler} points to a trap located before \funcname{entry}!
\item What installed this very first trap there?
\end{itemize}
\bigskip
\listocaml{../src/default/default.ml}{default/default.ml}
\bigskip
\inputminted[fontsize=\footnotesize]{text}{../src/default/default.out}
\end{column}
\begin{column}{0.5\textwidth}
\centering
\only<1>{\providecommand\step{0}\input{diagrams/default/default.tex}}%
\only<2>{\providecommand\step{4}\input{diagrams/default/default.tex}}%
\end{column}
\end{columns}
\end{frame}


\begin{frame}{Startup}{How did we get there by the way?}
\begin{columns}
\begin{column}{0.45\textwidth}
\begin{itemize}
\item The entry point address of the binary is \funcname{main} (\funcname{\_start}) from the runtime
\item The program starts like a C program:
\begin{itemize}
\item \funcname{caml\_start\_program}
\item \funcname{caml\_startup\_common}
\item \funcname{caml\_startup\_exn}
\item \funcname{caml\_startup}
\item \funcname{caml\_main}
\item \funcname{main}
\end{itemize}
%\item \funcname{caml\_startup\_common}: initializes GC, domains \dots
% Also: codefrag, locale, custom opeartions, frame_descr, signals, stack
\item \funcname{caml\_start\_program} is the transition point between the C realm and the OCaml one
\end{itemize}
\bigskip
\listocaml{../src/default/default.ml}{default/default.ml}
\end{column}
\begin{column}{0.55\textwidth}
\listasm[lastline=24,highlightlines={22-24}]{src/ocaml/caml\_start\_program.s}{runtime/amd64.S:604}
\end{column}
\end{columns}
\end{frame}


\begin{frame}{Setup to jump into OCaml entry point}{\funcname{caml\_start\_program} initializes the main fiber}
\begin{columns}
\begin{column}{0.6\textwidth}
\only<1,3,5>{
\begin{itemize}
\item<1-> Stores information to allow switching from OCaml stack to the C stack
\item<3-> Constructs the default exception handler
\item<5-> Switches to the OCaml stack and calls \funcname{caml\_program}
\end{itemize}
\bigskip
\listocaml{../src/default/default.ml}{default/default.ml}
}%
\only<2>{
\listasm[firstline=22,lastline=41,highlightlines={25-27}]{src/ocaml/caml\_start\_program.s}{caml\_start\_program}
}%
\only<4>{
\mintasm[firstline=22,lastline=41,highlightlines={31-38}]{src/ocaml/caml\_start\_program.s}
\listasm[firstline=66,lastline=72]{src/ocaml/caml\_start\_program.s}{caml\_start\_program}
}%
\only<6>{
\listasm[firstline=22,lastline=41,highlightlines={39-41}]{src/ocaml/caml\_start\_program.s}{caml\_start\_program}
}%
\end{column}
\begin{column}{0.4\textwidth}
\centering
\only<1-2>{\providecommand\step{2}\input{diagrams/default/default.tex}}%
\only<3-5>{\providecommand\step{3}\input{diagrams/default/default.tex}}%
\onslide*<6->{\providecommand\step{4}\input{diagrams/default/default.tex}}%
\end{column}
\end{columns}
\end{frame}

\begin{frame}{Executing the OCaml program}{And raising an exception without handler}
\begin{columns}
\begin{column}{0.6\textwidth}
\only<1-3>{
\begin{itemize}
\item<1-> Execution continues with OCaml code
\begin{itemize}
\item \funcname{camlDefault\_\_main\_269}
\item \funcname{camlDefault\_\_entry}
\item \funcname{caml\_program}
\end{itemize}
\item<1-> \funcname{camlDefault\_\_main\_269} raises a \typename{Failure} exception
\item<1-> \typename{Failure} is caught by \funcname{caml\_start\_program}
\begin{itemize}
\item<1-> The handler set the 2nd LSB of the exception value to \funcarg{1}
\item<3-> And then forces \funcname{caml\_start\_program} to terminate
\end{itemize}
\end{itemize}
}
\bigskip
\only<1,3>{\listocaml[highlightlines=3]{../src/default/default.ml}{default/default.ml}}%
\only<2>{\listasm[firstline=66,lastline=72,highlightlines=69]{src/ocaml/caml\_start\_program.s}{caml\_start\_program}}%
\end{column}
\begin{column}{0.4\textwidth}
\centering
\providecommand\step{4}\input{diagrams/default/default.tex}
\end{column}
\end{columns}
\end{frame}
%\only<>{\listasm[firstline=47,lastline=65]{src/ocaml/caml\_start\_program.s}{caml\_start\_program}}


\begin{frame}{Back to C}{Clean up, complain and terminate}
\begin{columns}
\begin{column}{0.45\textwidth}
\begin{itemize}
\item Backtrace:
\begin{itemize}
\item \funcname{caml\_start\_program} $\rightarrow$ OCaml code
\item \funcname{caml\_startup\_common}
\item \funcname{caml\_startup\_exn}
\item \funcname{caml\_startup}
\item \funcname{caml\_main}
\item \funcname{main}
\end{itemize}
\smallskip
\item \funcname{caml\_startup} checks the return value of \funcname{caml\_startup\_exn}
\begin{itemize}
\item The return value has been specially marked by \funcname{caml\_start\_program} handler
\item Calls \funcname{caml\_fatal\_uncaught\_exception}
\end{itemize}
\item<2-> \funcname{caml\_fatal\_uncaught\_exception} either
\begin{itemize}
\item<2-> Calls the user custom uncaught exception handler, if set with \mintinline{ocaml}{Printexc.set_uncaught_exception_handler fn}\footnotemark
\item<2-> Or falls back to the default one
\item<2-> Which notifies about the uncaught exception
\end{itemize}
\item<4-> Terminates the program either with \funcname{abort} or with a non-zero exit code
\end{itemize}
\end{column}
\begin{column}{0.55\textwidth}
\only<1>{
\listc[firstline=84,lastline=89,highlightlines=88]{../ocaml/runtime/caml/mlvalues.h}{runtime/caml/mlvalues.h:84}
\listc[firstline=138,lastline=143,highlightlines={141-142}]{../ocaml/runtime/startup\_nat.c}{runtime/startup\_nat.c:138}
}%
\only<2>{
\listc[firstline=138,highlightlines={147,149}]{../ocaml/runtime/printexc.c}{runtime/printexc.c:138}
}%
\only<3>{
\listc[firstline=110,lastline=134,highlightlines=129]{../ocaml/runtime/printexc.c}{runtime/printexc.c:138}
}%
\only<4>{
\listc[firstline=138,highlightlines={152,154}]{../ocaml/runtime/printexc.c}{runtime/printexc.c:138}
}%
\end{column}
\end{columns}
\footnotetext[1]{\url{https://v2.ocaml.org/api/Printexc.html\#1\_Uncaughtexceptions}}
\end{frame}
}%
\only<3-5>{\providecommand\step{3}%
% Runtime's default exception handler
%
\subsection*{Runtime's default exception handler}
\frameSubsection{pictures/The\_Architect.png}{}


\begin{frame}{Default exception handler}
\begin{columns}
\begin{column}{0.5\textwidth}
\begin{itemize}
\item \localname{domain\_state->exn\_handler} points to a trap located before \funcname{entry}!
\item What installed this very first trap there?
\end{itemize}
\bigskip
\listocaml{../src/default/default.ml}{default/default.ml}
\bigskip
\inputminted[fontsize=\footnotesize]{text}{../src/default/default.out}
\end{column}
\begin{column}{0.5\textwidth}
\centering
\only<1>{\providecommand\step{0}\input{diagrams/default/default.tex}}%
\only<2>{\providecommand\step{4}\input{diagrams/default/default.tex}}%
\end{column}
\end{columns}
\end{frame}


\begin{frame}{Startup}{How did we get there by the way?}
\begin{columns}
\begin{column}{0.45\textwidth}
\begin{itemize}
\item The entry point address of the binary is \funcname{main} (\funcname{\_start}) from the runtime
\item The program starts like a C program:
\begin{itemize}
\item \funcname{caml\_start\_program}
\item \funcname{caml\_startup\_common}
\item \funcname{caml\_startup\_exn}
\item \funcname{caml\_startup}
\item \funcname{caml\_main}
\item \funcname{main}
\end{itemize}
%\item \funcname{caml\_startup\_common}: initializes GC, domains \dots
% Also: codefrag, locale, custom opeartions, frame_descr, signals, stack
\item \funcname{caml\_start\_program} is the transition point between the C realm and the OCaml one
\end{itemize}
\bigskip
\listocaml{../src/default/default.ml}{default/default.ml}
\end{column}
\begin{column}{0.55\textwidth}
\listasm[lastline=24,highlightlines={22-24}]{src/ocaml/caml\_start\_program.s}{runtime/amd64.S:604}
\end{column}
\end{columns}
\end{frame}


\begin{frame}{Setup to jump into OCaml entry point}{\funcname{caml\_start\_program} initializes the main fiber}
\begin{columns}
\begin{column}{0.6\textwidth}
\only<1,3,5>{
\begin{itemize}
\item<1-> Stores information to allow switching from OCaml stack to the C stack
\item<3-> Constructs the default exception handler
\item<5-> Switches to the OCaml stack and calls \funcname{caml\_program}
\end{itemize}
\bigskip
\listocaml{../src/default/default.ml}{default/default.ml}
}%
\only<2>{
\listasm[firstline=22,lastline=41,highlightlines={25-27}]{src/ocaml/caml\_start\_program.s}{caml\_start\_program}
}%
\only<4>{
\mintasm[firstline=22,lastline=41,highlightlines={31-38}]{src/ocaml/caml\_start\_program.s}
\listasm[firstline=66,lastline=72]{src/ocaml/caml\_start\_program.s}{caml\_start\_program}
}%
\only<6>{
\listasm[firstline=22,lastline=41,highlightlines={39-41}]{src/ocaml/caml\_start\_program.s}{caml\_start\_program}
}%
\end{column}
\begin{column}{0.4\textwidth}
\centering
\only<1-2>{\providecommand\step{2}\input{diagrams/default/default.tex}}%
\only<3-5>{\providecommand\step{3}\input{diagrams/default/default.tex}}%
\onslide*<6->{\providecommand\step{4}\input{diagrams/default/default.tex}}%
\end{column}
\end{columns}
\end{frame}

\begin{frame}{Executing the OCaml program}{And raising an exception without handler}
\begin{columns}
\begin{column}{0.6\textwidth}
\only<1-3>{
\begin{itemize}
\item<1-> Execution continues with OCaml code
\begin{itemize}
\item \funcname{camlDefault\_\_main\_269}
\item \funcname{camlDefault\_\_entry}
\item \funcname{caml\_program}
\end{itemize}
\item<1-> \funcname{camlDefault\_\_main\_269} raises a \typename{Failure} exception
\item<1-> \typename{Failure} is caught by \funcname{caml\_start\_program}
\begin{itemize}
\item<1-> The handler set the 2nd LSB of the exception value to \funcarg{1}
\item<3-> And then forces \funcname{caml\_start\_program} to terminate
\end{itemize}
\end{itemize}
}
\bigskip
\only<1,3>{\listocaml[highlightlines=3]{../src/default/default.ml}{default/default.ml}}%
\only<2>{\listasm[firstline=66,lastline=72,highlightlines=69]{src/ocaml/caml\_start\_program.s}{caml\_start\_program}}%
\end{column}
\begin{column}{0.4\textwidth}
\centering
\providecommand\step{4}\input{diagrams/default/default.tex}
\end{column}
\end{columns}
\end{frame}
%\only<>{\listasm[firstline=47,lastline=65]{src/ocaml/caml\_start\_program.s}{caml\_start\_program}}


\begin{frame}{Back to C}{Clean up, complain and terminate}
\begin{columns}
\begin{column}{0.45\textwidth}
\begin{itemize}
\item Backtrace:
\begin{itemize}
\item \funcname{caml\_start\_program} $\rightarrow$ OCaml code
\item \funcname{caml\_startup\_common}
\item \funcname{caml\_startup\_exn}
\item \funcname{caml\_startup}
\item \funcname{caml\_main}
\item \funcname{main}
\end{itemize}
\smallskip
\item \funcname{caml\_startup} checks the return value of \funcname{caml\_startup\_exn}
\begin{itemize}
\item The return value has been specially marked by \funcname{caml\_start\_program} handler
\item Calls \funcname{caml\_fatal\_uncaught\_exception}
\end{itemize}
\item<2-> \funcname{caml\_fatal\_uncaught\_exception} either
\begin{itemize}
\item<2-> Calls the user custom uncaught exception handler, if set with \mintinline{ocaml}{Printexc.set_uncaught_exception_handler fn}\footnotemark
\item<2-> Or falls back to the default one
\item<2-> Which notifies about the uncaught exception
\end{itemize}
\item<4-> Terminates the program either with \funcname{abort} or with a non-zero exit code
\end{itemize}
\end{column}
\begin{column}{0.55\textwidth}
\only<1>{
\listc[firstline=84,lastline=89,highlightlines=88]{../ocaml/runtime/caml/mlvalues.h}{runtime/caml/mlvalues.h:84}
\listc[firstline=138,lastline=143,highlightlines={141-142}]{../ocaml/runtime/startup\_nat.c}{runtime/startup\_nat.c:138}
}%
\only<2>{
\listc[firstline=138,highlightlines={147,149}]{../ocaml/runtime/printexc.c}{runtime/printexc.c:138}
}%
\only<3>{
\listc[firstline=110,lastline=134,highlightlines=129]{../ocaml/runtime/printexc.c}{runtime/printexc.c:138}
}%
\only<4>{
\listc[firstline=138,highlightlines={152,154}]{../ocaml/runtime/printexc.c}{runtime/printexc.c:138}
}%
\end{column}
\end{columns}
\footnotetext[1]{\url{https://v2.ocaml.org/api/Printexc.html\#1\_Uncaughtexceptions}}
\end{frame}
}%
\onslide*<6->{\providecommand\step{4}%
% Runtime's default exception handler
%
\subsection*{Runtime's default exception handler}
\frameSubsection{pictures/The\_Architect.png}{}


\begin{frame}{Default exception handler}
\begin{columns}
\begin{column}{0.5\textwidth}
\begin{itemize}
\item \localname{domain\_state->exn\_handler} points to a trap located before \funcname{entry}!
\item What installed this very first trap there?
\end{itemize}
\bigskip
\listocaml{../src/default/default.ml}{default/default.ml}
\bigskip
\inputminted[fontsize=\footnotesize]{text}{../src/default/default.out}
\end{column}
\begin{column}{0.5\textwidth}
\centering
\only<1>{\providecommand\step{0}\input{diagrams/default/default.tex}}%
\only<2>{\providecommand\step{4}\input{diagrams/default/default.tex}}%
\end{column}
\end{columns}
\end{frame}


\begin{frame}{Startup}{How did we get there by the way?}
\begin{columns}
\begin{column}{0.45\textwidth}
\begin{itemize}
\item The entry point address of the binary is \funcname{main} (\funcname{\_start}) from the runtime
\item The program starts like a C program:
\begin{itemize}
\item \funcname{caml\_start\_program}
\item \funcname{caml\_startup\_common}
\item \funcname{caml\_startup\_exn}
\item \funcname{caml\_startup}
\item \funcname{caml\_main}
\item \funcname{main}
\end{itemize}
%\item \funcname{caml\_startup\_common}: initializes GC, domains \dots
% Also: codefrag, locale, custom opeartions, frame_descr, signals, stack
\item \funcname{caml\_start\_program} is the transition point between the C realm and the OCaml one
\end{itemize}
\bigskip
\listocaml{../src/default/default.ml}{default/default.ml}
\end{column}
\begin{column}{0.55\textwidth}
\listasm[lastline=24,highlightlines={22-24}]{src/ocaml/caml\_start\_program.s}{runtime/amd64.S:604}
\end{column}
\end{columns}
\end{frame}


\begin{frame}{Setup to jump into OCaml entry point}{\funcname{caml\_start\_program} initializes the main fiber}
\begin{columns}
\begin{column}{0.6\textwidth}
\only<1,3,5>{
\begin{itemize}
\item<1-> Stores information to allow switching from OCaml stack to the C stack
\item<3-> Constructs the default exception handler
\item<5-> Switches to the OCaml stack and calls \funcname{caml\_program}
\end{itemize}
\bigskip
\listocaml{../src/default/default.ml}{default/default.ml}
}%
\only<2>{
\listasm[firstline=22,lastline=41,highlightlines={25-27}]{src/ocaml/caml\_start\_program.s}{caml\_start\_program}
}%
\only<4>{
\mintasm[firstline=22,lastline=41,highlightlines={31-38}]{src/ocaml/caml\_start\_program.s}
\listasm[firstline=66,lastline=72]{src/ocaml/caml\_start\_program.s}{caml\_start\_program}
}%
\only<6>{
\listasm[firstline=22,lastline=41,highlightlines={39-41}]{src/ocaml/caml\_start\_program.s}{caml\_start\_program}
}%
\end{column}
\begin{column}{0.4\textwidth}
\centering
\only<1-2>{\providecommand\step{2}\input{diagrams/default/default.tex}}%
\only<3-5>{\providecommand\step{3}\input{diagrams/default/default.tex}}%
\onslide*<6->{\providecommand\step{4}\input{diagrams/default/default.tex}}%
\end{column}
\end{columns}
\end{frame}

\begin{frame}{Executing the OCaml program}{And raising an exception without handler}
\begin{columns}
\begin{column}{0.6\textwidth}
\only<1-3>{
\begin{itemize}
\item<1-> Execution continues with OCaml code
\begin{itemize}
\item \funcname{camlDefault\_\_main\_269}
\item \funcname{camlDefault\_\_entry}
\item \funcname{caml\_program}
\end{itemize}
\item<1-> \funcname{camlDefault\_\_main\_269} raises a \typename{Failure} exception
\item<1-> \typename{Failure} is caught by \funcname{caml\_start\_program}
\begin{itemize}
\item<1-> The handler set the 2nd LSB of the exception value to \funcarg{1}
\item<3-> And then forces \funcname{caml\_start\_program} to terminate
\end{itemize}
\end{itemize}
}
\bigskip
\only<1,3>{\listocaml[highlightlines=3]{../src/default/default.ml}{default/default.ml}}%
\only<2>{\listasm[firstline=66,lastline=72,highlightlines=69]{src/ocaml/caml\_start\_program.s}{caml\_start\_program}}%
\end{column}
\begin{column}{0.4\textwidth}
\centering
\providecommand\step{4}\input{diagrams/default/default.tex}
\end{column}
\end{columns}
\end{frame}
%\only<>{\listasm[firstline=47,lastline=65]{src/ocaml/caml\_start\_program.s}{caml\_start\_program}}


\begin{frame}{Back to C}{Clean up, complain and terminate}
\begin{columns}
\begin{column}{0.45\textwidth}
\begin{itemize}
\item Backtrace:
\begin{itemize}
\item \funcname{caml\_start\_program} $\rightarrow$ OCaml code
\item \funcname{caml\_startup\_common}
\item \funcname{caml\_startup\_exn}
\item \funcname{caml\_startup}
\item \funcname{caml\_main}
\item \funcname{main}
\end{itemize}
\smallskip
\item \funcname{caml\_startup} checks the return value of \funcname{caml\_startup\_exn}
\begin{itemize}
\item The return value has been specially marked by \funcname{caml\_start\_program} handler
\item Calls \funcname{caml\_fatal\_uncaught\_exception}
\end{itemize}
\item<2-> \funcname{caml\_fatal\_uncaught\_exception} either
\begin{itemize}
\item<2-> Calls the user custom uncaught exception handler, if set with \mintinline{ocaml}{Printexc.set_uncaught_exception_handler fn}\footnotemark
\item<2-> Or falls back to the default one
\item<2-> Which notifies about the uncaught exception
\end{itemize}
\item<4-> Terminates the program either with \funcname{abort} or with a non-zero exit code
\end{itemize}
\end{column}
\begin{column}{0.55\textwidth}
\only<1>{
\listc[firstline=84,lastline=89,highlightlines=88]{../ocaml/runtime/caml/mlvalues.h}{runtime/caml/mlvalues.h:84}
\listc[firstline=138,lastline=143,highlightlines={141-142}]{../ocaml/runtime/startup\_nat.c}{runtime/startup\_nat.c:138}
}%
\only<2>{
\listc[firstline=138,highlightlines={147,149}]{../ocaml/runtime/printexc.c}{runtime/printexc.c:138}
}%
\only<3>{
\listc[firstline=110,lastline=134,highlightlines=129]{../ocaml/runtime/printexc.c}{runtime/printexc.c:138}
}%
\only<4>{
\listc[firstline=138,highlightlines={152,154}]{../ocaml/runtime/printexc.c}{runtime/printexc.c:138}
}%
\end{column}
\end{columns}
\footnotetext[1]{\url{https://v2.ocaml.org/api/Printexc.html\#1\_Uncaughtexceptions}}
\end{frame}
}%
\end{column}
\end{columns}
\end{frame}

\begin{frame}{Executing the OCaml program}{And raising an exception without handler}
\begin{columns}
\begin{column}{0.6\textwidth}
\only<1-3>{
\begin{itemize}
\item<1-> Execution continues with OCaml code
\begin{itemize}
\item \funcname{camlDefault\_\_main\_269}
\item \funcname{camlDefault\_\_entry}
\item \funcname{caml\_program}
\end{itemize}
\item<1-> \funcname{camlDefault\_\_main\_269} raises a \typename{Failure} exception
\item<1-> \typename{Failure} is caught by \funcname{caml\_start\_program}
\begin{itemize}
\item<1-> The handler set the 2nd LSB of the exception value to \funcarg{1}
\item<3-> And then forces \funcname{caml\_start\_program} to terminate
\end{itemize}
\end{itemize}
}
\bigskip
\only<1,3>{\listocaml[highlightlines=3]{../src/default/default.ml}{default/default.ml}}%
\only<2>{\listasm[firstline=66,lastline=72,highlightlines=69]{src/ocaml/caml\_start\_program.s}{caml\_start\_program}}%
\end{column}
\begin{column}{0.4\textwidth}
\centering
\providecommand\step{4}%
% Runtime's default exception handler
%
\subsection*{Runtime's default exception handler}
\frameSubsection{pictures/The\_Architect.png}{}


\begin{frame}{Default exception handler}
\begin{columns}
\begin{column}{0.5\textwidth}
\begin{itemize}
\item \localname{domain\_state->exn\_handler} points to a trap located before \funcname{entry}!
\item What installed this very first trap there?
\end{itemize}
\bigskip
\listocaml{../src/default/default.ml}{default/default.ml}
\bigskip
\inputminted[fontsize=\footnotesize]{text}{../src/default/default.out}
\end{column}
\begin{column}{0.5\textwidth}
\centering
\only<1>{\providecommand\step{0}\input{diagrams/default/default.tex}}%
\only<2>{\providecommand\step{4}\input{diagrams/default/default.tex}}%
\end{column}
\end{columns}
\end{frame}


\begin{frame}{Startup}{How did we get there by the way?}
\begin{columns}
\begin{column}{0.45\textwidth}
\begin{itemize}
\item The entry point address of the binary is \funcname{main} (\funcname{\_start}) from the runtime
\item The program starts like a C program:
\begin{itemize}
\item \funcname{caml\_start\_program}
\item \funcname{caml\_startup\_common}
\item \funcname{caml\_startup\_exn}
\item \funcname{caml\_startup}
\item \funcname{caml\_main}
\item \funcname{main}
\end{itemize}
%\item \funcname{caml\_startup\_common}: initializes GC, domains \dots
% Also: codefrag, locale, custom opeartions, frame_descr, signals, stack
\item \funcname{caml\_start\_program} is the transition point between the C realm and the OCaml one
\end{itemize}
\bigskip
\listocaml{../src/default/default.ml}{default/default.ml}
\end{column}
\begin{column}{0.55\textwidth}
\listasm[lastline=24,highlightlines={22-24}]{src/ocaml/caml\_start\_program.s}{runtime/amd64.S:604}
\end{column}
\end{columns}
\end{frame}


\begin{frame}{Setup to jump into OCaml entry point}{\funcname{caml\_start\_program} initializes the main fiber}
\begin{columns}
\begin{column}{0.6\textwidth}
\only<1,3,5>{
\begin{itemize}
\item<1-> Stores information to allow switching from OCaml stack to the C stack
\item<3-> Constructs the default exception handler
\item<5-> Switches to the OCaml stack and calls \funcname{caml\_program}
\end{itemize}
\bigskip
\listocaml{../src/default/default.ml}{default/default.ml}
}%
\only<2>{
\listasm[firstline=22,lastline=41,highlightlines={25-27}]{src/ocaml/caml\_start\_program.s}{caml\_start\_program}
}%
\only<4>{
\mintasm[firstline=22,lastline=41,highlightlines={31-38}]{src/ocaml/caml\_start\_program.s}
\listasm[firstline=66,lastline=72]{src/ocaml/caml\_start\_program.s}{caml\_start\_program}
}%
\only<6>{
\listasm[firstline=22,lastline=41,highlightlines={39-41}]{src/ocaml/caml\_start\_program.s}{caml\_start\_program}
}%
\end{column}
\begin{column}{0.4\textwidth}
\centering
\only<1-2>{\providecommand\step{2}\input{diagrams/default/default.tex}}%
\only<3-5>{\providecommand\step{3}\input{diagrams/default/default.tex}}%
\onslide*<6->{\providecommand\step{4}\input{diagrams/default/default.tex}}%
\end{column}
\end{columns}
\end{frame}

\begin{frame}{Executing the OCaml program}{And raising an exception without handler}
\begin{columns}
\begin{column}{0.6\textwidth}
\only<1-3>{
\begin{itemize}
\item<1-> Execution continues with OCaml code
\begin{itemize}
\item \funcname{camlDefault\_\_main\_269}
\item \funcname{camlDefault\_\_entry}
\item \funcname{caml\_program}
\end{itemize}
\item<1-> \funcname{camlDefault\_\_main\_269} raises a \typename{Failure} exception
\item<1-> \typename{Failure} is caught by \funcname{caml\_start\_program}
\begin{itemize}
\item<1-> The handler set the 2nd LSB of the exception value to \funcarg{1}
\item<3-> And then forces \funcname{caml\_start\_program} to terminate
\end{itemize}
\end{itemize}
}
\bigskip
\only<1,3>{\listocaml[highlightlines=3]{../src/default/default.ml}{default/default.ml}}%
\only<2>{\listasm[firstline=66,lastline=72,highlightlines=69]{src/ocaml/caml\_start\_program.s}{caml\_start\_program}}%
\end{column}
\begin{column}{0.4\textwidth}
\centering
\providecommand\step{4}\input{diagrams/default/default.tex}
\end{column}
\end{columns}
\end{frame}
%\only<>{\listasm[firstline=47,lastline=65]{src/ocaml/caml\_start\_program.s}{caml\_start\_program}}


\begin{frame}{Back to C}{Clean up, complain and terminate}
\begin{columns}
\begin{column}{0.45\textwidth}
\begin{itemize}
\item Backtrace:
\begin{itemize}
\item \funcname{caml\_start\_program} $\rightarrow$ OCaml code
\item \funcname{caml\_startup\_common}
\item \funcname{caml\_startup\_exn}
\item \funcname{caml\_startup}
\item \funcname{caml\_main}
\item \funcname{main}
\end{itemize}
\smallskip
\item \funcname{caml\_startup} checks the return value of \funcname{caml\_startup\_exn}
\begin{itemize}
\item The return value has been specially marked by \funcname{caml\_start\_program} handler
\item Calls \funcname{caml\_fatal\_uncaught\_exception}
\end{itemize}
\item<2-> \funcname{caml\_fatal\_uncaught\_exception} either
\begin{itemize}
\item<2-> Calls the user custom uncaught exception handler, if set with \mintinline{ocaml}{Printexc.set_uncaught_exception_handler fn}\footnotemark
\item<2-> Or falls back to the default one
\item<2-> Which notifies about the uncaught exception
\end{itemize}
\item<4-> Terminates the program either with \funcname{abort} or with a non-zero exit code
\end{itemize}
\end{column}
\begin{column}{0.55\textwidth}
\only<1>{
\listc[firstline=84,lastline=89,highlightlines=88]{../ocaml/runtime/caml/mlvalues.h}{runtime/caml/mlvalues.h:84}
\listc[firstline=138,lastline=143,highlightlines={141-142}]{../ocaml/runtime/startup\_nat.c}{runtime/startup\_nat.c:138}
}%
\only<2>{
\listc[firstline=138,highlightlines={147,149}]{../ocaml/runtime/printexc.c}{runtime/printexc.c:138}
}%
\only<3>{
\listc[firstline=110,lastline=134,highlightlines=129]{../ocaml/runtime/printexc.c}{runtime/printexc.c:138}
}%
\only<4>{
\listc[firstline=138,highlightlines={152,154}]{../ocaml/runtime/printexc.c}{runtime/printexc.c:138}
}%
\end{column}
\end{columns}
\footnotetext[1]{\url{https://v2.ocaml.org/api/Printexc.html\#1\_Uncaughtexceptions}}
\end{frame}

\end{column}
\end{columns}
\end{frame}
%\only<>{\listasm[firstline=47,lastline=65]{src/ocaml/caml\_start\_program.s}{caml\_start\_program}}


\begin{frame}{Back to C}{Clean up, complain and terminate}
\begin{columns}
\begin{column}{0.45\textwidth}
\begin{itemize}
\item Backtrace:
\begin{itemize}
\item \funcname{caml\_start\_program} $\rightarrow$ OCaml code
\item \funcname{caml\_startup\_common}
\item \funcname{caml\_startup\_exn}
\item \funcname{caml\_startup}
\item \funcname{caml\_main}
\item \funcname{main}
\end{itemize}
\smallskip
\item \funcname{caml\_startup} checks the return value of \funcname{caml\_startup\_exn}
\begin{itemize}
\item The return value has been specially marked by \funcname{caml\_start\_program} handler
\item Calls \funcname{caml\_fatal\_uncaught\_exception}
\end{itemize}
\item<2-> \funcname{caml\_fatal\_uncaught\_exception} either
\begin{itemize}
\item<2-> Calls the user custom uncaught exception handler, if set with \mintinline{ocaml}{Printexc.set_uncaught_exception_handler fn}\footnotemark
\item<2-> Or falls back to the default one
\item<2-> Which notifies about the uncaught exception
\end{itemize}
\item<4-> Terminates the program either with \funcname{abort} or with a non-zero exit code
\end{itemize}
\end{column}
\begin{column}{0.55\textwidth}
\only<1>{
\listc[firstline=84,lastline=89,highlightlines=88]{../ocaml/runtime/caml/mlvalues.h}{runtime/caml/mlvalues.h:84}
\listc[firstline=138,lastline=143,highlightlines={141-142}]{../ocaml/runtime/startup\_nat.c}{runtime/startup\_nat.c:138}
}%
\only<2>{
\listc[firstline=138,highlightlines={147,149}]{../ocaml/runtime/printexc.c}{runtime/printexc.c:138}
}%
\only<3>{
\listc[firstline=110,lastline=134,highlightlines=129]{../ocaml/runtime/printexc.c}{runtime/printexc.c:138}
}%
\only<4>{
\listc[firstline=138,highlightlines={152,154}]{../ocaml/runtime/printexc.c}{runtime/printexc.c:138}
}%
\end{column}
\end{columns}
\footnotetext[1]{\url{https://v2.ocaml.org/api/Printexc.html\#1\_Uncaughtexceptions}}
\end{frame}
}%
\end{column}
\end{columns}
\end{frame}


\begin{frame}{Startup}{How did we get there by the way?}
\begin{columns}
\begin{column}{0.45\textwidth}
\begin{itemize}
\item The entry point address of the binary is \funcname{main} (\funcname{\_start}) from the runtime
\item The program starts like a C program:
\begin{itemize}
\item \funcname{caml\_start\_program}
\item \funcname{caml\_startup\_common}
\item \funcname{caml\_startup\_exn}
\item \funcname{caml\_startup}
\item \funcname{caml\_main}
\item \funcname{main}
\end{itemize}
%\item \funcname{caml\_startup\_common}: initializes GC, domains \dots
% Also: codefrag, locale, custom opeartions, frame_descr, signals, stack
\item \funcname{caml\_start\_program} is the transition point between the C realm and the OCaml one
\end{itemize}
\bigskip
\listocaml{../src/default/default.ml}{default/default.ml}
\end{column}
\begin{column}{0.55\textwidth}
\listasm[lastline=24,highlightlines={22-24}]{src/ocaml/caml\_start\_program.s}{runtime/amd64.S:604}
\end{column}
\end{columns}
\end{frame}


\begin{frame}{Setup to jump into OCaml entry point}{\funcname{caml\_start\_program} initializes the main fiber}
\begin{columns}
\begin{column}{0.6\textwidth}
\only<1,3,5>{
\begin{itemize}
\item<1-> Stores information to allow switching from OCaml stack to the C stack
\item<3-> Constructs the default exception handler
\item<5-> Switches to the OCaml stack and calls \funcname{caml\_program}
\end{itemize}
\bigskip
\listocaml{../src/default/default.ml}{default/default.ml}
}%
\only<2>{
\listasm[firstline=22,lastline=41,highlightlines={25-27}]{src/ocaml/caml\_start\_program.s}{caml\_start\_program}
}%
\only<4>{
\mintasm[firstline=22,lastline=41,highlightlines={31-38}]{src/ocaml/caml\_start\_program.s}
\listasm[firstline=66,lastline=72]{src/ocaml/caml\_start\_program.s}{caml\_start\_program}
}%
\only<6>{
\listasm[firstline=22,lastline=41,highlightlines={39-41}]{src/ocaml/caml\_start\_program.s}{caml\_start\_program}
}%
\end{column}
\begin{column}{0.4\textwidth}
\centering
\only<1-2>{\providecommand\step{2}%
% Runtime's default exception handler
%
\subsection*{Runtime's default exception handler}
\frameSubsection{pictures/The\_Architect.png}{}


\begin{frame}{Default exception handler}
\begin{columns}
\begin{column}{0.5\textwidth}
\begin{itemize}
\item \localname{domain\_state->exn\_handler} points to a trap located before \funcname{entry}!
\item What installed this very first trap there?
\end{itemize}
\bigskip
\listocaml{../src/default/default.ml}{default/default.ml}
\bigskip
\inputminted[fontsize=\footnotesize]{text}{../src/default/default.out}
\end{column}
\begin{column}{0.5\textwidth}
\centering
\only<1>{\providecommand\step{0}%
% Runtime's default exception handler
%
\subsection*{Runtime's default exception handler}
\frameSubsection{pictures/The\_Architect.png}{}


\begin{frame}{Default exception handler}
\begin{columns}
\begin{column}{0.5\textwidth}
\begin{itemize}
\item \localname{domain\_state->exn\_handler} points to a trap located before \funcname{entry}!
\item What installed this very first trap there?
\end{itemize}
\bigskip
\listocaml{../src/default/default.ml}{default/default.ml}
\bigskip
\inputminted[fontsize=\footnotesize]{text}{../src/default/default.out}
\end{column}
\begin{column}{0.5\textwidth}
\centering
\only<1>{\providecommand\step{0}\input{diagrams/default/default.tex}}%
\only<2>{\providecommand\step{4}\input{diagrams/default/default.tex}}%
\end{column}
\end{columns}
\end{frame}


\begin{frame}{Startup}{How did we get there by the way?}
\begin{columns}
\begin{column}{0.45\textwidth}
\begin{itemize}
\item The entry point address of the binary is \funcname{main} (\funcname{\_start}) from the runtime
\item The program starts like a C program:
\begin{itemize}
\item \funcname{caml\_start\_program}
\item \funcname{caml\_startup\_common}
\item \funcname{caml\_startup\_exn}
\item \funcname{caml\_startup}
\item \funcname{caml\_main}
\item \funcname{main}
\end{itemize}
%\item \funcname{caml\_startup\_common}: initializes GC, domains \dots
% Also: codefrag, locale, custom opeartions, frame_descr, signals, stack
\item \funcname{caml\_start\_program} is the transition point between the C realm and the OCaml one
\end{itemize}
\bigskip
\listocaml{../src/default/default.ml}{default/default.ml}
\end{column}
\begin{column}{0.55\textwidth}
\listasm[lastline=24,highlightlines={22-24}]{src/ocaml/caml\_start\_program.s}{runtime/amd64.S:604}
\end{column}
\end{columns}
\end{frame}


\begin{frame}{Setup to jump into OCaml entry point}{\funcname{caml\_start\_program} initializes the main fiber}
\begin{columns}
\begin{column}{0.6\textwidth}
\only<1,3,5>{
\begin{itemize}
\item<1-> Stores information to allow switching from OCaml stack to the C stack
\item<3-> Constructs the default exception handler
\item<5-> Switches to the OCaml stack and calls \funcname{caml\_program}
\end{itemize}
\bigskip
\listocaml{../src/default/default.ml}{default/default.ml}
}%
\only<2>{
\listasm[firstline=22,lastline=41,highlightlines={25-27}]{src/ocaml/caml\_start\_program.s}{caml\_start\_program}
}%
\only<4>{
\mintasm[firstline=22,lastline=41,highlightlines={31-38}]{src/ocaml/caml\_start\_program.s}
\listasm[firstline=66,lastline=72]{src/ocaml/caml\_start\_program.s}{caml\_start\_program}
}%
\only<6>{
\listasm[firstline=22,lastline=41,highlightlines={39-41}]{src/ocaml/caml\_start\_program.s}{caml\_start\_program}
}%
\end{column}
\begin{column}{0.4\textwidth}
\centering
\only<1-2>{\providecommand\step{2}\input{diagrams/default/default.tex}}%
\only<3-5>{\providecommand\step{3}\input{diagrams/default/default.tex}}%
\onslide*<6->{\providecommand\step{4}\input{diagrams/default/default.tex}}%
\end{column}
\end{columns}
\end{frame}

\begin{frame}{Executing the OCaml program}{And raising an exception without handler}
\begin{columns}
\begin{column}{0.6\textwidth}
\only<1-3>{
\begin{itemize}
\item<1-> Execution continues with OCaml code
\begin{itemize}
\item \funcname{camlDefault\_\_main\_269}
\item \funcname{camlDefault\_\_entry}
\item \funcname{caml\_program}
\end{itemize}
\item<1-> \funcname{camlDefault\_\_main\_269} raises a \typename{Failure} exception
\item<1-> \typename{Failure} is caught by \funcname{caml\_start\_program}
\begin{itemize}
\item<1-> The handler set the 2nd LSB of the exception value to \funcarg{1}
\item<3-> And then forces \funcname{caml\_start\_program} to terminate
\end{itemize}
\end{itemize}
}
\bigskip
\only<1,3>{\listocaml[highlightlines=3]{../src/default/default.ml}{default/default.ml}}%
\only<2>{\listasm[firstline=66,lastline=72,highlightlines=69]{src/ocaml/caml\_start\_program.s}{caml\_start\_program}}%
\end{column}
\begin{column}{0.4\textwidth}
\centering
\providecommand\step{4}\input{diagrams/default/default.tex}
\end{column}
\end{columns}
\end{frame}
%\only<>{\listasm[firstline=47,lastline=65]{src/ocaml/caml\_start\_program.s}{caml\_start\_program}}


\begin{frame}{Back to C}{Clean up, complain and terminate}
\begin{columns}
\begin{column}{0.45\textwidth}
\begin{itemize}
\item Backtrace:
\begin{itemize}
\item \funcname{caml\_start\_program} $\rightarrow$ OCaml code
\item \funcname{caml\_startup\_common}
\item \funcname{caml\_startup\_exn}
\item \funcname{caml\_startup}
\item \funcname{caml\_main}
\item \funcname{main}
\end{itemize}
\smallskip
\item \funcname{caml\_startup} checks the return value of \funcname{caml\_startup\_exn}
\begin{itemize}
\item The return value has been specially marked by \funcname{caml\_start\_program} handler
\item Calls \funcname{caml\_fatal\_uncaught\_exception}
\end{itemize}
\item<2-> \funcname{caml\_fatal\_uncaught\_exception} either
\begin{itemize}
\item<2-> Calls the user custom uncaught exception handler, if set with \mintinline{ocaml}{Printexc.set_uncaught_exception_handler fn}\footnotemark
\item<2-> Or falls back to the default one
\item<2-> Which notifies about the uncaught exception
\end{itemize}
\item<4-> Terminates the program either with \funcname{abort} or with a non-zero exit code
\end{itemize}
\end{column}
\begin{column}{0.55\textwidth}
\only<1>{
\listc[firstline=84,lastline=89,highlightlines=88]{../ocaml/runtime/caml/mlvalues.h}{runtime/caml/mlvalues.h:84}
\listc[firstline=138,lastline=143,highlightlines={141-142}]{../ocaml/runtime/startup\_nat.c}{runtime/startup\_nat.c:138}
}%
\only<2>{
\listc[firstline=138,highlightlines={147,149}]{../ocaml/runtime/printexc.c}{runtime/printexc.c:138}
}%
\only<3>{
\listc[firstline=110,lastline=134,highlightlines=129]{../ocaml/runtime/printexc.c}{runtime/printexc.c:138}
}%
\only<4>{
\listc[firstline=138,highlightlines={152,154}]{../ocaml/runtime/printexc.c}{runtime/printexc.c:138}
}%
\end{column}
\end{columns}
\footnotetext[1]{\url{https://v2.ocaml.org/api/Printexc.html\#1\_Uncaughtexceptions}}
\end{frame}
}%
\only<2>{\providecommand\step{4}%
% Runtime's default exception handler
%
\subsection*{Runtime's default exception handler}
\frameSubsection{pictures/The\_Architect.png}{}


\begin{frame}{Default exception handler}
\begin{columns}
\begin{column}{0.5\textwidth}
\begin{itemize}
\item \localname{domain\_state->exn\_handler} points to a trap located before \funcname{entry}!
\item What installed this very first trap there?
\end{itemize}
\bigskip
\listocaml{../src/default/default.ml}{default/default.ml}
\bigskip
\inputminted[fontsize=\footnotesize]{text}{../src/default/default.out}
\end{column}
\begin{column}{0.5\textwidth}
\centering
\only<1>{\providecommand\step{0}\input{diagrams/default/default.tex}}%
\only<2>{\providecommand\step{4}\input{diagrams/default/default.tex}}%
\end{column}
\end{columns}
\end{frame}


\begin{frame}{Startup}{How did we get there by the way?}
\begin{columns}
\begin{column}{0.45\textwidth}
\begin{itemize}
\item The entry point address of the binary is \funcname{main} (\funcname{\_start}) from the runtime
\item The program starts like a C program:
\begin{itemize}
\item \funcname{caml\_start\_program}
\item \funcname{caml\_startup\_common}
\item \funcname{caml\_startup\_exn}
\item \funcname{caml\_startup}
\item \funcname{caml\_main}
\item \funcname{main}
\end{itemize}
%\item \funcname{caml\_startup\_common}: initializes GC, domains \dots
% Also: codefrag, locale, custom opeartions, frame_descr, signals, stack
\item \funcname{caml\_start\_program} is the transition point between the C realm and the OCaml one
\end{itemize}
\bigskip
\listocaml{../src/default/default.ml}{default/default.ml}
\end{column}
\begin{column}{0.55\textwidth}
\listasm[lastline=24,highlightlines={22-24}]{src/ocaml/caml\_start\_program.s}{runtime/amd64.S:604}
\end{column}
\end{columns}
\end{frame}


\begin{frame}{Setup to jump into OCaml entry point}{\funcname{caml\_start\_program} initializes the main fiber}
\begin{columns}
\begin{column}{0.6\textwidth}
\only<1,3,5>{
\begin{itemize}
\item<1-> Stores information to allow switching from OCaml stack to the C stack
\item<3-> Constructs the default exception handler
\item<5-> Switches to the OCaml stack and calls \funcname{caml\_program}
\end{itemize}
\bigskip
\listocaml{../src/default/default.ml}{default/default.ml}
}%
\only<2>{
\listasm[firstline=22,lastline=41,highlightlines={25-27}]{src/ocaml/caml\_start\_program.s}{caml\_start\_program}
}%
\only<4>{
\mintasm[firstline=22,lastline=41,highlightlines={31-38}]{src/ocaml/caml\_start\_program.s}
\listasm[firstline=66,lastline=72]{src/ocaml/caml\_start\_program.s}{caml\_start\_program}
}%
\only<6>{
\listasm[firstline=22,lastline=41,highlightlines={39-41}]{src/ocaml/caml\_start\_program.s}{caml\_start\_program}
}%
\end{column}
\begin{column}{0.4\textwidth}
\centering
\only<1-2>{\providecommand\step{2}\input{diagrams/default/default.tex}}%
\only<3-5>{\providecommand\step{3}\input{diagrams/default/default.tex}}%
\onslide*<6->{\providecommand\step{4}\input{diagrams/default/default.tex}}%
\end{column}
\end{columns}
\end{frame}

\begin{frame}{Executing the OCaml program}{And raising an exception without handler}
\begin{columns}
\begin{column}{0.6\textwidth}
\only<1-3>{
\begin{itemize}
\item<1-> Execution continues with OCaml code
\begin{itemize}
\item \funcname{camlDefault\_\_main\_269}
\item \funcname{camlDefault\_\_entry}
\item \funcname{caml\_program}
\end{itemize}
\item<1-> \funcname{camlDefault\_\_main\_269} raises a \typename{Failure} exception
\item<1-> \typename{Failure} is caught by \funcname{caml\_start\_program}
\begin{itemize}
\item<1-> The handler set the 2nd LSB of the exception value to \funcarg{1}
\item<3-> And then forces \funcname{caml\_start\_program} to terminate
\end{itemize}
\end{itemize}
}
\bigskip
\only<1,3>{\listocaml[highlightlines=3]{../src/default/default.ml}{default/default.ml}}%
\only<2>{\listasm[firstline=66,lastline=72,highlightlines=69]{src/ocaml/caml\_start\_program.s}{caml\_start\_program}}%
\end{column}
\begin{column}{0.4\textwidth}
\centering
\providecommand\step{4}\input{diagrams/default/default.tex}
\end{column}
\end{columns}
\end{frame}
%\only<>{\listasm[firstline=47,lastline=65]{src/ocaml/caml\_start\_program.s}{caml\_start\_program}}


\begin{frame}{Back to C}{Clean up, complain and terminate}
\begin{columns}
\begin{column}{0.45\textwidth}
\begin{itemize}
\item Backtrace:
\begin{itemize}
\item \funcname{caml\_start\_program} $\rightarrow$ OCaml code
\item \funcname{caml\_startup\_common}
\item \funcname{caml\_startup\_exn}
\item \funcname{caml\_startup}
\item \funcname{caml\_main}
\item \funcname{main}
\end{itemize}
\smallskip
\item \funcname{caml\_startup} checks the return value of \funcname{caml\_startup\_exn}
\begin{itemize}
\item The return value has been specially marked by \funcname{caml\_start\_program} handler
\item Calls \funcname{caml\_fatal\_uncaught\_exception}
\end{itemize}
\item<2-> \funcname{caml\_fatal\_uncaught\_exception} either
\begin{itemize}
\item<2-> Calls the user custom uncaught exception handler, if set with \mintinline{ocaml}{Printexc.set_uncaught_exception_handler fn}\footnotemark
\item<2-> Or falls back to the default one
\item<2-> Which notifies about the uncaught exception
\end{itemize}
\item<4-> Terminates the program either with \funcname{abort} or with a non-zero exit code
\end{itemize}
\end{column}
\begin{column}{0.55\textwidth}
\only<1>{
\listc[firstline=84,lastline=89,highlightlines=88]{../ocaml/runtime/caml/mlvalues.h}{runtime/caml/mlvalues.h:84}
\listc[firstline=138,lastline=143,highlightlines={141-142}]{../ocaml/runtime/startup\_nat.c}{runtime/startup\_nat.c:138}
}%
\only<2>{
\listc[firstline=138,highlightlines={147,149}]{../ocaml/runtime/printexc.c}{runtime/printexc.c:138}
}%
\only<3>{
\listc[firstline=110,lastline=134,highlightlines=129]{../ocaml/runtime/printexc.c}{runtime/printexc.c:138}
}%
\only<4>{
\listc[firstline=138,highlightlines={152,154}]{../ocaml/runtime/printexc.c}{runtime/printexc.c:138}
}%
\end{column}
\end{columns}
\footnotetext[1]{\url{https://v2.ocaml.org/api/Printexc.html\#1\_Uncaughtexceptions}}
\end{frame}
}%
\end{column}
\end{columns}
\end{frame}


\begin{frame}{Startup}{How did we get there by the way?}
\begin{columns}
\begin{column}{0.45\textwidth}
\begin{itemize}
\item The entry point address of the binary is \funcname{main} (\funcname{\_start}) from the runtime
\item The program starts like a C program:
\begin{itemize}
\item \funcname{caml\_start\_program}
\item \funcname{caml\_startup\_common}
\item \funcname{caml\_startup\_exn}
\item \funcname{caml\_startup}
\item \funcname{caml\_main}
\item \funcname{main}
\end{itemize}
%\item \funcname{caml\_startup\_common}: initializes GC, domains \dots
% Also: codefrag, locale, custom opeartions, frame_descr, signals, stack
\item \funcname{caml\_start\_program} is the transition point between the C realm and the OCaml one
\end{itemize}
\bigskip
\listocaml{../src/default/default.ml}{default/default.ml}
\end{column}
\begin{column}{0.55\textwidth}
\listasm[lastline=24,highlightlines={22-24}]{src/ocaml/caml\_start\_program.s}{runtime/amd64.S:604}
\end{column}
\end{columns}
\end{frame}


\begin{frame}{Setup to jump into OCaml entry point}{\funcname{caml\_start\_program} initializes the main fiber}
\begin{columns}
\begin{column}{0.6\textwidth}
\only<1,3,5>{
\begin{itemize}
\item<1-> Stores information to allow switching from OCaml stack to the C stack
\item<3-> Constructs the default exception handler
\item<5-> Switches to the OCaml stack and calls \funcname{caml\_program}
\end{itemize}
\bigskip
\listocaml{../src/default/default.ml}{default/default.ml}
}%
\only<2>{
\listasm[firstline=22,lastline=41,highlightlines={25-27}]{src/ocaml/caml\_start\_program.s}{caml\_start\_program}
}%
\only<4>{
\mintasm[firstline=22,lastline=41,highlightlines={31-38}]{src/ocaml/caml\_start\_program.s}
\listasm[firstline=66,lastline=72]{src/ocaml/caml\_start\_program.s}{caml\_start\_program}
}%
\only<6>{
\listasm[firstline=22,lastline=41,highlightlines={39-41}]{src/ocaml/caml\_start\_program.s}{caml\_start\_program}
}%
\end{column}
\begin{column}{0.4\textwidth}
\centering
\only<1-2>{\providecommand\step{2}%
% Runtime's default exception handler
%
\subsection*{Runtime's default exception handler}
\frameSubsection{pictures/The\_Architect.png}{}


\begin{frame}{Default exception handler}
\begin{columns}
\begin{column}{0.5\textwidth}
\begin{itemize}
\item \localname{domain\_state->exn\_handler} points to a trap located before \funcname{entry}!
\item What installed this very first trap there?
\end{itemize}
\bigskip
\listocaml{../src/default/default.ml}{default/default.ml}
\bigskip
\inputminted[fontsize=\footnotesize]{text}{../src/default/default.out}
\end{column}
\begin{column}{0.5\textwidth}
\centering
\only<1>{\providecommand\step{0}\input{diagrams/default/default.tex}}%
\only<2>{\providecommand\step{4}\input{diagrams/default/default.tex}}%
\end{column}
\end{columns}
\end{frame}


\begin{frame}{Startup}{How did we get there by the way?}
\begin{columns}
\begin{column}{0.45\textwidth}
\begin{itemize}
\item The entry point address of the binary is \funcname{main} (\funcname{\_start}) from the runtime
\item The program starts like a C program:
\begin{itemize}
\item \funcname{caml\_start\_program}
\item \funcname{caml\_startup\_common}
\item \funcname{caml\_startup\_exn}
\item \funcname{caml\_startup}
\item \funcname{caml\_main}
\item \funcname{main}
\end{itemize}
%\item \funcname{caml\_startup\_common}: initializes GC, domains \dots
% Also: codefrag, locale, custom opeartions, frame_descr, signals, stack
\item \funcname{caml\_start\_program} is the transition point between the C realm and the OCaml one
\end{itemize}
\bigskip
\listocaml{../src/default/default.ml}{default/default.ml}
\end{column}
\begin{column}{0.55\textwidth}
\listasm[lastline=24,highlightlines={22-24}]{src/ocaml/caml\_start\_program.s}{runtime/amd64.S:604}
\end{column}
\end{columns}
\end{frame}


\begin{frame}{Setup to jump into OCaml entry point}{\funcname{caml\_start\_program} initializes the main fiber}
\begin{columns}
\begin{column}{0.6\textwidth}
\only<1,3,5>{
\begin{itemize}
\item<1-> Stores information to allow switching from OCaml stack to the C stack
\item<3-> Constructs the default exception handler
\item<5-> Switches to the OCaml stack and calls \funcname{caml\_program}
\end{itemize}
\bigskip
\listocaml{../src/default/default.ml}{default/default.ml}
}%
\only<2>{
\listasm[firstline=22,lastline=41,highlightlines={25-27}]{src/ocaml/caml\_start\_program.s}{caml\_start\_program}
}%
\only<4>{
\mintasm[firstline=22,lastline=41,highlightlines={31-38}]{src/ocaml/caml\_start\_program.s}
\listasm[firstline=66,lastline=72]{src/ocaml/caml\_start\_program.s}{caml\_start\_program}
}%
\only<6>{
\listasm[firstline=22,lastline=41,highlightlines={39-41}]{src/ocaml/caml\_start\_program.s}{caml\_start\_program}
}%
\end{column}
\begin{column}{0.4\textwidth}
\centering
\only<1-2>{\providecommand\step{2}\input{diagrams/default/default.tex}}%
\only<3-5>{\providecommand\step{3}\input{diagrams/default/default.tex}}%
\onslide*<6->{\providecommand\step{4}\input{diagrams/default/default.tex}}%
\end{column}
\end{columns}
\end{frame}

\begin{frame}{Executing the OCaml program}{And raising an exception without handler}
\begin{columns}
\begin{column}{0.6\textwidth}
\only<1-3>{
\begin{itemize}
\item<1-> Execution continues with OCaml code
\begin{itemize}
\item \funcname{camlDefault\_\_main\_269}
\item \funcname{camlDefault\_\_entry}
\item \funcname{caml\_program}
\end{itemize}
\item<1-> \funcname{camlDefault\_\_main\_269} raises a \typename{Failure} exception
\item<1-> \typename{Failure} is caught by \funcname{caml\_start\_program}
\begin{itemize}
\item<1-> The handler set the 2nd LSB of the exception value to \funcarg{1}
\item<3-> And then forces \funcname{caml\_start\_program} to terminate
\end{itemize}
\end{itemize}
}
\bigskip
\only<1,3>{\listocaml[highlightlines=3]{../src/default/default.ml}{default/default.ml}}%
\only<2>{\listasm[firstline=66,lastline=72,highlightlines=69]{src/ocaml/caml\_start\_program.s}{caml\_start\_program}}%
\end{column}
\begin{column}{0.4\textwidth}
\centering
\providecommand\step{4}\input{diagrams/default/default.tex}
\end{column}
\end{columns}
\end{frame}
%\only<>{\listasm[firstline=47,lastline=65]{src/ocaml/caml\_start\_program.s}{caml\_start\_program}}


\begin{frame}{Back to C}{Clean up, complain and terminate}
\begin{columns}
\begin{column}{0.45\textwidth}
\begin{itemize}
\item Backtrace:
\begin{itemize}
\item \funcname{caml\_start\_program} $\rightarrow$ OCaml code
\item \funcname{caml\_startup\_common}
\item \funcname{caml\_startup\_exn}
\item \funcname{caml\_startup}
\item \funcname{caml\_main}
\item \funcname{main}
\end{itemize}
\smallskip
\item \funcname{caml\_startup} checks the return value of \funcname{caml\_startup\_exn}
\begin{itemize}
\item The return value has been specially marked by \funcname{caml\_start\_program} handler
\item Calls \funcname{caml\_fatal\_uncaught\_exception}
\end{itemize}
\item<2-> \funcname{caml\_fatal\_uncaught\_exception} either
\begin{itemize}
\item<2-> Calls the user custom uncaught exception handler, if set with \mintinline{ocaml}{Printexc.set_uncaught_exception_handler fn}\footnotemark
\item<2-> Or falls back to the default one
\item<2-> Which notifies about the uncaught exception
\end{itemize}
\item<4-> Terminates the program either with \funcname{abort} or with a non-zero exit code
\end{itemize}
\end{column}
\begin{column}{0.55\textwidth}
\only<1>{
\listc[firstline=84,lastline=89,highlightlines=88]{../ocaml/runtime/caml/mlvalues.h}{runtime/caml/mlvalues.h:84}
\listc[firstline=138,lastline=143,highlightlines={141-142}]{../ocaml/runtime/startup\_nat.c}{runtime/startup\_nat.c:138}
}%
\only<2>{
\listc[firstline=138,highlightlines={147,149}]{../ocaml/runtime/printexc.c}{runtime/printexc.c:138}
}%
\only<3>{
\listc[firstline=110,lastline=134,highlightlines=129]{../ocaml/runtime/printexc.c}{runtime/printexc.c:138}
}%
\only<4>{
\listc[firstline=138,highlightlines={152,154}]{../ocaml/runtime/printexc.c}{runtime/printexc.c:138}
}%
\end{column}
\end{columns}
\footnotetext[1]{\url{https://v2.ocaml.org/api/Printexc.html\#1\_Uncaughtexceptions}}
\end{frame}
}%
\only<3-5>{\providecommand\step{3}%
% Runtime's default exception handler
%
\subsection*{Runtime's default exception handler}
\frameSubsection{pictures/The\_Architect.png}{}


\begin{frame}{Default exception handler}
\begin{columns}
\begin{column}{0.5\textwidth}
\begin{itemize}
\item \localname{domain\_state->exn\_handler} points to a trap located before \funcname{entry}!
\item What installed this very first trap there?
\end{itemize}
\bigskip
\listocaml{../src/default/default.ml}{default/default.ml}
\bigskip
\inputminted[fontsize=\footnotesize]{text}{../src/default/default.out}
\end{column}
\begin{column}{0.5\textwidth}
\centering
\only<1>{\providecommand\step{0}\input{diagrams/default/default.tex}}%
\only<2>{\providecommand\step{4}\input{diagrams/default/default.tex}}%
\end{column}
\end{columns}
\end{frame}


\begin{frame}{Startup}{How did we get there by the way?}
\begin{columns}
\begin{column}{0.45\textwidth}
\begin{itemize}
\item The entry point address of the binary is \funcname{main} (\funcname{\_start}) from the runtime
\item The program starts like a C program:
\begin{itemize}
\item \funcname{caml\_start\_program}
\item \funcname{caml\_startup\_common}
\item \funcname{caml\_startup\_exn}
\item \funcname{caml\_startup}
\item \funcname{caml\_main}
\item \funcname{main}
\end{itemize}
%\item \funcname{caml\_startup\_common}: initializes GC, domains \dots
% Also: codefrag, locale, custom opeartions, frame_descr, signals, stack
\item \funcname{caml\_start\_program} is the transition point between the C realm and the OCaml one
\end{itemize}
\bigskip
\listocaml{../src/default/default.ml}{default/default.ml}
\end{column}
\begin{column}{0.55\textwidth}
\listasm[lastline=24,highlightlines={22-24}]{src/ocaml/caml\_start\_program.s}{runtime/amd64.S:604}
\end{column}
\end{columns}
\end{frame}


\begin{frame}{Setup to jump into OCaml entry point}{\funcname{caml\_start\_program} initializes the main fiber}
\begin{columns}
\begin{column}{0.6\textwidth}
\only<1,3,5>{
\begin{itemize}
\item<1-> Stores information to allow switching from OCaml stack to the C stack
\item<3-> Constructs the default exception handler
\item<5-> Switches to the OCaml stack and calls \funcname{caml\_program}
\end{itemize}
\bigskip
\listocaml{../src/default/default.ml}{default/default.ml}
}%
\only<2>{
\listasm[firstline=22,lastline=41,highlightlines={25-27}]{src/ocaml/caml\_start\_program.s}{caml\_start\_program}
}%
\only<4>{
\mintasm[firstline=22,lastline=41,highlightlines={31-38}]{src/ocaml/caml\_start\_program.s}
\listasm[firstline=66,lastline=72]{src/ocaml/caml\_start\_program.s}{caml\_start\_program}
}%
\only<6>{
\listasm[firstline=22,lastline=41,highlightlines={39-41}]{src/ocaml/caml\_start\_program.s}{caml\_start\_program}
}%
\end{column}
\begin{column}{0.4\textwidth}
\centering
\only<1-2>{\providecommand\step{2}\input{diagrams/default/default.tex}}%
\only<3-5>{\providecommand\step{3}\input{diagrams/default/default.tex}}%
\onslide*<6->{\providecommand\step{4}\input{diagrams/default/default.tex}}%
\end{column}
\end{columns}
\end{frame}

\begin{frame}{Executing the OCaml program}{And raising an exception without handler}
\begin{columns}
\begin{column}{0.6\textwidth}
\only<1-3>{
\begin{itemize}
\item<1-> Execution continues with OCaml code
\begin{itemize}
\item \funcname{camlDefault\_\_main\_269}
\item \funcname{camlDefault\_\_entry}
\item \funcname{caml\_program}
\end{itemize}
\item<1-> \funcname{camlDefault\_\_main\_269} raises a \typename{Failure} exception
\item<1-> \typename{Failure} is caught by \funcname{caml\_start\_program}
\begin{itemize}
\item<1-> The handler set the 2nd LSB of the exception value to \funcarg{1}
\item<3-> And then forces \funcname{caml\_start\_program} to terminate
\end{itemize}
\end{itemize}
}
\bigskip
\only<1,3>{\listocaml[highlightlines=3]{../src/default/default.ml}{default/default.ml}}%
\only<2>{\listasm[firstline=66,lastline=72,highlightlines=69]{src/ocaml/caml\_start\_program.s}{caml\_start\_program}}%
\end{column}
\begin{column}{0.4\textwidth}
\centering
\providecommand\step{4}\input{diagrams/default/default.tex}
\end{column}
\end{columns}
\end{frame}
%\only<>{\listasm[firstline=47,lastline=65]{src/ocaml/caml\_start\_program.s}{caml\_start\_program}}


\begin{frame}{Back to C}{Clean up, complain and terminate}
\begin{columns}
\begin{column}{0.45\textwidth}
\begin{itemize}
\item Backtrace:
\begin{itemize}
\item \funcname{caml\_start\_program} $\rightarrow$ OCaml code
\item \funcname{caml\_startup\_common}
\item \funcname{caml\_startup\_exn}
\item \funcname{caml\_startup}
\item \funcname{caml\_main}
\item \funcname{main}
\end{itemize}
\smallskip
\item \funcname{caml\_startup} checks the return value of \funcname{caml\_startup\_exn}
\begin{itemize}
\item The return value has been specially marked by \funcname{caml\_start\_program} handler
\item Calls \funcname{caml\_fatal\_uncaught\_exception}
\end{itemize}
\item<2-> \funcname{caml\_fatal\_uncaught\_exception} either
\begin{itemize}
\item<2-> Calls the user custom uncaught exception handler, if set with \mintinline{ocaml}{Printexc.set_uncaught_exception_handler fn}\footnotemark
\item<2-> Or falls back to the default one
\item<2-> Which notifies about the uncaught exception
\end{itemize}
\item<4-> Terminates the program either with \funcname{abort} or with a non-zero exit code
\end{itemize}
\end{column}
\begin{column}{0.55\textwidth}
\only<1>{
\listc[firstline=84,lastline=89,highlightlines=88]{../ocaml/runtime/caml/mlvalues.h}{runtime/caml/mlvalues.h:84}
\listc[firstline=138,lastline=143,highlightlines={141-142}]{../ocaml/runtime/startup\_nat.c}{runtime/startup\_nat.c:138}
}%
\only<2>{
\listc[firstline=138,highlightlines={147,149}]{../ocaml/runtime/printexc.c}{runtime/printexc.c:138}
}%
\only<3>{
\listc[firstline=110,lastline=134,highlightlines=129]{../ocaml/runtime/printexc.c}{runtime/printexc.c:138}
}%
\only<4>{
\listc[firstline=138,highlightlines={152,154}]{../ocaml/runtime/printexc.c}{runtime/printexc.c:138}
}%
\end{column}
\end{columns}
\footnotetext[1]{\url{https://v2.ocaml.org/api/Printexc.html\#1\_Uncaughtexceptions}}
\end{frame}
}%
\onslide*<6->{\providecommand\step{4}%
% Runtime's default exception handler
%
\subsection*{Runtime's default exception handler}
\frameSubsection{pictures/The\_Architect.png}{}


\begin{frame}{Default exception handler}
\begin{columns}
\begin{column}{0.5\textwidth}
\begin{itemize}
\item \localname{domain\_state->exn\_handler} points to a trap located before \funcname{entry}!
\item What installed this very first trap there?
\end{itemize}
\bigskip
\listocaml{../src/default/default.ml}{default/default.ml}
\bigskip
\inputminted[fontsize=\footnotesize]{text}{../src/default/default.out}
\end{column}
\begin{column}{0.5\textwidth}
\centering
\only<1>{\providecommand\step{0}\input{diagrams/default/default.tex}}%
\only<2>{\providecommand\step{4}\input{diagrams/default/default.tex}}%
\end{column}
\end{columns}
\end{frame}


\begin{frame}{Startup}{How did we get there by the way?}
\begin{columns}
\begin{column}{0.45\textwidth}
\begin{itemize}
\item The entry point address of the binary is \funcname{main} (\funcname{\_start}) from the runtime
\item The program starts like a C program:
\begin{itemize}
\item \funcname{caml\_start\_program}
\item \funcname{caml\_startup\_common}
\item \funcname{caml\_startup\_exn}
\item \funcname{caml\_startup}
\item \funcname{caml\_main}
\item \funcname{main}
\end{itemize}
%\item \funcname{caml\_startup\_common}: initializes GC, domains \dots
% Also: codefrag, locale, custom opeartions, frame_descr, signals, stack
\item \funcname{caml\_start\_program} is the transition point between the C realm and the OCaml one
\end{itemize}
\bigskip
\listocaml{../src/default/default.ml}{default/default.ml}
\end{column}
\begin{column}{0.55\textwidth}
\listasm[lastline=24,highlightlines={22-24}]{src/ocaml/caml\_start\_program.s}{runtime/amd64.S:604}
\end{column}
\end{columns}
\end{frame}


\begin{frame}{Setup to jump into OCaml entry point}{\funcname{caml\_start\_program} initializes the main fiber}
\begin{columns}
\begin{column}{0.6\textwidth}
\only<1,3,5>{
\begin{itemize}
\item<1-> Stores information to allow switching from OCaml stack to the C stack
\item<3-> Constructs the default exception handler
\item<5-> Switches to the OCaml stack and calls \funcname{caml\_program}
\end{itemize}
\bigskip
\listocaml{../src/default/default.ml}{default/default.ml}
}%
\only<2>{
\listasm[firstline=22,lastline=41,highlightlines={25-27}]{src/ocaml/caml\_start\_program.s}{caml\_start\_program}
}%
\only<4>{
\mintasm[firstline=22,lastline=41,highlightlines={31-38}]{src/ocaml/caml\_start\_program.s}
\listasm[firstline=66,lastline=72]{src/ocaml/caml\_start\_program.s}{caml\_start\_program}
}%
\only<6>{
\listasm[firstline=22,lastline=41,highlightlines={39-41}]{src/ocaml/caml\_start\_program.s}{caml\_start\_program}
}%
\end{column}
\begin{column}{0.4\textwidth}
\centering
\only<1-2>{\providecommand\step{2}\input{diagrams/default/default.tex}}%
\only<3-5>{\providecommand\step{3}\input{diagrams/default/default.tex}}%
\onslide*<6->{\providecommand\step{4}\input{diagrams/default/default.tex}}%
\end{column}
\end{columns}
\end{frame}

\begin{frame}{Executing the OCaml program}{And raising an exception without handler}
\begin{columns}
\begin{column}{0.6\textwidth}
\only<1-3>{
\begin{itemize}
\item<1-> Execution continues with OCaml code
\begin{itemize}
\item \funcname{camlDefault\_\_main\_269}
\item \funcname{camlDefault\_\_entry}
\item \funcname{caml\_program}
\end{itemize}
\item<1-> \funcname{camlDefault\_\_main\_269} raises a \typename{Failure} exception
\item<1-> \typename{Failure} is caught by \funcname{caml\_start\_program}
\begin{itemize}
\item<1-> The handler set the 2nd LSB of the exception value to \funcarg{1}
\item<3-> And then forces \funcname{caml\_start\_program} to terminate
\end{itemize}
\end{itemize}
}
\bigskip
\only<1,3>{\listocaml[highlightlines=3]{../src/default/default.ml}{default/default.ml}}%
\only<2>{\listasm[firstline=66,lastline=72,highlightlines=69]{src/ocaml/caml\_start\_program.s}{caml\_start\_program}}%
\end{column}
\begin{column}{0.4\textwidth}
\centering
\providecommand\step{4}\input{diagrams/default/default.tex}
\end{column}
\end{columns}
\end{frame}
%\only<>{\listasm[firstline=47,lastline=65]{src/ocaml/caml\_start\_program.s}{caml\_start\_program}}


\begin{frame}{Back to C}{Clean up, complain and terminate}
\begin{columns}
\begin{column}{0.45\textwidth}
\begin{itemize}
\item Backtrace:
\begin{itemize}
\item \funcname{caml\_start\_program} $\rightarrow$ OCaml code
\item \funcname{caml\_startup\_common}
\item \funcname{caml\_startup\_exn}
\item \funcname{caml\_startup}
\item \funcname{caml\_main}
\item \funcname{main}
\end{itemize}
\smallskip
\item \funcname{caml\_startup} checks the return value of \funcname{caml\_startup\_exn}
\begin{itemize}
\item The return value has been specially marked by \funcname{caml\_start\_program} handler
\item Calls \funcname{caml\_fatal\_uncaught\_exception}
\end{itemize}
\item<2-> \funcname{caml\_fatal\_uncaught\_exception} either
\begin{itemize}
\item<2-> Calls the user custom uncaught exception handler, if set with \mintinline{ocaml}{Printexc.set_uncaught_exception_handler fn}\footnotemark
\item<2-> Or falls back to the default one
\item<2-> Which notifies about the uncaught exception
\end{itemize}
\item<4-> Terminates the program either with \funcname{abort} or with a non-zero exit code
\end{itemize}
\end{column}
\begin{column}{0.55\textwidth}
\only<1>{
\listc[firstline=84,lastline=89,highlightlines=88]{../ocaml/runtime/caml/mlvalues.h}{runtime/caml/mlvalues.h:84}
\listc[firstline=138,lastline=143,highlightlines={141-142}]{../ocaml/runtime/startup\_nat.c}{runtime/startup\_nat.c:138}
}%
\only<2>{
\listc[firstline=138,highlightlines={147,149}]{../ocaml/runtime/printexc.c}{runtime/printexc.c:138}
}%
\only<3>{
\listc[firstline=110,lastline=134,highlightlines=129]{../ocaml/runtime/printexc.c}{runtime/printexc.c:138}
}%
\only<4>{
\listc[firstline=138,highlightlines={152,154}]{../ocaml/runtime/printexc.c}{runtime/printexc.c:138}
}%
\end{column}
\end{columns}
\footnotetext[1]{\url{https://v2.ocaml.org/api/Printexc.html\#1\_Uncaughtexceptions}}
\end{frame}
}%
\end{column}
\end{columns}
\end{frame}

\begin{frame}{Executing the OCaml program}{And raising an exception without handler}
\begin{columns}
\begin{column}{0.6\textwidth}
\only<1-3>{
\begin{itemize}
\item<1-> Execution continues with OCaml code
\begin{itemize}
\item \funcname{camlDefault\_\_main\_269}
\item \funcname{camlDefault\_\_entry}
\item \funcname{caml\_program}
\end{itemize}
\item<1-> \funcname{camlDefault\_\_main\_269} raises a \typename{Failure} exception
\item<1-> \typename{Failure} is caught by \funcname{caml\_start\_program}
\begin{itemize}
\item<1-> The handler set the 2nd LSB of the exception value to \funcarg{1}
\item<3-> And then forces \funcname{caml\_start\_program} to terminate
\end{itemize}
\end{itemize}
}
\bigskip
\only<1,3>{\listocaml[highlightlines=3]{../src/default/default.ml}{default/default.ml}}%
\only<2>{\listasm[firstline=66,lastline=72,highlightlines=69]{src/ocaml/caml\_start\_program.s}{caml\_start\_program}}%
\end{column}
\begin{column}{0.4\textwidth}
\centering
\providecommand\step{4}%
% Runtime's default exception handler
%
\subsection*{Runtime's default exception handler}
\frameSubsection{pictures/The\_Architect.png}{}


\begin{frame}{Default exception handler}
\begin{columns}
\begin{column}{0.5\textwidth}
\begin{itemize}
\item \localname{domain\_state->exn\_handler} points to a trap located before \funcname{entry}!
\item What installed this very first trap there?
\end{itemize}
\bigskip
\listocaml{../src/default/default.ml}{default/default.ml}
\bigskip
\inputminted[fontsize=\footnotesize]{text}{../src/default/default.out}
\end{column}
\begin{column}{0.5\textwidth}
\centering
\only<1>{\providecommand\step{0}\input{diagrams/default/default.tex}}%
\only<2>{\providecommand\step{4}\input{diagrams/default/default.tex}}%
\end{column}
\end{columns}
\end{frame}


\begin{frame}{Startup}{How did we get there by the way?}
\begin{columns}
\begin{column}{0.45\textwidth}
\begin{itemize}
\item The entry point address of the binary is \funcname{main} (\funcname{\_start}) from the runtime
\item The program starts like a C program:
\begin{itemize}
\item \funcname{caml\_start\_program}
\item \funcname{caml\_startup\_common}
\item \funcname{caml\_startup\_exn}
\item \funcname{caml\_startup}
\item \funcname{caml\_main}
\item \funcname{main}
\end{itemize}
%\item \funcname{caml\_startup\_common}: initializes GC, domains \dots
% Also: codefrag, locale, custom opeartions, frame_descr, signals, stack
\item \funcname{caml\_start\_program} is the transition point between the C realm and the OCaml one
\end{itemize}
\bigskip
\listocaml{../src/default/default.ml}{default/default.ml}
\end{column}
\begin{column}{0.55\textwidth}
\listasm[lastline=24,highlightlines={22-24}]{src/ocaml/caml\_start\_program.s}{runtime/amd64.S:604}
\end{column}
\end{columns}
\end{frame}


\begin{frame}{Setup to jump into OCaml entry point}{\funcname{caml\_start\_program} initializes the main fiber}
\begin{columns}
\begin{column}{0.6\textwidth}
\only<1,3,5>{
\begin{itemize}
\item<1-> Stores information to allow switching from OCaml stack to the C stack
\item<3-> Constructs the default exception handler
\item<5-> Switches to the OCaml stack and calls \funcname{caml\_program}
\end{itemize}
\bigskip
\listocaml{../src/default/default.ml}{default/default.ml}
}%
\only<2>{
\listasm[firstline=22,lastline=41,highlightlines={25-27}]{src/ocaml/caml\_start\_program.s}{caml\_start\_program}
}%
\only<4>{
\mintasm[firstline=22,lastline=41,highlightlines={31-38}]{src/ocaml/caml\_start\_program.s}
\listasm[firstline=66,lastline=72]{src/ocaml/caml\_start\_program.s}{caml\_start\_program}
}%
\only<6>{
\listasm[firstline=22,lastline=41,highlightlines={39-41}]{src/ocaml/caml\_start\_program.s}{caml\_start\_program}
}%
\end{column}
\begin{column}{0.4\textwidth}
\centering
\only<1-2>{\providecommand\step{2}\input{diagrams/default/default.tex}}%
\only<3-5>{\providecommand\step{3}\input{diagrams/default/default.tex}}%
\onslide*<6->{\providecommand\step{4}\input{diagrams/default/default.tex}}%
\end{column}
\end{columns}
\end{frame}

\begin{frame}{Executing the OCaml program}{And raising an exception without handler}
\begin{columns}
\begin{column}{0.6\textwidth}
\only<1-3>{
\begin{itemize}
\item<1-> Execution continues with OCaml code
\begin{itemize}
\item \funcname{camlDefault\_\_main\_269}
\item \funcname{camlDefault\_\_entry}
\item \funcname{caml\_program}
\end{itemize}
\item<1-> \funcname{camlDefault\_\_main\_269} raises a \typename{Failure} exception
\item<1-> \typename{Failure} is caught by \funcname{caml\_start\_program}
\begin{itemize}
\item<1-> The handler set the 2nd LSB of the exception value to \funcarg{1}
\item<3-> And then forces \funcname{caml\_start\_program} to terminate
\end{itemize}
\end{itemize}
}
\bigskip
\only<1,3>{\listocaml[highlightlines=3]{../src/default/default.ml}{default/default.ml}}%
\only<2>{\listasm[firstline=66,lastline=72,highlightlines=69]{src/ocaml/caml\_start\_program.s}{caml\_start\_program}}%
\end{column}
\begin{column}{0.4\textwidth}
\centering
\providecommand\step{4}\input{diagrams/default/default.tex}
\end{column}
\end{columns}
\end{frame}
%\only<>{\listasm[firstline=47,lastline=65]{src/ocaml/caml\_start\_program.s}{caml\_start\_program}}


\begin{frame}{Back to C}{Clean up, complain and terminate}
\begin{columns}
\begin{column}{0.45\textwidth}
\begin{itemize}
\item Backtrace:
\begin{itemize}
\item \funcname{caml\_start\_program} $\rightarrow$ OCaml code
\item \funcname{caml\_startup\_common}
\item \funcname{caml\_startup\_exn}
\item \funcname{caml\_startup}
\item \funcname{caml\_main}
\item \funcname{main}
\end{itemize}
\smallskip
\item \funcname{caml\_startup} checks the return value of \funcname{caml\_startup\_exn}
\begin{itemize}
\item The return value has been specially marked by \funcname{caml\_start\_program} handler
\item Calls \funcname{caml\_fatal\_uncaught\_exception}
\end{itemize}
\item<2-> \funcname{caml\_fatal\_uncaught\_exception} either
\begin{itemize}
\item<2-> Calls the user custom uncaught exception handler, if set with \mintinline{ocaml}{Printexc.set_uncaught_exception_handler fn}\footnotemark
\item<2-> Or falls back to the default one
\item<2-> Which notifies about the uncaught exception
\end{itemize}
\item<4-> Terminates the program either with \funcname{abort} or with a non-zero exit code
\end{itemize}
\end{column}
\begin{column}{0.55\textwidth}
\only<1>{
\listc[firstline=84,lastline=89,highlightlines=88]{../ocaml/runtime/caml/mlvalues.h}{runtime/caml/mlvalues.h:84}
\listc[firstline=138,lastline=143,highlightlines={141-142}]{../ocaml/runtime/startup\_nat.c}{runtime/startup\_nat.c:138}
}%
\only<2>{
\listc[firstline=138,highlightlines={147,149}]{../ocaml/runtime/printexc.c}{runtime/printexc.c:138}
}%
\only<3>{
\listc[firstline=110,lastline=134,highlightlines=129]{../ocaml/runtime/printexc.c}{runtime/printexc.c:138}
}%
\only<4>{
\listc[firstline=138,highlightlines={152,154}]{../ocaml/runtime/printexc.c}{runtime/printexc.c:138}
}%
\end{column}
\end{columns}
\footnotetext[1]{\url{https://v2.ocaml.org/api/Printexc.html\#1\_Uncaughtexceptions}}
\end{frame}

\end{column}
\end{columns}
\end{frame}
%\only<>{\listasm[firstline=47,lastline=65]{src/ocaml/caml\_start\_program.s}{caml\_start\_program}}


\begin{frame}{Back to C}{Clean up, complain and terminate}
\begin{columns}
\begin{column}{0.45\textwidth}
\begin{itemize}
\item Backtrace:
\begin{itemize}
\item \funcname{caml\_start\_program} $\rightarrow$ OCaml code
\item \funcname{caml\_startup\_common}
\item \funcname{caml\_startup\_exn}
\item \funcname{caml\_startup}
\item \funcname{caml\_main}
\item \funcname{main}
\end{itemize}
\smallskip
\item \funcname{caml\_startup} checks the return value of \funcname{caml\_startup\_exn}
\begin{itemize}
\item The return value has been specially marked by \funcname{caml\_start\_program} handler
\item Calls \funcname{caml\_fatal\_uncaught\_exception}
\end{itemize}
\item<2-> \funcname{caml\_fatal\_uncaught\_exception} either
\begin{itemize}
\item<2-> Calls the user custom uncaught exception handler, if set with \mintinline{ocaml}{Printexc.set_uncaught_exception_handler fn}\footnotemark
\item<2-> Or falls back to the default one
\item<2-> Which notifies about the uncaught exception
\end{itemize}
\item<4-> Terminates the program either with \funcname{abort} or with a non-zero exit code
\end{itemize}
\end{column}
\begin{column}{0.55\textwidth}
\only<1>{
\listc[firstline=84,lastline=89,highlightlines=88]{../ocaml/runtime/caml/mlvalues.h}{runtime/caml/mlvalues.h:84}
\listc[firstline=138,lastline=143,highlightlines={141-142}]{../ocaml/runtime/startup\_nat.c}{runtime/startup\_nat.c:138}
}%
\only<2>{
\listc[firstline=138,highlightlines={147,149}]{../ocaml/runtime/printexc.c}{runtime/printexc.c:138}
}%
\only<3>{
\listc[firstline=110,lastline=134,highlightlines=129]{../ocaml/runtime/printexc.c}{runtime/printexc.c:138}
}%
\only<4>{
\listc[firstline=138,highlightlines={152,154}]{../ocaml/runtime/printexc.c}{runtime/printexc.c:138}
}%
\end{column}
\end{columns}
\footnotetext[1]{\url{https://v2.ocaml.org/api/Printexc.html\#1\_Uncaughtexceptions}}
\end{frame}
}%
\only<3-5>{\providecommand\step{3}%
% Runtime's default exception handler
%
\subsection*{Runtime's default exception handler}
\frameSubsection{pictures/The\_Architect.png}{}


\begin{frame}{Default exception handler}
\begin{columns}
\begin{column}{0.5\textwidth}
\begin{itemize}
\item \localname{domain\_state->exn\_handler} points to a trap located before \funcname{entry}!
\item What installed this very first trap there?
\end{itemize}
\bigskip
\listocaml{../src/default/default.ml}{default/default.ml}
\bigskip
\inputminted[fontsize=\footnotesize]{text}{../src/default/default.out}
\end{column}
\begin{column}{0.5\textwidth}
\centering
\only<1>{\providecommand\step{0}%
% Runtime's default exception handler
%
\subsection*{Runtime's default exception handler}
\frameSubsection{pictures/The\_Architect.png}{}


\begin{frame}{Default exception handler}
\begin{columns}
\begin{column}{0.5\textwidth}
\begin{itemize}
\item \localname{domain\_state->exn\_handler} points to a trap located before \funcname{entry}!
\item What installed this very first trap there?
\end{itemize}
\bigskip
\listocaml{../src/default/default.ml}{default/default.ml}
\bigskip
\inputminted[fontsize=\footnotesize]{text}{../src/default/default.out}
\end{column}
\begin{column}{0.5\textwidth}
\centering
\only<1>{\providecommand\step{0}\input{diagrams/default/default.tex}}%
\only<2>{\providecommand\step{4}\input{diagrams/default/default.tex}}%
\end{column}
\end{columns}
\end{frame}


\begin{frame}{Startup}{How did we get there by the way?}
\begin{columns}
\begin{column}{0.45\textwidth}
\begin{itemize}
\item The entry point address of the binary is \funcname{main} (\funcname{\_start}) from the runtime
\item The program starts like a C program:
\begin{itemize}
\item \funcname{caml\_start\_program}
\item \funcname{caml\_startup\_common}
\item \funcname{caml\_startup\_exn}
\item \funcname{caml\_startup}
\item \funcname{caml\_main}
\item \funcname{main}
\end{itemize}
%\item \funcname{caml\_startup\_common}: initializes GC, domains \dots
% Also: codefrag, locale, custom opeartions, frame_descr, signals, stack
\item \funcname{caml\_start\_program} is the transition point between the C realm and the OCaml one
\end{itemize}
\bigskip
\listocaml{../src/default/default.ml}{default/default.ml}
\end{column}
\begin{column}{0.55\textwidth}
\listasm[lastline=24,highlightlines={22-24}]{src/ocaml/caml\_start\_program.s}{runtime/amd64.S:604}
\end{column}
\end{columns}
\end{frame}


\begin{frame}{Setup to jump into OCaml entry point}{\funcname{caml\_start\_program} initializes the main fiber}
\begin{columns}
\begin{column}{0.6\textwidth}
\only<1,3,5>{
\begin{itemize}
\item<1-> Stores information to allow switching from OCaml stack to the C stack
\item<3-> Constructs the default exception handler
\item<5-> Switches to the OCaml stack and calls \funcname{caml\_program}
\end{itemize}
\bigskip
\listocaml{../src/default/default.ml}{default/default.ml}
}%
\only<2>{
\listasm[firstline=22,lastline=41,highlightlines={25-27}]{src/ocaml/caml\_start\_program.s}{caml\_start\_program}
}%
\only<4>{
\mintasm[firstline=22,lastline=41,highlightlines={31-38}]{src/ocaml/caml\_start\_program.s}
\listasm[firstline=66,lastline=72]{src/ocaml/caml\_start\_program.s}{caml\_start\_program}
}%
\only<6>{
\listasm[firstline=22,lastline=41,highlightlines={39-41}]{src/ocaml/caml\_start\_program.s}{caml\_start\_program}
}%
\end{column}
\begin{column}{0.4\textwidth}
\centering
\only<1-2>{\providecommand\step{2}\input{diagrams/default/default.tex}}%
\only<3-5>{\providecommand\step{3}\input{diagrams/default/default.tex}}%
\onslide*<6->{\providecommand\step{4}\input{diagrams/default/default.tex}}%
\end{column}
\end{columns}
\end{frame}

\begin{frame}{Executing the OCaml program}{And raising an exception without handler}
\begin{columns}
\begin{column}{0.6\textwidth}
\only<1-3>{
\begin{itemize}
\item<1-> Execution continues with OCaml code
\begin{itemize}
\item \funcname{camlDefault\_\_main\_269}
\item \funcname{camlDefault\_\_entry}
\item \funcname{caml\_program}
\end{itemize}
\item<1-> \funcname{camlDefault\_\_main\_269} raises a \typename{Failure} exception
\item<1-> \typename{Failure} is caught by \funcname{caml\_start\_program}
\begin{itemize}
\item<1-> The handler set the 2nd LSB of the exception value to \funcarg{1}
\item<3-> And then forces \funcname{caml\_start\_program} to terminate
\end{itemize}
\end{itemize}
}
\bigskip
\only<1,3>{\listocaml[highlightlines=3]{../src/default/default.ml}{default/default.ml}}%
\only<2>{\listasm[firstline=66,lastline=72,highlightlines=69]{src/ocaml/caml\_start\_program.s}{caml\_start\_program}}%
\end{column}
\begin{column}{0.4\textwidth}
\centering
\providecommand\step{4}\input{diagrams/default/default.tex}
\end{column}
\end{columns}
\end{frame}
%\only<>{\listasm[firstline=47,lastline=65]{src/ocaml/caml\_start\_program.s}{caml\_start\_program}}


\begin{frame}{Back to C}{Clean up, complain and terminate}
\begin{columns}
\begin{column}{0.45\textwidth}
\begin{itemize}
\item Backtrace:
\begin{itemize}
\item \funcname{caml\_start\_program} $\rightarrow$ OCaml code
\item \funcname{caml\_startup\_common}
\item \funcname{caml\_startup\_exn}
\item \funcname{caml\_startup}
\item \funcname{caml\_main}
\item \funcname{main}
\end{itemize}
\smallskip
\item \funcname{caml\_startup} checks the return value of \funcname{caml\_startup\_exn}
\begin{itemize}
\item The return value has been specially marked by \funcname{caml\_start\_program} handler
\item Calls \funcname{caml\_fatal\_uncaught\_exception}
\end{itemize}
\item<2-> \funcname{caml\_fatal\_uncaught\_exception} either
\begin{itemize}
\item<2-> Calls the user custom uncaught exception handler, if set with \mintinline{ocaml}{Printexc.set_uncaught_exception_handler fn}\footnotemark
\item<2-> Or falls back to the default one
\item<2-> Which notifies about the uncaught exception
\end{itemize}
\item<4-> Terminates the program either with \funcname{abort} or with a non-zero exit code
\end{itemize}
\end{column}
\begin{column}{0.55\textwidth}
\only<1>{
\listc[firstline=84,lastline=89,highlightlines=88]{../ocaml/runtime/caml/mlvalues.h}{runtime/caml/mlvalues.h:84}
\listc[firstline=138,lastline=143,highlightlines={141-142}]{../ocaml/runtime/startup\_nat.c}{runtime/startup\_nat.c:138}
}%
\only<2>{
\listc[firstline=138,highlightlines={147,149}]{../ocaml/runtime/printexc.c}{runtime/printexc.c:138}
}%
\only<3>{
\listc[firstline=110,lastline=134,highlightlines=129]{../ocaml/runtime/printexc.c}{runtime/printexc.c:138}
}%
\only<4>{
\listc[firstline=138,highlightlines={152,154}]{../ocaml/runtime/printexc.c}{runtime/printexc.c:138}
}%
\end{column}
\end{columns}
\footnotetext[1]{\url{https://v2.ocaml.org/api/Printexc.html\#1\_Uncaughtexceptions}}
\end{frame}
}%
\only<2>{\providecommand\step{4}%
% Runtime's default exception handler
%
\subsection*{Runtime's default exception handler}
\frameSubsection{pictures/The\_Architect.png}{}


\begin{frame}{Default exception handler}
\begin{columns}
\begin{column}{0.5\textwidth}
\begin{itemize}
\item \localname{domain\_state->exn\_handler} points to a trap located before \funcname{entry}!
\item What installed this very first trap there?
\end{itemize}
\bigskip
\listocaml{../src/default/default.ml}{default/default.ml}
\bigskip
\inputminted[fontsize=\footnotesize]{text}{../src/default/default.out}
\end{column}
\begin{column}{0.5\textwidth}
\centering
\only<1>{\providecommand\step{0}\input{diagrams/default/default.tex}}%
\only<2>{\providecommand\step{4}\input{diagrams/default/default.tex}}%
\end{column}
\end{columns}
\end{frame}


\begin{frame}{Startup}{How did we get there by the way?}
\begin{columns}
\begin{column}{0.45\textwidth}
\begin{itemize}
\item The entry point address of the binary is \funcname{main} (\funcname{\_start}) from the runtime
\item The program starts like a C program:
\begin{itemize}
\item \funcname{caml\_start\_program}
\item \funcname{caml\_startup\_common}
\item \funcname{caml\_startup\_exn}
\item \funcname{caml\_startup}
\item \funcname{caml\_main}
\item \funcname{main}
\end{itemize}
%\item \funcname{caml\_startup\_common}: initializes GC, domains \dots
% Also: codefrag, locale, custom opeartions, frame_descr, signals, stack
\item \funcname{caml\_start\_program} is the transition point between the C realm and the OCaml one
\end{itemize}
\bigskip
\listocaml{../src/default/default.ml}{default/default.ml}
\end{column}
\begin{column}{0.55\textwidth}
\listasm[lastline=24,highlightlines={22-24}]{src/ocaml/caml\_start\_program.s}{runtime/amd64.S:604}
\end{column}
\end{columns}
\end{frame}


\begin{frame}{Setup to jump into OCaml entry point}{\funcname{caml\_start\_program} initializes the main fiber}
\begin{columns}
\begin{column}{0.6\textwidth}
\only<1,3,5>{
\begin{itemize}
\item<1-> Stores information to allow switching from OCaml stack to the C stack
\item<3-> Constructs the default exception handler
\item<5-> Switches to the OCaml stack and calls \funcname{caml\_program}
\end{itemize}
\bigskip
\listocaml{../src/default/default.ml}{default/default.ml}
}%
\only<2>{
\listasm[firstline=22,lastline=41,highlightlines={25-27}]{src/ocaml/caml\_start\_program.s}{caml\_start\_program}
}%
\only<4>{
\mintasm[firstline=22,lastline=41,highlightlines={31-38}]{src/ocaml/caml\_start\_program.s}
\listasm[firstline=66,lastline=72]{src/ocaml/caml\_start\_program.s}{caml\_start\_program}
}%
\only<6>{
\listasm[firstline=22,lastline=41,highlightlines={39-41}]{src/ocaml/caml\_start\_program.s}{caml\_start\_program}
}%
\end{column}
\begin{column}{0.4\textwidth}
\centering
\only<1-2>{\providecommand\step{2}\input{diagrams/default/default.tex}}%
\only<3-5>{\providecommand\step{3}\input{diagrams/default/default.tex}}%
\onslide*<6->{\providecommand\step{4}\input{diagrams/default/default.tex}}%
\end{column}
\end{columns}
\end{frame}

\begin{frame}{Executing the OCaml program}{And raising an exception without handler}
\begin{columns}
\begin{column}{0.6\textwidth}
\only<1-3>{
\begin{itemize}
\item<1-> Execution continues with OCaml code
\begin{itemize}
\item \funcname{camlDefault\_\_main\_269}
\item \funcname{camlDefault\_\_entry}
\item \funcname{caml\_program}
\end{itemize}
\item<1-> \funcname{camlDefault\_\_main\_269} raises a \typename{Failure} exception
\item<1-> \typename{Failure} is caught by \funcname{caml\_start\_program}
\begin{itemize}
\item<1-> The handler set the 2nd LSB of the exception value to \funcarg{1}
\item<3-> And then forces \funcname{caml\_start\_program} to terminate
\end{itemize}
\end{itemize}
}
\bigskip
\only<1,3>{\listocaml[highlightlines=3]{../src/default/default.ml}{default/default.ml}}%
\only<2>{\listasm[firstline=66,lastline=72,highlightlines=69]{src/ocaml/caml\_start\_program.s}{caml\_start\_program}}%
\end{column}
\begin{column}{0.4\textwidth}
\centering
\providecommand\step{4}\input{diagrams/default/default.tex}
\end{column}
\end{columns}
\end{frame}
%\only<>{\listasm[firstline=47,lastline=65]{src/ocaml/caml\_start\_program.s}{caml\_start\_program}}


\begin{frame}{Back to C}{Clean up, complain and terminate}
\begin{columns}
\begin{column}{0.45\textwidth}
\begin{itemize}
\item Backtrace:
\begin{itemize}
\item \funcname{caml\_start\_program} $\rightarrow$ OCaml code
\item \funcname{caml\_startup\_common}
\item \funcname{caml\_startup\_exn}
\item \funcname{caml\_startup}
\item \funcname{caml\_main}
\item \funcname{main}
\end{itemize}
\smallskip
\item \funcname{caml\_startup} checks the return value of \funcname{caml\_startup\_exn}
\begin{itemize}
\item The return value has been specially marked by \funcname{caml\_start\_program} handler
\item Calls \funcname{caml\_fatal\_uncaught\_exception}
\end{itemize}
\item<2-> \funcname{caml\_fatal\_uncaught\_exception} either
\begin{itemize}
\item<2-> Calls the user custom uncaught exception handler, if set with \mintinline{ocaml}{Printexc.set_uncaught_exception_handler fn}\footnotemark
\item<2-> Or falls back to the default one
\item<2-> Which notifies about the uncaught exception
\end{itemize}
\item<4-> Terminates the program either with \funcname{abort} or with a non-zero exit code
\end{itemize}
\end{column}
\begin{column}{0.55\textwidth}
\only<1>{
\listc[firstline=84,lastline=89,highlightlines=88]{../ocaml/runtime/caml/mlvalues.h}{runtime/caml/mlvalues.h:84}
\listc[firstline=138,lastline=143,highlightlines={141-142}]{../ocaml/runtime/startup\_nat.c}{runtime/startup\_nat.c:138}
}%
\only<2>{
\listc[firstline=138,highlightlines={147,149}]{../ocaml/runtime/printexc.c}{runtime/printexc.c:138}
}%
\only<3>{
\listc[firstline=110,lastline=134,highlightlines=129]{../ocaml/runtime/printexc.c}{runtime/printexc.c:138}
}%
\only<4>{
\listc[firstline=138,highlightlines={152,154}]{../ocaml/runtime/printexc.c}{runtime/printexc.c:138}
}%
\end{column}
\end{columns}
\footnotetext[1]{\url{https://v2.ocaml.org/api/Printexc.html\#1\_Uncaughtexceptions}}
\end{frame}
}%
\end{column}
\end{columns}
\end{frame}


\begin{frame}{Startup}{How did we get there by the way?}
\begin{columns}
\begin{column}{0.45\textwidth}
\begin{itemize}
\item The entry point address of the binary is \funcname{main} (\funcname{\_start}) from the runtime
\item The program starts like a C program:
\begin{itemize}
\item \funcname{caml\_start\_program}
\item \funcname{caml\_startup\_common}
\item \funcname{caml\_startup\_exn}
\item \funcname{caml\_startup}
\item \funcname{caml\_main}
\item \funcname{main}
\end{itemize}
%\item \funcname{caml\_startup\_common}: initializes GC, domains \dots
% Also: codefrag, locale, custom opeartions, frame_descr, signals, stack
\item \funcname{caml\_start\_program} is the transition point between the C realm and the OCaml one
\end{itemize}
\bigskip
\listocaml{../src/default/default.ml}{default/default.ml}
\end{column}
\begin{column}{0.55\textwidth}
\listasm[lastline=24,highlightlines={22-24}]{src/ocaml/caml\_start\_program.s}{runtime/amd64.S:604}
\end{column}
\end{columns}
\end{frame}


\begin{frame}{Setup to jump into OCaml entry point}{\funcname{caml\_start\_program} initializes the main fiber}
\begin{columns}
\begin{column}{0.6\textwidth}
\only<1,3,5>{
\begin{itemize}
\item<1-> Stores information to allow switching from OCaml stack to the C stack
\item<3-> Constructs the default exception handler
\item<5-> Switches to the OCaml stack and calls \funcname{caml\_program}
\end{itemize}
\bigskip
\listocaml{../src/default/default.ml}{default/default.ml}
}%
\only<2>{
\listasm[firstline=22,lastline=41,highlightlines={25-27}]{src/ocaml/caml\_start\_program.s}{caml\_start\_program}
}%
\only<4>{
\mintasm[firstline=22,lastline=41,highlightlines={31-38}]{src/ocaml/caml\_start\_program.s}
\listasm[firstline=66,lastline=72]{src/ocaml/caml\_start\_program.s}{caml\_start\_program}
}%
\only<6>{
\listasm[firstline=22,lastline=41,highlightlines={39-41}]{src/ocaml/caml\_start\_program.s}{caml\_start\_program}
}%
\end{column}
\begin{column}{0.4\textwidth}
\centering
\only<1-2>{\providecommand\step{2}%
% Runtime's default exception handler
%
\subsection*{Runtime's default exception handler}
\frameSubsection{pictures/The\_Architect.png}{}


\begin{frame}{Default exception handler}
\begin{columns}
\begin{column}{0.5\textwidth}
\begin{itemize}
\item \localname{domain\_state->exn\_handler} points to a trap located before \funcname{entry}!
\item What installed this very first trap there?
\end{itemize}
\bigskip
\listocaml{../src/default/default.ml}{default/default.ml}
\bigskip
\inputminted[fontsize=\footnotesize]{text}{../src/default/default.out}
\end{column}
\begin{column}{0.5\textwidth}
\centering
\only<1>{\providecommand\step{0}\input{diagrams/default/default.tex}}%
\only<2>{\providecommand\step{4}\input{diagrams/default/default.tex}}%
\end{column}
\end{columns}
\end{frame}


\begin{frame}{Startup}{How did we get there by the way?}
\begin{columns}
\begin{column}{0.45\textwidth}
\begin{itemize}
\item The entry point address of the binary is \funcname{main} (\funcname{\_start}) from the runtime
\item The program starts like a C program:
\begin{itemize}
\item \funcname{caml\_start\_program}
\item \funcname{caml\_startup\_common}
\item \funcname{caml\_startup\_exn}
\item \funcname{caml\_startup}
\item \funcname{caml\_main}
\item \funcname{main}
\end{itemize}
%\item \funcname{caml\_startup\_common}: initializes GC, domains \dots
% Also: codefrag, locale, custom opeartions, frame_descr, signals, stack
\item \funcname{caml\_start\_program} is the transition point between the C realm and the OCaml one
\end{itemize}
\bigskip
\listocaml{../src/default/default.ml}{default/default.ml}
\end{column}
\begin{column}{0.55\textwidth}
\listasm[lastline=24,highlightlines={22-24}]{src/ocaml/caml\_start\_program.s}{runtime/amd64.S:604}
\end{column}
\end{columns}
\end{frame}


\begin{frame}{Setup to jump into OCaml entry point}{\funcname{caml\_start\_program} initializes the main fiber}
\begin{columns}
\begin{column}{0.6\textwidth}
\only<1,3,5>{
\begin{itemize}
\item<1-> Stores information to allow switching from OCaml stack to the C stack
\item<3-> Constructs the default exception handler
\item<5-> Switches to the OCaml stack and calls \funcname{caml\_program}
\end{itemize}
\bigskip
\listocaml{../src/default/default.ml}{default/default.ml}
}%
\only<2>{
\listasm[firstline=22,lastline=41,highlightlines={25-27}]{src/ocaml/caml\_start\_program.s}{caml\_start\_program}
}%
\only<4>{
\mintasm[firstline=22,lastline=41,highlightlines={31-38}]{src/ocaml/caml\_start\_program.s}
\listasm[firstline=66,lastline=72]{src/ocaml/caml\_start\_program.s}{caml\_start\_program}
}%
\only<6>{
\listasm[firstline=22,lastline=41,highlightlines={39-41}]{src/ocaml/caml\_start\_program.s}{caml\_start\_program}
}%
\end{column}
\begin{column}{0.4\textwidth}
\centering
\only<1-2>{\providecommand\step{2}\input{diagrams/default/default.tex}}%
\only<3-5>{\providecommand\step{3}\input{diagrams/default/default.tex}}%
\onslide*<6->{\providecommand\step{4}\input{diagrams/default/default.tex}}%
\end{column}
\end{columns}
\end{frame}

\begin{frame}{Executing the OCaml program}{And raising an exception without handler}
\begin{columns}
\begin{column}{0.6\textwidth}
\only<1-3>{
\begin{itemize}
\item<1-> Execution continues with OCaml code
\begin{itemize}
\item \funcname{camlDefault\_\_main\_269}
\item \funcname{camlDefault\_\_entry}
\item \funcname{caml\_program}
\end{itemize}
\item<1-> \funcname{camlDefault\_\_main\_269} raises a \typename{Failure} exception
\item<1-> \typename{Failure} is caught by \funcname{caml\_start\_program}
\begin{itemize}
\item<1-> The handler set the 2nd LSB of the exception value to \funcarg{1}
\item<3-> And then forces \funcname{caml\_start\_program} to terminate
\end{itemize}
\end{itemize}
}
\bigskip
\only<1,3>{\listocaml[highlightlines=3]{../src/default/default.ml}{default/default.ml}}%
\only<2>{\listasm[firstline=66,lastline=72,highlightlines=69]{src/ocaml/caml\_start\_program.s}{caml\_start\_program}}%
\end{column}
\begin{column}{0.4\textwidth}
\centering
\providecommand\step{4}\input{diagrams/default/default.tex}
\end{column}
\end{columns}
\end{frame}
%\only<>{\listasm[firstline=47,lastline=65]{src/ocaml/caml\_start\_program.s}{caml\_start\_program}}


\begin{frame}{Back to C}{Clean up, complain and terminate}
\begin{columns}
\begin{column}{0.45\textwidth}
\begin{itemize}
\item Backtrace:
\begin{itemize}
\item \funcname{caml\_start\_program} $\rightarrow$ OCaml code
\item \funcname{caml\_startup\_common}
\item \funcname{caml\_startup\_exn}
\item \funcname{caml\_startup}
\item \funcname{caml\_main}
\item \funcname{main}
\end{itemize}
\smallskip
\item \funcname{caml\_startup} checks the return value of \funcname{caml\_startup\_exn}
\begin{itemize}
\item The return value has been specially marked by \funcname{caml\_start\_program} handler
\item Calls \funcname{caml\_fatal\_uncaught\_exception}
\end{itemize}
\item<2-> \funcname{caml\_fatal\_uncaught\_exception} either
\begin{itemize}
\item<2-> Calls the user custom uncaught exception handler, if set with \mintinline{ocaml}{Printexc.set_uncaught_exception_handler fn}\footnotemark
\item<2-> Or falls back to the default one
\item<2-> Which notifies about the uncaught exception
\end{itemize}
\item<4-> Terminates the program either with \funcname{abort} or with a non-zero exit code
\end{itemize}
\end{column}
\begin{column}{0.55\textwidth}
\only<1>{
\listc[firstline=84,lastline=89,highlightlines=88]{../ocaml/runtime/caml/mlvalues.h}{runtime/caml/mlvalues.h:84}
\listc[firstline=138,lastline=143,highlightlines={141-142}]{../ocaml/runtime/startup\_nat.c}{runtime/startup\_nat.c:138}
}%
\only<2>{
\listc[firstline=138,highlightlines={147,149}]{../ocaml/runtime/printexc.c}{runtime/printexc.c:138}
}%
\only<3>{
\listc[firstline=110,lastline=134,highlightlines=129]{../ocaml/runtime/printexc.c}{runtime/printexc.c:138}
}%
\only<4>{
\listc[firstline=138,highlightlines={152,154}]{../ocaml/runtime/printexc.c}{runtime/printexc.c:138}
}%
\end{column}
\end{columns}
\footnotetext[1]{\url{https://v2.ocaml.org/api/Printexc.html\#1\_Uncaughtexceptions}}
\end{frame}
}%
\only<3-5>{\providecommand\step{3}%
% Runtime's default exception handler
%
\subsection*{Runtime's default exception handler}
\frameSubsection{pictures/The\_Architect.png}{}


\begin{frame}{Default exception handler}
\begin{columns}
\begin{column}{0.5\textwidth}
\begin{itemize}
\item \localname{domain\_state->exn\_handler} points to a trap located before \funcname{entry}!
\item What installed this very first trap there?
\end{itemize}
\bigskip
\listocaml{../src/default/default.ml}{default/default.ml}
\bigskip
\inputminted[fontsize=\footnotesize]{text}{../src/default/default.out}
\end{column}
\begin{column}{0.5\textwidth}
\centering
\only<1>{\providecommand\step{0}\input{diagrams/default/default.tex}}%
\only<2>{\providecommand\step{4}\input{diagrams/default/default.tex}}%
\end{column}
\end{columns}
\end{frame}


\begin{frame}{Startup}{How did we get there by the way?}
\begin{columns}
\begin{column}{0.45\textwidth}
\begin{itemize}
\item The entry point address of the binary is \funcname{main} (\funcname{\_start}) from the runtime
\item The program starts like a C program:
\begin{itemize}
\item \funcname{caml\_start\_program}
\item \funcname{caml\_startup\_common}
\item \funcname{caml\_startup\_exn}
\item \funcname{caml\_startup}
\item \funcname{caml\_main}
\item \funcname{main}
\end{itemize}
%\item \funcname{caml\_startup\_common}: initializes GC, domains \dots
% Also: codefrag, locale, custom opeartions, frame_descr, signals, stack
\item \funcname{caml\_start\_program} is the transition point between the C realm and the OCaml one
\end{itemize}
\bigskip
\listocaml{../src/default/default.ml}{default/default.ml}
\end{column}
\begin{column}{0.55\textwidth}
\listasm[lastline=24,highlightlines={22-24}]{src/ocaml/caml\_start\_program.s}{runtime/amd64.S:604}
\end{column}
\end{columns}
\end{frame}


\begin{frame}{Setup to jump into OCaml entry point}{\funcname{caml\_start\_program} initializes the main fiber}
\begin{columns}
\begin{column}{0.6\textwidth}
\only<1,3,5>{
\begin{itemize}
\item<1-> Stores information to allow switching from OCaml stack to the C stack
\item<3-> Constructs the default exception handler
\item<5-> Switches to the OCaml stack and calls \funcname{caml\_program}
\end{itemize}
\bigskip
\listocaml{../src/default/default.ml}{default/default.ml}
}%
\only<2>{
\listasm[firstline=22,lastline=41,highlightlines={25-27}]{src/ocaml/caml\_start\_program.s}{caml\_start\_program}
}%
\only<4>{
\mintasm[firstline=22,lastline=41,highlightlines={31-38}]{src/ocaml/caml\_start\_program.s}
\listasm[firstline=66,lastline=72]{src/ocaml/caml\_start\_program.s}{caml\_start\_program}
}%
\only<6>{
\listasm[firstline=22,lastline=41,highlightlines={39-41}]{src/ocaml/caml\_start\_program.s}{caml\_start\_program}
}%
\end{column}
\begin{column}{0.4\textwidth}
\centering
\only<1-2>{\providecommand\step{2}\input{diagrams/default/default.tex}}%
\only<3-5>{\providecommand\step{3}\input{diagrams/default/default.tex}}%
\onslide*<6->{\providecommand\step{4}\input{diagrams/default/default.tex}}%
\end{column}
\end{columns}
\end{frame}

\begin{frame}{Executing the OCaml program}{And raising an exception without handler}
\begin{columns}
\begin{column}{0.6\textwidth}
\only<1-3>{
\begin{itemize}
\item<1-> Execution continues with OCaml code
\begin{itemize}
\item \funcname{camlDefault\_\_main\_269}
\item \funcname{camlDefault\_\_entry}
\item \funcname{caml\_program}
\end{itemize}
\item<1-> \funcname{camlDefault\_\_main\_269} raises a \typename{Failure} exception
\item<1-> \typename{Failure} is caught by \funcname{caml\_start\_program}
\begin{itemize}
\item<1-> The handler set the 2nd LSB of the exception value to \funcarg{1}
\item<3-> And then forces \funcname{caml\_start\_program} to terminate
\end{itemize}
\end{itemize}
}
\bigskip
\only<1,3>{\listocaml[highlightlines=3]{../src/default/default.ml}{default/default.ml}}%
\only<2>{\listasm[firstline=66,lastline=72,highlightlines=69]{src/ocaml/caml\_start\_program.s}{caml\_start\_program}}%
\end{column}
\begin{column}{0.4\textwidth}
\centering
\providecommand\step{4}\input{diagrams/default/default.tex}
\end{column}
\end{columns}
\end{frame}
%\only<>{\listasm[firstline=47,lastline=65]{src/ocaml/caml\_start\_program.s}{caml\_start\_program}}


\begin{frame}{Back to C}{Clean up, complain and terminate}
\begin{columns}
\begin{column}{0.45\textwidth}
\begin{itemize}
\item Backtrace:
\begin{itemize}
\item \funcname{caml\_start\_program} $\rightarrow$ OCaml code
\item \funcname{caml\_startup\_common}
\item \funcname{caml\_startup\_exn}
\item \funcname{caml\_startup}
\item \funcname{caml\_main}
\item \funcname{main}
\end{itemize}
\smallskip
\item \funcname{caml\_startup} checks the return value of \funcname{caml\_startup\_exn}
\begin{itemize}
\item The return value has been specially marked by \funcname{caml\_start\_program} handler
\item Calls \funcname{caml\_fatal\_uncaught\_exception}
\end{itemize}
\item<2-> \funcname{caml\_fatal\_uncaught\_exception} either
\begin{itemize}
\item<2-> Calls the user custom uncaught exception handler, if set with \mintinline{ocaml}{Printexc.set_uncaught_exception_handler fn}\footnotemark
\item<2-> Or falls back to the default one
\item<2-> Which notifies about the uncaught exception
\end{itemize}
\item<4-> Terminates the program either with \funcname{abort} or with a non-zero exit code
\end{itemize}
\end{column}
\begin{column}{0.55\textwidth}
\only<1>{
\listc[firstline=84,lastline=89,highlightlines=88]{../ocaml/runtime/caml/mlvalues.h}{runtime/caml/mlvalues.h:84}
\listc[firstline=138,lastline=143,highlightlines={141-142}]{../ocaml/runtime/startup\_nat.c}{runtime/startup\_nat.c:138}
}%
\only<2>{
\listc[firstline=138,highlightlines={147,149}]{../ocaml/runtime/printexc.c}{runtime/printexc.c:138}
}%
\only<3>{
\listc[firstline=110,lastline=134,highlightlines=129]{../ocaml/runtime/printexc.c}{runtime/printexc.c:138}
}%
\only<4>{
\listc[firstline=138,highlightlines={152,154}]{../ocaml/runtime/printexc.c}{runtime/printexc.c:138}
}%
\end{column}
\end{columns}
\footnotetext[1]{\url{https://v2.ocaml.org/api/Printexc.html\#1\_Uncaughtexceptions}}
\end{frame}
}%
\onslide*<6->{\providecommand\step{4}%
% Runtime's default exception handler
%
\subsection*{Runtime's default exception handler}
\frameSubsection{pictures/The\_Architect.png}{}


\begin{frame}{Default exception handler}
\begin{columns}
\begin{column}{0.5\textwidth}
\begin{itemize}
\item \localname{domain\_state->exn\_handler} points to a trap located before \funcname{entry}!
\item What installed this very first trap there?
\end{itemize}
\bigskip
\listocaml{../src/default/default.ml}{default/default.ml}
\bigskip
\inputminted[fontsize=\footnotesize]{text}{../src/default/default.out}
\end{column}
\begin{column}{0.5\textwidth}
\centering
\only<1>{\providecommand\step{0}\input{diagrams/default/default.tex}}%
\only<2>{\providecommand\step{4}\input{diagrams/default/default.tex}}%
\end{column}
\end{columns}
\end{frame}


\begin{frame}{Startup}{How did we get there by the way?}
\begin{columns}
\begin{column}{0.45\textwidth}
\begin{itemize}
\item The entry point address of the binary is \funcname{main} (\funcname{\_start}) from the runtime
\item The program starts like a C program:
\begin{itemize}
\item \funcname{caml\_start\_program}
\item \funcname{caml\_startup\_common}
\item \funcname{caml\_startup\_exn}
\item \funcname{caml\_startup}
\item \funcname{caml\_main}
\item \funcname{main}
\end{itemize}
%\item \funcname{caml\_startup\_common}: initializes GC, domains \dots
% Also: codefrag, locale, custom opeartions, frame_descr, signals, stack
\item \funcname{caml\_start\_program} is the transition point between the C realm and the OCaml one
\end{itemize}
\bigskip
\listocaml{../src/default/default.ml}{default/default.ml}
\end{column}
\begin{column}{0.55\textwidth}
\listasm[lastline=24,highlightlines={22-24}]{src/ocaml/caml\_start\_program.s}{runtime/amd64.S:604}
\end{column}
\end{columns}
\end{frame}


\begin{frame}{Setup to jump into OCaml entry point}{\funcname{caml\_start\_program} initializes the main fiber}
\begin{columns}
\begin{column}{0.6\textwidth}
\only<1,3,5>{
\begin{itemize}
\item<1-> Stores information to allow switching from OCaml stack to the C stack
\item<3-> Constructs the default exception handler
\item<5-> Switches to the OCaml stack and calls \funcname{caml\_program}
\end{itemize}
\bigskip
\listocaml{../src/default/default.ml}{default/default.ml}
}%
\only<2>{
\listasm[firstline=22,lastline=41,highlightlines={25-27}]{src/ocaml/caml\_start\_program.s}{caml\_start\_program}
}%
\only<4>{
\mintasm[firstline=22,lastline=41,highlightlines={31-38}]{src/ocaml/caml\_start\_program.s}
\listasm[firstline=66,lastline=72]{src/ocaml/caml\_start\_program.s}{caml\_start\_program}
}%
\only<6>{
\listasm[firstline=22,lastline=41,highlightlines={39-41}]{src/ocaml/caml\_start\_program.s}{caml\_start\_program}
}%
\end{column}
\begin{column}{0.4\textwidth}
\centering
\only<1-2>{\providecommand\step{2}\input{diagrams/default/default.tex}}%
\only<3-5>{\providecommand\step{3}\input{diagrams/default/default.tex}}%
\onslide*<6->{\providecommand\step{4}\input{diagrams/default/default.tex}}%
\end{column}
\end{columns}
\end{frame}

\begin{frame}{Executing the OCaml program}{And raising an exception without handler}
\begin{columns}
\begin{column}{0.6\textwidth}
\only<1-3>{
\begin{itemize}
\item<1-> Execution continues with OCaml code
\begin{itemize}
\item \funcname{camlDefault\_\_main\_269}
\item \funcname{camlDefault\_\_entry}
\item \funcname{caml\_program}
\end{itemize}
\item<1-> \funcname{camlDefault\_\_main\_269} raises a \typename{Failure} exception
\item<1-> \typename{Failure} is caught by \funcname{caml\_start\_program}
\begin{itemize}
\item<1-> The handler set the 2nd LSB of the exception value to \funcarg{1}
\item<3-> And then forces \funcname{caml\_start\_program} to terminate
\end{itemize}
\end{itemize}
}
\bigskip
\only<1,3>{\listocaml[highlightlines=3]{../src/default/default.ml}{default/default.ml}}%
\only<2>{\listasm[firstline=66,lastline=72,highlightlines=69]{src/ocaml/caml\_start\_program.s}{caml\_start\_program}}%
\end{column}
\begin{column}{0.4\textwidth}
\centering
\providecommand\step{4}\input{diagrams/default/default.tex}
\end{column}
\end{columns}
\end{frame}
%\only<>{\listasm[firstline=47,lastline=65]{src/ocaml/caml\_start\_program.s}{caml\_start\_program}}


\begin{frame}{Back to C}{Clean up, complain and terminate}
\begin{columns}
\begin{column}{0.45\textwidth}
\begin{itemize}
\item Backtrace:
\begin{itemize}
\item \funcname{caml\_start\_program} $\rightarrow$ OCaml code
\item \funcname{caml\_startup\_common}
\item \funcname{caml\_startup\_exn}
\item \funcname{caml\_startup}
\item \funcname{caml\_main}
\item \funcname{main}
\end{itemize}
\smallskip
\item \funcname{caml\_startup} checks the return value of \funcname{caml\_startup\_exn}
\begin{itemize}
\item The return value has been specially marked by \funcname{caml\_start\_program} handler
\item Calls \funcname{caml\_fatal\_uncaught\_exception}
\end{itemize}
\item<2-> \funcname{caml\_fatal\_uncaught\_exception} either
\begin{itemize}
\item<2-> Calls the user custom uncaught exception handler, if set with \mintinline{ocaml}{Printexc.set_uncaught_exception_handler fn}\footnotemark
\item<2-> Or falls back to the default one
\item<2-> Which notifies about the uncaught exception
\end{itemize}
\item<4-> Terminates the program either with \funcname{abort} or with a non-zero exit code
\end{itemize}
\end{column}
\begin{column}{0.55\textwidth}
\only<1>{
\listc[firstline=84,lastline=89,highlightlines=88]{../ocaml/runtime/caml/mlvalues.h}{runtime/caml/mlvalues.h:84}
\listc[firstline=138,lastline=143,highlightlines={141-142}]{../ocaml/runtime/startup\_nat.c}{runtime/startup\_nat.c:138}
}%
\only<2>{
\listc[firstline=138,highlightlines={147,149}]{../ocaml/runtime/printexc.c}{runtime/printexc.c:138}
}%
\only<3>{
\listc[firstline=110,lastline=134,highlightlines=129]{../ocaml/runtime/printexc.c}{runtime/printexc.c:138}
}%
\only<4>{
\listc[firstline=138,highlightlines={152,154}]{../ocaml/runtime/printexc.c}{runtime/printexc.c:138}
}%
\end{column}
\end{columns}
\footnotetext[1]{\url{https://v2.ocaml.org/api/Printexc.html\#1\_Uncaughtexceptions}}
\end{frame}
}%
\end{column}
\end{columns}
\end{frame}

\begin{frame}{Executing the OCaml program}{And raising an exception without handler}
\begin{columns}
\begin{column}{0.6\textwidth}
\only<1-3>{
\begin{itemize}
\item<1-> Execution continues with OCaml code
\begin{itemize}
\item \funcname{camlDefault\_\_main\_269}
\item \funcname{camlDefault\_\_entry}
\item \funcname{caml\_program}
\end{itemize}
\item<1-> \funcname{camlDefault\_\_main\_269} raises a \typename{Failure} exception
\item<1-> \typename{Failure} is caught by \funcname{caml\_start\_program}
\begin{itemize}
\item<1-> The handler set the 2nd LSB of the exception value to \funcarg{1}
\item<3-> And then forces \funcname{caml\_start\_program} to terminate
\end{itemize}
\end{itemize}
}
\bigskip
\only<1,3>{\listocaml[highlightlines=3]{../src/default/default.ml}{default/default.ml}}%
\only<2>{\listasm[firstline=66,lastline=72,highlightlines=69]{src/ocaml/caml\_start\_program.s}{caml\_start\_program}}%
\end{column}
\begin{column}{0.4\textwidth}
\centering
\providecommand\step{4}%
% Runtime's default exception handler
%
\subsection*{Runtime's default exception handler}
\frameSubsection{pictures/The\_Architect.png}{}


\begin{frame}{Default exception handler}
\begin{columns}
\begin{column}{0.5\textwidth}
\begin{itemize}
\item \localname{domain\_state->exn\_handler} points to a trap located before \funcname{entry}!
\item What installed this very first trap there?
\end{itemize}
\bigskip
\listocaml{../src/default/default.ml}{default/default.ml}
\bigskip
\inputminted[fontsize=\footnotesize]{text}{../src/default/default.out}
\end{column}
\begin{column}{0.5\textwidth}
\centering
\only<1>{\providecommand\step{0}\input{diagrams/default/default.tex}}%
\only<2>{\providecommand\step{4}\input{diagrams/default/default.tex}}%
\end{column}
\end{columns}
\end{frame}


\begin{frame}{Startup}{How did we get there by the way?}
\begin{columns}
\begin{column}{0.45\textwidth}
\begin{itemize}
\item The entry point address of the binary is \funcname{main} (\funcname{\_start}) from the runtime
\item The program starts like a C program:
\begin{itemize}
\item \funcname{caml\_start\_program}
\item \funcname{caml\_startup\_common}
\item \funcname{caml\_startup\_exn}
\item \funcname{caml\_startup}
\item \funcname{caml\_main}
\item \funcname{main}
\end{itemize}
%\item \funcname{caml\_startup\_common}: initializes GC, domains \dots
% Also: codefrag, locale, custom opeartions, frame_descr, signals, stack
\item \funcname{caml\_start\_program} is the transition point between the C realm and the OCaml one
\end{itemize}
\bigskip
\listocaml{../src/default/default.ml}{default/default.ml}
\end{column}
\begin{column}{0.55\textwidth}
\listasm[lastline=24,highlightlines={22-24}]{src/ocaml/caml\_start\_program.s}{runtime/amd64.S:604}
\end{column}
\end{columns}
\end{frame}


\begin{frame}{Setup to jump into OCaml entry point}{\funcname{caml\_start\_program} initializes the main fiber}
\begin{columns}
\begin{column}{0.6\textwidth}
\only<1,3,5>{
\begin{itemize}
\item<1-> Stores information to allow switching from OCaml stack to the C stack
\item<3-> Constructs the default exception handler
\item<5-> Switches to the OCaml stack and calls \funcname{caml\_program}
\end{itemize}
\bigskip
\listocaml{../src/default/default.ml}{default/default.ml}
}%
\only<2>{
\listasm[firstline=22,lastline=41,highlightlines={25-27}]{src/ocaml/caml\_start\_program.s}{caml\_start\_program}
}%
\only<4>{
\mintasm[firstline=22,lastline=41,highlightlines={31-38}]{src/ocaml/caml\_start\_program.s}
\listasm[firstline=66,lastline=72]{src/ocaml/caml\_start\_program.s}{caml\_start\_program}
}%
\only<6>{
\listasm[firstline=22,lastline=41,highlightlines={39-41}]{src/ocaml/caml\_start\_program.s}{caml\_start\_program}
}%
\end{column}
\begin{column}{0.4\textwidth}
\centering
\only<1-2>{\providecommand\step{2}\input{diagrams/default/default.tex}}%
\only<3-5>{\providecommand\step{3}\input{diagrams/default/default.tex}}%
\onslide*<6->{\providecommand\step{4}\input{diagrams/default/default.tex}}%
\end{column}
\end{columns}
\end{frame}

\begin{frame}{Executing the OCaml program}{And raising an exception without handler}
\begin{columns}
\begin{column}{0.6\textwidth}
\only<1-3>{
\begin{itemize}
\item<1-> Execution continues with OCaml code
\begin{itemize}
\item \funcname{camlDefault\_\_main\_269}
\item \funcname{camlDefault\_\_entry}
\item \funcname{caml\_program}
\end{itemize}
\item<1-> \funcname{camlDefault\_\_main\_269} raises a \typename{Failure} exception
\item<1-> \typename{Failure} is caught by \funcname{caml\_start\_program}
\begin{itemize}
\item<1-> The handler set the 2nd LSB of the exception value to \funcarg{1}
\item<3-> And then forces \funcname{caml\_start\_program} to terminate
\end{itemize}
\end{itemize}
}
\bigskip
\only<1,3>{\listocaml[highlightlines=3]{../src/default/default.ml}{default/default.ml}}%
\only<2>{\listasm[firstline=66,lastline=72,highlightlines=69]{src/ocaml/caml\_start\_program.s}{caml\_start\_program}}%
\end{column}
\begin{column}{0.4\textwidth}
\centering
\providecommand\step{4}\input{diagrams/default/default.tex}
\end{column}
\end{columns}
\end{frame}
%\only<>{\listasm[firstline=47,lastline=65]{src/ocaml/caml\_start\_program.s}{caml\_start\_program}}


\begin{frame}{Back to C}{Clean up, complain and terminate}
\begin{columns}
\begin{column}{0.45\textwidth}
\begin{itemize}
\item Backtrace:
\begin{itemize}
\item \funcname{caml\_start\_program} $\rightarrow$ OCaml code
\item \funcname{caml\_startup\_common}
\item \funcname{caml\_startup\_exn}
\item \funcname{caml\_startup}
\item \funcname{caml\_main}
\item \funcname{main}
\end{itemize}
\smallskip
\item \funcname{caml\_startup} checks the return value of \funcname{caml\_startup\_exn}
\begin{itemize}
\item The return value has been specially marked by \funcname{caml\_start\_program} handler
\item Calls \funcname{caml\_fatal\_uncaught\_exception}
\end{itemize}
\item<2-> \funcname{caml\_fatal\_uncaught\_exception} either
\begin{itemize}
\item<2-> Calls the user custom uncaught exception handler, if set with \mintinline{ocaml}{Printexc.set_uncaught_exception_handler fn}\footnotemark
\item<2-> Or falls back to the default one
\item<2-> Which notifies about the uncaught exception
\end{itemize}
\item<4-> Terminates the program either with \funcname{abort} or with a non-zero exit code
\end{itemize}
\end{column}
\begin{column}{0.55\textwidth}
\only<1>{
\listc[firstline=84,lastline=89,highlightlines=88]{../ocaml/runtime/caml/mlvalues.h}{runtime/caml/mlvalues.h:84}
\listc[firstline=138,lastline=143,highlightlines={141-142}]{../ocaml/runtime/startup\_nat.c}{runtime/startup\_nat.c:138}
}%
\only<2>{
\listc[firstline=138,highlightlines={147,149}]{../ocaml/runtime/printexc.c}{runtime/printexc.c:138}
}%
\only<3>{
\listc[firstline=110,lastline=134,highlightlines=129]{../ocaml/runtime/printexc.c}{runtime/printexc.c:138}
}%
\only<4>{
\listc[firstline=138,highlightlines={152,154}]{../ocaml/runtime/printexc.c}{runtime/printexc.c:138}
}%
\end{column}
\end{columns}
\footnotetext[1]{\url{https://v2.ocaml.org/api/Printexc.html\#1\_Uncaughtexceptions}}
\end{frame}

\end{column}
\end{columns}
\end{frame}
%\only<>{\listasm[firstline=47,lastline=65]{src/ocaml/caml\_start\_program.s}{caml\_start\_program}}


\begin{frame}{Back to C}{Clean up, complain and terminate}
\begin{columns}
\begin{column}{0.45\textwidth}
\begin{itemize}
\item Backtrace:
\begin{itemize}
\item \funcname{caml\_start\_program} $\rightarrow$ OCaml code
\item \funcname{caml\_startup\_common}
\item \funcname{caml\_startup\_exn}
\item \funcname{caml\_startup}
\item \funcname{caml\_main}
\item \funcname{main}
\end{itemize}
\smallskip
\item \funcname{caml\_startup} checks the return value of \funcname{caml\_startup\_exn}
\begin{itemize}
\item The return value has been specially marked by \funcname{caml\_start\_program} handler
\item Calls \funcname{caml\_fatal\_uncaught\_exception}
\end{itemize}
\item<2-> \funcname{caml\_fatal\_uncaught\_exception} either
\begin{itemize}
\item<2-> Calls the user custom uncaught exception handler, if set with \mintinline{ocaml}{Printexc.set_uncaught_exception_handler fn}\footnotemark
\item<2-> Or falls back to the default one
\item<2-> Which notifies about the uncaught exception
\end{itemize}
\item<4-> Terminates the program either with \funcname{abort} or with a non-zero exit code
\end{itemize}
\end{column}
\begin{column}{0.55\textwidth}
\only<1>{
\listc[firstline=84,lastline=89,highlightlines=88]{../ocaml/runtime/caml/mlvalues.h}{runtime/caml/mlvalues.h:84}
\listc[firstline=138,lastline=143,highlightlines={141-142}]{../ocaml/runtime/startup\_nat.c}{runtime/startup\_nat.c:138}
}%
\only<2>{
\listc[firstline=138,highlightlines={147,149}]{../ocaml/runtime/printexc.c}{runtime/printexc.c:138}
}%
\only<3>{
\listc[firstline=110,lastline=134,highlightlines=129]{../ocaml/runtime/printexc.c}{runtime/printexc.c:138}
}%
\only<4>{
\listc[firstline=138,highlightlines={152,154}]{../ocaml/runtime/printexc.c}{runtime/printexc.c:138}
}%
\end{column}
\end{columns}
\footnotetext[1]{\url{https://v2.ocaml.org/api/Printexc.html\#1\_Uncaughtexceptions}}
\end{frame}
}%
\onslide*<6->{\providecommand\step{4}%
% Runtime's default exception handler
%
\subsection*{Runtime's default exception handler}
\frameSubsection{pictures/The\_Architect.png}{}


\begin{frame}{Default exception handler}
\begin{columns}
\begin{column}{0.5\textwidth}
\begin{itemize}
\item \localname{domain\_state->exn\_handler} points to a trap located before \funcname{entry}!
\item What installed this very first trap there?
\end{itemize}
\bigskip
\listocaml{../src/default/default.ml}{default/default.ml}
\bigskip
\inputminted[fontsize=\footnotesize]{text}{../src/default/default.out}
\end{column}
\begin{column}{0.5\textwidth}
\centering
\only<1>{\providecommand\step{0}%
% Runtime's default exception handler
%
\subsection*{Runtime's default exception handler}
\frameSubsection{pictures/The\_Architect.png}{}


\begin{frame}{Default exception handler}
\begin{columns}
\begin{column}{0.5\textwidth}
\begin{itemize}
\item \localname{domain\_state->exn\_handler} points to a trap located before \funcname{entry}!
\item What installed this very first trap there?
\end{itemize}
\bigskip
\listocaml{../src/default/default.ml}{default/default.ml}
\bigskip
\inputminted[fontsize=\footnotesize]{text}{../src/default/default.out}
\end{column}
\begin{column}{0.5\textwidth}
\centering
\only<1>{\providecommand\step{0}\input{diagrams/default/default.tex}}%
\only<2>{\providecommand\step{4}\input{diagrams/default/default.tex}}%
\end{column}
\end{columns}
\end{frame}


\begin{frame}{Startup}{How did we get there by the way?}
\begin{columns}
\begin{column}{0.45\textwidth}
\begin{itemize}
\item The entry point address of the binary is \funcname{main} (\funcname{\_start}) from the runtime
\item The program starts like a C program:
\begin{itemize}
\item \funcname{caml\_start\_program}
\item \funcname{caml\_startup\_common}
\item \funcname{caml\_startup\_exn}
\item \funcname{caml\_startup}
\item \funcname{caml\_main}
\item \funcname{main}
\end{itemize}
%\item \funcname{caml\_startup\_common}: initializes GC, domains \dots
% Also: codefrag, locale, custom opeartions, frame_descr, signals, stack
\item \funcname{caml\_start\_program} is the transition point between the C realm and the OCaml one
\end{itemize}
\bigskip
\listocaml{../src/default/default.ml}{default/default.ml}
\end{column}
\begin{column}{0.55\textwidth}
\listasm[lastline=24,highlightlines={22-24}]{src/ocaml/caml\_start\_program.s}{runtime/amd64.S:604}
\end{column}
\end{columns}
\end{frame}


\begin{frame}{Setup to jump into OCaml entry point}{\funcname{caml\_start\_program} initializes the main fiber}
\begin{columns}
\begin{column}{0.6\textwidth}
\only<1,3,5>{
\begin{itemize}
\item<1-> Stores information to allow switching from OCaml stack to the C stack
\item<3-> Constructs the default exception handler
\item<5-> Switches to the OCaml stack and calls \funcname{caml\_program}
\end{itemize}
\bigskip
\listocaml{../src/default/default.ml}{default/default.ml}
}%
\only<2>{
\listasm[firstline=22,lastline=41,highlightlines={25-27}]{src/ocaml/caml\_start\_program.s}{caml\_start\_program}
}%
\only<4>{
\mintasm[firstline=22,lastline=41,highlightlines={31-38}]{src/ocaml/caml\_start\_program.s}
\listasm[firstline=66,lastline=72]{src/ocaml/caml\_start\_program.s}{caml\_start\_program}
}%
\only<6>{
\listasm[firstline=22,lastline=41,highlightlines={39-41}]{src/ocaml/caml\_start\_program.s}{caml\_start\_program}
}%
\end{column}
\begin{column}{0.4\textwidth}
\centering
\only<1-2>{\providecommand\step{2}\input{diagrams/default/default.tex}}%
\only<3-5>{\providecommand\step{3}\input{diagrams/default/default.tex}}%
\onslide*<6->{\providecommand\step{4}\input{diagrams/default/default.tex}}%
\end{column}
\end{columns}
\end{frame}

\begin{frame}{Executing the OCaml program}{And raising an exception without handler}
\begin{columns}
\begin{column}{0.6\textwidth}
\only<1-3>{
\begin{itemize}
\item<1-> Execution continues with OCaml code
\begin{itemize}
\item \funcname{camlDefault\_\_main\_269}
\item \funcname{camlDefault\_\_entry}
\item \funcname{caml\_program}
\end{itemize}
\item<1-> \funcname{camlDefault\_\_main\_269} raises a \typename{Failure} exception
\item<1-> \typename{Failure} is caught by \funcname{caml\_start\_program}
\begin{itemize}
\item<1-> The handler set the 2nd LSB of the exception value to \funcarg{1}
\item<3-> And then forces \funcname{caml\_start\_program} to terminate
\end{itemize}
\end{itemize}
}
\bigskip
\only<1,3>{\listocaml[highlightlines=3]{../src/default/default.ml}{default/default.ml}}%
\only<2>{\listasm[firstline=66,lastline=72,highlightlines=69]{src/ocaml/caml\_start\_program.s}{caml\_start\_program}}%
\end{column}
\begin{column}{0.4\textwidth}
\centering
\providecommand\step{4}\input{diagrams/default/default.tex}
\end{column}
\end{columns}
\end{frame}
%\only<>{\listasm[firstline=47,lastline=65]{src/ocaml/caml\_start\_program.s}{caml\_start\_program}}


\begin{frame}{Back to C}{Clean up, complain and terminate}
\begin{columns}
\begin{column}{0.45\textwidth}
\begin{itemize}
\item Backtrace:
\begin{itemize}
\item \funcname{caml\_start\_program} $\rightarrow$ OCaml code
\item \funcname{caml\_startup\_common}
\item \funcname{caml\_startup\_exn}
\item \funcname{caml\_startup}
\item \funcname{caml\_main}
\item \funcname{main}
\end{itemize}
\smallskip
\item \funcname{caml\_startup} checks the return value of \funcname{caml\_startup\_exn}
\begin{itemize}
\item The return value has been specially marked by \funcname{caml\_start\_program} handler
\item Calls \funcname{caml\_fatal\_uncaught\_exception}
\end{itemize}
\item<2-> \funcname{caml\_fatal\_uncaught\_exception} either
\begin{itemize}
\item<2-> Calls the user custom uncaught exception handler, if set with \mintinline{ocaml}{Printexc.set_uncaught_exception_handler fn}\footnotemark
\item<2-> Or falls back to the default one
\item<2-> Which notifies about the uncaught exception
\end{itemize}
\item<4-> Terminates the program either with \funcname{abort} or with a non-zero exit code
\end{itemize}
\end{column}
\begin{column}{0.55\textwidth}
\only<1>{
\listc[firstline=84,lastline=89,highlightlines=88]{../ocaml/runtime/caml/mlvalues.h}{runtime/caml/mlvalues.h:84}
\listc[firstline=138,lastline=143,highlightlines={141-142}]{../ocaml/runtime/startup\_nat.c}{runtime/startup\_nat.c:138}
}%
\only<2>{
\listc[firstline=138,highlightlines={147,149}]{../ocaml/runtime/printexc.c}{runtime/printexc.c:138}
}%
\only<3>{
\listc[firstline=110,lastline=134,highlightlines=129]{../ocaml/runtime/printexc.c}{runtime/printexc.c:138}
}%
\only<4>{
\listc[firstline=138,highlightlines={152,154}]{../ocaml/runtime/printexc.c}{runtime/printexc.c:138}
}%
\end{column}
\end{columns}
\footnotetext[1]{\url{https://v2.ocaml.org/api/Printexc.html\#1\_Uncaughtexceptions}}
\end{frame}
}%
\only<2>{\providecommand\step{4}%
% Runtime's default exception handler
%
\subsection*{Runtime's default exception handler}
\frameSubsection{pictures/The\_Architect.png}{}


\begin{frame}{Default exception handler}
\begin{columns}
\begin{column}{0.5\textwidth}
\begin{itemize}
\item \localname{domain\_state->exn\_handler} points to a trap located before \funcname{entry}!
\item What installed this very first trap there?
\end{itemize}
\bigskip
\listocaml{../src/default/default.ml}{default/default.ml}
\bigskip
\inputminted[fontsize=\footnotesize]{text}{../src/default/default.out}
\end{column}
\begin{column}{0.5\textwidth}
\centering
\only<1>{\providecommand\step{0}\input{diagrams/default/default.tex}}%
\only<2>{\providecommand\step{4}\input{diagrams/default/default.tex}}%
\end{column}
\end{columns}
\end{frame}


\begin{frame}{Startup}{How did we get there by the way?}
\begin{columns}
\begin{column}{0.45\textwidth}
\begin{itemize}
\item The entry point address of the binary is \funcname{main} (\funcname{\_start}) from the runtime
\item The program starts like a C program:
\begin{itemize}
\item \funcname{caml\_start\_program}
\item \funcname{caml\_startup\_common}
\item \funcname{caml\_startup\_exn}
\item \funcname{caml\_startup}
\item \funcname{caml\_main}
\item \funcname{main}
\end{itemize}
%\item \funcname{caml\_startup\_common}: initializes GC, domains \dots
% Also: codefrag, locale, custom opeartions, frame_descr, signals, stack
\item \funcname{caml\_start\_program} is the transition point between the C realm and the OCaml one
\end{itemize}
\bigskip
\listocaml{../src/default/default.ml}{default/default.ml}
\end{column}
\begin{column}{0.55\textwidth}
\listasm[lastline=24,highlightlines={22-24}]{src/ocaml/caml\_start\_program.s}{runtime/amd64.S:604}
\end{column}
\end{columns}
\end{frame}


\begin{frame}{Setup to jump into OCaml entry point}{\funcname{caml\_start\_program} initializes the main fiber}
\begin{columns}
\begin{column}{0.6\textwidth}
\only<1,3,5>{
\begin{itemize}
\item<1-> Stores information to allow switching from OCaml stack to the C stack
\item<3-> Constructs the default exception handler
\item<5-> Switches to the OCaml stack and calls \funcname{caml\_program}
\end{itemize}
\bigskip
\listocaml{../src/default/default.ml}{default/default.ml}
}%
\only<2>{
\listasm[firstline=22,lastline=41,highlightlines={25-27}]{src/ocaml/caml\_start\_program.s}{caml\_start\_program}
}%
\only<4>{
\mintasm[firstline=22,lastline=41,highlightlines={31-38}]{src/ocaml/caml\_start\_program.s}
\listasm[firstline=66,lastline=72]{src/ocaml/caml\_start\_program.s}{caml\_start\_program}
}%
\only<6>{
\listasm[firstline=22,lastline=41,highlightlines={39-41}]{src/ocaml/caml\_start\_program.s}{caml\_start\_program}
}%
\end{column}
\begin{column}{0.4\textwidth}
\centering
\only<1-2>{\providecommand\step{2}\input{diagrams/default/default.tex}}%
\only<3-5>{\providecommand\step{3}\input{diagrams/default/default.tex}}%
\onslide*<6->{\providecommand\step{4}\input{diagrams/default/default.tex}}%
\end{column}
\end{columns}
\end{frame}

\begin{frame}{Executing the OCaml program}{And raising an exception without handler}
\begin{columns}
\begin{column}{0.6\textwidth}
\only<1-3>{
\begin{itemize}
\item<1-> Execution continues with OCaml code
\begin{itemize}
\item \funcname{camlDefault\_\_main\_269}
\item \funcname{camlDefault\_\_entry}
\item \funcname{caml\_program}
\end{itemize}
\item<1-> \funcname{camlDefault\_\_main\_269} raises a \typename{Failure} exception
\item<1-> \typename{Failure} is caught by \funcname{caml\_start\_program}
\begin{itemize}
\item<1-> The handler set the 2nd LSB of the exception value to \funcarg{1}
\item<3-> And then forces \funcname{caml\_start\_program} to terminate
\end{itemize}
\end{itemize}
}
\bigskip
\only<1,3>{\listocaml[highlightlines=3]{../src/default/default.ml}{default/default.ml}}%
\only<2>{\listasm[firstline=66,lastline=72,highlightlines=69]{src/ocaml/caml\_start\_program.s}{caml\_start\_program}}%
\end{column}
\begin{column}{0.4\textwidth}
\centering
\providecommand\step{4}\input{diagrams/default/default.tex}
\end{column}
\end{columns}
\end{frame}
%\only<>{\listasm[firstline=47,lastline=65]{src/ocaml/caml\_start\_program.s}{caml\_start\_program}}


\begin{frame}{Back to C}{Clean up, complain and terminate}
\begin{columns}
\begin{column}{0.45\textwidth}
\begin{itemize}
\item Backtrace:
\begin{itemize}
\item \funcname{caml\_start\_program} $\rightarrow$ OCaml code
\item \funcname{caml\_startup\_common}
\item \funcname{caml\_startup\_exn}
\item \funcname{caml\_startup}
\item \funcname{caml\_main}
\item \funcname{main}
\end{itemize}
\smallskip
\item \funcname{caml\_startup} checks the return value of \funcname{caml\_startup\_exn}
\begin{itemize}
\item The return value has been specially marked by \funcname{caml\_start\_program} handler
\item Calls \funcname{caml\_fatal\_uncaught\_exception}
\end{itemize}
\item<2-> \funcname{caml\_fatal\_uncaught\_exception} either
\begin{itemize}
\item<2-> Calls the user custom uncaught exception handler, if set with \mintinline{ocaml}{Printexc.set_uncaught_exception_handler fn}\footnotemark
\item<2-> Or falls back to the default one
\item<2-> Which notifies about the uncaught exception
\end{itemize}
\item<4-> Terminates the program either with \funcname{abort} or with a non-zero exit code
\end{itemize}
\end{column}
\begin{column}{0.55\textwidth}
\only<1>{
\listc[firstline=84,lastline=89,highlightlines=88]{../ocaml/runtime/caml/mlvalues.h}{runtime/caml/mlvalues.h:84}
\listc[firstline=138,lastline=143,highlightlines={141-142}]{../ocaml/runtime/startup\_nat.c}{runtime/startup\_nat.c:138}
}%
\only<2>{
\listc[firstline=138,highlightlines={147,149}]{../ocaml/runtime/printexc.c}{runtime/printexc.c:138}
}%
\only<3>{
\listc[firstline=110,lastline=134,highlightlines=129]{../ocaml/runtime/printexc.c}{runtime/printexc.c:138}
}%
\only<4>{
\listc[firstline=138,highlightlines={152,154}]{../ocaml/runtime/printexc.c}{runtime/printexc.c:138}
}%
\end{column}
\end{columns}
\footnotetext[1]{\url{https://v2.ocaml.org/api/Printexc.html\#1\_Uncaughtexceptions}}
\end{frame}
}%
\end{column}
\end{columns}
\end{frame}


\begin{frame}{Startup}{How did we get there by the way?}
\begin{columns}
\begin{column}{0.45\textwidth}
\begin{itemize}
\item The entry point address of the binary is \funcname{main} (\funcname{\_start}) from the runtime
\item The program starts like a C program:
\begin{itemize}
\item \funcname{caml\_start\_program}
\item \funcname{caml\_startup\_common}
\item \funcname{caml\_startup\_exn}
\item \funcname{caml\_startup}
\item \funcname{caml\_main}
\item \funcname{main}
\end{itemize}
%\item \funcname{caml\_startup\_common}: initializes GC, domains \dots
% Also: codefrag, locale, custom opeartions, frame_descr, signals, stack
\item \funcname{caml\_start\_program} is the transition point between the C realm and the OCaml one
\end{itemize}
\bigskip
\listocaml{../src/default/default.ml}{default/default.ml}
\end{column}
\begin{column}{0.55\textwidth}
\listasm[lastline=24,highlightlines={22-24}]{src/ocaml/caml\_start\_program.s}{runtime/amd64.S:604}
\end{column}
\end{columns}
\end{frame}


\begin{frame}{Setup to jump into OCaml entry point}{\funcname{caml\_start\_program} initializes the main fiber}
\begin{columns}
\begin{column}{0.6\textwidth}
\only<1,3,5>{
\begin{itemize}
\item<1-> Stores information to allow switching from OCaml stack to the C stack
\item<3-> Constructs the default exception handler
\item<5-> Switches to the OCaml stack and calls \funcname{caml\_program}
\end{itemize}
\bigskip
\listocaml{../src/default/default.ml}{default/default.ml}
}%
\only<2>{
\listasm[firstline=22,lastline=41,highlightlines={25-27}]{src/ocaml/caml\_start\_program.s}{caml\_start\_program}
}%
\only<4>{
\mintasm[firstline=22,lastline=41,highlightlines={31-38}]{src/ocaml/caml\_start\_program.s}
\listasm[firstline=66,lastline=72]{src/ocaml/caml\_start\_program.s}{caml\_start\_program}
}%
\only<6>{
\listasm[firstline=22,lastline=41,highlightlines={39-41}]{src/ocaml/caml\_start\_program.s}{caml\_start\_program}
}%
\end{column}
\begin{column}{0.4\textwidth}
\centering
\only<1-2>{\providecommand\step{2}%
% Runtime's default exception handler
%
\subsection*{Runtime's default exception handler}
\frameSubsection{pictures/The\_Architect.png}{}


\begin{frame}{Default exception handler}
\begin{columns}
\begin{column}{0.5\textwidth}
\begin{itemize}
\item \localname{domain\_state->exn\_handler} points to a trap located before \funcname{entry}!
\item What installed this very first trap there?
\end{itemize}
\bigskip
\listocaml{../src/default/default.ml}{default/default.ml}
\bigskip
\inputminted[fontsize=\footnotesize]{text}{../src/default/default.out}
\end{column}
\begin{column}{0.5\textwidth}
\centering
\only<1>{\providecommand\step{0}\input{diagrams/default/default.tex}}%
\only<2>{\providecommand\step{4}\input{diagrams/default/default.tex}}%
\end{column}
\end{columns}
\end{frame}


\begin{frame}{Startup}{How did we get there by the way?}
\begin{columns}
\begin{column}{0.45\textwidth}
\begin{itemize}
\item The entry point address of the binary is \funcname{main} (\funcname{\_start}) from the runtime
\item The program starts like a C program:
\begin{itemize}
\item \funcname{caml\_start\_program}
\item \funcname{caml\_startup\_common}
\item \funcname{caml\_startup\_exn}
\item \funcname{caml\_startup}
\item \funcname{caml\_main}
\item \funcname{main}
\end{itemize}
%\item \funcname{caml\_startup\_common}: initializes GC, domains \dots
% Also: codefrag, locale, custom opeartions, frame_descr, signals, stack
\item \funcname{caml\_start\_program} is the transition point between the C realm and the OCaml one
\end{itemize}
\bigskip
\listocaml{../src/default/default.ml}{default/default.ml}
\end{column}
\begin{column}{0.55\textwidth}
\listasm[lastline=24,highlightlines={22-24}]{src/ocaml/caml\_start\_program.s}{runtime/amd64.S:604}
\end{column}
\end{columns}
\end{frame}


\begin{frame}{Setup to jump into OCaml entry point}{\funcname{caml\_start\_program} initializes the main fiber}
\begin{columns}
\begin{column}{0.6\textwidth}
\only<1,3,5>{
\begin{itemize}
\item<1-> Stores information to allow switching from OCaml stack to the C stack
\item<3-> Constructs the default exception handler
\item<5-> Switches to the OCaml stack and calls \funcname{caml\_program}
\end{itemize}
\bigskip
\listocaml{../src/default/default.ml}{default/default.ml}
}%
\only<2>{
\listasm[firstline=22,lastline=41,highlightlines={25-27}]{src/ocaml/caml\_start\_program.s}{caml\_start\_program}
}%
\only<4>{
\mintasm[firstline=22,lastline=41,highlightlines={31-38}]{src/ocaml/caml\_start\_program.s}
\listasm[firstline=66,lastline=72]{src/ocaml/caml\_start\_program.s}{caml\_start\_program}
}%
\only<6>{
\listasm[firstline=22,lastline=41,highlightlines={39-41}]{src/ocaml/caml\_start\_program.s}{caml\_start\_program}
}%
\end{column}
\begin{column}{0.4\textwidth}
\centering
\only<1-2>{\providecommand\step{2}\input{diagrams/default/default.tex}}%
\only<3-5>{\providecommand\step{3}\input{diagrams/default/default.tex}}%
\onslide*<6->{\providecommand\step{4}\input{diagrams/default/default.tex}}%
\end{column}
\end{columns}
\end{frame}

\begin{frame}{Executing the OCaml program}{And raising an exception without handler}
\begin{columns}
\begin{column}{0.6\textwidth}
\only<1-3>{
\begin{itemize}
\item<1-> Execution continues with OCaml code
\begin{itemize}
\item \funcname{camlDefault\_\_main\_269}
\item \funcname{camlDefault\_\_entry}
\item \funcname{caml\_program}
\end{itemize}
\item<1-> \funcname{camlDefault\_\_main\_269} raises a \typename{Failure} exception
\item<1-> \typename{Failure} is caught by \funcname{caml\_start\_program}
\begin{itemize}
\item<1-> The handler set the 2nd LSB of the exception value to \funcarg{1}
\item<3-> And then forces \funcname{caml\_start\_program} to terminate
\end{itemize}
\end{itemize}
}
\bigskip
\only<1,3>{\listocaml[highlightlines=3]{../src/default/default.ml}{default/default.ml}}%
\only<2>{\listasm[firstline=66,lastline=72,highlightlines=69]{src/ocaml/caml\_start\_program.s}{caml\_start\_program}}%
\end{column}
\begin{column}{0.4\textwidth}
\centering
\providecommand\step{4}\input{diagrams/default/default.tex}
\end{column}
\end{columns}
\end{frame}
%\only<>{\listasm[firstline=47,lastline=65]{src/ocaml/caml\_start\_program.s}{caml\_start\_program}}


\begin{frame}{Back to C}{Clean up, complain and terminate}
\begin{columns}
\begin{column}{0.45\textwidth}
\begin{itemize}
\item Backtrace:
\begin{itemize}
\item \funcname{caml\_start\_program} $\rightarrow$ OCaml code
\item \funcname{caml\_startup\_common}
\item \funcname{caml\_startup\_exn}
\item \funcname{caml\_startup}
\item \funcname{caml\_main}
\item \funcname{main}
\end{itemize}
\smallskip
\item \funcname{caml\_startup} checks the return value of \funcname{caml\_startup\_exn}
\begin{itemize}
\item The return value has been specially marked by \funcname{caml\_start\_program} handler
\item Calls \funcname{caml\_fatal\_uncaught\_exception}
\end{itemize}
\item<2-> \funcname{caml\_fatal\_uncaught\_exception} either
\begin{itemize}
\item<2-> Calls the user custom uncaught exception handler, if set with \mintinline{ocaml}{Printexc.set_uncaught_exception_handler fn}\footnotemark
\item<2-> Or falls back to the default one
\item<2-> Which notifies about the uncaught exception
\end{itemize}
\item<4-> Terminates the program either with \funcname{abort} or with a non-zero exit code
\end{itemize}
\end{column}
\begin{column}{0.55\textwidth}
\only<1>{
\listc[firstline=84,lastline=89,highlightlines=88]{../ocaml/runtime/caml/mlvalues.h}{runtime/caml/mlvalues.h:84}
\listc[firstline=138,lastline=143,highlightlines={141-142}]{../ocaml/runtime/startup\_nat.c}{runtime/startup\_nat.c:138}
}%
\only<2>{
\listc[firstline=138,highlightlines={147,149}]{../ocaml/runtime/printexc.c}{runtime/printexc.c:138}
}%
\only<3>{
\listc[firstline=110,lastline=134,highlightlines=129]{../ocaml/runtime/printexc.c}{runtime/printexc.c:138}
}%
\only<4>{
\listc[firstline=138,highlightlines={152,154}]{../ocaml/runtime/printexc.c}{runtime/printexc.c:138}
}%
\end{column}
\end{columns}
\footnotetext[1]{\url{https://v2.ocaml.org/api/Printexc.html\#1\_Uncaughtexceptions}}
\end{frame}
}%
\only<3-5>{\providecommand\step{3}%
% Runtime's default exception handler
%
\subsection*{Runtime's default exception handler}
\frameSubsection{pictures/The\_Architect.png}{}


\begin{frame}{Default exception handler}
\begin{columns}
\begin{column}{0.5\textwidth}
\begin{itemize}
\item \localname{domain\_state->exn\_handler} points to a trap located before \funcname{entry}!
\item What installed this very first trap there?
\end{itemize}
\bigskip
\listocaml{../src/default/default.ml}{default/default.ml}
\bigskip
\inputminted[fontsize=\footnotesize]{text}{../src/default/default.out}
\end{column}
\begin{column}{0.5\textwidth}
\centering
\only<1>{\providecommand\step{0}\input{diagrams/default/default.tex}}%
\only<2>{\providecommand\step{4}\input{diagrams/default/default.tex}}%
\end{column}
\end{columns}
\end{frame}


\begin{frame}{Startup}{How did we get there by the way?}
\begin{columns}
\begin{column}{0.45\textwidth}
\begin{itemize}
\item The entry point address of the binary is \funcname{main} (\funcname{\_start}) from the runtime
\item The program starts like a C program:
\begin{itemize}
\item \funcname{caml\_start\_program}
\item \funcname{caml\_startup\_common}
\item \funcname{caml\_startup\_exn}
\item \funcname{caml\_startup}
\item \funcname{caml\_main}
\item \funcname{main}
\end{itemize}
%\item \funcname{caml\_startup\_common}: initializes GC, domains \dots
% Also: codefrag, locale, custom opeartions, frame_descr, signals, stack
\item \funcname{caml\_start\_program} is the transition point between the C realm and the OCaml one
\end{itemize}
\bigskip
\listocaml{../src/default/default.ml}{default/default.ml}
\end{column}
\begin{column}{0.55\textwidth}
\listasm[lastline=24,highlightlines={22-24}]{src/ocaml/caml\_start\_program.s}{runtime/amd64.S:604}
\end{column}
\end{columns}
\end{frame}


\begin{frame}{Setup to jump into OCaml entry point}{\funcname{caml\_start\_program} initializes the main fiber}
\begin{columns}
\begin{column}{0.6\textwidth}
\only<1,3,5>{
\begin{itemize}
\item<1-> Stores information to allow switching from OCaml stack to the C stack
\item<3-> Constructs the default exception handler
\item<5-> Switches to the OCaml stack and calls \funcname{caml\_program}
\end{itemize}
\bigskip
\listocaml{../src/default/default.ml}{default/default.ml}
}%
\only<2>{
\listasm[firstline=22,lastline=41,highlightlines={25-27}]{src/ocaml/caml\_start\_program.s}{caml\_start\_program}
}%
\only<4>{
\mintasm[firstline=22,lastline=41,highlightlines={31-38}]{src/ocaml/caml\_start\_program.s}
\listasm[firstline=66,lastline=72]{src/ocaml/caml\_start\_program.s}{caml\_start\_program}
}%
\only<6>{
\listasm[firstline=22,lastline=41,highlightlines={39-41}]{src/ocaml/caml\_start\_program.s}{caml\_start\_program}
}%
\end{column}
\begin{column}{0.4\textwidth}
\centering
\only<1-2>{\providecommand\step{2}\input{diagrams/default/default.tex}}%
\only<3-5>{\providecommand\step{3}\input{diagrams/default/default.tex}}%
\onslide*<6->{\providecommand\step{4}\input{diagrams/default/default.tex}}%
\end{column}
\end{columns}
\end{frame}

\begin{frame}{Executing the OCaml program}{And raising an exception without handler}
\begin{columns}
\begin{column}{0.6\textwidth}
\only<1-3>{
\begin{itemize}
\item<1-> Execution continues with OCaml code
\begin{itemize}
\item \funcname{camlDefault\_\_main\_269}
\item \funcname{camlDefault\_\_entry}
\item \funcname{caml\_program}
\end{itemize}
\item<1-> \funcname{camlDefault\_\_main\_269} raises a \typename{Failure} exception
\item<1-> \typename{Failure} is caught by \funcname{caml\_start\_program}
\begin{itemize}
\item<1-> The handler set the 2nd LSB of the exception value to \funcarg{1}
\item<3-> And then forces \funcname{caml\_start\_program} to terminate
\end{itemize}
\end{itemize}
}
\bigskip
\only<1,3>{\listocaml[highlightlines=3]{../src/default/default.ml}{default/default.ml}}%
\only<2>{\listasm[firstline=66,lastline=72,highlightlines=69]{src/ocaml/caml\_start\_program.s}{caml\_start\_program}}%
\end{column}
\begin{column}{0.4\textwidth}
\centering
\providecommand\step{4}\input{diagrams/default/default.tex}
\end{column}
\end{columns}
\end{frame}
%\only<>{\listasm[firstline=47,lastline=65]{src/ocaml/caml\_start\_program.s}{caml\_start\_program}}


\begin{frame}{Back to C}{Clean up, complain and terminate}
\begin{columns}
\begin{column}{0.45\textwidth}
\begin{itemize}
\item Backtrace:
\begin{itemize}
\item \funcname{caml\_start\_program} $\rightarrow$ OCaml code
\item \funcname{caml\_startup\_common}
\item \funcname{caml\_startup\_exn}
\item \funcname{caml\_startup}
\item \funcname{caml\_main}
\item \funcname{main}
\end{itemize}
\smallskip
\item \funcname{caml\_startup} checks the return value of \funcname{caml\_startup\_exn}
\begin{itemize}
\item The return value has been specially marked by \funcname{caml\_start\_program} handler
\item Calls \funcname{caml\_fatal\_uncaught\_exception}
\end{itemize}
\item<2-> \funcname{caml\_fatal\_uncaught\_exception} either
\begin{itemize}
\item<2-> Calls the user custom uncaught exception handler, if set with \mintinline{ocaml}{Printexc.set_uncaught_exception_handler fn}\footnotemark
\item<2-> Or falls back to the default one
\item<2-> Which notifies about the uncaught exception
\end{itemize}
\item<4-> Terminates the program either with \funcname{abort} or with a non-zero exit code
\end{itemize}
\end{column}
\begin{column}{0.55\textwidth}
\only<1>{
\listc[firstline=84,lastline=89,highlightlines=88]{../ocaml/runtime/caml/mlvalues.h}{runtime/caml/mlvalues.h:84}
\listc[firstline=138,lastline=143,highlightlines={141-142}]{../ocaml/runtime/startup\_nat.c}{runtime/startup\_nat.c:138}
}%
\only<2>{
\listc[firstline=138,highlightlines={147,149}]{../ocaml/runtime/printexc.c}{runtime/printexc.c:138}
}%
\only<3>{
\listc[firstline=110,lastline=134,highlightlines=129]{../ocaml/runtime/printexc.c}{runtime/printexc.c:138}
}%
\only<4>{
\listc[firstline=138,highlightlines={152,154}]{../ocaml/runtime/printexc.c}{runtime/printexc.c:138}
}%
\end{column}
\end{columns}
\footnotetext[1]{\url{https://v2.ocaml.org/api/Printexc.html\#1\_Uncaughtexceptions}}
\end{frame}
}%
\onslide*<6->{\providecommand\step{4}%
% Runtime's default exception handler
%
\subsection*{Runtime's default exception handler}
\frameSubsection{pictures/The\_Architect.png}{}


\begin{frame}{Default exception handler}
\begin{columns}
\begin{column}{0.5\textwidth}
\begin{itemize}
\item \localname{domain\_state->exn\_handler} points to a trap located before \funcname{entry}!
\item What installed this very first trap there?
\end{itemize}
\bigskip
\listocaml{../src/default/default.ml}{default/default.ml}
\bigskip
\inputminted[fontsize=\footnotesize]{text}{../src/default/default.out}
\end{column}
\begin{column}{0.5\textwidth}
\centering
\only<1>{\providecommand\step{0}\input{diagrams/default/default.tex}}%
\only<2>{\providecommand\step{4}\input{diagrams/default/default.tex}}%
\end{column}
\end{columns}
\end{frame}


\begin{frame}{Startup}{How did we get there by the way?}
\begin{columns}
\begin{column}{0.45\textwidth}
\begin{itemize}
\item The entry point address of the binary is \funcname{main} (\funcname{\_start}) from the runtime
\item The program starts like a C program:
\begin{itemize}
\item \funcname{caml\_start\_program}
\item \funcname{caml\_startup\_common}
\item \funcname{caml\_startup\_exn}
\item \funcname{caml\_startup}
\item \funcname{caml\_main}
\item \funcname{main}
\end{itemize}
%\item \funcname{caml\_startup\_common}: initializes GC, domains \dots
% Also: codefrag, locale, custom opeartions, frame_descr, signals, stack
\item \funcname{caml\_start\_program} is the transition point between the C realm and the OCaml one
\end{itemize}
\bigskip
\listocaml{../src/default/default.ml}{default/default.ml}
\end{column}
\begin{column}{0.55\textwidth}
\listasm[lastline=24,highlightlines={22-24}]{src/ocaml/caml\_start\_program.s}{runtime/amd64.S:604}
\end{column}
\end{columns}
\end{frame}


\begin{frame}{Setup to jump into OCaml entry point}{\funcname{caml\_start\_program} initializes the main fiber}
\begin{columns}
\begin{column}{0.6\textwidth}
\only<1,3,5>{
\begin{itemize}
\item<1-> Stores information to allow switching from OCaml stack to the C stack
\item<3-> Constructs the default exception handler
\item<5-> Switches to the OCaml stack and calls \funcname{caml\_program}
\end{itemize}
\bigskip
\listocaml{../src/default/default.ml}{default/default.ml}
}%
\only<2>{
\listasm[firstline=22,lastline=41,highlightlines={25-27}]{src/ocaml/caml\_start\_program.s}{caml\_start\_program}
}%
\only<4>{
\mintasm[firstline=22,lastline=41,highlightlines={31-38}]{src/ocaml/caml\_start\_program.s}
\listasm[firstline=66,lastline=72]{src/ocaml/caml\_start\_program.s}{caml\_start\_program}
}%
\only<6>{
\listasm[firstline=22,lastline=41,highlightlines={39-41}]{src/ocaml/caml\_start\_program.s}{caml\_start\_program}
}%
\end{column}
\begin{column}{0.4\textwidth}
\centering
\only<1-2>{\providecommand\step{2}\input{diagrams/default/default.tex}}%
\only<3-5>{\providecommand\step{3}\input{diagrams/default/default.tex}}%
\onslide*<6->{\providecommand\step{4}\input{diagrams/default/default.tex}}%
\end{column}
\end{columns}
\end{frame}

\begin{frame}{Executing the OCaml program}{And raising an exception without handler}
\begin{columns}
\begin{column}{0.6\textwidth}
\only<1-3>{
\begin{itemize}
\item<1-> Execution continues with OCaml code
\begin{itemize}
\item \funcname{camlDefault\_\_main\_269}
\item \funcname{camlDefault\_\_entry}
\item \funcname{caml\_program}
\end{itemize}
\item<1-> \funcname{camlDefault\_\_main\_269} raises a \typename{Failure} exception
\item<1-> \typename{Failure} is caught by \funcname{caml\_start\_program}
\begin{itemize}
\item<1-> The handler set the 2nd LSB of the exception value to \funcarg{1}
\item<3-> And then forces \funcname{caml\_start\_program} to terminate
\end{itemize}
\end{itemize}
}
\bigskip
\only<1,3>{\listocaml[highlightlines=3]{../src/default/default.ml}{default/default.ml}}%
\only<2>{\listasm[firstline=66,lastline=72,highlightlines=69]{src/ocaml/caml\_start\_program.s}{caml\_start\_program}}%
\end{column}
\begin{column}{0.4\textwidth}
\centering
\providecommand\step{4}\input{diagrams/default/default.tex}
\end{column}
\end{columns}
\end{frame}
%\only<>{\listasm[firstline=47,lastline=65]{src/ocaml/caml\_start\_program.s}{caml\_start\_program}}


\begin{frame}{Back to C}{Clean up, complain and terminate}
\begin{columns}
\begin{column}{0.45\textwidth}
\begin{itemize}
\item Backtrace:
\begin{itemize}
\item \funcname{caml\_start\_program} $\rightarrow$ OCaml code
\item \funcname{caml\_startup\_common}
\item \funcname{caml\_startup\_exn}
\item \funcname{caml\_startup}
\item \funcname{caml\_main}
\item \funcname{main}
\end{itemize}
\smallskip
\item \funcname{caml\_startup} checks the return value of \funcname{caml\_startup\_exn}
\begin{itemize}
\item The return value has been specially marked by \funcname{caml\_start\_program} handler
\item Calls \funcname{caml\_fatal\_uncaught\_exception}
\end{itemize}
\item<2-> \funcname{caml\_fatal\_uncaught\_exception} either
\begin{itemize}
\item<2-> Calls the user custom uncaught exception handler, if set with \mintinline{ocaml}{Printexc.set_uncaught_exception_handler fn}\footnotemark
\item<2-> Or falls back to the default one
\item<2-> Which notifies about the uncaught exception
\end{itemize}
\item<4-> Terminates the program either with \funcname{abort} or with a non-zero exit code
\end{itemize}
\end{column}
\begin{column}{0.55\textwidth}
\only<1>{
\listc[firstline=84,lastline=89,highlightlines=88]{../ocaml/runtime/caml/mlvalues.h}{runtime/caml/mlvalues.h:84}
\listc[firstline=138,lastline=143,highlightlines={141-142}]{../ocaml/runtime/startup\_nat.c}{runtime/startup\_nat.c:138}
}%
\only<2>{
\listc[firstline=138,highlightlines={147,149}]{../ocaml/runtime/printexc.c}{runtime/printexc.c:138}
}%
\only<3>{
\listc[firstline=110,lastline=134,highlightlines=129]{../ocaml/runtime/printexc.c}{runtime/printexc.c:138}
}%
\only<4>{
\listc[firstline=138,highlightlines={152,154}]{../ocaml/runtime/printexc.c}{runtime/printexc.c:138}
}%
\end{column}
\end{columns}
\footnotetext[1]{\url{https://v2.ocaml.org/api/Printexc.html\#1\_Uncaughtexceptions}}
\end{frame}
}%
\end{column}
\end{columns}
\end{frame}

\begin{frame}{Executing the OCaml program}{And raising an exception without handler}
\begin{columns}
\begin{column}{0.6\textwidth}
\only<1-3>{
\begin{itemize}
\item<1-> Execution continues with OCaml code
\begin{itemize}
\item \funcname{camlDefault\_\_main\_269}
\item \funcname{camlDefault\_\_entry}
\item \funcname{caml\_program}
\end{itemize}
\item<1-> \funcname{camlDefault\_\_main\_269} raises a \typename{Failure} exception
\item<1-> \typename{Failure} is caught by \funcname{caml\_start\_program}
\begin{itemize}
\item<1-> The handler set the 2nd LSB of the exception value to \funcarg{1}
\item<3-> And then forces \funcname{caml\_start\_program} to terminate
\end{itemize}
\end{itemize}
}
\bigskip
\only<1,3>{\listocaml[highlightlines=3]{../src/default/default.ml}{default/default.ml}}%
\only<2>{\listasm[firstline=66,lastline=72,highlightlines=69]{src/ocaml/caml\_start\_program.s}{caml\_start\_program}}%
\end{column}
\begin{column}{0.4\textwidth}
\centering
\providecommand\step{4}%
% Runtime's default exception handler
%
\subsection*{Runtime's default exception handler}
\frameSubsection{pictures/The\_Architect.png}{}


\begin{frame}{Default exception handler}
\begin{columns}
\begin{column}{0.5\textwidth}
\begin{itemize}
\item \localname{domain\_state->exn\_handler} points to a trap located before \funcname{entry}!
\item What installed this very first trap there?
\end{itemize}
\bigskip
\listocaml{../src/default/default.ml}{default/default.ml}
\bigskip
\inputminted[fontsize=\footnotesize]{text}{../src/default/default.out}
\end{column}
\begin{column}{0.5\textwidth}
\centering
\only<1>{\providecommand\step{0}\input{diagrams/default/default.tex}}%
\only<2>{\providecommand\step{4}\input{diagrams/default/default.tex}}%
\end{column}
\end{columns}
\end{frame}


\begin{frame}{Startup}{How did we get there by the way?}
\begin{columns}
\begin{column}{0.45\textwidth}
\begin{itemize}
\item The entry point address of the binary is \funcname{main} (\funcname{\_start}) from the runtime
\item The program starts like a C program:
\begin{itemize}
\item \funcname{caml\_start\_program}
\item \funcname{caml\_startup\_common}
\item \funcname{caml\_startup\_exn}
\item \funcname{caml\_startup}
\item \funcname{caml\_main}
\item \funcname{main}
\end{itemize}
%\item \funcname{caml\_startup\_common}: initializes GC, domains \dots
% Also: codefrag, locale, custom opeartions, frame_descr, signals, stack
\item \funcname{caml\_start\_program} is the transition point between the C realm and the OCaml one
\end{itemize}
\bigskip
\listocaml{../src/default/default.ml}{default/default.ml}
\end{column}
\begin{column}{0.55\textwidth}
\listasm[lastline=24,highlightlines={22-24}]{src/ocaml/caml\_start\_program.s}{runtime/amd64.S:604}
\end{column}
\end{columns}
\end{frame}


\begin{frame}{Setup to jump into OCaml entry point}{\funcname{caml\_start\_program} initializes the main fiber}
\begin{columns}
\begin{column}{0.6\textwidth}
\only<1,3,5>{
\begin{itemize}
\item<1-> Stores information to allow switching from OCaml stack to the C stack
\item<3-> Constructs the default exception handler
\item<5-> Switches to the OCaml stack and calls \funcname{caml\_program}
\end{itemize}
\bigskip
\listocaml{../src/default/default.ml}{default/default.ml}
}%
\only<2>{
\listasm[firstline=22,lastline=41,highlightlines={25-27}]{src/ocaml/caml\_start\_program.s}{caml\_start\_program}
}%
\only<4>{
\mintasm[firstline=22,lastline=41,highlightlines={31-38}]{src/ocaml/caml\_start\_program.s}
\listasm[firstline=66,lastline=72]{src/ocaml/caml\_start\_program.s}{caml\_start\_program}
}%
\only<6>{
\listasm[firstline=22,lastline=41,highlightlines={39-41}]{src/ocaml/caml\_start\_program.s}{caml\_start\_program}
}%
\end{column}
\begin{column}{0.4\textwidth}
\centering
\only<1-2>{\providecommand\step{2}\input{diagrams/default/default.tex}}%
\only<3-5>{\providecommand\step{3}\input{diagrams/default/default.tex}}%
\onslide*<6->{\providecommand\step{4}\input{diagrams/default/default.tex}}%
\end{column}
\end{columns}
\end{frame}

\begin{frame}{Executing the OCaml program}{And raising an exception without handler}
\begin{columns}
\begin{column}{0.6\textwidth}
\only<1-3>{
\begin{itemize}
\item<1-> Execution continues with OCaml code
\begin{itemize}
\item \funcname{camlDefault\_\_main\_269}
\item \funcname{camlDefault\_\_entry}
\item \funcname{caml\_program}
\end{itemize}
\item<1-> \funcname{camlDefault\_\_main\_269} raises a \typename{Failure} exception
\item<1-> \typename{Failure} is caught by \funcname{caml\_start\_program}
\begin{itemize}
\item<1-> The handler set the 2nd LSB of the exception value to \funcarg{1}
\item<3-> And then forces \funcname{caml\_start\_program} to terminate
\end{itemize}
\end{itemize}
}
\bigskip
\only<1,3>{\listocaml[highlightlines=3]{../src/default/default.ml}{default/default.ml}}%
\only<2>{\listasm[firstline=66,lastline=72,highlightlines=69]{src/ocaml/caml\_start\_program.s}{caml\_start\_program}}%
\end{column}
\begin{column}{0.4\textwidth}
\centering
\providecommand\step{4}\input{diagrams/default/default.tex}
\end{column}
\end{columns}
\end{frame}
%\only<>{\listasm[firstline=47,lastline=65]{src/ocaml/caml\_start\_program.s}{caml\_start\_program}}


\begin{frame}{Back to C}{Clean up, complain and terminate}
\begin{columns}
\begin{column}{0.45\textwidth}
\begin{itemize}
\item Backtrace:
\begin{itemize}
\item \funcname{caml\_start\_program} $\rightarrow$ OCaml code
\item \funcname{caml\_startup\_common}
\item \funcname{caml\_startup\_exn}
\item \funcname{caml\_startup}
\item \funcname{caml\_main}
\item \funcname{main}
\end{itemize}
\smallskip
\item \funcname{caml\_startup} checks the return value of \funcname{caml\_startup\_exn}
\begin{itemize}
\item The return value has been specially marked by \funcname{caml\_start\_program} handler
\item Calls \funcname{caml\_fatal\_uncaught\_exception}
\end{itemize}
\item<2-> \funcname{caml\_fatal\_uncaught\_exception} either
\begin{itemize}
\item<2-> Calls the user custom uncaught exception handler, if set with \mintinline{ocaml}{Printexc.set_uncaught_exception_handler fn}\footnotemark
\item<2-> Or falls back to the default one
\item<2-> Which notifies about the uncaught exception
\end{itemize}
\item<4-> Terminates the program either with \funcname{abort} or with a non-zero exit code
\end{itemize}
\end{column}
\begin{column}{0.55\textwidth}
\only<1>{
\listc[firstline=84,lastline=89,highlightlines=88]{../ocaml/runtime/caml/mlvalues.h}{runtime/caml/mlvalues.h:84}
\listc[firstline=138,lastline=143,highlightlines={141-142}]{../ocaml/runtime/startup\_nat.c}{runtime/startup\_nat.c:138}
}%
\only<2>{
\listc[firstline=138,highlightlines={147,149}]{../ocaml/runtime/printexc.c}{runtime/printexc.c:138}
}%
\only<3>{
\listc[firstline=110,lastline=134,highlightlines=129]{../ocaml/runtime/printexc.c}{runtime/printexc.c:138}
}%
\only<4>{
\listc[firstline=138,highlightlines={152,154}]{../ocaml/runtime/printexc.c}{runtime/printexc.c:138}
}%
\end{column}
\end{columns}
\footnotetext[1]{\url{https://v2.ocaml.org/api/Printexc.html\#1\_Uncaughtexceptions}}
\end{frame}

\end{column}
\end{columns}
\end{frame}
%\only<>{\listasm[firstline=47,lastline=65]{src/ocaml/caml\_start\_program.s}{caml\_start\_program}}


\begin{frame}{Back to C}{Clean up, complain and terminate}
\begin{columns}
\begin{column}{0.45\textwidth}
\begin{itemize}
\item Backtrace:
\begin{itemize}
\item \funcname{caml\_start\_program} $\rightarrow$ OCaml code
\item \funcname{caml\_startup\_common}
\item \funcname{caml\_startup\_exn}
\item \funcname{caml\_startup}
\item \funcname{caml\_main}
\item \funcname{main}
\end{itemize}
\smallskip
\item \funcname{caml\_startup} checks the return value of \funcname{caml\_startup\_exn}
\begin{itemize}
\item The return value has been specially marked by \funcname{caml\_start\_program} handler
\item Calls \funcname{caml\_fatal\_uncaught\_exception}
\end{itemize}
\item<2-> \funcname{caml\_fatal\_uncaught\_exception} either
\begin{itemize}
\item<2-> Calls the user custom uncaught exception handler, if set with \mintinline{ocaml}{Printexc.set_uncaught_exception_handler fn}\footnotemark
\item<2-> Or falls back to the default one
\item<2-> Which notifies about the uncaught exception
\end{itemize}
\item<4-> Terminates the program either with \funcname{abort} or with a non-zero exit code
\end{itemize}
\end{column}
\begin{column}{0.55\textwidth}
\only<1>{
\listc[firstline=84,lastline=89,highlightlines=88]{../ocaml/runtime/caml/mlvalues.h}{runtime/caml/mlvalues.h:84}
\listc[firstline=138,lastline=143,highlightlines={141-142}]{../ocaml/runtime/startup\_nat.c}{runtime/startup\_nat.c:138}
}%
\only<2>{
\listc[firstline=138,highlightlines={147,149}]{../ocaml/runtime/printexc.c}{runtime/printexc.c:138}
}%
\only<3>{
\listc[firstline=110,lastline=134,highlightlines=129]{../ocaml/runtime/printexc.c}{runtime/printexc.c:138}
}%
\only<4>{
\listc[firstline=138,highlightlines={152,154}]{../ocaml/runtime/printexc.c}{runtime/printexc.c:138}
}%
\end{column}
\end{columns}
\footnotetext[1]{\url{https://v2.ocaml.org/api/Printexc.html\#1\_Uncaughtexceptions}}
\end{frame}
}%
\end{column}
\end{columns}
\end{frame}

\begin{frame}{Executing the OCaml program}{And raising an exception without handler}
\begin{columns}
\begin{column}{0.6\textwidth}
\only<1-3>{
\begin{itemize}
\item<1-> Execution continues with OCaml code
\begin{itemize}
\item \funcname{camlDefault\_\_main\_269}
\item \funcname{camlDefault\_\_entry}
\item \funcname{caml\_program}
\end{itemize}
\item<1-> \funcname{camlDefault\_\_main\_269} raises a \typename{Failure} exception
\item<1-> \typename{Failure} is caught by \funcname{caml\_start\_program}
\begin{itemize}
\item<1-> The handler set the 2nd LSB of the exception value to \funcarg{1}
\item<3-> And then forces \funcname{caml\_start\_program} to terminate
\end{itemize}
\end{itemize}
}
\bigskip
\only<1,3>{\listocaml[highlightlines=3]{../src/default/default.ml}{default/default.ml}}%
\only<2>{\listasm[firstline=66,lastline=72,highlightlines=69]{src/ocaml/caml\_start\_program.s}{caml\_start\_program}}%
\end{column}
\begin{column}{0.4\textwidth}
\centering
\providecommand\step{4}%
% Runtime's default exception handler
%
\subsection*{Runtime's default exception handler}
\frameSubsection{pictures/The\_Architect.png}{}


\begin{frame}{Default exception handler}
\begin{columns}
\begin{column}{0.5\textwidth}
\begin{itemize}
\item \localname{domain\_state->exn\_handler} points to a trap located before \funcname{entry}!
\item What installed this very first trap there?
\end{itemize}
\bigskip
\listocaml{../src/default/default.ml}{default/default.ml}
\bigskip
\inputminted[fontsize=\footnotesize]{text}{../src/default/default.out}
\end{column}
\begin{column}{0.5\textwidth}
\centering
\only<1>{\providecommand\step{0}%
% Runtime's default exception handler
%
\subsection*{Runtime's default exception handler}
\frameSubsection{pictures/The\_Architect.png}{}


\begin{frame}{Default exception handler}
\begin{columns}
\begin{column}{0.5\textwidth}
\begin{itemize}
\item \localname{domain\_state->exn\_handler} points to a trap located before \funcname{entry}!
\item What installed this very first trap there?
\end{itemize}
\bigskip
\listocaml{../src/default/default.ml}{default/default.ml}
\bigskip
\inputminted[fontsize=\footnotesize]{text}{../src/default/default.out}
\end{column}
\begin{column}{0.5\textwidth}
\centering
\only<1>{\providecommand\step{0}\input{diagrams/default/default.tex}}%
\only<2>{\providecommand\step{4}\input{diagrams/default/default.tex}}%
\end{column}
\end{columns}
\end{frame}


\begin{frame}{Startup}{How did we get there by the way?}
\begin{columns}
\begin{column}{0.45\textwidth}
\begin{itemize}
\item The entry point address of the binary is \funcname{main} (\funcname{\_start}) from the runtime
\item The program starts like a C program:
\begin{itemize}
\item \funcname{caml\_start\_program}
\item \funcname{caml\_startup\_common}
\item \funcname{caml\_startup\_exn}
\item \funcname{caml\_startup}
\item \funcname{caml\_main}
\item \funcname{main}
\end{itemize}
%\item \funcname{caml\_startup\_common}: initializes GC, domains \dots
% Also: codefrag, locale, custom opeartions, frame_descr, signals, stack
\item \funcname{caml\_start\_program} is the transition point between the C realm and the OCaml one
\end{itemize}
\bigskip
\listocaml{../src/default/default.ml}{default/default.ml}
\end{column}
\begin{column}{0.55\textwidth}
\listasm[lastline=24,highlightlines={22-24}]{src/ocaml/caml\_start\_program.s}{runtime/amd64.S:604}
\end{column}
\end{columns}
\end{frame}


\begin{frame}{Setup to jump into OCaml entry point}{\funcname{caml\_start\_program} initializes the main fiber}
\begin{columns}
\begin{column}{0.6\textwidth}
\only<1,3,5>{
\begin{itemize}
\item<1-> Stores information to allow switching from OCaml stack to the C stack
\item<3-> Constructs the default exception handler
\item<5-> Switches to the OCaml stack and calls \funcname{caml\_program}
\end{itemize}
\bigskip
\listocaml{../src/default/default.ml}{default/default.ml}
}%
\only<2>{
\listasm[firstline=22,lastline=41,highlightlines={25-27}]{src/ocaml/caml\_start\_program.s}{caml\_start\_program}
}%
\only<4>{
\mintasm[firstline=22,lastline=41,highlightlines={31-38}]{src/ocaml/caml\_start\_program.s}
\listasm[firstline=66,lastline=72]{src/ocaml/caml\_start\_program.s}{caml\_start\_program}
}%
\only<6>{
\listasm[firstline=22,lastline=41,highlightlines={39-41}]{src/ocaml/caml\_start\_program.s}{caml\_start\_program}
}%
\end{column}
\begin{column}{0.4\textwidth}
\centering
\only<1-2>{\providecommand\step{2}\input{diagrams/default/default.tex}}%
\only<3-5>{\providecommand\step{3}\input{diagrams/default/default.tex}}%
\onslide*<6->{\providecommand\step{4}\input{diagrams/default/default.tex}}%
\end{column}
\end{columns}
\end{frame}

\begin{frame}{Executing the OCaml program}{And raising an exception without handler}
\begin{columns}
\begin{column}{0.6\textwidth}
\only<1-3>{
\begin{itemize}
\item<1-> Execution continues with OCaml code
\begin{itemize}
\item \funcname{camlDefault\_\_main\_269}
\item \funcname{camlDefault\_\_entry}
\item \funcname{caml\_program}
\end{itemize}
\item<1-> \funcname{camlDefault\_\_main\_269} raises a \typename{Failure} exception
\item<1-> \typename{Failure} is caught by \funcname{caml\_start\_program}
\begin{itemize}
\item<1-> The handler set the 2nd LSB of the exception value to \funcarg{1}
\item<3-> And then forces \funcname{caml\_start\_program} to terminate
\end{itemize}
\end{itemize}
}
\bigskip
\only<1,3>{\listocaml[highlightlines=3]{../src/default/default.ml}{default/default.ml}}%
\only<2>{\listasm[firstline=66,lastline=72,highlightlines=69]{src/ocaml/caml\_start\_program.s}{caml\_start\_program}}%
\end{column}
\begin{column}{0.4\textwidth}
\centering
\providecommand\step{4}\input{diagrams/default/default.tex}
\end{column}
\end{columns}
\end{frame}
%\only<>{\listasm[firstline=47,lastline=65]{src/ocaml/caml\_start\_program.s}{caml\_start\_program}}


\begin{frame}{Back to C}{Clean up, complain and terminate}
\begin{columns}
\begin{column}{0.45\textwidth}
\begin{itemize}
\item Backtrace:
\begin{itemize}
\item \funcname{caml\_start\_program} $\rightarrow$ OCaml code
\item \funcname{caml\_startup\_common}
\item \funcname{caml\_startup\_exn}
\item \funcname{caml\_startup}
\item \funcname{caml\_main}
\item \funcname{main}
\end{itemize}
\smallskip
\item \funcname{caml\_startup} checks the return value of \funcname{caml\_startup\_exn}
\begin{itemize}
\item The return value has been specially marked by \funcname{caml\_start\_program} handler
\item Calls \funcname{caml\_fatal\_uncaught\_exception}
\end{itemize}
\item<2-> \funcname{caml\_fatal\_uncaught\_exception} either
\begin{itemize}
\item<2-> Calls the user custom uncaught exception handler, if set with \mintinline{ocaml}{Printexc.set_uncaught_exception_handler fn}\footnotemark
\item<2-> Or falls back to the default one
\item<2-> Which notifies about the uncaught exception
\end{itemize}
\item<4-> Terminates the program either with \funcname{abort} or with a non-zero exit code
\end{itemize}
\end{column}
\begin{column}{0.55\textwidth}
\only<1>{
\listc[firstline=84,lastline=89,highlightlines=88]{../ocaml/runtime/caml/mlvalues.h}{runtime/caml/mlvalues.h:84}
\listc[firstline=138,lastline=143,highlightlines={141-142}]{../ocaml/runtime/startup\_nat.c}{runtime/startup\_nat.c:138}
}%
\only<2>{
\listc[firstline=138,highlightlines={147,149}]{../ocaml/runtime/printexc.c}{runtime/printexc.c:138}
}%
\only<3>{
\listc[firstline=110,lastline=134,highlightlines=129]{../ocaml/runtime/printexc.c}{runtime/printexc.c:138}
}%
\only<4>{
\listc[firstline=138,highlightlines={152,154}]{../ocaml/runtime/printexc.c}{runtime/printexc.c:138}
}%
\end{column}
\end{columns}
\footnotetext[1]{\url{https://v2.ocaml.org/api/Printexc.html\#1\_Uncaughtexceptions}}
\end{frame}
}%
\only<2>{\providecommand\step{4}%
% Runtime's default exception handler
%
\subsection*{Runtime's default exception handler}
\frameSubsection{pictures/The\_Architect.png}{}


\begin{frame}{Default exception handler}
\begin{columns}
\begin{column}{0.5\textwidth}
\begin{itemize}
\item \localname{domain\_state->exn\_handler} points to a trap located before \funcname{entry}!
\item What installed this very first trap there?
\end{itemize}
\bigskip
\listocaml{../src/default/default.ml}{default/default.ml}
\bigskip
\inputminted[fontsize=\footnotesize]{text}{../src/default/default.out}
\end{column}
\begin{column}{0.5\textwidth}
\centering
\only<1>{\providecommand\step{0}\input{diagrams/default/default.tex}}%
\only<2>{\providecommand\step{4}\input{diagrams/default/default.tex}}%
\end{column}
\end{columns}
\end{frame}


\begin{frame}{Startup}{How did we get there by the way?}
\begin{columns}
\begin{column}{0.45\textwidth}
\begin{itemize}
\item The entry point address of the binary is \funcname{main} (\funcname{\_start}) from the runtime
\item The program starts like a C program:
\begin{itemize}
\item \funcname{caml\_start\_program}
\item \funcname{caml\_startup\_common}
\item \funcname{caml\_startup\_exn}
\item \funcname{caml\_startup}
\item \funcname{caml\_main}
\item \funcname{main}
\end{itemize}
%\item \funcname{caml\_startup\_common}: initializes GC, domains \dots
% Also: codefrag, locale, custom opeartions, frame_descr, signals, stack
\item \funcname{caml\_start\_program} is the transition point between the C realm and the OCaml one
\end{itemize}
\bigskip
\listocaml{../src/default/default.ml}{default/default.ml}
\end{column}
\begin{column}{0.55\textwidth}
\listasm[lastline=24,highlightlines={22-24}]{src/ocaml/caml\_start\_program.s}{runtime/amd64.S:604}
\end{column}
\end{columns}
\end{frame}


\begin{frame}{Setup to jump into OCaml entry point}{\funcname{caml\_start\_program} initializes the main fiber}
\begin{columns}
\begin{column}{0.6\textwidth}
\only<1,3,5>{
\begin{itemize}
\item<1-> Stores information to allow switching from OCaml stack to the C stack
\item<3-> Constructs the default exception handler
\item<5-> Switches to the OCaml stack and calls \funcname{caml\_program}
\end{itemize}
\bigskip
\listocaml{../src/default/default.ml}{default/default.ml}
}%
\only<2>{
\listasm[firstline=22,lastline=41,highlightlines={25-27}]{src/ocaml/caml\_start\_program.s}{caml\_start\_program}
}%
\only<4>{
\mintasm[firstline=22,lastline=41,highlightlines={31-38}]{src/ocaml/caml\_start\_program.s}
\listasm[firstline=66,lastline=72]{src/ocaml/caml\_start\_program.s}{caml\_start\_program}
}%
\only<6>{
\listasm[firstline=22,lastline=41,highlightlines={39-41}]{src/ocaml/caml\_start\_program.s}{caml\_start\_program}
}%
\end{column}
\begin{column}{0.4\textwidth}
\centering
\only<1-2>{\providecommand\step{2}\input{diagrams/default/default.tex}}%
\only<3-5>{\providecommand\step{3}\input{diagrams/default/default.tex}}%
\onslide*<6->{\providecommand\step{4}\input{diagrams/default/default.tex}}%
\end{column}
\end{columns}
\end{frame}

\begin{frame}{Executing the OCaml program}{And raising an exception without handler}
\begin{columns}
\begin{column}{0.6\textwidth}
\only<1-3>{
\begin{itemize}
\item<1-> Execution continues with OCaml code
\begin{itemize}
\item \funcname{camlDefault\_\_main\_269}
\item \funcname{camlDefault\_\_entry}
\item \funcname{caml\_program}
\end{itemize}
\item<1-> \funcname{camlDefault\_\_main\_269} raises a \typename{Failure} exception
\item<1-> \typename{Failure} is caught by \funcname{caml\_start\_program}
\begin{itemize}
\item<1-> The handler set the 2nd LSB of the exception value to \funcarg{1}
\item<3-> And then forces \funcname{caml\_start\_program} to terminate
\end{itemize}
\end{itemize}
}
\bigskip
\only<1,3>{\listocaml[highlightlines=3]{../src/default/default.ml}{default/default.ml}}%
\only<2>{\listasm[firstline=66,lastline=72,highlightlines=69]{src/ocaml/caml\_start\_program.s}{caml\_start\_program}}%
\end{column}
\begin{column}{0.4\textwidth}
\centering
\providecommand\step{4}\input{diagrams/default/default.tex}
\end{column}
\end{columns}
\end{frame}
%\only<>{\listasm[firstline=47,lastline=65]{src/ocaml/caml\_start\_program.s}{caml\_start\_program}}


\begin{frame}{Back to C}{Clean up, complain and terminate}
\begin{columns}
\begin{column}{0.45\textwidth}
\begin{itemize}
\item Backtrace:
\begin{itemize}
\item \funcname{caml\_start\_program} $\rightarrow$ OCaml code
\item \funcname{caml\_startup\_common}
\item \funcname{caml\_startup\_exn}
\item \funcname{caml\_startup}
\item \funcname{caml\_main}
\item \funcname{main}
\end{itemize}
\smallskip
\item \funcname{caml\_startup} checks the return value of \funcname{caml\_startup\_exn}
\begin{itemize}
\item The return value has been specially marked by \funcname{caml\_start\_program} handler
\item Calls \funcname{caml\_fatal\_uncaught\_exception}
\end{itemize}
\item<2-> \funcname{caml\_fatal\_uncaught\_exception} either
\begin{itemize}
\item<2-> Calls the user custom uncaught exception handler, if set with \mintinline{ocaml}{Printexc.set_uncaught_exception_handler fn}\footnotemark
\item<2-> Or falls back to the default one
\item<2-> Which notifies about the uncaught exception
\end{itemize}
\item<4-> Terminates the program either with \funcname{abort} or with a non-zero exit code
\end{itemize}
\end{column}
\begin{column}{0.55\textwidth}
\only<1>{
\listc[firstline=84,lastline=89,highlightlines=88]{../ocaml/runtime/caml/mlvalues.h}{runtime/caml/mlvalues.h:84}
\listc[firstline=138,lastline=143,highlightlines={141-142}]{../ocaml/runtime/startup\_nat.c}{runtime/startup\_nat.c:138}
}%
\only<2>{
\listc[firstline=138,highlightlines={147,149}]{../ocaml/runtime/printexc.c}{runtime/printexc.c:138}
}%
\only<3>{
\listc[firstline=110,lastline=134,highlightlines=129]{../ocaml/runtime/printexc.c}{runtime/printexc.c:138}
}%
\only<4>{
\listc[firstline=138,highlightlines={152,154}]{../ocaml/runtime/printexc.c}{runtime/printexc.c:138}
}%
\end{column}
\end{columns}
\footnotetext[1]{\url{https://v2.ocaml.org/api/Printexc.html\#1\_Uncaughtexceptions}}
\end{frame}
}%
\end{column}
\end{columns}
\end{frame}


\begin{frame}{Startup}{How did we get there by the way?}
\begin{columns}
\begin{column}{0.45\textwidth}
\begin{itemize}
\item The entry point address of the binary is \funcname{main} (\funcname{\_start}) from the runtime
\item The program starts like a C program:
\begin{itemize}
\item \funcname{caml\_start\_program}
\item \funcname{caml\_startup\_common}
\item \funcname{caml\_startup\_exn}
\item \funcname{caml\_startup}
\item \funcname{caml\_main}
\item \funcname{main}
\end{itemize}
%\item \funcname{caml\_startup\_common}: initializes GC, domains \dots
% Also: codefrag, locale, custom opeartions, frame_descr, signals, stack
\item \funcname{caml\_start\_program} is the transition point between the C realm and the OCaml one
\end{itemize}
\bigskip
\listocaml{../src/default/default.ml}{default/default.ml}
\end{column}
\begin{column}{0.55\textwidth}
\listasm[lastline=24,highlightlines={22-24}]{src/ocaml/caml\_start\_program.s}{runtime/amd64.S:604}
\end{column}
\end{columns}
\end{frame}


\begin{frame}{Setup to jump into OCaml entry point}{\funcname{caml\_start\_program} initializes the main fiber}
\begin{columns}
\begin{column}{0.6\textwidth}
\only<1,3,5>{
\begin{itemize}
\item<1-> Stores information to allow switching from OCaml stack to the C stack
\item<3-> Constructs the default exception handler
\item<5-> Switches to the OCaml stack and calls \funcname{caml\_program}
\end{itemize}
\bigskip
\listocaml{../src/default/default.ml}{default/default.ml}
}%
\only<2>{
\listasm[firstline=22,lastline=41,highlightlines={25-27}]{src/ocaml/caml\_start\_program.s}{caml\_start\_program}
}%
\only<4>{
\mintasm[firstline=22,lastline=41,highlightlines={31-38}]{src/ocaml/caml\_start\_program.s}
\listasm[firstline=66,lastline=72]{src/ocaml/caml\_start\_program.s}{caml\_start\_program}
}%
\only<6>{
\listasm[firstline=22,lastline=41,highlightlines={39-41}]{src/ocaml/caml\_start\_program.s}{caml\_start\_program}
}%
\end{column}
\begin{column}{0.4\textwidth}
\centering
\only<1-2>{\providecommand\step{2}%
% Runtime's default exception handler
%
\subsection*{Runtime's default exception handler}
\frameSubsection{pictures/The\_Architect.png}{}


\begin{frame}{Default exception handler}
\begin{columns}
\begin{column}{0.5\textwidth}
\begin{itemize}
\item \localname{domain\_state->exn\_handler} points to a trap located before \funcname{entry}!
\item What installed this very first trap there?
\end{itemize}
\bigskip
\listocaml{../src/default/default.ml}{default/default.ml}
\bigskip
\inputminted[fontsize=\footnotesize]{text}{../src/default/default.out}
\end{column}
\begin{column}{0.5\textwidth}
\centering
\only<1>{\providecommand\step{0}\input{diagrams/default/default.tex}}%
\only<2>{\providecommand\step{4}\input{diagrams/default/default.tex}}%
\end{column}
\end{columns}
\end{frame}


\begin{frame}{Startup}{How did we get there by the way?}
\begin{columns}
\begin{column}{0.45\textwidth}
\begin{itemize}
\item The entry point address of the binary is \funcname{main} (\funcname{\_start}) from the runtime
\item The program starts like a C program:
\begin{itemize}
\item \funcname{caml\_start\_program}
\item \funcname{caml\_startup\_common}
\item \funcname{caml\_startup\_exn}
\item \funcname{caml\_startup}
\item \funcname{caml\_main}
\item \funcname{main}
\end{itemize}
%\item \funcname{caml\_startup\_common}: initializes GC, domains \dots
% Also: codefrag, locale, custom opeartions, frame_descr, signals, stack
\item \funcname{caml\_start\_program} is the transition point between the C realm and the OCaml one
\end{itemize}
\bigskip
\listocaml{../src/default/default.ml}{default/default.ml}
\end{column}
\begin{column}{0.55\textwidth}
\listasm[lastline=24,highlightlines={22-24}]{src/ocaml/caml\_start\_program.s}{runtime/amd64.S:604}
\end{column}
\end{columns}
\end{frame}


\begin{frame}{Setup to jump into OCaml entry point}{\funcname{caml\_start\_program} initializes the main fiber}
\begin{columns}
\begin{column}{0.6\textwidth}
\only<1,3,5>{
\begin{itemize}
\item<1-> Stores information to allow switching from OCaml stack to the C stack
\item<3-> Constructs the default exception handler
\item<5-> Switches to the OCaml stack and calls \funcname{caml\_program}
\end{itemize}
\bigskip
\listocaml{../src/default/default.ml}{default/default.ml}
}%
\only<2>{
\listasm[firstline=22,lastline=41,highlightlines={25-27}]{src/ocaml/caml\_start\_program.s}{caml\_start\_program}
}%
\only<4>{
\mintasm[firstline=22,lastline=41,highlightlines={31-38}]{src/ocaml/caml\_start\_program.s}
\listasm[firstline=66,lastline=72]{src/ocaml/caml\_start\_program.s}{caml\_start\_program}
}%
\only<6>{
\listasm[firstline=22,lastline=41,highlightlines={39-41}]{src/ocaml/caml\_start\_program.s}{caml\_start\_program}
}%
\end{column}
\begin{column}{0.4\textwidth}
\centering
\only<1-2>{\providecommand\step{2}\input{diagrams/default/default.tex}}%
\only<3-5>{\providecommand\step{3}\input{diagrams/default/default.tex}}%
\onslide*<6->{\providecommand\step{4}\input{diagrams/default/default.tex}}%
\end{column}
\end{columns}
\end{frame}

\begin{frame}{Executing the OCaml program}{And raising an exception without handler}
\begin{columns}
\begin{column}{0.6\textwidth}
\only<1-3>{
\begin{itemize}
\item<1-> Execution continues with OCaml code
\begin{itemize}
\item \funcname{camlDefault\_\_main\_269}
\item \funcname{camlDefault\_\_entry}
\item \funcname{caml\_program}
\end{itemize}
\item<1-> \funcname{camlDefault\_\_main\_269} raises a \typename{Failure} exception
\item<1-> \typename{Failure} is caught by \funcname{caml\_start\_program}
\begin{itemize}
\item<1-> The handler set the 2nd LSB of the exception value to \funcarg{1}
\item<3-> And then forces \funcname{caml\_start\_program} to terminate
\end{itemize}
\end{itemize}
}
\bigskip
\only<1,3>{\listocaml[highlightlines=3]{../src/default/default.ml}{default/default.ml}}%
\only<2>{\listasm[firstline=66,lastline=72,highlightlines=69]{src/ocaml/caml\_start\_program.s}{caml\_start\_program}}%
\end{column}
\begin{column}{0.4\textwidth}
\centering
\providecommand\step{4}\input{diagrams/default/default.tex}
\end{column}
\end{columns}
\end{frame}
%\only<>{\listasm[firstline=47,lastline=65]{src/ocaml/caml\_start\_program.s}{caml\_start\_program}}


\begin{frame}{Back to C}{Clean up, complain and terminate}
\begin{columns}
\begin{column}{0.45\textwidth}
\begin{itemize}
\item Backtrace:
\begin{itemize}
\item \funcname{caml\_start\_program} $\rightarrow$ OCaml code
\item \funcname{caml\_startup\_common}
\item \funcname{caml\_startup\_exn}
\item \funcname{caml\_startup}
\item \funcname{caml\_main}
\item \funcname{main}
\end{itemize}
\smallskip
\item \funcname{caml\_startup} checks the return value of \funcname{caml\_startup\_exn}
\begin{itemize}
\item The return value has been specially marked by \funcname{caml\_start\_program} handler
\item Calls \funcname{caml\_fatal\_uncaught\_exception}
\end{itemize}
\item<2-> \funcname{caml\_fatal\_uncaught\_exception} either
\begin{itemize}
\item<2-> Calls the user custom uncaught exception handler, if set with \mintinline{ocaml}{Printexc.set_uncaught_exception_handler fn}\footnotemark
\item<2-> Or falls back to the default one
\item<2-> Which notifies about the uncaught exception
\end{itemize}
\item<4-> Terminates the program either with \funcname{abort} or with a non-zero exit code
\end{itemize}
\end{column}
\begin{column}{0.55\textwidth}
\only<1>{
\listc[firstline=84,lastline=89,highlightlines=88]{../ocaml/runtime/caml/mlvalues.h}{runtime/caml/mlvalues.h:84}
\listc[firstline=138,lastline=143,highlightlines={141-142}]{../ocaml/runtime/startup\_nat.c}{runtime/startup\_nat.c:138}
}%
\only<2>{
\listc[firstline=138,highlightlines={147,149}]{../ocaml/runtime/printexc.c}{runtime/printexc.c:138}
}%
\only<3>{
\listc[firstline=110,lastline=134,highlightlines=129]{../ocaml/runtime/printexc.c}{runtime/printexc.c:138}
}%
\only<4>{
\listc[firstline=138,highlightlines={152,154}]{../ocaml/runtime/printexc.c}{runtime/printexc.c:138}
}%
\end{column}
\end{columns}
\footnotetext[1]{\url{https://v2.ocaml.org/api/Printexc.html\#1\_Uncaughtexceptions}}
\end{frame}
}%
\only<3-5>{\providecommand\step{3}%
% Runtime's default exception handler
%
\subsection*{Runtime's default exception handler}
\frameSubsection{pictures/The\_Architect.png}{}


\begin{frame}{Default exception handler}
\begin{columns}
\begin{column}{0.5\textwidth}
\begin{itemize}
\item \localname{domain\_state->exn\_handler} points to a trap located before \funcname{entry}!
\item What installed this very first trap there?
\end{itemize}
\bigskip
\listocaml{../src/default/default.ml}{default/default.ml}
\bigskip
\inputminted[fontsize=\footnotesize]{text}{../src/default/default.out}
\end{column}
\begin{column}{0.5\textwidth}
\centering
\only<1>{\providecommand\step{0}\input{diagrams/default/default.tex}}%
\only<2>{\providecommand\step{4}\input{diagrams/default/default.tex}}%
\end{column}
\end{columns}
\end{frame}


\begin{frame}{Startup}{How did we get there by the way?}
\begin{columns}
\begin{column}{0.45\textwidth}
\begin{itemize}
\item The entry point address of the binary is \funcname{main} (\funcname{\_start}) from the runtime
\item The program starts like a C program:
\begin{itemize}
\item \funcname{caml\_start\_program}
\item \funcname{caml\_startup\_common}
\item \funcname{caml\_startup\_exn}
\item \funcname{caml\_startup}
\item \funcname{caml\_main}
\item \funcname{main}
\end{itemize}
%\item \funcname{caml\_startup\_common}: initializes GC, domains \dots
% Also: codefrag, locale, custom opeartions, frame_descr, signals, stack
\item \funcname{caml\_start\_program} is the transition point between the C realm and the OCaml one
\end{itemize}
\bigskip
\listocaml{../src/default/default.ml}{default/default.ml}
\end{column}
\begin{column}{0.55\textwidth}
\listasm[lastline=24,highlightlines={22-24}]{src/ocaml/caml\_start\_program.s}{runtime/amd64.S:604}
\end{column}
\end{columns}
\end{frame}


\begin{frame}{Setup to jump into OCaml entry point}{\funcname{caml\_start\_program} initializes the main fiber}
\begin{columns}
\begin{column}{0.6\textwidth}
\only<1,3,5>{
\begin{itemize}
\item<1-> Stores information to allow switching from OCaml stack to the C stack
\item<3-> Constructs the default exception handler
\item<5-> Switches to the OCaml stack and calls \funcname{caml\_program}
\end{itemize}
\bigskip
\listocaml{../src/default/default.ml}{default/default.ml}
}%
\only<2>{
\listasm[firstline=22,lastline=41,highlightlines={25-27}]{src/ocaml/caml\_start\_program.s}{caml\_start\_program}
}%
\only<4>{
\mintasm[firstline=22,lastline=41,highlightlines={31-38}]{src/ocaml/caml\_start\_program.s}
\listasm[firstline=66,lastline=72]{src/ocaml/caml\_start\_program.s}{caml\_start\_program}
}%
\only<6>{
\listasm[firstline=22,lastline=41,highlightlines={39-41}]{src/ocaml/caml\_start\_program.s}{caml\_start\_program}
}%
\end{column}
\begin{column}{0.4\textwidth}
\centering
\only<1-2>{\providecommand\step{2}\input{diagrams/default/default.tex}}%
\only<3-5>{\providecommand\step{3}\input{diagrams/default/default.tex}}%
\onslide*<6->{\providecommand\step{4}\input{diagrams/default/default.tex}}%
\end{column}
\end{columns}
\end{frame}

\begin{frame}{Executing the OCaml program}{And raising an exception without handler}
\begin{columns}
\begin{column}{0.6\textwidth}
\only<1-3>{
\begin{itemize}
\item<1-> Execution continues with OCaml code
\begin{itemize}
\item \funcname{camlDefault\_\_main\_269}
\item \funcname{camlDefault\_\_entry}
\item \funcname{caml\_program}
\end{itemize}
\item<1-> \funcname{camlDefault\_\_main\_269} raises a \typename{Failure} exception
\item<1-> \typename{Failure} is caught by \funcname{caml\_start\_program}
\begin{itemize}
\item<1-> The handler set the 2nd LSB of the exception value to \funcarg{1}
\item<3-> And then forces \funcname{caml\_start\_program} to terminate
\end{itemize}
\end{itemize}
}
\bigskip
\only<1,3>{\listocaml[highlightlines=3]{../src/default/default.ml}{default/default.ml}}%
\only<2>{\listasm[firstline=66,lastline=72,highlightlines=69]{src/ocaml/caml\_start\_program.s}{caml\_start\_program}}%
\end{column}
\begin{column}{0.4\textwidth}
\centering
\providecommand\step{4}\input{diagrams/default/default.tex}
\end{column}
\end{columns}
\end{frame}
%\only<>{\listasm[firstline=47,lastline=65]{src/ocaml/caml\_start\_program.s}{caml\_start\_program}}


\begin{frame}{Back to C}{Clean up, complain and terminate}
\begin{columns}
\begin{column}{0.45\textwidth}
\begin{itemize}
\item Backtrace:
\begin{itemize}
\item \funcname{caml\_start\_program} $\rightarrow$ OCaml code
\item \funcname{caml\_startup\_common}
\item \funcname{caml\_startup\_exn}
\item \funcname{caml\_startup}
\item \funcname{caml\_main}
\item \funcname{main}
\end{itemize}
\smallskip
\item \funcname{caml\_startup} checks the return value of \funcname{caml\_startup\_exn}
\begin{itemize}
\item The return value has been specially marked by \funcname{caml\_start\_program} handler
\item Calls \funcname{caml\_fatal\_uncaught\_exception}
\end{itemize}
\item<2-> \funcname{caml\_fatal\_uncaught\_exception} either
\begin{itemize}
\item<2-> Calls the user custom uncaught exception handler, if set with \mintinline{ocaml}{Printexc.set_uncaught_exception_handler fn}\footnotemark
\item<2-> Or falls back to the default one
\item<2-> Which notifies about the uncaught exception
\end{itemize}
\item<4-> Terminates the program either with \funcname{abort} or with a non-zero exit code
\end{itemize}
\end{column}
\begin{column}{0.55\textwidth}
\only<1>{
\listc[firstline=84,lastline=89,highlightlines=88]{../ocaml/runtime/caml/mlvalues.h}{runtime/caml/mlvalues.h:84}
\listc[firstline=138,lastline=143,highlightlines={141-142}]{../ocaml/runtime/startup\_nat.c}{runtime/startup\_nat.c:138}
}%
\only<2>{
\listc[firstline=138,highlightlines={147,149}]{../ocaml/runtime/printexc.c}{runtime/printexc.c:138}
}%
\only<3>{
\listc[firstline=110,lastline=134,highlightlines=129]{../ocaml/runtime/printexc.c}{runtime/printexc.c:138}
}%
\only<4>{
\listc[firstline=138,highlightlines={152,154}]{../ocaml/runtime/printexc.c}{runtime/printexc.c:138}
}%
\end{column}
\end{columns}
\footnotetext[1]{\url{https://v2.ocaml.org/api/Printexc.html\#1\_Uncaughtexceptions}}
\end{frame}
}%
\onslide*<6->{\providecommand\step{4}%
% Runtime's default exception handler
%
\subsection*{Runtime's default exception handler}
\frameSubsection{pictures/The\_Architect.png}{}


\begin{frame}{Default exception handler}
\begin{columns}
\begin{column}{0.5\textwidth}
\begin{itemize}
\item \localname{domain\_state->exn\_handler} points to a trap located before \funcname{entry}!
\item What installed this very first trap there?
\end{itemize}
\bigskip
\listocaml{../src/default/default.ml}{default/default.ml}
\bigskip
\inputminted[fontsize=\footnotesize]{text}{../src/default/default.out}
\end{column}
\begin{column}{0.5\textwidth}
\centering
\only<1>{\providecommand\step{0}\input{diagrams/default/default.tex}}%
\only<2>{\providecommand\step{4}\input{diagrams/default/default.tex}}%
\end{column}
\end{columns}
\end{frame}


\begin{frame}{Startup}{How did we get there by the way?}
\begin{columns}
\begin{column}{0.45\textwidth}
\begin{itemize}
\item The entry point address of the binary is \funcname{main} (\funcname{\_start}) from the runtime
\item The program starts like a C program:
\begin{itemize}
\item \funcname{caml\_start\_program}
\item \funcname{caml\_startup\_common}
\item \funcname{caml\_startup\_exn}
\item \funcname{caml\_startup}
\item \funcname{caml\_main}
\item \funcname{main}
\end{itemize}
%\item \funcname{caml\_startup\_common}: initializes GC, domains \dots
% Also: codefrag, locale, custom opeartions, frame_descr, signals, stack
\item \funcname{caml\_start\_program} is the transition point between the C realm and the OCaml one
\end{itemize}
\bigskip
\listocaml{../src/default/default.ml}{default/default.ml}
\end{column}
\begin{column}{0.55\textwidth}
\listasm[lastline=24,highlightlines={22-24}]{src/ocaml/caml\_start\_program.s}{runtime/amd64.S:604}
\end{column}
\end{columns}
\end{frame}


\begin{frame}{Setup to jump into OCaml entry point}{\funcname{caml\_start\_program} initializes the main fiber}
\begin{columns}
\begin{column}{0.6\textwidth}
\only<1,3,5>{
\begin{itemize}
\item<1-> Stores information to allow switching from OCaml stack to the C stack
\item<3-> Constructs the default exception handler
\item<5-> Switches to the OCaml stack and calls \funcname{caml\_program}
\end{itemize}
\bigskip
\listocaml{../src/default/default.ml}{default/default.ml}
}%
\only<2>{
\listasm[firstline=22,lastline=41,highlightlines={25-27}]{src/ocaml/caml\_start\_program.s}{caml\_start\_program}
}%
\only<4>{
\mintasm[firstline=22,lastline=41,highlightlines={31-38}]{src/ocaml/caml\_start\_program.s}
\listasm[firstline=66,lastline=72]{src/ocaml/caml\_start\_program.s}{caml\_start\_program}
}%
\only<6>{
\listasm[firstline=22,lastline=41,highlightlines={39-41}]{src/ocaml/caml\_start\_program.s}{caml\_start\_program}
}%
\end{column}
\begin{column}{0.4\textwidth}
\centering
\only<1-2>{\providecommand\step{2}\input{diagrams/default/default.tex}}%
\only<3-5>{\providecommand\step{3}\input{diagrams/default/default.tex}}%
\onslide*<6->{\providecommand\step{4}\input{diagrams/default/default.tex}}%
\end{column}
\end{columns}
\end{frame}

\begin{frame}{Executing the OCaml program}{And raising an exception without handler}
\begin{columns}
\begin{column}{0.6\textwidth}
\only<1-3>{
\begin{itemize}
\item<1-> Execution continues with OCaml code
\begin{itemize}
\item \funcname{camlDefault\_\_main\_269}
\item \funcname{camlDefault\_\_entry}
\item \funcname{caml\_program}
\end{itemize}
\item<1-> \funcname{camlDefault\_\_main\_269} raises a \typename{Failure} exception
\item<1-> \typename{Failure} is caught by \funcname{caml\_start\_program}
\begin{itemize}
\item<1-> The handler set the 2nd LSB of the exception value to \funcarg{1}
\item<3-> And then forces \funcname{caml\_start\_program} to terminate
\end{itemize}
\end{itemize}
}
\bigskip
\only<1,3>{\listocaml[highlightlines=3]{../src/default/default.ml}{default/default.ml}}%
\only<2>{\listasm[firstline=66,lastline=72,highlightlines=69]{src/ocaml/caml\_start\_program.s}{caml\_start\_program}}%
\end{column}
\begin{column}{0.4\textwidth}
\centering
\providecommand\step{4}\input{diagrams/default/default.tex}
\end{column}
\end{columns}
\end{frame}
%\only<>{\listasm[firstline=47,lastline=65]{src/ocaml/caml\_start\_program.s}{caml\_start\_program}}


\begin{frame}{Back to C}{Clean up, complain and terminate}
\begin{columns}
\begin{column}{0.45\textwidth}
\begin{itemize}
\item Backtrace:
\begin{itemize}
\item \funcname{caml\_start\_program} $\rightarrow$ OCaml code
\item \funcname{caml\_startup\_common}
\item \funcname{caml\_startup\_exn}
\item \funcname{caml\_startup}
\item \funcname{caml\_main}
\item \funcname{main}
\end{itemize}
\smallskip
\item \funcname{caml\_startup} checks the return value of \funcname{caml\_startup\_exn}
\begin{itemize}
\item The return value has been specially marked by \funcname{caml\_start\_program} handler
\item Calls \funcname{caml\_fatal\_uncaught\_exception}
\end{itemize}
\item<2-> \funcname{caml\_fatal\_uncaught\_exception} either
\begin{itemize}
\item<2-> Calls the user custom uncaught exception handler, if set with \mintinline{ocaml}{Printexc.set_uncaught_exception_handler fn}\footnotemark
\item<2-> Or falls back to the default one
\item<2-> Which notifies about the uncaught exception
\end{itemize}
\item<4-> Terminates the program either with \funcname{abort} or with a non-zero exit code
\end{itemize}
\end{column}
\begin{column}{0.55\textwidth}
\only<1>{
\listc[firstline=84,lastline=89,highlightlines=88]{../ocaml/runtime/caml/mlvalues.h}{runtime/caml/mlvalues.h:84}
\listc[firstline=138,lastline=143,highlightlines={141-142}]{../ocaml/runtime/startup\_nat.c}{runtime/startup\_nat.c:138}
}%
\only<2>{
\listc[firstline=138,highlightlines={147,149}]{../ocaml/runtime/printexc.c}{runtime/printexc.c:138}
}%
\only<3>{
\listc[firstline=110,lastline=134,highlightlines=129]{../ocaml/runtime/printexc.c}{runtime/printexc.c:138}
}%
\only<4>{
\listc[firstline=138,highlightlines={152,154}]{../ocaml/runtime/printexc.c}{runtime/printexc.c:138}
}%
\end{column}
\end{columns}
\footnotetext[1]{\url{https://v2.ocaml.org/api/Printexc.html\#1\_Uncaughtexceptions}}
\end{frame}
}%
\end{column}
\end{columns}
\end{frame}

\begin{frame}{Executing the OCaml program}{And raising an exception without handler}
\begin{columns}
\begin{column}{0.6\textwidth}
\only<1-3>{
\begin{itemize}
\item<1-> Execution continues with OCaml code
\begin{itemize}
\item \funcname{camlDefault\_\_main\_269}
\item \funcname{camlDefault\_\_entry}
\item \funcname{caml\_program}
\end{itemize}
\item<1-> \funcname{camlDefault\_\_main\_269} raises a \typename{Failure} exception
\item<1-> \typename{Failure} is caught by \funcname{caml\_start\_program}
\begin{itemize}
\item<1-> The handler set the 2nd LSB of the exception value to \funcarg{1}
\item<3-> And then forces \funcname{caml\_start\_program} to terminate
\end{itemize}
\end{itemize}
}
\bigskip
\only<1,3>{\listocaml[highlightlines=3]{../src/default/default.ml}{default/default.ml}}%
\only<2>{\listasm[firstline=66,lastline=72,highlightlines=69]{src/ocaml/caml\_start\_program.s}{caml\_start\_program}}%
\end{column}
\begin{column}{0.4\textwidth}
\centering
\providecommand\step{4}%
% Runtime's default exception handler
%
\subsection*{Runtime's default exception handler}
\frameSubsection{pictures/The\_Architect.png}{}


\begin{frame}{Default exception handler}
\begin{columns}
\begin{column}{0.5\textwidth}
\begin{itemize}
\item \localname{domain\_state->exn\_handler} points to a trap located before \funcname{entry}!
\item What installed this very first trap there?
\end{itemize}
\bigskip
\listocaml{../src/default/default.ml}{default/default.ml}
\bigskip
\inputminted[fontsize=\footnotesize]{text}{../src/default/default.out}
\end{column}
\begin{column}{0.5\textwidth}
\centering
\only<1>{\providecommand\step{0}\input{diagrams/default/default.tex}}%
\only<2>{\providecommand\step{4}\input{diagrams/default/default.tex}}%
\end{column}
\end{columns}
\end{frame}


\begin{frame}{Startup}{How did we get there by the way?}
\begin{columns}
\begin{column}{0.45\textwidth}
\begin{itemize}
\item The entry point address of the binary is \funcname{main} (\funcname{\_start}) from the runtime
\item The program starts like a C program:
\begin{itemize}
\item \funcname{caml\_start\_program}
\item \funcname{caml\_startup\_common}
\item \funcname{caml\_startup\_exn}
\item \funcname{caml\_startup}
\item \funcname{caml\_main}
\item \funcname{main}
\end{itemize}
%\item \funcname{caml\_startup\_common}: initializes GC, domains \dots
% Also: codefrag, locale, custom opeartions, frame_descr, signals, stack
\item \funcname{caml\_start\_program} is the transition point between the C realm and the OCaml one
\end{itemize}
\bigskip
\listocaml{../src/default/default.ml}{default/default.ml}
\end{column}
\begin{column}{0.55\textwidth}
\listasm[lastline=24,highlightlines={22-24}]{src/ocaml/caml\_start\_program.s}{runtime/amd64.S:604}
\end{column}
\end{columns}
\end{frame}


\begin{frame}{Setup to jump into OCaml entry point}{\funcname{caml\_start\_program} initializes the main fiber}
\begin{columns}
\begin{column}{0.6\textwidth}
\only<1,3,5>{
\begin{itemize}
\item<1-> Stores information to allow switching from OCaml stack to the C stack
\item<3-> Constructs the default exception handler
\item<5-> Switches to the OCaml stack and calls \funcname{caml\_program}
\end{itemize}
\bigskip
\listocaml{../src/default/default.ml}{default/default.ml}
}%
\only<2>{
\listasm[firstline=22,lastline=41,highlightlines={25-27}]{src/ocaml/caml\_start\_program.s}{caml\_start\_program}
}%
\only<4>{
\mintasm[firstline=22,lastline=41,highlightlines={31-38}]{src/ocaml/caml\_start\_program.s}
\listasm[firstline=66,lastline=72]{src/ocaml/caml\_start\_program.s}{caml\_start\_program}
}%
\only<6>{
\listasm[firstline=22,lastline=41,highlightlines={39-41}]{src/ocaml/caml\_start\_program.s}{caml\_start\_program}
}%
\end{column}
\begin{column}{0.4\textwidth}
\centering
\only<1-2>{\providecommand\step{2}\input{diagrams/default/default.tex}}%
\only<3-5>{\providecommand\step{3}\input{diagrams/default/default.tex}}%
\onslide*<6->{\providecommand\step{4}\input{diagrams/default/default.tex}}%
\end{column}
\end{columns}
\end{frame}

\begin{frame}{Executing the OCaml program}{And raising an exception without handler}
\begin{columns}
\begin{column}{0.6\textwidth}
\only<1-3>{
\begin{itemize}
\item<1-> Execution continues with OCaml code
\begin{itemize}
\item \funcname{camlDefault\_\_main\_269}
\item \funcname{camlDefault\_\_entry}
\item \funcname{caml\_program}
\end{itemize}
\item<1-> \funcname{camlDefault\_\_main\_269} raises a \typename{Failure} exception
\item<1-> \typename{Failure} is caught by \funcname{caml\_start\_program}
\begin{itemize}
\item<1-> The handler set the 2nd LSB of the exception value to \funcarg{1}
\item<3-> And then forces \funcname{caml\_start\_program} to terminate
\end{itemize}
\end{itemize}
}
\bigskip
\only<1,3>{\listocaml[highlightlines=3]{../src/default/default.ml}{default/default.ml}}%
\only<2>{\listasm[firstline=66,lastline=72,highlightlines=69]{src/ocaml/caml\_start\_program.s}{caml\_start\_program}}%
\end{column}
\begin{column}{0.4\textwidth}
\centering
\providecommand\step{4}\input{diagrams/default/default.tex}
\end{column}
\end{columns}
\end{frame}
%\only<>{\listasm[firstline=47,lastline=65]{src/ocaml/caml\_start\_program.s}{caml\_start\_program}}


\begin{frame}{Back to C}{Clean up, complain and terminate}
\begin{columns}
\begin{column}{0.45\textwidth}
\begin{itemize}
\item Backtrace:
\begin{itemize}
\item \funcname{caml\_start\_program} $\rightarrow$ OCaml code
\item \funcname{caml\_startup\_common}
\item \funcname{caml\_startup\_exn}
\item \funcname{caml\_startup}
\item \funcname{caml\_main}
\item \funcname{main}
\end{itemize}
\smallskip
\item \funcname{caml\_startup} checks the return value of \funcname{caml\_startup\_exn}
\begin{itemize}
\item The return value has been specially marked by \funcname{caml\_start\_program} handler
\item Calls \funcname{caml\_fatal\_uncaught\_exception}
\end{itemize}
\item<2-> \funcname{caml\_fatal\_uncaught\_exception} either
\begin{itemize}
\item<2-> Calls the user custom uncaught exception handler, if set with \mintinline{ocaml}{Printexc.set_uncaught_exception_handler fn}\footnotemark
\item<2-> Or falls back to the default one
\item<2-> Which notifies about the uncaught exception
\end{itemize}
\item<4-> Terminates the program either with \funcname{abort} or with a non-zero exit code
\end{itemize}
\end{column}
\begin{column}{0.55\textwidth}
\only<1>{
\listc[firstline=84,lastline=89,highlightlines=88]{../ocaml/runtime/caml/mlvalues.h}{runtime/caml/mlvalues.h:84}
\listc[firstline=138,lastline=143,highlightlines={141-142}]{../ocaml/runtime/startup\_nat.c}{runtime/startup\_nat.c:138}
}%
\only<2>{
\listc[firstline=138,highlightlines={147,149}]{../ocaml/runtime/printexc.c}{runtime/printexc.c:138}
}%
\only<3>{
\listc[firstline=110,lastline=134,highlightlines=129]{../ocaml/runtime/printexc.c}{runtime/printexc.c:138}
}%
\only<4>{
\listc[firstline=138,highlightlines={152,154}]{../ocaml/runtime/printexc.c}{runtime/printexc.c:138}
}%
\end{column}
\end{columns}
\footnotetext[1]{\url{https://v2.ocaml.org/api/Printexc.html\#1\_Uncaughtexceptions}}
\end{frame}

\end{column}
\end{columns}
\end{frame}
%\only<>{\listasm[firstline=47,lastline=65]{src/ocaml/caml\_start\_program.s}{caml\_start\_program}}


\begin{frame}{Back to C}{Clean up, complain and terminate}
\begin{columns}
\begin{column}{0.45\textwidth}
\begin{itemize}
\item Backtrace:
\begin{itemize}
\item \funcname{caml\_start\_program} $\rightarrow$ OCaml code
\item \funcname{caml\_startup\_common}
\item \funcname{caml\_startup\_exn}
\item \funcname{caml\_startup}
\item \funcname{caml\_main}
\item \funcname{main}
\end{itemize}
\smallskip
\item \funcname{caml\_startup} checks the return value of \funcname{caml\_startup\_exn}
\begin{itemize}
\item The return value has been specially marked by \funcname{caml\_start\_program} handler
\item Calls \funcname{caml\_fatal\_uncaught\_exception}
\end{itemize}
\item<2-> \funcname{caml\_fatal\_uncaught\_exception} either
\begin{itemize}
\item<2-> Calls the user custom uncaught exception handler, if set with \mintinline{ocaml}{Printexc.set_uncaught_exception_handler fn}\footnotemark
\item<2-> Or falls back to the default one
\item<2-> Which notifies about the uncaught exception
\end{itemize}
\item<4-> Terminates the program either with \funcname{abort} or with a non-zero exit code
\end{itemize}
\end{column}
\begin{column}{0.55\textwidth}
\only<1>{
\listc[firstline=84,lastline=89,highlightlines=88]{../ocaml/runtime/caml/mlvalues.h}{runtime/caml/mlvalues.h:84}
\listc[firstline=138,lastline=143,highlightlines={141-142}]{../ocaml/runtime/startup\_nat.c}{runtime/startup\_nat.c:138}
}%
\only<2>{
\listc[firstline=138,highlightlines={147,149}]{../ocaml/runtime/printexc.c}{runtime/printexc.c:138}
}%
\only<3>{
\listc[firstline=110,lastline=134,highlightlines=129]{../ocaml/runtime/printexc.c}{runtime/printexc.c:138}
}%
\only<4>{
\listc[firstline=138,highlightlines={152,154}]{../ocaml/runtime/printexc.c}{runtime/printexc.c:138}
}%
\end{column}
\end{columns}
\footnotetext[1]{\url{https://v2.ocaml.org/api/Printexc.html\#1\_Uncaughtexceptions}}
\end{frame}

\end{column}
\end{columns}
\end{frame}
%\only<>{\listasm[firstline=47,lastline=65]{src/ocaml/caml\_start\_program.s}{caml\_start\_program}}


\begin{frame}{Back to C}{Clean up, complain and terminate}
\begin{columns}
\begin{column}{0.45\textwidth}
\begin{itemize}
\item Backtrace:
\begin{itemize}
\item \funcname{caml\_start\_program} $\rightarrow$ OCaml code
\item \funcname{caml\_startup\_common}
\item \funcname{caml\_startup\_exn}
\item \funcname{caml\_startup}
\item \funcname{caml\_main}
\item \funcname{main}
\end{itemize}
\smallskip
\item \funcname{caml\_startup} checks the return value of \funcname{caml\_startup\_exn}
\begin{itemize}
\item The return value has been specially marked by \funcname{caml\_start\_program} handler
\item Calls \funcname{caml\_fatal\_uncaught\_exception}
\end{itemize}
\item<2-> \funcname{caml\_fatal\_uncaught\_exception} either
\begin{itemize}
\item<2-> Calls the user custom uncaught exception handler, if set with \mintinline{ocaml}{Printexc.set_uncaught_exception_handler fn}\footnotemark
\item<2-> Or falls back to the default one
\item<2-> Which notifies about the uncaught exception
\end{itemize}
\item<4-> Terminates the program either with \funcname{abort} or with a non-zero exit code
\end{itemize}
\end{column}
\begin{column}{0.55\textwidth}
\only<1>{
\listc[firstline=84,lastline=89,highlightlines=88]{../ocaml/runtime/caml/mlvalues.h}{runtime/caml/mlvalues.h:84}
\listc[firstline=138,lastline=143,highlightlines={141-142}]{../ocaml/runtime/startup\_nat.c}{runtime/startup\_nat.c:138}
}%
\only<2>{
\listc[firstline=138,highlightlines={147,149}]{../ocaml/runtime/printexc.c}{runtime/printexc.c:138}
}%
\only<3>{
\listc[firstline=110,lastline=134,highlightlines=129]{../ocaml/runtime/printexc.c}{runtime/printexc.c:138}
}%
\only<4>{
\listc[firstline=138,highlightlines={152,154}]{../ocaml/runtime/printexc.c}{runtime/printexc.c:138}
}%
\end{column}
\end{columns}
\footnotetext[1]{\url{https://v2.ocaml.org/api/Printexc.html\#1\_Uncaughtexceptions}}
\end{frame}
}%
\end{column}
\end{columns}
\end{frame}

\begin{frame}{Executing the OCaml program}{And raising an exception without handler}
\begin{columns}
\begin{column}{0.6\textwidth}
\only<1-3>{
\begin{itemize}
\item<1-> Execution continues with OCaml code
\begin{itemize}
\item \funcname{camlDefault\_\_main\_269}
\item \funcname{camlDefault\_\_entry}
\item \funcname{caml\_program}
\end{itemize}
\item<1-> \funcname{camlDefault\_\_main\_269} raises a \typename{Failure} exception
\item<1-> \typename{Failure} is caught by \funcname{caml\_start\_program}
\begin{itemize}
\item<1-> The handler set the 2nd LSB of the exception value to \funcarg{1}
\item<3-> And then forces \funcname{caml\_start\_program} to terminate
\end{itemize}
\end{itemize}
}
\bigskip
\only<1,3>{\listocaml[highlightlines=3]{../src/default/default.ml}{default/default.ml}}%
\only<2>{\listasm[firstline=66,lastline=72,highlightlines=69]{src/ocaml/caml\_start\_program.s}{caml\_start\_program}}%
\end{column}
\begin{column}{0.4\textwidth}
\centering
\providecommand\step{4}%
% Runtime's default exception handler
%
\subsection*{Runtime's default exception handler}
\frameSubsection{pictures/The\_Architect.png}{}


\begin{frame}{Default exception handler}
\begin{columns}
\begin{column}{0.5\textwidth}
\begin{itemize}
\item \localname{domain\_state->exn\_handler} points to a trap located before \funcname{entry}!
\item What installed this very first trap there?
\end{itemize}
\bigskip
\listocaml{../src/default/default.ml}{default/default.ml}
\bigskip
\inputminted[fontsize=\footnotesize]{text}{../src/default/default.out}
\end{column}
\begin{column}{0.5\textwidth}
\centering
\only<1>{\providecommand\step{0}%
% Runtime's default exception handler
%
\subsection*{Runtime's default exception handler}
\frameSubsection{pictures/The\_Architect.png}{}


\begin{frame}{Default exception handler}
\begin{columns}
\begin{column}{0.5\textwidth}
\begin{itemize}
\item \localname{domain\_state->exn\_handler} points to a trap located before \funcname{entry}!
\item What installed this very first trap there?
\end{itemize}
\bigskip
\listocaml{../src/default/default.ml}{default/default.ml}
\bigskip
\inputminted[fontsize=\footnotesize]{text}{../src/default/default.out}
\end{column}
\begin{column}{0.5\textwidth}
\centering
\only<1>{\providecommand\step{0}%
% Runtime's default exception handler
%
\subsection*{Runtime's default exception handler}
\frameSubsection{pictures/The\_Architect.png}{}


\begin{frame}{Default exception handler}
\begin{columns}
\begin{column}{0.5\textwidth}
\begin{itemize}
\item \localname{domain\_state->exn\_handler} points to a trap located before \funcname{entry}!
\item What installed this very first trap there?
\end{itemize}
\bigskip
\listocaml{../src/default/default.ml}{default/default.ml}
\bigskip
\inputminted[fontsize=\footnotesize]{text}{../src/default/default.out}
\end{column}
\begin{column}{0.5\textwidth}
\centering
\only<1>{\providecommand\step{0}\input{diagrams/default/default.tex}}%
\only<2>{\providecommand\step{4}\input{diagrams/default/default.tex}}%
\end{column}
\end{columns}
\end{frame}


\begin{frame}{Startup}{How did we get there by the way?}
\begin{columns}
\begin{column}{0.45\textwidth}
\begin{itemize}
\item The entry point address of the binary is \funcname{main} (\funcname{\_start}) from the runtime
\item The program starts like a C program:
\begin{itemize}
\item \funcname{caml\_start\_program}
\item \funcname{caml\_startup\_common}
\item \funcname{caml\_startup\_exn}
\item \funcname{caml\_startup}
\item \funcname{caml\_main}
\item \funcname{main}
\end{itemize}
%\item \funcname{caml\_startup\_common}: initializes GC, domains \dots
% Also: codefrag, locale, custom opeartions, frame_descr, signals, stack
\item \funcname{caml\_start\_program} is the transition point between the C realm and the OCaml one
\end{itemize}
\bigskip
\listocaml{../src/default/default.ml}{default/default.ml}
\end{column}
\begin{column}{0.55\textwidth}
\listasm[lastline=24,highlightlines={22-24}]{src/ocaml/caml\_start\_program.s}{runtime/amd64.S:604}
\end{column}
\end{columns}
\end{frame}


\begin{frame}{Setup to jump into OCaml entry point}{\funcname{caml\_start\_program} initializes the main fiber}
\begin{columns}
\begin{column}{0.6\textwidth}
\only<1,3,5>{
\begin{itemize}
\item<1-> Stores information to allow switching from OCaml stack to the C stack
\item<3-> Constructs the default exception handler
\item<5-> Switches to the OCaml stack and calls \funcname{caml\_program}
\end{itemize}
\bigskip
\listocaml{../src/default/default.ml}{default/default.ml}
}%
\only<2>{
\listasm[firstline=22,lastline=41,highlightlines={25-27}]{src/ocaml/caml\_start\_program.s}{caml\_start\_program}
}%
\only<4>{
\mintasm[firstline=22,lastline=41,highlightlines={31-38}]{src/ocaml/caml\_start\_program.s}
\listasm[firstline=66,lastline=72]{src/ocaml/caml\_start\_program.s}{caml\_start\_program}
}%
\only<6>{
\listasm[firstline=22,lastline=41,highlightlines={39-41}]{src/ocaml/caml\_start\_program.s}{caml\_start\_program}
}%
\end{column}
\begin{column}{0.4\textwidth}
\centering
\only<1-2>{\providecommand\step{2}\input{diagrams/default/default.tex}}%
\only<3-5>{\providecommand\step{3}\input{diagrams/default/default.tex}}%
\onslide*<6->{\providecommand\step{4}\input{diagrams/default/default.tex}}%
\end{column}
\end{columns}
\end{frame}

\begin{frame}{Executing the OCaml program}{And raising an exception without handler}
\begin{columns}
\begin{column}{0.6\textwidth}
\only<1-3>{
\begin{itemize}
\item<1-> Execution continues with OCaml code
\begin{itemize}
\item \funcname{camlDefault\_\_main\_269}
\item \funcname{camlDefault\_\_entry}
\item \funcname{caml\_program}
\end{itemize}
\item<1-> \funcname{camlDefault\_\_main\_269} raises a \typename{Failure} exception
\item<1-> \typename{Failure} is caught by \funcname{caml\_start\_program}
\begin{itemize}
\item<1-> The handler set the 2nd LSB of the exception value to \funcarg{1}
\item<3-> And then forces \funcname{caml\_start\_program} to terminate
\end{itemize}
\end{itemize}
}
\bigskip
\only<1,3>{\listocaml[highlightlines=3]{../src/default/default.ml}{default/default.ml}}%
\only<2>{\listasm[firstline=66,lastline=72,highlightlines=69]{src/ocaml/caml\_start\_program.s}{caml\_start\_program}}%
\end{column}
\begin{column}{0.4\textwidth}
\centering
\providecommand\step{4}\input{diagrams/default/default.tex}
\end{column}
\end{columns}
\end{frame}
%\only<>{\listasm[firstline=47,lastline=65]{src/ocaml/caml\_start\_program.s}{caml\_start\_program}}


\begin{frame}{Back to C}{Clean up, complain and terminate}
\begin{columns}
\begin{column}{0.45\textwidth}
\begin{itemize}
\item Backtrace:
\begin{itemize}
\item \funcname{caml\_start\_program} $\rightarrow$ OCaml code
\item \funcname{caml\_startup\_common}
\item \funcname{caml\_startup\_exn}
\item \funcname{caml\_startup}
\item \funcname{caml\_main}
\item \funcname{main}
\end{itemize}
\smallskip
\item \funcname{caml\_startup} checks the return value of \funcname{caml\_startup\_exn}
\begin{itemize}
\item The return value has been specially marked by \funcname{caml\_start\_program} handler
\item Calls \funcname{caml\_fatal\_uncaught\_exception}
\end{itemize}
\item<2-> \funcname{caml\_fatal\_uncaught\_exception} either
\begin{itemize}
\item<2-> Calls the user custom uncaught exception handler, if set with \mintinline{ocaml}{Printexc.set_uncaught_exception_handler fn}\footnotemark
\item<2-> Or falls back to the default one
\item<2-> Which notifies about the uncaught exception
\end{itemize}
\item<4-> Terminates the program either with \funcname{abort} or with a non-zero exit code
\end{itemize}
\end{column}
\begin{column}{0.55\textwidth}
\only<1>{
\listc[firstline=84,lastline=89,highlightlines=88]{../ocaml/runtime/caml/mlvalues.h}{runtime/caml/mlvalues.h:84}
\listc[firstline=138,lastline=143,highlightlines={141-142}]{../ocaml/runtime/startup\_nat.c}{runtime/startup\_nat.c:138}
}%
\only<2>{
\listc[firstline=138,highlightlines={147,149}]{../ocaml/runtime/printexc.c}{runtime/printexc.c:138}
}%
\only<3>{
\listc[firstline=110,lastline=134,highlightlines=129]{../ocaml/runtime/printexc.c}{runtime/printexc.c:138}
}%
\only<4>{
\listc[firstline=138,highlightlines={152,154}]{../ocaml/runtime/printexc.c}{runtime/printexc.c:138}
}%
\end{column}
\end{columns}
\footnotetext[1]{\url{https://v2.ocaml.org/api/Printexc.html\#1\_Uncaughtexceptions}}
\end{frame}
}%
\only<2>{\providecommand\step{4}%
% Runtime's default exception handler
%
\subsection*{Runtime's default exception handler}
\frameSubsection{pictures/The\_Architect.png}{}


\begin{frame}{Default exception handler}
\begin{columns}
\begin{column}{0.5\textwidth}
\begin{itemize}
\item \localname{domain\_state->exn\_handler} points to a trap located before \funcname{entry}!
\item What installed this very first trap there?
\end{itemize}
\bigskip
\listocaml{../src/default/default.ml}{default/default.ml}
\bigskip
\inputminted[fontsize=\footnotesize]{text}{../src/default/default.out}
\end{column}
\begin{column}{0.5\textwidth}
\centering
\only<1>{\providecommand\step{0}\input{diagrams/default/default.tex}}%
\only<2>{\providecommand\step{4}\input{diagrams/default/default.tex}}%
\end{column}
\end{columns}
\end{frame}


\begin{frame}{Startup}{How did we get there by the way?}
\begin{columns}
\begin{column}{0.45\textwidth}
\begin{itemize}
\item The entry point address of the binary is \funcname{main} (\funcname{\_start}) from the runtime
\item The program starts like a C program:
\begin{itemize}
\item \funcname{caml\_start\_program}
\item \funcname{caml\_startup\_common}
\item \funcname{caml\_startup\_exn}
\item \funcname{caml\_startup}
\item \funcname{caml\_main}
\item \funcname{main}
\end{itemize}
%\item \funcname{caml\_startup\_common}: initializes GC, domains \dots
% Also: codefrag, locale, custom opeartions, frame_descr, signals, stack
\item \funcname{caml\_start\_program} is the transition point between the C realm and the OCaml one
\end{itemize}
\bigskip
\listocaml{../src/default/default.ml}{default/default.ml}
\end{column}
\begin{column}{0.55\textwidth}
\listasm[lastline=24,highlightlines={22-24}]{src/ocaml/caml\_start\_program.s}{runtime/amd64.S:604}
\end{column}
\end{columns}
\end{frame}


\begin{frame}{Setup to jump into OCaml entry point}{\funcname{caml\_start\_program} initializes the main fiber}
\begin{columns}
\begin{column}{0.6\textwidth}
\only<1,3,5>{
\begin{itemize}
\item<1-> Stores information to allow switching from OCaml stack to the C stack
\item<3-> Constructs the default exception handler
\item<5-> Switches to the OCaml stack and calls \funcname{caml\_program}
\end{itemize}
\bigskip
\listocaml{../src/default/default.ml}{default/default.ml}
}%
\only<2>{
\listasm[firstline=22,lastline=41,highlightlines={25-27}]{src/ocaml/caml\_start\_program.s}{caml\_start\_program}
}%
\only<4>{
\mintasm[firstline=22,lastline=41,highlightlines={31-38}]{src/ocaml/caml\_start\_program.s}
\listasm[firstline=66,lastline=72]{src/ocaml/caml\_start\_program.s}{caml\_start\_program}
}%
\only<6>{
\listasm[firstline=22,lastline=41,highlightlines={39-41}]{src/ocaml/caml\_start\_program.s}{caml\_start\_program}
}%
\end{column}
\begin{column}{0.4\textwidth}
\centering
\only<1-2>{\providecommand\step{2}\input{diagrams/default/default.tex}}%
\only<3-5>{\providecommand\step{3}\input{diagrams/default/default.tex}}%
\onslide*<6->{\providecommand\step{4}\input{diagrams/default/default.tex}}%
\end{column}
\end{columns}
\end{frame}

\begin{frame}{Executing the OCaml program}{And raising an exception without handler}
\begin{columns}
\begin{column}{0.6\textwidth}
\only<1-3>{
\begin{itemize}
\item<1-> Execution continues with OCaml code
\begin{itemize}
\item \funcname{camlDefault\_\_main\_269}
\item \funcname{camlDefault\_\_entry}
\item \funcname{caml\_program}
\end{itemize}
\item<1-> \funcname{camlDefault\_\_main\_269} raises a \typename{Failure} exception
\item<1-> \typename{Failure} is caught by \funcname{caml\_start\_program}
\begin{itemize}
\item<1-> The handler set the 2nd LSB of the exception value to \funcarg{1}
\item<3-> And then forces \funcname{caml\_start\_program} to terminate
\end{itemize}
\end{itemize}
}
\bigskip
\only<1,3>{\listocaml[highlightlines=3]{../src/default/default.ml}{default/default.ml}}%
\only<2>{\listasm[firstline=66,lastline=72,highlightlines=69]{src/ocaml/caml\_start\_program.s}{caml\_start\_program}}%
\end{column}
\begin{column}{0.4\textwidth}
\centering
\providecommand\step{4}\input{diagrams/default/default.tex}
\end{column}
\end{columns}
\end{frame}
%\only<>{\listasm[firstline=47,lastline=65]{src/ocaml/caml\_start\_program.s}{caml\_start\_program}}


\begin{frame}{Back to C}{Clean up, complain and terminate}
\begin{columns}
\begin{column}{0.45\textwidth}
\begin{itemize}
\item Backtrace:
\begin{itemize}
\item \funcname{caml\_start\_program} $\rightarrow$ OCaml code
\item \funcname{caml\_startup\_common}
\item \funcname{caml\_startup\_exn}
\item \funcname{caml\_startup}
\item \funcname{caml\_main}
\item \funcname{main}
\end{itemize}
\smallskip
\item \funcname{caml\_startup} checks the return value of \funcname{caml\_startup\_exn}
\begin{itemize}
\item The return value has been specially marked by \funcname{caml\_start\_program} handler
\item Calls \funcname{caml\_fatal\_uncaught\_exception}
\end{itemize}
\item<2-> \funcname{caml\_fatal\_uncaught\_exception} either
\begin{itemize}
\item<2-> Calls the user custom uncaught exception handler, if set with \mintinline{ocaml}{Printexc.set_uncaught_exception_handler fn}\footnotemark
\item<2-> Or falls back to the default one
\item<2-> Which notifies about the uncaught exception
\end{itemize}
\item<4-> Terminates the program either with \funcname{abort} or with a non-zero exit code
\end{itemize}
\end{column}
\begin{column}{0.55\textwidth}
\only<1>{
\listc[firstline=84,lastline=89,highlightlines=88]{../ocaml/runtime/caml/mlvalues.h}{runtime/caml/mlvalues.h:84}
\listc[firstline=138,lastline=143,highlightlines={141-142}]{../ocaml/runtime/startup\_nat.c}{runtime/startup\_nat.c:138}
}%
\only<2>{
\listc[firstline=138,highlightlines={147,149}]{../ocaml/runtime/printexc.c}{runtime/printexc.c:138}
}%
\only<3>{
\listc[firstline=110,lastline=134,highlightlines=129]{../ocaml/runtime/printexc.c}{runtime/printexc.c:138}
}%
\only<4>{
\listc[firstline=138,highlightlines={152,154}]{../ocaml/runtime/printexc.c}{runtime/printexc.c:138}
}%
\end{column}
\end{columns}
\footnotetext[1]{\url{https://v2.ocaml.org/api/Printexc.html\#1\_Uncaughtexceptions}}
\end{frame}
}%
\end{column}
\end{columns}
\end{frame}


\begin{frame}{Startup}{How did we get there by the way?}
\begin{columns}
\begin{column}{0.45\textwidth}
\begin{itemize}
\item The entry point address of the binary is \funcname{main} (\funcname{\_start}) from the runtime
\item The program starts like a C program:
\begin{itemize}
\item \funcname{caml\_start\_program}
\item \funcname{caml\_startup\_common}
\item \funcname{caml\_startup\_exn}
\item \funcname{caml\_startup}
\item \funcname{caml\_main}
\item \funcname{main}
\end{itemize}
%\item \funcname{caml\_startup\_common}: initializes GC, domains \dots
% Also: codefrag, locale, custom opeartions, frame_descr, signals, stack
\item \funcname{caml\_start\_program} is the transition point between the C realm and the OCaml one
\end{itemize}
\bigskip
\listocaml{../src/default/default.ml}{default/default.ml}
\end{column}
\begin{column}{0.55\textwidth}
\listasm[lastline=24,highlightlines={22-24}]{src/ocaml/caml\_start\_program.s}{runtime/amd64.S:604}
\end{column}
\end{columns}
\end{frame}


\begin{frame}{Setup to jump into OCaml entry point}{\funcname{caml\_start\_program} initializes the main fiber}
\begin{columns}
\begin{column}{0.6\textwidth}
\only<1,3,5>{
\begin{itemize}
\item<1-> Stores information to allow switching from OCaml stack to the C stack
\item<3-> Constructs the default exception handler
\item<5-> Switches to the OCaml stack and calls \funcname{caml\_program}
\end{itemize}
\bigskip
\listocaml{../src/default/default.ml}{default/default.ml}
}%
\only<2>{
\listasm[firstline=22,lastline=41,highlightlines={25-27}]{src/ocaml/caml\_start\_program.s}{caml\_start\_program}
}%
\only<4>{
\mintasm[firstline=22,lastline=41,highlightlines={31-38}]{src/ocaml/caml\_start\_program.s}
\listasm[firstline=66,lastline=72]{src/ocaml/caml\_start\_program.s}{caml\_start\_program}
}%
\only<6>{
\listasm[firstline=22,lastline=41,highlightlines={39-41}]{src/ocaml/caml\_start\_program.s}{caml\_start\_program}
}%
\end{column}
\begin{column}{0.4\textwidth}
\centering
\only<1-2>{\providecommand\step{2}%
% Runtime's default exception handler
%
\subsection*{Runtime's default exception handler}
\frameSubsection{pictures/The\_Architect.png}{}


\begin{frame}{Default exception handler}
\begin{columns}
\begin{column}{0.5\textwidth}
\begin{itemize}
\item \localname{domain\_state->exn\_handler} points to a trap located before \funcname{entry}!
\item What installed this very first trap there?
\end{itemize}
\bigskip
\listocaml{../src/default/default.ml}{default/default.ml}
\bigskip
\inputminted[fontsize=\footnotesize]{text}{../src/default/default.out}
\end{column}
\begin{column}{0.5\textwidth}
\centering
\only<1>{\providecommand\step{0}\input{diagrams/default/default.tex}}%
\only<2>{\providecommand\step{4}\input{diagrams/default/default.tex}}%
\end{column}
\end{columns}
\end{frame}


\begin{frame}{Startup}{How did we get there by the way?}
\begin{columns}
\begin{column}{0.45\textwidth}
\begin{itemize}
\item The entry point address of the binary is \funcname{main} (\funcname{\_start}) from the runtime
\item The program starts like a C program:
\begin{itemize}
\item \funcname{caml\_start\_program}
\item \funcname{caml\_startup\_common}
\item \funcname{caml\_startup\_exn}
\item \funcname{caml\_startup}
\item \funcname{caml\_main}
\item \funcname{main}
\end{itemize}
%\item \funcname{caml\_startup\_common}: initializes GC, domains \dots
% Also: codefrag, locale, custom opeartions, frame_descr, signals, stack
\item \funcname{caml\_start\_program} is the transition point between the C realm and the OCaml one
\end{itemize}
\bigskip
\listocaml{../src/default/default.ml}{default/default.ml}
\end{column}
\begin{column}{0.55\textwidth}
\listasm[lastline=24,highlightlines={22-24}]{src/ocaml/caml\_start\_program.s}{runtime/amd64.S:604}
\end{column}
\end{columns}
\end{frame}


\begin{frame}{Setup to jump into OCaml entry point}{\funcname{caml\_start\_program} initializes the main fiber}
\begin{columns}
\begin{column}{0.6\textwidth}
\only<1,3,5>{
\begin{itemize}
\item<1-> Stores information to allow switching from OCaml stack to the C stack
\item<3-> Constructs the default exception handler
\item<5-> Switches to the OCaml stack and calls \funcname{caml\_program}
\end{itemize}
\bigskip
\listocaml{../src/default/default.ml}{default/default.ml}
}%
\only<2>{
\listasm[firstline=22,lastline=41,highlightlines={25-27}]{src/ocaml/caml\_start\_program.s}{caml\_start\_program}
}%
\only<4>{
\mintasm[firstline=22,lastline=41,highlightlines={31-38}]{src/ocaml/caml\_start\_program.s}
\listasm[firstline=66,lastline=72]{src/ocaml/caml\_start\_program.s}{caml\_start\_program}
}%
\only<6>{
\listasm[firstline=22,lastline=41,highlightlines={39-41}]{src/ocaml/caml\_start\_program.s}{caml\_start\_program}
}%
\end{column}
\begin{column}{0.4\textwidth}
\centering
\only<1-2>{\providecommand\step{2}\input{diagrams/default/default.tex}}%
\only<3-5>{\providecommand\step{3}\input{diagrams/default/default.tex}}%
\onslide*<6->{\providecommand\step{4}\input{diagrams/default/default.tex}}%
\end{column}
\end{columns}
\end{frame}

\begin{frame}{Executing the OCaml program}{And raising an exception without handler}
\begin{columns}
\begin{column}{0.6\textwidth}
\only<1-3>{
\begin{itemize}
\item<1-> Execution continues with OCaml code
\begin{itemize}
\item \funcname{camlDefault\_\_main\_269}
\item \funcname{camlDefault\_\_entry}
\item \funcname{caml\_program}
\end{itemize}
\item<1-> \funcname{camlDefault\_\_main\_269} raises a \typename{Failure} exception
\item<1-> \typename{Failure} is caught by \funcname{caml\_start\_program}
\begin{itemize}
\item<1-> The handler set the 2nd LSB of the exception value to \funcarg{1}
\item<3-> And then forces \funcname{caml\_start\_program} to terminate
\end{itemize}
\end{itemize}
}
\bigskip
\only<1,3>{\listocaml[highlightlines=3]{../src/default/default.ml}{default/default.ml}}%
\only<2>{\listasm[firstline=66,lastline=72,highlightlines=69]{src/ocaml/caml\_start\_program.s}{caml\_start\_program}}%
\end{column}
\begin{column}{0.4\textwidth}
\centering
\providecommand\step{4}\input{diagrams/default/default.tex}
\end{column}
\end{columns}
\end{frame}
%\only<>{\listasm[firstline=47,lastline=65]{src/ocaml/caml\_start\_program.s}{caml\_start\_program}}


\begin{frame}{Back to C}{Clean up, complain and terminate}
\begin{columns}
\begin{column}{0.45\textwidth}
\begin{itemize}
\item Backtrace:
\begin{itemize}
\item \funcname{caml\_start\_program} $\rightarrow$ OCaml code
\item \funcname{caml\_startup\_common}
\item \funcname{caml\_startup\_exn}
\item \funcname{caml\_startup}
\item \funcname{caml\_main}
\item \funcname{main}
\end{itemize}
\smallskip
\item \funcname{caml\_startup} checks the return value of \funcname{caml\_startup\_exn}
\begin{itemize}
\item The return value has been specially marked by \funcname{caml\_start\_program} handler
\item Calls \funcname{caml\_fatal\_uncaught\_exception}
\end{itemize}
\item<2-> \funcname{caml\_fatal\_uncaught\_exception} either
\begin{itemize}
\item<2-> Calls the user custom uncaught exception handler, if set with \mintinline{ocaml}{Printexc.set_uncaught_exception_handler fn}\footnotemark
\item<2-> Or falls back to the default one
\item<2-> Which notifies about the uncaught exception
\end{itemize}
\item<4-> Terminates the program either with \funcname{abort} or with a non-zero exit code
\end{itemize}
\end{column}
\begin{column}{0.55\textwidth}
\only<1>{
\listc[firstline=84,lastline=89,highlightlines=88]{../ocaml/runtime/caml/mlvalues.h}{runtime/caml/mlvalues.h:84}
\listc[firstline=138,lastline=143,highlightlines={141-142}]{../ocaml/runtime/startup\_nat.c}{runtime/startup\_nat.c:138}
}%
\only<2>{
\listc[firstline=138,highlightlines={147,149}]{../ocaml/runtime/printexc.c}{runtime/printexc.c:138}
}%
\only<3>{
\listc[firstline=110,lastline=134,highlightlines=129]{../ocaml/runtime/printexc.c}{runtime/printexc.c:138}
}%
\only<4>{
\listc[firstline=138,highlightlines={152,154}]{../ocaml/runtime/printexc.c}{runtime/printexc.c:138}
}%
\end{column}
\end{columns}
\footnotetext[1]{\url{https://v2.ocaml.org/api/Printexc.html\#1\_Uncaughtexceptions}}
\end{frame}
}%
\only<3-5>{\providecommand\step{3}%
% Runtime's default exception handler
%
\subsection*{Runtime's default exception handler}
\frameSubsection{pictures/The\_Architect.png}{}


\begin{frame}{Default exception handler}
\begin{columns}
\begin{column}{0.5\textwidth}
\begin{itemize}
\item \localname{domain\_state->exn\_handler} points to a trap located before \funcname{entry}!
\item What installed this very first trap there?
\end{itemize}
\bigskip
\listocaml{../src/default/default.ml}{default/default.ml}
\bigskip
\inputminted[fontsize=\footnotesize]{text}{../src/default/default.out}
\end{column}
\begin{column}{0.5\textwidth}
\centering
\only<1>{\providecommand\step{0}\input{diagrams/default/default.tex}}%
\only<2>{\providecommand\step{4}\input{diagrams/default/default.tex}}%
\end{column}
\end{columns}
\end{frame}


\begin{frame}{Startup}{How did we get there by the way?}
\begin{columns}
\begin{column}{0.45\textwidth}
\begin{itemize}
\item The entry point address of the binary is \funcname{main} (\funcname{\_start}) from the runtime
\item The program starts like a C program:
\begin{itemize}
\item \funcname{caml\_start\_program}
\item \funcname{caml\_startup\_common}
\item \funcname{caml\_startup\_exn}
\item \funcname{caml\_startup}
\item \funcname{caml\_main}
\item \funcname{main}
\end{itemize}
%\item \funcname{caml\_startup\_common}: initializes GC, domains \dots
% Also: codefrag, locale, custom opeartions, frame_descr, signals, stack
\item \funcname{caml\_start\_program} is the transition point between the C realm and the OCaml one
\end{itemize}
\bigskip
\listocaml{../src/default/default.ml}{default/default.ml}
\end{column}
\begin{column}{0.55\textwidth}
\listasm[lastline=24,highlightlines={22-24}]{src/ocaml/caml\_start\_program.s}{runtime/amd64.S:604}
\end{column}
\end{columns}
\end{frame}


\begin{frame}{Setup to jump into OCaml entry point}{\funcname{caml\_start\_program} initializes the main fiber}
\begin{columns}
\begin{column}{0.6\textwidth}
\only<1,3,5>{
\begin{itemize}
\item<1-> Stores information to allow switching from OCaml stack to the C stack
\item<3-> Constructs the default exception handler
\item<5-> Switches to the OCaml stack and calls \funcname{caml\_program}
\end{itemize}
\bigskip
\listocaml{../src/default/default.ml}{default/default.ml}
}%
\only<2>{
\listasm[firstline=22,lastline=41,highlightlines={25-27}]{src/ocaml/caml\_start\_program.s}{caml\_start\_program}
}%
\only<4>{
\mintasm[firstline=22,lastline=41,highlightlines={31-38}]{src/ocaml/caml\_start\_program.s}
\listasm[firstline=66,lastline=72]{src/ocaml/caml\_start\_program.s}{caml\_start\_program}
}%
\only<6>{
\listasm[firstline=22,lastline=41,highlightlines={39-41}]{src/ocaml/caml\_start\_program.s}{caml\_start\_program}
}%
\end{column}
\begin{column}{0.4\textwidth}
\centering
\only<1-2>{\providecommand\step{2}\input{diagrams/default/default.tex}}%
\only<3-5>{\providecommand\step{3}\input{diagrams/default/default.tex}}%
\onslide*<6->{\providecommand\step{4}\input{diagrams/default/default.tex}}%
\end{column}
\end{columns}
\end{frame}

\begin{frame}{Executing the OCaml program}{And raising an exception without handler}
\begin{columns}
\begin{column}{0.6\textwidth}
\only<1-3>{
\begin{itemize}
\item<1-> Execution continues with OCaml code
\begin{itemize}
\item \funcname{camlDefault\_\_main\_269}
\item \funcname{camlDefault\_\_entry}
\item \funcname{caml\_program}
\end{itemize}
\item<1-> \funcname{camlDefault\_\_main\_269} raises a \typename{Failure} exception
\item<1-> \typename{Failure} is caught by \funcname{caml\_start\_program}
\begin{itemize}
\item<1-> The handler set the 2nd LSB of the exception value to \funcarg{1}
\item<3-> And then forces \funcname{caml\_start\_program} to terminate
\end{itemize}
\end{itemize}
}
\bigskip
\only<1,3>{\listocaml[highlightlines=3]{../src/default/default.ml}{default/default.ml}}%
\only<2>{\listasm[firstline=66,lastline=72,highlightlines=69]{src/ocaml/caml\_start\_program.s}{caml\_start\_program}}%
\end{column}
\begin{column}{0.4\textwidth}
\centering
\providecommand\step{4}\input{diagrams/default/default.tex}
\end{column}
\end{columns}
\end{frame}
%\only<>{\listasm[firstline=47,lastline=65]{src/ocaml/caml\_start\_program.s}{caml\_start\_program}}


\begin{frame}{Back to C}{Clean up, complain and terminate}
\begin{columns}
\begin{column}{0.45\textwidth}
\begin{itemize}
\item Backtrace:
\begin{itemize}
\item \funcname{caml\_start\_program} $\rightarrow$ OCaml code
\item \funcname{caml\_startup\_common}
\item \funcname{caml\_startup\_exn}
\item \funcname{caml\_startup}
\item \funcname{caml\_main}
\item \funcname{main}
\end{itemize}
\smallskip
\item \funcname{caml\_startup} checks the return value of \funcname{caml\_startup\_exn}
\begin{itemize}
\item The return value has been specially marked by \funcname{caml\_start\_program} handler
\item Calls \funcname{caml\_fatal\_uncaught\_exception}
\end{itemize}
\item<2-> \funcname{caml\_fatal\_uncaught\_exception} either
\begin{itemize}
\item<2-> Calls the user custom uncaught exception handler, if set with \mintinline{ocaml}{Printexc.set_uncaught_exception_handler fn}\footnotemark
\item<2-> Or falls back to the default one
\item<2-> Which notifies about the uncaught exception
\end{itemize}
\item<4-> Terminates the program either with \funcname{abort} or with a non-zero exit code
\end{itemize}
\end{column}
\begin{column}{0.55\textwidth}
\only<1>{
\listc[firstline=84,lastline=89,highlightlines=88]{../ocaml/runtime/caml/mlvalues.h}{runtime/caml/mlvalues.h:84}
\listc[firstline=138,lastline=143,highlightlines={141-142}]{../ocaml/runtime/startup\_nat.c}{runtime/startup\_nat.c:138}
}%
\only<2>{
\listc[firstline=138,highlightlines={147,149}]{../ocaml/runtime/printexc.c}{runtime/printexc.c:138}
}%
\only<3>{
\listc[firstline=110,lastline=134,highlightlines=129]{../ocaml/runtime/printexc.c}{runtime/printexc.c:138}
}%
\only<4>{
\listc[firstline=138,highlightlines={152,154}]{../ocaml/runtime/printexc.c}{runtime/printexc.c:138}
}%
\end{column}
\end{columns}
\footnotetext[1]{\url{https://v2.ocaml.org/api/Printexc.html\#1\_Uncaughtexceptions}}
\end{frame}
}%
\onslide*<6->{\providecommand\step{4}%
% Runtime's default exception handler
%
\subsection*{Runtime's default exception handler}
\frameSubsection{pictures/The\_Architect.png}{}


\begin{frame}{Default exception handler}
\begin{columns}
\begin{column}{0.5\textwidth}
\begin{itemize}
\item \localname{domain\_state->exn\_handler} points to a trap located before \funcname{entry}!
\item What installed this very first trap there?
\end{itemize}
\bigskip
\listocaml{../src/default/default.ml}{default/default.ml}
\bigskip
\inputminted[fontsize=\footnotesize]{text}{../src/default/default.out}
\end{column}
\begin{column}{0.5\textwidth}
\centering
\only<1>{\providecommand\step{0}\input{diagrams/default/default.tex}}%
\only<2>{\providecommand\step{4}\input{diagrams/default/default.tex}}%
\end{column}
\end{columns}
\end{frame}


\begin{frame}{Startup}{How did we get there by the way?}
\begin{columns}
\begin{column}{0.45\textwidth}
\begin{itemize}
\item The entry point address of the binary is \funcname{main} (\funcname{\_start}) from the runtime
\item The program starts like a C program:
\begin{itemize}
\item \funcname{caml\_start\_program}
\item \funcname{caml\_startup\_common}
\item \funcname{caml\_startup\_exn}
\item \funcname{caml\_startup}
\item \funcname{caml\_main}
\item \funcname{main}
\end{itemize}
%\item \funcname{caml\_startup\_common}: initializes GC, domains \dots
% Also: codefrag, locale, custom opeartions, frame_descr, signals, stack
\item \funcname{caml\_start\_program} is the transition point between the C realm and the OCaml one
\end{itemize}
\bigskip
\listocaml{../src/default/default.ml}{default/default.ml}
\end{column}
\begin{column}{0.55\textwidth}
\listasm[lastline=24,highlightlines={22-24}]{src/ocaml/caml\_start\_program.s}{runtime/amd64.S:604}
\end{column}
\end{columns}
\end{frame}


\begin{frame}{Setup to jump into OCaml entry point}{\funcname{caml\_start\_program} initializes the main fiber}
\begin{columns}
\begin{column}{0.6\textwidth}
\only<1,3,5>{
\begin{itemize}
\item<1-> Stores information to allow switching from OCaml stack to the C stack
\item<3-> Constructs the default exception handler
\item<5-> Switches to the OCaml stack and calls \funcname{caml\_program}
\end{itemize}
\bigskip
\listocaml{../src/default/default.ml}{default/default.ml}
}%
\only<2>{
\listasm[firstline=22,lastline=41,highlightlines={25-27}]{src/ocaml/caml\_start\_program.s}{caml\_start\_program}
}%
\only<4>{
\mintasm[firstline=22,lastline=41,highlightlines={31-38}]{src/ocaml/caml\_start\_program.s}
\listasm[firstline=66,lastline=72]{src/ocaml/caml\_start\_program.s}{caml\_start\_program}
}%
\only<6>{
\listasm[firstline=22,lastline=41,highlightlines={39-41}]{src/ocaml/caml\_start\_program.s}{caml\_start\_program}
}%
\end{column}
\begin{column}{0.4\textwidth}
\centering
\only<1-2>{\providecommand\step{2}\input{diagrams/default/default.tex}}%
\only<3-5>{\providecommand\step{3}\input{diagrams/default/default.tex}}%
\onslide*<6->{\providecommand\step{4}\input{diagrams/default/default.tex}}%
\end{column}
\end{columns}
\end{frame}

\begin{frame}{Executing the OCaml program}{And raising an exception without handler}
\begin{columns}
\begin{column}{0.6\textwidth}
\only<1-3>{
\begin{itemize}
\item<1-> Execution continues with OCaml code
\begin{itemize}
\item \funcname{camlDefault\_\_main\_269}
\item \funcname{camlDefault\_\_entry}
\item \funcname{caml\_program}
\end{itemize}
\item<1-> \funcname{camlDefault\_\_main\_269} raises a \typename{Failure} exception
\item<1-> \typename{Failure} is caught by \funcname{caml\_start\_program}
\begin{itemize}
\item<1-> The handler set the 2nd LSB of the exception value to \funcarg{1}
\item<3-> And then forces \funcname{caml\_start\_program} to terminate
\end{itemize}
\end{itemize}
}
\bigskip
\only<1,3>{\listocaml[highlightlines=3]{../src/default/default.ml}{default/default.ml}}%
\only<2>{\listasm[firstline=66,lastline=72,highlightlines=69]{src/ocaml/caml\_start\_program.s}{caml\_start\_program}}%
\end{column}
\begin{column}{0.4\textwidth}
\centering
\providecommand\step{4}\input{diagrams/default/default.tex}
\end{column}
\end{columns}
\end{frame}
%\only<>{\listasm[firstline=47,lastline=65]{src/ocaml/caml\_start\_program.s}{caml\_start\_program}}


\begin{frame}{Back to C}{Clean up, complain and terminate}
\begin{columns}
\begin{column}{0.45\textwidth}
\begin{itemize}
\item Backtrace:
\begin{itemize}
\item \funcname{caml\_start\_program} $\rightarrow$ OCaml code
\item \funcname{caml\_startup\_common}
\item \funcname{caml\_startup\_exn}
\item \funcname{caml\_startup}
\item \funcname{caml\_main}
\item \funcname{main}
\end{itemize}
\smallskip
\item \funcname{caml\_startup} checks the return value of \funcname{caml\_startup\_exn}
\begin{itemize}
\item The return value has been specially marked by \funcname{caml\_start\_program} handler
\item Calls \funcname{caml\_fatal\_uncaught\_exception}
\end{itemize}
\item<2-> \funcname{caml\_fatal\_uncaught\_exception} either
\begin{itemize}
\item<2-> Calls the user custom uncaught exception handler, if set with \mintinline{ocaml}{Printexc.set_uncaught_exception_handler fn}\footnotemark
\item<2-> Or falls back to the default one
\item<2-> Which notifies about the uncaught exception
\end{itemize}
\item<4-> Terminates the program either with \funcname{abort} or with a non-zero exit code
\end{itemize}
\end{column}
\begin{column}{0.55\textwidth}
\only<1>{
\listc[firstline=84,lastline=89,highlightlines=88]{../ocaml/runtime/caml/mlvalues.h}{runtime/caml/mlvalues.h:84}
\listc[firstline=138,lastline=143,highlightlines={141-142}]{../ocaml/runtime/startup\_nat.c}{runtime/startup\_nat.c:138}
}%
\only<2>{
\listc[firstline=138,highlightlines={147,149}]{../ocaml/runtime/printexc.c}{runtime/printexc.c:138}
}%
\only<3>{
\listc[firstline=110,lastline=134,highlightlines=129]{../ocaml/runtime/printexc.c}{runtime/printexc.c:138}
}%
\only<4>{
\listc[firstline=138,highlightlines={152,154}]{../ocaml/runtime/printexc.c}{runtime/printexc.c:138}
}%
\end{column}
\end{columns}
\footnotetext[1]{\url{https://v2.ocaml.org/api/Printexc.html\#1\_Uncaughtexceptions}}
\end{frame}
}%
\end{column}
\end{columns}
\end{frame}

\begin{frame}{Executing the OCaml program}{And raising an exception without handler}
\begin{columns}
\begin{column}{0.6\textwidth}
\only<1-3>{
\begin{itemize}
\item<1-> Execution continues with OCaml code
\begin{itemize}
\item \funcname{camlDefault\_\_main\_269}
\item \funcname{camlDefault\_\_entry}
\item \funcname{caml\_program}
\end{itemize}
\item<1-> \funcname{camlDefault\_\_main\_269} raises a \typename{Failure} exception
\item<1-> \typename{Failure} is caught by \funcname{caml\_start\_program}
\begin{itemize}
\item<1-> The handler set the 2nd LSB of the exception value to \funcarg{1}
\item<3-> And then forces \funcname{caml\_start\_program} to terminate
\end{itemize}
\end{itemize}
}
\bigskip
\only<1,3>{\listocaml[highlightlines=3]{../src/default/default.ml}{default/default.ml}}%
\only<2>{\listasm[firstline=66,lastline=72,highlightlines=69]{src/ocaml/caml\_start\_program.s}{caml\_start\_program}}%
\end{column}
\begin{column}{0.4\textwidth}
\centering
\providecommand\step{4}%
% Runtime's default exception handler
%
\subsection*{Runtime's default exception handler}
\frameSubsection{pictures/The\_Architect.png}{}


\begin{frame}{Default exception handler}
\begin{columns}
\begin{column}{0.5\textwidth}
\begin{itemize}
\item \localname{domain\_state->exn\_handler} points to a trap located before \funcname{entry}!
\item What installed this very first trap there?
\end{itemize}
\bigskip
\listocaml{../src/default/default.ml}{default/default.ml}
\bigskip
\inputminted[fontsize=\footnotesize]{text}{../src/default/default.out}
\end{column}
\begin{column}{0.5\textwidth}
\centering
\only<1>{\providecommand\step{0}\input{diagrams/default/default.tex}}%
\only<2>{\providecommand\step{4}\input{diagrams/default/default.tex}}%
\end{column}
\end{columns}
\end{frame}


\begin{frame}{Startup}{How did we get there by the way?}
\begin{columns}
\begin{column}{0.45\textwidth}
\begin{itemize}
\item The entry point address of the binary is \funcname{main} (\funcname{\_start}) from the runtime
\item The program starts like a C program:
\begin{itemize}
\item \funcname{caml\_start\_program}
\item \funcname{caml\_startup\_common}
\item \funcname{caml\_startup\_exn}
\item \funcname{caml\_startup}
\item \funcname{caml\_main}
\item \funcname{main}
\end{itemize}
%\item \funcname{caml\_startup\_common}: initializes GC, domains \dots
% Also: codefrag, locale, custom opeartions, frame_descr, signals, stack
\item \funcname{caml\_start\_program} is the transition point between the C realm and the OCaml one
\end{itemize}
\bigskip
\listocaml{../src/default/default.ml}{default/default.ml}
\end{column}
\begin{column}{0.55\textwidth}
\listasm[lastline=24,highlightlines={22-24}]{src/ocaml/caml\_start\_program.s}{runtime/amd64.S:604}
\end{column}
\end{columns}
\end{frame}


\begin{frame}{Setup to jump into OCaml entry point}{\funcname{caml\_start\_program} initializes the main fiber}
\begin{columns}
\begin{column}{0.6\textwidth}
\only<1,3,5>{
\begin{itemize}
\item<1-> Stores information to allow switching from OCaml stack to the C stack
\item<3-> Constructs the default exception handler
\item<5-> Switches to the OCaml stack and calls \funcname{caml\_program}
\end{itemize}
\bigskip
\listocaml{../src/default/default.ml}{default/default.ml}
}%
\only<2>{
\listasm[firstline=22,lastline=41,highlightlines={25-27}]{src/ocaml/caml\_start\_program.s}{caml\_start\_program}
}%
\only<4>{
\mintasm[firstline=22,lastline=41,highlightlines={31-38}]{src/ocaml/caml\_start\_program.s}
\listasm[firstline=66,lastline=72]{src/ocaml/caml\_start\_program.s}{caml\_start\_program}
}%
\only<6>{
\listasm[firstline=22,lastline=41,highlightlines={39-41}]{src/ocaml/caml\_start\_program.s}{caml\_start\_program}
}%
\end{column}
\begin{column}{0.4\textwidth}
\centering
\only<1-2>{\providecommand\step{2}\input{diagrams/default/default.tex}}%
\only<3-5>{\providecommand\step{3}\input{diagrams/default/default.tex}}%
\onslide*<6->{\providecommand\step{4}\input{diagrams/default/default.tex}}%
\end{column}
\end{columns}
\end{frame}

\begin{frame}{Executing the OCaml program}{And raising an exception without handler}
\begin{columns}
\begin{column}{0.6\textwidth}
\only<1-3>{
\begin{itemize}
\item<1-> Execution continues with OCaml code
\begin{itemize}
\item \funcname{camlDefault\_\_main\_269}
\item \funcname{camlDefault\_\_entry}
\item \funcname{caml\_program}
\end{itemize}
\item<1-> \funcname{camlDefault\_\_main\_269} raises a \typename{Failure} exception
\item<1-> \typename{Failure} is caught by \funcname{caml\_start\_program}
\begin{itemize}
\item<1-> The handler set the 2nd LSB of the exception value to \funcarg{1}
\item<3-> And then forces \funcname{caml\_start\_program} to terminate
\end{itemize}
\end{itemize}
}
\bigskip
\only<1,3>{\listocaml[highlightlines=3]{../src/default/default.ml}{default/default.ml}}%
\only<2>{\listasm[firstline=66,lastline=72,highlightlines=69]{src/ocaml/caml\_start\_program.s}{caml\_start\_program}}%
\end{column}
\begin{column}{0.4\textwidth}
\centering
\providecommand\step{4}\input{diagrams/default/default.tex}
\end{column}
\end{columns}
\end{frame}
%\only<>{\listasm[firstline=47,lastline=65]{src/ocaml/caml\_start\_program.s}{caml\_start\_program}}


\begin{frame}{Back to C}{Clean up, complain and terminate}
\begin{columns}
\begin{column}{0.45\textwidth}
\begin{itemize}
\item Backtrace:
\begin{itemize}
\item \funcname{caml\_start\_program} $\rightarrow$ OCaml code
\item \funcname{caml\_startup\_common}
\item \funcname{caml\_startup\_exn}
\item \funcname{caml\_startup}
\item \funcname{caml\_main}
\item \funcname{main}
\end{itemize}
\smallskip
\item \funcname{caml\_startup} checks the return value of \funcname{caml\_startup\_exn}
\begin{itemize}
\item The return value has been specially marked by \funcname{caml\_start\_program} handler
\item Calls \funcname{caml\_fatal\_uncaught\_exception}
\end{itemize}
\item<2-> \funcname{caml\_fatal\_uncaught\_exception} either
\begin{itemize}
\item<2-> Calls the user custom uncaught exception handler, if set with \mintinline{ocaml}{Printexc.set_uncaught_exception_handler fn}\footnotemark
\item<2-> Or falls back to the default one
\item<2-> Which notifies about the uncaught exception
\end{itemize}
\item<4-> Terminates the program either with \funcname{abort} or with a non-zero exit code
\end{itemize}
\end{column}
\begin{column}{0.55\textwidth}
\only<1>{
\listc[firstline=84,lastline=89,highlightlines=88]{../ocaml/runtime/caml/mlvalues.h}{runtime/caml/mlvalues.h:84}
\listc[firstline=138,lastline=143,highlightlines={141-142}]{../ocaml/runtime/startup\_nat.c}{runtime/startup\_nat.c:138}
}%
\only<2>{
\listc[firstline=138,highlightlines={147,149}]{../ocaml/runtime/printexc.c}{runtime/printexc.c:138}
}%
\only<3>{
\listc[firstline=110,lastline=134,highlightlines=129]{../ocaml/runtime/printexc.c}{runtime/printexc.c:138}
}%
\only<4>{
\listc[firstline=138,highlightlines={152,154}]{../ocaml/runtime/printexc.c}{runtime/printexc.c:138}
}%
\end{column}
\end{columns}
\footnotetext[1]{\url{https://v2.ocaml.org/api/Printexc.html\#1\_Uncaughtexceptions}}
\end{frame}

\end{column}
\end{columns}
\end{frame}
%\only<>{\listasm[firstline=47,lastline=65]{src/ocaml/caml\_start\_program.s}{caml\_start\_program}}


\begin{frame}{Back to C}{Clean up, complain and terminate}
\begin{columns}
\begin{column}{0.45\textwidth}
\begin{itemize}
\item Backtrace:
\begin{itemize}
\item \funcname{caml\_start\_program} $\rightarrow$ OCaml code
\item \funcname{caml\_startup\_common}
\item \funcname{caml\_startup\_exn}
\item \funcname{caml\_startup}
\item \funcname{caml\_main}
\item \funcname{main}
\end{itemize}
\smallskip
\item \funcname{caml\_startup} checks the return value of \funcname{caml\_startup\_exn}
\begin{itemize}
\item The return value has been specially marked by \funcname{caml\_start\_program} handler
\item Calls \funcname{caml\_fatal\_uncaught\_exception}
\end{itemize}
\item<2-> \funcname{caml\_fatal\_uncaught\_exception} either
\begin{itemize}
\item<2-> Calls the user custom uncaught exception handler, if set with \mintinline{ocaml}{Printexc.set_uncaught_exception_handler fn}\footnotemark
\item<2-> Or falls back to the default one
\item<2-> Which notifies about the uncaught exception
\end{itemize}
\item<4-> Terminates the program either with \funcname{abort} or with a non-zero exit code
\end{itemize}
\end{column}
\begin{column}{0.55\textwidth}
\only<1>{
\listc[firstline=84,lastline=89,highlightlines=88]{../ocaml/runtime/caml/mlvalues.h}{runtime/caml/mlvalues.h:84}
\listc[firstline=138,lastline=143,highlightlines={141-142}]{../ocaml/runtime/startup\_nat.c}{runtime/startup\_nat.c:138}
}%
\only<2>{
\listc[firstline=138,highlightlines={147,149}]{../ocaml/runtime/printexc.c}{runtime/printexc.c:138}
}%
\only<3>{
\listc[firstline=110,lastline=134,highlightlines=129]{../ocaml/runtime/printexc.c}{runtime/printexc.c:138}
}%
\only<4>{
\listc[firstline=138,highlightlines={152,154}]{../ocaml/runtime/printexc.c}{runtime/printexc.c:138}
}%
\end{column}
\end{columns}
\footnotetext[1]{\url{https://v2.ocaml.org/api/Printexc.html\#1\_Uncaughtexceptions}}
\end{frame}
}%
\only<2>{\providecommand\step{4}%
% Runtime's default exception handler
%
\subsection*{Runtime's default exception handler}
\frameSubsection{pictures/The\_Architect.png}{}


\begin{frame}{Default exception handler}
\begin{columns}
\begin{column}{0.5\textwidth}
\begin{itemize}
\item \localname{domain\_state->exn\_handler} points to a trap located before \funcname{entry}!
\item What installed this very first trap there?
\end{itemize}
\bigskip
\listocaml{../src/default/default.ml}{default/default.ml}
\bigskip
\inputminted[fontsize=\footnotesize]{text}{../src/default/default.out}
\end{column}
\begin{column}{0.5\textwidth}
\centering
\only<1>{\providecommand\step{0}%
% Runtime's default exception handler
%
\subsection*{Runtime's default exception handler}
\frameSubsection{pictures/The\_Architect.png}{}


\begin{frame}{Default exception handler}
\begin{columns}
\begin{column}{0.5\textwidth}
\begin{itemize}
\item \localname{domain\_state->exn\_handler} points to a trap located before \funcname{entry}!
\item What installed this very first trap there?
\end{itemize}
\bigskip
\listocaml{../src/default/default.ml}{default/default.ml}
\bigskip
\inputminted[fontsize=\footnotesize]{text}{../src/default/default.out}
\end{column}
\begin{column}{0.5\textwidth}
\centering
\only<1>{\providecommand\step{0}\input{diagrams/default/default.tex}}%
\only<2>{\providecommand\step{4}\input{diagrams/default/default.tex}}%
\end{column}
\end{columns}
\end{frame}


\begin{frame}{Startup}{How did we get there by the way?}
\begin{columns}
\begin{column}{0.45\textwidth}
\begin{itemize}
\item The entry point address of the binary is \funcname{main} (\funcname{\_start}) from the runtime
\item The program starts like a C program:
\begin{itemize}
\item \funcname{caml\_start\_program}
\item \funcname{caml\_startup\_common}
\item \funcname{caml\_startup\_exn}
\item \funcname{caml\_startup}
\item \funcname{caml\_main}
\item \funcname{main}
\end{itemize}
%\item \funcname{caml\_startup\_common}: initializes GC, domains \dots
% Also: codefrag, locale, custom opeartions, frame_descr, signals, stack
\item \funcname{caml\_start\_program} is the transition point between the C realm and the OCaml one
\end{itemize}
\bigskip
\listocaml{../src/default/default.ml}{default/default.ml}
\end{column}
\begin{column}{0.55\textwidth}
\listasm[lastline=24,highlightlines={22-24}]{src/ocaml/caml\_start\_program.s}{runtime/amd64.S:604}
\end{column}
\end{columns}
\end{frame}


\begin{frame}{Setup to jump into OCaml entry point}{\funcname{caml\_start\_program} initializes the main fiber}
\begin{columns}
\begin{column}{0.6\textwidth}
\only<1,3,5>{
\begin{itemize}
\item<1-> Stores information to allow switching from OCaml stack to the C stack
\item<3-> Constructs the default exception handler
\item<5-> Switches to the OCaml stack and calls \funcname{caml\_program}
\end{itemize}
\bigskip
\listocaml{../src/default/default.ml}{default/default.ml}
}%
\only<2>{
\listasm[firstline=22,lastline=41,highlightlines={25-27}]{src/ocaml/caml\_start\_program.s}{caml\_start\_program}
}%
\only<4>{
\mintasm[firstline=22,lastline=41,highlightlines={31-38}]{src/ocaml/caml\_start\_program.s}
\listasm[firstline=66,lastline=72]{src/ocaml/caml\_start\_program.s}{caml\_start\_program}
}%
\only<6>{
\listasm[firstline=22,lastline=41,highlightlines={39-41}]{src/ocaml/caml\_start\_program.s}{caml\_start\_program}
}%
\end{column}
\begin{column}{0.4\textwidth}
\centering
\only<1-2>{\providecommand\step{2}\input{diagrams/default/default.tex}}%
\only<3-5>{\providecommand\step{3}\input{diagrams/default/default.tex}}%
\onslide*<6->{\providecommand\step{4}\input{diagrams/default/default.tex}}%
\end{column}
\end{columns}
\end{frame}

\begin{frame}{Executing the OCaml program}{And raising an exception without handler}
\begin{columns}
\begin{column}{0.6\textwidth}
\only<1-3>{
\begin{itemize}
\item<1-> Execution continues with OCaml code
\begin{itemize}
\item \funcname{camlDefault\_\_main\_269}
\item \funcname{camlDefault\_\_entry}
\item \funcname{caml\_program}
\end{itemize}
\item<1-> \funcname{camlDefault\_\_main\_269} raises a \typename{Failure} exception
\item<1-> \typename{Failure} is caught by \funcname{caml\_start\_program}
\begin{itemize}
\item<1-> The handler set the 2nd LSB of the exception value to \funcarg{1}
\item<3-> And then forces \funcname{caml\_start\_program} to terminate
\end{itemize}
\end{itemize}
}
\bigskip
\only<1,3>{\listocaml[highlightlines=3]{../src/default/default.ml}{default/default.ml}}%
\only<2>{\listasm[firstline=66,lastline=72,highlightlines=69]{src/ocaml/caml\_start\_program.s}{caml\_start\_program}}%
\end{column}
\begin{column}{0.4\textwidth}
\centering
\providecommand\step{4}\input{diagrams/default/default.tex}
\end{column}
\end{columns}
\end{frame}
%\only<>{\listasm[firstline=47,lastline=65]{src/ocaml/caml\_start\_program.s}{caml\_start\_program}}


\begin{frame}{Back to C}{Clean up, complain and terminate}
\begin{columns}
\begin{column}{0.45\textwidth}
\begin{itemize}
\item Backtrace:
\begin{itemize}
\item \funcname{caml\_start\_program} $\rightarrow$ OCaml code
\item \funcname{caml\_startup\_common}
\item \funcname{caml\_startup\_exn}
\item \funcname{caml\_startup}
\item \funcname{caml\_main}
\item \funcname{main}
\end{itemize}
\smallskip
\item \funcname{caml\_startup} checks the return value of \funcname{caml\_startup\_exn}
\begin{itemize}
\item The return value has been specially marked by \funcname{caml\_start\_program} handler
\item Calls \funcname{caml\_fatal\_uncaught\_exception}
\end{itemize}
\item<2-> \funcname{caml\_fatal\_uncaught\_exception} either
\begin{itemize}
\item<2-> Calls the user custom uncaught exception handler, if set with \mintinline{ocaml}{Printexc.set_uncaught_exception_handler fn}\footnotemark
\item<2-> Or falls back to the default one
\item<2-> Which notifies about the uncaught exception
\end{itemize}
\item<4-> Terminates the program either with \funcname{abort} or with a non-zero exit code
\end{itemize}
\end{column}
\begin{column}{0.55\textwidth}
\only<1>{
\listc[firstline=84,lastline=89,highlightlines=88]{../ocaml/runtime/caml/mlvalues.h}{runtime/caml/mlvalues.h:84}
\listc[firstline=138,lastline=143,highlightlines={141-142}]{../ocaml/runtime/startup\_nat.c}{runtime/startup\_nat.c:138}
}%
\only<2>{
\listc[firstline=138,highlightlines={147,149}]{../ocaml/runtime/printexc.c}{runtime/printexc.c:138}
}%
\only<3>{
\listc[firstline=110,lastline=134,highlightlines=129]{../ocaml/runtime/printexc.c}{runtime/printexc.c:138}
}%
\only<4>{
\listc[firstline=138,highlightlines={152,154}]{../ocaml/runtime/printexc.c}{runtime/printexc.c:138}
}%
\end{column}
\end{columns}
\footnotetext[1]{\url{https://v2.ocaml.org/api/Printexc.html\#1\_Uncaughtexceptions}}
\end{frame}
}%
\only<2>{\providecommand\step{4}%
% Runtime's default exception handler
%
\subsection*{Runtime's default exception handler}
\frameSubsection{pictures/The\_Architect.png}{}


\begin{frame}{Default exception handler}
\begin{columns}
\begin{column}{0.5\textwidth}
\begin{itemize}
\item \localname{domain\_state->exn\_handler} points to a trap located before \funcname{entry}!
\item What installed this very first trap there?
\end{itemize}
\bigskip
\listocaml{../src/default/default.ml}{default/default.ml}
\bigskip
\inputminted[fontsize=\footnotesize]{text}{../src/default/default.out}
\end{column}
\begin{column}{0.5\textwidth}
\centering
\only<1>{\providecommand\step{0}\input{diagrams/default/default.tex}}%
\only<2>{\providecommand\step{4}\input{diagrams/default/default.tex}}%
\end{column}
\end{columns}
\end{frame}


\begin{frame}{Startup}{How did we get there by the way?}
\begin{columns}
\begin{column}{0.45\textwidth}
\begin{itemize}
\item The entry point address of the binary is \funcname{main} (\funcname{\_start}) from the runtime
\item The program starts like a C program:
\begin{itemize}
\item \funcname{caml\_start\_program}
\item \funcname{caml\_startup\_common}
\item \funcname{caml\_startup\_exn}
\item \funcname{caml\_startup}
\item \funcname{caml\_main}
\item \funcname{main}
\end{itemize}
%\item \funcname{caml\_startup\_common}: initializes GC, domains \dots
% Also: codefrag, locale, custom opeartions, frame_descr, signals, stack
\item \funcname{caml\_start\_program} is the transition point between the C realm and the OCaml one
\end{itemize}
\bigskip
\listocaml{../src/default/default.ml}{default/default.ml}
\end{column}
\begin{column}{0.55\textwidth}
\listasm[lastline=24,highlightlines={22-24}]{src/ocaml/caml\_start\_program.s}{runtime/amd64.S:604}
\end{column}
\end{columns}
\end{frame}


\begin{frame}{Setup to jump into OCaml entry point}{\funcname{caml\_start\_program} initializes the main fiber}
\begin{columns}
\begin{column}{0.6\textwidth}
\only<1,3,5>{
\begin{itemize}
\item<1-> Stores information to allow switching from OCaml stack to the C stack
\item<3-> Constructs the default exception handler
\item<5-> Switches to the OCaml stack and calls \funcname{caml\_program}
\end{itemize}
\bigskip
\listocaml{../src/default/default.ml}{default/default.ml}
}%
\only<2>{
\listasm[firstline=22,lastline=41,highlightlines={25-27}]{src/ocaml/caml\_start\_program.s}{caml\_start\_program}
}%
\only<4>{
\mintasm[firstline=22,lastline=41,highlightlines={31-38}]{src/ocaml/caml\_start\_program.s}
\listasm[firstline=66,lastline=72]{src/ocaml/caml\_start\_program.s}{caml\_start\_program}
}%
\only<6>{
\listasm[firstline=22,lastline=41,highlightlines={39-41}]{src/ocaml/caml\_start\_program.s}{caml\_start\_program}
}%
\end{column}
\begin{column}{0.4\textwidth}
\centering
\only<1-2>{\providecommand\step{2}\input{diagrams/default/default.tex}}%
\only<3-5>{\providecommand\step{3}\input{diagrams/default/default.tex}}%
\onslide*<6->{\providecommand\step{4}\input{diagrams/default/default.tex}}%
\end{column}
\end{columns}
\end{frame}

\begin{frame}{Executing the OCaml program}{And raising an exception without handler}
\begin{columns}
\begin{column}{0.6\textwidth}
\only<1-3>{
\begin{itemize}
\item<1-> Execution continues with OCaml code
\begin{itemize}
\item \funcname{camlDefault\_\_main\_269}
\item \funcname{camlDefault\_\_entry}
\item \funcname{caml\_program}
\end{itemize}
\item<1-> \funcname{camlDefault\_\_main\_269} raises a \typename{Failure} exception
\item<1-> \typename{Failure} is caught by \funcname{caml\_start\_program}
\begin{itemize}
\item<1-> The handler set the 2nd LSB of the exception value to \funcarg{1}
\item<3-> And then forces \funcname{caml\_start\_program} to terminate
\end{itemize}
\end{itemize}
}
\bigskip
\only<1,3>{\listocaml[highlightlines=3]{../src/default/default.ml}{default/default.ml}}%
\only<2>{\listasm[firstline=66,lastline=72,highlightlines=69]{src/ocaml/caml\_start\_program.s}{caml\_start\_program}}%
\end{column}
\begin{column}{0.4\textwidth}
\centering
\providecommand\step{4}\input{diagrams/default/default.tex}
\end{column}
\end{columns}
\end{frame}
%\only<>{\listasm[firstline=47,lastline=65]{src/ocaml/caml\_start\_program.s}{caml\_start\_program}}


\begin{frame}{Back to C}{Clean up, complain and terminate}
\begin{columns}
\begin{column}{0.45\textwidth}
\begin{itemize}
\item Backtrace:
\begin{itemize}
\item \funcname{caml\_start\_program} $\rightarrow$ OCaml code
\item \funcname{caml\_startup\_common}
\item \funcname{caml\_startup\_exn}
\item \funcname{caml\_startup}
\item \funcname{caml\_main}
\item \funcname{main}
\end{itemize}
\smallskip
\item \funcname{caml\_startup} checks the return value of \funcname{caml\_startup\_exn}
\begin{itemize}
\item The return value has been specially marked by \funcname{caml\_start\_program} handler
\item Calls \funcname{caml\_fatal\_uncaught\_exception}
\end{itemize}
\item<2-> \funcname{caml\_fatal\_uncaught\_exception} either
\begin{itemize}
\item<2-> Calls the user custom uncaught exception handler, if set with \mintinline{ocaml}{Printexc.set_uncaught_exception_handler fn}\footnotemark
\item<2-> Or falls back to the default one
\item<2-> Which notifies about the uncaught exception
\end{itemize}
\item<4-> Terminates the program either with \funcname{abort} or with a non-zero exit code
\end{itemize}
\end{column}
\begin{column}{0.55\textwidth}
\only<1>{
\listc[firstline=84,lastline=89,highlightlines=88]{../ocaml/runtime/caml/mlvalues.h}{runtime/caml/mlvalues.h:84}
\listc[firstline=138,lastline=143,highlightlines={141-142}]{../ocaml/runtime/startup\_nat.c}{runtime/startup\_nat.c:138}
}%
\only<2>{
\listc[firstline=138,highlightlines={147,149}]{../ocaml/runtime/printexc.c}{runtime/printexc.c:138}
}%
\only<3>{
\listc[firstline=110,lastline=134,highlightlines=129]{../ocaml/runtime/printexc.c}{runtime/printexc.c:138}
}%
\only<4>{
\listc[firstline=138,highlightlines={152,154}]{../ocaml/runtime/printexc.c}{runtime/printexc.c:138}
}%
\end{column}
\end{columns}
\footnotetext[1]{\url{https://v2.ocaml.org/api/Printexc.html\#1\_Uncaughtexceptions}}
\end{frame}
}%
\end{column}
\end{columns}
\end{frame}


\begin{frame}{Startup}{How did we get there by the way?}
\begin{columns}
\begin{column}{0.45\textwidth}
\begin{itemize}
\item The entry point address of the binary is \funcname{main} (\funcname{\_start}) from the runtime
\item The program starts like a C program:
\begin{itemize}
\item \funcname{caml\_start\_program}
\item \funcname{caml\_startup\_common}
\item \funcname{caml\_startup\_exn}
\item \funcname{caml\_startup}
\item \funcname{caml\_main}
\item \funcname{main}
\end{itemize}
%\item \funcname{caml\_startup\_common}: initializes GC, domains \dots
% Also: codefrag, locale, custom opeartions, frame_descr, signals, stack
\item \funcname{caml\_start\_program} is the transition point between the C realm and the OCaml one
\end{itemize}
\bigskip
\listocaml{../src/default/default.ml}{default/default.ml}
\end{column}
\begin{column}{0.55\textwidth}
\listasm[lastline=24,highlightlines={22-24}]{src/ocaml/caml\_start\_program.s}{runtime/amd64.S:604}
\end{column}
\end{columns}
\end{frame}


\begin{frame}{Setup to jump into OCaml entry point}{\funcname{caml\_start\_program} initializes the main fiber}
\begin{columns}
\begin{column}{0.6\textwidth}
\only<1,3,5>{
\begin{itemize}
\item<1-> Stores information to allow switching from OCaml stack to the C stack
\item<3-> Constructs the default exception handler
\item<5-> Switches to the OCaml stack and calls \funcname{caml\_program}
\end{itemize}
\bigskip
\listocaml{../src/default/default.ml}{default/default.ml}
}%
\only<2>{
\listasm[firstline=22,lastline=41,highlightlines={25-27}]{src/ocaml/caml\_start\_program.s}{caml\_start\_program}
}%
\only<4>{
\mintasm[firstline=22,lastline=41,highlightlines={31-38}]{src/ocaml/caml\_start\_program.s}
\listasm[firstline=66,lastline=72]{src/ocaml/caml\_start\_program.s}{caml\_start\_program}
}%
\only<6>{
\listasm[firstline=22,lastline=41,highlightlines={39-41}]{src/ocaml/caml\_start\_program.s}{caml\_start\_program}
}%
\end{column}
\begin{column}{0.4\textwidth}
\centering
\only<1-2>{\providecommand\step{2}%
% Runtime's default exception handler
%
\subsection*{Runtime's default exception handler}
\frameSubsection{pictures/The\_Architect.png}{}


\begin{frame}{Default exception handler}
\begin{columns}
\begin{column}{0.5\textwidth}
\begin{itemize}
\item \localname{domain\_state->exn\_handler} points to a trap located before \funcname{entry}!
\item What installed this very first trap there?
\end{itemize}
\bigskip
\listocaml{../src/default/default.ml}{default/default.ml}
\bigskip
\inputminted[fontsize=\footnotesize]{text}{../src/default/default.out}
\end{column}
\begin{column}{0.5\textwidth}
\centering
\only<1>{\providecommand\step{0}\input{diagrams/default/default.tex}}%
\only<2>{\providecommand\step{4}\input{diagrams/default/default.tex}}%
\end{column}
\end{columns}
\end{frame}


\begin{frame}{Startup}{How did we get there by the way?}
\begin{columns}
\begin{column}{0.45\textwidth}
\begin{itemize}
\item The entry point address of the binary is \funcname{main} (\funcname{\_start}) from the runtime
\item The program starts like a C program:
\begin{itemize}
\item \funcname{caml\_start\_program}
\item \funcname{caml\_startup\_common}
\item \funcname{caml\_startup\_exn}
\item \funcname{caml\_startup}
\item \funcname{caml\_main}
\item \funcname{main}
\end{itemize}
%\item \funcname{caml\_startup\_common}: initializes GC, domains \dots
% Also: codefrag, locale, custom opeartions, frame_descr, signals, stack
\item \funcname{caml\_start\_program} is the transition point between the C realm and the OCaml one
\end{itemize}
\bigskip
\listocaml{../src/default/default.ml}{default/default.ml}
\end{column}
\begin{column}{0.55\textwidth}
\listasm[lastline=24,highlightlines={22-24}]{src/ocaml/caml\_start\_program.s}{runtime/amd64.S:604}
\end{column}
\end{columns}
\end{frame}


\begin{frame}{Setup to jump into OCaml entry point}{\funcname{caml\_start\_program} initializes the main fiber}
\begin{columns}
\begin{column}{0.6\textwidth}
\only<1,3,5>{
\begin{itemize}
\item<1-> Stores information to allow switching from OCaml stack to the C stack
\item<3-> Constructs the default exception handler
\item<5-> Switches to the OCaml stack and calls \funcname{caml\_program}
\end{itemize}
\bigskip
\listocaml{../src/default/default.ml}{default/default.ml}
}%
\only<2>{
\listasm[firstline=22,lastline=41,highlightlines={25-27}]{src/ocaml/caml\_start\_program.s}{caml\_start\_program}
}%
\only<4>{
\mintasm[firstline=22,lastline=41,highlightlines={31-38}]{src/ocaml/caml\_start\_program.s}
\listasm[firstline=66,lastline=72]{src/ocaml/caml\_start\_program.s}{caml\_start\_program}
}%
\only<6>{
\listasm[firstline=22,lastline=41,highlightlines={39-41}]{src/ocaml/caml\_start\_program.s}{caml\_start\_program}
}%
\end{column}
\begin{column}{0.4\textwidth}
\centering
\only<1-2>{\providecommand\step{2}\input{diagrams/default/default.tex}}%
\only<3-5>{\providecommand\step{3}\input{diagrams/default/default.tex}}%
\onslide*<6->{\providecommand\step{4}\input{diagrams/default/default.tex}}%
\end{column}
\end{columns}
\end{frame}

\begin{frame}{Executing the OCaml program}{And raising an exception without handler}
\begin{columns}
\begin{column}{0.6\textwidth}
\only<1-3>{
\begin{itemize}
\item<1-> Execution continues with OCaml code
\begin{itemize}
\item \funcname{camlDefault\_\_main\_269}
\item \funcname{camlDefault\_\_entry}
\item \funcname{caml\_program}
\end{itemize}
\item<1-> \funcname{camlDefault\_\_main\_269} raises a \typename{Failure} exception
\item<1-> \typename{Failure} is caught by \funcname{caml\_start\_program}
\begin{itemize}
\item<1-> The handler set the 2nd LSB of the exception value to \funcarg{1}
\item<3-> And then forces \funcname{caml\_start\_program} to terminate
\end{itemize}
\end{itemize}
}
\bigskip
\only<1,3>{\listocaml[highlightlines=3]{../src/default/default.ml}{default/default.ml}}%
\only<2>{\listasm[firstline=66,lastline=72,highlightlines=69]{src/ocaml/caml\_start\_program.s}{caml\_start\_program}}%
\end{column}
\begin{column}{0.4\textwidth}
\centering
\providecommand\step{4}\input{diagrams/default/default.tex}
\end{column}
\end{columns}
\end{frame}
%\only<>{\listasm[firstline=47,lastline=65]{src/ocaml/caml\_start\_program.s}{caml\_start\_program}}


\begin{frame}{Back to C}{Clean up, complain and terminate}
\begin{columns}
\begin{column}{0.45\textwidth}
\begin{itemize}
\item Backtrace:
\begin{itemize}
\item \funcname{caml\_start\_program} $\rightarrow$ OCaml code
\item \funcname{caml\_startup\_common}
\item \funcname{caml\_startup\_exn}
\item \funcname{caml\_startup}
\item \funcname{caml\_main}
\item \funcname{main}
\end{itemize}
\smallskip
\item \funcname{caml\_startup} checks the return value of \funcname{caml\_startup\_exn}
\begin{itemize}
\item The return value has been specially marked by \funcname{caml\_start\_program} handler
\item Calls \funcname{caml\_fatal\_uncaught\_exception}
\end{itemize}
\item<2-> \funcname{caml\_fatal\_uncaught\_exception} either
\begin{itemize}
\item<2-> Calls the user custom uncaught exception handler, if set with \mintinline{ocaml}{Printexc.set_uncaught_exception_handler fn}\footnotemark
\item<2-> Or falls back to the default one
\item<2-> Which notifies about the uncaught exception
\end{itemize}
\item<4-> Terminates the program either with \funcname{abort} or with a non-zero exit code
\end{itemize}
\end{column}
\begin{column}{0.55\textwidth}
\only<1>{
\listc[firstline=84,lastline=89,highlightlines=88]{../ocaml/runtime/caml/mlvalues.h}{runtime/caml/mlvalues.h:84}
\listc[firstline=138,lastline=143,highlightlines={141-142}]{../ocaml/runtime/startup\_nat.c}{runtime/startup\_nat.c:138}
}%
\only<2>{
\listc[firstline=138,highlightlines={147,149}]{../ocaml/runtime/printexc.c}{runtime/printexc.c:138}
}%
\only<3>{
\listc[firstline=110,lastline=134,highlightlines=129]{../ocaml/runtime/printexc.c}{runtime/printexc.c:138}
}%
\only<4>{
\listc[firstline=138,highlightlines={152,154}]{../ocaml/runtime/printexc.c}{runtime/printexc.c:138}
}%
\end{column}
\end{columns}
\footnotetext[1]{\url{https://v2.ocaml.org/api/Printexc.html\#1\_Uncaughtexceptions}}
\end{frame}
}%
\only<3-5>{\providecommand\step{3}%
% Runtime's default exception handler
%
\subsection*{Runtime's default exception handler}
\frameSubsection{pictures/The\_Architect.png}{}


\begin{frame}{Default exception handler}
\begin{columns}
\begin{column}{0.5\textwidth}
\begin{itemize}
\item \localname{domain\_state->exn\_handler} points to a trap located before \funcname{entry}!
\item What installed this very first trap there?
\end{itemize}
\bigskip
\listocaml{../src/default/default.ml}{default/default.ml}
\bigskip
\inputminted[fontsize=\footnotesize]{text}{../src/default/default.out}
\end{column}
\begin{column}{0.5\textwidth}
\centering
\only<1>{\providecommand\step{0}\input{diagrams/default/default.tex}}%
\only<2>{\providecommand\step{4}\input{diagrams/default/default.tex}}%
\end{column}
\end{columns}
\end{frame}


\begin{frame}{Startup}{How did we get there by the way?}
\begin{columns}
\begin{column}{0.45\textwidth}
\begin{itemize}
\item The entry point address of the binary is \funcname{main} (\funcname{\_start}) from the runtime
\item The program starts like a C program:
\begin{itemize}
\item \funcname{caml\_start\_program}
\item \funcname{caml\_startup\_common}
\item \funcname{caml\_startup\_exn}
\item \funcname{caml\_startup}
\item \funcname{caml\_main}
\item \funcname{main}
\end{itemize}
%\item \funcname{caml\_startup\_common}: initializes GC, domains \dots
% Also: codefrag, locale, custom opeartions, frame_descr, signals, stack
\item \funcname{caml\_start\_program} is the transition point between the C realm and the OCaml one
\end{itemize}
\bigskip
\listocaml{../src/default/default.ml}{default/default.ml}
\end{column}
\begin{column}{0.55\textwidth}
\listasm[lastline=24,highlightlines={22-24}]{src/ocaml/caml\_start\_program.s}{runtime/amd64.S:604}
\end{column}
\end{columns}
\end{frame}


\begin{frame}{Setup to jump into OCaml entry point}{\funcname{caml\_start\_program} initializes the main fiber}
\begin{columns}
\begin{column}{0.6\textwidth}
\only<1,3,5>{
\begin{itemize}
\item<1-> Stores information to allow switching from OCaml stack to the C stack
\item<3-> Constructs the default exception handler
\item<5-> Switches to the OCaml stack and calls \funcname{caml\_program}
\end{itemize}
\bigskip
\listocaml{../src/default/default.ml}{default/default.ml}
}%
\only<2>{
\listasm[firstline=22,lastline=41,highlightlines={25-27}]{src/ocaml/caml\_start\_program.s}{caml\_start\_program}
}%
\only<4>{
\mintasm[firstline=22,lastline=41,highlightlines={31-38}]{src/ocaml/caml\_start\_program.s}
\listasm[firstline=66,lastline=72]{src/ocaml/caml\_start\_program.s}{caml\_start\_program}
}%
\only<6>{
\listasm[firstline=22,lastline=41,highlightlines={39-41}]{src/ocaml/caml\_start\_program.s}{caml\_start\_program}
}%
\end{column}
\begin{column}{0.4\textwidth}
\centering
\only<1-2>{\providecommand\step{2}\input{diagrams/default/default.tex}}%
\only<3-5>{\providecommand\step{3}\input{diagrams/default/default.tex}}%
\onslide*<6->{\providecommand\step{4}\input{diagrams/default/default.tex}}%
\end{column}
\end{columns}
\end{frame}

\begin{frame}{Executing the OCaml program}{And raising an exception without handler}
\begin{columns}
\begin{column}{0.6\textwidth}
\only<1-3>{
\begin{itemize}
\item<1-> Execution continues with OCaml code
\begin{itemize}
\item \funcname{camlDefault\_\_main\_269}
\item \funcname{camlDefault\_\_entry}
\item \funcname{caml\_program}
\end{itemize}
\item<1-> \funcname{camlDefault\_\_main\_269} raises a \typename{Failure} exception
\item<1-> \typename{Failure} is caught by \funcname{caml\_start\_program}
\begin{itemize}
\item<1-> The handler set the 2nd LSB of the exception value to \funcarg{1}
\item<3-> And then forces \funcname{caml\_start\_program} to terminate
\end{itemize}
\end{itemize}
}
\bigskip
\only<1,3>{\listocaml[highlightlines=3]{../src/default/default.ml}{default/default.ml}}%
\only<2>{\listasm[firstline=66,lastline=72,highlightlines=69]{src/ocaml/caml\_start\_program.s}{caml\_start\_program}}%
\end{column}
\begin{column}{0.4\textwidth}
\centering
\providecommand\step{4}\input{diagrams/default/default.tex}
\end{column}
\end{columns}
\end{frame}
%\only<>{\listasm[firstline=47,lastline=65]{src/ocaml/caml\_start\_program.s}{caml\_start\_program}}


\begin{frame}{Back to C}{Clean up, complain and terminate}
\begin{columns}
\begin{column}{0.45\textwidth}
\begin{itemize}
\item Backtrace:
\begin{itemize}
\item \funcname{caml\_start\_program} $\rightarrow$ OCaml code
\item \funcname{caml\_startup\_common}
\item \funcname{caml\_startup\_exn}
\item \funcname{caml\_startup}
\item \funcname{caml\_main}
\item \funcname{main}
\end{itemize}
\smallskip
\item \funcname{caml\_startup} checks the return value of \funcname{caml\_startup\_exn}
\begin{itemize}
\item The return value has been specially marked by \funcname{caml\_start\_program} handler
\item Calls \funcname{caml\_fatal\_uncaught\_exception}
\end{itemize}
\item<2-> \funcname{caml\_fatal\_uncaught\_exception} either
\begin{itemize}
\item<2-> Calls the user custom uncaught exception handler, if set with \mintinline{ocaml}{Printexc.set_uncaught_exception_handler fn}\footnotemark
\item<2-> Or falls back to the default one
\item<2-> Which notifies about the uncaught exception
\end{itemize}
\item<4-> Terminates the program either with \funcname{abort} or with a non-zero exit code
\end{itemize}
\end{column}
\begin{column}{0.55\textwidth}
\only<1>{
\listc[firstline=84,lastline=89,highlightlines=88]{../ocaml/runtime/caml/mlvalues.h}{runtime/caml/mlvalues.h:84}
\listc[firstline=138,lastline=143,highlightlines={141-142}]{../ocaml/runtime/startup\_nat.c}{runtime/startup\_nat.c:138}
}%
\only<2>{
\listc[firstline=138,highlightlines={147,149}]{../ocaml/runtime/printexc.c}{runtime/printexc.c:138}
}%
\only<3>{
\listc[firstline=110,lastline=134,highlightlines=129]{../ocaml/runtime/printexc.c}{runtime/printexc.c:138}
}%
\only<4>{
\listc[firstline=138,highlightlines={152,154}]{../ocaml/runtime/printexc.c}{runtime/printexc.c:138}
}%
\end{column}
\end{columns}
\footnotetext[1]{\url{https://v2.ocaml.org/api/Printexc.html\#1\_Uncaughtexceptions}}
\end{frame}
}%
\onslide*<6->{\providecommand\step{4}%
% Runtime's default exception handler
%
\subsection*{Runtime's default exception handler}
\frameSubsection{pictures/The\_Architect.png}{}


\begin{frame}{Default exception handler}
\begin{columns}
\begin{column}{0.5\textwidth}
\begin{itemize}
\item \localname{domain\_state->exn\_handler} points to a trap located before \funcname{entry}!
\item What installed this very first trap there?
\end{itemize}
\bigskip
\listocaml{../src/default/default.ml}{default/default.ml}
\bigskip
\inputminted[fontsize=\footnotesize]{text}{../src/default/default.out}
\end{column}
\begin{column}{0.5\textwidth}
\centering
\only<1>{\providecommand\step{0}\input{diagrams/default/default.tex}}%
\only<2>{\providecommand\step{4}\input{diagrams/default/default.tex}}%
\end{column}
\end{columns}
\end{frame}


\begin{frame}{Startup}{How did we get there by the way?}
\begin{columns}
\begin{column}{0.45\textwidth}
\begin{itemize}
\item The entry point address of the binary is \funcname{main} (\funcname{\_start}) from the runtime
\item The program starts like a C program:
\begin{itemize}
\item \funcname{caml\_start\_program}
\item \funcname{caml\_startup\_common}
\item \funcname{caml\_startup\_exn}
\item \funcname{caml\_startup}
\item \funcname{caml\_main}
\item \funcname{main}
\end{itemize}
%\item \funcname{caml\_startup\_common}: initializes GC, domains \dots
% Also: codefrag, locale, custom opeartions, frame_descr, signals, stack
\item \funcname{caml\_start\_program} is the transition point between the C realm and the OCaml one
\end{itemize}
\bigskip
\listocaml{../src/default/default.ml}{default/default.ml}
\end{column}
\begin{column}{0.55\textwidth}
\listasm[lastline=24,highlightlines={22-24}]{src/ocaml/caml\_start\_program.s}{runtime/amd64.S:604}
\end{column}
\end{columns}
\end{frame}


\begin{frame}{Setup to jump into OCaml entry point}{\funcname{caml\_start\_program} initializes the main fiber}
\begin{columns}
\begin{column}{0.6\textwidth}
\only<1,3,5>{
\begin{itemize}
\item<1-> Stores information to allow switching from OCaml stack to the C stack
\item<3-> Constructs the default exception handler
\item<5-> Switches to the OCaml stack and calls \funcname{caml\_program}
\end{itemize}
\bigskip
\listocaml{../src/default/default.ml}{default/default.ml}
}%
\only<2>{
\listasm[firstline=22,lastline=41,highlightlines={25-27}]{src/ocaml/caml\_start\_program.s}{caml\_start\_program}
}%
\only<4>{
\mintasm[firstline=22,lastline=41,highlightlines={31-38}]{src/ocaml/caml\_start\_program.s}
\listasm[firstline=66,lastline=72]{src/ocaml/caml\_start\_program.s}{caml\_start\_program}
}%
\only<6>{
\listasm[firstline=22,lastline=41,highlightlines={39-41}]{src/ocaml/caml\_start\_program.s}{caml\_start\_program}
}%
\end{column}
\begin{column}{0.4\textwidth}
\centering
\only<1-2>{\providecommand\step{2}\input{diagrams/default/default.tex}}%
\only<3-5>{\providecommand\step{3}\input{diagrams/default/default.tex}}%
\onslide*<6->{\providecommand\step{4}\input{diagrams/default/default.tex}}%
\end{column}
\end{columns}
\end{frame}

\begin{frame}{Executing the OCaml program}{And raising an exception without handler}
\begin{columns}
\begin{column}{0.6\textwidth}
\only<1-3>{
\begin{itemize}
\item<1-> Execution continues with OCaml code
\begin{itemize}
\item \funcname{camlDefault\_\_main\_269}
\item \funcname{camlDefault\_\_entry}
\item \funcname{caml\_program}
\end{itemize}
\item<1-> \funcname{camlDefault\_\_main\_269} raises a \typename{Failure} exception
\item<1-> \typename{Failure} is caught by \funcname{caml\_start\_program}
\begin{itemize}
\item<1-> The handler set the 2nd LSB of the exception value to \funcarg{1}
\item<3-> And then forces \funcname{caml\_start\_program} to terminate
\end{itemize}
\end{itemize}
}
\bigskip
\only<1,3>{\listocaml[highlightlines=3]{../src/default/default.ml}{default/default.ml}}%
\only<2>{\listasm[firstline=66,lastline=72,highlightlines=69]{src/ocaml/caml\_start\_program.s}{caml\_start\_program}}%
\end{column}
\begin{column}{0.4\textwidth}
\centering
\providecommand\step{4}\input{diagrams/default/default.tex}
\end{column}
\end{columns}
\end{frame}
%\only<>{\listasm[firstline=47,lastline=65]{src/ocaml/caml\_start\_program.s}{caml\_start\_program}}


\begin{frame}{Back to C}{Clean up, complain and terminate}
\begin{columns}
\begin{column}{0.45\textwidth}
\begin{itemize}
\item Backtrace:
\begin{itemize}
\item \funcname{caml\_start\_program} $\rightarrow$ OCaml code
\item \funcname{caml\_startup\_common}
\item \funcname{caml\_startup\_exn}
\item \funcname{caml\_startup}
\item \funcname{caml\_main}
\item \funcname{main}
\end{itemize}
\smallskip
\item \funcname{caml\_startup} checks the return value of \funcname{caml\_startup\_exn}
\begin{itemize}
\item The return value has been specially marked by \funcname{caml\_start\_program} handler
\item Calls \funcname{caml\_fatal\_uncaught\_exception}
\end{itemize}
\item<2-> \funcname{caml\_fatal\_uncaught\_exception} either
\begin{itemize}
\item<2-> Calls the user custom uncaught exception handler, if set with \mintinline{ocaml}{Printexc.set_uncaught_exception_handler fn}\footnotemark
\item<2-> Or falls back to the default one
\item<2-> Which notifies about the uncaught exception
\end{itemize}
\item<4-> Terminates the program either with \funcname{abort} or with a non-zero exit code
\end{itemize}
\end{column}
\begin{column}{0.55\textwidth}
\only<1>{
\listc[firstline=84,lastline=89,highlightlines=88]{../ocaml/runtime/caml/mlvalues.h}{runtime/caml/mlvalues.h:84}
\listc[firstline=138,lastline=143,highlightlines={141-142}]{../ocaml/runtime/startup\_nat.c}{runtime/startup\_nat.c:138}
}%
\only<2>{
\listc[firstline=138,highlightlines={147,149}]{../ocaml/runtime/printexc.c}{runtime/printexc.c:138}
}%
\only<3>{
\listc[firstline=110,lastline=134,highlightlines=129]{../ocaml/runtime/printexc.c}{runtime/printexc.c:138}
}%
\only<4>{
\listc[firstline=138,highlightlines={152,154}]{../ocaml/runtime/printexc.c}{runtime/printexc.c:138}
}%
\end{column}
\end{columns}
\footnotetext[1]{\url{https://v2.ocaml.org/api/Printexc.html\#1\_Uncaughtexceptions}}
\end{frame}
}%
\end{column}
\end{columns}
\end{frame}

\begin{frame}{Executing the OCaml program}{And raising an exception without handler}
\begin{columns}
\begin{column}{0.6\textwidth}
\only<1-3>{
\begin{itemize}
\item<1-> Execution continues with OCaml code
\begin{itemize}
\item \funcname{camlDefault\_\_main\_269}
\item \funcname{camlDefault\_\_entry}
\item \funcname{caml\_program}
\end{itemize}
\item<1-> \funcname{camlDefault\_\_main\_269} raises a \typename{Failure} exception
\item<1-> \typename{Failure} is caught by \funcname{caml\_start\_program}
\begin{itemize}
\item<1-> The handler set the 2nd LSB of the exception value to \funcarg{1}
\item<3-> And then forces \funcname{caml\_start\_program} to terminate
\end{itemize}
\end{itemize}
}
\bigskip
\only<1,3>{\listocaml[highlightlines=3]{../src/default/default.ml}{default/default.ml}}%
\only<2>{\listasm[firstline=66,lastline=72,highlightlines=69]{src/ocaml/caml\_start\_program.s}{caml\_start\_program}}%
\end{column}
\begin{column}{0.4\textwidth}
\centering
\providecommand\step{4}%
% Runtime's default exception handler
%
\subsection*{Runtime's default exception handler}
\frameSubsection{pictures/The\_Architect.png}{}


\begin{frame}{Default exception handler}
\begin{columns}
\begin{column}{0.5\textwidth}
\begin{itemize}
\item \localname{domain\_state->exn\_handler} points to a trap located before \funcname{entry}!
\item What installed this very first trap there?
\end{itemize}
\bigskip
\listocaml{../src/default/default.ml}{default/default.ml}
\bigskip
\inputminted[fontsize=\footnotesize]{text}{../src/default/default.out}
\end{column}
\begin{column}{0.5\textwidth}
\centering
\only<1>{\providecommand\step{0}\input{diagrams/default/default.tex}}%
\only<2>{\providecommand\step{4}\input{diagrams/default/default.tex}}%
\end{column}
\end{columns}
\end{frame}


\begin{frame}{Startup}{How did we get there by the way?}
\begin{columns}
\begin{column}{0.45\textwidth}
\begin{itemize}
\item The entry point address of the binary is \funcname{main} (\funcname{\_start}) from the runtime
\item The program starts like a C program:
\begin{itemize}
\item \funcname{caml\_start\_program}
\item \funcname{caml\_startup\_common}
\item \funcname{caml\_startup\_exn}
\item \funcname{caml\_startup}
\item \funcname{caml\_main}
\item \funcname{main}
\end{itemize}
%\item \funcname{caml\_startup\_common}: initializes GC, domains \dots
% Also: codefrag, locale, custom opeartions, frame_descr, signals, stack
\item \funcname{caml\_start\_program} is the transition point between the C realm and the OCaml one
\end{itemize}
\bigskip
\listocaml{../src/default/default.ml}{default/default.ml}
\end{column}
\begin{column}{0.55\textwidth}
\listasm[lastline=24,highlightlines={22-24}]{src/ocaml/caml\_start\_program.s}{runtime/amd64.S:604}
\end{column}
\end{columns}
\end{frame}


\begin{frame}{Setup to jump into OCaml entry point}{\funcname{caml\_start\_program} initializes the main fiber}
\begin{columns}
\begin{column}{0.6\textwidth}
\only<1,3,5>{
\begin{itemize}
\item<1-> Stores information to allow switching from OCaml stack to the C stack
\item<3-> Constructs the default exception handler
\item<5-> Switches to the OCaml stack and calls \funcname{caml\_program}
\end{itemize}
\bigskip
\listocaml{../src/default/default.ml}{default/default.ml}
}%
\only<2>{
\listasm[firstline=22,lastline=41,highlightlines={25-27}]{src/ocaml/caml\_start\_program.s}{caml\_start\_program}
}%
\only<4>{
\mintasm[firstline=22,lastline=41,highlightlines={31-38}]{src/ocaml/caml\_start\_program.s}
\listasm[firstline=66,lastline=72]{src/ocaml/caml\_start\_program.s}{caml\_start\_program}
}%
\only<6>{
\listasm[firstline=22,lastline=41,highlightlines={39-41}]{src/ocaml/caml\_start\_program.s}{caml\_start\_program}
}%
\end{column}
\begin{column}{0.4\textwidth}
\centering
\only<1-2>{\providecommand\step{2}\input{diagrams/default/default.tex}}%
\only<3-5>{\providecommand\step{3}\input{diagrams/default/default.tex}}%
\onslide*<6->{\providecommand\step{4}\input{diagrams/default/default.tex}}%
\end{column}
\end{columns}
\end{frame}

\begin{frame}{Executing the OCaml program}{And raising an exception without handler}
\begin{columns}
\begin{column}{0.6\textwidth}
\only<1-3>{
\begin{itemize}
\item<1-> Execution continues with OCaml code
\begin{itemize}
\item \funcname{camlDefault\_\_main\_269}
\item \funcname{camlDefault\_\_entry}
\item \funcname{caml\_program}
\end{itemize}
\item<1-> \funcname{camlDefault\_\_main\_269} raises a \typename{Failure} exception
\item<1-> \typename{Failure} is caught by \funcname{caml\_start\_program}
\begin{itemize}
\item<1-> The handler set the 2nd LSB of the exception value to \funcarg{1}
\item<3-> And then forces \funcname{caml\_start\_program} to terminate
\end{itemize}
\end{itemize}
}
\bigskip
\only<1,3>{\listocaml[highlightlines=3]{../src/default/default.ml}{default/default.ml}}%
\only<2>{\listasm[firstline=66,lastline=72,highlightlines=69]{src/ocaml/caml\_start\_program.s}{caml\_start\_program}}%
\end{column}
\begin{column}{0.4\textwidth}
\centering
\providecommand\step{4}\input{diagrams/default/default.tex}
\end{column}
\end{columns}
\end{frame}
%\only<>{\listasm[firstline=47,lastline=65]{src/ocaml/caml\_start\_program.s}{caml\_start\_program}}


\begin{frame}{Back to C}{Clean up, complain and terminate}
\begin{columns}
\begin{column}{0.45\textwidth}
\begin{itemize}
\item Backtrace:
\begin{itemize}
\item \funcname{caml\_start\_program} $\rightarrow$ OCaml code
\item \funcname{caml\_startup\_common}
\item \funcname{caml\_startup\_exn}
\item \funcname{caml\_startup}
\item \funcname{caml\_main}
\item \funcname{main}
\end{itemize}
\smallskip
\item \funcname{caml\_startup} checks the return value of \funcname{caml\_startup\_exn}
\begin{itemize}
\item The return value has been specially marked by \funcname{caml\_start\_program} handler
\item Calls \funcname{caml\_fatal\_uncaught\_exception}
\end{itemize}
\item<2-> \funcname{caml\_fatal\_uncaught\_exception} either
\begin{itemize}
\item<2-> Calls the user custom uncaught exception handler, if set with \mintinline{ocaml}{Printexc.set_uncaught_exception_handler fn}\footnotemark
\item<2-> Or falls back to the default one
\item<2-> Which notifies about the uncaught exception
\end{itemize}
\item<4-> Terminates the program either with \funcname{abort} or with a non-zero exit code
\end{itemize}
\end{column}
\begin{column}{0.55\textwidth}
\only<1>{
\listc[firstline=84,lastline=89,highlightlines=88]{../ocaml/runtime/caml/mlvalues.h}{runtime/caml/mlvalues.h:84}
\listc[firstline=138,lastline=143,highlightlines={141-142}]{../ocaml/runtime/startup\_nat.c}{runtime/startup\_nat.c:138}
}%
\only<2>{
\listc[firstline=138,highlightlines={147,149}]{../ocaml/runtime/printexc.c}{runtime/printexc.c:138}
}%
\only<3>{
\listc[firstline=110,lastline=134,highlightlines=129]{../ocaml/runtime/printexc.c}{runtime/printexc.c:138}
}%
\only<4>{
\listc[firstline=138,highlightlines={152,154}]{../ocaml/runtime/printexc.c}{runtime/printexc.c:138}
}%
\end{column}
\end{columns}
\footnotetext[1]{\url{https://v2.ocaml.org/api/Printexc.html\#1\_Uncaughtexceptions}}
\end{frame}

\end{column}
\end{columns}
\end{frame}
%\only<>{\listasm[firstline=47,lastline=65]{src/ocaml/caml\_start\_program.s}{caml\_start\_program}}


\begin{frame}{Back to C}{Clean up, complain and terminate}
\begin{columns}
\begin{column}{0.45\textwidth}
\begin{itemize}
\item Backtrace:
\begin{itemize}
\item \funcname{caml\_start\_program} $\rightarrow$ OCaml code
\item \funcname{caml\_startup\_common}
\item \funcname{caml\_startup\_exn}
\item \funcname{caml\_startup}
\item \funcname{caml\_main}
\item \funcname{main}
\end{itemize}
\smallskip
\item \funcname{caml\_startup} checks the return value of \funcname{caml\_startup\_exn}
\begin{itemize}
\item The return value has been specially marked by \funcname{caml\_start\_program} handler
\item Calls \funcname{caml\_fatal\_uncaught\_exception}
\end{itemize}
\item<2-> \funcname{caml\_fatal\_uncaught\_exception} either
\begin{itemize}
\item<2-> Calls the user custom uncaught exception handler, if set with \mintinline{ocaml}{Printexc.set_uncaught_exception_handler fn}\footnotemark
\item<2-> Or falls back to the default one
\item<2-> Which notifies about the uncaught exception
\end{itemize}
\item<4-> Terminates the program either with \funcname{abort} or with a non-zero exit code
\end{itemize}
\end{column}
\begin{column}{0.55\textwidth}
\only<1>{
\listc[firstline=84,lastline=89,highlightlines=88]{../ocaml/runtime/caml/mlvalues.h}{runtime/caml/mlvalues.h:84}
\listc[firstline=138,lastline=143,highlightlines={141-142}]{../ocaml/runtime/startup\_nat.c}{runtime/startup\_nat.c:138}
}%
\only<2>{
\listc[firstline=138,highlightlines={147,149}]{../ocaml/runtime/printexc.c}{runtime/printexc.c:138}
}%
\only<3>{
\listc[firstline=110,lastline=134,highlightlines=129]{../ocaml/runtime/printexc.c}{runtime/printexc.c:138}
}%
\only<4>{
\listc[firstline=138,highlightlines={152,154}]{../ocaml/runtime/printexc.c}{runtime/printexc.c:138}
}%
\end{column}
\end{columns}
\footnotetext[1]{\url{https://v2.ocaml.org/api/Printexc.html\#1\_Uncaughtexceptions}}
\end{frame}
}%
\end{column}
\end{columns}
\end{frame}


\begin{frame}{Startup}{How did we get there by the way?}
\begin{columns}
\begin{column}{0.45\textwidth}
\begin{itemize}
\item The entry point address of the binary is \funcname{main} (\funcname{\_start}) from the runtime
\item The program starts like a C program:
\begin{itemize}
\item \funcname{caml\_start\_program}
\item \funcname{caml\_startup\_common}
\item \funcname{caml\_startup\_exn}
\item \funcname{caml\_startup}
\item \funcname{caml\_main}
\item \funcname{main}
\end{itemize}
%\item \funcname{caml\_startup\_common}: initializes GC, domains \dots
% Also: codefrag, locale, custom opeartions, frame_descr, signals, stack
\item \funcname{caml\_start\_program} is the transition point between the C realm and the OCaml one
\end{itemize}
\bigskip
\listocaml{../src/default/default.ml}{default/default.ml}
\end{column}
\begin{column}{0.55\textwidth}
\listasm[lastline=24,highlightlines={22-24}]{src/ocaml/caml\_start\_program.s}{runtime/amd64.S:604}
\end{column}
\end{columns}
\end{frame}


\begin{frame}{Setup to jump into OCaml entry point}{\funcname{caml\_start\_program} initializes the main fiber}
\begin{columns}
\begin{column}{0.6\textwidth}
\only<1,3,5>{
\begin{itemize}
\item<1-> Stores information to allow switching from OCaml stack to the C stack
\item<3-> Constructs the default exception handler
\item<5-> Switches to the OCaml stack and calls \funcname{caml\_program}
\end{itemize}
\bigskip
\listocaml{../src/default/default.ml}{default/default.ml}
}%
\only<2>{
\listasm[firstline=22,lastline=41,highlightlines={25-27}]{src/ocaml/caml\_start\_program.s}{caml\_start\_program}
}%
\only<4>{
\mintasm[firstline=22,lastline=41,highlightlines={31-38}]{src/ocaml/caml\_start\_program.s}
\listasm[firstline=66,lastline=72]{src/ocaml/caml\_start\_program.s}{caml\_start\_program}
}%
\only<6>{
\listasm[firstline=22,lastline=41,highlightlines={39-41}]{src/ocaml/caml\_start\_program.s}{caml\_start\_program}
}%
\end{column}
\begin{column}{0.4\textwidth}
\centering
\only<1-2>{\providecommand\step{2}%
% Runtime's default exception handler
%
\subsection*{Runtime's default exception handler}
\frameSubsection{pictures/The\_Architect.png}{}


\begin{frame}{Default exception handler}
\begin{columns}
\begin{column}{0.5\textwidth}
\begin{itemize}
\item \localname{domain\_state->exn\_handler} points to a trap located before \funcname{entry}!
\item What installed this very first trap there?
\end{itemize}
\bigskip
\listocaml{../src/default/default.ml}{default/default.ml}
\bigskip
\inputminted[fontsize=\footnotesize]{text}{../src/default/default.out}
\end{column}
\begin{column}{0.5\textwidth}
\centering
\only<1>{\providecommand\step{0}%
% Runtime's default exception handler
%
\subsection*{Runtime's default exception handler}
\frameSubsection{pictures/The\_Architect.png}{}


\begin{frame}{Default exception handler}
\begin{columns}
\begin{column}{0.5\textwidth}
\begin{itemize}
\item \localname{domain\_state->exn\_handler} points to a trap located before \funcname{entry}!
\item What installed this very first trap there?
\end{itemize}
\bigskip
\listocaml{../src/default/default.ml}{default/default.ml}
\bigskip
\inputminted[fontsize=\footnotesize]{text}{../src/default/default.out}
\end{column}
\begin{column}{0.5\textwidth}
\centering
\only<1>{\providecommand\step{0}\input{diagrams/default/default.tex}}%
\only<2>{\providecommand\step{4}\input{diagrams/default/default.tex}}%
\end{column}
\end{columns}
\end{frame}


\begin{frame}{Startup}{How did we get there by the way?}
\begin{columns}
\begin{column}{0.45\textwidth}
\begin{itemize}
\item The entry point address of the binary is \funcname{main} (\funcname{\_start}) from the runtime
\item The program starts like a C program:
\begin{itemize}
\item \funcname{caml\_start\_program}
\item \funcname{caml\_startup\_common}
\item \funcname{caml\_startup\_exn}
\item \funcname{caml\_startup}
\item \funcname{caml\_main}
\item \funcname{main}
\end{itemize}
%\item \funcname{caml\_startup\_common}: initializes GC, domains \dots
% Also: codefrag, locale, custom opeartions, frame_descr, signals, stack
\item \funcname{caml\_start\_program} is the transition point between the C realm and the OCaml one
\end{itemize}
\bigskip
\listocaml{../src/default/default.ml}{default/default.ml}
\end{column}
\begin{column}{0.55\textwidth}
\listasm[lastline=24,highlightlines={22-24}]{src/ocaml/caml\_start\_program.s}{runtime/amd64.S:604}
\end{column}
\end{columns}
\end{frame}


\begin{frame}{Setup to jump into OCaml entry point}{\funcname{caml\_start\_program} initializes the main fiber}
\begin{columns}
\begin{column}{0.6\textwidth}
\only<1,3,5>{
\begin{itemize}
\item<1-> Stores information to allow switching from OCaml stack to the C stack
\item<3-> Constructs the default exception handler
\item<5-> Switches to the OCaml stack and calls \funcname{caml\_program}
\end{itemize}
\bigskip
\listocaml{../src/default/default.ml}{default/default.ml}
}%
\only<2>{
\listasm[firstline=22,lastline=41,highlightlines={25-27}]{src/ocaml/caml\_start\_program.s}{caml\_start\_program}
}%
\only<4>{
\mintasm[firstline=22,lastline=41,highlightlines={31-38}]{src/ocaml/caml\_start\_program.s}
\listasm[firstline=66,lastline=72]{src/ocaml/caml\_start\_program.s}{caml\_start\_program}
}%
\only<6>{
\listasm[firstline=22,lastline=41,highlightlines={39-41}]{src/ocaml/caml\_start\_program.s}{caml\_start\_program}
}%
\end{column}
\begin{column}{0.4\textwidth}
\centering
\only<1-2>{\providecommand\step{2}\input{diagrams/default/default.tex}}%
\only<3-5>{\providecommand\step{3}\input{diagrams/default/default.tex}}%
\onslide*<6->{\providecommand\step{4}\input{diagrams/default/default.tex}}%
\end{column}
\end{columns}
\end{frame}

\begin{frame}{Executing the OCaml program}{And raising an exception without handler}
\begin{columns}
\begin{column}{0.6\textwidth}
\only<1-3>{
\begin{itemize}
\item<1-> Execution continues with OCaml code
\begin{itemize}
\item \funcname{camlDefault\_\_main\_269}
\item \funcname{camlDefault\_\_entry}
\item \funcname{caml\_program}
\end{itemize}
\item<1-> \funcname{camlDefault\_\_main\_269} raises a \typename{Failure} exception
\item<1-> \typename{Failure} is caught by \funcname{caml\_start\_program}
\begin{itemize}
\item<1-> The handler set the 2nd LSB of the exception value to \funcarg{1}
\item<3-> And then forces \funcname{caml\_start\_program} to terminate
\end{itemize}
\end{itemize}
}
\bigskip
\only<1,3>{\listocaml[highlightlines=3]{../src/default/default.ml}{default/default.ml}}%
\only<2>{\listasm[firstline=66,lastline=72,highlightlines=69]{src/ocaml/caml\_start\_program.s}{caml\_start\_program}}%
\end{column}
\begin{column}{0.4\textwidth}
\centering
\providecommand\step{4}\input{diagrams/default/default.tex}
\end{column}
\end{columns}
\end{frame}
%\only<>{\listasm[firstline=47,lastline=65]{src/ocaml/caml\_start\_program.s}{caml\_start\_program}}


\begin{frame}{Back to C}{Clean up, complain and terminate}
\begin{columns}
\begin{column}{0.45\textwidth}
\begin{itemize}
\item Backtrace:
\begin{itemize}
\item \funcname{caml\_start\_program} $\rightarrow$ OCaml code
\item \funcname{caml\_startup\_common}
\item \funcname{caml\_startup\_exn}
\item \funcname{caml\_startup}
\item \funcname{caml\_main}
\item \funcname{main}
\end{itemize}
\smallskip
\item \funcname{caml\_startup} checks the return value of \funcname{caml\_startup\_exn}
\begin{itemize}
\item The return value has been specially marked by \funcname{caml\_start\_program} handler
\item Calls \funcname{caml\_fatal\_uncaught\_exception}
\end{itemize}
\item<2-> \funcname{caml\_fatal\_uncaught\_exception} either
\begin{itemize}
\item<2-> Calls the user custom uncaught exception handler, if set with \mintinline{ocaml}{Printexc.set_uncaught_exception_handler fn}\footnotemark
\item<2-> Or falls back to the default one
\item<2-> Which notifies about the uncaught exception
\end{itemize}
\item<4-> Terminates the program either with \funcname{abort} or with a non-zero exit code
\end{itemize}
\end{column}
\begin{column}{0.55\textwidth}
\only<1>{
\listc[firstline=84,lastline=89,highlightlines=88]{../ocaml/runtime/caml/mlvalues.h}{runtime/caml/mlvalues.h:84}
\listc[firstline=138,lastline=143,highlightlines={141-142}]{../ocaml/runtime/startup\_nat.c}{runtime/startup\_nat.c:138}
}%
\only<2>{
\listc[firstline=138,highlightlines={147,149}]{../ocaml/runtime/printexc.c}{runtime/printexc.c:138}
}%
\only<3>{
\listc[firstline=110,lastline=134,highlightlines=129]{../ocaml/runtime/printexc.c}{runtime/printexc.c:138}
}%
\only<4>{
\listc[firstline=138,highlightlines={152,154}]{../ocaml/runtime/printexc.c}{runtime/printexc.c:138}
}%
\end{column}
\end{columns}
\footnotetext[1]{\url{https://v2.ocaml.org/api/Printexc.html\#1\_Uncaughtexceptions}}
\end{frame}
}%
\only<2>{\providecommand\step{4}%
% Runtime's default exception handler
%
\subsection*{Runtime's default exception handler}
\frameSubsection{pictures/The\_Architect.png}{}


\begin{frame}{Default exception handler}
\begin{columns}
\begin{column}{0.5\textwidth}
\begin{itemize}
\item \localname{domain\_state->exn\_handler} points to a trap located before \funcname{entry}!
\item What installed this very first trap there?
\end{itemize}
\bigskip
\listocaml{../src/default/default.ml}{default/default.ml}
\bigskip
\inputminted[fontsize=\footnotesize]{text}{../src/default/default.out}
\end{column}
\begin{column}{0.5\textwidth}
\centering
\only<1>{\providecommand\step{0}\input{diagrams/default/default.tex}}%
\only<2>{\providecommand\step{4}\input{diagrams/default/default.tex}}%
\end{column}
\end{columns}
\end{frame}


\begin{frame}{Startup}{How did we get there by the way?}
\begin{columns}
\begin{column}{0.45\textwidth}
\begin{itemize}
\item The entry point address of the binary is \funcname{main} (\funcname{\_start}) from the runtime
\item The program starts like a C program:
\begin{itemize}
\item \funcname{caml\_start\_program}
\item \funcname{caml\_startup\_common}
\item \funcname{caml\_startup\_exn}
\item \funcname{caml\_startup}
\item \funcname{caml\_main}
\item \funcname{main}
\end{itemize}
%\item \funcname{caml\_startup\_common}: initializes GC, domains \dots
% Also: codefrag, locale, custom opeartions, frame_descr, signals, stack
\item \funcname{caml\_start\_program} is the transition point between the C realm and the OCaml one
\end{itemize}
\bigskip
\listocaml{../src/default/default.ml}{default/default.ml}
\end{column}
\begin{column}{0.55\textwidth}
\listasm[lastline=24,highlightlines={22-24}]{src/ocaml/caml\_start\_program.s}{runtime/amd64.S:604}
\end{column}
\end{columns}
\end{frame}


\begin{frame}{Setup to jump into OCaml entry point}{\funcname{caml\_start\_program} initializes the main fiber}
\begin{columns}
\begin{column}{0.6\textwidth}
\only<1,3,5>{
\begin{itemize}
\item<1-> Stores information to allow switching from OCaml stack to the C stack
\item<3-> Constructs the default exception handler
\item<5-> Switches to the OCaml stack and calls \funcname{caml\_program}
\end{itemize}
\bigskip
\listocaml{../src/default/default.ml}{default/default.ml}
}%
\only<2>{
\listasm[firstline=22,lastline=41,highlightlines={25-27}]{src/ocaml/caml\_start\_program.s}{caml\_start\_program}
}%
\only<4>{
\mintasm[firstline=22,lastline=41,highlightlines={31-38}]{src/ocaml/caml\_start\_program.s}
\listasm[firstline=66,lastline=72]{src/ocaml/caml\_start\_program.s}{caml\_start\_program}
}%
\only<6>{
\listasm[firstline=22,lastline=41,highlightlines={39-41}]{src/ocaml/caml\_start\_program.s}{caml\_start\_program}
}%
\end{column}
\begin{column}{0.4\textwidth}
\centering
\only<1-2>{\providecommand\step{2}\input{diagrams/default/default.tex}}%
\only<3-5>{\providecommand\step{3}\input{diagrams/default/default.tex}}%
\onslide*<6->{\providecommand\step{4}\input{diagrams/default/default.tex}}%
\end{column}
\end{columns}
\end{frame}

\begin{frame}{Executing the OCaml program}{And raising an exception without handler}
\begin{columns}
\begin{column}{0.6\textwidth}
\only<1-3>{
\begin{itemize}
\item<1-> Execution continues with OCaml code
\begin{itemize}
\item \funcname{camlDefault\_\_main\_269}
\item \funcname{camlDefault\_\_entry}
\item \funcname{caml\_program}
\end{itemize}
\item<1-> \funcname{camlDefault\_\_main\_269} raises a \typename{Failure} exception
\item<1-> \typename{Failure} is caught by \funcname{caml\_start\_program}
\begin{itemize}
\item<1-> The handler set the 2nd LSB of the exception value to \funcarg{1}
\item<3-> And then forces \funcname{caml\_start\_program} to terminate
\end{itemize}
\end{itemize}
}
\bigskip
\only<1,3>{\listocaml[highlightlines=3]{../src/default/default.ml}{default/default.ml}}%
\only<2>{\listasm[firstline=66,lastline=72,highlightlines=69]{src/ocaml/caml\_start\_program.s}{caml\_start\_program}}%
\end{column}
\begin{column}{0.4\textwidth}
\centering
\providecommand\step{4}\input{diagrams/default/default.tex}
\end{column}
\end{columns}
\end{frame}
%\only<>{\listasm[firstline=47,lastline=65]{src/ocaml/caml\_start\_program.s}{caml\_start\_program}}


\begin{frame}{Back to C}{Clean up, complain and terminate}
\begin{columns}
\begin{column}{0.45\textwidth}
\begin{itemize}
\item Backtrace:
\begin{itemize}
\item \funcname{caml\_start\_program} $\rightarrow$ OCaml code
\item \funcname{caml\_startup\_common}
\item \funcname{caml\_startup\_exn}
\item \funcname{caml\_startup}
\item \funcname{caml\_main}
\item \funcname{main}
\end{itemize}
\smallskip
\item \funcname{caml\_startup} checks the return value of \funcname{caml\_startup\_exn}
\begin{itemize}
\item The return value has been specially marked by \funcname{caml\_start\_program} handler
\item Calls \funcname{caml\_fatal\_uncaught\_exception}
\end{itemize}
\item<2-> \funcname{caml\_fatal\_uncaught\_exception} either
\begin{itemize}
\item<2-> Calls the user custom uncaught exception handler, if set with \mintinline{ocaml}{Printexc.set_uncaught_exception_handler fn}\footnotemark
\item<2-> Or falls back to the default one
\item<2-> Which notifies about the uncaught exception
\end{itemize}
\item<4-> Terminates the program either with \funcname{abort} or with a non-zero exit code
\end{itemize}
\end{column}
\begin{column}{0.55\textwidth}
\only<1>{
\listc[firstline=84,lastline=89,highlightlines=88]{../ocaml/runtime/caml/mlvalues.h}{runtime/caml/mlvalues.h:84}
\listc[firstline=138,lastline=143,highlightlines={141-142}]{../ocaml/runtime/startup\_nat.c}{runtime/startup\_nat.c:138}
}%
\only<2>{
\listc[firstline=138,highlightlines={147,149}]{../ocaml/runtime/printexc.c}{runtime/printexc.c:138}
}%
\only<3>{
\listc[firstline=110,lastline=134,highlightlines=129]{../ocaml/runtime/printexc.c}{runtime/printexc.c:138}
}%
\only<4>{
\listc[firstline=138,highlightlines={152,154}]{../ocaml/runtime/printexc.c}{runtime/printexc.c:138}
}%
\end{column}
\end{columns}
\footnotetext[1]{\url{https://v2.ocaml.org/api/Printexc.html\#1\_Uncaughtexceptions}}
\end{frame}
}%
\end{column}
\end{columns}
\end{frame}


\begin{frame}{Startup}{How did we get there by the way?}
\begin{columns}
\begin{column}{0.45\textwidth}
\begin{itemize}
\item The entry point address of the binary is \funcname{main} (\funcname{\_start}) from the runtime
\item The program starts like a C program:
\begin{itemize}
\item \funcname{caml\_start\_program}
\item \funcname{caml\_startup\_common}
\item \funcname{caml\_startup\_exn}
\item \funcname{caml\_startup}
\item \funcname{caml\_main}
\item \funcname{main}
\end{itemize}
%\item \funcname{caml\_startup\_common}: initializes GC, domains \dots
% Also: codefrag, locale, custom opeartions, frame_descr, signals, stack
\item \funcname{caml\_start\_program} is the transition point between the C realm and the OCaml one
\end{itemize}
\bigskip
\listocaml{../src/default/default.ml}{default/default.ml}
\end{column}
\begin{column}{0.55\textwidth}
\listasm[lastline=24,highlightlines={22-24}]{src/ocaml/caml\_start\_program.s}{runtime/amd64.S:604}
\end{column}
\end{columns}
\end{frame}


\begin{frame}{Setup to jump into OCaml entry point}{\funcname{caml\_start\_program} initializes the main fiber}
\begin{columns}
\begin{column}{0.6\textwidth}
\only<1,3,5>{
\begin{itemize}
\item<1-> Stores information to allow switching from OCaml stack to the C stack
\item<3-> Constructs the default exception handler
\item<5-> Switches to the OCaml stack and calls \funcname{caml\_program}
\end{itemize}
\bigskip
\listocaml{../src/default/default.ml}{default/default.ml}
}%
\only<2>{
\listasm[firstline=22,lastline=41,highlightlines={25-27}]{src/ocaml/caml\_start\_program.s}{caml\_start\_program}
}%
\only<4>{
\mintasm[firstline=22,lastline=41,highlightlines={31-38}]{src/ocaml/caml\_start\_program.s}
\listasm[firstline=66,lastline=72]{src/ocaml/caml\_start\_program.s}{caml\_start\_program}
}%
\only<6>{
\listasm[firstline=22,lastline=41,highlightlines={39-41}]{src/ocaml/caml\_start\_program.s}{caml\_start\_program}
}%
\end{column}
\begin{column}{0.4\textwidth}
\centering
\only<1-2>{\providecommand\step{2}%
% Runtime's default exception handler
%
\subsection*{Runtime's default exception handler}
\frameSubsection{pictures/The\_Architect.png}{}


\begin{frame}{Default exception handler}
\begin{columns}
\begin{column}{0.5\textwidth}
\begin{itemize}
\item \localname{domain\_state->exn\_handler} points to a trap located before \funcname{entry}!
\item What installed this very first trap there?
\end{itemize}
\bigskip
\listocaml{../src/default/default.ml}{default/default.ml}
\bigskip
\inputminted[fontsize=\footnotesize]{text}{../src/default/default.out}
\end{column}
\begin{column}{0.5\textwidth}
\centering
\only<1>{\providecommand\step{0}\input{diagrams/default/default.tex}}%
\only<2>{\providecommand\step{4}\input{diagrams/default/default.tex}}%
\end{column}
\end{columns}
\end{frame}


\begin{frame}{Startup}{How did we get there by the way?}
\begin{columns}
\begin{column}{0.45\textwidth}
\begin{itemize}
\item The entry point address of the binary is \funcname{main} (\funcname{\_start}) from the runtime
\item The program starts like a C program:
\begin{itemize}
\item \funcname{caml\_start\_program}
\item \funcname{caml\_startup\_common}
\item \funcname{caml\_startup\_exn}
\item \funcname{caml\_startup}
\item \funcname{caml\_main}
\item \funcname{main}
\end{itemize}
%\item \funcname{caml\_startup\_common}: initializes GC, domains \dots
% Also: codefrag, locale, custom opeartions, frame_descr, signals, stack
\item \funcname{caml\_start\_program} is the transition point between the C realm and the OCaml one
\end{itemize}
\bigskip
\listocaml{../src/default/default.ml}{default/default.ml}
\end{column}
\begin{column}{0.55\textwidth}
\listasm[lastline=24,highlightlines={22-24}]{src/ocaml/caml\_start\_program.s}{runtime/amd64.S:604}
\end{column}
\end{columns}
\end{frame}


\begin{frame}{Setup to jump into OCaml entry point}{\funcname{caml\_start\_program} initializes the main fiber}
\begin{columns}
\begin{column}{0.6\textwidth}
\only<1,3,5>{
\begin{itemize}
\item<1-> Stores information to allow switching from OCaml stack to the C stack
\item<3-> Constructs the default exception handler
\item<5-> Switches to the OCaml stack and calls \funcname{caml\_program}
\end{itemize}
\bigskip
\listocaml{../src/default/default.ml}{default/default.ml}
}%
\only<2>{
\listasm[firstline=22,lastline=41,highlightlines={25-27}]{src/ocaml/caml\_start\_program.s}{caml\_start\_program}
}%
\only<4>{
\mintasm[firstline=22,lastline=41,highlightlines={31-38}]{src/ocaml/caml\_start\_program.s}
\listasm[firstline=66,lastline=72]{src/ocaml/caml\_start\_program.s}{caml\_start\_program}
}%
\only<6>{
\listasm[firstline=22,lastline=41,highlightlines={39-41}]{src/ocaml/caml\_start\_program.s}{caml\_start\_program}
}%
\end{column}
\begin{column}{0.4\textwidth}
\centering
\only<1-2>{\providecommand\step{2}\input{diagrams/default/default.tex}}%
\only<3-5>{\providecommand\step{3}\input{diagrams/default/default.tex}}%
\onslide*<6->{\providecommand\step{4}\input{diagrams/default/default.tex}}%
\end{column}
\end{columns}
\end{frame}

\begin{frame}{Executing the OCaml program}{And raising an exception without handler}
\begin{columns}
\begin{column}{0.6\textwidth}
\only<1-3>{
\begin{itemize}
\item<1-> Execution continues with OCaml code
\begin{itemize}
\item \funcname{camlDefault\_\_main\_269}
\item \funcname{camlDefault\_\_entry}
\item \funcname{caml\_program}
\end{itemize}
\item<1-> \funcname{camlDefault\_\_main\_269} raises a \typename{Failure} exception
\item<1-> \typename{Failure} is caught by \funcname{caml\_start\_program}
\begin{itemize}
\item<1-> The handler set the 2nd LSB of the exception value to \funcarg{1}
\item<3-> And then forces \funcname{caml\_start\_program} to terminate
\end{itemize}
\end{itemize}
}
\bigskip
\only<1,3>{\listocaml[highlightlines=3]{../src/default/default.ml}{default/default.ml}}%
\only<2>{\listasm[firstline=66,lastline=72,highlightlines=69]{src/ocaml/caml\_start\_program.s}{caml\_start\_program}}%
\end{column}
\begin{column}{0.4\textwidth}
\centering
\providecommand\step{4}\input{diagrams/default/default.tex}
\end{column}
\end{columns}
\end{frame}
%\only<>{\listasm[firstline=47,lastline=65]{src/ocaml/caml\_start\_program.s}{caml\_start\_program}}


\begin{frame}{Back to C}{Clean up, complain and terminate}
\begin{columns}
\begin{column}{0.45\textwidth}
\begin{itemize}
\item Backtrace:
\begin{itemize}
\item \funcname{caml\_start\_program} $\rightarrow$ OCaml code
\item \funcname{caml\_startup\_common}
\item \funcname{caml\_startup\_exn}
\item \funcname{caml\_startup}
\item \funcname{caml\_main}
\item \funcname{main}
\end{itemize}
\smallskip
\item \funcname{caml\_startup} checks the return value of \funcname{caml\_startup\_exn}
\begin{itemize}
\item The return value has been specially marked by \funcname{caml\_start\_program} handler
\item Calls \funcname{caml\_fatal\_uncaught\_exception}
\end{itemize}
\item<2-> \funcname{caml\_fatal\_uncaught\_exception} either
\begin{itemize}
\item<2-> Calls the user custom uncaught exception handler, if set with \mintinline{ocaml}{Printexc.set_uncaught_exception_handler fn}\footnotemark
\item<2-> Or falls back to the default one
\item<2-> Which notifies about the uncaught exception
\end{itemize}
\item<4-> Terminates the program either with \funcname{abort} or with a non-zero exit code
\end{itemize}
\end{column}
\begin{column}{0.55\textwidth}
\only<1>{
\listc[firstline=84,lastline=89,highlightlines=88]{../ocaml/runtime/caml/mlvalues.h}{runtime/caml/mlvalues.h:84}
\listc[firstline=138,lastline=143,highlightlines={141-142}]{../ocaml/runtime/startup\_nat.c}{runtime/startup\_nat.c:138}
}%
\only<2>{
\listc[firstline=138,highlightlines={147,149}]{../ocaml/runtime/printexc.c}{runtime/printexc.c:138}
}%
\only<3>{
\listc[firstline=110,lastline=134,highlightlines=129]{../ocaml/runtime/printexc.c}{runtime/printexc.c:138}
}%
\only<4>{
\listc[firstline=138,highlightlines={152,154}]{../ocaml/runtime/printexc.c}{runtime/printexc.c:138}
}%
\end{column}
\end{columns}
\footnotetext[1]{\url{https://v2.ocaml.org/api/Printexc.html\#1\_Uncaughtexceptions}}
\end{frame}
}%
\only<3-5>{\providecommand\step{3}%
% Runtime's default exception handler
%
\subsection*{Runtime's default exception handler}
\frameSubsection{pictures/The\_Architect.png}{}


\begin{frame}{Default exception handler}
\begin{columns}
\begin{column}{0.5\textwidth}
\begin{itemize}
\item \localname{domain\_state->exn\_handler} points to a trap located before \funcname{entry}!
\item What installed this very first trap there?
\end{itemize}
\bigskip
\listocaml{../src/default/default.ml}{default/default.ml}
\bigskip
\inputminted[fontsize=\footnotesize]{text}{../src/default/default.out}
\end{column}
\begin{column}{0.5\textwidth}
\centering
\only<1>{\providecommand\step{0}\input{diagrams/default/default.tex}}%
\only<2>{\providecommand\step{4}\input{diagrams/default/default.tex}}%
\end{column}
\end{columns}
\end{frame}


\begin{frame}{Startup}{How did we get there by the way?}
\begin{columns}
\begin{column}{0.45\textwidth}
\begin{itemize}
\item The entry point address of the binary is \funcname{main} (\funcname{\_start}) from the runtime
\item The program starts like a C program:
\begin{itemize}
\item \funcname{caml\_start\_program}
\item \funcname{caml\_startup\_common}
\item \funcname{caml\_startup\_exn}
\item \funcname{caml\_startup}
\item \funcname{caml\_main}
\item \funcname{main}
\end{itemize}
%\item \funcname{caml\_startup\_common}: initializes GC, domains \dots
% Also: codefrag, locale, custom opeartions, frame_descr, signals, stack
\item \funcname{caml\_start\_program} is the transition point between the C realm and the OCaml one
\end{itemize}
\bigskip
\listocaml{../src/default/default.ml}{default/default.ml}
\end{column}
\begin{column}{0.55\textwidth}
\listasm[lastline=24,highlightlines={22-24}]{src/ocaml/caml\_start\_program.s}{runtime/amd64.S:604}
\end{column}
\end{columns}
\end{frame}


\begin{frame}{Setup to jump into OCaml entry point}{\funcname{caml\_start\_program} initializes the main fiber}
\begin{columns}
\begin{column}{0.6\textwidth}
\only<1,3,5>{
\begin{itemize}
\item<1-> Stores information to allow switching from OCaml stack to the C stack
\item<3-> Constructs the default exception handler
\item<5-> Switches to the OCaml stack and calls \funcname{caml\_program}
\end{itemize}
\bigskip
\listocaml{../src/default/default.ml}{default/default.ml}
}%
\only<2>{
\listasm[firstline=22,lastline=41,highlightlines={25-27}]{src/ocaml/caml\_start\_program.s}{caml\_start\_program}
}%
\only<4>{
\mintasm[firstline=22,lastline=41,highlightlines={31-38}]{src/ocaml/caml\_start\_program.s}
\listasm[firstline=66,lastline=72]{src/ocaml/caml\_start\_program.s}{caml\_start\_program}
}%
\only<6>{
\listasm[firstline=22,lastline=41,highlightlines={39-41}]{src/ocaml/caml\_start\_program.s}{caml\_start\_program}
}%
\end{column}
\begin{column}{0.4\textwidth}
\centering
\only<1-2>{\providecommand\step{2}\input{diagrams/default/default.tex}}%
\only<3-5>{\providecommand\step{3}\input{diagrams/default/default.tex}}%
\onslide*<6->{\providecommand\step{4}\input{diagrams/default/default.tex}}%
\end{column}
\end{columns}
\end{frame}

\begin{frame}{Executing the OCaml program}{And raising an exception without handler}
\begin{columns}
\begin{column}{0.6\textwidth}
\only<1-3>{
\begin{itemize}
\item<1-> Execution continues with OCaml code
\begin{itemize}
\item \funcname{camlDefault\_\_main\_269}
\item \funcname{camlDefault\_\_entry}
\item \funcname{caml\_program}
\end{itemize}
\item<1-> \funcname{camlDefault\_\_main\_269} raises a \typename{Failure} exception
\item<1-> \typename{Failure} is caught by \funcname{caml\_start\_program}
\begin{itemize}
\item<1-> The handler set the 2nd LSB of the exception value to \funcarg{1}
\item<3-> And then forces \funcname{caml\_start\_program} to terminate
\end{itemize}
\end{itemize}
}
\bigskip
\only<1,3>{\listocaml[highlightlines=3]{../src/default/default.ml}{default/default.ml}}%
\only<2>{\listasm[firstline=66,lastline=72,highlightlines=69]{src/ocaml/caml\_start\_program.s}{caml\_start\_program}}%
\end{column}
\begin{column}{0.4\textwidth}
\centering
\providecommand\step{4}\input{diagrams/default/default.tex}
\end{column}
\end{columns}
\end{frame}
%\only<>{\listasm[firstline=47,lastline=65]{src/ocaml/caml\_start\_program.s}{caml\_start\_program}}


\begin{frame}{Back to C}{Clean up, complain and terminate}
\begin{columns}
\begin{column}{0.45\textwidth}
\begin{itemize}
\item Backtrace:
\begin{itemize}
\item \funcname{caml\_start\_program} $\rightarrow$ OCaml code
\item \funcname{caml\_startup\_common}
\item \funcname{caml\_startup\_exn}
\item \funcname{caml\_startup}
\item \funcname{caml\_main}
\item \funcname{main}
\end{itemize}
\smallskip
\item \funcname{caml\_startup} checks the return value of \funcname{caml\_startup\_exn}
\begin{itemize}
\item The return value has been specially marked by \funcname{caml\_start\_program} handler
\item Calls \funcname{caml\_fatal\_uncaught\_exception}
\end{itemize}
\item<2-> \funcname{caml\_fatal\_uncaught\_exception} either
\begin{itemize}
\item<2-> Calls the user custom uncaught exception handler, if set with \mintinline{ocaml}{Printexc.set_uncaught_exception_handler fn}\footnotemark
\item<2-> Or falls back to the default one
\item<2-> Which notifies about the uncaught exception
\end{itemize}
\item<4-> Terminates the program either with \funcname{abort} or with a non-zero exit code
\end{itemize}
\end{column}
\begin{column}{0.55\textwidth}
\only<1>{
\listc[firstline=84,lastline=89,highlightlines=88]{../ocaml/runtime/caml/mlvalues.h}{runtime/caml/mlvalues.h:84}
\listc[firstline=138,lastline=143,highlightlines={141-142}]{../ocaml/runtime/startup\_nat.c}{runtime/startup\_nat.c:138}
}%
\only<2>{
\listc[firstline=138,highlightlines={147,149}]{../ocaml/runtime/printexc.c}{runtime/printexc.c:138}
}%
\only<3>{
\listc[firstline=110,lastline=134,highlightlines=129]{../ocaml/runtime/printexc.c}{runtime/printexc.c:138}
}%
\only<4>{
\listc[firstline=138,highlightlines={152,154}]{../ocaml/runtime/printexc.c}{runtime/printexc.c:138}
}%
\end{column}
\end{columns}
\footnotetext[1]{\url{https://v2.ocaml.org/api/Printexc.html\#1\_Uncaughtexceptions}}
\end{frame}
}%
\onslide*<6->{\providecommand\step{4}%
% Runtime's default exception handler
%
\subsection*{Runtime's default exception handler}
\frameSubsection{pictures/The\_Architect.png}{}


\begin{frame}{Default exception handler}
\begin{columns}
\begin{column}{0.5\textwidth}
\begin{itemize}
\item \localname{domain\_state->exn\_handler} points to a trap located before \funcname{entry}!
\item What installed this very first trap there?
\end{itemize}
\bigskip
\listocaml{../src/default/default.ml}{default/default.ml}
\bigskip
\inputminted[fontsize=\footnotesize]{text}{../src/default/default.out}
\end{column}
\begin{column}{0.5\textwidth}
\centering
\only<1>{\providecommand\step{0}\input{diagrams/default/default.tex}}%
\only<2>{\providecommand\step{4}\input{diagrams/default/default.tex}}%
\end{column}
\end{columns}
\end{frame}


\begin{frame}{Startup}{How did we get there by the way?}
\begin{columns}
\begin{column}{0.45\textwidth}
\begin{itemize}
\item The entry point address of the binary is \funcname{main} (\funcname{\_start}) from the runtime
\item The program starts like a C program:
\begin{itemize}
\item \funcname{caml\_start\_program}
\item \funcname{caml\_startup\_common}
\item \funcname{caml\_startup\_exn}
\item \funcname{caml\_startup}
\item \funcname{caml\_main}
\item \funcname{main}
\end{itemize}
%\item \funcname{caml\_startup\_common}: initializes GC, domains \dots
% Also: codefrag, locale, custom opeartions, frame_descr, signals, stack
\item \funcname{caml\_start\_program} is the transition point between the C realm and the OCaml one
\end{itemize}
\bigskip
\listocaml{../src/default/default.ml}{default/default.ml}
\end{column}
\begin{column}{0.55\textwidth}
\listasm[lastline=24,highlightlines={22-24}]{src/ocaml/caml\_start\_program.s}{runtime/amd64.S:604}
\end{column}
\end{columns}
\end{frame}


\begin{frame}{Setup to jump into OCaml entry point}{\funcname{caml\_start\_program} initializes the main fiber}
\begin{columns}
\begin{column}{0.6\textwidth}
\only<1,3,5>{
\begin{itemize}
\item<1-> Stores information to allow switching from OCaml stack to the C stack
\item<3-> Constructs the default exception handler
\item<5-> Switches to the OCaml stack and calls \funcname{caml\_program}
\end{itemize}
\bigskip
\listocaml{../src/default/default.ml}{default/default.ml}
}%
\only<2>{
\listasm[firstline=22,lastline=41,highlightlines={25-27}]{src/ocaml/caml\_start\_program.s}{caml\_start\_program}
}%
\only<4>{
\mintasm[firstline=22,lastline=41,highlightlines={31-38}]{src/ocaml/caml\_start\_program.s}
\listasm[firstline=66,lastline=72]{src/ocaml/caml\_start\_program.s}{caml\_start\_program}
}%
\only<6>{
\listasm[firstline=22,lastline=41,highlightlines={39-41}]{src/ocaml/caml\_start\_program.s}{caml\_start\_program}
}%
\end{column}
\begin{column}{0.4\textwidth}
\centering
\only<1-2>{\providecommand\step{2}\input{diagrams/default/default.tex}}%
\only<3-5>{\providecommand\step{3}\input{diagrams/default/default.tex}}%
\onslide*<6->{\providecommand\step{4}\input{diagrams/default/default.tex}}%
\end{column}
\end{columns}
\end{frame}

\begin{frame}{Executing the OCaml program}{And raising an exception without handler}
\begin{columns}
\begin{column}{0.6\textwidth}
\only<1-3>{
\begin{itemize}
\item<1-> Execution continues with OCaml code
\begin{itemize}
\item \funcname{camlDefault\_\_main\_269}
\item \funcname{camlDefault\_\_entry}
\item \funcname{caml\_program}
\end{itemize}
\item<1-> \funcname{camlDefault\_\_main\_269} raises a \typename{Failure} exception
\item<1-> \typename{Failure} is caught by \funcname{caml\_start\_program}
\begin{itemize}
\item<1-> The handler set the 2nd LSB of the exception value to \funcarg{1}
\item<3-> And then forces \funcname{caml\_start\_program} to terminate
\end{itemize}
\end{itemize}
}
\bigskip
\only<1,3>{\listocaml[highlightlines=3]{../src/default/default.ml}{default/default.ml}}%
\only<2>{\listasm[firstline=66,lastline=72,highlightlines=69]{src/ocaml/caml\_start\_program.s}{caml\_start\_program}}%
\end{column}
\begin{column}{0.4\textwidth}
\centering
\providecommand\step{4}\input{diagrams/default/default.tex}
\end{column}
\end{columns}
\end{frame}
%\only<>{\listasm[firstline=47,lastline=65]{src/ocaml/caml\_start\_program.s}{caml\_start\_program}}


\begin{frame}{Back to C}{Clean up, complain and terminate}
\begin{columns}
\begin{column}{0.45\textwidth}
\begin{itemize}
\item Backtrace:
\begin{itemize}
\item \funcname{caml\_start\_program} $\rightarrow$ OCaml code
\item \funcname{caml\_startup\_common}
\item \funcname{caml\_startup\_exn}
\item \funcname{caml\_startup}
\item \funcname{caml\_main}
\item \funcname{main}
\end{itemize}
\smallskip
\item \funcname{caml\_startup} checks the return value of \funcname{caml\_startup\_exn}
\begin{itemize}
\item The return value has been specially marked by \funcname{caml\_start\_program} handler
\item Calls \funcname{caml\_fatal\_uncaught\_exception}
\end{itemize}
\item<2-> \funcname{caml\_fatal\_uncaught\_exception} either
\begin{itemize}
\item<2-> Calls the user custom uncaught exception handler, if set with \mintinline{ocaml}{Printexc.set_uncaught_exception_handler fn}\footnotemark
\item<2-> Or falls back to the default one
\item<2-> Which notifies about the uncaught exception
\end{itemize}
\item<4-> Terminates the program either with \funcname{abort} or with a non-zero exit code
\end{itemize}
\end{column}
\begin{column}{0.55\textwidth}
\only<1>{
\listc[firstline=84,lastline=89,highlightlines=88]{../ocaml/runtime/caml/mlvalues.h}{runtime/caml/mlvalues.h:84}
\listc[firstline=138,lastline=143,highlightlines={141-142}]{../ocaml/runtime/startup\_nat.c}{runtime/startup\_nat.c:138}
}%
\only<2>{
\listc[firstline=138,highlightlines={147,149}]{../ocaml/runtime/printexc.c}{runtime/printexc.c:138}
}%
\only<3>{
\listc[firstline=110,lastline=134,highlightlines=129]{../ocaml/runtime/printexc.c}{runtime/printexc.c:138}
}%
\only<4>{
\listc[firstline=138,highlightlines={152,154}]{../ocaml/runtime/printexc.c}{runtime/printexc.c:138}
}%
\end{column}
\end{columns}
\footnotetext[1]{\url{https://v2.ocaml.org/api/Printexc.html\#1\_Uncaughtexceptions}}
\end{frame}
}%
\end{column}
\end{columns}
\end{frame}

\begin{frame}{Executing the OCaml program}{And raising an exception without handler}
\begin{columns}
\begin{column}{0.6\textwidth}
\only<1-3>{
\begin{itemize}
\item<1-> Execution continues with OCaml code
\begin{itemize}
\item \funcname{camlDefault\_\_main\_269}
\item \funcname{camlDefault\_\_entry}
\item \funcname{caml\_program}
\end{itemize}
\item<1-> \funcname{camlDefault\_\_main\_269} raises a \typename{Failure} exception
\item<1-> \typename{Failure} is caught by \funcname{caml\_start\_program}
\begin{itemize}
\item<1-> The handler set the 2nd LSB of the exception value to \funcarg{1}
\item<3-> And then forces \funcname{caml\_start\_program} to terminate
\end{itemize}
\end{itemize}
}
\bigskip
\only<1,3>{\listocaml[highlightlines=3]{../src/default/default.ml}{default/default.ml}}%
\only<2>{\listasm[firstline=66,lastline=72,highlightlines=69]{src/ocaml/caml\_start\_program.s}{caml\_start\_program}}%
\end{column}
\begin{column}{0.4\textwidth}
\centering
\providecommand\step{4}%
% Runtime's default exception handler
%
\subsection*{Runtime's default exception handler}
\frameSubsection{pictures/The\_Architect.png}{}


\begin{frame}{Default exception handler}
\begin{columns}
\begin{column}{0.5\textwidth}
\begin{itemize}
\item \localname{domain\_state->exn\_handler} points to a trap located before \funcname{entry}!
\item What installed this very first trap there?
\end{itemize}
\bigskip
\listocaml{../src/default/default.ml}{default/default.ml}
\bigskip
\inputminted[fontsize=\footnotesize]{text}{../src/default/default.out}
\end{column}
\begin{column}{0.5\textwidth}
\centering
\only<1>{\providecommand\step{0}\input{diagrams/default/default.tex}}%
\only<2>{\providecommand\step{4}\input{diagrams/default/default.tex}}%
\end{column}
\end{columns}
\end{frame}


\begin{frame}{Startup}{How did we get there by the way?}
\begin{columns}
\begin{column}{0.45\textwidth}
\begin{itemize}
\item The entry point address of the binary is \funcname{main} (\funcname{\_start}) from the runtime
\item The program starts like a C program:
\begin{itemize}
\item \funcname{caml\_start\_program}
\item \funcname{caml\_startup\_common}
\item \funcname{caml\_startup\_exn}
\item \funcname{caml\_startup}
\item \funcname{caml\_main}
\item \funcname{main}
\end{itemize}
%\item \funcname{caml\_startup\_common}: initializes GC, domains \dots
% Also: codefrag, locale, custom opeartions, frame_descr, signals, stack
\item \funcname{caml\_start\_program} is the transition point between the C realm and the OCaml one
\end{itemize}
\bigskip
\listocaml{../src/default/default.ml}{default/default.ml}
\end{column}
\begin{column}{0.55\textwidth}
\listasm[lastline=24,highlightlines={22-24}]{src/ocaml/caml\_start\_program.s}{runtime/amd64.S:604}
\end{column}
\end{columns}
\end{frame}


\begin{frame}{Setup to jump into OCaml entry point}{\funcname{caml\_start\_program} initializes the main fiber}
\begin{columns}
\begin{column}{0.6\textwidth}
\only<1,3,5>{
\begin{itemize}
\item<1-> Stores information to allow switching from OCaml stack to the C stack
\item<3-> Constructs the default exception handler
\item<5-> Switches to the OCaml stack and calls \funcname{caml\_program}
\end{itemize}
\bigskip
\listocaml{../src/default/default.ml}{default/default.ml}
}%
\only<2>{
\listasm[firstline=22,lastline=41,highlightlines={25-27}]{src/ocaml/caml\_start\_program.s}{caml\_start\_program}
}%
\only<4>{
\mintasm[firstline=22,lastline=41,highlightlines={31-38}]{src/ocaml/caml\_start\_program.s}
\listasm[firstline=66,lastline=72]{src/ocaml/caml\_start\_program.s}{caml\_start\_program}
}%
\only<6>{
\listasm[firstline=22,lastline=41,highlightlines={39-41}]{src/ocaml/caml\_start\_program.s}{caml\_start\_program}
}%
\end{column}
\begin{column}{0.4\textwidth}
\centering
\only<1-2>{\providecommand\step{2}\input{diagrams/default/default.tex}}%
\only<3-5>{\providecommand\step{3}\input{diagrams/default/default.tex}}%
\onslide*<6->{\providecommand\step{4}\input{diagrams/default/default.tex}}%
\end{column}
\end{columns}
\end{frame}

\begin{frame}{Executing the OCaml program}{And raising an exception without handler}
\begin{columns}
\begin{column}{0.6\textwidth}
\only<1-3>{
\begin{itemize}
\item<1-> Execution continues with OCaml code
\begin{itemize}
\item \funcname{camlDefault\_\_main\_269}
\item \funcname{camlDefault\_\_entry}
\item \funcname{caml\_program}
\end{itemize}
\item<1-> \funcname{camlDefault\_\_main\_269} raises a \typename{Failure} exception
\item<1-> \typename{Failure} is caught by \funcname{caml\_start\_program}
\begin{itemize}
\item<1-> The handler set the 2nd LSB of the exception value to \funcarg{1}
\item<3-> And then forces \funcname{caml\_start\_program} to terminate
\end{itemize}
\end{itemize}
}
\bigskip
\only<1,3>{\listocaml[highlightlines=3]{../src/default/default.ml}{default/default.ml}}%
\only<2>{\listasm[firstline=66,lastline=72,highlightlines=69]{src/ocaml/caml\_start\_program.s}{caml\_start\_program}}%
\end{column}
\begin{column}{0.4\textwidth}
\centering
\providecommand\step{4}\input{diagrams/default/default.tex}
\end{column}
\end{columns}
\end{frame}
%\only<>{\listasm[firstline=47,lastline=65]{src/ocaml/caml\_start\_program.s}{caml\_start\_program}}


\begin{frame}{Back to C}{Clean up, complain and terminate}
\begin{columns}
\begin{column}{0.45\textwidth}
\begin{itemize}
\item Backtrace:
\begin{itemize}
\item \funcname{caml\_start\_program} $\rightarrow$ OCaml code
\item \funcname{caml\_startup\_common}
\item \funcname{caml\_startup\_exn}
\item \funcname{caml\_startup}
\item \funcname{caml\_main}
\item \funcname{main}
\end{itemize}
\smallskip
\item \funcname{caml\_startup} checks the return value of \funcname{caml\_startup\_exn}
\begin{itemize}
\item The return value has been specially marked by \funcname{caml\_start\_program} handler
\item Calls \funcname{caml\_fatal\_uncaught\_exception}
\end{itemize}
\item<2-> \funcname{caml\_fatal\_uncaught\_exception} either
\begin{itemize}
\item<2-> Calls the user custom uncaught exception handler, if set with \mintinline{ocaml}{Printexc.set_uncaught_exception_handler fn}\footnotemark
\item<2-> Or falls back to the default one
\item<2-> Which notifies about the uncaught exception
\end{itemize}
\item<4-> Terminates the program either with \funcname{abort} or with a non-zero exit code
\end{itemize}
\end{column}
\begin{column}{0.55\textwidth}
\only<1>{
\listc[firstline=84,lastline=89,highlightlines=88]{../ocaml/runtime/caml/mlvalues.h}{runtime/caml/mlvalues.h:84}
\listc[firstline=138,lastline=143,highlightlines={141-142}]{../ocaml/runtime/startup\_nat.c}{runtime/startup\_nat.c:138}
}%
\only<2>{
\listc[firstline=138,highlightlines={147,149}]{../ocaml/runtime/printexc.c}{runtime/printexc.c:138}
}%
\only<3>{
\listc[firstline=110,lastline=134,highlightlines=129]{../ocaml/runtime/printexc.c}{runtime/printexc.c:138}
}%
\only<4>{
\listc[firstline=138,highlightlines={152,154}]{../ocaml/runtime/printexc.c}{runtime/printexc.c:138}
}%
\end{column}
\end{columns}
\footnotetext[1]{\url{https://v2.ocaml.org/api/Printexc.html\#1\_Uncaughtexceptions}}
\end{frame}

\end{column}
\end{columns}
\end{frame}
%\only<>{\listasm[firstline=47,lastline=65]{src/ocaml/caml\_start\_program.s}{caml\_start\_program}}


\begin{frame}{Back to C}{Clean up, complain and terminate}
\begin{columns}
\begin{column}{0.45\textwidth}
\begin{itemize}
\item Backtrace:
\begin{itemize}
\item \funcname{caml\_start\_program} $\rightarrow$ OCaml code
\item \funcname{caml\_startup\_common}
\item \funcname{caml\_startup\_exn}
\item \funcname{caml\_startup}
\item \funcname{caml\_main}
\item \funcname{main}
\end{itemize}
\smallskip
\item \funcname{caml\_startup} checks the return value of \funcname{caml\_startup\_exn}
\begin{itemize}
\item The return value has been specially marked by \funcname{caml\_start\_program} handler
\item Calls \funcname{caml\_fatal\_uncaught\_exception}
\end{itemize}
\item<2-> \funcname{caml\_fatal\_uncaught\_exception} either
\begin{itemize}
\item<2-> Calls the user custom uncaught exception handler, if set with \mintinline{ocaml}{Printexc.set_uncaught_exception_handler fn}\footnotemark
\item<2-> Or falls back to the default one
\item<2-> Which notifies about the uncaught exception
\end{itemize}
\item<4-> Terminates the program either with \funcname{abort} or with a non-zero exit code
\end{itemize}
\end{column}
\begin{column}{0.55\textwidth}
\only<1>{
\listc[firstline=84,lastline=89,highlightlines=88]{../ocaml/runtime/caml/mlvalues.h}{runtime/caml/mlvalues.h:84}
\listc[firstline=138,lastline=143,highlightlines={141-142}]{../ocaml/runtime/startup\_nat.c}{runtime/startup\_nat.c:138}
}%
\only<2>{
\listc[firstline=138,highlightlines={147,149}]{../ocaml/runtime/printexc.c}{runtime/printexc.c:138}
}%
\only<3>{
\listc[firstline=110,lastline=134,highlightlines=129]{../ocaml/runtime/printexc.c}{runtime/printexc.c:138}
}%
\only<4>{
\listc[firstline=138,highlightlines={152,154}]{../ocaml/runtime/printexc.c}{runtime/printexc.c:138}
}%
\end{column}
\end{columns}
\footnotetext[1]{\url{https://v2.ocaml.org/api/Printexc.html\#1\_Uncaughtexceptions}}
\end{frame}
}%
\only<3-5>{\providecommand\step{3}%
% Runtime's default exception handler
%
\subsection*{Runtime's default exception handler}
\frameSubsection{pictures/The\_Architect.png}{}


\begin{frame}{Default exception handler}
\begin{columns}
\begin{column}{0.5\textwidth}
\begin{itemize}
\item \localname{domain\_state->exn\_handler} points to a trap located before \funcname{entry}!
\item What installed this very first trap there?
\end{itemize}
\bigskip
\listocaml{../src/default/default.ml}{default/default.ml}
\bigskip
\inputminted[fontsize=\footnotesize]{text}{../src/default/default.out}
\end{column}
\begin{column}{0.5\textwidth}
\centering
\only<1>{\providecommand\step{0}%
% Runtime's default exception handler
%
\subsection*{Runtime's default exception handler}
\frameSubsection{pictures/The\_Architect.png}{}


\begin{frame}{Default exception handler}
\begin{columns}
\begin{column}{0.5\textwidth}
\begin{itemize}
\item \localname{domain\_state->exn\_handler} points to a trap located before \funcname{entry}!
\item What installed this very first trap there?
\end{itemize}
\bigskip
\listocaml{../src/default/default.ml}{default/default.ml}
\bigskip
\inputminted[fontsize=\footnotesize]{text}{../src/default/default.out}
\end{column}
\begin{column}{0.5\textwidth}
\centering
\only<1>{\providecommand\step{0}\input{diagrams/default/default.tex}}%
\only<2>{\providecommand\step{4}\input{diagrams/default/default.tex}}%
\end{column}
\end{columns}
\end{frame}


\begin{frame}{Startup}{How did we get there by the way?}
\begin{columns}
\begin{column}{0.45\textwidth}
\begin{itemize}
\item The entry point address of the binary is \funcname{main} (\funcname{\_start}) from the runtime
\item The program starts like a C program:
\begin{itemize}
\item \funcname{caml\_start\_program}
\item \funcname{caml\_startup\_common}
\item \funcname{caml\_startup\_exn}
\item \funcname{caml\_startup}
\item \funcname{caml\_main}
\item \funcname{main}
\end{itemize}
%\item \funcname{caml\_startup\_common}: initializes GC, domains \dots
% Also: codefrag, locale, custom opeartions, frame_descr, signals, stack
\item \funcname{caml\_start\_program} is the transition point between the C realm and the OCaml one
\end{itemize}
\bigskip
\listocaml{../src/default/default.ml}{default/default.ml}
\end{column}
\begin{column}{0.55\textwidth}
\listasm[lastline=24,highlightlines={22-24}]{src/ocaml/caml\_start\_program.s}{runtime/amd64.S:604}
\end{column}
\end{columns}
\end{frame}


\begin{frame}{Setup to jump into OCaml entry point}{\funcname{caml\_start\_program} initializes the main fiber}
\begin{columns}
\begin{column}{0.6\textwidth}
\only<1,3,5>{
\begin{itemize}
\item<1-> Stores information to allow switching from OCaml stack to the C stack
\item<3-> Constructs the default exception handler
\item<5-> Switches to the OCaml stack and calls \funcname{caml\_program}
\end{itemize}
\bigskip
\listocaml{../src/default/default.ml}{default/default.ml}
}%
\only<2>{
\listasm[firstline=22,lastline=41,highlightlines={25-27}]{src/ocaml/caml\_start\_program.s}{caml\_start\_program}
}%
\only<4>{
\mintasm[firstline=22,lastline=41,highlightlines={31-38}]{src/ocaml/caml\_start\_program.s}
\listasm[firstline=66,lastline=72]{src/ocaml/caml\_start\_program.s}{caml\_start\_program}
}%
\only<6>{
\listasm[firstline=22,lastline=41,highlightlines={39-41}]{src/ocaml/caml\_start\_program.s}{caml\_start\_program}
}%
\end{column}
\begin{column}{0.4\textwidth}
\centering
\only<1-2>{\providecommand\step{2}\input{diagrams/default/default.tex}}%
\only<3-5>{\providecommand\step{3}\input{diagrams/default/default.tex}}%
\onslide*<6->{\providecommand\step{4}\input{diagrams/default/default.tex}}%
\end{column}
\end{columns}
\end{frame}

\begin{frame}{Executing the OCaml program}{And raising an exception without handler}
\begin{columns}
\begin{column}{0.6\textwidth}
\only<1-3>{
\begin{itemize}
\item<1-> Execution continues with OCaml code
\begin{itemize}
\item \funcname{camlDefault\_\_main\_269}
\item \funcname{camlDefault\_\_entry}
\item \funcname{caml\_program}
\end{itemize}
\item<1-> \funcname{camlDefault\_\_main\_269} raises a \typename{Failure} exception
\item<1-> \typename{Failure} is caught by \funcname{caml\_start\_program}
\begin{itemize}
\item<1-> The handler set the 2nd LSB of the exception value to \funcarg{1}
\item<3-> And then forces \funcname{caml\_start\_program} to terminate
\end{itemize}
\end{itemize}
}
\bigskip
\only<1,3>{\listocaml[highlightlines=3]{../src/default/default.ml}{default/default.ml}}%
\only<2>{\listasm[firstline=66,lastline=72,highlightlines=69]{src/ocaml/caml\_start\_program.s}{caml\_start\_program}}%
\end{column}
\begin{column}{0.4\textwidth}
\centering
\providecommand\step{4}\input{diagrams/default/default.tex}
\end{column}
\end{columns}
\end{frame}
%\only<>{\listasm[firstline=47,lastline=65]{src/ocaml/caml\_start\_program.s}{caml\_start\_program}}


\begin{frame}{Back to C}{Clean up, complain and terminate}
\begin{columns}
\begin{column}{0.45\textwidth}
\begin{itemize}
\item Backtrace:
\begin{itemize}
\item \funcname{caml\_start\_program} $\rightarrow$ OCaml code
\item \funcname{caml\_startup\_common}
\item \funcname{caml\_startup\_exn}
\item \funcname{caml\_startup}
\item \funcname{caml\_main}
\item \funcname{main}
\end{itemize}
\smallskip
\item \funcname{caml\_startup} checks the return value of \funcname{caml\_startup\_exn}
\begin{itemize}
\item The return value has been specially marked by \funcname{caml\_start\_program} handler
\item Calls \funcname{caml\_fatal\_uncaught\_exception}
\end{itemize}
\item<2-> \funcname{caml\_fatal\_uncaught\_exception} either
\begin{itemize}
\item<2-> Calls the user custom uncaught exception handler, if set with \mintinline{ocaml}{Printexc.set_uncaught_exception_handler fn}\footnotemark
\item<2-> Or falls back to the default one
\item<2-> Which notifies about the uncaught exception
\end{itemize}
\item<4-> Terminates the program either with \funcname{abort} or with a non-zero exit code
\end{itemize}
\end{column}
\begin{column}{0.55\textwidth}
\only<1>{
\listc[firstline=84,lastline=89,highlightlines=88]{../ocaml/runtime/caml/mlvalues.h}{runtime/caml/mlvalues.h:84}
\listc[firstline=138,lastline=143,highlightlines={141-142}]{../ocaml/runtime/startup\_nat.c}{runtime/startup\_nat.c:138}
}%
\only<2>{
\listc[firstline=138,highlightlines={147,149}]{../ocaml/runtime/printexc.c}{runtime/printexc.c:138}
}%
\only<3>{
\listc[firstline=110,lastline=134,highlightlines=129]{../ocaml/runtime/printexc.c}{runtime/printexc.c:138}
}%
\only<4>{
\listc[firstline=138,highlightlines={152,154}]{../ocaml/runtime/printexc.c}{runtime/printexc.c:138}
}%
\end{column}
\end{columns}
\footnotetext[1]{\url{https://v2.ocaml.org/api/Printexc.html\#1\_Uncaughtexceptions}}
\end{frame}
}%
\only<2>{\providecommand\step{4}%
% Runtime's default exception handler
%
\subsection*{Runtime's default exception handler}
\frameSubsection{pictures/The\_Architect.png}{}


\begin{frame}{Default exception handler}
\begin{columns}
\begin{column}{0.5\textwidth}
\begin{itemize}
\item \localname{domain\_state->exn\_handler} points to a trap located before \funcname{entry}!
\item What installed this very first trap there?
\end{itemize}
\bigskip
\listocaml{../src/default/default.ml}{default/default.ml}
\bigskip
\inputminted[fontsize=\footnotesize]{text}{../src/default/default.out}
\end{column}
\begin{column}{0.5\textwidth}
\centering
\only<1>{\providecommand\step{0}\input{diagrams/default/default.tex}}%
\only<2>{\providecommand\step{4}\input{diagrams/default/default.tex}}%
\end{column}
\end{columns}
\end{frame}


\begin{frame}{Startup}{How did we get there by the way?}
\begin{columns}
\begin{column}{0.45\textwidth}
\begin{itemize}
\item The entry point address of the binary is \funcname{main} (\funcname{\_start}) from the runtime
\item The program starts like a C program:
\begin{itemize}
\item \funcname{caml\_start\_program}
\item \funcname{caml\_startup\_common}
\item \funcname{caml\_startup\_exn}
\item \funcname{caml\_startup}
\item \funcname{caml\_main}
\item \funcname{main}
\end{itemize}
%\item \funcname{caml\_startup\_common}: initializes GC, domains \dots
% Also: codefrag, locale, custom opeartions, frame_descr, signals, stack
\item \funcname{caml\_start\_program} is the transition point between the C realm and the OCaml one
\end{itemize}
\bigskip
\listocaml{../src/default/default.ml}{default/default.ml}
\end{column}
\begin{column}{0.55\textwidth}
\listasm[lastline=24,highlightlines={22-24}]{src/ocaml/caml\_start\_program.s}{runtime/amd64.S:604}
\end{column}
\end{columns}
\end{frame}


\begin{frame}{Setup to jump into OCaml entry point}{\funcname{caml\_start\_program} initializes the main fiber}
\begin{columns}
\begin{column}{0.6\textwidth}
\only<1,3,5>{
\begin{itemize}
\item<1-> Stores information to allow switching from OCaml stack to the C stack
\item<3-> Constructs the default exception handler
\item<5-> Switches to the OCaml stack and calls \funcname{caml\_program}
\end{itemize}
\bigskip
\listocaml{../src/default/default.ml}{default/default.ml}
}%
\only<2>{
\listasm[firstline=22,lastline=41,highlightlines={25-27}]{src/ocaml/caml\_start\_program.s}{caml\_start\_program}
}%
\only<4>{
\mintasm[firstline=22,lastline=41,highlightlines={31-38}]{src/ocaml/caml\_start\_program.s}
\listasm[firstline=66,lastline=72]{src/ocaml/caml\_start\_program.s}{caml\_start\_program}
}%
\only<6>{
\listasm[firstline=22,lastline=41,highlightlines={39-41}]{src/ocaml/caml\_start\_program.s}{caml\_start\_program}
}%
\end{column}
\begin{column}{0.4\textwidth}
\centering
\only<1-2>{\providecommand\step{2}\input{diagrams/default/default.tex}}%
\only<3-5>{\providecommand\step{3}\input{diagrams/default/default.tex}}%
\onslide*<6->{\providecommand\step{4}\input{diagrams/default/default.tex}}%
\end{column}
\end{columns}
\end{frame}

\begin{frame}{Executing the OCaml program}{And raising an exception without handler}
\begin{columns}
\begin{column}{0.6\textwidth}
\only<1-3>{
\begin{itemize}
\item<1-> Execution continues with OCaml code
\begin{itemize}
\item \funcname{camlDefault\_\_main\_269}
\item \funcname{camlDefault\_\_entry}
\item \funcname{caml\_program}
\end{itemize}
\item<1-> \funcname{camlDefault\_\_main\_269} raises a \typename{Failure} exception
\item<1-> \typename{Failure} is caught by \funcname{caml\_start\_program}
\begin{itemize}
\item<1-> The handler set the 2nd LSB of the exception value to \funcarg{1}
\item<3-> And then forces \funcname{caml\_start\_program} to terminate
\end{itemize}
\end{itemize}
}
\bigskip
\only<1,3>{\listocaml[highlightlines=3]{../src/default/default.ml}{default/default.ml}}%
\only<2>{\listasm[firstline=66,lastline=72,highlightlines=69]{src/ocaml/caml\_start\_program.s}{caml\_start\_program}}%
\end{column}
\begin{column}{0.4\textwidth}
\centering
\providecommand\step{4}\input{diagrams/default/default.tex}
\end{column}
\end{columns}
\end{frame}
%\only<>{\listasm[firstline=47,lastline=65]{src/ocaml/caml\_start\_program.s}{caml\_start\_program}}


\begin{frame}{Back to C}{Clean up, complain and terminate}
\begin{columns}
\begin{column}{0.45\textwidth}
\begin{itemize}
\item Backtrace:
\begin{itemize}
\item \funcname{caml\_start\_program} $\rightarrow$ OCaml code
\item \funcname{caml\_startup\_common}
\item \funcname{caml\_startup\_exn}
\item \funcname{caml\_startup}
\item \funcname{caml\_main}
\item \funcname{main}
\end{itemize}
\smallskip
\item \funcname{caml\_startup} checks the return value of \funcname{caml\_startup\_exn}
\begin{itemize}
\item The return value has been specially marked by \funcname{caml\_start\_program} handler
\item Calls \funcname{caml\_fatal\_uncaught\_exception}
\end{itemize}
\item<2-> \funcname{caml\_fatal\_uncaught\_exception} either
\begin{itemize}
\item<2-> Calls the user custom uncaught exception handler, if set with \mintinline{ocaml}{Printexc.set_uncaught_exception_handler fn}\footnotemark
\item<2-> Or falls back to the default one
\item<2-> Which notifies about the uncaught exception
\end{itemize}
\item<4-> Terminates the program either with \funcname{abort} or with a non-zero exit code
\end{itemize}
\end{column}
\begin{column}{0.55\textwidth}
\only<1>{
\listc[firstline=84,lastline=89,highlightlines=88]{../ocaml/runtime/caml/mlvalues.h}{runtime/caml/mlvalues.h:84}
\listc[firstline=138,lastline=143,highlightlines={141-142}]{../ocaml/runtime/startup\_nat.c}{runtime/startup\_nat.c:138}
}%
\only<2>{
\listc[firstline=138,highlightlines={147,149}]{../ocaml/runtime/printexc.c}{runtime/printexc.c:138}
}%
\only<3>{
\listc[firstline=110,lastline=134,highlightlines=129]{../ocaml/runtime/printexc.c}{runtime/printexc.c:138}
}%
\only<4>{
\listc[firstline=138,highlightlines={152,154}]{../ocaml/runtime/printexc.c}{runtime/printexc.c:138}
}%
\end{column}
\end{columns}
\footnotetext[1]{\url{https://v2.ocaml.org/api/Printexc.html\#1\_Uncaughtexceptions}}
\end{frame}
}%
\end{column}
\end{columns}
\end{frame}


\begin{frame}{Startup}{How did we get there by the way?}
\begin{columns}
\begin{column}{0.45\textwidth}
\begin{itemize}
\item The entry point address of the binary is \funcname{main} (\funcname{\_start}) from the runtime
\item The program starts like a C program:
\begin{itemize}
\item \funcname{caml\_start\_program}
\item \funcname{caml\_startup\_common}
\item \funcname{caml\_startup\_exn}
\item \funcname{caml\_startup}
\item \funcname{caml\_main}
\item \funcname{main}
\end{itemize}
%\item \funcname{caml\_startup\_common}: initializes GC, domains \dots
% Also: codefrag, locale, custom opeartions, frame_descr, signals, stack
\item \funcname{caml\_start\_program} is the transition point between the C realm and the OCaml one
\end{itemize}
\bigskip
\listocaml{../src/default/default.ml}{default/default.ml}
\end{column}
\begin{column}{0.55\textwidth}
\listasm[lastline=24,highlightlines={22-24}]{src/ocaml/caml\_start\_program.s}{runtime/amd64.S:604}
\end{column}
\end{columns}
\end{frame}


\begin{frame}{Setup to jump into OCaml entry point}{\funcname{caml\_start\_program} initializes the main fiber}
\begin{columns}
\begin{column}{0.6\textwidth}
\only<1,3,5>{
\begin{itemize}
\item<1-> Stores information to allow switching from OCaml stack to the C stack
\item<3-> Constructs the default exception handler
\item<5-> Switches to the OCaml stack and calls \funcname{caml\_program}
\end{itemize}
\bigskip
\listocaml{../src/default/default.ml}{default/default.ml}
}%
\only<2>{
\listasm[firstline=22,lastline=41,highlightlines={25-27}]{src/ocaml/caml\_start\_program.s}{caml\_start\_program}
}%
\only<4>{
\mintasm[firstline=22,lastline=41,highlightlines={31-38}]{src/ocaml/caml\_start\_program.s}
\listasm[firstline=66,lastline=72]{src/ocaml/caml\_start\_program.s}{caml\_start\_program}
}%
\only<6>{
\listasm[firstline=22,lastline=41,highlightlines={39-41}]{src/ocaml/caml\_start\_program.s}{caml\_start\_program}
}%
\end{column}
\begin{column}{0.4\textwidth}
\centering
\only<1-2>{\providecommand\step{2}%
% Runtime's default exception handler
%
\subsection*{Runtime's default exception handler}
\frameSubsection{pictures/The\_Architect.png}{}


\begin{frame}{Default exception handler}
\begin{columns}
\begin{column}{0.5\textwidth}
\begin{itemize}
\item \localname{domain\_state->exn\_handler} points to a trap located before \funcname{entry}!
\item What installed this very first trap there?
\end{itemize}
\bigskip
\listocaml{../src/default/default.ml}{default/default.ml}
\bigskip
\inputminted[fontsize=\footnotesize]{text}{../src/default/default.out}
\end{column}
\begin{column}{0.5\textwidth}
\centering
\only<1>{\providecommand\step{0}\input{diagrams/default/default.tex}}%
\only<2>{\providecommand\step{4}\input{diagrams/default/default.tex}}%
\end{column}
\end{columns}
\end{frame}


\begin{frame}{Startup}{How did we get there by the way?}
\begin{columns}
\begin{column}{0.45\textwidth}
\begin{itemize}
\item The entry point address of the binary is \funcname{main} (\funcname{\_start}) from the runtime
\item The program starts like a C program:
\begin{itemize}
\item \funcname{caml\_start\_program}
\item \funcname{caml\_startup\_common}
\item \funcname{caml\_startup\_exn}
\item \funcname{caml\_startup}
\item \funcname{caml\_main}
\item \funcname{main}
\end{itemize}
%\item \funcname{caml\_startup\_common}: initializes GC, domains \dots
% Also: codefrag, locale, custom opeartions, frame_descr, signals, stack
\item \funcname{caml\_start\_program} is the transition point between the C realm and the OCaml one
\end{itemize}
\bigskip
\listocaml{../src/default/default.ml}{default/default.ml}
\end{column}
\begin{column}{0.55\textwidth}
\listasm[lastline=24,highlightlines={22-24}]{src/ocaml/caml\_start\_program.s}{runtime/amd64.S:604}
\end{column}
\end{columns}
\end{frame}


\begin{frame}{Setup to jump into OCaml entry point}{\funcname{caml\_start\_program} initializes the main fiber}
\begin{columns}
\begin{column}{0.6\textwidth}
\only<1,3,5>{
\begin{itemize}
\item<1-> Stores information to allow switching from OCaml stack to the C stack
\item<3-> Constructs the default exception handler
\item<5-> Switches to the OCaml stack and calls \funcname{caml\_program}
\end{itemize}
\bigskip
\listocaml{../src/default/default.ml}{default/default.ml}
}%
\only<2>{
\listasm[firstline=22,lastline=41,highlightlines={25-27}]{src/ocaml/caml\_start\_program.s}{caml\_start\_program}
}%
\only<4>{
\mintasm[firstline=22,lastline=41,highlightlines={31-38}]{src/ocaml/caml\_start\_program.s}
\listasm[firstline=66,lastline=72]{src/ocaml/caml\_start\_program.s}{caml\_start\_program}
}%
\only<6>{
\listasm[firstline=22,lastline=41,highlightlines={39-41}]{src/ocaml/caml\_start\_program.s}{caml\_start\_program}
}%
\end{column}
\begin{column}{0.4\textwidth}
\centering
\only<1-2>{\providecommand\step{2}\input{diagrams/default/default.tex}}%
\only<3-5>{\providecommand\step{3}\input{diagrams/default/default.tex}}%
\onslide*<6->{\providecommand\step{4}\input{diagrams/default/default.tex}}%
\end{column}
\end{columns}
\end{frame}

\begin{frame}{Executing the OCaml program}{And raising an exception without handler}
\begin{columns}
\begin{column}{0.6\textwidth}
\only<1-3>{
\begin{itemize}
\item<1-> Execution continues with OCaml code
\begin{itemize}
\item \funcname{camlDefault\_\_main\_269}
\item \funcname{camlDefault\_\_entry}
\item \funcname{caml\_program}
\end{itemize}
\item<1-> \funcname{camlDefault\_\_main\_269} raises a \typename{Failure} exception
\item<1-> \typename{Failure} is caught by \funcname{caml\_start\_program}
\begin{itemize}
\item<1-> The handler set the 2nd LSB of the exception value to \funcarg{1}
\item<3-> And then forces \funcname{caml\_start\_program} to terminate
\end{itemize}
\end{itemize}
}
\bigskip
\only<1,3>{\listocaml[highlightlines=3]{../src/default/default.ml}{default/default.ml}}%
\only<2>{\listasm[firstline=66,lastline=72,highlightlines=69]{src/ocaml/caml\_start\_program.s}{caml\_start\_program}}%
\end{column}
\begin{column}{0.4\textwidth}
\centering
\providecommand\step{4}\input{diagrams/default/default.tex}
\end{column}
\end{columns}
\end{frame}
%\only<>{\listasm[firstline=47,lastline=65]{src/ocaml/caml\_start\_program.s}{caml\_start\_program}}


\begin{frame}{Back to C}{Clean up, complain and terminate}
\begin{columns}
\begin{column}{0.45\textwidth}
\begin{itemize}
\item Backtrace:
\begin{itemize}
\item \funcname{caml\_start\_program} $\rightarrow$ OCaml code
\item \funcname{caml\_startup\_common}
\item \funcname{caml\_startup\_exn}
\item \funcname{caml\_startup}
\item \funcname{caml\_main}
\item \funcname{main}
\end{itemize}
\smallskip
\item \funcname{caml\_startup} checks the return value of \funcname{caml\_startup\_exn}
\begin{itemize}
\item The return value has been specially marked by \funcname{caml\_start\_program} handler
\item Calls \funcname{caml\_fatal\_uncaught\_exception}
\end{itemize}
\item<2-> \funcname{caml\_fatal\_uncaught\_exception} either
\begin{itemize}
\item<2-> Calls the user custom uncaught exception handler, if set with \mintinline{ocaml}{Printexc.set_uncaught_exception_handler fn}\footnotemark
\item<2-> Or falls back to the default one
\item<2-> Which notifies about the uncaught exception
\end{itemize}
\item<4-> Terminates the program either with \funcname{abort} or with a non-zero exit code
\end{itemize}
\end{column}
\begin{column}{0.55\textwidth}
\only<1>{
\listc[firstline=84,lastline=89,highlightlines=88]{../ocaml/runtime/caml/mlvalues.h}{runtime/caml/mlvalues.h:84}
\listc[firstline=138,lastline=143,highlightlines={141-142}]{../ocaml/runtime/startup\_nat.c}{runtime/startup\_nat.c:138}
}%
\only<2>{
\listc[firstline=138,highlightlines={147,149}]{../ocaml/runtime/printexc.c}{runtime/printexc.c:138}
}%
\only<3>{
\listc[firstline=110,lastline=134,highlightlines=129]{../ocaml/runtime/printexc.c}{runtime/printexc.c:138}
}%
\only<4>{
\listc[firstline=138,highlightlines={152,154}]{../ocaml/runtime/printexc.c}{runtime/printexc.c:138}
}%
\end{column}
\end{columns}
\footnotetext[1]{\url{https://v2.ocaml.org/api/Printexc.html\#1\_Uncaughtexceptions}}
\end{frame}
}%
\only<3-5>{\providecommand\step{3}%
% Runtime's default exception handler
%
\subsection*{Runtime's default exception handler}
\frameSubsection{pictures/The\_Architect.png}{}


\begin{frame}{Default exception handler}
\begin{columns}
\begin{column}{0.5\textwidth}
\begin{itemize}
\item \localname{domain\_state->exn\_handler} points to a trap located before \funcname{entry}!
\item What installed this very first trap there?
\end{itemize}
\bigskip
\listocaml{../src/default/default.ml}{default/default.ml}
\bigskip
\inputminted[fontsize=\footnotesize]{text}{../src/default/default.out}
\end{column}
\begin{column}{0.5\textwidth}
\centering
\only<1>{\providecommand\step{0}\input{diagrams/default/default.tex}}%
\only<2>{\providecommand\step{4}\input{diagrams/default/default.tex}}%
\end{column}
\end{columns}
\end{frame}


\begin{frame}{Startup}{How did we get there by the way?}
\begin{columns}
\begin{column}{0.45\textwidth}
\begin{itemize}
\item The entry point address of the binary is \funcname{main} (\funcname{\_start}) from the runtime
\item The program starts like a C program:
\begin{itemize}
\item \funcname{caml\_start\_program}
\item \funcname{caml\_startup\_common}
\item \funcname{caml\_startup\_exn}
\item \funcname{caml\_startup}
\item \funcname{caml\_main}
\item \funcname{main}
\end{itemize}
%\item \funcname{caml\_startup\_common}: initializes GC, domains \dots
% Also: codefrag, locale, custom opeartions, frame_descr, signals, stack
\item \funcname{caml\_start\_program} is the transition point between the C realm and the OCaml one
\end{itemize}
\bigskip
\listocaml{../src/default/default.ml}{default/default.ml}
\end{column}
\begin{column}{0.55\textwidth}
\listasm[lastline=24,highlightlines={22-24}]{src/ocaml/caml\_start\_program.s}{runtime/amd64.S:604}
\end{column}
\end{columns}
\end{frame}


\begin{frame}{Setup to jump into OCaml entry point}{\funcname{caml\_start\_program} initializes the main fiber}
\begin{columns}
\begin{column}{0.6\textwidth}
\only<1,3,5>{
\begin{itemize}
\item<1-> Stores information to allow switching from OCaml stack to the C stack
\item<3-> Constructs the default exception handler
\item<5-> Switches to the OCaml stack and calls \funcname{caml\_program}
\end{itemize}
\bigskip
\listocaml{../src/default/default.ml}{default/default.ml}
}%
\only<2>{
\listasm[firstline=22,lastline=41,highlightlines={25-27}]{src/ocaml/caml\_start\_program.s}{caml\_start\_program}
}%
\only<4>{
\mintasm[firstline=22,lastline=41,highlightlines={31-38}]{src/ocaml/caml\_start\_program.s}
\listasm[firstline=66,lastline=72]{src/ocaml/caml\_start\_program.s}{caml\_start\_program}
}%
\only<6>{
\listasm[firstline=22,lastline=41,highlightlines={39-41}]{src/ocaml/caml\_start\_program.s}{caml\_start\_program}
}%
\end{column}
\begin{column}{0.4\textwidth}
\centering
\only<1-2>{\providecommand\step{2}\input{diagrams/default/default.tex}}%
\only<3-5>{\providecommand\step{3}\input{diagrams/default/default.tex}}%
\onslide*<6->{\providecommand\step{4}\input{diagrams/default/default.tex}}%
\end{column}
\end{columns}
\end{frame}

\begin{frame}{Executing the OCaml program}{And raising an exception without handler}
\begin{columns}
\begin{column}{0.6\textwidth}
\only<1-3>{
\begin{itemize}
\item<1-> Execution continues with OCaml code
\begin{itemize}
\item \funcname{camlDefault\_\_main\_269}
\item \funcname{camlDefault\_\_entry}
\item \funcname{caml\_program}
\end{itemize}
\item<1-> \funcname{camlDefault\_\_main\_269} raises a \typename{Failure} exception
\item<1-> \typename{Failure} is caught by \funcname{caml\_start\_program}
\begin{itemize}
\item<1-> The handler set the 2nd LSB of the exception value to \funcarg{1}
\item<3-> And then forces \funcname{caml\_start\_program} to terminate
\end{itemize}
\end{itemize}
}
\bigskip
\only<1,3>{\listocaml[highlightlines=3]{../src/default/default.ml}{default/default.ml}}%
\only<2>{\listasm[firstline=66,lastline=72,highlightlines=69]{src/ocaml/caml\_start\_program.s}{caml\_start\_program}}%
\end{column}
\begin{column}{0.4\textwidth}
\centering
\providecommand\step{4}\input{diagrams/default/default.tex}
\end{column}
\end{columns}
\end{frame}
%\only<>{\listasm[firstline=47,lastline=65]{src/ocaml/caml\_start\_program.s}{caml\_start\_program}}


\begin{frame}{Back to C}{Clean up, complain and terminate}
\begin{columns}
\begin{column}{0.45\textwidth}
\begin{itemize}
\item Backtrace:
\begin{itemize}
\item \funcname{caml\_start\_program} $\rightarrow$ OCaml code
\item \funcname{caml\_startup\_common}
\item \funcname{caml\_startup\_exn}
\item \funcname{caml\_startup}
\item \funcname{caml\_main}
\item \funcname{main}
\end{itemize}
\smallskip
\item \funcname{caml\_startup} checks the return value of \funcname{caml\_startup\_exn}
\begin{itemize}
\item The return value has been specially marked by \funcname{caml\_start\_program} handler
\item Calls \funcname{caml\_fatal\_uncaught\_exception}
\end{itemize}
\item<2-> \funcname{caml\_fatal\_uncaught\_exception} either
\begin{itemize}
\item<2-> Calls the user custom uncaught exception handler, if set with \mintinline{ocaml}{Printexc.set_uncaught_exception_handler fn}\footnotemark
\item<2-> Or falls back to the default one
\item<2-> Which notifies about the uncaught exception
\end{itemize}
\item<4-> Terminates the program either with \funcname{abort} or with a non-zero exit code
\end{itemize}
\end{column}
\begin{column}{0.55\textwidth}
\only<1>{
\listc[firstline=84,lastline=89,highlightlines=88]{../ocaml/runtime/caml/mlvalues.h}{runtime/caml/mlvalues.h:84}
\listc[firstline=138,lastline=143,highlightlines={141-142}]{../ocaml/runtime/startup\_nat.c}{runtime/startup\_nat.c:138}
}%
\only<2>{
\listc[firstline=138,highlightlines={147,149}]{../ocaml/runtime/printexc.c}{runtime/printexc.c:138}
}%
\only<3>{
\listc[firstline=110,lastline=134,highlightlines=129]{../ocaml/runtime/printexc.c}{runtime/printexc.c:138}
}%
\only<4>{
\listc[firstline=138,highlightlines={152,154}]{../ocaml/runtime/printexc.c}{runtime/printexc.c:138}
}%
\end{column}
\end{columns}
\footnotetext[1]{\url{https://v2.ocaml.org/api/Printexc.html\#1\_Uncaughtexceptions}}
\end{frame}
}%
\onslide*<6->{\providecommand\step{4}%
% Runtime's default exception handler
%
\subsection*{Runtime's default exception handler}
\frameSubsection{pictures/The\_Architect.png}{}


\begin{frame}{Default exception handler}
\begin{columns}
\begin{column}{0.5\textwidth}
\begin{itemize}
\item \localname{domain\_state->exn\_handler} points to a trap located before \funcname{entry}!
\item What installed this very first trap there?
\end{itemize}
\bigskip
\listocaml{../src/default/default.ml}{default/default.ml}
\bigskip
\inputminted[fontsize=\footnotesize]{text}{../src/default/default.out}
\end{column}
\begin{column}{0.5\textwidth}
\centering
\only<1>{\providecommand\step{0}\input{diagrams/default/default.tex}}%
\only<2>{\providecommand\step{4}\input{diagrams/default/default.tex}}%
\end{column}
\end{columns}
\end{frame}


\begin{frame}{Startup}{How did we get there by the way?}
\begin{columns}
\begin{column}{0.45\textwidth}
\begin{itemize}
\item The entry point address of the binary is \funcname{main} (\funcname{\_start}) from the runtime
\item The program starts like a C program:
\begin{itemize}
\item \funcname{caml\_start\_program}
\item \funcname{caml\_startup\_common}
\item \funcname{caml\_startup\_exn}
\item \funcname{caml\_startup}
\item \funcname{caml\_main}
\item \funcname{main}
\end{itemize}
%\item \funcname{caml\_startup\_common}: initializes GC, domains \dots
% Also: codefrag, locale, custom opeartions, frame_descr, signals, stack
\item \funcname{caml\_start\_program} is the transition point between the C realm and the OCaml one
\end{itemize}
\bigskip
\listocaml{../src/default/default.ml}{default/default.ml}
\end{column}
\begin{column}{0.55\textwidth}
\listasm[lastline=24,highlightlines={22-24}]{src/ocaml/caml\_start\_program.s}{runtime/amd64.S:604}
\end{column}
\end{columns}
\end{frame}


\begin{frame}{Setup to jump into OCaml entry point}{\funcname{caml\_start\_program} initializes the main fiber}
\begin{columns}
\begin{column}{0.6\textwidth}
\only<1,3,5>{
\begin{itemize}
\item<1-> Stores information to allow switching from OCaml stack to the C stack
\item<3-> Constructs the default exception handler
\item<5-> Switches to the OCaml stack and calls \funcname{caml\_program}
\end{itemize}
\bigskip
\listocaml{../src/default/default.ml}{default/default.ml}
}%
\only<2>{
\listasm[firstline=22,lastline=41,highlightlines={25-27}]{src/ocaml/caml\_start\_program.s}{caml\_start\_program}
}%
\only<4>{
\mintasm[firstline=22,lastline=41,highlightlines={31-38}]{src/ocaml/caml\_start\_program.s}
\listasm[firstline=66,lastline=72]{src/ocaml/caml\_start\_program.s}{caml\_start\_program}
}%
\only<6>{
\listasm[firstline=22,lastline=41,highlightlines={39-41}]{src/ocaml/caml\_start\_program.s}{caml\_start\_program}
}%
\end{column}
\begin{column}{0.4\textwidth}
\centering
\only<1-2>{\providecommand\step{2}\input{diagrams/default/default.tex}}%
\only<3-5>{\providecommand\step{3}\input{diagrams/default/default.tex}}%
\onslide*<6->{\providecommand\step{4}\input{diagrams/default/default.tex}}%
\end{column}
\end{columns}
\end{frame}

\begin{frame}{Executing the OCaml program}{And raising an exception without handler}
\begin{columns}
\begin{column}{0.6\textwidth}
\only<1-3>{
\begin{itemize}
\item<1-> Execution continues with OCaml code
\begin{itemize}
\item \funcname{camlDefault\_\_main\_269}
\item \funcname{camlDefault\_\_entry}
\item \funcname{caml\_program}
\end{itemize}
\item<1-> \funcname{camlDefault\_\_main\_269} raises a \typename{Failure} exception
\item<1-> \typename{Failure} is caught by \funcname{caml\_start\_program}
\begin{itemize}
\item<1-> The handler set the 2nd LSB of the exception value to \funcarg{1}
\item<3-> And then forces \funcname{caml\_start\_program} to terminate
\end{itemize}
\end{itemize}
}
\bigskip
\only<1,3>{\listocaml[highlightlines=3]{../src/default/default.ml}{default/default.ml}}%
\only<2>{\listasm[firstline=66,lastline=72,highlightlines=69]{src/ocaml/caml\_start\_program.s}{caml\_start\_program}}%
\end{column}
\begin{column}{0.4\textwidth}
\centering
\providecommand\step{4}\input{diagrams/default/default.tex}
\end{column}
\end{columns}
\end{frame}
%\only<>{\listasm[firstline=47,lastline=65]{src/ocaml/caml\_start\_program.s}{caml\_start\_program}}


\begin{frame}{Back to C}{Clean up, complain and terminate}
\begin{columns}
\begin{column}{0.45\textwidth}
\begin{itemize}
\item Backtrace:
\begin{itemize}
\item \funcname{caml\_start\_program} $\rightarrow$ OCaml code
\item \funcname{caml\_startup\_common}
\item \funcname{caml\_startup\_exn}
\item \funcname{caml\_startup}
\item \funcname{caml\_main}
\item \funcname{main}
\end{itemize}
\smallskip
\item \funcname{caml\_startup} checks the return value of \funcname{caml\_startup\_exn}
\begin{itemize}
\item The return value has been specially marked by \funcname{caml\_start\_program} handler
\item Calls \funcname{caml\_fatal\_uncaught\_exception}
\end{itemize}
\item<2-> \funcname{caml\_fatal\_uncaught\_exception} either
\begin{itemize}
\item<2-> Calls the user custom uncaught exception handler, if set with \mintinline{ocaml}{Printexc.set_uncaught_exception_handler fn}\footnotemark
\item<2-> Or falls back to the default one
\item<2-> Which notifies about the uncaught exception
\end{itemize}
\item<4-> Terminates the program either with \funcname{abort} or with a non-zero exit code
\end{itemize}
\end{column}
\begin{column}{0.55\textwidth}
\only<1>{
\listc[firstline=84,lastline=89,highlightlines=88]{../ocaml/runtime/caml/mlvalues.h}{runtime/caml/mlvalues.h:84}
\listc[firstline=138,lastline=143,highlightlines={141-142}]{../ocaml/runtime/startup\_nat.c}{runtime/startup\_nat.c:138}
}%
\only<2>{
\listc[firstline=138,highlightlines={147,149}]{../ocaml/runtime/printexc.c}{runtime/printexc.c:138}
}%
\only<3>{
\listc[firstline=110,lastline=134,highlightlines=129]{../ocaml/runtime/printexc.c}{runtime/printexc.c:138}
}%
\only<4>{
\listc[firstline=138,highlightlines={152,154}]{../ocaml/runtime/printexc.c}{runtime/printexc.c:138}
}%
\end{column}
\end{columns}
\footnotetext[1]{\url{https://v2.ocaml.org/api/Printexc.html\#1\_Uncaughtexceptions}}
\end{frame}
}%
\end{column}
\end{columns}
\end{frame}

\begin{frame}{Executing the OCaml program}{And raising an exception without handler}
\begin{columns}
\begin{column}{0.6\textwidth}
\only<1-3>{
\begin{itemize}
\item<1-> Execution continues with OCaml code
\begin{itemize}
\item \funcname{camlDefault\_\_main\_269}
\item \funcname{camlDefault\_\_entry}
\item \funcname{caml\_program}
\end{itemize}
\item<1-> \funcname{camlDefault\_\_main\_269} raises a \typename{Failure} exception
\item<1-> \typename{Failure} is caught by \funcname{caml\_start\_program}
\begin{itemize}
\item<1-> The handler set the 2nd LSB of the exception value to \funcarg{1}
\item<3-> And then forces \funcname{caml\_start\_program} to terminate
\end{itemize}
\end{itemize}
}
\bigskip
\only<1,3>{\listocaml[highlightlines=3]{../src/default/default.ml}{default/default.ml}}%
\only<2>{\listasm[firstline=66,lastline=72,highlightlines=69]{src/ocaml/caml\_start\_program.s}{caml\_start\_program}}%
\end{column}
\begin{column}{0.4\textwidth}
\centering
\providecommand\step{4}%
% Runtime's default exception handler
%
\subsection*{Runtime's default exception handler}
\frameSubsection{pictures/The\_Architect.png}{}


\begin{frame}{Default exception handler}
\begin{columns}
\begin{column}{0.5\textwidth}
\begin{itemize}
\item \localname{domain\_state->exn\_handler} points to a trap located before \funcname{entry}!
\item What installed this very first trap there?
\end{itemize}
\bigskip
\listocaml{../src/default/default.ml}{default/default.ml}
\bigskip
\inputminted[fontsize=\footnotesize]{text}{../src/default/default.out}
\end{column}
\begin{column}{0.5\textwidth}
\centering
\only<1>{\providecommand\step{0}\input{diagrams/default/default.tex}}%
\only<2>{\providecommand\step{4}\input{diagrams/default/default.tex}}%
\end{column}
\end{columns}
\end{frame}


\begin{frame}{Startup}{How did we get there by the way?}
\begin{columns}
\begin{column}{0.45\textwidth}
\begin{itemize}
\item The entry point address of the binary is \funcname{main} (\funcname{\_start}) from the runtime
\item The program starts like a C program:
\begin{itemize}
\item \funcname{caml\_start\_program}
\item \funcname{caml\_startup\_common}
\item \funcname{caml\_startup\_exn}
\item \funcname{caml\_startup}
\item \funcname{caml\_main}
\item \funcname{main}
\end{itemize}
%\item \funcname{caml\_startup\_common}: initializes GC, domains \dots
% Also: codefrag, locale, custom opeartions, frame_descr, signals, stack
\item \funcname{caml\_start\_program} is the transition point between the C realm and the OCaml one
\end{itemize}
\bigskip
\listocaml{../src/default/default.ml}{default/default.ml}
\end{column}
\begin{column}{0.55\textwidth}
\listasm[lastline=24,highlightlines={22-24}]{src/ocaml/caml\_start\_program.s}{runtime/amd64.S:604}
\end{column}
\end{columns}
\end{frame}


\begin{frame}{Setup to jump into OCaml entry point}{\funcname{caml\_start\_program} initializes the main fiber}
\begin{columns}
\begin{column}{0.6\textwidth}
\only<1,3,5>{
\begin{itemize}
\item<1-> Stores information to allow switching from OCaml stack to the C stack
\item<3-> Constructs the default exception handler
\item<5-> Switches to the OCaml stack and calls \funcname{caml\_program}
\end{itemize}
\bigskip
\listocaml{../src/default/default.ml}{default/default.ml}
}%
\only<2>{
\listasm[firstline=22,lastline=41,highlightlines={25-27}]{src/ocaml/caml\_start\_program.s}{caml\_start\_program}
}%
\only<4>{
\mintasm[firstline=22,lastline=41,highlightlines={31-38}]{src/ocaml/caml\_start\_program.s}
\listasm[firstline=66,lastline=72]{src/ocaml/caml\_start\_program.s}{caml\_start\_program}
}%
\only<6>{
\listasm[firstline=22,lastline=41,highlightlines={39-41}]{src/ocaml/caml\_start\_program.s}{caml\_start\_program}
}%
\end{column}
\begin{column}{0.4\textwidth}
\centering
\only<1-2>{\providecommand\step{2}\input{diagrams/default/default.tex}}%
\only<3-5>{\providecommand\step{3}\input{diagrams/default/default.tex}}%
\onslide*<6->{\providecommand\step{4}\input{diagrams/default/default.tex}}%
\end{column}
\end{columns}
\end{frame}

\begin{frame}{Executing the OCaml program}{And raising an exception without handler}
\begin{columns}
\begin{column}{0.6\textwidth}
\only<1-3>{
\begin{itemize}
\item<1-> Execution continues with OCaml code
\begin{itemize}
\item \funcname{camlDefault\_\_main\_269}
\item \funcname{camlDefault\_\_entry}
\item \funcname{caml\_program}
\end{itemize}
\item<1-> \funcname{camlDefault\_\_main\_269} raises a \typename{Failure} exception
\item<1-> \typename{Failure} is caught by \funcname{caml\_start\_program}
\begin{itemize}
\item<1-> The handler set the 2nd LSB of the exception value to \funcarg{1}
\item<3-> And then forces \funcname{caml\_start\_program} to terminate
\end{itemize}
\end{itemize}
}
\bigskip
\only<1,3>{\listocaml[highlightlines=3]{../src/default/default.ml}{default/default.ml}}%
\only<2>{\listasm[firstline=66,lastline=72,highlightlines=69]{src/ocaml/caml\_start\_program.s}{caml\_start\_program}}%
\end{column}
\begin{column}{0.4\textwidth}
\centering
\providecommand\step{4}\input{diagrams/default/default.tex}
\end{column}
\end{columns}
\end{frame}
%\only<>{\listasm[firstline=47,lastline=65]{src/ocaml/caml\_start\_program.s}{caml\_start\_program}}


\begin{frame}{Back to C}{Clean up, complain and terminate}
\begin{columns}
\begin{column}{0.45\textwidth}
\begin{itemize}
\item Backtrace:
\begin{itemize}
\item \funcname{caml\_start\_program} $\rightarrow$ OCaml code
\item \funcname{caml\_startup\_common}
\item \funcname{caml\_startup\_exn}
\item \funcname{caml\_startup}
\item \funcname{caml\_main}
\item \funcname{main}
\end{itemize}
\smallskip
\item \funcname{caml\_startup} checks the return value of \funcname{caml\_startup\_exn}
\begin{itemize}
\item The return value has been specially marked by \funcname{caml\_start\_program} handler
\item Calls \funcname{caml\_fatal\_uncaught\_exception}
\end{itemize}
\item<2-> \funcname{caml\_fatal\_uncaught\_exception} either
\begin{itemize}
\item<2-> Calls the user custom uncaught exception handler, if set with \mintinline{ocaml}{Printexc.set_uncaught_exception_handler fn}\footnotemark
\item<2-> Or falls back to the default one
\item<2-> Which notifies about the uncaught exception
\end{itemize}
\item<4-> Terminates the program either with \funcname{abort} or with a non-zero exit code
\end{itemize}
\end{column}
\begin{column}{0.55\textwidth}
\only<1>{
\listc[firstline=84,lastline=89,highlightlines=88]{../ocaml/runtime/caml/mlvalues.h}{runtime/caml/mlvalues.h:84}
\listc[firstline=138,lastline=143,highlightlines={141-142}]{../ocaml/runtime/startup\_nat.c}{runtime/startup\_nat.c:138}
}%
\only<2>{
\listc[firstline=138,highlightlines={147,149}]{../ocaml/runtime/printexc.c}{runtime/printexc.c:138}
}%
\only<3>{
\listc[firstline=110,lastline=134,highlightlines=129]{../ocaml/runtime/printexc.c}{runtime/printexc.c:138}
}%
\only<4>{
\listc[firstline=138,highlightlines={152,154}]{../ocaml/runtime/printexc.c}{runtime/printexc.c:138}
}%
\end{column}
\end{columns}
\footnotetext[1]{\url{https://v2.ocaml.org/api/Printexc.html\#1\_Uncaughtexceptions}}
\end{frame}

\end{column}
\end{columns}
\end{frame}
%\only<>{\listasm[firstline=47,lastline=65]{src/ocaml/caml\_start\_program.s}{caml\_start\_program}}


\begin{frame}{Back to C}{Clean up, complain and terminate}
\begin{columns}
\begin{column}{0.45\textwidth}
\begin{itemize}
\item Backtrace:
\begin{itemize}
\item \funcname{caml\_start\_program} $\rightarrow$ OCaml code
\item \funcname{caml\_startup\_common}
\item \funcname{caml\_startup\_exn}
\item \funcname{caml\_startup}
\item \funcname{caml\_main}
\item \funcname{main}
\end{itemize}
\smallskip
\item \funcname{caml\_startup} checks the return value of \funcname{caml\_startup\_exn}
\begin{itemize}
\item The return value has been specially marked by \funcname{caml\_start\_program} handler
\item Calls \funcname{caml\_fatal\_uncaught\_exception}
\end{itemize}
\item<2-> \funcname{caml\_fatal\_uncaught\_exception} either
\begin{itemize}
\item<2-> Calls the user custom uncaught exception handler, if set with \mintinline{ocaml}{Printexc.set_uncaught_exception_handler fn}\footnotemark
\item<2-> Or falls back to the default one
\item<2-> Which notifies about the uncaught exception
\end{itemize}
\item<4-> Terminates the program either with \funcname{abort} or with a non-zero exit code
\end{itemize}
\end{column}
\begin{column}{0.55\textwidth}
\only<1>{
\listc[firstline=84,lastline=89,highlightlines=88]{../ocaml/runtime/caml/mlvalues.h}{runtime/caml/mlvalues.h:84}
\listc[firstline=138,lastline=143,highlightlines={141-142}]{../ocaml/runtime/startup\_nat.c}{runtime/startup\_nat.c:138}
}%
\only<2>{
\listc[firstline=138,highlightlines={147,149}]{../ocaml/runtime/printexc.c}{runtime/printexc.c:138}
}%
\only<3>{
\listc[firstline=110,lastline=134,highlightlines=129]{../ocaml/runtime/printexc.c}{runtime/printexc.c:138}
}%
\only<4>{
\listc[firstline=138,highlightlines={152,154}]{../ocaml/runtime/printexc.c}{runtime/printexc.c:138}
}%
\end{column}
\end{columns}
\footnotetext[1]{\url{https://v2.ocaml.org/api/Printexc.html\#1\_Uncaughtexceptions}}
\end{frame}
}%
\onslide*<6->{\providecommand\step{4}%
% Runtime's default exception handler
%
\subsection*{Runtime's default exception handler}
\frameSubsection{pictures/The\_Architect.png}{}


\begin{frame}{Default exception handler}
\begin{columns}
\begin{column}{0.5\textwidth}
\begin{itemize}
\item \localname{domain\_state->exn\_handler} points to a trap located before \funcname{entry}!
\item What installed this very first trap there?
\end{itemize}
\bigskip
\listocaml{../src/default/default.ml}{default/default.ml}
\bigskip
\inputminted[fontsize=\footnotesize]{text}{../src/default/default.out}
\end{column}
\begin{column}{0.5\textwidth}
\centering
\only<1>{\providecommand\step{0}%
% Runtime's default exception handler
%
\subsection*{Runtime's default exception handler}
\frameSubsection{pictures/The\_Architect.png}{}


\begin{frame}{Default exception handler}
\begin{columns}
\begin{column}{0.5\textwidth}
\begin{itemize}
\item \localname{domain\_state->exn\_handler} points to a trap located before \funcname{entry}!
\item What installed this very first trap there?
\end{itemize}
\bigskip
\listocaml{../src/default/default.ml}{default/default.ml}
\bigskip
\inputminted[fontsize=\footnotesize]{text}{../src/default/default.out}
\end{column}
\begin{column}{0.5\textwidth}
\centering
\only<1>{\providecommand\step{0}\input{diagrams/default/default.tex}}%
\only<2>{\providecommand\step{4}\input{diagrams/default/default.tex}}%
\end{column}
\end{columns}
\end{frame}


\begin{frame}{Startup}{How did we get there by the way?}
\begin{columns}
\begin{column}{0.45\textwidth}
\begin{itemize}
\item The entry point address of the binary is \funcname{main} (\funcname{\_start}) from the runtime
\item The program starts like a C program:
\begin{itemize}
\item \funcname{caml\_start\_program}
\item \funcname{caml\_startup\_common}
\item \funcname{caml\_startup\_exn}
\item \funcname{caml\_startup}
\item \funcname{caml\_main}
\item \funcname{main}
\end{itemize}
%\item \funcname{caml\_startup\_common}: initializes GC, domains \dots
% Also: codefrag, locale, custom opeartions, frame_descr, signals, stack
\item \funcname{caml\_start\_program} is the transition point between the C realm and the OCaml one
\end{itemize}
\bigskip
\listocaml{../src/default/default.ml}{default/default.ml}
\end{column}
\begin{column}{0.55\textwidth}
\listasm[lastline=24,highlightlines={22-24}]{src/ocaml/caml\_start\_program.s}{runtime/amd64.S:604}
\end{column}
\end{columns}
\end{frame}


\begin{frame}{Setup to jump into OCaml entry point}{\funcname{caml\_start\_program} initializes the main fiber}
\begin{columns}
\begin{column}{0.6\textwidth}
\only<1,3,5>{
\begin{itemize}
\item<1-> Stores information to allow switching from OCaml stack to the C stack
\item<3-> Constructs the default exception handler
\item<5-> Switches to the OCaml stack and calls \funcname{caml\_program}
\end{itemize}
\bigskip
\listocaml{../src/default/default.ml}{default/default.ml}
}%
\only<2>{
\listasm[firstline=22,lastline=41,highlightlines={25-27}]{src/ocaml/caml\_start\_program.s}{caml\_start\_program}
}%
\only<4>{
\mintasm[firstline=22,lastline=41,highlightlines={31-38}]{src/ocaml/caml\_start\_program.s}
\listasm[firstline=66,lastline=72]{src/ocaml/caml\_start\_program.s}{caml\_start\_program}
}%
\only<6>{
\listasm[firstline=22,lastline=41,highlightlines={39-41}]{src/ocaml/caml\_start\_program.s}{caml\_start\_program}
}%
\end{column}
\begin{column}{0.4\textwidth}
\centering
\only<1-2>{\providecommand\step{2}\input{diagrams/default/default.tex}}%
\only<3-5>{\providecommand\step{3}\input{diagrams/default/default.tex}}%
\onslide*<6->{\providecommand\step{4}\input{diagrams/default/default.tex}}%
\end{column}
\end{columns}
\end{frame}

\begin{frame}{Executing the OCaml program}{And raising an exception without handler}
\begin{columns}
\begin{column}{0.6\textwidth}
\only<1-3>{
\begin{itemize}
\item<1-> Execution continues with OCaml code
\begin{itemize}
\item \funcname{camlDefault\_\_main\_269}
\item \funcname{camlDefault\_\_entry}
\item \funcname{caml\_program}
\end{itemize}
\item<1-> \funcname{camlDefault\_\_main\_269} raises a \typename{Failure} exception
\item<1-> \typename{Failure} is caught by \funcname{caml\_start\_program}
\begin{itemize}
\item<1-> The handler set the 2nd LSB of the exception value to \funcarg{1}
\item<3-> And then forces \funcname{caml\_start\_program} to terminate
\end{itemize}
\end{itemize}
}
\bigskip
\only<1,3>{\listocaml[highlightlines=3]{../src/default/default.ml}{default/default.ml}}%
\only<2>{\listasm[firstline=66,lastline=72,highlightlines=69]{src/ocaml/caml\_start\_program.s}{caml\_start\_program}}%
\end{column}
\begin{column}{0.4\textwidth}
\centering
\providecommand\step{4}\input{diagrams/default/default.tex}
\end{column}
\end{columns}
\end{frame}
%\only<>{\listasm[firstline=47,lastline=65]{src/ocaml/caml\_start\_program.s}{caml\_start\_program}}


\begin{frame}{Back to C}{Clean up, complain and terminate}
\begin{columns}
\begin{column}{0.45\textwidth}
\begin{itemize}
\item Backtrace:
\begin{itemize}
\item \funcname{caml\_start\_program} $\rightarrow$ OCaml code
\item \funcname{caml\_startup\_common}
\item \funcname{caml\_startup\_exn}
\item \funcname{caml\_startup}
\item \funcname{caml\_main}
\item \funcname{main}
\end{itemize}
\smallskip
\item \funcname{caml\_startup} checks the return value of \funcname{caml\_startup\_exn}
\begin{itemize}
\item The return value has been specially marked by \funcname{caml\_start\_program} handler
\item Calls \funcname{caml\_fatal\_uncaught\_exception}
\end{itemize}
\item<2-> \funcname{caml\_fatal\_uncaught\_exception} either
\begin{itemize}
\item<2-> Calls the user custom uncaught exception handler, if set with \mintinline{ocaml}{Printexc.set_uncaught_exception_handler fn}\footnotemark
\item<2-> Or falls back to the default one
\item<2-> Which notifies about the uncaught exception
\end{itemize}
\item<4-> Terminates the program either with \funcname{abort} or with a non-zero exit code
\end{itemize}
\end{column}
\begin{column}{0.55\textwidth}
\only<1>{
\listc[firstline=84,lastline=89,highlightlines=88]{../ocaml/runtime/caml/mlvalues.h}{runtime/caml/mlvalues.h:84}
\listc[firstline=138,lastline=143,highlightlines={141-142}]{../ocaml/runtime/startup\_nat.c}{runtime/startup\_nat.c:138}
}%
\only<2>{
\listc[firstline=138,highlightlines={147,149}]{../ocaml/runtime/printexc.c}{runtime/printexc.c:138}
}%
\only<3>{
\listc[firstline=110,lastline=134,highlightlines=129]{../ocaml/runtime/printexc.c}{runtime/printexc.c:138}
}%
\only<4>{
\listc[firstline=138,highlightlines={152,154}]{../ocaml/runtime/printexc.c}{runtime/printexc.c:138}
}%
\end{column}
\end{columns}
\footnotetext[1]{\url{https://v2.ocaml.org/api/Printexc.html\#1\_Uncaughtexceptions}}
\end{frame}
}%
\only<2>{\providecommand\step{4}%
% Runtime's default exception handler
%
\subsection*{Runtime's default exception handler}
\frameSubsection{pictures/The\_Architect.png}{}


\begin{frame}{Default exception handler}
\begin{columns}
\begin{column}{0.5\textwidth}
\begin{itemize}
\item \localname{domain\_state->exn\_handler} points to a trap located before \funcname{entry}!
\item What installed this very first trap there?
\end{itemize}
\bigskip
\listocaml{../src/default/default.ml}{default/default.ml}
\bigskip
\inputminted[fontsize=\footnotesize]{text}{../src/default/default.out}
\end{column}
\begin{column}{0.5\textwidth}
\centering
\only<1>{\providecommand\step{0}\input{diagrams/default/default.tex}}%
\only<2>{\providecommand\step{4}\input{diagrams/default/default.tex}}%
\end{column}
\end{columns}
\end{frame}


\begin{frame}{Startup}{How did we get there by the way?}
\begin{columns}
\begin{column}{0.45\textwidth}
\begin{itemize}
\item The entry point address of the binary is \funcname{main} (\funcname{\_start}) from the runtime
\item The program starts like a C program:
\begin{itemize}
\item \funcname{caml\_start\_program}
\item \funcname{caml\_startup\_common}
\item \funcname{caml\_startup\_exn}
\item \funcname{caml\_startup}
\item \funcname{caml\_main}
\item \funcname{main}
\end{itemize}
%\item \funcname{caml\_startup\_common}: initializes GC, domains \dots
% Also: codefrag, locale, custom opeartions, frame_descr, signals, stack
\item \funcname{caml\_start\_program} is the transition point between the C realm and the OCaml one
\end{itemize}
\bigskip
\listocaml{../src/default/default.ml}{default/default.ml}
\end{column}
\begin{column}{0.55\textwidth}
\listasm[lastline=24,highlightlines={22-24}]{src/ocaml/caml\_start\_program.s}{runtime/amd64.S:604}
\end{column}
\end{columns}
\end{frame}


\begin{frame}{Setup to jump into OCaml entry point}{\funcname{caml\_start\_program} initializes the main fiber}
\begin{columns}
\begin{column}{0.6\textwidth}
\only<1,3,5>{
\begin{itemize}
\item<1-> Stores information to allow switching from OCaml stack to the C stack
\item<3-> Constructs the default exception handler
\item<5-> Switches to the OCaml stack and calls \funcname{caml\_program}
\end{itemize}
\bigskip
\listocaml{../src/default/default.ml}{default/default.ml}
}%
\only<2>{
\listasm[firstline=22,lastline=41,highlightlines={25-27}]{src/ocaml/caml\_start\_program.s}{caml\_start\_program}
}%
\only<4>{
\mintasm[firstline=22,lastline=41,highlightlines={31-38}]{src/ocaml/caml\_start\_program.s}
\listasm[firstline=66,lastline=72]{src/ocaml/caml\_start\_program.s}{caml\_start\_program}
}%
\only<6>{
\listasm[firstline=22,lastline=41,highlightlines={39-41}]{src/ocaml/caml\_start\_program.s}{caml\_start\_program}
}%
\end{column}
\begin{column}{0.4\textwidth}
\centering
\only<1-2>{\providecommand\step{2}\input{diagrams/default/default.tex}}%
\only<3-5>{\providecommand\step{3}\input{diagrams/default/default.tex}}%
\onslide*<6->{\providecommand\step{4}\input{diagrams/default/default.tex}}%
\end{column}
\end{columns}
\end{frame}

\begin{frame}{Executing the OCaml program}{And raising an exception without handler}
\begin{columns}
\begin{column}{0.6\textwidth}
\only<1-3>{
\begin{itemize}
\item<1-> Execution continues with OCaml code
\begin{itemize}
\item \funcname{camlDefault\_\_main\_269}
\item \funcname{camlDefault\_\_entry}
\item \funcname{caml\_program}
\end{itemize}
\item<1-> \funcname{camlDefault\_\_main\_269} raises a \typename{Failure} exception
\item<1-> \typename{Failure} is caught by \funcname{caml\_start\_program}
\begin{itemize}
\item<1-> The handler set the 2nd LSB of the exception value to \funcarg{1}
\item<3-> And then forces \funcname{caml\_start\_program} to terminate
\end{itemize}
\end{itemize}
}
\bigskip
\only<1,3>{\listocaml[highlightlines=3]{../src/default/default.ml}{default/default.ml}}%
\only<2>{\listasm[firstline=66,lastline=72,highlightlines=69]{src/ocaml/caml\_start\_program.s}{caml\_start\_program}}%
\end{column}
\begin{column}{0.4\textwidth}
\centering
\providecommand\step{4}\input{diagrams/default/default.tex}
\end{column}
\end{columns}
\end{frame}
%\only<>{\listasm[firstline=47,lastline=65]{src/ocaml/caml\_start\_program.s}{caml\_start\_program}}


\begin{frame}{Back to C}{Clean up, complain and terminate}
\begin{columns}
\begin{column}{0.45\textwidth}
\begin{itemize}
\item Backtrace:
\begin{itemize}
\item \funcname{caml\_start\_program} $\rightarrow$ OCaml code
\item \funcname{caml\_startup\_common}
\item \funcname{caml\_startup\_exn}
\item \funcname{caml\_startup}
\item \funcname{caml\_main}
\item \funcname{main}
\end{itemize}
\smallskip
\item \funcname{caml\_startup} checks the return value of \funcname{caml\_startup\_exn}
\begin{itemize}
\item The return value has been specially marked by \funcname{caml\_start\_program} handler
\item Calls \funcname{caml\_fatal\_uncaught\_exception}
\end{itemize}
\item<2-> \funcname{caml\_fatal\_uncaught\_exception} either
\begin{itemize}
\item<2-> Calls the user custom uncaught exception handler, if set with \mintinline{ocaml}{Printexc.set_uncaught_exception_handler fn}\footnotemark
\item<2-> Or falls back to the default one
\item<2-> Which notifies about the uncaught exception
\end{itemize}
\item<4-> Terminates the program either with \funcname{abort} or with a non-zero exit code
\end{itemize}
\end{column}
\begin{column}{0.55\textwidth}
\only<1>{
\listc[firstline=84,lastline=89,highlightlines=88]{../ocaml/runtime/caml/mlvalues.h}{runtime/caml/mlvalues.h:84}
\listc[firstline=138,lastline=143,highlightlines={141-142}]{../ocaml/runtime/startup\_nat.c}{runtime/startup\_nat.c:138}
}%
\only<2>{
\listc[firstline=138,highlightlines={147,149}]{../ocaml/runtime/printexc.c}{runtime/printexc.c:138}
}%
\only<3>{
\listc[firstline=110,lastline=134,highlightlines=129]{../ocaml/runtime/printexc.c}{runtime/printexc.c:138}
}%
\only<4>{
\listc[firstline=138,highlightlines={152,154}]{../ocaml/runtime/printexc.c}{runtime/printexc.c:138}
}%
\end{column}
\end{columns}
\footnotetext[1]{\url{https://v2.ocaml.org/api/Printexc.html\#1\_Uncaughtexceptions}}
\end{frame}
}%
\end{column}
\end{columns}
\end{frame}


\begin{frame}{Startup}{How did we get there by the way?}
\begin{columns}
\begin{column}{0.45\textwidth}
\begin{itemize}
\item The entry point address of the binary is \funcname{main} (\funcname{\_start}) from the runtime
\item The program starts like a C program:
\begin{itemize}
\item \funcname{caml\_start\_program}
\item \funcname{caml\_startup\_common}
\item \funcname{caml\_startup\_exn}
\item \funcname{caml\_startup}
\item \funcname{caml\_main}
\item \funcname{main}
\end{itemize}
%\item \funcname{caml\_startup\_common}: initializes GC, domains \dots
% Also: codefrag, locale, custom opeartions, frame_descr, signals, stack
\item \funcname{caml\_start\_program} is the transition point between the C realm and the OCaml one
\end{itemize}
\bigskip
\listocaml{../src/default/default.ml}{default/default.ml}
\end{column}
\begin{column}{0.55\textwidth}
\listasm[lastline=24,highlightlines={22-24}]{src/ocaml/caml\_start\_program.s}{runtime/amd64.S:604}
\end{column}
\end{columns}
\end{frame}


\begin{frame}{Setup to jump into OCaml entry point}{\funcname{caml\_start\_program} initializes the main fiber}
\begin{columns}
\begin{column}{0.6\textwidth}
\only<1,3,5>{
\begin{itemize}
\item<1-> Stores information to allow switching from OCaml stack to the C stack
\item<3-> Constructs the default exception handler
\item<5-> Switches to the OCaml stack and calls \funcname{caml\_program}
\end{itemize}
\bigskip
\listocaml{../src/default/default.ml}{default/default.ml}
}%
\only<2>{
\listasm[firstline=22,lastline=41,highlightlines={25-27}]{src/ocaml/caml\_start\_program.s}{caml\_start\_program}
}%
\only<4>{
\mintasm[firstline=22,lastline=41,highlightlines={31-38}]{src/ocaml/caml\_start\_program.s}
\listasm[firstline=66,lastline=72]{src/ocaml/caml\_start\_program.s}{caml\_start\_program}
}%
\only<6>{
\listasm[firstline=22,lastline=41,highlightlines={39-41}]{src/ocaml/caml\_start\_program.s}{caml\_start\_program}
}%
\end{column}
\begin{column}{0.4\textwidth}
\centering
\only<1-2>{\providecommand\step{2}%
% Runtime's default exception handler
%
\subsection*{Runtime's default exception handler}
\frameSubsection{pictures/The\_Architect.png}{}


\begin{frame}{Default exception handler}
\begin{columns}
\begin{column}{0.5\textwidth}
\begin{itemize}
\item \localname{domain\_state->exn\_handler} points to a trap located before \funcname{entry}!
\item What installed this very first trap there?
\end{itemize}
\bigskip
\listocaml{../src/default/default.ml}{default/default.ml}
\bigskip
\inputminted[fontsize=\footnotesize]{text}{../src/default/default.out}
\end{column}
\begin{column}{0.5\textwidth}
\centering
\only<1>{\providecommand\step{0}\input{diagrams/default/default.tex}}%
\only<2>{\providecommand\step{4}\input{diagrams/default/default.tex}}%
\end{column}
\end{columns}
\end{frame}


\begin{frame}{Startup}{How did we get there by the way?}
\begin{columns}
\begin{column}{0.45\textwidth}
\begin{itemize}
\item The entry point address of the binary is \funcname{main} (\funcname{\_start}) from the runtime
\item The program starts like a C program:
\begin{itemize}
\item \funcname{caml\_start\_program}
\item \funcname{caml\_startup\_common}
\item \funcname{caml\_startup\_exn}
\item \funcname{caml\_startup}
\item \funcname{caml\_main}
\item \funcname{main}
\end{itemize}
%\item \funcname{caml\_startup\_common}: initializes GC, domains \dots
% Also: codefrag, locale, custom opeartions, frame_descr, signals, stack
\item \funcname{caml\_start\_program} is the transition point between the C realm and the OCaml one
\end{itemize}
\bigskip
\listocaml{../src/default/default.ml}{default/default.ml}
\end{column}
\begin{column}{0.55\textwidth}
\listasm[lastline=24,highlightlines={22-24}]{src/ocaml/caml\_start\_program.s}{runtime/amd64.S:604}
\end{column}
\end{columns}
\end{frame}


\begin{frame}{Setup to jump into OCaml entry point}{\funcname{caml\_start\_program} initializes the main fiber}
\begin{columns}
\begin{column}{0.6\textwidth}
\only<1,3,5>{
\begin{itemize}
\item<1-> Stores information to allow switching from OCaml stack to the C stack
\item<3-> Constructs the default exception handler
\item<5-> Switches to the OCaml stack and calls \funcname{caml\_program}
\end{itemize}
\bigskip
\listocaml{../src/default/default.ml}{default/default.ml}
}%
\only<2>{
\listasm[firstline=22,lastline=41,highlightlines={25-27}]{src/ocaml/caml\_start\_program.s}{caml\_start\_program}
}%
\only<4>{
\mintasm[firstline=22,lastline=41,highlightlines={31-38}]{src/ocaml/caml\_start\_program.s}
\listasm[firstline=66,lastline=72]{src/ocaml/caml\_start\_program.s}{caml\_start\_program}
}%
\only<6>{
\listasm[firstline=22,lastline=41,highlightlines={39-41}]{src/ocaml/caml\_start\_program.s}{caml\_start\_program}
}%
\end{column}
\begin{column}{0.4\textwidth}
\centering
\only<1-2>{\providecommand\step{2}\input{diagrams/default/default.tex}}%
\only<3-5>{\providecommand\step{3}\input{diagrams/default/default.tex}}%
\onslide*<6->{\providecommand\step{4}\input{diagrams/default/default.tex}}%
\end{column}
\end{columns}
\end{frame}

\begin{frame}{Executing the OCaml program}{And raising an exception without handler}
\begin{columns}
\begin{column}{0.6\textwidth}
\only<1-3>{
\begin{itemize}
\item<1-> Execution continues with OCaml code
\begin{itemize}
\item \funcname{camlDefault\_\_main\_269}
\item \funcname{camlDefault\_\_entry}
\item \funcname{caml\_program}
\end{itemize}
\item<1-> \funcname{camlDefault\_\_main\_269} raises a \typename{Failure} exception
\item<1-> \typename{Failure} is caught by \funcname{caml\_start\_program}
\begin{itemize}
\item<1-> The handler set the 2nd LSB of the exception value to \funcarg{1}
\item<3-> And then forces \funcname{caml\_start\_program} to terminate
\end{itemize}
\end{itemize}
}
\bigskip
\only<1,3>{\listocaml[highlightlines=3]{../src/default/default.ml}{default/default.ml}}%
\only<2>{\listasm[firstline=66,lastline=72,highlightlines=69]{src/ocaml/caml\_start\_program.s}{caml\_start\_program}}%
\end{column}
\begin{column}{0.4\textwidth}
\centering
\providecommand\step{4}\input{diagrams/default/default.tex}
\end{column}
\end{columns}
\end{frame}
%\only<>{\listasm[firstline=47,lastline=65]{src/ocaml/caml\_start\_program.s}{caml\_start\_program}}


\begin{frame}{Back to C}{Clean up, complain and terminate}
\begin{columns}
\begin{column}{0.45\textwidth}
\begin{itemize}
\item Backtrace:
\begin{itemize}
\item \funcname{caml\_start\_program} $\rightarrow$ OCaml code
\item \funcname{caml\_startup\_common}
\item \funcname{caml\_startup\_exn}
\item \funcname{caml\_startup}
\item \funcname{caml\_main}
\item \funcname{main}
\end{itemize}
\smallskip
\item \funcname{caml\_startup} checks the return value of \funcname{caml\_startup\_exn}
\begin{itemize}
\item The return value has been specially marked by \funcname{caml\_start\_program} handler
\item Calls \funcname{caml\_fatal\_uncaught\_exception}
\end{itemize}
\item<2-> \funcname{caml\_fatal\_uncaught\_exception} either
\begin{itemize}
\item<2-> Calls the user custom uncaught exception handler, if set with \mintinline{ocaml}{Printexc.set_uncaught_exception_handler fn}\footnotemark
\item<2-> Or falls back to the default one
\item<2-> Which notifies about the uncaught exception
\end{itemize}
\item<4-> Terminates the program either with \funcname{abort} or with a non-zero exit code
\end{itemize}
\end{column}
\begin{column}{0.55\textwidth}
\only<1>{
\listc[firstline=84,lastline=89,highlightlines=88]{../ocaml/runtime/caml/mlvalues.h}{runtime/caml/mlvalues.h:84}
\listc[firstline=138,lastline=143,highlightlines={141-142}]{../ocaml/runtime/startup\_nat.c}{runtime/startup\_nat.c:138}
}%
\only<2>{
\listc[firstline=138,highlightlines={147,149}]{../ocaml/runtime/printexc.c}{runtime/printexc.c:138}
}%
\only<3>{
\listc[firstline=110,lastline=134,highlightlines=129]{../ocaml/runtime/printexc.c}{runtime/printexc.c:138}
}%
\only<4>{
\listc[firstline=138,highlightlines={152,154}]{../ocaml/runtime/printexc.c}{runtime/printexc.c:138}
}%
\end{column}
\end{columns}
\footnotetext[1]{\url{https://v2.ocaml.org/api/Printexc.html\#1\_Uncaughtexceptions}}
\end{frame}
}%
\only<3-5>{\providecommand\step{3}%
% Runtime's default exception handler
%
\subsection*{Runtime's default exception handler}
\frameSubsection{pictures/The\_Architect.png}{}


\begin{frame}{Default exception handler}
\begin{columns}
\begin{column}{0.5\textwidth}
\begin{itemize}
\item \localname{domain\_state->exn\_handler} points to a trap located before \funcname{entry}!
\item What installed this very first trap there?
\end{itemize}
\bigskip
\listocaml{../src/default/default.ml}{default/default.ml}
\bigskip
\inputminted[fontsize=\footnotesize]{text}{../src/default/default.out}
\end{column}
\begin{column}{0.5\textwidth}
\centering
\only<1>{\providecommand\step{0}\input{diagrams/default/default.tex}}%
\only<2>{\providecommand\step{4}\input{diagrams/default/default.tex}}%
\end{column}
\end{columns}
\end{frame}


\begin{frame}{Startup}{How did we get there by the way?}
\begin{columns}
\begin{column}{0.45\textwidth}
\begin{itemize}
\item The entry point address of the binary is \funcname{main} (\funcname{\_start}) from the runtime
\item The program starts like a C program:
\begin{itemize}
\item \funcname{caml\_start\_program}
\item \funcname{caml\_startup\_common}
\item \funcname{caml\_startup\_exn}
\item \funcname{caml\_startup}
\item \funcname{caml\_main}
\item \funcname{main}
\end{itemize}
%\item \funcname{caml\_startup\_common}: initializes GC, domains \dots
% Also: codefrag, locale, custom opeartions, frame_descr, signals, stack
\item \funcname{caml\_start\_program} is the transition point between the C realm and the OCaml one
\end{itemize}
\bigskip
\listocaml{../src/default/default.ml}{default/default.ml}
\end{column}
\begin{column}{0.55\textwidth}
\listasm[lastline=24,highlightlines={22-24}]{src/ocaml/caml\_start\_program.s}{runtime/amd64.S:604}
\end{column}
\end{columns}
\end{frame}


\begin{frame}{Setup to jump into OCaml entry point}{\funcname{caml\_start\_program} initializes the main fiber}
\begin{columns}
\begin{column}{0.6\textwidth}
\only<1,3,5>{
\begin{itemize}
\item<1-> Stores information to allow switching from OCaml stack to the C stack
\item<3-> Constructs the default exception handler
\item<5-> Switches to the OCaml stack and calls \funcname{caml\_program}
\end{itemize}
\bigskip
\listocaml{../src/default/default.ml}{default/default.ml}
}%
\only<2>{
\listasm[firstline=22,lastline=41,highlightlines={25-27}]{src/ocaml/caml\_start\_program.s}{caml\_start\_program}
}%
\only<4>{
\mintasm[firstline=22,lastline=41,highlightlines={31-38}]{src/ocaml/caml\_start\_program.s}
\listasm[firstline=66,lastline=72]{src/ocaml/caml\_start\_program.s}{caml\_start\_program}
}%
\only<6>{
\listasm[firstline=22,lastline=41,highlightlines={39-41}]{src/ocaml/caml\_start\_program.s}{caml\_start\_program}
}%
\end{column}
\begin{column}{0.4\textwidth}
\centering
\only<1-2>{\providecommand\step{2}\input{diagrams/default/default.tex}}%
\only<3-5>{\providecommand\step{3}\input{diagrams/default/default.tex}}%
\onslide*<6->{\providecommand\step{4}\input{diagrams/default/default.tex}}%
\end{column}
\end{columns}
\end{frame}

\begin{frame}{Executing the OCaml program}{And raising an exception without handler}
\begin{columns}
\begin{column}{0.6\textwidth}
\only<1-3>{
\begin{itemize}
\item<1-> Execution continues with OCaml code
\begin{itemize}
\item \funcname{camlDefault\_\_main\_269}
\item \funcname{camlDefault\_\_entry}
\item \funcname{caml\_program}
\end{itemize}
\item<1-> \funcname{camlDefault\_\_main\_269} raises a \typename{Failure} exception
\item<1-> \typename{Failure} is caught by \funcname{caml\_start\_program}
\begin{itemize}
\item<1-> The handler set the 2nd LSB of the exception value to \funcarg{1}
\item<3-> And then forces \funcname{caml\_start\_program} to terminate
\end{itemize}
\end{itemize}
}
\bigskip
\only<1,3>{\listocaml[highlightlines=3]{../src/default/default.ml}{default/default.ml}}%
\only<2>{\listasm[firstline=66,lastline=72,highlightlines=69]{src/ocaml/caml\_start\_program.s}{caml\_start\_program}}%
\end{column}
\begin{column}{0.4\textwidth}
\centering
\providecommand\step{4}\input{diagrams/default/default.tex}
\end{column}
\end{columns}
\end{frame}
%\only<>{\listasm[firstline=47,lastline=65]{src/ocaml/caml\_start\_program.s}{caml\_start\_program}}


\begin{frame}{Back to C}{Clean up, complain and terminate}
\begin{columns}
\begin{column}{0.45\textwidth}
\begin{itemize}
\item Backtrace:
\begin{itemize}
\item \funcname{caml\_start\_program} $\rightarrow$ OCaml code
\item \funcname{caml\_startup\_common}
\item \funcname{caml\_startup\_exn}
\item \funcname{caml\_startup}
\item \funcname{caml\_main}
\item \funcname{main}
\end{itemize}
\smallskip
\item \funcname{caml\_startup} checks the return value of \funcname{caml\_startup\_exn}
\begin{itemize}
\item The return value has been specially marked by \funcname{caml\_start\_program} handler
\item Calls \funcname{caml\_fatal\_uncaught\_exception}
\end{itemize}
\item<2-> \funcname{caml\_fatal\_uncaught\_exception} either
\begin{itemize}
\item<2-> Calls the user custom uncaught exception handler, if set with \mintinline{ocaml}{Printexc.set_uncaught_exception_handler fn}\footnotemark
\item<2-> Or falls back to the default one
\item<2-> Which notifies about the uncaught exception
\end{itemize}
\item<4-> Terminates the program either with \funcname{abort} or with a non-zero exit code
\end{itemize}
\end{column}
\begin{column}{0.55\textwidth}
\only<1>{
\listc[firstline=84,lastline=89,highlightlines=88]{../ocaml/runtime/caml/mlvalues.h}{runtime/caml/mlvalues.h:84}
\listc[firstline=138,lastline=143,highlightlines={141-142}]{../ocaml/runtime/startup\_nat.c}{runtime/startup\_nat.c:138}
}%
\only<2>{
\listc[firstline=138,highlightlines={147,149}]{../ocaml/runtime/printexc.c}{runtime/printexc.c:138}
}%
\only<3>{
\listc[firstline=110,lastline=134,highlightlines=129]{../ocaml/runtime/printexc.c}{runtime/printexc.c:138}
}%
\only<4>{
\listc[firstline=138,highlightlines={152,154}]{../ocaml/runtime/printexc.c}{runtime/printexc.c:138}
}%
\end{column}
\end{columns}
\footnotetext[1]{\url{https://v2.ocaml.org/api/Printexc.html\#1\_Uncaughtexceptions}}
\end{frame}
}%
\onslide*<6->{\providecommand\step{4}%
% Runtime's default exception handler
%
\subsection*{Runtime's default exception handler}
\frameSubsection{pictures/The\_Architect.png}{}


\begin{frame}{Default exception handler}
\begin{columns}
\begin{column}{0.5\textwidth}
\begin{itemize}
\item \localname{domain\_state->exn\_handler} points to a trap located before \funcname{entry}!
\item What installed this very first trap there?
\end{itemize}
\bigskip
\listocaml{../src/default/default.ml}{default/default.ml}
\bigskip
\inputminted[fontsize=\footnotesize]{text}{../src/default/default.out}
\end{column}
\begin{column}{0.5\textwidth}
\centering
\only<1>{\providecommand\step{0}\input{diagrams/default/default.tex}}%
\only<2>{\providecommand\step{4}\input{diagrams/default/default.tex}}%
\end{column}
\end{columns}
\end{frame}


\begin{frame}{Startup}{How did we get there by the way?}
\begin{columns}
\begin{column}{0.45\textwidth}
\begin{itemize}
\item The entry point address of the binary is \funcname{main} (\funcname{\_start}) from the runtime
\item The program starts like a C program:
\begin{itemize}
\item \funcname{caml\_start\_program}
\item \funcname{caml\_startup\_common}
\item \funcname{caml\_startup\_exn}
\item \funcname{caml\_startup}
\item \funcname{caml\_main}
\item \funcname{main}
\end{itemize}
%\item \funcname{caml\_startup\_common}: initializes GC, domains \dots
% Also: codefrag, locale, custom opeartions, frame_descr, signals, stack
\item \funcname{caml\_start\_program} is the transition point between the C realm and the OCaml one
\end{itemize}
\bigskip
\listocaml{../src/default/default.ml}{default/default.ml}
\end{column}
\begin{column}{0.55\textwidth}
\listasm[lastline=24,highlightlines={22-24}]{src/ocaml/caml\_start\_program.s}{runtime/amd64.S:604}
\end{column}
\end{columns}
\end{frame}


\begin{frame}{Setup to jump into OCaml entry point}{\funcname{caml\_start\_program} initializes the main fiber}
\begin{columns}
\begin{column}{0.6\textwidth}
\only<1,3,5>{
\begin{itemize}
\item<1-> Stores information to allow switching from OCaml stack to the C stack
\item<3-> Constructs the default exception handler
\item<5-> Switches to the OCaml stack and calls \funcname{caml\_program}
\end{itemize}
\bigskip
\listocaml{../src/default/default.ml}{default/default.ml}
}%
\only<2>{
\listasm[firstline=22,lastline=41,highlightlines={25-27}]{src/ocaml/caml\_start\_program.s}{caml\_start\_program}
}%
\only<4>{
\mintasm[firstline=22,lastline=41,highlightlines={31-38}]{src/ocaml/caml\_start\_program.s}
\listasm[firstline=66,lastline=72]{src/ocaml/caml\_start\_program.s}{caml\_start\_program}
}%
\only<6>{
\listasm[firstline=22,lastline=41,highlightlines={39-41}]{src/ocaml/caml\_start\_program.s}{caml\_start\_program}
}%
\end{column}
\begin{column}{0.4\textwidth}
\centering
\only<1-2>{\providecommand\step{2}\input{diagrams/default/default.tex}}%
\only<3-5>{\providecommand\step{3}\input{diagrams/default/default.tex}}%
\onslide*<6->{\providecommand\step{4}\input{diagrams/default/default.tex}}%
\end{column}
\end{columns}
\end{frame}

\begin{frame}{Executing the OCaml program}{And raising an exception without handler}
\begin{columns}
\begin{column}{0.6\textwidth}
\only<1-3>{
\begin{itemize}
\item<1-> Execution continues with OCaml code
\begin{itemize}
\item \funcname{camlDefault\_\_main\_269}
\item \funcname{camlDefault\_\_entry}
\item \funcname{caml\_program}
\end{itemize}
\item<1-> \funcname{camlDefault\_\_main\_269} raises a \typename{Failure} exception
\item<1-> \typename{Failure} is caught by \funcname{caml\_start\_program}
\begin{itemize}
\item<1-> The handler set the 2nd LSB of the exception value to \funcarg{1}
\item<3-> And then forces \funcname{caml\_start\_program} to terminate
\end{itemize}
\end{itemize}
}
\bigskip
\only<1,3>{\listocaml[highlightlines=3]{../src/default/default.ml}{default/default.ml}}%
\only<2>{\listasm[firstline=66,lastline=72,highlightlines=69]{src/ocaml/caml\_start\_program.s}{caml\_start\_program}}%
\end{column}
\begin{column}{0.4\textwidth}
\centering
\providecommand\step{4}\input{diagrams/default/default.tex}
\end{column}
\end{columns}
\end{frame}
%\only<>{\listasm[firstline=47,lastline=65]{src/ocaml/caml\_start\_program.s}{caml\_start\_program}}


\begin{frame}{Back to C}{Clean up, complain and terminate}
\begin{columns}
\begin{column}{0.45\textwidth}
\begin{itemize}
\item Backtrace:
\begin{itemize}
\item \funcname{caml\_start\_program} $\rightarrow$ OCaml code
\item \funcname{caml\_startup\_common}
\item \funcname{caml\_startup\_exn}
\item \funcname{caml\_startup}
\item \funcname{caml\_main}
\item \funcname{main}
\end{itemize}
\smallskip
\item \funcname{caml\_startup} checks the return value of \funcname{caml\_startup\_exn}
\begin{itemize}
\item The return value has been specially marked by \funcname{caml\_start\_program} handler
\item Calls \funcname{caml\_fatal\_uncaught\_exception}
\end{itemize}
\item<2-> \funcname{caml\_fatal\_uncaught\_exception} either
\begin{itemize}
\item<2-> Calls the user custom uncaught exception handler, if set with \mintinline{ocaml}{Printexc.set_uncaught_exception_handler fn}\footnotemark
\item<2-> Or falls back to the default one
\item<2-> Which notifies about the uncaught exception
\end{itemize}
\item<4-> Terminates the program either with \funcname{abort} or with a non-zero exit code
\end{itemize}
\end{column}
\begin{column}{0.55\textwidth}
\only<1>{
\listc[firstline=84,lastline=89,highlightlines=88]{../ocaml/runtime/caml/mlvalues.h}{runtime/caml/mlvalues.h:84}
\listc[firstline=138,lastline=143,highlightlines={141-142}]{../ocaml/runtime/startup\_nat.c}{runtime/startup\_nat.c:138}
}%
\only<2>{
\listc[firstline=138,highlightlines={147,149}]{../ocaml/runtime/printexc.c}{runtime/printexc.c:138}
}%
\only<3>{
\listc[firstline=110,lastline=134,highlightlines=129]{../ocaml/runtime/printexc.c}{runtime/printexc.c:138}
}%
\only<4>{
\listc[firstline=138,highlightlines={152,154}]{../ocaml/runtime/printexc.c}{runtime/printexc.c:138}
}%
\end{column}
\end{columns}
\footnotetext[1]{\url{https://v2.ocaml.org/api/Printexc.html\#1\_Uncaughtexceptions}}
\end{frame}
}%
\end{column}
\end{columns}
\end{frame}

\begin{frame}{Executing the OCaml program}{And raising an exception without handler}
\begin{columns}
\begin{column}{0.6\textwidth}
\only<1-3>{
\begin{itemize}
\item<1-> Execution continues with OCaml code
\begin{itemize}
\item \funcname{camlDefault\_\_main\_269}
\item \funcname{camlDefault\_\_entry}
\item \funcname{caml\_program}
\end{itemize}
\item<1-> \funcname{camlDefault\_\_main\_269} raises a \typename{Failure} exception
\item<1-> \typename{Failure} is caught by \funcname{caml\_start\_program}
\begin{itemize}
\item<1-> The handler set the 2nd LSB of the exception value to \funcarg{1}
\item<3-> And then forces \funcname{caml\_start\_program} to terminate
\end{itemize}
\end{itemize}
}
\bigskip
\only<1,3>{\listocaml[highlightlines=3]{../src/default/default.ml}{default/default.ml}}%
\only<2>{\listasm[firstline=66,lastline=72,highlightlines=69]{src/ocaml/caml\_start\_program.s}{caml\_start\_program}}%
\end{column}
\begin{column}{0.4\textwidth}
\centering
\providecommand\step{4}%
% Runtime's default exception handler
%
\subsection*{Runtime's default exception handler}
\frameSubsection{pictures/The\_Architect.png}{}


\begin{frame}{Default exception handler}
\begin{columns}
\begin{column}{0.5\textwidth}
\begin{itemize}
\item \localname{domain\_state->exn\_handler} points to a trap located before \funcname{entry}!
\item What installed this very first trap there?
\end{itemize}
\bigskip
\listocaml{../src/default/default.ml}{default/default.ml}
\bigskip
\inputminted[fontsize=\footnotesize]{text}{../src/default/default.out}
\end{column}
\begin{column}{0.5\textwidth}
\centering
\only<1>{\providecommand\step{0}\input{diagrams/default/default.tex}}%
\only<2>{\providecommand\step{4}\input{diagrams/default/default.tex}}%
\end{column}
\end{columns}
\end{frame}


\begin{frame}{Startup}{How did we get there by the way?}
\begin{columns}
\begin{column}{0.45\textwidth}
\begin{itemize}
\item The entry point address of the binary is \funcname{main} (\funcname{\_start}) from the runtime
\item The program starts like a C program:
\begin{itemize}
\item \funcname{caml\_start\_program}
\item \funcname{caml\_startup\_common}
\item \funcname{caml\_startup\_exn}
\item \funcname{caml\_startup}
\item \funcname{caml\_main}
\item \funcname{main}
\end{itemize}
%\item \funcname{caml\_startup\_common}: initializes GC, domains \dots
% Also: codefrag, locale, custom opeartions, frame_descr, signals, stack
\item \funcname{caml\_start\_program} is the transition point between the C realm and the OCaml one
\end{itemize}
\bigskip
\listocaml{../src/default/default.ml}{default/default.ml}
\end{column}
\begin{column}{0.55\textwidth}
\listasm[lastline=24,highlightlines={22-24}]{src/ocaml/caml\_start\_program.s}{runtime/amd64.S:604}
\end{column}
\end{columns}
\end{frame}


\begin{frame}{Setup to jump into OCaml entry point}{\funcname{caml\_start\_program} initializes the main fiber}
\begin{columns}
\begin{column}{0.6\textwidth}
\only<1,3,5>{
\begin{itemize}
\item<1-> Stores information to allow switching from OCaml stack to the C stack
\item<3-> Constructs the default exception handler
\item<5-> Switches to the OCaml stack and calls \funcname{caml\_program}
\end{itemize}
\bigskip
\listocaml{../src/default/default.ml}{default/default.ml}
}%
\only<2>{
\listasm[firstline=22,lastline=41,highlightlines={25-27}]{src/ocaml/caml\_start\_program.s}{caml\_start\_program}
}%
\only<4>{
\mintasm[firstline=22,lastline=41,highlightlines={31-38}]{src/ocaml/caml\_start\_program.s}
\listasm[firstline=66,lastline=72]{src/ocaml/caml\_start\_program.s}{caml\_start\_program}
}%
\only<6>{
\listasm[firstline=22,lastline=41,highlightlines={39-41}]{src/ocaml/caml\_start\_program.s}{caml\_start\_program}
}%
\end{column}
\begin{column}{0.4\textwidth}
\centering
\only<1-2>{\providecommand\step{2}\input{diagrams/default/default.tex}}%
\only<3-5>{\providecommand\step{3}\input{diagrams/default/default.tex}}%
\onslide*<6->{\providecommand\step{4}\input{diagrams/default/default.tex}}%
\end{column}
\end{columns}
\end{frame}

\begin{frame}{Executing the OCaml program}{And raising an exception without handler}
\begin{columns}
\begin{column}{0.6\textwidth}
\only<1-3>{
\begin{itemize}
\item<1-> Execution continues with OCaml code
\begin{itemize}
\item \funcname{camlDefault\_\_main\_269}
\item \funcname{camlDefault\_\_entry}
\item \funcname{caml\_program}
\end{itemize}
\item<1-> \funcname{camlDefault\_\_main\_269} raises a \typename{Failure} exception
\item<1-> \typename{Failure} is caught by \funcname{caml\_start\_program}
\begin{itemize}
\item<1-> The handler set the 2nd LSB of the exception value to \funcarg{1}
\item<3-> And then forces \funcname{caml\_start\_program} to terminate
\end{itemize}
\end{itemize}
}
\bigskip
\only<1,3>{\listocaml[highlightlines=3]{../src/default/default.ml}{default/default.ml}}%
\only<2>{\listasm[firstline=66,lastline=72,highlightlines=69]{src/ocaml/caml\_start\_program.s}{caml\_start\_program}}%
\end{column}
\begin{column}{0.4\textwidth}
\centering
\providecommand\step{4}\input{diagrams/default/default.tex}
\end{column}
\end{columns}
\end{frame}
%\only<>{\listasm[firstline=47,lastline=65]{src/ocaml/caml\_start\_program.s}{caml\_start\_program}}


\begin{frame}{Back to C}{Clean up, complain and terminate}
\begin{columns}
\begin{column}{0.45\textwidth}
\begin{itemize}
\item Backtrace:
\begin{itemize}
\item \funcname{caml\_start\_program} $\rightarrow$ OCaml code
\item \funcname{caml\_startup\_common}
\item \funcname{caml\_startup\_exn}
\item \funcname{caml\_startup}
\item \funcname{caml\_main}
\item \funcname{main}
\end{itemize}
\smallskip
\item \funcname{caml\_startup} checks the return value of \funcname{caml\_startup\_exn}
\begin{itemize}
\item The return value has been specially marked by \funcname{caml\_start\_program} handler
\item Calls \funcname{caml\_fatal\_uncaught\_exception}
\end{itemize}
\item<2-> \funcname{caml\_fatal\_uncaught\_exception} either
\begin{itemize}
\item<2-> Calls the user custom uncaught exception handler, if set with \mintinline{ocaml}{Printexc.set_uncaught_exception_handler fn}\footnotemark
\item<2-> Or falls back to the default one
\item<2-> Which notifies about the uncaught exception
\end{itemize}
\item<4-> Terminates the program either with \funcname{abort} or with a non-zero exit code
\end{itemize}
\end{column}
\begin{column}{0.55\textwidth}
\only<1>{
\listc[firstline=84,lastline=89,highlightlines=88]{../ocaml/runtime/caml/mlvalues.h}{runtime/caml/mlvalues.h:84}
\listc[firstline=138,lastline=143,highlightlines={141-142}]{../ocaml/runtime/startup\_nat.c}{runtime/startup\_nat.c:138}
}%
\only<2>{
\listc[firstline=138,highlightlines={147,149}]{../ocaml/runtime/printexc.c}{runtime/printexc.c:138}
}%
\only<3>{
\listc[firstline=110,lastline=134,highlightlines=129]{../ocaml/runtime/printexc.c}{runtime/printexc.c:138}
}%
\only<4>{
\listc[firstline=138,highlightlines={152,154}]{../ocaml/runtime/printexc.c}{runtime/printexc.c:138}
}%
\end{column}
\end{columns}
\footnotetext[1]{\url{https://v2.ocaml.org/api/Printexc.html\#1\_Uncaughtexceptions}}
\end{frame}

\end{column}
\end{columns}
\end{frame}
%\only<>{\listasm[firstline=47,lastline=65]{src/ocaml/caml\_start\_program.s}{caml\_start\_program}}


\begin{frame}{Back to C}{Clean up, complain and terminate}
\begin{columns}
\begin{column}{0.45\textwidth}
\begin{itemize}
\item Backtrace:
\begin{itemize}
\item \funcname{caml\_start\_program} $\rightarrow$ OCaml code
\item \funcname{caml\_startup\_common}
\item \funcname{caml\_startup\_exn}
\item \funcname{caml\_startup}
\item \funcname{caml\_main}
\item \funcname{main}
\end{itemize}
\smallskip
\item \funcname{caml\_startup} checks the return value of \funcname{caml\_startup\_exn}
\begin{itemize}
\item The return value has been specially marked by \funcname{caml\_start\_program} handler
\item Calls \funcname{caml\_fatal\_uncaught\_exception}
\end{itemize}
\item<2-> \funcname{caml\_fatal\_uncaught\_exception} either
\begin{itemize}
\item<2-> Calls the user custom uncaught exception handler, if set with \mintinline{ocaml}{Printexc.set_uncaught_exception_handler fn}\footnotemark
\item<2-> Or falls back to the default one
\item<2-> Which notifies about the uncaught exception
\end{itemize}
\item<4-> Terminates the program either with \funcname{abort} or with a non-zero exit code
\end{itemize}
\end{column}
\begin{column}{0.55\textwidth}
\only<1>{
\listc[firstline=84,lastline=89,highlightlines=88]{../ocaml/runtime/caml/mlvalues.h}{runtime/caml/mlvalues.h:84}
\listc[firstline=138,lastline=143,highlightlines={141-142}]{../ocaml/runtime/startup\_nat.c}{runtime/startup\_nat.c:138}
}%
\only<2>{
\listc[firstline=138,highlightlines={147,149}]{../ocaml/runtime/printexc.c}{runtime/printexc.c:138}
}%
\only<3>{
\listc[firstline=110,lastline=134,highlightlines=129]{../ocaml/runtime/printexc.c}{runtime/printexc.c:138}
}%
\only<4>{
\listc[firstline=138,highlightlines={152,154}]{../ocaml/runtime/printexc.c}{runtime/printexc.c:138}
}%
\end{column}
\end{columns}
\footnotetext[1]{\url{https://v2.ocaml.org/api/Printexc.html\#1\_Uncaughtexceptions}}
\end{frame}
}%
\end{column}
\end{columns}
\end{frame}

\begin{frame}{Executing the OCaml program}{And raising an exception without handler}
\begin{columns}
\begin{column}{0.6\textwidth}
\only<1-3>{
\begin{itemize}
\item<1-> Execution continues with OCaml code
\begin{itemize}
\item \funcname{camlDefault\_\_main\_269}
\item \funcname{camlDefault\_\_entry}
\item \funcname{caml\_program}
\end{itemize}
\item<1-> \funcname{camlDefault\_\_main\_269} raises a \typename{Failure} exception
\item<1-> \typename{Failure} is caught by \funcname{caml\_start\_program}
\begin{itemize}
\item<1-> The handler set the 2nd LSB of the exception value to \funcarg{1}
\item<3-> And then forces \funcname{caml\_start\_program} to terminate
\end{itemize}
\end{itemize}
}
\bigskip
\only<1,3>{\listocaml[highlightlines=3]{../src/default/default.ml}{default/default.ml}}%
\only<2>{\listasm[firstline=66,lastline=72,highlightlines=69]{src/ocaml/caml\_start\_program.s}{caml\_start\_program}}%
\end{column}
\begin{column}{0.4\textwidth}
\centering
\providecommand\step{4}%
% Runtime's default exception handler
%
\subsection*{Runtime's default exception handler}
\frameSubsection{pictures/The\_Architect.png}{}


\begin{frame}{Default exception handler}
\begin{columns}
\begin{column}{0.5\textwidth}
\begin{itemize}
\item \localname{domain\_state->exn\_handler} points to a trap located before \funcname{entry}!
\item What installed this very first trap there?
\end{itemize}
\bigskip
\listocaml{../src/default/default.ml}{default/default.ml}
\bigskip
\inputminted[fontsize=\footnotesize]{text}{../src/default/default.out}
\end{column}
\begin{column}{0.5\textwidth}
\centering
\only<1>{\providecommand\step{0}%
% Runtime's default exception handler
%
\subsection*{Runtime's default exception handler}
\frameSubsection{pictures/The\_Architect.png}{}


\begin{frame}{Default exception handler}
\begin{columns}
\begin{column}{0.5\textwidth}
\begin{itemize}
\item \localname{domain\_state->exn\_handler} points to a trap located before \funcname{entry}!
\item What installed this very first trap there?
\end{itemize}
\bigskip
\listocaml{../src/default/default.ml}{default/default.ml}
\bigskip
\inputminted[fontsize=\footnotesize]{text}{../src/default/default.out}
\end{column}
\begin{column}{0.5\textwidth}
\centering
\only<1>{\providecommand\step{0}\input{diagrams/default/default.tex}}%
\only<2>{\providecommand\step{4}\input{diagrams/default/default.tex}}%
\end{column}
\end{columns}
\end{frame}


\begin{frame}{Startup}{How did we get there by the way?}
\begin{columns}
\begin{column}{0.45\textwidth}
\begin{itemize}
\item The entry point address of the binary is \funcname{main} (\funcname{\_start}) from the runtime
\item The program starts like a C program:
\begin{itemize}
\item \funcname{caml\_start\_program}
\item \funcname{caml\_startup\_common}
\item \funcname{caml\_startup\_exn}
\item \funcname{caml\_startup}
\item \funcname{caml\_main}
\item \funcname{main}
\end{itemize}
%\item \funcname{caml\_startup\_common}: initializes GC, domains \dots
% Also: codefrag, locale, custom opeartions, frame_descr, signals, stack
\item \funcname{caml\_start\_program} is the transition point between the C realm and the OCaml one
\end{itemize}
\bigskip
\listocaml{../src/default/default.ml}{default/default.ml}
\end{column}
\begin{column}{0.55\textwidth}
\listasm[lastline=24,highlightlines={22-24}]{src/ocaml/caml\_start\_program.s}{runtime/amd64.S:604}
\end{column}
\end{columns}
\end{frame}


\begin{frame}{Setup to jump into OCaml entry point}{\funcname{caml\_start\_program} initializes the main fiber}
\begin{columns}
\begin{column}{0.6\textwidth}
\only<1,3,5>{
\begin{itemize}
\item<1-> Stores information to allow switching from OCaml stack to the C stack
\item<3-> Constructs the default exception handler
\item<5-> Switches to the OCaml stack and calls \funcname{caml\_program}
\end{itemize}
\bigskip
\listocaml{../src/default/default.ml}{default/default.ml}
}%
\only<2>{
\listasm[firstline=22,lastline=41,highlightlines={25-27}]{src/ocaml/caml\_start\_program.s}{caml\_start\_program}
}%
\only<4>{
\mintasm[firstline=22,lastline=41,highlightlines={31-38}]{src/ocaml/caml\_start\_program.s}
\listasm[firstline=66,lastline=72]{src/ocaml/caml\_start\_program.s}{caml\_start\_program}
}%
\only<6>{
\listasm[firstline=22,lastline=41,highlightlines={39-41}]{src/ocaml/caml\_start\_program.s}{caml\_start\_program}
}%
\end{column}
\begin{column}{0.4\textwidth}
\centering
\only<1-2>{\providecommand\step{2}\input{diagrams/default/default.tex}}%
\only<3-5>{\providecommand\step{3}\input{diagrams/default/default.tex}}%
\onslide*<6->{\providecommand\step{4}\input{diagrams/default/default.tex}}%
\end{column}
\end{columns}
\end{frame}

\begin{frame}{Executing the OCaml program}{And raising an exception without handler}
\begin{columns}
\begin{column}{0.6\textwidth}
\only<1-3>{
\begin{itemize}
\item<1-> Execution continues with OCaml code
\begin{itemize}
\item \funcname{camlDefault\_\_main\_269}
\item \funcname{camlDefault\_\_entry}
\item \funcname{caml\_program}
\end{itemize}
\item<1-> \funcname{camlDefault\_\_main\_269} raises a \typename{Failure} exception
\item<1-> \typename{Failure} is caught by \funcname{caml\_start\_program}
\begin{itemize}
\item<1-> The handler set the 2nd LSB of the exception value to \funcarg{1}
\item<3-> And then forces \funcname{caml\_start\_program} to terminate
\end{itemize}
\end{itemize}
}
\bigskip
\only<1,3>{\listocaml[highlightlines=3]{../src/default/default.ml}{default/default.ml}}%
\only<2>{\listasm[firstline=66,lastline=72,highlightlines=69]{src/ocaml/caml\_start\_program.s}{caml\_start\_program}}%
\end{column}
\begin{column}{0.4\textwidth}
\centering
\providecommand\step{4}\input{diagrams/default/default.tex}
\end{column}
\end{columns}
\end{frame}
%\only<>{\listasm[firstline=47,lastline=65]{src/ocaml/caml\_start\_program.s}{caml\_start\_program}}


\begin{frame}{Back to C}{Clean up, complain and terminate}
\begin{columns}
\begin{column}{0.45\textwidth}
\begin{itemize}
\item Backtrace:
\begin{itemize}
\item \funcname{caml\_start\_program} $\rightarrow$ OCaml code
\item \funcname{caml\_startup\_common}
\item \funcname{caml\_startup\_exn}
\item \funcname{caml\_startup}
\item \funcname{caml\_main}
\item \funcname{main}
\end{itemize}
\smallskip
\item \funcname{caml\_startup} checks the return value of \funcname{caml\_startup\_exn}
\begin{itemize}
\item The return value has been specially marked by \funcname{caml\_start\_program} handler
\item Calls \funcname{caml\_fatal\_uncaught\_exception}
\end{itemize}
\item<2-> \funcname{caml\_fatal\_uncaught\_exception} either
\begin{itemize}
\item<2-> Calls the user custom uncaught exception handler, if set with \mintinline{ocaml}{Printexc.set_uncaught_exception_handler fn}\footnotemark
\item<2-> Or falls back to the default one
\item<2-> Which notifies about the uncaught exception
\end{itemize}
\item<4-> Terminates the program either with \funcname{abort} or with a non-zero exit code
\end{itemize}
\end{column}
\begin{column}{0.55\textwidth}
\only<1>{
\listc[firstline=84,lastline=89,highlightlines=88]{../ocaml/runtime/caml/mlvalues.h}{runtime/caml/mlvalues.h:84}
\listc[firstline=138,lastline=143,highlightlines={141-142}]{../ocaml/runtime/startup\_nat.c}{runtime/startup\_nat.c:138}
}%
\only<2>{
\listc[firstline=138,highlightlines={147,149}]{../ocaml/runtime/printexc.c}{runtime/printexc.c:138}
}%
\only<3>{
\listc[firstline=110,lastline=134,highlightlines=129]{../ocaml/runtime/printexc.c}{runtime/printexc.c:138}
}%
\only<4>{
\listc[firstline=138,highlightlines={152,154}]{../ocaml/runtime/printexc.c}{runtime/printexc.c:138}
}%
\end{column}
\end{columns}
\footnotetext[1]{\url{https://v2.ocaml.org/api/Printexc.html\#1\_Uncaughtexceptions}}
\end{frame}
}%
\only<2>{\providecommand\step{4}%
% Runtime's default exception handler
%
\subsection*{Runtime's default exception handler}
\frameSubsection{pictures/The\_Architect.png}{}


\begin{frame}{Default exception handler}
\begin{columns}
\begin{column}{0.5\textwidth}
\begin{itemize}
\item \localname{domain\_state->exn\_handler} points to a trap located before \funcname{entry}!
\item What installed this very first trap there?
\end{itemize}
\bigskip
\listocaml{../src/default/default.ml}{default/default.ml}
\bigskip
\inputminted[fontsize=\footnotesize]{text}{../src/default/default.out}
\end{column}
\begin{column}{0.5\textwidth}
\centering
\only<1>{\providecommand\step{0}\input{diagrams/default/default.tex}}%
\only<2>{\providecommand\step{4}\input{diagrams/default/default.tex}}%
\end{column}
\end{columns}
\end{frame}


\begin{frame}{Startup}{How did we get there by the way?}
\begin{columns}
\begin{column}{0.45\textwidth}
\begin{itemize}
\item The entry point address of the binary is \funcname{main} (\funcname{\_start}) from the runtime
\item The program starts like a C program:
\begin{itemize}
\item \funcname{caml\_start\_program}
\item \funcname{caml\_startup\_common}
\item \funcname{caml\_startup\_exn}
\item \funcname{caml\_startup}
\item \funcname{caml\_main}
\item \funcname{main}
\end{itemize}
%\item \funcname{caml\_startup\_common}: initializes GC, domains \dots
% Also: codefrag, locale, custom opeartions, frame_descr, signals, stack
\item \funcname{caml\_start\_program} is the transition point between the C realm and the OCaml one
\end{itemize}
\bigskip
\listocaml{../src/default/default.ml}{default/default.ml}
\end{column}
\begin{column}{0.55\textwidth}
\listasm[lastline=24,highlightlines={22-24}]{src/ocaml/caml\_start\_program.s}{runtime/amd64.S:604}
\end{column}
\end{columns}
\end{frame}


\begin{frame}{Setup to jump into OCaml entry point}{\funcname{caml\_start\_program} initializes the main fiber}
\begin{columns}
\begin{column}{0.6\textwidth}
\only<1,3,5>{
\begin{itemize}
\item<1-> Stores information to allow switching from OCaml stack to the C stack
\item<3-> Constructs the default exception handler
\item<5-> Switches to the OCaml stack and calls \funcname{caml\_program}
\end{itemize}
\bigskip
\listocaml{../src/default/default.ml}{default/default.ml}
}%
\only<2>{
\listasm[firstline=22,lastline=41,highlightlines={25-27}]{src/ocaml/caml\_start\_program.s}{caml\_start\_program}
}%
\only<4>{
\mintasm[firstline=22,lastline=41,highlightlines={31-38}]{src/ocaml/caml\_start\_program.s}
\listasm[firstline=66,lastline=72]{src/ocaml/caml\_start\_program.s}{caml\_start\_program}
}%
\only<6>{
\listasm[firstline=22,lastline=41,highlightlines={39-41}]{src/ocaml/caml\_start\_program.s}{caml\_start\_program}
}%
\end{column}
\begin{column}{0.4\textwidth}
\centering
\only<1-2>{\providecommand\step{2}\input{diagrams/default/default.tex}}%
\only<3-5>{\providecommand\step{3}\input{diagrams/default/default.tex}}%
\onslide*<6->{\providecommand\step{4}\input{diagrams/default/default.tex}}%
\end{column}
\end{columns}
\end{frame}

\begin{frame}{Executing the OCaml program}{And raising an exception without handler}
\begin{columns}
\begin{column}{0.6\textwidth}
\only<1-3>{
\begin{itemize}
\item<1-> Execution continues with OCaml code
\begin{itemize}
\item \funcname{camlDefault\_\_main\_269}
\item \funcname{camlDefault\_\_entry}
\item \funcname{caml\_program}
\end{itemize}
\item<1-> \funcname{camlDefault\_\_main\_269} raises a \typename{Failure} exception
\item<1-> \typename{Failure} is caught by \funcname{caml\_start\_program}
\begin{itemize}
\item<1-> The handler set the 2nd LSB of the exception value to \funcarg{1}
\item<3-> And then forces \funcname{caml\_start\_program} to terminate
\end{itemize}
\end{itemize}
}
\bigskip
\only<1,3>{\listocaml[highlightlines=3]{../src/default/default.ml}{default/default.ml}}%
\only<2>{\listasm[firstline=66,lastline=72,highlightlines=69]{src/ocaml/caml\_start\_program.s}{caml\_start\_program}}%
\end{column}
\begin{column}{0.4\textwidth}
\centering
\providecommand\step{4}\input{diagrams/default/default.tex}
\end{column}
\end{columns}
\end{frame}
%\only<>{\listasm[firstline=47,lastline=65]{src/ocaml/caml\_start\_program.s}{caml\_start\_program}}


\begin{frame}{Back to C}{Clean up, complain and terminate}
\begin{columns}
\begin{column}{0.45\textwidth}
\begin{itemize}
\item Backtrace:
\begin{itemize}
\item \funcname{caml\_start\_program} $\rightarrow$ OCaml code
\item \funcname{caml\_startup\_common}
\item \funcname{caml\_startup\_exn}
\item \funcname{caml\_startup}
\item \funcname{caml\_main}
\item \funcname{main}
\end{itemize}
\smallskip
\item \funcname{caml\_startup} checks the return value of \funcname{caml\_startup\_exn}
\begin{itemize}
\item The return value has been specially marked by \funcname{caml\_start\_program} handler
\item Calls \funcname{caml\_fatal\_uncaught\_exception}
\end{itemize}
\item<2-> \funcname{caml\_fatal\_uncaught\_exception} either
\begin{itemize}
\item<2-> Calls the user custom uncaught exception handler, if set with \mintinline{ocaml}{Printexc.set_uncaught_exception_handler fn}\footnotemark
\item<2-> Or falls back to the default one
\item<2-> Which notifies about the uncaught exception
\end{itemize}
\item<4-> Terminates the program either with \funcname{abort} or with a non-zero exit code
\end{itemize}
\end{column}
\begin{column}{0.55\textwidth}
\only<1>{
\listc[firstline=84,lastline=89,highlightlines=88]{../ocaml/runtime/caml/mlvalues.h}{runtime/caml/mlvalues.h:84}
\listc[firstline=138,lastline=143,highlightlines={141-142}]{../ocaml/runtime/startup\_nat.c}{runtime/startup\_nat.c:138}
}%
\only<2>{
\listc[firstline=138,highlightlines={147,149}]{../ocaml/runtime/printexc.c}{runtime/printexc.c:138}
}%
\only<3>{
\listc[firstline=110,lastline=134,highlightlines=129]{../ocaml/runtime/printexc.c}{runtime/printexc.c:138}
}%
\only<4>{
\listc[firstline=138,highlightlines={152,154}]{../ocaml/runtime/printexc.c}{runtime/printexc.c:138}
}%
\end{column}
\end{columns}
\footnotetext[1]{\url{https://v2.ocaml.org/api/Printexc.html\#1\_Uncaughtexceptions}}
\end{frame}
}%
\end{column}
\end{columns}
\end{frame}


\begin{frame}{Startup}{How did we get there by the way?}
\begin{columns}
\begin{column}{0.45\textwidth}
\begin{itemize}
\item The entry point address of the binary is \funcname{main} (\funcname{\_start}) from the runtime
\item The program starts like a C program:
\begin{itemize}
\item \funcname{caml\_start\_program}
\item \funcname{caml\_startup\_common}
\item \funcname{caml\_startup\_exn}
\item \funcname{caml\_startup}
\item \funcname{caml\_main}
\item \funcname{main}
\end{itemize}
%\item \funcname{caml\_startup\_common}: initializes GC, domains \dots
% Also: codefrag, locale, custom opeartions, frame_descr, signals, stack
\item \funcname{caml\_start\_program} is the transition point between the C realm and the OCaml one
\end{itemize}
\bigskip
\listocaml{../src/default/default.ml}{default/default.ml}
\end{column}
\begin{column}{0.55\textwidth}
\listasm[lastline=24,highlightlines={22-24}]{src/ocaml/caml\_start\_program.s}{runtime/amd64.S:604}
\end{column}
\end{columns}
\end{frame}


\begin{frame}{Setup to jump into OCaml entry point}{\funcname{caml\_start\_program} initializes the main fiber}
\begin{columns}
\begin{column}{0.6\textwidth}
\only<1,3,5>{
\begin{itemize}
\item<1-> Stores information to allow switching from OCaml stack to the C stack
\item<3-> Constructs the default exception handler
\item<5-> Switches to the OCaml stack and calls \funcname{caml\_program}
\end{itemize}
\bigskip
\listocaml{../src/default/default.ml}{default/default.ml}
}%
\only<2>{
\listasm[firstline=22,lastline=41,highlightlines={25-27}]{src/ocaml/caml\_start\_program.s}{caml\_start\_program}
}%
\only<4>{
\mintasm[firstline=22,lastline=41,highlightlines={31-38}]{src/ocaml/caml\_start\_program.s}
\listasm[firstline=66,lastline=72]{src/ocaml/caml\_start\_program.s}{caml\_start\_program}
}%
\only<6>{
\listasm[firstline=22,lastline=41,highlightlines={39-41}]{src/ocaml/caml\_start\_program.s}{caml\_start\_program}
}%
\end{column}
\begin{column}{0.4\textwidth}
\centering
\only<1-2>{\providecommand\step{2}%
% Runtime's default exception handler
%
\subsection*{Runtime's default exception handler}
\frameSubsection{pictures/The\_Architect.png}{}


\begin{frame}{Default exception handler}
\begin{columns}
\begin{column}{0.5\textwidth}
\begin{itemize}
\item \localname{domain\_state->exn\_handler} points to a trap located before \funcname{entry}!
\item What installed this very first trap there?
\end{itemize}
\bigskip
\listocaml{../src/default/default.ml}{default/default.ml}
\bigskip
\inputminted[fontsize=\footnotesize]{text}{../src/default/default.out}
\end{column}
\begin{column}{0.5\textwidth}
\centering
\only<1>{\providecommand\step{0}\input{diagrams/default/default.tex}}%
\only<2>{\providecommand\step{4}\input{diagrams/default/default.tex}}%
\end{column}
\end{columns}
\end{frame}


\begin{frame}{Startup}{How did we get there by the way?}
\begin{columns}
\begin{column}{0.45\textwidth}
\begin{itemize}
\item The entry point address of the binary is \funcname{main} (\funcname{\_start}) from the runtime
\item The program starts like a C program:
\begin{itemize}
\item \funcname{caml\_start\_program}
\item \funcname{caml\_startup\_common}
\item \funcname{caml\_startup\_exn}
\item \funcname{caml\_startup}
\item \funcname{caml\_main}
\item \funcname{main}
\end{itemize}
%\item \funcname{caml\_startup\_common}: initializes GC, domains \dots
% Also: codefrag, locale, custom opeartions, frame_descr, signals, stack
\item \funcname{caml\_start\_program} is the transition point between the C realm and the OCaml one
\end{itemize}
\bigskip
\listocaml{../src/default/default.ml}{default/default.ml}
\end{column}
\begin{column}{0.55\textwidth}
\listasm[lastline=24,highlightlines={22-24}]{src/ocaml/caml\_start\_program.s}{runtime/amd64.S:604}
\end{column}
\end{columns}
\end{frame}


\begin{frame}{Setup to jump into OCaml entry point}{\funcname{caml\_start\_program} initializes the main fiber}
\begin{columns}
\begin{column}{0.6\textwidth}
\only<1,3,5>{
\begin{itemize}
\item<1-> Stores information to allow switching from OCaml stack to the C stack
\item<3-> Constructs the default exception handler
\item<5-> Switches to the OCaml stack and calls \funcname{caml\_program}
\end{itemize}
\bigskip
\listocaml{../src/default/default.ml}{default/default.ml}
}%
\only<2>{
\listasm[firstline=22,lastline=41,highlightlines={25-27}]{src/ocaml/caml\_start\_program.s}{caml\_start\_program}
}%
\only<4>{
\mintasm[firstline=22,lastline=41,highlightlines={31-38}]{src/ocaml/caml\_start\_program.s}
\listasm[firstline=66,lastline=72]{src/ocaml/caml\_start\_program.s}{caml\_start\_program}
}%
\only<6>{
\listasm[firstline=22,lastline=41,highlightlines={39-41}]{src/ocaml/caml\_start\_program.s}{caml\_start\_program}
}%
\end{column}
\begin{column}{0.4\textwidth}
\centering
\only<1-2>{\providecommand\step{2}\input{diagrams/default/default.tex}}%
\only<3-5>{\providecommand\step{3}\input{diagrams/default/default.tex}}%
\onslide*<6->{\providecommand\step{4}\input{diagrams/default/default.tex}}%
\end{column}
\end{columns}
\end{frame}

\begin{frame}{Executing the OCaml program}{And raising an exception without handler}
\begin{columns}
\begin{column}{0.6\textwidth}
\only<1-3>{
\begin{itemize}
\item<1-> Execution continues with OCaml code
\begin{itemize}
\item \funcname{camlDefault\_\_main\_269}
\item \funcname{camlDefault\_\_entry}
\item \funcname{caml\_program}
\end{itemize}
\item<1-> \funcname{camlDefault\_\_main\_269} raises a \typename{Failure} exception
\item<1-> \typename{Failure} is caught by \funcname{caml\_start\_program}
\begin{itemize}
\item<1-> The handler set the 2nd LSB of the exception value to \funcarg{1}
\item<3-> And then forces \funcname{caml\_start\_program} to terminate
\end{itemize}
\end{itemize}
}
\bigskip
\only<1,3>{\listocaml[highlightlines=3]{../src/default/default.ml}{default/default.ml}}%
\only<2>{\listasm[firstline=66,lastline=72,highlightlines=69]{src/ocaml/caml\_start\_program.s}{caml\_start\_program}}%
\end{column}
\begin{column}{0.4\textwidth}
\centering
\providecommand\step{4}\input{diagrams/default/default.tex}
\end{column}
\end{columns}
\end{frame}
%\only<>{\listasm[firstline=47,lastline=65]{src/ocaml/caml\_start\_program.s}{caml\_start\_program}}


\begin{frame}{Back to C}{Clean up, complain and terminate}
\begin{columns}
\begin{column}{0.45\textwidth}
\begin{itemize}
\item Backtrace:
\begin{itemize}
\item \funcname{caml\_start\_program} $\rightarrow$ OCaml code
\item \funcname{caml\_startup\_common}
\item \funcname{caml\_startup\_exn}
\item \funcname{caml\_startup}
\item \funcname{caml\_main}
\item \funcname{main}
\end{itemize}
\smallskip
\item \funcname{caml\_startup} checks the return value of \funcname{caml\_startup\_exn}
\begin{itemize}
\item The return value has been specially marked by \funcname{caml\_start\_program} handler
\item Calls \funcname{caml\_fatal\_uncaught\_exception}
\end{itemize}
\item<2-> \funcname{caml\_fatal\_uncaught\_exception} either
\begin{itemize}
\item<2-> Calls the user custom uncaught exception handler, if set with \mintinline{ocaml}{Printexc.set_uncaught_exception_handler fn}\footnotemark
\item<2-> Or falls back to the default one
\item<2-> Which notifies about the uncaught exception
\end{itemize}
\item<4-> Terminates the program either with \funcname{abort} or with a non-zero exit code
\end{itemize}
\end{column}
\begin{column}{0.55\textwidth}
\only<1>{
\listc[firstline=84,lastline=89,highlightlines=88]{../ocaml/runtime/caml/mlvalues.h}{runtime/caml/mlvalues.h:84}
\listc[firstline=138,lastline=143,highlightlines={141-142}]{../ocaml/runtime/startup\_nat.c}{runtime/startup\_nat.c:138}
}%
\only<2>{
\listc[firstline=138,highlightlines={147,149}]{../ocaml/runtime/printexc.c}{runtime/printexc.c:138}
}%
\only<3>{
\listc[firstline=110,lastline=134,highlightlines=129]{../ocaml/runtime/printexc.c}{runtime/printexc.c:138}
}%
\only<4>{
\listc[firstline=138,highlightlines={152,154}]{../ocaml/runtime/printexc.c}{runtime/printexc.c:138}
}%
\end{column}
\end{columns}
\footnotetext[1]{\url{https://v2.ocaml.org/api/Printexc.html\#1\_Uncaughtexceptions}}
\end{frame}
}%
\only<3-5>{\providecommand\step{3}%
% Runtime's default exception handler
%
\subsection*{Runtime's default exception handler}
\frameSubsection{pictures/The\_Architect.png}{}


\begin{frame}{Default exception handler}
\begin{columns}
\begin{column}{0.5\textwidth}
\begin{itemize}
\item \localname{domain\_state->exn\_handler} points to a trap located before \funcname{entry}!
\item What installed this very first trap there?
\end{itemize}
\bigskip
\listocaml{../src/default/default.ml}{default/default.ml}
\bigskip
\inputminted[fontsize=\footnotesize]{text}{../src/default/default.out}
\end{column}
\begin{column}{0.5\textwidth}
\centering
\only<1>{\providecommand\step{0}\input{diagrams/default/default.tex}}%
\only<2>{\providecommand\step{4}\input{diagrams/default/default.tex}}%
\end{column}
\end{columns}
\end{frame}


\begin{frame}{Startup}{How did we get there by the way?}
\begin{columns}
\begin{column}{0.45\textwidth}
\begin{itemize}
\item The entry point address of the binary is \funcname{main} (\funcname{\_start}) from the runtime
\item The program starts like a C program:
\begin{itemize}
\item \funcname{caml\_start\_program}
\item \funcname{caml\_startup\_common}
\item \funcname{caml\_startup\_exn}
\item \funcname{caml\_startup}
\item \funcname{caml\_main}
\item \funcname{main}
\end{itemize}
%\item \funcname{caml\_startup\_common}: initializes GC, domains \dots
% Also: codefrag, locale, custom opeartions, frame_descr, signals, stack
\item \funcname{caml\_start\_program} is the transition point between the C realm and the OCaml one
\end{itemize}
\bigskip
\listocaml{../src/default/default.ml}{default/default.ml}
\end{column}
\begin{column}{0.55\textwidth}
\listasm[lastline=24,highlightlines={22-24}]{src/ocaml/caml\_start\_program.s}{runtime/amd64.S:604}
\end{column}
\end{columns}
\end{frame}


\begin{frame}{Setup to jump into OCaml entry point}{\funcname{caml\_start\_program} initializes the main fiber}
\begin{columns}
\begin{column}{0.6\textwidth}
\only<1,3,5>{
\begin{itemize}
\item<1-> Stores information to allow switching from OCaml stack to the C stack
\item<3-> Constructs the default exception handler
\item<5-> Switches to the OCaml stack and calls \funcname{caml\_program}
\end{itemize}
\bigskip
\listocaml{../src/default/default.ml}{default/default.ml}
}%
\only<2>{
\listasm[firstline=22,lastline=41,highlightlines={25-27}]{src/ocaml/caml\_start\_program.s}{caml\_start\_program}
}%
\only<4>{
\mintasm[firstline=22,lastline=41,highlightlines={31-38}]{src/ocaml/caml\_start\_program.s}
\listasm[firstline=66,lastline=72]{src/ocaml/caml\_start\_program.s}{caml\_start\_program}
}%
\only<6>{
\listasm[firstline=22,lastline=41,highlightlines={39-41}]{src/ocaml/caml\_start\_program.s}{caml\_start\_program}
}%
\end{column}
\begin{column}{0.4\textwidth}
\centering
\only<1-2>{\providecommand\step{2}\input{diagrams/default/default.tex}}%
\only<3-5>{\providecommand\step{3}\input{diagrams/default/default.tex}}%
\onslide*<6->{\providecommand\step{4}\input{diagrams/default/default.tex}}%
\end{column}
\end{columns}
\end{frame}

\begin{frame}{Executing the OCaml program}{And raising an exception without handler}
\begin{columns}
\begin{column}{0.6\textwidth}
\only<1-3>{
\begin{itemize}
\item<1-> Execution continues with OCaml code
\begin{itemize}
\item \funcname{camlDefault\_\_main\_269}
\item \funcname{camlDefault\_\_entry}
\item \funcname{caml\_program}
\end{itemize}
\item<1-> \funcname{camlDefault\_\_main\_269} raises a \typename{Failure} exception
\item<1-> \typename{Failure} is caught by \funcname{caml\_start\_program}
\begin{itemize}
\item<1-> The handler set the 2nd LSB of the exception value to \funcarg{1}
\item<3-> And then forces \funcname{caml\_start\_program} to terminate
\end{itemize}
\end{itemize}
}
\bigskip
\only<1,3>{\listocaml[highlightlines=3]{../src/default/default.ml}{default/default.ml}}%
\only<2>{\listasm[firstline=66,lastline=72,highlightlines=69]{src/ocaml/caml\_start\_program.s}{caml\_start\_program}}%
\end{column}
\begin{column}{0.4\textwidth}
\centering
\providecommand\step{4}\input{diagrams/default/default.tex}
\end{column}
\end{columns}
\end{frame}
%\only<>{\listasm[firstline=47,lastline=65]{src/ocaml/caml\_start\_program.s}{caml\_start\_program}}


\begin{frame}{Back to C}{Clean up, complain and terminate}
\begin{columns}
\begin{column}{0.45\textwidth}
\begin{itemize}
\item Backtrace:
\begin{itemize}
\item \funcname{caml\_start\_program} $\rightarrow$ OCaml code
\item \funcname{caml\_startup\_common}
\item \funcname{caml\_startup\_exn}
\item \funcname{caml\_startup}
\item \funcname{caml\_main}
\item \funcname{main}
\end{itemize}
\smallskip
\item \funcname{caml\_startup} checks the return value of \funcname{caml\_startup\_exn}
\begin{itemize}
\item The return value has been specially marked by \funcname{caml\_start\_program} handler
\item Calls \funcname{caml\_fatal\_uncaught\_exception}
\end{itemize}
\item<2-> \funcname{caml\_fatal\_uncaught\_exception} either
\begin{itemize}
\item<2-> Calls the user custom uncaught exception handler, if set with \mintinline{ocaml}{Printexc.set_uncaught_exception_handler fn}\footnotemark
\item<2-> Or falls back to the default one
\item<2-> Which notifies about the uncaught exception
\end{itemize}
\item<4-> Terminates the program either with \funcname{abort} or with a non-zero exit code
\end{itemize}
\end{column}
\begin{column}{0.55\textwidth}
\only<1>{
\listc[firstline=84,lastline=89,highlightlines=88]{../ocaml/runtime/caml/mlvalues.h}{runtime/caml/mlvalues.h:84}
\listc[firstline=138,lastline=143,highlightlines={141-142}]{../ocaml/runtime/startup\_nat.c}{runtime/startup\_nat.c:138}
}%
\only<2>{
\listc[firstline=138,highlightlines={147,149}]{../ocaml/runtime/printexc.c}{runtime/printexc.c:138}
}%
\only<3>{
\listc[firstline=110,lastline=134,highlightlines=129]{../ocaml/runtime/printexc.c}{runtime/printexc.c:138}
}%
\only<4>{
\listc[firstline=138,highlightlines={152,154}]{../ocaml/runtime/printexc.c}{runtime/printexc.c:138}
}%
\end{column}
\end{columns}
\footnotetext[1]{\url{https://v2.ocaml.org/api/Printexc.html\#1\_Uncaughtexceptions}}
\end{frame}
}%
\onslide*<6->{\providecommand\step{4}%
% Runtime's default exception handler
%
\subsection*{Runtime's default exception handler}
\frameSubsection{pictures/The\_Architect.png}{}


\begin{frame}{Default exception handler}
\begin{columns}
\begin{column}{0.5\textwidth}
\begin{itemize}
\item \localname{domain\_state->exn\_handler} points to a trap located before \funcname{entry}!
\item What installed this very first trap there?
\end{itemize}
\bigskip
\listocaml{../src/default/default.ml}{default/default.ml}
\bigskip
\inputminted[fontsize=\footnotesize]{text}{../src/default/default.out}
\end{column}
\begin{column}{0.5\textwidth}
\centering
\only<1>{\providecommand\step{0}\input{diagrams/default/default.tex}}%
\only<2>{\providecommand\step{4}\input{diagrams/default/default.tex}}%
\end{column}
\end{columns}
\end{frame}


\begin{frame}{Startup}{How did we get there by the way?}
\begin{columns}
\begin{column}{0.45\textwidth}
\begin{itemize}
\item The entry point address of the binary is \funcname{main} (\funcname{\_start}) from the runtime
\item The program starts like a C program:
\begin{itemize}
\item \funcname{caml\_start\_program}
\item \funcname{caml\_startup\_common}
\item \funcname{caml\_startup\_exn}
\item \funcname{caml\_startup}
\item \funcname{caml\_main}
\item \funcname{main}
\end{itemize}
%\item \funcname{caml\_startup\_common}: initializes GC, domains \dots
% Also: codefrag, locale, custom opeartions, frame_descr, signals, stack
\item \funcname{caml\_start\_program} is the transition point between the C realm and the OCaml one
\end{itemize}
\bigskip
\listocaml{../src/default/default.ml}{default/default.ml}
\end{column}
\begin{column}{0.55\textwidth}
\listasm[lastline=24,highlightlines={22-24}]{src/ocaml/caml\_start\_program.s}{runtime/amd64.S:604}
\end{column}
\end{columns}
\end{frame}


\begin{frame}{Setup to jump into OCaml entry point}{\funcname{caml\_start\_program} initializes the main fiber}
\begin{columns}
\begin{column}{0.6\textwidth}
\only<1,3,5>{
\begin{itemize}
\item<1-> Stores information to allow switching from OCaml stack to the C stack
\item<3-> Constructs the default exception handler
\item<5-> Switches to the OCaml stack and calls \funcname{caml\_program}
\end{itemize}
\bigskip
\listocaml{../src/default/default.ml}{default/default.ml}
}%
\only<2>{
\listasm[firstline=22,lastline=41,highlightlines={25-27}]{src/ocaml/caml\_start\_program.s}{caml\_start\_program}
}%
\only<4>{
\mintasm[firstline=22,lastline=41,highlightlines={31-38}]{src/ocaml/caml\_start\_program.s}
\listasm[firstline=66,lastline=72]{src/ocaml/caml\_start\_program.s}{caml\_start\_program}
}%
\only<6>{
\listasm[firstline=22,lastline=41,highlightlines={39-41}]{src/ocaml/caml\_start\_program.s}{caml\_start\_program}
}%
\end{column}
\begin{column}{0.4\textwidth}
\centering
\only<1-2>{\providecommand\step{2}\input{diagrams/default/default.tex}}%
\only<3-5>{\providecommand\step{3}\input{diagrams/default/default.tex}}%
\onslide*<6->{\providecommand\step{4}\input{diagrams/default/default.tex}}%
\end{column}
\end{columns}
\end{frame}

\begin{frame}{Executing the OCaml program}{And raising an exception without handler}
\begin{columns}
\begin{column}{0.6\textwidth}
\only<1-3>{
\begin{itemize}
\item<1-> Execution continues with OCaml code
\begin{itemize}
\item \funcname{camlDefault\_\_main\_269}
\item \funcname{camlDefault\_\_entry}
\item \funcname{caml\_program}
\end{itemize}
\item<1-> \funcname{camlDefault\_\_main\_269} raises a \typename{Failure} exception
\item<1-> \typename{Failure} is caught by \funcname{caml\_start\_program}
\begin{itemize}
\item<1-> The handler set the 2nd LSB of the exception value to \funcarg{1}
\item<3-> And then forces \funcname{caml\_start\_program} to terminate
\end{itemize}
\end{itemize}
}
\bigskip
\only<1,3>{\listocaml[highlightlines=3]{../src/default/default.ml}{default/default.ml}}%
\only<2>{\listasm[firstline=66,lastline=72,highlightlines=69]{src/ocaml/caml\_start\_program.s}{caml\_start\_program}}%
\end{column}
\begin{column}{0.4\textwidth}
\centering
\providecommand\step{4}\input{diagrams/default/default.tex}
\end{column}
\end{columns}
\end{frame}
%\only<>{\listasm[firstline=47,lastline=65]{src/ocaml/caml\_start\_program.s}{caml\_start\_program}}


\begin{frame}{Back to C}{Clean up, complain and terminate}
\begin{columns}
\begin{column}{0.45\textwidth}
\begin{itemize}
\item Backtrace:
\begin{itemize}
\item \funcname{caml\_start\_program} $\rightarrow$ OCaml code
\item \funcname{caml\_startup\_common}
\item \funcname{caml\_startup\_exn}
\item \funcname{caml\_startup}
\item \funcname{caml\_main}
\item \funcname{main}
\end{itemize}
\smallskip
\item \funcname{caml\_startup} checks the return value of \funcname{caml\_startup\_exn}
\begin{itemize}
\item The return value has been specially marked by \funcname{caml\_start\_program} handler
\item Calls \funcname{caml\_fatal\_uncaught\_exception}
\end{itemize}
\item<2-> \funcname{caml\_fatal\_uncaught\_exception} either
\begin{itemize}
\item<2-> Calls the user custom uncaught exception handler, if set with \mintinline{ocaml}{Printexc.set_uncaught_exception_handler fn}\footnotemark
\item<2-> Or falls back to the default one
\item<2-> Which notifies about the uncaught exception
\end{itemize}
\item<4-> Terminates the program either with \funcname{abort} or with a non-zero exit code
\end{itemize}
\end{column}
\begin{column}{0.55\textwidth}
\only<1>{
\listc[firstline=84,lastline=89,highlightlines=88]{../ocaml/runtime/caml/mlvalues.h}{runtime/caml/mlvalues.h:84}
\listc[firstline=138,lastline=143,highlightlines={141-142}]{../ocaml/runtime/startup\_nat.c}{runtime/startup\_nat.c:138}
}%
\only<2>{
\listc[firstline=138,highlightlines={147,149}]{../ocaml/runtime/printexc.c}{runtime/printexc.c:138}
}%
\only<3>{
\listc[firstline=110,lastline=134,highlightlines=129]{../ocaml/runtime/printexc.c}{runtime/printexc.c:138}
}%
\only<4>{
\listc[firstline=138,highlightlines={152,154}]{../ocaml/runtime/printexc.c}{runtime/printexc.c:138}
}%
\end{column}
\end{columns}
\footnotetext[1]{\url{https://v2.ocaml.org/api/Printexc.html\#1\_Uncaughtexceptions}}
\end{frame}
}%
\end{column}
\end{columns}
\end{frame}

\begin{frame}{Executing the OCaml program}{And raising an exception without handler}
\begin{columns}
\begin{column}{0.6\textwidth}
\only<1-3>{
\begin{itemize}
\item<1-> Execution continues with OCaml code
\begin{itemize}
\item \funcname{camlDefault\_\_main\_269}
\item \funcname{camlDefault\_\_entry}
\item \funcname{caml\_program}
\end{itemize}
\item<1-> \funcname{camlDefault\_\_main\_269} raises a \typename{Failure} exception
\item<1-> \typename{Failure} is caught by \funcname{caml\_start\_program}
\begin{itemize}
\item<1-> The handler set the 2nd LSB of the exception value to \funcarg{1}
\item<3-> And then forces \funcname{caml\_start\_program} to terminate
\end{itemize}
\end{itemize}
}
\bigskip
\only<1,3>{\listocaml[highlightlines=3]{../src/default/default.ml}{default/default.ml}}%
\only<2>{\listasm[firstline=66,lastline=72,highlightlines=69]{src/ocaml/caml\_start\_program.s}{caml\_start\_program}}%
\end{column}
\begin{column}{0.4\textwidth}
\centering
\providecommand\step{4}%
% Runtime's default exception handler
%
\subsection*{Runtime's default exception handler}
\frameSubsection{pictures/The\_Architect.png}{}


\begin{frame}{Default exception handler}
\begin{columns}
\begin{column}{0.5\textwidth}
\begin{itemize}
\item \localname{domain\_state->exn\_handler} points to a trap located before \funcname{entry}!
\item What installed this very first trap there?
\end{itemize}
\bigskip
\listocaml{../src/default/default.ml}{default/default.ml}
\bigskip
\inputminted[fontsize=\footnotesize]{text}{../src/default/default.out}
\end{column}
\begin{column}{0.5\textwidth}
\centering
\only<1>{\providecommand\step{0}\input{diagrams/default/default.tex}}%
\only<2>{\providecommand\step{4}\input{diagrams/default/default.tex}}%
\end{column}
\end{columns}
\end{frame}


\begin{frame}{Startup}{How did we get there by the way?}
\begin{columns}
\begin{column}{0.45\textwidth}
\begin{itemize}
\item The entry point address of the binary is \funcname{main} (\funcname{\_start}) from the runtime
\item The program starts like a C program:
\begin{itemize}
\item \funcname{caml\_start\_program}
\item \funcname{caml\_startup\_common}
\item \funcname{caml\_startup\_exn}
\item \funcname{caml\_startup}
\item \funcname{caml\_main}
\item \funcname{main}
\end{itemize}
%\item \funcname{caml\_startup\_common}: initializes GC, domains \dots
% Also: codefrag, locale, custom opeartions, frame_descr, signals, stack
\item \funcname{caml\_start\_program} is the transition point between the C realm and the OCaml one
\end{itemize}
\bigskip
\listocaml{../src/default/default.ml}{default/default.ml}
\end{column}
\begin{column}{0.55\textwidth}
\listasm[lastline=24,highlightlines={22-24}]{src/ocaml/caml\_start\_program.s}{runtime/amd64.S:604}
\end{column}
\end{columns}
\end{frame}


\begin{frame}{Setup to jump into OCaml entry point}{\funcname{caml\_start\_program} initializes the main fiber}
\begin{columns}
\begin{column}{0.6\textwidth}
\only<1,3,5>{
\begin{itemize}
\item<1-> Stores information to allow switching from OCaml stack to the C stack
\item<3-> Constructs the default exception handler
\item<5-> Switches to the OCaml stack and calls \funcname{caml\_program}
\end{itemize}
\bigskip
\listocaml{../src/default/default.ml}{default/default.ml}
}%
\only<2>{
\listasm[firstline=22,lastline=41,highlightlines={25-27}]{src/ocaml/caml\_start\_program.s}{caml\_start\_program}
}%
\only<4>{
\mintasm[firstline=22,lastline=41,highlightlines={31-38}]{src/ocaml/caml\_start\_program.s}
\listasm[firstline=66,lastline=72]{src/ocaml/caml\_start\_program.s}{caml\_start\_program}
}%
\only<6>{
\listasm[firstline=22,lastline=41,highlightlines={39-41}]{src/ocaml/caml\_start\_program.s}{caml\_start\_program}
}%
\end{column}
\begin{column}{0.4\textwidth}
\centering
\only<1-2>{\providecommand\step{2}\input{diagrams/default/default.tex}}%
\only<3-5>{\providecommand\step{3}\input{diagrams/default/default.tex}}%
\onslide*<6->{\providecommand\step{4}\input{diagrams/default/default.tex}}%
\end{column}
\end{columns}
\end{frame}

\begin{frame}{Executing the OCaml program}{And raising an exception without handler}
\begin{columns}
\begin{column}{0.6\textwidth}
\only<1-3>{
\begin{itemize}
\item<1-> Execution continues with OCaml code
\begin{itemize}
\item \funcname{camlDefault\_\_main\_269}
\item \funcname{camlDefault\_\_entry}
\item \funcname{caml\_program}
\end{itemize}
\item<1-> \funcname{camlDefault\_\_main\_269} raises a \typename{Failure} exception
\item<1-> \typename{Failure} is caught by \funcname{caml\_start\_program}
\begin{itemize}
\item<1-> The handler set the 2nd LSB of the exception value to \funcarg{1}
\item<3-> And then forces \funcname{caml\_start\_program} to terminate
\end{itemize}
\end{itemize}
}
\bigskip
\only<1,3>{\listocaml[highlightlines=3]{../src/default/default.ml}{default/default.ml}}%
\only<2>{\listasm[firstline=66,lastline=72,highlightlines=69]{src/ocaml/caml\_start\_program.s}{caml\_start\_program}}%
\end{column}
\begin{column}{0.4\textwidth}
\centering
\providecommand\step{4}\input{diagrams/default/default.tex}
\end{column}
\end{columns}
\end{frame}
%\only<>{\listasm[firstline=47,lastline=65]{src/ocaml/caml\_start\_program.s}{caml\_start\_program}}


\begin{frame}{Back to C}{Clean up, complain and terminate}
\begin{columns}
\begin{column}{0.45\textwidth}
\begin{itemize}
\item Backtrace:
\begin{itemize}
\item \funcname{caml\_start\_program} $\rightarrow$ OCaml code
\item \funcname{caml\_startup\_common}
\item \funcname{caml\_startup\_exn}
\item \funcname{caml\_startup}
\item \funcname{caml\_main}
\item \funcname{main}
\end{itemize}
\smallskip
\item \funcname{caml\_startup} checks the return value of \funcname{caml\_startup\_exn}
\begin{itemize}
\item The return value has been specially marked by \funcname{caml\_start\_program} handler
\item Calls \funcname{caml\_fatal\_uncaught\_exception}
\end{itemize}
\item<2-> \funcname{caml\_fatal\_uncaught\_exception} either
\begin{itemize}
\item<2-> Calls the user custom uncaught exception handler, if set with \mintinline{ocaml}{Printexc.set_uncaught_exception_handler fn}\footnotemark
\item<2-> Or falls back to the default one
\item<2-> Which notifies about the uncaught exception
\end{itemize}
\item<4-> Terminates the program either with \funcname{abort} or with a non-zero exit code
\end{itemize}
\end{column}
\begin{column}{0.55\textwidth}
\only<1>{
\listc[firstline=84,lastline=89,highlightlines=88]{../ocaml/runtime/caml/mlvalues.h}{runtime/caml/mlvalues.h:84}
\listc[firstline=138,lastline=143,highlightlines={141-142}]{../ocaml/runtime/startup\_nat.c}{runtime/startup\_nat.c:138}
}%
\only<2>{
\listc[firstline=138,highlightlines={147,149}]{../ocaml/runtime/printexc.c}{runtime/printexc.c:138}
}%
\only<3>{
\listc[firstline=110,lastline=134,highlightlines=129]{../ocaml/runtime/printexc.c}{runtime/printexc.c:138}
}%
\only<4>{
\listc[firstline=138,highlightlines={152,154}]{../ocaml/runtime/printexc.c}{runtime/printexc.c:138}
}%
\end{column}
\end{columns}
\footnotetext[1]{\url{https://v2.ocaml.org/api/Printexc.html\#1\_Uncaughtexceptions}}
\end{frame}

\end{column}
\end{columns}
\end{frame}
%\only<>{\listasm[firstline=47,lastline=65]{src/ocaml/caml\_start\_program.s}{caml\_start\_program}}


\begin{frame}{Back to C}{Clean up, complain and terminate}
\begin{columns}
\begin{column}{0.45\textwidth}
\begin{itemize}
\item Backtrace:
\begin{itemize}
\item \funcname{caml\_start\_program} $\rightarrow$ OCaml code
\item \funcname{caml\_startup\_common}
\item \funcname{caml\_startup\_exn}
\item \funcname{caml\_startup}
\item \funcname{caml\_main}
\item \funcname{main}
\end{itemize}
\smallskip
\item \funcname{caml\_startup} checks the return value of \funcname{caml\_startup\_exn}
\begin{itemize}
\item The return value has been specially marked by \funcname{caml\_start\_program} handler
\item Calls \funcname{caml\_fatal\_uncaught\_exception}
\end{itemize}
\item<2-> \funcname{caml\_fatal\_uncaught\_exception} either
\begin{itemize}
\item<2-> Calls the user custom uncaught exception handler, if set with \mintinline{ocaml}{Printexc.set_uncaught_exception_handler fn}\footnotemark
\item<2-> Or falls back to the default one
\item<2-> Which notifies about the uncaught exception
\end{itemize}
\item<4-> Terminates the program either with \funcname{abort} or with a non-zero exit code
\end{itemize}
\end{column}
\begin{column}{0.55\textwidth}
\only<1>{
\listc[firstline=84,lastline=89,highlightlines=88]{../ocaml/runtime/caml/mlvalues.h}{runtime/caml/mlvalues.h:84}
\listc[firstline=138,lastline=143,highlightlines={141-142}]{../ocaml/runtime/startup\_nat.c}{runtime/startup\_nat.c:138}
}%
\only<2>{
\listc[firstline=138,highlightlines={147,149}]{../ocaml/runtime/printexc.c}{runtime/printexc.c:138}
}%
\only<3>{
\listc[firstline=110,lastline=134,highlightlines=129]{../ocaml/runtime/printexc.c}{runtime/printexc.c:138}
}%
\only<4>{
\listc[firstline=138,highlightlines={152,154}]{../ocaml/runtime/printexc.c}{runtime/printexc.c:138}
}%
\end{column}
\end{columns}
\footnotetext[1]{\url{https://v2.ocaml.org/api/Printexc.html\#1\_Uncaughtexceptions}}
\end{frame}

\end{column}
\end{columns}
\end{frame}
%\only<>{\listasm[firstline=47,lastline=65]{src/ocaml/caml\_start\_program.s}{caml\_start\_program}}


\begin{frame}{Back to C}{Clean up, complain and terminate}
\begin{columns}
\begin{column}{0.45\textwidth}
\begin{itemize}
\item Backtrace:
\begin{itemize}
\item \funcname{caml\_start\_program} $\rightarrow$ OCaml code
\item \funcname{caml\_startup\_common}
\item \funcname{caml\_startup\_exn}
\item \funcname{caml\_startup}
\item \funcname{caml\_main}
\item \funcname{main}
\end{itemize}
\smallskip
\item \funcname{caml\_startup} checks the return value of \funcname{caml\_startup\_exn}
\begin{itemize}
\item The return value has been specially marked by \funcname{caml\_start\_program} handler
\item Calls \funcname{caml\_fatal\_uncaught\_exception}
\end{itemize}
\item<2-> \funcname{caml\_fatal\_uncaught\_exception} either
\begin{itemize}
\item<2-> Calls the user custom uncaught exception handler, if set with \mintinline{ocaml}{Printexc.set_uncaught_exception_handler fn}\footnotemark
\item<2-> Or falls back to the default one
\item<2-> Which notifies about the uncaught exception
\end{itemize}
\item<4-> Terminates the program either with \funcname{abort} or with a non-zero exit code
\end{itemize}
\end{column}
\begin{column}{0.55\textwidth}
\only<1>{
\listc[firstline=84,lastline=89,highlightlines=88]{../ocaml/runtime/caml/mlvalues.h}{runtime/caml/mlvalues.h:84}
\listc[firstline=138,lastline=143,highlightlines={141-142}]{../ocaml/runtime/startup\_nat.c}{runtime/startup\_nat.c:138}
}%
\only<2>{
\listc[firstline=138,highlightlines={147,149}]{../ocaml/runtime/printexc.c}{runtime/printexc.c:138}
}%
\only<3>{
\listc[firstline=110,lastline=134,highlightlines=129]{../ocaml/runtime/printexc.c}{runtime/printexc.c:138}
}%
\only<4>{
\listc[firstline=138,highlightlines={152,154}]{../ocaml/runtime/printexc.c}{runtime/printexc.c:138}
}%
\end{column}
\end{columns}
\footnotetext[1]{\url{https://v2.ocaml.org/api/Printexc.html\#1\_Uncaughtexceptions}}
\end{frame}

\end{column}
\end{columns}
\end{frame}
%\only<>{\listasm[firstline=47,lastline=65]{src/ocaml/caml\_start\_program.s}{caml\_start\_program}}


\begin{frame}{Back to C}{Clean up, complain and terminate}
\begin{columns}
\begin{column}{0.45\textwidth}
\begin{itemize}
\item Backtrace:
\begin{itemize}
\item \funcname{caml\_start\_program} $\rightarrow$ OCaml code
\item \funcname{caml\_startup\_common}
\item \funcname{caml\_startup\_exn}
\item \funcname{caml\_startup}
\item \funcname{caml\_main}
\item \funcname{main}
\end{itemize}
\smallskip
\item \funcname{caml\_startup} checks the return value of \funcname{caml\_startup\_exn}
\begin{itemize}
\item The return value has been specially marked by \funcname{caml\_start\_program} handler
\item Calls \funcname{caml\_fatal\_uncaught\_exception}
\end{itemize}
\item<2-> \funcname{caml\_fatal\_uncaught\_exception} either
\begin{itemize}
\item<2-> Calls the user custom uncaught exception handler, if set with \mintinline{ocaml}{Printexc.set_uncaught_exception_handler fn}\footnotemark
\item<2-> Or falls back to the default one
\item<2-> Which notifies about the uncaught exception
\end{itemize}
\item<4-> Terminates the program either with \funcname{abort} or with a non-zero exit code
\end{itemize}
\end{column}
\begin{column}{0.55\textwidth}
\only<1>{
\listc[firstline=84,lastline=89,highlightlines=88]{../ocaml/runtime/caml/mlvalues.h}{runtime/caml/mlvalues.h:84}
\listc[firstline=138,lastline=143,highlightlines={141-142}]{../ocaml/runtime/startup\_nat.c}{runtime/startup\_nat.c:138}
}%
\only<2>{
\listc[firstline=138,highlightlines={147,149}]{../ocaml/runtime/printexc.c}{runtime/printexc.c:138}
}%
\only<3>{
\listc[firstline=110,lastline=134,highlightlines=129]{../ocaml/runtime/printexc.c}{runtime/printexc.c:138}
}%
\only<4>{
\listc[firstline=138,highlightlines={152,154}]{../ocaml/runtime/printexc.c}{runtime/printexc.c:138}
}%
\end{column}
\end{columns}
\footnotetext[1]{\url{https://v2.ocaml.org/api/Printexc.html\#1\_Uncaughtexceptions}}
\end{frame}
