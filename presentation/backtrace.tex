 of the previous frame
        \item And the return address of the caller of this function
        \bigskip
        \onslide<3->{
        \item Note that traps (here the reddish \funcname{ExnA}, \funcname{ExnB} and \funcname{ExnC} blocks) belong to the frame that installed them, and so the frame size changes when traps are removed
        }
      \end{itemize}
    \end{column}
    \begin{column}{0.5\textwidth}
      \raggedleft
      \onslide*<1>{\providecommand\step{2}\input{diagrams/backtrace/backtrace.tex}}%
      \onslide*<2>{\providecommand\step{1}\input{diagrams/backtrace/scanstack.tex}}%
      \onslide*<3>{\providecommand\step{2}\input{diagrams/backtrace/scanstack.tex}}%
      \onslide*<4>{\providecommand\step{3}\input{diagrams/backtrace/scanstack.tex}}%
      \onslide*<5>{\providecommand\step{4}\input{diagrams/backtrace/scanstack.tex}}%
      \onslide*<6>{\providecommand\step{5}\input{diagrams/backtrace/scanstack.tex}}%
    \end{column}
  \end{columns}
\end{frame}

% Duplicate of previous-previous slide
\begin{frame}{Backtraces}{Continuing on \funcname{caml\_stash\_backtrace}}
  \begin{columns}[c]
    \begin{column}{0.5\textwidth}
      \minipage[c][0.75\textheight][s]{\columnwidth}
      \begin{itemize}
          \color{gray}
        \item A C function provided by the runtime (specific to native code)
        \item It scans the stack, between the stack pointer at the raising point up to the next trap, in order to collect the names of the detected function frames
        \item It uses the same technique and information as the Garbage Collector (GC) uses to scan the values live on the stack
      \end{itemize}
      \begin{itemize}
        \item For each function frame detected, store a pointer to the \typename{frame\_descr} in the \localname{domain\_state->backtrace\_buffer} array, effectively constructing the backtrace
      \end{itemize}
      \onslide<2->{
        \begin{itemize}
            \smallskip
          \item Constructed backtrace:
            \funcname{definitely\_raise},
            \onslide<3->{\funcname{probably\_raise}}
        \end{itemize}
      }
      \vfill
      \mintocaml[firstline=9,lastline=13,highlightlines=12]{../src/backtrace/backtrace.ml}
      \endminipage
    \end{column}
    \begin{column}{0.5\textwidth}
      \raggedleft
      \onslide*<1>{\listStashBacktrace[highlightlines={114-115}]{}}
      \onslide*<2>{\providecommand\step{3}\input{diagrams/backtrace/backtrace.tex}}%
      \onslide*<3>{\providecommand\step{4}\input{diagrams/backtrace/backtrace.tex}}%
    \end{column}
  \end{columns}
\end{frame}

\begin{frame}{Backtraces}{Back to \funcname{caml\_raise\_exn}}
  \begin{columns}[c]
    \begin{column}{0.5\textwidth}
      \minipage[c][0.66\textheight][s]{\columnwidth}
      \begin{itemize}\color{gray}
        \item Checks wheteher exceptions backtrace should be recorded, by checking the \funcarg{domain\_state->backtrace\_active} value
        \item Switches to the C stack to be able to execute C code
        \item Calls \funcname{caml\_stash\_backtrace}, to collect the backtrace up to the next trap
      \end{itemize}
      \onslide*<2->{
        \begin{itemize}
          \item<2-> Executes the exception handler in \funcname{probably\_raise} with \funcname{RESTORE\_EXN\_HANDLER\_OCAML}
            \begin{itemize}
              \item Set \regname{RSP} to \localname{domain\_state->exn\_handler} value
              \item Restore \localname{domain\_state->exn\_handler} from trap
              \item Jump to the handler code with \funcname{ret}
            \end{itemize}
        \end{itemize}
      }
      \vfill
      \onslide*<1>{\mintocaml[firstline=9,lastline=13,highlightlines=12]{../src/backtrace/backtrace.ml}}
      \onslide*<2->{\mintocaml[firstline=15,lastline=21,highlightlines={19-21}]{../src/backtrace/backtrace.ml}}
      \endminipage
    \end{column}
    \begin{column}{0.5\textwidth}
      \centering
      \onslide*<1>{
        \mintc[firstline=190,lastline=192]{../ocaml/runtime/amd64.S}
        \mintc[firstline=234,lastline=237]{../ocaml/runtime/amd64.S}
      }
      \onslide*<2>{
        \mintc[firstline=190,lastline=192,highlightlines={191-192}]{../ocaml/runtime/amd64.S}
        \mintc[firstline=234,lastline=237,highlightlines={235-237}]{../ocaml/runtime/amd64.S}
      }
      \onslide*<1>{\listCamlRaiseExn[highlightlines=720]{}}
      \onslide*<2>{\listCamlRaiseExn[highlightlines=722]{}}
      \onslide*<3->{\providecommand\step{5}\input{diagrams/backtrace/backtrace.tex}}
    \end{column}
  \end{columns}
\end{frame}

\begin{frame}{Backtraces}{\funcname{caml\_probably\_raise}'s handler doesn't catch \typename{ExnC}}
  \begin{columns}[c]
    \begin{column}{0.45\textwidth}
      \minipage[c][0.75\textheight][s]{\columnwidth}
      \begin{itemize}
        \item<1-> \funcname{match \dots with}\footnotemark pattern matching can also match exceptions
        \item<1-> The CMM of \funcname{probably\_raise} shows that this \funcname{match \dots with} is implemented with a pattern matching and a \funcname{try \dots with}
        \item<1-> But this one only matches on \typename{ExnA} exception
        \item<2-> Since the pattern matching fails to catch the exception, \funcname{reraise} is called with our current \typename{ExnC} exception
        \item<3-> Disassembly shows that \funcname{reraise} is actually a call to \funcname{caml\_reraise\_exn}, which is also an assembly function provided by the runtime
      \end{itemize}
      \vfill
      \mintocaml[firstline=15,lastline=21,highlightlines={18-21}]{../src/backtrace/backtrace.ml}
      \endminipage
    \end{column}
    \begin{column}{0.55\textwidth}
      \centering
      \onslide*<1>{\mintcmm[highlightlines={10-18}]{../src/backtrace/camlBacktrace\_\_probably\_raise\_351.cmm}}
      \onslide*<2>{\listcmm[highlightlines=18]{../src/backtrace/camlBacktrace\_\_probably\_raise_351.cmm}{CMM of probably\_raise}}
      \onslide*<3>{\mintobjdump[firstline=5,lastline=35,highlightlines=31]{../src/backtrace/camlBacktrace\_\_probably\_raise_351.objdump}}
    \end{column}
  \end{columns}
  \only<1>{\footnotetext[1]{https://blog.janestreet.com/pattern-matching-and-exception-handling-unite/}}
\end{frame}

\newcommand\listCamlReraiseExn[1][]{
  \listasm[firstline=727,lastline=734,#1]{../ocaml/runtime/amd64.S}{runtime/amd64.S:727}}

\begin{frame}{Backtraces}{Introducing \funcname{caml\_reraise\_exn}}
  \begin{columns}[c]
    \begin{column}{0.5\textwidth}
      \minipage[c][0.66\textheight][s]{\columnwidth}
      \begin{itemize}
        \item<1-> The exception have to be reraised to the next handler
        \item<2-> First, checks if the backtrace should be collected with \funcarg{domain\_state->backtrace\_active}
        \only<3>{
        \begin{itemize}
            \item If not, behaves like a \funcname{raise\_notrace} with \funcname{RESTORE\_EXN\_HANDLER\_OCAML}
        \end{itemize}
        }
        \item<4-> Jumps into \funcname{caml\_raise\_exn}
        \item<5-> Calls \funcname{caml\_stash\_backtrace} again, to scan the next part of the stack: from the trap just pop-ed to the next trap
      \end{itemize}
      \vfill
      \mintocaml[firstline=15,lastline=21,highlightlines={18-21}]{../src/backtrace/backtrace.ml}
      \endminipage
    \end{column}
    \begin{column}{0.5\textwidth}
      \raggedleft
      \onslide*<1>{\listCamlReraiseExn{}}
      \onslide*<2>{\listCamlReraiseExn[highlightlines=729]{}}
      \onslide*<3>{
        \mintc[firstline=190,lastline=192,highlightlines={191-192}]{../ocaml/runtime/amd64.S}
        \mintc[firstline=234,lastline=237,highlightlines={235-237}]{../ocaml/runtime/amd64.S}
        \medskip
        \listCamlReraiseExn[highlightlines=731]{}
      }
      \onslide*<4-5>{
        % TODO refactor with \listCamlReraiseExn
        \mintasm[firstline=727,lastline=734,highlightlines=730]{../ocaml/runtime/amd64.S}
        \medskip
      }
      \onslide*<4>{\listCamlRaiseExn[highlightlines=711]{}}
      \onslide*<5>{\listCamlRaiseExn[highlightlines=720]{}}
    \end{column}
  \end{columns}
\end{frame}

\begin{frame}{Backtraces}{Backtrace collection continues}
  \begin{columns}[c]
    \begin{column}{0.55\textwidth}
      \minipage[c][0.75\textheight][s]{\columnwidth}
      \begin{itemize}
        \item<1-> \funcname{caml\_stash\_backtrace} scans the older part of the stack, up to the next trap
        \item<6-> Controls jumps to the nested exception handler in \funcname{maybe\_raise}, which matches only on \typename{ExnB}
        \item<7-> \funcname{caml\_reraise\_exn} is called again, backtrace collection resumes with \funcname{caml\_stash\_backtrace}
      \end{itemize}
      \medskip
      \begin{itemize}
        \item<2-> Constructed backtrace:
        \begin{itemize}
          \item <2-> \funcname{definitely\_raise}
          \item <2-> \funcname{probably\_raise}
          \item <3-> \funcname{probably\_raise} (re-raise)
          \item <4-> \funcname{maybe\_raise}
          \item <5-> \funcname{main}
          \item <8-> \funcname{main} (re-raise)
        \end{itemize}
      \end{itemize}
      \vfill
      \onslide*<1-5>{\mintocaml[firstline=15,lastline=21]{../src/backtrace/backtrace.ml}}
      \onslide*<6>{\mintocaml[firstline=28,lastline=34,highlightlines=31]{../src/backtrace/backtrace.ml}}
      \onslide*<7->{\mintocaml[firstline=28,lastline=34,highlightlines=33]{../src/backtrace/backtrace.ml}}
      \endminipage
    \end{column}
    \begin{column}{0.45\textwidth}
      \raggedleft
      \onslide*<1>{\providecommand\step{6}\input{diagrams/backtrace/backtrace.tex}}%
      \onslide*<2>{\providecommand\step{6}\input{diagrams/backtrace/backtrace.tex}}%
      \onslide*<3>{\providecommand\step{7}\input{diagrams/backtrace/backtrace.tex}}%
      \onslide*<4>{\providecommand\step{8}\input{diagrams/backtrace/backtrace.tex}}%
      \onslide*<5>{\providecommand\step{9}\input{diagrams/backtrace/backtrace.tex}}%
      \onslide*<6>{\providecommand\step{10}\input{diagrams/backtrace/backtrace.tex}}%
      \onslide*<7>{\providecommand\step{11}\input{diagrams/backtrace/backtrace.tex}}%
      \onslide*<8>{\providecommand\step{12}\input{diagrams/backtrace/backtrace.tex}}%
      \onslide*<9>{\providecommand\step{13}\input{diagrams/backtrace/backtrace.tex}}%
    \end{column}
  \end{columns}
\end{frame}

\begin{frame}{Backtraces}{Printing the backtrace}
  \begin{columns}[c]
    \begin{column}{0.45\textwidth}
      \minipage[c][0.66\textheight][s]{\columnwidth}
      \begin{itemize}
        \item<1-> The exception backtrace can now be printed to \funcarg{stdout} with \funcname{Printexc.print_backtrace}
        \item<2-> First the exception backtrace is retrieved into an OCaml array with \funcname{get_raw_backtrace }
        \item<3-> Which is implemented as the \funcname{caml_get_exception_raw_backtrace} C function in the runtime
        \item<4-> And finally printed with \funcname{print_exception_backtrace}
      \end{itemize}
      \vfill
      \mintocaml[firstline=28,lastline=34,highlightlines=33]{../src/backtrace/backtrace.ml}
      \endminipage
    \end{column}
    \begin{column}{0.55\textwidth}
      \onslide*<1,2>{
        \mintocaml[firstline=100,lastline=101]{../ocaml/stdlib/printexc.ml}
      }
      \only<1>{
        \listocaml[firstline=170,lastline=172]{../ocaml/stdlib/printexc.ml}{stdlib/printexc.ml:170}
      }
      \only<2>{
        \listocaml[firstline=170,lastline=172,highlightlines=172]{../ocaml/stdlib/printexc.ml}{stdlib/printexc.ml:170}
      }
      \only<3>{
        \mintc[firstline=154,lastline=156]{../ocaml/runtime/backtrace.c}
        \listc[firstline=171,lastline=192,highlightlines={184-187}]{../ocaml/runtime/backtrace.c}{runtime/backtrace.c:154}
      }
      \only<4>{
        \listocaml[firstline=155,lastline=165]{../ocaml/stdlib/printexc.ml}{stdlib/printexc.ml:55}
      }
    \end{column}
  \end{columns}
\end{frame}


\subsection{The multiple kinds of raise}
\frameSubsection{}{}

\begin{frame}{Backtraces}{When backtraces are not needed}
  \begin{columns}[c]
    \begin{column}{0.45\textwidth}
      \minipage[c][0.66\textheight][s]{\columnwidth}
      \begin{itemize}
        \item<1-> What if the backtrace for an exception is never needed?
        \item<2-> It's possible to raise the exception explicitly with \funcname{raise\_notrace} to make raising as cheap as possible
        \item<3-> An example of use in the \funcname{dynlink} library of the OCaml compiler
      \end{itemize}
      \vfill
      \onslide<3->{
      \listocaml[firstline=205,lastline=209,highlightlines=207]{../ocaml/otherlibs/dynlink/dynlink\_compilerlibs/misc.ml}{otherlibs/dynlink/dynlink\_compilerlibs/misc.ml:205}
      }
      \endminipage
    \end{column}
    \begin{column}{0.55\textwidth}
    \only<4>{
      \mintcmm[highlightlines=3]{../src/notrace/camlNotrace\_\_fun_347.cmm}
      \listcmm{../src/notrace/camlNotrace\_\_all\_somes\_327.cmm}{all\_somes}
    }
    \onslide*<5->{
      \listobjdump[highlightlines={6-9}]{../src/notrace/camlNotrace\_\_fun_347.objdump}{anonymous function from \funcname{all\_somes}}
    }
    \end{column}
  \end{columns}
\end{frame}

\begin{frame}{Backtraces}{Summary table of the multiple kinds of raise}
  \begin{itemize}
    \item When debugging information is enabled and if the backtrace of an exception isn't needed, it's possible to raise with \funcname{raise\_notrace} for performance
    \item When debugging information is \emph{not} enabled, the compiler optimises \funcname{raise} calls to inline assembly (same effect as using \funcname{raise\_notrace})
  \end{itemize}
  \bigskip
  \centering
  \begin{tabular}{| l | c | c | c | c |}
    \hline
    \parbox{10em}{Debug information \\ (\funcarg{-g} flag)}  & \multicolumn{2}{|c|}{Disabled}                                     & \multicolumn{2}{|c|}{Enabled} \\
    \hline
    Raise kind                                               & \funcname{raise} & \funcname{raise\_notrace}                       & \funcname{raise} & \funcname{raise\_notrace} \\
    \hline
    Compilation                                              & \parbox{8em}{inline assembly\\ (optimization)} & inline assembly   & \funcname{caml\_raise\_exn} & inline assembly \\
    \hline
  \end{tabular}
\end{frame}


%
% Takeaway #5
%
\subsection*{Takeaway \#5}
\frameSubsectionTakeaway{}
\againframe<5>{takeaway}
}%
    \end{column}
  \end{columns}
\end{frame}

\begin{frame}{Backtraces}{Printing the backtrace}
  \begin{columns}[c]
    \begin{column}{0.45\textwidth}
      \minipage[c][0.66\textheight][s]{\columnwidth}
      \begin{itemize}
        \item<1-> The exception backtrace can now be printed to \funcarg{stdout} with \funcname{Printexc.print_backtrace}
        \item<2-> First the exception backtrace is retrieved into an OCaml array with \funcname{get_raw_backtrace }
        \item<3-> Which is implemented as the \funcname{caml_get_exception_raw_backtrace} C function in the runtime
        \item<4-> And finally printed with \funcname{print_exception_backtrace}
      \end{itemize}
      \vfill
      \mintocaml[firstline=28,lastline=34,highlightlines=33]{../src/backtrace/backtrace.ml}
      \endminipage
    \end{column}
    \begin{column}{0.55\textwidth}
      \onslide*<1,2>{
        \mintocaml[firstline=100,lastline=101]{../ocaml/stdlib/printexc.ml}
      }
      \only<1>{
        \listocaml[firstline=170,lastline=172]{../ocaml/stdlib/printexc.ml}{stdlib/printexc.ml:170}
      }
      \only<2>{
        \listocaml[firstline=170,lastline=172,highlightlines=172]{../ocaml/stdlib/printexc.ml}{stdlib/printexc.ml:170}
      }
      \only<3>{
        \mintc[firstline=154,lastline=156]{../ocaml/runtime/backtrace.c}
        \listc[firstline=171,lastline=192,highlightlines={184-187}]{../ocaml/runtime/backtrace.c}{runtime/backtrace.c:154}
      }
      \only<4>{
        \listocaml[firstline=155,lastline=165]{../ocaml/stdlib/printexc.ml}{stdlib/printexc.ml:55}
      }
    \end{column}
  \end{columns}
\end{frame}


\subsection{The multiple kinds of raise}
\frameSubsection{}{}

\begin{frame}{Backtraces}{When backtraces are not needed}
  \begin{columns}[c]
    \begin{column}{0.45\textwidth}
      \minipage[c][0.66\textheight][s]{\columnwidth}
      \begin{itemize}
        \item<1-> What if the backtrace for an exception is never needed?
        \item<2-> It's possible to raise the exception explicitly with \funcname{raise\_notrace} to make raising as cheap as possible
        \item<3-> An example of use in the \funcname{dynlink} library of the OCaml compiler
      \end{itemize}
      \vfill
      \onslide<3->{
      \listocaml[firstline=205,lastline=209,highlightlines=207]{../ocaml/otherlibs/dynlink/dynlink\_compilerlibs/misc.ml}{otherlibs/dynlink/dynlink\_compilerlibs/misc.ml:205}
      }
      \endminipage
    \end{column}
    \begin{column}{0.55\textwidth}
    \only<4>{
      \mintcmm[highlightlines=3]{../src/notrace/camlNotrace\_\_fun_347.cmm}
      \listcmm{../src/notrace/camlNotrace\_\_all\_somes\_327.cmm}{all\_somes}
    }
    \onslide*<5->{
      \listobjdump[highlightlines={6-9}]{../src/notrace/camlNotrace\_\_fun_347.objdump}{anonymous function from \funcname{all\_somes}}
    }
    \end{column}
  \end{columns}
\end{frame}

\begin{frame}{Backtraces}{Summary table of the multiple kinds of raise}
  \begin{itemize}
    \item When debugging information is enabled and if the backtrace of an exception isn't needed, it's possible to raise with \funcname{raise\_notrace} for performance
    \item When debugging information is \emph{not} enabled, the compiler optimises \funcname{raise} calls to inline assembly (same effect as using \funcname{raise\_notrace})
  \end{itemize}
  \bigskip
  \centering
  \begin{tabular}{| l | c | c | c | c |}
    \hline
    \parbox{10em}{Debug information \\ (\funcarg{-g} flag)}  & \multicolumn{2}{|c|}{Disabled}                                     & \multicolumn{2}{|c|}{Enabled} \\
    \hline
    Raise kind                                               & \funcname{raise} & \funcname{raise\_notrace}                       & \funcname{raise} & \funcname{raise\_notrace} \\
    \hline
    Compilation                                              & \parbox{8em}{inline assembly\\ (optimization)} & inline assembly   & \funcname{caml\_raise\_exn} & inline assembly \\
    \hline
  \end{tabular}
\end{frame}


%
% Takeaway #5
%
\subsection*{Takeaway \#5}
\frameSubsectionTakeaway{}
\againframe<5>{takeaway}
}%
      \onslide*<8>{\providecommand\step{12}%
% Backtraces
%
\subsection{One advantage of raising with debugging information}
\frameSubsection{}{}

\begin{frame}{Backtraces}{Debugging information \& source code}
  \begin{columns}[c]
    \begin{column}{0.5\textwidth}
      \begin{itemize}
        \item Raising an exception is fun and games, but how do we know where the exception originated? $\rightarrow$ With the backtrace associated with the exception!
        \item In order to get a backtrace from the point where an exception was raised, it's necessary to compile with debugging information enabled (\funcarg{-g} flag for the compiler)
        \item Same example as in the `Nested handlers' section
      \end{itemize}
    \end{column}
    \begin{column}{0.5\textwidth}
      \listocaml[firstline=9,lastline=34]{../src/backtrace/backtrace.ml}{backtrace/backtrace.ml}
    \end{column}
  \end{columns}
\end{frame}

\begin{frame}{Backtraces}{CMM and disassembly of \funcname{definitely\_raise}}
  \begin{columns}[c]
    \begin{column}{0.5\textwidth}
      \minipage[c][0.66\textheight][s]{\columnwidth}
      \begin{itemize}
        \item<1-> An effect of compiling with debugging information (\funcarg{-g}): the CMM no longer uses \funcname{raise\_notrace} to raise the exception but \funcname{raise} instead
        \item<2-> Disassembly shows that CMM \funcname{raise} is actually a call to \funcname{caml\_raise\_exn}, a function in assembler provided by the runtime
      \end{itemize}
      \vfill
      \mintocaml[firstline=9,lastline=13,highlightlines=12]{../src/backtrace/backtrace.ml}
      \endminipage
    \end{column}
    \begin{column}{0.5\textwidth}
      \onslide*<1>{
        \mintcmm[highlightlines=5]{../src/backtrace/camlBacktrace\_\_definitely\_raise\_347.cmm}
      }
      \onslide*<2>{
        \mintobjdump[firstline=2,highlightlines=14]{../src/backtrace/camlBacktrace\_\_definitely\_raise\_347.objdump}
      }
    \end{column}
  \end{columns}
\end{frame}

\begin{frame}{Backtraces}{Introducing \funcname{caml\_raise\_exn}}
  \begin{columns}[c]
    \begin{column}{0.5\textwidth}
      \minipage[c][0.66\textheight][s]{\columnwidth}
      \onslide*<1>{
        \begin{itemize}
          \item Raising the exception is performed using \funcname{caml\_raise\_exn} which is
            \begin{itemize}
              \item An assembly function provided by the runtime
              \item Architecture specific
            \end{itemize}
          \item Let's see what it does differently from \funcname{raise\_notrace}\dots
          \item We are about to raise \typename{ExnC} with \funcname{caml\_raise\_exn} from \funcname{definitely\_raise}
        \end{itemize}
      }
      \onslide*<2>{
        \mintobjdump[firstline=2,highlightlines=14]{../src/backtrace/camlBacktrace\_\_definitely\_raise\_347.objdump}
      }
      \vfill
      \mintocaml[firstline=9,lastline=13,highlightlines=12]{../src/backtrace/backtrace.ml}
      \endminipage
    \end{column}
    \begin{column}{0.5\textwidth}
      \centering
      \providecommand\step{1}%
% Backtraces
%
\subsection{One advantage of raising with debugging information}
\frameSubsection{}{}

\begin{frame}{Backtraces}{Debugging information \& source code}
  \begin{columns}[c]
    \begin{column}{0.5\textwidth}
      \begin{itemize}
        \item Raising an exception is fun and games, but how do we know where the exception originated? $\rightarrow$ With the backtrace associated with the exception!
        \item In order to get a backtrace from the point where an exception was raised, it's necessary to compile with debugging information enabled (\funcarg{-g} flag for the compiler)
        \item Same example as in the `Nested handlers' section
      \end{itemize}
    \end{column}
    \begin{column}{0.5\textwidth}
      \listocaml[firstline=9,lastline=34]{../src/backtrace/backtrace.ml}{backtrace/backtrace.ml}
    \end{column}
  \end{columns}
\end{frame}

\begin{frame}{Backtraces}{CMM and disassembly of \funcname{definitely\_raise}}
  \begin{columns}[c]
    \begin{column}{0.5\textwidth}
      \minipage[c][0.66\textheight][s]{\columnwidth}
      \begin{itemize}
        \item<1-> An effect of compiling with debugging information (\funcarg{-g}): the CMM no longer uses \funcname{raise\_notrace} to raise the exception but \funcname{raise} instead
        \item<2-> Disassembly shows that CMM \funcname{raise} is actually a call to \funcname{caml\_raise\_exn}, a function in assembler provided by the runtime
      \end{itemize}
      \vfill
      \mintocaml[firstline=9,lastline=13,highlightlines=12]{../src/backtrace/backtrace.ml}
      \endminipage
    \end{column}
    \begin{column}{0.5\textwidth}
      \onslide*<1>{
        \mintcmm[highlightlines=5]{../src/backtrace/camlBacktrace\_\_definitely\_raise\_347.cmm}
      }
      \onslide*<2>{
        \mintobjdump[firstline=2,highlightlines=14]{../src/backtrace/camlBacktrace\_\_definitely\_raise\_347.objdump}
      }
    \end{column}
  \end{columns}
\end{frame}

\begin{frame}{Backtraces}{Introducing \funcname{caml\_raise\_exn}}
  \begin{columns}[c]
    \begin{column}{0.5\textwidth}
      \minipage[c][0.66\textheight][s]{\columnwidth}
      \onslide*<1>{
        \begin{itemize}
          \item Raising the exception is performed using \funcname{caml\_raise\_exn} which is
            \begin{itemize}
              \item An assembly function provided by the runtime
              \item Architecture specific
            \end{itemize}
          \item Let's see what it does differently from \funcname{raise\_notrace}\dots
          \item We are about to raise \typename{ExnC} with \funcname{caml\_raise\_exn} from \funcname{definitely\_raise}
        \end{itemize}
      }
      \onslide*<2>{
        \mintobjdump[firstline=2,highlightlines=14]{../src/backtrace/camlBacktrace\_\_definitely\_raise\_347.objdump}
      }
      \vfill
      \mintocaml[firstline=9,lastline=13,highlightlines=12]{../src/backtrace/backtrace.ml}
      \endminipage
    \end{column}
    \begin{column}{0.5\textwidth}
      \centering
      \providecommand\step{1}\input{diagrams/backtrace/backtrace.tex}
    \end{column}
  \end{columns}
\end{frame}

\begin{frame}{Backtraces}{But before that, we need more fields from \typename{caml\_domain\_state}}
  \begin{columns}
    \begin{column}{0.5\textwidth}
      \begin{itemize}
        \item Remember \typename{caml\_domain\_state} the per-domain runtime state?
        \item Some other members are of interest to us now\dots
        \item \localname{backtrace\_active} is used to know if the runtime should collect backtraces
        \item \localname{backtrace\_active} can be set by
          \begin{itemize}
            \item The \funcarg{b} flag in the \funcname{OCAMLRUNPARAM} environment variable
            \item \mintinline{ocaml}{Printexc.record_backtrace : bool -> unit} \footnotemark
          \end{itemize}
      \end{itemize}
    \end{column}
    \begin{column}{0.5\textwidth}
      \begin{listing}
        \mintc[highlightlines={9-11}]{src/ocaml/domain\_state\_backtrace.h}
        \caption{runtime/caml/domain\_state.h}
      \end{listing}
    \end{column}
  \end{columns}
  \footnotetext[1]{https://v2.ocaml.org/api/Printexc.html}
\end{frame}

\newcommand\listCamlRaiseExn[1][]{
  \listasm[firstline=702,lastline=725,#1]{../ocaml/runtime/amd64.S}{runtime/amd64.S:702}}

\begin{frame}{Backtraces}{Walk through \funcname{caml\_raise\_exn}}
  \begin{columns}[T]
    \begin{column}{0.5\textwidth}
      \begin{itemize}
        \item<1-> First checks whether exceptions backtrace should be recorded
        \item<1-> By checking the \funcarg{domain\_state->backtrace\_active} value
        \item<2-> If not, acts as \funcname{raise\_notrace} with \funcname{RESTORE\_EXN\_HANDLER\_OCAML}
          \begin{itemize}
            \item Set \regname{RSP} to \localname{domain\_state->exn\_handler} value
            \item Restore \localname{domain\_state->exn\_handler} from trap
            \item Jump with \funcname{ret} (same effect as using \funcname{pop} and \funcname{jmp})
          \end{itemize}
        \item<3-> Otherwise, switches to the C stack to be able to execute C code
        \item<4-> Calls \funcname{caml\_stash\_backtrace}, a C function provided by the runtime\ignorespaces
          \onslide<6->{, with
            \begin{itemize}
              \item \funcarg{exn}: exception value
              \item \funcarg{pc}: code address of the raising point
              \item \funcarg{sp}: stack address of the raising function
              \item \funcarg{trapsp}: stack address of the next handler
            \end{itemize}
          }
      \end{itemize}
      \bigskip
      \mintocaml[firstline=9,lastline=13,highlightlines=12]{../src/backtrace/backtrace.ml}
    \end{column}
    \begin{column}{0.5\textwidth}
      \raggedleft
      \onslide*<1,3-4>{
        \mintc[firstline=190,lastline=192]{../ocaml/runtime/amd64.S}
        \mintc[firstline=234,lastline=237]{../ocaml/runtime/amd64.S}
      }
      \onslide*<2>{
        \mintc[firstline=190,lastline=192,highlightlines={191-192}]{../ocaml/runtime/amd64.S}
        \mintc[firstline=234,lastline=237,highlightlines={235-237}]{../ocaml/runtime/amd64.S}
      }
      \onslide*<1>{\listCamlRaiseExn[highlightlines=705]{}}%
      \onslide*<2>{\listCamlRaiseExn[highlightlines={707-708}]{}}%
      \onslide*<3>{\listCamlRaiseExn[highlightlines={712-714}]{}}%
      \onslide*<4>{\listCamlRaiseExn[highlightlines={715-720}]{}}%
      \onslide*<5>{\providecommand\step{1}\input{diagrams/backtrace/backtrace.tex}}%
      \onslide*<6>{\providecommand\step{2}\input{diagrams/backtrace/backtrace.tex}}%
    \end{column}
  \end{columns}
\end{frame}

\newcommand\listStashBacktrace[1][]{
  \listc[firstline=91,lastline=120,#1]{../ocaml/runtime/backtrace\_nat.c}{runtime/backtrace\_nat.c:91}}

\begin{frame}{Backtraces}{Introducing \funcname{caml\_stash\_backtrace}}
  \begin{columns}[c]
    \begin{column}{0.5\textwidth}
      \minipage[c][0.66\textheight][s]{\columnwidth}
      \begin{itemize}
        \item<1-> A C function provided by the runtime (specific to native code)
        \item<2-> It scans the stack, between the stack pointer at the raising point up to the next trap, in order to collect the names of the detected function frames
          \onslide*<2>{
            \begin{itemize}
              \item And more information, like line number, column number...
            \end{itemize}
          }
        \item<3-> It uses the same technique and information as the Garbage Collector (GC) uses to scan the values live on the stack
          \onslide*<4>{
            \begin{itemize}
              \item \typename{frame\_descr} are at the core of stack unwinding
            \end{itemize}
          }
      \end{itemize}
      \vfill
      \mintocaml[firstline=9,lastline=13,highlightlines=12]{../src/backtrace/backtrace.ml}
      \endminipage
    \end{column}
    \begin{column}{0.5\textwidth}
      \onslide*<1>{\listStashBacktrace[highlightlines=91]{}}
      \onslide*<2>{\listStashBacktrace[highlightlines={91,117-118}]{}}
      \onslide*<3->{\listStashBacktrace[highlightlines={94,106,109-110}]{}}
    \end{column}
  \end{columns}
\end{frame}

\newcommand\listFrameDescr[1][]{
  \listocaml[firstline=111,lastline=124,#1]{../ocaml/asmcomp/emitaux.ml}{asmcomp/emitaux.ml:111}}
\newcommand\listDebugInfo[1][]{
  \listocaml[firstline=38,lastline=49,#1]{../ocaml/lambda/debuginfo.mli}{lambda/debuginfo.mli:38}}

\begin{frame}{Backtraces}{\typename{frame\_descr} and \typename{frame\_debuginfo} in a nutshell}
  \begin{columns}[c]
    \begin{column}{0.5\textwidth}
      \begin{itemize}
        \item<1-> For every return address in an OCaml program, there is a associated \typename{frame\_descr}
        \item<2-> A \typename{frame\_descr} specifies
        \begin{itemize}
          \item<2-> The size of the function stack frame
          \item<3-> Where live values are on the stack (so that the GC can mark them as live and prevent from being swept)
        \end{itemize}
        \item<4-> When compiled with debug information, \typename{frame\_descr} will be linked to a \typename{frame\_debuginfo}
        \begin{itemize}
          \item<5-> Contains information about the position in the source code (file name, line and column number)
        \end{itemize}
      \end{itemize}
    \end{column}
    \begin{column}{0.5\textwidth}
        \onslide*<1>{\listFrameDescr[highlightlines={118-122}]{}}
        \onslide*<2>{\listFrameDescr[highlightlines={120}]{}}
        \onslide*<3>{\listFrameDescr[highlightlines={121}]{}}
        \onslide*<4->{\listFrameDescr[highlightlines={122}]{}}
        \medskip
        \onslide*<1-3>{\listDebugInfo{}}
        \onslide*<4>{\listDebugInfo[highlightlines={38,49}]{}}
        \onslide*<5>{\listDebugInfo[highlightlines={39-46}]{}}
    \end{column}
  \end{columns}
\end{frame}

\begin{frame}{Backtraces}{Scanning the stack with \typename{frame\_descr}}
  \begin{columns}[c]
    \begin{column}{0.5\textwidth}
      \begin{itemize}
        \item Given the return address of the code that raises the exception by calling \funcname{caml\_raise\_exn} (\funcarg{pc} argument of \funcname{caml\_stash\_backtrace})
        \item And the position of the stack pointer (\funcarg{sp} argument of \funcname{caml\_stash\_backtrace})
        \item The frame size given by \typename{frame\_descr}.\funcarg{frame\_size}, allows to find the end of the previous frame
        \item And the return address of the caller of this function
        \bigskip
        \onslide<3->{
        \item Note that traps (here the reddish \funcname{ExnA}, \funcname{ExnB} and \funcname{ExnC} blocks) belong to the frame that installed them, and so the frame size changes when traps are removed
        }
      \end{itemize}
    \end{column}
    \begin{column}{0.5\textwidth}
      \raggedleft
      \onslide*<1>{\providecommand\step{2}\input{diagrams/backtrace/backtrace.tex}}%
      \onslide*<2>{\providecommand\step{1}\input{diagrams/backtrace/scanstack.tex}}%
      \onslide*<3>{\providecommand\step{2}\input{diagrams/backtrace/scanstack.tex}}%
      \onslide*<4>{\providecommand\step{3}\input{diagrams/backtrace/scanstack.tex}}%
      \onslide*<5>{\providecommand\step{4}\input{diagrams/backtrace/scanstack.tex}}%
      \onslide*<6>{\providecommand\step{5}\input{diagrams/backtrace/scanstack.tex}}%
    \end{column}
  \end{columns}
\end{frame}

% Duplicate of previous-previous slide
\begin{frame}{Backtraces}{Continuing on \funcname{caml\_stash\_backtrace}}
  \begin{columns}[c]
    \begin{column}{0.5\textwidth}
      \minipage[c][0.75\textheight][s]{\columnwidth}
      \begin{itemize}
          \color{gray}
        \item A C function provided by the runtime (specific to native code)
        \item It scans the stack, between the stack pointer at the raising point up to the next trap, in order to collect the names of the detected function frames
        \item It uses the same technique and information as the Garbage Collector (GC) uses to scan the values live on the stack
      \end{itemize}
      \begin{itemize}
        \item For each function frame detected, store a pointer to the \typename{frame\_descr} in the \localname{domain\_state->backtrace\_buffer} array, effectively constructing the backtrace
      \end{itemize}
      \onslide<2->{
        \begin{itemize}
            \smallskip
          \item Constructed backtrace:
            \funcname{definitely\_raise},
            \onslide<3->{\funcname{probably\_raise}}
        \end{itemize}
      }
      \vfill
      \mintocaml[firstline=9,lastline=13,highlightlines=12]{../src/backtrace/backtrace.ml}
      \endminipage
    \end{column}
    \begin{column}{0.5\textwidth}
      \raggedleft
      \onslide*<1>{\listStashBacktrace[highlightlines={114-115}]{}}
      \onslide*<2>{\providecommand\step{3}\input{diagrams/backtrace/backtrace.tex}}%
      \onslide*<3>{\providecommand\step{4}\input{diagrams/backtrace/backtrace.tex}}%
    \end{column}
  \end{columns}
\end{frame}

\begin{frame}{Backtraces}{Back to \funcname{caml\_raise\_exn}}
  \begin{columns}[c]
    \begin{column}{0.5\textwidth}
      \minipage[c][0.66\textheight][s]{\columnwidth}
      \begin{itemize}\color{gray}
        \item Checks wheteher exceptions backtrace should be recorded, by checking the \funcarg{domain\_state->backtrace\_active} value
        \item Switches to the C stack to be able to execute C code
        \item Calls \funcname{caml\_stash\_backtrace}, to collect the backtrace up to the next trap
      \end{itemize}
      \onslide*<2->{
        \begin{itemize}
          \item<2-> Executes the exception handler in \funcname{probably\_raise} with \funcname{RESTORE\_EXN\_HANDLER\_OCAML}
            \begin{itemize}
              \item Set \regname{RSP} to \localname{domain\_state->exn\_handler} value
              \item Restore \localname{domain\_state->exn\_handler} from trap
              \item Jump to the handler code with \funcname{ret}
            \end{itemize}
        \end{itemize}
      }
      \vfill
      \onslide*<1>{\mintocaml[firstline=9,lastline=13,highlightlines=12]{../src/backtrace/backtrace.ml}}
      \onslide*<2->{\mintocaml[firstline=15,lastline=21,highlightlines={19-21}]{../src/backtrace/backtrace.ml}}
      \endminipage
    \end{column}
    \begin{column}{0.5\textwidth}
      \centering
      \onslide*<1>{
        \mintc[firstline=190,lastline=192]{../ocaml/runtime/amd64.S}
        \mintc[firstline=234,lastline=237]{../ocaml/runtime/amd64.S}
      }
      \onslide*<2>{
        \mintc[firstline=190,lastline=192,highlightlines={191-192}]{../ocaml/runtime/amd64.S}
        \mintc[firstline=234,lastline=237,highlightlines={235-237}]{../ocaml/runtime/amd64.S}
      }
      \onslide*<1>{\listCamlRaiseExn[highlightlines=720]{}}
      \onslide*<2>{\listCamlRaiseExn[highlightlines=722]{}}
      \onslide*<3->{\providecommand\step{5}\input{diagrams/backtrace/backtrace.tex}}
    \end{column}
  \end{columns}
\end{frame}

\begin{frame}{Backtraces}{\funcname{caml\_probably\_raise}'s handler doesn't catch \typename{ExnC}}
  \begin{columns}[c]
    \begin{column}{0.45\textwidth}
      \minipage[c][0.75\textheight][s]{\columnwidth}
      \begin{itemize}
        \item<1-> \funcname{match \dots with}\footnotemark pattern matching can also match exceptions
        \item<1-> The CMM of \funcname{probably\_raise} shows that this \funcname{match \dots with} is implemented with a pattern matching and a \funcname{try \dots with}
        \item<1-> But this one only matches on \typename{ExnA} exception
        \item<2-> Since the pattern matching fails to catch the exception, \funcname{reraise} is called with our current \typename{ExnC} exception
        \item<3-> Disassembly shows that \funcname{reraise} is actually a call to \funcname{caml\_reraise\_exn}, which is also an assembly function provided by the runtime
      \end{itemize}
      \vfill
      \mintocaml[firstline=15,lastline=21,highlightlines={18-21}]{../src/backtrace/backtrace.ml}
      \endminipage
    \end{column}
    \begin{column}{0.55\textwidth}
      \centering
      \onslide*<1>{\mintcmm[highlightlines={10-18}]{../src/backtrace/camlBacktrace\_\_probably\_raise\_351.cmm}}
      \onslide*<2>{\listcmm[highlightlines=18]{../src/backtrace/camlBacktrace\_\_probably\_raise_351.cmm}{CMM of probably\_raise}}
      \onslide*<3>{\mintobjdump[firstline=5,lastline=35,highlightlines=31]{../src/backtrace/camlBacktrace\_\_probably\_raise_351.objdump}}
    \end{column}
  \end{columns}
  \only<1>{\footnotetext[1]{https://blog.janestreet.com/pattern-matching-and-exception-handling-unite/}}
\end{frame}

\newcommand\listCamlReraiseExn[1][]{
  \listasm[firstline=727,lastline=734,#1]{../ocaml/runtime/amd64.S}{runtime/amd64.S:727}}

\begin{frame}{Backtraces}{Introducing \funcname{caml\_reraise\_exn}}
  \begin{columns}[c]
    \begin{column}{0.5\textwidth}
      \minipage[c][0.66\textheight][s]{\columnwidth}
      \begin{itemize}
        \item<1-> The exception have to be reraised to the next handler
        \item<2-> First, checks if the backtrace should be collected with \funcarg{domain\_state->backtrace\_active}
        \only<3>{
        \begin{itemize}
            \item If not, behaves like a \funcname{raise\_notrace} with \funcname{RESTORE\_EXN\_HANDLER\_OCAML}
        \end{itemize}
        }
        \item<4-> Jumps into \funcname{caml\_raise\_exn}
        \item<5-> Calls \funcname{caml\_stash\_backtrace} again, to scan the next part of the stack: from the trap just pop-ed to the next trap
      \end{itemize}
      \vfill
      \mintocaml[firstline=15,lastline=21,highlightlines={18-21}]{../src/backtrace/backtrace.ml}
      \endminipage
    \end{column}
    \begin{column}{0.5\textwidth}
      \raggedleft
      \onslide*<1>{\listCamlReraiseExn{}}
      \onslide*<2>{\listCamlReraiseExn[highlightlines=729]{}}
      \onslide*<3>{
        \mintc[firstline=190,lastline=192,highlightlines={191-192}]{../ocaml/runtime/amd64.S}
        \mintc[firstline=234,lastline=237,highlightlines={235-237}]{../ocaml/runtime/amd64.S}
        \medskip
        \listCamlReraiseExn[highlightlines=731]{}
      }
      \onslide*<4-5>{
        % TODO refactor with \listCamlReraiseExn
        \mintasm[firstline=727,lastline=734,highlightlines=730]{../ocaml/runtime/amd64.S}
        \medskip
      }
      \onslide*<4>{\listCamlRaiseExn[highlightlines=711]{}}
      \onslide*<5>{\listCamlRaiseExn[highlightlines=720]{}}
    \end{column}
  \end{columns}
\end{frame}

\begin{frame}{Backtraces}{Backtrace collection continues}
  \begin{columns}[c]
    \begin{column}{0.55\textwidth}
      \minipage[c][0.75\textheight][s]{\columnwidth}
      \begin{itemize}
        \item<1-> \funcname{caml\_stash\_backtrace} scans the older part of the stack, up to the next trap
        \item<6-> Controls jumps to the nested exception handler in \funcname{maybe\_raise}, which matches only on \typename{ExnB}
        \item<7-> \funcname{caml\_reraise\_exn} is called again, backtrace collection resumes with \funcname{caml\_stash\_backtrace}
      \end{itemize}
      \medskip
      \begin{itemize}
        \item<2-> Constructed backtrace:
        \begin{itemize}
          \item <2-> \funcname{definitely\_raise}
          \item <2-> \funcname{probably\_raise}
          \item <3-> \funcname{probably\_raise} (re-raise)
          \item <4-> \funcname{maybe\_raise}
          \item <5-> \funcname{main}
          \item <8-> \funcname{main} (re-raise)
        \end{itemize}
      \end{itemize}
      \vfill
      \onslide*<1-5>{\mintocaml[firstline=15,lastline=21]{../src/backtrace/backtrace.ml}}
      \onslide*<6>{\mintocaml[firstline=28,lastline=34,highlightlines=31]{../src/backtrace/backtrace.ml}}
      \onslide*<7->{\mintocaml[firstline=28,lastline=34,highlightlines=33]{../src/backtrace/backtrace.ml}}
      \endminipage
    \end{column}
    \begin{column}{0.45\textwidth}
      \raggedleft
      \onslide*<1>{\providecommand\step{6}\input{diagrams/backtrace/backtrace.tex}}%
      \onslide*<2>{\providecommand\step{6}\input{diagrams/backtrace/backtrace.tex}}%
      \onslide*<3>{\providecommand\step{7}\input{diagrams/backtrace/backtrace.tex}}%
      \onslide*<4>{\providecommand\step{8}\input{diagrams/backtrace/backtrace.tex}}%
      \onslide*<5>{\providecommand\step{9}\input{diagrams/backtrace/backtrace.tex}}%
      \onslide*<6>{\providecommand\step{10}\input{diagrams/backtrace/backtrace.tex}}%
      \onslide*<7>{\providecommand\step{11}\input{diagrams/backtrace/backtrace.tex}}%
      \onslide*<8>{\providecommand\step{12}\input{diagrams/backtrace/backtrace.tex}}%
      \onslide*<9>{\providecommand\step{13}\input{diagrams/backtrace/backtrace.tex}}%
    \end{column}
  \end{columns}
\end{frame}

\begin{frame}{Backtraces}{Printing the backtrace}
  \begin{columns}[c]
    \begin{column}{0.45\textwidth}
      \minipage[c][0.66\textheight][s]{\columnwidth}
      \begin{itemize}
        \item<1-> The exception backtrace can now be printed to \funcarg{stdout} with \funcname{Printexc.print_backtrace}
        \item<2-> First the exception backtrace is retrieved into an OCaml array with \funcname{get_raw_backtrace }
        \item<3-> Which is implemented as the \funcname{caml_get_exception_raw_backtrace} C function in the runtime
        \item<4-> And finally printed with \funcname{print_exception_backtrace}
      \end{itemize}
      \vfill
      \mintocaml[firstline=28,lastline=34,highlightlines=33]{../src/backtrace/backtrace.ml}
      \endminipage
    \end{column}
    \begin{column}{0.55\textwidth}
      \onslide*<1,2>{
        \mintocaml[firstline=100,lastline=101]{../ocaml/stdlib/printexc.ml}
      }
      \only<1>{
        \listocaml[firstline=170,lastline=172]{../ocaml/stdlib/printexc.ml}{stdlib/printexc.ml:170}
      }
      \only<2>{
        \listocaml[firstline=170,lastline=172,highlightlines=172]{../ocaml/stdlib/printexc.ml}{stdlib/printexc.ml:170}
      }
      \only<3>{
        \mintc[firstline=154,lastline=156]{../ocaml/runtime/backtrace.c}
        \listc[firstline=171,lastline=192,highlightlines={184-187}]{../ocaml/runtime/backtrace.c}{runtime/backtrace.c:154}
      }
      \only<4>{
        \listocaml[firstline=155,lastline=165]{../ocaml/stdlib/printexc.ml}{stdlib/printexc.ml:55}
      }
    \end{column}
  \end{columns}
\end{frame}


\subsection{The multiple kinds of raise}
\frameSubsection{}{}

\begin{frame}{Backtraces}{When backtraces are not needed}
  \begin{columns}[c]
    \begin{column}{0.45\textwidth}
      \minipage[c][0.66\textheight][s]{\columnwidth}
      \begin{itemize}
        \item<1-> What if the backtrace for an exception is never needed?
        \item<2-> It's possible to raise the exception explicitly with \funcname{raise\_notrace} to make raising as cheap as possible
        \item<3-> An example of use in the \funcname{dynlink} library of the OCaml compiler
      \end{itemize}
      \vfill
      \onslide<3->{
      \listocaml[firstline=205,lastline=209,highlightlines=207]{../ocaml/otherlibs/dynlink/dynlink\_compilerlibs/misc.ml}{otherlibs/dynlink/dynlink\_compilerlibs/misc.ml:205}
      }
      \endminipage
    \end{column}
    \begin{column}{0.55\textwidth}
    \only<4>{
      \mintcmm[highlightlines=3]{../src/notrace/camlNotrace\_\_fun_347.cmm}
      \listcmm{../src/notrace/camlNotrace\_\_all\_somes\_327.cmm}{all\_somes}
    }
    \onslide*<5->{
      \listobjdump[highlightlines={6-9}]{../src/notrace/camlNotrace\_\_fun_347.objdump}{anonymous function from \funcname{all\_somes}}
    }
    \end{column}
  \end{columns}
\end{frame}

\begin{frame}{Backtraces}{Summary table of the multiple kinds of raise}
  \begin{itemize}
    \item When debugging information is enabled and if the backtrace of an exception isn't needed, it's possible to raise with \funcname{raise\_notrace} for performance
    \item When debugging information is \emph{not} enabled, the compiler optimises \funcname{raise} calls to inline assembly (same effect as using \funcname{raise\_notrace})
  \end{itemize}
  \bigskip
  \centering
  \begin{tabular}{| l | c | c | c | c |}
    \hline
    \parbox{10em}{Debug information \\ (\funcarg{-g} flag)}  & \multicolumn{2}{|c|}{Disabled}                                     & \multicolumn{2}{|c|}{Enabled} \\
    \hline
    Raise kind                                               & \funcname{raise} & \funcname{raise\_notrace}                       & \funcname{raise} & \funcname{raise\_notrace} \\
    \hline
    Compilation                                              & \parbox{8em}{inline assembly\\ (optimization)} & inline assembly   & \funcname{caml\_raise\_exn} & inline assembly \\
    \hline
  \end{tabular}
\end{frame}


%
% Takeaway #5
%
\subsection*{Takeaway \#5}
\frameSubsectionTakeaway{}
\againframe<5>{takeaway}

    \end{column}
  \end{columns}
\end{frame}

\begin{frame}{Backtraces}{But before that, we need more fields from \typename{caml\_domain\_state}}
  \begin{columns}
    \begin{column}{0.5\textwidth}
      \begin{itemize}
        \item Remember \typename{caml\_domain\_state} the per-domain runtime state?
        \item Some other members are of interest to us now\dots
        \item \localname{backtrace\_active} is used to know if the runtime should collect backtraces
        \item \localname{backtrace\_active} can be set by
          \begin{itemize}
            \item The \funcarg{b} flag in the \funcname{OCAMLRUNPARAM} environment variable
            \item \mintinline{ocaml}{Printexc.record_backtrace : bool -> unit} \footnotemark
          \end{itemize}
      \end{itemize}
    \end{column}
    \begin{column}{0.5\textwidth}
      \begin{listing}
        \mintc[highlightlines={9-11}]{src/ocaml/domain\_state\_backtrace.h}
        \caption{runtime/caml/domain\_state.h}
      \end{listing}
    \end{column}
  \end{columns}
  \footnotetext[1]{https://v2.ocaml.org/api/Printexc.html}
\end{frame}

\newcommand\listCamlRaiseExn[1][]{
  \listasm[firstline=702,lastline=725,#1]{../ocaml/runtime/amd64.S}{runtime/amd64.S:702}}

\begin{frame}{Backtraces}{Walk through \funcname{caml\_raise\_exn}}
  \begin{columns}[T]
    \begin{column}{0.5\textwidth}
      \begin{itemize}
        \item<1-> First checks whether exceptions backtrace should be recorded
        \item<1-> By checking the \funcarg{domain\_state->backtrace\_active} value
        \item<2-> If not, acts as \funcname{raise\_notrace} with \funcname{RESTORE\_EXN\_HANDLER\_OCAML}
          \begin{itemize}
            \item Set \regname{RSP} to \localname{domain\_state->exn\_handler} value
            \item Restore \localname{domain\_state->exn\_handler} from trap
            \item Jump with \funcname{ret} (same effect as using \funcname{pop} and \funcname{jmp})
          \end{itemize}
        \item<3-> Otherwise, switches to the C stack to be able to execute C code
        \item<4-> Calls \funcname{caml\_stash\_backtrace}, a C function provided by the runtime\ignorespaces
          \onslide<6->{, with
            \begin{itemize}
              \item \funcarg{exn}: exception value
              \item \funcarg{pc}: code address of the raising point
              \item \funcarg{sp}: stack address of the raising function
              \item \funcarg{trapsp}: stack address of the next handler
            \end{itemize}
          }
      \end{itemize}
      \bigskip
      \mintocaml[firstline=9,lastline=13,highlightlines=12]{../src/backtrace/backtrace.ml}
    \end{column}
    \begin{column}{0.5\textwidth}
      \raggedleft
      \onslide*<1,3-4>{
        \mintc[firstline=190,lastline=192]{../ocaml/runtime/amd64.S}
        \mintc[firstline=234,lastline=237]{../ocaml/runtime/amd64.S}
      }
      \onslide*<2>{
        \mintc[firstline=190,lastline=192,highlightlines={191-192}]{../ocaml/runtime/amd64.S}
        \mintc[firstline=234,lastline=237,highlightlines={235-237}]{../ocaml/runtime/amd64.S}
      }
      \onslide*<1>{\listCamlRaiseExn[highlightlines=705]{}}%
      \onslide*<2>{\listCamlRaiseExn[highlightlines={707-708}]{}}%
      \onslide*<3>{\listCamlRaiseExn[highlightlines={712-714}]{}}%
      \onslide*<4>{\listCamlRaiseExn[highlightlines={715-720}]{}}%
      \onslide*<5>{\providecommand\step{1}%
% Backtraces
%
\subsection{One advantage of raising with debugging information}
\frameSubsection{}{}

\begin{frame}{Backtraces}{Debugging information \& source code}
  \begin{columns}[c]
    \begin{column}{0.5\textwidth}
      \begin{itemize}
        \item Raising an exception is fun and games, but how do we know where the exception originated? $\rightarrow$ With the backtrace associated with the exception!
        \item In order to get a backtrace from the point where an exception was raised, it's necessary to compile with debugging information enabled (\funcarg{-g} flag for the compiler)
        \item Same example as in the `Nested handlers' section
      \end{itemize}
    \end{column}
    \begin{column}{0.5\textwidth}
      \listocaml[firstline=9,lastline=34]{../src/backtrace/backtrace.ml}{backtrace/backtrace.ml}
    \end{column}
  \end{columns}
\end{frame}

\begin{frame}{Backtraces}{CMM and disassembly of \funcname{definitely\_raise}}
  \begin{columns}[c]
    \begin{column}{0.5\textwidth}
      \minipage[c][0.66\textheight][s]{\columnwidth}
      \begin{itemize}
        \item<1-> An effect of compiling with debugging information (\funcarg{-g}): the CMM no longer uses \funcname{raise\_notrace} to raise the exception but \funcname{raise} instead
        \item<2-> Disassembly shows that CMM \funcname{raise} is actually a call to \funcname{caml\_raise\_exn}, a function in assembler provided by the runtime
      \end{itemize}
      \vfill
      \mintocaml[firstline=9,lastline=13,highlightlines=12]{../src/backtrace/backtrace.ml}
      \endminipage
    \end{column}
    \begin{column}{0.5\textwidth}
      \onslide*<1>{
        \mintcmm[highlightlines=5]{../src/backtrace/camlBacktrace\_\_definitely\_raise\_347.cmm}
      }
      \onslide*<2>{
        \mintobjdump[firstline=2,highlightlines=14]{../src/backtrace/camlBacktrace\_\_definitely\_raise\_347.objdump}
      }
    \end{column}
  \end{columns}
\end{frame}

\begin{frame}{Backtraces}{Introducing \funcname{caml\_raise\_exn}}
  \begin{columns}[c]
    \begin{column}{0.5\textwidth}
      \minipage[c][0.66\textheight][s]{\columnwidth}
      \onslide*<1>{
        \begin{itemize}
          \item Raising the exception is performed using \funcname{caml\_raise\_exn} which is
            \begin{itemize}
              \item An assembly function provided by the runtime
              \item Architecture specific
            \end{itemize}
          \item Let's see what it does differently from \funcname{raise\_notrace}\dots
          \item We are about to raise \typename{ExnC} with \funcname{caml\_raise\_exn} from \funcname{definitely\_raise}
        \end{itemize}
      }
      \onslide*<2>{
        \mintobjdump[firstline=2,highlightlines=14]{../src/backtrace/camlBacktrace\_\_definitely\_raise\_347.objdump}
      }
      \vfill
      \mintocaml[firstline=9,lastline=13,highlightlines=12]{../src/backtrace/backtrace.ml}
      \endminipage
    \end{column}
    \begin{column}{0.5\textwidth}
      \centering
      \providecommand\step{1}\input{diagrams/backtrace/backtrace.tex}
    \end{column}
  \end{columns}
\end{frame}

\begin{frame}{Backtraces}{But before that, we need more fields from \typename{caml\_domain\_state}}
  \begin{columns}
    \begin{column}{0.5\textwidth}
      \begin{itemize}
        \item Remember \typename{caml\_domain\_state} the per-domain runtime state?
        \item Some other members are of interest to us now\dots
        \item \localname{backtrace\_active} is used to know if the runtime should collect backtraces
        \item \localname{backtrace\_active} can be set by
          \begin{itemize}
            \item The \funcarg{b} flag in the \funcname{OCAMLRUNPARAM} environment variable
            \item \mintinline{ocaml}{Printexc.record_backtrace : bool -> unit} \footnotemark
          \end{itemize}
      \end{itemize}
    \end{column}
    \begin{column}{0.5\textwidth}
      \begin{listing}
        \mintc[highlightlines={9-11}]{src/ocaml/domain\_state\_backtrace.h}
        \caption{runtime/caml/domain\_state.h}
      \end{listing}
    \end{column}
  \end{columns}
  \footnotetext[1]{https://v2.ocaml.org/api/Printexc.html}
\end{frame}

\newcommand\listCamlRaiseExn[1][]{
  \listasm[firstline=702,lastline=725,#1]{../ocaml/runtime/amd64.S}{runtime/amd64.S:702}}

\begin{frame}{Backtraces}{Walk through \funcname{caml\_raise\_exn}}
  \begin{columns}[T]
    \begin{column}{0.5\textwidth}
      \begin{itemize}
        \item<1-> First checks whether exceptions backtrace should be recorded
        \item<1-> By checking the \funcarg{domain\_state->backtrace\_active} value
        \item<2-> If not, acts as \funcname{raise\_notrace} with \funcname{RESTORE\_EXN\_HANDLER\_OCAML}
          \begin{itemize}
            \item Set \regname{RSP} to \localname{domain\_state->exn\_handler} value
            \item Restore \localname{domain\_state->exn\_handler} from trap
            \item Jump with \funcname{ret} (same effect as using \funcname{pop} and \funcname{jmp})
          \end{itemize}
        \item<3-> Otherwise, switches to the C stack to be able to execute C code
        \item<4-> Calls \funcname{caml\_stash\_backtrace}, a C function provided by the runtime\ignorespaces
          \onslide<6->{, with
            \begin{itemize}
              \item \funcarg{exn}: exception value
              \item \funcarg{pc}: code address of the raising point
              \item \funcarg{sp}: stack address of the raising function
              \item \funcarg{trapsp}: stack address of the next handler
            \end{itemize}
          }
      \end{itemize}
      \bigskip
      \mintocaml[firstline=9,lastline=13,highlightlines=12]{../src/backtrace/backtrace.ml}
    \end{column}
    \begin{column}{0.5\textwidth}
      \raggedleft
      \onslide*<1,3-4>{
        \mintc[firstline=190,lastline=192]{../ocaml/runtime/amd64.S}
        \mintc[firstline=234,lastline=237]{../ocaml/runtime/amd64.S}
      }
      \onslide*<2>{
        \mintc[firstline=190,lastline=192,highlightlines={191-192}]{../ocaml/runtime/amd64.S}
        \mintc[firstline=234,lastline=237,highlightlines={235-237}]{../ocaml/runtime/amd64.S}
      }
      \onslide*<1>{\listCamlRaiseExn[highlightlines=705]{}}%
      \onslide*<2>{\listCamlRaiseExn[highlightlines={707-708}]{}}%
      \onslide*<3>{\listCamlRaiseExn[highlightlines={712-714}]{}}%
      \onslide*<4>{\listCamlRaiseExn[highlightlines={715-720}]{}}%
      \onslide*<5>{\providecommand\step{1}\input{diagrams/backtrace/backtrace.tex}}%
      \onslide*<6>{\providecommand\step{2}\input{diagrams/backtrace/backtrace.tex}}%
    \end{column}
  \end{columns}
\end{frame}

\newcommand\listStashBacktrace[1][]{
  \listc[firstline=91,lastline=120,#1]{../ocaml/runtime/backtrace\_nat.c}{runtime/backtrace\_nat.c:91}}

\begin{frame}{Backtraces}{Introducing \funcname{caml\_stash\_backtrace}}
  \begin{columns}[c]
    \begin{column}{0.5\textwidth}
      \minipage[c][0.66\textheight][s]{\columnwidth}
      \begin{itemize}
        \item<1-> A C function provided by the runtime (specific to native code)
        \item<2-> It scans the stack, between the stack pointer at the raising point up to the next trap, in order to collect the names of the detected function frames
          \onslide*<2>{
            \begin{itemize}
              \item And more information, like line number, column number...
            \end{itemize}
          }
        \item<3-> It uses the same technique and information as the Garbage Collector (GC) uses to scan the values live on the stack
          \onslide*<4>{
            \begin{itemize}
              \item \typename{frame\_descr} are at the core of stack unwinding
            \end{itemize}
          }
      \end{itemize}
      \vfill
      \mintocaml[firstline=9,lastline=13,highlightlines=12]{../src/backtrace/backtrace.ml}
      \endminipage
    \end{column}
    \begin{column}{0.5\textwidth}
      \onslide*<1>{\listStashBacktrace[highlightlines=91]{}}
      \onslide*<2>{\listStashBacktrace[highlightlines={91,117-118}]{}}
      \onslide*<3->{\listStashBacktrace[highlightlines={94,106,109-110}]{}}
    \end{column}
  \end{columns}
\end{frame}

\newcommand\listFrameDescr[1][]{
  \listocaml[firstline=111,lastline=124,#1]{../ocaml/asmcomp/emitaux.ml}{asmcomp/emitaux.ml:111}}
\newcommand\listDebugInfo[1][]{
  \listocaml[firstline=38,lastline=49,#1]{../ocaml/lambda/debuginfo.mli}{lambda/debuginfo.mli:38}}

\begin{frame}{Backtraces}{\typename{frame\_descr} and \typename{frame\_debuginfo} in a nutshell}
  \begin{columns}[c]
    \begin{column}{0.5\textwidth}
      \begin{itemize}
        \item<1-> For every return address in an OCaml program, there is a associated \typename{frame\_descr}
        \item<2-> A \typename{frame\_descr} specifies
        \begin{itemize}
          \item<2-> The size of the function stack frame
          \item<3-> Where live values are on the stack (so that the GC can mark them as live and prevent from being swept)
        \end{itemize}
        \item<4-> When compiled with debug information, \typename{frame\_descr} will be linked to a \typename{frame\_debuginfo}
        \begin{itemize}
          \item<5-> Contains information about the position in the source code (file name, line and column number)
        \end{itemize}
      \end{itemize}
    \end{column}
    \begin{column}{0.5\textwidth}
        \onslide*<1>{\listFrameDescr[highlightlines={118-122}]{}}
        \onslide*<2>{\listFrameDescr[highlightlines={120}]{}}
        \onslide*<3>{\listFrameDescr[highlightlines={121}]{}}
        \onslide*<4->{\listFrameDescr[highlightlines={122}]{}}
        \medskip
        \onslide*<1-3>{\listDebugInfo{}}
        \onslide*<4>{\listDebugInfo[highlightlines={38,49}]{}}
        \onslide*<5>{\listDebugInfo[highlightlines={39-46}]{}}
    \end{column}
  \end{columns}
\end{frame}

\begin{frame}{Backtraces}{Scanning the stack with \typename{frame\_descr}}
  \begin{columns}[c]
    \begin{column}{0.5\textwidth}
      \begin{itemize}
        \item Given the return address of the code that raises the exception by calling \funcname{caml\_raise\_exn} (\funcarg{pc} argument of \funcname{caml\_stash\_backtrace})
        \item And the position of the stack pointer (\funcarg{sp} argument of \funcname{caml\_stash\_backtrace})
        \item The frame size given by \typename{frame\_descr}.\funcarg{frame\_size}, allows to find the end of the previous frame
        \item And the return address of the caller of this function
        \bigskip
        \onslide<3->{
        \item Note that traps (here the reddish \funcname{ExnA}, \funcname{ExnB} and \funcname{ExnC} blocks) belong to the frame that installed them, and so the frame size changes when traps are removed
        }
      \end{itemize}
    \end{column}
    \begin{column}{0.5\textwidth}
      \raggedleft
      \onslide*<1>{\providecommand\step{2}\input{diagrams/backtrace/backtrace.tex}}%
      \onslide*<2>{\providecommand\step{1}\input{diagrams/backtrace/scanstack.tex}}%
      \onslide*<3>{\providecommand\step{2}\input{diagrams/backtrace/scanstack.tex}}%
      \onslide*<4>{\providecommand\step{3}\input{diagrams/backtrace/scanstack.tex}}%
      \onslide*<5>{\providecommand\step{4}\input{diagrams/backtrace/scanstack.tex}}%
      \onslide*<6>{\providecommand\step{5}\input{diagrams/backtrace/scanstack.tex}}%
    \end{column}
  \end{columns}
\end{frame}

% Duplicate of previous-previous slide
\begin{frame}{Backtraces}{Continuing on \funcname{caml\_stash\_backtrace}}
  \begin{columns}[c]
    \begin{column}{0.5\textwidth}
      \minipage[c][0.75\textheight][s]{\columnwidth}
      \begin{itemize}
          \color{gray}
        \item A C function provided by the runtime (specific to native code)
        \item It scans the stack, between the stack pointer at the raising point up to the next trap, in order to collect the names of the detected function frames
        \item It uses the same technique and information as the Garbage Collector (GC) uses to scan the values live on the stack
      \end{itemize}
      \begin{itemize}
        \item For each function frame detected, store a pointer to the \typename{frame\_descr} in the \localname{domain\_state->backtrace\_buffer} array, effectively constructing the backtrace
      \end{itemize}
      \onslide<2->{
        \begin{itemize}
            \smallskip
          \item Constructed backtrace:
            \funcname{definitely\_raise},
            \onslide<3->{\funcname{probably\_raise}}
        \end{itemize}
      }
      \vfill
      \mintocaml[firstline=9,lastline=13,highlightlines=12]{../src/backtrace/backtrace.ml}
      \endminipage
    \end{column}
    \begin{column}{0.5\textwidth}
      \raggedleft
      \onslide*<1>{\listStashBacktrace[highlightlines={114-115}]{}}
      \onslide*<2>{\providecommand\step{3}\input{diagrams/backtrace/backtrace.tex}}%
      \onslide*<3>{\providecommand\step{4}\input{diagrams/backtrace/backtrace.tex}}%
    \end{column}
  \end{columns}
\end{frame}

\begin{frame}{Backtraces}{Back to \funcname{caml\_raise\_exn}}
  \begin{columns}[c]
    \begin{column}{0.5\textwidth}
      \minipage[c][0.66\textheight][s]{\columnwidth}
      \begin{itemize}\color{gray}
        \item Checks wheteher exceptions backtrace should be recorded, by checking the \funcarg{domain\_state->backtrace\_active} value
        \item Switches to the C stack to be able to execute C code
        \item Calls \funcname{caml\_stash\_backtrace}, to collect the backtrace up to the next trap
      \end{itemize}
      \onslide*<2->{
        \begin{itemize}
          \item<2-> Executes the exception handler in \funcname{probably\_raise} with \funcname{RESTORE\_EXN\_HANDLER\_OCAML}
            \begin{itemize}
              \item Set \regname{RSP} to \localname{domain\_state->exn\_handler} value
              \item Restore \localname{domain\_state->exn\_handler} from trap
              \item Jump to the handler code with \funcname{ret}
            \end{itemize}
        \end{itemize}
      }
      \vfill
      \onslide*<1>{\mintocaml[firstline=9,lastline=13,highlightlines=12]{../src/backtrace/backtrace.ml}}
      \onslide*<2->{\mintocaml[firstline=15,lastline=21,highlightlines={19-21}]{../src/backtrace/backtrace.ml}}
      \endminipage
    \end{column}
    \begin{column}{0.5\textwidth}
      \centering
      \onslide*<1>{
        \mintc[firstline=190,lastline=192]{../ocaml/runtime/amd64.S}
        \mintc[firstline=234,lastline=237]{../ocaml/runtime/amd64.S}
      }
      \onslide*<2>{
        \mintc[firstline=190,lastline=192,highlightlines={191-192}]{../ocaml/runtime/amd64.S}
        \mintc[firstline=234,lastline=237,highlightlines={235-237}]{../ocaml/runtime/amd64.S}
      }
      \onslide*<1>{\listCamlRaiseExn[highlightlines=720]{}}
      \onslide*<2>{\listCamlRaiseExn[highlightlines=722]{}}
      \onslide*<3->{\providecommand\step{5}\input{diagrams/backtrace/backtrace.tex}}
    \end{column}
  \end{columns}
\end{frame}

\begin{frame}{Backtraces}{\funcname{caml\_probably\_raise}'s handler doesn't catch \typename{ExnC}}
  \begin{columns}[c]
    \begin{column}{0.45\textwidth}
      \minipage[c][0.75\textheight][s]{\columnwidth}
      \begin{itemize}
        \item<1-> \funcname{match \dots with}\footnotemark pattern matching can also match exceptions
        \item<1-> The CMM of \funcname{probably\_raise} shows that this \funcname{match \dots with} is implemented with a pattern matching and a \funcname{try \dots with}
        \item<1-> But this one only matches on \typename{ExnA} exception
        \item<2-> Since the pattern matching fails to catch the exception, \funcname{reraise} is called with our current \typename{ExnC} exception
        \item<3-> Disassembly shows that \funcname{reraise} is actually a call to \funcname{caml\_reraise\_exn}, which is also an assembly function provided by the runtime
      \end{itemize}
      \vfill
      \mintocaml[firstline=15,lastline=21,highlightlines={18-21}]{../src/backtrace/backtrace.ml}
      \endminipage
    \end{column}
    \begin{column}{0.55\textwidth}
      \centering
      \onslide*<1>{\mintcmm[highlightlines={10-18}]{../src/backtrace/camlBacktrace\_\_probably\_raise\_351.cmm}}
      \onslide*<2>{\listcmm[highlightlines=18]{../src/backtrace/camlBacktrace\_\_probably\_raise_351.cmm}{CMM of probably\_raise}}
      \onslide*<3>{\mintobjdump[firstline=5,lastline=35,highlightlines=31]{../src/backtrace/camlBacktrace\_\_probably\_raise_351.objdump}}
    \end{column}
  \end{columns}
  \only<1>{\footnotetext[1]{https://blog.janestreet.com/pattern-matching-and-exception-handling-unite/}}
\end{frame}

\newcommand\listCamlReraiseExn[1][]{
  \listasm[firstline=727,lastline=734,#1]{../ocaml/runtime/amd64.S}{runtime/amd64.S:727}}

\begin{frame}{Backtraces}{Introducing \funcname{caml\_reraise\_exn}}
  \begin{columns}[c]
    \begin{column}{0.5\textwidth}
      \minipage[c][0.66\textheight][s]{\columnwidth}
      \begin{itemize}
        \item<1-> The exception have to be reraised to the next handler
        \item<2-> First, checks if the backtrace should be collected with \funcarg{domain\_state->backtrace\_active}
        \only<3>{
        \begin{itemize}
            \item If not, behaves like a \funcname{raise\_notrace} with \funcname{RESTORE\_EXN\_HANDLER\_OCAML}
        \end{itemize}
        }
        \item<4-> Jumps into \funcname{caml\_raise\_exn}
        \item<5-> Calls \funcname{caml\_stash\_backtrace} again, to scan the next part of the stack: from the trap just pop-ed to the next trap
      \end{itemize}
      \vfill
      \mintocaml[firstline=15,lastline=21,highlightlines={18-21}]{../src/backtrace/backtrace.ml}
      \endminipage
    \end{column}
    \begin{column}{0.5\textwidth}
      \raggedleft
      \onslide*<1>{\listCamlReraiseExn{}}
      \onslide*<2>{\listCamlReraiseExn[highlightlines=729]{}}
      \onslide*<3>{
        \mintc[firstline=190,lastline=192,highlightlines={191-192}]{../ocaml/runtime/amd64.S}
        \mintc[firstline=234,lastline=237,highlightlines={235-237}]{../ocaml/runtime/amd64.S}
        \medskip
        \listCamlReraiseExn[highlightlines=731]{}
      }
      \onslide*<4-5>{
        % TODO refactor with \listCamlReraiseExn
        \mintasm[firstline=727,lastline=734,highlightlines=730]{../ocaml/runtime/amd64.S}
        \medskip
      }
      \onslide*<4>{\listCamlRaiseExn[highlightlines=711]{}}
      \onslide*<5>{\listCamlRaiseExn[highlightlines=720]{}}
    \end{column}
  \end{columns}
\end{frame}

\begin{frame}{Backtraces}{Backtrace collection continues}
  \begin{columns}[c]
    \begin{column}{0.55\textwidth}
      \minipage[c][0.75\textheight][s]{\columnwidth}
      \begin{itemize}
        \item<1-> \funcname{caml\_stash\_backtrace} scans the older part of the stack, up to the next trap
        \item<6-> Controls jumps to the nested exception handler in \funcname{maybe\_raise}, which matches only on \typename{ExnB}
        \item<7-> \funcname{caml\_reraise\_exn} is called again, backtrace collection resumes with \funcname{caml\_stash\_backtrace}
      \end{itemize}
      \medskip
      \begin{itemize}
        \item<2-> Constructed backtrace:
        \begin{itemize}
          \item <2-> \funcname{definitely\_raise}
          \item <2-> \funcname{probably\_raise}
          \item <3-> \funcname{probably\_raise} (re-raise)
          \item <4-> \funcname{maybe\_raise}
          \item <5-> \funcname{main}
          \item <8-> \funcname{main} (re-raise)
        \end{itemize}
      \end{itemize}
      \vfill
      \onslide*<1-5>{\mintocaml[firstline=15,lastline=21]{../src/backtrace/backtrace.ml}}
      \onslide*<6>{\mintocaml[firstline=28,lastline=34,highlightlines=31]{../src/backtrace/backtrace.ml}}
      \onslide*<7->{\mintocaml[firstline=28,lastline=34,highlightlines=33]{../src/backtrace/backtrace.ml}}
      \endminipage
    \end{column}
    \begin{column}{0.45\textwidth}
      \raggedleft
      \onslide*<1>{\providecommand\step{6}\input{diagrams/backtrace/backtrace.tex}}%
      \onslide*<2>{\providecommand\step{6}\input{diagrams/backtrace/backtrace.tex}}%
      \onslide*<3>{\providecommand\step{7}\input{diagrams/backtrace/backtrace.tex}}%
      \onslide*<4>{\providecommand\step{8}\input{diagrams/backtrace/backtrace.tex}}%
      \onslide*<5>{\providecommand\step{9}\input{diagrams/backtrace/backtrace.tex}}%
      \onslide*<6>{\providecommand\step{10}\input{diagrams/backtrace/backtrace.tex}}%
      \onslide*<7>{\providecommand\step{11}\input{diagrams/backtrace/backtrace.tex}}%
      \onslide*<8>{\providecommand\step{12}\input{diagrams/backtrace/backtrace.tex}}%
      \onslide*<9>{\providecommand\step{13}\input{diagrams/backtrace/backtrace.tex}}%
    \end{column}
  \end{columns}
\end{frame}

\begin{frame}{Backtraces}{Printing the backtrace}
  \begin{columns}[c]
    \begin{column}{0.45\textwidth}
      \minipage[c][0.66\textheight][s]{\columnwidth}
      \begin{itemize}
        \item<1-> The exception backtrace can now be printed to \funcarg{stdout} with \funcname{Printexc.print_backtrace}
        \item<2-> First the exception backtrace is retrieved into an OCaml array with \funcname{get_raw_backtrace }
        \item<3-> Which is implemented as the \funcname{caml_get_exception_raw_backtrace} C function in the runtime
        \item<4-> And finally printed with \funcname{print_exception_backtrace}
      \end{itemize}
      \vfill
      \mintocaml[firstline=28,lastline=34,highlightlines=33]{../src/backtrace/backtrace.ml}
      \endminipage
    \end{column}
    \begin{column}{0.55\textwidth}
      \onslide*<1,2>{
        \mintocaml[firstline=100,lastline=101]{../ocaml/stdlib/printexc.ml}
      }
      \only<1>{
        \listocaml[firstline=170,lastline=172]{../ocaml/stdlib/printexc.ml}{stdlib/printexc.ml:170}
      }
      \only<2>{
        \listocaml[firstline=170,lastline=172,highlightlines=172]{../ocaml/stdlib/printexc.ml}{stdlib/printexc.ml:170}
      }
      \only<3>{
        \mintc[firstline=154,lastline=156]{../ocaml/runtime/backtrace.c}
        \listc[firstline=171,lastline=192,highlightlines={184-187}]{../ocaml/runtime/backtrace.c}{runtime/backtrace.c:154}
      }
      \only<4>{
        \listocaml[firstline=155,lastline=165]{../ocaml/stdlib/printexc.ml}{stdlib/printexc.ml:55}
      }
    \end{column}
  \end{columns}
\end{frame}


\subsection{The multiple kinds of raise}
\frameSubsection{}{}

\begin{frame}{Backtraces}{When backtraces are not needed}
  \begin{columns}[c]
    \begin{column}{0.45\textwidth}
      \minipage[c][0.66\textheight][s]{\columnwidth}
      \begin{itemize}
        \item<1-> What if the backtrace for an exception is never needed?
        \item<2-> It's possible to raise the exception explicitly with \funcname{raise\_notrace} to make raising as cheap as possible
        \item<3-> An example of use in the \funcname{dynlink} library of the OCaml compiler
      \end{itemize}
      \vfill
      \onslide<3->{
      \listocaml[firstline=205,lastline=209,highlightlines=207]{../ocaml/otherlibs/dynlink/dynlink\_compilerlibs/misc.ml}{otherlibs/dynlink/dynlink\_compilerlibs/misc.ml:205}
      }
      \endminipage
    \end{column}
    \begin{column}{0.55\textwidth}
    \only<4>{
      \mintcmm[highlightlines=3]{../src/notrace/camlNotrace\_\_fun_347.cmm}
      \listcmm{../src/notrace/camlNotrace\_\_all\_somes\_327.cmm}{all\_somes}
    }
    \onslide*<5->{
      \listobjdump[highlightlines={6-9}]{../src/notrace/camlNotrace\_\_fun_347.objdump}{anonymous function from \funcname{all\_somes}}
    }
    \end{column}
  \end{columns}
\end{frame}

\begin{frame}{Backtraces}{Summary table of the multiple kinds of raise}
  \begin{itemize}
    \item When debugging information is enabled and if the backtrace of an exception isn't needed, it's possible to raise with \funcname{raise\_notrace} for performance
    \item When debugging information is \emph{not} enabled, the compiler optimises \funcname{raise} calls to inline assembly (same effect as using \funcname{raise\_notrace})
  \end{itemize}
  \bigskip
  \centering
  \begin{tabular}{| l | c | c | c | c |}
    \hline
    \parbox{10em}{Debug information \\ (\funcarg{-g} flag)}  & \multicolumn{2}{|c|}{Disabled}                                     & \multicolumn{2}{|c|}{Enabled} \\
    \hline
    Raise kind                                               & \funcname{raise} & \funcname{raise\_notrace}                       & \funcname{raise} & \funcname{raise\_notrace} \\
    \hline
    Compilation                                              & \parbox{8em}{inline assembly\\ (optimization)} & inline assembly   & \funcname{caml\_raise\_exn} & inline assembly \\
    \hline
  \end{tabular}
\end{frame}


%
% Takeaway #5
%
\subsection*{Takeaway \#5}
\frameSubsectionTakeaway{}
\againframe<5>{takeaway}
}%
      \onslide*<6>{\providecommand\step{2}%
% Backtraces
%
\subsection{One advantage of raising with debugging information}
\frameSubsection{}{}

\begin{frame}{Backtraces}{Debugging information \& source code}
  \begin{columns}[c]
    \begin{column}{0.5\textwidth}
      \begin{itemize}
        \item Raising an exception is fun and games, but how do we know where the exception originated? $\rightarrow$ With the backtrace associated with the exception!
        \item In order to get a backtrace from the point where an exception was raised, it's necessary to compile with debugging information enabled (\funcarg{-g} flag for the compiler)
        \item Same example as in the `Nested handlers' section
      \end{itemize}
    \end{column}
    \begin{column}{0.5\textwidth}
      \listocaml[firstline=9,lastline=34]{../src/backtrace/backtrace.ml}{backtrace/backtrace.ml}
    \end{column}
  \end{columns}
\end{frame}

\begin{frame}{Backtraces}{CMM and disassembly of \funcname{definitely\_raise}}
  \begin{columns}[c]
    \begin{column}{0.5\textwidth}
      \minipage[c][0.66\textheight][s]{\columnwidth}
      \begin{itemize}
        \item<1-> An effect of compiling with debugging information (\funcarg{-g}): the CMM no longer uses \funcname{raise\_notrace} to raise the exception but \funcname{raise} instead
        \item<2-> Disassembly shows that CMM \funcname{raise} is actually a call to \funcname{caml\_raise\_exn}, a function in assembler provided by the runtime
      \end{itemize}
      \vfill
      \mintocaml[firstline=9,lastline=13,highlightlines=12]{../src/backtrace/backtrace.ml}
      \endminipage
    \end{column}
    \begin{column}{0.5\textwidth}
      \onslide*<1>{
        \mintcmm[highlightlines=5]{../src/backtrace/camlBacktrace\_\_definitely\_raise\_347.cmm}
      }
      \onslide*<2>{
        \mintobjdump[firstline=2,highlightlines=14]{../src/backtrace/camlBacktrace\_\_definitely\_raise\_347.objdump}
      }
    \end{column}
  \end{columns}
\end{frame}

\begin{frame}{Backtraces}{Introducing \funcname{caml\_raise\_exn}}
  \begin{columns}[c]
    \begin{column}{0.5\textwidth}
      \minipage[c][0.66\textheight][s]{\columnwidth}
      \onslide*<1>{
        \begin{itemize}
          \item Raising the exception is performed using \funcname{caml\_raise\_exn} which is
            \begin{itemize}
              \item An assembly function provided by the runtime
              \item Architecture specific
            \end{itemize}
          \item Let's see what it does differently from \funcname{raise\_notrace}\dots
          \item We are about to raise \typename{ExnC} with \funcname{caml\_raise\_exn} from \funcname{definitely\_raise}
        \end{itemize}
      }
      \onslide*<2>{
        \mintobjdump[firstline=2,highlightlines=14]{../src/backtrace/camlBacktrace\_\_definitely\_raise\_347.objdump}
      }
      \vfill
      \mintocaml[firstline=9,lastline=13,highlightlines=12]{../src/backtrace/backtrace.ml}
      \endminipage
    \end{column}
    \begin{column}{0.5\textwidth}
      \centering
      \providecommand\step{1}\input{diagrams/backtrace/backtrace.tex}
    \end{column}
  \end{columns}
\end{frame}

\begin{frame}{Backtraces}{But before that, we need more fields from \typename{caml\_domain\_state}}
  \begin{columns}
    \begin{column}{0.5\textwidth}
      \begin{itemize}
        \item Remember \typename{caml\_domain\_state} the per-domain runtime state?
        \item Some other members are of interest to us now\dots
        \item \localname{backtrace\_active} is used to know if the runtime should collect backtraces
        \item \localname{backtrace\_active} can be set by
          \begin{itemize}
            \item The \funcarg{b} flag in the \funcname{OCAMLRUNPARAM} environment variable
            \item \mintinline{ocaml}{Printexc.record_backtrace : bool -> unit} \footnotemark
          \end{itemize}
      \end{itemize}
    \end{column}
    \begin{column}{0.5\textwidth}
      \begin{listing}
        \mintc[highlightlines={9-11}]{src/ocaml/domain\_state\_backtrace.h}
        \caption{runtime/caml/domain\_state.h}
      \end{listing}
    \end{column}
  \end{columns}
  \footnotetext[1]{https://v2.ocaml.org/api/Printexc.html}
\end{frame}

\newcommand\listCamlRaiseExn[1][]{
  \listasm[firstline=702,lastline=725,#1]{../ocaml/runtime/amd64.S}{runtime/amd64.S:702}}

\begin{frame}{Backtraces}{Walk through \funcname{caml\_raise\_exn}}
  \begin{columns}[T]
    \begin{column}{0.5\textwidth}
      \begin{itemize}
        \item<1-> First checks whether exceptions backtrace should be recorded
        \item<1-> By checking the \funcarg{domain\_state->backtrace\_active} value
        \item<2-> If not, acts as \funcname{raise\_notrace} with \funcname{RESTORE\_EXN\_HANDLER\_OCAML}
          \begin{itemize}
            \item Set \regname{RSP} to \localname{domain\_state->exn\_handler} value
            \item Restore \localname{domain\_state->exn\_handler} from trap
            \item Jump with \funcname{ret} (same effect as using \funcname{pop} and \funcname{jmp})
          \end{itemize}
        \item<3-> Otherwise, switches to the C stack to be able to execute C code
        \item<4-> Calls \funcname{caml\_stash\_backtrace}, a C function provided by the runtime\ignorespaces
          \onslide<6->{, with
            \begin{itemize}
              \item \funcarg{exn}: exception value
              \item \funcarg{pc}: code address of the raising point
              \item \funcarg{sp}: stack address of the raising function
              \item \funcarg{trapsp}: stack address of the next handler
            \end{itemize}
          }
      \end{itemize}
      \bigskip
      \mintocaml[firstline=9,lastline=13,highlightlines=12]{../src/backtrace/backtrace.ml}
    \end{column}
    \begin{column}{0.5\textwidth}
      \raggedleft
      \onslide*<1,3-4>{
        \mintc[firstline=190,lastline=192]{../ocaml/runtime/amd64.S}
        \mintc[firstline=234,lastline=237]{../ocaml/runtime/amd64.S}
      }
      \onslide*<2>{
        \mintc[firstline=190,lastline=192,highlightlines={191-192}]{../ocaml/runtime/amd64.S}
        \mintc[firstline=234,lastline=237,highlightlines={235-237}]{../ocaml/runtime/amd64.S}
      }
      \onslide*<1>{\listCamlRaiseExn[highlightlines=705]{}}%
      \onslide*<2>{\listCamlRaiseExn[highlightlines={707-708}]{}}%
      \onslide*<3>{\listCamlRaiseExn[highlightlines={712-714}]{}}%
      \onslide*<4>{\listCamlRaiseExn[highlightlines={715-720}]{}}%
      \onslide*<5>{\providecommand\step{1}\input{diagrams/backtrace/backtrace.tex}}%
      \onslide*<6>{\providecommand\step{2}\input{diagrams/backtrace/backtrace.tex}}%
    \end{column}
  \end{columns}
\end{frame}

\newcommand\listStashBacktrace[1][]{
  \listc[firstline=91,lastline=120,#1]{../ocaml/runtime/backtrace\_nat.c}{runtime/backtrace\_nat.c:91}}

\begin{frame}{Backtraces}{Introducing \funcname{caml\_stash\_backtrace}}
  \begin{columns}[c]
    \begin{column}{0.5\textwidth}
      \minipage[c][0.66\textheight][s]{\columnwidth}
      \begin{itemize}
        \item<1-> A C function provided by the runtime (specific to native code)
        \item<2-> It scans the stack, between the stack pointer at the raising point up to the next trap, in order to collect the names of the detected function frames
          \onslide*<2>{
            \begin{itemize}
              \item And more information, like line number, column number...
            \end{itemize}
          }
        \item<3-> It uses the same technique and information as the Garbage Collector (GC) uses to scan the values live on the stack
          \onslide*<4>{
            \begin{itemize}
              \item \typename{frame\_descr} are at the core of stack unwinding
            \end{itemize}
          }
      \end{itemize}
      \vfill
      \mintocaml[firstline=9,lastline=13,highlightlines=12]{../src/backtrace/backtrace.ml}
      \endminipage
    \end{column}
    \begin{column}{0.5\textwidth}
      \onslide*<1>{\listStashBacktrace[highlightlines=91]{}}
      \onslide*<2>{\listStashBacktrace[highlightlines={91,117-118}]{}}
      \onslide*<3->{\listStashBacktrace[highlightlines={94,106,109-110}]{}}
    \end{column}
  \end{columns}
\end{frame}

\newcommand\listFrameDescr[1][]{
  \listocaml[firstline=111,lastline=124,#1]{../ocaml/asmcomp/emitaux.ml}{asmcomp/emitaux.ml:111}}
\newcommand\listDebugInfo[1][]{
  \listocaml[firstline=38,lastline=49,#1]{../ocaml/lambda/debuginfo.mli}{lambda/debuginfo.mli:38}}

\begin{frame}{Backtraces}{\typename{frame\_descr} and \typename{frame\_debuginfo} in a nutshell}
  \begin{columns}[c]
    \begin{column}{0.5\textwidth}
      \begin{itemize}
        \item<1-> For every return address in an OCaml program, there is a associated \typename{frame\_descr}
        \item<2-> A \typename{frame\_descr} specifies
        \begin{itemize}
          \item<2-> The size of the function stack frame
          \item<3-> Where live values are on the stack (so that the GC can mark them as live and prevent from being swept)
        \end{itemize}
        \item<4-> When compiled with debug information, \typename{frame\_descr} will be linked to a \typename{frame\_debuginfo}
        \begin{itemize}
          \item<5-> Contains information about the position in the source code (file name, line and column number)
        \end{itemize}
      \end{itemize}
    \end{column}
    \begin{column}{0.5\textwidth}
        \onslide*<1>{\listFrameDescr[highlightlines={118-122}]{}}
        \onslide*<2>{\listFrameDescr[highlightlines={120}]{}}
        \onslide*<3>{\listFrameDescr[highlightlines={121}]{}}
        \onslide*<4->{\listFrameDescr[highlightlines={122}]{}}
        \medskip
        \onslide*<1-3>{\listDebugInfo{}}
        \onslide*<4>{\listDebugInfo[highlightlines={38,49}]{}}
        \onslide*<5>{\listDebugInfo[highlightlines={39-46}]{}}
    \end{column}
  \end{columns}
\end{frame}

\begin{frame}{Backtraces}{Scanning the stack with \typename{frame\_descr}}
  \begin{columns}[c]
    \begin{column}{0.5\textwidth}
      \begin{itemize}
        \item Given the return address of the code that raises the exception by calling \funcname{caml\_raise\_exn} (\funcarg{pc} argument of \funcname{caml\_stash\_backtrace})
        \item And the position of the stack pointer (\funcarg{sp} argument of \funcname{caml\_stash\_backtrace})
        \item The frame size given by \typename{frame\_descr}.\funcarg{frame\_size}, allows to find the end of the previous frame
        \item And the return address of the caller of this function
        \bigskip
        \onslide<3->{
        \item Note that traps (here the reddish \funcname{ExnA}, \funcname{ExnB} and \funcname{ExnC} blocks) belong to the frame that installed them, and so the frame size changes when traps are removed
        }
      \end{itemize}
    \end{column}
    \begin{column}{0.5\textwidth}
      \raggedleft
      \onslide*<1>{\providecommand\step{2}\input{diagrams/backtrace/backtrace.tex}}%
      \onslide*<2>{\providecommand\step{1}\input{diagrams/backtrace/scanstack.tex}}%
      \onslide*<3>{\providecommand\step{2}\input{diagrams/backtrace/scanstack.tex}}%
      \onslide*<4>{\providecommand\step{3}\input{diagrams/backtrace/scanstack.tex}}%
      \onslide*<5>{\providecommand\step{4}\input{diagrams/backtrace/scanstack.tex}}%
      \onslide*<6>{\providecommand\step{5}\input{diagrams/backtrace/scanstack.tex}}%
    \end{column}
  \end{columns}
\end{frame}

% Duplicate of previous-previous slide
\begin{frame}{Backtraces}{Continuing on \funcname{caml\_stash\_backtrace}}
  \begin{columns}[c]
    \begin{column}{0.5\textwidth}
      \minipage[c][0.75\textheight][s]{\columnwidth}
      \begin{itemize}
          \color{gray}
        \item A C function provided by the runtime (specific to native code)
        \item It scans the stack, between the stack pointer at the raising point up to the next trap, in order to collect the names of the detected function frames
        \item It uses the same technique and information as the Garbage Collector (GC) uses to scan the values live on the stack
      \end{itemize}
      \begin{itemize}
        \item For each function frame detected, store a pointer to the \typename{frame\_descr} in the \localname{domain\_state->backtrace\_buffer} array, effectively constructing the backtrace
      \end{itemize}
      \onslide<2->{
        \begin{itemize}
            \smallskip
          \item Constructed backtrace:
            \funcname{definitely\_raise},
            \onslide<3->{\funcname{probably\_raise}}
        \end{itemize}
      }
      \vfill
      \mintocaml[firstline=9,lastline=13,highlightlines=12]{../src/backtrace/backtrace.ml}
      \endminipage
    \end{column}
    \begin{column}{0.5\textwidth}
      \raggedleft
      \onslide*<1>{\listStashBacktrace[highlightlines={114-115}]{}}
      \onslide*<2>{\providecommand\step{3}\input{diagrams/backtrace/backtrace.tex}}%
      \onslide*<3>{\providecommand\step{4}\input{diagrams/backtrace/backtrace.tex}}%
    \end{column}
  \end{columns}
\end{frame}

\begin{frame}{Backtraces}{Back to \funcname{caml\_raise\_exn}}
  \begin{columns}[c]
    \begin{column}{0.5\textwidth}
      \minipage[c][0.66\textheight][s]{\columnwidth}
      \begin{itemize}\color{gray}
        \item Checks wheteher exceptions backtrace should be recorded, by checking the \funcarg{domain\_state->backtrace\_active} value
        \item Switches to the C stack to be able to execute C code
        \item Calls \funcname{caml\_stash\_backtrace}, to collect the backtrace up to the next trap
      \end{itemize}
      \onslide*<2->{
        \begin{itemize}
          \item<2-> Executes the exception handler in \funcname{probably\_raise} with \funcname{RESTORE\_EXN\_HANDLER\_OCAML}
            \begin{itemize}
              \item Set \regname{RSP} to \localname{domain\_state->exn\_handler} value
              \item Restore \localname{domain\_state->exn\_handler} from trap
              \item Jump to the handler code with \funcname{ret}
            \end{itemize}
        \end{itemize}
      }
      \vfill
      \onslide*<1>{\mintocaml[firstline=9,lastline=13,highlightlines=12]{../src/backtrace/backtrace.ml}}
      \onslide*<2->{\mintocaml[firstline=15,lastline=21,highlightlines={19-21}]{../src/backtrace/backtrace.ml}}
      \endminipage
    \end{column}
    \begin{column}{0.5\textwidth}
      \centering
      \onslide*<1>{
        \mintc[firstline=190,lastline=192]{../ocaml/runtime/amd64.S}
        \mintc[firstline=234,lastline=237]{../ocaml/runtime/amd64.S}
      }
      \onslide*<2>{
        \mintc[firstline=190,lastline=192,highlightlines={191-192}]{../ocaml/runtime/amd64.S}
        \mintc[firstline=234,lastline=237,highlightlines={235-237}]{../ocaml/runtime/amd64.S}
      }
      \onslide*<1>{\listCamlRaiseExn[highlightlines=720]{}}
      \onslide*<2>{\listCamlRaiseExn[highlightlines=722]{}}
      \onslide*<3->{\providecommand\step{5}\input{diagrams/backtrace/backtrace.tex}}
    \end{column}
  \end{columns}
\end{frame}

\begin{frame}{Backtraces}{\funcname{caml\_probably\_raise}'s handler doesn't catch \typename{ExnC}}
  \begin{columns}[c]
    \begin{column}{0.45\textwidth}
      \minipage[c][0.75\textheight][s]{\columnwidth}
      \begin{itemize}
        \item<1-> \funcname{match \dots with}\footnotemark pattern matching can also match exceptions
        \item<1-> The CMM of \funcname{probably\_raise} shows that this \funcname{match \dots with} is implemented with a pattern matching and a \funcname{try \dots with}
        \item<1-> But this one only matches on \typename{ExnA} exception
        \item<2-> Since the pattern matching fails to catch the exception, \funcname{reraise} is called with our current \typename{ExnC} exception
        \item<3-> Disassembly shows that \funcname{reraise} is actually a call to \funcname{caml\_reraise\_exn}, which is also an assembly function provided by the runtime
      \end{itemize}
      \vfill
      \mintocaml[firstline=15,lastline=21,highlightlines={18-21}]{../src/backtrace/backtrace.ml}
      \endminipage
    \end{column}
    \begin{column}{0.55\textwidth}
      \centering
      \onslide*<1>{\mintcmm[highlightlines={10-18}]{../src/backtrace/camlBacktrace\_\_probably\_raise\_351.cmm}}
      \onslide*<2>{\listcmm[highlightlines=18]{../src/backtrace/camlBacktrace\_\_probably\_raise_351.cmm}{CMM of probably\_raise}}
      \onslide*<3>{\mintobjdump[firstline=5,lastline=35,highlightlines=31]{../src/backtrace/camlBacktrace\_\_probably\_raise_351.objdump}}
    \end{column}
  \end{columns}
  \only<1>{\footnotetext[1]{https://blog.janestreet.com/pattern-matching-and-exception-handling-unite/}}
\end{frame}

\newcommand\listCamlReraiseExn[1][]{
  \listasm[firstline=727,lastline=734,#1]{../ocaml/runtime/amd64.S}{runtime/amd64.S:727}}

\begin{frame}{Backtraces}{Introducing \funcname{caml\_reraise\_exn}}
  \begin{columns}[c]
    \begin{column}{0.5\textwidth}
      \minipage[c][0.66\textheight][s]{\columnwidth}
      \begin{itemize}
        \item<1-> The exception have to be reraised to the next handler
        \item<2-> First, checks if the backtrace should be collected with \funcarg{domain\_state->backtrace\_active}
        \only<3>{
        \begin{itemize}
            \item If not, behaves like a \funcname{raise\_notrace} with \funcname{RESTORE\_EXN\_HANDLER\_OCAML}
        \end{itemize}
        }
        \item<4-> Jumps into \funcname{caml\_raise\_exn}
        \item<5-> Calls \funcname{caml\_stash\_backtrace} again, to scan the next part of the stack: from the trap just pop-ed to the next trap
      \end{itemize}
      \vfill
      \mintocaml[firstline=15,lastline=21,highlightlines={18-21}]{../src/backtrace/backtrace.ml}
      \endminipage
    \end{column}
    \begin{column}{0.5\textwidth}
      \raggedleft
      \onslide*<1>{\listCamlReraiseExn{}}
      \onslide*<2>{\listCamlReraiseExn[highlightlines=729]{}}
      \onslide*<3>{
        \mintc[firstline=190,lastline=192,highlightlines={191-192}]{../ocaml/runtime/amd64.S}
        \mintc[firstline=234,lastline=237,highlightlines={235-237}]{../ocaml/runtime/amd64.S}
        \medskip
        \listCamlReraiseExn[highlightlines=731]{}
      }
      \onslide*<4-5>{
        % TODO refactor with \listCamlReraiseExn
        \mintasm[firstline=727,lastline=734,highlightlines=730]{../ocaml/runtime/amd64.S}
        \medskip
      }
      \onslide*<4>{\listCamlRaiseExn[highlightlines=711]{}}
      \onslide*<5>{\listCamlRaiseExn[highlightlines=720]{}}
    \end{column}
  \end{columns}
\end{frame}

\begin{frame}{Backtraces}{Backtrace collection continues}
  \begin{columns}[c]
    \begin{column}{0.55\textwidth}
      \minipage[c][0.75\textheight][s]{\columnwidth}
      \begin{itemize}
        \item<1-> \funcname{caml\_stash\_backtrace} scans the older part of the stack, up to the next trap
        \item<6-> Controls jumps to the nested exception handler in \funcname{maybe\_raise}, which matches only on \typename{ExnB}
        \item<7-> \funcname{caml\_reraise\_exn} is called again, backtrace collection resumes with \funcname{caml\_stash\_backtrace}
      \end{itemize}
      \medskip
      \begin{itemize}
        \item<2-> Constructed backtrace:
        \begin{itemize}
          \item <2-> \funcname{definitely\_raise}
          \item <2-> \funcname{probably\_raise}
          \item <3-> \funcname{probably\_raise} (re-raise)
          \item <4-> \funcname{maybe\_raise}
          \item <5-> \funcname{main}
          \item <8-> \funcname{main} (re-raise)
        \end{itemize}
      \end{itemize}
      \vfill
      \onslide*<1-5>{\mintocaml[firstline=15,lastline=21]{../src/backtrace/backtrace.ml}}
      \onslide*<6>{\mintocaml[firstline=28,lastline=34,highlightlines=31]{../src/backtrace/backtrace.ml}}
      \onslide*<7->{\mintocaml[firstline=28,lastline=34,highlightlines=33]{../src/backtrace/backtrace.ml}}
      \endminipage
    \end{column}
    \begin{column}{0.45\textwidth}
      \raggedleft
      \onslide*<1>{\providecommand\step{6}\input{diagrams/backtrace/backtrace.tex}}%
      \onslide*<2>{\providecommand\step{6}\input{diagrams/backtrace/backtrace.tex}}%
      \onslide*<3>{\providecommand\step{7}\input{diagrams/backtrace/backtrace.tex}}%
      \onslide*<4>{\providecommand\step{8}\input{diagrams/backtrace/backtrace.tex}}%
      \onslide*<5>{\providecommand\step{9}\input{diagrams/backtrace/backtrace.tex}}%
      \onslide*<6>{\providecommand\step{10}\input{diagrams/backtrace/backtrace.tex}}%
      \onslide*<7>{\providecommand\step{11}\input{diagrams/backtrace/backtrace.tex}}%
      \onslide*<8>{\providecommand\step{12}\input{diagrams/backtrace/backtrace.tex}}%
      \onslide*<9>{\providecommand\step{13}\input{diagrams/backtrace/backtrace.tex}}%
    \end{column}
  \end{columns}
\end{frame}

\begin{frame}{Backtraces}{Printing the backtrace}
  \begin{columns}[c]
    \begin{column}{0.45\textwidth}
      \minipage[c][0.66\textheight][s]{\columnwidth}
      \begin{itemize}
        \item<1-> The exception backtrace can now be printed to \funcarg{stdout} with \funcname{Printexc.print_backtrace}
        \item<2-> First the exception backtrace is retrieved into an OCaml array with \funcname{get_raw_backtrace }
        \item<3-> Which is implemented as the \funcname{caml_get_exception_raw_backtrace} C function in the runtime
        \item<4-> And finally printed with \funcname{print_exception_backtrace}
      \end{itemize}
      \vfill
      \mintocaml[firstline=28,lastline=34,highlightlines=33]{../src/backtrace/backtrace.ml}
      \endminipage
    \end{column}
    \begin{column}{0.55\textwidth}
      \onslide*<1,2>{
        \mintocaml[firstline=100,lastline=101]{../ocaml/stdlib/printexc.ml}
      }
      \only<1>{
        \listocaml[firstline=170,lastline=172]{../ocaml/stdlib/printexc.ml}{stdlib/printexc.ml:170}
      }
      \only<2>{
        \listocaml[firstline=170,lastline=172,highlightlines=172]{../ocaml/stdlib/printexc.ml}{stdlib/printexc.ml:170}
      }
      \only<3>{
        \mintc[firstline=154,lastline=156]{../ocaml/runtime/backtrace.c}
        \listc[firstline=171,lastline=192,highlightlines={184-187}]{../ocaml/runtime/backtrace.c}{runtime/backtrace.c:154}
      }
      \only<4>{
        \listocaml[firstline=155,lastline=165]{../ocaml/stdlib/printexc.ml}{stdlib/printexc.ml:55}
      }
    \end{column}
  \end{columns}
\end{frame}


\subsection{The multiple kinds of raise}
\frameSubsection{}{}

\begin{frame}{Backtraces}{When backtraces are not needed}
  \begin{columns}[c]
    \begin{column}{0.45\textwidth}
      \minipage[c][0.66\textheight][s]{\columnwidth}
      \begin{itemize}
        \item<1-> What if the backtrace for an exception is never needed?
        \item<2-> It's possible to raise the exception explicitly with \funcname{raise\_notrace} to make raising as cheap as possible
        \item<3-> An example of use in the \funcname{dynlink} library of the OCaml compiler
      \end{itemize}
      \vfill
      \onslide<3->{
      \listocaml[firstline=205,lastline=209,highlightlines=207]{../ocaml/otherlibs/dynlink/dynlink\_compilerlibs/misc.ml}{otherlibs/dynlink/dynlink\_compilerlibs/misc.ml:205}
      }
      \endminipage
    \end{column}
    \begin{column}{0.55\textwidth}
    \only<4>{
      \mintcmm[highlightlines=3]{../src/notrace/camlNotrace\_\_fun_347.cmm}
      \listcmm{../src/notrace/camlNotrace\_\_all\_somes\_327.cmm}{all\_somes}
    }
    \onslide*<5->{
      \listobjdump[highlightlines={6-9}]{../src/notrace/camlNotrace\_\_fun_347.objdump}{anonymous function from \funcname{all\_somes}}
    }
    \end{column}
  \end{columns}
\end{frame}

\begin{frame}{Backtraces}{Summary table of the multiple kinds of raise}
  \begin{itemize}
    \item When debugging information is enabled and if the backtrace of an exception isn't needed, it's possible to raise with \funcname{raise\_notrace} for performance
    \item When debugging information is \emph{not} enabled, the compiler optimises \funcname{raise} calls to inline assembly (same effect as using \funcname{raise\_notrace})
  \end{itemize}
  \bigskip
  \centering
  \begin{tabular}{| l | c | c | c | c |}
    \hline
    \parbox{10em}{Debug information \\ (\funcarg{-g} flag)}  & \multicolumn{2}{|c|}{Disabled}                                     & \multicolumn{2}{|c|}{Enabled} \\
    \hline
    Raise kind                                               & \funcname{raise} & \funcname{raise\_notrace}                       & \funcname{raise} & \funcname{raise\_notrace} \\
    \hline
    Compilation                                              & \parbox{8em}{inline assembly\\ (optimization)} & inline assembly   & \funcname{caml\_raise\_exn} & inline assembly \\
    \hline
  \end{tabular}
\end{frame}


%
% Takeaway #5
%
\subsection*{Takeaway \#5}
\frameSubsectionTakeaway{}
\againframe<5>{takeaway}
}%
    \end{column}
  \end{columns}
\end{frame}

\newcommand\listStashBacktrace[1][]{
  \listc[firstline=91,lastline=120,#1]{../ocaml/runtime/backtrace\_nat.c}{runtime/backtrace\_nat.c:91}}

\begin{frame}{Backtraces}{Introducing \funcname{caml\_stash\_backtrace}}
  \begin{columns}[c]
    \begin{column}{0.5\textwidth}
      \minipage[c][0.66\textheight][s]{\columnwidth}
      \begin{itemize}
        \item<1-> A C function provided by the runtime (specific to native code)
        \item<2-> It scans the stack, between the stack pointer at the raising point up to the next trap, in order to collect the names of the detected function frames
          \onslide*<2>{
            \begin{itemize}
              \item And more information, like line number, column number...
            \end{itemize}
          }
        \item<3-> It uses the same technique and information as the Garbage Collector (GC) uses to scan the values live on the stack
          \onslide*<4>{
            \begin{itemize}
              \item \typename{frame\_descr} are at the core of stack unwinding
            \end{itemize}
          }
      \end{itemize}
      \vfill
      \mintocaml[firstline=9,lastline=13,highlightlines=12]{../src/backtrace/backtrace.ml}
      \endminipage
    \end{column}
    \begin{column}{0.5\textwidth}
      \onslide*<1>{\listStashBacktrace[highlightlines=91]{}}
      \onslide*<2>{\listStashBacktrace[highlightlines={91,117-118}]{}}
      \onslide*<3->{\listStashBacktrace[highlightlines={94,106,109-110}]{}}
    \end{column}
  \end{columns}
\end{frame}

\newcommand\listFrameDescr[1][]{
  \listocaml[firstline=111,lastline=124,#1]{../ocaml/asmcomp/emitaux.ml}{asmcomp/emitaux.ml:111}}
\newcommand\listDebugInfo[1][]{
  \listocaml[firstline=38,lastline=49,#1]{../ocaml/lambda/debuginfo.mli}{lambda/debuginfo.mli:38}}

\begin{frame}{Backtraces}{\typename{frame\_descr} and \typename{frame\_debuginfo} in a nutshell}
  \begin{columns}[c]
    \begin{column}{0.5\textwidth}
      \begin{itemize}
        \item<1-> For every return address in an OCaml program, there is a associated \typename{frame\_descr}
        \item<2-> A \typename{frame\_descr} specifies
        \begin{itemize}
          \item<2-> The size of the function stack frame
          \item<3-> Where live values are on the stack (so that the GC can mark them as live and prevent from being swept)
        \end{itemize}
        \item<4-> When compiled with debug information, \typename{frame\_descr} will be linked to a \typename{frame\_debuginfo}
        \begin{itemize}
          \item<5-> Contains information about the position in the source code (file name, line and column number)
        \end{itemize}
      \end{itemize}
    \end{column}
    \begin{column}{0.5\textwidth}
        \onslide*<1>{\listFrameDescr[highlightlines={118-122}]{}}
        \onslide*<2>{\listFrameDescr[highlightlines={120}]{}}
        \onslide*<3>{\listFrameDescr[highlightlines={121}]{}}
        \onslide*<4->{\listFrameDescr[highlightlines={122}]{}}
        \medskip
        \onslide*<1-3>{\listDebugInfo{}}
        \onslide*<4>{\listDebugInfo[highlightlines={38,49}]{}}
        \onslide*<5>{\listDebugInfo[highlightlines={39-46}]{}}
    \end{column}
  \end{columns}
\end{frame}

\begin{frame}{Backtraces}{Scanning the stack with \typename{frame\_descr}}
  \begin{columns}[c]
    \begin{column}{0.5\textwidth}
      \begin{itemize}
        \item Given the return address of the code that raises the exception by calling \funcname{caml\_raise\_exn} (\funcarg{pc} argument of \funcname{caml\_stash\_backtrace})
        \item And the position of the stack pointer (\funcarg{sp} argument of \funcname{caml\_stash\_backtrace})
        \item The frame size given by \typename{frame\_descr}.\funcarg{frame\_size}, allows to find the end of the previous frame
        \item And the return address of the caller of this function
        \bigskip
        \onslide<3->{
        \item Note that traps (here the reddish \funcname{ExnA}, \funcname{ExnB} and \funcname{ExnC} blocks) belong to the frame that installed them, and so the frame size changes when traps are removed
        }
      \end{itemize}
    \end{column}
    \begin{column}{0.5\textwidth}
      \raggedleft
      \onslide*<1>{\providecommand\step{2}%
% Backtraces
%
\subsection{One advantage of raising with debugging information}
\frameSubsection{}{}

\begin{frame}{Backtraces}{Debugging information \& source code}
  \begin{columns}[c]
    \begin{column}{0.5\textwidth}
      \begin{itemize}
        \item Raising an exception is fun and games, but how do we know where the exception originated? $\rightarrow$ With the backtrace associated with the exception!
        \item In order to get a backtrace from the point where an exception was raised, it's necessary to compile with debugging information enabled (\funcarg{-g} flag for the compiler)
        \item Same example as in the `Nested handlers' section
      \end{itemize}
    \end{column}
    \begin{column}{0.5\textwidth}
      \listocaml[firstline=9,lastline=34]{../src/backtrace/backtrace.ml}{backtrace/backtrace.ml}
    \end{column}
  \end{columns}
\end{frame}

\begin{frame}{Backtraces}{CMM and disassembly of \funcname{definitely\_raise}}
  \begin{columns}[c]
    \begin{column}{0.5\textwidth}
      \minipage[c][0.66\textheight][s]{\columnwidth}
      \begin{itemize}
        \item<1-> An effect of compiling with debugging information (\funcarg{-g}): the CMM no longer uses \funcname{raise\_notrace} to raise the exception but \funcname{raise} instead
        \item<2-> Disassembly shows that CMM \funcname{raise} is actually a call to \funcname{caml\_raise\_exn}, a function in assembler provided by the runtime
      \end{itemize}
      \vfill
      \mintocaml[firstline=9,lastline=13,highlightlines=12]{../src/backtrace/backtrace.ml}
      \endminipage
    \end{column}
    \begin{column}{0.5\textwidth}
      \onslide*<1>{
        \mintcmm[highlightlines=5]{../src/backtrace/camlBacktrace\_\_definitely\_raise\_347.cmm}
      }
      \onslide*<2>{
        \mintobjdump[firstline=2,highlightlines=14]{../src/backtrace/camlBacktrace\_\_definitely\_raise\_347.objdump}
      }
    \end{column}
  \end{columns}
\end{frame}

\begin{frame}{Backtraces}{Introducing \funcname{caml\_raise\_exn}}
  \begin{columns}[c]
    \begin{column}{0.5\textwidth}
      \minipage[c][0.66\textheight][s]{\columnwidth}
      \onslide*<1>{
        \begin{itemize}
          \item Raising the exception is performed using \funcname{caml\_raise\_exn} which is
            \begin{itemize}
              \item An assembly function provided by the runtime
              \item Architecture specific
            \end{itemize}
          \item Let's see what it does differently from \funcname{raise\_notrace}\dots
          \item We are about to raise \typename{ExnC} with \funcname{caml\_raise\_exn} from \funcname{definitely\_raise}
        \end{itemize}
      }
      \onslide*<2>{
        \mintobjdump[firstline=2,highlightlines=14]{../src/backtrace/camlBacktrace\_\_definitely\_raise\_347.objdump}
      }
      \vfill
      \mintocaml[firstline=9,lastline=13,highlightlines=12]{../src/backtrace/backtrace.ml}
      \endminipage
    \end{column}
    \begin{column}{0.5\textwidth}
      \centering
      \providecommand\step{1}\input{diagrams/backtrace/backtrace.tex}
    \end{column}
  \end{columns}
\end{frame}

\begin{frame}{Backtraces}{But before that, we need more fields from \typename{caml\_domain\_state}}
  \begin{columns}
    \begin{column}{0.5\textwidth}
      \begin{itemize}
        \item Remember \typename{caml\_domain\_state} the per-domain runtime state?
        \item Some other members are of interest to us now\dots
        \item \localname{backtrace\_active} is used to know if the runtime should collect backtraces
        \item \localname{backtrace\_active} can be set by
          \begin{itemize}
            \item The \funcarg{b} flag in the \funcname{OCAMLRUNPARAM} environment variable
            \item \mintinline{ocaml}{Printexc.record_backtrace : bool -> unit} \footnotemark
          \end{itemize}
      \end{itemize}
    \end{column}
    \begin{column}{0.5\textwidth}
      \begin{listing}
        \mintc[highlightlines={9-11}]{src/ocaml/domain\_state\_backtrace.h}
        \caption{runtime/caml/domain\_state.h}
      \end{listing}
    \end{column}
  \end{columns}
  \footnotetext[1]{https://v2.ocaml.org/api/Printexc.html}
\end{frame}

\newcommand\listCamlRaiseExn[1][]{
  \listasm[firstline=702,lastline=725,#1]{../ocaml/runtime/amd64.S}{runtime/amd64.S:702}}

\begin{frame}{Backtraces}{Walk through \funcname{caml\_raise\_exn}}
  \begin{columns}[T]
    \begin{column}{0.5\textwidth}
      \begin{itemize}
        \item<1-> First checks whether exceptions backtrace should be recorded
        \item<1-> By checking the \funcarg{domain\_state->backtrace\_active} value
        \item<2-> If not, acts as \funcname{raise\_notrace} with \funcname{RESTORE\_EXN\_HANDLER\_OCAML}
          \begin{itemize}
            \item Set \regname{RSP} to \localname{domain\_state->exn\_handler} value
            \item Restore \localname{domain\_state->exn\_handler} from trap
            \item Jump with \funcname{ret} (same effect as using \funcname{pop} and \funcname{jmp})
          \end{itemize}
        \item<3-> Otherwise, switches to the C stack to be able to execute C code
        \item<4-> Calls \funcname{caml\_stash\_backtrace}, a C function provided by the runtime\ignorespaces
          \onslide<6->{, with
            \begin{itemize}
              \item \funcarg{exn}: exception value
              \item \funcarg{pc}: code address of the raising point
              \item \funcarg{sp}: stack address of the raising function
              \item \funcarg{trapsp}: stack address of the next handler
            \end{itemize}
          }
      \end{itemize}
      \bigskip
      \mintocaml[firstline=9,lastline=13,highlightlines=12]{../src/backtrace/backtrace.ml}
    \end{column}
    \begin{column}{0.5\textwidth}
      \raggedleft
      \onslide*<1,3-4>{
        \mintc[firstline=190,lastline=192]{../ocaml/runtime/amd64.S}
        \mintc[firstline=234,lastline=237]{../ocaml/runtime/amd64.S}
      }
      \onslide*<2>{
        \mintc[firstline=190,lastline=192,highlightlines={191-192}]{../ocaml/runtime/amd64.S}
        \mintc[firstline=234,lastline=237,highlightlines={235-237}]{../ocaml/runtime/amd64.S}
      }
      \onslide*<1>{\listCamlRaiseExn[highlightlines=705]{}}%
      \onslide*<2>{\listCamlRaiseExn[highlightlines={707-708}]{}}%
      \onslide*<3>{\listCamlRaiseExn[highlightlines={712-714}]{}}%
      \onslide*<4>{\listCamlRaiseExn[highlightlines={715-720}]{}}%
      \onslide*<5>{\providecommand\step{1}\input{diagrams/backtrace/backtrace.tex}}%
      \onslide*<6>{\providecommand\step{2}\input{diagrams/backtrace/backtrace.tex}}%
    \end{column}
  \end{columns}
\end{frame}

\newcommand\listStashBacktrace[1][]{
  \listc[firstline=91,lastline=120,#1]{../ocaml/runtime/backtrace\_nat.c}{runtime/backtrace\_nat.c:91}}

\begin{frame}{Backtraces}{Introducing \funcname{caml\_stash\_backtrace}}
  \begin{columns}[c]
    \begin{column}{0.5\textwidth}
      \minipage[c][0.66\textheight][s]{\columnwidth}
      \begin{itemize}
        \item<1-> A C function provided by the runtime (specific to native code)
        \item<2-> It scans the stack, between the stack pointer at the raising point up to the next trap, in order to collect the names of the detected function frames
          \onslide*<2>{
            \begin{itemize}
              \item And more information, like line number, column number...
            \end{itemize}
          }
        \item<3-> It uses the same technique and information as the Garbage Collector (GC) uses to scan the values live on the stack
          \onslide*<4>{
            \begin{itemize}
              \item \typename{frame\_descr} are at the core of stack unwinding
            \end{itemize}
          }
      \end{itemize}
      \vfill
      \mintocaml[firstline=9,lastline=13,highlightlines=12]{../src/backtrace/backtrace.ml}
      \endminipage
    \end{column}
    \begin{column}{0.5\textwidth}
      \onslide*<1>{\listStashBacktrace[highlightlines=91]{}}
      \onslide*<2>{\listStashBacktrace[highlightlines={91,117-118}]{}}
      \onslide*<3->{\listStashBacktrace[highlightlines={94,106,109-110}]{}}
    \end{column}
  \end{columns}
\end{frame}

\newcommand\listFrameDescr[1][]{
  \listocaml[firstline=111,lastline=124,#1]{../ocaml/asmcomp/emitaux.ml}{asmcomp/emitaux.ml:111}}
\newcommand\listDebugInfo[1][]{
  \listocaml[firstline=38,lastline=49,#1]{../ocaml/lambda/debuginfo.mli}{lambda/debuginfo.mli:38}}

\begin{frame}{Backtraces}{\typename{frame\_descr} and \typename{frame\_debuginfo} in a nutshell}
  \begin{columns}[c]
    \begin{column}{0.5\textwidth}
      \begin{itemize}
        \item<1-> For every return address in an OCaml program, there is a associated \typename{frame\_descr}
        \item<2-> A \typename{frame\_descr} specifies
        \begin{itemize}
          \item<2-> The size of the function stack frame
          \item<3-> Where live values are on the stack (so that the GC can mark them as live and prevent from being swept)
        \end{itemize}
        \item<4-> When compiled with debug information, \typename{frame\_descr} will be linked to a \typename{frame\_debuginfo}
        \begin{itemize}
          \item<5-> Contains information about the position in the source code (file name, line and column number)
        \end{itemize}
      \end{itemize}
    \end{column}
    \begin{column}{0.5\textwidth}
        \onslide*<1>{\listFrameDescr[highlightlines={118-122}]{}}
        \onslide*<2>{\listFrameDescr[highlightlines={120}]{}}
        \onslide*<3>{\listFrameDescr[highlightlines={121}]{}}
        \onslide*<4->{\listFrameDescr[highlightlines={122}]{}}
        \medskip
        \onslide*<1-3>{\listDebugInfo{}}
        \onslide*<4>{\listDebugInfo[highlightlines={38,49}]{}}
        \onslide*<5>{\listDebugInfo[highlightlines={39-46}]{}}
    \end{column}
  \end{columns}
\end{frame}

\begin{frame}{Backtraces}{Scanning the stack with \typename{frame\_descr}}
  \begin{columns}[c]
    \begin{column}{0.5\textwidth}
      \begin{itemize}
        \item Given the return address of the code that raises the exception by calling \funcname{caml\_raise\_exn} (\funcarg{pc} argument of \funcname{caml\_stash\_backtrace})
        \item And the position of the stack pointer (\funcarg{sp} argument of \funcname{caml\_stash\_backtrace})
        \item The frame size given by \typename{frame\_descr}.\funcarg{frame\_size}, allows to find the end of the previous frame
        \item And the return address of the caller of this function
        \bigskip
        \onslide<3->{
        \item Note that traps (here the reddish \funcname{ExnA}, \funcname{ExnB} and \funcname{ExnC} blocks) belong to the frame that installed them, and so the frame size changes when traps are removed
        }
      \end{itemize}
    \end{column}
    \begin{column}{0.5\textwidth}
      \raggedleft
      \onslide*<1>{\providecommand\step{2}\input{diagrams/backtrace/backtrace.tex}}%
      \onslide*<2>{\providecommand\step{1}\input{diagrams/backtrace/scanstack.tex}}%
      \onslide*<3>{\providecommand\step{2}\input{diagrams/backtrace/scanstack.tex}}%
      \onslide*<4>{\providecommand\step{3}\input{diagrams/backtrace/scanstack.tex}}%
      \onslide*<5>{\providecommand\step{4}\input{diagrams/backtrace/scanstack.tex}}%
      \onslide*<6>{\providecommand\step{5}\input{diagrams/backtrace/scanstack.tex}}%
    \end{column}
  \end{columns}
\end{frame}

% Duplicate of previous-previous slide
\begin{frame}{Backtraces}{Continuing on \funcname{caml\_stash\_backtrace}}
  \begin{columns}[c]
    \begin{column}{0.5\textwidth}
      \minipage[c][0.75\textheight][s]{\columnwidth}
      \begin{itemize}
          \color{gray}
        \item A C function provided by the runtime (specific to native code)
        \item It scans the stack, between the stack pointer at the raising point up to the next trap, in order to collect the names of the detected function frames
        \item It uses the same technique and information as the Garbage Collector (GC) uses to scan the values live on the stack
      \end{itemize}
      \begin{itemize}
        \item For each function frame detected, store a pointer to the \typename{frame\_descr} in the \localname{domain\_state->backtrace\_buffer} array, effectively constructing the backtrace
      \end{itemize}
      \onslide<2->{
        \begin{itemize}
            \smallskip
          \item Constructed backtrace:
            \funcname{definitely\_raise},
            \onslide<3->{\funcname{probably\_raise}}
        \end{itemize}
      }
      \vfill
      \mintocaml[firstline=9,lastline=13,highlightlines=12]{../src/backtrace/backtrace.ml}
      \endminipage
    \end{column}
    \begin{column}{0.5\textwidth}
      \raggedleft
      \onslide*<1>{\listStashBacktrace[highlightlines={114-115}]{}}
      \onslide*<2>{\providecommand\step{3}\input{diagrams/backtrace/backtrace.tex}}%
      \onslide*<3>{\providecommand\step{4}\input{diagrams/backtrace/backtrace.tex}}%
    \end{column}
  \end{columns}
\end{frame}

\begin{frame}{Backtraces}{Back to \funcname{caml\_raise\_exn}}
  \begin{columns}[c]
    \begin{column}{0.5\textwidth}
      \minipage[c][0.66\textheight][s]{\columnwidth}
      \begin{itemize}\color{gray}
        \item Checks wheteher exceptions backtrace should be recorded, by checking the \funcarg{domain\_state->backtrace\_active} value
        \item Switches to the C stack to be able to execute C code
        \item Calls \funcname{caml\_stash\_backtrace}, to collect the backtrace up to the next trap
      \end{itemize}
      \onslide*<2->{
        \begin{itemize}
          \item<2-> Executes the exception handler in \funcname{probably\_raise} with \funcname{RESTORE\_EXN\_HANDLER\_OCAML}
            \begin{itemize}
              \item Set \regname{RSP} to \localname{domain\_state->exn\_handler} value
              \item Restore \localname{domain\_state->exn\_handler} from trap
              \item Jump to the handler code with \funcname{ret}
            \end{itemize}
        \end{itemize}
      }
      \vfill
      \onslide*<1>{\mintocaml[firstline=9,lastline=13,highlightlines=12]{../src/backtrace/backtrace.ml}}
      \onslide*<2->{\mintocaml[firstline=15,lastline=21,highlightlines={19-21}]{../src/backtrace/backtrace.ml}}
      \endminipage
    \end{column}
    \begin{column}{0.5\textwidth}
      \centering
      \onslide*<1>{
        \mintc[firstline=190,lastline=192]{../ocaml/runtime/amd64.S}
        \mintc[firstline=234,lastline=237]{../ocaml/runtime/amd64.S}
      }
      \onslide*<2>{
        \mintc[firstline=190,lastline=192,highlightlines={191-192}]{../ocaml/runtime/amd64.S}
        \mintc[firstline=234,lastline=237,highlightlines={235-237}]{../ocaml/runtime/amd64.S}
      }
      \onslide*<1>{\listCamlRaiseExn[highlightlines=720]{}}
      \onslide*<2>{\listCamlRaiseExn[highlightlines=722]{}}
      \onslide*<3->{\providecommand\step{5}\input{diagrams/backtrace/backtrace.tex}}
    \end{column}
  \end{columns}
\end{frame}

\begin{frame}{Backtraces}{\funcname{caml\_probably\_raise}'s handler doesn't catch \typename{ExnC}}
  \begin{columns}[c]
    \begin{column}{0.45\textwidth}
      \minipage[c][0.75\textheight][s]{\columnwidth}
      \begin{itemize}
        \item<1-> \funcname{match \dots with}\footnotemark pattern matching can also match exceptions
        \item<1-> The CMM of \funcname{probably\_raise} shows that this \funcname{match \dots with} is implemented with a pattern matching and a \funcname{try \dots with}
        \item<1-> But this one only matches on \typename{ExnA} exception
        \item<2-> Since the pattern matching fails to catch the exception, \funcname{reraise} is called with our current \typename{ExnC} exception
        \item<3-> Disassembly shows that \funcname{reraise} is actually a call to \funcname{caml\_reraise\_exn}, which is also an assembly function provided by the runtime
      \end{itemize}
      \vfill
      \mintocaml[firstline=15,lastline=21,highlightlines={18-21}]{../src/backtrace/backtrace.ml}
      \endminipage
    \end{column}
    \begin{column}{0.55\textwidth}
      \centering
      \onslide*<1>{\mintcmm[highlightlines={10-18}]{../src/backtrace/camlBacktrace\_\_probably\_raise\_351.cmm}}
      \onslide*<2>{\listcmm[highlightlines=18]{../src/backtrace/camlBacktrace\_\_probably\_raise_351.cmm}{CMM of probably\_raise}}
      \onslide*<3>{\mintobjdump[firstline=5,lastline=35,highlightlines=31]{../src/backtrace/camlBacktrace\_\_probably\_raise_351.objdump}}
    \end{column}
  \end{columns}
  \only<1>{\footnotetext[1]{https://blog.janestreet.com/pattern-matching-and-exception-handling-unite/}}
\end{frame}

\newcommand\listCamlReraiseExn[1][]{
  \listasm[firstline=727,lastline=734,#1]{../ocaml/runtime/amd64.S}{runtime/amd64.S:727}}

\begin{frame}{Backtraces}{Introducing \funcname{caml\_reraise\_exn}}
  \begin{columns}[c]
    \begin{column}{0.5\textwidth}
      \minipage[c][0.66\textheight][s]{\columnwidth}
      \begin{itemize}
        \item<1-> The exception have to be reraised to the next handler
        \item<2-> First, checks if the backtrace should be collected with \funcarg{domain\_state->backtrace\_active}
        \only<3>{
        \begin{itemize}
            \item If not, behaves like a \funcname{raise\_notrace} with \funcname{RESTORE\_EXN\_HANDLER\_OCAML}
        \end{itemize}
        }
        \item<4-> Jumps into \funcname{caml\_raise\_exn}
        \item<5-> Calls \funcname{caml\_stash\_backtrace} again, to scan the next part of the stack: from the trap just pop-ed to the next trap
      \end{itemize}
      \vfill
      \mintocaml[firstline=15,lastline=21,highlightlines={18-21}]{../src/backtrace/backtrace.ml}
      \endminipage
    \end{column}
    \begin{column}{0.5\textwidth}
      \raggedleft
      \onslide*<1>{\listCamlReraiseExn{}}
      \onslide*<2>{\listCamlReraiseExn[highlightlines=729]{}}
      \onslide*<3>{
        \mintc[firstline=190,lastline=192,highlightlines={191-192}]{../ocaml/runtime/amd64.S}
        \mintc[firstline=234,lastline=237,highlightlines={235-237}]{../ocaml/runtime/amd64.S}
        \medskip
        \listCamlReraiseExn[highlightlines=731]{}
      }
      \onslide*<4-5>{
        % TODO refactor with \listCamlReraiseExn
        \mintasm[firstline=727,lastline=734,highlightlines=730]{../ocaml/runtime/amd64.S}
        \medskip
      }
      \onslide*<4>{\listCamlRaiseExn[highlightlines=711]{}}
      \onslide*<5>{\listCamlRaiseExn[highlightlines=720]{}}
    \end{column}
  \end{columns}
\end{frame}

\begin{frame}{Backtraces}{Backtrace collection continues}
  \begin{columns}[c]
    \begin{column}{0.55\textwidth}
      \minipage[c][0.75\textheight][s]{\columnwidth}
      \begin{itemize}
        \item<1-> \funcname{caml\_stash\_backtrace} scans the older part of the stack, up to the next trap
        \item<6-> Controls jumps to the nested exception handler in \funcname{maybe\_raise}, which matches only on \typename{ExnB}
        \item<7-> \funcname{caml\_reraise\_exn} is called again, backtrace collection resumes with \funcname{caml\_stash\_backtrace}
      \end{itemize}
      \medskip
      \begin{itemize}
        \item<2-> Constructed backtrace:
        \begin{itemize}
          \item <2-> \funcname{definitely\_raise}
          \item <2-> \funcname{probably\_raise}
          \item <3-> \funcname{probably\_raise} (re-raise)
          \item <4-> \funcname{maybe\_raise}
          \item <5-> \funcname{main}
          \item <8-> \funcname{main} (re-raise)
        \end{itemize}
      \end{itemize}
      \vfill
      \onslide*<1-5>{\mintocaml[firstline=15,lastline=21]{../src/backtrace/backtrace.ml}}
      \onslide*<6>{\mintocaml[firstline=28,lastline=34,highlightlines=31]{../src/backtrace/backtrace.ml}}
      \onslide*<7->{\mintocaml[firstline=28,lastline=34,highlightlines=33]{../src/backtrace/backtrace.ml}}
      \endminipage
    \end{column}
    \begin{column}{0.45\textwidth}
      \raggedleft
      \onslide*<1>{\providecommand\step{6}\input{diagrams/backtrace/backtrace.tex}}%
      \onslide*<2>{\providecommand\step{6}\input{diagrams/backtrace/backtrace.tex}}%
      \onslide*<3>{\providecommand\step{7}\input{diagrams/backtrace/backtrace.tex}}%
      \onslide*<4>{\providecommand\step{8}\input{diagrams/backtrace/backtrace.tex}}%
      \onslide*<5>{\providecommand\step{9}\input{diagrams/backtrace/backtrace.tex}}%
      \onslide*<6>{\providecommand\step{10}\input{diagrams/backtrace/backtrace.tex}}%
      \onslide*<7>{\providecommand\step{11}\input{diagrams/backtrace/backtrace.tex}}%
      \onslide*<8>{\providecommand\step{12}\input{diagrams/backtrace/backtrace.tex}}%
      \onslide*<9>{\providecommand\step{13}\input{diagrams/backtrace/backtrace.tex}}%
    \end{column}
  \end{columns}
\end{frame}

\begin{frame}{Backtraces}{Printing the backtrace}
  \begin{columns}[c]
    \begin{column}{0.45\textwidth}
      \minipage[c][0.66\textheight][s]{\columnwidth}
      \begin{itemize}
        \item<1-> The exception backtrace can now be printed to \funcarg{stdout} with \funcname{Printexc.print_backtrace}
        \item<2-> First the exception backtrace is retrieved into an OCaml array with \funcname{get_raw_backtrace }
        \item<3-> Which is implemented as the \funcname{caml_get_exception_raw_backtrace} C function in the runtime
        \item<4-> And finally printed with \funcname{print_exception_backtrace}
      \end{itemize}
      \vfill
      \mintocaml[firstline=28,lastline=34,highlightlines=33]{../src/backtrace/backtrace.ml}
      \endminipage
    \end{column}
    \begin{column}{0.55\textwidth}
      \onslide*<1,2>{
        \mintocaml[firstline=100,lastline=101]{../ocaml/stdlib/printexc.ml}
      }
      \only<1>{
        \listocaml[firstline=170,lastline=172]{../ocaml/stdlib/printexc.ml}{stdlib/printexc.ml:170}
      }
      \only<2>{
        \listocaml[firstline=170,lastline=172,highlightlines=172]{../ocaml/stdlib/printexc.ml}{stdlib/printexc.ml:170}
      }
      \only<3>{
        \mintc[firstline=154,lastline=156]{../ocaml/runtime/backtrace.c}
        \listc[firstline=171,lastline=192,highlightlines={184-187}]{../ocaml/runtime/backtrace.c}{runtime/backtrace.c:154}
      }
      \only<4>{
        \listocaml[firstline=155,lastline=165]{../ocaml/stdlib/printexc.ml}{stdlib/printexc.ml:55}
      }
    \end{column}
  \end{columns}
\end{frame}


\subsection{The multiple kinds of raise}
\frameSubsection{}{}

\begin{frame}{Backtraces}{When backtraces are not needed}
  \begin{columns}[c]
    \begin{column}{0.45\textwidth}
      \minipage[c][0.66\textheight][s]{\columnwidth}
      \begin{itemize}
        \item<1-> What if the backtrace for an exception is never needed?
        \item<2-> It's possible to raise the exception explicitly with \funcname{raise\_notrace} to make raising as cheap as possible
        \item<3-> An example of use in the \funcname{dynlink} library of the OCaml compiler
      \end{itemize}
      \vfill
      \onslide<3->{
      \listocaml[firstline=205,lastline=209,highlightlines=207]{../ocaml/otherlibs/dynlink/dynlink\_compilerlibs/misc.ml}{otherlibs/dynlink/dynlink\_compilerlibs/misc.ml:205}
      }
      \endminipage
    \end{column}
    \begin{column}{0.55\textwidth}
    \only<4>{
      \mintcmm[highlightlines=3]{../src/notrace/camlNotrace\_\_fun_347.cmm}
      \listcmm{../src/notrace/camlNotrace\_\_all\_somes\_327.cmm}{all\_somes}
    }
    \onslide*<5->{
      \listobjdump[highlightlines={6-9}]{../src/notrace/camlNotrace\_\_fun_347.objdump}{anonymous function from \funcname{all\_somes}}
    }
    \end{column}
  \end{columns}
\end{frame}

\begin{frame}{Backtraces}{Summary table of the multiple kinds of raise}
  \begin{itemize}
    \item When debugging information is enabled and if the backtrace of an exception isn't needed, it's possible to raise with \funcname{raise\_notrace} for performance
    \item When debugging information is \emph{not} enabled, the compiler optimises \funcname{raise} calls to inline assembly (same effect as using \funcname{raise\_notrace})
  \end{itemize}
  \bigskip
  \centering
  \begin{tabular}{| l | c | c | c | c |}
    \hline
    \parbox{10em}{Debug information \\ (\funcarg{-g} flag)}  & \multicolumn{2}{|c|}{Disabled}                                     & \multicolumn{2}{|c|}{Enabled} \\
    \hline
    Raise kind                                               & \funcname{raise} & \funcname{raise\_notrace}                       & \funcname{raise} & \funcname{raise\_notrace} \\
    \hline
    Compilation                                              & \parbox{8em}{inline assembly\\ (optimization)} & inline assembly   & \funcname{caml\_raise\_exn} & inline assembly \\
    \hline
  \end{tabular}
\end{frame}


%
% Takeaway #5
%
\subsection*{Takeaway \#5}
\frameSubsectionTakeaway{}
\againframe<5>{takeaway}
}%
      \onslide*<2>{\providecommand\step{1}\input{diagrams/backtrace/scanstack.tex}}%
      \onslide*<3>{\providecommand\step{2}\input{diagrams/backtrace/scanstack.tex}}%
      \onslide*<4>{\providecommand\step{3}\input{diagrams/backtrace/scanstack.tex}}%
      \onslide*<5>{\providecommand\step{4}\input{diagrams/backtrace/scanstack.tex}}%
      \onslide*<6>{\providecommand\step{5}\input{diagrams/backtrace/scanstack.tex}}%
    \end{column}
  \end{columns}
\end{frame}

% Duplicate of previous-previous slide
\begin{frame}{Backtraces}{Continuing on \funcname{caml\_stash\_backtrace}}
  \begin{columns}[c]
    \begin{column}{0.5\textwidth}
      \minipage[c][0.75\textheight][s]{\columnwidth}
      \begin{itemize}
          \color{gray}
        \item A C function provided by the runtime (specific to native code)
        \item It scans the stack, between the stack pointer at the raising point up to the next trap, in order to collect the names of the detected function frames
        \item It uses the same technique and information as the Garbage Collector (GC) uses to scan the values live on the stack
      \end{itemize}
      \begin{itemize}
        \item For each function frame detected, store a pointer to the \typename{frame\_descr} in the \localname{domain\_state->backtrace\_buffer} array, effectively constructing the backtrace
      \end{itemize}
      \onslide<2->{
        \begin{itemize}
            \smallskip
          \item Constructed backtrace:
            \funcname{definitely\_raise},
            \onslide<3->{\funcname{probably\_raise}}
        \end{itemize}
      }
      \vfill
      \mintocaml[firstline=9,lastline=13,highlightlines=12]{../src/backtrace/backtrace.ml}
      \endminipage
    \end{column}
    \begin{column}{0.5\textwidth}
      \raggedleft
      \onslide*<1>{\listStashBacktrace[highlightlines={114-115}]{}}
      \onslide*<2>{\providecommand\step{3}%
% Backtraces
%
\subsection{One advantage of raising with debugging information}
\frameSubsection{}{}

\begin{frame}{Backtraces}{Debugging information \& source code}
  \begin{columns}[c]
    \begin{column}{0.5\textwidth}
      \begin{itemize}
        \item Raising an exception is fun and games, but how do we know where the exception originated? $\rightarrow$ With the backtrace associated with the exception!
        \item In order to get a backtrace from the point where an exception was raised, it's necessary to compile with debugging information enabled (\funcarg{-g} flag for the compiler)
        \item Same example as in the `Nested handlers' section
      \end{itemize}
    \end{column}
    \begin{column}{0.5\textwidth}
      \listocaml[firstline=9,lastline=34]{../src/backtrace/backtrace.ml}{backtrace/backtrace.ml}
    \end{column}
  \end{columns}
\end{frame}

\begin{frame}{Backtraces}{CMM and disassembly of \funcname{definitely\_raise}}
  \begin{columns}[c]
    \begin{column}{0.5\textwidth}
      \minipage[c][0.66\textheight][s]{\columnwidth}
      \begin{itemize}
        \item<1-> An effect of compiling with debugging information (\funcarg{-g}): the CMM no longer uses \funcname{raise\_notrace} to raise the exception but \funcname{raise} instead
        \item<2-> Disassembly shows that CMM \funcname{raise} is actually a call to \funcname{caml\_raise\_exn}, a function in assembler provided by the runtime
      \end{itemize}
      \vfill
      \mintocaml[firstline=9,lastline=13,highlightlines=12]{../src/backtrace/backtrace.ml}
      \endminipage
    \end{column}
    \begin{column}{0.5\textwidth}
      \onslide*<1>{
        \mintcmm[highlightlines=5]{../src/backtrace/camlBacktrace\_\_definitely\_raise\_347.cmm}
      }
      \onslide*<2>{
        \mintobjdump[firstline=2,highlightlines=14]{../src/backtrace/camlBacktrace\_\_definitely\_raise\_347.objdump}
      }
    \end{column}
  \end{columns}
\end{frame}

\begin{frame}{Backtraces}{Introducing \funcname{caml\_raise\_exn}}
  \begin{columns}[c]
    \begin{column}{0.5\textwidth}
      \minipage[c][0.66\textheight][s]{\columnwidth}
      \onslide*<1>{
        \begin{itemize}
          \item Raising the exception is performed using \funcname{caml\_raise\_exn} which is
            \begin{itemize}
              \item An assembly function provided by the runtime
              \item Architecture specific
            \end{itemize}
          \item Let's see what it does differently from \funcname{raise\_notrace}\dots
          \item We are about to raise \typename{ExnC} with \funcname{caml\_raise\_exn} from \funcname{definitely\_raise}
        \end{itemize}
      }
      \onslide*<2>{
        \mintobjdump[firstline=2,highlightlines=14]{../src/backtrace/camlBacktrace\_\_definitely\_raise\_347.objdump}
      }
      \vfill
      \mintocaml[firstline=9,lastline=13,highlightlines=12]{../src/backtrace/backtrace.ml}
      \endminipage
    \end{column}
    \begin{column}{0.5\textwidth}
      \centering
      \providecommand\step{1}\input{diagrams/backtrace/backtrace.tex}
    \end{column}
  \end{columns}
\end{frame}

\begin{frame}{Backtraces}{But before that, we need more fields from \typename{caml\_domain\_state}}
  \begin{columns}
    \begin{column}{0.5\textwidth}
      \begin{itemize}
        \item Remember \typename{caml\_domain\_state} the per-domain runtime state?
        \item Some other members are of interest to us now\dots
        \item \localname{backtrace\_active} is used to know if the runtime should collect backtraces
        \item \localname{backtrace\_active} can be set by
          \begin{itemize}
            \item The \funcarg{b} flag in the \funcname{OCAMLRUNPARAM} environment variable
            \item \mintinline{ocaml}{Printexc.record_backtrace : bool -> unit} \footnotemark
          \end{itemize}
      \end{itemize}
    \end{column}
    \begin{column}{0.5\textwidth}
      \begin{listing}
        \mintc[highlightlines={9-11}]{src/ocaml/domain\_state\_backtrace.h}
        \caption{runtime/caml/domain\_state.h}
      \end{listing}
    \end{column}
  \end{columns}
  \footnotetext[1]{https://v2.ocaml.org/api/Printexc.html}
\end{frame}

\newcommand\listCamlRaiseExn[1][]{
  \listasm[firstline=702,lastline=725,#1]{../ocaml/runtime/amd64.S}{runtime/amd64.S:702}}

\begin{frame}{Backtraces}{Walk through \funcname{caml\_raise\_exn}}
  \begin{columns}[T]
    \begin{column}{0.5\textwidth}
      \begin{itemize}
        \item<1-> First checks whether exceptions backtrace should be recorded
        \item<1-> By checking the \funcarg{domain\_state->backtrace\_active} value
        \item<2-> If not, acts as \funcname{raise\_notrace} with \funcname{RESTORE\_EXN\_HANDLER\_OCAML}
          \begin{itemize}
            \item Set \regname{RSP} to \localname{domain\_state->exn\_handler} value
            \item Restore \localname{domain\_state->exn\_handler} from trap
            \item Jump with \funcname{ret} (same effect as using \funcname{pop} and \funcname{jmp})
          \end{itemize}
        \item<3-> Otherwise, switches to the C stack to be able to execute C code
        \item<4-> Calls \funcname{caml\_stash\_backtrace}, a C function provided by the runtime\ignorespaces
          \onslide<6->{, with
            \begin{itemize}
              \item \funcarg{exn}: exception value
              \item \funcarg{pc}: code address of the raising point
              \item \funcarg{sp}: stack address of the raising function
              \item \funcarg{trapsp}: stack address of the next handler
            \end{itemize}
          }
      \end{itemize}
      \bigskip
      \mintocaml[firstline=9,lastline=13,highlightlines=12]{../src/backtrace/backtrace.ml}
    \end{column}
    \begin{column}{0.5\textwidth}
      \raggedleft
      \onslide*<1,3-4>{
        \mintc[firstline=190,lastline=192]{../ocaml/runtime/amd64.S}
        \mintc[firstline=234,lastline=237]{../ocaml/runtime/amd64.S}
      }
      \onslide*<2>{
        \mintc[firstline=190,lastline=192,highlightlines={191-192}]{../ocaml/runtime/amd64.S}
        \mintc[firstline=234,lastline=237,highlightlines={235-237}]{../ocaml/runtime/amd64.S}
      }
      \onslide*<1>{\listCamlRaiseExn[highlightlines=705]{}}%
      \onslide*<2>{\listCamlRaiseExn[highlightlines={707-708}]{}}%
      \onslide*<3>{\listCamlRaiseExn[highlightlines={712-714}]{}}%
      \onslide*<4>{\listCamlRaiseExn[highlightlines={715-720}]{}}%
      \onslide*<5>{\providecommand\step{1}\input{diagrams/backtrace/backtrace.tex}}%
      \onslide*<6>{\providecommand\step{2}\input{diagrams/backtrace/backtrace.tex}}%
    \end{column}
  \end{columns}
\end{frame}

\newcommand\listStashBacktrace[1][]{
  \listc[firstline=91,lastline=120,#1]{../ocaml/runtime/backtrace\_nat.c}{runtime/backtrace\_nat.c:91}}

\begin{frame}{Backtraces}{Introducing \funcname{caml\_stash\_backtrace}}
  \begin{columns}[c]
    \begin{column}{0.5\textwidth}
      \minipage[c][0.66\textheight][s]{\columnwidth}
      \begin{itemize}
        \item<1-> A C function provided by the runtime (specific to native code)
        \item<2-> It scans the stack, between the stack pointer at the raising point up to the next trap, in order to collect the names of the detected function frames
          \onslide*<2>{
            \begin{itemize}
              \item And more information, like line number, column number...
            \end{itemize}
          }
        \item<3-> It uses the same technique and information as the Garbage Collector (GC) uses to scan the values live on the stack
          \onslide*<4>{
            \begin{itemize}
              \item \typename{frame\_descr} are at the core of stack unwinding
            \end{itemize}
          }
      \end{itemize}
      \vfill
      \mintocaml[firstline=9,lastline=13,highlightlines=12]{../src/backtrace/backtrace.ml}
      \endminipage
    \end{column}
    \begin{column}{0.5\textwidth}
      \onslide*<1>{\listStashBacktrace[highlightlines=91]{}}
      \onslide*<2>{\listStashBacktrace[highlightlines={91,117-118}]{}}
      \onslide*<3->{\listStashBacktrace[highlightlines={94,106,109-110}]{}}
    \end{column}
  \end{columns}
\end{frame}

\newcommand\listFrameDescr[1][]{
  \listocaml[firstline=111,lastline=124,#1]{../ocaml/asmcomp/emitaux.ml}{asmcomp/emitaux.ml:111}}
\newcommand\listDebugInfo[1][]{
  \listocaml[firstline=38,lastline=49,#1]{../ocaml/lambda/debuginfo.mli}{lambda/debuginfo.mli:38}}

\begin{frame}{Backtraces}{\typename{frame\_descr} and \typename{frame\_debuginfo} in a nutshell}
  \begin{columns}[c]
    \begin{column}{0.5\textwidth}
      \begin{itemize}
        \item<1-> For every return address in an OCaml program, there is a associated \typename{frame\_descr}
        \item<2-> A \typename{frame\_descr} specifies
        \begin{itemize}
          \item<2-> The size of the function stack frame
          \item<3-> Where live values are on the stack (so that the GC can mark them as live and prevent from being swept)
        \end{itemize}
        \item<4-> When compiled with debug information, \typename{frame\_descr} will be linked to a \typename{frame\_debuginfo}
        \begin{itemize}
          \item<5-> Contains information about the position in the source code (file name, line and column number)
        \end{itemize}
      \end{itemize}
    \end{column}
    \begin{column}{0.5\textwidth}
        \onslide*<1>{\listFrameDescr[highlightlines={118-122}]{}}
        \onslide*<2>{\listFrameDescr[highlightlines={120}]{}}
        \onslide*<3>{\listFrameDescr[highlightlines={121}]{}}
        \onslide*<4->{\listFrameDescr[highlightlines={122}]{}}
        \medskip
        \onslide*<1-3>{\listDebugInfo{}}
        \onslide*<4>{\listDebugInfo[highlightlines={38,49}]{}}
        \onslide*<5>{\listDebugInfo[highlightlines={39-46}]{}}
    \end{column}
  \end{columns}
\end{frame}

\begin{frame}{Backtraces}{Scanning the stack with \typename{frame\_descr}}
  \begin{columns}[c]
    \begin{column}{0.5\textwidth}
      \begin{itemize}
        \item Given the return address of the code that raises the exception by calling \funcname{caml\_raise\_exn} (\funcarg{pc} argument of \funcname{caml\_stash\_backtrace})
        \item And the position of the stack pointer (\funcarg{sp} argument of \funcname{caml\_stash\_backtrace})
        \item The frame size given by \typename{frame\_descr}.\funcarg{frame\_size}, allows to find the end of the previous frame
        \item And the return address of the caller of this function
        \bigskip
        \onslide<3->{
        \item Note that traps (here the reddish \funcname{ExnA}, \funcname{ExnB} and \funcname{ExnC} blocks) belong to the frame that installed them, and so the frame size changes when traps are removed
        }
      \end{itemize}
    \end{column}
    \begin{column}{0.5\textwidth}
      \raggedleft
      \onslide*<1>{\providecommand\step{2}\input{diagrams/backtrace/backtrace.tex}}%
      \onslide*<2>{\providecommand\step{1}\input{diagrams/backtrace/scanstack.tex}}%
      \onslide*<3>{\providecommand\step{2}\input{diagrams/backtrace/scanstack.tex}}%
      \onslide*<4>{\providecommand\step{3}\input{diagrams/backtrace/scanstack.tex}}%
      \onslide*<5>{\providecommand\step{4}\input{diagrams/backtrace/scanstack.tex}}%
      \onslide*<6>{\providecommand\step{5}\input{diagrams/backtrace/scanstack.tex}}%
    \end{column}
  \end{columns}
\end{frame}

% Duplicate of previous-previous slide
\begin{frame}{Backtraces}{Continuing on \funcname{caml\_stash\_backtrace}}
  \begin{columns}[c]
    \begin{column}{0.5\textwidth}
      \minipage[c][0.75\textheight][s]{\columnwidth}
      \begin{itemize}
          \color{gray}
        \item A C function provided by the runtime (specific to native code)
        \item It scans the stack, between the stack pointer at the raising point up to the next trap, in order to collect the names of the detected function frames
        \item It uses the same technique and information as the Garbage Collector (GC) uses to scan the values live on the stack
      \end{itemize}
      \begin{itemize}
        \item For each function frame detected, store a pointer to the \typename{frame\_descr} in the \localname{domain\_state->backtrace\_buffer} array, effectively constructing the backtrace
      \end{itemize}
      \onslide<2->{
        \begin{itemize}
            \smallskip
          \item Constructed backtrace:
            \funcname{definitely\_raise},
            \onslide<3->{\funcname{probably\_raise}}
        \end{itemize}
      }
      \vfill
      \mintocaml[firstline=9,lastline=13,highlightlines=12]{../src/backtrace/backtrace.ml}
      \endminipage
    \end{column}
    \begin{column}{0.5\textwidth}
      \raggedleft
      \onslide*<1>{\listStashBacktrace[highlightlines={114-115}]{}}
      \onslide*<2>{\providecommand\step{3}\input{diagrams/backtrace/backtrace.tex}}%
      \onslide*<3>{\providecommand\step{4}\input{diagrams/backtrace/backtrace.tex}}%
    \end{column}
  \end{columns}
\end{frame}

\begin{frame}{Backtraces}{Back to \funcname{caml\_raise\_exn}}
  \begin{columns}[c]
    \begin{column}{0.5\textwidth}
      \minipage[c][0.66\textheight][s]{\columnwidth}
      \begin{itemize}\color{gray}
        \item Checks wheteher exceptions backtrace should be recorded, by checking the \funcarg{domain\_state->backtrace\_active} value
        \item Switches to the C stack to be able to execute C code
        \item Calls \funcname{caml\_stash\_backtrace}, to collect the backtrace up to the next trap
      \end{itemize}
      \onslide*<2->{
        \begin{itemize}
          \item<2-> Executes the exception handler in \funcname{probably\_raise} with \funcname{RESTORE\_EXN\_HANDLER\_OCAML}
            \begin{itemize}
              \item Set \regname{RSP} to \localname{domain\_state->exn\_handler} value
              \item Restore \localname{domain\_state->exn\_handler} from trap
              \item Jump to the handler code with \funcname{ret}
            \end{itemize}
        \end{itemize}
      }
      \vfill
      \onslide*<1>{\mintocaml[firstline=9,lastline=13,highlightlines=12]{../src/backtrace/backtrace.ml}}
      \onslide*<2->{\mintocaml[firstline=15,lastline=21,highlightlines={19-21}]{../src/backtrace/backtrace.ml}}
      \endminipage
    \end{column}
    \begin{column}{0.5\textwidth}
      \centering
      \onslide*<1>{
        \mintc[firstline=190,lastline=192]{../ocaml/runtime/amd64.S}
        \mintc[firstline=234,lastline=237]{../ocaml/runtime/amd64.S}
      }
      \onslide*<2>{
        \mintc[firstline=190,lastline=192,highlightlines={191-192}]{../ocaml/runtime/amd64.S}
        \mintc[firstline=234,lastline=237,highlightlines={235-237}]{../ocaml/runtime/amd64.S}
      }
      \onslide*<1>{\listCamlRaiseExn[highlightlines=720]{}}
      \onslide*<2>{\listCamlRaiseExn[highlightlines=722]{}}
      \onslide*<3->{\providecommand\step{5}\input{diagrams/backtrace/backtrace.tex}}
    \end{column}
  \end{columns}
\end{frame}

\begin{frame}{Backtraces}{\funcname{caml\_probably\_raise}'s handler doesn't catch \typename{ExnC}}
  \begin{columns}[c]
    \begin{column}{0.45\textwidth}
      \minipage[c][0.75\textheight][s]{\columnwidth}
      \begin{itemize}
        \item<1-> \funcname{match \dots with}\footnotemark pattern matching can also match exceptions
        \item<1-> The CMM of \funcname{probably\_raise} shows that this \funcname{match \dots with} is implemented with a pattern matching and a \funcname{try \dots with}
        \item<1-> But this one only matches on \typename{ExnA} exception
        \item<2-> Since the pattern matching fails to catch the exception, \funcname{reraise} is called with our current \typename{ExnC} exception
        \item<3-> Disassembly shows that \funcname{reraise} is actually a call to \funcname{caml\_reraise\_exn}, which is also an assembly function provided by the runtime
      \end{itemize}
      \vfill
      \mintocaml[firstline=15,lastline=21,highlightlines={18-21}]{../src/backtrace/backtrace.ml}
      \endminipage
    \end{column}
    \begin{column}{0.55\textwidth}
      \centering
      \onslide*<1>{\mintcmm[highlightlines={10-18}]{../src/backtrace/camlBacktrace\_\_probably\_raise\_351.cmm}}
      \onslide*<2>{\listcmm[highlightlines=18]{../src/backtrace/camlBacktrace\_\_probably\_raise_351.cmm}{CMM of probably\_raise}}
      \onslide*<3>{\mintobjdump[firstline=5,lastline=35,highlightlines=31]{../src/backtrace/camlBacktrace\_\_probably\_raise_351.objdump}}
    \end{column}
  \end{columns}
  \only<1>{\footnotetext[1]{https://blog.janestreet.com/pattern-matching-and-exception-handling-unite/}}
\end{frame}

\newcommand\listCamlReraiseExn[1][]{
  \listasm[firstline=727,lastline=734,#1]{../ocaml/runtime/amd64.S}{runtime/amd64.S:727}}

\begin{frame}{Backtraces}{Introducing \funcname{caml\_reraise\_exn}}
  \begin{columns}[c]
    \begin{column}{0.5\textwidth}
      \minipage[c][0.66\textheight][s]{\columnwidth}
      \begin{itemize}
        \item<1-> The exception have to be reraised to the next handler
        \item<2-> First, checks if the backtrace should be collected with \funcarg{domain\_state->backtrace\_active}
        \only<3>{
        \begin{itemize}
            \item If not, behaves like a \funcname{raise\_notrace} with \funcname{RESTORE\_EXN\_HANDLER\_OCAML}
        \end{itemize}
        }
        \item<4-> Jumps into \funcname{caml\_raise\_exn}
        \item<5-> Calls \funcname{caml\_stash\_backtrace} again, to scan the next part of the stack: from the trap just pop-ed to the next trap
      \end{itemize}
      \vfill
      \mintocaml[firstline=15,lastline=21,highlightlines={18-21}]{../src/backtrace/backtrace.ml}
      \endminipage
    \end{column}
    \begin{column}{0.5\textwidth}
      \raggedleft
      \onslide*<1>{\listCamlReraiseExn{}}
      \onslide*<2>{\listCamlReraiseExn[highlightlines=729]{}}
      \onslide*<3>{
        \mintc[firstline=190,lastline=192,highlightlines={191-192}]{../ocaml/runtime/amd64.S}
        \mintc[firstline=234,lastline=237,highlightlines={235-237}]{../ocaml/runtime/amd64.S}
        \medskip
        \listCamlReraiseExn[highlightlines=731]{}
      }
      \onslide*<4-5>{
        % TODO refactor with \listCamlReraiseExn
        \mintasm[firstline=727,lastline=734,highlightlines=730]{../ocaml/runtime/amd64.S}
        \medskip
      }
      \onslide*<4>{\listCamlRaiseExn[highlightlines=711]{}}
      \onslide*<5>{\listCamlRaiseExn[highlightlines=720]{}}
    \end{column}
  \end{columns}
\end{frame}

\begin{frame}{Backtraces}{Backtrace collection continues}
  \begin{columns}[c]
    \begin{column}{0.55\textwidth}
      \minipage[c][0.75\textheight][s]{\columnwidth}
      \begin{itemize}
        \item<1-> \funcname{caml\_stash\_backtrace} scans the older part of the stack, up to the next trap
        \item<6-> Controls jumps to the nested exception handler in \funcname{maybe\_raise}, which matches only on \typename{ExnB}
        \item<7-> \funcname{caml\_reraise\_exn} is called again, backtrace collection resumes with \funcname{caml\_stash\_backtrace}
      \end{itemize}
      \medskip
      \begin{itemize}
        \item<2-> Constructed backtrace:
        \begin{itemize}
          \item <2-> \funcname{definitely\_raise}
          \item <2-> \funcname{probably\_raise}
          \item <3-> \funcname{probably\_raise} (re-raise)
          \item <4-> \funcname{maybe\_raise}
          \item <5-> \funcname{main}
          \item <8-> \funcname{main} (re-raise)
        \end{itemize}
      \end{itemize}
      \vfill
      \onslide*<1-5>{\mintocaml[firstline=15,lastline=21]{../src/backtrace/backtrace.ml}}
      \onslide*<6>{\mintocaml[firstline=28,lastline=34,highlightlines=31]{../src/backtrace/backtrace.ml}}
      \onslide*<7->{\mintocaml[firstline=28,lastline=34,highlightlines=33]{../src/backtrace/backtrace.ml}}
      \endminipage
    \end{column}
    \begin{column}{0.45\textwidth}
      \raggedleft
      \onslide*<1>{\providecommand\step{6}\input{diagrams/backtrace/backtrace.tex}}%
      \onslide*<2>{\providecommand\step{6}\input{diagrams/backtrace/backtrace.tex}}%
      \onslide*<3>{\providecommand\step{7}\input{diagrams/backtrace/backtrace.tex}}%
      \onslide*<4>{\providecommand\step{8}\input{diagrams/backtrace/backtrace.tex}}%
      \onslide*<5>{\providecommand\step{9}\input{diagrams/backtrace/backtrace.tex}}%
      \onslide*<6>{\providecommand\step{10}\input{diagrams/backtrace/backtrace.tex}}%
      \onslide*<7>{\providecommand\step{11}\input{diagrams/backtrace/backtrace.tex}}%
      \onslide*<8>{\providecommand\step{12}\input{diagrams/backtrace/backtrace.tex}}%
      \onslide*<9>{\providecommand\step{13}\input{diagrams/backtrace/backtrace.tex}}%
    \end{column}
  \end{columns}
\end{frame}

\begin{frame}{Backtraces}{Printing the backtrace}
  \begin{columns}[c]
    \begin{column}{0.45\textwidth}
      \minipage[c][0.66\textheight][s]{\columnwidth}
      \begin{itemize}
        \item<1-> The exception backtrace can now be printed to \funcarg{stdout} with \funcname{Printexc.print_backtrace}
        \item<2-> First the exception backtrace is retrieved into an OCaml array with \funcname{get_raw_backtrace }
        \item<3-> Which is implemented as the \funcname{caml_get_exception_raw_backtrace} C function in the runtime
        \item<4-> And finally printed with \funcname{print_exception_backtrace}
      \end{itemize}
      \vfill
      \mintocaml[firstline=28,lastline=34,highlightlines=33]{../src/backtrace/backtrace.ml}
      \endminipage
    \end{column}
    \begin{column}{0.55\textwidth}
      \onslide*<1,2>{
        \mintocaml[firstline=100,lastline=101]{../ocaml/stdlib/printexc.ml}
      }
      \only<1>{
        \listocaml[firstline=170,lastline=172]{../ocaml/stdlib/printexc.ml}{stdlib/printexc.ml:170}
      }
      \only<2>{
        \listocaml[firstline=170,lastline=172,highlightlines=172]{../ocaml/stdlib/printexc.ml}{stdlib/printexc.ml:170}
      }
      \only<3>{
        \mintc[firstline=154,lastline=156]{../ocaml/runtime/backtrace.c}
        \listc[firstline=171,lastline=192,highlightlines={184-187}]{../ocaml/runtime/backtrace.c}{runtime/backtrace.c:154}
      }
      \only<4>{
        \listocaml[firstline=155,lastline=165]{../ocaml/stdlib/printexc.ml}{stdlib/printexc.ml:55}
      }
    \end{column}
  \end{columns}
\end{frame}


\subsection{The multiple kinds of raise}
\frameSubsection{}{}

\begin{frame}{Backtraces}{When backtraces are not needed}
  \begin{columns}[c]
    \begin{column}{0.45\textwidth}
      \minipage[c][0.66\textheight][s]{\columnwidth}
      \begin{itemize}
        \item<1-> What if the backtrace for an exception is never needed?
        \item<2-> It's possible to raise the exception explicitly with \funcname{raise\_notrace} to make raising as cheap as possible
        \item<3-> An example of use in the \funcname{dynlink} library of the OCaml compiler
      \end{itemize}
      \vfill
      \onslide<3->{
      \listocaml[firstline=205,lastline=209,highlightlines=207]{../ocaml/otherlibs/dynlink/dynlink\_compilerlibs/misc.ml}{otherlibs/dynlink/dynlink\_compilerlibs/misc.ml:205}
      }
      \endminipage
    \end{column}
    \begin{column}{0.55\textwidth}
    \only<4>{
      \mintcmm[highlightlines=3]{../src/notrace/camlNotrace\_\_fun_347.cmm}
      \listcmm{../src/notrace/camlNotrace\_\_all\_somes\_327.cmm}{all\_somes}
    }
    \onslide*<5->{
      \listobjdump[highlightlines={6-9}]{../src/notrace/camlNotrace\_\_fun_347.objdump}{anonymous function from \funcname{all\_somes}}
    }
    \end{column}
  \end{columns}
\end{frame}

\begin{frame}{Backtraces}{Summary table of the multiple kinds of raise}
  \begin{itemize}
    \item When debugging information is enabled and if the backtrace of an exception isn't needed, it's possible to raise with \funcname{raise\_notrace} for performance
    \item When debugging information is \emph{not} enabled, the compiler optimises \funcname{raise} calls to inline assembly (same effect as using \funcname{raise\_notrace})
  \end{itemize}
  \bigskip
  \centering
  \begin{tabular}{| l | c | c | c | c |}
    \hline
    \parbox{10em}{Debug information \\ (\funcarg{-g} flag)}  & \multicolumn{2}{|c|}{Disabled}                                     & \multicolumn{2}{|c|}{Enabled} \\
    \hline
    Raise kind                                               & \funcname{raise} & \funcname{raise\_notrace}                       & \funcname{raise} & \funcname{raise\_notrace} \\
    \hline
    Compilation                                              & \parbox{8em}{inline assembly\\ (optimization)} & inline assembly   & \funcname{caml\_raise\_exn} & inline assembly \\
    \hline
  \end{tabular}
\end{frame}


%
% Takeaway #5
%
\subsection*{Takeaway \#5}
\frameSubsectionTakeaway{}
\againframe<5>{takeaway}
}%
      \onslide*<3>{\providecommand\step{4}%
% Backtraces
%
\subsection{One advantage of raising with debugging information}
\frameSubsection{}{}

\begin{frame}{Backtraces}{Debugging information \& source code}
  \begin{columns}[c]
    \begin{column}{0.5\textwidth}
      \begin{itemize}
        \item Raising an exception is fun and games, but how do we know where the exception originated? $\rightarrow$ With the backtrace associated with the exception!
        \item In order to get a backtrace from the point where an exception was raised, it's necessary to compile with debugging information enabled (\funcarg{-g} flag for the compiler)
        \item Same example as in the `Nested handlers' section
      \end{itemize}
    \end{column}
    \begin{column}{0.5\textwidth}
      \listocaml[firstline=9,lastline=34]{../src/backtrace/backtrace.ml}{backtrace/backtrace.ml}
    \end{column}
  \end{columns}
\end{frame}

\begin{frame}{Backtraces}{CMM and disassembly of \funcname{definitely\_raise}}
  \begin{columns}[c]
    \begin{column}{0.5\textwidth}
      \minipage[c][0.66\textheight][s]{\columnwidth}
      \begin{itemize}
        \item<1-> An effect of compiling with debugging information (\funcarg{-g}): the CMM no longer uses \funcname{raise\_notrace} to raise the exception but \funcname{raise} instead
        \item<2-> Disassembly shows that CMM \funcname{raise} is actually a call to \funcname{caml\_raise\_exn}, a function in assembler provided by the runtime
      \end{itemize}
      \vfill
      \mintocaml[firstline=9,lastline=13,highlightlines=12]{../src/backtrace/backtrace.ml}
      \endminipage
    \end{column}
    \begin{column}{0.5\textwidth}
      \onslide*<1>{
        \mintcmm[highlightlines=5]{../src/backtrace/camlBacktrace\_\_definitely\_raise\_347.cmm}
      }
      \onslide*<2>{
        \mintobjdump[firstline=2,highlightlines=14]{../src/backtrace/camlBacktrace\_\_definitely\_raise\_347.objdump}
      }
    \end{column}
  \end{columns}
\end{frame}

\begin{frame}{Backtraces}{Introducing \funcname{caml\_raise\_exn}}
  \begin{columns}[c]
    \begin{column}{0.5\textwidth}
      \minipage[c][0.66\textheight][s]{\columnwidth}
      \onslide*<1>{
        \begin{itemize}
          \item Raising the exception is performed using \funcname{caml\_raise\_exn} which is
            \begin{itemize}
              \item An assembly function provided by the runtime
              \item Architecture specific
            \end{itemize}
          \item Let's see what it does differently from \funcname{raise\_notrace}\dots
          \item We are about to raise \typename{ExnC} with \funcname{caml\_raise\_exn} from \funcname{definitely\_raise}
        \end{itemize}
      }
      \onslide*<2>{
        \mintobjdump[firstline=2,highlightlines=14]{../src/backtrace/camlBacktrace\_\_definitely\_raise\_347.objdump}
      }
      \vfill
      \mintocaml[firstline=9,lastline=13,highlightlines=12]{../src/backtrace/backtrace.ml}
      \endminipage
    \end{column}
    \begin{column}{0.5\textwidth}
      \centering
      \providecommand\step{1}\input{diagrams/backtrace/backtrace.tex}
    \end{column}
  \end{columns}
\end{frame}

\begin{frame}{Backtraces}{But before that, we need more fields from \typename{caml\_domain\_state}}
  \begin{columns}
    \begin{column}{0.5\textwidth}
      \begin{itemize}
        \item Remember \typename{caml\_domain\_state} the per-domain runtime state?
        \item Some other members are of interest to us now\dots
        \item \localname{backtrace\_active} is used to know if the runtime should collect backtraces
        \item \localname{backtrace\_active} can be set by
          \begin{itemize}
            \item The \funcarg{b} flag in the \funcname{OCAMLRUNPARAM} environment variable
            \item \mintinline{ocaml}{Printexc.record_backtrace : bool -> unit} \footnotemark
          \end{itemize}
      \end{itemize}
    \end{column}
    \begin{column}{0.5\textwidth}
      \begin{listing}
        \mintc[highlightlines={9-11}]{src/ocaml/domain\_state\_backtrace.h}
        \caption{runtime/caml/domain\_state.h}
      \end{listing}
    \end{column}
  \end{columns}
  \footnotetext[1]{https://v2.ocaml.org/api/Printexc.html}
\end{frame}

\newcommand\listCamlRaiseExn[1][]{
  \listasm[firstline=702,lastline=725,#1]{../ocaml/runtime/amd64.S}{runtime/amd64.S:702}}

\begin{frame}{Backtraces}{Walk through \funcname{caml\_raise\_exn}}
  \begin{columns}[T]
    \begin{column}{0.5\textwidth}
      \begin{itemize}
        \item<1-> First checks whether exceptions backtrace should be recorded
        \item<1-> By checking the \funcarg{domain\_state->backtrace\_active} value
        \item<2-> If not, acts as \funcname{raise\_notrace} with \funcname{RESTORE\_EXN\_HANDLER\_OCAML}
          \begin{itemize}
            \item Set \regname{RSP} to \localname{domain\_state->exn\_handler} value
            \item Restore \localname{domain\_state->exn\_handler} from trap
            \item Jump with \funcname{ret} (same effect as using \funcname{pop} and \funcname{jmp})
          \end{itemize}
        \item<3-> Otherwise, switches to the C stack to be able to execute C code
        \item<4-> Calls \funcname{caml\_stash\_backtrace}, a C function provided by the runtime\ignorespaces
          \onslide<6->{, with
            \begin{itemize}
              \item \funcarg{exn}: exception value
              \item \funcarg{pc}: code address of the raising point
              \item \funcarg{sp}: stack address of the raising function
              \item \funcarg{trapsp}: stack address of the next handler
            \end{itemize}
          }
      \end{itemize}
      \bigskip
      \mintocaml[firstline=9,lastline=13,highlightlines=12]{../src/backtrace/backtrace.ml}
    \end{column}
    \begin{column}{0.5\textwidth}
      \raggedleft
      \onslide*<1,3-4>{
        \mintc[firstline=190,lastline=192]{../ocaml/runtime/amd64.S}
        \mintc[firstline=234,lastline=237]{../ocaml/runtime/amd64.S}
      }
      \onslide*<2>{
        \mintc[firstline=190,lastline=192,highlightlines={191-192}]{../ocaml/runtime/amd64.S}
        \mintc[firstline=234,lastline=237,highlightlines={235-237}]{../ocaml/runtime/amd64.S}
      }
      \onslide*<1>{\listCamlRaiseExn[highlightlines=705]{}}%
      \onslide*<2>{\listCamlRaiseExn[highlightlines={707-708}]{}}%
      \onslide*<3>{\listCamlRaiseExn[highlightlines={712-714}]{}}%
      \onslide*<4>{\listCamlRaiseExn[highlightlines={715-720}]{}}%
      \onslide*<5>{\providecommand\step{1}\input{diagrams/backtrace/backtrace.tex}}%
      \onslide*<6>{\providecommand\step{2}\input{diagrams/backtrace/backtrace.tex}}%
    \end{column}
  \end{columns}
\end{frame}

\newcommand\listStashBacktrace[1][]{
  \listc[firstline=91,lastline=120,#1]{../ocaml/runtime/backtrace\_nat.c}{runtime/backtrace\_nat.c:91}}

\begin{frame}{Backtraces}{Introducing \funcname{caml\_stash\_backtrace}}
  \begin{columns}[c]
    \begin{column}{0.5\textwidth}
      \minipage[c][0.66\textheight][s]{\columnwidth}
      \begin{itemize}
        \item<1-> A C function provided by the runtime (specific to native code)
        \item<2-> It scans the stack, between the stack pointer at the raising point up to the next trap, in order to collect the names of the detected function frames
          \onslide*<2>{
            \begin{itemize}
              \item And more information, like line number, column number...
            \end{itemize}
          }
        \item<3-> It uses the same technique and information as the Garbage Collector (GC) uses to scan the values live on the stack
          \onslide*<4>{
            \begin{itemize}
              \item \typename{frame\_descr} are at the core of stack unwinding
            \end{itemize}
          }
      \end{itemize}
      \vfill
      \mintocaml[firstline=9,lastline=13,highlightlines=12]{../src/backtrace/backtrace.ml}
      \endminipage
    \end{column}
    \begin{column}{0.5\textwidth}
      \onslide*<1>{\listStashBacktrace[highlightlines=91]{}}
      \onslide*<2>{\listStashBacktrace[highlightlines={91,117-118}]{}}
      \onslide*<3->{\listStashBacktrace[highlightlines={94,106,109-110}]{}}
    \end{column}
  \end{columns}
\end{frame}

\newcommand\listFrameDescr[1][]{
  \listocaml[firstline=111,lastline=124,#1]{../ocaml/asmcomp/emitaux.ml}{asmcomp/emitaux.ml:111}}
\newcommand\listDebugInfo[1][]{
  \listocaml[firstline=38,lastline=49,#1]{../ocaml/lambda/debuginfo.mli}{lambda/debuginfo.mli:38}}

\begin{frame}{Backtraces}{\typename{frame\_descr} and \typename{frame\_debuginfo} in a nutshell}
  \begin{columns}[c]
    \begin{column}{0.5\textwidth}
      \begin{itemize}
        \item<1-> For every return address in an OCaml program, there is a associated \typename{frame\_descr}
        \item<2-> A \typename{frame\_descr} specifies
        \begin{itemize}
          \item<2-> The size of the function stack frame
          \item<3-> Where live values are on the stack (so that the GC can mark them as live and prevent from being swept)
        \end{itemize}
        \item<4-> When compiled with debug information, \typename{frame\_descr} will be linked to a \typename{frame\_debuginfo}
        \begin{itemize}
          \item<5-> Contains information about the position in the source code (file name, line and column number)
        \end{itemize}
      \end{itemize}
    \end{column}
    \begin{column}{0.5\textwidth}
        \onslide*<1>{\listFrameDescr[highlightlines={118-122}]{}}
        \onslide*<2>{\listFrameDescr[highlightlines={120}]{}}
        \onslide*<3>{\listFrameDescr[highlightlines={121}]{}}
        \onslide*<4->{\listFrameDescr[highlightlines={122}]{}}
        \medskip
        \onslide*<1-3>{\listDebugInfo{}}
        \onslide*<4>{\listDebugInfo[highlightlines={38,49}]{}}
        \onslide*<5>{\listDebugInfo[highlightlines={39-46}]{}}
    \end{column}
  \end{columns}
\end{frame}

\begin{frame}{Backtraces}{Scanning the stack with \typename{frame\_descr}}
  \begin{columns}[c]
    \begin{column}{0.5\textwidth}
      \begin{itemize}
        \item Given the return address of the code that raises the exception by calling \funcname{caml\_raise\_exn} (\funcarg{pc} argument of \funcname{caml\_stash\_backtrace})
        \item And the position of the stack pointer (\funcarg{sp} argument of \funcname{caml\_stash\_backtrace})
        \item The frame size given by \typename{frame\_descr}.\funcarg{frame\_size}, allows to find the end of the previous frame
        \item And the return address of the caller of this function
        \bigskip
        \onslide<3->{
        \item Note that traps (here the reddish \funcname{ExnA}, \funcname{ExnB} and \funcname{ExnC} blocks) belong to the frame that installed them, and so the frame size changes when traps are removed
        }
      \end{itemize}
    \end{column}
    \begin{column}{0.5\textwidth}
      \raggedleft
      \onslide*<1>{\providecommand\step{2}\input{diagrams/backtrace/backtrace.tex}}%
      \onslide*<2>{\providecommand\step{1}\input{diagrams/backtrace/scanstack.tex}}%
      \onslide*<3>{\providecommand\step{2}\input{diagrams/backtrace/scanstack.tex}}%
      \onslide*<4>{\providecommand\step{3}\input{diagrams/backtrace/scanstack.tex}}%
      \onslide*<5>{\providecommand\step{4}\input{diagrams/backtrace/scanstack.tex}}%
      \onslide*<6>{\providecommand\step{5}\input{diagrams/backtrace/scanstack.tex}}%
    \end{column}
  \end{columns}
\end{frame}

% Duplicate of previous-previous slide
\begin{frame}{Backtraces}{Continuing on \funcname{caml\_stash\_backtrace}}
  \begin{columns}[c]
    \begin{column}{0.5\textwidth}
      \minipage[c][0.75\textheight][s]{\columnwidth}
      \begin{itemize}
          \color{gray}
        \item A C function provided by the runtime (specific to native code)
        \item It scans the stack, between the stack pointer at the raising point up to the next trap, in order to collect the names of the detected function frames
        \item It uses the same technique and information as the Garbage Collector (GC) uses to scan the values live on the stack
      \end{itemize}
      \begin{itemize}
        \item For each function frame detected, store a pointer to the \typename{frame\_descr} in the \localname{domain\_state->backtrace\_buffer} array, effectively constructing the backtrace
      \end{itemize}
      \onslide<2->{
        \begin{itemize}
            \smallskip
          \item Constructed backtrace:
            \funcname{definitely\_raise},
            \onslide<3->{\funcname{probably\_raise}}
        \end{itemize}
      }
      \vfill
      \mintocaml[firstline=9,lastline=13,highlightlines=12]{../src/backtrace/backtrace.ml}
      \endminipage
    \end{column}
    \begin{column}{0.5\textwidth}
      \raggedleft
      \onslide*<1>{\listStashBacktrace[highlightlines={114-115}]{}}
      \onslide*<2>{\providecommand\step{3}\input{diagrams/backtrace/backtrace.tex}}%
      \onslide*<3>{\providecommand\step{4}\input{diagrams/backtrace/backtrace.tex}}%
    \end{column}
  \end{columns}
\end{frame}

\begin{frame}{Backtraces}{Back to \funcname{caml\_raise\_exn}}
  \begin{columns}[c]
    \begin{column}{0.5\textwidth}
      \minipage[c][0.66\textheight][s]{\columnwidth}
      \begin{itemize}\color{gray}
        \item Checks wheteher exceptions backtrace should be recorded, by checking the \funcarg{domain\_state->backtrace\_active} value
        \item Switches to the C stack to be able to execute C code
        \item Calls \funcname{caml\_stash\_backtrace}, to collect the backtrace up to the next trap
      \end{itemize}
      \onslide*<2->{
        \begin{itemize}
          \item<2-> Executes the exception handler in \funcname{probably\_raise} with \funcname{RESTORE\_EXN\_HANDLER\_OCAML}
            \begin{itemize}
              \item Set \regname{RSP} to \localname{domain\_state->exn\_handler} value
              \item Restore \localname{domain\_state->exn\_handler} from trap
              \item Jump to the handler code with \funcname{ret}
            \end{itemize}
        \end{itemize}
      }
      \vfill
      \onslide*<1>{\mintocaml[firstline=9,lastline=13,highlightlines=12]{../src/backtrace/backtrace.ml}}
      \onslide*<2->{\mintocaml[firstline=15,lastline=21,highlightlines={19-21}]{../src/backtrace/backtrace.ml}}
      \endminipage
    \end{column}
    \begin{column}{0.5\textwidth}
      \centering
      \onslide*<1>{
        \mintc[firstline=190,lastline=192]{../ocaml/runtime/amd64.S}
        \mintc[firstline=234,lastline=237]{../ocaml/runtime/amd64.S}
      }
      \onslide*<2>{
        \mintc[firstline=190,lastline=192,highlightlines={191-192}]{../ocaml/runtime/amd64.S}
        \mintc[firstline=234,lastline=237,highlightlines={235-237}]{../ocaml/runtime/amd64.S}
      }
      \onslide*<1>{\listCamlRaiseExn[highlightlines=720]{}}
      \onslide*<2>{\listCamlRaiseExn[highlightlines=722]{}}
      \onslide*<3->{\providecommand\step{5}\input{diagrams/backtrace/backtrace.tex}}
    \end{column}
  \end{columns}
\end{frame}

\begin{frame}{Backtraces}{\funcname{caml\_probably\_raise}'s handler doesn't catch \typename{ExnC}}
  \begin{columns}[c]
    \begin{column}{0.45\textwidth}
      \minipage[c][0.75\textheight][s]{\columnwidth}
      \begin{itemize}
        \item<1-> \funcname{match \dots with}\footnotemark pattern matching can also match exceptions
        \item<1-> The CMM of \funcname{probably\_raise} shows that this \funcname{match \dots with} is implemented with a pattern matching and a \funcname{try \dots with}
        \item<1-> But this one only matches on \typename{ExnA} exception
        \item<2-> Since the pattern matching fails to catch the exception, \funcname{reraise} is called with our current \typename{ExnC} exception
        \item<3-> Disassembly shows that \funcname{reraise} is actually a call to \funcname{caml\_reraise\_exn}, which is also an assembly function provided by the runtime
      \end{itemize}
      \vfill
      \mintocaml[firstline=15,lastline=21,highlightlines={18-21}]{../src/backtrace/backtrace.ml}
      \endminipage
    \end{column}
    \begin{column}{0.55\textwidth}
      \centering
      \onslide*<1>{\mintcmm[highlightlines={10-18}]{../src/backtrace/camlBacktrace\_\_probably\_raise\_351.cmm}}
      \onslide*<2>{\listcmm[highlightlines=18]{../src/backtrace/camlBacktrace\_\_probably\_raise_351.cmm}{CMM of probably\_raise}}
      \onslide*<3>{\mintobjdump[firstline=5,lastline=35,highlightlines=31]{../src/backtrace/camlBacktrace\_\_probably\_raise_351.objdump}}
    \end{column}
  \end{columns}
  \only<1>{\footnotetext[1]{https://blog.janestreet.com/pattern-matching-and-exception-handling-unite/}}
\end{frame}

\newcommand\listCamlReraiseExn[1][]{
  \listasm[firstline=727,lastline=734,#1]{../ocaml/runtime/amd64.S}{runtime/amd64.S:727}}

\begin{frame}{Backtraces}{Introducing \funcname{caml\_reraise\_exn}}
  \begin{columns}[c]
    \begin{column}{0.5\textwidth}
      \minipage[c][0.66\textheight][s]{\columnwidth}
      \begin{itemize}
        \item<1-> The exception have to be reraised to the next handler
        \item<2-> First, checks if the backtrace should be collected with \funcarg{domain\_state->backtrace\_active}
        \only<3>{
        \begin{itemize}
            \item If not, behaves like a \funcname{raise\_notrace} with \funcname{RESTORE\_EXN\_HANDLER\_OCAML}
        \end{itemize}
        }
        \item<4-> Jumps into \funcname{caml\_raise\_exn}
        \item<5-> Calls \funcname{caml\_stash\_backtrace} again, to scan the next part of the stack: from the trap just pop-ed to the next trap
      \end{itemize}
      \vfill
      \mintocaml[firstline=15,lastline=21,highlightlines={18-21}]{../src/backtrace/backtrace.ml}
      \endminipage
    \end{column}
    \begin{column}{0.5\textwidth}
      \raggedleft
      \onslide*<1>{\listCamlReraiseExn{}}
      \onslide*<2>{\listCamlReraiseExn[highlightlines=729]{}}
      \onslide*<3>{
        \mintc[firstline=190,lastline=192,highlightlines={191-192}]{../ocaml/runtime/amd64.S}
        \mintc[firstline=234,lastline=237,highlightlines={235-237}]{../ocaml/runtime/amd64.S}
        \medskip
        \listCamlReraiseExn[highlightlines=731]{}
      }
      \onslide*<4-5>{
        % TODO refactor with \listCamlReraiseExn
        \mintasm[firstline=727,lastline=734,highlightlines=730]{../ocaml/runtime/amd64.S}
        \medskip
      }
      \onslide*<4>{\listCamlRaiseExn[highlightlines=711]{}}
      \onslide*<5>{\listCamlRaiseExn[highlightlines=720]{}}
    \end{column}
  \end{columns}
\end{frame}

\begin{frame}{Backtraces}{Backtrace collection continues}
  \begin{columns}[c]
    \begin{column}{0.55\textwidth}
      \minipage[c][0.75\textheight][s]{\columnwidth}
      \begin{itemize}
        \item<1-> \funcname{caml\_stash\_backtrace} scans the older part of the stack, up to the next trap
        \item<6-> Controls jumps to the nested exception handler in \funcname{maybe\_raise}, which matches only on \typename{ExnB}
        \item<7-> \funcname{caml\_reraise\_exn} is called again, backtrace collection resumes with \funcname{caml\_stash\_backtrace}
      \end{itemize}
      \medskip
      \begin{itemize}
        \item<2-> Constructed backtrace:
        \begin{itemize}
          \item <2-> \funcname{definitely\_raise}
          \item <2-> \funcname{probably\_raise}
          \item <3-> \funcname{probably\_raise} (re-raise)
          \item <4-> \funcname{maybe\_raise}
          \item <5-> \funcname{main}
          \item <8-> \funcname{main} (re-raise)
        \end{itemize}
      \end{itemize}
      \vfill
      \onslide*<1-5>{\mintocaml[firstline=15,lastline=21]{../src/backtrace/backtrace.ml}}
      \onslide*<6>{\mintocaml[firstline=28,lastline=34,highlightlines=31]{../src/backtrace/backtrace.ml}}
      \onslide*<7->{\mintocaml[firstline=28,lastline=34,highlightlines=33]{../src/backtrace/backtrace.ml}}
      \endminipage
    \end{column}
    \begin{column}{0.45\textwidth}
      \raggedleft
      \onslide*<1>{\providecommand\step{6}\input{diagrams/backtrace/backtrace.tex}}%
      \onslide*<2>{\providecommand\step{6}\input{diagrams/backtrace/backtrace.tex}}%
      \onslide*<3>{\providecommand\step{7}\input{diagrams/backtrace/backtrace.tex}}%
      \onslide*<4>{\providecommand\step{8}\input{diagrams/backtrace/backtrace.tex}}%
      \onslide*<5>{\providecommand\step{9}\input{diagrams/backtrace/backtrace.tex}}%
      \onslide*<6>{\providecommand\step{10}\input{diagrams/backtrace/backtrace.tex}}%
      \onslide*<7>{\providecommand\step{11}\input{diagrams/backtrace/backtrace.tex}}%
      \onslide*<8>{\providecommand\step{12}\input{diagrams/backtrace/backtrace.tex}}%
      \onslide*<9>{\providecommand\step{13}\input{diagrams/backtrace/backtrace.tex}}%
    \end{column}
  \end{columns}
\end{frame}

\begin{frame}{Backtraces}{Printing the backtrace}
  \begin{columns}[c]
    \begin{column}{0.45\textwidth}
      \minipage[c][0.66\textheight][s]{\columnwidth}
      \begin{itemize}
        \item<1-> The exception backtrace can now be printed to \funcarg{stdout} with \funcname{Printexc.print_backtrace}
        \item<2-> First the exception backtrace is retrieved into an OCaml array with \funcname{get_raw_backtrace }
        \item<3-> Which is implemented as the \funcname{caml_get_exception_raw_backtrace} C function in the runtime
        \item<4-> And finally printed with \funcname{print_exception_backtrace}
      \end{itemize}
      \vfill
      \mintocaml[firstline=28,lastline=34,highlightlines=33]{../src/backtrace/backtrace.ml}
      \endminipage
    \end{column}
    \begin{column}{0.55\textwidth}
      \onslide*<1,2>{
        \mintocaml[firstline=100,lastline=101]{../ocaml/stdlib/printexc.ml}
      }
      \only<1>{
        \listocaml[firstline=170,lastline=172]{../ocaml/stdlib/printexc.ml}{stdlib/printexc.ml:170}
      }
      \only<2>{
        \listocaml[firstline=170,lastline=172,highlightlines=172]{../ocaml/stdlib/printexc.ml}{stdlib/printexc.ml:170}
      }
      \only<3>{
        \mintc[firstline=154,lastline=156]{../ocaml/runtime/backtrace.c}
        \listc[firstline=171,lastline=192,highlightlines={184-187}]{../ocaml/runtime/backtrace.c}{runtime/backtrace.c:154}
      }
      \only<4>{
        \listocaml[firstline=155,lastline=165]{../ocaml/stdlib/printexc.ml}{stdlib/printexc.ml:55}
      }
    \end{column}
  \end{columns}
\end{frame}


\subsection{The multiple kinds of raise}
\frameSubsection{}{}

\begin{frame}{Backtraces}{When backtraces are not needed}
  \begin{columns}[c]
    \begin{column}{0.45\textwidth}
      \minipage[c][0.66\textheight][s]{\columnwidth}
      \begin{itemize}
        \item<1-> What if the backtrace for an exception is never needed?
        \item<2-> It's possible to raise the exception explicitly with \funcname{raise\_notrace} to make raising as cheap as possible
        \item<3-> An example of use in the \funcname{dynlink} library of the OCaml compiler
      \end{itemize}
      \vfill
      \onslide<3->{
      \listocaml[firstline=205,lastline=209,highlightlines=207]{../ocaml/otherlibs/dynlink/dynlink\_compilerlibs/misc.ml}{otherlibs/dynlink/dynlink\_compilerlibs/misc.ml:205}
      }
      \endminipage
    \end{column}
    \begin{column}{0.55\textwidth}
    \only<4>{
      \mintcmm[highlightlines=3]{../src/notrace/camlNotrace\_\_fun_347.cmm}
      \listcmm{../src/notrace/camlNotrace\_\_all\_somes\_327.cmm}{all\_somes}
    }
    \onslide*<5->{
      \listobjdump[highlightlines={6-9}]{../src/notrace/camlNotrace\_\_fun_347.objdump}{anonymous function from \funcname{all\_somes}}
    }
    \end{column}
  \end{columns}
\end{frame}

\begin{frame}{Backtraces}{Summary table of the multiple kinds of raise}
  \begin{itemize}
    \item When debugging information is enabled and if the backtrace of an exception isn't needed, it's possible to raise with \funcname{raise\_notrace} for performance
    \item When debugging information is \emph{not} enabled, the compiler optimises \funcname{raise} calls to inline assembly (same effect as using \funcname{raise\_notrace})
  \end{itemize}
  \bigskip
  \centering
  \begin{tabular}{| l | c | c | c | c |}
    \hline
    \parbox{10em}{Debug information \\ (\funcarg{-g} flag)}  & \multicolumn{2}{|c|}{Disabled}                                     & \multicolumn{2}{|c|}{Enabled} \\
    \hline
    Raise kind                                               & \funcname{raise} & \funcname{raise\_notrace}                       & \funcname{raise} & \funcname{raise\_notrace} \\
    \hline
    Compilation                                              & \parbox{8em}{inline assembly\\ (optimization)} & inline assembly   & \funcname{caml\_raise\_exn} & inline assembly \\
    \hline
  \end{tabular}
\end{frame}


%
% Takeaway #5
%
\subsection*{Takeaway \#5}
\frameSubsectionTakeaway{}
\againframe<5>{takeaway}
}%
    \end{column}
  \end{columns}
\end{frame}

\begin{frame}{Backtraces}{Back to \funcname{caml\_raise\_exn}}
  \begin{columns}[c]
    \begin{column}{0.5\textwidth}
      \minipage[c][0.66\textheight][s]{\columnwidth}
      \begin{itemize}\color{gray}
        \item Checks wheteher exceptions backtrace should be recorded, by checking the \funcarg{domain\_state->backtrace\_active} value
        \item Switches to the C stack to be able to execute C code
        \item Calls \funcname{caml\_stash\_backtrace}, to collect the backtrace up to the next trap
      \end{itemize}
      \onslide*<2->{
        \begin{itemize}
          \item<2-> Executes the exception handler in \funcname{probably\_raise} with \funcname{RESTORE\_EXN\_HANDLER\_OCAML}
            \begin{itemize}
              \item Set \regname{RSP} to \localname{domain\_state->exn\_handler} value
              \item Restore \localname{domain\_state->exn\_handler} from trap
              \item Jump to the handler code with \funcname{ret}
            \end{itemize}
        \end{itemize}
      }
      \vfill
      \onslide*<1>{\mintocaml[firstline=9,lastline=13,highlightlines=12]{../src/backtrace/backtrace.ml}}
      \onslide*<2->{\mintocaml[firstline=15,lastline=21,highlightlines={19-21}]{../src/backtrace/backtrace.ml}}
      \endminipage
    \end{column}
    \begin{column}{0.5\textwidth}
      \centering
      \onslide*<1>{
        \mintc[firstline=190,lastline=192]{../ocaml/runtime/amd64.S}
        \mintc[firstline=234,lastline=237]{../ocaml/runtime/amd64.S}
      }
      \onslide*<2>{
        \mintc[firstline=190,lastline=192,highlightlines={191-192}]{../ocaml/runtime/amd64.S}
        \mintc[firstline=234,lastline=237,highlightlines={235-237}]{../ocaml/runtime/amd64.S}
      }
      \onslide*<1>{\listCamlRaiseExn[highlightlines=720]{}}
      \onslide*<2>{\listCamlRaiseExn[highlightlines=722]{}}
      \onslide*<3->{\providecommand\step{5}%
% Backtraces
%
\subsection{One advantage of raising with debugging information}
\frameSubsection{}{}

\begin{frame}{Backtraces}{Debugging information \& source code}
  \begin{columns}[c]
    \begin{column}{0.5\textwidth}
      \begin{itemize}
        \item Raising an exception is fun and games, but how do we know where the exception originated? $\rightarrow$ With the backtrace associated with the exception!
        \item In order to get a backtrace from the point where an exception was raised, it's necessary to compile with debugging information enabled (\funcarg{-g} flag for the compiler)
        \item Same example as in the `Nested handlers' section
      \end{itemize}
    \end{column}
    \begin{column}{0.5\textwidth}
      \listocaml[firstline=9,lastline=34]{../src/backtrace/backtrace.ml}{backtrace/backtrace.ml}
    \end{column}
  \end{columns}
\end{frame}

\begin{frame}{Backtraces}{CMM and disassembly of \funcname{definitely\_raise}}
  \begin{columns}[c]
    \begin{column}{0.5\textwidth}
      \minipage[c][0.66\textheight][s]{\columnwidth}
      \begin{itemize}
        \item<1-> An effect of compiling with debugging information (\funcarg{-g}): the CMM no longer uses \funcname{raise\_notrace} to raise the exception but \funcname{raise} instead
        \item<2-> Disassembly shows that CMM \funcname{raise} is actually a call to \funcname{caml\_raise\_exn}, a function in assembler provided by the runtime
      \end{itemize}
      \vfill
      \mintocaml[firstline=9,lastline=13,highlightlines=12]{../src/backtrace/backtrace.ml}
      \endminipage
    \end{column}
    \begin{column}{0.5\textwidth}
      \onslide*<1>{
        \mintcmm[highlightlines=5]{../src/backtrace/camlBacktrace\_\_definitely\_raise\_347.cmm}
      }
      \onslide*<2>{
        \mintobjdump[firstline=2,highlightlines=14]{../src/backtrace/camlBacktrace\_\_definitely\_raise\_347.objdump}
      }
    \end{column}
  \end{columns}
\end{frame}

\begin{frame}{Backtraces}{Introducing \funcname{caml\_raise\_exn}}
  \begin{columns}[c]
    \begin{column}{0.5\textwidth}
      \minipage[c][0.66\textheight][s]{\columnwidth}
      \onslide*<1>{
        \begin{itemize}
          \item Raising the exception is performed using \funcname{caml\_raise\_exn} which is
            \begin{itemize}
              \item An assembly function provided by the runtime
              \item Architecture specific
            \end{itemize}
          \item Let's see what it does differently from \funcname{raise\_notrace}\dots
          \item We are about to raise \typename{ExnC} with \funcname{caml\_raise\_exn} from \funcname{definitely\_raise}
        \end{itemize}
      }
      \onslide*<2>{
        \mintobjdump[firstline=2,highlightlines=14]{../src/backtrace/camlBacktrace\_\_definitely\_raise\_347.objdump}
      }
      \vfill
      \mintocaml[firstline=9,lastline=13,highlightlines=12]{../src/backtrace/backtrace.ml}
      \endminipage
    \end{column}
    \begin{column}{0.5\textwidth}
      \centering
      \providecommand\step{1}\input{diagrams/backtrace/backtrace.tex}
    \end{column}
  \end{columns}
\end{frame}

\begin{frame}{Backtraces}{But before that, we need more fields from \typename{caml\_domain\_state}}
  \begin{columns}
    \begin{column}{0.5\textwidth}
      \begin{itemize}
        \item Remember \typename{caml\_domain\_state} the per-domain runtime state?
        \item Some other members are of interest to us now\dots
        \item \localname{backtrace\_active} is used to know if the runtime should collect backtraces
        \item \localname{backtrace\_active} can be set by
          \begin{itemize}
            \item The \funcarg{b} flag in the \funcname{OCAMLRUNPARAM} environment variable
            \item \mintinline{ocaml}{Printexc.record_backtrace : bool -> unit} \footnotemark
          \end{itemize}
      \end{itemize}
    \end{column}
    \begin{column}{0.5\textwidth}
      \begin{listing}
        \mintc[highlightlines={9-11}]{src/ocaml/domain\_state\_backtrace.h}
        \caption{runtime/caml/domain\_state.h}
      \end{listing}
    \end{column}
  \end{columns}
  \footnotetext[1]{https://v2.ocaml.org/api/Printexc.html}
\end{frame}

\newcommand\listCamlRaiseExn[1][]{
  \listasm[firstline=702,lastline=725,#1]{../ocaml/runtime/amd64.S}{runtime/amd64.S:702}}

\begin{frame}{Backtraces}{Walk through \funcname{caml\_raise\_exn}}
  \begin{columns}[T]
    \begin{column}{0.5\textwidth}
      \begin{itemize}
        \item<1-> First checks whether exceptions backtrace should be recorded
        \item<1-> By checking the \funcarg{domain\_state->backtrace\_active} value
        \item<2-> If not, acts as \funcname{raise\_notrace} with \funcname{RESTORE\_EXN\_HANDLER\_OCAML}
          \begin{itemize}
            \item Set \regname{RSP} to \localname{domain\_state->exn\_handler} value
            \item Restore \localname{domain\_state->exn\_handler} from trap
            \item Jump with \funcname{ret} (same effect as using \funcname{pop} and \funcname{jmp})
          \end{itemize}
        \item<3-> Otherwise, switches to the C stack to be able to execute C code
        \item<4-> Calls \funcname{caml\_stash\_backtrace}, a C function provided by the runtime\ignorespaces
          \onslide<6->{, with
            \begin{itemize}
              \item \funcarg{exn}: exception value
              \item \funcarg{pc}: code address of the raising point
              \item \funcarg{sp}: stack address of the raising function
              \item \funcarg{trapsp}: stack address of the next handler
            \end{itemize}
          }
      \end{itemize}
      \bigskip
      \mintocaml[firstline=9,lastline=13,highlightlines=12]{../src/backtrace/backtrace.ml}
    \end{column}
    \begin{column}{0.5\textwidth}
      \raggedleft
      \onslide*<1,3-4>{
        \mintc[firstline=190,lastline=192]{../ocaml/runtime/amd64.S}
        \mintc[firstline=234,lastline=237]{../ocaml/runtime/amd64.S}
      }
      \onslide*<2>{
        \mintc[firstline=190,lastline=192,highlightlines={191-192}]{../ocaml/runtime/amd64.S}
        \mintc[firstline=234,lastline=237,highlightlines={235-237}]{../ocaml/runtime/amd64.S}
      }
      \onslide*<1>{\listCamlRaiseExn[highlightlines=705]{}}%
      \onslide*<2>{\listCamlRaiseExn[highlightlines={707-708}]{}}%
      \onslide*<3>{\listCamlRaiseExn[highlightlines={712-714}]{}}%
      \onslide*<4>{\listCamlRaiseExn[highlightlines={715-720}]{}}%
      \onslide*<5>{\providecommand\step{1}\input{diagrams/backtrace/backtrace.tex}}%
      \onslide*<6>{\providecommand\step{2}\input{diagrams/backtrace/backtrace.tex}}%
    \end{column}
  \end{columns}
\end{frame}

\newcommand\listStashBacktrace[1][]{
  \listc[firstline=91,lastline=120,#1]{../ocaml/runtime/backtrace\_nat.c}{runtime/backtrace\_nat.c:91}}

\begin{frame}{Backtraces}{Introducing \funcname{caml\_stash\_backtrace}}
  \begin{columns}[c]
    \begin{column}{0.5\textwidth}
      \minipage[c][0.66\textheight][s]{\columnwidth}
      \begin{itemize}
        \item<1-> A C function provided by the runtime (specific to native code)
        \item<2-> It scans the stack, between the stack pointer at the raising point up to the next trap, in order to collect the names of the detected function frames
          \onslide*<2>{
            \begin{itemize}
              \item And more information, like line number, column number...
            \end{itemize}
          }
        \item<3-> It uses the same technique and information as the Garbage Collector (GC) uses to scan the values live on the stack
          \onslide*<4>{
            \begin{itemize}
              \item \typename{frame\_descr} are at the core of stack unwinding
            \end{itemize}
          }
      \end{itemize}
      \vfill
      \mintocaml[firstline=9,lastline=13,highlightlines=12]{../src/backtrace/backtrace.ml}
      \endminipage
    \end{column}
    \begin{column}{0.5\textwidth}
      \onslide*<1>{\listStashBacktrace[highlightlines=91]{}}
      \onslide*<2>{\listStashBacktrace[highlightlines={91,117-118}]{}}
      \onslide*<3->{\listStashBacktrace[highlightlines={94,106,109-110}]{}}
    \end{column}
  \end{columns}
\end{frame}

\newcommand\listFrameDescr[1][]{
  \listocaml[firstline=111,lastline=124,#1]{../ocaml/asmcomp/emitaux.ml}{asmcomp/emitaux.ml:111}}
\newcommand\listDebugInfo[1][]{
  \listocaml[firstline=38,lastline=49,#1]{../ocaml/lambda/debuginfo.mli}{lambda/debuginfo.mli:38}}

\begin{frame}{Backtraces}{\typename{frame\_descr} and \typename{frame\_debuginfo} in a nutshell}
  \begin{columns}[c]
    \begin{column}{0.5\textwidth}
      \begin{itemize}
        \item<1-> For every return address in an OCaml program, there is a associated \typename{frame\_descr}
        \item<2-> A \typename{frame\_descr} specifies
        \begin{itemize}
          \item<2-> The size of the function stack frame
          \item<3-> Where live values are on the stack (so that the GC can mark them as live and prevent from being swept)
        \end{itemize}
        \item<4-> When compiled with debug information, \typename{frame\_descr} will be linked to a \typename{frame\_debuginfo}
        \begin{itemize}
          \item<5-> Contains information about the position in the source code (file name, line and column number)
        \end{itemize}
      \end{itemize}
    \end{column}
    \begin{column}{0.5\textwidth}
        \onslide*<1>{\listFrameDescr[highlightlines={118-122}]{}}
        \onslide*<2>{\listFrameDescr[highlightlines={120}]{}}
        \onslide*<3>{\listFrameDescr[highlightlines={121}]{}}
        \onslide*<4->{\listFrameDescr[highlightlines={122}]{}}
        \medskip
        \onslide*<1-3>{\listDebugInfo{}}
        \onslide*<4>{\listDebugInfo[highlightlines={38,49}]{}}
        \onslide*<5>{\listDebugInfo[highlightlines={39-46}]{}}
    \end{column}
  \end{columns}
\end{frame}

\begin{frame}{Backtraces}{Scanning the stack with \typename{frame\_descr}}
  \begin{columns}[c]
    \begin{column}{0.5\textwidth}
      \begin{itemize}
        \item Given the return address of the code that raises the exception by calling \funcname{caml\_raise\_exn} (\funcarg{pc} argument of \funcname{caml\_stash\_backtrace})
        \item And the position of the stack pointer (\funcarg{sp} argument of \funcname{caml\_stash\_backtrace})
        \item The frame size given by \typename{frame\_descr}.\funcarg{frame\_size}, allows to find the end of the previous frame
        \item And the return address of the caller of this function
        \bigskip
        \onslide<3->{
        \item Note that traps (here the reddish \funcname{ExnA}, \funcname{ExnB} and \funcname{ExnC} blocks) belong to the frame that installed them, and so the frame size changes when traps are removed
        }
      \end{itemize}
    \end{column}
    \begin{column}{0.5\textwidth}
      \raggedleft
      \onslide*<1>{\providecommand\step{2}\input{diagrams/backtrace/backtrace.tex}}%
      \onslide*<2>{\providecommand\step{1}\input{diagrams/backtrace/scanstack.tex}}%
      \onslide*<3>{\providecommand\step{2}\input{diagrams/backtrace/scanstack.tex}}%
      \onslide*<4>{\providecommand\step{3}\input{diagrams/backtrace/scanstack.tex}}%
      \onslide*<5>{\providecommand\step{4}\input{diagrams/backtrace/scanstack.tex}}%
      \onslide*<6>{\providecommand\step{5}\input{diagrams/backtrace/scanstack.tex}}%
    \end{column}
  \end{columns}
\end{frame}

% Duplicate of previous-previous slide
\begin{frame}{Backtraces}{Continuing on \funcname{caml\_stash\_backtrace}}
  \begin{columns}[c]
    \begin{column}{0.5\textwidth}
      \minipage[c][0.75\textheight][s]{\columnwidth}
      \begin{itemize}
          \color{gray}
        \item A C function provided by the runtime (specific to native code)
        \item It scans the stack, between the stack pointer at the raising point up to the next trap, in order to collect the names of the detected function frames
        \item It uses the same technique and information as the Garbage Collector (GC) uses to scan the values live on the stack
      \end{itemize}
      \begin{itemize}
        \item For each function frame detected, store a pointer to the \typename{frame\_descr} in the \localname{domain\_state->backtrace\_buffer} array, effectively constructing the backtrace
      \end{itemize}
      \onslide<2->{
        \begin{itemize}
            \smallskip
          \item Constructed backtrace:
            \funcname{definitely\_raise},
            \onslide<3->{\funcname{probably\_raise}}
        \end{itemize}
      }
      \vfill
      \mintocaml[firstline=9,lastline=13,highlightlines=12]{../src/backtrace/backtrace.ml}
      \endminipage
    \end{column}
    \begin{column}{0.5\textwidth}
      \raggedleft
      \onslide*<1>{\listStashBacktrace[highlightlines={114-115}]{}}
      \onslide*<2>{\providecommand\step{3}\input{diagrams/backtrace/backtrace.tex}}%
      \onslide*<3>{\providecommand\step{4}\input{diagrams/backtrace/backtrace.tex}}%
    \end{column}
  \end{columns}
\end{frame}

\begin{frame}{Backtraces}{Back to \funcname{caml\_raise\_exn}}
  \begin{columns}[c]
    \begin{column}{0.5\textwidth}
      \minipage[c][0.66\textheight][s]{\columnwidth}
      \begin{itemize}\color{gray}
        \item Checks wheteher exceptions backtrace should be recorded, by checking the \funcarg{domain\_state->backtrace\_active} value
        \item Switches to the C stack to be able to execute C code
        \item Calls \funcname{caml\_stash\_backtrace}, to collect the backtrace up to the next trap
      \end{itemize}
      \onslide*<2->{
        \begin{itemize}
          \item<2-> Executes the exception handler in \funcname{probably\_raise} with \funcname{RESTORE\_EXN\_HANDLER\_OCAML}
            \begin{itemize}
              \item Set \regname{RSP} to \localname{domain\_state->exn\_handler} value
              \item Restore \localname{domain\_state->exn\_handler} from trap
              \item Jump to the handler code with \funcname{ret}
            \end{itemize}
        \end{itemize}
      }
      \vfill
      \onslide*<1>{\mintocaml[firstline=9,lastline=13,highlightlines=12]{../src/backtrace/backtrace.ml}}
      \onslide*<2->{\mintocaml[firstline=15,lastline=21,highlightlines={19-21}]{../src/backtrace/backtrace.ml}}
      \endminipage
    \end{column}
    \begin{column}{0.5\textwidth}
      \centering
      \onslide*<1>{
        \mintc[firstline=190,lastline=192]{../ocaml/runtime/amd64.S}
        \mintc[firstline=234,lastline=237]{../ocaml/runtime/amd64.S}
      }
      \onslide*<2>{
        \mintc[firstline=190,lastline=192,highlightlines={191-192}]{../ocaml/runtime/amd64.S}
        \mintc[firstline=234,lastline=237,highlightlines={235-237}]{../ocaml/runtime/amd64.S}
      }
      \onslide*<1>{\listCamlRaiseExn[highlightlines=720]{}}
      \onslide*<2>{\listCamlRaiseExn[highlightlines=722]{}}
      \onslide*<3->{\providecommand\step{5}\input{diagrams/backtrace/backtrace.tex}}
    \end{column}
  \end{columns}
\end{frame}

\begin{frame}{Backtraces}{\funcname{caml\_probably\_raise}'s handler doesn't catch \typename{ExnC}}
  \begin{columns}[c]
    \begin{column}{0.45\textwidth}
      \minipage[c][0.75\textheight][s]{\columnwidth}
      \begin{itemize}
        \item<1-> \funcname{match \dots with}\footnotemark pattern matching can also match exceptions
        \item<1-> The CMM of \funcname{probably\_raise} shows that this \funcname{match \dots with} is implemented with a pattern matching and a \funcname{try \dots with}
        \item<1-> But this one only matches on \typename{ExnA} exception
        \item<2-> Since the pattern matching fails to catch the exception, \funcname{reraise} is called with our current \typename{ExnC} exception
        \item<3-> Disassembly shows that \funcname{reraise} is actually a call to \funcname{caml\_reraise\_exn}, which is also an assembly function provided by the runtime
      \end{itemize}
      \vfill
      \mintocaml[firstline=15,lastline=21,highlightlines={18-21}]{../src/backtrace/backtrace.ml}
      \endminipage
    \end{column}
    \begin{column}{0.55\textwidth}
      \centering
      \onslide*<1>{\mintcmm[highlightlines={10-18}]{../src/backtrace/camlBacktrace\_\_probably\_raise\_351.cmm}}
      \onslide*<2>{\listcmm[highlightlines=18]{../src/backtrace/camlBacktrace\_\_probably\_raise_351.cmm}{CMM of probably\_raise}}
      \onslide*<3>{\mintobjdump[firstline=5,lastline=35,highlightlines=31]{../src/backtrace/camlBacktrace\_\_probably\_raise_351.objdump}}
    \end{column}
  \end{columns}
  \only<1>{\footnotetext[1]{https://blog.janestreet.com/pattern-matching-and-exception-handling-unite/}}
\end{frame}

\newcommand\listCamlReraiseExn[1][]{
  \listasm[firstline=727,lastline=734,#1]{../ocaml/runtime/amd64.S}{runtime/amd64.S:727}}

\begin{frame}{Backtraces}{Introducing \funcname{caml\_reraise\_exn}}
  \begin{columns}[c]
    \begin{column}{0.5\textwidth}
      \minipage[c][0.66\textheight][s]{\columnwidth}
      \begin{itemize}
        \item<1-> The exception have to be reraised to the next handler
        \item<2-> First, checks if the backtrace should be collected with \funcarg{domain\_state->backtrace\_active}
        \only<3>{
        \begin{itemize}
            \item If not, behaves like a \funcname{raise\_notrace} with \funcname{RESTORE\_EXN\_HANDLER\_OCAML}
        \end{itemize}
        }
        \item<4-> Jumps into \funcname{caml\_raise\_exn}
        \item<5-> Calls \funcname{caml\_stash\_backtrace} again, to scan the next part of the stack: from the trap just pop-ed to the next trap
      \end{itemize}
      \vfill
      \mintocaml[firstline=15,lastline=21,highlightlines={18-21}]{../src/backtrace/backtrace.ml}
      \endminipage
    \end{column}
    \begin{column}{0.5\textwidth}
      \raggedleft
      \onslide*<1>{\listCamlReraiseExn{}}
      \onslide*<2>{\listCamlReraiseExn[highlightlines=729]{}}
      \onslide*<3>{
        \mintc[firstline=190,lastline=192,highlightlines={191-192}]{../ocaml/runtime/amd64.S}
        \mintc[firstline=234,lastline=237,highlightlines={235-237}]{../ocaml/runtime/amd64.S}
        \medskip
        \listCamlReraiseExn[highlightlines=731]{}
      }
      \onslide*<4-5>{
        % TODO refactor with \listCamlReraiseExn
        \mintasm[firstline=727,lastline=734,highlightlines=730]{../ocaml/runtime/amd64.S}
        \medskip
      }
      \onslide*<4>{\listCamlRaiseExn[highlightlines=711]{}}
      \onslide*<5>{\listCamlRaiseExn[highlightlines=720]{}}
    \end{column}
  \end{columns}
\end{frame}

\begin{frame}{Backtraces}{Backtrace collection continues}
  \begin{columns}[c]
    \begin{column}{0.55\textwidth}
      \minipage[c][0.75\textheight][s]{\columnwidth}
      \begin{itemize}
        \item<1-> \funcname{caml\_stash\_backtrace} scans the older part of the stack, up to the next trap
        \item<6-> Controls jumps to the nested exception handler in \funcname{maybe\_raise}, which matches only on \typename{ExnB}
        \item<7-> \funcname{caml\_reraise\_exn} is called again, backtrace collection resumes with \funcname{caml\_stash\_backtrace}
      \end{itemize}
      \medskip
      \begin{itemize}
        \item<2-> Constructed backtrace:
        \begin{itemize}
          \item <2-> \funcname{definitely\_raise}
          \item <2-> \funcname{probably\_raise}
          \item <3-> \funcname{probably\_raise} (re-raise)
          \item <4-> \funcname{maybe\_raise}
          \item <5-> \funcname{main}
          \item <8-> \funcname{main} (re-raise)
        \end{itemize}
      \end{itemize}
      \vfill
      \onslide*<1-5>{\mintocaml[firstline=15,lastline=21]{../src/backtrace/backtrace.ml}}
      \onslide*<6>{\mintocaml[firstline=28,lastline=34,highlightlines=31]{../src/backtrace/backtrace.ml}}
      \onslide*<7->{\mintocaml[firstline=28,lastline=34,highlightlines=33]{../src/backtrace/backtrace.ml}}
      \endminipage
    \end{column}
    \begin{column}{0.45\textwidth}
      \raggedleft
      \onslide*<1>{\providecommand\step{6}\input{diagrams/backtrace/backtrace.tex}}%
      \onslide*<2>{\providecommand\step{6}\input{diagrams/backtrace/backtrace.tex}}%
      \onslide*<3>{\providecommand\step{7}\input{diagrams/backtrace/backtrace.tex}}%
      \onslide*<4>{\providecommand\step{8}\input{diagrams/backtrace/backtrace.tex}}%
      \onslide*<5>{\providecommand\step{9}\input{diagrams/backtrace/backtrace.tex}}%
      \onslide*<6>{\providecommand\step{10}\input{diagrams/backtrace/backtrace.tex}}%
      \onslide*<7>{\providecommand\step{11}\input{diagrams/backtrace/backtrace.tex}}%
      \onslide*<8>{\providecommand\step{12}\input{diagrams/backtrace/backtrace.tex}}%
      \onslide*<9>{\providecommand\step{13}\input{diagrams/backtrace/backtrace.tex}}%
    \end{column}
  \end{columns}
\end{frame}

\begin{frame}{Backtraces}{Printing the backtrace}
  \begin{columns}[c]
    \begin{column}{0.45\textwidth}
      \minipage[c][0.66\textheight][s]{\columnwidth}
      \begin{itemize}
        \item<1-> The exception backtrace can now be printed to \funcarg{stdout} with \funcname{Printexc.print_backtrace}
        \item<2-> First the exception backtrace is retrieved into an OCaml array with \funcname{get_raw_backtrace }
        \item<3-> Which is implemented as the \funcname{caml_get_exception_raw_backtrace} C function in the runtime
        \item<4-> And finally printed with \funcname{print_exception_backtrace}
      \end{itemize}
      \vfill
      \mintocaml[firstline=28,lastline=34,highlightlines=33]{../src/backtrace/backtrace.ml}
      \endminipage
    \end{column}
    \begin{column}{0.55\textwidth}
      \onslide*<1,2>{
        \mintocaml[firstline=100,lastline=101]{../ocaml/stdlib/printexc.ml}
      }
      \only<1>{
        \listocaml[firstline=170,lastline=172]{../ocaml/stdlib/printexc.ml}{stdlib/printexc.ml:170}
      }
      \only<2>{
        \listocaml[firstline=170,lastline=172,highlightlines=172]{../ocaml/stdlib/printexc.ml}{stdlib/printexc.ml:170}
      }
      \only<3>{
        \mintc[firstline=154,lastline=156]{../ocaml/runtime/backtrace.c}
        \listc[firstline=171,lastline=192,highlightlines={184-187}]{../ocaml/runtime/backtrace.c}{runtime/backtrace.c:154}
      }
      \only<4>{
        \listocaml[firstline=155,lastline=165]{../ocaml/stdlib/printexc.ml}{stdlib/printexc.ml:55}
      }
    \end{column}
  \end{columns}
\end{frame}


\subsection{The multiple kinds of raise}
\frameSubsection{}{}

\begin{frame}{Backtraces}{When backtraces are not needed}
  \begin{columns}[c]
    \begin{column}{0.45\textwidth}
      \minipage[c][0.66\textheight][s]{\columnwidth}
      \begin{itemize}
        \item<1-> What if the backtrace for an exception is never needed?
        \item<2-> It's possible to raise the exception explicitly with \funcname{raise\_notrace} to make raising as cheap as possible
        \item<3-> An example of use in the \funcname{dynlink} library of the OCaml compiler
      \end{itemize}
      \vfill
      \onslide<3->{
      \listocaml[firstline=205,lastline=209,highlightlines=207]{../ocaml/otherlibs/dynlink/dynlink\_compilerlibs/misc.ml}{otherlibs/dynlink/dynlink\_compilerlibs/misc.ml:205}
      }
      \endminipage
    \end{column}
    \begin{column}{0.55\textwidth}
    \only<4>{
      \mintcmm[highlightlines=3]{../src/notrace/camlNotrace\_\_fun_347.cmm}
      \listcmm{../src/notrace/camlNotrace\_\_all\_somes\_327.cmm}{all\_somes}
    }
    \onslide*<5->{
      \listobjdump[highlightlines={6-9}]{../src/notrace/camlNotrace\_\_fun_347.objdump}{anonymous function from \funcname{all\_somes}}
    }
    \end{column}
  \end{columns}
\end{frame}

\begin{frame}{Backtraces}{Summary table of the multiple kinds of raise}
  \begin{itemize}
    \item When debugging information is enabled and if the backtrace of an exception isn't needed, it's possible to raise with \funcname{raise\_notrace} for performance
    \item When debugging information is \emph{not} enabled, the compiler optimises \funcname{raise} calls to inline assembly (same effect as using \funcname{raise\_notrace})
  \end{itemize}
  \bigskip
  \centering
  \begin{tabular}{| l | c | c | c | c |}
    \hline
    \parbox{10em}{Debug information \\ (\funcarg{-g} flag)}  & \multicolumn{2}{|c|}{Disabled}                                     & \multicolumn{2}{|c|}{Enabled} \\
    \hline
    Raise kind                                               & \funcname{raise} & \funcname{raise\_notrace}                       & \funcname{raise} & \funcname{raise\_notrace} \\
    \hline
    Compilation                                              & \parbox{8em}{inline assembly\\ (optimization)} & inline assembly   & \funcname{caml\_raise\_exn} & inline assembly \\
    \hline
  \end{tabular}
\end{frame}


%
% Takeaway #5
%
\subsection*{Takeaway \#5}
\frameSubsectionTakeaway{}
\againframe<5>{takeaway}
}
    \end{column}
  \end{columns}
\end{frame}

\begin{frame}{Backtraces}{\funcname{caml\_probably\_raise}'s handler doesn't catch \typename{ExnC}}
  \begin{columns}[c]
    \begin{column}{0.45\textwidth}
      \minipage[c][0.75\textheight][s]{\columnwidth}
      \begin{itemize}
        \item<1-> \funcname{match \dots with}\footnotemark pattern matching can also match exceptions
        \item<1-> The CMM of \funcname{probably\_raise} shows that this \funcname{match \dots with} is implemented with a pattern matching and a \funcname{try \dots with}
        \item<1-> But this one only matches on \typename{ExnA} exception
        \item<2-> Since the pattern matching fails to catch the exception, \funcname{reraise} is called with our current \typename{ExnC} exception
        \item<3-> Disassembly shows that \funcname{reraise} is actually a call to \funcname{caml\_reraise\_exn}, which is also an assembly function provided by the runtime
      \end{itemize}
      \vfill
      \mintocaml[firstline=15,lastline=21,highlightlines={18-21}]{../src/backtrace/backtrace.ml}
      \endminipage
    \end{column}
    \begin{column}{0.55\textwidth}
      \centering
      \onslide*<1>{\mintcmm[highlightlines={10-18}]{../src/backtrace/camlBacktrace\_\_probably\_raise\_351.cmm}}
      \onslide*<2>{\listcmm[highlightlines=18]{../src/backtrace/camlBacktrace\_\_probably\_raise_351.cmm}{CMM of probably\_raise}}
      \onslide*<3>{\mintobjdump[firstline=5,lastline=35,highlightlines=31]{../src/backtrace/camlBacktrace\_\_probably\_raise_351.objdump}}
    \end{column}
  \end{columns}
  \only<1>{\footnotetext[1]{https://blog.janestreet.com/pattern-matching-and-exception-handling-unite/}}
\end{frame}

\newcommand\listCamlReraiseExn[1][]{
  \listasm[firstline=727,lastline=734,#1]{../ocaml/runtime/amd64.S}{runtime/amd64.S:727}}

\begin{frame}{Backtraces}{Introducing \funcname{caml\_reraise\_exn}}
  \begin{columns}[c]
    \begin{column}{0.5\textwidth}
      \minipage[c][0.66\textheight][s]{\columnwidth}
      \begin{itemize}
        \item<1-> The exception have to be reraised to the next handler
        \item<2-> First, checks if the backtrace should be collected with \funcarg{domain\_state->backtrace\_active}
        \only<3>{
        \begin{itemize}
            \item If not, behaves like a \funcname{raise\_notrace} with \funcname{RESTORE\_EXN\_HANDLER\_OCAML}
        \end{itemize}
        }
        \item<4-> Jumps into \funcname{caml\_raise\_exn}
        \item<5-> Calls \funcname{caml\_stash\_backtrace} again, to scan the next part of the stack: from the trap just pop-ed to the next trap
      \end{itemize}
      \vfill
      \mintocaml[firstline=15,lastline=21,highlightlines={18-21}]{../src/backtrace/backtrace.ml}
      \endminipage
    \end{column}
    \begin{column}{0.5\textwidth}
      \raggedleft
      \onslide*<1>{\listCamlReraiseExn{}}
      \onslide*<2>{\listCamlReraiseExn[highlightlines=729]{}}
      \onslide*<3>{
        \mintc[firstline=190,lastline=192,highlightlines={191-192}]{../ocaml/runtime/amd64.S}
        \mintc[firstline=234,lastline=237,highlightlines={235-237}]{../ocaml/runtime/amd64.S}
        \medskip
        \listCamlReraiseExn[highlightlines=731]{}
      }
      \onslide*<4-5>{
        % TODO refactor with \listCamlReraiseExn
        \mintasm[firstline=727,lastline=734,highlightlines=730]{../ocaml/runtime/amd64.S}
        \medskip
      }
      \onslide*<4>{\listCamlRaiseExn[highlightlines=711]{}}
      \onslide*<5>{\listCamlRaiseExn[highlightlines=720]{}}
    \end{column}
  \end{columns}
\end{frame}

\begin{frame}{Backtraces}{Backtrace collection continues}
  \begin{columns}[c]
    \begin{column}{0.55\textwidth}
      \minipage[c][0.75\textheight][s]{\columnwidth}
      \begin{itemize}
        \item<1-> \funcname{caml\_stash\_backtrace} scans the older part of the stack, up to the next trap
        \item<6-> Controls jumps to the nested exception handler in \funcname{maybe\_raise}, which matches only on \typename{ExnB}
        \item<7-> \funcname{caml\_reraise\_exn} is called again, backtrace collection resumes with \funcname{caml\_stash\_backtrace}
      \end{itemize}
      \medskip
      \begin{itemize}
        \item<2-> Constructed backtrace:
        \begin{itemize}
          \item <2-> \funcname{definitely\_raise}
          \item <2-> \funcname{probably\_raise}
          \item <3-> \funcname{probably\_raise} (re-raise)
          \item <4-> \funcname{maybe\_raise}
          \item <5-> \funcname{main}
          \item <8-> \funcname{main} (re-raise)
        \end{itemize}
      \end{itemize}
      \vfill
      \onslide*<1-5>{\mintocaml[firstline=15,lastline=21]{../src/backtrace/backtrace.ml}}
      \onslide*<6>{\mintocaml[firstline=28,lastline=34,highlightlines=31]{../src/backtrace/backtrace.ml}}
      \onslide*<7->{\mintocaml[firstline=28,lastline=34,highlightlines=33]{../src/backtrace/backtrace.ml}}
      \endminipage
    \end{column}
    \begin{column}{0.45\textwidth}
      \raggedleft
      \onslide*<1>{\providecommand\step{6}%
% Backtraces
%
\subsection{One advantage of raising with debugging information}
\frameSubsection{}{}

\begin{frame}{Backtraces}{Debugging information \& source code}
  \begin{columns}[c]
    \begin{column}{0.5\textwidth}
      \begin{itemize}
        \item Raising an exception is fun and games, but how do we know where the exception originated? $\rightarrow$ With the backtrace associated with the exception!
        \item In order to get a backtrace from the point where an exception was raised, it's necessary to compile with debugging information enabled (\funcarg{-g} flag for the compiler)
        \item Same example as in the `Nested handlers' section
      \end{itemize}
    \end{column}
    \begin{column}{0.5\textwidth}
      \listocaml[firstline=9,lastline=34]{../src/backtrace/backtrace.ml}{backtrace/backtrace.ml}
    \end{column}
  \end{columns}
\end{frame}

\begin{frame}{Backtraces}{CMM and disassembly of \funcname{definitely\_raise}}
  \begin{columns}[c]
    \begin{column}{0.5\textwidth}
      \minipage[c][0.66\textheight][s]{\columnwidth}
      \begin{itemize}
        \item<1-> An effect of compiling with debugging information (\funcarg{-g}): the CMM no longer uses \funcname{raise\_notrace} to raise the exception but \funcname{raise} instead
        \item<2-> Disassembly shows that CMM \funcname{raise} is actually a call to \funcname{caml\_raise\_exn}, a function in assembler provided by the runtime
      \end{itemize}
      \vfill
      \mintocaml[firstline=9,lastline=13,highlightlines=12]{../src/backtrace/backtrace.ml}
      \endminipage
    \end{column}
    \begin{column}{0.5\textwidth}
      \onslide*<1>{
        \mintcmm[highlightlines=5]{../src/backtrace/camlBacktrace\_\_definitely\_raise\_347.cmm}
      }
      \onslide*<2>{
        \mintobjdump[firstline=2,highlightlines=14]{../src/backtrace/camlBacktrace\_\_definitely\_raise\_347.objdump}
      }
    \end{column}
  \end{columns}
\end{frame}

\begin{frame}{Backtraces}{Introducing \funcname{caml\_raise\_exn}}
  \begin{columns}[c]
    \begin{column}{0.5\textwidth}
      \minipage[c][0.66\textheight][s]{\columnwidth}
      \onslide*<1>{
        \begin{itemize}
          \item Raising the exception is performed using \funcname{caml\_raise\_exn} which is
            \begin{itemize}
              \item An assembly function provided by the runtime
              \item Architecture specific
            \end{itemize}
          \item Let's see what it does differently from \funcname{raise\_notrace}\dots
          \item We are about to raise \typename{ExnC} with \funcname{caml\_raise\_exn} from \funcname{definitely\_raise}
        \end{itemize}
      }
      \onslide*<2>{
        \mintobjdump[firstline=2,highlightlines=14]{../src/backtrace/camlBacktrace\_\_definitely\_raise\_347.objdump}
      }
      \vfill
      \mintocaml[firstline=9,lastline=13,highlightlines=12]{../src/backtrace/backtrace.ml}
      \endminipage
    \end{column}
    \begin{column}{0.5\textwidth}
      \centering
      \providecommand\step{1}\input{diagrams/backtrace/backtrace.tex}
    \end{column}
  \end{columns}
\end{frame}

\begin{frame}{Backtraces}{But before that, we need more fields from \typename{caml\_domain\_state}}
  \begin{columns}
    \begin{column}{0.5\textwidth}
      \begin{itemize}
        \item Remember \typename{caml\_domain\_state} the per-domain runtime state?
        \item Some other members are of interest to us now\dots
        \item \localname{backtrace\_active} is used to know if the runtime should collect backtraces
        \item \localname{backtrace\_active} can be set by
          \begin{itemize}
            \item The \funcarg{b} flag in the \funcname{OCAMLRUNPARAM} environment variable
            \item \mintinline{ocaml}{Printexc.record_backtrace : bool -> unit} \footnotemark
          \end{itemize}
      \end{itemize}
    \end{column}
    \begin{column}{0.5\textwidth}
      \begin{listing}
        \mintc[highlightlines={9-11}]{src/ocaml/domain\_state\_backtrace.h}
        \caption{runtime/caml/domain\_state.h}
      \end{listing}
    \end{column}
  \end{columns}
  \footnotetext[1]{https://v2.ocaml.org/api/Printexc.html}
\end{frame}

\newcommand\listCamlRaiseExn[1][]{
  \listasm[firstline=702,lastline=725,#1]{../ocaml/runtime/amd64.S}{runtime/amd64.S:702}}

\begin{frame}{Backtraces}{Walk through \funcname{caml\_raise\_exn}}
  \begin{columns}[T]
    \begin{column}{0.5\textwidth}
      \begin{itemize}
        \item<1-> First checks whether exceptions backtrace should be recorded
        \item<1-> By checking the \funcarg{domain\_state->backtrace\_active} value
        \item<2-> If not, acts as \funcname{raise\_notrace} with \funcname{RESTORE\_EXN\_HANDLER\_OCAML}
          \begin{itemize}
            \item Set \regname{RSP} to \localname{domain\_state->exn\_handler} value
            \item Restore \localname{domain\_state->exn\_handler} from trap
            \item Jump with \funcname{ret} (same effect as using \funcname{pop} and \funcname{jmp})
          \end{itemize}
        \item<3-> Otherwise, switches to the C stack to be able to execute C code
        \item<4-> Calls \funcname{caml\_stash\_backtrace}, a C function provided by the runtime\ignorespaces
          \onslide<6->{, with
            \begin{itemize}
              \item \funcarg{exn}: exception value
              \item \funcarg{pc}: code address of the raising point
              \item \funcarg{sp}: stack address of the raising function
              \item \funcarg{trapsp}: stack address of the next handler
            \end{itemize}
          }
      \end{itemize}
      \bigskip
      \mintocaml[firstline=9,lastline=13,highlightlines=12]{../src/backtrace/backtrace.ml}
    \end{column}
    \begin{column}{0.5\textwidth}
      \raggedleft
      \onslide*<1,3-4>{
        \mintc[firstline=190,lastline=192]{../ocaml/runtime/amd64.S}
        \mintc[firstline=234,lastline=237]{../ocaml/runtime/amd64.S}
      }
      \onslide*<2>{
        \mintc[firstline=190,lastline=192,highlightlines={191-192}]{../ocaml/runtime/amd64.S}
        \mintc[firstline=234,lastline=237,highlightlines={235-237}]{../ocaml/runtime/amd64.S}
      }
      \onslide*<1>{\listCamlRaiseExn[highlightlines=705]{}}%
      \onslide*<2>{\listCamlRaiseExn[highlightlines={707-708}]{}}%
      \onslide*<3>{\listCamlRaiseExn[highlightlines={712-714}]{}}%
      \onslide*<4>{\listCamlRaiseExn[highlightlines={715-720}]{}}%
      \onslide*<5>{\providecommand\step{1}\input{diagrams/backtrace/backtrace.tex}}%
      \onslide*<6>{\providecommand\step{2}\input{diagrams/backtrace/backtrace.tex}}%
    \end{column}
  \end{columns}
\end{frame}

\newcommand\listStashBacktrace[1][]{
  \listc[firstline=91,lastline=120,#1]{../ocaml/runtime/backtrace\_nat.c}{runtime/backtrace\_nat.c:91}}

\begin{frame}{Backtraces}{Introducing \funcname{caml\_stash\_backtrace}}
  \begin{columns}[c]
    \begin{column}{0.5\textwidth}
      \minipage[c][0.66\textheight][s]{\columnwidth}
      \begin{itemize}
        \item<1-> A C function provided by the runtime (specific to native code)
        \item<2-> It scans the stack, between the stack pointer at the raising point up to the next trap, in order to collect the names of the detected function frames
          \onslide*<2>{
            \begin{itemize}
              \item And more information, like line number, column number...
            \end{itemize}
          }
        \item<3-> It uses the same technique and information as the Garbage Collector (GC) uses to scan the values live on the stack
          \onslide*<4>{
            \begin{itemize}
              \item \typename{frame\_descr} are at the core of stack unwinding
            \end{itemize}
          }
      \end{itemize}
      \vfill
      \mintocaml[firstline=9,lastline=13,highlightlines=12]{../src/backtrace/backtrace.ml}
      \endminipage
    \end{column}
    \begin{column}{0.5\textwidth}
      \onslide*<1>{\listStashBacktrace[highlightlines=91]{}}
      \onslide*<2>{\listStashBacktrace[highlightlines={91,117-118}]{}}
      \onslide*<3->{\listStashBacktrace[highlightlines={94,106,109-110}]{}}
    \end{column}
  \end{columns}
\end{frame}

\newcommand\listFrameDescr[1][]{
  \listocaml[firstline=111,lastline=124,#1]{../ocaml/asmcomp/emitaux.ml}{asmcomp/emitaux.ml:111}}
\newcommand\listDebugInfo[1][]{
  \listocaml[firstline=38,lastline=49,#1]{../ocaml/lambda/debuginfo.mli}{lambda/debuginfo.mli:38}}

\begin{frame}{Backtraces}{\typename{frame\_descr} and \typename{frame\_debuginfo} in a nutshell}
  \begin{columns}[c]
    \begin{column}{0.5\textwidth}
      \begin{itemize}
        \item<1-> For every return address in an OCaml program, there is a associated \typename{frame\_descr}
        \item<2-> A \typename{frame\_descr} specifies
        \begin{itemize}
          \item<2-> The size of the function stack frame
          \item<3-> Where live values are on the stack (so that the GC can mark them as live and prevent from being swept)
        \end{itemize}
        \item<4-> When compiled with debug information, \typename{frame\_descr} will be linked to a \typename{frame\_debuginfo}
        \begin{itemize}
          \item<5-> Contains information about the position in the source code (file name, line and column number)
        \end{itemize}
      \end{itemize}
    \end{column}
    \begin{column}{0.5\textwidth}
        \onslide*<1>{\listFrameDescr[highlightlines={118-122}]{}}
        \onslide*<2>{\listFrameDescr[highlightlines={120}]{}}
        \onslide*<3>{\listFrameDescr[highlightlines={121}]{}}
        \onslide*<4->{\listFrameDescr[highlightlines={122}]{}}
        \medskip
        \onslide*<1-3>{\listDebugInfo{}}
        \onslide*<4>{\listDebugInfo[highlightlines={38,49}]{}}
        \onslide*<5>{\listDebugInfo[highlightlines={39-46}]{}}
    \end{column}
  \end{columns}
\end{frame}

\begin{frame}{Backtraces}{Scanning the stack with \typename{frame\_descr}}
  \begin{columns}[c]
    \begin{column}{0.5\textwidth}
      \begin{itemize}
        \item Given the return address of the code that raises the exception by calling \funcname{caml\_raise\_exn} (\funcarg{pc} argument of \funcname{caml\_stash\_backtrace})
        \item And the position of the stack pointer (\funcarg{sp} argument of \funcname{caml\_stash\_backtrace})
        \item The frame size given by \typename{frame\_descr}.\funcarg{frame\_size}, allows to find the end of the previous frame
        \item And the return address of the caller of this function
        \bigskip
        \onslide<3->{
        \item Note that traps (here the reddish \funcname{ExnA}, \funcname{ExnB} and \funcname{ExnC} blocks) belong to the frame that installed them, and so the frame size changes when traps are removed
        }
      \end{itemize}
    \end{column}
    \begin{column}{0.5\textwidth}
      \raggedleft
      \onslide*<1>{\providecommand\step{2}\input{diagrams/backtrace/backtrace.tex}}%
      \onslide*<2>{\providecommand\step{1}\input{diagrams/backtrace/scanstack.tex}}%
      \onslide*<3>{\providecommand\step{2}\input{diagrams/backtrace/scanstack.tex}}%
      \onslide*<4>{\providecommand\step{3}\input{diagrams/backtrace/scanstack.tex}}%
      \onslide*<5>{\providecommand\step{4}\input{diagrams/backtrace/scanstack.tex}}%
      \onslide*<6>{\providecommand\step{5}\input{diagrams/backtrace/scanstack.tex}}%
    \end{column}
  \end{columns}
\end{frame}

% Duplicate of previous-previous slide
\begin{frame}{Backtraces}{Continuing on \funcname{caml\_stash\_backtrace}}
  \begin{columns}[c]
    \begin{column}{0.5\textwidth}
      \minipage[c][0.75\textheight][s]{\columnwidth}
      \begin{itemize}
          \color{gray}
        \item A C function provided by the runtime (specific to native code)
        \item It scans the stack, between the stack pointer at the raising point up to the next trap, in order to collect the names of the detected function frames
        \item It uses the same technique and information as the Garbage Collector (GC) uses to scan the values live on the stack
      \end{itemize}
      \begin{itemize}
        \item For each function frame detected, store a pointer to the \typename{frame\_descr} in the \localname{domain\_state->backtrace\_buffer} array, effectively constructing the backtrace
      \end{itemize}
      \onslide<2->{
        \begin{itemize}
            \smallskip
          \item Constructed backtrace:
            \funcname{definitely\_raise},
            \onslide<3->{\funcname{probably\_raise}}
        \end{itemize}
      }
      \vfill
      \mintocaml[firstline=9,lastline=13,highlightlines=12]{../src/backtrace/backtrace.ml}
      \endminipage
    \end{column}
    \begin{column}{0.5\textwidth}
      \raggedleft
      \onslide*<1>{\listStashBacktrace[highlightlines={114-115}]{}}
      \onslide*<2>{\providecommand\step{3}\input{diagrams/backtrace/backtrace.tex}}%
      \onslide*<3>{\providecommand\step{4}\input{diagrams/backtrace/backtrace.tex}}%
    \end{column}
  \end{columns}
\end{frame}

\begin{frame}{Backtraces}{Back to \funcname{caml\_raise\_exn}}
  \begin{columns}[c]
    \begin{column}{0.5\textwidth}
      \minipage[c][0.66\textheight][s]{\columnwidth}
      \begin{itemize}\color{gray}
        \item Checks wheteher exceptions backtrace should be recorded, by checking the \funcarg{domain\_state->backtrace\_active} value
        \item Switches to the C stack to be able to execute C code
        \item Calls \funcname{caml\_stash\_backtrace}, to collect the backtrace up to the next trap
      \end{itemize}
      \onslide*<2->{
        \begin{itemize}
          \item<2-> Executes the exception handler in \funcname{probably\_raise} with \funcname{RESTORE\_EXN\_HANDLER\_OCAML}
            \begin{itemize}
              \item Set \regname{RSP} to \localname{domain\_state->exn\_handler} value
              \item Restore \localname{domain\_state->exn\_handler} from trap
              \item Jump to the handler code with \funcname{ret}
            \end{itemize}
        \end{itemize}
      }
      \vfill
      \onslide*<1>{\mintocaml[firstline=9,lastline=13,highlightlines=12]{../src/backtrace/backtrace.ml}}
      \onslide*<2->{\mintocaml[firstline=15,lastline=21,highlightlines={19-21}]{../src/backtrace/backtrace.ml}}
      \endminipage
    \end{column}
    \begin{column}{0.5\textwidth}
      \centering
      \onslide*<1>{
        \mintc[firstline=190,lastline=192]{../ocaml/runtime/amd64.S}
        \mintc[firstline=234,lastline=237]{../ocaml/runtime/amd64.S}
      }
      \onslide*<2>{
        \mintc[firstline=190,lastline=192,highlightlines={191-192}]{../ocaml/runtime/amd64.S}
        \mintc[firstline=234,lastline=237,highlightlines={235-237}]{../ocaml/runtime/amd64.S}
      }
      \onslide*<1>{\listCamlRaiseExn[highlightlines=720]{}}
      \onslide*<2>{\listCamlRaiseExn[highlightlines=722]{}}
      \onslide*<3->{\providecommand\step{5}\input{diagrams/backtrace/backtrace.tex}}
    \end{column}
  \end{columns}
\end{frame}

\begin{frame}{Backtraces}{\funcname{caml\_probably\_raise}'s handler doesn't catch \typename{ExnC}}
  \begin{columns}[c]
    \begin{column}{0.45\textwidth}
      \minipage[c][0.75\textheight][s]{\columnwidth}
      \begin{itemize}
        \item<1-> \funcname{match \dots with}\footnotemark pattern matching can also match exceptions
        \item<1-> The CMM of \funcname{probably\_raise} shows that this \funcname{match \dots with} is implemented with a pattern matching and a \funcname{try \dots with}
        \item<1-> But this one only matches on \typename{ExnA} exception
        \item<2-> Since the pattern matching fails to catch the exception, \funcname{reraise} is called with our current \typename{ExnC} exception
        \item<3-> Disassembly shows that \funcname{reraise} is actually a call to \funcname{caml\_reraise\_exn}, which is also an assembly function provided by the runtime
      \end{itemize}
      \vfill
      \mintocaml[firstline=15,lastline=21,highlightlines={18-21}]{../src/backtrace/backtrace.ml}
      \endminipage
    \end{column}
    \begin{column}{0.55\textwidth}
      \centering
      \onslide*<1>{\mintcmm[highlightlines={10-18}]{../src/backtrace/camlBacktrace\_\_probably\_raise\_351.cmm}}
      \onslide*<2>{\listcmm[highlightlines=18]{../src/backtrace/camlBacktrace\_\_probably\_raise_351.cmm}{CMM of probably\_raise}}
      \onslide*<3>{\mintobjdump[firstline=5,lastline=35,highlightlines=31]{../src/backtrace/camlBacktrace\_\_probably\_raise_351.objdump}}
    \end{column}
  \end{columns}
  \only<1>{\footnotetext[1]{https://blog.janestreet.com/pattern-matching-and-exception-handling-unite/}}
\end{frame}

\newcommand\listCamlReraiseExn[1][]{
  \listasm[firstline=727,lastline=734,#1]{../ocaml/runtime/amd64.S}{runtime/amd64.S:727}}

\begin{frame}{Backtraces}{Introducing \funcname{caml\_reraise\_exn}}
  \begin{columns}[c]
    \begin{column}{0.5\textwidth}
      \minipage[c][0.66\textheight][s]{\columnwidth}
      \begin{itemize}
        \item<1-> The exception have to be reraised to the next handler
        \item<2-> First, checks if the backtrace should be collected with \funcarg{domain\_state->backtrace\_active}
        \only<3>{
        \begin{itemize}
            \item If not, behaves like a \funcname{raise\_notrace} with \funcname{RESTORE\_EXN\_HANDLER\_OCAML}
        \end{itemize}
        }
        \item<4-> Jumps into \funcname{caml\_raise\_exn}
        \item<5-> Calls \funcname{caml\_stash\_backtrace} again, to scan the next part of the stack: from the trap just pop-ed to the next trap
      \end{itemize}
      \vfill
      \mintocaml[firstline=15,lastline=21,highlightlines={18-21}]{../src/backtrace/backtrace.ml}
      \endminipage
    \end{column}
    \begin{column}{0.5\textwidth}
      \raggedleft
      \onslide*<1>{\listCamlReraiseExn{}}
      \onslide*<2>{\listCamlReraiseExn[highlightlines=729]{}}
      \onslide*<3>{
        \mintc[firstline=190,lastline=192,highlightlines={191-192}]{../ocaml/runtime/amd64.S}
        \mintc[firstline=234,lastline=237,highlightlines={235-237}]{../ocaml/runtime/amd64.S}
        \medskip
        \listCamlReraiseExn[highlightlines=731]{}
      }
      \onslide*<4-5>{
        % TODO refactor with \listCamlReraiseExn
        \mintasm[firstline=727,lastline=734,highlightlines=730]{../ocaml/runtime/amd64.S}
        \medskip
      }
      \onslide*<4>{\listCamlRaiseExn[highlightlines=711]{}}
      \onslide*<5>{\listCamlRaiseExn[highlightlines=720]{}}
    \end{column}
  \end{columns}
\end{frame}

\begin{frame}{Backtraces}{Backtrace collection continues}
  \begin{columns}[c]
    \begin{column}{0.55\textwidth}
      \minipage[c][0.75\textheight][s]{\columnwidth}
      \begin{itemize}
        \item<1-> \funcname{caml\_stash\_backtrace} scans the older part of the stack, up to the next trap
        \item<6-> Controls jumps to the nested exception handler in \funcname{maybe\_raise}, which matches only on \typename{ExnB}
        \item<7-> \funcname{caml\_reraise\_exn} is called again, backtrace collection resumes with \funcname{caml\_stash\_backtrace}
      \end{itemize}
      \medskip
      \begin{itemize}
        \item<2-> Constructed backtrace:
        \begin{itemize}
          \item <2-> \funcname{definitely\_raise}
          \item <2-> \funcname{probably\_raise}
          \item <3-> \funcname{probably\_raise} (re-raise)
          \item <4-> \funcname{maybe\_raise}
          \item <5-> \funcname{main}
          \item <8-> \funcname{main} (re-raise)
        \end{itemize}
      \end{itemize}
      \vfill
      \onslide*<1-5>{\mintocaml[firstline=15,lastline=21]{../src/backtrace/backtrace.ml}}
      \onslide*<6>{\mintocaml[firstline=28,lastline=34,highlightlines=31]{../src/backtrace/backtrace.ml}}
      \onslide*<7->{\mintocaml[firstline=28,lastline=34,highlightlines=33]{../src/backtrace/backtrace.ml}}
      \endminipage
    \end{column}
    \begin{column}{0.45\textwidth}
      \raggedleft
      \onslide*<1>{\providecommand\step{6}\input{diagrams/backtrace/backtrace.tex}}%
      \onslide*<2>{\providecommand\step{6}\input{diagrams/backtrace/backtrace.tex}}%
      \onslide*<3>{\providecommand\step{7}\input{diagrams/backtrace/backtrace.tex}}%
      \onslide*<4>{\providecommand\step{8}\input{diagrams/backtrace/backtrace.tex}}%
      \onslide*<5>{\providecommand\step{9}\input{diagrams/backtrace/backtrace.tex}}%
      \onslide*<6>{\providecommand\step{10}\input{diagrams/backtrace/backtrace.tex}}%
      \onslide*<7>{\providecommand\step{11}\input{diagrams/backtrace/backtrace.tex}}%
      \onslide*<8>{\providecommand\step{12}\input{diagrams/backtrace/backtrace.tex}}%
      \onslide*<9>{\providecommand\step{13}\input{diagrams/backtrace/backtrace.tex}}%
    \end{column}
  \end{columns}
\end{frame}

\begin{frame}{Backtraces}{Printing the backtrace}
  \begin{columns}[c]
    \begin{column}{0.45\textwidth}
      \minipage[c][0.66\textheight][s]{\columnwidth}
      \begin{itemize}
        \item<1-> The exception backtrace can now be printed to \funcarg{stdout} with \funcname{Printexc.print_backtrace}
        \item<2-> First the exception backtrace is retrieved into an OCaml array with \funcname{get_raw_backtrace }
        \item<3-> Which is implemented as the \funcname{caml_get_exception_raw_backtrace} C function in the runtime
        \item<4-> And finally printed with \funcname{print_exception_backtrace}
      \end{itemize}
      \vfill
      \mintocaml[firstline=28,lastline=34,highlightlines=33]{../src/backtrace/backtrace.ml}
      \endminipage
    \end{column}
    \begin{column}{0.55\textwidth}
      \onslide*<1,2>{
        \mintocaml[firstline=100,lastline=101]{../ocaml/stdlib/printexc.ml}
      }
      \only<1>{
        \listocaml[firstline=170,lastline=172]{../ocaml/stdlib/printexc.ml}{stdlib/printexc.ml:170}
      }
      \only<2>{
        \listocaml[firstline=170,lastline=172,highlightlines=172]{../ocaml/stdlib/printexc.ml}{stdlib/printexc.ml:170}
      }
      \only<3>{
        \mintc[firstline=154,lastline=156]{../ocaml/runtime/backtrace.c}
        \listc[firstline=171,lastline=192,highlightlines={184-187}]{../ocaml/runtime/backtrace.c}{runtime/backtrace.c:154}
      }
      \only<4>{
        \listocaml[firstline=155,lastline=165]{../ocaml/stdlib/printexc.ml}{stdlib/printexc.ml:55}
      }
    \end{column}
  \end{columns}
\end{frame}


\subsection{The multiple kinds of raise}
\frameSubsection{}{}

\begin{frame}{Backtraces}{When backtraces are not needed}
  \begin{columns}[c]
    \begin{column}{0.45\textwidth}
      \minipage[c][0.66\textheight][s]{\columnwidth}
      \begin{itemize}
        \item<1-> What if the backtrace for an exception is never needed?
        \item<2-> It's possible to raise the exception explicitly with \funcname{raise\_notrace} to make raising as cheap as possible
        \item<3-> An example of use in the \funcname{dynlink} library of the OCaml compiler
      \end{itemize}
      \vfill
      \onslide<3->{
      \listocaml[firstline=205,lastline=209,highlightlines=207]{../ocaml/otherlibs/dynlink/dynlink\_compilerlibs/misc.ml}{otherlibs/dynlink/dynlink\_compilerlibs/misc.ml:205}
      }
      \endminipage
    \end{column}
    \begin{column}{0.55\textwidth}
    \only<4>{
      \mintcmm[highlightlines=3]{../src/notrace/camlNotrace\_\_fun_347.cmm}
      \listcmm{../src/notrace/camlNotrace\_\_all\_somes\_327.cmm}{all\_somes}
    }
    \onslide*<5->{
      \listobjdump[highlightlines={6-9}]{../src/notrace/camlNotrace\_\_fun_347.objdump}{anonymous function from \funcname{all\_somes}}
    }
    \end{column}
  \end{columns}
\end{frame}

\begin{frame}{Backtraces}{Summary table of the multiple kinds of raise}
  \begin{itemize}
    \item When debugging information is enabled and if the backtrace of an exception isn't needed, it's possible to raise with \funcname{raise\_notrace} for performance
    \item When debugging information is \emph{not} enabled, the compiler optimises \funcname{raise} calls to inline assembly (same effect as using \funcname{raise\_notrace})
  \end{itemize}
  \bigskip
  \centering
  \begin{tabular}{| l | c | c | c | c |}
    \hline
    \parbox{10em}{Debug information \\ (\funcarg{-g} flag)}  & \multicolumn{2}{|c|}{Disabled}                                     & \multicolumn{2}{|c|}{Enabled} \\
    \hline
    Raise kind                                               & \funcname{raise} & \funcname{raise\_notrace}                       & \funcname{raise} & \funcname{raise\_notrace} \\
    \hline
    Compilation                                              & \parbox{8em}{inline assembly\\ (optimization)} & inline assembly   & \funcname{caml\_raise\_exn} & inline assembly \\
    \hline
  \end{tabular}
\end{frame}


%
% Takeaway #5
%
\subsection*{Takeaway \#5}
\frameSubsectionTakeaway{}
\againframe<5>{takeaway}
}%
      \onslide*<2>{\providecommand\step{6}%
% Backtraces
%
\subsection{One advantage of raising with debugging information}
\frameSubsection{}{}

\begin{frame}{Backtraces}{Debugging information \& source code}
  \begin{columns}[c]
    \begin{column}{0.5\textwidth}
      \begin{itemize}
        \item Raising an exception is fun and games, but how do we know where the exception originated? $\rightarrow$ With the backtrace associated with the exception!
        \item In order to get a backtrace from the point where an exception was raised, it's necessary to compile with debugging information enabled (\funcarg{-g} flag for the compiler)
        \item Same example as in the `Nested handlers' section
      \end{itemize}
    \end{column}
    \begin{column}{0.5\textwidth}
      \listocaml[firstline=9,lastline=34]{../src/backtrace/backtrace.ml}{backtrace/backtrace.ml}
    \end{column}
  \end{columns}
\end{frame}

\begin{frame}{Backtraces}{CMM and disassembly of \funcname{definitely\_raise}}
  \begin{columns}[c]
    \begin{column}{0.5\textwidth}
      \minipage[c][0.66\textheight][s]{\columnwidth}
      \begin{itemize}
        \item<1-> An effect of compiling with debugging information (\funcarg{-g}): the CMM no longer uses \funcname{raise\_notrace} to raise the exception but \funcname{raise} instead
        \item<2-> Disassembly shows that CMM \funcname{raise} is actually a call to \funcname{caml\_raise\_exn}, a function in assembler provided by the runtime
      \end{itemize}
      \vfill
      \mintocaml[firstline=9,lastline=13,highlightlines=12]{../src/backtrace/backtrace.ml}
      \endminipage
    \end{column}
    \begin{column}{0.5\textwidth}
      \onslide*<1>{
        \mintcmm[highlightlines=5]{../src/backtrace/camlBacktrace\_\_definitely\_raise\_347.cmm}
      }
      \onslide*<2>{
        \mintobjdump[firstline=2,highlightlines=14]{../src/backtrace/camlBacktrace\_\_definitely\_raise\_347.objdump}
      }
    \end{column}
  \end{columns}
\end{frame}

\begin{frame}{Backtraces}{Introducing \funcname{caml\_raise\_exn}}
  \begin{columns}[c]
    \begin{column}{0.5\textwidth}
      \minipage[c][0.66\textheight][s]{\columnwidth}
      \onslide*<1>{
        \begin{itemize}
          \item Raising the exception is performed using \funcname{caml\_raise\_exn} which is
            \begin{itemize}
              \item An assembly function provided by the runtime
              \item Architecture specific
            \end{itemize}
          \item Let's see what it does differently from \funcname{raise\_notrace}\dots
          \item We are about to raise \typename{ExnC} with \funcname{caml\_raise\_exn} from \funcname{definitely\_raise}
        \end{itemize}
      }
      \onslide*<2>{
        \mintobjdump[firstline=2,highlightlines=14]{../src/backtrace/camlBacktrace\_\_definitely\_raise\_347.objdump}
      }
      \vfill
      \mintocaml[firstline=9,lastline=13,highlightlines=12]{../src/backtrace/backtrace.ml}
      \endminipage
    \end{column}
    \begin{column}{0.5\textwidth}
      \centering
      \providecommand\step{1}\input{diagrams/backtrace/backtrace.tex}
    \end{column}
  \end{columns}
\end{frame}

\begin{frame}{Backtraces}{But before that, we need more fields from \typename{caml\_domain\_state}}
  \begin{columns}
    \begin{column}{0.5\textwidth}
      \begin{itemize}
        \item Remember \typename{caml\_domain\_state} the per-domain runtime state?
        \item Some other members are of interest to us now\dots
        \item \localname{backtrace\_active} is used to know if the runtime should collect backtraces
        \item \localname{backtrace\_active} can be set by
          \begin{itemize}
            \item The \funcarg{b} flag in the \funcname{OCAMLRUNPARAM} environment variable
            \item \mintinline{ocaml}{Printexc.record_backtrace : bool -> unit} \footnotemark
          \end{itemize}
      \end{itemize}
    \end{column}
    \begin{column}{0.5\textwidth}
      \begin{listing}
        \mintc[highlightlines={9-11}]{src/ocaml/domain\_state\_backtrace.h}
        \caption{runtime/caml/domain\_state.h}
      \end{listing}
    \end{column}
  \end{columns}
  \footnotetext[1]{https://v2.ocaml.org/api/Printexc.html}
\end{frame}

\newcommand\listCamlRaiseExn[1][]{
  \listasm[firstline=702,lastline=725,#1]{../ocaml/runtime/amd64.S}{runtime/amd64.S:702}}

\begin{frame}{Backtraces}{Walk through \funcname{caml\_raise\_exn}}
  \begin{columns}[T]
    \begin{column}{0.5\textwidth}
      \begin{itemize}
        \item<1-> First checks whether exceptions backtrace should be recorded
        \item<1-> By checking the \funcarg{domain\_state->backtrace\_active} value
        \item<2-> If not, acts as \funcname{raise\_notrace} with \funcname{RESTORE\_EXN\_HANDLER\_OCAML}
          \begin{itemize}
            \item Set \regname{RSP} to \localname{domain\_state->exn\_handler} value
            \item Restore \localname{domain\_state->exn\_handler} from trap
            \item Jump with \funcname{ret} (same effect as using \funcname{pop} and \funcname{jmp})
          \end{itemize}
        \item<3-> Otherwise, switches to the C stack to be able to execute C code
        \item<4-> Calls \funcname{caml\_stash\_backtrace}, a C function provided by the runtime\ignorespaces
          \onslide<6->{, with
            \begin{itemize}
              \item \funcarg{exn}: exception value
              \item \funcarg{pc}: code address of the raising point
              \item \funcarg{sp}: stack address of the raising function
              \item \funcarg{trapsp}: stack address of the next handler
            \end{itemize}
          }
      \end{itemize}
      \bigskip
      \mintocaml[firstline=9,lastline=13,highlightlines=12]{../src/backtrace/backtrace.ml}
    \end{column}
    \begin{column}{0.5\textwidth}
      \raggedleft
      \onslide*<1,3-4>{
        \mintc[firstline=190,lastline=192]{../ocaml/runtime/amd64.S}
        \mintc[firstline=234,lastline=237]{../ocaml/runtime/amd64.S}
      }
      \onslide*<2>{
        \mintc[firstline=190,lastline=192,highlightlines={191-192}]{../ocaml/runtime/amd64.S}
        \mintc[firstline=234,lastline=237,highlightlines={235-237}]{../ocaml/runtime/amd64.S}
      }
      \onslide*<1>{\listCamlRaiseExn[highlightlines=705]{}}%
      \onslide*<2>{\listCamlRaiseExn[highlightlines={707-708}]{}}%
      \onslide*<3>{\listCamlRaiseExn[highlightlines={712-714}]{}}%
      \onslide*<4>{\listCamlRaiseExn[highlightlines={715-720}]{}}%
      \onslide*<5>{\providecommand\step{1}\input{diagrams/backtrace/backtrace.tex}}%
      \onslide*<6>{\providecommand\step{2}\input{diagrams/backtrace/backtrace.tex}}%
    \end{column}
  \end{columns}
\end{frame}

\newcommand\listStashBacktrace[1][]{
  \listc[firstline=91,lastline=120,#1]{../ocaml/runtime/backtrace\_nat.c}{runtime/backtrace\_nat.c:91}}

\begin{frame}{Backtraces}{Introducing \funcname{caml\_stash\_backtrace}}
  \begin{columns}[c]
    \begin{column}{0.5\textwidth}
      \minipage[c][0.66\textheight][s]{\columnwidth}
      \begin{itemize}
        \item<1-> A C function provided by the runtime (specific to native code)
        \item<2-> It scans the stack, between the stack pointer at the raising point up to the next trap, in order to collect the names of the detected function frames
          \onslide*<2>{
            \begin{itemize}
              \item And more information, like line number, column number...
            \end{itemize}
          }
        \item<3-> It uses the same technique and information as the Garbage Collector (GC) uses to scan the values live on the stack
          \onslide*<4>{
            \begin{itemize}
              \item \typename{frame\_descr} are at the core of stack unwinding
            \end{itemize}
          }
      \end{itemize}
      \vfill
      \mintocaml[firstline=9,lastline=13,highlightlines=12]{../src/backtrace/backtrace.ml}
      \endminipage
    \end{column}
    \begin{column}{0.5\textwidth}
      \onslide*<1>{\listStashBacktrace[highlightlines=91]{}}
      \onslide*<2>{\listStashBacktrace[highlightlines={91,117-118}]{}}
      \onslide*<3->{\listStashBacktrace[highlightlines={94,106,109-110}]{}}
    \end{column}
  \end{columns}
\end{frame}

\newcommand\listFrameDescr[1][]{
  \listocaml[firstline=111,lastline=124,#1]{../ocaml/asmcomp/emitaux.ml}{asmcomp/emitaux.ml:111}}
\newcommand\listDebugInfo[1][]{
  \listocaml[firstline=38,lastline=49,#1]{../ocaml/lambda/debuginfo.mli}{lambda/debuginfo.mli:38}}

\begin{frame}{Backtraces}{\typename{frame\_descr} and \typename{frame\_debuginfo} in a nutshell}
  \begin{columns}[c]
    \begin{column}{0.5\textwidth}
      \begin{itemize}
        \item<1-> For every return address in an OCaml program, there is a associated \typename{frame\_descr}
        \item<2-> A \typename{frame\_descr} specifies
        \begin{itemize}
          \item<2-> The size of the function stack frame
          \item<3-> Where live values are on the stack (so that the GC can mark them as live and prevent from being swept)
        \end{itemize}
        \item<4-> When compiled with debug information, \typename{frame\_descr} will be linked to a \typename{frame\_debuginfo}
        \begin{itemize}
          \item<5-> Contains information about the position in the source code (file name, line and column number)
        \end{itemize}
      \end{itemize}
    \end{column}
    \begin{column}{0.5\textwidth}
        \onslide*<1>{\listFrameDescr[highlightlines={118-122}]{}}
        \onslide*<2>{\listFrameDescr[highlightlines={120}]{}}
        \onslide*<3>{\listFrameDescr[highlightlines={121}]{}}
        \onslide*<4->{\listFrameDescr[highlightlines={122}]{}}
        \medskip
        \onslide*<1-3>{\listDebugInfo{}}
        \onslide*<4>{\listDebugInfo[highlightlines={38,49}]{}}
        \onslide*<5>{\listDebugInfo[highlightlines={39-46}]{}}
    \end{column}
  \end{columns}
\end{frame}

\begin{frame}{Backtraces}{Scanning the stack with \typename{frame\_descr}}
  \begin{columns}[c]
    \begin{column}{0.5\textwidth}
      \begin{itemize}
        \item Given the return address of the code that raises the exception by calling \funcname{caml\_raise\_exn} (\funcarg{pc} argument of \funcname{caml\_stash\_backtrace})
        \item And the position of the stack pointer (\funcarg{sp} argument of \funcname{caml\_stash\_backtrace})
        \item The frame size given by \typename{frame\_descr}.\funcarg{frame\_size}, allows to find the end of the previous frame
        \item And the return address of the caller of this function
        \bigskip
        \onslide<3->{
        \item Note that traps (here the reddish \funcname{ExnA}, \funcname{ExnB} and \funcname{ExnC} blocks) belong to the frame that installed them, and so the frame size changes when traps are removed
        }
      \end{itemize}
    \end{column}
    \begin{column}{0.5\textwidth}
      \raggedleft
      \onslide*<1>{\providecommand\step{2}\input{diagrams/backtrace/backtrace.tex}}%
      \onslide*<2>{\providecommand\step{1}\input{diagrams/backtrace/scanstack.tex}}%
      \onslide*<3>{\providecommand\step{2}\input{diagrams/backtrace/scanstack.tex}}%
      \onslide*<4>{\providecommand\step{3}\input{diagrams/backtrace/scanstack.tex}}%
      \onslide*<5>{\providecommand\step{4}\input{diagrams/backtrace/scanstack.tex}}%
      \onslide*<6>{\providecommand\step{5}\input{diagrams/backtrace/scanstack.tex}}%
    \end{column}
  \end{columns}
\end{frame}

% Duplicate of previous-previous slide
\begin{frame}{Backtraces}{Continuing on \funcname{caml\_stash\_backtrace}}
  \begin{columns}[c]
    \begin{column}{0.5\textwidth}
      \minipage[c][0.75\textheight][s]{\columnwidth}
      \begin{itemize}
          \color{gray}
        \item A C function provided by the runtime (specific to native code)
        \item It scans the stack, between the stack pointer at the raising point up to the next trap, in order to collect the names of the detected function frames
        \item It uses the same technique and information as the Garbage Collector (GC) uses to scan the values live on the stack
      \end{itemize}
      \begin{itemize}
        \item For each function frame detected, store a pointer to the \typename{frame\_descr} in the \localname{domain\_state->backtrace\_buffer} array, effectively constructing the backtrace
      \end{itemize}
      \onslide<2->{
        \begin{itemize}
            \smallskip
          \item Constructed backtrace:
            \funcname{definitely\_raise},
            \onslide<3->{\funcname{probably\_raise}}
        \end{itemize}
      }
      \vfill
      \mintocaml[firstline=9,lastline=13,highlightlines=12]{../src/backtrace/backtrace.ml}
      \endminipage
    \end{column}
    \begin{column}{0.5\textwidth}
      \raggedleft
      \onslide*<1>{\listStashBacktrace[highlightlines={114-115}]{}}
      \onslide*<2>{\providecommand\step{3}\input{diagrams/backtrace/backtrace.tex}}%
      \onslide*<3>{\providecommand\step{4}\input{diagrams/backtrace/backtrace.tex}}%
    \end{column}
  \end{columns}
\end{frame}

\begin{frame}{Backtraces}{Back to \funcname{caml\_raise\_exn}}
  \begin{columns}[c]
    \begin{column}{0.5\textwidth}
      \minipage[c][0.66\textheight][s]{\columnwidth}
      \begin{itemize}\color{gray}
        \item Checks wheteher exceptions backtrace should be recorded, by checking the \funcarg{domain\_state->backtrace\_active} value
        \item Switches to the C stack to be able to execute C code
        \item Calls \funcname{caml\_stash\_backtrace}, to collect the backtrace up to the next trap
      \end{itemize}
      \onslide*<2->{
        \begin{itemize}
          \item<2-> Executes the exception handler in \funcname{probably\_raise} with \funcname{RESTORE\_EXN\_HANDLER\_OCAML}
            \begin{itemize}
              \item Set \regname{RSP} to \localname{domain\_state->exn\_handler} value
              \item Restore \localname{domain\_state->exn\_handler} from trap
              \item Jump to the handler code with \funcname{ret}
            \end{itemize}
        \end{itemize}
      }
      \vfill
      \onslide*<1>{\mintocaml[firstline=9,lastline=13,highlightlines=12]{../src/backtrace/backtrace.ml}}
      \onslide*<2->{\mintocaml[firstline=15,lastline=21,highlightlines={19-21}]{../src/backtrace/backtrace.ml}}
      \endminipage
    \end{column}
    \begin{column}{0.5\textwidth}
      \centering
      \onslide*<1>{
        \mintc[firstline=190,lastline=192]{../ocaml/runtime/amd64.S}
        \mintc[firstline=234,lastline=237]{../ocaml/runtime/amd64.S}
      }
      \onslide*<2>{
        \mintc[firstline=190,lastline=192,highlightlines={191-192}]{../ocaml/runtime/amd64.S}
        \mintc[firstline=234,lastline=237,highlightlines={235-237}]{../ocaml/runtime/amd64.S}
      }
      \onslide*<1>{\listCamlRaiseExn[highlightlines=720]{}}
      \onslide*<2>{\listCamlRaiseExn[highlightlines=722]{}}
      \onslide*<3->{\providecommand\step{5}\input{diagrams/backtrace/backtrace.tex}}
    \end{column}
  \end{columns}
\end{frame}

\begin{frame}{Backtraces}{\funcname{caml\_probably\_raise}'s handler doesn't catch \typename{ExnC}}
  \begin{columns}[c]
    \begin{column}{0.45\textwidth}
      \minipage[c][0.75\textheight][s]{\columnwidth}
      \begin{itemize}
        \item<1-> \funcname{match \dots with}\footnotemark pattern matching can also match exceptions
        \item<1-> The CMM of \funcname{probably\_raise} shows that this \funcname{match \dots with} is implemented with a pattern matching and a \funcname{try \dots with}
        \item<1-> But this one only matches on \typename{ExnA} exception
        \item<2-> Since the pattern matching fails to catch the exception, \funcname{reraise} is called with our current \typename{ExnC} exception
        \item<3-> Disassembly shows that \funcname{reraise} is actually a call to \funcname{caml\_reraise\_exn}, which is also an assembly function provided by the runtime
      \end{itemize}
      \vfill
      \mintocaml[firstline=15,lastline=21,highlightlines={18-21}]{../src/backtrace/backtrace.ml}
      \endminipage
    \end{column}
    \begin{column}{0.55\textwidth}
      \centering
      \onslide*<1>{\mintcmm[highlightlines={10-18}]{../src/backtrace/camlBacktrace\_\_probably\_raise\_351.cmm}}
      \onslide*<2>{\listcmm[highlightlines=18]{../src/backtrace/camlBacktrace\_\_probably\_raise_351.cmm}{CMM of probably\_raise}}
      \onslide*<3>{\mintobjdump[firstline=5,lastline=35,highlightlines=31]{../src/backtrace/camlBacktrace\_\_probably\_raise_351.objdump}}
    \end{column}
  \end{columns}
  \only<1>{\footnotetext[1]{https://blog.janestreet.com/pattern-matching-and-exception-handling-unite/}}
\end{frame}

\newcommand\listCamlReraiseExn[1][]{
  \listasm[firstline=727,lastline=734,#1]{../ocaml/runtime/amd64.S}{runtime/amd64.S:727}}

\begin{frame}{Backtraces}{Introducing \funcname{caml\_reraise\_exn}}
  \begin{columns}[c]
    \begin{column}{0.5\textwidth}
      \minipage[c][0.66\textheight][s]{\columnwidth}
      \begin{itemize}
        \item<1-> The exception have to be reraised to the next handler
        \item<2-> First, checks if the backtrace should be collected with \funcarg{domain\_state->backtrace\_active}
        \only<3>{
        \begin{itemize}
            \item If not, behaves like a \funcname{raise\_notrace} with \funcname{RESTORE\_EXN\_HANDLER\_OCAML}
        \end{itemize}
        }
        \item<4-> Jumps into \funcname{caml\_raise\_exn}
        \item<5-> Calls \funcname{caml\_stash\_backtrace} again, to scan the next part of the stack: from the trap just pop-ed to the next trap
      \end{itemize}
      \vfill
      \mintocaml[firstline=15,lastline=21,highlightlines={18-21}]{../src/backtrace/backtrace.ml}
      \endminipage
    \end{column}
    \begin{column}{0.5\textwidth}
      \raggedleft
      \onslide*<1>{\listCamlReraiseExn{}}
      \onslide*<2>{\listCamlReraiseExn[highlightlines=729]{}}
      \onslide*<3>{
        \mintc[firstline=190,lastline=192,highlightlines={191-192}]{../ocaml/runtime/amd64.S}
        \mintc[firstline=234,lastline=237,highlightlines={235-237}]{../ocaml/runtime/amd64.S}
        \medskip
        \listCamlReraiseExn[highlightlines=731]{}
      }
      \onslide*<4-5>{
        % TODO refactor with \listCamlReraiseExn
        \mintasm[firstline=727,lastline=734,highlightlines=730]{../ocaml/runtime/amd64.S}
        \medskip
      }
      \onslide*<4>{\listCamlRaiseExn[highlightlines=711]{}}
      \onslide*<5>{\listCamlRaiseExn[highlightlines=720]{}}
    \end{column}
  \end{columns}
\end{frame}

\begin{frame}{Backtraces}{Backtrace collection continues}
  \begin{columns}[c]
    \begin{column}{0.55\textwidth}
      \minipage[c][0.75\textheight][s]{\columnwidth}
      \begin{itemize}
        \item<1-> \funcname{caml\_stash\_backtrace} scans the older part of the stack, up to the next trap
        \item<6-> Controls jumps to the nested exception handler in \funcname{maybe\_raise}, which matches only on \typename{ExnB}
        \item<7-> \funcname{caml\_reraise\_exn} is called again, backtrace collection resumes with \funcname{caml\_stash\_backtrace}
      \end{itemize}
      \medskip
      \begin{itemize}
        \item<2-> Constructed backtrace:
        \begin{itemize}
          \item <2-> \funcname{definitely\_raise}
          \item <2-> \funcname{probably\_raise}
          \item <3-> \funcname{probably\_raise} (re-raise)
          \item <4-> \funcname{maybe\_raise}
          \item <5-> \funcname{main}
          \item <8-> \funcname{main} (re-raise)
        \end{itemize}
      \end{itemize}
      \vfill
      \onslide*<1-5>{\mintocaml[firstline=15,lastline=21]{../src/backtrace/backtrace.ml}}
      \onslide*<6>{\mintocaml[firstline=28,lastline=34,highlightlines=31]{../src/backtrace/backtrace.ml}}
      \onslide*<7->{\mintocaml[firstline=28,lastline=34,highlightlines=33]{../src/backtrace/backtrace.ml}}
      \endminipage
    \end{column}
    \begin{column}{0.45\textwidth}
      \raggedleft
      \onslide*<1>{\providecommand\step{6}\input{diagrams/backtrace/backtrace.tex}}%
      \onslide*<2>{\providecommand\step{6}\input{diagrams/backtrace/backtrace.tex}}%
      \onslide*<3>{\providecommand\step{7}\input{diagrams/backtrace/backtrace.tex}}%
      \onslide*<4>{\providecommand\step{8}\input{diagrams/backtrace/backtrace.tex}}%
      \onslide*<5>{\providecommand\step{9}\input{diagrams/backtrace/backtrace.tex}}%
      \onslide*<6>{\providecommand\step{10}\input{diagrams/backtrace/backtrace.tex}}%
      \onslide*<7>{\providecommand\step{11}\input{diagrams/backtrace/backtrace.tex}}%
      \onslide*<8>{\providecommand\step{12}\input{diagrams/backtrace/backtrace.tex}}%
      \onslide*<9>{\providecommand\step{13}\input{diagrams/backtrace/backtrace.tex}}%
    \end{column}
  \end{columns}
\end{frame}

\begin{frame}{Backtraces}{Printing the backtrace}
  \begin{columns}[c]
    \begin{column}{0.45\textwidth}
      \minipage[c][0.66\textheight][s]{\columnwidth}
      \begin{itemize}
        \item<1-> The exception backtrace can now be printed to \funcarg{stdout} with \funcname{Printexc.print_backtrace}
        \item<2-> First the exception backtrace is retrieved into an OCaml array with \funcname{get_raw_backtrace }
        \item<3-> Which is implemented as the \funcname{caml_get_exception_raw_backtrace} C function in the runtime
        \item<4-> And finally printed with \funcname{print_exception_backtrace}
      \end{itemize}
      \vfill
      \mintocaml[firstline=28,lastline=34,highlightlines=33]{../src/backtrace/backtrace.ml}
      \endminipage
    \end{column}
    \begin{column}{0.55\textwidth}
      \onslide*<1,2>{
        \mintocaml[firstline=100,lastline=101]{../ocaml/stdlib/printexc.ml}
      }
      \only<1>{
        \listocaml[firstline=170,lastline=172]{../ocaml/stdlib/printexc.ml}{stdlib/printexc.ml:170}
      }
      \only<2>{
        \listocaml[firstline=170,lastline=172,highlightlines=172]{../ocaml/stdlib/printexc.ml}{stdlib/printexc.ml:170}
      }
      \only<3>{
        \mintc[firstline=154,lastline=156]{../ocaml/runtime/backtrace.c}
        \listc[firstline=171,lastline=192,highlightlines={184-187}]{../ocaml/runtime/backtrace.c}{runtime/backtrace.c:154}
      }
      \only<4>{
        \listocaml[firstline=155,lastline=165]{../ocaml/stdlib/printexc.ml}{stdlib/printexc.ml:55}
      }
    \end{column}
  \end{columns}
\end{frame}


\subsection{The multiple kinds of raise}
\frameSubsection{}{}

\begin{frame}{Backtraces}{When backtraces are not needed}
  \begin{columns}[c]
    \begin{column}{0.45\textwidth}
      \minipage[c][0.66\textheight][s]{\columnwidth}
      \begin{itemize}
        \item<1-> What if the backtrace for an exception is never needed?
        \item<2-> It's possible to raise the exception explicitly with \funcname{raise\_notrace} to make raising as cheap as possible
        \item<3-> An example of use in the \funcname{dynlink} library of the OCaml compiler
      \end{itemize}
      \vfill
      \onslide<3->{
      \listocaml[firstline=205,lastline=209,highlightlines=207]{../ocaml/otherlibs/dynlink/dynlink\_compilerlibs/misc.ml}{otherlibs/dynlink/dynlink\_compilerlibs/misc.ml:205}
      }
      \endminipage
    \end{column}
    \begin{column}{0.55\textwidth}
    \only<4>{
      \mintcmm[highlightlines=3]{../src/notrace/camlNotrace\_\_fun_347.cmm}
      \listcmm{../src/notrace/camlNotrace\_\_all\_somes\_327.cmm}{all\_somes}
    }
    \onslide*<5->{
      \listobjdump[highlightlines={6-9}]{../src/notrace/camlNotrace\_\_fun_347.objdump}{anonymous function from \funcname{all\_somes}}
    }
    \end{column}
  \end{columns}
\end{frame}

\begin{frame}{Backtraces}{Summary table of the multiple kinds of raise}
  \begin{itemize}
    \item When debugging information is enabled and if the backtrace of an exception isn't needed, it's possible to raise with \funcname{raise\_notrace} for performance
    \item When debugging information is \emph{not} enabled, the compiler optimises \funcname{raise} calls to inline assembly (same effect as using \funcname{raise\_notrace})
  \end{itemize}
  \bigskip
  \centering
  \begin{tabular}{| l | c | c | c | c |}
    \hline
    \parbox{10em}{Debug information \\ (\funcarg{-g} flag)}  & \multicolumn{2}{|c|}{Disabled}                                     & \multicolumn{2}{|c|}{Enabled} \\
    \hline
    Raise kind                                               & \funcname{raise} & \funcname{raise\_notrace}                       & \funcname{raise} & \funcname{raise\_notrace} \\
    \hline
    Compilation                                              & \parbox{8em}{inline assembly\\ (optimization)} & inline assembly   & \funcname{caml\_raise\_exn} & inline assembly \\
    \hline
  \end{tabular}
\end{frame}


%
% Takeaway #5
%
\subsection*{Takeaway \#5}
\frameSubsectionTakeaway{}
\againframe<5>{takeaway}
}%
      \onslide*<3>{\providecommand\step{7}%
% Backtraces
%
\subsection{One advantage of raising with debugging information}
\frameSubsection{}{}

\begin{frame}{Backtraces}{Debugging information \& source code}
  \begin{columns}[c]
    \begin{column}{0.5\textwidth}
      \begin{itemize}
        \item Raising an exception is fun and games, but how do we know where the exception originated? $\rightarrow$ With the backtrace associated with the exception!
        \item In order to get a backtrace from the point where an exception was raised, it's necessary to compile with debugging information enabled (\funcarg{-g} flag for the compiler)
        \item Same example as in the `Nested handlers' section
      \end{itemize}
    \end{column}
    \begin{column}{0.5\textwidth}
      \listocaml[firstline=9,lastline=34]{../src/backtrace/backtrace.ml}{backtrace/backtrace.ml}
    \end{column}
  \end{columns}
\end{frame}

\begin{frame}{Backtraces}{CMM and disassembly of \funcname{definitely\_raise}}
  \begin{columns}[c]
    \begin{column}{0.5\textwidth}
      \minipage[c][0.66\textheight][s]{\columnwidth}
      \begin{itemize}
        \item<1-> An effect of compiling with debugging information (\funcarg{-g}): the CMM no longer uses \funcname{raise\_notrace} to raise the exception but \funcname{raise} instead
        \item<2-> Disassembly shows that CMM \funcname{raise} is actually a call to \funcname{caml\_raise\_exn}, a function in assembler provided by the runtime
      \end{itemize}
      \vfill
      \mintocaml[firstline=9,lastline=13,highlightlines=12]{../src/backtrace/backtrace.ml}
      \endminipage
    \end{column}
    \begin{column}{0.5\textwidth}
      \onslide*<1>{
        \mintcmm[highlightlines=5]{../src/backtrace/camlBacktrace\_\_definitely\_raise\_347.cmm}
      }
      \onslide*<2>{
        \mintobjdump[firstline=2,highlightlines=14]{../src/backtrace/camlBacktrace\_\_definitely\_raise\_347.objdump}
      }
    \end{column}
  \end{columns}
\end{frame}

\begin{frame}{Backtraces}{Introducing \funcname{caml\_raise\_exn}}
  \begin{columns}[c]
    \begin{column}{0.5\textwidth}
      \minipage[c][0.66\textheight][s]{\columnwidth}
      \onslide*<1>{
        \begin{itemize}
          \item Raising the exception is performed using \funcname{caml\_raise\_exn} which is
            \begin{itemize}
              \item An assembly function provided by the runtime
              \item Architecture specific
            \end{itemize}
          \item Let's see what it does differently from \funcname{raise\_notrace}\dots
          \item We are about to raise \typename{ExnC} with \funcname{caml\_raise\_exn} from \funcname{definitely\_raise}
        \end{itemize}
      }
      \onslide*<2>{
        \mintobjdump[firstline=2,highlightlines=14]{../src/backtrace/camlBacktrace\_\_definitely\_raise\_347.objdump}
      }
      \vfill
      \mintocaml[firstline=9,lastline=13,highlightlines=12]{../src/backtrace/backtrace.ml}
      \endminipage
    \end{column}
    \begin{column}{0.5\textwidth}
      \centering
      \providecommand\step{1}\input{diagrams/backtrace/backtrace.tex}
    \end{column}
  \end{columns}
\end{frame}

\begin{frame}{Backtraces}{But before that, we need more fields from \typename{caml\_domain\_state}}
  \begin{columns}
    \begin{column}{0.5\textwidth}
      \begin{itemize}
        \item Remember \typename{caml\_domain\_state} the per-domain runtime state?
        \item Some other members are of interest to us now\dots
        \item \localname{backtrace\_active} is used to know if the runtime should collect backtraces
        \item \localname{backtrace\_active} can be set by
          \begin{itemize}
            \item The \funcarg{b} flag in the \funcname{OCAMLRUNPARAM} environment variable
            \item \mintinline{ocaml}{Printexc.record_backtrace : bool -> unit} \footnotemark
          \end{itemize}
      \end{itemize}
    \end{column}
    \begin{column}{0.5\textwidth}
      \begin{listing}
        \mintc[highlightlines={9-11}]{src/ocaml/domain\_state\_backtrace.h}
        \caption{runtime/caml/domain\_state.h}
      \end{listing}
    \end{column}
  \end{columns}
  \footnotetext[1]{https://v2.ocaml.org/api/Printexc.html}
\end{frame}

\newcommand\listCamlRaiseExn[1][]{
  \listasm[firstline=702,lastline=725,#1]{../ocaml/runtime/amd64.S}{runtime/amd64.S:702}}

\begin{frame}{Backtraces}{Walk through \funcname{caml\_raise\_exn}}
  \begin{columns}[T]
    \begin{column}{0.5\textwidth}
      \begin{itemize}
        \item<1-> First checks whether exceptions backtrace should be recorded
        \item<1-> By checking the \funcarg{domain\_state->backtrace\_active} value
        \item<2-> If not, acts as \funcname{raise\_notrace} with \funcname{RESTORE\_EXN\_HANDLER\_OCAML}
          \begin{itemize}
            \item Set \regname{RSP} to \localname{domain\_state->exn\_handler} value
            \item Restore \localname{domain\_state->exn\_handler} from trap
            \item Jump with \funcname{ret} (same effect as using \funcname{pop} and \funcname{jmp})
          \end{itemize}
        \item<3-> Otherwise, switches to the C stack to be able to execute C code
        \item<4-> Calls \funcname{caml\_stash\_backtrace}, a C function provided by the runtime\ignorespaces
          \onslide<6->{, with
            \begin{itemize}
              \item \funcarg{exn}: exception value
              \item \funcarg{pc}: code address of the raising point
              \item \funcarg{sp}: stack address of the raising function
              \item \funcarg{trapsp}: stack address of the next handler
            \end{itemize}
          }
      \end{itemize}
      \bigskip
      \mintocaml[firstline=9,lastline=13,highlightlines=12]{../src/backtrace/backtrace.ml}
    \end{column}
    \begin{column}{0.5\textwidth}
      \raggedleft
      \onslide*<1,3-4>{
        \mintc[firstline=190,lastline=192]{../ocaml/runtime/amd64.S}
        \mintc[firstline=234,lastline=237]{../ocaml/runtime/amd64.S}
      }
      \onslide*<2>{
        \mintc[firstline=190,lastline=192,highlightlines={191-192}]{../ocaml/runtime/amd64.S}
        \mintc[firstline=234,lastline=237,highlightlines={235-237}]{../ocaml/runtime/amd64.S}
      }
      \onslide*<1>{\listCamlRaiseExn[highlightlines=705]{}}%
      \onslide*<2>{\listCamlRaiseExn[highlightlines={707-708}]{}}%
      \onslide*<3>{\listCamlRaiseExn[highlightlines={712-714}]{}}%
      \onslide*<4>{\listCamlRaiseExn[highlightlines={715-720}]{}}%
      \onslide*<5>{\providecommand\step{1}\input{diagrams/backtrace/backtrace.tex}}%
      \onslide*<6>{\providecommand\step{2}\input{diagrams/backtrace/backtrace.tex}}%
    \end{column}
  \end{columns}
\end{frame}

\newcommand\listStashBacktrace[1][]{
  \listc[firstline=91,lastline=120,#1]{../ocaml/runtime/backtrace\_nat.c}{runtime/backtrace\_nat.c:91}}

\begin{frame}{Backtraces}{Introducing \funcname{caml\_stash\_backtrace}}
  \begin{columns}[c]
    \begin{column}{0.5\textwidth}
      \minipage[c][0.66\textheight][s]{\columnwidth}
      \begin{itemize}
        \item<1-> A C function provided by the runtime (specific to native code)
        \item<2-> It scans the stack, between the stack pointer at the raising point up to the next trap, in order to collect the names of the detected function frames
          \onslide*<2>{
            \begin{itemize}
              \item And more information, like line number, column number...
            \end{itemize}
          }
        \item<3-> It uses the same technique and information as the Garbage Collector (GC) uses to scan the values live on the stack
          \onslide*<4>{
            \begin{itemize}
              \item \typename{frame\_descr} are at the core of stack unwinding
            \end{itemize}
          }
      \end{itemize}
      \vfill
      \mintocaml[firstline=9,lastline=13,highlightlines=12]{../src/backtrace/backtrace.ml}
      \endminipage
    \end{column}
    \begin{column}{0.5\textwidth}
      \onslide*<1>{\listStashBacktrace[highlightlines=91]{}}
      \onslide*<2>{\listStashBacktrace[highlightlines={91,117-118}]{}}
      \onslide*<3->{\listStashBacktrace[highlightlines={94,106,109-110}]{}}
    \end{column}
  \end{columns}
\end{frame}

\newcommand\listFrameDescr[1][]{
  \listocaml[firstline=111,lastline=124,#1]{../ocaml/asmcomp/emitaux.ml}{asmcomp/emitaux.ml:111}}
\newcommand\listDebugInfo[1][]{
  \listocaml[firstline=38,lastline=49,#1]{../ocaml/lambda/debuginfo.mli}{lambda/debuginfo.mli:38}}

\begin{frame}{Backtraces}{\typename{frame\_descr} and \typename{frame\_debuginfo} in a nutshell}
  \begin{columns}[c]
    \begin{column}{0.5\textwidth}
      \begin{itemize}
        \item<1-> For every return address in an OCaml program, there is a associated \typename{frame\_descr}
        \item<2-> A \typename{frame\_descr} specifies
        \begin{itemize}
          \item<2-> The size of the function stack frame
          \item<3-> Where live values are on the stack (so that the GC can mark them as live and prevent from being swept)
        \end{itemize}
        \item<4-> When compiled with debug information, \typename{frame\_descr} will be linked to a \typename{frame\_debuginfo}
        \begin{itemize}
          \item<5-> Contains information about the position in the source code (file name, line and column number)
        \end{itemize}
      \end{itemize}
    \end{column}
    \begin{column}{0.5\textwidth}
        \onslide*<1>{\listFrameDescr[highlightlines={118-122}]{}}
        \onslide*<2>{\listFrameDescr[highlightlines={120}]{}}
        \onslide*<3>{\listFrameDescr[highlightlines={121}]{}}
        \onslide*<4->{\listFrameDescr[highlightlines={122}]{}}
        \medskip
        \onslide*<1-3>{\listDebugInfo{}}
        \onslide*<4>{\listDebugInfo[highlightlines={38,49}]{}}
        \onslide*<5>{\listDebugInfo[highlightlines={39-46}]{}}
    \end{column}
  \end{columns}
\end{frame}

\begin{frame}{Backtraces}{Scanning the stack with \typename{frame\_descr}}
  \begin{columns}[c]
    \begin{column}{0.5\textwidth}
      \begin{itemize}
        \item Given the return address of the code that raises the exception by calling \funcname{caml\_raise\_exn} (\funcarg{pc} argument of \funcname{caml\_stash\_backtrace})
        \item And the position of the stack pointer (\funcarg{sp} argument of \funcname{caml\_stash\_backtrace})
        \item The frame size given by \typename{frame\_descr}.\funcarg{frame\_size}, allows to find the end of the previous frame
        \item And the return address of the caller of this function
        \bigskip
        \onslide<3->{
        \item Note that traps (here the reddish \funcname{ExnA}, \funcname{ExnB} and \funcname{ExnC} blocks) belong to the frame that installed them, and so the frame size changes when traps are removed
        }
      \end{itemize}
    \end{column}
    \begin{column}{0.5\textwidth}
      \raggedleft
      \onslide*<1>{\providecommand\step{2}\input{diagrams/backtrace/backtrace.tex}}%
      \onslide*<2>{\providecommand\step{1}\input{diagrams/backtrace/scanstack.tex}}%
      \onslide*<3>{\providecommand\step{2}\input{diagrams/backtrace/scanstack.tex}}%
      \onslide*<4>{\providecommand\step{3}\input{diagrams/backtrace/scanstack.tex}}%
      \onslide*<5>{\providecommand\step{4}\input{diagrams/backtrace/scanstack.tex}}%
      \onslide*<6>{\providecommand\step{5}\input{diagrams/backtrace/scanstack.tex}}%
    \end{column}
  \end{columns}
\end{frame}

% Duplicate of previous-previous slide
\begin{frame}{Backtraces}{Continuing on \funcname{caml\_stash\_backtrace}}
  \begin{columns}[c]
    \begin{column}{0.5\textwidth}
      \minipage[c][0.75\textheight][s]{\columnwidth}
      \begin{itemize}
          \color{gray}
        \item A C function provided by the runtime (specific to native code)
        \item It scans the stack, between the stack pointer at the raising point up to the next trap, in order to collect the names of the detected function frames
        \item It uses the same technique and information as the Garbage Collector (GC) uses to scan the values live on the stack
      \end{itemize}
      \begin{itemize}
        \item For each function frame detected, store a pointer to the \typename{frame\_descr} in the \localname{domain\_state->backtrace\_buffer} array, effectively constructing the backtrace
      \end{itemize}
      \onslide<2->{
        \begin{itemize}
            \smallskip
          \item Constructed backtrace:
            \funcname{definitely\_raise},
            \onslide<3->{\funcname{probably\_raise}}
        \end{itemize}
      }
      \vfill
      \mintocaml[firstline=9,lastline=13,highlightlines=12]{../src/backtrace/backtrace.ml}
      \endminipage
    \end{column}
    \begin{column}{0.5\textwidth}
      \raggedleft
      \onslide*<1>{\listStashBacktrace[highlightlines={114-115}]{}}
      \onslide*<2>{\providecommand\step{3}\input{diagrams/backtrace/backtrace.tex}}%
      \onslide*<3>{\providecommand\step{4}\input{diagrams/backtrace/backtrace.tex}}%
    \end{column}
  \end{columns}
\end{frame}

\begin{frame}{Backtraces}{Back to \funcname{caml\_raise\_exn}}
  \begin{columns}[c]
    \begin{column}{0.5\textwidth}
      \minipage[c][0.66\textheight][s]{\columnwidth}
      \begin{itemize}\color{gray}
        \item Checks wheteher exceptions backtrace should be recorded, by checking the \funcarg{domain\_state->backtrace\_active} value
        \item Switches to the C stack to be able to execute C code
        \item Calls \funcname{caml\_stash\_backtrace}, to collect the backtrace up to the next trap
      \end{itemize}
      \onslide*<2->{
        \begin{itemize}
          \item<2-> Executes the exception handler in \funcname{probably\_raise} with \funcname{RESTORE\_EXN\_HANDLER\_OCAML}
            \begin{itemize}
              \item Set \regname{RSP} to \localname{domain\_state->exn\_handler} value
              \item Restore \localname{domain\_state->exn\_handler} from trap
              \item Jump to the handler code with \funcname{ret}
            \end{itemize}
        \end{itemize}
      }
      \vfill
      \onslide*<1>{\mintocaml[firstline=9,lastline=13,highlightlines=12]{../src/backtrace/backtrace.ml}}
      \onslide*<2->{\mintocaml[firstline=15,lastline=21,highlightlines={19-21}]{../src/backtrace/backtrace.ml}}
      \endminipage
    \end{column}
    \begin{column}{0.5\textwidth}
      \centering
      \onslide*<1>{
        \mintc[firstline=190,lastline=192]{../ocaml/runtime/amd64.S}
        \mintc[firstline=234,lastline=237]{../ocaml/runtime/amd64.S}
      }
      \onslide*<2>{
        \mintc[firstline=190,lastline=192,highlightlines={191-192}]{../ocaml/runtime/amd64.S}
        \mintc[firstline=234,lastline=237,highlightlines={235-237}]{../ocaml/runtime/amd64.S}
      }
      \onslide*<1>{\listCamlRaiseExn[highlightlines=720]{}}
      \onslide*<2>{\listCamlRaiseExn[highlightlines=722]{}}
      \onslide*<3->{\providecommand\step{5}\input{diagrams/backtrace/backtrace.tex}}
    \end{column}
  \end{columns}
\end{frame}

\begin{frame}{Backtraces}{\funcname{caml\_probably\_raise}'s handler doesn't catch \typename{ExnC}}
  \begin{columns}[c]
    \begin{column}{0.45\textwidth}
      \minipage[c][0.75\textheight][s]{\columnwidth}
      \begin{itemize}
        \item<1-> \funcname{match \dots with}\footnotemark pattern matching can also match exceptions
        \item<1-> The CMM of \funcname{probably\_raise} shows that this \funcname{match \dots with} is implemented with a pattern matching and a \funcname{try \dots with}
        \item<1-> But this one only matches on \typename{ExnA} exception
        \item<2-> Since the pattern matching fails to catch the exception, \funcname{reraise} is called with our current \typename{ExnC} exception
        \item<3-> Disassembly shows that \funcname{reraise} is actually a call to \funcname{caml\_reraise\_exn}, which is also an assembly function provided by the runtime
      \end{itemize}
      \vfill
      \mintocaml[firstline=15,lastline=21,highlightlines={18-21}]{../src/backtrace/backtrace.ml}
      \endminipage
    \end{column}
    \begin{column}{0.55\textwidth}
      \centering
      \onslide*<1>{\mintcmm[highlightlines={10-18}]{../src/backtrace/camlBacktrace\_\_probably\_raise\_351.cmm}}
      \onslide*<2>{\listcmm[highlightlines=18]{../src/backtrace/camlBacktrace\_\_probably\_raise_351.cmm}{CMM of probably\_raise}}
      \onslide*<3>{\mintobjdump[firstline=5,lastline=35,highlightlines=31]{../src/backtrace/camlBacktrace\_\_probably\_raise_351.objdump}}
    \end{column}
  \end{columns}
  \only<1>{\footnotetext[1]{https://blog.janestreet.com/pattern-matching-and-exception-handling-unite/}}
\end{frame}

\newcommand\listCamlReraiseExn[1][]{
  \listasm[firstline=727,lastline=734,#1]{../ocaml/runtime/amd64.S}{runtime/amd64.S:727}}

\begin{frame}{Backtraces}{Introducing \funcname{caml\_reraise\_exn}}
  \begin{columns}[c]
    \begin{column}{0.5\textwidth}
      \minipage[c][0.66\textheight][s]{\columnwidth}
      \begin{itemize}
        \item<1-> The exception have to be reraised to the next handler
        \item<2-> First, checks if the backtrace should be collected with \funcarg{domain\_state->backtrace\_active}
        \only<3>{
        \begin{itemize}
            \item If not, behaves like a \funcname{raise\_notrace} with \funcname{RESTORE\_EXN\_HANDLER\_OCAML}
        \end{itemize}
        }
        \item<4-> Jumps into \funcname{caml\_raise\_exn}
        \item<5-> Calls \funcname{caml\_stash\_backtrace} again, to scan the next part of the stack: from the trap just pop-ed to the next trap
      \end{itemize}
      \vfill
      \mintocaml[firstline=15,lastline=21,highlightlines={18-21}]{../src/backtrace/backtrace.ml}
      \endminipage
    \end{column}
    \begin{column}{0.5\textwidth}
      \raggedleft
      \onslide*<1>{\listCamlReraiseExn{}}
      \onslide*<2>{\listCamlReraiseExn[highlightlines=729]{}}
      \onslide*<3>{
        \mintc[firstline=190,lastline=192,highlightlines={191-192}]{../ocaml/runtime/amd64.S}
        \mintc[firstline=234,lastline=237,highlightlines={235-237}]{../ocaml/runtime/amd64.S}
        \medskip
        \listCamlReraiseExn[highlightlines=731]{}
      }
      \onslide*<4-5>{
        % TODO refactor with \listCamlReraiseExn
        \mintasm[firstline=727,lastline=734,highlightlines=730]{../ocaml/runtime/amd64.S}
        \medskip
      }
      \onslide*<4>{\listCamlRaiseExn[highlightlines=711]{}}
      \onslide*<5>{\listCamlRaiseExn[highlightlines=720]{}}
    \end{column}
  \end{columns}
\end{frame}

\begin{frame}{Backtraces}{Backtrace collection continues}
  \begin{columns}[c]
    \begin{column}{0.55\textwidth}
      \minipage[c][0.75\textheight][s]{\columnwidth}
      \begin{itemize}
        \item<1-> \funcname{caml\_stash\_backtrace} scans the older part of the stack, up to the next trap
        \item<6-> Controls jumps to the nested exception handler in \funcname{maybe\_raise}, which matches only on \typename{ExnB}
        \item<7-> \funcname{caml\_reraise\_exn} is called again, backtrace collection resumes with \funcname{caml\_stash\_backtrace}
      \end{itemize}
      \medskip
      \begin{itemize}
        \item<2-> Constructed backtrace:
        \begin{itemize}
          \item <2-> \funcname{definitely\_raise}
          \item <2-> \funcname{probably\_raise}
          \item <3-> \funcname{probably\_raise} (re-raise)
          \item <4-> \funcname{maybe\_raise}
          \item <5-> \funcname{main}
          \item <8-> \funcname{main} (re-raise)
        \end{itemize}
      \end{itemize}
      \vfill
      \onslide*<1-5>{\mintocaml[firstline=15,lastline=21]{../src/backtrace/backtrace.ml}}
      \onslide*<6>{\mintocaml[firstline=28,lastline=34,highlightlines=31]{../src/backtrace/backtrace.ml}}
      \onslide*<7->{\mintocaml[firstline=28,lastline=34,highlightlines=33]{../src/backtrace/backtrace.ml}}
      \endminipage
    \end{column}
    \begin{column}{0.45\textwidth}
      \raggedleft
      \onslide*<1>{\providecommand\step{6}\input{diagrams/backtrace/backtrace.tex}}%
      \onslide*<2>{\providecommand\step{6}\input{diagrams/backtrace/backtrace.tex}}%
      \onslide*<3>{\providecommand\step{7}\input{diagrams/backtrace/backtrace.tex}}%
      \onslide*<4>{\providecommand\step{8}\input{diagrams/backtrace/backtrace.tex}}%
      \onslide*<5>{\providecommand\step{9}\input{diagrams/backtrace/backtrace.tex}}%
      \onslide*<6>{\providecommand\step{10}\input{diagrams/backtrace/backtrace.tex}}%
      \onslide*<7>{\providecommand\step{11}\input{diagrams/backtrace/backtrace.tex}}%
      \onslide*<8>{\providecommand\step{12}\input{diagrams/backtrace/backtrace.tex}}%
      \onslide*<9>{\providecommand\step{13}\input{diagrams/backtrace/backtrace.tex}}%
    \end{column}
  \end{columns}
\end{frame}

\begin{frame}{Backtraces}{Printing the backtrace}
  \begin{columns}[c]
    \begin{column}{0.45\textwidth}
      \minipage[c][0.66\textheight][s]{\columnwidth}
      \begin{itemize}
        \item<1-> The exception backtrace can now be printed to \funcarg{stdout} with \funcname{Printexc.print_backtrace}
        \item<2-> First the exception backtrace is retrieved into an OCaml array with \funcname{get_raw_backtrace }
        \item<3-> Which is implemented as the \funcname{caml_get_exception_raw_backtrace} C function in the runtime
        \item<4-> And finally printed with \funcname{print_exception_backtrace}
      \end{itemize}
      \vfill
      \mintocaml[firstline=28,lastline=34,highlightlines=33]{../src/backtrace/backtrace.ml}
      \endminipage
    \end{column}
    \begin{column}{0.55\textwidth}
      \onslide*<1,2>{
        \mintocaml[firstline=100,lastline=101]{../ocaml/stdlib/printexc.ml}
      }
      \only<1>{
        \listocaml[firstline=170,lastline=172]{../ocaml/stdlib/printexc.ml}{stdlib/printexc.ml:170}
      }
      \only<2>{
        \listocaml[firstline=170,lastline=172,highlightlines=172]{../ocaml/stdlib/printexc.ml}{stdlib/printexc.ml:170}
      }
      \only<3>{
        \mintc[firstline=154,lastline=156]{../ocaml/runtime/backtrace.c}
        \listc[firstline=171,lastline=192,highlightlines={184-187}]{../ocaml/runtime/backtrace.c}{runtime/backtrace.c:154}
      }
      \only<4>{
        \listocaml[firstline=155,lastline=165]{../ocaml/stdlib/printexc.ml}{stdlib/printexc.ml:55}
      }
    \end{column}
  \end{columns}
\end{frame}


\subsection{The multiple kinds of raise}
\frameSubsection{}{}

\begin{frame}{Backtraces}{When backtraces are not needed}
  \begin{columns}[c]
    \begin{column}{0.45\textwidth}
      \minipage[c][0.66\textheight][s]{\columnwidth}
      \begin{itemize}
        \item<1-> What if the backtrace for an exception is never needed?
        \item<2-> It's possible to raise the exception explicitly with \funcname{raise\_notrace} to make raising as cheap as possible
        \item<3-> An example of use in the \funcname{dynlink} library of the OCaml compiler
      \end{itemize}
      \vfill
      \onslide<3->{
      \listocaml[firstline=205,lastline=209,highlightlines=207]{../ocaml/otherlibs/dynlink/dynlink\_compilerlibs/misc.ml}{otherlibs/dynlink/dynlink\_compilerlibs/misc.ml:205}
      }
      \endminipage
    \end{column}
    \begin{column}{0.55\textwidth}
    \only<4>{
      \mintcmm[highlightlines=3]{../src/notrace/camlNotrace\_\_fun_347.cmm}
      \listcmm{../src/notrace/camlNotrace\_\_all\_somes\_327.cmm}{all\_somes}
    }
    \onslide*<5->{
      \listobjdump[highlightlines={6-9}]{../src/notrace/camlNotrace\_\_fun_347.objdump}{anonymous function from \funcname{all\_somes}}
    }
    \end{column}
  \end{columns}
\end{frame}

\begin{frame}{Backtraces}{Summary table of the multiple kinds of raise}
  \begin{itemize}
    \item When debugging information is enabled and if the backtrace of an exception isn't needed, it's possible to raise with \funcname{raise\_notrace} for performance
    \item When debugging information is \emph{not} enabled, the compiler optimises \funcname{raise} calls to inline assembly (same effect as using \funcname{raise\_notrace})
  \end{itemize}
  \bigskip
  \centering
  \begin{tabular}{| l | c | c | c | c |}
    \hline
    \parbox{10em}{Debug information \\ (\funcarg{-g} flag)}  & \multicolumn{2}{|c|}{Disabled}                                     & \multicolumn{2}{|c|}{Enabled} \\
    \hline
    Raise kind                                               & \funcname{raise} & \funcname{raise\_notrace}                       & \funcname{raise} & \funcname{raise\_notrace} \\
    \hline
    Compilation                                              & \parbox{8em}{inline assembly\\ (optimization)} & inline assembly   & \funcname{caml\_raise\_exn} & inline assembly \\
    \hline
  \end{tabular}
\end{frame}


%
% Takeaway #5
%
\subsection*{Takeaway \#5}
\frameSubsectionTakeaway{}
\againframe<5>{takeaway}
}%
      \onslide*<4>{\providecommand\step{8}%
% Backtraces
%
\subsection{One advantage of raising with debugging information}
\frameSubsection{}{}

\begin{frame}{Backtraces}{Debugging information \& source code}
  \begin{columns}[c]
    \begin{column}{0.5\textwidth}
      \begin{itemize}
        \item Raising an exception is fun and games, but how do we know where the exception originated? $\rightarrow$ With the backtrace associated with the exception!
        \item In order to get a backtrace from the point where an exception was raised, it's necessary to compile with debugging information enabled (\funcarg{-g} flag for the compiler)
        \item Same example as in the `Nested handlers' section
      \end{itemize}
    \end{column}
    \begin{column}{0.5\textwidth}
      \listocaml[firstline=9,lastline=34]{../src/backtrace/backtrace.ml}{backtrace/backtrace.ml}
    \end{column}
  \end{columns}
\end{frame}

\begin{frame}{Backtraces}{CMM and disassembly of \funcname{definitely\_raise}}
  \begin{columns}[c]
    \begin{column}{0.5\textwidth}
      \minipage[c][0.66\textheight][s]{\columnwidth}
      \begin{itemize}
        \item<1-> An effect of compiling with debugging information (\funcarg{-g}): the CMM no longer uses \funcname{raise\_notrace} to raise the exception but \funcname{raise} instead
        \item<2-> Disassembly shows that CMM \funcname{raise} is actually a call to \funcname{caml\_raise\_exn}, a function in assembler provided by the runtime
      \end{itemize}
      \vfill
      \mintocaml[firstline=9,lastline=13,highlightlines=12]{../src/backtrace/backtrace.ml}
      \endminipage
    \end{column}
    \begin{column}{0.5\textwidth}
      \onslide*<1>{
        \mintcmm[highlightlines=5]{../src/backtrace/camlBacktrace\_\_definitely\_raise\_347.cmm}
      }
      \onslide*<2>{
        \mintobjdump[firstline=2,highlightlines=14]{../src/backtrace/camlBacktrace\_\_definitely\_raise\_347.objdump}
      }
    \end{column}
  \end{columns}
\end{frame}

\begin{frame}{Backtraces}{Introducing \funcname{caml\_raise\_exn}}
  \begin{columns}[c]
    \begin{column}{0.5\textwidth}
      \minipage[c][0.66\textheight][s]{\columnwidth}
      \onslide*<1>{
        \begin{itemize}
          \item Raising the exception is performed using \funcname{caml\_raise\_exn} which is
            \begin{itemize}
              \item An assembly function provided by the runtime
              \item Architecture specific
            \end{itemize}
          \item Let's see what it does differently from \funcname{raise\_notrace}\dots
          \item We are about to raise \typename{ExnC} with \funcname{caml\_raise\_exn} from \funcname{definitely\_raise}
        \end{itemize}
      }
      \onslide*<2>{
        \mintobjdump[firstline=2,highlightlines=14]{../src/backtrace/camlBacktrace\_\_definitely\_raise\_347.objdump}
      }
      \vfill
      \mintocaml[firstline=9,lastline=13,highlightlines=12]{../src/backtrace/backtrace.ml}
      \endminipage
    \end{column}
    \begin{column}{0.5\textwidth}
      \centering
      \providecommand\step{1}\input{diagrams/backtrace/backtrace.tex}
    \end{column}
  \end{columns}
\end{frame}

\begin{frame}{Backtraces}{But before that, we need more fields from \typename{caml\_domain\_state}}
  \begin{columns}
    \begin{column}{0.5\textwidth}
      \begin{itemize}
        \item Remember \typename{caml\_domain\_state} the per-domain runtime state?
        \item Some other members are of interest to us now\dots
        \item \localname{backtrace\_active} is used to know if the runtime should collect backtraces
        \item \localname{backtrace\_active} can be set by
          \begin{itemize}
            \item The \funcarg{b} flag in the \funcname{OCAMLRUNPARAM} environment variable
            \item \mintinline{ocaml}{Printexc.record_backtrace : bool -> unit} \footnotemark
          \end{itemize}
      \end{itemize}
    \end{column}
    \begin{column}{0.5\textwidth}
      \begin{listing}
        \mintc[highlightlines={9-11}]{src/ocaml/domain\_state\_backtrace.h}
        \caption{runtime/caml/domain\_state.h}
      \end{listing}
    \end{column}
  \end{columns}
  \footnotetext[1]{https://v2.ocaml.org/api/Printexc.html}
\end{frame}

\newcommand\listCamlRaiseExn[1][]{
  \listasm[firstline=702,lastline=725,#1]{../ocaml/runtime/amd64.S}{runtime/amd64.S:702}}

\begin{frame}{Backtraces}{Walk through \funcname{caml\_raise\_exn}}
  \begin{columns}[T]
    \begin{column}{0.5\textwidth}
      \begin{itemize}
        \item<1-> First checks whether exceptions backtrace should be recorded
        \item<1-> By checking the \funcarg{domain\_state->backtrace\_active} value
        \item<2-> If not, acts as \funcname{raise\_notrace} with \funcname{RESTORE\_EXN\_HANDLER\_OCAML}
          \begin{itemize}
            \item Set \regname{RSP} to \localname{domain\_state->exn\_handler} value
            \item Restore \localname{domain\_state->exn\_handler} from trap
            \item Jump with \funcname{ret} (same effect as using \funcname{pop} and \funcname{jmp})
          \end{itemize}
        \item<3-> Otherwise, switches to the C stack to be able to execute C code
        \item<4-> Calls \funcname{caml\_stash\_backtrace}, a C function provided by the runtime\ignorespaces
          \onslide<6->{, with
            \begin{itemize}
              \item \funcarg{exn}: exception value
              \item \funcarg{pc}: code address of the raising point
              \item \funcarg{sp}: stack address of the raising function
              \item \funcarg{trapsp}: stack address of the next handler
            \end{itemize}
          }
      \end{itemize}
      \bigskip
      \mintocaml[firstline=9,lastline=13,highlightlines=12]{../src/backtrace/backtrace.ml}
    \end{column}
    \begin{column}{0.5\textwidth}
      \raggedleft
      \onslide*<1,3-4>{
        \mintc[firstline=190,lastline=192]{../ocaml/runtime/amd64.S}
        \mintc[firstline=234,lastline=237]{../ocaml/runtime/amd64.S}
      }
      \onslide*<2>{
        \mintc[firstline=190,lastline=192,highlightlines={191-192}]{../ocaml/runtime/amd64.S}
        \mintc[firstline=234,lastline=237,highlightlines={235-237}]{../ocaml/runtime/amd64.S}
      }
      \onslide*<1>{\listCamlRaiseExn[highlightlines=705]{}}%
      \onslide*<2>{\listCamlRaiseExn[highlightlines={707-708}]{}}%
      \onslide*<3>{\listCamlRaiseExn[highlightlines={712-714}]{}}%
      \onslide*<4>{\listCamlRaiseExn[highlightlines={715-720}]{}}%
      \onslide*<5>{\providecommand\step{1}\input{diagrams/backtrace/backtrace.tex}}%
      \onslide*<6>{\providecommand\step{2}\input{diagrams/backtrace/backtrace.tex}}%
    \end{column}
  \end{columns}
\end{frame}

\newcommand\listStashBacktrace[1][]{
  \listc[firstline=91,lastline=120,#1]{../ocaml/runtime/backtrace\_nat.c}{runtime/backtrace\_nat.c:91}}

\begin{frame}{Backtraces}{Introducing \funcname{caml\_stash\_backtrace}}
  \begin{columns}[c]
    \begin{column}{0.5\textwidth}
      \minipage[c][0.66\textheight][s]{\columnwidth}
      \begin{itemize}
        \item<1-> A C function provided by the runtime (specific to native code)
        \item<2-> It scans the stack, between the stack pointer at the raising point up to the next trap, in order to collect the names of the detected function frames
          \onslide*<2>{
            \begin{itemize}
              \item And more information, like line number, column number...
            \end{itemize}
          }
        \item<3-> It uses the same technique and information as the Garbage Collector (GC) uses to scan the values live on the stack
          \onslide*<4>{
            \begin{itemize}
              \item \typename{frame\_descr} are at the core of stack unwinding
            \end{itemize}
          }
      \end{itemize}
      \vfill
      \mintocaml[firstline=9,lastline=13,highlightlines=12]{../src/backtrace/backtrace.ml}
      \endminipage
    \end{column}
    \begin{column}{0.5\textwidth}
      \onslide*<1>{\listStashBacktrace[highlightlines=91]{}}
      \onslide*<2>{\listStashBacktrace[highlightlines={91,117-118}]{}}
      \onslide*<3->{\listStashBacktrace[highlightlines={94,106,109-110}]{}}
    \end{column}
  \end{columns}
\end{frame}

\newcommand\listFrameDescr[1][]{
  \listocaml[firstline=111,lastline=124,#1]{../ocaml/asmcomp/emitaux.ml}{asmcomp/emitaux.ml:111}}
\newcommand\listDebugInfo[1][]{
  \listocaml[firstline=38,lastline=49,#1]{../ocaml/lambda/debuginfo.mli}{lambda/debuginfo.mli:38}}

\begin{frame}{Backtraces}{\typename{frame\_descr} and \typename{frame\_debuginfo} in a nutshell}
  \begin{columns}[c]
    \begin{column}{0.5\textwidth}
      \begin{itemize}
        \item<1-> For every return address in an OCaml program, there is a associated \typename{frame\_descr}
        \item<2-> A \typename{frame\_descr} specifies
        \begin{itemize}
          \item<2-> The size of the function stack frame
          \item<3-> Where live values are on the stack (so that the GC can mark them as live and prevent from being swept)
        \end{itemize}
        \item<4-> When compiled with debug information, \typename{frame\_descr} will be linked to a \typename{frame\_debuginfo}
        \begin{itemize}
          \item<5-> Contains information about the position in the source code (file name, line and column number)
        \end{itemize}
      \end{itemize}
    \end{column}
    \begin{column}{0.5\textwidth}
        \onslide*<1>{\listFrameDescr[highlightlines={118-122}]{}}
        \onslide*<2>{\listFrameDescr[highlightlines={120}]{}}
        \onslide*<3>{\listFrameDescr[highlightlines={121}]{}}
        \onslide*<4->{\listFrameDescr[highlightlines={122}]{}}
        \medskip
        \onslide*<1-3>{\listDebugInfo{}}
        \onslide*<4>{\listDebugInfo[highlightlines={38,49}]{}}
        \onslide*<5>{\listDebugInfo[highlightlines={39-46}]{}}
    \end{column}
  \end{columns}
\end{frame}

\begin{frame}{Backtraces}{Scanning the stack with \typename{frame\_descr}}
  \begin{columns}[c]
    \begin{column}{0.5\textwidth}
      \begin{itemize}
        \item Given the return address of the code that raises the exception by calling \funcname{caml\_raise\_exn} (\funcarg{pc} argument of \funcname{caml\_stash\_backtrace})
        \item And the position of the stack pointer (\funcarg{sp} argument of \funcname{caml\_stash\_backtrace})
        \item The frame size given by \typename{frame\_descr}.\funcarg{frame\_size}, allows to find the end of the previous frame
        \item And the return address of the caller of this function
        \bigskip
        \onslide<3->{
        \item Note that traps (here the reddish \funcname{ExnA}, \funcname{ExnB} and \funcname{ExnC} blocks) belong to the frame that installed them, and so the frame size changes when traps are removed
        }
      \end{itemize}
    \end{column}
    \begin{column}{0.5\textwidth}
      \raggedleft
      \onslide*<1>{\providecommand\step{2}\input{diagrams/backtrace/backtrace.tex}}%
      \onslide*<2>{\providecommand\step{1}\input{diagrams/backtrace/scanstack.tex}}%
      \onslide*<3>{\providecommand\step{2}\input{diagrams/backtrace/scanstack.tex}}%
      \onslide*<4>{\providecommand\step{3}\input{diagrams/backtrace/scanstack.tex}}%
      \onslide*<5>{\providecommand\step{4}\input{diagrams/backtrace/scanstack.tex}}%
      \onslide*<6>{\providecommand\step{5}\input{diagrams/backtrace/scanstack.tex}}%
    \end{column}
  \end{columns}
\end{frame}

% Duplicate of previous-previous slide
\begin{frame}{Backtraces}{Continuing on \funcname{caml\_stash\_backtrace}}
  \begin{columns}[c]
    \begin{column}{0.5\textwidth}
      \minipage[c][0.75\textheight][s]{\columnwidth}
      \begin{itemize}
          \color{gray}
        \item A C function provided by the runtime (specific to native code)
        \item It scans the stack, between the stack pointer at the raising point up to the next trap, in order to collect the names of the detected function frames
        \item It uses the same technique and information as the Garbage Collector (GC) uses to scan the values live on the stack
      \end{itemize}
      \begin{itemize}
        \item For each function frame detected, store a pointer to the \typename{frame\_descr} in the \localname{domain\_state->backtrace\_buffer} array, effectively constructing the backtrace
      \end{itemize}
      \onslide<2->{
        \begin{itemize}
            \smallskip
          \item Constructed backtrace:
            \funcname{definitely\_raise},
            \onslide<3->{\funcname{probably\_raise}}
        \end{itemize}
      }
      \vfill
      \mintocaml[firstline=9,lastline=13,highlightlines=12]{../src/backtrace/backtrace.ml}
      \endminipage
    \end{column}
    \begin{column}{0.5\textwidth}
      \raggedleft
      \onslide*<1>{\listStashBacktrace[highlightlines={114-115}]{}}
      \onslide*<2>{\providecommand\step{3}\input{diagrams/backtrace/backtrace.tex}}%
      \onslide*<3>{\providecommand\step{4}\input{diagrams/backtrace/backtrace.tex}}%
    \end{column}
  \end{columns}
\end{frame}

\begin{frame}{Backtraces}{Back to \funcname{caml\_raise\_exn}}
  \begin{columns}[c]
    \begin{column}{0.5\textwidth}
      \minipage[c][0.66\textheight][s]{\columnwidth}
      \begin{itemize}\color{gray}
        \item Checks wheteher exceptions backtrace should be recorded, by checking the \funcarg{domain\_state->backtrace\_active} value
        \item Switches to the C stack to be able to execute C code
        \item Calls \funcname{caml\_stash\_backtrace}, to collect the backtrace up to the next trap
      \end{itemize}
      \onslide*<2->{
        \begin{itemize}
          \item<2-> Executes the exception handler in \funcname{probably\_raise} with \funcname{RESTORE\_EXN\_HANDLER\_OCAML}
            \begin{itemize}
              \item Set \regname{RSP} to \localname{domain\_state->exn\_handler} value
              \item Restore \localname{domain\_state->exn\_handler} from trap
              \item Jump to the handler code with \funcname{ret}
            \end{itemize}
        \end{itemize}
      }
      \vfill
      \onslide*<1>{\mintocaml[firstline=9,lastline=13,highlightlines=12]{../src/backtrace/backtrace.ml}}
      \onslide*<2->{\mintocaml[firstline=15,lastline=21,highlightlines={19-21}]{../src/backtrace/backtrace.ml}}
      \endminipage
    \end{column}
    \begin{column}{0.5\textwidth}
      \centering
      \onslide*<1>{
        \mintc[firstline=190,lastline=192]{../ocaml/runtime/amd64.S}
        \mintc[firstline=234,lastline=237]{../ocaml/runtime/amd64.S}
      }
      \onslide*<2>{
        \mintc[firstline=190,lastline=192,highlightlines={191-192}]{../ocaml/runtime/amd64.S}
        \mintc[firstline=234,lastline=237,highlightlines={235-237}]{../ocaml/runtime/amd64.S}
      }
      \onslide*<1>{\listCamlRaiseExn[highlightlines=720]{}}
      \onslide*<2>{\listCamlRaiseExn[highlightlines=722]{}}
      \onslide*<3->{\providecommand\step{5}\input{diagrams/backtrace/backtrace.tex}}
    \end{column}
  \end{columns}
\end{frame}

\begin{frame}{Backtraces}{\funcname{caml\_probably\_raise}'s handler doesn't catch \typename{ExnC}}
  \begin{columns}[c]
    \begin{column}{0.45\textwidth}
      \minipage[c][0.75\textheight][s]{\columnwidth}
      \begin{itemize}
        \item<1-> \funcname{match \dots with}\footnotemark pattern matching can also match exceptions
        \item<1-> The CMM of \funcname{probably\_raise} shows that this \funcname{match \dots with} is implemented with a pattern matching and a \funcname{try \dots with}
        \item<1-> But this one only matches on \typename{ExnA} exception
        \item<2-> Since the pattern matching fails to catch the exception, \funcname{reraise} is called with our current \typename{ExnC} exception
        \item<3-> Disassembly shows that \funcname{reraise} is actually a call to \funcname{caml\_reraise\_exn}, which is also an assembly function provided by the runtime
      \end{itemize}
      \vfill
      \mintocaml[firstline=15,lastline=21,highlightlines={18-21}]{../src/backtrace/backtrace.ml}
      \endminipage
    \end{column}
    \begin{column}{0.55\textwidth}
      \centering
      \onslide*<1>{\mintcmm[highlightlines={10-18}]{../src/backtrace/camlBacktrace\_\_probably\_raise\_351.cmm}}
      \onslide*<2>{\listcmm[highlightlines=18]{../src/backtrace/camlBacktrace\_\_probably\_raise_351.cmm}{CMM of probably\_raise}}
      \onslide*<3>{\mintobjdump[firstline=5,lastline=35,highlightlines=31]{../src/backtrace/camlBacktrace\_\_probably\_raise_351.objdump}}
    \end{column}
  \end{columns}
  \only<1>{\footnotetext[1]{https://blog.janestreet.com/pattern-matching-and-exception-handling-unite/}}
\end{frame}

\newcommand\listCamlReraiseExn[1][]{
  \listasm[firstline=727,lastline=734,#1]{../ocaml/runtime/amd64.S}{runtime/amd64.S:727}}

\begin{frame}{Backtraces}{Introducing \funcname{caml\_reraise\_exn}}
  \begin{columns}[c]
    \begin{column}{0.5\textwidth}
      \minipage[c][0.66\textheight][s]{\columnwidth}
      \begin{itemize}
        \item<1-> The exception have to be reraised to the next handler
        \item<2-> First, checks if the backtrace should be collected with \funcarg{domain\_state->backtrace\_active}
        \only<3>{
        \begin{itemize}
            \item If not, behaves like a \funcname{raise\_notrace} with \funcname{RESTORE\_EXN\_HANDLER\_OCAML}
        \end{itemize}
        }
        \item<4-> Jumps into \funcname{caml\_raise\_exn}
        \item<5-> Calls \funcname{caml\_stash\_backtrace} again, to scan the next part of the stack: from the trap just pop-ed to the next trap
      \end{itemize}
      \vfill
      \mintocaml[firstline=15,lastline=21,highlightlines={18-21}]{../src/backtrace/backtrace.ml}
      \endminipage
    \end{column}
    \begin{column}{0.5\textwidth}
      \raggedleft
      \onslide*<1>{\listCamlReraiseExn{}}
      \onslide*<2>{\listCamlReraiseExn[highlightlines=729]{}}
      \onslide*<3>{
        \mintc[firstline=190,lastline=192,highlightlines={191-192}]{../ocaml/runtime/amd64.S}
        \mintc[firstline=234,lastline=237,highlightlines={235-237}]{../ocaml/runtime/amd64.S}
        \medskip
        \listCamlReraiseExn[highlightlines=731]{}
      }
      \onslide*<4-5>{
        % TODO refactor with \listCamlReraiseExn
        \mintasm[firstline=727,lastline=734,highlightlines=730]{../ocaml/runtime/amd64.S}
        \medskip
      }
      \onslide*<4>{\listCamlRaiseExn[highlightlines=711]{}}
      \onslide*<5>{\listCamlRaiseExn[highlightlines=720]{}}
    \end{column}
  \end{columns}
\end{frame}

\begin{frame}{Backtraces}{Backtrace collection continues}
  \begin{columns}[c]
    \begin{column}{0.55\textwidth}
      \minipage[c][0.75\textheight][s]{\columnwidth}
      \begin{itemize}
        \item<1-> \funcname{caml\_stash\_backtrace} scans the older part of the stack, up to the next trap
        \item<6-> Controls jumps to the nested exception handler in \funcname{maybe\_raise}, which matches only on \typename{ExnB}
        \item<7-> \funcname{caml\_reraise\_exn} is called again, backtrace collection resumes with \funcname{caml\_stash\_backtrace}
      \end{itemize}
      \medskip
      \begin{itemize}
        \item<2-> Constructed backtrace:
        \begin{itemize}
          \item <2-> \funcname{definitely\_raise}
          \item <2-> \funcname{probably\_raise}
          \item <3-> \funcname{probably\_raise} (re-raise)
          \item <4-> \funcname{maybe\_raise}
          \item <5-> \funcname{main}
          \item <8-> \funcname{main} (re-raise)
        \end{itemize}
      \end{itemize}
      \vfill
      \onslide*<1-5>{\mintocaml[firstline=15,lastline=21]{../src/backtrace/backtrace.ml}}
      \onslide*<6>{\mintocaml[firstline=28,lastline=34,highlightlines=31]{../src/backtrace/backtrace.ml}}
      \onslide*<7->{\mintocaml[firstline=28,lastline=34,highlightlines=33]{../src/backtrace/backtrace.ml}}
      \endminipage
    \end{column}
    \begin{column}{0.45\textwidth}
      \raggedleft
      \onslide*<1>{\providecommand\step{6}\input{diagrams/backtrace/backtrace.tex}}%
      \onslide*<2>{\providecommand\step{6}\input{diagrams/backtrace/backtrace.tex}}%
      \onslide*<3>{\providecommand\step{7}\input{diagrams/backtrace/backtrace.tex}}%
      \onslide*<4>{\providecommand\step{8}\input{diagrams/backtrace/backtrace.tex}}%
      \onslide*<5>{\providecommand\step{9}\input{diagrams/backtrace/backtrace.tex}}%
      \onslide*<6>{\providecommand\step{10}\input{diagrams/backtrace/backtrace.tex}}%
      \onslide*<7>{\providecommand\step{11}\input{diagrams/backtrace/backtrace.tex}}%
      \onslide*<8>{\providecommand\step{12}\input{diagrams/backtrace/backtrace.tex}}%
      \onslide*<9>{\providecommand\step{13}\input{diagrams/backtrace/backtrace.tex}}%
    \end{column}
  \end{columns}
\end{frame}

\begin{frame}{Backtraces}{Printing the backtrace}
  \begin{columns}[c]
    \begin{column}{0.45\textwidth}
      \minipage[c][0.66\textheight][s]{\columnwidth}
      \begin{itemize}
        \item<1-> The exception backtrace can now be printed to \funcarg{stdout} with \funcname{Printexc.print_backtrace}
        \item<2-> First the exception backtrace is retrieved into an OCaml array with \funcname{get_raw_backtrace }
        \item<3-> Which is implemented as the \funcname{caml_get_exception_raw_backtrace} C function in the runtime
        \item<4-> And finally printed with \funcname{print_exception_backtrace}
      \end{itemize}
      \vfill
      \mintocaml[firstline=28,lastline=34,highlightlines=33]{../src/backtrace/backtrace.ml}
      \endminipage
    \end{column}
    \begin{column}{0.55\textwidth}
      \onslide*<1,2>{
        \mintocaml[firstline=100,lastline=101]{../ocaml/stdlib/printexc.ml}
      }
      \only<1>{
        \listocaml[firstline=170,lastline=172]{../ocaml/stdlib/printexc.ml}{stdlib/printexc.ml:170}
      }
      \only<2>{
        \listocaml[firstline=170,lastline=172,highlightlines=172]{../ocaml/stdlib/printexc.ml}{stdlib/printexc.ml:170}
      }
      \only<3>{
        \mintc[firstline=154,lastline=156]{../ocaml/runtime/backtrace.c}
        \listc[firstline=171,lastline=192,highlightlines={184-187}]{../ocaml/runtime/backtrace.c}{runtime/backtrace.c:154}
      }
      \only<4>{
        \listocaml[firstline=155,lastline=165]{../ocaml/stdlib/printexc.ml}{stdlib/printexc.ml:55}
      }
    \end{column}
  \end{columns}
\end{frame}


\subsection{The multiple kinds of raise}
\frameSubsection{}{}

\begin{frame}{Backtraces}{When backtraces are not needed}
  \begin{columns}[c]
    \begin{column}{0.45\textwidth}
      \minipage[c][0.66\textheight][s]{\columnwidth}
      \begin{itemize}
        \item<1-> What if the backtrace for an exception is never needed?
        \item<2-> It's possible to raise the exception explicitly with \funcname{raise\_notrace} to make raising as cheap as possible
        \item<3-> An example of use in the \funcname{dynlink} library of the OCaml compiler
      \end{itemize}
      \vfill
      \onslide<3->{
      \listocaml[firstline=205,lastline=209,highlightlines=207]{../ocaml/otherlibs/dynlink/dynlink\_compilerlibs/misc.ml}{otherlibs/dynlink/dynlink\_compilerlibs/misc.ml:205}
      }
      \endminipage
    \end{column}
    \begin{column}{0.55\textwidth}
    \only<4>{
      \mintcmm[highlightlines=3]{../src/notrace/camlNotrace\_\_fun_347.cmm}
      \listcmm{../src/notrace/camlNotrace\_\_all\_somes\_327.cmm}{all\_somes}
    }
    \onslide*<5->{
      \listobjdump[highlightlines={6-9}]{../src/notrace/camlNotrace\_\_fun_347.objdump}{anonymous function from \funcname{all\_somes}}
    }
    \end{column}
  \end{columns}
\end{frame}

\begin{frame}{Backtraces}{Summary table of the multiple kinds of raise}
  \begin{itemize}
    \item When debugging information is enabled and if the backtrace of an exception isn't needed, it's possible to raise with \funcname{raise\_notrace} for performance
    \item When debugging information is \emph{not} enabled, the compiler optimises \funcname{raise} calls to inline assembly (same effect as using \funcname{raise\_notrace})
  \end{itemize}
  \bigskip
  \centering
  \begin{tabular}{| l | c | c | c | c |}
    \hline
    \parbox{10em}{Debug information \\ (\funcarg{-g} flag)}  & \multicolumn{2}{|c|}{Disabled}                                     & \multicolumn{2}{|c|}{Enabled} \\
    \hline
    Raise kind                                               & \funcname{raise} & \funcname{raise\_notrace}                       & \funcname{raise} & \funcname{raise\_notrace} \\
    \hline
    Compilation                                              & \parbox{8em}{inline assembly\\ (optimization)} & inline assembly   & \funcname{caml\_raise\_exn} & inline assembly \\
    \hline
  \end{tabular}
\end{frame}


%
% Takeaway #5
%
\subsection*{Takeaway \#5}
\frameSubsectionTakeaway{}
\againframe<5>{takeaway}
}%
      \onslide*<5>{\providecommand\step{9}%
% Backtraces
%
\subsection{One advantage of raising with debugging information}
\frameSubsection{}{}

\begin{frame}{Backtraces}{Debugging information \& source code}
  \begin{columns}[c]
    \begin{column}{0.5\textwidth}
      \begin{itemize}
        \item Raising an exception is fun and games, but how do we know where the exception originated? $\rightarrow$ With the backtrace associated with the exception!
        \item In order to get a backtrace from the point where an exception was raised, it's necessary to compile with debugging information enabled (\funcarg{-g} flag for the compiler)
        \item Same example as in the `Nested handlers' section
      \end{itemize}
    \end{column}
    \begin{column}{0.5\textwidth}
      \listocaml[firstline=9,lastline=34]{../src/backtrace/backtrace.ml}{backtrace/backtrace.ml}
    \end{column}
  \end{columns}
\end{frame}

\begin{frame}{Backtraces}{CMM and disassembly of \funcname{definitely\_raise}}
  \begin{columns}[c]
    \begin{column}{0.5\textwidth}
      \minipage[c][0.66\textheight][s]{\columnwidth}
      \begin{itemize}
        \item<1-> An effect of compiling with debugging information (\funcarg{-g}): the CMM no longer uses \funcname{raise\_notrace} to raise the exception but \funcname{raise} instead
        \item<2-> Disassembly shows that CMM \funcname{raise} is actually a call to \funcname{caml\_raise\_exn}, a function in assembler provided by the runtime
      \end{itemize}
      \vfill
      \mintocaml[firstline=9,lastline=13,highlightlines=12]{../src/backtrace/backtrace.ml}
      \endminipage
    \end{column}
    \begin{column}{0.5\textwidth}
      \onslide*<1>{
        \mintcmm[highlightlines=5]{../src/backtrace/camlBacktrace\_\_definitely\_raise\_347.cmm}
      }
      \onslide*<2>{
        \mintobjdump[firstline=2,highlightlines=14]{../src/backtrace/camlBacktrace\_\_definitely\_raise\_347.objdump}
      }
    \end{column}
  \end{columns}
\end{frame}

\begin{frame}{Backtraces}{Introducing \funcname{caml\_raise\_exn}}
  \begin{columns}[c]
    \begin{column}{0.5\textwidth}
      \minipage[c][0.66\textheight][s]{\columnwidth}
      \onslide*<1>{
        \begin{itemize}
          \item Raising the exception is performed using \funcname{caml\_raise\_exn} which is
            \begin{itemize}
              \item An assembly function provided by the runtime
              \item Architecture specific
            \end{itemize}
          \item Let's see what it does differently from \funcname{raise\_notrace}\dots
          \item We are about to raise \typename{ExnC} with \funcname{caml\_raise\_exn} from \funcname{definitely\_raise}
        \end{itemize}
      }
      \onslide*<2>{
        \mintobjdump[firstline=2,highlightlines=14]{../src/backtrace/camlBacktrace\_\_definitely\_raise\_347.objdump}
      }
      \vfill
      \mintocaml[firstline=9,lastline=13,highlightlines=12]{../src/backtrace/backtrace.ml}
      \endminipage
    \end{column}
    \begin{column}{0.5\textwidth}
      \centering
      \providecommand\step{1}\input{diagrams/backtrace/backtrace.tex}
    \end{column}
  \end{columns}
\end{frame}

\begin{frame}{Backtraces}{But before that, we need more fields from \typename{caml\_domain\_state}}
  \begin{columns}
    \begin{column}{0.5\textwidth}
      \begin{itemize}
        \item Remember \typename{caml\_domain\_state} the per-domain runtime state?
        \item Some other members are of interest to us now\dots
        \item \localname{backtrace\_active} is used to know if the runtime should collect backtraces
        \item \localname{backtrace\_active} can be set by
          \begin{itemize}
            \item The \funcarg{b} flag in the \funcname{OCAMLRUNPARAM} environment variable
            \item \mintinline{ocaml}{Printexc.record_backtrace : bool -> unit} \footnotemark
          \end{itemize}
      \end{itemize}
    \end{column}
    \begin{column}{0.5\textwidth}
      \begin{listing}
        \mintc[highlightlines={9-11}]{src/ocaml/domain\_state\_backtrace.h}
        \caption{runtime/caml/domain\_state.h}
      \end{listing}
    \end{column}
  \end{columns}
  \footnotetext[1]{https://v2.ocaml.org/api/Printexc.html}
\end{frame}

\newcommand\listCamlRaiseExn[1][]{
  \listasm[firstline=702,lastline=725,#1]{../ocaml/runtime/amd64.S}{runtime/amd64.S:702}}

\begin{frame}{Backtraces}{Walk through \funcname{caml\_raise\_exn}}
  \begin{columns}[T]
    \begin{column}{0.5\textwidth}
      \begin{itemize}
        \item<1-> First checks whether exceptions backtrace should be recorded
        \item<1-> By checking the \funcarg{domain\_state->backtrace\_active} value
        \item<2-> If not, acts as \funcname{raise\_notrace} with \funcname{RESTORE\_EXN\_HANDLER\_OCAML}
          \begin{itemize}
            \item Set \regname{RSP} to \localname{domain\_state->exn\_handler} value
            \item Restore \localname{domain\_state->exn\_handler} from trap
            \item Jump with \funcname{ret} (same effect as using \funcname{pop} and \funcname{jmp})
          \end{itemize}
        \item<3-> Otherwise, switches to the C stack to be able to execute C code
        \item<4-> Calls \funcname{caml\_stash\_backtrace}, a C function provided by the runtime\ignorespaces
          \onslide<6->{, with
            \begin{itemize}
              \item \funcarg{exn}: exception value
              \item \funcarg{pc}: code address of the raising point
              \item \funcarg{sp}: stack address of the raising function
              \item \funcarg{trapsp}: stack address of the next handler
            \end{itemize}
          }
      \end{itemize}
      \bigskip
      \mintocaml[firstline=9,lastline=13,highlightlines=12]{../src/backtrace/backtrace.ml}
    \end{column}
    \begin{column}{0.5\textwidth}
      \raggedleft
      \onslide*<1,3-4>{
        \mintc[firstline=190,lastline=192]{../ocaml/runtime/amd64.S}
        \mintc[firstline=234,lastline=237]{../ocaml/runtime/amd64.S}
      }
      \onslide*<2>{
        \mintc[firstline=190,lastline=192,highlightlines={191-192}]{../ocaml/runtime/amd64.S}
        \mintc[firstline=234,lastline=237,highlightlines={235-237}]{../ocaml/runtime/amd64.S}
      }
      \onslide*<1>{\listCamlRaiseExn[highlightlines=705]{}}%
      \onslide*<2>{\listCamlRaiseExn[highlightlines={707-708}]{}}%
      \onslide*<3>{\listCamlRaiseExn[highlightlines={712-714}]{}}%
      \onslide*<4>{\listCamlRaiseExn[highlightlines={715-720}]{}}%
      \onslide*<5>{\providecommand\step{1}\input{diagrams/backtrace/backtrace.tex}}%
      \onslide*<6>{\providecommand\step{2}\input{diagrams/backtrace/backtrace.tex}}%
    \end{column}
  \end{columns}
\end{frame}

\newcommand\listStashBacktrace[1][]{
  \listc[firstline=91,lastline=120,#1]{../ocaml/runtime/backtrace\_nat.c}{runtime/backtrace\_nat.c:91}}

\begin{frame}{Backtraces}{Introducing \funcname{caml\_stash\_backtrace}}
  \begin{columns}[c]
    \begin{column}{0.5\textwidth}
      \minipage[c][0.66\textheight][s]{\columnwidth}
      \begin{itemize}
        \item<1-> A C function provided by the runtime (specific to native code)
        \item<2-> It scans the stack, between the stack pointer at the raising point up to the next trap, in order to collect the names of the detected function frames
          \onslide*<2>{
            \begin{itemize}
              \item And more information, like line number, column number...
            \end{itemize}
          }
        \item<3-> It uses the same technique and information as the Garbage Collector (GC) uses to scan the values live on the stack
          \onslide*<4>{
            \begin{itemize}
              \item \typename{frame\_descr} are at the core of stack unwinding
            \end{itemize}
          }
      \end{itemize}
      \vfill
      \mintocaml[firstline=9,lastline=13,highlightlines=12]{../src/backtrace/backtrace.ml}
      \endminipage
    \end{column}
    \begin{column}{0.5\textwidth}
      \onslide*<1>{\listStashBacktrace[highlightlines=91]{}}
      \onslide*<2>{\listStashBacktrace[highlightlines={91,117-118}]{}}
      \onslide*<3->{\listStashBacktrace[highlightlines={94,106,109-110}]{}}
    \end{column}
  \end{columns}
\end{frame}

\newcommand\listFrameDescr[1][]{
  \listocaml[firstline=111,lastline=124,#1]{../ocaml/asmcomp/emitaux.ml}{asmcomp/emitaux.ml:111}}
\newcommand\listDebugInfo[1][]{
  \listocaml[firstline=38,lastline=49,#1]{../ocaml/lambda/debuginfo.mli}{lambda/debuginfo.mli:38}}

\begin{frame}{Backtraces}{\typename{frame\_descr} and \typename{frame\_debuginfo} in a nutshell}
  \begin{columns}[c]
    \begin{column}{0.5\textwidth}
      \begin{itemize}
        \item<1-> For every return address in an OCaml program, there is a associated \typename{frame\_descr}
        \item<2-> A \typename{frame\_descr} specifies
        \begin{itemize}
          \item<2-> The size of the function stack frame
          \item<3-> Where live values are on the stack (so that the GC can mark them as live and prevent from being swept)
        \end{itemize}
        \item<4-> When compiled with debug information, \typename{frame\_descr} will be linked to a \typename{frame\_debuginfo}
        \begin{itemize}
          \item<5-> Contains information about the position in the source code (file name, line and column number)
        \end{itemize}
      \end{itemize}
    \end{column}
    \begin{column}{0.5\textwidth}
        \onslide*<1>{\listFrameDescr[highlightlines={118-122}]{}}
        \onslide*<2>{\listFrameDescr[highlightlines={120}]{}}
        \onslide*<3>{\listFrameDescr[highlightlines={121}]{}}
        \onslide*<4->{\listFrameDescr[highlightlines={122}]{}}
        \medskip
        \onslide*<1-3>{\listDebugInfo{}}
        \onslide*<4>{\listDebugInfo[highlightlines={38,49}]{}}
        \onslide*<5>{\listDebugInfo[highlightlines={39-46}]{}}
    \end{column}
  \end{columns}
\end{frame}

\begin{frame}{Backtraces}{Scanning the stack with \typename{frame\_descr}}
  \begin{columns}[c]
    \begin{column}{0.5\textwidth}
      \begin{itemize}
        \item Given the return address of the code that raises the exception by calling \funcname{caml\_raise\_exn} (\funcarg{pc} argument of \funcname{caml\_stash\_backtrace})
        \item And the position of the stack pointer (\funcarg{sp} argument of \funcname{caml\_stash\_backtrace})
        \item The frame size given by \typename{frame\_descr}.\funcarg{frame\_size}, allows to find the end of the previous frame
        \item And the return address of the caller of this function
        \bigskip
        \onslide<3->{
        \item Note that traps (here the reddish \funcname{ExnA}, \funcname{ExnB} and \funcname{ExnC} blocks) belong to the frame that installed them, and so the frame size changes when traps are removed
        }
      \end{itemize}
    \end{column}
    \begin{column}{0.5\textwidth}
      \raggedleft
      \onslide*<1>{\providecommand\step{2}\input{diagrams/backtrace/backtrace.tex}}%
      \onslide*<2>{\providecommand\step{1}\input{diagrams/backtrace/scanstack.tex}}%
      \onslide*<3>{\providecommand\step{2}\input{diagrams/backtrace/scanstack.tex}}%
      \onslide*<4>{\providecommand\step{3}\input{diagrams/backtrace/scanstack.tex}}%
      \onslide*<5>{\providecommand\step{4}\input{diagrams/backtrace/scanstack.tex}}%
      \onslide*<6>{\providecommand\step{5}\input{diagrams/backtrace/scanstack.tex}}%
    \end{column}
  \end{columns}
\end{frame}

% Duplicate of previous-previous slide
\begin{frame}{Backtraces}{Continuing on \funcname{caml\_stash\_backtrace}}
  \begin{columns}[c]
    \begin{column}{0.5\textwidth}
      \minipage[c][0.75\textheight][s]{\columnwidth}
      \begin{itemize}
          \color{gray}
        \item A C function provided by the runtime (specific to native code)
        \item It scans the stack, between the stack pointer at the raising point up to the next trap, in order to collect the names of the detected function frames
        \item It uses the same technique and information as the Garbage Collector (GC) uses to scan the values live on the stack
      \end{itemize}
      \begin{itemize}
        \item For each function frame detected, store a pointer to the \typename{frame\_descr} in the \localname{domain\_state->backtrace\_buffer} array, effectively constructing the backtrace
      \end{itemize}
      \onslide<2->{
        \begin{itemize}
            \smallskip
          \item Constructed backtrace:
            \funcname{definitely\_raise},
            \onslide<3->{\funcname{probably\_raise}}
        \end{itemize}
      }
      \vfill
      \mintocaml[firstline=9,lastline=13,highlightlines=12]{../src/backtrace/backtrace.ml}
      \endminipage
    \end{column}
    \begin{column}{0.5\textwidth}
      \raggedleft
      \onslide*<1>{\listStashBacktrace[highlightlines={114-115}]{}}
      \onslide*<2>{\providecommand\step{3}\input{diagrams/backtrace/backtrace.tex}}%
      \onslide*<3>{\providecommand\step{4}\input{diagrams/backtrace/backtrace.tex}}%
    \end{column}
  \end{columns}
\end{frame}

\begin{frame}{Backtraces}{Back to \funcname{caml\_raise\_exn}}
  \begin{columns}[c]
    \begin{column}{0.5\textwidth}
      \minipage[c][0.66\textheight][s]{\columnwidth}
      \begin{itemize}\color{gray}
        \item Checks wheteher exceptions backtrace should be recorded, by checking the \funcarg{domain\_state->backtrace\_active} value
        \item Switches to the C stack to be able to execute C code
        \item Calls \funcname{caml\_stash\_backtrace}, to collect the backtrace up to the next trap
      \end{itemize}
      \onslide*<2->{
        \begin{itemize}
          \item<2-> Executes the exception handler in \funcname{probably\_raise} with \funcname{RESTORE\_EXN\_HANDLER\_OCAML}
            \begin{itemize}
              \item Set \regname{RSP} to \localname{domain\_state->exn\_handler} value
              \item Restore \localname{domain\_state->exn\_handler} from trap
              \item Jump to the handler code with \funcname{ret}
            \end{itemize}
        \end{itemize}
      }
      \vfill
      \onslide*<1>{\mintocaml[firstline=9,lastline=13,highlightlines=12]{../src/backtrace/backtrace.ml}}
      \onslide*<2->{\mintocaml[firstline=15,lastline=21,highlightlines={19-21}]{../src/backtrace/backtrace.ml}}
      \endminipage
    \end{column}
    \begin{column}{0.5\textwidth}
      \centering
      \onslide*<1>{
        \mintc[firstline=190,lastline=192]{../ocaml/runtime/amd64.S}
        \mintc[firstline=234,lastline=237]{../ocaml/runtime/amd64.S}
      }
      \onslide*<2>{
        \mintc[firstline=190,lastline=192,highlightlines={191-192}]{../ocaml/runtime/amd64.S}
        \mintc[firstline=234,lastline=237,highlightlines={235-237}]{../ocaml/runtime/amd64.S}
      }
      \onslide*<1>{\listCamlRaiseExn[highlightlines=720]{}}
      \onslide*<2>{\listCamlRaiseExn[highlightlines=722]{}}
      \onslide*<3->{\providecommand\step{5}\input{diagrams/backtrace/backtrace.tex}}
    \end{column}
  \end{columns}
\end{frame}

\begin{frame}{Backtraces}{\funcname{caml\_probably\_raise}'s handler doesn't catch \typename{ExnC}}
  \begin{columns}[c]
    \begin{column}{0.45\textwidth}
      \minipage[c][0.75\textheight][s]{\columnwidth}
      \begin{itemize}
        \item<1-> \funcname{match \dots with}\footnotemark pattern matching can also match exceptions
        \item<1-> The CMM of \funcname{probably\_raise} shows that this \funcname{match \dots with} is implemented with a pattern matching and a \funcname{try \dots with}
        \item<1-> But this one only matches on \typename{ExnA} exception
        \item<2-> Since the pattern matching fails to catch the exception, \funcname{reraise} is called with our current \typename{ExnC} exception
        \item<3-> Disassembly shows that \funcname{reraise} is actually a call to \funcname{caml\_reraise\_exn}, which is also an assembly function provided by the runtime
      \end{itemize}
      \vfill
      \mintocaml[firstline=15,lastline=21,highlightlines={18-21}]{../src/backtrace/backtrace.ml}
      \endminipage
    \end{column}
    \begin{column}{0.55\textwidth}
      \centering
      \onslide*<1>{\mintcmm[highlightlines={10-18}]{../src/backtrace/camlBacktrace\_\_probably\_raise\_351.cmm}}
      \onslide*<2>{\listcmm[highlightlines=18]{../src/backtrace/camlBacktrace\_\_probably\_raise_351.cmm}{CMM of probably\_raise}}
      \onslide*<3>{\mintobjdump[firstline=5,lastline=35,highlightlines=31]{../src/backtrace/camlBacktrace\_\_probably\_raise_351.objdump}}
    \end{column}
  \end{columns}
  \only<1>{\footnotetext[1]{https://blog.janestreet.com/pattern-matching-and-exception-handling-unite/}}
\end{frame}

\newcommand\listCamlReraiseExn[1][]{
  \listasm[firstline=727,lastline=734,#1]{../ocaml/runtime/amd64.S}{runtime/amd64.S:727}}

\begin{frame}{Backtraces}{Introducing \funcname{caml\_reraise\_exn}}
  \begin{columns}[c]
    \begin{column}{0.5\textwidth}
      \minipage[c][0.66\textheight][s]{\columnwidth}
      \begin{itemize}
        \item<1-> The exception have to be reraised to the next handler
        \item<2-> First, checks if the backtrace should be collected with \funcarg{domain\_state->backtrace\_active}
        \only<3>{
        \begin{itemize}
            \item If not, behaves like a \funcname{raise\_notrace} with \funcname{RESTORE\_EXN\_HANDLER\_OCAML}
        \end{itemize}
        }
        \item<4-> Jumps into \funcname{caml\_raise\_exn}
        \item<5-> Calls \funcname{caml\_stash\_backtrace} again, to scan the next part of the stack: from the trap just pop-ed to the next trap
      \end{itemize}
      \vfill
      \mintocaml[firstline=15,lastline=21,highlightlines={18-21}]{../src/backtrace/backtrace.ml}
      \endminipage
    \end{column}
    \begin{column}{0.5\textwidth}
      \raggedleft
      \onslide*<1>{\listCamlReraiseExn{}}
      \onslide*<2>{\listCamlReraiseExn[highlightlines=729]{}}
      \onslide*<3>{
        \mintc[firstline=190,lastline=192,highlightlines={191-192}]{../ocaml/runtime/amd64.S}
        \mintc[firstline=234,lastline=237,highlightlines={235-237}]{../ocaml/runtime/amd64.S}
        \medskip
        \listCamlReraiseExn[highlightlines=731]{}
      }
      \onslide*<4-5>{
        % TODO refactor with \listCamlReraiseExn
        \mintasm[firstline=727,lastline=734,highlightlines=730]{../ocaml/runtime/amd64.S}
        \medskip
      }
      \onslide*<4>{\listCamlRaiseExn[highlightlines=711]{}}
      \onslide*<5>{\listCamlRaiseExn[highlightlines=720]{}}
    \end{column}
  \end{columns}
\end{frame}

\begin{frame}{Backtraces}{Backtrace collection continues}
  \begin{columns}[c]
    \begin{column}{0.55\textwidth}
      \minipage[c][0.75\textheight][s]{\columnwidth}
      \begin{itemize}
        \item<1-> \funcname{caml\_stash\_backtrace} scans the older part of the stack, up to the next trap
        \item<6-> Controls jumps to the nested exception handler in \funcname{maybe\_raise}, which matches only on \typename{ExnB}
        \item<7-> \funcname{caml\_reraise\_exn} is called again, backtrace collection resumes with \funcname{caml\_stash\_backtrace}
      \end{itemize}
      \medskip
      \begin{itemize}
        \item<2-> Constructed backtrace:
        \begin{itemize}
          \item <2-> \funcname{definitely\_raise}
          \item <2-> \funcname{probably\_raise}
          \item <3-> \funcname{probably\_raise} (re-raise)
          \item <4-> \funcname{maybe\_raise}
          \item <5-> \funcname{main}
          \item <8-> \funcname{main} (re-raise)
        \end{itemize}
      \end{itemize}
      \vfill
      \onslide*<1-5>{\mintocaml[firstline=15,lastline=21]{../src/backtrace/backtrace.ml}}
      \onslide*<6>{\mintocaml[firstline=28,lastline=34,highlightlines=31]{../src/backtrace/backtrace.ml}}
      \onslide*<7->{\mintocaml[firstline=28,lastline=34,highlightlines=33]{../src/backtrace/backtrace.ml}}
      \endminipage
    \end{column}
    \begin{column}{0.45\textwidth}
      \raggedleft
      \onslide*<1>{\providecommand\step{6}\input{diagrams/backtrace/backtrace.tex}}%
      \onslide*<2>{\providecommand\step{6}\input{diagrams/backtrace/backtrace.tex}}%
      \onslide*<3>{\providecommand\step{7}\input{diagrams/backtrace/backtrace.tex}}%
      \onslide*<4>{\providecommand\step{8}\input{diagrams/backtrace/backtrace.tex}}%
      \onslide*<5>{\providecommand\step{9}\input{diagrams/backtrace/backtrace.tex}}%
      \onslide*<6>{\providecommand\step{10}\input{diagrams/backtrace/backtrace.tex}}%
      \onslide*<7>{\providecommand\step{11}\input{diagrams/backtrace/backtrace.tex}}%
      \onslide*<8>{\providecommand\step{12}\input{diagrams/backtrace/backtrace.tex}}%
      \onslide*<9>{\providecommand\step{13}\input{diagrams/backtrace/backtrace.tex}}%
    \end{column}
  \end{columns}
\end{frame}

\begin{frame}{Backtraces}{Printing the backtrace}
  \begin{columns}[c]
    \begin{column}{0.45\textwidth}
      \minipage[c][0.66\textheight][s]{\columnwidth}
      \begin{itemize}
        \item<1-> The exception backtrace can now be printed to \funcarg{stdout} with \funcname{Printexc.print_backtrace}
        \item<2-> First the exception backtrace is retrieved into an OCaml array with \funcname{get_raw_backtrace }
        \item<3-> Which is implemented as the \funcname{caml_get_exception_raw_backtrace} C function in the runtime
        \item<4-> And finally printed with \funcname{print_exception_backtrace}
      \end{itemize}
      \vfill
      \mintocaml[firstline=28,lastline=34,highlightlines=33]{../src/backtrace/backtrace.ml}
      \endminipage
    \end{column}
    \begin{column}{0.55\textwidth}
      \onslide*<1,2>{
        \mintocaml[firstline=100,lastline=101]{../ocaml/stdlib/printexc.ml}
      }
      \only<1>{
        \listocaml[firstline=170,lastline=172]{../ocaml/stdlib/printexc.ml}{stdlib/printexc.ml:170}
      }
      \only<2>{
        \listocaml[firstline=170,lastline=172,highlightlines=172]{../ocaml/stdlib/printexc.ml}{stdlib/printexc.ml:170}
      }
      \only<3>{
        \mintc[firstline=154,lastline=156]{../ocaml/runtime/backtrace.c}
        \listc[firstline=171,lastline=192,highlightlines={184-187}]{../ocaml/runtime/backtrace.c}{runtime/backtrace.c:154}
      }
      \only<4>{
        \listocaml[firstline=155,lastline=165]{../ocaml/stdlib/printexc.ml}{stdlib/printexc.ml:55}
      }
    \end{column}
  \end{columns}
\end{frame}


\subsection{The multiple kinds of raise}
\frameSubsection{}{}

\begin{frame}{Backtraces}{When backtraces are not needed}
  \begin{columns}[c]
    \begin{column}{0.45\textwidth}
      \minipage[c][0.66\textheight][s]{\columnwidth}
      \begin{itemize}
        \item<1-> What if the backtrace for an exception is never needed?
        \item<2-> It's possible to raise the exception explicitly with \funcname{raise\_notrace} to make raising as cheap as possible
        \item<3-> An example of use in the \funcname{dynlink} library of the OCaml compiler
      \end{itemize}
      \vfill
      \onslide<3->{
      \listocaml[firstline=205,lastline=209,highlightlines=207]{../ocaml/otherlibs/dynlink/dynlink\_compilerlibs/misc.ml}{otherlibs/dynlink/dynlink\_compilerlibs/misc.ml:205}
      }
      \endminipage
    \end{column}
    \begin{column}{0.55\textwidth}
    \only<4>{
      \mintcmm[highlightlines=3]{../src/notrace/camlNotrace\_\_fun_347.cmm}
      \listcmm{../src/notrace/camlNotrace\_\_all\_somes\_327.cmm}{all\_somes}
    }
    \onslide*<5->{
      \listobjdump[highlightlines={6-9}]{../src/notrace/camlNotrace\_\_fun_347.objdump}{anonymous function from \funcname{all\_somes}}
    }
    \end{column}
  \end{columns}
\end{frame}

\begin{frame}{Backtraces}{Summary table of the multiple kinds of raise}
  \begin{itemize}
    \item When debugging information is enabled and if the backtrace of an exception isn't needed, it's possible to raise with \funcname{raise\_notrace} for performance
    \item When debugging information is \emph{not} enabled, the compiler optimises \funcname{raise} calls to inline assembly (same effect as using \funcname{raise\_notrace})
  \end{itemize}
  \bigskip
  \centering
  \begin{tabular}{| l | c | c | c | c |}
    \hline
    \parbox{10em}{Debug information \\ (\funcarg{-g} flag)}  & \multicolumn{2}{|c|}{Disabled}                                     & \multicolumn{2}{|c|}{Enabled} \\
    \hline
    Raise kind                                               & \funcname{raise} & \funcname{raise\_notrace}                       & \funcname{raise} & \funcname{raise\_notrace} \\
    \hline
    Compilation                                              & \parbox{8em}{inline assembly\\ (optimization)} & inline assembly   & \funcname{caml\_raise\_exn} & inline assembly \\
    \hline
  \end{tabular}
\end{frame}


%
% Takeaway #5
%
\subsection*{Takeaway \#5}
\frameSubsectionTakeaway{}
\againframe<5>{takeaway}
}%
      \onslide*<6>{\providecommand\step{10}%
% Backtraces
%
\subsection{One advantage of raising with debugging information}
\frameSubsection{}{}

\begin{frame}{Backtraces}{Debugging information \& source code}
  \begin{columns}[c]
    \begin{column}{0.5\textwidth}
      \begin{itemize}
        \item Raising an exception is fun and games, but how do we know where the exception originated? $\rightarrow$ With the backtrace associated with the exception!
        \item In order to get a backtrace from the point where an exception was raised, it's necessary to compile with debugging information enabled (\funcarg{-g} flag for the compiler)
        \item Same example as in the `Nested handlers' section
      \end{itemize}
    \end{column}
    \begin{column}{0.5\textwidth}
      \listocaml[firstline=9,lastline=34]{../src/backtrace/backtrace.ml}{backtrace/backtrace.ml}
    \end{column}
  \end{columns}
\end{frame}

\begin{frame}{Backtraces}{CMM and disassembly of \funcname{definitely\_raise}}
  \begin{columns}[c]
    \begin{column}{0.5\textwidth}
      \minipage[c][0.66\textheight][s]{\columnwidth}
      \begin{itemize}
        \item<1-> An effect of compiling with debugging information (\funcarg{-g}): the CMM no longer uses \funcname{raise\_notrace} to raise the exception but \funcname{raise} instead
        \item<2-> Disassembly shows that CMM \funcname{raise} is actually a call to \funcname{caml\_raise\_exn}, a function in assembler provided by the runtime
      \end{itemize}
      \vfill
      \mintocaml[firstline=9,lastline=13,highlightlines=12]{../src/backtrace/backtrace.ml}
      \endminipage
    \end{column}
    \begin{column}{0.5\textwidth}
      \onslide*<1>{
        \mintcmm[highlightlines=5]{../src/backtrace/camlBacktrace\_\_definitely\_raise\_347.cmm}
      }
      \onslide*<2>{
        \mintobjdump[firstline=2,highlightlines=14]{../src/backtrace/camlBacktrace\_\_definitely\_raise\_347.objdump}
      }
    \end{column}
  \end{columns}
\end{frame}

\begin{frame}{Backtraces}{Introducing \funcname{caml\_raise\_exn}}
  \begin{columns}[c]
    \begin{column}{0.5\textwidth}
      \minipage[c][0.66\textheight][s]{\columnwidth}
      \onslide*<1>{
        \begin{itemize}
          \item Raising the exception is performed using \funcname{caml\_raise\_exn} which is
            \begin{itemize}
              \item An assembly function provided by the runtime
              \item Architecture specific
            \end{itemize}
          \item Let's see what it does differently from \funcname{raise\_notrace}\dots
          \item We are about to raise \typename{ExnC} with \funcname{caml\_raise\_exn} from \funcname{definitely\_raise}
        \end{itemize}
      }
      \onslide*<2>{
        \mintobjdump[firstline=2,highlightlines=14]{../src/backtrace/camlBacktrace\_\_definitely\_raise\_347.objdump}
      }
      \vfill
      \mintocaml[firstline=9,lastline=13,highlightlines=12]{../src/backtrace/backtrace.ml}
      \endminipage
    \end{column}
    \begin{column}{0.5\textwidth}
      \centering
      \providecommand\step{1}\input{diagrams/backtrace/backtrace.tex}
    \end{column}
  \end{columns}
\end{frame}

\begin{frame}{Backtraces}{But before that, we need more fields from \typename{caml\_domain\_state}}
  \begin{columns}
    \begin{column}{0.5\textwidth}
      \begin{itemize}
        \item Remember \typename{caml\_domain\_state} the per-domain runtime state?
        \item Some other members are of interest to us now\dots
        \item \localname{backtrace\_active} is used to know if the runtime should collect backtraces
        \item \localname{backtrace\_active} can be set by
          \begin{itemize}
            \item The \funcarg{b} flag in the \funcname{OCAMLRUNPARAM} environment variable
            \item \mintinline{ocaml}{Printexc.record_backtrace : bool -> unit} \footnotemark
          \end{itemize}
      \end{itemize}
    \end{column}
    \begin{column}{0.5\textwidth}
      \begin{listing}
        \mintc[highlightlines={9-11}]{src/ocaml/domain\_state\_backtrace.h}
        \caption{runtime/caml/domain\_state.h}
      \end{listing}
    \end{column}
  \end{columns}
  \footnotetext[1]{https://v2.ocaml.org/api/Printexc.html}
\end{frame}

\newcommand\listCamlRaiseExn[1][]{
  \listasm[firstline=702,lastline=725,#1]{../ocaml/runtime/amd64.S}{runtime/amd64.S:702}}

\begin{frame}{Backtraces}{Walk through \funcname{caml\_raise\_exn}}
  \begin{columns}[T]
    \begin{column}{0.5\textwidth}
      \begin{itemize}
        \item<1-> First checks whether exceptions backtrace should be recorded
        \item<1-> By checking the \funcarg{domain\_state->backtrace\_active} value
        \item<2-> If not, acts as \funcname{raise\_notrace} with \funcname{RESTORE\_EXN\_HANDLER\_OCAML}
          \begin{itemize}
            \item Set \regname{RSP} to \localname{domain\_state->exn\_handler} value
            \item Restore \localname{domain\_state->exn\_handler} from trap
            \item Jump with \funcname{ret} (same effect as using \funcname{pop} and \funcname{jmp})
          \end{itemize}
        \item<3-> Otherwise, switches to the C stack to be able to execute C code
        \item<4-> Calls \funcname{caml\_stash\_backtrace}, a C function provided by the runtime\ignorespaces
          \onslide<6->{, with
            \begin{itemize}
              \item \funcarg{exn}: exception value
              \item \funcarg{pc}: code address of the raising point
              \item \funcarg{sp}: stack address of the raising function
              \item \funcarg{trapsp}: stack address of the next handler
            \end{itemize}
          }
      \end{itemize}
      \bigskip
      \mintocaml[firstline=9,lastline=13,highlightlines=12]{../src/backtrace/backtrace.ml}
    \end{column}
    \begin{column}{0.5\textwidth}
      \raggedleft
      \onslide*<1,3-4>{
        \mintc[firstline=190,lastline=192]{../ocaml/runtime/amd64.S}
        \mintc[firstline=234,lastline=237]{../ocaml/runtime/amd64.S}
      }
      \onslide*<2>{
        \mintc[firstline=190,lastline=192,highlightlines={191-192}]{../ocaml/runtime/amd64.S}
        \mintc[firstline=234,lastline=237,highlightlines={235-237}]{../ocaml/runtime/amd64.S}
      }
      \onslide*<1>{\listCamlRaiseExn[highlightlines=705]{}}%
      \onslide*<2>{\listCamlRaiseExn[highlightlines={707-708}]{}}%
      \onslide*<3>{\listCamlRaiseExn[highlightlines={712-714}]{}}%
      \onslide*<4>{\listCamlRaiseExn[highlightlines={715-720}]{}}%
      \onslide*<5>{\providecommand\step{1}\input{diagrams/backtrace/backtrace.tex}}%
      \onslide*<6>{\providecommand\step{2}\input{diagrams/backtrace/backtrace.tex}}%
    \end{column}
  \end{columns}
\end{frame}

\newcommand\listStashBacktrace[1][]{
  \listc[firstline=91,lastline=120,#1]{../ocaml/runtime/backtrace\_nat.c}{runtime/backtrace\_nat.c:91}}

\begin{frame}{Backtraces}{Introducing \funcname{caml\_stash\_backtrace}}
  \begin{columns}[c]
    \begin{column}{0.5\textwidth}
      \minipage[c][0.66\textheight][s]{\columnwidth}
      \begin{itemize}
        \item<1-> A C function provided by the runtime (specific to native code)
        \item<2-> It scans the stack, between the stack pointer at the raising point up to the next trap, in order to collect the names of the detected function frames
          \onslide*<2>{
            \begin{itemize}
              \item And more information, like line number, column number...
            \end{itemize}
          }
        \item<3-> It uses the same technique and information as the Garbage Collector (GC) uses to scan the values live on the stack
          \onslide*<4>{
            \begin{itemize}
              \item \typename{frame\_descr} are at the core of stack unwinding
            \end{itemize}
          }
      \end{itemize}
      \vfill
      \mintocaml[firstline=9,lastline=13,highlightlines=12]{../src/backtrace/backtrace.ml}
      \endminipage
    \end{column}
    \begin{column}{0.5\textwidth}
      \onslide*<1>{\listStashBacktrace[highlightlines=91]{}}
      \onslide*<2>{\listStashBacktrace[highlightlines={91,117-118}]{}}
      \onslide*<3->{\listStashBacktrace[highlightlines={94,106,109-110}]{}}
    \end{column}
  \end{columns}
\end{frame}

\newcommand\listFrameDescr[1][]{
  \listocaml[firstline=111,lastline=124,#1]{../ocaml/asmcomp/emitaux.ml}{asmcomp/emitaux.ml:111}}
\newcommand\listDebugInfo[1][]{
  \listocaml[firstline=38,lastline=49,#1]{../ocaml/lambda/debuginfo.mli}{lambda/debuginfo.mli:38}}

\begin{frame}{Backtraces}{\typename{frame\_descr} and \typename{frame\_debuginfo} in a nutshell}
  \begin{columns}[c]
    \begin{column}{0.5\textwidth}
      \begin{itemize}
        \item<1-> For every return address in an OCaml program, there is a associated \typename{frame\_descr}
        \item<2-> A \typename{frame\_descr} specifies
        \begin{itemize}
          \item<2-> The size of the function stack frame
          \item<3-> Where live values are on the stack (so that the GC can mark them as live and prevent from being swept)
        \end{itemize}
        \item<4-> When compiled with debug information, \typename{frame\_descr} will be linked to a \typename{frame\_debuginfo}
        \begin{itemize}
          \item<5-> Contains information about the position in the source code (file name, line and column number)
        \end{itemize}
      \end{itemize}
    \end{column}
    \begin{column}{0.5\textwidth}
        \onslide*<1>{\listFrameDescr[highlightlines={118-122}]{}}
        \onslide*<2>{\listFrameDescr[highlightlines={120}]{}}
        \onslide*<3>{\listFrameDescr[highlightlines={121}]{}}
        \onslide*<4->{\listFrameDescr[highlightlines={122}]{}}
        \medskip
        \onslide*<1-3>{\listDebugInfo{}}
        \onslide*<4>{\listDebugInfo[highlightlines={38,49}]{}}
        \onslide*<5>{\listDebugInfo[highlightlines={39-46}]{}}
    \end{column}
  \end{columns}
\end{frame}

\begin{frame}{Backtraces}{Scanning the stack with \typename{frame\_descr}}
  \begin{columns}[c]
    \begin{column}{0.5\textwidth}
      \begin{itemize}
        \item Given the return address of the code that raises the exception by calling \funcname{caml\_raise\_exn} (\funcarg{pc} argument of \funcname{caml\_stash\_backtrace})
        \item And the position of the stack pointer (\funcarg{sp} argument of \funcname{caml\_stash\_backtrace})
        \item The frame size given by \typename{frame\_descr}.\funcarg{frame\_size}, allows to find the end of the previous frame
        \item And the return address of the caller of this function
        \bigskip
        \onslide<3->{
        \item Note that traps (here the reddish \funcname{ExnA}, \funcname{ExnB} and \funcname{ExnC} blocks) belong to the frame that installed them, and so the frame size changes when traps are removed
        }
      \end{itemize}
    \end{column}
    \begin{column}{0.5\textwidth}
      \raggedleft
      \onslide*<1>{\providecommand\step{2}\input{diagrams/backtrace/backtrace.tex}}%
      \onslide*<2>{\providecommand\step{1}\input{diagrams/backtrace/scanstack.tex}}%
      \onslide*<3>{\providecommand\step{2}\input{diagrams/backtrace/scanstack.tex}}%
      \onslide*<4>{\providecommand\step{3}\input{diagrams/backtrace/scanstack.tex}}%
      \onslide*<5>{\providecommand\step{4}\input{diagrams/backtrace/scanstack.tex}}%
      \onslide*<6>{\providecommand\step{5}\input{diagrams/backtrace/scanstack.tex}}%
    \end{column}
  \end{columns}
\end{frame}

% Duplicate of previous-previous slide
\begin{frame}{Backtraces}{Continuing on \funcname{caml\_stash\_backtrace}}
  \begin{columns}[c]
    \begin{column}{0.5\textwidth}
      \minipage[c][0.75\textheight][s]{\columnwidth}
      \begin{itemize}
          \color{gray}
        \item A C function provided by the runtime (specific to native code)
        \item It scans the stack, between the stack pointer at the raising point up to the next trap, in order to collect the names of the detected function frames
        \item It uses the same technique and information as the Garbage Collector (GC) uses to scan the values live on the stack
      \end{itemize}
      \begin{itemize}
        \item For each function frame detected, store a pointer to the \typename{frame\_descr} in the \localname{domain\_state->backtrace\_buffer} array, effectively constructing the backtrace
      \end{itemize}
      \onslide<2->{
        \begin{itemize}
            \smallskip
          \item Constructed backtrace:
            \funcname{definitely\_raise},
            \onslide<3->{\funcname{probably\_raise}}
        \end{itemize}
      }
      \vfill
      \mintocaml[firstline=9,lastline=13,highlightlines=12]{../src/backtrace/backtrace.ml}
      \endminipage
    \end{column}
    \begin{column}{0.5\textwidth}
      \raggedleft
      \onslide*<1>{\listStashBacktrace[highlightlines={114-115}]{}}
      \onslide*<2>{\providecommand\step{3}\input{diagrams/backtrace/backtrace.tex}}%
      \onslide*<3>{\providecommand\step{4}\input{diagrams/backtrace/backtrace.tex}}%
    \end{column}
  \end{columns}
\end{frame}

\begin{frame}{Backtraces}{Back to \funcname{caml\_raise\_exn}}
  \begin{columns}[c]
    \begin{column}{0.5\textwidth}
      \minipage[c][0.66\textheight][s]{\columnwidth}
      \begin{itemize}\color{gray}
        \item Checks wheteher exceptions backtrace should be recorded, by checking the \funcarg{domain\_state->backtrace\_active} value
        \item Switches to the C stack to be able to execute C code
        \item Calls \funcname{caml\_stash\_backtrace}, to collect the backtrace up to the next trap
      \end{itemize}
      \onslide*<2->{
        \begin{itemize}
          \item<2-> Executes the exception handler in \funcname{probably\_raise} with \funcname{RESTORE\_EXN\_HANDLER\_OCAML}
            \begin{itemize}
              \item Set \regname{RSP} to \localname{domain\_state->exn\_handler} value
              \item Restore \localname{domain\_state->exn\_handler} from trap
              \item Jump to the handler code with \funcname{ret}
            \end{itemize}
        \end{itemize}
      }
      \vfill
      \onslide*<1>{\mintocaml[firstline=9,lastline=13,highlightlines=12]{../src/backtrace/backtrace.ml}}
      \onslide*<2->{\mintocaml[firstline=15,lastline=21,highlightlines={19-21}]{../src/backtrace/backtrace.ml}}
      \endminipage
    \end{column}
    \begin{column}{0.5\textwidth}
      \centering
      \onslide*<1>{
        \mintc[firstline=190,lastline=192]{../ocaml/runtime/amd64.S}
        \mintc[firstline=234,lastline=237]{../ocaml/runtime/amd64.S}
      }
      \onslide*<2>{
        \mintc[firstline=190,lastline=192,highlightlines={191-192}]{../ocaml/runtime/amd64.S}
        \mintc[firstline=234,lastline=237,highlightlines={235-237}]{../ocaml/runtime/amd64.S}
      }
      \onslide*<1>{\listCamlRaiseExn[highlightlines=720]{}}
      \onslide*<2>{\listCamlRaiseExn[highlightlines=722]{}}
      \onslide*<3->{\providecommand\step{5}\input{diagrams/backtrace/backtrace.tex}}
    \end{column}
  \end{columns}
\end{frame}

\begin{frame}{Backtraces}{\funcname{caml\_probably\_raise}'s handler doesn't catch \typename{ExnC}}
  \begin{columns}[c]
    \begin{column}{0.45\textwidth}
      \minipage[c][0.75\textheight][s]{\columnwidth}
      \begin{itemize}
        \item<1-> \funcname{match \dots with}\footnotemark pattern matching can also match exceptions
        \item<1-> The CMM of \funcname{probably\_raise} shows that this \funcname{match \dots with} is implemented with a pattern matching and a \funcname{try \dots with}
        \item<1-> But this one only matches on \typename{ExnA} exception
        \item<2-> Since the pattern matching fails to catch the exception, \funcname{reraise} is called with our current \typename{ExnC} exception
        \item<3-> Disassembly shows that \funcname{reraise} is actually a call to \funcname{caml\_reraise\_exn}, which is also an assembly function provided by the runtime
      \end{itemize}
      \vfill
      \mintocaml[firstline=15,lastline=21,highlightlines={18-21}]{../src/backtrace/backtrace.ml}
      \endminipage
    \end{column}
    \begin{column}{0.55\textwidth}
      \centering
      \onslide*<1>{\mintcmm[highlightlines={10-18}]{../src/backtrace/camlBacktrace\_\_probably\_raise\_351.cmm}}
      \onslide*<2>{\listcmm[highlightlines=18]{../src/backtrace/camlBacktrace\_\_probably\_raise_351.cmm}{CMM of probably\_raise}}
      \onslide*<3>{\mintobjdump[firstline=5,lastline=35,highlightlines=31]{../src/backtrace/camlBacktrace\_\_probably\_raise_351.objdump}}
    \end{column}
  \end{columns}
  \only<1>{\footnotetext[1]{https://blog.janestreet.com/pattern-matching-and-exception-handling-unite/}}
\end{frame}

\newcommand\listCamlReraiseExn[1][]{
  \listasm[firstline=727,lastline=734,#1]{../ocaml/runtime/amd64.S}{runtime/amd64.S:727}}

\begin{frame}{Backtraces}{Introducing \funcname{caml\_reraise\_exn}}
  \begin{columns}[c]
    \begin{column}{0.5\textwidth}
      \minipage[c][0.66\textheight][s]{\columnwidth}
      \begin{itemize}
        \item<1-> The exception have to be reraised to the next handler
        \item<2-> First, checks if the backtrace should be collected with \funcarg{domain\_state->backtrace\_active}
        \only<3>{
        \begin{itemize}
            \item If not, behaves like a \funcname{raise\_notrace} with \funcname{RESTORE\_EXN\_HANDLER\_OCAML}
        \end{itemize}
        }
        \item<4-> Jumps into \funcname{caml\_raise\_exn}
        \item<5-> Calls \funcname{caml\_stash\_backtrace} again, to scan the next part of the stack: from the trap just pop-ed to the next trap
      \end{itemize}
      \vfill
      \mintocaml[firstline=15,lastline=21,highlightlines={18-21}]{../src/backtrace/backtrace.ml}
      \endminipage
    \end{column}
    \begin{column}{0.5\textwidth}
      \raggedleft
      \onslide*<1>{\listCamlReraiseExn{}}
      \onslide*<2>{\listCamlReraiseExn[highlightlines=729]{}}
      \onslide*<3>{
        \mintc[firstline=190,lastline=192,highlightlines={191-192}]{../ocaml/runtime/amd64.S}
        \mintc[firstline=234,lastline=237,highlightlines={235-237}]{../ocaml/runtime/amd64.S}
        \medskip
        \listCamlReraiseExn[highlightlines=731]{}
      }
      \onslide*<4-5>{
        % TODO refactor with \listCamlReraiseExn
        \mintasm[firstline=727,lastline=734,highlightlines=730]{../ocaml/runtime/amd64.S}
        \medskip
      }
      \onslide*<4>{\listCamlRaiseExn[highlightlines=711]{}}
      \onslide*<5>{\listCamlRaiseExn[highlightlines=720]{}}
    \end{column}
  \end{columns}
\end{frame}

\begin{frame}{Backtraces}{Backtrace collection continues}
  \begin{columns}[c]
    \begin{column}{0.55\textwidth}
      \minipage[c][0.75\textheight][s]{\columnwidth}
      \begin{itemize}
        \item<1-> \funcname{caml\_stash\_backtrace} scans the older part of the stack, up to the next trap
        \item<6-> Controls jumps to the nested exception handler in \funcname{maybe\_raise}, which matches only on \typename{ExnB}
        \item<7-> \funcname{caml\_reraise\_exn} is called again, backtrace collection resumes with \funcname{caml\_stash\_backtrace}
      \end{itemize}
      \medskip
      \begin{itemize}
        \item<2-> Constructed backtrace:
        \begin{itemize}
          \item <2-> \funcname{definitely\_raise}
          \item <2-> \funcname{probably\_raise}
          \item <3-> \funcname{probably\_raise} (re-raise)
          \item <4-> \funcname{maybe\_raise}
          \item <5-> \funcname{main}
          \item <8-> \funcname{main} (re-raise)
        \end{itemize}
      \end{itemize}
      \vfill
      \onslide*<1-5>{\mintocaml[firstline=15,lastline=21]{../src/backtrace/backtrace.ml}}
      \onslide*<6>{\mintocaml[firstline=28,lastline=34,highlightlines=31]{../src/backtrace/backtrace.ml}}
      \onslide*<7->{\mintocaml[firstline=28,lastline=34,highlightlines=33]{../src/backtrace/backtrace.ml}}
      \endminipage
    \end{column}
    \begin{column}{0.45\textwidth}
      \raggedleft
      \onslide*<1>{\providecommand\step{6}\input{diagrams/backtrace/backtrace.tex}}%
      \onslide*<2>{\providecommand\step{6}\input{diagrams/backtrace/backtrace.tex}}%
      \onslide*<3>{\providecommand\step{7}\input{diagrams/backtrace/backtrace.tex}}%
      \onslide*<4>{\providecommand\step{8}\input{diagrams/backtrace/backtrace.tex}}%
      \onslide*<5>{\providecommand\step{9}\input{diagrams/backtrace/backtrace.tex}}%
      \onslide*<6>{\providecommand\step{10}\input{diagrams/backtrace/backtrace.tex}}%
      \onslide*<7>{\providecommand\step{11}\input{diagrams/backtrace/backtrace.tex}}%
      \onslide*<8>{\providecommand\step{12}\input{diagrams/backtrace/backtrace.tex}}%
      \onslide*<9>{\providecommand\step{13}\input{diagrams/backtrace/backtrace.tex}}%
    \end{column}
  \end{columns}
\end{frame}

\begin{frame}{Backtraces}{Printing the backtrace}
  \begin{columns}[c]
    \begin{column}{0.45\textwidth}
      \minipage[c][0.66\textheight][s]{\columnwidth}
      \begin{itemize}
        \item<1-> The exception backtrace can now be printed to \funcarg{stdout} with \funcname{Printexc.print_backtrace}
        \item<2-> First the exception backtrace is retrieved into an OCaml array with \funcname{get_raw_backtrace }
        \item<3-> Which is implemented as the \funcname{caml_get_exception_raw_backtrace} C function in the runtime
        \item<4-> And finally printed with \funcname{print_exception_backtrace}
      \end{itemize}
      \vfill
      \mintocaml[firstline=28,lastline=34,highlightlines=33]{../src/backtrace/backtrace.ml}
      \endminipage
    \end{column}
    \begin{column}{0.55\textwidth}
      \onslide*<1,2>{
        \mintocaml[firstline=100,lastline=101]{../ocaml/stdlib/printexc.ml}
      }
      \only<1>{
        \listocaml[firstline=170,lastline=172]{../ocaml/stdlib/printexc.ml}{stdlib/printexc.ml:170}
      }
      \only<2>{
        \listocaml[firstline=170,lastline=172,highlightlines=172]{../ocaml/stdlib/printexc.ml}{stdlib/printexc.ml:170}
      }
      \only<3>{
        \mintc[firstline=154,lastline=156]{../ocaml/runtime/backtrace.c}
        \listc[firstline=171,lastline=192,highlightlines={184-187}]{../ocaml/runtime/backtrace.c}{runtime/backtrace.c:154}
      }
      \only<4>{
        \listocaml[firstline=155,lastline=165]{../ocaml/stdlib/printexc.ml}{stdlib/printexc.ml:55}
      }
    \end{column}
  \end{columns}
\end{frame}


\subsection{The multiple kinds of raise}
\frameSubsection{}{}

\begin{frame}{Backtraces}{When backtraces are not needed}
  \begin{columns}[c]
    \begin{column}{0.45\textwidth}
      \minipage[c][0.66\textheight][s]{\columnwidth}
      \begin{itemize}
        \item<1-> What if the backtrace for an exception is never needed?
        \item<2-> It's possible to raise the exception explicitly with \funcname{raise\_notrace} to make raising as cheap as possible
        \item<3-> An example of use in the \funcname{dynlink} library of the OCaml compiler
      \end{itemize}
      \vfill
      \onslide<3->{
      \listocaml[firstline=205,lastline=209,highlightlines=207]{../ocaml/otherlibs/dynlink/dynlink\_compilerlibs/misc.ml}{otherlibs/dynlink/dynlink\_compilerlibs/misc.ml:205}
      }
      \endminipage
    \end{column}
    \begin{column}{0.55\textwidth}
    \only<4>{
      \mintcmm[highlightlines=3]{../src/notrace/camlNotrace\_\_fun_347.cmm}
      \listcmm{../src/notrace/camlNotrace\_\_all\_somes\_327.cmm}{all\_somes}
    }
    \onslide*<5->{
      \listobjdump[highlightlines={6-9}]{../src/notrace/camlNotrace\_\_fun_347.objdump}{anonymous function from \funcname{all\_somes}}
    }
    \end{column}
  \end{columns}
\end{frame}

\begin{frame}{Backtraces}{Summary table of the multiple kinds of raise}
  \begin{itemize}
    \item When debugging information is enabled and if the backtrace of an exception isn't needed, it's possible to raise with \funcname{raise\_notrace} for performance
    \item When debugging information is \emph{not} enabled, the compiler optimises \funcname{raise} calls to inline assembly (same effect as using \funcname{raise\_notrace})
  \end{itemize}
  \bigskip
  \centering
  \begin{tabular}{| l | c | c | c | c |}
    \hline
    \parbox{10em}{Debug information \\ (\funcarg{-g} flag)}  & \multicolumn{2}{|c|}{Disabled}                                     & \multicolumn{2}{|c|}{Enabled} \\
    \hline
    Raise kind                                               & \funcname{raise} & \funcname{raise\_notrace}                       & \funcname{raise} & \funcname{raise\_notrace} \\
    \hline
    Compilation                                              & \parbox{8em}{inline assembly\\ (optimization)} & inline assembly   & \funcname{caml\_raise\_exn} & inline assembly \\
    \hline
  \end{tabular}
\end{frame}


%
% Takeaway #5
%
\subsection*{Takeaway \#5}
\frameSubsectionTakeaway{}
\againframe<5>{takeaway}
}%
      \onslide*<7>{\providecommand\step{11}%
% Backtraces
%
\subsection{One advantage of raising with debugging information}
\frameSubsection{}{}

\begin{frame}{Backtraces}{Debugging information \& source code}
  \begin{columns}[c]
    \begin{column}{0.5\textwidth}
      \begin{itemize}
        \item Raising an exception is fun and games, but how do we know where the exception originated? $\rightarrow$ With the backtrace associated with the exception!
        \item In order to get a backtrace from the point where an exception was raised, it's necessary to compile with debugging information enabled (\funcarg{-g} flag for the compiler)
        \item Same example as in the `Nested handlers' section
      \end{itemize}
    \end{column}
    \begin{column}{0.5\textwidth}
      \listocaml[firstline=9,lastline=34]{../src/backtrace/backtrace.ml}{backtrace/backtrace.ml}
    \end{column}
  \end{columns}
\end{frame}

\begin{frame}{Backtraces}{CMM and disassembly of \funcname{definitely\_raise}}
  \begin{columns}[c]
    \begin{column}{0.5\textwidth}
      \minipage[c][0.66\textheight][s]{\columnwidth}
      \begin{itemize}
        \item<1-> An effect of compiling with debugging information (\funcarg{-g}): the CMM no longer uses \funcname{raise\_notrace} to raise the exception but \funcname{raise} instead
        \item<2-> Disassembly shows that CMM \funcname{raise} is actually a call to \funcname{caml\_raise\_exn}, a function in assembler provided by the runtime
      \end{itemize}
      \vfill
      \mintocaml[firstline=9,lastline=13,highlightlines=12]{../src/backtrace/backtrace.ml}
      \endminipage
    \end{column}
    \begin{column}{0.5\textwidth}
      \onslide*<1>{
        \mintcmm[highlightlines=5]{../src/backtrace/camlBacktrace\_\_definitely\_raise\_347.cmm}
      }
      \onslide*<2>{
        \mintobjdump[firstline=2,highlightlines=14]{../src/backtrace/camlBacktrace\_\_definitely\_raise\_347.objdump}
      }
    \end{column}
  \end{columns}
\end{frame}

\begin{frame}{Backtraces}{Introducing \funcname{caml\_raise\_exn}}
  \begin{columns}[c]
    \begin{column}{0.5\textwidth}
      \minipage[c][0.66\textheight][s]{\columnwidth}
      \onslide*<1>{
        \begin{itemize}
          \item Raising the exception is performed using \funcname{caml\_raise\_exn} which is
            \begin{itemize}
              \item An assembly function provided by the runtime
              \item Architecture specific
            \end{itemize}
          \item Let's see what it does differently from \funcname{raise\_notrace}\dots
          \item We are about to raise \typename{ExnC} with \funcname{caml\_raise\_exn} from \funcname{definitely\_raise}
        \end{itemize}
      }
      \onslide*<2>{
        \mintobjdump[firstline=2,highlightlines=14]{../src/backtrace/camlBacktrace\_\_definitely\_raise\_347.objdump}
      }
      \vfill
      \mintocaml[firstline=9,lastline=13,highlightlines=12]{../src/backtrace/backtrace.ml}
      \endminipage
    \end{column}
    \begin{column}{0.5\textwidth}
      \centering
      \providecommand\step{1}\input{diagrams/backtrace/backtrace.tex}
    \end{column}
  \end{columns}
\end{frame}

\begin{frame}{Backtraces}{But before that, we need more fields from \typename{caml\_domain\_state}}
  \begin{columns}
    \begin{column}{0.5\textwidth}
      \begin{itemize}
        \item Remember \typename{caml\_domain\_state} the per-domain runtime state?
        \item Some other members are of interest to us now\dots
        \item \localname{backtrace\_active} is used to know if the runtime should collect backtraces
        \item \localname{backtrace\_active} can be set by
          \begin{itemize}
            \item The \funcarg{b} flag in the \funcname{OCAMLRUNPARAM} environment variable
            \item \mintinline{ocaml}{Printexc.record_backtrace : bool -> unit} \footnotemark
          \end{itemize}
      \end{itemize}
    \end{column}
    \begin{column}{0.5\textwidth}
      \begin{listing}
        \mintc[highlightlines={9-11}]{src/ocaml/domain\_state\_backtrace.h}
        \caption{runtime/caml/domain\_state.h}
      \end{listing}
    \end{column}
  \end{columns}
  \footnotetext[1]{https://v2.ocaml.org/api/Printexc.html}
\end{frame}

\newcommand\listCamlRaiseExn[1][]{
  \listasm[firstline=702,lastline=725,#1]{../ocaml/runtime/amd64.S}{runtime/amd64.S:702}}

\begin{frame}{Backtraces}{Walk through \funcname{caml\_raise\_exn}}
  \begin{columns}[T]
    \begin{column}{0.5\textwidth}
      \begin{itemize}
        \item<1-> First checks whether exceptions backtrace should be recorded
        \item<1-> By checking the \funcarg{domain\_state->backtrace\_active} value
        \item<2-> If not, acts as \funcname{raise\_notrace} with \funcname{RESTORE\_EXN\_HANDLER\_OCAML}
          \begin{itemize}
            \item Set \regname{RSP} to \localname{domain\_state->exn\_handler} value
            \item Restore \localname{domain\_state->exn\_handler} from trap
            \item Jump with \funcname{ret} (same effect as using \funcname{pop} and \funcname{jmp})
          \end{itemize}
        \item<3-> Otherwise, switches to the C stack to be able to execute C code
        \item<4-> Calls \funcname{caml\_stash\_backtrace}, a C function provided by the runtime\ignorespaces
          \onslide<6->{, with
            \begin{itemize}
              \item \funcarg{exn}: exception value
              \item \funcarg{pc}: code address of the raising point
              \item \funcarg{sp}: stack address of the raising function
              \item \funcarg{trapsp}: stack address of the next handler
            \end{itemize}
          }
      \end{itemize}
      \bigskip
      \mintocaml[firstline=9,lastline=13,highlightlines=12]{../src/backtrace/backtrace.ml}
    \end{column}
    \begin{column}{0.5\textwidth}
      \raggedleft
      \onslide*<1,3-4>{
        \mintc[firstline=190,lastline=192]{../ocaml/runtime/amd64.S}
        \mintc[firstline=234,lastline=237]{../ocaml/runtime/amd64.S}
      }
      \onslide*<2>{
        \mintc[firstline=190,lastline=192,highlightlines={191-192}]{../ocaml/runtime/amd64.S}
        \mintc[firstline=234,lastline=237,highlightlines={235-237}]{../ocaml/runtime/amd64.S}
      }
      \onslide*<1>{\listCamlRaiseExn[highlightlines=705]{}}%
      \onslide*<2>{\listCamlRaiseExn[highlightlines={707-708}]{}}%
      \onslide*<3>{\listCamlRaiseExn[highlightlines={712-714}]{}}%
      \onslide*<4>{\listCamlRaiseExn[highlightlines={715-720}]{}}%
      \onslide*<5>{\providecommand\step{1}\input{diagrams/backtrace/backtrace.tex}}%
      \onslide*<6>{\providecommand\step{2}\input{diagrams/backtrace/backtrace.tex}}%
    \end{column}
  \end{columns}
\end{frame}

\newcommand\listStashBacktrace[1][]{
  \listc[firstline=91,lastline=120,#1]{../ocaml/runtime/backtrace\_nat.c}{runtime/backtrace\_nat.c:91}}

\begin{frame}{Backtraces}{Introducing \funcname{caml\_stash\_backtrace}}
  \begin{columns}[c]
    \begin{column}{0.5\textwidth}
      \minipage[c][0.66\textheight][s]{\columnwidth}
      \begin{itemize}
        \item<1-> A C function provided by the runtime (specific to native code)
        \item<2-> It scans the stack, between the stack pointer at the raising point up to the next trap, in order to collect the names of the detected function frames
          \onslide*<2>{
            \begin{itemize}
              \item And more information, like line number, column number...
            \end{itemize}
          }
        \item<3-> It uses the same technique and information as the Garbage Collector (GC) uses to scan the values live on the stack
          \onslide*<4>{
            \begin{itemize}
              \item \typename{frame\_descr} are at the core of stack unwinding
            \end{itemize}
          }
      \end{itemize}
      \vfill
      \mintocaml[firstline=9,lastline=13,highlightlines=12]{../src/backtrace/backtrace.ml}
      \endminipage
    \end{column}
    \begin{column}{0.5\textwidth}
      \onslide*<1>{\listStashBacktrace[highlightlines=91]{}}
      \onslide*<2>{\listStashBacktrace[highlightlines={91,117-118}]{}}
      \onslide*<3->{\listStashBacktrace[highlightlines={94,106,109-110}]{}}
    \end{column}
  \end{columns}
\end{frame}

\newcommand\listFrameDescr[1][]{
  \listocaml[firstline=111,lastline=124,#1]{../ocaml/asmcomp/emitaux.ml}{asmcomp/emitaux.ml:111}}
\newcommand\listDebugInfo[1][]{
  \listocaml[firstline=38,lastline=49,#1]{../ocaml/lambda/debuginfo.mli}{lambda/debuginfo.mli:38}}

\begin{frame}{Backtraces}{\typename{frame\_descr} and \typename{frame\_debuginfo} in a nutshell}
  \begin{columns}[c]
    \begin{column}{0.5\textwidth}
      \begin{itemize}
        \item<1-> For every return address in an OCaml program, there is a associated \typename{frame\_descr}
        \item<2-> A \typename{frame\_descr} specifies
        \begin{itemize}
          \item<2-> The size of the function stack frame
          \item<3-> Where live values are on the stack (so that the GC can mark them as live and prevent from being swept)
        \end{itemize}
        \item<4-> When compiled with debug information, \typename{frame\_descr} will be linked to a \typename{frame\_debuginfo}
        \begin{itemize}
          \item<5-> Contains information about the position in the source code (file name, line and column number)
        \end{itemize}
      \end{itemize}
    \end{column}
    \begin{column}{0.5\textwidth}
        \onslide*<1>{\listFrameDescr[highlightlines={118-122}]{}}
        \onslide*<2>{\listFrameDescr[highlightlines={120}]{}}
        \onslide*<3>{\listFrameDescr[highlightlines={121}]{}}
        \onslide*<4->{\listFrameDescr[highlightlines={122}]{}}
        \medskip
        \onslide*<1-3>{\listDebugInfo{}}
        \onslide*<4>{\listDebugInfo[highlightlines={38,49}]{}}
        \onslide*<5>{\listDebugInfo[highlightlines={39-46}]{}}
    \end{column}
  \end{columns}
\end{frame}

\begin{frame}{Backtraces}{Scanning the stack with \typename{frame\_descr}}
  \begin{columns}[c]
    \begin{column}{0.5\textwidth}
      \begin{itemize}
        \item Given the return address of the code that raises the exception by calling \funcname{caml\_raise\_exn} (\funcarg{pc} argument of \funcname{caml\_stash\_backtrace})
        \item And the position of the stack pointer (\funcarg{sp} argument of \funcname{caml\_stash\_backtrace})
        \item The frame size given by \typename{frame\_descr}.\funcarg{frame\_size}, allows to find the end of the previous frame
        \item And the return address of the caller of this function
        \bigskip
        \onslide<3->{
        \item Note that traps (here the reddish \funcname{ExnA}, \funcname{ExnB} and \funcname{ExnC} blocks) belong to the frame that installed them, and so the frame size changes when traps are removed
        }
      \end{itemize}
    \end{column}
    \begin{column}{0.5\textwidth}
      \raggedleft
      \onslide*<1>{\providecommand\step{2}\input{diagrams/backtrace/backtrace.tex}}%
      \onslide*<2>{\providecommand\step{1}\input{diagrams/backtrace/scanstack.tex}}%
      \onslide*<3>{\providecommand\step{2}\input{diagrams/backtrace/scanstack.tex}}%
      \onslide*<4>{\providecommand\step{3}\input{diagrams/backtrace/scanstack.tex}}%
      \onslide*<5>{\providecommand\step{4}\input{diagrams/backtrace/scanstack.tex}}%
      \onslide*<6>{\providecommand\step{5}\input{diagrams/backtrace/scanstack.tex}}%
    \end{column}
  \end{columns}
\end{frame}

% Duplicate of previous-previous slide
\begin{frame}{Backtraces}{Continuing on \funcname{caml\_stash\_backtrace}}
  \begin{columns}[c]
    \begin{column}{0.5\textwidth}
      \minipage[c][0.75\textheight][s]{\columnwidth}
      \begin{itemize}
          \color{gray}
        \item A C function provided by the runtime (specific to native code)
        \item It scans the stack, between the stack pointer at the raising point up to the next trap, in order to collect the names of the detected function frames
        \item It uses the same technique and information as the Garbage Collector (GC) uses to scan the values live on the stack
      \end{itemize}
      \begin{itemize}
        \item For each function frame detected, store a pointer to the \typename{frame\_descr} in the \localname{domain\_state->backtrace\_buffer} array, effectively constructing the backtrace
      \end{itemize}
      \onslide<2->{
        \begin{itemize}
            \smallskip
          \item Constructed backtrace:
            \funcname{definitely\_raise},
            \onslide<3->{\funcname{probably\_raise}}
        \end{itemize}
      }
      \vfill
      \mintocaml[firstline=9,lastline=13,highlightlines=12]{../src/backtrace/backtrace.ml}
      \endminipage
    \end{column}
    \begin{column}{0.5\textwidth}
      \raggedleft
      \onslide*<1>{\listStashBacktrace[highlightlines={114-115}]{}}
      \onslide*<2>{\providecommand\step{3}\input{diagrams/backtrace/backtrace.tex}}%
      \onslide*<3>{\providecommand\step{4}\input{diagrams/backtrace/backtrace.tex}}%
    \end{column}
  \end{columns}
\end{frame}

\begin{frame}{Backtraces}{Back to \funcname{caml\_raise\_exn}}
  \begin{columns}[c]
    \begin{column}{0.5\textwidth}
      \minipage[c][0.66\textheight][s]{\columnwidth}
      \begin{itemize}\color{gray}
        \item Checks wheteher exceptions backtrace should be recorded, by checking the \funcarg{domain\_state->backtrace\_active} value
        \item Switches to the C stack to be able to execute C code
        \item Calls \funcname{caml\_stash\_backtrace}, to collect the backtrace up to the next trap
      \end{itemize}
      \onslide*<2->{
        \begin{itemize}
          \item<2-> Executes the exception handler in \funcname{probably\_raise} with \funcname{RESTORE\_EXN\_HANDLER\_OCAML}
            \begin{itemize}
              \item Set \regname{RSP} to \localname{domain\_state->exn\_handler} value
              \item Restore \localname{domain\_state->exn\_handler} from trap
              \item Jump to the handler code with \funcname{ret}
            \end{itemize}
        \end{itemize}
      }
      \vfill
      \onslide*<1>{\mintocaml[firstline=9,lastline=13,highlightlines=12]{../src/backtrace/backtrace.ml}}
      \onslide*<2->{\mintocaml[firstline=15,lastline=21,highlightlines={19-21}]{../src/backtrace/backtrace.ml}}
      \endminipage
    \end{column}
    \begin{column}{0.5\textwidth}
      \centering
      \onslide*<1>{
        \mintc[firstline=190,lastline=192]{../ocaml/runtime/amd64.S}
        \mintc[firstline=234,lastline=237]{../ocaml/runtime/amd64.S}
      }
      \onslide*<2>{
        \mintc[firstline=190,lastline=192,highlightlines={191-192}]{../ocaml/runtime/amd64.S}
        \mintc[firstline=234,lastline=237,highlightlines={235-237}]{../ocaml/runtime/amd64.S}
      }
      \onslide*<1>{\listCamlRaiseExn[highlightlines=720]{}}
      \onslide*<2>{\listCamlRaiseExn[highlightlines=722]{}}
      \onslide*<3->{\providecommand\step{5}\input{diagrams/backtrace/backtrace.tex}}
    \end{column}
  \end{columns}
\end{frame}

\begin{frame}{Backtraces}{\funcname{caml\_probably\_raise}'s handler doesn't catch \typename{ExnC}}
  \begin{columns}[c]
    \begin{column}{0.45\textwidth}
      \minipage[c][0.75\textheight][s]{\columnwidth}
      \begin{itemize}
        \item<1-> \funcname{match \dots with}\footnotemark pattern matching can also match exceptions
        \item<1-> The CMM of \funcname{probably\_raise} shows that this \funcname{match \dots with} is implemented with a pattern matching and a \funcname{try \dots with}
        \item<1-> But this one only matches on \typename{ExnA} exception
        \item<2-> Since the pattern matching fails to catch the exception, \funcname{reraise} is called with our current \typename{ExnC} exception
        \item<3-> Disassembly shows that \funcname{reraise} is actually a call to \funcname{caml\_reraise\_exn}, which is also an assembly function provided by the runtime
      \end{itemize}
      \vfill
      \mintocaml[firstline=15,lastline=21,highlightlines={18-21}]{../src/backtrace/backtrace.ml}
      \endminipage
    \end{column}
    \begin{column}{0.55\textwidth}
      \centering
      \onslide*<1>{\mintcmm[highlightlines={10-18}]{../src/backtrace/camlBacktrace\_\_probably\_raise\_351.cmm}}
      \onslide*<2>{\listcmm[highlightlines=18]{../src/backtrace/camlBacktrace\_\_probably\_raise_351.cmm}{CMM of probably\_raise}}
      \onslide*<3>{\mintobjdump[firstline=5,lastline=35,highlightlines=31]{../src/backtrace/camlBacktrace\_\_probably\_raise_351.objdump}}
    \end{column}
  \end{columns}
  \only<1>{\footnotetext[1]{https://blog.janestreet.com/pattern-matching-and-exception-handling-unite/}}
\end{frame}

\newcommand\listCamlReraiseExn[1][]{
  \listasm[firstline=727,lastline=734,#1]{../ocaml/runtime/amd64.S}{runtime/amd64.S:727}}

\begin{frame}{Backtraces}{Introducing \funcname{caml\_reraise\_exn}}
  \begin{columns}[c]
    \begin{column}{0.5\textwidth}
      \minipage[c][0.66\textheight][s]{\columnwidth}
      \begin{itemize}
        \item<1-> The exception have to be reraised to the next handler
        \item<2-> First, checks if the backtrace should be collected with \funcarg{domain\_state->backtrace\_active}
        \only<3>{
        \begin{itemize}
            \item If not, behaves like a \funcname{raise\_notrace} with \funcname{RESTORE\_EXN\_HANDLER\_OCAML}
        \end{itemize}
        }
        \item<4-> Jumps into \funcname{caml\_raise\_exn}
        \item<5-> Calls \funcname{caml\_stash\_backtrace} again, to scan the next part of the stack: from the trap just pop-ed to the next trap
      \end{itemize}
      \vfill
      \mintocaml[firstline=15,lastline=21,highlightlines={18-21}]{../src/backtrace/backtrace.ml}
      \endminipage
    \end{column}
    \begin{column}{0.5\textwidth}
      \raggedleft
      \onslide*<1>{\listCamlReraiseExn{}}
      \onslide*<2>{\listCamlReraiseExn[highlightlines=729]{}}
      \onslide*<3>{
        \mintc[firstline=190,lastline=192,highlightlines={191-192}]{../ocaml/runtime/amd64.S}
        \mintc[firstline=234,lastline=237,highlightlines={235-237}]{../ocaml/runtime/amd64.S}
        \medskip
        \listCamlReraiseExn[highlightlines=731]{}
      }
      \onslide*<4-5>{
        % TODO refactor with \listCamlReraiseExn
        \mintasm[firstline=727,lastline=734,highlightlines=730]{../ocaml/runtime/amd64.S}
        \medskip
      }
      \onslide*<4>{\listCamlRaiseExn[highlightlines=711]{}}
      \onslide*<5>{\listCamlRaiseExn[highlightlines=720]{}}
    \end{column}
  \end{columns}
\end{frame}

\begin{frame}{Backtraces}{Backtrace collection continues}
  \begin{columns}[c]
    \begin{column}{0.55\textwidth}
      \minipage[c][0.75\textheight][s]{\columnwidth}
      \begin{itemize}
        \item<1-> \funcname{caml\_stash\_backtrace} scans the older part of the stack, up to the next trap
        \item<6-> Controls jumps to the nested exception handler in \funcname{maybe\_raise}, which matches only on \typename{ExnB}
        \item<7-> \funcname{caml\_reraise\_exn} is called again, backtrace collection resumes with \funcname{caml\_stash\_backtrace}
      \end{itemize}
      \medskip
      \begin{itemize}
        \item<2-> Constructed backtrace:
        \begin{itemize}
          \item <2-> \funcname{definitely\_raise}
          \item <2-> \funcname{probably\_raise}
          \item <3-> \funcname{probably\_raise} (re-raise)
          \item <4-> \funcname{maybe\_raise}
          \item <5-> \funcname{main}
          \item <8-> \funcname{main} (re-raise)
        \end{itemize}
      \end{itemize}
      \vfill
      \onslide*<1-5>{\mintocaml[firstline=15,lastline=21]{../src/backtrace/backtrace.ml}}
      \onslide*<6>{\mintocaml[firstline=28,lastline=34,highlightlines=31]{../src/backtrace/backtrace.ml}}
      \onslide*<7->{\mintocaml[firstline=28,lastline=34,highlightlines=33]{../src/backtrace/backtrace.ml}}
      \endminipage
    \end{column}
    \begin{column}{0.45\textwidth}
      \raggedleft
      \onslide*<1>{\providecommand\step{6}\input{diagrams/backtrace/backtrace.tex}}%
      \onslide*<2>{\providecommand\step{6}\input{diagrams/backtrace/backtrace.tex}}%
      \onslide*<3>{\providecommand\step{7}\input{diagrams/backtrace/backtrace.tex}}%
      \onslide*<4>{\providecommand\step{8}\input{diagrams/backtrace/backtrace.tex}}%
      \onslide*<5>{\providecommand\step{9}\input{diagrams/backtrace/backtrace.tex}}%
      \onslide*<6>{\providecommand\step{10}\input{diagrams/backtrace/backtrace.tex}}%
      \onslide*<7>{\providecommand\step{11}\input{diagrams/backtrace/backtrace.tex}}%
      \onslide*<8>{\providecommand\step{12}\input{diagrams/backtrace/backtrace.tex}}%
      \onslide*<9>{\providecommand\step{13}\input{diagrams/backtrace/backtrace.tex}}%
    \end{column}
  \end{columns}
\end{frame}

\begin{frame}{Backtraces}{Printing the backtrace}
  \begin{columns}[c]
    \begin{column}{0.45\textwidth}
      \minipage[c][0.66\textheight][s]{\columnwidth}
      \begin{itemize}
        \item<1-> The exception backtrace can now be printed to \funcarg{stdout} with \funcname{Printexc.print_backtrace}
        \item<2-> First the exception backtrace is retrieved into an OCaml array with \funcname{get_raw_backtrace }
        \item<3-> Which is implemented as the \funcname{caml_get_exception_raw_backtrace} C function in the runtime
        \item<4-> And finally printed with \funcname{print_exception_backtrace}
      \end{itemize}
      \vfill
      \mintocaml[firstline=28,lastline=34,highlightlines=33]{../src/backtrace/backtrace.ml}
      \endminipage
    \end{column}
    \begin{column}{0.55\textwidth}
      \onslide*<1,2>{
        \mintocaml[firstline=100,lastline=101]{../ocaml/stdlib/printexc.ml}
      }
      \only<1>{
        \listocaml[firstline=170,lastline=172]{../ocaml/stdlib/printexc.ml}{stdlib/printexc.ml:170}
      }
      \only<2>{
        \listocaml[firstline=170,lastline=172,highlightlines=172]{../ocaml/stdlib/printexc.ml}{stdlib/printexc.ml:170}
      }
      \only<3>{
        \mintc[firstline=154,lastline=156]{../ocaml/runtime/backtrace.c}
        \listc[firstline=171,lastline=192,highlightlines={184-187}]{../ocaml/runtime/backtrace.c}{runtime/backtrace.c:154}
      }
      \only<4>{
        \listocaml[firstline=155,lastline=165]{../ocaml/stdlib/printexc.ml}{stdlib/printexc.ml:55}
      }
    \end{column}
  \end{columns}
\end{frame}


\subsection{The multiple kinds of raise}
\frameSubsection{}{}

\begin{frame}{Backtraces}{When backtraces are not needed}
  \begin{columns}[c]
    \begin{column}{0.45\textwidth}
      \minipage[c][0.66\textheight][s]{\columnwidth}
      \begin{itemize}
        \item<1-> What if the backtrace for an exception is never needed?
        \item<2-> It's possible to raise the exception explicitly with \funcname{raise\_notrace} to make raising as cheap as possible
        \item<3-> An example of use in the \funcname{dynlink} library of the OCaml compiler
      \end{itemize}
      \vfill
      \onslide<3->{
      \listocaml[firstline=205,lastline=209,highlightlines=207]{../ocaml/otherlibs/dynlink/dynlink\_compilerlibs/misc.ml}{otherlibs/dynlink/dynlink\_compilerlibs/misc.ml:205}
      }
      \endminipage
    \end{column}
    \begin{column}{0.55\textwidth}
    \only<4>{
      \mintcmm[highlightlines=3]{../src/notrace/camlNotrace\_\_fun_347.cmm}
      \listcmm{../src/notrace/camlNotrace\_\_all\_somes\_327.cmm}{all\_somes}
    }
    \onslide*<5->{
      \listobjdump[highlightlines={6-9}]{../src/notrace/camlNotrace\_\_fun_347.objdump}{anonymous function from \funcname{all\_somes}}
    }
    \end{column}
  \end{columns}
\end{frame}

\begin{frame}{Backtraces}{Summary table of the multiple kinds of raise}
  \begin{itemize}
    \item When debugging information is enabled and if the backtrace of an exception isn't needed, it's possible to raise with \funcname{raise\_notrace} for performance
    \item When debugging information is \emph{not} enabled, the compiler optimises \funcname{raise} calls to inline assembly (same effect as using \funcname{raise\_notrace})
  \end{itemize}
  \bigskip
  \centering
  \begin{tabular}{| l | c | c | c | c |}
    \hline
    \parbox{10em}{Debug information \\ (\funcarg{-g} flag)}  & \multicolumn{2}{|c|}{Disabled}                                     & \multicolumn{2}{|c|}{Enabled} \\
    \hline
    Raise kind                                               & \funcname{raise} & \funcname{raise\_notrace}                       & \funcname{raise} & \funcname{raise\_notrace} \\
    \hline
    Compilation                                              & \parbox{8em}{inline assembly\\ (optimization)} & inline assembly   & \funcname{caml\_raise\_exn} & inline assembly \\
    \hline
  \end{tabular}
\end{frame}


%
% Takeaway #5
%
\subsection*{Takeaway \#5}
\frameSubsectionTakeaway{}
\againframe<5>{takeaway}
}%
      \onslide*<8>{\providecommand\step{12}%
% Backtraces
%
\subsection{One advantage of raising with debugging information}
\frameSubsection{}{}

\begin{frame}{Backtraces}{Debugging information \& source code}
  \begin{columns}[c]
    \begin{column}{0.5\textwidth}
      \begin{itemize}
        \item Raising an exception is fun and games, but how do we know where the exception originated? $\rightarrow$ With the backtrace associated with the exception!
        \item In order to get a backtrace from the point where an exception was raised, it's necessary to compile with debugging information enabled (\funcarg{-g} flag for the compiler)
        \item Same example as in the `Nested handlers' section
      \end{itemize}
    \end{column}
    \begin{column}{0.5\textwidth}
      \listocaml[firstline=9,lastline=34]{../src/backtrace/backtrace.ml}{backtrace/backtrace.ml}
    \end{column}
  \end{columns}
\end{frame}

\begin{frame}{Backtraces}{CMM and disassembly of \funcname{definitely\_raise}}
  \begin{columns}[c]
    \begin{column}{0.5\textwidth}
      \minipage[c][0.66\textheight][s]{\columnwidth}
      \begin{itemize}
        \item<1-> An effect of compiling with debugging information (\funcarg{-g}): the CMM no longer uses \funcname{raise\_notrace} to raise the exception but \funcname{raise} instead
        \item<2-> Disassembly shows that CMM \funcname{raise} is actually a call to \funcname{caml\_raise\_exn}, a function in assembler provided by the runtime
      \end{itemize}
      \vfill
      \mintocaml[firstline=9,lastline=13,highlightlines=12]{../src/backtrace/backtrace.ml}
      \endminipage
    \end{column}
    \begin{column}{0.5\textwidth}
      \onslide*<1>{
        \mintcmm[highlightlines=5]{../src/backtrace/camlBacktrace\_\_definitely\_raise\_347.cmm}
      }
      \onslide*<2>{
        \mintobjdump[firstline=2,highlightlines=14]{../src/backtrace/camlBacktrace\_\_definitely\_raise\_347.objdump}
      }
    \end{column}
  \end{columns}
\end{frame}

\begin{frame}{Backtraces}{Introducing \funcname{caml\_raise\_exn}}
  \begin{columns}[c]
    \begin{column}{0.5\textwidth}
      \minipage[c][0.66\textheight][s]{\columnwidth}
      \onslide*<1>{
        \begin{itemize}
          \item Raising the exception is performed using \funcname{caml\_raise\_exn} which is
            \begin{itemize}
              \item An assembly function provided by the runtime
              \item Architecture specific
            \end{itemize}
          \item Let's see what it does differently from \funcname{raise\_notrace}\dots
          \item We are about to raise \typename{ExnC} with \funcname{caml\_raise\_exn} from \funcname{definitely\_raise}
        \end{itemize}
      }
      \onslide*<2>{
        \mintobjdump[firstline=2,highlightlines=14]{../src/backtrace/camlBacktrace\_\_definitely\_raise\_347.objdump}
      }
      \vfill
      \mintocaml[firstline=9,lastline=13,highlightlines=12]{../src/backtrace/backtrace.ml}
      \endminipage
    \end{column}
    \begin{column}{0.5\textwidth}
      \centering
      \providecommand\step{1}\input{diagrams/backtrace/backtrace.tex}
    \end{column}
  \end{columns}
\end{frame}

\begin{frame}{Backtraces}{But before that, we need more fields from \typename{caml\_domain\_state}}
  \begin{columns}
    \begin{column}{0.5\textwidth}
      \begin{itemize}
        \item Remember \typename{caml\_domain\_state} the per-domain runtime state?
        \item Some other members are of interest to us now\dots
        \item \localname{backtrace\_active} is used to know if the runtime should collect backtraces
        \item \localname{backtrace\_active} can be set by
          \begin{itemize}
            \item The \funcarg{b} flag in the \funcname{OCAMLRUNPARAM} environment variable
            \item \mintinline{ocaml}{Printexc.record_backtrace : bool -> unit} \footnotemark
          \end{itemize}
      \end{itemize}
    \end{column}
    \begin{column}{0.5\textwidth}
      \begin{listing}
        \mintc[highlightlines={9-11}]{src/ocaml/domain\_state\_backtrace.h}
        \caption{runtime/caml/domain\_state.h}
      \end{listing}
    \end{column}
  \end{columns}
  \footnotetext[1]{https://v2.ocaml.org/api/Printexc.html}
\end{frame}

\newcommand\listCamlRaiseExn[1][]{
  \listasm[firstline=702,lastline=725,#1]{../ocaml/runtime/amd64.S}{runtime/amd64.S:702}}

\begin{frame}{Backtraces}{Walk through \funcname{caml\_raise\_exn}}
  \begin{columns}[T]
    \begin{column}{0.5\textwidth}
      \begin{itemize}
        \item<1-> First checks whether exceptions backtrace should be recorded
        \item<1-> By checking the \funcarg{domain\_state->backtrace\_active} value
        \item<2-> If not, acts as \funcname{raise\_notrace} with \funcname{RESTORE\_EXN\_HANDLER\_OCAML}
          \begin{itemize}
            \item Set \regname{RSP} to \localname{domain\_state->exn\_handler} value
            \item Restore \localname{domain\_state->exn\_handler} from trap
            \item Jump with \funcname{ret} (same effect as using \funcname{pop} and \funcname{jmp})
          \end{itemize}
        \item<3-> Otherwise, switches to the C stack to be able to execute C code
        \item<4-> Calls \funcname{caml\_stash\_backtrace}, a C function provided by the runtime\ignorespaces
          \onslide<6->{, with
            \begin{itemize}
              \item \funcarg{exn}: exception value
              \item \funcarg{pc}: code address of the raising point
              \item \funcarg{sp}: stack address of the raising function
              \item \funcarg{trapsp}: stack address of the next handler
            \end{itemize}
          }
      \end{itemize}
      \bigskip
      \mintocaml[firstline=9,lastline=13,highlightlines=12]{../src/backtrace/backtrace.ml}
    \end{column}
    \begin{column}{0.5\textwidth}
      \raggedleft
      \onslide*<1,3-4>{
        \mintc[firstline=190,lastline=192]{../ocaml/runtime/amd64.S}
        \mintc[firstline=234,lastline=237]{../ocaml/runtime/amd64.S}
      }
      \onslide*<2>{
        \mintc[firstline=190,lastline=192,highlightlines={191-192}]{../ocaml/runtime/amd64.S}
        \mintc[firstline=234,lastline=237,highlightlines={235-237}]{../ocaml/runtime/amd64.S}
      }
      \onslide*<1>{\listCamlRaiseExn[highlightlines=705]{}}%
      \onslide*<2>{\listCamlRaiseExn[highlightlines={707-708}]{}}%
      \onslide*<3>{\listCamlRaiseExn[highlightlines={712-714}]{}}%
      \onslide*<4>{\listCamlRaiseExn[highlightlines={715-720}]{}}%
      \onslide*<5>{\providecommand\step{1}\input{diagrams/backtrace/backtrace.tex}}%
      \onslide*<6>{\providecommand\step{2}\input{diagrams/backtrace/backtrace.tex}}%
    \end{column}
  \end{columns}
\end{frame}

\newcommand\listStashBacktrace[1][]{
  \listc[firstline=91,lastline=120,#1]{../ocaml/runtime/backtrace\_nat.c}{runtime/backtrace\_nat.c:91}}

\begin{frame}{Backtraces}{Introducing \funcname{caml\_stash\_backtrace}}
  \begin{columns}[c]
    \begin{column}{0.5\textwidth}
      \minipage[c][0.66\textheight][s]{\columnwidth}
      \begin{itemize}
        \item<1-> A C function provided by the runtime (specific to native code)
        \item<2-> It scans the stack, between the stack pointer at the raising point up to the next trap, in order to collect the names of the detected function frames
          \onslide*<2>{
            \begin{itemize}
              \item And more information, like line number, column number...
            \end{itemize}
          }
        \item<3-> It uses the same technique and information as the Garbage Collector (GC) uses to scan the values live on the stack
          \onslide*<4>{
            \begin{itemize}
              \item \typename{frame\_descr} are at the core of stack unwinding
            \end{itemize}
          }
      \end{itemize}
      \vfill
      \mintocaml[firstline=9,lastline=13,highlightlines=12]{../src/backtrace/backtrace.ml}
      \endminipage
    \end{column}
    \begin{column}{0.5\textwidth}
      \onslide*<1>{\listStashBacktrace[highlightlines=91]{}}
      \onslide*<2>{\listStashBacktrace[highlightlines={91,117-118}]{}}
      \onslide*<3->{\listStashBacktrace[highlightlines={94,106,109-110}]{}}
    \end{column}
  \end{columns}
\end{frame}

\newcommand\listFrameDescr[1][]{
  \listocaml[firstline=111,lastline=124,#1]{../ocaml/asmcomp/emitaux.ml}{asmcomp/emitaux.ml:111}}
\newcommand\listDebugInfo[1][]{
  \listocaml[firstline=38,lastline=49,#1]{../ocaml/lambda/debuginfo.mli}{lambda/debuginfo.mli:38}}

\begin{frame}{Backtraces}{\typename{frame\_descr} and \typename{frame\_debuginfo} in a nutshell}
  \begin{columns}[c]
    \begin{column}{0.5\textwidth}
      \begin{itemize}
        \item<1-> For every return address in an OCaml program, there is a associated \typename{frame\_descr}
        \item<2-> A \typename{frame\_descr} specifies
        \begin{itemize}
          \item<2-> The size of the function stack frame
          \item<3-> Where live values are on the stack (so that the GC can mark them as live and prevent from being swept)
        \end{itemize}
        \item<4-> When compiled with debug information, \typename{frame\_descr} will be linked to a \typename{frame\_debuginfo}
        \begin{itemize}
          \item<5-> Contains information about the position in the source code (file name, line and column number)
        \end{itemize}
      \end{itemize}
    \end{column}
    \begin{column}{0.5\textwidth}
        \onslide*<1>{\listFrameDescr[highlightlines={118-122}]{}}
        \onslide*<2>{\listFrameDescr[highlightlines={120}]{}}
        \onslide*<3>{\listFrameDescr[highlightlines={121}]{}}
        \onslide*<4->{\listFrameDescr[highlightlines={122}]{}}
        \medskip
        \onslide*<1-3>{\listDebugInfo{}}
        \onslide*<4>{\listDebugInfo[highlightlines={38,49}]{}}
        \onslide*<5>{\listDebugInfo[highlightlines={39-46}]{}}
    \end{column}
  \end{columns}
\end{frame}

\begin{frame}{Backtraces}{Scanning the stack with \typename{frame\_descr}}
  \begin{columns}[c]
    \begin{column}{0.5\textwidth}
      \begin{itemize}
        \item Given the return address of the code that raises the exception by calling \funcname{caml\_raise\_exn} (\funcarg{pc} argument of \funcname{caml\_stash\_backtrace})
        \item And the position of the stack pointer (\funcarg{sp} argument of \funcname{caml\_stash\_backtrace})
        \item The frame size given by \typename{frame\_descr}.\funcarg{frame\_size}, allows to find the end of the previous frame
        \item And the return address of the caller of this function
        \bigskip
        \onslide<3->{
        \item Note that traps (here the reddish \funcname{ExnA}, \funcname{ExnB} and \funcname{ExnC} blocks) belong to the frame that installed them, and so the frame size changes when traps are removed
        }
      \end{itemize}
    \end{column}
    \begin{column}{0.5\textwidth}
      \raggedleft
      \onslide*<1>{\providecommand\step{2}\input{diagrams/backtrace/backtrace.tex}}%
      \onslide*<2>{\providecommand\step{1}\input{diagrams/backtrace/scanstack.tex}}%
      \onslide*<3>{\providecommand\step{2}\input{diagrams/backtrace/scanstack.tex}}%
      \onslide*<4>{\providecommand\step{3}\input{diagrams/backtrace/scanstack.tex}}%
      \onslide*<5>{\providecommand\step{4}\input{diagrams/backtrace/scanstack.tex}}%
      \onslide*<6>{\providecommand\step{5}\input{diagrams/backtrace/scanstack.tex}}%
    \end{column}
  \end{columns}
\end{frame}

% Duplicate of previous-previous slide
\begin{frame}{Backtraces}{Continuing on \funcname{caml\_stash\_backtrace}}
  \begin{columns}[c]
    \begin{column}{0.5\textwidth}
      \minipage[c][0.75\textheight][s]{\columnwidth}
      \begin{itemize}
          \color{gray}
        \item A C function provided by the runtime (specific to native code)
        \item It scans the stack, between the stack pointer at the raising point up to the next trap, in order to collect the names of the detected function frames
        \item It uses the same technique and information as the Garbage Collector (GC) uses to scan the values live on the stack
      \end{itemize}
      \begin{itemize}
        \item For each function frame detected, store a pointer to the \typename{frame\_descr} in the \localname{domain\_state->backtrace\_buffer} array, effectively constructing the backtrace
      \end{itemize}
      \onslide<2->{
        \begin{itemize}
            \smallskip
          \item Constructed backtrace:
            \funcname{definitely\_raise},
            \onslide<3->{\funcname{probably\_raise}}
        \end{itemize}
      }
      \vfill
      \mintocaml[firstline=9,lastline=13,highlightlines=12]{../src/backtrace/backtrace.ml}
      \endminipage
    \end{column}
    \begin{column}{0.5\textwidth}
      \raggedleft
      \onslide*<1>{\listStashBacktrace[highlightlines={114-115}]{}}
      \onslide*<2>{\providecommand\step{3}\input{diagrams/backtrace/backtrace.tex}}%
      \onslide*<3>{\providecommand\step{4}\input{diagrams/backtrace/backtrace.tex}}%
    \end{column}
  \end{columns}
\end{frame}

\begin{frame}{Backtraces}{Back to \funcname{caml\_raise\_exn}}
  \begin{columns}[c]
    \begin{column}{0.5\textwidth}
      \minipage[c][0.66\textheight][s]{\columnwidth}
      \begin{itemize}\color{gray}
        \item Checks wheteher exceptions backtrace should be recorded, by checking the \funcarg{domain\_state->backtrace\_active} value
        \item Switches to the C stack to be able to execute C code
        \item Calls \funcname{caml\_stash\_backtrace}, to collect the backtrace up to the next trap
      \end{itemize}
      \onslide*<2->{
        \begin{itemize}
          \item<2-> Executes the exception handler in \funcname{probably\_raise} with \funcname{RESTORE\_EXN\_HANDLER\_OCAML}
            \begin{itemize}
              \item Set \regname{RSP} to \localname{domain\_state->exn\_handler} value
              \item Restore \localname{domain\_state->exn\_handler} from trap
              \item Jump to the handler code with \funcname{ret}
            \end{itemize}
        \end{itemize}
      }
      \vfill
      \onslide*<1>{\mintocaml[firstline=9,lastline=13,highlightlines=12]{../src/backtrace/backtrace.ml}}
      \onslide*<2->{\mintocaml[firstline=15,lastline=21,highlightlines={19-21}]{../src/backtrace/backtrace.ml}}
      \endminipage
    \end{column}
    \begin{column}{0.5\textwidth}
      \centering
      \onslide*<1>{
        \mintc[firstline=190,lastline=192]{../ocaml/runtime/amd64.S}
        \mintc[firstline=234,lastline=237]{../ocaml/runtime/amd64.S}
      }
      \onslide*<2>{
        \mintc[firstline=190,lastline=192,highlightlines={191-192}]{../ocaml/runtime/amd64.S}
        \mintc[firstline=234,lastline=237,highlightlines={235-237}]{../ocaml/runtime/amd64.S}
      }
      \onslide*<1>{\listCamlRaiseExn[highlightlines=720]{}}
      \onslide*<2>{\listCamlRaiseExn[highlightlines=722]{}}
      \onslide*<3->{\providecommand\step{5}\input{diagrams/backtrace/backtrace.tex}}
    \end{column}
  \end{columns}
\end{frame}

\begin{frame}{Backtraces}{\funcname{caml\_probably\_raise}'s handler doesn't catch \typename{ExnC}}
  \begin{columns}[c]
    \begin{column}{0.45\textwidth}
      \minipage[c][0.75\textheight][s]{\columnwidth}
      \begin{itemize}
        \item<1-> \funcname{match \dots with}\footnotemark pattern matching can also match exceptions
        \item<1-> The CMM of \funcname{probably\_raise} shows that this \funcname{match \dots with} is implemented with a pattern matching and a \funcname{try \dots with}
        \item<1-> But this one only matches on \typename{ExnA} exception
        \item<2-> Since the pattern matching fails to catch the exception, \funcname{reraise} is called with our current \typename{ExnC} exception
        \item<3-> Disassembly shows that \funcname{reraise} is actually a call to \funcname{caml\_reraise\_exn}, which is also an assembly function provided by the runtime
      \end{itemize}
      \vfill
      \mintocaml[firstline=15,lastline=21,highlightlines={18-21}]{../src/backtrace/backtrace.ml}
      \endminipage
    \end{column}
    \begin{column}{0.55\textwidth}
      \centering
      \onslide*<1>{\mintcmm[highlightlines={10-18}]{../src/backtrace/camlBacktrace\_\_probably\_raise\_351.cmm}}
      \onslide*<2>{\listcmm[highlightlines=18]{../src/backtrace/camlBacktrace\_\_probably\_raise_351.cmm}{CMM of probably\_raise}}
      \onslide*<3>{\mintobjdump[firstline=5,lastline=35,highlightlines=31]{../src/backtrace/camlBacktrace\_\_probably\_raise_351.objdump}}
    \end{column}
  \end{columns}
  \only<1>{\footnotetext[1]{https://blog.janestreet.com/pattern-matching-and-exception-handling-unite/}}
\end{frame}

\newcommand\listCamlReraiseExn[1][]{
  \listasm[firstline=727,lastline=734,#1]{../ocaml/runtime/amd64.S}{runtime/amd64.S:727}}

\begin{frame}{Backtraces}{Introducing \funcname{caml\_reraise\_exn}}
  \begin{columns}[c]
    \begin{column}{0.5\textwidth}
      \minipage[c][0.66\textheight][s]{\columnwidth}
      \begin{itemize}
        \item<1-> The exception have to be reraised to the next handler
        \item<2-> First, checks if the backtrace should be collected with \funcarg{domain\_state->backtrace\_active}
        \only<3>{
        \begin{itemize}
            \item If not, behaves like a \funcname{raise\_notrace} with \funcname{RESTORE\_EXN\_HANDLER\_OCAML}
        \end{itemize}
        }
        \item<4-> Jumps into \funcname{caml\_raise\_exn}
        \item<5-> Calls \funcname{caml\_stash\_backtrace} again, to scan the next part of the stack: from the trap just pop-ed to the next trap
      \end{itemize}
      \vfill
      \mintocaml[firstline=15,lastline=21,highlightlines={18-21}]{../src/backtrace/backtrace.ml}
      \endminipage
    \end{column}
    \begin{column}{0.5\textwidth}
      \raggedleft
      \onslide*<1>{\listCamlReraiseExn{}}
      \onslide*<2>{\listCamlReraiseExn[highlightlines=729]{}}
      \onslide*<3>{
        \mintc[firstline=190,lastline=192,highlightlines={191-192}]{../ocaml/runtime/amd64.S}
        \mintc[firstline=234,lastline=237,highlightlines={235-237}]{../ocaml/runtime/amd64.S}
        \medskip
        \listCamlReraiseExn[highlightlines=731]{}
      }
      \onslide*<4-5>{
        % TODO refactor with \listCamlReraiseExn
        \mintasm[firstline=727,lastline=734,highlightlines=730]{../ocaml/runtime/amd64.S}
        \medskip
      }
      \onslide*<4>{\listCamlRaiseExn[highlightlines=711]{}}
      \onslide*<5>{\listCamlRaiseExn[highlightlines=720]{}}
    \end{column}
  \end{columns}
\end{frame}

\begin{frame}{Backtraces}{Backtrace collection continues}
  \begin{columns}[c]
    \begin{column}{0.55\textwidth}
      \minipage[c][0.75\textheight][s]{\columnwidth}
      \begin{itemize}
        \item<1-> \funcname{caml\_stash\_backtrace} scans the older part of the stack, up to the next trap
        \item<6-> Controls jumps to the nested exception handler in \funcname{maybe\_raise}, which matches only on \typename{ExnB}
        \item<7-> \funcname{caml\_reraise\_exn} is called again, backtrace collection resumes with \funcname{caml\_stash\_backtrace}
      \end{itemize}
      \medskip
      \begin{itemize}
        \item<2-> Constructed backtrace:
        \begin{itemize}
          \item <2-> \funcname{definitely\_raise}
          \item <2-> \funcname{probably\_raise}
          \item <3-> \funcname{probably\_raise} (re-raise)
          \item <4-> \funcname{maybe\_raise}
          \item <5-> \funcname{main}
          \item <8-> \funcname{main} (re-raise)
        \end{itemize}
      \end{itemize}
      \vfill
      \onslide*<1-5>{\mintocaml[firstline=15,lastline=21]{../src/backtrace/backtrace.ml}}
      \onslide*<6>{\mintocaml[firstline=28,lastline=34,highlightlines=31]{../src/backtrace/backtrace.ml}}
      \onslide*<7->{\mintocaml[firstline=28,lastline=34,highlightlines=33]{../src/backtrace/backtrace.ml}}
      \endminipage
    \end{column}
    \begin{column}{0.45\textwidth}
      \raggedleft
      \onslide*<1>{\providecommand\step{6}\input{diagrams/backtrace/backtrace.tex}}%
      \onslide*<2>{\providecommand\step{6}\input{diagrams/backtrace/backtrace.tex}}%
      \onslide*<3>{\providecommand\step{7}\input{diagrams/backtrace/backtrace.tex}}%
      \onslide*<4>{\providecommand\step{8}\input{diagrams/backtrace/backtrace.tex}}%
      \onslide*<5>{\providecommand\step{9}\input{diagrams/backtrace/backtrace.tex}}%
      \onslide*<6>{\providecommand\step{10}\input{diagrams/backtrace/backtrace.tex}}%
      \onslide*<7>{\providecommand\step{11}\input{diagrams/backtrace/backtrace.tex}}%
      \onslide*<8>{\providecommand\step{12}\input{diagrams/backtrace/backtrace.tex}}%
      \onslide*<9>{\providecommand\step{13}\input{diagrams/backtrace/backtrace.tex}}%
    \end{column}
  \end{columns}
\end{frame}

\begin{frame}{Backtraces}{Printing the backtrace}
  \begin{columns}[c]
    \begin{column}{0.45\textwidth}
      \minipage[c][0.66\textheight][s]{\columnwidth}
      \begin{itemize}
        \item<1-> The exception backtrace can now be printed to \funcarg{stdout} with \funcname{Printexc.print_backtrace}
        \item<2-> First the exception backtrace is retrieved into an OCaml array with \funcname{get_raw_backtrace }
        \item<3-> Which is implemented as the \funcname{caml_get_exception_raw_backtrace} C function in the runtime
        \item<4-> And finally printed with \funcname{print_exception_backtrace}
      \end{itemize}
      \vfill
      \mintocaml[firstline=28,lastline=34,highlightlines=33]{../src/backtrace/backtrace.ml}
      \endminipage
    \end{column}
    \begin{column}{0.55\textwidth}
      \onslide*<1,2>{
        \mintocaml[firstline=100,lastline=101]{../ocaml/stdlib/printexc.ml}
      }
      \only<1>{
        \listocaml[firstline=170,lastline=172]{../ocaml/stdlib/printexc.ml}{stdlib/printexc.ml:170}
      }
      \only<2>{
        \listocaml[firstline=170,lastline=172,highlightlines=172]{../ocaml/stdlib/printexc.ml}{stdlib/printexc.ml:170}
      }
      \only<3>{
        \mintc[firstline=154,lastline=156]{../ocaml/runtime/backtrace.c}
        \listc[firstline=171,lastline=192,highlightlines={184-187}]{../ocaml/runtime/backtrace.c}{runtime/backtrace.c:154}
      }
      \only<4>{
        \listocaml[firstline=155,lastline=165]{../ocaml/stdlib/printexc.ml}{stdlib/printexc.ml:55}
      }
    \end{column}
  \end{columns}
\end{frame}


\subsection{The multiple kinds of raise}
\frameSubsection{}{}

\begin{frame}{Backtraces}{When backtraces are not needed}
  \begin{columns}[c]
    \begin{column}{0.45\textwidth}
      \minipage[c][0.66\textheight][s]{\columnwidth}
      \begin{itemize}
        \item<1-> What if the backtrace for an exception is never needed?
        \item<2-> It's possible to raise the exception explicitly with \funcname{raise\_notrace} to make raising as cheap as possible
        \item<3-> An example of use in the \funcname{dynlink} library of the OCaml compiler
      \end{itemize}
      \vfill
      \onslide<3->{
      \listocaml[firstline=205,lastline=209,highlightlines=207]{../ocaml/otherlibs/dynlink/dynlink\_compilerlibs/misc.ml}{otherlibs/dynlink/dynlink\_compilerlibs/misc.ml:205}
      }
      \endminipage
    \end{column}
    \begin{column}{0.55\textwidth}
    \only<4>{
      \mintcmm[highlightlines=3]{../src/notrace/camlNotrace\_\_fun_347.cmm}
      \listcmm{../src/notrace/camlNotrace\_\_all\_somes\_327.cmm}{all\_somes}
    }
    \onslide*<5->{
      \listobjdump[highlightlines={6-9}]{../src/notrace/camlNotrace\_\_fun_347.objdump}{anonymous function from \funcname{all\_somes}}
    }
    \end{column}
  \end{columns}
\end{frame}

\begin{frame}{Backtraces}{Summary table of the multiple kinds of raise}
  \begin{itemize}
    \item When debugging information is enabled and if the backtrace of an exception isn't needed, it's possible to raise with \funcname{raise\_notrace} for performance
    \item When debugging information is \emph{not} enabled, the compiler optimises \funcname{raise} calls to inline assembly (same effect as using \funcname{raise\_notrace})
  \end{itemize}
  \bigskip
  \centering
  \begin{tabular}{| l | c | c | c | c |}
    \hline
    \parbox{10em}{Debug information \\ (\funcarg{-g} flag)}  & \multicolumn{2}{|c|}{Disabled}                                     & \multicolumn{2}{|c|}{Enabled} \\
    \hline
    Raise kind                                               & \funcname{raise} & \funcname{raise\_notrace}                       & \funcname{raise} & \funcname{raise\_notrace} \\
    \hline
    Compilation                                              & \parbox{8em}{inline assembly\\ (optimization)} & inline assembly   & \funcname{caml\_raise\_exn} & inline assembly \\
    \hline
  \end{tabular}
\end{frame}


%
% Takeaway #5
%
\subsection*{Takeaway \#5}
\frameSubsectionTakeaway{}
\againframe<5>{takeaway}
}%
      \onslide*<9>{\providecommand\step{13}%
% Backtraces
%
\subsection{One advantage of raising with debugging information}
\frameSubsection{}{}

\begin{frame}{Backtraces}{Debugging information \& source code}
  \begin{columns}[c]
    \begin{column}{0.5\textwidth}
      \begin{itemize}
        \item Raising an exception is fun and games, but how do we know where the exception originated? $\rightarrow$ With the backtrace associated with the exception!
        \item In order to get a backtrace from the point where an exception was raised, it's necessary to compile with debugging information enabled (\funcarg{-g} flag for the compiler)
        \item Same example as in the `Nested handlers' section
      \end{itemize}
    \end{column}
    \begin{column}{0.5\textwidth}
      \listocaml[firstline=9,lastline=34]{../src/backtrace/backtrace.ml}{backtrace/backtrace.ml}
    \end{column}
  \end{columns}
\end{frame}

\begin{frame}{Backtraces}{CMM and disassembly of \funcname{definitely\_raise}}
  \begin{columns}[c]
    \begin{column}{0.5\textwidth}
      \minipage[c][0.66\textheight][s]{\columnwidth}
      \begin{itemize}
        \item<1-> An effect of compiling with debugging information (\funcarg{-g}): the CMM no longer uses \funcname{raise\_notrace} to raise the exception but \funcname{raise} instead
        \item<2-> Disassembly shows that CMM \funcname{raise} is actually a call to \funcname{caml\_raise\_exn}, a function in assembler provided by the runtime
      \end{itemize}
      \vfill
      \mintocaml[firstline=9,lastline=13,highlightlines=12]{../src/backtrace/backtrace.ml}
      \endminipage
    \end{column}
    \begin{column}{0.5\textwidth}
      \onslide*<1>{
        \mintcmm[highlightlines=5]{../src/backtrace/camlBacktrace\_\_definitely\_raise\_347.cmm}
      }
      \onslide*<2>{
        \mintobjdump[firstline=2,highlightlines=14]{../src/backtrace/camlBacktrace\_\_definitely\_raise\_347.objdump}
      }
    \end{column}
  \end{columns}
\end{frame}

\begin{frame}{Backtraces}{Introducing \funcname{caml\_raise\_exn}}
  \begin{columns}[c]
    \begin{column}{0.5\textwidth}
      \minipage[c][0.66\textheight][s]{\columnwidth}
      \onslide*<1>{
        \begin{itemize}
          \item Raising the exception is performed using \funcname{caml\_raise\_exn} which is
            \begin{itemize}
              \item An assembly function provided by the runtime
              \item Architecture specific
            \end{itemize}
          \item Let's see what it does differently from \funcname{raise\_notrace}\dots
          \item We are about to raise \typename{ExnC} with \funcname{caml\_raise\_exn} from \funcname{definitely\_raise}
        \end{itemize}
      }
      \onslide*<2>{
        \mintobjdump[firstline=2,highlightlines=14]{../src/backtrace/camlBacktrace\_\_definitely\_raise\_347.objdump}
      }
      \vfill
      \mintocaml[firstline=9,lastline=13,highlightlines=12]{../src/backtrace/backtrace.ml}
      \endminipage
    \end{column}
    \begin{column}{0.5\textwidth}
      \centering
      \providecommand\step{1}\input{diagrams/backtrace/backtrace.tex}
    \end{column}
  \end{columns}
\end{frame}

\begin{frame}{Backtraces}{But before that, we need more fields from \typename{caml\_domain\_state}}
  \begin{columns}
    \begin{column}{0.5\textwidth}
      \begin{itemize}
        \item Remember \typename{caml\_domain\_state} the per-domain runtime state?
        \item Some other members are of interest to us now\dots
        \item \localname{backtrace\_active} is used to know if the runtime should collect backtraces
        \item \localname{backtrace\_active} can be set by
          \begin{itemize}
            \item The \funcarg{b} flag in the \funcname{OCAMLRUNPARAM} environment variable
            \item \mintinline{ocaml}{Printexc.record_backtrace : bool -> unit} \footnotemark
          \end{itemize}
      \end{itemize}
    \end{column}
    \begin{column}{0.5\textwidth}
      \begin{listing}
        \mintc[highlightlines={9-11}]{src/ocaml/domain\_state\_backtrace.h}
        \caption{runtime/caml/domain\_state.h}
      \end{listing}
    \end{column}
  \end{columns}
  \footnotetext[1]{https://v2.ocaml.org/api/Printexc.html}
\end{frame}

\newcommand\listCamlRaiseExn[1][]{
  \listasm[firstline=702,lastline=725,#1]{../ocaml/runtime/amd64.S}{runtime/amd64.S:702}}

\begin{frame}{Backtraces}{Walk through \funcname{caml\_raise\_exn}}
  \begin{columns}[T]
    \begin{column}{0.5\textwidth}
      \begin{itemize}
        \item<1-> First checks whether exceptions backtrace should be recorded
        \item<1-> By checking the \funcarg{domain\_state->backtrace\_active} value
        \item<2-> If not, acts as \funcname{raise\_notrace} with \funcname{RESTORE\_EXN\_HANDLER\_OCAML}
          \begin{itemize}
            \item Set \regname{RSP} to \localname{domain\_state->exn\_handler} value
            \item Restore \localname{domain\_state->exn\_handler} from trap
            \item Jump with \funcname{ret} (same effect as using \funcname{pop} and \funcname{jmp})
          \end{itemize}
        \item<3-> Otherwise, switches to the C stack to be able to execute C code
        \item<4-> Calls \funcname{caml\_stash\_backtrace}, a C function provided by the runtime\ignorespaces
          \onslide<6->{, with
            \begin{itemize}
              \item \funcarg{exn}: exception value
              \item \funcarg{pc}: code address of the raising point
              \item \funcarg{sp}: stack address of the raising function
              \item \funcarg{trapsp}: stack address of the next handler
            \end{itemize}
          }
      \end{itemize}
      \bigskip
      \mintocaml[firstline=9,lastline=13,highlightlines=12]{../src/backtrace/backtrace.ml}
    \end{column}
    \begin{column}{0.5\textwidth}
      \raggedleft
      \onslide*<1,3-4>{
        \mintc[firstline=190,lastline=192]{../ocaml/runtime/amd64.S}
        \mintc[firstline=234,lastline=237]{../ocaml/runtime/amd64.S}
      }
      \onslide*<2>{
        \mintc[firstline=190,lastline=192,highlightlines={191-192}]{../ocaml/runtime/amd64.S}
        \mintc[firstline=234,lastline=237,highlightlines={235-237}]{../ocaml/runtime/amd64.S}
      }
      \onslide*<1>{\listCamlRaiseExn[highlightlines=705]{}}%
      \onslide*<2>{\listCamlRaiseExn[highlightlines={707-708}]{}}%
      \onslide*<3>{\listCamlRaiseExn[highlightlines={712-714}]{}}%
      \onslide*<4>{\listCamlRaiseExn[highlightlines={715-720}]{}}%
      \onslide*<5>{\providecommand\step{1}\input{diagrams/backtrace/backtrace.tex}}%
      \onslide*<6>{\providecommand\step{2}\input{diagrams/backtrace/backtrace.tex}}%
    \end{column}
  \end{columns}
\end{frame}

\newcommand\listStashBacktrace[1][]{
  \listc[firstline=91,lastline=120,#1]{../ocaml/runtime/backtrace\_nat.c}{runtime/backtrace\_nat.c:91}}

\begin{frame}{Backtraces}{Introducing \funcname{caml\_stash\_backtrace}}
  \begin{columns}[c]
    \begin{column}{0.5\textwidth}
      \minipage[c][0.66\textheight][s]{\columnwidth}
      \begin{itemize}
        \item<1-> A C function provided by the runtime (specific to native code)
        \item<2-> It scans the stack, between the stack pointer at the raising point up to the next trap, in order to collect the names of the detected function frames
          \onslide*<2>{
            \begin{itemize}
              \item And more information, like line number, column number...
            \end{itemize}
          }
        \item<3-> It uses the same technique and information as the Garbage Collector (GC) uses to scan the values live on the stack
          \onslide*<4>{
            \begin{itemize}
              \item \typename{frame\_descr} are at the core of stack unwinding
            \end{itemize}
          }
      \end{itemize}
      \vfill
      \mintocaml[firstline=9,lastline=13,highlightlines=12]{../src/backtrace/backtrace.ml}
      \endminipage
    \end{column}
    \begin{column}{0.5\textwidth}
      \onslide*<1>{\listStashBacktrace[highlightlines=91]{}}
      \onslide*<2>{\listStashBacktrace[highlightlines={91,117-118}]{}}
      \onslide*<3->{\listStashBacktrace[highlightlines={94,106,109-110}]{}}
    \end{column}
  \end{columns}
\end{frame}

\newcommand\listFrameDescr[1][]{
  \listocaml[firstline=111,lastline=124,#1]{../ocaml/asmcomp/emitaux.ml}{asmcomp/emitaux.ml:111}}
\newcommand\listDebugInfo[1][]{
  \listocaml[firstline=38,lastline=49,#1]{../ocaml/lambda/debuginfo.mli}{lambda/debuginfo.mli:38}}

\begin{frame}{Backtraces}{\typename{frame\_descr} and \typename{frame\_debuginfo} in a nutshell}
  \begin{columns}[c]
    \begin{column}{0.5\textwidth}
      \begin{itemize}
        \item<1-> For every return address in an OCaml program, there is a associated \typename{frame\_descr}
        \item<2-> A \typename{frame\_descr} specifies
        \begin{itemize}
          \item<2-> The size of the function stack frame
          \item<3-> Where live values are on the stack (so that the GC can mark them as live and prevent from being swept)
        \end{itemize}
        \item<4-> When compiled with debug information, \typename{frame\_descr} will be linked to a \typename{frame\_debuginfo}
        \begin{itemize}
          \item<5-> Contains information about the position in the source code (file name, line and column number)
        \end{itemize}
      \end{itemize}
    \end{column}
    \begin{column}{0.5\textwidth}
        \onslide*<1>{\listFrameDescr[highlightlines={118-122}]{}}
        \onslide*<2>{\listFrameDescr[highlightlines={120}]{}}
        \onslide*<3>{\listFrameDescr[highlightlines={121}]{}}
        \onslide*<4->{\listFrameDescr[highlightlines={122}]{}}
        \medskip
        \onslide*<1-3>{\listDebugInfo{}}
        \onslide*<4>{\listDebugInfo[highlightlines={38,49}]{}}
        \onslide*<5>{\listDebugInfo[highlightlines={39-46}]{}}
    \end{column}
  \end{columns}
\end{frame}

\begin{frame}{Backtraces}{Scanning the stack with \typename{frame\_descr}}
  \begin{columns}[c]
    \begin{column}{0.5\textwidth}
      \begin{itemize}
        \item Given the return address of the code that raises the exception by calling \funcname{caml\_raise\_exn} (\funcarg{pc} argument of \funcname{caml\_stash\_backtrace})
        \item And the position of the stack pointer (\funcarg{sp} argument of \funcname{caml\_stash\_backtrace})
        \item The frame size given by \typename{frame\_descr}.\funcarg{frame\_size}, allows to find the end of the previous frame
        \item And the return address of the caller of this function
        \bigskip
        \onslide<3->{
        \item Note that traps (here the reddish \funcname{ExnA}, \funcname{ExnB} and \funcname{ExnC} blocks) belong to the frame that installed them, and so the frame size changes when traps are removed
        }
      \end{itemize}
    \end{column}
    \begin{column}{0.5\textwidth}
      \raggedleft
      \onslide*<1>{\providecommand\step{2}\input{diagrams/backtrace/backtrace.tex}}%
      \onslide*<2>{\providecommand\step{1}\input{diagrams/backtrace/scanstack.tex}}%
      \onslide*<3>{\providecommand\step{2}\input{diagrams/backtrace/scanstack.tex}}%
      \onslide*<4>{\providecommand\step{3}\input{diagrams/backtrace/scanstack.tex}}%
      \onslide*<5>{\providecommand\step{4}\input{diagrams/backtrace/scanstack.tex}}%
      \onslide*<6>{\providecommand\step{5}\input{diagrams/backtrace/scanstack.tex}}%
    \end{column}
  \end{columns}
\end{frame}

% Duplicate of previous-previous slide
\begin{frame}{Backtraces}{Continuing on \funcname{caml\_stash\_backtrace}}
  \begin{columns}[c]
    \begin{column}{0.5\textwidth}
      \minipage[c][0.75\textheight][s]{\columnwidth}
      \begin{itemize}
          \color{gray}
        \item A C function provided by the runtime (specific to native code)
        \item It scans the stack, between the stack pointer at the raising point up to the next trap, in order to collect the names of the detected function frames
        \item It uses the same technique and information as the Garbage Collector (GC) uses to scan the values live on the stack
      \end{itemize}
      \begin{itemize}
        \item For each function frame detected, store a pointer to the \typename{frame\_descr} in the \localname{domain\_state->backtrace\_buffer} array, effectively constructing the backtrace
      \end{itemize}
      \onslide<2->{
        \begin{itemize}
            \smallskip
          \item Constructed backtrace:
            \funcname{definitely\_raise},
            \onslide<3->{\funcname{probably\_raise}}
        \end{itemize}
      }
      \vfill
      \mintocaml[firstline=9,lastline=13,highlightlines=12]{../src/backtrace/backtrace.ml}
      \endminipage
    \end{column}
    \begin{column}{0.5\textwidth}
      \raggedleft
      \onslide*<1>{\listStashBacktrace[highlightlines={114-115}]{}}
      \onslide*<2>{\providecommand\step{3}\input{diagrams/backtrace/backtrace.tex}}%
      \onslide*<3>{\providecommand\step{4}\input{diagrams/backtrace/backtrace.tex}}%
    \end{column}
  \end{columns}
\end{frame}

\begin{frame}{Backtraces}{Back to \funcname{caml\_raise\_exn}}
  \begin{columns}[c]
    \begin{column}{0.5\textwidth}
      \minipage[c][0.66\textheight][s]{\columnwidth}
      \begin{itemize}\color{gray}
        \item Checks wheteher exceptions backtrace should be recorded, by checking the \funcarg{domain\_state->backtrace\_active} value
        \item Switches to the C stack to be able to execute C code
        \item Calls \funcname{caml\_stash\_backtrace}, to collect the backtrace up to the next trap
      \end{itemize}
      \onslide*<2->{
        \begin{itemize}
          \item<2-> Executes the exception handler in \funcname{probably\_raise} with \funcname{RESTORE\_EXN\_HANDLER\_OCAML}
            \begin{itemize}
              \item Set \regname{RSP} to \localname{domain\_state->exn\_handler} value
              \item Restore \localname{domain\_state->exn\_handler} from trap
              \item Jump to the handler code with \funcname{ret}
            \end{itemize}
        \end{itemize}
      }
      \vfill
      \onslide*<1>{\mintocaml[firstline=9,lastline=13,highlightlines=12]{../src/backtrace/backtrace.ml}}
      \onslide*<2->{\mintocaml[firstline=15,lastline=21,highlightlines={19-21}]{../src/backtrace/backtrace.ml}}
      \endminipage
    \end{column}
    \begin{column}{0.5\textwidth}
      \centering
      \onslide*<1>{
        \mintc[firstline=190,lastline=192]{../ocaml/runtime/amd64.S}
        \mintc[firstline=234,lastline=237]{../ocaml/runtime/amd64.S}
      }
      \onslide*<2>{
        \mintc[firstline=190,lastline=192,highlightlines={191-192}]{../ocaml/runtime/amd64.S}
        \mintc[firstline=234,lastline=237,highlightlines={235-237}]{../ocaml/runtime/amd64.S}
      }
      \onslide*<1>{\listCamlRaiseExn[highlightlines=720]{}}
      \onslide*<2>{\listCamlRaiseExn[highlightlines=722]{}}
      \onslide*<3->{\providecommand\step{5}\input{diagrams/backtrace/backtrace.tex}}
    \end{column}
  \end{columns}
\end{frame}

\begin{frame}{Backtraces}{\funcname{caml\_probably\_raise}'s handler doesn't catch \typename{ExnC}}
  \begin{columns}[c]
    \begin{column}{0.45\textwidth}
      \minipage[c][0.75\textheight][s]{\columnwidth}
      \begin{itemize}
        \item<1-> \funcname{match \dots with}\footnotemark pattern matching can also match exceptions
        \item<1-> The CMM of \funcname{probably\_raise} shows that this \funcname{match \dots with} is implemented with a pattern matching and a \funcname{try \dots with}
        \item<1-> But this one only matches on \typename{ExnA} exception
        \item<2-> Since the pattern matching fails to catch the exception, \funcname{reraise} is called with our current \typename{ExnC} exception
        \item<3-> Disassembly shows that \funcname{reraise} is actually a call to \funcname{caml\_reraise\_exn}, which is also an assembly function provided by the runtime
      \end{itemize}
      \vfill
      \mintocaml[firstline=15,lastline=21,highlightlines={18-21}]{../src/backtrace/backtrace.ml}
      \endminipage
    \end{column}
    \begin{column}{0.55\textwidth}
      \centering
      \onslide*<1>{\mintcmm[highlightlines={10-18}]{../src/backtrace/camlBacktrace\_\_probably\_raise\_351.cmm}}
      \onslide*<2>{\listcmm[highlightlines=18]{../src/backtrace/camlBacktrace\_\_probably\_raise_351.cmm}{CMM of probably\_raise}}
      \onslide*<3>{\mintobjdump[firstline=5,lastline=35,highlightlines=31]{../src/backtrace/camlBacktrace\_\_probably\_raise_351.objdump}}
    \end{column}
  \end{columns}
  \only<1>{\footnotetext[1]{https://blog.janestreet.com/pattern-matching-and-exception-handling-unite/}}
\end{frame}

\newcommand\listCamlReraiseExn[1][]{
  \listasm[firstline=727,lastline=734,#1]{../ocaml/runtime/amd64.S}{runtime/amd64.S:727}}

\begin{frame}{Backtraces}{Introducing \funcname{caml\_reraise\_exn}}
  \begin{columns}[c]
    \begin{column}{0.5\textwidth}
      \minipage[c][0.66\textheight][s]{\columnwidth}
      \begin{itemize}
        \item<1-> The exception have to be reraised to the next handler
        \item<2-> First, checks if the backtrace should be collected with \funcarg{domain\_state->backtrace\_active}
        \only<3>{
        \begin{itemize}
            \item If not, behaves like a \funcname{raise\_notrace} with \funcname{RESTORE\_EXN\_HANDLER\_OCAML}
        \end{itemize}
        }
        \item<4-> Jumps into \funcname{caml\_raise\_exn}
        \item<5-> Calls \funcname{caml\_stash\_backtrace} again, to scan the next part of the stack: from the trap just pop-ed to the next trap
      \end{itemize}
      \vfill
      \mintocaml[firstline=15,lastline=21,highlightlines={18-21}]{../src/backtrace/backtrace.ml}
      \endminipage
    \end{column}
    \begin{column}{0.5\textwidth}
      \raggedleft
      \onslide*<1>{\listCamlReraiseExn{}}
      \onslide*<2>{\listCamlReraiseExn[highlightlines=729]{}}
      \onslide*<3>{
        \mintc[firstline=190,lastline=192,highlightlines={191-192}]{../ocaml/runtime/amd64.S}
        \mintc[firstline=234,lastline=237,highlightlines={235-237}]{../ocaml/runtime/amd64.S}
        \medskip
        \listCamlReraiseExn[highlightlines=731]{}
      }
      \onslide*<4-5>{
        % TODO refactor with \listCamlReraiseExn
        \mintasm[firstline=727,lastline=734,highlightlines=730]{../ocaml/runtime/amd64.S}
        \medskip
      }
      \onslide*<4>{\listCamlRaiseExn[highlightlines=711]{}}
      \onslide*<5>{\listCamlRaiseExn[highlightlines=720]{}}
    \end{column}
  \end{columns}
\end{frame}

\begin{frame}{Backtraces}{Backtrace collection continues}
  \begin{columns}[c]
    \begin{column}{0.55\textwidth}
      \minipage[c][0.75\textheight][s]{\columnwidth}
      \begin{itemize}
        \item<1-> \funcname{caml\_stash\_backtrace} scans the older part of the stack, up to the next trap
        \item<6-> Controls jumps to the nested exception handler in \funcname{maybe\_raise}, which matches only on \typename{ExnB}
        \item<7-> \funcname{caml\_reraise\_exn} is called again, backtrace collection resumes with \funcname{caml\_stash\_backtrace}
      \end{itemize}
      \medskip
      \begin{itemize}
        \item<2-> Constructed backtrace:
        \begin{itemize}
          \item <2-> \funcname{definitely\_raise}
          \item <2-> \funcname{probably\_raise}
          \item <3-> \funcname{probably\_raise} (re-raise)
          \item <4-> \funcname{maybe\_raise}
          \item <5-> \funcname{main}
          \item <8-> \funcname{main} (re-raise)
        \end{itemize}
      \end{itemize}
      \vfill
      \onslide*<1-5>{\mintocaml[firstline=15,lastline=21]{../src/backtrace/backtrace.ml}}
      \onslide*<6>{\mintocaml[firstline=28,lastline=34,highlightlines=31]{../src/backtrace/backtrace.ml}}
      \onslide*<7->{\mintocaml[firstline=28,lastline=34,highlightlines=33]{../src/backtrace/backtrace.ml}}
      \endminipage
    \end{column}
    \begin{column}{0.45\textwidth}
      \raggedleft
      \onslide*<1>{\providecommand\step{6}\input{diagrams/backtrace/backtrace.tex}}%
      \onslide*<2>{\providecommand\step{6}\input{diagrams/backtrace/backtrace.tex}}%
      \onslide*<3>{\providecommand\step{7}\input{diagrams/backtrace/backtrace.tex}}%
      \onslide*<4>{\providecommand\step{8}\input{diagrams/backtrace/backtrace.tex}}%
      \onslide*<5>{\providecommand\step{9}\input{diagrams/backtrace/backtrace.tex}}%
      \onslide*<6>{\providecommand\step{10}\input{diagrams/backtrace/backtrace.tex}}%
      \onslide*<7>{\providecommand\step{11}\input{diagrams/backtrace/backtrace.tex}}%
      \onslide*<8>{\providecommand\step{12}\input{diagrams/backtrace/backtrace.tex}}%
      \onslide*<9>{\providecommand\step{13}\input{diagrams/backtrace/backtrace.tex}}%
    \end{column}
  \end{columns}
\end{frame}

\begin{frame}{Backtraces}{Printing the backtrace}
  \begin{columns}[c]
    \begin{column}{0.45\textwidth}
      \minipage[c][0.66\textheight][s]{\columnwidth}
      \begin{itemize}
        \item<1-> The exception backtrace can now be printed to \funcarg{stdout} with \funcname{Printexc.print_backtrace}
        \item<2-> First the exception backtrace is retrieved into an OCaml array with \funcname{get_raw_backtrace }
        \item<3-> Which is implemented as the \funcname{caml_get_exception_raw_backtrace} C function in the runtime
        \item<4-> And finally printed with \funcname{print_exception_backtrace}
      \end{itemize}
      \vfill
      \mintocaml[firstline=28,lastline=34,highlightlines=33]{../src/backtrace/backtrace.ml}
      \endminipage
    \end{column}
    \begin{column}{0.55\textwidth}
      \onslide*<1,2>{
        \mintocaml[firstline=100,lastline=101]{../ocaml/stdlib/printexc.ml}
      }
      \only<1>{
        \listocaml[firstline=170,lastline=172]{../ocaml/stdlib/printexc.ml}{stdlib/printexc.ml:170}
      }
      \only<2>{
        \listocaml[firstline=170,lastline=172,highlightlines=172]{../ocaml/stdlib/printexc.ml}{stdlib/printexc.ml:170}
      }
      \only<3>{
        \mintc[firstline=154,lastline=156]{../ocaml/runtime/backtrace.c}
        \listc[firstline=171,lastline=192,highlightlines={184-187}]{../ocaml/runtime/backtrace.c}{runtime/backtrace.c:154}
      }
      \only<4>{
        \listocaml[firstline=155,lastline=165]{../ocaml/stdlib/printexc.ml}{stdlib/printexc.ml:55}
      }
    \end{column}
  \end{columns}
\end{frame}


\subsection{The multiple kinds of raise}
\frameSubsection{}{}

\begin{frame}{Backtraces}{When backtraces are not needed}
  \begin{columns}[c]
    \begin{column}{0.45\textwidth}
      \minipage[c][0.66\textheight][s]{\columnwidth}
      \begin{itemize}
        \item<1-> What if the backtrace for an exception is never needed?
        \item<2-> It's possible to raise the exception explicitly with \funcname{raise\_notrace} to make raising as cheap as possible
        \item<3-> An example of use in the \funcname{dynlink} library of the OCaml compiler
      \end{itemize}
      \vfill
      \onslide<3->{
      \listocaml[firstline=205,lastline=209,highlightlines=207]{../ocaml/otherlibs/dynlink/dynlink\_compilerlibs/misc.ml}{otherlibs/dynlink/dynlink\_compilerlibs/misc.ml:205}
      }
      \endminipage
    \end{column}
    \begin{column}{0.55\textwidth}
    \only<4>{
      \mintcmm[highlightlines=3]{../src/notrace/camlNotrace\_\_fun_347.cmm}
      \listcmm{../src/notrace/camlNotrace\_\_all\_somes\_327.cmm}{all\_somes}
    }
    \onslide*<5->{
      \listobjdump[highlightlines={6-9}]{../src/notrace/camlNotrace\_\_fun_347.objdump}{anonymous function from \funcname{all\_somes}}
    }
    \end{column}
  \end{columns}
\end{frame}

\begin{frame}{Backtraces}{Summary table of the multiple kinds of raise}
  \begin{itemize}
    \item When debugging information is enabled and if the backtrace of an exception isn't needed, it's possible to raise with \funcname{raise\_notrace} for performance
    \item When debugging information is \emph{not} enabled, the compiler optimises \funcname{raise} calls to inline assembly (same effect as using \funcname{raise\_notrace})
  \end{itemize}
  \bigskip
  \centering
  \begin{tabular}{| l | c | c | c | c |}
    \hline
    \parbox{10em}{Debug information \\ (\funcarg{-g} flag)}  & \multicolumn{2}{|c|}{Disabled}                                     & \multicolumn{2}{|c|}{Enabled} \\
    \hline
    Raise kind                                               & \funcname{raise} & \funcname{raise\_notrace}                       & \funcname{raise} & \funcname{raise\_notrace} \\
    \hline
    Compilation                                              & \parbox{8em}{inline assembly\\ (optimization)} & inline assembly   & \funcname{caml\_raise\_exn} & inline assembly \\
    \hline
  \end{tabular}
\end{frame}


%
% Takeaway #5
%
\subsection*{Takeaway \#5}
\frameSubsectionTakeaway{}
\againframe<5>{takeaway}
}%
    \end{column}
  \end{columns}
\end{frame}

\begin{frame}{Backtraces}{Printing the backtrace}
  \begin{columns}[c]
    \begin{column}{0.45\textwidth}
      \minipage[c][0.66\textheight][s]{\columnwidth}
      \begin{itemize}
        \item<1-> The exception backtrace can now be printed to \funcarg{stdout} with \funcname{Printexc.print_backtrace}
        \item<2-> First the exception backtrace is retrieved into an OCaml array with \funcname{get_raw_backtrace }
        \item<3-> Which is implemented as the \funcname{caml_get_exception_raw_backtrace} C function in the runtime
        \item<4-> And finally printed with \funcname{print_exception_backtrace}
      \end{itemize}
      \vfill
      \mintocaml[firstline=28,lastline=34,highlightlines=33]{../src/backtrace/backtrace.ml}
      \endminipage
    \end{column}
    \begin{column}{0.55\textwidth}
      \onslide*<1,2>{
        \mintocaml[firstline=100,lastline=101]{../ocaml/stdlib/printexc.ml}
      }
      \only<1>{
        \listocaml[firstline=170,lastline=172]{../ocaml/stdlib/printexc.ml}{stdlib/printexc.ml:170}
      }
      \only<2>{
        \listocaml[firstline=170,lastline=172,highlightlines=172]{../ocaml/stdlib/printexc.ml}{stdlib/printexc.ml:170}
      }
      \only<3>{
        \mintc[firstline=154,lastline=156]{../ocaml/runtime/backtrace.c}
        \listc[firstline=171,lastline=192,highlightlines={184-187}]{../ocaml/runtime/backtrace.c}{runtime/backtrace.c:154}
      }
      \only<4>{
        \listocaml[firstline=155,lastline=165]{../ocaml/stdlib/printexc.ml}{stdlib/printexc.ml:55}
      }
    \end{column}
  \end{columns}
\end{frame}


\subsection{The multiple kinds of raise}
\frameSubsection{}{}

\begin{frame}{Backtraces}{When backtraces are not needed}
  \begin{columns}[c]
    \begin{column}{0.45\textwidth}
      \minipage[c][0.66\textheight][s]{\columnwidth}
      \begin{itemize}
        \item<1-> What if the backtrace for an exception is never needed?
        \item<2-> It's possible to raise the exception explicitly with \funcname{raise\_notrace} to make raising as cheap as possible
        \item<3-> An example of use in the \funcname{dynlink} library of the OCaml compiler
      \end{itemize}
      \vfill
      \onslide<3->{
      \listocaml[firstline=205,lastline=209,highlightlines=207]{../ocaml/otherlibs/dynlink/dynlink\_compilerlibs/misc.ml}{otherlibs/dynlink/dynlink\_compilerlibs/misc.ml:205}
      }
      \endminipage
    \end{column}
    \begin{column}{0.55\textwidth}
    \only<4>{
      \mintcmm[highlightlines=3]{../src/notrace/camlNotrace\_\_fun_347.cmm}
      \listcmm{../src/notrace/camlNotrace\_\_all\_somes\_327.cmm}{all\_somes}
    }
    \onslide*<5->{
      \listobjdump[highlightlines={6-9}]{../src/notrace/camlNotrace\_\_fun_347.objdump}{anonymous function from \funcname{all\_somes}}
    }
    \end{column}
  \end{columns}
\end{frame}

\begin{frame}{Backtraces}{Summary table of the multiple kinds of raise}
  \begin{itemize}
    \item When debugging information is enabled and if the backtrace of an exception isn't needed, it's possible to raise with \funcname{raise\_notrace} for performance
    \item When debugging information is \emph{not} enabled, the compiler optimises \funcname{raise} calls to inline assembly (same effect as using \funcname{raise\_notrace})
  \end{itemize}
  \bigskip
  \centering
  \begin{tabular}{| l | c | c | c | c |}
    \hline
    \parbox{10em}{Debug information \\ (\funcarg{-g} flag)}  & \multicolumn{2}{|c|}{Disabled}                                     & \multicolumn{2}{|c|}{Enabled} \\
    \hline
    Raise kind                                               & \funcname{raise} & \funcname{raise\_notrace}                       & \funcname{raise} & \funcname{raise\_notrace} \\
    \hline
    Compilation                                              & \parbox{8em}{inline assembly\\ (optimization)} & inline assembly   & \funcname{caml\_raise\_exn} & inline assembly \\
    \hline
  \end{tabular}
\end{frame}


%
% Takeaway #5
%
\subsection*{Takeaway \#5}
\frameSubsectionTakeaway{}
\againframe<5>{takeaway}
}%
      \onslide*<9>{\providecommand\step{13}%
% Backtraces
%
\subsection{One advantage of raising with debugging information}
\frameSubsection{}{}

\begin{frame}{Backtraces}{Debugging information \& source code}
  \begin{columns}[c]
    \begin{column}{0.5\textwidth}
      \begin{itemize}
        \item Raising an exception is fun and games, but how do we know where the exception originated? $\rightarrow$ With the backtrace associated with the exception!
        \item In order to get a backtrace from the point where an exception was raised, it's necessary to compile with debugging information enabled (\funcarg{-g} flag for the compiler)
        \item Same example as in the `Nested handlers' section
      \end{itemize}
    \end{column}
    \begin{column}{0.5\textwidth}
      \listocaml[firstline=9,lastline=34]{../src/backtrace/backtrace.ml}{backtrace/backtrace.ml}
    \end{column}
  \end{columns}
\end{frame}

\begin{frame}{Backtraces}{CMM and disassembly of \funcname{definitely\_raise}}
  \begin{columns}[c]
    \begin{column}{0.5\textwidth}
      \minipage[c][0.66\textheight][s]{\columnwidth}
      \begin{itemize}
        \item<1-> An effect of compiling with debugging information (\funcarg{-g}): the CMM no longer uses \funcname{raise\_notrace} to raise the exception but \funcname{raise} instead
        \item<2-> Disassembly shows that CMM \funcname{raise} is actually a call to \funcname{caml\_raise\_exn}, a function in assembler provided by the runtime
      \end{itemize}
      \vfill
      \mintocaml[firstline=9,lastline=13,highlightlines=12]{../src/backtrace/backtrace.ml}
      \endminipage
    \end{column}
    \begin{column}{0.5\textwidth}
      \onslide*<1>{
        \mintcmm[highlightlines=5]{../src/backtrace/camlBacktrace\_\_definitely\_raise\_347.cmm}
      }
      \onslide*<2>{
        \mintobjdump[firstline=2,highlightlines=14]{../src/backtrace/camlBacktrace\_\_definitely\_raise\_347.objdump}
      }
    \end{column}
  \end{columns}
\end{frame}

\begin{frame}{Backtraces}{Introducing \funcname{caml\_raise\_exn}}
  \begin{columns}[c]
    \begin{column}{0.5\textwidth}
      \minipage[c][0.66\textheight][s]{\columnwidth}
      \onslide*<1>{
        \begin{itemize}
          \item Raising the exception is performed using \funcname{caml\_raise\_exn} which is
            \begin{itemize}
              \item An assembly function provided by the runtime
              \item Architecture specific
            \end{itemize}
          \item Let's see what it does differently from \funcname{raise\_notrace}\dots
          \item We are about to raise \typename{ExnC} with \funcname{caml\_raise\_exn} from \funcname{definitely\_raise}
        \end{itemize}
      }
      \onslide*<2>{
        \mintobjdump[firstline=2,highlightlines=14]{../src/backtrace/camlBacktrace\_\_definitely\_raise\_347.objdump}
      }
      \vfill
      \mintocaml[firstline=9,lastline=13,highlightlines=12]{../src/backtrace/backtrace.ml}
      \endminipage
    \end{column}
    \begin{column}{0.5\textwidth}
      \centering
      \providecommand\step{1}%
% Backtraces
%
\subsection{One advantage of raising with debugging information}
\frameSubsection{}{}

\begin{frame}{Backtraces}{Debugging information \& source code}
  \begin{columns}[c]
    \begin{column}{0.5\textwidth}
      \begin{itemize}
        \item Raising an exception is fun and games, but how do we know where the exception originated? $\rightarrow$ With the backtrace associated with the exception!
        \item In order to get a backtrace from the point where an exception was raised, it's necessary to compile with debugging information enabled (\funcarg{-g} flag for the compiler)
        \item Same example as in the `Nested handlers' section
      \end{itemize}
    \end{column}
    \begin{column}{0.5\textwidth}
      \listocaml[firstline=9,lastline=34]{../src/backtrace/backtrace.ml}{backtrace/backtrace.ml}
    \end{column}
  \end{columns}
\end{frame}

\begin{frame}{Backtraces}{CMM and disassembly of \funcname{definitely\_raise}}
  \begin{columns}[c]
    \begin{column}{0.5\textwidth}
      \minipage[c][0.66\textheight][s]{\columnwidth}
      \begin{itemize}
        \item<1-> An effect of compiling with debugging information (\funcarg{-g}): the CMM no longer uses \funcname{raise\_notrace} to raise the exception but \funcname{raise} instead
        \item<2-> Disassembly shows that CMM \funcname{raise} is actually a call to \funcname{caml\_raise\_exn}, a function in assembler provided by the runtime
      \end{itemize}
      \vfill
      \mintocaml[firstline=9,lastline=13,highlightlines=12]{../src/backtrace/backtrace.ml}
      \endminipage
    \end{column}
    \begin{column}{0.5\textwidth}
      \onslide*<1>{
        \mintcmm[highlightlines=5]{../src/backtrace/camlBacktrace\_\_definitely\_raise\_347.cmm}
      }
      \onslide*<2>{
        \mintobjdump[firstline=2,highlightlines=14]{../src/backtrace/camlBacktrace\_\_definitely\_raise\_347.objdump}
      }
    \end{column}
  \end{columns}
\end{frame}

\begin{frame}{Backtraces}{Introducing \funcname{caml\_raise\_exn}}
  \begin{columns}[c]
    \begin{column}{0.5\textwidth}
      \minipage[c][0.66\textheight][s]{\columnwidth}
      \onslide*<1>{
        \begin{itemize}
          \item Raising the exception is performed using \funcname{caml\_raise\_exn} which is
            \begin{itemize}
              \item An assembly function provided by the runtime
              \item Architecture specific
            \end{itemize}
          \item Let's see what it does differently from \funcname{raise\_notrace}\dots
          \item We are about to raise \typename{ExnC} with \funcname{caml\_raise\_exn} from \funcname{definitely\_raise}
        \end{itemize}
      }
      \onslide*<2>{
        \mintobjdump[firstline=2,highlightlines=14]{../src/backtrace/camlBacktrace\_\_definitely\_raise\_347.objdump}
      }
      \vfill
      \mintocaml[firstline=9,lastline=13,highlightlines=12]{../src/backtrace/backtrace.ml}
      \endminipage
    \end{column}
    \begin{column}{0.5\textwidth}
      \centering
      \providecommand\step{1}\input{diagrams/backtrace/backtrace.tex}
    \end{column}
  \end{columns}
\end{frame}

\begin{frame}{Backtraces}{But before that, we need more fields from \typename{caml\_domain\_state}}
  \begin{columns}
    \begin{column}{0.5\textwidth}
      \begin{itemize}
        \item Remember \typename{caml\_domain\_state} the per-domain runtime state?
        \item Some other members are of interest to us now\dots
        \item \localname{backtrace\_active} is used to know if the runtime should collect backtraces
        \item \localname{backtrace\_active} can be set by
          \begin{itemize}
            \item The \funcarg{b} flag in the \funcname{OCAMLRUNPARAM} environment variable
            \item \mintinline{ocaml}{Printexc.record_backtrace : bool -> unit} \footnotemark
          \end{itemize}
      \end{itemize}
    \end{column}
    \begin{column}{0.5\textwidth}
      \begin{listing}
        \mintc[highlightlines={9-11}]{src/ocaml/domain\_state\_backtrace.h}
        \caption{runtime/caml/domain\_state.h}
      \end{listing}
    \end{column}
  \end{columns}
  \footnotetext[1]{https://v2.ocaml.org/api/Printexc.html}
\end{frame}

\newcommand\listCamlRaiseExn[1][]{
  \listasm[firstline=702,lastline=725,#1]{../ocaml/runtime/amd64.S}{runtime/amd64.S:702}}

\begin{frame}{Backtraces}{Walk through \funcname{caml\_raise\_exn}}
  \begin{columns}[T]
    \begin{column}{0.5\textwidth}
      \begin{itemize}
        \item<1-> First checks whether exceptions backtrace should be recorded
        \item<1-> By checking the \funcarg{domain\_state->backtrace\_active} value
        \item<2-> If not, acts as \funcname{raise\_notrace} with \funcname{RESTORE\_EXN\_HANDLER\_OCAML}
          \begin{itemize}
            \item Set \regname{RSP} to \localname{domain\_state->exn\_handler} value
            \item Restore \localname{domain\_state->exn\_handler} from trap
            \item Jump with \funcname{ret} (same effect as using \funcname{pop} and \funcname{jmp})
          \end{itemize}
        \item<3-> Otherwise, switches to the C stack to be able to execute C code
        \item<4-> Calls \funcname{caml\_stash\_backtrace}, a C function provided by the runtime\ignorespaces
          \onslide<6->{, with
            \begin{itemize}
              \item \funcarg{exn}: exception value
              \item \funcarg{pc}: code address of the raising point
              \item \funcarg{sp}: stack address of the raising function
              \item \funcarg{trapsp}: stack address of the next handler
            \end{itemize}
          }
      \end{itemize}
      \bigskip
      \mintocaml[firstline=9,lastline=13,highlightlines=12]{../src/backtrace/backtrace.ml}
    \end{column}
    \begin{column}{0.5\textwidth}
      \raggedleft
      \onslide*<1,3-4>{
        \mintc[firstline=190,lastline=192]{../ocaml/runtime/amd64.S}
        \mintc[firstline=234,lastline=237]{../ocaml/runtime/amd64.S}
      }
      \onslide*<2>{
        \mintc[firstline=190,lastline=192,highlightlines={191-192}]{../ocaml/runtime/amd64.S}
        \mintc[firstline=234,lastline=237,highlightlines={235-237}]{../ocaml/runtime/amd64.S}
      }
      \onslide*<1>{\listCamlRaiseExn[highlightlines=705]{}}%
      \onslide*<2>{\listCamlRaiseExn[highlightlines={707-708}]{}}%
      \onslide*<3>{\listCamlRaiseExn[highlightlines={712-714}]{}}%
      \onslide*<4>{\listCamlRaiseExn[highlightlines={715-720}]{}}%
      \onslide*<5>{\providecommand\step{1}\input{diagrams/backtrace/backtrace.tex}}%
      \onslide*<6>{\providecommand\step{2}\input{diagrams/backtrace/backtrace.tex}}%
    \end{column}
  \end{columns}
\end{frame}

\newcommand\listStashBacktrace[1][]{
  \listc[firstline=91,lastline=120,#1]{../ocaml/runtime/backtrace\_nat.c}{runtime/backtrace\_nat.c:91}}

\begin{frame}{Backtraces}{Introducing \funcname{caml\_stash\_backtrace}}
  \begin{columns}[c]
    \begin{column}{0.5\textwidth}
      \minipage[c][0.66\textheight][s]{\columnwidth}
      \begin{itemize}
        \item<1-> A C function provided by the runtime (specific to native code)
        \item<2-> It scans the stack, between the stack pointer at the raising point up to the next trap, in order to collect the names of the detected function frames
          \onslide*<2>{
            \begin{itemize}
              \item And more information, like line number, column number...
            \end{itemize}
          }
        \item<3-> It uses the same technique and information as the Garbage Collector (GC) uses to scan the values live on the stack
          \onslide*<4>{
            \begin{itemize}
              \item \typename{frame\_descr} are at the core of stack unwinding
            \end{itemize}
          }
      \end{itemize}
      \vfill
      \mintocaml[firstline=9,lastline=13,highlightlines=12]{../src/backtrace/backtrace.ml}
      \endminipage
    \end{column}
    \begin{column}{0.5\textwidth}
      \onslide*<1>{\listStashBacktrace[highlightlines=91]{}}
      \onslide*<2>{\listStashBacktrace[highlightlines={91,117-118}]{}}
      \onslide*<3->{\listStashBacktrace[highlightlines={94,106,109-110}]{}}
    \end{column}
  \end{columns}
\end{frame}

\newcommand\listFrameDescr[1][]{
  \listocaml[firstline=111,lastline=124,#1]{../ocaml/asmcomp/emitaux.ml}{asmcomp/emitaux.ml:111}}
\newcommand\listDebugInfo[1][]{
  \listocaml[firstline=38,lastline=49,#1]{../ocaml/lambda/debuginfo.mli}{lambda/debuginfo.mli:38}}

\begin{frame}{Backtraces}{\typename{frame\_descr} and \typename{frame\_debuginfo} in a nutshell}
  \begin{columns}[c]
    \begin{column}{0.5\textwidth}
      \begin{itemize}
        \item<1-> For every return address in an OCaml program, there is a associated \typename{frame\_descr}
        \item<2-> A \typename{frame\_descr} specifies
        \begin{itemize}
          \item<2-> The size of the function stack frame
          \item<3-> Where live values are on the stack (so that the GC can mark them as live and prevent from being swept)
        \end{itemize}
        \item<4-> When compiled with debug information, \typename{frame\_descr} will be linked to a \typename{frame\_debuginfo}
        \begin{itemize}
          \item<5-> Contains information about the position in the source code (file name, line and column number)
        \end{itemize}
      \end{itemize}
    \end{column}
    \begin{column}{0.5\textwidth}
        \onslide*<1>{\listFrameDescr[highlightlines={118-122}]{}}
        \onslide*<2>{\listFrameDescr[highlightlines={120}]{}}
        \onslide*<3>{\listFrameDescr[highlightlines={121}]{}}
        \onslide*<4->{\listFrameDescr[highlightlines={122}]{}}
        \medskip
        \onslide*<1-3>{\listDebugInfo{}}
        \onslide*<4>{\listDebugInfo[highlightlines={38,49}]{}}
        \onslide*<5>{\listDebugInfo[highlightlines={39-46}]{}}
    \end{column}
  \end{columns}
\end{frame}

\begin{frame}{Backtraces}{Scanning the stack with \typename{frame\_descr}}
  \begin{columns}[c]
    \begin{column}{0.5\textwidth}
      \begin{itemize}
        \item Given the return address of the code that raises the exception by calling \funcname{caml\_raise\_exn} (\funcarg{pc} argument of \funcname{caml\_stash\_backtrace})
        \item And the position of the stack pointer (\funcarg{sp} argument of \funcname{caml\_stash\_backtrace})
        \item The frame size given by \typename{frame\_descr}.\funcarg{frame\_size}, allows to find the end of the previous frame
        \item And the return address of the caller of this function
        \bigskip
        \onslide<3->{
        \item Note that traps (here the reddish \funcname{ExnA}, \funcname{ExnB} and \funcname{ExnC} blocks) belong to the frame that installed them, and so the frame size changes when traps are removed
        }
      \end{itemize}
    \end{column}
    \begin{column}{0.5\textwidth}
      \raggedleft
      \onslide*<1>{\providecommand\step{2}\input{diagrams/backtrace/backtrace.tex}}%
      \onslide*<2>{\providecommand\step{1}\input{diagrams/backtrace/scanstack.tex}}%
      \onslide*<3>{\providecommand\step{2}\input{diagrams/backtrace/scanstack.tex}}%
      \onslide*<4>{\providecommand\step{3}\input{diagrams/backtrace/scanstack.tex}}%
      \onslide*<5>{\providecommand\step{4}\input{diagrams/backtrace/scanstack.tex}}%
      \onslide*<6>{\providecommand\step{5}\input{diagrams/backtrace/scanstack.tex}}%
    \end{column}
  \end{columns}
\end{frame}

% Duplicate of previous-previous slide
\begin{frame}{Backtraces}{Continuing on \funcname{caml\_stash\_backtrace}}
  \begin{columns}[c]
    \begin{column}{0.5\textwidth}
      \minipage[c][0.75\textheight][s]{\columnwidth}
      \begin{itemize}
          \color{gray}
        \item A C function provided by the runtime (specific to native code)
        \item It scans the stack, between the stack pointer at the raising point up to the next trap, in order to collect the names of the detected function frames
        \item It uses the same technique and information as the Garbage Collector (GC) uses to scan the values live on the stack
      \end{itemize}
      \begin{itemize}
        \item For each function frame detected, store a pointer to the \typename{frame\_descr} in the \localname{domain\_state->backtrace\_buffer} array, effectively constructing the backtrace
      \end{itemize}
      \onslide<2->{
        \begin{itemize}
            \smallskip
          \item Constructed backtrace:
            \funcname{definitely\_raise},
            \onslide<3->{\funcname{probably\_raise}}
        \end{itemize}
      }
      \vfill
      \mintocaml[firstline=9,lastline=13,highlightlines=12]{../src/backtrace/backtrace.ml}
      \endminipage
    \end{column}
    \begin{column}{0.5\textwidth}
      \raggedleft
      \onslide*<1>{\listStashBacktrace[highlightlines={114-115}]{}}
      \onslide*<2>{\providecommand\step{3}\input{diagrams/backtrace/backtrace.tex}}%
      \onslide*<3>{\providecommand\step{4}\input{diagrams/backtrace/backtrace.tex}}%
    \end{column}
  \end{columns}
\end{frame}

\begin{frame}{Backtraces}{Back to \funcname{caml\_raise\_exn}}
  \begin{columns}[c]
    \begin{column}{0.5\textwidth}
      \minipage[c][0.66\textheight][s]{\columnwidth}
      \begin{itemize}\color{gray}
        \item Checks wheteher exceptions backtrace should be recorded, by checking the \funcarg{domain\_state->backtrace\_active} value
        \item Switches to the C stack to be able to execute C code
        \item Calls \funcname{caml\_stash\_backtrace}, to collect the backtrace up to the next trap
      \end{itemize}
      \onslide*<2->{
        \begin{itemize}
          \item<2-> Executes the exception handler in \funcname{probably\_raise} with \funcname{RESTORE\_EXN\_HANDLER\_OCAML}
            \begin{itemize}
              \item Set \regname{RSP} to \localname{domain\_state->exn\_handler} value
              \item Restore \localname{domain\_state->exn\_handler} from trap
              \item Jump to the handler code with \funcname{ret}
            \end{itemize}
        \end{itemize}
      }
      \vfill
      \onslide*<1>{\mintocaml[firstline=9,lastline=13,highlightlines=12]{../src/backtrace/backtrace.ml}}
      \onslide*<2->{\mintocaml[firstline=15,lastline=21,highlightlines={19-21}]{../src/backtrace/backtrace.ml}}
      \endminipage
    \end{column}
    \begin{column}{0.5\textwidth}
      \centering
      \onslide*<1>{
        \mintc[firstline=190,lastline=192]{../ocaml/runtime/amd64.S}
        \mintc[firstline=234,lastline=237]{../ocaml/runtime/amd64.S}
      }
      \onslide*<2>{
        \mintc[firstline=190,lastline=192,highlightlines={191-192}]{../ocaml/runtime/amd64.S}
        \mintc[firstline=234,lastline=237,highlightlines={235-237}]{../ocaml/runtime/amd64.S}
      }
      \onslide*<1>{\listCamlRaiseExn[highlightlines=720]{}}
      \onslide*<2>{\listCamlRaiseExn[highlightlines=722]{}}
      \onslide*<3->{\providecommand\step{5}\input{diagrams/backtrace/backtrace.tex}}
    \end{column}
  \end{columns}
\end{frame}

\begin{frame}{Backtraces}{\funcname{caml\_probably\_raise}'s handler doesn't catch \typename{ExnC}}
  \begin{columns}[c]
    \begin{column}{0.45\textwidth}
      \minipage[c][0.75\textheight][s]{\columnwidth}
      \begin{itemize}
        \item<1-> \funcname{match \dots with}\footnotemark pattern matching can also match exceptions
        \item<1-> The CMM of \funcname{probably\_raise} shows that this \funcname{match \dots with} is implemented with a pattern matching and a \funcname{try \dots with}
        \item<1-> But this one only matches on \typename{ExnA} exception
        \item<2-> Since the pattern matching fails to catch the exception, \funcname{reraise} is called with our current \typename{ExnC} exception
        \item<3-> Disassembly shows that \funcname{reraise} is actually a call to \funcname{caml\_reraise\_exn}, which is also an assembly function provided by the runtime
      \end{itemize}
      \vfill
      \mintocaml[firstline=15,lastline=21,highlightlines={18-21}]{../src/backtrace/backtrace.ml}
      \endminipage
    \end{column}
    \begin{column}{0.55\textwidth}
      \centering
      \onslide*<1>{\mintcmm[highlightlines={10-18}]{../src/backtrace/camlBacktrace\_\_probably\_raise\_351.cmm}}
      \onslide*<2>{\listcmm[highlightlines=18]{../src/backtrace/camlBacktrace\_\_probably\_raise_351.cmm}{CMM of probably\_raise}}
      \onslide*<3>{\mintobjdump[firstline=5,lastline=35,highlightlines=31]{../src/backtrace/camlBacktrace\_\_probably\_raise_351.objdump}}
    \end{column}
  \end{columns}
  \only<1>{\footnotetext[1]{https://blog.janestreet.com/pattern-matching-and-exception-handling-unite/}}
\end{frame}

\newcommand\listCamlReraiseExn[1][]{
  \listasm[firstline=727,lastline=734,#1]{../ocaml/runtime/amd64.S}{runtime/amd64.S:727}}

\begin{frame}{Backtraces}{Introducing \funcname{caml\_reraise\_exn}}
  \begin{columns}[c]
    \begin{column}{0.5\textwidth}
      \minipage[c][0.66\textheight][s]{\columnwidth}
      \begin{itemize}
        \item<1-> The exception have to be reraised to the next handler
        \item<2-> First, checks if the backtrace should be collected with \funcarg{domain\_state->backtrace\_active}
        \only<3>{
        \begin{itemize}
            \item If not, behaves like a \funcname{raise\_notrace} with \funcname{RESTORE\_EXN\_HANDLER\_OCAML}
        \end{itemize}
        }
        \item<4-> Jumps into \funcname{caml\_raise\_exn}
        \item<5-> Calls \funcname{caml\_stash\_backtrace} again, to scan the next part of the stack: from the trap just pop-ed to the next trap
      \end{itemize}
      \vfill
      \mintocaml[firstline=15,lastline=21,highlightlines={18-21}]{../src/backtrace/backtrace.ml}
      \endminipage
    \end{column}
    \begin{column}{0.5\textwidth}
      \raggedleft
      \onslide*<1>{\listCamlReraiseExn{}}
      \onslide*<2>{\listCamlReraiseExn[highlightlines=729]{}}
      \onslide*<3>{
        \mintc[firstline=190,lastline=192,highlightlines={191-192}]{../ocaml/runtime/amd64.S}
        \mintc[firstline=234,lastline=237,highlightlines={235-237}]{../ocaml/runtime/amd64.S}
        \medskip
        \listCamlReraiseExn[highlightlines=731]{}
      }
      \onslide*<4-5>{
        % TODO refactor with \listCamlReraiseExn
        \mintasm[firstline=727,lastline=734,highlightlines=730]{../ocaml/runtime/amd64.S}
        \medskip
      }
      \onslide*<4>{\listCamlRaiseExn[highlightlines=711]{}}
      \onslide*<5>{\listCamlRaiseExn[highlightlines=720]{}}
    \end{column}
  \end{columns}
\end{frame}

\begin{frame}{Backtraces}{Backtrace collection continues}
  \begin{columns}[c]
    \begin{column}{0.55\textwidth}
      \minipage[c][0.75\textheight][s]{\columnwidth}
      \begin{itemize}
        \item<1-> \funcname{caml\_stash\_backtrace} scans the older part of the stack, up to the next trap
        \item<6-> Controls jumps to the nested exception handler in \funcname{maybe\_raise}, which matches only on \typename{ExnB}
        \item<7-> \funcname{caml\_reraise\_exn} is called again, backtrace collection resumes with \funcname{caml\_stash\_backtrace}
      \end{itemize}
      \medskip
      \begin{itemize}
        \item<2-> Constructed backtrace:
        \begin{itemize}
          \item <2-> \funcname{definitely\_raise}
          \item <2-> \funcname{probably\_raise}
          \item <3-> \funcname{probably\_raise} (re-raise)
          \item <4-> \funcname{maybe\_raise}
          \item <5-> \funcname{main}
          \item <8-> \funcname{main} (re-raise)
        \end{itemize}
      \end{itemize}
      \vfill
      \onslide*<1-5>{\mintocaml[firstline=15,lastline=21]{../src/backtrace/backtrace.ml}}
      \onslide*<6>{\mintocaml[firstline=28,lastline=34,highlightlines=31]{../src/backtrace/backtrace.ml}}
      \onslide*<7->{\mintocaml[firstline=28,lastline=34,highlightlines=33]{../src/backtrace/backtrace.ml}}
      \endminipage
    \end{column}
    \begin{column}{0.45\textwidth}
      \raggedleft
      \onslide*<1>{\providecommand\step{6}\input{diagrams/backtrace/backtrace.tex}}%
      \onslide*<2>{\providecommand\step{6}\input{diagrams/backtrace/backtrace.tex}}%
      \onslide*<3>{\providecommand\step{7}\input{diagrams/backtrace/backtrace.tex}}%
      \onslide*<4>{\providecommand\step{8}\input{diagrams/backtrace/backtrace.tex}}%
      \onslide*<5>{\providecommand\step{9}\input{diagrams/backtrace/backtrace.tex}}%
      \onslide*<6>{\providecommand\step{10}\input{diagrams/backtrace/backtrace.tex}}%
      \onslide*<7>{\providecommand\step{11}\input{diagrams/backtrace/backtrace.tex}}%
      \onslide*<8>{\providecommand\step{12}\input{diagrams/backtrace/backtrace.tex}}%
      \onslide*<9>{\providecommand\step{13}\input{diagrams/backtrace/backtrace.tex}}%
    \end{column}
  \end{columns}
\end{frame}

\begin{frame}{Backtraces}{Printing the backtrace}
  \begin{columns}[c]
    \begin{column}{0.45\textwidth}
      \minipage[c][0.66\textheight][s]{\columnwidth}
      \begin{itemize}
        \item<1-> The exception backtrace can now be printed to \funcarg{stdout} with \funcname{Printexc.print_backtrace}
        \item<2-> First the exception backtrace is retrieved into an OCaml array with \funcname{get_raw_backtrace }
        \item<3-> Which is implemented as the \funcname{caml_get_exception_raw_backtrace} C function in the runtime
        \item<4-> And finally printed with \funcname{print_exception_backtrace}
      \end{itemize}
      \vfill
      \mintocaml[firstline=28,lastline=34,highlightlines=33]{../src/backtrace/backtrace.ml}
      \endminipage
    \end{column}
    \begin{column}{0.55\textwidth}
      \onslide*<1,2>{
        \mintocaml[firstline=100,lastline=101]{../ocaml/stdlib/printexc.ml}
      }
      \only<1>{
        \listocaml[firstline=170,lastline=172]{../ocaml/stdlib/printexc.ml}{stdlib/printexc.ml:170}
      }
      \only<2>{
        \listocaml[firstline=170,lastline=172,highlightlines=172]{../ocaml/stdlib/printexc.ml}{stdlib/printexc.ml:170}
      }
      \only<3>{
        \mintc[firstline=154,lastline=156]{../ocaml/runtime/backtrace.c}
        \listc[firstline=171,lastline=192,highlightlines={184-187}]{../ocaml/runtime/backtrace.c}{runtime/backtrace.c:154}
      }
      \only<4>{
        \listocaml[firstline=155,lastline=165]{../ocaml/stdlib/printexc.ml}{stdlib/printexc.ml:55}
      }
    \end{column}
  \end{columns}
\end{frame}


\subsection{The multiple kinds of raise}
\frameSubsection{}{}

\begin{frame}{Backtraces}{When backtraces are not needed}
  \begin{columns}[c]
    \begin{column}{0.45\textwidth}
      \minipage[c][0.66\textheight][s]{\columnwidth}
      \begin{itemize}
        \item<1-> What if the backtrace for an exception is never needed?
        \item<2-> It's possible to raise the exception explicitly with \funcname{raise\_notrace} to make raising as cheap as possible
        \item<3-> An example of use in the \funcname{dynlink} library of the OCaml compiler
      \end{itemize}
      \vfill
      \onslide<3->{
      \listocaml[firstline=205,lastline=209,highlightlines=207]{../ocaml/otherlibs/dynlink/dynlink\_compilerlibs/misc.ml}{otherlibs/dynlink/dynlink\_compilerlibs/misc.ml:205}
      }
      \endminipage
    \end{column}
    \begin{column}{0.55\textwidth}
    \only<4>{
      \mintcmm[highlightlines=3]{../src/notrace/camlNotrace\_\_fun_347.cmm}
      \listcmm{../src/notrace/camlNotrace\_\_all\_somes\_327.cmm}{all\_somes}
    }
    \onslide*<5->{
      \listobjdump[highlightlines={6-9}]{../src/notrace/camlNotrace\_\_fun_347.objdump}{anonymous function from \funcname{all\_somes}}
    }
    \end{column}
  \end{columns}
\end{frame}

\begin{frame}{Backtraces}{Summary table of the multiple kinds of raise}
  \begin{itemize}
    \item When debugging information is enabled and if the backtrace of an exception isn't needed, it's possible to raise with \funcname{raise\_notrace} for performance
    \item When debugging information is \emph{not} enabled, the compiler optimises \funcname{raise} calls to inline assembly (same effect as using \funcname{raise\_notrace})
  \end{itemize}
  \bigskip
  \centering
  \begin{tabular}{| l | c | c | c | c |}
    \hline
    \parbox{10em}{Debug information \\ (\funcarg{-g} flag)}  & \multicolumn{2}{|c|}{Disabled}                                     & \multicolumn{2}{|c|}{Enabled} \\
    \hline
    Raise kind                                               & \funcname{raise} & \funcname{raise\_notrace}                       & \funcname{raise} & \funcname{raise\_notrace} \\
    \hline
    Compilation                                              & \parbox{8em}{inline assembly\\ (optimization)} & inline assembly   & \funcname{caml\_raise\_exn} & inline assembly \\
    \hline
  \end{tabular}
\end{frame}


%
% Takeaway #5
%
\subsection*{Takeaway \#5}
\frameSubsectionTakeaway{}
\againframe<5>{takeaway}

    \end{column}
  \end{columns}
\end{frame}

\begin{frame}{Backtraces}{But before that, we need more fields from \typename{caml\_domain\_state}}
  \begin{columns}
    \begin{column}{0.5\textwidth}
      \begin{itemize}
        \item Remember \typename{caml\_domain\_state} the per-domain runtime state?
        \item Some other members are of interest to us now\dots
        \item \localname{backtrace\_active} is used to know if the runtime should collect backtraces
        \item \localname{backtrace\_active} can be set by
          \begin{itemize}
            \item The \funcarg{b} flag in the \funcname{OCAMLRUNPARAM} environment variable
            \item \mintinline{ocaml}{Printexc.record_backtrace : bool -> unit} \footnotemark
          \end{itemize}
      \end{itemize}
    \end{column}
    \begin{column}{0.5\textwidth}
      \begin{listing}
        \mintc[highlightlines={9-11}]{src/ocaml/domain\_state\_backtrace.h}
        \caption{runtime/caml/domain\_state.h}
      \end{listing}
    \end{column}
  \end{columns}
  \footnotetext[1]{https://v2.ocaml.org/api/Printexc.html}
\end{frame}

\newcommand\listCamlRaiseExn[1][]{
  \listasm[firstline=702,lastline=725,#1]{../ocaml/runtime/amd64.S}{runtime/amd64.S:702}}

\begin{frame}{Backtraces}{Walk through \funcname{caml\_raise\_exn}}
  \begin{columns}[T]
    \begin{column}{0.5\textwidth}
      \begin{itemize}
        \item<1-> First checks whether exceptions backtrace should be recorded
        \item<1-> By checking the \funcarg{domain\_state->backtrace\_active} value
        \item<2-> If not, acts as \funcname{raise\_notrace} with \funcname{RESTORE\_EXN\_HANDLER\_OCAML}
          \begin{itemize}
            \item Set \regname{RSP} to \localname{domain\_state->exn\_handler} value
            \item Restore \localname{domain\_state->exn\_handler} from trap
            \item Jump with \funcname{ret} (same effect as using \funcname{pop} and \funcname{jmp})
          \end{itemize}
        \item<3-> Otherwise, switches to the C stack to be able to execute C code
        \item<4-> Calls \funcname{caml\_stash\_backtrace}, a C function provided by the runtime\ignorespaces
          \onslide<6->{, with
            \begin{itemize}
              \item \funcarg{exn}: exception value
              \item \funcarg{pc}: code address of the raising point
              \item \funcarg{sp}: stack address of the raising function
              \item \funcarg{trapsp}: stack address of the next handler
            \end{itemize}
          }
      \end{itemize}
      \bigskip
      \mintocaml[firstline=9,lastline=13,highlightlines=12]{../src/backtrace/backtrace.ml}
    \end{column}
    \begin{column}{0.5\textwidth}
      \raggedleft
      \onslide*<1,3-4>{
        \mintc[firstline=190,lastline=192]{../ocaml/runtime/amd64.S}
        \mintc[firstline=234,lastline=237]{../ocaml/runtime/amd64.S}
      }
      \onslide*<2>{
        \mintc[firstline=190,lastline=192,highlightlines={191-192}]{../ocaml/runtime/amd64.S}
        \mintc[firstline=234,lastline=237,highlightlines={235-237}]{../ocaml/runtime/amd64.S}
      }
      \onslide*<1>{\listCamlRaiseExn[highlightlines=705]{}}%
      \onslide*<2>{\listCamlRaiseExn[highlightlines={707-708}]{}}%
      \onslide*<3>{\listCamlRaiseExn[highlightlines={712-714}]{}}%
      \onslide*<4>{\listCamlRaiseExn[highlightlines={715-720}]{}}%
      \onslide*<5>{\providecommand\step{1}%
% Backtraces
%
\subsection{One advantage of raising with debugging information}
\frameSubsection{}{}

\begin{frame}{Backtraces}{Debugging information \& source code}
  \begin{columns}[c]
    \begin{column}{0.5\textwidth}
      \begin{itemize}
        \item Raising an exception is fun and games, but how do we know where the exception originated? $\rightarrow$ With the backtrace associated with the exception!
        \item In order to get a backtrace from the point where an exception was raised, it's necessary to compile with debugging information enabled (\funcarg{-g} flag for the compiler)
        \item Same example as in the `Nested handlers' section
      \end{itemize}
    \end{column}
    \begin{column}{0.5\textwidth}
      \listocaml[firstline=9,lastline=34]{../src/backtrace/backtrace.ml}{backtrace/backtrace.ml}
    \end{column}
  \end{columns}
\end{frame}

\begin{frame}{Backtraces}{CMM and disassembly of \funcname{definitely\_raise}}
  \begin{columns}[c]
    \begin{column}{0.5\textwidth}
      \minipage[c][0.66\textheight][s]{\columnwidth}
      \begin{itemize}
        \item<1-> An effect of compiling with debugging information (\funcarg{-g}): the CMM no longer uses \funcname{raise\_notrace} to raise the exception but \funcname{raise} instead
        \item<2-> Disassembly shows that CMM \funcname{raise} is actually a call to \funcname{caml\_raise\_exn}, a function in assembler provided by the runtime
      \end{itemize}
      \vfill
      \mintocaml[firstline=9,lastline=13,highlightlines=12]{../src/backtrace/backtrace.ml}
      \endminipage
    \end{column}
    \begin{column}{0.5\textwidth}
      \onslide*<1>{
        \mintcmm[highlightlines=5]{../src/backtrace/camlBacktrace\_\_definitely\_raise\_347.cmm}
      }
      \onslide*<2>{
        \mintobjdump[firstline=2,highlightlines=14]{../src/backtrace/camlBacktrace\_\_definitely\_raise\_347.objdump}
      }
    \end{column}
  \end{columns}
\end{frame}

\begin{frame}{Backtraces}{Introducing \funcname{caml\_raise\_exn}}
  \begin{columns}[c]
    \begin{column}{0.5\textwidth}
      \minipage[c][0.66\textheight][s]{\columnwidth}
      \onslide*<1>{
        \begin{itemize}
          \item Raising the exception is performed using \funcname{caml\_raise\_exn} which is
            \begin{itemize}
              \item An assembly function provided by the runtime
              \item Architecture specific
            \end{itemize}
          \item Let's see what it does differently from \funcname{raise\_notrace}\dots
          \item We are about to raise \typename{ExnC} with \funcname{caml\_raise\_exn} from \funcname{definitely\_raise}
        \end{itemize}
      }
      \onslide*<2>{
        \mintobjdump[firstline=2,highlightlines=14]{../src/backtrace/camlBacktrace\_\_definitely\_raise\_347.objdump}
      }
      \vfill
      \mintocaml[firstline=9,lastline=13,highlightlines=12]{../src/backtrace/backtrace.ml}
      \endminipage
    \end{column}
    \begin{column}{0.5\textwidth}
      \centering
      \providecommand\step{1}\input{diagrams/backtrace/backtrace.tex}
    \end{column}
  \end{columns}
\end{frame}

\begin{frame}{Backtraces}{But before that, we need more fields from \typename{caml\_domain\_state}}
  \begin{columns}
    \begin{column}{0.5\textwidth}
      \begin{itemize}
        \item Remember \typename{caml\_domain\_state} the per-domain runtime state?
        \item Some other members are of interest to us now\dots
        \item \localname{backtrace\_active} is used to know if the runtime should collect backtraces
        \item \localname{backtrace\_active} can be set by
          \begin{itemize}
            \item The \funcarg{b} flag in the \funcname{OCAMLRUNPARAM} environment variable
            \item \mintinline{ocaml}{Printexc.record_backtrace : bool -> unit} \footnotemark
          \end{itemize}
      \end{itemize}
    \end{column}
    \begin{column}{0.5\textwidth}
      \begin{listing}
        \mintc[highlightlines={9-11}]{src/ocaml/domain\_state\_backtrace.h}
        \caption{runtime/caml/domain\_state.h}
      \end{listing}
    \end{column}
  \end{columns}
  \footnotetext[1]{https://v2.ocaml.org/api/Printexc.html}
\end{frame}

\newcommand\listCamlRaiseExn[1][]{
  \listasm[firstline=702,lastline=725,#1]{../ocaml/runtime/amd64.S}{runtime/amd64.S:702}}

\begin{frame}{Backtraces}{Walk through \funcname{caml\_raise\_exn}}
  \begin{columns}[T]
    \begin{column}{0.5\textwidth}
      \begin{itemize}
        \item<1-> First checks whether exceptions backtrace should be recorded
        \item<1-> By checking the \funcarg{domain\_state->backtrace\_active} value
        \item<2-> If not, acts as \funcname{raise\_notrace} with \funcname{RESTORE\_EXN\_HANDLER\_OCAML}
          \begin{itemize}
            \item Set \regname{RSP} to \localname{domain\_state->exn\_handler} value
            \item Restore \localname{domain\_state->exn\_handler} from trap
            \item Jump with \funcname{ret} (same effect as using \funcname{pop} and \funcname{jmp})
          \end{itemize}
        \item<3-> Otherwise, switches to the C stack to be able to execute C code
        \item<4-> Calls \funcname{caml\_stash\_backtrace}, a C function provided by the runtime\ignorespaces
          \onslide<6->{, with
            \begin{itemize}
              \item \funcarg{exn}: exception value
              \item \funcarg{pc}: code address of the raising point
              \item \funcarg{sp}: stack address of the raising function
              \item \funcarg{trapsp}: stack address of the next handler
            \end{itemize}
          }
      \end{itemize}
      \bigskip
      \mintocaml[firstline=9,lastline=13,highlightlines=12]{../src/backtrace/backtrace.ml}
    \end{column}
    \begin{column}{0.5\textwidth}
      \raggedleft
      \onslide*<1,3-4>{
        \mintc[firstline=190,lastline=192]{../ocaml/runtime/amd64.S}
        \mintc[firstline=234,lastline=237]{../ocaml/runtime/amd64.S}
      }
      \onslide*<2>{
        \mintc[firstline=190,lastline=192,highlightlines={191-192}]{../ocaml/runtime/amd64.S}
        \mintc[firstline=234,lastline=237,highlightlines={235-237}]{../ocaml/runtime/amd64.S}
      }
      \onslide*<1>{\listCamlRaiseExn[highlightlines=705]{}}%
      \onslide*<2>{\listCamlRaiseExn[highlightlines={707-708}]{}}%
      \onslide*<3>{\listCamlRaiseExn[highlightlines={712-714}]{}}%
      \onslide*<4>{\listCamlRaiseExn[highlightlines={715-720}]{}}%
      \onslide*<5>{\providecommand\step{1}\input{diagrams/backtrace/backtrace.tex}}%
      \onslide*<6>{\providecommand\step{2}\input{diagrams/backtrace/backtrace.tex}}%
    \end{column}
  \end{columns}
\end{frame}

\newcommand\listStashBacktrace[1][]{
  \listc[firstline=91,lastline=120,#1]{../ocaml/runtime/backtrace\_nat.c}{runtime/backtrace\_nat.c:91}}

\begin{frame}{Backtraces}{Introducing \funcname{caml\_stash\_backtrace}}
  \begin{columns}[c]
    \begin{column}{0.5\textwidth}
      \minipage[c][0.66\textheight][s]{\columnwidth}
      \begin{itemize}
        \item<1-> A C function provided by the runtime (specific to native code)
        \item<2-> It scans the stack, between the stack pointer at the raising point up to the next trap, in order to collect the names of the detected function frames
          \onslide*<2>{
            \begin{itemize}
              \item And more information, like line number, column number...
            \end{itemize}
          }
        \item<3-> It uses the same technique and information as the Garbage Collector (GC) uses to scan the values live on the stack
          \onslide*<4>{
            \begin{itemize}
              \item \typename{frame\_descr} are at the core of stack unwinding
            \end{itemize}
          }
      \end{itemize}
      \vfill
      \mintocaml[firstline=9,lastline=13,highlightlines=12]{../src/backtrace/backtrace.ml}
      \endminipage
    \end{column}
    \begin{column}{0.5\textwidth}
      \onslide*<1>{\listStashBacktrace[highlightlines=91]{}}
      \onslide*<2>{\listStashBacktrace[highlightlines={91,117-118}]{}}
      \onslide*<3->{\listStashBacktrace[highlightlines={94,106,109-110}]{}}
    \end{column}
  \end{columns}
\end{frame}

\newcommand\listFrameDescr[1][]{
  \listocaml[firstline=111,lastline=124,#1]{../ocaml/asmcomp/emitaux.ml}{asmcomp/emitaux.ml:111}}
\newcommand\listDebugInfo[1][]{
  \listocaml[firstline=38,lastline=49,#1]{../ocaml/lambda/debuginfo.mli}{lambda/debuginfo.mli:38}}

\begin{frame}{Backtraces}{\typename{frame\_descr} and \typename{frame\_debuginfo} in a nutshell}
  \begin{columns}[c]
    \begin{column}{0.5\textwidth}
      \begin{itemize}
        \item<1-> For every return address in an OCaml program, there is a associated \typename{frame\_descr}
        \item<2-> A \typename{frame\_descr} specifies
        \begin{itemize}
          \item<2-> The size of the function stack frame
          \item<3-> Where live values are on the stack (so that the GC can mark them as live and prevent from being swept)
        \end{itemize}
        \item<4-> When compiled with debug information, \typename{frame\_descr} will be linked to a \typename{frame\_debuginfo}
        \begin{itemize}
          \item<5-> Contains information about the position in the source code (file name, line and column number)
        \end{itemize}
      \end{itemize}
    \end{column}
    \begin{column}{0.5\textwidth}
        \onslide*<1>{\listFrameDescr[highlightlines={118-122}]{}}
        \onslide*<2>{\listFrameDescr[highlightlines={120}]{}}
        \onslide*<3>{\listFrameDescr[highlightlines={121}]{}}
        \onslide*<4->{\listFrameDescr[highlightlines={122}]{}}
        \medskip
        \onslide*<1-3>{\listDebugInfo{}}
        \onslide*<4>{\listDebugInfo[highlightlines={38,49}]{}}
        \onslide*<5>{\listDebugInfo[highlightlines={39-46}]{}}
    \end{column}
  \end{columns}
\end{frame}

\begin{frame}{Backtraces}{Scanning the stack with \typename{frame\_descr}}
  \begin{columns}[c]
    \begin{column}{0.5\textwidth}
      \begin{itemize}
        \item Given the return address of the code that raises the exception by calling \funcname{caml\_raise\_exn} (\funcarg{pc} argument of \funcname{caml\_stash\_backtrace})
        \item And the position of the stack pointer (\funcarg{sp} argument of \funcname{caml\_stash\_backtrace})
        \item The frame size given by \typename{frame\_descr}.\funcarg{frame\_size}, allows to find the end of the previous frame
        \item And the return address of the caller of this function
        \bigskip
        \onslide<3->{
        \item Note that traps (here the reddish \funcname{ExnA}, \funcname{ExnB} and \funcname{ExnC} blocks) belong to the frame that installed them, and so the frame size changes when traps are removed
        }
      \end{itemize}
    \end{column}
    \begin{column}{0.5\textwidth}
      \raggedleft
      \onslide*<1>{\providecommand\step{2}\input{diagrams/backtrace/backtrace.tex}}%
      \onslide*<2>{\providecommand\step{1}\input{diagrams/backtrace/scanstack.tex}}%
      \onslide*<3>{\providecommand\step{2}\input{diagrams/backtrace/scanstack.tex}}%
      \onslide*<4>{\providecommand\step{3}\input{diagrams/backtrace/scanstack.tex}}%
      \onslide*<5>{\providecommand\step{4}\input{diagrams/backtrace/scanstack.tex}}%
      \onslide*<6>{\providecommand\step{5}\input{diagrams/backtrace/scanstack.tex}}%
    \end{column}
  \end{columns}
\end{frame}

% Duplicate of previous-previous slide
\begin{frame}{Backtraces}{Continuing on \funcname{caml\_stash\_backtrace}}
  \begin{columns}[c]
    \begin{column}{0.5\textwidth}
      \minipage[c][0.75\textheight][s]{\columnwidth}
      \begin{itemize}
          \color{gray}
        \item A C function provided by the runtime (specific to native code)
        \item It scans the stack, between the stack pointer at the raising point up to the next trap, in order to collect the names of the detected function frames
        \item It uses the same technique and information as the Garbage Collector (GC) uses to scan the values live on the stack
      \end{itemize}
      \begin{itemize}
        \item For each function frame detected, store a pointer to the \typename{frame\_descr} in the \localname{domain\_state->backtrace\_buffer} array, effectively constructing the backtrace
      \end{itemize}
      \onslide<2->{
        \begin{itemize}
            \smallskip
          \item Constructed backtrace:
            \funcname{definitely\_raise},
            \onslide<3->{\funcname{probably\_raise}}
        \end{itemize}
      }
      \vfill
      \mintocaml[firstline=9,lastline=13,highlightlines=12]{../src/backtrace/backtrace.ml}
      \endminipage
    \end{column}
    \begin{column}{0.5\textwidth}
      \raggedleft
      \onslide*<1>{\listStashBacktrace[highlightlines={114-115}]{}}
      \onslide*<2>{\providecommand\step{3}\input{diagrams/backtrace/backtrace.tex}}%
      \onslide*<3>{\providecommand\step{4}\input{diagrams/backtrace/backtrace.tex}}%
    \end{column}
  \end{columns}
\end{frame}

\begin{frame}{Backtraces}{Back to \funcname{caml\_raise\_exn}}
  \begin{columns}[c]
    \begin{column}{0.5\textwidth}
      \minipage[c][0.66\textheight][s]{\columnwidth}
      \begin{itemize}\color{gray}
        \item Checks wheteher exceptions backtrace should be recorded, by checking the \funcarg{domain\_state->backtrace\_active} value
        \item Switches to the C stack to be able to execute C code
        \item Calls \funcname{caml\_stash\_backtrace}, to collect the backtrace up to the next trap
      \end{itemize}
      \onslide*<2->{
        \begin{itemize}
          \item<2-> Executes the exception handler in \funcname{probably\_raise} with \funcname{RESTORE\_EXN\_HANDLER\_OCAML}
            \begin{itemize}
              \item Set \regname{RSP} to \localname{domain\_state->exn\_handler} value
              \item Restore \localname{domain\_state->exn\_handler} from trap
              \item Jump to the handler code with \funcname{ret}
            \end{itemize}
        \end{itemize}
      }
      \vfill
      \onslide*<1>{\mintocaml[firstline=9,lastline=13,highlightlines=12]{../src/backtrace/backtrace.ml}}
      \onslide*<2->{\mintocaml[firstline=15,lastline=21,highlightlines={19-21}]{../src/backtrace/backtrace.ml}}
      \endminipage
    \end{column}
    \begin{column}{0.5\textwidth}
      \centering
      \onslide*<1>{
        \mintc[firstline=190,lastline=192]{../ocaml/runtime/amd64.S}
        \mintc[firstline=234,lastline=237]{../ocaml/runtime/amd64.S}
      }
      \onslide*<2>{
        \mintc[firstline=190,lastline=192,highlightlines={191-192}]{../ocaml/runtime/amd64.S}
        \mintc[firstline=234,lastline=237,highlightlines={235-237}]{../ocaml/runtime/amd64.S}
      }
      \onslide*<1>{\listCamlRaiseExn[highlightlines=720]{}}
      \onslide*<2>{\listCamlRaiseExn[highlightlines=722]{}}
      \onslide*<3->{\providecommand\step{5}\input{diagrams/backtrace/backtrace.tex}}
    \end{column}
  \end{columns}
\end{frame}

\begin{frame}{Backtraces}{\funcname{caml\_probably\_raise}'s handler doesn't catch \typename{ExnC}}
  \begin{columns}[c]
    \begin{column}{0.45\textwidth}
      \minipage[c][0.75\textheight][s]{\columnwidth}
      \begin{itemize}
        \item<1-> \funcname{match \dots with}\footnotemark pattern matching can also match exceptions
        \item<1-> The CMM of \funcname{probably\_raise} shows that this \funcname{match \dots with} is implemented with a pattern matching and a \funcname{try \dots with}
        \item<1-> But this one only matches on \typename{ExnA} exception
        \item<2-> Since the pattern matching fails to catch the exception, \funcname{reraise} is called with our current \typename{ExnC} exception
        \item<3-> Disassembly shows that \funcname{reraise} is actually a call to \funcname{caml\_reraise\_exn}, which is also an assembly function provided by the runtime
      \end{itemize}
      \vfill
      \mintocaml[firstline=15,lastline=21,highlightlines={18-21}]{../src/backtrace/backtrace.ml}
      \endminipage
    \end{column}
    \begin{column}{0.55\textwidth}
      \centering
      \onslide*<1>{\mintcmm[highlightlines={10-18}]{../src/backtrace/camlBacktrace\_\_probably\_raise\_351.cmm}}
      \onslide*<2>{\listcmm[highlightlines=18]{../src/backtrace/camlBacktrace\_\_probably\_raise_351.cmm}{CMM of probably\_raise}}
      \onslide*<3>{\mintobjdump[firstline=5,lastline=35,highlightlines=31]{../src/backtrace/camlBacktrace\_\_probably\_raise_351.objdump}}
    \end{column}
  \end{columns}
  \only<1>{\footnotetext[1]{https://blog.janestreet.com/pattern-matching-and-exception-handling-unite/}}
\end{frame}

\newcommand\listCamlReraiseExn[1][]{
  \listasm[firstline=727,lastline=734,#1]{../ocaml/runtime/amd64.S}{runtime/amd64.S:727}}

\begin{frame}{Backtraces}{Introducing \funcname{caml\_reraise\_exn}}
  \begin{columns}[c]
    \begin{column}{0.5\textwidth}
      \minipage[c][0.66\textheight][s]{\columnwidth}
      \begin{itemize}
        \item<1-> The exception have to be reraised to the next handler
        \item<2-> First, checks if the backtrace should be collected with \funcarg{domain\_state->backtrace\_active}
        \only<3>{
        \begin{itemize}
            \item If not, behaves like a \funcname{raise\_notrace} with \funcname{RESTORE\_EXN\_HANDLER\_OCAML}
        \end{itemize}
        }
        \item<4-> Jumps into \funcname{caml\_raise\_exn}
        \item<5-> Calls \funcname{caml\_stash\_backtrace} again, to scan the next part of the stack: from the trap just pop-ed to the next trap
      \end{itemize}
      \vfill
      \mintocaml[firstline=15,lastline=21,highlightlines={18-21}]{../src/backtrace/backtrace.ml}
      \endminipage
    \end{column}
    \begin{column}{0.5\textwidth}
      \raggedleft
      \onslide*<1>{\listCamlReraiseExn{}}
      \onslide*<2>{\listCamlReraiseExn[highlightlines=729]{}}
      \onslide*<3>{
        \mintc[firstline=190,lastline=192,highlightlines={191-192}]{../ocaml/runtime/amd64.S}
        \mintc[firstline=234,lastline=237,highlightlines={235-237}]{../ocaml/runtime/amd64.S}
        \medskip
        \listCamlReraiseExn[highlightlines=731]{}
      }
      \onslide*<4-5>{
        % TODO refactor with \listCamlReraiseExn
        \mintasm[firstline=727,lastline=734,highlightlines=730]{../ocaml/runtime/amd64.S}
        \medskip
      }
      \onslide*<4>{\listCamlRaiseExn[highlightlines=711]{}}
      \onslide*<5>{\listCamlRaiseExn[highlightlines=720]{}}
    \end{column}
  \end{columns}
\end{frame}

\begin{frame}{Backtraces}{Backtrace collection continues}
  \begin{columns}[c]
    \begin{column}{0.55\textwidth}
      \minipage[c][0.75\textheight][s]{\columnwidth}
      \begin{itemize}
        \item<1-> \funcname{caml\_stash\_backtrace} scans the older part of the stack, up to the next trap
        \item<6-> Controls jumps to the nested exception handler in \funcname{maybe\_raise}, which matches only on \typename{ExnB}
        \item<7-> \funcname{caml\_reraise\_exn} is called again, backtrace collection resumes with \funcname{caml\_stash\_backtrace}
      \end{itemize}
      \medskip
      \begin{itemize}
        \item<2-> Constructed backtrace:
        \begin{itemize}
          \item <2-> \funcname{definitely\_raise}
          \item <2-> \funcname{probably\_raise}
          \item <3-> \funcname{probably\_raise} (re-raise)
          \item <4-> \funcname{maybe\_raise}
          \item <5-> \funcname{main}
          \item <8-> \funcname{main} (re-raise)
        \end{itemize}
      \end{itemize}
      \vfill
      \onslide*<1-5>{\mintocaml[firstline=15,lastline=21]{../src/backtrace/backtrace.ml}}
      \onslide*<6>{\mintocaml[firstline=28,lastline=34,highlightlines=31]{../src/backtrace/backtrace.ml}}
      \onslide*<7->{\mintocaml[firstline=28,lastline=34,highlightlines=33]{../src/backtrace/backtrace.ml}}
      \endminipage
    \end{column}
    \begin{column}{0.45\textwidth}
      \raggedleft
      \onslide*<1>{\providecommand\step{6}\input{diagrams/backtrace/backtrace.tex}}%
      \onslide*<2>{\providecommand\step{6}\input{diagrams/backtrace/backtrace.tex}}%
      \onslide*<3>{\providecommand\step{7}\input{diagrams/backtrace/backtrace.tex}}%
      \onslide*<4>{\providecommand\step{8}\input{diagrams/backtrace/backtrace.tex}}%
      \onslide*<5>{\providecommand\step{9}\input{diagrams/backtrace/backtrace.tex}}%
      \onslide*<6>{\providecommand\step{10}\input{diagrams/backtrace/backtrace.tex}}%
      \onslide*<7>{\providecommand\step{11}\input{diagrams/backtrace/backtrace.tex}}%
      \onslide*<8>{\providecommand\step{12}\input{diagrams/backtrace/backtrace.tex}}%
      \onslide*<9>{\providecommand\step{13}\input{diagrams/backtrace/backtrace.tex}}%
    \end{column}
  \end{columns}
\end{frame}

\begin{frame}{Backtraces}{Printing the backtrace}
  \begin{columns}[c]
    \begin{column}{0.45\textwidth}
      \minipage[c][0.66\textheight][s]{\columnwidth}
      \begin{itemize}
        \item<1-> The exception backtrace can now be printed to \funcarg{stdout} with \funcname{Printexc.print_backtrace}
        \item<2-> First the exception backtrace is retrieved into an OCaml array with \funcname{get_raw_backtrace }
        \item<3-> Which is implemented as the \funcname{caml_get_exception_raw_backtrace} C function in the runtime
        \item<4-> And finally printed with \funcname{print_exception_backtrace}
      \end{itemize}
      \vfill
      \mintocaml[firstline=28,lastline=34,highlightlines=33]{../src/backtrace/backtrace.ml}
      \endminipage
    \end{column}
    \begin{column}{0.55\textwidth}
      \onslide*<1,2>{
        \mintocaml[firstline=100,lastline=101]{../ocaml/stdlib/printexc.ml}
      }
      \only<1>{
        \listocaml[firstline=170,lastline=172]{../ocaml/stdlib/printexc.ml}{stdlib/printexc.ml:170}
      }
      \only<2>{
        \listocaml[firstline=170,lastline=172,highlightlines=172]{../ocaml/stdlib/printexc.ml}{stdlib/printexc.ml:170}
      }
      \only<3>{
        \mintc[firstline=154,lastline=156]{../ocaml/runtime/backtrace.c}
        \listc[firstline=171,lastline=192,highlightlines={184-187}]{../ocaml/runtime/backtrace.c}{runtime/backtrace.c:154}
      }
      \only<4>{
        \listocaml[firstline=155,lastline=165]{../ocaml/stdlib/printexc.ml}{stdlib/printexc.ml:55}
      }
    \end{column}
  \end{columns}
\end{frame}


\subsection{The multiple kinds of raise}
\frameSubsection{}{}

\begin{frame}{Backtraces}{When backtraces are not needed}
  \begin{columns}[c]
    \begin{column}{0.45\textwidth}
      \minipage[c][0.66\textheight][s]{\columnwidth}
      \begin{itemize}
        \item<1-> What if the backtrace for an exception is never needed?
        \item<2-> It's possible to raise the exception explicitly with \funcname{raise\_notrace} to make raising as cheap as possible
        \item<3-> An example of use in the \funcname{dynlink} library of the OCaml compiler
      \end{itemize}
      \vfill
      \onslide<3->{
      \listocaml[firstline=205,lastline=209,highlightlines=207]{../ocaml/otherlibs/dynlink/dynlink\_compilerlibs/misc.ml}{otherlibs/dynlink/dynlink\_compilerlibs/misc.ml:205}
      }
      \endminipage
    \end{column}
    \begin{column}{0.55\textwidth}
    \only<4>{
      \mintcmm[highlightlines=3]{../src/notrace/camlNotrace\_\_fun_347.cmm}
      \listcmm{../src/notrace/camlNotrace\_\_all\_somes\_327.cmm}{all\_somes}
    }
    \onslide*<5->{
      \listobjdump[highlightlines={6-9}]{../src/notrace/camlNotrace\_\_fun_347.objdump}{anonymous function from \funcname{all\_somes}}
    }
    \end{column}
  \end{columns}
\end{frame}

\begin{frame}{Backtraces}{Summary table of the multiple kinds of raise}
  \begin{itemize}
    \item When debugging information is enabled and if the backtrace of an exception isn't needed, it's possible to raise with \funcname{raise\_notrace} for performance
    \item When debugging information is \emph{not} enabled, the compiler optimises \funcname{raise} calls to inline assembly (same effect as using \funcname{raise\_notrace})
  \end{itemize}
  \bigskip
  \centering
  \begin{tabular}{| l | c | c | c | c |}
    \hline
    \parbox{10em}{Debug information \\ (\funcarg{-g} flag)}  & \multicolumn{2}{|c|}{Disabled}                                     & \multicolumn{2}{|c|}{Enabled} \\
    \hline
    Raise kind                                               & \funcname{raise} & \funcname{raise\_notrace}                       & \funcname{raise} & \funcname{raise\_notrace} \\
    \hline
    Compilation                                              & \parbox{8em}{inline assembly\\ (optimization)} & inline assembly   & \funcname{caml\_raise\_exn} & inline assembly \\
    \hline
  \end{tabular}
\end{frame}


%
% Takeaway #5
%
\subsection*{Takeaway \#5}
\frameSubsectionTakeaway{}
\againframe<5>{takeaway}
}%
      \onslide*<6>{\providecommand\step{2}%
% Backtraces
%
\subsection{One advantage of raising with debugging information}
\frameSubsection{}{}

\begin{frame}{Backtraces}{Debugging information \& source code}
  \begin{columns}[c]
    \begin{column}{0.5\textwidth}
      \begin{itemize}
        \item Raising an exception is fun and games, but how do we know where the exception originated? $\rightarrow$ With the backtrace associated with the exception!
        \item In order to get a backtrace from the point where an exception was raised, it's necessary to compile with debugging information enabled (\funcarg{-g} flag for the compiler)
        \item Same example as in the `Nested handlers' section
      \end{itemize}
    \end{column}
    \begin{column}{0.5\textwidth}
      \listocaml[firstline=9,lastline=34]{../src/backtrace/backtrace.ml}{backtrace/backtrace.ml}
    \end{column}
  \end{columns}
\end{frame}

\begin{frame}{Backtraces}{CMM and disassembly of \funcname{definitely\_raise}}
  \begin{columns}[c]
    \begin{column}{0.5\textwidth}
      \minipage[c][0.66\textheight][s]{\columnwidth}
      \begin{itemize}
        \item<1-> An effect of compiling with debugging information (\funcarg{-g}): the CMM no longer uses \funcname{raise\_notrace} to raise the exception but \funcname{raise} instead
        \item<2-> Disassembly shows that CMM \funcname{raise} is actually a call to \funcname{caml\_raise\_exn}, a function in assembler provided by the runtime
      \end{itemize}
      \vfill
      \mintocaml[firstline=9,lastline=13,highlightlines=12]{../src/backtrace/backtrace.ml}
      \endminipage
    \end{column}
    \begin{column}{0.5\textwidth}
      \onslide*<1>{
        \mintcmm[highlightlines=5]{../src/backtrace/camlBacktrace\_\_definitely\_raise\_347.cmm}
      }
      \onslide*<2>{
        \mintobjdump[firstline=2,highlightlines=14]{../src/backtrace/camlBacktrace\_\_definitely\_raise\_347.objdump}
      }
    \end{column}
  \end{columns}
\end{frame}

\begin{frame}{Backtraces}{Introducing \funcname{caml\_raise\_exn}}
  \begin{columns}[c]
    \begin{column}{0.5\textwidth}
      \minipage[c][0.66\textheight][s]{\columnwidth}
      \onslide*<1>{
        \begin{itemize}
          \item Raising the exception is performed using \funcname{caml\_raise\_exn} which is
            \begin{itemize}
              \item An assembly function provided by the runtime
              \item Architecture specific
            \end{itemize}
          \item Let's see what it does differently from \funcname{raise\_notrace}\dots
          \item We are about to raise \typename{ExnC} with \funcname{caml\_raise\_exn} from \funcname{definitely\_raise}
        \end{itemize}
      }
      \onslide*<2>{
        \mintobjdump[firstline=2,highlightlines=14]{../src/backtrace/camlBacktrace\_\_definitely\_raise\_347.objdump}
      }
      \vfill
      \mintocaml[firstline=9,lastline=13,highlightlines=12]{../src/backtrace/backtrace.ml}
      \endminipage
    \end{column}
    \begin{column}{0.5\textwidth}
      \centering
      \providecommand\step{1}\input{diagrams/backtrace/backtrace.tex}
    \end{column}
  \end{columns}
\end{frame}

\begin{frame}{Backtraces}{But before that, we need more fields from \typename{caml\_domain\_state}}
  \begin{columns}
    \begin{column}{0.5\textwidth}
      \begin{itemize}
        \item Remember \typename{caml\_domain\_state} the per-domain runtime state?
        \item Some other members are of interest to us now\dots
        \item \localname{backtrace\_active} is used to know if the runtime should collect backtraces
        \item \localname{backtrace\_active} can be set by
          \begin{itemize}
            \item The \funcarg{b} flag in the \funcname{OCAMLRUNPARAM} environment variable
            \item \mintinline{ocaml}{Printexc.record_backtrace : bool -> unit} \footnotemark
          \end{itemize}
      \end{itemize}
    \end{column}
    \begin{column}{0.5\textwidth}
      \begin{listing}
        \mintc[highlightlines={9-11}]{src/ocaml/domain\_state\_backtrace.h}
        \caption{runtime/caml/domain\_state.h}
      \end{listing}
    \end{column}
  \end{columns}
  \footnotetext[1]{https://v2.ocaml.org/api/Printexc.html}
\end{frame}

\newcommand\listCamlRaiseExn[1][]{
  \listasm[firstline=702,lastline=725,#1]{../ocaml/runtime/amd64.S}{runtime/amd64.S:702}}

\begin{frame}{Backtraces}{Walk through \funcname{caml\_raise\_exn}}
  \begin{columns}[T]
    \begin{column}{0.5\textwidth}
      \begin{itemize}
        \item<1-> First checks whether exceptions backtrace should be recorded
        \item<1-> By checking the \funcarg{domain\_state->backtrace\_active} value
        \item<2-> If not, acts as \funcname{raise\_notrace} with \funcname{RESTORE\_EXN\_HANDLER\_OCAML}
          \begin{itemize}
            \item Set \regname{RSP} to \localname{domain\_state->exn\_handler} value
            \item Restore \localname{domain\_state->exn\_handler} from trap
            \item Jump with \funcname{ret} (same effect as using \funcname{pop} and \funcname{jmp})
          \end{itemize}
        \item<3-> Otherwise, switches to the C stack to be able to execute C code
        \item<4-> Calls \funcname{caml\_stash\_backtrace}, a C function provided by the runtime\ignorespaces
          \onslide<6->{, with
            \begin{itemize}
              \item \funcarg{exn}: exception value
              \item \funcarg{pc}: code address of the raising point
              \item \funcarg{sp}: stack address of the raising function
              \item \funcarg{trapsp}: stack address of the next handler
            \end{itemize}
          }
      \end{itemize}
      \bigskip
      \mintocaml[firstline=9,lastline=13,highlightlines=12]{../src/backtrace/backtrace.ml}
    \end{column}
    \begin{column}{0.5\textwidth}
      \raggedleft
      \onslide*<1,3-4>{
        \mintc[firstline=190,lastline=192]{../ocaml/runtime/amd64.S}
        \mintc[firstline=234,lastline=237]{../ocaml/runtime/amd64.S}
      }
      \onslide*<2>{
        \mintc[firstline=190,lastline=192,highlightlines={191-192}]{../ocaml/runtime/amd64.S}
        \mintc[firstline=234,lastline=237,highlightlines={235-237}]{../ocaml/runtime/amd64.S}
      }
      \onslide*<1>{\listCamlRaiseExn[highlightlines=705]{}}%
      \onslide*<2>{\listCamlRaiseExn[highlightlines={707-708}]{}}%
      \onslide*<3>{\listCamlRaiseExn[highlightlines={712-714}]{}}%
      \onslide*<4>{\listCamlRaiseExn[highlightlines={715-720}]{}}%
      \onslide*<5>{\providecommand\step{1}\input{diagrams/backtrace/backtrace.tex}}%
      \onslide*<6>{\providecommand\step{2}\input{diagrams/backtrace/backtrace.tex}}%
    \end{column}
  \end{columns}
\end{frame}

\newcommand\listStashBacktrace[1][]{
  \listc[firstline=91,lastline=120,#1]{../ocaml/runtime/backtrace\_nat.c}{runtime/backtrace\_nat.c:91}}

\begin{frame}{Backtraces}{Introducing \funcname{caml\_stash\_backtrace}}
  \begin{columns}[c]
    \begin{column}{0.5\textwidth}
      \minipage[c][0.66\textheight][s]{\columnwidth}
      \begin{itemize}
        \item<1-> A C function provided by the runtime (specific to native code)
        \item<2-> It scans the stack, between the stack pointer at the raising point up to the next trap, in order to collect the names of the detected function frames
          \onslide*<2>{
            \begin{itemize}
              \item And more information, like line number, column number...
            \end{itemize}
          }
        \item<3-> It uses the same technique and information as the Garbage Collector (GC) uses to scan the values live on the stack
          \onslide*<4>{
            \begin{itemize}
              \item \typename{frame\_descr} are at the core of stack unwinding
            \end{itemize}
          }
      \end{itemize}
      \vfill
      \mintocaml[firstline=9,lastline=13,highlightlines=12]{../src/backtrace/backtrace.ml}
      \endminipage
    \end{column}
    \begin{column}{0.5\textwidth}
      \onslide*<1>{\listStashBacktrace[highlightlines=91]{}}
      \onslide*<2>{\listStashBacktrace[highlightlines={91,117-118}]{}}
      \onslide*<3->{\listStashBacktrace[highlightlines={94,106,109-110}]{}}
    \end{column}
  \end{columns}
\end{frame}

\newcommand\listFrameDescr[1][]{
  \listocaml[firstline=111,lastline=124,#1]{../ocaml/asmcomp/emitaux.ml}{asmcomp/emitaux.ml:111}}
\newcommand\listDebugInfo[1][]{
  \listocaml[firstline=38,lastline=49,#1]{../ocaml/lambda/debuginfo.mli}{lambda/debuginfo.mli:38}}

\begin{frame}{Backtraces}{\typename{frame\_descr} and \typename{frame\_debuginfo} in a nutshell}
  \begin{columns}[c]
    \begin{column}{0.5\textwidth}
      \begin{itemize}
        \item<1-> For every return address in an OCaml program, there is a associated \typename{frame\_descr}
        \item<2-> A \typename{frame\_descr} specifies
        \begin{itemize}
          \item<2-> The size of the function stack frame
          \item<3-> Where live values are on the stack (so that the GC can mark them as live and prevent from being swept)
        \end{itemize}
        \item<4-> When compiled with debug information, \typename{frame\_descr} will be linked to a \typename{frame\_debuginfo}
        \begin{itemize}
          \item<5-> Contains information about the position in the source code (file name, line and column number)
        \end{itemize}
      \end{itemize}
    \end{column}
    \begin{column}{0.5\textwidth}
        \onslide*<1>{\listFrameDescr[highlightlines={118-122}]{}}
        \onslide*<2>{\listFrameDescr[highlightlines={120}]{}}
        \onslide*<3>{\listFrameDescr[highlightlines={121}]{}}
        \onslide*<4->{\listFrameDescr[highlightlines={122}]{}}
        \medskip
        \onslide*<1-3>{\listDebugInfo{}}
        \onslide*<4>{\listDebugInfo[highlightlines={38,49}]{}}
        \onslide*<5>{\listDebugInfo[highlightlines={39-46}]{}}
    \end{column}
  \end{columns}
\end{frame}

\begin{frame}{Backtraces}{Scanning the stack with \typename{frame\_descr}}
  \begin{columns}[c]
    \begin{column}{0.5\textwidth}
      \begin{itemize}
        \item Given the return address of the code that raises the exception by calling \funcname{caml\_raise\_exn} (\funcarg{pc} argument of \funcname{caml\_stash\_backtrace})
        \item And the position of the stack pointer (\funcarg{sp} argument of \funcname{caml\_stash\_backtrace})
        \item The frame size given by \typename{frame\_descr}.\funcarg{frame\_size}, allows to find the end of the previous frame
        \item And the return address of the caller of this function
        \bigskip
        \onslide<3->{
        \item Note that traps (here the reddish \funcname{ExnA}, \funcname{ExnB} and \funcname{ExnC} blocks) belong to the frame that installed them, and so the frame size changes when traps are removed
        }
      \end{itemize}
    \end{column}
    \begin{column}{0.5\textwidth}
      \raggedleft
      \onslide*<1>{\providecommand\step{2}\input{diagrams/backtrace/backtrace.tex}}%
      \onslide*<2>{\providecommand\step{1}\input{diagrams/backtrace/scanstack.tex}}%
      \onslide*<3>{\providecommand\step{2}\input{diagrams/backtrace/scanstack.tex}}%
      \onslide*<4>{\providecommand\step{3}\input{diagrams/backtrace/scanstack.tex}}%
      \onslide*<5>{\providecommand\step{4}\input{diagrams/backtrace/scanstack.tex}}%
      \onslide*<6>{\providecommand\step{5}\input{diagrams/backtrace/scanstack.tex}}%
    \end{column}
  \end{columns}
\end{frame}

% Duplicate of previous-previous slide
\begin{frame}{Backtraces}{Continuing on \funcname{caml\_stash\_backtrace}}
  \begin{columns}[c]
    \begin{column}{0.5\textwidth}
      \minipage[c][0.75\textheight][s]{\columnwidth}
      \begin{itemize}
          \color{gray}
        \item A C function provided by the runtime (specific to native code)
        \item It scans the stack, between the stack pointer at the raising point up to the next trap, in order to collect the names of the detected function frames
        \item It uses the same technique and information as the Garbage Collector (GC) uses to scan the values live on the stack
      \end{itemize}
      \begin{itemize}
        \item For each function frame detected, store a pointer to the \typename{frame\_descr} in the \localname{domain\_state->backtrace\_buffer} array, effectively constructing the backtrace
      \end{itemize}
      \onslide<2->{
        \begin{itemize}
            \smallskip
          \item Constructed backtrace:
            \funcname{definitely\_raise},
            \onslide<3->{\funcname{probably\_raise}}
        \end{itemize}
      }
      \vfill
      \mintocaml[firstline=9,lastline=13,highlightlines=12]{../src/backtrace/backtrace.ml}
      \endminipage
    \end{column}
    \begin{column}{0.5\textwidth}
      \raggedleft
      \onslide*<1>{\listStashBacktrace[highlightlines={114-115}]{}}
      \onslide*<2>{\providecommand\step{3}\input{diagrams/backtrace/backtrace.tex}}%
      \onslide*<3>{\providecommand\step{4}\input{diagrams/backtrace/backtrace.tex}}%
    \end{column}
  \end{columns}
\end{frame}

\begin{frame}{Backtraces}{Back to \funcname{caml\_raise\_exn}}
  \begin{columns}[c]
    \begin{column}{0.5\textwidth}
      \minipage[c][0.66\textheight][s]{\columnwidth}
      \begin{itemize}\color{gray}
        \item Checks wheteher exceptions backtrace should be recorded, by checking the \funcarg{domain\_state->backtrace\_active} value
        \item Switches to the C stack to be able to execute C code
        \item Calls \funcname{caml\_stash\_backtrace}, to collect the backtrace up to the next trap
      \end{itemize}
      \onslide*<2->{
        \begin{itemize}
          \item<2-> Executes the exception handler in \funcname{probably\_raise} with \funcname{RESTORE\_EXN\_HANDLER\_OCAML}
            \begin{itemize}
              \item Set \regname{RSP} to \localname{domain\_state->exn\_handler} value
              \item Restore \localname{domain\_state->exn\_handler} from trap
              \item Jump to the handler code with \funcname{ret}
            \end{itemize}
        \end{itemize}
      }
      \vfill
      \onslide*<1>{\mintocaml[firstline=9,lastline=13,highlightlines=12]{../src/backtrace/backtrace.ml}}
      \onslide*<2->{\mintocaml[firstline=15,lastline=21,highlightlines={19-21}]{../src/backtrace/backtrace.ml}}
      \endminipage
    \end{column}
    \begin{column}{0.5\textwidth}
      \centering
      \onslide*<1>{
        \mintc[firstline=190,lastline=192]{../ocaml/runtime/amd64.S}
        \mintc[firstline=234,lastline=237]{../ocaml/runtime/amd64.S}
      }
      \onslide*<2>{
        \mintc[firstline=190,lastline=192,highlightlines={191-192}]{../ocaml/runtime/amd64.S}
        \mintc[firstline=234,lastline=237,highlightlines={235-237}]{../ocaml/runtime/amd64.S}
      }
      \onslide*<1>{\listCamlRaiseExn[highlightlines=720]{}}
      \onslide*<2>{\listCamlRaiseExn[highlightlines=722]{}}
      \onslide*<3->{\providecommand\step{5}\input{diagrams/backtrace/backtrace.tex}}
    \end{column}
  \end{columns}
\end{frame}

\begin{frame}{Backtraces}{\funcname{caml\_probably\_raise}'s handler doesn't catch \typename{ExnC}}
  \begin{columns}[c]
    \begin{column}{0.45\textwidth}
      \minipage[c][0.75\textheight][s]{\columnwidth}
      \begin{itemize}
        \item<1-> \funcname{match \dots with}\footnotemark pattern matching can also match exceptions
        \item<1-> The CMM of \funcname{probably\_raise} shows that this \funcname{match \dots with} is implemented with a pattern matching and a \funcname{try \dots with}
        \item<1-> But this one only matches on \typename{ExnA} exception
        \item<2-> Since the pattern matching fails to catch the exception, \funcname{reraise} is called with our current \typename{ExnC} exception
        \item<3-> Disassembly shows that \funcname{reraise} is actually a call to \funcname{caml\_reraise\_exn}, which is also an assembly function provided by the runtime
      \end{itemize}
      \vfill
      \mintocaml[firstline=15,lastline=21,highlightlines={18-21}]{../src/backtrace/backtrace.ml}
      \endminipage
    \end{column}
    \begin{column}{0.55\textwidth}
      \centering
      \onslide*<1>{\mintcmm[highlightlines={10-18}]{../src/backtrace/camlBacktrace\_\_probably\_raise\_351.cmm}}
      \onslide*<2>{\listcmm[highlightlines=18]{../src/backtrace/camlBacktrace\_\_probably\_raise_351.cmm}{CMM of probably\_raise}}
      \onslide*<3>{\mintobjdump[firstline=5,lastline=35,highlightlines=31]{../src/backtrace/camlBacktrace\_\_probably\_raise_351.objdump}}
    \end{column}
  \end{columns}
  \only<1>{\footnotetext[1]{https://blog.janestreet.com/pattern-matching-and-exception-handling-unite/}}
\end{frame}

\newcommand\listCamlReraiseExn[1][]{
  \listasm[firstline=727,lastline=734,#1]{../ocaml/runtime/amd64.S}{runtime/amd64.S:727}}

\begin{frame}{Backtraces}{Introducing \funcname{caml\_reraise\_exn}}
  \begin{columns}[c]
    \begin{column}{0.5\textwidth}
      \minipage[c][0.66\textheight][s]{\columnwidth}
      \begin{itemize}
        \item<1-> The exception have to be reraised to the next handler
        \item<2-> First, checks if the backtrace should be collected with \funcarg{domain\_state->backtrace\_active}
        \only<3>{
        \begin{itemize}
            \item If not, behaves like a \funcname{raise\_notrace} with \funcname{RESTORE\_EXN\_HANDLER\_OCAML}
        \end{itemize}
        }
        \item<4-> Jumps into \funcname{caml\_raise\_exn}
        \item<5-> Calls \funcname{caml\_stash\_backtrace} again, to scan the next part of the stack: from the trap just pop-ed to the next trap
      \end{itemize}
      \vfill
      \mintocaml[firstline=15,lastline=21,highlightlines={18-21}]{../src/backtrace/backtrace.ml}
      \endminipage
    \end{column}
    \begin{column}{0.5\textwidth}
      \raggedleft
      \onslide*<1>{\listCamlReraiseExn{}}
      \onslide*<2>{\listCamlReraiseExn[highlightlines=729]{}}
      \onslide*<3>{
        \mintc[firstline=190,lastline=192,highlightlines={191-192}]{../ocaml/runtime/amd64.S}
        \mintc[firstline=234,lastline=237,highlightlines={235-237}]{../ocaml/runtime/amd64.S}
        \medskip
        \listCamlReraiseExn[highlightlines=731]{}
      }
      \onslide*<4-5>{
        % TODO refactor with \listCamlReraiseExn
        \mintasm[firstline=727,lastline=734,highlightlines=730]{../ocaml/runtime/amd64.S}
        \medskip
      }
      \onslide*<4>{\listCamlRaiseExn[highlightlines=711]{}}
      \onslide*<5>{\listCamlRaiseExn[highlightlines=720]{}}
    \end{column}
  \end{columns}
\end{frame}

\begin{frame}{Backtraces}{Backtrace collection continues}
  \begin{columns}[c]
    \begin{column}{0.55\textwidth}
      \minipage[c][0.75\textheight][s]{\columnwidth}
      \begin{itemize}
        \item<1-> \funcname{caml\_stash\_backtrace} scans the older part of the stack, up to the next trap
        \item<6-> Controls jumps to the nested exception handler in \funcname{maybe\_raise}, which matches only on \typename{ExnB}
        \item<7-> \funcname{caml\_reraise\_exn} is called again, backtrace collection resumes with \funcname{caml\_stash\_backtrace}
      \end{itemize}
      \medskip
      \begin{itemize}
        \item<2-> Constructed backtrace:
        \begin{itemize}
          \item <2-> \funcname{definitely\_raise}
          \item <2-> \funcname{probably\_raise}
          \item <3-> \funcname{probably\_raise} (re-raise)
          \item <4-> \funcname{maybe\_raise}
          \item <5-> \funcname{main}
          \item <8-> \funcname{main} (re-raise)
        \end{itemize}
      \end{itemize}
      \vfill
      \onslide*<1-5>{\mintocaml[firstline=15,lastline=21]{../src/backtrace/backtrace.ml}}
      \onslide*<6>{\mintocaml[firstline=28,lastline=34,highlightlines=31]{../src/backtrace/backtrace.ml}}
      \onslide*<7->{\mintocaml[firstline=28,lastline=34,highlightlines=33]{../src/backtrace/backtrace.ml}}
      \endminipage
    \end{column}
    \begin{column}{0.45\textwidth}
      \raggedleft
      \onslide*<1>{\providecommand\step{6}\input{diagrams/backtrace/backtrace.tex}}%
      \onslide*<2>{\providecommand\step{6}\input{diagrams/backtrace/backtrace.tex}}%
      \onslide*<3>{\providecommand\step{7}\input{diagrams/backtrace/backtrace.tex}}%
      \onslide*<4>{\providecommand\step{8}\input{diagrams/backtrace/backtrace.tex}}%
      \onslide*<5>{\providecommand\step{9}\input{diagrams/backtrace/backtrace.tex}}%
      \onslide*<6>{\providecommand\step{10}\input{diagrams/backtrace/backtrace.tex}}%
      \onslide*<7>{\providecommand\step{11}\input{diagrams/backtrace/backtrace.tex}}%
      \onslide*<8>{\providecommand\step{12}\input{diagrams/backtrace/backtrace.tex}}%
      \onslide*<9>{\providecommand\step{13}\input{diagrams/backtrace/backtrace.tex}}%
    \end{column}
  \end{columns}
\end{frame}

\begin{frame}{Backtraces}{Printing the backtrace}
  \begin{columns}[c]
    \begin{column}{0.45\textwidth}
      \minipage[c][0.66\textheight][s]{\columnwidth}
      \begin{itemize}
        \item<1-> The exception backtrace can now be printed to \funcarg{stdout} with \funcname{Printexc.print_backtrace}
        \item<2-> First the exception backtrace is retrieved into an OCaml array with \funcname{get_raw_backtrace }
        \item<3-> Which is implemented as the \funcname{caml_get_exception_raw_backtrace} C function in the runtime
        \item<4-> And finally printed with \funcname{print_exception_backtrace}
      \end{itemize}
      \vfill
      \mintocaml[firstline=28,lastline=34,highlightlines=33]{../src/backtrace/backtrace.ml}
      \endminipage
    \end{column}
    \begin{column}{0.55\textwidth}
      \onslide*<1,2>{
        \mintocaml[firstline=100,lastline=101]{../ocaml/stdlib/printexc.ml}
      }
      \only<1>{
        \listocaml[firstline=170,lastline=172]{../ocaml/stdlib/printexc.ml}{stdlib/printexc.ml:170}
      }
      \only<2>{
        \listocaml[firstline=170,lastline=172,highlightlines=172]{../ocaml/stdlib/printexc.ml}{stdlib/printexc.ml:170}
      }
      \only<3>{
        \mintc[firstline=154,lastline=156]{../ocaml/runtime/backtrace.c}
        \listc[firstline=171,lastline=192,highlightlines={184-187}]{../ocaml/runtime/backtrace.c}{runtime/backtrace.c:154}
      }
      \only<4>{
        \listocaml[firstline=155,lastline=165]{../ocaml/stdlib/printexc.ml}{stdlib/printexc.ml:55}
      }
    \end{column}
  \end{columns}
\end{frame}


\subsection{The multiple kinds of raise}
\frameSubsection{}{}

\begin{frame}{Backtraces}{When backtraces are not needed}
  \begin{columns}[c]
    \begin{column}{0.45\textwidth}
      \minipage[c][0.66\textheight][s]{\columnwidth}
      \begin{itemize}
        \item<1-> What if the backtrace for an exception is never needed?
        \item<2-> It's possible to raise the exception explicitly with \funcname{raise\_notrace} to make raising as cheap as possible
        \item<3-> An example of use in the \funcname{dynlink} library of the OCaml compiler
      \end{itemize}
      \vfill
      \onslide<3->{
      \listocaml[firstline=205,lastline=209,highlightlines=207]{../ocaml/otherlibs/dynlink/dynlink\_compilerlibs/misc.ml}{otherlibs/dynlink/dynlink\_compilerlibs/misc.ml:205}
      }
      \endminipage
    \end{column}
    \begin{column}{0.55\textwidth}
    \only<4>{
      \mintcmm[highlightlines=3]{../src/notrace/camlNotrace\_\_fun_347.cmm}
      \listcmm{../src/notrace/camlNotrace\_\_all\_somes\_327.cmm}{all\_somes}
    }
    \onslide*<5->{
      \listobjdump[highlightlines={6-9}]{../src/notrace/camlNotrace\_\_fun_347.objdump}{anonymous function from \funcname{all\_somes}}
    }
    \end{column}
  \end{columns}
\end{frame}

\begin{frame}{Backtraces}{Summary table of the multiple kinds of raise}
  \begin{itemize}
    \item When debugging information is enabled and if the backtrace of an exception isn't needed, it's possible to raise with \funcname{raise\_notrace} for performance
    \item When debugging information is \emph{not} enabled, the compiler optimises \funcname{raise} calls to inline assembly (same effect as using \funcname{raise\_notrace})
  \end{itemize}
  \bigskip
  \centering
  \begin{tabular}{| l | c | c | c | c |}
    \hline
    \parbox{10em}{Debug information \\ (\funcarg{-g} flag)}  & \multicolumn{2}{|c|}{Disabled}                                     & \multicolumn{2}{|c|}{Enabled} \\
    \hline
    Raise kind                                               & \funcname{raise} & \funcname{raise\_notrace}                       & \funcname{raise} & \funcname{raise\_notrace} \\
    \hline
    Compilation                                              & \parbox{8em}{inline assembly\\ (optimization)} & inline assembly   & \funcname{caml\_raise\_exn} & inline assembly \\
    \hline
  \end{tabular}
\end{frame}


%
% Takeaway #5
%
\subsection*{Takeaway \#5}
\frameSubsectionTakeaway{}
\againframe<5>{takeaway}
}%
    \end{column}
  \end{columns}
\end{frame}

\newcommand\listStashBacktrace[1][]{
  \listc[firstline=91,lastline=120,#1]{../ocaml/runtime/backtrace\_nat.c}{runtime/backtrace\_nat.c:91}}

\begin{frame}{Backtraces}{Introducing \funcname{caml\_stash\_backtrace}}
  \begin{columns}[c]
    \begin{column}{0.5\textwidth}
      \minipage[c][0.66\textheight][s]{\columnwidth}
      \begin{itemize}
        \item<1-> A C function provided by the runtime (specific to native code)
        \item<2-> It scans the stack, between the stack pointer at the raising point up to the next trap, in order to collect the names of the detected function frames
          \onslide*<2>{
            \begin{itemize}
              \item And more information, like line number, column number...
            \end{itemize}
          }
        \item<3-> It uses the same technique and information as the Garbage Collector (GC) uses to scan the values live on the stack
          \onslide*<4>{
            \begin{itemize}
              \item \typename{frame\_descr} are at the core of stack unwinding
            \end{itemize}
          }
      \end{itemize}
      \vfill
      \mintocaml[firstline=9,lastline=13,highlightlines=12]{../src/backtrace/backtrace.ml}
      \endminipage
    \end{column}
    \begin{column}{0.5\textwidth}
      \onslide*<1>{\listStashBacktrace[highlightlines=91]{}}
      \onslide*<2>{\listStashBacktrace[highlightlines={91,117-118}]{}}
      \onslide*<3->{\listStashBacktrace[highlightlines={94,106,109-110}]{}}
    \end{column}
  \end{columns}
\end{frame}

\newcommand\listFrameDescr[1][]{
  \listocaml[firstline=111,lastline=124,#1]{../ocaml/asmcomp/emitaux.ml}{asmcomp/emitaux.ml:111}}
\newcommand\listDebugInfo[1][]{
  \listocaml[firstline=38,lastline=49,#1]{../ocaml/lambda/debuginfo.mli}{lambda/debuginfo.mli:38}}

\begin{frame}{Backtraces}{\typename{frame\_descr} and \typename{frame\_debuginfo} in a nutshell}
  \begin{columns}[c]
    \begin{column}{0.5\textwidth}
      \begin{itemize}
        \item<1-> For every return address in an OCaml program, there is a associated \typename{frame\_descr}
        \item<2-> A \typename{frame\_descr} specifies
        \begin{itemize}
          \item<2-> The size of the function stack frame
          \item<3-> Where live values are on the stack (so that the GC can mark them as live and prevent from being swept)
        \end{itemize}
        \item<4-> When compiled with debug information, \typename{frame\_descr} will be linked to a \typename{frame\_debuginfo}
        \begin{itemize}
          \item<5-> Contains information about the position in the source code (file name, line and column number)
        \end{itemize}
      \end{itemize}
    \end{column}
    \begin{column}{0.5\textwidth}
        \onslide*<1>{\listFrameDescr[highlightlines={118-122}]{}}
        \onslide*<2>{\listFrameDescr[highlightlines={120}]{}}
        \onslide*<3>{\listFrameDescr[highlightlines={121}]{}}
        \onslide*<4->{\listFrameDescr[highlightlines={122}]{}}
        \medskip
        \onslide*<1-3>{\listDebugInfo{}}
        \onslide*<4>{\listDebugInfo[highlightlines={38,49}]{}}
        \onslide*<5>{\listDebugInfo[highlightlines={39-46}]{}}
    \end{column}
  \end{columns}
\end{frame}

\begin{frame}{Backtraces}{Scanning the stack with \typename{frame\_descr}}
  \begin{columns}[c]
    \begin{column}{0.5\textwidth}
      \begin{itemize}
        \item Given the return address of the code that raises the exception by calling \funcname{caml\_raise\_exn} (\funcarg{pc} argument of \funcname{caml\_stash\_backtrace})
        \item And the position of the stack pointer (\funcarg{sp} argument of \funcname{caml\_stash\_backtrace})
        \item The frame size given by \typename{frame\_descr}.\funcarg{frame\_size}, allows to find the end of the previous frame
        \item And the return address of the caller of this function
        \bigskip
        \onslide<3->{
        \item Note that traps (here the reddish \funcname{ExnA}, \funcname{ExnB} and \funcname{ExnC} blocks) belong to the frame that installed them, and so the frame size changes when traps are removed
        }
      \end{itemize}
    \end{column}
    \begin{column}{0.5\textwidth}
      \raggedleft
      \onslide*<1>{\providecommand\step{2}%
% Backtraces
%
\subsection{One advantage of raising with debugging information}
\frameSubsection{}{}

\begin{frame}{Backtraces}{Debugging information \& source code}
  \begin{columns}[c]
    \begin{column}{0.5\textwidth}
      \begin{itemize}
        \item Raising an exception is fun and games, but how do we know where the exception originated? $\rightarrow$ With the backtrace associated with the exception!
        \item In order to get a backtrace from the point where an exception was raised, it's necessary to compile with debugging information enabled (\funcarg{-g} flag for the compiler)
        \item Same example as in the `Nested handlers' section
      \end{itemize}
    \end{column}
    \begin{column}{0.5\textwidth}
      \listocaml[firstline=9,lastline=34]{../src/backtrace/backtrace.ml}{backtrace/backtrace.ml}
    \end{column}
  \end{columns}
\end{frame}

\begin{frame}{Backtraces}{CMM and disassembly of \funcname{definitely\_raise}}
  \begin{columns}[c]
    \begin{column}{0.5\textwidth}
      \minipage[c][0.66\textheight][s]{\columnwidth}
      \begin{itemize}
        \item<1-> An effect of compiling with debugging information (\funcarg{-g}): the CMM no longer uses \funcname{raise\_notrace} to raise the exception but \funcname{raise} instead
        \item<2-> Disassembly shows that CMM \funcname{raise} is actually a call to \funcname{caml\_raise\_exn}, a function in assembler provided by the runtime
      \end{itemize}
      \vfill
      \mintocaml[firstline=9,lastline=13,highlightlines=12]{../src/backtrace/backtrace.ml}
      \endminipage
    \end{column}
    \begin{column}{0.5\textwidth}
      \onslide*<1>{
        \mintcmm[highlightlines=5]{../src/backtrace/camlBacktrace\_\_definitely\_raise\_347.cmm}
      }
      \onslide*<2>{
        \mintobjdump[firstline=2,highlightlines=14]{../src/backtrace/camlBacktrace\_\_definitely\_raise\_347.objdump}
      }
    \end{column}
  \end{columns}
\end{frame}

\begin{frame}{Backtraces}{Introducing \funcname{caml\_raise\_exn}}
  \begin{columns}[c]
    \begin{column}{0.5\textwidth}
      \minipage[c][0.66\textheight][s]{\columnwidth}
      \onslide*<1>{
        \begin{itemize}
          \item Raising the exception is performed using \funcname{caml\_raise\_exn} which is
            \begin{itemize}
              \item An assembly function provided by the runtime
              \item Architecture specific
            \end{itemize}
          \item Let's see what it does differently from \funcname{raise\_notrace}\dots
          \item We are about to raise \typename{ExnC} with \funcname{caml\_raise\_exn} from \funcname{definitely\_raise}
        \end{itemize}
      }
      \onslide*<2>{
        \mintobjdump[firstline=2,highlightlines=14]{../src/backtrace/camlBacktrace\_\_definitely\_raise\_347.objdump}
      }
      \vfill
      \mintocaml[firstline=9,lastline=13,highlightlines=12]{../src/backtrace/backtrace.ml}
      \endminipage
    \end{column}
    \begin{column}{0.5\textwidth}
      \centering
      \providecommand\step{1}\input{diagrams/backtrace/backtrace.tex}
    \end{column}
  \end{columns}
\end{frame}

\begin{frame}{Backtraces}{But before that, we need more fields from \typename{caml\_domain\_state}}
  \begin{columns}
    \begin{column}{0.5\textwidth}
      \begin{itemize}
        \item Remember \typename{caml\_domain\_state} the per-domain runtime state?
        \item Some other members are of interest to us now\dots
        \item \localname{backtrace\_active} is used to know if the runtime should collect backtraces
        \item \localname{backtrace\_active} can be set by
          \begin{itemize}
            \item The \funcarg{b} flag in the \funcname{OCAMLRUNPARAM} environment variable
            \item \mintinline{ocaml}{Printexc.record_backtrace : bool -> unit} \footnotemark
          \end{itemize}
      \end{itemize}
    \end{column}
    \begin{column}{0.5\textwidth}
      \begin{listing}
        \mintc[highlightlines={9-11}]{src/ocaml/domain\_state\_backtrace.h}
        \caption{runtime/caml/domain\_state.h}
      \end{listing}
    \end{column}
  \end{columns}
  \footnotetext[1]{https://v2.ocaml.org/api/Printexc.html}
\end{frame}

\newcommand\listCamlRaiseExn[1][]{
  \listasm[firstline=702,lastline=725,#1]{../ocaml/runtime/amd64.S}{runtime/amd64.S:702}}

\begin{frame}{Backtraces}{Walk through \funcname{caml\_raise\_exn}}
  \begin{columns}[T]
    \begin{column}{0.5\textwidth}
      \begin{itemize}
        \item<1-> First checks whether exceptions backtrace should be recorded
        \item<1-> By checking the \funcarg{domain\_state->backtrace\_active} value
        \item<2-> If not, acts as \funcname{raise\_notrace} with \funcname{RESTORE\_EXN\_HANDLER\_OCAML}
          \begin{itemize}
            \item Set \regname{RSP} to \localname{domain\_state->exn\_handler} value
            \item Restore \localname{domain\_state->exn\_handler} from trap
            \item Jump with \funcname{ret} (same effect as using \funcname{pop} and \funcname{jmp})
          \end{itemize}
        \item<3-> Otherwise, switches to the C stack to be able to execute C code
        \item<4-> Calls \funcname{caml\_stash\_backtrace}, a C function provided by the runtime\ignorespaces
          \onslide<6->{, with
            \begin{itemize}
              \item \funcarg{exn}: exception value
              \item \funcarg{pc}: code address of the raising point
              \item \funcarg{sp}: stack address of the raising function
              \item \funcarg{trapsp}: stack address of the next handler
            \end{itemize}
          }
      \end{itemize}
      \bigskip
      \mintocaml[firstline=9,lastline=13,highlightlines=12]{../src/backtrace/backtrace.ml}
    \end{column}
    \begin{column}{0.5\textwidth}
      \raggedleft
      \onslide*<1,3-4>{
        \mintc[firstline=190,lastline=192]{../ocaml/runtime/amd64.S}
        \mintc[firstline=234,lastline=237]{../ocaml/runtime/amd64.S}
      }
      \onslide*<2>{
        \mintc[firstline=190,lastline=192,highlightlines={191-192}]{../ocaml/runtime/amd64.S}
        \mintc[firstline=234,lastline=237,highlightlines={235-237}]{../ocaml/runtime/amd64.S}
      }
      \onslide*<1>{\listCamlRaiseExn[highlightlines=705]{}}%
      \onslide*<2>{\listCamlRaiseExn[highlightlines={707-708}]{}}%
      \onslide*<3>{\listCamlRaiseExn[highlightlines={712-714}]{}}%
      \onslide*<4>{\listCamlRaiseExn[highlightlines={715-720}]{}}%
      \onslide*<5>{\providecommand\step{1}\input{diagrams/backtrace/backtrace.tex}}%
      \onslide*<6>{\providecommand\step{2}\input{diagrams/backtrace/backtrace.tex}}%
    \end{column}
  \end{columns}
\end{frame}

\newcommand\listStashBacktrace[1][]{
  \listc[firstline=91,lastline=120,#1]{../ocaml/runtime/backtrace\_nat.c}{runtime/backtrace\_nat.c:91}}

\begin{frame}{Backtraces}{Introducing \funcname{caml\_stash\_backtrace}}
  \begin{columns}[c]
    \begin{column}{0.5\textwidth}
      \minipage[c][0.66\textheight][s]{\columnwidth}
      \begin{itemize}
        \item<1-> A C function provided by the runtime (specific to native code)
        \item<2-> It scans the stack, between the stack pointer at the raising point up to the next trap, in order to collect the names of the detected function frames
          \onslide*<2>{
            \begin{itemize}
              \item And more information, like line number, column number...
            \end{itemize}
          }
        \item<3-> It uses the same technique and information as the Garbage Collector (GC) uses to scan the values live on the stack
          \onslide*<4>{
            \begin{itemize}
              \item \typename{frame\_descr} are at the core of stack unwinding
            \end{itemize}
          }
      \end{itemize}
      \vfill
      \mintocaml[firstline=9,lastline=13,highlightlines=12]{../src/backtrace/backtrace.ml}
      \endminipage
    \end{column}
    \begin{column}{0.5\textwidth}
      \onslide*<1>{\listStashBacktrace[highlightlines=91]{}}
      \onslide*<2>{\listStashBacktrace[highlightlines={91,117-118}]{}}
      \onslide*<3->{\listStashBacktrace[highlightlines={94,106,109-110}]{}}
    \end{column}
  \end{columns}
\end{frame}

\newcommand\listFrameDescr[1][]{
  \listocaml[firstline=111,lastline=124,#1]{../ocaml/asmcomp/emitaux.ml}{asmcomp/emitaux.ml:111}}
\newcommand\listDebugInfo[1][]{
  \listocaml[firstline=38,lastline=49,#1]{../ocaml/lambda/debuginfo.mli}{lambda/debuginfo.mli:38}}

\begin{frame}{Backtraces}{\typename{frame\_descr} and \typename{frame\_debuginfo} in a nutshell}
  \begin{columns}[c]
    \begin{column}{0.5\textwidth}
      \begin{itemize}
        \item<1-> For every return address in an OCaml program, there is a associated \typename{frame\_descr}
        \item<2-> A \typename{frame\_descr} specifies
        \begin{itemize}
          \item<2-> The size of the function stack frame
          \item<3-> Where live values are on the stack (so that the GC can mark them as live and prevent from being swept)
        \end{itemize}
        \item<4-> When compiled with debug information, \typename{frame\_descr} will be linked to a \typename{frame\_debuginfo}
        \begin{itemize}
          \item<5-> Contains information about the position in the source code (file name, line and column number)
        \end{itemize}
      \end{itemize}
    \end{column}
    \begin{column}{0.5\textwidth}
        \onslide*<1>{\listFrameDescr[highlightlines={118-122}]{}}
        \onslide*<2>{\listFrameDescr[highlightlines={120}]{}}
        \onslide*<3>{\listFrameDescr[highlightlines={121}]{}}
        \onslide*<4->{\listFrameDescr[highlightlines={122}]{}}
        \medskip
        \onslide*<1-3>{\listDebugInfo{}}
        \onslide*<4>{\listDebugInfo[highlightlines={38,49}]{}}
        \onslide*<5>{\listDebugInfo[highlightlines={39-46}]{}}
    \end{column}
  \end{columns}
\end{frame}

\begin{frame}{Backtraces}{Scanning the stack with \typename{frame\_descr}}
  \begin{columns}[c]
    \begin{column}{0.5\textwidth}
      \begin{itemize}
        \item Given the return address of the code that raises the exception by calling \funcname{caml\_raise\_exn} (\funcarg{pc} argument of \funcname{caml\_stash\_backtrace})
        \item And the position of the stack pointer (\funcarg{sp} argument of \funcname{caml\_stash\_backtrace})
        \item The frame size given by \typename{frame\_descr}.\funcarg{frame\_size}, allows to find the end of the previous frame
        \item And the return address of the caller of this function
        \bigskip
        \onslide<3->{
        \item Note that traps (here the reddish \funcname{ExnA}, \funcname{ExnB} and \funcname{ExnC} blocks) belong to the frame that installed them, and so the frame size changes when traps are removed
        }
      \end{itemize}
    \end{column}
    \begin{column}{0.5\textwidth}
      \raggedleft
      \onslide*<1>{\providecommand\step{2}\input{diagrams/backtrace/backtrace.tex}}%
      \onslide*<2>{\providecommand\step{1}\input{diagrams/backtrace/scanstack.tex}}%
      \onslide*<3>{\providecommand\step{2}\input{diagrams/backtrace/scanstack.tex}}%
      \onslide*<4>{\providecommand\step{3}\input{diagrams/backtrace/scanstack.tex}}%
      \onslide*<5>{\providecommand\step{4}\input{diagrams/backtrace/scanstack.tex}}%
      \onslide*<6>{\providecommand\step{5}\input{diagrams/backtrace/scanstack.tex}}%
    \end{column}
  \end{columns}
\end{frame}

% Duplicate of previous-previous slide
\begin{frame}{Backtraces}{Continuing on \funcname{caml\_stash\_backtrace}}
  \begin{columns}[c]
    \begin{column}{0.5\textwidth}
      \minipage[c][0.75\textheight][s]{\columnwidth}
      \begin{itemize}
          \color{gray}
        \item A C function provided by the runtime (specific to native code)
        \item It scans the stack, between the stack pointer at the raising point up to the next trap, in order to collect the names of the detected function frames
        \item It uses the same technique and information as the Garbage Collector (GC) uses to scan the values live on the stack
      \end{itemize}
      \begin{itemize}
        \item For each function frame detected, store a pointer to the \typename{frame\_descr} in the \localname{domain\_state->backtrace\_buffer} array, effectively constructing the backtrace
      \end{itemize}
      \onslide<2->{
        \begin{itemize}
            \smallskip
          \item Constructed backtrace:
            \funcname{definitely\_raise},
            \onslide<3->{\funcname{probably\_raise}}
        \end{itemize}
      }
      \vfill
      \mintocaml[firstline=9,lastline=13,highlightlines=12]{../src/backtrace/backtrace.ml}
      \endminipage
    \end{column}
    \begin{column}{0.5\textwidth}
      \raggedleft
      \onslide*<1>{\listStashBacktrace[highlightlines={114-115}]{}}
      \onslide*<2>{\providecommand\step{3}\input{diagrams/backtrace/backtrace.tex}}%
      \onslide*<3>{\providecommand\step{4}\input{diagrams/backtrace/backtrace.tex}}%
    \end{column}
  \end{columns}
\end{frame}

\begin{frame}{Backtraces}{Back to \funcname{caml\_raise\_exn}}
  \begin{columns}[c]
    \begin{column}{0.5\textwidth}
      \minipage[c][0.66\textheight][s]{\columnwidth}
      \begin{itemize}\color{gray}
        \item Checks wheteher exceptions backtrace should be recorded, by checking the \funcarg{domain\_state->backtrace\_active} value
        \item Switches to the C stack to be able to execute C code
        \item Calls \funcname{caml\_stash\_backtrace}, to collect the backtrace up to the next trap
      \end{itemize}
      \onslide*<2->{
        \begin{itemize}
          \item<2-> Executes the exception handler in \funcname{probably\_raise} with \funcname{RESTORE\_EXN\_HANDLER\_OCAML}
            \begin{itemize}
              \item Set \regname{RSP} to \localname{domain\_state->exn\_handler} value
              \item Restore \localname{domain\_state->exn\_handler} from trap
              \item Jump to the handler code with \funcname{ret}
            \end{itemize}
        \end{itemize}
      }
      \vfill
      \onslide*<1>{\mintocaml[firstline=9,lastline=13,highlightlines=12]{../src/backtrace/backtrace.ml}}
      \onslide*<2->{\mintocaml[firstline=15,lastline=21,highlightlines={19-21}]{../src/backtrace/backtrace.ml}}
      \endminipage
    \end{column}
    \begin{column}{0.5\textwidth}
      \centering
      \onslide*<1>{
        \mintc[firstline=190,lastline=192]{../ocaml/runtime/amd64.S}
        \mintc[firstline=234,lastline=237]{../ocaml/runtime/amd64.S}
      }
      \onslide*<2>{
        \mintc[firstline=190,lastline=192,highlightlines={191-192}]{../ocaml/runtime/amd64.S}
        \mintc[firstline=234,lastline=237,highlightlines={235-237}]{../ocaml/runtime/amd64.S}
      }
      \onslide*<1>{\listCamlRaiseExn[highlightlines=720]{}}
      \onslide*<2>{\listCamlRaiseExn[highlightlines=722]{}}
      \onslide*<3->{\providecommand\step{5}\input{diagrams/backtrace/backtrace.tex}}
    \end{column}
  \end{columns}
\end{frame}

\begin{frame}{Backtraces}{\funcname{caml\_probably\_raise}'s handler doesn't catch \typename{ExnC}}
  \begin{columns}[c]
    \begin{column}{0.45\textwidth}
      \minipage[c][0.75\textheight][s]{\columnwidth}
      \begin{itemize}
        \item<1-> \funcname{match \dots with}\footnotemark pattern matching can also match exceptions
        \item<1-> The CMM of \funcname{probably\_raise} shows that this \funcname{match \dots with} is implemented with a pattern matching and a \funcname{try \dots with}
        \item<1-> But this one only matches on \typename{ExnA} exception
        \item<2-> Since the pattern matching fails to catch the exception, \funcname{reraise} is called with our current \typename{ExnC} exception
        \item<3-> Disassembly shows that \funcname{reraise} is actually a call to \funcname{caml\_reraise\_exn}, which is also an assembly function provided by the runtime
      \end{itemize}
      \vfill
      \mintocaml[firstline=15,lastline=21,highlightlines={18-21}]{../src/backtrace/backtrace.ml}
      \endminipage
    \end{column}
    \begin{column}{0.55\textwidth}
      \centering
      \onslide*<1>{\mintcmm[highlightlines={10-18}]{../src/backtrace/camlBacktrace\_\_probably\_raise\_351.cmm}}
      \onslide*<2>{\listcmm[highlightlines=18]{../src/backtrace/camlBacktrace\_\_probably\_raise_351.cmm}{CMM of probably\_raise}}
      \onslide*<3>{\mintobjdump[firstline=5,lastline=35,highlightlines=31]{../src/backtrace/camlBacktrace\_\_probably\_raise_351.objdump}}
    \end{column}
  \end{columns}
  \only<1>{\footnotetext[1]{https://blog.janestreet.com/pattern-matching-and-exception-handling-unite/}}
\end{frame}

\newcommand\listCamlReraiseExn[1][]{
  \listasm[firstline=727,lastline=734,#1]{../ocaml/runtime/amd64.S}{runtime/amd64.S:727}}

\begin{frame}{Backtraces}{Introducing \funcname{caml\_reraise\_exn}}
  \begin{columns}[c]
    \begin{column}{0.5\textwidth}
      \minipage[c][0.66\textheight][s]{\columnwidth}
      \begin{itemize}
        \item<1-> The exception have to be reraised to the next handler
        \item<2-> First, checks if the backtrace should be collected with \funcarg{domain\_state->backtrace\_active}
        \only<3>{
        \begin{itemize}
            \item If not, behaves like a \funcname{raise\_notrace} with \funcname{RESTORE\_EXN\_HANDLER\_OCAML}
        \end{itemize}
        }
        \item<4-> Jumps into \funcname{caml\_raise\_exn}
        \item<5-> Calls \funcname{caml\_stash\_backtrace} again, to scan the next part of the stack: from the trap just pop-ed to the next trap
      \end{itemize}
      \vfill
      \mintocaml[firstline=15,lastline=21,highlightlines={18-21}]{../src/backtrace/backtrace.ml}
      \endminipage
    \end{column}
    \begin{column}{0.5\textwidth}
      \raggedleft
      \onslide*<1>{\listCamlReraiseExn{}}
      \onslide*<2>{\listCamlReraiseExn[highlightlines=729]{}}
      \onslide*<3>{
        \mintc[firstline=190,lastline=192,highlightlines={191-192}]{../ocaml/runtime/amd64.S}
        \mintc[firstline=234,lastline=237,highlightlines={235-237}]{../ocaml/runtime/amd64.S}
        \medskip
        \listCamlReraiseExn[highlightlines=731]{}
      }
      \onslide*<4-5>{
        % TODO refactor with \listCamlReraiseExn
        \mintasm[firstline=727,lastline=734,highlightlines=730]{../ocaml/runtime/amd64.S}
        \medskip
      }
      \onslide*<4>{\listCamlRaiseExn[highlightlines=711]{}}
      \onslide*<5>{\listCamlRaiseExn[highlightlines=720]{}}
    \end{column}
  \end{columns}
\end{frame}

\begin{frame}{Backtraces}{Backtrace collection continues}
  \begin{columns}[c]
    \begin{column}{0.55\textwidth}
      \minipage[c][0.75\textheight][s]{\columnwidth}
      \begin{itemize}
        \item<1-> \funcname{caml\_stash\_backtrace} scans the older part of the stack, up to the next trap
        \item<6-> Controls jumps to the nested exception handler in \funcname{maybe\_raise}, which matches only on \typename{ExnB}
        \item<7-> \funcname{caml\_reraise\_exn} is called again, backtrace collection resumes with \funcname{caml\_stash\_backtrace}
      \end{itemize}
      \medskip
      \begin{itemize}
        \item<2-> Constructed backtrace:
        \begin{itemize}
          \item <2-> \funcname{definitely\_raise}
          \item <2-> \funcname{probably\_raise}
          \item <3-> \funcname{probably\_raise} (re-raise)
          \item <4-> \funcname{maybe\_raise}
          \item <5-> \funcname{main}
          \item <8-> \funcname{main} (re-raise)
        \end{itemize}
      \end{itemize}
      \vfill
      \onslide*<1-5>{\mintocaml[firstline=15,lastline=21]{../src/backtrace/backtrace.ml}}
      \onslide*<6>{\mintocaml[firstline=28,lastline=34,highlightlines=31]{../src/backtrace/backtrace.ml}}
      \onslide*<7->{\mintocaml[firstline=28,lastline=34,highlightlines=33]{../src/backtrace/backtrace.ml}}
      \endminipage
    \end{column}
    \begin{column}{0.45\textwidth}
      \raggedleft
      \onslide*<1>{\providecommand\step{6}\input{diagrams/backtrace/backtrace.tex}}%
      \onslide*<2>{\providecommand\step{6}\input{diagrams/backtrace/backtrace.tex}}%
      \onslide*<3>{\providecommand\step{7}\input{diagrams/backtrace/backtrace.tex}}%
      \onslide*<4>{\providecommand\step{8}\input{diagrams/backtrace/backtrace.tex}}%
      \onslide*<5>{\providecommand\step{9}\input{diagrams/backtrace/backtrace.tex}}%
      \onslide*<6>{\providecommand\step{10}\input{diagrams/backtrace/backtrace.tex}}%
      \onslide*<7>{\providecommand\step{11}\input{diagrams/backtrace/backtrace.tex}}%
      \onslide*<8>{\providecommand\step{12}\input{diagrams/backtrace/backtrace.tex}}%
      \onslide*<9>{\providecommand\step{13}\input{diagrams/backtrace/backtrace.tex}}%
    \end{column}
  \end{columns}
\end{frame}

\begin{frame}{Backtraces}{Printing the backtrace}
  \begin{columns}[c]
    \begin{column}{0.45\textwidth}
      \minipage[c][0.66\textheight][s]{\columnwidth}
      \begin{itemize}
        \item<1-> The exception backtrace can now be printed to \funcarg{stdout} with \funcname{Printexc.print_backtrace}
        \item<2-> First the exception backtrace is retrieved into an OCaml array with \funcname{get_raw_backtrace }
        \item<3-> Which is implemented as the \funcname{caml_get_exception_raw_backtrace} C function in the runtime
        \item<4-> And finally printed with \funcname{print_exception_backtrace}
      \end{itemize}
      \vfill
      \mintocaml[firstline=28,lastline=34,highlightlines=33]{../src/backtrace/backtrace.ml}
      \endminipage
    \end{column}
    \begin{column}{0.55\textwidth}
      \onslide*<1,2>{
        \mintocaml[firstline=100,lastline=101]{../ocaml/stdlib/printexc.ml}
      }
      \only<1>{
        \listocaml[firstline=170,lastline=172]{../ocaml/stdlib/printexc.ml}{stdlib/printexc.ml:170}
      }
      \only<2>{
        \listocaml[firstline=170,lastline=172,highlightlines=172]{../ocaml/stdlib/printexc.ml}{stdlib/printexc.ml:170}
      }
      \only<3>{
        \mintc[firstline=154,lastline=156]{../ocaml/runtime/backtrace.c}
        \listc[firstline=171,lastline=192,highlightlines={184-187}]{../ocaml/runtime/backtrace.c}{runtime/backtrace.c:154}
      }
      \only<4>{
        \listocaml[firstline=155,lastline=165]{../ocaml/stdlib/printexc.ml}{stdlib/printexc.ml:55}
      }
    \end{column}
  \end{columns}
\end{frame}


\subsection{The multiple kinds of raise}
\frameSubsection{}{}

\begin{frame}{Backtraces}{When backtraces are not needed}
  \begin{columns}[c]
    \begin{column}{0.45\textwidth}
      \minipage[c][0.66\textheight][s]{\columnwidth}
      \begin{itemize}
        \item<1-> What if the backtrace for an exception is never needed?
        \item<2-> It's possible to raise the exception explicitly with \funcname{raise\_notrace} to make raising as cheap as possible
        \item<3-> An example of use in the \funcname{dynlink} library of the OCaml compiler
      \end{itemize}
      \vfill
      \onslide<3->{
      \listocaml[firstline=205,lastline=209,highlightlines=207]{../ocaml/otherlibs/dynlink/dynlink\_compilerlibs/misc.ml}{otherlibs/dynlink/dynlink\_compilerlibs/misc.ml:205}
      }
      \endminipage
    \end{column}
    \begin{column}{0.55\textwidth}
    \only<4>{
      \mintcmm[highlightlines=3]{../src/notrace/camlNotrace\_\_fun_347.cmm}
      \listcmm{../src/notrace/camlNotrace\_\_all\_somes\_327.cmm}{all\_somes}
    }
    \onslide*<5->{
      \listobjdump[highlightlines={6-9}]{../src/notrace/camlNotrace\_\_fun_347.objdump}{anonymous function from \funcname{all\_somes}}
    }
    \end{column}
  \end{columns}
\end{frame}

\begin{frame}{Backtraces}{Summary table of the multiple kinds of raise}
  \begin{itemize}
    \item When debugging information is enabled and if the backtrace of an exception isn't needed, it's possible to raise with \funcname{raise\_notrace} for performance
    \item When debugging information is \emph{not} enabled, the compiler optimises \funcname{raise} calls to inline assembly (same effect as using \funcname{raise\_notrace})
  \end{itemize}
  \bigskip
  \centering
  \begin{tabular}{| l | c | c | c | c |}
    \hline
    \parbox{10em}{Debug information \\ (\funcarg{-g} flag)}  & \multicolumn{2}{|c|}{Disabled}                                     & \multicolumn{2}{|c|}{Enabled} \\
    \hline
    Raise kind                                               & \funcname{raise} & \funcname{raise\_notrace}                       & \funcname{raise} & \funcname{raise\_notrace} \\
    \hline
    Compilation                                              & \parbox{8em}{inline assembly\\ (optimization)} & inline assembly   & \funcname{caml\_raise\_exn} & inline assembly \\
    \hline
  \end{tabular}
\end{frame}


%
% Takeaway #5
%
\subsection*{Takeaway \#5}
\frameSubsectionTakeaway{}
\againframe<5>{takeaway}
}%
      \onslide*<2>{\providecommand\step{1}\input{diagrams/backtrace/scanstack.tex}}%
      \onslide*<3>{\providecommand\step{2}\input{diagrams/backtrace/scanstack.tex}}%
      \onslide*<4>{\providecommand\step{3}\input{diagrams/backtrace/scanstack.tex}}%
      \onslide*<5>{\providecommand\step{4}\input{diagrams/backtrace/scanstack.tex}}%
      \onslide*<6>{\providecommand\step{5}\input{diagrams/backtrace/scanstack.tex}}%
    \end{column}
  \end{columns}
\end{frame}

% Duplicate of previous-previous slide
\begin{frame}{Backtraces}{Continuing on \funcname{caml\_stash\_backtrace}}
  \begin{columns}[c]
    \begin{column}{0.5\textwidth}
      \minipage[c][0.75\textheight][s]{\columnwidth}
      \begin{itemize}
          \color{gray}
        \item A C function provided by the runtime (specific to native code)
        \item It scans the stack, between the stack pointer at the raising point up to the next trap, in order to collect the names of the detected function frames
        \item It uses the same technique and information as the Garbage Collector (GC) uses to scan the values live on the stack
      \end{itemize}
      \begin{itemize}
        \item For each function frame detected, store a pointer to the \typename{frame\_descr} in the \localname{domain\_state->backtrace\_buffer} array, effectively constructing the backtrace
      \end{itemize}
      \onslide<2->{
        \begin{itemize}
            \smallskip
          \item Constructed backtrace:
            \funcname{definitely\_raise},
            \onslide<3->{\funcname{probably\_raise}}
        \end{itemize}
      }
      \vfill
      \mintocaml[firstline=9,lastline=13,highlightlines=12]{../src/backtrace/backtrace.ml}
      \endminipage
    \end{column}
    \begin{column}{0.5\textwidth}
      \raggedleft
      \onslide*<1>{\listStashBacktrace[highlightlines={114-115}]{}}
      \onslide*<2>{\providecommand\step{3}%
% Backtraces
%
\subsection{One advantage of raising with debugging information}
\frameSubsection{}{}

\begin{frame}{Backtraces}{Debugging information \& source code}
  \begin{columns}[c]
    \begin{column}{0.5\textwidth}
      \begin{itemize}
        \item Raising an exception is fun and games, but how do we know where the exception originated? $\rightarrow$ With the backtrace associated with the exception!
        \item In order to get a backtrace from the point where an exception was raised, it's necessary to compile with debugging information enabled (\funcarg{-g} flag for the compiler)
        \item Same example as in the `Nested handlers' section
      \end{itemize}
    \end{column}
    \begin{column}{0.5\textwidth}
      \listocaml[firstline=9,lastline=34]{../src/backtrace/backtrace.ml}{backtrace/backtrace.ml}
    \end{column}
  \end{columns}
\end{frame}

\begin{frame}{Backtraces}{CMM and disassembly of \funcname{definitely\_raise}}
  \begin{columns}[c]
    \begin{column}{0.5\textwidth}
      \minipage[c][0.66\textheight][s]{\columnwidth}
      \begin{itemize}
        \item<1-> An effect of compiling with debugging information (\funcarg{-g}): the CMM no longer uses \funcname{raise\_notrace} to raise the exception but \funcname{raise} instead
        \item<2-> Disassembly shows that CMM \funcname{raise} is actually a call to \funcname{caml\_raise\_exn}, a function in assembler provided by the runtime
      \end{itemize}
      \vfill
      \mintocaml[firstline=9,lastline=13,highlightlines=12]{../src/backtrace/backtrace.ml}
      \endminipage
    \end{column}
    \begin{column}{0.5\textwidth}
      \onslide*<1>{
        \mintcmm[highlightlines=5]{../src/backtrace/camlBacktrace\_\_definitely\_raise\_347.cmm}
      }
      \onslide*<2>{
        \mintobjdump[firstline=2,highlightlines=14]{../src/backtrace/camlBacktrace\_\_definitely\_raise\_347.objdump}
      }
    \end{column}
  \end{columns}
\end{frame}

\begin{frame}{Backtraces}{Introducing \funcname{caml\_raise\_exn}}
  \begin{columns}[c]
    \begin{column}{0.5\textwidth}
      \minipage[c][0.66\textheight][s]{\columnwidth}
      \onslide*<1>{
        \begin{itemize}
          \item Raising the exception is performed using \funcname{caml\_raise\_exn} which is
            \begin{itemize}
              \item An assembly function provided by the runtime
              \item Architecture specific
            \end{itemize}
          \item Let's see what it does differently from \funcname{raise\_notrace}\dots
          \item We are about to raise \typename{ExnC} with \funcname{caml\_raise\_exn} from \funcname{definitely\_raise}
        \end{itemize}
      }
      \onslide*<2>{
        \mintobjdump[firstline=2,highlightlines=14]{../src/backtrace/camlBacktrace\_\_definitely\_raise\_347.objdump}
      }
      \vfill
      \mintocaml[firstline=9,lastline=13,highlightlines=12]{../src/backtrace/backtrace.ml}
      \endminipage
    \end{column}
    \begin{column}{0.5\textwidth}
      \centering
      \providecommand\step{1}\input{diagrams/backtrace/backtrace.tex}
    \end{column}
  \end{columns}
\end{frame}

\begin{frame}{Backtraces}{But before that, we need more fields from \typename{caml\_domain\_state}}
  \begin{columns}
    \begin{column}{0.5\textwidth}
      \begin{itemize}
        \item Remember \typename{caml\_domain\_state} the per-domain runtime state?
        \item Some other members are of interest to us now\dots
        \item \localname{backtrace\_active} is used to know if the runtime should collect backtraces
        \item \localname{backtrace\_active} can be set by
          \begin{itemize}
            \item The \funcarg{b} flag in the \funcname{OCAMLRUNPARAM} environment variable
            \item \mintinline{ocaml}{Printexc.record_backtrace : bool -> unit} \footnotemark
          \end{itemize}
      \end{itemize}
    \end{column}
    \begin{column}{0.5\textwidth}
      \begin{listing}
        \mintc[highlightlines={9-11}]{src/ocaml/domain\_state\_backtrace.h}
        \caption{runtime/caml/domain\_state.h}
      \end{listing}
    \end{column}
  \end{columns}
  \footnotetext[1]{https://v2.ocaml.org/api/Printexc.html}
\end{frame}

\newcommand\listCamlRaiseExn[1][]{
  \listasm[firstline=702,lastline=725,#1]{../ocaml/runtime/amd64.S}{runtime/amd64.S:702}}

\begin{frame}{Backtraces}{Walk through \funcname{caml\_raise\_exn}}
  \begin{columns}[T]
    \begin{column}{0.5\textwidth}
      \begin{itemize}
        \item<1-> First checks whether exceptions backtrace should be recorded
        \item<1-> By checking the \funcarg{domain\_state->backtrace\_active} value
        \item<2-> If not, acts as \funcname{raise\_notrace} with \funcname{RESTORE\_EXN\_HANDLER\_OCAML}
          \begin{itemize}
            \item Set \regname{RSP} to \localname{domain\_state->exn\_handler} value
            \item Restore \localname{domain\_state->exn\_handler} from trap
            \item Jump with \funcname{ret} (same effect as using \funcname{pop} and \funcname{jmp})
          \end{itemize}
        \item<3-> Otherwise, switches to the C stack to be able to execute C code
        \item<4-> Calls \funcname{caml\_stash\_backtrace}, a C function provided by the runtime\ignorespaces
          \onslide<6->{, with
            \begin{itemize}
              \item \funcarg{exn}: exception value
              \item \funcarg{pc}: code address of the raising point
              \item \funcarg{sp}: stack address of the raising function
              \item \funcarg{trapsp}: stack address of the next handler
            \end{itemize}
          }
      \end{itemize}
      \bigskip
      \mintocaml[firstline=9,lastline=13,highlightlines=12]{../src/backtrace/backtrace.ml}
    \end{column}
    \begin{column}{0.5\textwidth}
      \raggedleft
      \onslide*<1,3-4>{
        \mintc[firstline=190,lastline=192]{../ocaml/runtime/amd64.S}
        \mintc[firstline=234,lastline=237]{../ocaml/runtime/amd64.S}
      }
      \onslide*<2>{
        \mintc[firstline=190,lastline=192,highlightlines={191-192}]{../ocaml/runtime/amd64.S}
        \mintc[firstline=234,lastline=237,highlightlines={235-237}]{../ocaml/runtime/amd64.S}
      }
      \onslide*<1>{\listCamlRaiseExn[highlightlines=705]{}}%
      \onslide*<2>{\listCamlRaiseExn[highlightlines={707-708}]{}}%
      \onslide*<3>{\listCamlRaiseExn[highlightlines={712-714}]{}}%
      \onslide*<4>{\listCamlRaiseExn[highlightlines={715-720}]{}}%
      \onslide*<5>{\providecommand\step{1}\input{diagrams/backtrace/backtrace.tex}}%
      \onslide*<6>{\providecommand\step{2}\input{diagrams/backtrace/backtrace.tex}}%
    \end{column}
  \end{columns}
\end{frame}

\newcommand\listStashBacktrace[1][]{
  \listc[firstline=91,lastline=120,#1]{../ocaml/runtime/backtrace\_nat.c}{runtime/backtrace\_nat.c:91}}

\begin{frame}{Backtraces}{Introducing \funcname{caml\_stash\_backtrace}}
  \begin{columns}[c]
    \begin{column}{0.5\textwidth}
      \minipage[c][0.66\textheight][s]{\columnwidth}
      \begin{itemize}
        \item<1-> A C function provided by the runtime (specific to native code)
        \item<2-> It scans the stack, between the stack pointer at the raising point up to the next trap, in order to collect the names of the detected function frames
          \onslide*<2>{
            \begin{itemize}
              \item And more information, like line number, column number...
            \end{itemize}
          }
        \item<3-> It uses the same technique and information as the Garbage Collector (GC) uses to scan the values live on the stack
          \onslide*<4>{
            \begin{itemize}
              \item \typename{frame\_descr} are at the core of stack unwinding
            \end{itemize}
          }
      \end{itemize}
      \vfill
      \mintocaml[firstline=9,lastline=13,highlightlines=12]{../src/backtrace/backtrace.ml}
      \endminipage
    \end{column}
    \begin{column}{0.5\textwidth}
      \onslide*<1>{\listStashBacktrace[highlightlines=91]{}}
      \onslide*<2>{\listStashBacktrace[highlightlines={91,117-118}]{}}
      \onslide*<3->{\listStashBacktrace[highlightlines={94,106,109-110}]{}}
    \end{column}
  \end{columns}
\end{frame}

\newcommand\listFrameDescr[1][]{
  \listocaml[firstline=111,lastline=124,#1]{../ocaml/asmcomp/emitaux.ml}{asmcomp/emitaux.ml:111}}
\newcommand\listDebugInfo[1][]{
  \listocaml[firstline=38,lastline=49,#1]{../ocaml/lambda/debuginfo.mli}{lambda/debuginfo.mli:38}}

\begin{frame}{Backtraces}{\typename{frame\_descr} and \typename{frame\_debuginfo} in a nutshell}
  \begin{columns}[c]
    \begin{column}{0.5\textwidth}
      \begin{itemize}
        \item<1-> For every return address in an OCaml program, there is a associated \typename{frame\_descr}
        \item<2-> A \typename{frame\_descr} specifies
        \begin{itemize}
          \item<2-> The size of the function stack frame
          \item<3-> Where live values are on the stack (so that the GC can mark them as live and prevent from being swept)
        \end{itemize}
        \item<4-> When compiled with debug information, \typename{frame\_descr} will be linked to a \typename{frame\_debuginfo}
        \begin{itemize}
          \item<5-> Contains information about the position in the source code (file name, line and column number)
        \end{itemize}
      \end{itemize}
    \end{column}
    \begin{column}{0.5\textwidth}
        \onslide*<1>{\listFrameDescr[highlightlines={118-122}]{}}
        \onslide*<2>{\listFrameDescr[highlightlines={120}]{}}
        \onslide*<3>{\listFrameDescr[highlightlines={121}]{}}
        \onslide*<4->{\listFrameDescr[highlightlines={122}]{}}
        \medskip
        \onslide*<1-3>{\listDebugInfo{}}
        \onslide*<4>{\listDebugInfo[highlightlines={38,49}]{}}
        \onslide*<5>{\listDebugInfo[highlightlines={39-46}]{}}
    \end{column}
  \end{columns}
\end{frame}

\begin{frame}{Backtraces}{Scanning the stack with \typename{frame\_descr}}
  \begin{columns}[c]
    \begin{column}{0.5\textwidth}
      \begin{itemize}
        \item Given the return address of the code that raises the exception by calling \funcname{caml\_raise\_exn} (\funcarg{pc} argument of \funcname{caml\_stash\_backtrace})
        \item And the position of the stack pointer (\funcarg{sp} argument of \funcname{caml\_stash\_backtrace})
        \item The frame size given by \typename{frame\_descr}.\funcarg{frame\_size}, allows to find the end of the previous frame
        \item And the return address of the caller of this function
        \bigskip
        \onslide<3->{
        \item Note that traps (here the reddish \funcname{ExnA}, \funcname{ExnB} and \funcname{ExnC} blocks) belong to the frame that installed them, and so the frame size changes when traps are removed
        }
      \end{itemize}
    \end{column}
    \begin{column}{0.5\textwidth}
      \raggedleft
      \onslide*<1>{\providecommand\step{2}\input{diagrams/backtrace/backtrace.tex}}%
      \onslide*<2>{\providecommand\step{1}\input{diagrams/backtrace/scanstack.tex}}%
      \onslide*<3>{\providecommand\step{2}\input{diagrams/backtrace/scanstack.tex}}%
      \onslide*<4>{\providecommand\step{3}\input{diagrams/backtrace/scanstack.tex}}%
      \onslide*<5>{\providecommand\step{4}\input{diagrams/backtrace/scanstack.tex}}%
      \onslide*<6>{\providecommand\step{5}\input{diagrams/backtrace/scanstack.tex}}%
    \end{column}
  \end{columns}
\end{frame}

% Duplicate of previous-previous slide
\begin{frame}{Backtraces}{Continuing on \funcname{caml\_stash\_backtrace}}
  \begin{columns}[c]
    \begin{column}{0.5\textwidth}
      \minipage[c][0.75\textheight][s]{\columnwidth}
      \begin{itemize}
          \color{gray}
        \item A C function provided by the runtime (specific to native code)
        \item It scans the stack, between the stack pointer at the raising point up to the next trap, in order to collect the names of the detected function frames
        \item It uses the same technique and information as the Garbage Collector (GC) uses to scan the values live on the stack
      \end{itemize}
      \begin{itemize}
        \item For each function frame detected, store a pointer to the \typename{frame\_descr} in the \localname{domain\_state->backtrace\_buffer} array, effectively constructing the backtrace
      \end{itemize}
      \onslide<2->{
        \begin{itemize}
            \smallskip
          \item Constructed backtrace:
            \funcname{definitely\_raise},
            \onslide<3->{\funcname{probably\_raise}}
        \end{itemize}
      }
      \vfill
      \mintocaml[firstline=9,lastline=13,highlightlines=12]{../src/backtrace/backtrace.ml}
      \endminipage
    \end{column}
    \begin{column}{0.5\textwidth}
      \raggedleft
      \onslide*<1>{\listStashBacktrace[highlightlines={114-115}]{}}
      \onslide*<2>{\providecommand\step{3}\input{diagrams/backtrace/backtrace.tex}}%
      \onslide*<3>{\providecommand\step{4}\input{diagrams/backtrace/backtrace.tex}}%
    \end{column}
  \end{columns}
\end{frame}

\begin{frame}{Backtraces}{Back to \funcname{caml\_raise\_exn}}
  \begin{columns}[c]
    \begin{column}{0.5\textwidth}
      \minipage[c][0.66\textheight][s]{\columnwidth}
      \begin{itemize}\color{gray}
        \item Checks wheteher exceptions backtrace should be recorded, by checking the \funcarg{domain\_state->backtrace\_active} value
        \item Switches to the C stack to be able to execute C code
        \item Calls \funcname{caml\_stash\_backtrace}, to collect the backtrace up to the next trap
      \end{itemize}
      \onslide*<2->{
        \begin{itemize}
          \item<2-> Executes the exception handler in \funcname{probably\_raise} with \funcname{RESTORE\_EXN\_HANDLER\_OCAML}
            \begin{itemize}
              \item Set \regname{RSP} to \localname{domain\_state->exn\_handler} value
              \item Restore \localname{domain\_state->exn\_handler} from trap
              \item Jump to the handler code with \funcname{ret}
            \end{itemize}
        \end{itemize}
      }
      \vfill
      \onslide*<1>{\mintocaml[firstline=9,lastline=13,highlightlines=12]{../src/backtrace/backtrace.ml}}
      \onslide*<2->{\mintocaml[firstline=15,lastline=21,highlightlines={19-21}]{../src/backtrace/backtrace.ml}}
      \endminipage
    \end{column}
    \begin{column}{0.5\textwidth}
      \centering
      \onslide*<1>{
        \mintc[firstline=190,lastline=192]{../ocaml/runtime/amd64.S}
        \mintc[firstline=234,lastline=237]{../ocaml/runtime/amd64.S}
      }
      \onslide*<2>{
        \mintc[firstline=190,lastline=192,highlightlines={191-192}]{../ocaml/runtime/amd64.S}
        \mintc[firstline=234,lastline=237,highlightlines={235-237}]{../ocaml/runtime/amd64.S}
      }
      \onslide*<1>{\listCamlRaiseExn[highlightlines=720]{}}
      \onslide*<2>{\listCamlRaiseExn[highlightlines=722]{}}
      \onslide*<3->{\providecommand\step{5}\input{diagrams/backtrace/backtrace.tex}}
    \end{column}
  \end{columns}
\end{frame}

\begin{frame}{Backtraces}{\funcname{caml\_probably\_raise}'s handler doesn't catch \typename{ExnC}}
  \begin{columns}[c]
    \begin{column}{0.45\textwidth}
      \minipage[c][0.75\textheight][s]{\columnwidth}
      \begin{itemize}
        \item<1-> \funcname{match \dots with}\footnotemark pattern matching can also match exceptions
        \item<1-> The CMM of \funcname{probably\_raise} shows that this \funcname{match \dots with} is implemented with a pattern matching and a \funcname{try \dots with}
        \item<1-> But this one only matches on \typename{ExnA} exception
        \item<2-> Since the pattern matching fails to catch the exception, \funcname{reraise} is called with our current \typename{ExnC} exception
        \item<3-> Disassembly shows that \funcname{reraise} is actually a call to \funcname{caml\_reraise\_exn}, which is also an assembly function provided by the runtime
      \end{itemize}
      \vfill
      \mintocaml[firstline=15,lastline=21,highlightlines={18-21}]{../src/backtrace/backtrace.ml}
      \endminipage
    \end{column}
    \begin{column}{0.55\textwidth}
      \centering
      \onslide*<1>{\mintcmm[highlightlines={10-18}]{../src/backtrace/camlBacktrace\_\_probably\_raise\_351.cmm}}
      \onslide*<2>{\listcmm[highlightlines=18]{../src/backtrace/camlBacktrace\_\_probably\_raise_351.cmm}{CMM of probably\_raise}}
      \onslide*<3>{\mintobjdump[firstline=5,lastline=35,highlightlines=31]{../src/backtrace/camlBacktrace\_\_probably\_raise_351.objdump}}
    \end{column}
  \end{columns}
  \only<1>{\footnotetext[1]{https://blog.janestreet.com/pattern-matching-and-exception-handling-unite/}}
\end{frame}

\newcommand\listCamlReraiseExn[1][]{
  \listasm[firstline=727,lastline=734,#1]{../ocaml/runtime/amd64.S}{runtime/amd64.S:727}}

\begin{frame}{Backtraces}{Introducing \funcname{caml\_reraise\_exn}}
  \begin{columns}[c]
    \begin{column}{0.5\textwidth}
      \minipage[c][0.66\textheight][s]{\columnwidth}
      \begin{itemize}
        \item<1-> The exception have to be reraised to the next handler
        \item<2-> First, checks if the backtrace should be collected with \funcarg{domain\_state->backtrace\_active}
        \only<3>{
        \begin{itemize}
            \item If not, behaves like a \funcname{raise\_notrace} with \funcname{RESTORE\_EXN\_HANDLER\_OCAML}
        \end{itemize}
        }
        \item<4-> Jumps into \funcname{caml\_raise\_exn}
        \item<5-> Calls \funcname{caml\_stash\_backtrace} again, to scan the next part of the stack: from the trap just pop-ed to the next trap
      \end{itemize}
      \vfill
      \mintocaml[firstline=15,lastline=21,highlightlines={18-21}]{../src/backtrace/backtrace.ml}
      \endminipage
    \end{column}
    \begin{column}{0.5\textwidth}
      \raggedleft
      \onslide*<1>{\listCamlReraiseExn{}}
      \onslide*<2>{\listCamlReraiseExn[highlightlines=729]{}}
      \onslide*<3>{
        \mintc[firstline=190,lastline=192,highlightlines={191-192}]{../ocaml/runtime/amd64.S}
        \mintc[firstline=234,lastline=237,highlightlines={235-237}]{../ocaml/runtime/amd64.S}
        \medskip
        \listCamlReraiseExn[highlightlines=731]{}
      }
      \onslide*<4-5>{
        % TODO refactor with \listCamlReraiseExn
        \mintasm[firstline=727,lastline=734,highlightlines=730]{../ocaml/runtime/amd64.S}
        \medskip
      }
      \onslide*<4>{\listCamlRaiseExn[highlightlines=711]{}}
      \onslide*<5>{\listCamlRaiseExn[highlightlines=720]{}}
    \end{column}
  \end{columns}
\end{frame}

\begin{frame}{Backtraces}{Backtrace collection continues}
  \begin{columns}[c]
    \begin{column}{0.55\textwidth}
      \minipage[c][0.75\textheight][s]{\columnwidth}
      \begin{itemize}
        \item<1-> \funcname{caml\_stash\_backtrace} scans the older part of the stack, up to the next trap
        \item<6-> Controls jumps to the nested exception handler in \funcname{maybe\_raise}, which matches only on \typename{ExnB}
        \item<7-> \funcname{caml\_reraise\_exn} is called again, backtrace collection resumes with \funcname{caml\_stash\_backtrace}
      \end{itemize}
      \medskip
      \begin{itemize}
        \item<2-> Constructed backtrace:
        \begin{itemize}
          \item <2-> \funcname{definitely\_raise}
          \item <2-> \funcname{probably\_raise}
          \item <3-> \funcname{probably\_raise} (re-raise)
          \item <4-> \funcname{maybe\_raise}
          \item <5-> \funcname{main}
          \item <8-> \funcname{main} (re-raise)
        \end{itemize}
      \end{itemize}
      \vfill
      \onslide*<1-5>{\mintocaml[firstline=15,lastline=21]{../src/backtrace/backtrace.ml}}
      \onslide*<6>{\mintocaml[firstline=28,lastline=34,highlightlines=31]{../src/backtrace/backtrace.ml}}
      \onslide*<7->{\mintocaml[firstline=28,lastline=34,highlightlines=33]{../src/backtrace/backtrace.ml}}
      \endminipage
    \end{column}
    \begin{column}{0.45\textwidth}
      \raggedleft
      \onslide*<1>{\providecommand\step{6}\input{diagrams/backtrace/backtrace.tex}}%
      \onslide*<2>{\providecommand\step{6}\input{diagrams/backtrace/backtrace.tex}}%
      \onslide*<3>{\providecommand\step{7}\input{diagrams/backtrace/backtrace.tex}}%
      \onslide*<4>{\providecommand\step{8}\input{diagrams/backtrace/backtrace.tex}}%
      \onslide*<5>{\providecommand\step{9}\input{diagrams/backtrace/backtrace.tex}}%
      \onslide*<6>{\providecommand\step{10}\input{diagrams/backtrace/backtrace.tex}}%
      \onslide*<7>{\providecommand\step{11}\input{diagrams/backtrace/backtrace.tex}}%
      \onslide*<8>{\providecommand\step{12}\input{diagrams/backtrace/backtrace.tex}}%
      \onslide*<9>{\providecommand\step{13}\input{diagrams/backtrace/backtrace.tex}}%
    \end{column}
  \end{columns}
\end{frame}

\begin{frame}{Backtraces}{Printing the backtrace}
  \begin{columns}[c]
    \begin{column}{0.45\textwidth}
      \minipage[c][0.66\textheight][s]{\columnwidth}
      \begin{itemize}
        \item<1-> The exception backtrace can now be printed to \funcarg{stdout} with \funcname{Printexc.print_backtrace}
        \item<2-> First the exception backtrace is retrieved into an OCaml array with \funcname{get_raw_backtrace }
        \item<3-> Which is implemented as the \funcname{caml_get_exception_raw_backtrace} C function in the runtime
        \item<4-> And finally printed with \funcname{print_exception_backtrace}
      \end{itemize}
      \vfill
      \mintocaml[firstline=28,lastline=34,highlightlines=33]{../src/backtrace/backtrace.ml}
      \endminipage
    \end{column}
    \begin{column}{0.55\textwidth}
      \onslide*<1,2>{
        \mintocaml[firstline=100,lastline=101]{../ocaml/stdlib/printexc.ml}
      }
      \only<1>{
        \listocaml[firstline=170,lastline=172]{../ocaml/stdlib/printexc.ml}{stdlib/printexc.ml:170}
      }
      \only<2>{
        \listocaml[firstline=170,lastline=172,highlightlines=172]{../ocaml/stdlib/printexc.ml}{stdlib/printexc.ml:170}
      }
      \only<3>{
        \mintc[firstline=154,lastline=156]{../ocaml/runtime/backtrace.c}
        \listc[firstline=171,lastline=192,highlightlines={184-187}]{../ocaml/runtime/backtrace.c}{runtime/backtrace.c:154}
      }
      \only<4>{
        \listocaml[firstline=155,lastline=165]{../ocaml/stdlib/printexc.ml}{stdlib/printexc.ml:55}
      }
    \end{column}
  \end{columns}
\end{frame}


\subsection{The multiple kinds of raise}
\frameSubsection{}{}

\begin{frame}{Backtraces}{When backtraces are not needed}
  \begin{columns}[c]
    \begin{column}{0.45\textwidth}
      \minipage[c][0.66\textheight][s]{\columnwidth}
      \begin{itemize}
        \item<1-> What if the backtrace for an exception is never needed?
        \item<2-> It's possible to raise the exception explicitly with \funcname{raise\_notrace} to make raising as cheap as possible
        \item<3-> An example of use in the \funcname{dynlink} library of the OCaml compiler
      \end{itemize}
      \vfill
      \onslide<3->{
      \listocaml[firstline=205,lastline=209,highlightlines=207]{../ocaml/otherlibs/dynlink/dynlink\_compilerlibs/misc.ml}{otherlibs/dynlink/dynlink\_compilerlibs/misc.ml:205}
      }
      \endminipage
    \end{column}
    \begin{column}{0.55\textwidth}
    \only<4>{
      \mintcmm[highlightlines=3]{../src/notrace/camlNotrace\_\_fun_347.cmm}
      \listcmm{../src/notrace/camlNotrace\_\_all\_somes\_327.cmm}{all\_somes}
    }
    \onslide*<5->{
      \listobjdump[highlightlines={6-9}]{../src/notrace/camlNotrace\_\_fun_347.objdump}{anonymous function from \funcname{all\_somes}}
    }
    \end{column}
  \end{columns}
\end{frame}

\begin{frame}{Backtraces}{Summary table of the multiple kinds of raise}
  \begin{itemize}
    \item When debugging information is enabled and if the backtrace of an exception isn't needed, it's possible to raise with \funcname{raise\_notrace} for performance
    \item When debugging information is \emph{not} enabled, the compiler optimises \funcname{raise} calls to inline assembly (same effect as using \funcname{raise\_notrace})
  \end{itemize}
  \bigskip
  \centering
  \begin{tabular}{| l | c | c | c | c |}
    \hline
    \parbox{10em}{Debug information \\ (\funcarg{-g} flag)}  & \multicolumn{2}{|c|}{Disabled}                                     & \multicolumn{2}{|c|}{Enabled} \\
    \hline
    Raise kind                                               & \funcname{raise} & \funcname{raise\_notrace}                       & \funcname{raise} & \funcname{raise\_notrace} \\
    \hline
    Compilation                                              & \parbox{8em}{inline assembly\\ (optimization)} & inline assembly   & \funcname{caml\_raise\_exn} & inline assembly \\
    \hline
  \end{tabular}
\end{frame}


%
% Takeaway #5
%
\subsection*{Takeaway \#5}
\frameSubsectionTakeaway{}
\againframe<5>{takeaway}
}%
      \onslide*<3>{\providecommand\step{4}%
% Backtraces
%
\subsection{One advantage of raising with debugging information}
\frameSubsection{}{}

\begin{frame}{Backtraces}{Debugging information \& source code}
  \begin{columns}[c]
    \begin{column}{0.5\textwidth}
      \begin{itemize}
        \item Raising an exception is fun and games, but how do we know where the exception originated? $\rightarrow$ With the backtrace associated with the exception!
        \item In order to get a backtrace from the point where an exception was raised, it's necessary to compile with debugging information enabled (\funcarg{-g} flag for the compiler)
        \item Same example as in the `Nested handlers' section
      \end{itemize}
    \end{column}
    \begin{column}{0.5\textwidth}
      \listocaml[firstline=9,lastline=34]{../src/backtrace/backtrace.ml}{backtrace/backtrace.ml}
    \end{column}
  \end{columns}
\end{frame}

\begin{frame}{Backtraces}{CMM and disassembly of \funcname{definitely\_raise}}
  \begin{columns}[c]
    \begin{column}{0.5\textwidth}
      \minipage[c][0.66\textheight][s]{\columnwidth}
      \begin{itemize}
        \item<1-> An effect of compiling with debugging information (\funcarg{-g}): the CMM no longer uses \funcname{raise\_notrace} to raise the exception but \funcname{raise} instead
        \item<2-> Disassembly shows that CMM \funcname{raise} is actually a call to \funcname{caml\_raise\_exn}, a function in assembler provided by the runtime
      \end{itemize}
      \vfill
      \mintocaml[firstline=9,lastline=13,highlightlines=12]{../src/backtrace/backtrace.ml}
      \endminipage
    \end{column}
    \begin{column}{0.5\textwidth}
      \onslide*<1>{
        \mintcmm[highlightlines=5]{../src/backtrace/camlBacktrace\_\_definitely\_raise\_347.cmm}
      }
      \onslide*<2>{
        \mintobjdump[firstline=2,highlightlines=14]{../src/backtrace/camlBacktrace\_\_definitely\_raise\_347.objdump}
      }
    \end{column}
  \end{columns}
\end{frame}

\begin{frame}{Backtraces}{Introducing \funcname{caml\_raise\_exn}}
  \begin{columns}[c]
    \begin{column}{0.5\textwidth}
      \minipage[c][0.66\textheight][s]{\columnwidth}
      \onslide*<1>{
        \begin{itemize}
          \item Raising the exception is performed using \funcname{caml\_raise\_exn} which is
            \begin{itemize}
              \item An assembly function provided by the runtime
              \item Architecture specific
            \end{itemize}
          \item Let's see what it does differently from \funcname{raise\_notrace}\dots
          \item We are about to raise \typename{ExnC} with \funcname{caml\_raise\_exn} from \funcname{definitely\_raise}
        \end{itemize}
      }
      \onslide*<2>{
        \mintobjdump[firstline=2,highlightlines=14]{../src/backtrace/camlBacktrace\_\_definitely\_raise\_347.objdump}
      }
      \vfill
      \mintocaml[firstline=9,lastline=13,highlightlines=12]{../src/backtrace/backtrace.ml}
      \endminipage
    \end{column}
    \begin{column}{0.5\textwidth}
      \centering
      \providecommand\step{1}\input{diagrams/backtrace/backtrace.tex}
    \end{column}
  \end{columns}
\end{frame}

\begin{frame}{Backtraces}{But before that, we need more fields from \typename{caml\_domain\_state}}
  \begin{columns}
    \begin{column}{0.5\textwidth}
      \begin{itemize}
        \item Remember \typename{caml\_domain\_state} the per-domain runtime state?
        \item Some other members are of interest to us now\dots
        \item \localname{backtrace\_active} is used to know if the runtime should collect backtraces
        \item \localname{backtrace\_active} can be set by
          \begin{itemize}
            \item The \funcarg{b} flag in the \funcname{OCAMLRUNPARAM} environment variable
            \item \mintinline{ocaml}{Printexc.record_backtrace : bool -> unit} \footnotemark
          \end{itemize}
      \end{itemize}
    \end{column}
    \begin{column}{0.5\textwidth}
      \begin{listing}
        \mintc[highlightlines={9-11}]{src/ocaml/domain\_state\_backtrace.h}
        \caption{runtime/caml/domain\_state.h}
      \end{listing}
    \end{column}
  \end{columns}
  \footnotetext[1]{https://v2.ocaml.org/api/Printexc.html}
\end{frame}

\newcommand\listCamlRaiseExn[1][]{
  \listasm[firstline=702,lastline=725,#1]{../ocaml/runtime/amd64.S}{runtime/amd64.S:702}}

\begin{frame}{Backtraces}{Walk through \funcname{caml\_raise\_exn}}
  \begin{columns}[T]
    \begin{column}{0.5\textwidth}
      \begin{itemize}
        \item<1-> First checks whether exceptions backtrace should be recorded
        \item<1-> By checking the \funcarg{domain\_state->backtrace\_active} value
        \item<2-> If not, acts as \funcname{raise\_notrace} with \funcname{RESTORE\_EXN\_HANDLER\_OCAML}
          \begin{itemize}
            \item Set \regname{RSP} to \localname{domain\_state->exn\_handler} value
            \item Restore \localname{domain\_state->exn\_handler} from trap
            \item Jump with \funcname{ret} (same effect as using \funcname{pop} and \funcname{jmp})
          \end{itemize}
        \item<3-> Otherwise, switches to the C stack to be able to execute C code
        \item<4-> Calls \funcname{caml\_stash\_backtrace}, a C function provided by the runtime\ignorespaces
          \onslide<6->{, with
            \begin{itemize}
              \item \funcarg{exn}: exception value
              \item \funcarg{pc}: code address of the raising point
              \item \funcarg{sp}: stack address of the raising function
              \item \funcarg{trapsp}: stack address of the next handler
            \end{itemize}
          }
      \end{itemize}
      \bigskip
      \mintocaml[firstline=9,lastline=13,highlightlines=12]{../src/backtrace/backtrace.ml}
    \end{column}
    \begin{column}{0.5\textwidth}
      \raggedleft
      \onslide*<1,3-4>{
        \mintc[firstline=190,lastline=192]{../ocaml/runtime/amd64.S}
        \mintc[firstline=234,lastline=237]{../ocaml/runtime/amd64.S}
      }
      \onslide*<2>{
        \mintc[firstline=190,lastline=192,highlightlines={191-192}]{../ocaml/runtime/amd64.S}
        \mintc[firstline=234,lastline=237,highlightlines={235-237}]{../ocaml/runtime/amd64.S}
      }
      \onslide*<1>{\listCamlRaiseExn[highlightlines=705]{}}%
      \onslide*<2>{\listCamlRaiseExn[highlightlines={707-708}]{}}%
      \onslide*<3>{\listCamlRaiseExn[highlightlines={712-714}]{}}%
      \onslide*<4>{\listCamlRaiseExn[highlightlines={715-720}]{}}%
      \onslide*<5>{\providecommand\step{1}\input{diagrams/backtrace/backtrace.tex}}%
      \onslide*<6>{\providecommand\step{2}\input{diagrams/backtrace/backtrace.tex}}%
    \end{column}
  \end{columns}
\end{frame}

\newcommand\listStashBacktrace[1][]{
  \listc[firstline=91,lastline=120,#1]{../ocaml/runtime/backtrace\_nat.c}{runtime/backtrace\_nat.c:91}}

\begin{frame}{Backtraces}{Introducing \funcname{caml\_stash\_backtrace}}
  \begin{columns}[c]
    \begin{column}{0.5\textwidth}
      \minipage[c][0.66\textheight][s]{\columnwidth}
      \begin{itemize}
        \item<1-> A C function provided by the runtime (specific to native code)
        \item<2-> It scans the stack, between the stack pointer at the raising point up to the next trap, in order to collect the names of the detected function frames
          \onslide*<2>{
            \begin{itemize}
              \item And more information, like line number, column number...
            \end{itemize}
          }
        \item<3-> It uses the same technique and information as the Garbage Collector (GC) uses to scan the values live on the stack
          \onslide*<4>{
            \begin{itemize}
              \item \typename{frame\_descr} are at the core of stack unwinding
            \end{itemize}
          }
      \end{itemize}
      \vfill
      \mintocaml[firstline=9,lastline=13,highlightlines=12]{../src/backtrace/backtrace.ml}
      \endminipage
    \end{column}
    \begin{column}{0.5\textwidth}
      \onslide*<1>{\listStashBacktrace[highlightlines=91]{}}
      \onslide*<2>{\listStashBacktrace[highlightlines={91,117-118}]{}}
      \onslide*<3->{\listStashBacktrace[highlightlines={94,106,109-110}]{}}
    \end{column}
  \end{columns}
\end{frame}

\newcommand\listFrameDescr[1][]{
  \listocaml[firstline=111,lastline=124,#1]{../ocaml/asmcomp/emitaux.ml}{asmcomp/emitaux.ml:111}}
\newcommand\listDebugInfo[1][]{
  \listocaml[firstline=38,lastline=49,#1]{../ocaml/lambda/debuginfo.mli}{lambda/debuginfo.mli:38}}

\begin{frame}{Backtraces}{\typename{frame\_descr} and \typename{frame\_debuginfo} in a nutshell}
  \begin{columns}[c]
    \begin{column}{0.5\textwidth}
      \begin{itemize}
        \item<1-> For every return address in an OCaml program, there is a associated \typename{frame\_descr}
        \item<2-> A \typename{frame\_descr} specifies
        \begin{itemize}
          \item<2-> The size of the function stack frame
          \item<3-> Where live values are on the stack (so that the GC can mark them as live and prevent from being swept)
        \end{itemize}
        \item<4-> When compiled with debug information, \typename{frame\_descr} will be linked to a \typename{frame\_debuginfo}
        \begin{itemize}
          \item<5-> Contains information about the position in the source code (file name, line and column number)
        \end{itemize}
      \end{itemize}
    \end{column}
    \begin{column}{0.5\textwidth}
        \onslide*<1>{\listFrameDescr[highlightlines={118-122}]{}}
        \onslide*<2>{\listFrameDescr[highlightlines={120}]{}}
        \onslide*<3>{\listFrameDescr[highlightlines={121}]{}}
        \onslide*<4->{\listFrameDescr[highlightlines={122}]{}}
        \medskip
        \onslide*<1-3>{\listDebugInfo{}}
        \onslide*<4>{\listDebugInfo[highlightlines={38,49}]{}}
        \onslide*<5>{\listDebugInfo[highlightlines={39-46}]{}}
    \end{column}
  \end{columns}
\end{frame}

\begin{frame}{Backtraces}{Scanning the stack with \typename{frame\_descr}}
  \begin{columns}[c]
    \begin{column}{0.5\textwidth}
      \begin{itemize}
        \item Given the return address of the code that raises the exception by calling \funcname{caml\_raise\_exn} (\funcarg{pc} argument of \funcname{caml\_stash\_backtrace})
        \item And the position of the stack pointer (\funcarg{sp} argument of \funcname{caml\_stash\_backtrace})
        \item The frame size given by \typename{frame\_descr}.\funcarg{frame\_size}, allows to find the end of the previous frame
        \item And the return address of the caller of this function
        \bigskip
        \onslide<3->{
        \item Note that traps (here the reddish \funcname{ExnA}, \funcname{ExnB} and \funcname{ExnC} blocks) belong to the frame that installed them, and so the frame size changes when traps are removed
        }
      \end{itemize}
    \end{column}
    \begin{column}{0.5\textwidth}
      \raggedleft
      \onslide*<1>{\providecommand\step{2}\input{diagrams/backtrace/backtrace.tex}}%
      \onslide*<2>{\providecommand\step{1}\input{diagrams/backtrace/scanstack.tex}}%
      \onslide*<3>{\providecommand\step{2}\input{diagrams/backtrace/scanstack.tex}}%
      \onslide*<4>{\providecommand\step{3}\input{diagrams/backtrace/scanstack.tex}}%
      \onslide*<5>{\providecommand\step{4}\input{diagrams/backtrace/scanstack.tex}}%
      \onslide*<6>{\providecommand\step{5}\input{diagrams/backtrace/scanstack.tex}}%
    \end{column}
  \end{columns}
\end{frame}

% Duplicate of previous-previous slide
\begin{frame}{Backtraces}{Continuing on \funcname{caml\_stash\_backtrace}}
  \begin{columns}[c]
    \begin{column}{0.5\textwidth}
      \minipage[c][0.75\textheight][s]{\columnwidth}
      \begin{itemize}
          \color{gray}
        \item A C function provided by the runtime (specific to native code)
        \item It scans the stack, between the stack pointer at the raising point up to the next trap, in order to collect the names of the detected function frames
        \item It uses the same technique and information as the Garbage Collector (GC) uses to scan the values live on the stack
      \end{itemize}
      \begin{itemize}
        \item For each function frame detected, store a pointer to the \typename{frame\_descr} in the \localname{domain\_state->backtrace\_buffer} array, effectively constructing the backtrace
      \end{itemize}
      \onslide<2->{
        \begin{itemize}
            \smallskip
          \item Constructed backtrace:
            \funcname{definitely\_raise},
            \onslide<3->{\funcname{probably\_raise}}
        \end{itemize}
      }
      \vfill
      \mintocaml[firstline=9,lastline=13,highlightlines=12]{../src/backtrace/backtrace.ml}
      \endminipage
    \end{column}
    \begin{column}{0.5\textwidth}
      \raggedleft
      \onslide*<1>{\listStashBacktrace[highlightlines={114-115}]{}}
      \onslide*<2>{\providecommand\step{3}\input{diagrams/backtrace/backtrace.tex}}%
      \onslide*<3>{\providecommand\step{4}\input{diagrams/backtrace/backtrace.tex}}%
    \end{column}
  \end{columns}
\end{frame}

\begin{frame}{Backtraces}{Back to \funcname{caml\_raise\_exn}}
  \begin{columns}[c]
    \begin{column}{0.5\textwidth}
      \minipage[c][0.66\textheight][s]{\columnwidth}
      \begin{itemize}\color{gray}
        \item Checks wheteher exceptions backtrace should be recorded, by checking the \funcarg{domain\_state->backtrace\_active} value
        \item Switches to the C stack to be able to execute C code
        \item Calls \funcname{caml\_stash\_backtrace}, to collect the backtrace up to the next trap
      \end{itemize}
      \onslide*<2->{
        \begin{itemize}
          \item<2-> Executes the exception handler in \funcname{probably\_raise} with \funcname{RESTORE\_EXN\_HANDLER\_OCAML}
            \begin{itemize}
              \item Set \regname{RSP} to \localname{domain\_state->exn\_handler} value
              \item Restore \localname{domain\_state->exn\_handler} from trap
              \item Jump to the handler code with \funcname{ret}
            \end{itemize}
        \end{itemize}
      }
      \vfill
      \onslide*<1>{\mintocaml[firstline=9,lastline=13,highlightlines=12]{../src/backtrace/backtrace.ml}}
      \onslide*<2->{\mintocaml[firstline=15,lastline=21,highlightlines={19-21}]{../src/backtrace/backtrace.ml}}
      \endminipage
    \end{column}
    \begin{column}{0.5\textwidth}
      \centering
      \onslide*<1>{
        \mintc[firstline=190,lastline=192]{../ocaml/runtime/amd64.S}
        \mintc[firstline=234,lastline=237]{../ocaml/runtime/amd64.S}
      }
      \onslide*<2>{
        \mintc[firstline=190,lastline=192,highlightlines={191-192}]{../ocaml/runtime/amd64.S}
        \mintc[firstline=234,lastline=237,highlightlines={235-237}]{../ocaml/runtime/amd64.S}
      }
      \onslide*<1>{\listCamlRaiseExn[highlightlines=720]{}}
      \onslide*<2>{\listCamlRaiseExn[highlightlines=722]{}}
      \onslide*<3->{\providecommand\step{5}\input{diagrams/backtrace/backtrace.tex}}
    \end{column}
  \end{columns}
\end{frame}

\begin{frame}{Backtraces}{\funcname{caml\_probably\_raise}'s handler doesn't catch \typename{ExnC}}
  \begin{columns}[c]
    \begin{column}{0.45\textwidth}
      \minipage[c][0.75\textheight][s]{\columnwidth}
      \begin{itemize}
        \item<1-> \funcname{match \dots with}\footnotemark pattern matching can also match exceptions
        \item<1-> The CMM of \funcname{probably\_raise} shows that this \funcname{match \dots with} is implemented with a pattern matching and a \funcname{try \dots with}
        \item<1-> But this one only matches on \typename{ExnA} exception
        \item<2-> Since the pattern matching fails to catch the exception, \funcname{reraise} is called with our current \typename{ExnC} exception
        \item<3-> Disassembly shows that \funcname{reraise} is actually a call to \funcname{caml\_reraise\_exn}, which is also an assembly function provided by the runtime
      \end{itemize}
      \vfill
      \mintocaml[firstline=15,lastline=21,highlightlines={18-21}]{../src/backtrace/backtrace.ml}
      \endminipage
    \end{column}
    \begin{column}{0.55\textwidth}
      \centering
      \onslide*<1>{\mintcmm[highlightlines={10-18}]{../src/backtrace/camlBacktrace\_\_probably\_raise\_351.cmm}}
      \onslide*<2>{\listcmm[highlightlines=18]{../src/backtrace/camlBacktrace\_\_probably\_raise_351.cmm}{CMM of probably\_raise}}
      \onslide*<3>{\mintobjdump[firstline=5,lastline=35,highlightlines=31]{../src/backtrace/camlBacktrace\_\_probably\_raise_351.objdump}}
    \end{column}
  \end{columns}
  \only<1>{\footnotetext[1]{https://blog.janestreet.com/pattern-matching-and-exception-handling-unite/}}
\end{frame}

\newcommand\listCamlReraiseExn[1][]{
  \listasm[firstline=727,lastline=734,#1]{../ocaml/runtime/amd64.S}{runtime/amd64.S:727}}

\begin{frame}{Backtraces}{Introducing \funcname{caml\_reraise\_exn}}
  \begin{columns}[c]
    \begin{column}{0.5\textwidth}
      \minipage[c][0.66\textheight][s]{\columnwidth}
      \begin{itemize}
        \item<1-> The exception have to be reraised to the next handler
        \item<2-> First, checks if the backtrace should be collected with \funcarg{domain\_state->backtrace\_active}
        \only<3>{
        \begin{itemize}
            \item If not, behaves like a \funcname{raise\_notrace} with \funcname{RESTORE\_EXN\_HANDLER\_OCAML}
        \end{itemize}
        }
        \item<4-> Jumps into \funcname{caml\_raise\_exn}
        \item<5-> Calls \funcname{caml\_stash\_backtrace} again, to scan the next part of the stack: from the trap just pop-ed to the next trap
      \end{itemize}
      \vfill
      \mintocaml[firstline=15,lastline=21,highlightlines={18-21}]{../src/backtrace/backtrace.ml}
      \endminipage
    \end{column}
    \begin{column}{0.5\textwidth}
      \raggedleft
      \onslide*<1>{\listCamlReraiseExn{}}
      \onslide*<2>{\listCamlReraiseExn[highlightlines=729]{}}
      \onslide*<3>{
        \mintc[firstline=190,lastline=192,highlightlines={191-192}]{../ocaml/runtime/amd64.S}
        \mintc[firstline=234,lastline=237,highlightlines={235-237}]{../ocaml/runtime/amd64.S}
        \medskip
        \listCamlReraiseExn[highlightlines=731]{}
      }
      \onslide*<4-5>{
        % TODO refactor with \listCamlReraiseExn
        \mintasm[firstline=727,lastline=734,highlightlines=730]{../ocaml/runtime/amd64.S}
        \medskip
      }
      \onslide*<4>{\listCamlRaiseExn[highlightlines=711]{}}
      \onslide*<5>{\listCamlRaiseExn[highlightlines=720]{}}
    \end{column}
  \end{columns}
\end{frame}

\begin{frame}{Backtraces}{Backtrace collection continues}
  \begin{columns}[c]
    \begin{column}{0.55\textwidth}
      \minipage[c][0.75\textheight][s]{\columnwidth}
      \begin{itemize}
        \item<1-> \funcname{caml\_stash\_backtrace} scans the older part of the stack, up to the next trap
        \item<6-> Controls jumps to the nested exception handler in \funcname{maybe\_raise}, which matches only on \typename{ExnB}
        \item<7-> \funcname{caml\_reraise\_exn} is called again, backtrace collection resumes with \funcname{caml\_stash\_backtrace}
      \end{itemize}
      \medskip
      \begin{itemize}
        \item<2-> Constructed backtrace:
        \begin{itemize}
          \item <2-> \funcname{definitely\_raise}
          \item <2-> \funcname{probably\_raise}
          \item <3-> \funcname{probably\_raise} (re-raise)
          \item <4-> \funcname{maybe\_raise}
          \item <5-> \funcname{main}
          \item <8-> \funcname{main} (re-raise)
        \end{itemize}
      \end{itemize}
      \vfill
      \onslide*<1-5>{\mintocaml[firstline=15,lastline=21]{../src/backtrace/backtrace.ml}}
      \onslide*<6>{\mintocaml[firstline=28,lastline=34,highlightlines=31]{../src/backtrace/backtrace.ml}}
      \onslide*<7->{\mintocaml[firstline=28,lastline=34,highlightlines=33]{../src/backtrace/backtrace.ml}}
      \endminipage
    \end{column}
    \begin{column}{0.45\textwidth}
      \raggedleft
      \onslide*<1>{\providecommand\step{6}\input{diagrams/backtrace/backtrace.tex}}%
      \onslide*<2>{\providecommand\step{6}\input{diagrams/backtrace/backtrace.tex}}%
      \onslide*<3>{\providecommand\step{7}\input{diagrams/backtrace/backtrace.tex}}%
      \onslide*<4>{\providecommand\step{8}\input{diagrams/backtrace/backtrace.tex}}%
      \onslide*<5>{\providecommand\step{9}\input{diagrams/backtrace/backtrace.tex}}%
      \onslide*<6>{\providecommand\step{10}\input{diagrams/backtrace/backtrace.tex}}%
      \onslide*<7>{\providecommand\step{11}\input{diagrams/backtrace/backtrace.tex}}%
      \onslide*<8>{\providecommand\step{12}\input{diagrams/backtrace/backtrace.tex}}%
      \onslide*<9>{\providecommand\step{13}\input{diagrams/backtrace/backtrace.tex}}%
    \end{column}
  \end{columns}
\end{frame}

\begin{frame}{Backtraces}{Printing the backtrace}
  \begin{columns}[c]
    \begin{column}{0.45\textwidth}
      \minipage[c][0.66\textheight][s]{\columnwidth}
      \begin{itemize}
        \item<1-> The exception backtrace can now be printed to \funcarg{stdout} with \funcname{Printexc.print_backtrace}
        \item<2-> First the exception backtrace is retrieved into an OCaml array with \funcname{get_raw_backtrace }
        \item<3-> Which is implemented as the \funcname{caml_get_exception_raw_backtrace} C function in the runtime
        \item<4-> And finally printed with \funcname{print_exception_backtrace}
      \end{itemize}
      \vfill
      \mintocaml[firstline=28,lastline=34,highlightlines=33]{../src/backtrace/backtrace.ml}
      \endminipage
    \end{column}
    \begin{column}{0.55\textwidth}
      \onslide*<1,2>{
        \mintocaml[firstline=100,lastline=101]{../ocaml/stdlib/printexc.ml}
      }
      \only<1>{
        \listocaml[firstline=170,lastline=172]{../ocaml/stdlib/printexc.ml}{stdlib/printexc.ml:170}
      }
      \only<2>{
        \listocaml[firstline=170,lastline=172,highlightlines=172]{../ocaml/stdlib/printexc.ml}{stdlib/printexc.ml:170}
      }
      \only<3>{
        \mintc[firstline=154,lastline=156]{../ocaml/runtime/backtrace.c}
        \listc[firstline=171,lastline=192,highlightlines={184-187}]{../ocaml/runtime/backtrace.c}{runtime/backtrace.c:154}
      }
      \only<4>{
        \listocaml[firstline=155,lastline=165]{../ocaml/stdlib/printexc.ml}{stdlib/printexc.ml:55}
      }
    \end{column}
  \end{columns}
\end{frame}


\subsection{The multiple kinds of raise}
\frameSubsection{}{}

\begin{frame}{Backtraces}{When backtraces are not needed}
  \begin{columns}[c]
    \begin{column}{0.45\textwidth}
      \minipage[c][0.66\textheight][s]{\columnwidth}
      \begin{itemize}
        \item<1-> What if the backtrace for an exception is never needed?
        \item<2-> It's possible to raise the exception explicitly with \funcname{raise\_notrace} to make raising as cheap as possible
        \item<3-> An example of use in the \funcname{dynlink} library of the OCaml compiler
      \end{itemize}
      \vfill
      \onslide<3->{
      \listocaml[firstline=205,lastline=209,highlightlines=207]{../ocaml/otherlibs/dynlink/dynlink\_compilerlibs/misc.ml}{otherlibs/dynlink/dynlink\_compilerlibs/misc.ml:205}
      }
      \endminipage
    \end{column}
    \begin{column}{0.55\textwidth}
    \only<4>{
      \mintcmm[highlightlines=3]{../src/notrace/camlNotrace\_\_fun_347.cmm}
      \listcmm{../src/notrace/camlNotrace\_\_all\_somes\_327.cmm}{all\_somes}
    }
    \onslide*<5->{
      \listobjdump[highlightlines={6-9}]{../src/notrace/camlNotrace\_\_fun_347.objdump}{anonymous function from \funcname{all\_somes}}
    }
    \end{column}
  \end{columns}
\end{frame}

\begin{frame}{Backtraces}{Summary table of the multiple kinds of raise}
  \begin{itemize}
    \item When debugging information is enabled and if the backtrace of an exception isn't needed, it's possible to raise with \funcname{raise\_notrace} for performance
    \item When debugging information is \emph{not} enabled, the compiler optimises \funcname{raise} calls to inline assembly (same effect as using \funcname{raise\_notrace})
  \end{itemize}
  \bigskip
  \centering
  \begin{tabular}{| l | c | c | c | c |}
    \hline
    \parbox{10em}{Debug information \\ (\funcarg{-g} flag)}  & \multicolumn{2}{|c|}{Disabled}                                     & \multicolumn{2}{|c|}{Enabled} \\
    \hline
    Raise kind                                               & \funcname{raise} & \funcname{raise\_notrace}                       & \funcname{raise} & \funcname{raise\_notrace} \\
    \hline
    Compilation                                              & \parbox{8em}{inline assembly\\ (optimization)} & inline assembly   & \funcname{caml\_raise\_exn} & inline assembly \\
    \hline
  \end{tabular}
\end{frame}


%
% Takeaway #5
%
\subsection*{Takeaway \#5}
\frameSubsectionTakeaway{}
\againframe<5>{takeaway}
}%
    \end{column}
  \end{columns}
\end{frame}

\begin{frame}{Backtraces}{Back to \funcname{caml\_raise\_exn}}
  \begin{columns}[c]
    \begin{column}{0.5\textwidth}
      \minipage[c][0.66\textheight][s]{\columnwidth}
      \begin{itemize}\color{gray}
        \item Checks wheteher exceptions backtrace should be recorded, by checking the \funcarg{domain\_state->backtrace\_active} value
        \item Switches to the C stack to be able to execute C code
        \item Calls \funcname{caml\_stash\_backtrace}, to collect the backtrace up to the next trap
      \end{itemize}
      \onslide*<2->{
        \begin{itemize}
          \item<2-> Executes the exception handler in \funcname{probably\_raise} with \funcname{RESTORE\_EXN\_HANDLER\_OCAML}
            \begin{itemize}
              \item Set \regname{RSP} to \localname{domain\_state->exn\_handler} value
              \item Restore \localname{domain\_state->exn\_handler} from trap
              \item Jump to the handler code with \funcname{ret}
            \end{itemize}
        \end{itemize}
      }
      \vfill
      \onslide*<1>{\mintocaml[firstline=9,lastline=13,highlightlines=12]{../src/backtrace/backtrace.ml}}
      \onslide*<2->{\mintocaml[firstline=15,lastline=21,highlightlines={19-21}]{../src/backtrace/backtrace.ml}}
      \endminipage
    \end{column}
    \begin{column}{0.5\textwidth}
      \centering
      \onslide*<1>{
        \mintc[firstline=190,lastline=192]{../ocaml/runtime/amd64.S}
        \mintc[firstline=234,lastline=237]{../ocaml/runtime/amd64.S}
      }
      \onslide*<2>{
        \mintc[firstline=190,lastline=192,highlightlines={191-192}]{../ocaml/runtime/amd64.S}
        \mintc[firstline=234,lastline=237,highlightlines={235-237}]{../ocaml/runtime/amd64.S}
      }
      \onslide*<1>{\listCamlRaiseExn[highlightlines=720]{}}
      \onslide*<2>{\listCamlRaiseExn[highlightlines=722]{}}
      \onslide*<3->{\providecommand\step{5}%
% Backtraces
%
\subsection{One advantage of raising with debugging information}
\frameSubsection{}{}

\begin{frame}{Backtraces}{Debugging information \& source code}
  \begin{columns}[c]
    \begin{column}{0.5\textwidth}
      \begin{itemize}
        \item Raising an exception is fun and games, but how do we know where the exception originated? $\rightarrow$ With the backtrace associated with the exception!
        \item In order to get a backtrace from the point where an exception was raised, it's necessary to compile with debugging information enabled (\funcarg{-g} flag for the compiler)
        \item Same example as in the `Nested handlers' section
      \end{itemize}
    \end{column}
    \begin{column}{0.5\textwidth}
      \listocaml[firstline=9,lastline=34]{../src/backtrace/backtrace.ml}{backtrace/backtrace.ml}
    \end{column}
  \end{columns}
\end{frame}

\begin{frame}{Backtraces}{CMM and disassembly of \funcname{definitely\_raise}}
  \begin{columns}[c]
    \begin{column}{0.5\textwidth}
      \minipage[c][0.66\textheight][s]{\columnwidth}
      \begin{itemize}
        \item<1-> An effect of compiling with debugging information (\funcarg{-g}): the CMM no longer uses \funcname{raise\_notrace} to raise the exception but \funcname{raise} instead
        \item<2-> Disassembly shows that CMM \funcname{raise} is actually a call to \funcname{caml\_raise\_exn}, a function in assembler provided by the runtime
      \end{itemize}
      \vfill
      \mintocaml[firstline=9,lastline=13,highlightlines=12]{../src/backtrace/backtrace.ml}
      \endminipage
    \end{column}
    \begin{column}{0.5\textwidth}
      \onslide*<1>{
        \mintcmm[highlightlines=5]{../src/backtrace/camlBacktrace\_\_definitely\_raise\_347.cmm}
      }
      \onslide*<2>{
        \mintobjdump[firstline=2,highlightlines=14]{../src/backtrace/camlBacktrace\_\_definitely\_raise\_347.objdump}
      }
    \end{column}
  \end{columns}
\end{frame}

\begin{frame}{Backtraces}{Introducing \funcname{caml\_raise\_exn}}
  \begin{columns}[c]
    \begin{column}{0.5\textwidth}
      \minipage[c][0.66\textheight][s]{\columnwidth}
      \onslide*<1>{
        \begin{itemize}
          \item Raising the exception is performed using \funcname{caml\_raise\_exn} which is
            \begin{itemize}
              \item An assembly function provided by the runtime
              \item Architecture specific
            \end{itemize}
          \item Let's see what it does differently from \funcname{raise\_notrace}\dots
          \item We are about to raise \typename{ExnC} with \funcname{caml\_raise\_exn} from \funcname{definitely\_raise}
        \end{itemize}
      }
      \onslide*<2>{
        \mintobjdump[firstline=2,highlightlines=14]{../src/backtrace/camlBacktrace\_\_definitely\_raise\_347.objdump}
      }
      \vfill
      \mintocaml[firstline=9,lastline=13,highlightlines=12]{../src/backtrace/backtrace.ml}
      \endminipage
    \end{column}
    \begin{column}{0.5\textwidth}
      \centering
      \providecommand\step{1}\input{diagrams/backtrace/backtrace.tex}
    \end{column}
  \end{columns}
\end{frame}

\begin{frame}{Backtraces}{But before that, we need more fields from \typename{caml\_domain\_state}}
  \begin{columns}
    \begin{column}{0.5\textwidth}
      \begin{itemize}
        \item Remember \typename{caml\_domain\_state} the per-domain runtime state?
        \item Some other members are of interest to us now\dots
        \item \localname{backtrace\_active} is used to know if the runtime should collect backtraces
        \item \localname{backtrace\_active} can be set by
          \begin{itemize}
            \item The \funcarg{b} flag in the \funcname{OCAMLRUNPARAM} environment variable
            \item \mintinline{ocaml}{Printexc.record_backtrace : bool -> unit} \footnotemark
          \end{itemize}
      \end{itemize}
    \end{column}
    \begin{column}{0.5\textwidth}
      \begin{listing}
        \mintc[highlightlines={9-11}]{src/ocaml/domain\_state\_backtrace.h}
        \caption{runtime/caml/domain\_state.h}
      \end{listing}
    \end{column}
  \end{columns}
  \footnotetext[1]{https://v2.ocaml.org/api/Printexc.html}
\end{frame}

\newcommand\listCamlRaiseExn[1][]{
  \listasm[firstline=702,lastline=725,#1]{../ocaml/runtime/amd64.S}{runtime/amd64.S:702}}

\begin{frame}{Backtraces}{Walk through \funcname{caml\_raise\_exn}}
  \begin{columns}[T]
    \begin{column}{0.5\textwidth}
      \begin{itemize}
        \item<1-> First checks whether exceptions backtrace should be recorded
        \item<1-> By checking the \funcarg{domain\_state->backtrace\_active} value
        \item<2-> If not, acts as \funcname{raise\_notrace} with \funcname{RESTORE\_EXN\_HANDLER\_OCAML}
          \begin{itemize}
            \item Set \regname{RSP} to \localname{domain\_state->exn\_handler} value
            \item Restore \localname{domain\_state->exn\_handler} from trap
            \item Jump with \funcname{ret} (same effect as using \funcname{pop} and \funcname{jmp})
          \end{itemize}
        \item<3-> Otherwise, switches to the C stack to be able to execute C code
        \item<4-> Calls \funcname{caml\_stash\_backtrace}, a C function provided by the runtime\ignorespaces
          \onslide<6->{, with
            \begin{itemize}
              \item \funcarg{exn}: exception value
              \item \funcarg{pc}: code address of the raising point
              \item \funcarg{sp}: stack address of the raising function
              \item \funcarg{trapsp}: stack address of the next handler
            \end{itemize}
          }
      \end{itemize}
      \bigskip
      \mintocaml[firstline=9,lastline=13,highlightlines=12]{../src/backtrace/backtrace.ml}
    \end{column}
    \begin{column}{0.5\textwidth}
      \raggedleft
      \onslide*<1,3-4>{
        \mintc[firstline=190,lastline=192]{../ocaml/runtime/amd64.S}
        \mintc[firstline=234,lastline=237]{../ocaml/runtime/amd64.S}
      }
      \onslide*<2>{
        \mintc[firstline=190,lastline=192,highlightlines={191-192}]{../ocaml/runtime/amd64.S}
        \mintc[firstline=234,lastline=237,highlightlines={235-237}]{../ocaml/runtime/amd64.S}
      }
      \onslide*<1>{\listCamlRaiseExn[highlightlines=705]{}}%
      \onslide*<2>{\listCamlRaiseExn[highlightlines={707-708}]{}}%
      \onslide*<3>{\listCamlRaiseExn[highlightlines={712-714}]{}}%
      \onslide*<4>{\listCamlRaiseExn[highlightlines={715-720}]{}}%
      \onslide*<5>{\providecommand\step{1}\input{diagrams/backtrace/backtrace.tex}}%
      \onslide*<6>{\providecommand\step{2}\input{diagrams/backtrace/backtrace.tex}}%
    \end{column}
  \end{columns}
\end{frame}

\newcommand\listStashBacktrace[1][]{
  \listc[firstline=91,lastline=120,#1]{../ocaml/runtime/backtrace\_nat.c}{runtime/backtrace\_nat.c:91}}

\begin{frame}{Backtraces}{Introducing \funcname{caml\_stash\_backtrace}}
  \begin{columns}[c]
    \begin{column}{0.5\textwidth}
      \minipage[c][0.66\textheight][s]{\columnwidth}
      \begin{itemize}
        \item<1-> A C function provided by the runtime (specific to native code)
        \item<2-> It scans the stack, between the stack pointer at the raising point up to the next trap, in order to collect the names of the detected function frames
          \onslide*<2>{
            \begin{itemize}
              \item And more information, like line number, column number...
            \end{itemize}
          }
        \item<3-> It uses the same technique and information as the Garbage Collector (GC) uses to scan the values live on the stack
          \onslide*<4>{
            \begin{itemize}
              \item \typename{frame\_descr} are at the core of stack unwinding
            \end{itemize}
          }
      \end{itemize}
      \vfill
      \mintocaml[firstline=9,lastline=13,highlightlines=12]{../src/backtrace/backtrace.ml}
      \endminipage
    \end{column}
    \begin{column}{0.5\textwidth}
      \onslide*<1>{\listStashBacktrace[highlightlines=91]{}}
      \onslide*<2>{\listStashBacktrace[highlightlines={91,117-118}]{}}
      \onslide*<3->{\listStashBacktrace[highlightlines={94,106,109-110}]{}}
    \end{column}
  \end{columns}
\end{frame}

\newcommand\listFrameDescr[1][]{
  \listocaml[firstline=111,lastline=124,#1]{../ocaml/asmcomp/emitaux.ml}{asmcomp/emitaux.ml:111}}
\newcommand\listDebugInfo[1][]{
  \listocaml[firstline=38,lastline=49,#1]{../ocaml/lambda/debuginfo.mli}{lambda/debuginfo.mli:38}}

\begin{frame}{Backtraces}{\typename{frame\_descr} and \typename{frame\_debuginfo} in a nutshell}
  \begin{columns}[c]
    \begin{column}{0.5\textwidth}
      \begin{itemize}
        \item<1-> For every return address in an OCaml program, there is a associated \typename{frame\_descr}
        \item<2-> A \typename{frame\_descr} specifies
        \begin{itemize}
          \item<2-> The size of the function stack frame
          \item<3-> Where live values are on the stack (so that the GC can mark them as live and prevent from being swept)
        \end{itemize}
        \item<4-> When compiled with debug information, \typename{frame\_descr} will be linked to a \typename{frame\_debuginfo}
        \begin{itemize}
          \item<5-> Contains information about the position in the source code (file name, line and column number)
        \end{itemize}
      \end{itemize}
    \end{column}
    \begin{column}{0.5\textwidth}
        \onslide*<1>{\listFrameDescr[highlightlines={118-122}]{}}
        \onslide*<2>{\listFrameDescr[highlightlines={120}]{}}
        \onslide*<3>{\listFrameDescr[highlightlines={121}]{}}
        \onslide*<4->{\listFrameDescr[highlightlines={122}]{}}
        \medskip
        \onslide*<1-3>{\listDebugInfo{}}
        \onslide*<4>{\listDebugInfo[highlightlines={38,49}]{}}
        \onslide*<5>{\listDebugInfo[highlightlines={39-46}]{}}
    \end{column}
  \end{columns}
\end{frame}

\begin{frame}{Backtraces}{Scanning the stack with \typename{frame\_descr}}
  \begin{columns}[c]
    \begin{column}{0.5\textwidth}
      \begin{itemize}
        \item Given the return address of the code that raises the exception by calling \funcname{caml\_raise\_exn} (\funcarg{pc} argument of \funcname{caml\_stash\_backtrace})
        \item And the position of the stack pointer (\funcarg{sp} argument of \funcname{caml\_stash\_backtrace})
        \item The frame size given by \typename{frame\_descr}.\funcarg{frame\_size}, allows to find the end of the previous frame
        \item And the return address of the caller of this function
        \bigskip
        \onslide<3->{
        \item Note that traps (here the reddish \funcname{ExnA}, \funcname{ExnB} and \funcname{ExnC} blocks) belong to the frame that installed them, and so the frame size changes when traps are removed
        }
      \end{itemize}
    \end{column}
    \begin{column}{0.5\textwidth}
      \raggedleft
      \onslide*<1>{\providecommand\step{2}\input{diagrams/backtrace/backtrace.tex}}%
      \onslide*<2>{\providecommand\step{1}\input{diagrams/backtrace/scanstack.tex}}%
      \onslide*<3>{\providecommand\step{2}\input{diagrams/backtrace/scanstack.tex}}%
      \onslide*<4>{\providecommand\step{3}\input{diagrams/backtrace/scanstack.tex}}%
      \onslide*<5>{\providecommand\step{4}\input{diagrams/backtrace/scanstack.tex}}%
      \onslide*<6>{\providecommand\step{5}\input{diagrams/backtrace/scanstack.tex}}%
    \end{column}
  \end{columns}
\end{frame}

% Duplicate of previous-previous slide
\begin{frame}{Backtraces}{Continuing on \funcname{caml\_stash\_backtrace}}
  \begin{columns}[c]
    \begin{column}{0.5\textwidth}
      \minipage[c][0.75\textheight][s]{\columnwidth}
      \begin{itemize}
          \color{gray}
        \item A C function provided by the runtime (specific to native code)
        \item It scans the stack, between the stack pointer at the raising point up to the next trap, in order to collect the names of the detected function frames
        \item It uses the same technique and information as the Garbage Collector (GC) uses to scan the values live on the stack
      \end{itemize}
      \begin{itemize}
        \item For each function frame detected, store a pointer to the \typename{frame\_descr} in the \localname{domain\_state->backtrace\_buffer} array, effectively constructing the backtrace
      \end{itemize}
      \onslide<2->{
        \begin{itemize}
            \smallskip
          \item Constructed backtrace:
            \funcname{definitely\_raise},
            \onslide<3->{\funcname{probably\_raise}}
        \end{itemize}
      }
      \vfill
      \mintocaml[firstline=9,lastline=13,highlightlines=12]{../src/backtrace/backtrace.ml}
      \endminipage
    \end{column}
    \begin{column}{0.5\textwidth}
      \raggedleft
      \onslide*<1>{\listStashBacktrace[highlightlines={114-115}]{}}
      \onslide*<2>{\providecommand\step{3}\input{diagrams/backtrace/backtrace.tex}}%
      \onslide*<3>{\providecommand\step{4}\input{diagrams/backtrace/backtrace.tex}}%
    \end{column}
  \end{columns}
\end{frame}

\begin{frame}{Backtraces}{Back to \funcname{caml\_raise\_exn}}
  \begin{columns}[c]
    \begin{column}{0.5\textwidth}
      \minipage[c][0.66\textheight][s]{\columnwidth}
      \begin{itemize}\color{gray}
        \item Checks wheteher exceptions backtrace should be recorded, by checking the \funcarg{domain\_state->backtrace\_active} value
        \item Switches to the C stack to be able to execute C code
        \item Calls \funcname{caml\_stash\_backtrace}, to collect the backtrace up to the next trap
      \end{itemize}
      \onslide*<2->{
        \begin{itemize}
          \item<2-> Executes the exception handler in \funcname{probably\_raise} with \funcname{RESTORE\_EXN\_HANDLER\_OCAML}
            \begin{itemize}
              \item Set \regname{RSP} to \localname{domain\_state->exn\_handler} value
              \item Restore \localname{domain\_state->exn\_handler} from trap
              \item Jump to the handler code with \funcname{ret}
            \end{itemize}
        \end{itemize}
      }
      \vfill
      \onslide*<1>{\mintocaml[firstline=9,lastline=13,highlightlines=12]{../src/backtrace/backtrace.ml}}
      \onslide*<2->{\mintocaml[firstline=15,lastline=21,highlightlines={19-21}]{../src/backtrace/backtrace.ml}}
      \endminipage
    \end{column}
    \begin{column}{0.5\textwidth}
      \centering
      \onslide*<1>{
        \mintc[firstline=190,lastline=192]{../ocaml/runtime/amd64.S}
        \mintc[firstline=234,lastline=237]{../ocaml/runtime/amd64.S}
      }
      \onslide*<2>{
        \mintc[firstline=190,lastline=192,highlightlines={191-192}]{../ocaml/runtime/amd64.S}
        \mintc[firstline=234,lastline=237,highlightlines={235-237}]{../ocaml/runtime/amd64.S}
      }
      \onslide*<1>{\listCamlRaiseExn[highlightlines=720]{}}
      \onslide*<2>{\listCamlRaiseExn[highlightlines=722]{}}
      \onslide*<3->{\providecommand\step{5}\input{diagrams/backtrace/backtrace.tex}}
    \end{column}
  \end{columns}
\end{frame}

\begin{frame}{Backtraces}{\funcname{caml\_probably\_raise}'s handler doesn't catch \typename{ExnC}}
  \begin{columns}[c]
    \begin{column}{0.45\textwidth}
      \minipage[c][0.75\textheight][s]{\columnwidth}
      \begin{itemize}
        \item<1-> \funcname{match \dots with}\footnotemark pattern matching can also match exceptions
        \item<1-> The CMM of \funcname{probably\_raise} shows that this \funcname{match \dots with} is implemented with a pattern matching and a \funcname{try \dots with}
        \item<1-> But this one only matches on \typename{ExnA} exception
        \item<2-> Since the pattern matching fails to catch the exception, \funcname{reraise} is called with our current \typename{ExnC} exception
        \item<3-> Disassembly shows that \funcname{reraise} is actually a call to \funcname{caml\_reraise\_exn}, which is also an assembly function provided by the runtime
      \end{itemize}
      \vfill
      \mintocaml[firstline=15,lastline=21,highlightlines={18-21}]{../src/backtrace/backtrace.ml}
      \endminipage
    \end{column}
    \begin{column}{0.55\textwidth}
      \centering
      \onslide*<1>{\mintcmm[highlightlines={10-18}]{../src/backtrace/camlBacktrace\_\_probably\_raise\_351.cmm}}
      \onslide*<2>{\listcmm[highlightlines=18]{../src/backtrace/camlBacktrace\_\_probably\_raise_351.cmm}{CMM of probably\_raise}}
      \onslide*<3>{\mintobjdump[firstline=5,lastline=35,highlightlines=31]{../src/backtrace/camlBacktrace\_\_probably\_raise_351.objdump}}
    \end{column}
  \end{columns}
  \only<1>{\footnotetext[1]{https://blog.janestreet.com/pattern-matching-and-exception-handling-unite/}}
\end{frame}

\newcommand\listCamlReraiseExn[1][]{
  \listasm[firstline=727,lastline=734,#1]{../ocaml/runtime/amd64.S}{runtime/amd64.S:727}}

\begin{frame}{Backtraces}{Introducing \funcname{caml\_reraise\_exn}}
  \begin{columns}[c]
    \begin{column}{0.5\textwidth}
      \minipage[c][0.66\textheight][s]{\columnwidth}
      \begin{itemize}
        \item<1-> The exception have to be reraised to the next handler
        \item<2-> First, checks if the backtrace should be collected with \funcarg{domain\_state->backtrace\_active}
        \only<3>{
        \begin{itemize}
            \item If not, behaves like a \funcname{raise\_notrace} with \funcname{RESTORE\_EXN\_HANDLER\_OCAML}
        \end{itemize}
        }
        \item<4-> Jumps into \funcname{caml\_raise\_exn}
        \item<5-> Calls \funcname{caml\_stash\_backtrace} again, to scan the next part of the stack: from the trap just pop-ed to the next trap
      \end{itemize}
      \vfill
      \mintocaml[firstline=15,lastline=21,highlightlines={18-21}]{../src/backtrace/backtrace.ml}
      \endminipage
    \end{column}
    \begin{column}{0.5\textwidth}
      \raggedleft
      \onslide*<1>{\listCamlReraiseExn{}}
      \onslide*<2>{\listCamlReraiseExn[highlightlines=729]{}}
      \onslide*<3>{
        \mintc[firstline=190,lastline=192,highlightlines={191-192}]{../ocaml/runtime/amd64.S}
        \mintc[firstline=234,lastline=237,highlightlines={235-237}]{../ocaml/runtime/amd64.S}
        \medskip
        \listCamlReraiseExn[highlightlines=731]{}
      }
      \onslide*<4-5>{
        % TODO refactor with \listCamlReraiseExn
        \mintasm[firstline=727,lastline=734,highlightlines=730]{../ocaml/runtime/amd64.S}
        \medskip
      }
      \onslide*<4>{\listCamlRaiseExn[highlightlines=711]{}}
      \onslide*<5>{\listCamlRaiseExn[highlightlines=720]{}}
    \end{column}
  \end{columns}
\end{frame}

\begin{frame}{Backtraces}{Backtrace collection continues}
  \begin{columns}[c]
    \begin{column}{0.55\textwidth}
      \minipage[c][0.75\textheight][s]{\columnwidth}
      \begin{itemize}
        \item<1-> \funcname{caml\_stash\_backtrace} scans the older part of the stack, up to the next trap
        \item<6-> Controls jumps to the nested exception handler in \funcname{maybe\_raise}, which matches only on \typename{ExnB}
        \item<7-> \funcname{caml\_reraise\_exn} is called again, backtrace collection resumes with \funcname{caml\_stash\_backtrace}
      \end{itemize}
      \medskip
      \begin{itemize}
        \item<2-> Constructed backtrace:
        \begin{itemize}
          \item <2-> \funcname{definitely\_raise}
          \item <2-> \funcname{probably\_raise}
          \item <3-> \funcname{probably\_raise} (re-raise)
          \item <4-> \funcname{maybe\_raise}
          \item <5-> \funcname{main}
          \item <8-> \funcname{main} (re-raise)
        \end{itemize}
      \end{itemize}
      \vfill
      \onslide*<1-5>{\mintocaml[firstline=15,lastline=21]{../src/backtrace/backtrace.ml}}
      \onslide*<6>{\mintocaml[firstline=28,lastline=34,highlightlines=31]{../src/backtrace/backtrace.ml}}
      \onslide*<7->{\mintocaml[firstline=28,lastline=34,highlightlines=33]{../src/backtrace/backtrace.ml}}
      \endminipage
    \end{column}
    \begin{column}{0.45\textwidth}
      \raggedleft
      \onslide*<1>{\providecommand\step{6}\input{diagrams/backtrace/backtrace.tex}}%
      \onslide*<2>{\providecommand\step{6}\input{diagrams/backtrace/backtrace.tex}}%
      \onslide*<3>{\providecommand\step{7}\input{diagrams/backtrace/backtrace.tex}}%
      \onslide*<4>{\providecommand\step{8}\input{diagrams/backtrace/backtrace.tex}}%
      \onslide*<5>{\providecommand\step{9}\input{diagrams/backtrace/backtrace.tex}}%
      \onslide*<6>{\providecommand\step{10}\input{diagrams/backtrace/backtrace.tex}}%
      \onslide*<7>{\providecommand\step{11}\input{diagrams/backtrace/backtrace.tex}}%
      \onslide*<8>{\providecommand\step{12}\input{diagrams/backtrace/backtrace.tex}}%
      \onslide*<9>{\providecommand\step{13}\input{diagrams/backtrace/backtrace.tex}}%
    \end{column}
  \end{columns}
\end{frame}

\begin{frame}{Backtraces}{Printing the backtrace}
  \begin{columns}[c]
    \begin{column}{0.45\textwidth}
      \minipage[c][0.66\textheight][s]{\columnwidth}
      \begin{itemize}
        \item<1-> The exception backtrace can now be printed to \funcarg{stdout} with \funcname{Printexc.print_backtrace}
        \item<2-> First the exception backtrace is retrieved into an OCaml array with \funcname{get_raw_backtrace }
        \item<3-> Which is implemented as the \funcname{caml_get_exception_raw_backtrace} C function in the runtime
        \item<4-> And finally printed with \funcname{print_exception_backtrace}
      \end{itemize}
      \vfill
      \mintocaml[firstline=28,lastline=34,highlightlines=33]{../src/backtrace/backtrace.ml}
      \endminipage
    \end{column}
    \begin{column}{0.55\textwidth}
      \onslide*<1,2>{
        \mintocaml[firstline=100,lastline=101]{../ocaml/stdlib/printexc.ml}
      }
      \only<1>{
        \listocaml[firstline=170,lastline=172]{../ocaml/stdlib/printexc.ml}{stdlib/printexc.ml:170}
      }
      \only<2>{
        \listocaml[firstline=170,lastline=172,highlightlines=172]{../ocaml/stdlib/printexc.ml}{stdlib/printexc.ml:170}
      }
      \only<3>{
        \mintc[firstline=154,lastline=156]{../ocaml/runtime/backtrace.c}
        \listc[firstline=171,lastline=192,highlightlines={184-187}]{../ocaml/runtime/backtrace.c}{runtime/backtrace.c:154}
      }
      \only<4>{
        \listocaml[firstline=155,lastline=165]{../ocaml/stdlib/printexc.ml}{stdlib/printexc.ml:55}
      }
    \end{column}
  \end{columns}
\end{frame}


\subsection{The multiple kinds of raise}
\frameSubsection{}{}

\begin{frame}{Backtraces}{When backtraces are not needed}
  \begin{columns}[c]
    \begin{column}{0.45\textwidth}
      \minipage[c][0.66\textheight][s]{\columnwidth}
      \begin{itemize}
        \item<1-> What if the backtrace for an exception is never needed?
        \item<2-> It's possible to raise the exception explicitly with \funcname{raise\_notrace} to make raising as cheap as possible
        \item<3-> An example of use in the \funcname{dynlink} library of the OCaml compiler
      \end{itemize}
      \vfill
      \onslide<3->{
      \listocaml[firstline=205,lastline=209,highlightlines=207]{../ocaml/otherlibs/dynlink/dynlink\_compilerlibs/misc.ml}{otherlibs/dynlink/dynlink\_compilerlibs/misc.ml:205}
      }
      \endminipage
    \end{column}
    \begin{column}{0.55\textwidth}
    \only<4>{
      \mintcmm[highlightlines=3]{../src/notrace/camlNotrace\_\_fun_347.cmm}
      \listcmm{../src/notrace/camlNotrace\_\_all\_somes\_327.cmm}{all\_somes}
    }
    \onslide*<5->{
      \listobjdump[highlightlines={6-9}]{../src/notrace/camlNotrace\_\_fun_347.objdump}{anonymous function from \funcname{all\_somes}}
    }
    \end{column}
  \end{columns}
\end{frame}

\begin{frame}{Backtraces}{Summary table of the multiple kinds of raise}
  \begin{itemize}
    \item When debugging information is enabled and if the backtrace of an exception isn't needed, it's possible to raise with \funcname{raise\_notrace} for performance
    \item When debugging information is \emph{not} enabled, the compiler optimises \funcname{raise} calls to inline assembly (same effect as using \funcname{raise\_notrace})
  \end{itemize}
  \bigskip
  \centering
  \begin{tabular}{| l | c | c | c | c |}
    \hline
    \parbox{10em}{Debug information \\ (\funcarg{-g} flag)}  & \multicolumn{2}{|c|}{Disabled}                                     & \multicolumn{2}{|c|}{Enabled} \\
    \hline
    Raise kind                                               & \funcname{raise} & \funcname{raise\_notrace}                       & \funcname{raise} & \funcname{raise\_notrace} \\
    \hline
    Compilation                                              & \parbox{8em}{inline assembly\\ (optimization)} & inline assembly   & \funcname{caml\_raise\_exn} & inline assembly \\
    \hline
  \end{tabular}
\end{frame}


%
% Takeaway #5
%
\subsection*{Takeaway \#5}
\frameSubsectionTakeaway{}
\againframe<5>{takeaway}
}
    \end{column}
  \end{columns}
\end{frame}

\begin{frame}{Backtraces}{\funcname{caml\_probably\_raise}'s handler doesn't catch \typename{ExnC}}
  \begin{columns}[c]
    \begin{column}{0.45\textwidth}
      \minipage[c][0.75\textheight][s]{\columnwidth}
      \begin{itemize}
        \item<1-> \funcname{match \dots with}\footnotemark pattern matching can also match exceptions
        \item<1-> The CMM of \funcname{probably\_raise} shows that this \funcname{match \dots with} is implemented with a pattern matching and a \funcname{try \dots with}
        \item<1-> But this one only matches on \typename{ExnA} exception
        \item<2-> Since the pattern matching fails to catch the exception, \funcname{reraise} is called with our current \typename{ExnC} exception
        \item<3-> Disassembly shows that \funcname{reraise} is actually a call to \funcname{caml\_reraise\_exn}, which is also an assembly function provided by the runtime
      \end{itemize}
      \vfill
      \mintocaml[firstline=15,lastline=21,highlightlines={18-21}]{../src/backtrace/backtrace.ml}
      \endminipage
    \end{column}
    \begin{column}{0.55\textwidth}
      \centering
      \onslide*<1>{\mintcmm[highlightlines={10-18}]{../src/backtrace/camlBacktrace\_\_probably\_raise\_351.cmm}}
      \onslide*<2>{\listcmm[highlightlines=18]{../src/backtrace/camlBacktrace\_\_probably\_raise_351.cmm}{CMM of probably\_raise}}
      \onslide*<3>{\mintobjdump[firstline=5,lastline=35,highlightlines=31]{../src/backtrace/camlBacktrace\_\_probably\_raise_351.objdump}}
    \end{column}
  \end{columns}
  \only<1>{\footnotetext[1]{https://blog.janestreet.com/pattern-matching-and-exception-handling-unite/}}
\end{frame}

\newcommand\listCamlReraiseExn[1][]{
  \listasm[firstline=727,lastline=734,#1]{../ocaml/runtime/amd64.S}{runtime/amd64.S:727}}

\begin{frame}{Backtraces}{Introducing \funcname{caml\_reraise\_exn}}
  \begin{columns}[c]
    \begin{column}{0.5\textwidth}
      \minipage[c][0.66\textheight][s]{\columnwidth}
      \begin{itemize}
        \item<1-> The exception have to be reraised to the next handler
        \item<2-> First, checks if the backtrace should be collected with \funcarg{domain\_state->backtrace\_active}
        \only<3>{
        \begin{itemize}
            \item If not, behaves like a \funcname{raise\_notrace} with \funcname{RESTORE\_EXN\_HANDLER\_OCAML}
        \end{itemize}
        }
        \item<4-> Jumps into \funcname{caml\_raise\_exn}
        \item<5-> Calls \funcname{caml\_stash\_backtrace} again, to scan the next part of the stack: from the trap just pop-ed to the next trap
      \end{itemize}
      \vfill
      \mintocaml[firstline=15,lastline=21,highlightlines={18-21}]{../src/backtrace/backtrace.ml}
      \endminipage
    \end{column}
    \begin{column}{0.5\textwidth}
      \raggedleft
      \onslide*<1>{\listCamlReraiseExn{}}
      \onslide*<2>{\listCamlReraiseExn[highlightlines=729]{}}
      \onslide*<3>{
        \mintc[firstline=190,lastline=192,highlightlines={191-192}]{../ocaml/runtime/amd64.S}
        \mintc[firstline=234,lastline=237,highlightlines={235-237}]{../ocaml/runtime/amd64.S}
        \medskip
        \listCamlReraiseExn[highlightlines=731]{}
      }
      \onslide*<4-5>{
        % TODO refactor with \listCamlReraiseExn
        \mintasm[firstline=727,lastline=734,highlightlines=730]{../ocaml/runtime/amd64.S}
        \medskip
      }
      \onslide*<4>{\listCamlRaiseExn[highlightlines=711]{}}
      \onslide*<5>{\listCamlRaiseExn[highlightlines=720]{}}
    \end{column}
  \end{columns}
\end{frame}

\begin{frame}{Backtraces}{Backtrace collection continues}
  \begin{columns}[c]
    \begin{column}{0.55\textwidth}
      \minipage[c][0.75\textheight][s]{\columnwidth}
      \begin{itemize}
        \item<1-> \funcname{caml\_stash\_backtrace} scans the older part of the stack, up to the next trap
        \item<6-> Controls jumps to the nested exception handler in \funcname{maybe\_raise}, which matches only on \typename{ExnB}
        \item<7-> \funcname{caml\_reraise\_exn} is called again, backtrace collection resumes with \funcname{caml\_stash\_backtrace}
      \end{itemize}
      \medskip
      \begin{itemize}
        \item<2-> Constructed backtrace:
        \begin{itemize}
          \item <2-> \funcname{definitely\_raise}
          \item <2-> \funcname{probably\_raise}
          \item <3-> \funcname{probably\_raise} (re-raise)
          \item <4-> \funcname{maybe\_raise}
          \item <5-> \funcname{main}
          \item <8-> \funcname{main} (re-raise)
        \end{itemize}
      \end{itemize}
      \vfill
      \onslide*<1-5>{\mintocaml[firstline=15,lastline=21]{../src/backtrace/backtrace.ml}}
      \onslide*<6>{\mintocaml[firstline=28,lastline=34,highlightlines=31]{../src/backtrace/backtrace.ml}}
      \onslide*<7->{\mintocaml[firstline=28,lastline=34,highlightlines=33]{../src/backtrace/backtrace.ml}}
      \endminipage
    \end{column}
    \begin{column}{0.45\textwidth}
      \raggedleft
      \onslide*<1>{\providecommand\step{6}%
% Backtraces
%
\subsection{One advantage of raising with debugging information}
\frameSubsection{}{}

\begin{frame}{Backtraces}{Debugging information \& source code}
  \begin{columns}[c]
    \begin{column}{0.5\textwidth}
      \begin{itemize}
        \item Raising an exception is fun and games, but how do we know where the exception originated? $\rightarrow$ With the backtrace associated with the exception!
        \item In order to get a backtrace from the point where an exception was raised, it's necessary to compile with debugging information enabled (\funcarg{-g} flag for the compiler)
        \item Same example as in the `Nested handlers' section
      \end{itemize}
    \end{column}
    \begin{column}{0.5\textwidth}
      \listocaml[firstline=9,lastline=34]{../src/backtrace/backtrace.ml}{backtrace/backtrace.ml}
    \end{column}
  \end{columns}
\end{frame}

\begin{frame}{Backtraces}{CMM and disassembly of \funcname{definitely\_raise}}
  \begin{columns}[c]
    \begin{column}{0.5\textwidth}
      \minipage[c][0.66\textheight][s]{\columnwidth}
      \begin{itemize}
        \item<1-> An effect of compiling with debugging information (\funcarg{-g}): the CMM no longer uses \funcname{raise\_notrace} to raise the exception but \funcname{raise} instead
        \item<2-> Disassembly shows that CMM \funcname{raise} is actually a call to \funcname{caml\_raise\_exn}, a function in assembler provided by the runtime
      \end{itemize}
      \vfill
      \mintocaml[firstline=9,lastline=13,highlightlines=12]{../src/backtrace/backtrace.ml}
      \endminipage
    \end{column}
    \begin{column}{0.5\textwidth}
      \onslide*<1>{
        \mintcmm[highlightlines=5]{../src/backtrace/camlBacktrace\_\_definitely\_raise\_347.cmm}
      }
      \onslide*<2>{
        \mintobjdump[firstline=2,highlightlines=14]{../src/backtrace/camlBacktrace\_\_definitely\_raise\_347.objdump}
      }
    \end{column}
  \end{columns}
\end{frame}

\begin{frame}{Backtraces}{Introducing \funcname{caml\_raise\_exn}}
  \begin{columns}[c]
    \begin{column}{0.5\textwidth}
      \minipage[c][0.66\textheight][s]{\columnwidth}
      \onslide*<1>{
        \begin{itemize}
          \item Raising the exception is performed using \funcname{caml\_raise\_exn} which is
            \begin{itemize}
              \item An assembly function provided by the runtime
              \item Architecture specific
            \end{itemize}
          \item Let's see what it does differently from \funcname{raise\_notrace}\dots
          \item We are about to raise \typename{ExnC} with \funcname{caml\_raise\_exn} from \funcname{definitely\_raise}
        \end{itemize}
      }
      \onslide*<2>{
        \mintobjdump[firstline=2,highlightlines=14]{../src/backtrace/camlBacktrace\_\_definitely\_raise\_347.objdump}
      }
      \vfill
      \mintocaml[firstline=9,lastline=13,highlightlines=12]{../src/backtrace/backtrace.ml}
      \endminipage
    \end{column}
    \begin{column}{0.5\textwidth}
      \centering
      \providecommand\step{1}\input{diagrams/backtrace/backtrace.tex}
    \end{column}
  \end{columns}
\end{frame}

\begin{frame}{Backtraces}{But before that, we need more fields from \typename{caml\_domain\_state}}
  \begin{columns}
    \begin{column}{0.5\textwidth}
      \begin{itemize}
        \item Remember \typename{caml\_domain\_state} the per-domain runtime state?
        \item Some other members are of interest to us now\dots
        \item \localname{backtrace\_active} is used to know if the runtime should collect backtraces
        \item \localname{backtrace\_active} can be set by
          \begin{itemize}
            \item The \funcarg{b} flag in the \funcname{OCAMLRUNPARAM} environment variable
            \item \mintinline{ocaml}{Printexc.record_backtrace : bool -> unit} \footnotemark
          \end{itemize}
      \end{itemize}
    \end{column}
    \begin{column}{0.5\textwidth}
      \begin{listing}
        \mintc[highlightlines={9-11}]{src/ocaml/domain\_state\_backtrace.h}
        \caption{runtime/caml/domain\_state.h}
      \end{listing}
    \end{column}
  \end{columns}
  \footnotetext[1]{https://v2.ocaml.org/api/Printexc.html}
\end{frame}

\newcommand\listCamlRaiseExn[1][]{
  \listasm[firstline=702,lastline=725,#1]{../ocaml/runtime/amd64.S}{runtime/amd64.S:702}}

\begin{frame}{Backtraces}{Walk through \funcname{caml\_raise\_exn}}
  \begin{columns}[T]
    \begin{column}{0.5\textwidth}
      \begin{itemize}
        \item<1-> First checks whether exceptions backtrace should be recorded
        \item<1-> By checking the \funcarg{domain\_state->backtrace\_active} value
        \item<2-> If not, acts as \funcname{raise\_notrace} with \funcname{RESTORE\_EXN\_HANDLER\_OCAML}
          \begin{itemize}
            \item Set \regname{RSP} to \localname{domain\_state->exn\_handler} value
            \item Restore \localname{domain\_state->exn\_handler} from trap
            \item Jump with \funcname{ret} (same effect as using \funcname{pop} and \funcname{jmp})
          \end{itemize}
        \item<3-> Otherwise, switches to the C stack to be able to execute C code
        \item<4-> Calls \funcname{caml\_stash\_backtrace}, a C function provided by the runtime\ignorespaces
          \onslide<6->{, with
            \begin{itemize}
              \item \funcarg{exn}: exception value
              \item \funcarg{pc}: code address of the raising point
              \item \funcarg{sp}: stack address of the raising function
              \item \funcarg{trapsp}: stack address of the next handler
            \end{itemize}
          }
      \end{itemize}
      \bigskip
      \mintocaml[firstline=9,lastline=13,highlightlines=12]{../src/backtrace/backtrace.ml}
    \end{column}
    \begin{column}{0.5\textwidth}
      \raggedleft
      \onslide*<1,3-4>{
        \mintc[firstline=190,lastline=192]{../ocaml/runtime/amd64.S}
        \mintc[firstline=234,lastline=237]{../ocaml/runtime/amd64.S}
      }
      \onslide*<2>{
        \mintc[firstline=190,lastline=192,highlightlines={191-192}]{../ocaml/runtime/amd64.S}
        \mintc[firstline=234,lastline=237,highlightlines={235-237}]{../ocaml/runtime/amd64.S}
      }
      \onslide*<1>{\listCamlRaiseExn[highlightlines=705]{}}%
      \onslide*<2>{\listCamlRaiseExn[highlightlines={707-708}]{}}%
      \onslide*<3>{\listCamlRaiseExn[highlightlines={712-714}]{}}%
      \onslide*<4>{\listCamlRaiseExn[highlightlines={715-720}]{}}%
      \onslide*<5>{\providecommand\step{1}\input{diagrams/backtrace/backtrace.tex}}%
      \onslide*<6>{\providecommand\step{2}\input{diagrams/backtrace/backtrace.tex}}%
    \end{column}
  \end{columns}
\end{frame}

\newcommand\listStashBacktrace[1][]{
  \listc[firstline=91,lastline=120,#1]{../ocaml/runtime/backtrace\_nat.c}{runtime/backtrace\_nat.c:91}}

\begin{frame}{Backtraces}{Introducing \funcname{caml\_stash\_backtrace}}
  \begin{columns}[c]
    \begin{column}{0.5\textwidth}
      \minipage[c][0.66\textheight][s]{\columnwidth}
      \begin{itemize}
        \item<1-> A C function provided by the runtime (specific to native code)
        \item<2-> It scans the stack, between the stack pointer at the raising point up to the next trap, in order to collect the names of the detected function frames
          \onslide*<2>{
            \begin{itemize}
              \item And more information, like line number, column number...
            \end{itemize}
          }
        \item<3-> It uses the same technique and information as the Garbage Collector (GC) uses to scan the values live on the stack
          \onslide*<4>{
            \begin{itemize}
              \item \typename{frame\_descr} are at the core of stack unwinding
            \end{itemize}
          }
      \end{itemize}
      \vfill
      \mintocaml[firstline=9,lastline=13,highlightlines=12]{../src/backtrace/backtrace.ml}
      \endminipage
    \end{column}
    \begin{column}{0.5\textwidth}
      \onslide*<1>{\listStashBacktrace[highlightlines=91]{}}
      \onslide*<2>{\listStashBacktrace[highlightlines={91,117-118}]{}}
      \onslide*<3->{\listStashBacktrace[highlightlines={94,106,109-110}]{}}
    \end{column}
  \end{columns}
\end{frame}

\newcommand\listFrameDescr[1][]{
  \listocaml[firstline=111,lastline=124,#1]{../ocaml/asmcomp/emitaux.ml}{asmcomp/emitaux.ml:111}}
\newcommand\listDebugInfo[1][]{
  \listocaml[firstline=38,lastline=49,#1]{../ocaml/lambda/debuginfo.mli}{lambda/debuginfo.mli:38}}

\begin{frame}{Backtraces}{\typename{frame\_descr} and \typename{frame\_debuginfo} in a nutshell}
  \begin{columns}[c]
    \begin{column}{0.5\textwidth}
      \begin{itemize}
        \item<1-> For every return address in an OCaml program, there is a associated \typename{frame\_descr}
        \item<2-> A \typename{frame\_descr} specifies
        \begin{itemize}
          \item<2-> The size of the function stack frame
          \item<3-> Where live values are on the stack (so that the GC can mark them as live and prevent from being swept)
        \end{itemize}
        \item<4-> When compiled with debug information, \typename{frame\_descr} will be linked to a \typename{frame\_debuginfo}
        \begin{itemize}
          \item<5-> Contains information about the position in the source code (file name, line and column number)
        \end{itemize}
      \end{itemize}
    \end{column}
    \begin{column}{0.5\textwidth}
        \onslide*<1>{\listFrameDescr[highlightlines={118-122}]{}}
        \onslide*<2>{\listFrameDescr[highlightlines={120}]{}}
        \onslide*<3>{\listFrameDescr[highlightlines={121}]{}}
        \onslide*<4->{\listFrameDescr[highlightlines={122}]{}}
        \medskip
        \onslide*<1-3>{\listDebugInfo{}}
        \onslide*<4>{\listDebugInfo[highlightlines={38,49}]{}}
        \onslide*<5>{\listDebugInfo[highlightlines={39-46}]{}}
    \end{column}
  \end{columns}
\end{frame}

\begin{frame}{Backtraces}{Scanning the stack with \typename{frame\_descr}}
  \begin{columns}[c]
    \begin{column}{0.5\textwidth}
      \begin{itemize}
        \item Given the return address of the code that raises the exception by calling \funcname{caml\_raise\_exn} (\funcarg{pc} argument of \funcname{caml\_stash\_backtrace})
        \item And the position of the stack pointer (\funcarg{sp} argument of \funcname{caml\_stash\_backtrace})
        \item The frame size given by \typename{frame\_descr}.\funcarg{frame\_size}, allows to find the end of the previous frame
        \item And the return address of the caller of this function
        \bigskip
        \onslide<3->{
        \item Note that traps (here the reddish \funcname{ExnA}, \funcname{ExnB} and \funcname{ExnC} blocks) belong to the frame that installed them, and so the frame size changes when traps are removed
        }
      \end{itemize}
    \end{column}
    \begin{column}{0.5\textwidth}
      \raggedleft
      \onslide*<1>{\providecommand\step{2}\input{diagrams/backtrace/backtrace.tex}}%
      \onslide*<2>{\providecommand\step{1}\input{diagrams/backtrace/scanstack.tex}}%
      \onslide*<3>{\providecommand\step{2}\input{diagrams/backtrace/scanstack.tex}}%
      \onslide*<4>{\providecommand\step{3}\input{diagrams/backtrace/scanstack.tex}}%
      \onslide*<5>{\providecommand\step{4}\input{diagrams/backtrace/scanstack.tex}}%
      \onslide*<6>{\providecommand\step{5}\input{diagrams/backtrace/scanstack.tex}}%
    \end{column}
  \end{columns}
\end{frame}

% Duplicate of previous-previous slide
\begin{frame}{Backtraces}{Continuing on \funcname{caml\_stash\_backtrace}}
  \begin{columns}[c]
    \begin{column}{0.5\textwidth}
      \minipage[c][0.75\textheight][s]{\columnwidth}
      \begin{itemize}
          \color{gray}
        \item A C function provided by the runtime (specific to native code)
        \item It scans the stack, between the stack pointer at the raising point up to the next trap, in order to collect the names of the detected function frames
        \item It uses the same technique and information as the Garbage Collector (GC) uses to scan the values live on the stack
      \end{itemize}
      \begin{itemize}
        \item For each function frame detected, store a pointer to the \typename{frame\_descr} in the \localname{domain\_state->backtrace\_buffer} array, effectively constructing the backtrace
      \end{itemize}
      \onslide<2->{
        \begin{itemize}
            \smallskip
          \item Constructed backtrace:
            \funcname{definitely\_raise},
            \onslide<3->{\funcname{probably\_raise}}
        \end{itemize}
      }
      \vfill
      \mintocaml[firstline=9,lastline=13,highlightlines=12]{../src/backtrace/backtrace.ml}
      \endminipage
    \end{column}
    \begin{column}{0.5\textwidth}
      \raggedleft
      \onslide*<1>{\listStashBacktrace[highlightlines={114-115}]{}}
      \onslide*<2>{\providecommand\step{3}\input{diagrams/backtrace/backtrace.tex}}%
      \onslide*<3>{\providecommand\step{4}\input{diagrams/backtrace/backtrace.tex}}%
    \end{column}
  \end{columns}
\end{frame}

\begin{frame}{Backtraces}{Back to \funcname{caml\_raise\_exn}}
  \begin{columns}[c]
    \begin{column}{0.5\textwidth}
      \minipage[c][0.66\textheight][s]{\columnwidth}
      \begin{itemize}\color{gray}
        \item Checks wheteher exceptions backtrace should be recorded, by checking the \funcarg{domain\_state->backtrace\_active} value
        \item Switches to the C stack to be able to execute C code
        \item Calls \funcname{caml\_stash\_backtrace}, to collect the backtrace up to the next trap
      \end{itemize}
      \onslide*<2->{
        \begin{itemize}
          \item<2-> Executes the exception handler in \funcname{probably\_raise} with \funcname{RESTORE\_EXN\_HANDLER\_OCAML}
            \begin{itemize}
              \item Set \regname{RSP} to \localname{domain\_state->exn\_handler} value
              \item Restore \localname{domain\_state->exn\_handler} from trap
              \item Jump to the handler code with \funcname{ret}
            \end{itemize}
        \end{itemize}
      }
      \vfill
      \onslide*<1>{\mintocaml[firstline=9,lastline=13,highlightlines=12]{../src/backtrace/backtrace.ml}}
      \onslide*<2->{\mintocaml[firstline=15,lastline=21,highlightlines={19-21}]{../src/backtrace/backtrace.ml}}
      \endminipage
    \end{column}
    \begin{column}{0.5\textwidth}
      \centering
      \onslide*<1>{
        \mintc[firstline=190,lastline=192]{../ocaml/runtime/amd64.S}
        \mintc[firstline=234,lastline=237]{../ocaml/runtime/amd64.S}
      }
      \onslide*<2>{
        \mintc[firstline=190,lastline=192,highlightlines={191-192}]{../ocaml/runtime/amd64.S}
        \mintc[firstline=234,lastline=237,highlightlines={235-237}]{../ocaml/runtime/amd64.S}
      }
      \onslide*<1>{\listCamlRaiseExn[highlightlines=720]{}}
      \onslide*<2>{\listCamlRaiseExn[highlightlines=722]{}}
      \onslide*<3->{\providecommand\step{5}\input{diagrams/backtrace/backtrace.tex}}
    \end{column}
  \end{columns}
\end{frame}

\begin{frame}{Backtraces}{\funcname{caml\_probably\_raise}'s handler doesn't catch \typename{ExnC}}
  \begin{columns}[c]
    \begin{column}{0.45\textwidth}
      \minipage[c][0.75\textheight][s]{\columnwidth}
      \begin{itemize}
        \item<1-> \funcname{match \dots with}\footnotemark pattern matching can also match exceptions
        \item<1-> The CMM of \funcname{probably\_raise} shows that this \funcname{match \dots with} is implemented with a pattern matching and a \funcname{try \dots with}
        \item<1-> But this one only matches on \typename{ExnA} exception
        \item<2-> Since the pattern matching fails to catch the exception, \funcname{reraise} is called with our current \typename{ExnC} exception
        \item<3-> Disassembly shows that \funcname{reraise} is actually a call to \funcname{caml\_reraise\_exn}, which is also an assembly function provided by the runtime
      \end{itemize}
      \vfill
      \mintocaml[firstline=15,lastline=21,highlightlines={18-21}]{../src/backtrace/backtrace.ml}
      \endminipage
    \end{column}
    \begin{column}{0.55\textwidth}
      \centering
      \onslide*<1>{\mintcmm[highlightlines={10-18}]{../src/backtrace/camlBacktrace\_\_probably\_raise\_351.cmm}}
      \onslide*<2>{\listcmm[highlightlines=18]{../src/backtrace/camlBacktrace\_\_probably\_raise_351.cmm}{CMM of probably\_raise}}
      \onslide*<3>{\mintobjdump[firstline=5,lastline=35,highlightlines=31]{../src/backtrace/camlBacktrace\_\_probably\_raise_351.objdump}}
    \end{column}
  \end{columns}
  \only<1>{\footnotetext[1]{https://blog.janestreet.com/pattern-matching-and-exception-handling-unite/}}
\end{frame}

\newcommand\listCamlReraiseExn[1][]{
  \listasm[firstline=727,lastline=734,#1]{../ocaml/runtime/amd64.S}{runtime/amd64.S:727}}

\begin{frame}{Backtraces}{Introducing \funcname{caml\_reraise\_exn}}
  \begin{columns}[c]
    \begin{column}{0.5\textwidth}
      \minipage[c][0.66\textheight][s]{\columnwidth}
      \begin{itemize}
        \item<1-> The exception have to be reraised to the next handler
        \item<2-> First, checks if the backtrace should be collected with \funcarg{domain\_state->backtrace\_active}
        \only<3>{
        \begin{itemize}
            \item If not, behaves like a \funcname{raise\_notrace} with \funcname{RESTORE\_EXN\_HANDLER\_OCAML}
        \end{itemize}
        }
        \item<4-> Jumps into \funcname{caml\_raise\_exn}
        \item<5-> Calls \funcname{caml\_stash\_backtrace} again, to scan the next part of the stack: from the trap just pop-ed to the next trap
      \end{itemize}
      \vfill
      \mintocaml[firstline=15,lastline=21,highlightlines={18-21}]{../src/backtrace/backtrace.ml}
      \endminipage
    \end{column}
    \begin{column}{0.5\textwidth}
      \raggedleft
      \onslide*<1>{\listCamlReraiseExn{}}
      \onslide*<2>{\listCamlReraiseExn[highlightlines=729]{}}
      \onslide*<3>{
        \mintc[firstline=190,lastline=192,highlightlines={191-192}]{../ocaml/runtime/amd64.S}
        \mintc[firstline=234,lastline=237,highlightlines={235-237}]{../ocaml/runtime/amd64.S}
        \medskip
        \listCamlReraiseExn[highlightlines=731]{}
      }
      \onslide*<4-5>{
        % TODO refactor with \listCamlReraiseExn
        \mintasm[firstline=727,lastline=734,highlightlines=730]{../ocaml/runtime/amd64.S}
        \medskip
      }
      \onslide*<4>{\listCamlRaiseExn[highlightlines=711]{}}
      \onslide*<5>{\listCamlRaiseExn[highlightlines=720]{}}
    \end{column}
  \end{columns}
\end{frame}

\begin{frame}{Backtraces}{Backtrace collection continues}
  \begin{columns}[c]
    \begin{column}{0.55\textwidth}
      \minipage[c][0.75\textheight][s]{\columnwidth}
      \begin{itemize}
        \item<1-> \funcname{caml\_stash\_backtrace} scans the older part of the stack, up to the next trap
        \item<6-> Controls jumps to the nested exception handler in \funcname{maybe\_raise}, which matches only on \typename{ExnB}
        \item<7-> \funcname{caml\_reraise\_exn} is called again, backtrace collection resumes with \funcname{caml\_stash\_backtrace}
      \end{itemize}
      \medskip
      \begin{itemize}
        \item<2-> Constructed backtrace:
        \begin{itemize}
          \item <2-> \funcname{definitely\_raise}
          \item <2-> \funcname{probably\_raise}
          \item <3-> \funcname{probably\_raise} (re-raise)
          \item <4-> \funcname{maybe\_raise}
          \item <5-> \funcname{main}
          \item <8-> \funcname{main} (re-raise)
        \end{itemize}
      \end{itemize}
      \vfill
      \onslide*<1-5>{\mintocaml[firstline=15,lastline=21]{../src/backtrace/backtrace.ml}}
      \onslide*<6>{\mintocaml[firstline=28,lastline=34,highlightlines=31]{../src/backtrace/backtrace.ml}}
      \onslide*<7->{\mintocaml[firstline=28,lastline=34,highlightlines=33]{../src/backtrace/backtrace.ml}}
      \endminipage
    \end{column}
    \begin{column}{0.45\textwidth}
      \raggedleft
      \onslide*<1>{\providecommand\step{6}\input{diagrams/backtrace/backtrace.tex}}%
      \onslide*<2>{\providecommand\step{6}\input{diagrams/backtrace/backtrace.tex}}%
      \onslide*<3>{\providecommand\step{7}\input{diagrams/backtrace/backtrace.tex}}%
      \onslide*<4>{\providecommand\step{8}\input{diagrams/backtrace/backtrace.tex}}%
      \onslide*<5>{\providecommand\step{9}\input{diagrams/backtrace/backtrace.tex}}%
      \onslide*<6>{\providecommand\step{10}\input{diagrams/backtrace/backtrace.tex}}%
      \onslide*<7>{\providecommand\step{11}\input{diagrams/backtrace/backtrace.tex}}%
      \onslide*<8>{\providecommand\step{12}\input{diagrams/backtrace/backtrace.tex}}%
      \onslide*<9>{\providecommand\step{13}\input{diagrams/backtrace/backtrace.tex}}%
    \end{column}
  \end{columns}
\end{frame}

\begin{frame}{Backtraces}{Printing the backtrace}
  \begin{columns}[c]
    \begin{column}{0.45\textwidth}
      \minipage[c][0.66\textheight][s]{\columnwidth}
      \begin{itemize}
        \item<1-> The exception backtrace can now be printed to \funcarg{stdout} with \funcname{Printexc.print_backtrace}
        \item<2-> First the exception backtrace is retrieved into an OCaml array with \funcname{get_raw_backtrace }
        \item<3-> Which is implemented as the \funcname{caml_get_exception_raw_backtrace} C function in the runtime
        \item<4-> And finally printed with \funcname{print_exception_backtrace}
      \end{itemize}
      \vfill
      \mintocaml[firstline=28,lastline=34,highlightlines=33]{../src/backtrace/backtrace.ml}
      \endminipage
    \end{column}
    \begin{column}{0.55\textwidth}
      \onslide*<1,2>{
        \mintocaml[firstline=100,lastline=101]{../ocaml/stdlib/printexc.ml}
      }
      \only<1>{
        \listocaml[firstline=170,lastline=172]{../ocaml/stdlib/printexc.ml}{stdlib/printexc.ml:170}
      }
      \only<2>{
        \listocaml[firstline=170,lastline=172,highlightlines=172]{../ocaml/stdlib/printexc.ml}{stdlib/printexc.ml:170}
      }
      \only<3>{
        \mintc[firstline=154,lastline=156]{../ocaml/runtime/backtrace.c}
        \listc[firstline=171,lastline=192,highlightlines={184-187}]{../ocaml/runtime/backtrace.c}{runtime/backtrace.c:154}
      }
      \only<4>{
        \listocaml[firstline=155,lastline=165]{../ocaml/stdlib/printexc.ml}{stdlib/printexc.ml:55}
      }
    \end{column}
  \end{columns}
\end{frame}


\subsection{The multiple kinds of raise}
\frameSubsection{}{}

\begin{frame}{Backtraces}{When backtraces are not needed}
  \begin{columns}[c]
    \begin{column}{0.45\textwidth}
      \minipage[c][0.66\textheight][s]{\columnwidth}
      \begin{itemize}
        \item<1-> What if the backtrace for an exception is never needed?
        \item<2-> It's possible to raise the exception explicitly with \funcname{raise\_notrace} to make raising as cheap as possible
        \item<3-> An example of use in the \funcname{dynlink} library of the OCaml compiler
      \end{itemize}
      \vfill
      \onslide<3->{
      \listocaml[firstline=205,lastline=209,highlightlines=207]{../ocaml/otherlibs/dynlink/dynlink\_compilerlibs/misc.ml}{otherlibs/dynlink/dynlink\_compilerlibs/misc.ml:205}
      }
      \endminipage
    \end{column}
    \begin{column}{0.55\textwidth}
    \only<4>{
      \mintcmm[highlightlines=3]{../src/notrace/camlNotrace\_\_fun_347.cmm}
      \listcmm{../src/notrace/camlNotrace\_\_all\_somes\_327.cmm}{all\_somes}
    }
    \onslide*<5->{
      \listobjdump[highlightlines={6-9}]{../src/notrace/camlNotrace\_\_fun_347.objdump}{anonymous function from \funcname{all\_somes}}
    }
    \end{column}
  \end{columns}
\end{frame}

\begin{frame}{Backtraces}{Summary table of the multiple kinds of raise}
  \begin{itemize}
    \item When debugging information is enabled and if the backtrace of an exception isn't needed, it's possible to raise with \funcname{raise\_notrace} for performance
    \item When debugging information is \emph{not} enabled, the compiler optimises \funcname{raise} calls to inline assembly (same effect as using \funcname{raise\_notrace})
  \end{itemize}
  \bigskip
  \centering
  \begin{tabular}{| l | c | c | c | c |}
    \hline
    \parbox{10em}{Debug information \\ (\funcarg{-g} flag)}  & \multicolumn{2}{|c|}{Disabled}                                     & \multicolumn{2}{|c|}{Enabled} \\
    \hline
    Raise kind                                               & \funcname{raise} & \funcname{raise\_notrace}                       & \funcname{raise} & \funcname{raise\_notrace} \\
    \hline
    Compilation                                              & \parbox{8em}{inline assembly\\ (optimization)} & inline assembly   & \funcname{caml\_raise\_exn} & inline assembly \\
    \hline
  \end{tabular}
\end{frame}


%
% Takeaway #5
%
\subsection*{Takeaway \#5}
\frameSubsectionTakeaway{}
\againframe<5>{takeaway}
}%
      \onslide*<2>{\providecommand\step{6}%
% Backtraces
%
\subsection{One advantage of raising with debugging information}
\frameSubsection{}{}

\begin{frame}{Backtraces}{Debugging information \& source code}
  \begin{columns}[c]
    \begin{column}{0.5\textwidth}
      \begin{itemize}
        \item Raising an exception is fun and games, but how do we know where the exception originated? $\rightarrow$ With the backtrace associated with the exception!
        \item In order to get a backtrace from the point where an exception was raised, it's necessary to compile with debugging information enabled (\funcarg{-g} flag for the compiler)
        \item Same example as in the `Nested handlers' section
      \end{itemize}
    \end{column}
    \begin{column}{0.5\textwidth}
      \listocaml[firstline=9,lastline=34]{../src/backtrace/backtrace.ml}{backtrace/backtrace.ml}
    \end{column}
  \end{columns}
\end{frame}

\begin{frame}{Backtraces}{CMM and disassembly of \funcname{definitely\_raise}}
  \begin{columns}[c]
    \begin{column}{0.5\textwidth}
      \minipage[c][0.66\textheight][s]{\columnwidth}
      \begin{itemize}
        \item<1-> An effect of compiling with debugging information (\funcarg{-g}): the CMM no longer uses \funcname{raise\_notrace} to raise the exception but \funcname{raise} instead
        \item<2-> Disassembly shows that CMM \funcname{raise} is actually a call to \funcname{caml\_raise\_exn}, a function in assembler provided by the runtime
      \end{itemize}
      \vfill
      \mintocaml[firstline=9,lastline=13,highlightlines=12]{../src/backtrace/backtrace.ml}
      \endminipage
    \end{column}
    \begin{column}{0.5\textwidth}
      \onslide*<1>{
        \mintcmm[highlightlines=5]{../src/backtrace/camlBacktrace\_\_definitely\_raise\_347.cmm}
      }
      \onslide*<2>{
        \mintobjdump[firstline=2,highlightlines=14]{../src/backtrace/camlBacktrace\_\_definitely\_raise\_347.objdump}
      }
    \end{column}
  \end{columns}
\end{frame}

\begin{frame}{Backtraces}{Introducing \funcname{caml\_raise\_exn}}
  \begin{columns}[c]
    \begin{column}{0.5\textwidth}
      \minipage[c][0.66\textheight][s]{\columnwidth}
      \onslide*<1>{
        \begin{itemize}
          \item Raising the exception is performed using \funcname{caml\_raise\_exn} which is
            \begin{itemize}
              \item An assembly function provided by the runtime
              \item Architecture specific
            \end{itemize}
          \item Let's see what it does differently from \funcname{raise\_notrace}\dots
          \item We are about to raise \typename{ExnC} with \funcname{caml\_raise\_exn} from \funcname{definitely\_raise}
        \end{itemize}
      }
      \onslide*<2>{
        \mintobjdump[firstline=2,highlightlines=14]{../src/backtrace/camlBacktrace\_\_definitely\_raise\_347.objdump}
      }
      \vfill
      \mintocaml[firstline=9,lastline=13,highlightlines=12]{../src/backtrace/backtrace.ml}
      \endminipage
    \end{column}
    \begin{column}{0.5\textwidth}
      \centering
      \providecommand\step{1}\input{diagrams/backtrace/backtrace.tex}
    \end{column}
  \end{columns}
\end{frame}

\begin{frame}{Backtraces}{But before that, we need more fields from \typename{caml\_domain\_state}}
  \begin{columns}
    \begin{column}{0.5\textwidth}
      \begin{itemize}
        \item Remember \typename{caml\_domain\_state} the per-domain runtime state?
        \item Some other members are of interest to us now\dots
        \item \localname{backtrace\_active} is used to know if the runtime should collect backtraces
        \item \localname{backtrace\_active} can be set by
          \begin{itemize}
            \item The \funcarg{b} flag in the \funcname{OCAMLRUNPARAM} environment variable
            \item \mintinline{ocaml}{Printexc.record_backtrace : bool -> unit} \footnotemark
          \end{itemize}
      \end{itemize}
    \end{column}
    \begin{column}{0.5\textwidth}
      \begin{listing}
        \mintc[highlightlines={9-11}]{src/ocaml/domain\_state\_backtrace.h}
        \caption{runtime/caml/domain\_state.h}
      \end{listing}
    \end{column}
  \end{columns}
  \footnotetext[1]{https://v2.ocaml.org/api/Printexc.html}
\end{frame}

\newcommand\listCamlRaiseExn[1][]{
  \listasm[firstline=702,lastline=725,#1]{../ocaml/runtime/amd64.S}{runtime/amd64.S:702}}

\begin{frame}{Backtraces}{Walk through \funcname{caml\_raise\_exn}}
  \begin{columns}[T]
    \begin{column}{0.5\textwidth}
      \begin{itemize}
        \item<1-> First checks whether exceptions backtrace should be recorded
        \item<1-> By checking the \funcarg{domain\_state->backtrace\_active} value
        \item<2-> If not, acts as \funcname{raise\_notrace} with \funcname{RESTORE\_EXN\_HANDLER\_OCAML}
          \begin{itemize}
            \item Set \regname{RSP} to \localname{domain\_state->exn\_handler} value
            \item Restore \localname{domain\_state->exn\_handler} from trap
            \item Jump with \funcname{ret} (same effect as using \funcname{pop} and \funcname{jmp})
          \end{itemize}
        \item<3-> Otherwise, switches to the C stack to be able to execute C code
        \item<4-> Calls \funcname{caml\_stash\_backtrace}, a C function provided by the runtime\ignorespaces
          \onslide<6->{, with
            \begin{itemize}
              \item \funcarg{exn}: exception value
              \item \funcarg{pc}: code address of the raising point
              \item \funcarg{sp}: stack address of the raising function
              \item \funcarg{trapsp}: stack address of the next handler
            \end{itemize}
          }
      \end{itemize}
      \bigskip
      \mintocaml[firstline=9,lastline=13,highlightlines=12]{../src/backtrace/backtrace.ml}
    \end{column}
    \begin{column}{0.5\textwidth}
      \raggedleft
      \onslide*<1,3-4>{
        \mintc[firstline=190,lastline=192]{../ocaml/runtime/amd64.S}
        \mintc[firstline=234,lastline=237]{../ocaml/runtime/amd64.S}
      }
      \onslide*<2>{
        \mintc[firstline=190,lastline=192,highlightlines={191-192}]{../ocaml/runtime/amd64.S}
        \mintc[firstline=234,lastline=237,highlightlines={235-237}]{../ocaml/runtime/amd64.S}
      }
      \onslide*<1>{\listCamlRaiseExn[highlightlines=705]{}}%
      \onslide*<2>{\listCamlRaiseExn[highlightlines={707-708}]{}}%
      \onslide*<3>{\listCamlRaiseExn[highlightlines={712-714}]{}}%
      \onslide*<4>{\listCamlRaiseExn[highlightlines={715-720}]{}}%
      \onslide*<5>{\providecommand\step{1}\input{diagrams/backtrace/backtrace.tex}}%
      \onslide*<6>{\providecommand\step{2}\input{diagrams/backtrace/backtrace.tex}}%
    \end{column}
  \end{columns}
\end{frame}

\newcommand\listStashBacktrace[1][]{
  \listc[firstline=91,lastline=120,#1]{../ocaml/runtime/backtrace\_nat.c}{runtime/backtrace\_nat.c:91}}

\begin{frame}{Backtraces}{Introducing \funcname{caml\_stash\_backtrace}}
  \begin{columns}[c]
    \begin{column}{0.5\textwidth}
      \minipage[c][0.66\textheight][s]{\columnwidth}
      \begin{itemize}
        \item<1-> A C function provided by the runtime (specific to native code)
        \item<2-> It scans the stack, between the stack pointer at the raising point up to the next trap, in order to collect the names of the detected function frames
          \onslide*<2>{
            \begin{itemize}
              \item And more information, like line number, column number...
            \end{itemize}
          }
        \item<3-> It uses the same technique and information as the Garbage Collector (GC) uses to scan the values live on the stack
          \onslide*<4>{
            \begin{itemize}
              \item \typename{frame\_descr} are at the core of stack unwinding
            \end{itemize}
          }
      \end{itemize}
      \vfill
      \mintocaml[firstline=9,lastline=13,highlightlines=12]{../src/backtrace/backtrace.ml}
      \endminipage
    \end{column}
    \begin{column}{0.5\textwidth}
      \onslide*<1>{\listStashBacktrace[highlightlines=91]{}}
      \onslide*<2>{\listStashBacktrace[highlightlines={91,117-118}]{}}
      \onslide*<3->{\listStashBacktrace[highlightlines={94,106,109-110}]{}}
    \end{column}
  \end{columns}
\end{frame}

\newcommand\listFrameDescr[1][]{
  \listocaml[firstline=111,lastline=124,#1]{../ocaml/asmcomp/emitaux.ml}{asmcomp/emitaux.ml:111}}
\newcommand\listDebugInfo[1][]{
  \listocaml[firstline=38,lastline=49,#1]{../ocaml/lambda/debuginfo.mli}{lambda/debuginfo.mli:38}}

\begin{frame}{Backtraces}{\typename{frame\_descr} and \typename{frame\_debuginfo} in a nutshell}
  \begin{columns}[c]
    \begin{column}{0.5\textwidth}
      \begin{itemize}
        \item<1-> For every return address in an OCaml program, there is a associated \typename{frame\_descr}
        \item<2-> A \typename{frame\_descr} specifies
        \begin{itemize}
          \item<2-> The size of the function stack frame
          \item<3-> Where live values are on the stack (so that the GC can mark them as live and prevent from being swept)
        \end{itemize}
        \item<4-> When compiled with debug information, \typename{frame\_descr} will be linked to a \typename{frame\_debuginfo}
        \begin{itemize}
          \item<5-> Contains information about the position in the source code (file name, line and column number)
        \end{itemize}
      \end{itemize}
    \end{column}
    \begin{column}{0.5\textwidth}
        \onslide*<1>{\listFrameDescr[highlightlines={118-122}]{}}
        \onslide*<2>{\listFrameDescr[highlightlines={120}]{}}
        \onslide*<3>{\listFrameDescr[highlightlines={121}]{}}
        \onslide*<4->{\listFrameDescr[highlightlines={122}]{}}
        \medskip
        \onslide*<1-3>{\listDebugInfo{}}
        \onslide*<4>{\listDebugInfo[highlightlines={38,49}]{}}
        \onslide*<5>{\listDebugInfo[highlightlines={39-46}]{}}
    \end{column}
  \end{columns}
\end{frame}

\begin{frame}{Backtraces}{Scanning the stack with \typename{frame\_descr}}
  \begin{columns}[c]
    \begin{column}{0.5\textwidth}
      \begin{itemize}
        \item Given the return address of the code that raises the exception by calling \funcname{caml\_raise\_exn} (\funcarg{pc} argument of \funcname{caml\_stash\_backtrace})
        \item And the position of the stack pointer (\funcarg{sp} argument of \funcname{caml\_stash\_backtrace})
        \item The frame size given by \typename{frame\_descr}.\funcarg{frame\_size}, allows to find the end of the previous frame
        \item And the return address of the caller of this function
        \bigskip
        \onslide<3->{
        \item Note that traps (here the reddish \funcname{ExnA}, \funcname{ExnB} and \funcname{ExnC} blocks) belong to the frame that installed them, and so the frame size changes when traps are removed
        }
      \end{itemize}
    \end{column}
    \begin{column}{0.5\textwidth}
      \raggedleft
      \onslide*<1>{\providecommand\step{2}\input{diagrams/backtrace/backtrace.tex}}%
      \onslide*<2>{\providecommand\step{1}\input{diagrams/backtrace/scanstack.tex}}%
      \onslide*<3>{\providecommand\step{2}\input{diagrams/backtrace/scanstack.tex}}%
      \onslide*<4>{\providecommand\step{3}\input{diagrams/backtrace/scanstack.tex}}%
      \onslide*<5>{\providecommand\step{4}\input{diagrams/backtrace/scanstack.tex}}%
      \onslide*<6>{\providecommand\step{5}\input{diagrams/backtrace/scanstack.tex}}%
    \end{column}
  \end{columns}
\end{frame}

% Duplicate of previous-previous slide
\begin{frame}{Backtraces}{Continuing on \funcname{caml\_stash\_backtrace}}
  \begin{columns}[c]
    \begin{column}{0.5\textwidth}
      \minipage[c][0.75\textheight][s]{\columnwidth}
      \begin{itemize}
          \color{gray}
        \item A C function provided by the runtime (specific to native code)
        \item It scans the stack, between the stack pointer at the raising point up to the next trap, in order to collect the names of the detected function frames
        \item It uses the same technique and information as the Garbage Collector (GC) uses to scan the values live on the stack
      \end{itemize}
      \begin{itemize}
        \item For each function frame detected, store a pointer to the \typename{frame\_descr} in the \localname{domain\_state->backtrace\_buffer} array, effectively constructing the backtrace
      \end{itemize}
      \onslide<2->{
        \begin{itemize}
            \smallskip
          \item Constructed backtrace:
            \funcname{definitely\_raise},
            \onslide<3->{\funcname{probably\_raise}}
        \end{itemize}
      }
      \vfill
      \mintocaml[firstline=9,lastline=13,highlightlines=12]{../src/backtrace/backtrace.ml}
      \endminipage
    \end{column}
    \begin{column}{0.5\textwidth}
      \raggedleft
      \onslide*<1>{\listStashBacktrace[highlightlines={114-115}]{}}
      \onslide*<2>{\providecommand\step{3}\input{diagrams/backtrace/backtrace.tex}}%
      \onslide*<3>{\providecommand\step{4}\input{diagrams/backtrace/backtrace.tex}}%
    \end{column}
  \end{columns}
\end{frame}

\begin{frame}{Backtraces}{Back to \funcname{caml\_raise\_exn}}
  \begin{columns}[c]
    \begin{column}{0.5\textwidth}
      \minipage[c][0.66\textheight][s]{\columnwidth}
      \begin{itemize}\color{gray}
        \item Checks wheteher exceptions backtrace should be recorded, by checking the \funcarg{domain\_state->backtrace\_active} value
        \item Switches to the C stack to be able to execute C code
        \item Calls \funcname{caml\_stash\_backtrace}, to collect the backtrace up to the next trap
      \end{itemize}
      \onslide*<2->{
        \begin{itemize}
          \item<2-> Executes the exception handler in \funcname{probably\_raise} with \funcname{RESTORE\_EXN\_HANDLER\_OCAML}
            \begin{itemize}
              \item Set \regname{RSP} to \localname{domain\_state->exn\_handler} value
              \item Restore \localname{domain\_state->exn\_handler} from trap
              \item Jump to the handler code with \funcname{ret}
            \end{itemize}
        \end{itemize}
      }
      \vfill
      \onslide*<1>{\mintocaml[firstline=9,lastline=13,highlightlines=12]{../src/backtrace/backtrace.ml}}
      \onslide*<2->{\mintocaml[firstline=15,lastline=21,highlightlines={19-21}]{../src/backtrace/backtrace.ml}}
      \endminipage
    \end{column}
    \begin{column}{0.5\textwidth}
      \centering
      \onslide*<1>{
        \mintc[firstline=190,lastline=192]{../ocaml/runtime/amd64.S}
        \mintc[firstline=234,lastline=237]{../ocaml/runtime/amd64.S}
      }
      \onslide*<2>{
        \mintc[firstline=190,lastline=192,highlightlines={191-192}]{../ocaml/runtime/amd64.S}
        \mintc[firstline=234,lastline=237,highlightlines={235-237}]{../ocaml/runtime/amd64.S}
      }
      \onslide*<1>{\listCamlRaiseExn[highlightlines=720]{}}
      \onslide*<2>{\listCamlRaiseExn[highlightlines=722]{}}
      \onslide*<3->{\providecommand\step{5}\input{diagrams/backtrace/backtrace.tex}}
    \end{column}
  \end{columns}
\end{frame}

\begin{frame}{Backtraces}{\funcname{caml\_probably\_raise}'s handler doesn't catch \typename{ExnC}}
  \begin{columns}[c]
    \begin{column}{0.45\textwidth}
      \minipage[c][0.75\textheight][s]{\columnwidth}
      \begin{itemize}
        \item<1-> \funcname{match \dots with}\footnotemark pattern matching can also match exceptions
        \item<1-> The CMM of \funcname{probably\_raise} shows that this \funcname{match \dots with} is implemented with a pattern matching and a \funcname{try \dots with}
        \item<1-> But this one only matches on \typename{ExnA} exception
        \item<2-> Since the pattern matching fails to catch the exception, \funcname{reraise} is called with our current \typename{ExnC} exception
        \item<3-> Disassembly shows that \funcname{reraise} is actually a call to \funcname{caml\_reraise\_exn}, which is also an assembly function provided by the runtime
      \end{itemize}
      \vfill
      \mintocaml[firstline=15,lastline=21,highlightlines={18-21}]{../src/backtrace/backtrace.ml}
      \endminipage
    \end{column}
    \begin{column}{0.55\textwidth}
      \centering
      \onslide*<1>{\mintcmm[highlightlines={10-18}]{../src/backtrace/camlBacktrace\_\_probably\_raise\_351.cmm}}
      \onslide*<2>{\listcmm[highlightlines=18]{../src/backtrace/camlBacktrace\_\_probably\_raise_351.cmm}{CMM of probably\_raise}}
      \onslide*<3>{\mintobjdump[firstline=5,lastline=35,highlightlines=31]{../src/backtrace/camlBacktrace\_\_probably\_raise_351.objdump}}
    \end{column}
  \end{columns}
  \only<1>{\footnotetext[1]{https://blog.janestreet.com/pattern-matching-and-exception-handling-unite/}}
\end{frame}

\newcommand\listCamlReraiseExn[1][]{
  \listasm[firstline=727,lastline=734,#1]{../ocaml/runtime/amd64.S}{runtime/amd64.S:727}}

\begin{frame}{Backtraces}{Introducing \funcname{caml\_reraise\_exn}}
  \begin{columns}[c]
    \begin{column}{0.5\textwidth}
      \minipage[c][0.66\textheight][s]{\columnwidth}
      \begin{itemize}
        \item<1-> The exception have to be reraised to the next handler
        \item<2-> First, checks if the backtrace should be collected with \funcarg{domain\_state->backtrace\_active}
        \only<3>{
        \begin{itemize}
            \item If not, behaves like a \funcname{raise\_notrace} with \funcname{RESTORE\_EXN\_HANDLER\_OCAML}
        \end{itemize}
        }
        \item<4-> Jumps into \funcname{caml\_raise\_exn}
        \item<5-> Calls \funcname{caml\_stash\_backtrace} again, to scan the next part of the stack: from the trap just pop-ed to the next trap
      \end{itemize}
      \vfill
      \mintocaml[firstline=15,lastline=21,highlightlines={18-21}]{../src/backtrace/backtrace.ml}
      \endminipage
    \end{column}
    \begin{column}{0.5\textwidth}
      \raggedleft
      \onslide*<1>{\listCamlReraiseExn{}}
      \onslide*<2>{\listCamlReraiseExn[highlightlines=729]{}}
      \onslide*<3>{
        \mintc[firstline=190,lastline=192,highlightlines={191-192}]{../ocaml/runtime/amd64.S}
        \mintc[firstline=234,lastline=237,highlightlines={235-237}]{../ocaml/runtime/amd64.S}
        \medskip
        \listCamlReraiseExn[highlightlines=731]{}
      }
      \onslide*<4-5>{
        % TODO refactor with \listCamlReraiseExn
        \mintasm[firstline=727,lastline=734,highlightlines=730]{../ocaml/runtime/amd64.S}
        \medskip
      }
      \onslide*<4>{\listCamlRaiseExn[highlightlines=711]{}}
      \onslide*<5>{\listCamlRaiseExn[highlightlines=720]{}}
    \end{column}
  \end{columns}
\end{frame}

\begin{frame}{Backtraces}{Backtrace collection continues}
  \begin{columns}[c]
    \begin{column}{0.55\textwidth}
      \minipage[c][0.75\textheight][s]{\columnwidth}
      \begin{itemize}
        \item<1-> \funcname{caml\_stash\_backtrace} scans the older part of the stack, up to the next trap
        \item<6-> Controls jumps to the nested exception handler in \funcname{maybe\_raise}, which matches only on \typename{ExnB}
        \item<7-> \funcname{caml\_reraise\_exn} is called again, backtrace collection resumes with \funcname{caml\_stash\_backtrace}
      \end{itemize}
      \medskip
      \begin{itemize}
        \item<2-> Constructed backtrace:
        \begin{itemize}
          \item <2-> \funcname{definitely\_raise}
          \item <2-> \funcname{probably\_raise}
          \item <3-> \funcname{probably\_raise} (re-raise)
          \item <4-> \funcname{maybe\_raise}
          \item <5-> \funcname{main}
          \item <8-> \funcname{main} (re-raise)
        \end{itemize}
      \end{itemize}
      \vfill
      \onslide*<1-5>{\mintocaml[firstline=15,lastline=21]{../src/backtrace/backtrace.ml}}
      \onslide*<6>{\mintocaml[firstline=28,lastline=34,highlightlines=31]{../src/backtrace/backtrace.ml}}
      \onslide*<7->{\mintocaml[firstline=28,lastline=34,highlightlines=33]{../src/backtrace/backtrace.ml}}
      \endminipage
    \end{column}
    \begin{column}{0.45\textwidth}
      \raggedleft
      \onslide*<1>{\providecommand\step{6}\input{diagrams/backtrace/backtrace.tex}}%
      \onslide*<2>{\providecommand\step{6}\input{diagrams/backtrace/backtrace.tex}}%
      \onslide*<3>{\providecommand\step{7}\input{diagrams/backtrace/backtrace.tex}}%
      \onslide*<4>{\providecommand\step{8}\input{diagrams/backtrace/backtrace.tex}}%
      \onslide*<5>{\providecommand\step{9}\input{diagrams/backtrace/backtrace.tex}}%
      \onslide*<6>{\providecommand\step{10}\input{diagrams/backtrace/backtrace.tex}}%
      \onslide*<7>{\providecommand\step{11}\input{diagrams/backtrace/backtrace.tex}}%
      \onslide*<8>{\providecommand\step{12}\input{diagrams/backtrace/backtrace.tex}}%
      \onslide*<9>{\providecommand\step{13}\input{diagrams/backtrace/backtrace.tex}}%
    \end{column}
  \end{columns}
\end{frame}

\begin{frame}{Backtraces}{Printing the backtrace}
  \begin{columns}[c]
    \begin{column}{0.45\textwidth}
      \minipage[c][0.66\textheight][s]{\columnwidth}
      \begin{itemize}
        \item<1-> The exception backtrace can now be printed to \funcarg{stdout} with \funcname{Printexc.print_backtrace}
        \item<2-> First the exception backtrace is retrieved into an OCaml array with \funcname{get_raw_backtrace }
        \item<3-> Which is implemented as the \funcname{caml_get_exception_raw_backtrace} C function in the runtime
        \item<4-> And finally printed with \funcname{print_exception_backtrace}
      \end{itemize}
      \vfill
      \mintocaml[firstline=28,lastline=34,highlightlines=33]{../src/backtrace/backtrace.ml}
      \endminipage
    \end{column}
    \begin{column}{0.55\textwidth}
      \onslide*<1,2>{
        \mintocaml[firstline=100,lastline=101]{../ocaml/stdlib/printexc.ml}
      }
      \only<1>{
        \listocaml[firstline=170,lastline=172]{../ocaml/stdlib/printexc.ml}{stdlib/printexc.ml:170}
      }
      \only<2>{
        \listocaml[firstline=170,lastline=172,highlightlines=172]{../ocaml/stdlib/printexc.ml}{stdlib/printexc.ml:170}
      }
      \only<3>{
        \mintc[firstline=154,lastline=156]{../ocaml/runtime/backtrace.c}
        \listc[firstline=171,lastline=192,highlightlines={184-187}]{../ocaml/runtime/backtrace.c}{runtime/backtrace.c:154}
      }
      \only<4>{
        \listocaml[firstline=155,lastline=165]{../ocaml/stdlib/printexc.ml}{stdlib/printexc.ml:55}
      }
    \end{column}
  \end{columns}
\end{frame}


\subsection{The multiple kinds of raise}
\frameSubsection{}{}

\begin{frame}{Backtraces}{When backtraces are not needed}
  \begin{columns}[c]
    \begin{column}{0.45\textwidth}
      \minipage[c][0.66\textheight][s]{\columnwidth}
      \begin{itemize}
        \item<1-> What if the backtrace for an exception is never needed?
        \item<2-> It's possible to raise the exception explicitly with \funcname{raise\_notrace} to make raising as cheap as possible
        \item<3-> An example of use in the \funcname{dynlink} library of the OCaml compiler
      \end{itemize}
      \vfill
      \onslide<3->{
      \listocaml[firstline=205,lastline=209,highlightlines=207]{../ocaml/otherlibs/dynlink/dynlink\_compilerlibs/misc.ml}{otherlibs/dynlink/dynlink\_compilerlibs/misc.ml:205}
      }
      \endminipage
    \end{column}
    \begin{column}{0.55\textwidth}
    \only<4>{
      \mintcmm[highlightlines=3]{../src/notrace/camlNotrace\_\_fun_347.cmm}
      \listcmm{../src/notrace/camlNotrace\_\_all\_somes\_327.cmm}{all\_somes}
    }
    \onslide*<5->{
      \listobjdump[highlightlines={6-9}]{../src/notrace/camlNotrace\_\_fun_347.objdump}{anonymous function from \funcname{all\_somes}}
    }
    \end{column}
  \end{columns}
\end{frame}

\begin{frame}{Backtraces}{Summary table of the multiple kinds of raise}
  \begin{itemize}
    \item When debugging information is enabled and if the backtrace of an exception isn't needed, it's possible to raise with \funcname{raise\_notrace} for performance
    \item When debugging information is \emph{not} enabled, the compiler optimises \funcname{raise} calls to inline assembly (same effect as using \funcname{raise\_notrace})
  \end{itemize}
  \bigskip
  \centering
  \begin{tabular}{| l | c | c | c | c |}
    \hline
    \parbox{10em}{Debug information \\ (\funcarg{-g} flag)}  & \multicolumn{2}{|c|}{Disabled}                                     & \multicolumn{2}{|c|}{Enabled} \\
    \hline
    Raise kind                                               & \funcname{raise} & \funcname{raise\_notrace}                       & \funcname{raise} & \funcname{raise\_notrace} \\
    \hline
    Compilation                                              & \parbox{8em}{inline assembly\\ (optimization)} & inline assembly   & \funcname{caml\_raise\_exn} & inline assembly \\
    \hline
  \end{tabular}
\end{frame}


%
% Takeaway #5
%
\subsection*{Takeaway \#5}
\frameSubsectionTakeaway{}
\againframe<5>{takeaway}
}%
      \onslide*<3>{\providecommand\step{7}%
% Backtraces
%
\subsection{One advantage of raising with debugging information}
\frameSubsection{}{}

\begin{frame}{Backtraces}{Debugging information \& source code}
  \begin{columns}[c]
    \begin{column}{0.5\textwidth}
      \begin{itemize}
        \item Raising an exception is fun and games, but how do we know where the exception originated? $\rightarrow$ With the backtrace associated with the exception!
        \item In order to get a backtrace from the point where an exception was raised, it's necessary to compile with debugging information enabled (\funcarg{-g} flag for the compiler)
        \item Same example as in the `Nested handlers' section
      \end{itemize}
    \end{column}
    \begin{column}{0.5\textwidth}
      \listocaml[firstline=9,lastline=34]{../src/backtrace/backtrace.ml}{backtrace/backtrace.ml}
    \end{column}
  \end{columns}
\end{frame}

\begin{frame}{Backtraces}{CMM and disassembly of \funcname{definitely\_raise}}
  \begin{columns}[c]
    \begin{column}{0.5\textwidth}
      \minipage[c][0.66\textheight][s]{\columnwidth}
      \begin{itemize}
        \item<1-> An effect of compiling with debugging information (\funcarg{-g}): the CMM no longer uses \funcname{raise\_notrace} to raise the exception but \funcname{raise} instead
        \item<2-> Disassembly shows that CMM \funcname{raise} is actually a call to \funcname{caml\_raise\_exn}, a function in assembler provided by the runtime
      \end{itemize}
      \vfill
      \mintocaml[firstline=9,lastline=13,highlightlines=12]{../src/backtrace/backtrace.ml}
      \endminipage
    \end{column}
    \begin{column}{0.5\textwidth}
      \onslide*<1>{
        \mintcmm[highlightlines=5]{../src/backtrace/camlBacktrace\_\_definitely\_raise\_347.cmm}
      }
      \onslide*<2>{
        \mintobjdump[firstline=2,highlightlines=14]{../src/backtrace/camlBacktrace\_\_definitely\_raise\_347.objdump}
      }
    \end{column}
  \end{columns}
\end{frame}

\begin{frame}{Backtraces}{Introducing \funcname{caml\_raise\_exn}}
  \begin{columns}[c]
    \begin{column}{0.5\textwidth}
      \minipage[c][0.66\textheight][s]{\columnwidth}
      \onslide*<1>{
        \begin{itemize}
          \item Raising the exception is performed using \funcname{caml\_raise\_exn} which is
            \begin{itemize}
              \item An assembly function provided by the runtime
              \item Architecture specific
            \end{itemize}
          \item Let's see what it does differently from \funcname{raise\_notrace}\dots
          \item We are about to raise \typename{ExnC} with \funcname{caml\_raise\_exn} from \funcname{definitely\_raise}
        \end{itemize}
      }
      \onslide*<2>{
        \mintobjdump[firstline=2,highlightlines=14]{../src/backtrace/camlBacktrace\_\_definitely\_raise\_347.objdump}
      }
      \vfill
      \mintocaml[firstline=9,lastline=13,highlightlines=12]{../src/backtrace/backtrace.ml}
      \endminipage
    \end{column}
    \begin{column}{0.5\textwidth}
      \centering
      \providecommand\step{1}\input{diagrams/backtrace/backtrace.tex}
    \end{column}
  \end{columns}
\end{frame}

\begin{frame}{Backtraces}{But before that, we need more fields from \typename{caml\_domain\_state}}
  \begin{columns}
    \begin{column}{0.5\textwidth}
      \begin{itemize}
        \item Remember \typename{caml\_domain\_state} the per-domain runtime state?
        \item Some other members are of interest to us now\dots
        \item \localname{backtrace\_active} is used to know if the runtime should collect backtraces
        \item \localname{backtrace\_active} can be set by
          \begin{itemize}
            \item The \funcarg{b} flag in the \funcname{OCAMLRUNPARAM} environment variable
            \item \mintinline{ocaml}{Printexc.record_backtrace : bool -> unit} \footnotemark
          \end{itemize}
      \end{itemize}
    \end{column}
    \begin{column}{0.5\textwidth}
      \begin{listing}
        \mintc[highlightlines={9-11}]{src/ocaml/domain\_state\_backtrace.h}
        \caption{runtime/caml/domain\_state.h}
      \end{listing}
    \end{column}
  \end{columns}
  \footnotetext[1]{https://v2.ocaml.org/api/Printexc.html}
\end{frame}

\newcommand\listCamlRaiseExn[1][]{
  \listasm[firstline=702,lastline=725,#1]{../ocaml/runtime/amd64.S}{runtime/amd64.S:702}}

\begin{frame}{Backtraces}{Walk through \funcname{caml\_raise\_exn}}
  \begin{columns}[T]
    \begin{column}{0.5\textwidth}
      \begin{itemize}
        \item<1-> First checks whether exceptions backtrace should be recorded
        \item<1-> By checking the \funcarg{domain\_state->backtrace\_active} value
        \item<2-> If not, acts as \funcname{raise\_notrace} with \funcname{RESTORE\_EXN\_HANDLER\_OCAML}
          \begin{itemize}
            \item Set \regname{RSP} to \localname{domain\_state->exn\_handler} value
            \item Restore \localname{domain\_state->exn\_handler} from trap
            \item Jump with \funcname{ret} (same effect as using \funcname{pop} and \funcname{jmp})
          \end{itemize}
        \item<3-> Otherwise, switches to the C stack to be able to execute C code
        \item<4-> Calls \funcname{caml\_stash\_backtrace}, a C function provided by the runtime\ignorespaces
          \onslide<6->{, with
            \begin{itemize}
              \item \funcarg{exn}: exception value
              \item \funcarg{pc}: code address of the raising point
              \item \funcarg{sp}: stack address of the raising function
              \item \funcarg{trapsp}: stack address of the next handler
            \end{itemize}
          }
      \end{itemize}
      \bigskip
      \mintocaml[firstline=9,lastline=13,highlightlines=12]{../src/backtrace/backtrace.ml}
    \end{column}
    \begin{column}{0.5\textwidth}
      \raggedleft
      \onslide*<1,3-4>{
        \mintc[firstline=190,lastline=192]{../ocaml/runtime/amd64.S}
        \mintc[firstline=234,lastline=237]{../ocaml/runtime/amd64.S}
      }
      \onslide*<2>{
        \mintc[firstline=190,lastline=192,highlightlines={191-192}]{../ocaml/runtime/amd64.S}
        \mintc[firstline=234,lastline=237,highlightlines={235-237}]{../ocaml/runtime/amd64.S}
      }
      \onslide*<1>{\listCamlRaiseExn[highlightlines=705]{}}%
      \onslide*<2>{\listCamlRaiseExn[highlightlines={707-708}]{}}%
      \onslide*<3>{\listCamlRaiseExn[highlightlines={712-714}]{}}%
      \onslide*<4>{\listCamlRaiseExn[highlightlines={715-720}]{}}%
      \onslide*<5>{\providecommand\step{1}\input{diagrams/backtrace/backtrace.tex}}%
      \onslide*<6>{\providecommand\step{2}\input{diagrams/backtrace/backtrace.tex}}%
    \end{column}
  \end{columns}
\end{frame}

\newcommand\listStashBacktrace[1][]{
  \listc[firstline=91,lastline=120,#1]{../ocaml/runtime/backtrace\_nat.c}{runtime/backtrace\_nat.c:91}}

\begin{frame}{Backtraces}{Introducing \funcname{caml\_stash\_backtrace}}
  \begin{columns}[c]
    \begin{column}{0.5\textwidth}
      \minipage[c][0.66\textheight][s]{\columnwidth}
      \begin{itemize}
        \item<1-> A C function provided by the runtime (specific to native code)
        \item<2-> It scans the stack, between the stack pointer at the raising point up to the next trap, in order to collect the names of the detected function frames
          \onslide*<2>{
            \begin{itemize}
              \item And more information, like line number, column number...
            \end{itemize}
          }
        \item<3-> It uses the same technique and information as the Garbage Collector (GC) uses to scan the values live on the stack
          \onslide*<4>{
            \begin{itemize}
              \item \typename{frame\_descr} are at the core of stack unwinding
            \end{itemize}
          }
      \end{itemize}
      \vfill
      \mintocaml[firstline=9,lastline=13,highlightlines=12]{../src/backtrace/backtrace.ml}
      \endminipage
    \end{column}
    \begin{column}{0.5\textwidth}
      \onslide*<1>{\listStashBacktrace[highlightlines=91]{}}
      \onslide*<2>{\listStashBacktrace[highlightlines={91,117-118}]{}}
      \onslide*<3->{\listStashBacktrace[highlightlines={94,106,109-110}]{}}
    \end{column}
  \end{columns}
\end{frame}

\newcommand\listFrameDescr[1][]{
  \listocaml[firstline=111,lastline=124,#1]{../ocaml/asmcomp/emitaux.ml}{asmcomp/emitaux.ml:111}}
\newcommand\listDebugInfo[1][]{
  \listocaml[firstline=38,lastline=49,#1]{../ocaml/lambda/debuginfo.mli}{lambda/debuginfo.mli:38}}

\begin{frame}{Backtraces}{\typename{frame\_descr} and \typename{frame\_debuginfo} in a nutshell}
  \begin{columns}[c]
    \begin{column}{0.5\textwidth}
      \begin{itemize}
        \item<1-> For every return address in an OCaml program, there is a associated \typename{frame\_descr}
        \item<2-> A \typename{frame\_descr} specifies
        \begin{itemize}
          \item<2-> The size of the function stack frame
          \item<3-> Where live values are on the stack (so that the GC can mark them as live and prevent from being swept)
        \end{itemize}
        \item<4-> When compiled with debug information, \typename{frame\_descr} will be linked to a \typename{frame\_debuginfo}
        \begin{itemize}
          \item<5-> Contains information about the position in the source code (file name, line and column number)
        \end{itemize}
      \end{itemize}
    \end{column}
    \begin{column}{0.5\textwidth}
        \onslide*<1>{\listFrameDescr[highlightlines={118-122}]{}}
        \onslide*<2>{\listFrameDescr[highlightlines={120}]{}}
        \onslide*<3>{\listFrameDescr[highlightlines={121}]{}}
        \onslide*<4->{\listFrameDescr[highlightlines={122}]{}}
        \medskip
        \onslide*<1-3>{\listDebugInfo{}}
        \onslide*<4>{\listDebugInfo[highlightlines={38,49}]{}}
        \onslide*<5>{\listDebugInfo[highlightlines={39-46}]{}}
    \end{column}
  \end{columns}
\end{frame}

\begin{frame}{Backtraces}{Scanning the stack with \typename{frame\_descr}}
  \begin{columns}[c]
    \begin{column}{0.5\textwidth}
      \begin{itemize}
        \item Given the return address of the code that raises the exception by calling \funcname{caml\_raise\_exn} (\funcarg{pc} argument of \funcname{caml\_stash\_backtrace})
        \item And the position of the stack pointer (\funcarg{sp} argument of \funcname{caml\_stash\_backtrace})
        \item The frame size given by \typename{frame\_descr}.\funcarg{frame\_size}, allows to find the end of the previous frame
        \item And the return address of the caller of this function
        \bigskip
        \onslide<3->{
        \item Note that traps (here the reddish \funcname{ExnA}, \funcname{ExnB} and \funcname{ExnC} blocks) belong to the frame that installed them, and so the frame size changes when traps are removed
        }
      \end{itemize}
    \end{column}
    \begin{column}{0.5\textwidth}
      \raggedleft
      \onslide*<1>{\providecommand\step{2}\input{diagrams/backtrace/backtrace.tex}}%
      \onslide*<2>{\providecommand\step{1}\input{diagrams/backtrace/scanstack.tex}}%
      \onslide*<3>{\providecommand\step{2}\input{diagrams/backtrace/scanstack.tex}}%
      \onslide*<4>{\providecommand\step{3}\input{diagrams/backtrace/scanstack.tex}}%
      \onslide*<5>{\providecommand\step{4}\input{diagrams/backtrace/scanstack.tex}}%
      \onslide*<6>{\providecommand\step{5}\input{diagrams/backtrace/scanstack.tex}}%
    \end{column}
  \end{columns}
\end{frame}

% Duplicate of previous-previous slide
\begin{frame}{Backtraces}{Continuing on \funcname{caml\_stash\_backtrace}}
  \begin{columns}[c]
    \begin{column}{0.5\textwidth}
      \minipage[c][0.75\textheight][s]{\columnwidth}
      \begin{itemize}
          \color{gray}
        \item A C function provided by the runtime (specific to native code)
        \item It scans the stack, between the stack pointer at the raising point up to the next trap, in order to collect the names of the detected function frames
        \item It uses the same technique and information as the Garbage Collector (GC) uses to scan the values live on the stack
      \end{itemize}
      \begin{itemize}
        \item For each function frame detected, store a pointer to the \typename{frame\_descr} in the \localname{domain\_state->backtrace\_buffer} array, effectively constructing the backtrace
      \end{itemize}
      \onslide<2->{
        \begin{itemize}
            \smallskip
          \item Constructed backtrace:
            \funcname{definitely\_raise},
            \onslide<3->{\funcname{probably\_raise}}
        \end{itemize}
      }
      \vfill
      \mintocaml[firstline=9,lastline=13,highlightlines=12]{../src/backtrace/backtrace.ml}
      \endminipage
    \end{column}
    \begin{column}{0.5\textwidth}
      \raggedleft
      \onslide*<1>{\listStashBacktrace[highlightlines={114-115}]{}}
      \onslide*<2>{\providecommand\step{3}\input{diagrams/backtrace/backtrace.tex}}%
      \onslide*<3>{\providecommand\step{4}\input{diagrams/backtrace/backtrace.tex}}%
    \end{column}
  \end{columns}
\end{frame}

\begin{frame}{Backtraces}{Back to \funcname{caml\_raise\_exn}}
  \begin{columns}[c]
    \begin{column}{0.5\textwidth}
      \minipage[c][0.66\textheight][s]{\columnwidth}
      \begin{itemize}\color{gray}
        \item Checks wheteher exceptions backtrace should be recorded, by checking the \funcarg{domain\_state->backtrace\_active} value
        \item Switches to the C stack to be able to execute C code
        \item Calls \funcname{caml\_stash\_backtrace}, to collect the backtrace up to the next trap
      \end{itemize}
      \onslide*<2->{
        \begin{itemize}
          \item<2-> Executes the exception handler in \funcname{probably\_raise} with \funcname{RESTORE\_EXN\_HANDLER\_OCAML}
            \begin{itemize}
              \item Set \regname{RSP} to \localname{domain\_state->exn\_handler} value
              \item Restore \localname{domain\_state->exn\_handler} from trap
              \item Jump to the handler code with \funcname{ret}
            \end{itemize}
        \end{itemize}
      }
      \vfill
      \onslide*<1>{\mintocaml[firstline=9,lastline=13,highlightlines=12]{../src/backtrace/backtrace.ml}}
      \onslide*<2->{\mintocaml[firstline=15,lastline=21,highlightlines={19-21}]{../src/backtrace/backtrace.ml}}
      \endminipage
    \end{column}
    \begin{column}{0.5\textwidth}
      \centering
      \onslide*<1>{
        \mintc[firstline=190,lastline=192]{../ocaml/runtime/amd64.S}
        \mintc[firstline=234,lastline=237]{../ocaml/runtime/amd64.S}
      }
      \onslide*<2>{
        \mintc[firstline=190,lastline=192,highlightlines={191-192}]{../ocaml/runtime/amd64.S}
        \mintc[firstline=234,lastline=237,highlightlines={235-237}]{../ocaml/runtime/amd64.S}
      }
      \onslide*<1>{\listCamlRaiseExn[highlightlines=720]{}}
      \onslide*<2>{\listCamlRaiseExn[highlightlines=722]{}}
      \onslide*<3->{\providecommand\step{5}\input{diagrams/backtrace/backtrace.tex}}
    \end{column}
  \end{columns}
\end{frame}

\begin{frame}{Backtraces}{\funcname{caml\_probably\_raise}'s handler doesn't catch \typename{ExnC}}
  \begin{columns}[c]
    \begin{column}{0.45\textwidth}
      \minipage[c][0.75\textheight][s]{\columnwidth}
      \begin{itemize}
        \item<1-> \funcname{match \dots with}\footnotemark pattern matching can also match exceptions
        \item<1-> The CMM of \funcname{probably\_raise} shows that this \funcname{match \dots with} is implemented with a pattern matching and a \funcname{try \dots with}
        \item<1-> But this one only matches on \typename{ExnA} exception
        \item<2-> Since the pattern matching fails to catch the exception, \funcname{reraise} is called with our current \typename{ExnC} exception
        \item<3-> Disassembly shows that \funcname{reraise} is actually a call to \funcname{caml\_reraise\_exn}, which is also an assembly function provided by the runtime
      \end{itemize}
      \vfill
      \mintocaml[firstline=15,lastline=21,highlightlines={18-21}]{../src/backtrace/backtrace.ml}
      \endminipage
    \end{column}
    \begin{column}{0.55\textwidth}
      \centering
      \onslide*<1>{\mintcmm[highlightlines={10-18}]{../src/backtrace/camlBacktrace\_\_probably\_raise\_351.cmm}}
      \onslide*<2>{\listcmm[highlightlines=18]{../src/backtrace/camlBacktrace\_\_probably\_raise_351.cmm}{CMM of probably\_raise}}
      \onslide*<3>{\mintobjdump[firstline=5,lastline=35,highlightlines=31]{../src/backtrace/camlBacktrace\_\_probably\_raise_351.objdump}}
    \end{column}
  \end{columns}
  \only<1>{\footnotetext[1]{https://blog.janestreet.com/pattern-matching-and-exception-handling-unite/}}
\end{frame}

\newcommand\listCamlReraiseExn[1][]{
  \listasm[firstline=727,lastline=734,#1]{../ocaml/runtime/amd64.S}{runtime/amd64.S:727}}

\begin{frame}{Backtraces}{Introducing \funcname{caml\_reraise\_exn}}
  \begin{columns}[c]
    \begin{column}{0.5\textwidth}
      \minipage[c][0.66\textheight][s]{\columnwidth}
      \begin{itemize}
        \item<1-> The exception have to be reraised to the next handler
        \item<2-> First, checks if the backtrace should be collected with \funcarg{domain\_state->backtrace\_active}
        \only<3>{
        \begin{itemize}
            \item If not, behaves like a \funcname{raise\_notrace} with \funcname{RESTORE\_EXN\_HANDLER\_OCAML}
        \end{itemize}
        }
        \item<4-> Jumps into \funcname{caml\_raise\_exn}
        \item<5-> Calls \funcname{caml\_stash\_backtrace} again, to scan the next part of the stack: from the trap just pop-ed to the next trap
      \end{itemize}
      \vfill
      \mintocaml[firstline=15,lastline=21,highlightlines={18-21}]{../src/backtrace/backtrace.ml}
      \endminipage
    \end{column}
    \begin{column}{0.5\textwidth}
      \raggedleft
      \onslide*<1>{\listCamlReraiseExn{}}
      \onslide*<2>{\listCamlReraiseExn[highlightlines=729]{}}
      \onslide*<3>{
        \mintc[firstline=190,lastline=192,highlightlines={191-192}]{../ocaml/runtime/amd64.S}
        \mintc[firstline=234,lastline=237,highlightlines={235-237}]{../ocaml/runtime/amd64.S}
        \medskip
        \listCamlReraiseExn[highlightlines=731]{}
      }
      \onslide*<4-5>{
        % TODO refactor with \listCamlReraiseExn
        \mintasm[firstline=727,lastline=734,highlightlines=730]{../ocaml/runtime/amd64.S}
        \medskip
      }
      \onslide*<4>{\listCamlRaiseExn[highlightlines=711]{}}
      \onslide*<5>{\listCamlRaiseExn[highlightlines=720]{}}
    \end{column}
  \end{columns}
\end{frame}

\begin{frame}{Backtraces}{Backtrace collection continues}
  \begin{columns}[c]
    \begin{column}{0.55\textwidth}
      \minipage[c][0.75\textheight][s]{\columnwidth}
      \begin{itemize}
        \item<1-> \funcname{caml\_stash\_backtrace} scans the older part of the stack, up to the next trap
        \item<6-> Controls jumps to the nested exception handler in \funcname{maybe\_raise}, which matches only on \typename{ExnB}
        \item<7-> \funcname{caml\_reraise\_exn} is called again, backtrace collection resumes with \funcname{caml\_stash\_backtrace}
      \end{itemize}
      \medskip
      \begin{itemize}
        \item<2-> Constructed backtrace:
        \begin{itemize}
          \item <2-> \funcname{definitely\_raise}
          \item <2-> \funcname{probably\_raise}
          \item <3-> \funcname{probably\_raise} (re-raise)
          \item <4-> \funcname{maybe\_raise}
          \item <5-> \funcname{main}
          \item <8-> \funcname{main} (re-raise)
        \end{itemize}
      \end{itemize}
      \vfill
      \onslide*<1-5>{\mintocaml[firstline=15,lastline=21]{../src/backtrace/backtrace.ml}}
      \onslide*<6>{\mintocaml[firstline=28,lastline=34,highlightlines=31]{../src/backtrace/backtrace.ml}}
      \onslide*<7->{\mintocaml[firstline=28,lastline=34,highlightlines=33]{../src/backtrace/backtrace.ml}}
      \endminipage
    \end{column}
    \begin{column}{0.45\textwidth}
      \raggedleft
      \onslide*<1>{\providecommand\step{6}\input{diagrams/backtrace/backtrace.tex}}%
      \onslide*<2>{\providecommand\step{6}\input{diagrams/backtrace/backtrace.tex}}%
      \onslide*<3>{\providecommand\step{7}\input{diagrams/backtrace/backtrace.tex}}%
      \onslide*<4>{\providecommand\step{8}\input{diagrams/backtrace/backtrace.tex}}%
      \onslide*<5>{\providecommand\step{9}\input{diagrams/backtrace/backtrace.tex}}%
      \onslide*<6>{\providecommand\step{10}\input{diagrams/backtrace/backtrace.tex}}%
      \onslide*<7>{\providecommand\step{11}\input{diagrams/backtrace/backtrace.tex}}%
      \onslide*<8>{\providecommand\step{12}\input{diagrams/backtrace/backtrace.tex}}%
      \onslide*<9>{\providecommand\step{13}\input{diagrams/backtrace/backtrace.tex}}%
    \end{column}
  \end{columns}
\end{frame}

\begin{frame}{Backtraces}{Printing the backtrace}
  \begin{columns}[c]
    \begin{column}{0.45\textwidth}
      \minipage[c][0.66\textheight][s]{\columnwidth}
      \begin{itemize}
        \item<1-> The exception backtrace can now be printed to \funcarg{stdout} with \funcname{Printexc.print_backtrace}
        \item<2-> First the exception backtrace is retrieved into an OCaml array with \funcname{get_raw_backtrace }
        \item<3-> Which is implemented as the \funcname{caml_get_exception_raw_backtrace} C function in the runtime
        \item<4-> And finally printed with \funcname{print_exception_backtrace}
      \end{itemize}
      \vfill
      \mintocaml[firstline=28,lastline=34,highlightlines=33]{../src/backtrace/backtrace.ml}
      \endminipage
    \end{column}
    \begin{column}{0.55\textwidth}
      \onslide*<1,2>{
        \mintocaml[firstline=100,lastline=101]{../ocaml/stdlib/printexc.ml}
      }
      \only<1>{
        \listocaml[firstline=170,lastline=172]{../ocaml/stdlib/printexc.ml}{stdlib/printexc.ml:170}
      }
      \only<2>{
        \listocaml[firstline=170,lastline=172,highlightlines=172]{../ocaml/stdlib/printexc.ml}{stdlib/printexc.ml:170}
      }
      \only<3>{
        \mintc[firstline=154,lastline=156]{../ocaml/runtime/backtrace.c}
        \listc[firstline=171,lastline=192,highlightlines={184-187}]{../ocaml/runtime/backtrace.c}{runtime/backtrace.c:154}
      }
      \only<4>{
        \listocaml[firstline=155,lastline=165]{../ocaml/stdlib/printexc.ml}{stdlib/printexc.ml:55}
      }
    \end{column}
  \end{columns}
\end{frame}


\subsection{The multiple kinds of raise}
\frameSubsection{}{}

\begin{frame}{Backtraces}{When backtraces are not needed}
  \begin{columns}[c]
    \begin{column}{0.45\textwidth}
      \minipage[c][0.66\textheight][s]{\columnwidth}
      \begin{itemize}
        \item<1-> What if the backtrace for an exception is never needed?
        \item<2-> It's possible to raise the exception explicitly with \funcname{raise\_notrace} to make raising as cheap as possible
        \item<3-> An example of use in the \funcname{dynlink} library of the OCaml compiler
      \end{itemize}
      \vfill
      \onslide<3->{
      \listocaml[firstline=205,lastline=209,highlightlines=207]{../ocaml/otherlibs/dynlink/dynlink\_compilerlibs/misc.ml}{otherlibs/dynlink/dynlink\_compilerlibs/misc.ml:205}
      }
      \endminipage
    \end{column}
    \begin{column}{0.55\textwidth}
    \only<4>{
      \mintcmm[highlightlines=3]{../src/notrace/camlNotrace\_\_fun_347.cmm}
      \listcmm{../src/notrace/camlNotrace\_\_all\_somes\_327.cmm}{all\_somes}
    }
    \onslide*<5->{
      \listobjdump[highlightlines={6-9}]{../src/notrace/camlNotrace\_\_fun_347.objdump}{anonymous function from \funcname{all\_somes}}
    }
    \end{column}
  \end{columns}
\end{frame}

\begin{frame}{Backtraces}{Summary table of the multiple kinds of raise}
  \begin{itemize}
    \item When debugging information is enabled and if the backtrace of an exception isn't needed, it's possible to raise with \funcname{raise\_notrace} for performance
    \item When debugging information is \emph{not} enabled, the compiler optimises \funcname{raise} calls to inline assembly (same effect as using \funcname{raise\_notrace})
  \end{itemize}
  \bigskip
  \centering
  \begin{tabular}{| l | c | c | c | c |}
    \hline
    \parbox{10em}{Debug information \\ (\funcarg{-g} flag)}  & \multicolumn{2}{|c|}{Disabled}                                     & \multicolumn{2}{|c|}{Enabled} \\
    \hline
    Raise kind                                               & \funcname{raise} & \funcname{raise\_notrace}                       & \funcname{raise} & \funcname{raise\_notrace} \\
    \hline
    Compilation                                              & \parbox{8em}{inline assembly\\ (optimization)} & inline assembly   & \funcname{caml\_raise\_exn} & inline assembly \\
    \hline
  \end{tabular}
\end{frame}


%
% Takeaway #5
%
\subsection*{Takeaway \#5}
\frameSubsectionTakeaway{}
\againframe<5>{takeaway}
}%
      \onslide*<4>{\providecommand\step{8}%
% Backtraces
%
\subsection{One advantage of raising with debugging information}
\frameSubsection{}{}

\begin{frame}{Backtraces}{Debugging information \& source code}
  \begin{columns}[c]
    \begin{column}{0.5\textwidth}
      \begin{itemize}
        \item Raising an exception is fun and games, but how do we know where the exception originated? $\rightarrow$ With the backtrace associated with the exception!
        \item In order to get a backtrace from the point where an exception was raised, it's necessary to compile with debugging information enabled (\funcarg{-g} flag for the compiler)
        \item Same example as in the `Nested handlers' section
      \end{itemize}
    \end{column}
    \begin{column}{0.5\textwidth}
      \listocaml[firstline=9,lastline=34]{../src/backtrace/backtrace.ml}{backtrace/backtrace.ml}
    \end{column}
  \end{columns}
\end{frame}

\begin{frame}{Backtraces}{CMM and disassembly of \funcname{definitely\_raise}}
  \begin{columns}[c]
    \begin{column}{0.5\textwidth}
      \minipage[c][0.66\textheight][s]{\columnwidth}
      \begin{itemize}
        \item<1-> An effect of compiling with debugging information (\funcarg{-g}): the CMM no longer uses \funcname{raise\_notrace} to raise the exception but \funcname{raise} instead
        \item<2-> Disassembly shows that CMM \funcname{raise} is actually a call to \funcname{caml\_raise\_exn}, a function in assembler provided by the runtime
      \end{itemize}
      \vfill
      \mintocaml[firstline=9,lastline=13,highlightlines=12]{../src/backtrace/backtrace.ml}
      \endminipage
    \end{column}
    \begin{column}{0.5\textwidth}
      \onslide*<1>{
        \mintcmm[highlightlines=5]{../src/backtrace/camlBacktrace\_\_definitely\_raise\_347.cmm}
      }
      \onslide*<2>{
        \mintobjdump[firstline=2,highlightlines=14]{../src/backtrace/camlBacktrace\_\_definitely\_raise\_347.objdump}
      }
    \end{column}
  \end{columns}
\end{frame}

\begin{frame}{Backtraces}{Introducing \funcname{caml\_raise\_exn}}
  \begin{columns}[c]
    \begin{column}{0.5\textwidth}
      \minipage[c][0.66\textheight][s]{\columnwidth}
      \onslide*<1>{
        \begin{itemize}
          \item Raising the exception is performed using \funcname{caml\_raise\_exn} which is
            \begin{itemize}
              \item An assembly function provided by the runtime
              \item Architecture specific
            \end{itemize}
          \item Let's see what it does differently from \funcname{raise\_notrace}\dots
          \item We are about to raise \typename{ExnC} with \funcname{caml\_raise\_exn} from \funcname{definitely\_raise}
        \end{itemize}
      }
      \onslide*<2>{
        \mintobjdump[firstline=2,highlightlines=14]{../src/backtrace/camlBacktrace\_\_definitely\_raise\_347.objdump}
      }
      \vfill
      \mintocaml[firstline=9,lastline=13,highlightlines=12]{../src/backtrace/backtrace.ml}
      \endminipage
    \end{column}
    \begin{column}{0.5\textwidth}
      \centering
      \providecommand\step{1}\input{diagrams/backtrace/backtrace.tex}
    \end{column}
  \end{columns}
\end{frame}

\begin{frame}{Backtraces}{But before that, we need more fields from \typename{caml\_domain\_state}}
  \begin{columns}
    \begin{column}{0.5\textwidth}
      \begin{itemize}
        \item Remember \typename{caml\_domain\_state} the per-domain runtime state?
        \item Some other members are of interest to us now\dots
        \item \localname{backtrace\_active} is used to know if the runtime should collect backtraces
        \item \localname{backtrace\_active} can be set by
          \begin{itemize}
            \item The \funcarg{b} flag in the \funcname{OCAMLRUNPARAM} environment variable
            \item \mintinline{ocaml}{Printexc.record_backtrace : bool -> unit} \footnotemark
          \end{itemize}
      \end{itemize}
    \end{column}
    \begin{column}{0.5\textwidth}
      \begin{listing}
        \mintc[highlightlines={9-11}]{src/ocaml/domain\_state\_backtrace.h}
        \caption{runtime/caml/domain\_state.h}
      \end{listing}
    \end{column}
  \end{columns}
  \footnotetext[1]{https://v2.ocaml.org/api/Printexc.html}
\end{frame}

\newcommand\listCamlRaiseExn[1][]{
  \listasm[firstline=702,lastline=725,#1]{../ocaml/runtime/amd64.S}{runtime/amd64.S:702}}

\begin{frame}{Backtraces}{Walk through \funcname{caml\_raise\_exn}}
  \begin{columns}[T]
    \begin{column}{0.5\textwidth}
      \begin{itemize}
        \item<1-> First checks whether exceptions backtrace should be recorded
        \item<1-> By checking the \funcarg{domain\_state->backtrace\_active} value
        \item<2-> If not, acts as \funcname{raise\_notrace} with \funcname{RESTORE\_EXN\_HANDLER\_OCAML}
          \begin{itemize}
            \item Set \regname{RSP} to \localname{domain\_state->exn\_handler} value
            \item Restore \localname{domain\_state->exn\_handler} from trap
            \item Jump with \funcname{ret} (same effect as using \funcname{pop} and \funcname{jmp})
          \end{itemize}
        \item<3-> Otherwise, switches to the C stack to be able to execute C code
        \item<4-> Calls \funcname{caml\_stash\_backtrace}, a C function provided by the runtime\ignorespaces
          \onslide<6->{, with
            \begin{itemize}
              \item \funcarg{exn}: exception value
              \item \funcarg{pc}: code address of the raising point
              \item \funcarg{sp}: stack address of the raising function
              \item \funcarg{trapsp}: stack address of the next handler
            \end{itemize}
          }
      \end{itemize}
      \bigskip
      \mintocaml[firstline=9,lastline=13,highlightlines=12]{../src/backtrace/backtrace.ml}
    \end{column}
    \begin{column}{0.5\textwidth}
      \raggedleft
      \onslide*<1,3-4>{
        \mintc[firstline=190,lastline=192]{../ocaml/runtime/amd64.S}
        \mintc[firstline=234,lastline=237]{../ocaml/runtime/amd64.S}
      }
      \onslide*<2>{
        \mintc[firstline=190,lastline=192,highlightlines={191-192}]{../ocaml/runtime/amd64.S}
        \mintc[firstline=234,lastline=237,highlightlines={235-237}]{../ocaml/runtime/amd64.S}
      }
      \onslide*<1>{\listCamlRaiseExn[highlightlines=705]{}}%
      \onslide*<2>{\listCamlRaiseExn[highlightlines={707-708}]{}}%
      \onslide*<3>{\listCamlRaiseExn[highlightlines={712-714}]{}}%
      \onslide*<4>{\listCamlRaiseExn[highlightlines={715-720}]{}}%
      \onslide*<5>{\providecommand\step{1}\input{diagrams/backtrace/backtrace.tex}}%
      \onslide*<6>{\providecommand\step{2}\input{diagrams/backtrace/backtrace.tex}}%
    \end{column}
  \end{columns}
\end{frame}

\newcommand\listStashBacktrace[1][]{
  \listc[firstline=91,lastline=120,#1]{../ocaml/runtime/backtrace\_nat.c}{runtime/backtrace\_nat.c:91}}

\begin{frame}{Backtraces}{Introducing \funcname{caml\_stash\_backtrace}}
  \begin{columns}[c]
    \begin{column}{0.5\textwidth}
      \minipage[c][0.66\textheight][s]{\columnwidth}
      \begin{itemize}
        \item<1-> A C function provided by the runtime (specific to native code)
        \item<2-> It scans the stack, between the stack pointer at the raising point up to the next trap, in order to collect the names of the detected function frames
          \onslide*<2>{
            \begin{itemize}
              \item And more information, like line number, column number...
            \end{itemize}
          }
        \item<3-> It uses the same technique and information as the Garbage Collector (GC) uses to scan the values live on the stack
          \onslide*<4>{
            \begin{itemize}
              \item \typename{frame\_descr} are at the core of stack unwinding
            \end{itemize}
          }
      \end{itemize}
      \vfill
      \mintocaml[firstline=9,lastline=13,highlightlines=12]{../src/backtrace/backtrace.ml}
      \endminipage
    \end{column}
    \begin{column}{0.5\textwidth}
      \onslide*<1>{\listStashBacktrace[highlightlines=91]{}}
      \onslide*<2>{\listStashBacktrace[highlightlines={91,117-118}]{}}
      \onslide*<3->{\listStashBacktrace[highlightlines={94,106,109-110}]{}}
    \end{column}
  \end{columns}
\end{frame}

\newcommand\listFrameDescr[1][]{
  \listocaml[firstline=111,lastline=124,#1]{../ocaml/asmcomp/emitaux.ml}{asmcomp/emitaux.ml:111}}
\newcommand\listDebugInfo[1][]{
  \listocaml[firstline=38,lastline=49,#1]{../ocaml/lambda/debuginfo.mli}{lambda/debuginfo.mli:38}}

\begin{frame}{Backtraces}{\typename{frame\_descr} and \typename{frame\_debuginfo} in a nutshell}
  \begin{columns}[c]
    \begin{column}{0.5\textwidth}
      \begin{itemize}
        \item<1-> For every return address in an OCaml program, there is a associated \typename{frame\_descr}
        \item<2-> A \typename{frame\_descr} specifies
        \begin{itemize}
          \item<2-> The size of the function stack frame
          \item<3-> Where live values are on the stack (so that the GC can mark them as live and prevent from being swept)
        \end{itemize}
        \item<4-> When compiled with debug information, \typename{frame\_descr} will be linked to a \typename{frame\_debuginfo}
        \begin{itemize}
          \item<5-> Contains information about the position in the source code (file name, line and column number)
        \end{itemize}
      \end{itemize}
    \end{column}
    \begin{column}{0.5\textwidth}
        \onslide*<1>{\listFrameDescr[highlightlines={118-122}]{}}
        \onslide*<2>{\listFrameDescr[highlightlines={120}]{}}
        \onslide*<3>{\listFrameDescr[highlightlines={121}]{}}
        \onslide*<4->{\listFrameDescr[highlightlines={122}]{}}
        \medskip
        \onslide*<1-3>{\listDebugInfo{}}
        \onslide*<4>{\listDebugInfo[highlightlines={38,49}]{}}
        \onslide*<5>{\listDebugInfo[highlightlines={39-46}]{}}
    \end{column}
  \end{columns}
\end{frame}

\begin{frame}{Backtraces}{Scanning the stack with \typename{frame\_descr}}
  \begin{columns}[c]
    \begin{column}{0.5\textwidth}
      \begin{itemize}
        \item Given the return address of the code that raises the exception by calling \funcname{caml\_raise\_exn} (\funcarg{pc} argument of \funcname{caml\_stash\_backtrace})
        \item And the position of the stack pointer (\funcarg{sp} argument of \funcname{caml\_stash\_backtrace})
        \item The frame size given by \typename{frame\_descr}.\funcarg{frame\_size}, allows to find the end of the previous frame
        \item And the return address of the caller of this function
        \bigskip
        \onslide<3->{
        \item Note that traps (here the reddish \funcname{ExnA}, \funcname{ExnB} and \funcname{ExnC} blocks) belong to the frame that installed them, and so the frame size changes when traps are removed
        }
      \end{itemize}
    \end{column}
    \begin{column}{0.5\textwidth}
      \raggedleft
      \onslide*<1>{\providecommand\step{2}\input{diagrams/backtrace/backtrace.tex}}%
      \onslide*<2>{\providecommand\step{1}\input{diagrams/backtrace/scanstack.tex}}%
      \onslide*<3>{\providecommand\step{2}\input{diagrams/backtrace/scanstack.tex}}%
      \onslide*<4>{\providecommand\step{3}\input{diagrams/backtrace/scanstack.tex}}%
      \onslide*<5>{\providecommand\step{4}\input{diagrams/backtrace/scanstack.tex}}%
      \onslide*<6>{\providecommand\step{5}\input{diagrams/backtrace/scanstack.tex}}%
    \end{column}
  \end{columns}
\end{frame}

% Duplicate of previous-previous slide
\begin{frame}{Backtraces}{Continuing on \funcname{caml\_stash\_backtrace}}
  \begin{columns}[c]
    \begin{column}{0.5\textwidth}
      \minipage[c][0.75\textheight][s]{\columnwidth}
      \begin{itemize}
          \color{gray}
        \item A C function provided by the runtime (specific to native code)
        \item It scans the stack, between the stack pointer at the raising point up to the next trap, in order to collect the names of the detected function frames
        \item It uses the same technique and information as the Garbage Collector (GC) uses to scan the values live on the stack
      \end{itemize}
      \begin{itemize}
        \item For each function frame detected, store a pointer to the \typename{frame\_descr} in the \localname{domain\_state->backtrace\_buffer} array, effectively constructing the backtrace
      \end{itemize}
      \onslide<2->{
        \begin{itemize}
            \smallskip
          \item Constructed backtrace:
            \funcname{definitely\_raise},
            \onslide<3->{\funcname{probably\_raise}}
        \end{itemize}
      }
      \vfill
      \mintocaml[firstline=9,lastline=13,highlightlines=12]{../src/backtrace/backtrace.ml}
      \endminipage
    \end{column}
    \begin{column}{0.5\textwidth}
      \raggedleft
      \onslide*<1>{\listStashBacktrace[highlightlines={114-115}]{}}
      \onslide*<2>{\providecommand\step{3}\input{diagrams/backtrace/backtrace.tex}}%
      \onslide*<3>{\providecommand\step{4}\input{diagrams/backtrace/backtrace.tex}}%
    \end{column}
  \end{columns}
\end{frame}

\begin{frame}{Backtraces}{Back to \funcname{caml\_raise\_exn}}
  \begin{columns}[c]
    \begin{column}{0.5\textwidth}
      \minipage[c][0.66\textheight][s]{\columnwidth}
      \begin{itemize}\color{gray}
        \item Checks wheteher exceptions backtrace should be recorded, by checking the \funcarg{domain\_state->backtrace\_active} value
        \item Switches to the C stack to be able to execute C code
        \item Calls \funcname{caml\_stash\_backtrace}, to collect the backtrace up to the next trap
      \end{itemize}
      \onslide*<2->{
        \begin{itemize}
          \item<2-> Executes the exception handler in \funcname{probably\_raise} with \funcname{RESTORE\_EXN\_HANDLER\_OCAML}
            \begin{itemize}
              \item Set \regname{RSP} to \localname{domain\_state->exn\_handler} value
              \item Restore \localname{domain\_state->exn\_handler} from trap
              \item Jump to the handler code with \funcname{ret}
            \end{itemize}
        \end{itemize}
      }
      \vfill
      \onslide*<1>{\mintocaml[firstline=9,lastline=13,highlightlines=12]{../src/backtrace/backtrace.ml}}
      \onslide*<2->{\mintocaml[firstline=15,lastline=21,highlightlines={19-21}]{../src/backtrace/backtrace.ml}}
      \endminipage
    \end{column}
    \begin{column}{0.5\textwidth}
      \centering
      \onslide*<1>{
        \mintc[firstline=190,lastline=192]{../ocaml/runtime/amd64.S}
        \mintc[firstline=234,lastline=237]{../ocaml/runtime/amd64.S}
      }
      \onslide*<2>{
        \mintc[firstline=190,lastline=192,highlightlines={191-192}]{../ocaml/runtime/amd64.S}
        \mintc[firstline=234,lastline=237,highlightlines={235-237}]{../ocaml/runtime/amd64.S}
      }
      \onslide*<1>{\listCamlRaiseExn[highlightlines=720]{}}
      \onslide*<2>{\listCamlRaiseExn[highlightlines=722]{}}
      \onslide*<3->{\providecommand\step{5}\input{diagrams/backtrace/backtrace.tex}}
    \end{column}
  \end{columns}
\end{frame}

\begin{frame}{Backtraces}{\funcname{caml\_probably\_raise}'s handler doesn't catch \typename{ExnC}}
  \begin{columns}[c]
    \begin{column}{0.45\textwidth}
      \minipage[c][0.75\textheight][s]{\columnwidth}
      \begin{itemize}
        \item<1-> \funcname{match \dots with}\footnotemark pattern matching can also match exceptions
        \item<1-> The CMM of \funcname{probably\_raise} shows that this \funcname{match \dots with} is implemented with a pattern matching and a \funcname{try \dots with}
        \item<1-> But this one only matches on \typename{ExnA} exception
        \item<2-> Since the pattern matching fails to catch the exception, \funcname{reraise} is called with our current \typename{ExnC} exception
        \item<3-> Disassembly shows that \funcname{reraise} is actually a call to \funcname{caml\_reraise\_exn}, which is also an assembly function provided by the runtime
      \end{itemize}
      \vfill
      \mintocaml[firstline=15,lastline=21,highlightlines={18-21}]{../src/backtrace/backtrace.ml}
      \endminipage
    \end{column}
    \begin{column}{0.55\textwidth}
      \centering
      \onslide*<1>{\mintcmm[highlightlines={10-18}]{../src/backtrace/camlBacktrace\_\_probably\_raise\_351.cmm}}
      \onslide*<2>{\listcmm[highlightlines=18]{../src/backtrace/camlBacktrace\_\_probably\_raise_351.cmm}{CMM of probably\_raise}}
      \onslide*<3>{\mintobjdump[firstline=5,lastline=35,highlightlines=31]{../src/backtrace/camlBacktrace\_\_probably\_raise_351.objdump}}
    \end{column}
  \end{columns}
  \only<1>{\footnotetext[1]{https://blog.janestreet.com/pattern-matching-and-exception-handling-unite/}}
\end{frame}

\newcommand\listCamlReraiseExn[1][]{
  \listasm[firstline=727,lastline=734,#1]{../ocaml/runtime/amd64.S}{runtime/amd64.S:727}}

\begin{frame}{Backtraces}{Introducing \funcname{caml\_reraise\_exn}}
  \begin{columns}[c]
    \begin{column}{0.5\textwidth}
      \minipage[c][0.66\textheight][s]{\columnwidth}
      \begin{itemize}
        \item<1-> The exception have to be reraised to the next handler
        \item<2-> First, checks if the backtrace should be collected with \funcarg{domain\_state->backtrace\_active}
        \only<3>{
        \begin{itemize}
            \item If not, behaves like a \funcname{raise\_notrace} with \funcname{RESTORE\_EXN\_HANDLER\_OCAML}
        \end{itemize}
        }
        \item<4-> Jumps into \funcname{caml\_raise\_exn}
        \item<5-> Calls \funcname{caml\_stash\_backtrace} again, to scan the next part of the stack: from the trap just pop-ed to the next trap
      \end{itemize}
      \vfill
      \mintocaml[firstline=15,lastline=21,highlightlines={18-21}]{../src/backtrace/backtrace.ml}
      \endminipage
    \end{column}
    \begin{column}{0.5\textwidth}
      \raggedleft
      \onslide*<1>{\listCamlReraiseExn{}}
      \onslide*<2>{\listCamlReraiseExn[highlightlines=729]{}}
      \onslide*<3>{
        \mintc[firstline=190,lastline=192,highlightlines={191-192}]{../ocaml/runtime/amd64.S}
        \mintc[firstline=234,lastline=237,highlightlines={235-237}]{../ocaml/runtime/amd64.S}
        \medskip
        \listCamlReraiseExn[highlightlines=731]{}
      }
      \onslide*<4-5>{
        % TODO refactor with \listCamlReraiseExn
        \mintasm[firstline=727,lastline=734,highlightlines=730]{../ocaml/runtime/amd64.S}
        \medskip
      }
      \onslide*<4>{\listCamlRaiseExn[highlightlines=711]{}}
      \onslide*<5>{\listCamlRaiseExn[highlightlines=720]{}}
    \end{column}
  \end{columns}
\end{frame}

\begin{frame}{Backtraces}{Backtrace collection continues}
  \begin{columns}[c]
    \begin{column}{0.55\textwidth}
      \minipage[c][0.75\textheight][s]{\columnwidth}
      \begin{itemize}
        \item<1-> \funcname{caml\_stash\_backtrace} scans the older part of the stack, up to the next trap
        \item<6-> Controls jumps to the nested exception handler in \funcname{maybe\_raise}, which matches only on \typename{ExnB}
        \item<7-> \funcname{caml\_reraise\_exn} is called again, backtrace collection resumes with \funcname{caml\_stash\_backtrace}
      \end{itemize}
      \medskip
      \begin{itemize}
        \item<2-> Constructed backtrace:
        \begin{itemize}
          \item <2-> \funcname{definitely\_raise}
          \item <2-> \funcname{probably\_raise}
          \item <3-> \funcname{probably\_raise} (re-raise)
          \item <4-> \funcname{maybe\_raise}
          \item <5-> \funcname{main}
          \item <8-> \funcname{main} (re-raise)
        \end{itemize}
      \end{itemize}
      \vfill
      \onslide*<1-5>{\mintocaml[firstline=15,lastline=21]{../src/backtrace/backtrace.ml}}
      \onslide*<6>{\mintocaml[firstline=28,lastline=34,highlightlines=31]{../src/backtrace/backtrace.ml}}
      \onslide*<7->{\mintocaml[firstline=28,lastline=34,highlightlines=33]{../src/backtrace/backtrace.ml}}
      \endminipage
    \end{column}
    \begin{column}{0.45\textwidth}
      \raggedleft
      \onslide*<1>{\providecommand\step{6}\input{diagrams/backtrace/backtrace.tex}}%
      \onslide*<2>{\providecommand\step{6}\input{diagrams/backtrace/backtrace.tex}}%
      \onslide*<3>{\providecommand\step{7}\input{diagrams/backtrace/backtrace.tex}}%
      \onslide*<4>{\providecommand\step{8}\input{diagrams/backtrace/backtrace.tex}}%
      \onslide*<5>{\providecommand\step{9}\input{diagrams/backtrace/backtrace.tex}}%
      \onslide*<6>{\providecommand\step{10}\input{diagrams/backtrace/backtrace.tex}}%
      \onslide*<7>{\providecommand\step{11}\input{diagrams/backtrace/backtrace.tex}}%
      \onslide*<8>{\providecommand\step{12}\input{diagrams/backtrace/backtrace.tex}}%
      \onslide*<9>{\providecommand\step{13}\input{diagrams/backtrace/backtrace.tex}}%
    \end{column}
  \end{columns}
\end{frame}

\begin{frame}{Backtraces}{Printing the backtrace}
  \begin{columns}[c]
    \begin{column}{0.45\textwidth}
      \minipage[c][0.66\textheight][s]{\columnwidth}
      \begin{itemize}
        \item<1-> The exception backtrace can now be printed to \funcarg{stdout} with \funcname{Printexc.print_backtrace}
        \item<2-> First the exception backtrace is retrieved into an OCaml array with \funcname{get_raw_backtrace }
        \item<3-> Which is implemented as the \funcname{caml_get_exception_raw_backtrace} C function in the runtime
        \item<4-> And finally printed with \funcname{print_exception_backtrace}
      \end{itemize}
      \vfill
      \mintocaml[firstline=28,lastline=34,highlightlines=33]{../src/backtrace/backtrace.ml}
      \endminipage
    \end{column}
    \begin{column}{0.55\textwidth}
      \onslide*<1,2>{
        \mintocaml[firstline=100,lastline=101]{../ocaml/stdlib/printexc.ml}
      }
      \only<1>{
        \listocaml[firstline=170,lastline=172]{../ocaml/stdlib/printexc.ml}{stdlib/printexc.ml:170}
      }
      \only<2>{
        \listocaml[firstline=170,lastline=172,highlightlines=172]{../ocaml/stdlib/printexc.ml}{stdlib/printexc.ml:170}
      }
      \only<3>{
        \mintc[firstline=154,lastline=156]{../ocaml/runtime/backtrace.c}
        \listc[firstline=171,lastline=192,highlightlines={184-187}]{../ocaml/runtime/backtrace.c}{runtime/backtrace.c:154}
      }
      \only<4>{
        \listocaml[firstline=155,lastline=165]{../ocaml/stdlib/printexc.ml}{stdlib/printexc.ml:55}
      }
    \end{column}
  \end{columns}
\end{frame}


\subsection{The multiple kinds of raise}
\frameSubsection{}{}

\begin{frame}{Backtraces}{When backtraces are not needed}
  \begin{columns}[c]
    \begin{column}{0.45\textwidth}
      \minipage[c][0.66\textheight][s]{\columnwidth}
      \begin{itemize}
        \item<1-> What if the backtrace for an exception is never needed?
        \item<2-> It's possible to raise the exception explicitly with \funcname{raise\_notrace} to make raising as cheap as possible
        \item<3-> An example of use in the \funcname{dynlink} library of the OCaml compiler
      \end{itemize}
      \vfill
      \onslide<3->{
      \listocaml[firstline=205,lastline=209,highlightlines=207]{../ocaml/otherlibs/dynlink/dynlink\_compilerlibs/misc.ml}{otherlibs/dynlink/dynlink\_compilerlibs/misc.ml:205}
      }
      \endminipage
    \end{column}
    \begin{column}{0.55\textwidth}
    \only<4>{
      \mintcmm[highlightlines=3]{../src/notrace/camlNotrace\_\_fun_347.cmm}
      \listcmm{../src/notrace/camlNotrace\_\_all\_somes\_327.cmm}{all\_somes}
    }
    \onslide*<5->{
      \listobjdump[highlightlines={6-9}]{../src/notrace/camlNotrace\_\_fun_347.objdump}{anonymous function from \funcname{all\_somes}}
    }
    \end{column}
  \end{columns}
\end{frame}

\begin{frame}{Backtraces}{Summary table of the multiple kinds of raise}
  \begin{itemize}
    \item When debugging information is enabled and if the backtrace of an exception isn't needed, it's possible to raise with \funcname{raise\_notrace} for performance
    \item When debugging information is \emph{not} enabled, the compiler optimises \funcname{raise} calls to inline assembly (same effect as using \funcname{raise\_notrace})
  \end{itemize}
  \bigskip
  \centering
  \begin{tabular}{| l | c | c | c | c |}
    \hline
    \parbox{10em}{Debug information \\ (\funcarg{-g} flag)}  & \multicolumn{2}{|c|}{Disabled}                                     & \multicolumn{2}{|c|}{Enabled} \\
    \hline
    Raise kind                                               & \funcname{raise} & \funcname{raise\_notrace}                       & \funcname{raise} & \funcname{raise\_notrace} \\
    \hline
    Compilation                                              & \parbox{8em}{inline assembly\\ (optimization)} & inline assembly   & \funcname{caml\_raise\_exn} & inline assembly \\
    \hline
  \end{tabular}
\end{frame}


%
% Takeaway #5
%
\subsection*{Takeaway \#5}
\frameSubsectionTakeaway{}
\againframe<5>{takeaway}
}%
      \onslide*<5>{\providecommand\step{9}%
% Backtraces
%
\subsection{One advantage of raising with debugging information}
\frameSubsection{}{}

\begin{frame}{Backtraces}{Debugging information \& source code}
  \begin{columns}[c]
    \begin{column}{0.5\textwidth}
      \begin{itemize}
        \item Raising an exception is fun and games, but how do we know where the exception originated? $\rightarrow$ With the backtrace associated with the exception!
        \item In order to get a backtrace from the point where an exception was raised, it's necessary to compile with debugging information enabled (\funcarg{-g} flag for the compiler)
        \item Same example as in the `Nested handlers' section
      \end{itemize}
    \end{column}
    \begin{column}{0.5\textwidth}
      \listocaml[firstline=9,lastline=34]{../src/backtrace/backtrace.ml}{backtrace/backtrace.ml}
    \end{column}
  \end{columns}
\end{frame}

\begin{frame}{Backtraces}{CMM and disassembly of \funcname{definitely\_raise}}
  \begin{columns}[c]
    \begin{column}{0.5\textwidth}
      \minipage[c][0.66\textheight][s]{\columnwidth}
      \begin{itemize}
        \item<1-> An effect of compiling with debugging information (\funcarg{-g}): the CMM no longer uses \funcname{raise\_notrace} to raise the exception but \funcname{raise} instead
        \item<2-> Disassembly shows that CMM \funcname{raise} is actually a call to \funcname{caml\_raise\_exn}, a function in assembler provided by the runtime
      \end{itemize}
      \vfill
      \mintocaml[firstline=9,lastline=13,highlightlines=12]{../src/backtrace/backtrace.ml}
      \endminipage
    \end{column}
    \begin{column}{0.5\textwidth}
      \onslide*<1>{
        \mintcmm[highlightlines=5]{../src/backtrace/camlBacktrace\_\_definitely\_raise\_347.cmm}
      }
      \onslide*<2>{
        \mintobjdump[firstline=2,highlightlines=14]{../src/backtrace/camlBacktrace\_\_definitely\_raise\_347.objdump}
      }
    \end{column}
  \end{columns}
\end{frame}

\begin{frame}{Backtraces}{Introducing \funcname{caml\_raise\_exn}}
  \begin{columns}[c]
    \begin{column}{0.5\textwidth}
      \minipage[c][0.66\textheight][s]{\columnwidth}
      \onslide*<1>{
        \begin{itemize}
          \item Raising the exception is performed using \funcname{caml\_raise\_exn} which is
            \begin{itemize}
              \item An assembly function provided by the runtime
              \item Architecture specific
            \end{itemize}
          \item Let's see what it does differently from \funcname{raise\_notrace}\dots
          \item We are about to raise \typename{ExnC} with \funcname{caml\_raise\_exn} from \funcname{definitely\_raise}
        \end{itemize}
      }
      \onslide*<2>{
        \mintobjdump[firstline=2,highlightlines=14]{../src/backtrace/camlBacktrace\_\_definitely\_raise\_347.objdump}
      }
      \vfill
      \mintocaml[firstline=9,lastline=13,highlightlines=12]{../src/backtrace/backtrace.ml}
      \endminipage
    \end{column}
    \begin{column}{0.5\textwidth}
      \centering
      \providecommand\step{1}\input{diagrams/backtrace/backtrace.tex}
    \end{column}
  \end{columns}
\end{frame}

\begin{frame}{Backtraces}{But before that, we need more fields from \typename{caml\_domain\_state}}
  \begin{columns}
    \begin{column}{0.5\textwidth}
      \begin{itemize}
        \item Remember \typename{caml\_domain\_state} the per-domain runtime state?
        \item Some other members are of interest to us now\dots
        \item \localname{backtrace\_active} is used to know if the runtime should collect backtraces
        \item \localname{backtrace\_active} can be set by
          \begin{itemize}
            \item The \funcarg{b} flag in the \funcname{OCAMLRUNPARAM} environment variable
            \item \mintinline{ocaml}{Printexc.record_backtrace : bool -> unit} \footnotemark
          \end{itemize}
      \end{itemize}
    \end{column}
    \begin{column}{0.5\textwidth}
      \begin{listing}
        \mintc[highlightlines={9-11}]{src/ocaml/domain\_state\_backtrace.h}
        \caption{runtime/caml/domain\_state.h}
      \end{listing}
    \end{column}
  \end{columns}
  \footnotetext[1]{https://v2.ocaml.org/api/Printexc.html}
\end{frame}

\newcommand\listCamlRaiseExn[1][]{
  \listasm[firstline=702,lastline=725,#1]{../ocaml/runtime/amd64.S}{runtime/amd64.S:702}}

\begin{frame}{Backtraces}{Walk through \funcname{caml\_raise\_exn}}
  \begin{columns}[T]
    \begin{column}{0.5\textwidth}
      \begin{itemize}
        \item<1-> First checks whether exceptions backtrace should be recorded
        \item<1-> By checking the \funcarg{domain\_state->backtrace\_active} value
        \item<2-> If not, acts as \funcname{raise\_notrace} with \funcname{RESTORE\_EXN\_HANDLER\_OCAML}
          \begin{itemize}
            \item Set \regname{RSP} to \localname{domain\_state->exn\_handler} value
            \item Restore \localname{domain\_state->exn\_handler} from trap
            \item Jump with \funcname{ret} (same effect as using \funcname{pop} and \funcname{jmp})
          \end{itemize}
        \item<3-> Otherwise, switches to the C stack to be able to execute C code
        \item<4-> Calls \funcname{caml\_stash\_backtrace}, a C function provided by the runtime\ignorespaces
          \onslide<6->{, with
            \begin{itemize}
              \item \funcarg{exn}: exception value
              \item \funcarg{pc}: code address of the raising point
              \item \funcarg{sp}: stack address of the raising function
              \item \funcarg{trapsp}: stack address of the next handler
            \end{itemize}
          }
      \end{itemize}
      \bigskip
      \mintocaml[firstline=9,lastline=13,highlightlines=12]{../src/backtrace/backtrace.ml}
    \end{column}
    \begin{column}{0.5\textwidth}
      \raggedleft
      \onslide*<1,3-4>{
        \mintc[firstline=190,lastline=192]{../ocaml/runtime/amd64.S}
        \mintc[firstline=234,lastline=237]{../ocaml/runtime/amd64.S}
      }
      \onslide*<2>{
        \mintc[firstline=190,lastline=192,highlightlines={191-192}]{../ocaml/runtime/amd64.S}
        \mintc[firstline=234,lastline=237,highlightlines={235-237}]{../ocaml/runtime/amd64.S}
      }
      \onslide*<1>{\listCamlRaiseExn[highlightlines=705]{}}%
      \onslide*<2>{\listCamlRaiseExn[highlightlines={707-708}]{}}%
      \onslide*<3>{\listCamlRaiseExn[highlightlines={712-714}]{}}%
      \onslide*<4>{\listCamlRaiseExn[highlightlines={715-720}]{}}%
      \onslide*<5>{\providecommand\step{1}\input{diagrams/backtrace/backtrace.tex}}%
      \onslide*<6>{\providecommand\step{2}\input{diagrams/backtrace/backtrace.tex}}%
    \end{column}
  \end{columns}
\end{frame}

\newcommand\listStashBacktrace[1][]{
  \listc[firstline=91,lastline=120,#1]{../ocaml/runtime/backtrace\_nat.c}{runtime/backtrace\_nat.c:91}}

\begin{frame}{Backtraces}{Introducing \funcname{caml\_stash\_backtrace}}
  \begin{columns}[c]
    \begin{column}{0.5\textwidth}
      \minipage[c][0.66\textheight][s]{\columnwidth}
      \begin{itemize}
        \item<1-> A C function provided by the runtime (specific to native code)
        \item<2-> It scans the stack, between the stack pointer at the raising point up to the next trap, in order to collect the names of the detected function frames
          \onslide*<2>{
            \begin{itemize}
              \item And more information, like line number, column number...
            \end{itemize}
          }
        \item<3-> It uses the same technique and information as the Garbage Collector (GC) uses to scan the values live on the stack
          \onslide*<4>{
            \begin{itemize}
              \item \typename{frame\_descr} are at the core of stack unwinding
            \end{itemize}
          }
      \end{itemize}
      \vfill
      \mintocaml[firstline=9,lastline=13,highlightlines=12]{../src/backtrace/backtrace.ml}
      \endminipage
    \end{column}
    \begin{column}{0.5\textwidth}
      \onslide*<1>{\listStashBacktrace[highlightlines=91]{}}
      \onslide*<2>{\listStashBacktrace[highlightlines={91,117-118}]{}}
      \onslide*<3->{\listStashBacktrace[highlightlines={94,106,109-110}]{}}
    \end{column}
  \end{columns}
\end{frame}

\newcommand\listFrameDescr[1][]{
  \listocaml[firstline=111,lastline=124,#1]{../ocaml/asmcomp/emitaux.ml}{asmcomp/emitaux.ml:111}}
\newcommand\listDebugInfo[1][]{
  \listocaml[firstline=38,lastline=49,#1]{../ocaml/lambda/debuginfo.mli}{lambda/debuginfo.mli:38}}

\begin{frame}{Backtraces}{\typename{frame\_descr} and \typename{frame\_debuginfo} in a nutshell}
  \begin{columns}[c]
    \begin{column}{0.5\textwidth}
      \begin{itemize}
        \item<1-> For every return address in an OCaml program, there is a associated \typename{frame\_descr}
        \item<2-> A \typename{frame\_descr} specifies
        \begin{itemize}
          \item<2-> The size of the function stack frame
          \item<3-> Where live values are on the stack (so that the GC can mark them as live and prevent from being swept)
        \end{itemize}
        \item<4-> When compiled with debug information, \typename{frame\_descr} will be linked to a \typename{frame\_debuginfo}
        \begin{itemize}
          \item<5-> Contains information about the position in the source code (file name, line and column number)
        \end{itemize}
      \end{itemize}
    \end{column}
    \begin{column}{0.5\textwidth}
        \onslide*<1>{\listFrameDescr[highlightlines={118-122}]{}}
        \onslide*<2>{\listFrameDescr[highlightlines={120}]{}}
        \onslide*<3>{\listFrameDescr[highlightlines={121}]{}}
        \onslide*<4->{\listFrameDescr[highlightlines={122}]{}}
        \medskip
        \onslide*<1-3>{\listDebugInfo{}}
        \onslide*<4>{\listDebugInfo[highlightlines={38,49}]{}}
        \onslide*<5>{\listDebugInfo[highlightlines={39-46}]{}}
    \end{column}
  \end{columns}
\end{frame}

\begin{frame}{Backtraces}{Scanning the stack with \typename{frame\_descr}}
  \begin{columns}[c]
    \begin{column}{0.5\textwidth}
      \begin{itemize}
        \item Given the return address of the code that raises the exception by calling \funcname{caml\_raise\_exn} (\funcarg{pc} argument of \funcname{caml\_stash\_backtrace})
        \item And the position of the stack pointer (\funcarg{sp} argument of \funcname{caml\_stash\_backtrace})
        \item The frame size given by \typename{frame\_descr}.\funcarg{frame\_size}, allows to find the end of the previous frame
        \item And the return address of the caller of this function
        \bigskip
        \onslide<3->{
        \item Note that traps (here the reddish \funcname{ExnA}, \funcname{ExnB} and \funcname{ExnC} blocks) belong to the frame that installed them, and so the frame size changes when traps are removed
        }
      \end{itemize}
    \end{column}
    \begin{column}{0.5\textwidth}
      \raggedleft
      \onslide*<1>{\providecommand\step{2}\input{diagrams/backtrace/backtrace.tex}}%
      \onslide*<2>{\providecommand\step{1}\input{diagrams/backtrace/scanstack.tex}}%
      \onslide*<3>{\providecommand\step{2}\input{diagrams/backtrace/scanstack.tex}}%
      \onslide*<4>{\providecommand\step{3}\input{diagrams/backtrace/scanstack.tex}}%
      \onslide*<5>{\providecommand\step{4}\input{diagrams/backtrace/scanstack.tex}}%
      \onslide*<6>{\providecommand\step{5}\input{diagrams/backtrace/scanstack.tex}}%
    \end{column}
  \end{columns}
\end{frame}

% Duplicate of previous-previous slide
\begin{frame}{Backtraces}{Continuing on \funcname{caml\_stash\_backtrace}}
  \begin{columns}[c]
    \begin{column}{0.5\textwidth}
      \minipage[c][0.75\textheight][s]{\columnwidth}
      \begin{itemize}
          \color{gray}
        \item A C function provided by the runtime (specific to native code)
        \item It scans the stack, between the stack pointer at the raising point up to the next trap, in order to collect the names of the detected function frames
        \item It uses the same technique and information as the Garbage Collector (GC) uses to scan the values live on the stack
      \end{itemize}
      \begin{itemize}
        \item For each function frame detected, store a pointer to the \typename{frame\_descr} in the \localname{domain\_state->backtrace\_buffer} array, effectively constructing the backtrace
      \end{itemize}
      \onslide<2->{
        \begin{itemize}
            \smallskip
          \item Constructed backtrace:
            \funcname{definitely\_raise},
            \onslide<3->{\funcname{probably\_raise}}
        \end{itemize}
      }
      \vfill
      \mintocaml[firstline=9,lastline=13,highlightlines=12]{../src/backtrace/backtrace.ml}
      \endminipage
    \end{column}
    \begin{column}{0.5\textwidth}
      \raggedleft
      \onslide*<1>{\listStashBacktrace[highlightlines={114-115}]{}}
      \onslide*<2>{\providecommand\step{3}\input{diagrams/backtrace/backtrace.tex}}%
      \onslide*<3>{\providecommand\step{4}\input{diagrams/backtrace/backtrace.tex}}%
    \end{column}
  \end{columns}
\end{frame}

\begin{frame}{Backtraces}{Back to \funcname{caml\_raise\_exn}}
  \begin{columns}[c]
    \begin{column}{0.5\textwidth}
      \minipage[c][0.66\textheight][s]{\columnwidth}
      \begin{itemize}\color{gray}
        \item Checks wheteher exceptions backtrace should be recorded, by checking the \funcarg{domain\_state->backtrace\_active} value
        \item Switches to the C stack to be able to execute C code
        \item Calls \funcname{caml\_stash\_backtrace}, to collect the backtrace up to the next trap
      \end{itemize}
      \onslide*<2->{
        \begin{itemize}
          \item<2-> Executes the exception handler in \funcname{probably\_raise} with \funcname{RESTORE\_EXN\_HANDLER\_OCAML}
            \begin{itemize}
              \item Set \regname{RSP} to \localname{domain\_state->exn\_handler} value
              \item Restore \localname{domain\_state->exn\_handler} from trap
              \item Jump to the handler code with \funcname{ret}
            \end{itemize}
        \end{itemize}
      }
      \vfill
      \onslide*<1>{\mintocaml[firstline=9,lastline=13,highlightlines=12]{../src/backtrace/backtrace.ml}}
      \onslide*<2->{\mintocaml[firstline=15,lastline=21,highlightlines={19-21}]{../src/backtrace/backtrace.ml}}
      \endminipage
    \end{column}
    \begin{column}{0.5\textwidth}
      \centering
      \onslide*<1>{
        \mintc[firstline=190,lastline=192]{../ocaml/runtime/amd64.S}
        \mintc[firstline=234,lastline=237]{../ocaml/runtime/amd64.S}
      }
      \onslide*<2>{
        \mintc[firstline=190,lastline=192,highlightlines={191-192}]{../ocaml/runtime/amd64.S}
        \mintc[firstline=234,lastline=237,highlightlines={235-237}]{../ocaml/runtime/amd64.S}
      }
      \onslide*<1>{\listCamlRaiseExn[highlightlines=720]{}}
      \onslide*<2>{\listCamlRaiseExn[highlightlines=722]{}}
      \onslide*<3->{\providecommand\step{5}\input{diagrams/backtrace/backtrace.tex}}
    \end{column}
  \end{columns}
\end{frame}

\begin{frame}{Backtraces}{\funcname{caml\_probably\_raise}'s handler doesn't catch \typename{ExnC}}
  \begin{columns}[c]
    \begin{column}{0.45\textwidth}
      \minipage[c][0.75\textheight][s]{\columnwidth}
      \begin{itemize}
        \item<1-> \funcname{match \dots with}\footnotemark pattern matching can also match exceptions
        \item<1-> The CMM of \funcname{probably\_raise} shows that this \funcname{match \dots with} is implemented with a pattern matching and a \funcname{try \dots with}
        \item<1-> But this one only matches on \typename{ExnA} exception
        \item<2-> Since the pattern matching fails to catch the exception, \funcname{reraise} is called with our current \typename{ExnC} exception
        \item<3-> Disassembly shows that \funcname{reraise} is actually a call to \funcname{caml\_reraise\_exn}, which is also an assembly function provided by the runtime
      \end{itemize}
      \vfill
      \mintocaml[firstline=15,lastline=21,highlightlines={18-21}]{../src/backtrace/backtrace.ml}
      \endminipage
    \end{column}
    \begin{column}{0.55\textwidth}
      \centering
      \onslide*<1>{\mintcmm[highlightlines={10-18}]{../src/backtrace/camlBacktrace\_\_probably\_raise\_351.cmm}}
      \onslide*<2>{\listcmm[highlightlines=18]{../src/backtrace/camlBacktrace\_\_probably\_raise_351.cmm}{CMM of probably\_raise}}
      \onslide*<3>{\mintobjdump[firstline=5,lastline=35,highlightlines=31]{../src/backtrace/camlBacktrace\_\_probably\_raise_351.objdump}}
    \end{column}
  \end{columns}
  \only<1>{\footnotetext[1]{https://blog.janestreet.com/pattern-matching-and-exception-handling-unite/}}
\end{frame}

\newcommand\listCamlReraiseExn[1][]{
  \listasm[firstline=727,lastline=734,#1]{../ocaml/runtime/amd64.S}{runtime/amd64.S:727}}

\begin{frame}{Backtraces}{Introducing \funcname{caml\_reraise\_exn}}
  \begin{columns}[c]
    \begin{column}{0.5\textwidth}
      \minipage[c][0.66\textheight][s]{\columnwidth}
      \begin{itemize}
        \item<1-> The exception have to be reraised to the next handler
        \item<2-> First, checks if the backtrace should be collected with \funcarg{domain\_state->backtrace\_active}
        \only<3>{
        \begin{itemize}
            \item If not, behaves like a \funcname{raise\_notrace} with \funcname{RESTORE\_EXN\_HANDLER\_OCAML}
        \end{itemize}
        }
        \item<4-> Jumps into \funcname{caml\_raise\_exn}
        \item<5-> Calls \funcname{caml\_stash\_backtrace} again, to scan the next part of the stack: from the trap just pop-ed to the next trap
      \end{itemize}
      \vfill
      \mintocaml[firstline=15,lastline=21,highlightlines={18-21}]{../src/backtrace/backtrace.ml}
      \endminipage
    \end{column}
    \begin{column}{0.5\textwidth}
      \raggedleft
      \onslide*<1>{\listCamlReraiseExn{}}
      \onslide*<2>{\listCamlReraiseExn[highlightlines=729]{}}
      \onslide*<3>{
        \mintc[firstline=190,lastline=192,highlightlines={191-192}]{../ocaml/runtime/amd64.S}
        \mintc[firstline=234,lastline=237,highlightlines={235-237}]{../ocaml/runtime/amd64.S}
        \medskip
        \listCamlReraiseExn[highlightlines=731]{}
      }
      \onslide*<4-5>{
        % TODO refactor with \listCamlReraiseExn
        \mintasm[firstline=727,lastline=734,highlightlines=730]{../ocaml/runtime/amd64.S}
        \medskip
      }
      \onslide*<4>{\listCamlRaiseExn[highlightlines=711]{}}
      \onslide*<5>{\listCamlRaiseExn[highlightlines=720]{}}
    \end{column}
  \end{columns}
\end{frame}

\begin{frame}{Backtraces}{Backtrace collection continues}
  \begin{columns}[c]
    \begin{column}{0.55\textwidth}
      \minipage[c][0.75\textheight][s]{\columnwidth}
      \begin{itemize}
        \item<1-> \funcname{caml\_stash\_backtrace} scans the older part of the stack, up to the next trap
        \item<6-> Controls jumps to the nested exception handler in \funcname{maybe\_raise}, which matches only on \typename{ExnB}
        \item<7-> \funcname{caml\_reraise\_exn} is called again, backtrace collection resumes with \funcname{caml\_stash\_backtrace}
      \end{itemize}
      \medskip
      \begin{itemize}
        \item<2-> Constructed backtrace:
        \begin{itemize}
          \item <2-> \funcname{definitely\_raise}
          \item <2-> \funcname{probably\_raise}
          \item <3-> \funcname{probably\_raise} (re-raise)
          \item <4-> \funcname{maybe\_raise}
          \item <5-> \funcname{main}
          \item <8-> \funcname{main} (re-raise)
        \end{itemize}
      \end{itemize}
      \vfill
      \onslide*<1-5>{\mintocaml[firstline=15,lastline=21]{../src/backtrace/backtrace.ml}}
      \onslide*<6>{\mintocaml[firstline=28,lastline=34,highlightlines=31]{../src/backtrace/backtrace.ml}}
      \onslide*<7->{\mintocaml[firstline=28,lastline=34,highlightlines=33]{../src/backtrace/backtrace.ml}}
      \endminipage
    \end{column}
    \begin{column}{0.45\textwidth}
      \raggedleft
      \onslide*<1>{\providecommand\step{6}\input{diagrams/backtrace/backtrace.tex}}%
      \onslide*<2>{\providecommand\step{6}\input{diagrams/backtrace/backtrace.tex}}%
      \onslide*<3>{\providecommand\step{7}\input{diagrams/backtrace/backtrace.tex}}%
      \onslide*<4>{\providecommand\step{8}\input{diagrams/backtrace/backtrace.tex}}%
      \onslide*<5>{\providecommand\step{9}\input{diagrams/backtrace/backtrace.tex}}%
      \onslide*<6>{\providecommand\step{10}\input{diagrams/backtrace/backtrace.tex}}%
      \onslide*<7>{\providecommand\step{11}\input{diagrams/backtrace/backtrace.tex}}%
      \onslide*<8>{\providecommand\step{12}\input{diagrams/backtrace/backtrace.tex}}%
      \onslide*<9>{\providecommand\step{13}\input{diagrams/backtrace/backtrace.tex}}%
    \end{column}
  \end{columns}
\end{frame}

\begin{frame}{Backtraces}{Printing the backtrace}
  \begin{columns}[c]
    \begin{column}{0.45\textwidth}
      \minipage[c][0.66\textheight][s]{\columnwidth}
      \begin{itemize}
        \item<1-> The exception backtrace can now be printed to \funcarg{stdout} with \funcname{Printexc.print_backtrace}
        \item<2-> First the exception backtrace is retrieved into an OCaml array with \funcname{get_raw_backtrace }
        \item<3-> Which is implemented as the \funcname{caml_get_exception_raw_backtrace} C function in the runtime
        \item<4-> And finally printed with \funcname{print_exception_backtrace}
      \end{itemize}
      \vfill
      \mintocaml[firstline=28,lastline=34,highlightlines=33]{../src/backtrace/backtrace.ml}
      \endminipage
    \end{column}
    \begin{column}{0.55\textwidth}
      \onslide*<1,2>{
        \mintocaml[firstline=100,lastline=101]{../ocaml/stdlib/printexc.ml}
      }
      \only<1>{
        \listocaml[firstline=170,lastline=172]{../ocaml/stdlib/printexc.ml}{stdlib/printexc.ml:170}
      }
      \only<2>{
        \listocaml[firstline=170,lastline=172,highlightlines=172]{../ocaml/stdlib/printexc.ml}{stdlib/printexc.ml:170}
      }
      \only<3>{
        \mintc[firstline=154,lastline=156]{../ocaml/runtime/backtrace.c}
        \listc[firstline=171,lastline=192,highlightlines={184-187}]{../ocaml/runtime/backtrace.c}{runtime/backtrace.c:154}
      }
      \only<4>{
        \listocaml[firstline=155,lastline=165]{../ocaml/stdlib/printexc.ml}{stdlib/printexc.ml:55}
      }
    \end{column}
  \end{columns}
\end{frame}


\subsection{The multiple kinds of raise}
\frameSubsection{}{}

\begin{frame}{Backtraces}{When backtraces are not needed}
  \begin{columns}[c]
    \begin{column}{0.45\textwidth}
      \minipage[c][0.66\textheight][s]{\columnwidth}
      \begin{itemize}
        \item<1-> What if the backtrace for an exception is never needed?
        \item<2-> It's possible to raise the exception explicitly with \funcname{raise\_notrace} to make raising as cheap as possible
        \item<3-> An example of use in the \funcname{dynlink} library of the OCaml compiler
      \end{itemize}
      \vfill
      \onslide<3->{
      \listocaml[firstline=205,lastline=209,highlightlines=207]{../ocaml/otherlibs/dynlink/dynlink\_compilerlibs/misc.ml}{otherlibs/dynlink/dynlink\_compilerlibs/misc.ml:205}
      }
      \endminipage
    \end{column}
    \begin{column}{0.55\textwidth}
    \only<4>{
      \mintcmm[highlightlines=3]{../src/notrace/camlNotrace\_\_fun_347.cmm}
      \listcmm{../src/notrace/camlNotrace\_\_all\_somes\_327.cmm}{all\_somes}
    }
    \onslide*<5->{
      \listobjdump[highlightlines={6-9}]{../src/notrace/camlNotrace\_\_fun_347.objdump}{anonymous function from \funcname{all\_somes}}
    }
    \end{column}
  \end{columns}
\end{frame}

\begin{frame}{Backtraces}{Summary table of the multiple kinds of raise}
  \begin{itemize}
    \item When debugging information is enabled and if the backtrace of an exception isn't needed, it's possible to raise with \funcname{raise\_notrace} for performance
    \item When debugging information is \emph{not} enabled, the compiler optimises \funcname{raise} calls to inline assembly (same effect as using \funcname{raise\_notrace})
  \end{itemize}
  \bigskip
  \centering
  \begin{tabular}{| l | c | c | c | c |}
    \hline
    \parbox{10em}{Debug information \\ (\funcarg{-g} flag)}  & \multicolumn{2}{|c|}{Disabled}                                     & \multicolumn{2}{|c|}{Enabled} \\
    \hline
    Raise kind                                               & \funcname{raise} & \funcname{raise\_notrace}                       & \funcname{raise} & \funcname{raise\_notrace} \\
    \hline
    Compilation                                              & \parbox{8em}{inline assembly\\ (optimization)} & inline assembly   & \funcname{caml\_raise\_exn} & inline assembly \\
    \hline
  \end{tabular}
\end{frame}


%
% Takeaway #5
%
\subsection*{Takeaway \#5}
\frameSubsectionTakeaway{}
\againframe<5>{takeaway}
}%
      \onslide*<6>{\providecommand\step{10}%
% Backtraces
%
\subsection{One advantage of raising with debugging information}
\frameSubsection{}{}

\begin{frame}{Backtraces}{Debugging information \& source code}
  \begin{columns}[c]
    \begin{column}{0.5\textwidth}
      \begin{itemize}
        \item Raising an exception is fun and games, but how do we know where the exception originated? $\rightarrow$ With the backtrace associated with the exception!
        \item In order to get a backtrace from the point where an exception was raised, it's necessary to compile with debugging information enabled (\funcarg{-g} flag for the compiler)
        \item Same example as in the `Nested handlers' section
      \end{itemize}
    \end{column}
    \begin{column}{0.5\textwidth}
      \listocaml[firstline=9,lastline=34]{../src/backtrace/backtrace.ml}{backtrace/backtrace.ml}
    \end{column}
  \end{columns}
\end{frame}

\begin{frame}{Backtraces}{CMM and disassembly of \funcname{definitely\_raise}}
  \begin{columns}[c]
    \begin{column}{0.5\textwidth}
      \minipage[c][0.66\textheight][s]{\columnwidth}
      \begin{itemize}
        \item<1-> An effect of compiling with debugging information (\funcarg{-g}): the CMM no longer uses \funcname{raise\_notrace} to raise the exception but \funcname{raise} instead
        \item<2-> Disassembly shows that CMM \funcname{raise} is actually a call to \funcname{caml\_raise\_exn}, a function in assembler provided by the runtime
      \end{itemize}
      \vfill
      \mintocaml[firstline=9,lastline=13,highlightlines=12]{../src/backtrace/backtrace.ml}
      \endminipage
    \end{column}
    \begin{column}{0.5\textwidth}
      \onslide*<1>{
        \mintcmm[highlightlines=5]{../src/backtrace/camlBacktrace\_\_definitely\_raise\_347.cmm}
      }
      \onslide*<2>{
        \mintobjdump[firstline=2,highlightlines=14]{../src/backtrace/camlBacktrace\_\_definitely\_raise\_347.objdump}
      }
    \end{column}
  \end{columns}
\end{frame}

\begin{frame}{Backtraces}{Introducing \funcname{caml\_raise\_exn}}
  \begin{columns}[c]
    \begin{column}{0.5\textwidth}
      \minipage[c][0.66\textheight][s]{\columnwidth}
      \onslide*<1>{
        \begin{itemize}
          \item Raising the exception is performed using \funcname{caml\_raise\_exn} which is
            \begin{itemize}
              \item An assembly function provided by the runtime
              \item Architecture specific
            \end{itemize}
          \item Let's see what it does differently from \funcname{raise\_notrace}\dots
          \item We are about to raise \typename{ExnC} with \funcname{caml\_raise\_exn} from \funcname{definitely\_raise}
        \end{itemize}
      }
      \onslide*<2>{
        \mintobjdump[firstline=2,highlightlines=14]{../src/backtrace/camlBacktrace\_\_definitely\_raise\_347.objdump}
      }
      \vfill
      \mintocaml[firstline=9,lastline=13,highlightlines=12]{../src/backtrace/backtrace.ml}
      \endminipage
    \end{column}
    \begin{column}{0.5\textwidth}
      \centering
      \providecommand\step{1}\input{diagrams/backtrace/backtrace.tex}
    \end{column}
  \end{columns}
\end{frame}

\begin{frame}{Backtraces}{But before that, we need more fields from \typename{caml\_domain\_state}}
  \begin{columns}
    \begin{column}{0.5\textwidth}
      \begin{itemize}
        \item Remember \typename{caml\_domain\_state} the per-domain runtime state?
        \item Some other members are of interest to us now\dots
        \item \localname{backtrace\_active} is used to know if the runtime should collect backtraces
        \item \localname{backtrace\_active} can be set by
          \begin{itemize}
            \item The \funcarg{b} flag in the \funcname{OCAMLRUNPARAM} environment variable
            \item \mintinline{ocaml}{Printexc.record_backtrace : bool -> unit} \footnotemark
          \end{itemize}
      \end{itemize}
    \end{column}
    \begin{column}{0.5\textwidth}
      \begin{listing}
        \mintc[highlightlines={9-11}]{src/ocaml/domain\_state\_backtrace.h}
        \caption{runtime/caml/domain\_state.h}
      \end{listing}
    \end{column}
  \end{columns}
  \footnotetext[1]{https://v2.ocaml.org/api/Printexc.html}
\end{frame}

\newcommand\listCamlRaiseExn[1][]{
  \listasm[firstline=702,lastline=725,#1]{../ocaml/runtime/amd64.S}{runtime/amd64.S:702}}

\begin{frame}{Backtraces}{Walk through \funcname{caml\_raise\_exn}}
  \begin{columns}[T]
    \begin{column}{0.5\textwidth}
      \begin{itemize}
        \item<1-> First checks whether exceptions backtrace should be recorded
        \item<1-> By checking the \funcarg{domain\_state->backtrace\_active} value
        \item<2-> If not, acts as \funcname{raise\_notrace} with \funcname{RESTORE\_EXN\_HANDLER\_OCAML}
          \begin{itemize}
            \item Set \regname{RSP} to \localname{domain\_state->exn\_handler} value
            \item Restore \localname{domain\_state->exn\_handler} from trap
            \item Jump with \funcname{ret} (same effect as using \funcname{pop} and \funcname{jmp})
          \end{itemize}
        \item<3-> Otherwise, switches to the C stack to be able to execute C code
        \item<4-> Calls \funcname{caml\_stash\_backtrace}, a C function provided by the runtime\ignorespaces
          \onslide<6->{, with
            \begin{itemize}
              \item \funcarg{exn}: exception value
              \item \funcarg{pc}: code address of the raising point
              \item \funcarg{sp}: stack address of the raising function
              \item \funcarg{trapsp}: stack address of the next handler
            \end{itemize}
          }
      \end{itemize}
      \bigskip
      \mintocaml[firstline=9,lastline=13,highlightlines=12]{../src/backtrace/backtrace.ml}
    \end{column}
    \begin{column}{0.5\textwidth}
      \raggedleft
      \onslide*<1,3-4>{
        \mintc[firstline=190,lastline=192]{../ocaml/runtime/amd64.S}
        \mintc[firstline=234,lastline=237]{../ocaml/runtime/amd64.S}
      }
      \onslide*<2>{
        \mintc[firstline=190,lastline=192,highlightlines={191-192}]{../ocaml/runtime/amd64.S}
        \mintc[firstline=234,lastline=237,highlightlines={235-237}]{../ocaml/runtime/amd64.S}
      }
      \onslide*<1>{\listCamlRaiseExn[highlightlines=705]{}}%
      \onslide*<2>{\listCamlRaiseExn[highlightlines={707-708}]{}}%
      \onslide*<3>{\listCamlRaiseExn[highlightlines={712-714}]{}}%
      \onslide*<4>{\listCamlRaiseExn[highlightlines={715-720}]{}}%
      \onslide*<5>{\providecommand\step{1}\input{diagrams/backtrace/backtrace.tex}}%
      \onslide*<6>{\providecommand\step{2}\input{diagrams/backtrace/backtrace.tex}}%
    \end{column}
  \end{columns}
\end{frame}

\newcommand\listStashBacktrace[1][]{
  \listc[firstline=91,lastline=120,#1]{../ocaml/runtime/backtrace\_nat.c}{runtime/backtrace\_nat.c:91}}

\begin{frame}{Backtraces}{Introducing \funcname{caml\_stash\_backtrace}}
  \begin{columns}[c]
    \begin{column}{0.5\textwidth}
      \minipage[c][0.66\textheight][s]{\columnwidth}
      \begin{itemize}
        \item<1-> A C function provided by the runtime (specific to native code)
        \item<2-> It scans the stack, between the stack pointer at the raising point up to the next trap, in order to collect the names of the detected function frames
          \onslide*<2>{
            \begin{itemize}
              \item And more information, like line number, column number...
            \end{itemize}
          }
        \item<3-> It uses the same technique and information as the Garbage Collector (GC) uses to scan the values live on the stack
          \onslide*<4>{
            \begin{itemize}
              \item \typename{frame\_descr} are at the core of stack unwinding
            \end{itemize}
          }
      \end{itemize}
      \vfill
      \mintocaml[firstline=9,lastline=13,highlightlines=12]{../src/backtrace/backtrace.ml}
      \endminipage
    \end{column}
    \begin{column}{0.5\textwidth}
      \onslide*<1>{\listStashBacktrace[highlightlines=91]{}}
      \onslide*<2>{\listStashBacktrace[highlightlines={91,117-118}]{}}
      \onslide*<3->{\listStashBacktrace[highlightlines={94,106,109-110}]{}}
    \end{column}
  \end{columns}
\end{frame}

\newcommand\listFrameDescr[1][]{
  \listocaml[firstline=111,lastline=124,#1]{../ocaml/asmcomp/emitaux.ml}{asmcomp/emitaux.ml:111}}
\newcommand\listDebugInfo[1][]{
  \listocaml[firstline=38,lastline=49,#1]{../ocaml/lambda/debuginfo.mli}{lambda/debuginfo.mli:38}}

\begin{frame}{Backtraces}{\typename{frame\_descr} and \typename{frame\_debuginfo} in a nutshell}
  \begin{columns}[c]
    \begin{column}{0.5\textwidth}
      \begin{itemize}
        \item<1-> For every return address in an OCaml program, there is a associated \typename{frame\_descr}
        \item<2-> A \typename{frame\_descr} specifies
        \begin{itemize}
          \item<2-> The size of the function stack frame
          \item<3-> Where live values are on the stack (so that the GC can mark them as live and prevent from being swept)
        \end{itemize}
        \item<4-> When compiled with debug information, \typename{frame\_descr} will be linked to a \typename{frame\_debuginfo}
        \begin{itemize}
          \item<5-> Contains information about the position in the source code (file name, line and column number)
        \end{itemize}
      \end{itemize}
    \end{column}
    \begin{column}{0.5\textwidth}
        \onslide*<1>{\listFrameDescr[highlightlines={118-122}]{}}
        \onslide*<2>{\listFrameDescr[highlightlines={120}]{}}
        \onslide*<3>{\listFrameDescr[highlightlines={121}]{}}
        \onslide*<4->{\listFrameDescr[highlightlines={122}]{}}
        \medskip
        \onslide*<1-3>{\listDebugInfo{}}
        \onslide*<4>{\listDebugInfo[highlightlines={38,49}]{}}
        \onslide*<5>{\listDebugInfo[highlightlines={39-46}]{}}
    \end{column}
  \end{columns}
\end{frame}

\begin{frame}{Backtraces}{Scanning the stack with \typename{frame\_descr}}
  \begin{columns}[c]
    \begin{column}{0.5\textwidth}
      \begin{itemize}
        \item Given the return address of the code that raises the exception by calling \funcname{caml\_raise\_exn} (\funcarg{pc} argument of \funcname{caml\_stash\_backtrace})
        \item And the position of the stack pointer (\funcarg{sp} argument of \funcname{caml\_stash\_backtrace})
        \item The frame size given by \typename{frame\_descr}.\funcarg{frame\_size}, allows to find the end of the previous frame
        \item And the return address of the caller of this function
        \bigskip
        \onslide<3->{
        \item Note that traps (here the reddish \funcname{ExnA}, \funcname{ExnB} and \funcname{ExnC} blocks) belong to the frame that installed them, and so the frame size changes when traps are removed
        }
      \end{itemize}
    \end{column}
    \begin{column}{0.5\textwidth}
      \raggedleft
      \onslide*<1>{\providecommand\step{2}\input{diagrams/backtrace/backtrace.tex}}%
      \onslide*<2>{\providecommand\step{1}\input{diagrams/backtrace/scanstack.tex}}%
      \onslide*<3>{\providecommand\step{2}\input{diagrams/backtrace/scanstack.tex}}%
      \onslide*<4>{\providecommand\step{3}\input{diagrams/backtrace/scanstack.tex}}%
      \onslide*<5>{\providecommand\step{4}\input{diagrams/backtrace/scanstack.tex}}%
      \onslide*<6>{\providecommand\step{5}\input{diagrams/backtrace/scanstack.tex}}%
    \end{column}
  \end{columns}
\end{frame}

% Duplicate of previous-previous slide
\begin{frame}{Backtraces}{Continuing on \funcname{caml\_stash\_backtrace}}
  \begin{columns}[c]
    \begin{column}{0.5\textwidth}
      \minipage[c][0.75\textheight][s]{\columnwidth}
      \begin{itemize}
          \color{gray}
        \item A C function provided by the runtime (specific to native code)
        \item It scans the stack, between the stack pointer at the raising point up to the next trap, in order to collect the names of the detected function frames
        \item It uses the same technique and information as the Garbage Collector (GC) uses to scan the values live on the stack
      \end{itemize}
      \begin{itemize}
        \item For each function frame detected, store a pointer to the \typename{frame\_descr} in the \localname{domain\_state->backtrace\_buffer} array, effectively constructing the backtrace
      \end{itemize}
      \onslide<2->{
        \begin{itemize}
            \smallskip
          \item Constructed backtrace:
            \funcname{definitely\_raise},
            \onslide<3->{\funcname{probably\_raise}}
        \end{itemize}
      }
      \vfill
      \mintocaml[firstline=9,lastline=13,highlightlines=12]{../src/backtrace/backtrace.ml}
      \endminipage
    \end{column}
    \begin{column}{0.5\textwidth}
      \raggedleft
      \onslide*<1>{\listStashBacktrace[highlightlines={114-115}]{}}
      \onslide*<2>{\providecommand\step{3}\input{diagrams/backtrace/backtrace.tex}}%
      \onslide*<3>{\providecommand\step{4}\input{diagrams/backtrace/backtrace.tex}}%
    \end{column}
  \end{columns}
\end{frame}

\begin{frame}{Backtraces}{Back to \funcname{caml\_raise\_exn}}
  \begin{columns}[c]
    \begin{column}{0.5\textwidth}
      \minipage[c][0.66\textheight][s]{\columnwidth}
      \begin{itemize}\color{gray}
        \item Checks wheteher exceptions backtrace should be recorded, by checking the \funcarg{domain\_state->backtrace\_active} value
        \item Switches to the C stack to be able to execute C code
        \item Calls \funcname{caml\_stash\_backtrace}, to collect the backtrace up to the next trap
      \end{itemize}
      \onslide*<2->{
        \begin{itemize}
          \item<2-> Executes the exception handler in \funcname{probably\_raise} with \funcname{RESTORE\_EXN\_HANDLER\_OCAML}
            \begin{itemize}
              \item Set \regname{RSP} to \localname{domain\_state->exn\_handler} value
              \item Restore \localname{domain\_state->exn\_handler} from trap
              \item Jump to the handler code with \funcname{ret}
            \end{itemize}
        \end{itemize}
      }
      \vfill
      \onslide*<1>{\mintocaml[firstline=9,lastline=13,highlightlines=12]{../src/backtrace/backtrace.ml}}
      \onslide*<2->{\mintocaml[firstline=15,lastline=21,highlightlines={19-21}]{../src/backtrace/backtrace.ml}}
      \endminipage
    \end{column}
    \begin{column}{0.5\textwidth}
      \centering
      \onslide*<1>{
        \mintc[firstline=190,lastline=192]{../ocaml/runtime/amd64.S}
        \mintc[firstline=234,lastline=237]{../ocaml/runtime/amd64.S}
      }
      \onslide*<2>{
        \mintc[firstline=190,lastline=192,highlightlines={191-192}]{../ocaml/runtime/amd64.S}
        \mintc[firstline=234,lastline=237,highlightlines={235-237}]{../ocaml/runtime/amd64.S}
      }
      \onslide*<1>{\listCamlRaiseExn[highlightlines=720]{}}
      \onslide*<2>{\listCamlRaiseExn[highlightlines=722]{}}
      \onslide*<3->{\providecommand\step{5}\input{diagrams/backtrace/backtrace.tex}}
    \end{column}
  \end{columns}
\end{frame}

\begin{frame}{Backtraces}{\funcname{caml\_probably\_raise}'s handler doesn't catch \typename{ExnC}}
  \begin{columns}[c]
    \begin{column}{0.45\textwidth}
      \minipage[c][0.75\textheight][s]{\columnwidth}
      \begin{itemize}
        \item<1-> \funcname{match \dots with}\footnotemark pattern matching can also match exceptions
        \item<1-> The CMM of \funcname{probably\_raise} shows that this \funcname{match \dots with} is implemented with a pattern matching and a \funcname{try \dots with}
        \item<1-> But this one only matches on \typename{ExnA} exception
        \item<2-> Since the pattern matching fails to catch the exception, \funcname{reraise} is called with our current \typename{ExnC} exception
        \item<3-> Disassembly shows that \funcname{reraise} is actually a call to \funcname{caml\_reraise\_exn}, which is also an assembly function provided by the runtime
      \end{itemize}
      \vfill
      \mintocaml[firstline=15,lastline=21,highlightlines={18-21}]{../src/backtrace/backtrace.ml}
      \endminipage
    \end{column}
    \begin{column}{0.55\textwidth}
      \centering
      \onslide*<1>{\mintcmm[highlightlines={10-18}]{../src/backtrace/camlBacktrace\_\_probably\_raise\_351.cmm}}
      \onslide*<2>{\listcmm[highlightlines=18]{../src/backtrace/camlBacktrace\_\_probably\_raise_351.cmm}{CMM of probably\_raise}}
      \onslide*<3>{\mintobjdump[firstline=5,lastline=35,highlightlines=31]{../src/backtrace/camlBacktrace\_\_probably\_raise_351.objdump}}
    \end{column}
  \end{columns}
  \only<1>{\footnotetext[1]{https://blog.janestreet.com/pattern-matching-and-exception-handling-unite/}}
\end{frame}

\newcommand\listCamlReraiseExn[1][]{
  \listasm[firstline=727,lastline=734,#1]{../ocaml/runtime/amd64.S}{runtime/amd64.S:727}}

\begin{frame}{Backtraces}{Introducing \funcname{caml\_reraise\_exn}}
  \begin{columns}[c]
    \begin{column}{0.5\textwidth}
      \minipage[c][0.66\textheight][s]{\columnwidth}
      \begin{itemize}
        \item<1-> The exception have to be reraised to the next handler
        \item<2-> First, checks if the backtrace should be collected with \funcarg{domain\_state->backtrace\_active}
        \only<3>{
        \begin{itemize}
            \item If not, behaves like a \funcname{raise\_notrace} with \funcname{RESTORE\_EXN\_HANDLER\_OCAML}
        \end{itemize}
        }
        \item<4-> Jumps into \funcname{caml\_raise\_exn}
        \item<5-> Calls \funcname{caml\_stash\_backtrace} again, to scan the next part of the stack: from the trap just pop-ed to the next trap
      \end{itemize}
      \vfill
      \mintocaml[firstline=15,lastline=21,highlightlines={18-21}]{../src/backtrace/backtrace.ml}
      \endminipage
    \end{column}
    \begin{column}{0.5\textwidth}
      \raggedleft
      \onslide*<1>{\listCamlReraiseExn{}}
      \onslide*<2>{\listCamlReraiseExn[highlightlines=729]{}}
      \onslide*<3>{
        \mintc[firstline=190,lastline=192,highlightlines={191-192}]{../ocaml/runtime/amd64.S}
        \mintc[firstline=234,lastline=237,highlightlines={235-237}]{../ocaml/runtime/amd64.S}
        \medskip
        \listCamlReraiseExn[highlightlines=731]{}
      }
      \onslide*<4-5>{
        % TODO refactor with \listCamlReraiseExn
        \mintasm[firstline=727,lastline=734,highlightlines=730]{../ocaml/runtime/amd64.S}
        \medskip
      }
      \onslide*<4>{\listCamlRaiseExn[highlightlines=711]{}}
      \onslide*<5>{\listCamlRaiseExn[highlightlines=720]{}}
    \end{column}
  \end{columns}
\end{frame}

\begin{frame}{Backtraces}{Backtrace collection continues}
  \begin{columns}[c]
    \begin{column}{0.55\textwidth}
      \minipage[c][0.75\textheight][s]{\columnwidth}
      \begin{itemize}
        \item<1-> \funcname{caml\_stash\_backtrace} scans the older part of the stack, up to the next trap
        \item<6-> Controls jumps to the nested exception handler in \funcname{maybe\_raise}, which matches only on \typename{ExnB}
        \item<7-> \funcname{caml\_reraise\_exn} is called again, backtrace collection resumes with \funcname{caml\_stash\_backtrace}
      \end{itemize}
      \medskip
      \begin{itemize}
        \item<2-> Constructed backtrace:
        \begin{itemize}
          \item <2-> \funcname{definitely\_raise}
          \item <2-> \funcname{probably\_raise}
          \item <3-> \funcname{probably\_raise} (re-raise)
          \item <4-> \funcname{maybe\_raise}
          \item <5-> \funcname{main}
          \item <8-> \funcname{main} (re-raise)
        \end{itemize}
      \end{itemize}
      \vfill
      \onslide*<1-5>{\mintocaml[firstline=15,lastline=21]{../src/backtrace/backtrace.ml}}
      \onslide*<6>{\mintocaml[firstline=28,lastline=34,highlightlines=31]{../src/backtrace/backtrace.ml}}
      \onslide*<7->{\mintocaml[firstline=28,lastline=34,highlightlines=33]{../src/backtrace/backtrace.ml}}
      \endminipage
    \end{column}
    \begin{column}{0.45\textwidth}
      \raggedleft
      \onslide*<1>{\providecommand\step{6}\input{diagrams/backtrace/backtrace.tex}}%
      \onslide*<2>{\providecommand\step{6}\input{diagrams/backtrace/backtrace.tex}}%
      \onslide*<3>{\providecommand\step{7}\input{diagrams/backtrace/backtrace.tex}}%
      \onslide*<4>{\providecommand\step{8}\input{diagrams/backtrace/backtrace.tex}}%
      \onslide*<5>{\providecommand\step{9}\input{diagrams/backtrace/backtrace.tex}}%
      \onslide*<6>{\providecommand\step{10}\input{diagrams/backtrace/backtrace.tex}}%
      \onslide*<7>{\providecommand\step{11}\input{diagrams/backtrace/backtrace.tex}}%
      \onslide*<8>{\providecommand\step{12}\input{diagrams/backtrace/backtrace.tex}}%
      \onslide*<9>{\providecommand\step{13}\input{diagrams/backtrace/backtrace.tex}}%
    \end{column}
  \end{columns}
\end{frame}

\begin{frame}{Backtraces}{Printing the backtrace}
  \begin{columns}[c]
    \begin{column}{0.45\textwidth}
      \minipage[c][0.66\textheight][s]{\columnwidth}
      \begin{itemize}
        \item<1-> The exception backtrace can now be printed to \funcarg{stdout} with \funcname{Printexc.print_backtrace}
        \item<2-> First the exception backtrace is retrieved into an OCaml array with \funcname{get_raw_backtrace }
        \item<3-> Which is implemented as the \funcname{caml_get_exception_raw_backtrace} C function in the runtime
        \item<4-> And finally printed with \funcname{print_exception_backtrace}
      \end{itemize}
      \vfill
      \mintocaml[firstline=28,lastline=34,highlightlines=33]{../src/backtrace/backtrace.ml}
      \endminipage
    \end{column}
    \begin{column}{0.55\textwidth}
      \onslide*<1,2>{
        \mintocaml[firstline=100,lastline=101]{../ocaml/stdlib/printexc.ml}
      }
      \only<1>{
        \listocaml[firstline=170,lastline=172]{../ocaml/stdlib/printexc.ml}{stdlib/printexc.ml:170}
      }
      \only<2>{
        \listocaml[firstline=170,lastline=172,highlightlines=172]{../ocaml/stdlib/printexc.ml}{stdlib/printexc.ml:170}
      }
      \only<3>{
        \mintc[firstline=154,lastline=156]{../ocaml/runtime/backtrace.c}
        \listc[firstline=171,lastline=192,highlightlines={184-187}]{../ocaml/runtime/backtrace.c}{runtime/backtrace.c:154}
      }
      \only<4>{
        \listocaml[firstline=155,lastline=165]{../ocaml/stdlib/printexc.ml}{stdlib/printexc.ml:55}
      }
    \end{column}
  \end{columns}
\end{frame}


\subsection{The multiple kinds of raise}
\frameSubsection{}{}

\begin{frame}{Backtraces}{When backtraces are not needed}
  \begin{columns}[c]
    \begin{column}{0.45\textwidth}
      \minipage[c][0.66\textheight][s]{\columnwidth}
      \begin{itemize}
        \item<1-> What if the backtrace for an exception is never needed?
        \item<2-> It's possible to raise the exception explicitly with \funcname{raise\_notrace} to make raising as cheap as possible
        \item<3-> An example of use in the \funcname{dynlink} library of the OCaml compiler
      \end{itemize}
      \vfill
      \onslide<3->{
      \listocaml[firstline=205,lastline=209,highlightlines=207]{../ocaml/otherlibs/dynlink/dynlink\_compilerlibs/misc.ml}{otherlibs/dynlink/dynlink\_compilerlibs/misc.ml:205}
      }
      \endminipage
    \end{column}
    \begin{column}{0.55\textwidth}
    \only<4>{
      \mintcmm[highlightlines=3]{../src/notrace/camlNotrace\_\_fun_347.cmm}
      \listcmm{../src/notrace/camlNotrace\_\_all\_somes\_327.cmm}{all\_somes}
    }
    \onslide*<5->{
      \listobjdump[highlightlines={6-9}]{../src/notrace/camlNotrace\_\_fun_347.objdump}{anonymous function from \funcname{all\_somes}}
    }
    \end{column}
  \end{columns}
\end{frame}

\begin{frame}{Backtraces}{Summary table of the multiple kinds of raise}
  \begin{itemize}
    \item When debugging information is enabled and if the backtrace of an exception isn't needed, it's possible to raise with \funcname{raise\_notrace} for performance
    \item When debugging information is \emph{not} enabled, the compiler optimises \funcname{raise} calls to inline assembly (same effect as using \funcname{raise\_notrace})
  \end{itemize}
  \bigskip
  \centering
  \begin{tabular}{| l | c | c | c | c |}
    \hline
    \parbox{10em}{Debug information \\ (\funcarg{-g} flag)}  & \multicolumn{2}{|c|}{Disabled}                                     & \multicolumn{2}{|c|}{Enabled} \\
    \hline
    Raise kind                                               & \funcname{raise} & \funcname{raise\_notrace}                       & \funcname{raise} & \funcname{raise\_notrace} \\
    \hline
    Compilation                                              & \parbox{8em}{inline assembly\\ (optimization)} & inline assembly   & \funcname{caml\_raise\_exn} & inline assembly \\
    \hline
  \end{tabular}
\end{frame}


%
% Takeaway #5
%
\subsection*{Takeaway \#5}
\frameSubsectionTakeaway{}
\againframe<5>{takeaway}
}%
      \onslide*<7>{\providecommand\step{11}%
% Backtraces
%
\subsection{One advantage of raising with debugging information}
\frameSubsection{}{}

\begin{frame}{Backtraces}{Debugging information \& source code}
  \begin{columns}[c]
    \begin{column}{0.5\textwidth}
      \begin{itemize}
        \item Raising an exception is fun and games, but how do we know where the exception originated? $\rightarrow$ With the backtrace associated with the exception!
        \item In order to get a backtrace from the point where an exception was raised, it's necessary to compile with debugging information enabled (\funcarg{-g} flag for the compiler)
        \item Same example as in the `Nested handlers' section
      \end{itemize}
    \end{column}
    \begin{column}{0.5\textwidth}
      \listocaml[firstline=9,lastline=34]{../src/backtrace/backtrace.ml}{backtrace/backtrace.ml}
    \end{column}
  \end{columns}
\end{frame}

\begin{frame}{Backtraces}{CMM and disassembly of \funcname{definitely\_raise}}
  \begin{columns}[c]
    \begin{column}{0.5\textwidth}
      \minipage[c][0.66\textheight][s]{\columnwidth}
      \begin{itemize}
        \item<1-> An effect of compiling with debugging information (\funcarg{-g}): the CMM no longer uses \funcname{raise\_notrace} to raise the exception but \funcname{raise} instead
        \item<2-> Disassembly shows that CMM \funcname{raise} is actually a call to \funcname{caml\_raise\_exn}, a function in assembler provided by the runtime
      \end{itemize}
      \vfill
      \mintocaml[firstline=9,lastline=13,highlightlines=12]{../src/backtrace/backtrace.ml}
      \endminipage
    \end{column}
    \begin{column}{0.5\textwidth}
      \onslide*<1>{
        \mintcmm[highlightlines=5]{../src/backtrace/camlBacktrace\_\_definitely\_raise\_347.cmm}
      }
      \onslide*<2>{
        \mintobjdump[firstline=2,highlightlines=14]{../src/backtrace/camlBacktrace\_\_definitely\_raise\_347.objdump}
      }
    \end{column}
  \end{columns}
\end{frame}

\begin{frame}{Backtraces}{Introducing \funcname{caml\_raise\_exn}}
  \begin{columns}[c]
    \begin{column}{0.5\textwidth}
      \minipage[c][0.66\textheight][s]{\columnwidth}
      \onslide*<1>{
        \begin{itemize}
          \item Raising the exception is performed using \funcname{caml\_raise\_exn} which is
            \begin{itemize}
              \item An assembly function provided by the runtime
              \item Architecture specific
            \end{itemize}
          \item Let's see what it does differently from \funcname{raise\_notrace}\dots
          \item We are about to raise \typename{ExnC} with \funcname{caml\_raise\_exn} from \funcname{definitely\_raise}
        \end{itemize}
      }
      \onslide*<2>{
        \mintobjdump[firstline=2,highlightlines=14]{../src/backtrace/camlBacktrace\_\_definitely\_raise\_347.objdump}
      }
      \vfill
      \mintocaml[firstline=9,lastline=13,highlightlines=12]{../src/backtrace/backtrace.ml}
      \endminipage
    \end{column}
    \begin{column}{0.5\textwidth}
      \centering
      \providecommand\step{1}\input{diagrams/backtrace/backtrace.tex}
    \end{column}
  \end{columns}
\end{frame}

\begin{frame}{Backtraces}{But before that, we need more fields from \typename{caml\_domain\_state}}
  \begin{columns}
    \begin{column}{0.5\textwidth}
      \begin{itemize}
        \item Remember \typename{caml\_domain\_state} the per-domain runtime state?
        \item Some other members are of interest to us now\dots
        \item \localname{backtrace\_active} is used to know if the runtime should collect backtraces
        \item \localname{backtrace\_active} can be set by
          \begin{itemize}
            \item The \funcarg{b} flag in the \funcname{OCAMLRUNPARAM} environment variable
            \item \mintinline{ocaml}{Printexc.record_backtrace : bool -> unit} \footnotemark
          \end{itemize}
      \end{itemize}
    \end{column}
    \begin{column}{0.5\textwidth}
      \begin{listing}
        \mintc[highlightlines={9-11}]{src/ocaml/domain\_state\_backtrace.h}
        \caption{runtime/caml/domain\_state.h}
      \end{listing}
    \end{column}
  \end{columns}
  \footnotetext[1]{https://v2.ocaml.org/api/Printexc.html}
\end{frame}

\newcommand\listCamlRaiseExn[1][]{
  \listasm[firstline=702,lastline=725,#1]{../ocaml/runtime/amd64.S}{runtime/amd64.S:702}}

\begin{frame}{Backtraces}{Walk through \funcname{caml\_raise\_exn}}
  \begin{columns}[T]
    \begin{column}{0.5\textwidth}
      \begin{itemize}
        \item<1-> First checks whether exceptions backtrace should be recorded
        \item<1-> By checking the \funcarg{domain\_state->backtrace\_active} value
        \item<2-> If not, acts as \funcname{raise\_notrace} with \funcname{RESTORE\_EXN\_HANDLER\_OCAML}
          \begin{itemize}
            \item Set \regname{RSP} to \localname{domain\_state->exn\_handler} value
            \item Restore \localname{domain\_state->exn\_handler} from trap
            \item Jump with \funcname{ret} (same effect as using \funcname{pop} and \funcname{jmp})
          \end{itemize}
        \item<3-> Otherwise, switches to the C stack to be able to execute C code
        \item<4-> Calls \funcname{caml\_stash\_backtrace}, a C function provided by the runtime\ignorespaces
          \onslide<6->{, with
            \begin{itemize}
              \item \funcarg{exn}: exception value
              \item \funcarg{pc}: code address of the raising point
              \item \funcarg{sp}: stack address of the raising function
              \item \funcarg{trapsp}: stack address of the next handler
            \end{itemize}
          }
      \end{itemize}
      \bigskip
      \mintocaml[firstline=9,lastline=13,highlightlines=12]{../src/backtrace/backtrace.ml}
    \end{column}
    \begin{column}{0.5\textwidth}
      \raggedleft
      \onslide*<1,3-4>{
        \mintc[firstline=190,lastline=192]{../ocaml/runtime/amd64.S}
        \mintc[firstline=234,lastline=237]{../ocaml/runtime/amd64.S}
      }
      \onslide*<2>{
        \mintc[firstline=190,lastline=192,highlightlines={191-192}]{../ocaml/runtime/amd64.S}
        \mintc[firstline=234,lastline=237,highlightlines={235-237}]{../ocaml/runtime/amd64.S}
      }
      \onslide*<1>{\listCamlRaiseExn[highlightlines=705]{}}%
      \onslide*<2>{\listCamlRaiseExn[highlightlines={707-708}]{}}%
      \onslide*<3>{\listCamlRaiseExn[highlightlines={712-714}]{}}%
      \onslide*<4>{\listCamlRaiseExn[highlightlines={715-720}]{}}%
      \onslide*<5>{\providecommand\step{1}\input{diagrams/backtrace/backtrace.tex}}%
      \onslide*<6>{\providecommand\step{2}\input{diagrams/backtrace/backtrace.tex}}%
    \end{column}
  \end{columns}
\end{frame}

\newcommand\listStashBacktrace[1][]{
  \listc[firstline=91,lastline=120,#1]{../ocaml/runtime/backtrace\_nat.c}{runtime/backtrace\_nat.c:91}}

\begin{frame}{Backtraces}{Introducing \funcname{caml\_stash\_backtrace}}
  \begin{columns}[c]
    \begin{column}{0.5\textwidth}
      \minipage[c][0.66\textheight][s]{\columnwidth}
      \begin{itemize}
        \item<1-> A C function provided by the runtime (specific to native code)
        \item<2-> It scans the stack, between the stack pointer at the raising point up to the next trap, in order to collect the names of the detected function frames
          \onslide*<2>{
            \begin{itemize}
              \item And more information, like line number, column number...
            \end{itemize}
          }
        \item<3-> It uses the same technique and information as the Garbage Collector (GC) uses to scan the values live on the stack
          \onslide*<4>{
            \begin{itemize}
              \item \typename{frame\_descr} are at the core of stack unwinding
            \end{itemize}
          }
      \end{itemize}
      \vfill
      \mintocaml[firstline=9,lastline=13,highlightlines=12]{../src/backtrace/backtrace.ml}
      \endminipage
    \end{column}
    \begin{column}{0.5\textwidth}
      \onslide*<1>{\listStashBacktrace[highlightlines=91]{}}
      \onslide*<2>{\listStashBacktrace[highlightlines={91,117-118}]{}}
      \onslide*<3->{\listStashBacktrace[highlightlines={94,106,109-110}]{}}
    \end{column}
  \end{columns}
\end{frame}

\newcommand\listFrameDescr[1][]{
  \listocaml[firstline=111,lastline=124,#1]{../ocaml/asmcomp/emitaux.ml}{asmcomp/emitaux.ml:111}}
\newcommand\listDebugInfo[1][]{
  \listocaml[firstline=38,lastline=49,#1]{../ocaml/lambda/debuginfo.mli}{lambda/debuginfo.mli:38}}

\begin{frame}{Backtraces}{\typename{frame\_descr} and \typename{frame\_debuginfo} in a nutshell}
  \begin{columns}[c]
    \begin{column}{0.5\textwidth}
      \begin{itemize}
        \item<1-> For every return address in an OCaml program, there is a associated \typename{frame\_descr}
        \item<2-> A \typename{frame\_descr} specifies
        \begin{itemize}
          \item<2-> The size of the function stack frame
          \item<3-> Where live values are on the stack (so that the GC can mark them as live and prevent from being swept)
        \end{itemize}
        \item<4-> When compiled with debug information, \typename{frame\_descr} will be linked to a \typename{frame\_debuginfo}
        \begin{itemize}
          \item<5-> Contains information about the position in the source code (file name, line and column number)
        \end{itemize}
      \end{itemize}
    \end{column}
    \begin{column}{0.5\textwidth}
        \onslide*<1>{\listFrameDescr[highlightlines={118-122}]{}}
        \onslide*<2>{\listFrameDescr[highlightlines={120}]{}}
        \onslide*<3>{\listFrameDescr[highlightlines={121}]{}}
        \onslide*<4->{\listFrameDescr[highlightlines={122}]{}}
        \medskip
        \onslide*<1-3>{\listDebugInfo{}}
        \onslide*<4>{\listDebugInfo[highlightlines={38,49}]{}}
        \onslide*<5>{\listDebugInfo[highlightlines={39-46}]{}}
    \end{column}
  \end{columns}
\end{frame}

\begin{frame}{Backtraces}{Scanning the stack with \typename{frame\_descr}}
  \begin{columns}[c]
    \begin{column}{0.5\textwidth}
      \begin{itemize}
        \item Given the return address of the code that raises the exception by calling \funcname{caml\_raise\_exn} (\funcarg{pc} argument of \funcname{caml\_stash\_backtrace})
        \item And the position of the stack pointer (\funcarg{sp} argument of \funcname{caml\_stash\_backtrace})
        \item The frame size given by \typename{frame\_descr}.\funcarg{frame\_size}, allows to find the end of the previous frame
        \item And the return address of the caller of this function
        \bigskip
        \onslide<3->{
        \item Note that traps (here the reddish \funcname{ExnA}, \funcname{ExnB} and \funcname{ExnC} blocks) belong to the frame that installed them, and so the frame size changes when traps are removed
        }
      \end{itemize}
    \end{column}
    \begin{column}{0.5\textwidth}
      \raggedleft
      \onslide*<1>{\providecommand\step{2}\input{diagrams/backtrace/backtrace.tex}}%
      \onslide*<2>{\providecommand\step{1}\input{diagrams/backtrace/scanstack.tex}}%
      \onslide*<3>{\providecommand\step{2}\input{diagrams/backtrace/scanstack.tex}}%
      \onslide*<4>{\providecommand\step{3}\input{diagrams/backtrace/scanstack.tex}}%
      \onslide*<5>{\providecommand\step{4}\input{diagrams/backtrace/scanstack.tex}}%
      \onslide*<6>{\providecommand\step{5}\input{diagrams/backtrace/scanstack.tex}}%
    \end{column}
  \end{columns}
\end{frame}

% Duplicate of previous-previous slide
\begin{frame}{Backtraces}{Continuing on \funcname{caml\_stash\_backtrace}}
  \begin{columns}[c]
    \begin{column}{0.5\textwidth}
      \minipage[c][0.75\textheight][s]{\columnwidth}
      \begin{itemize}
          \color{gray}
        \item A C function provided by the runtime (specific to native code)
        \item It scans the stack, between the stack pointer at the raising point up to the next trap, in order to collect the names of the detected function frames
        \item It uses the same technique and information as the Garbage Collector (GC) uses to scan the values live on the stack
      \end{itemize}
      \begin{itemize}
        \item For each function frame detected, store a pointer to the \typename{frame\_descr} in the \localname{domain\_state->backtrace\_buffer} array, effectively constructing the backtrace
      \end{itemize}
      \onslide<2->{
        \begin{itemize}
            \smallskip
          \item Constructed backtrace:
            \funcname{definitely\_raise},
            \onslide<3->{\funcname{probably\_raise}}
        \end{itemize}
      }
      \vfill
      \mintocaml[firstline=9,lastline=13,highlightlines=12]{../src/backtrace/backtrace.ml}
      \endminipage
    \end{column}
    \begin{column}{0.5\textwidth}
      \raggedleft
      \onslide*<1>{\listStashBacktrace[highlightlines={114-115}]{}}
      \onslide*<2>{\providecommand\step{3}\input{diagrams/backtrace/backtrace.tex}}%
      \onslide*<3>{\providecommand\step{4}\input{diagrams/backtrace/backtrace.tex}}%
    \end{column}
  \end{columns}
\end{frame}

\begin{frame}{Backtraces}{Back to \funcname{caml\_raise\_exn}}
  \begin{columns}[c]
    \begin{column}{0.5\textwidth}
      \minipage[c][0.66\textheight][s]{\columnwidth}
      \begin{itemize}\color{gray}
        \item Checks wheteher exceptions backtrace should be recorded, by checking the \funcarg{domain\_state->backtrace\_active} value
        \item Switches to the C stack to be able to execute C code
        \item Calls \funcname{caml\_stash\_backtrace}, to collect the backtrace up to the next trap
      \end{itemize}
      \onslide*<2->{
        \begin{itemize}
          \item<2-> Executes the exception handler in \funcname{probably\_raise} with \funcname{RESTORE\_EXN\_HANDLER\_OCAML}
            \begin{itemize}
              \item Set \regname{RSP} to \localname{domain\_state->exn\_handler} value
              \item Restore \localname{domain\_state->exn\_handler} from trap
              \item Jump to the handler code with \funcname{ret}
            \end{itemize}
        \end{itemize}
      }
      \vfill
      \onslide*<1>{\mintocaml[firstline=9,lastline=13,highlightlines=12]{../src/backtrace/backtrace.ml}}
      \onslide*<2->{\mintocaml[firstline=15,lastline=21,highlightlines={19-21}]{../src/backtrace/backtrace.ml}}
      \endminipage
    \end{column}
    \begin{column}{0.5\textwidth}
      \centering
      \onslide*<1>{
        \mintc[firstline=190,lastline=192]{../ocaml/runtime/amd64.S}
        \mintc[firstline=234,lastline=237]{../ocaml/runtime/amd64.S}
      }
      \onslide*<2>{
        \mintc[firstline=190,lastline=192,highlightlines={191-192}]{../ocaml/runtime/amd64.S}
        \mintc[firstline=234,lastline=237,highlightlines={235-237}]{../ocaml/runtime/amd64.S}
      }
      \onslide*<1>{\listCamlRaiseExn[highlightlines=720]{}}
      \onslide*<2>{\listCamlRaiseExn[highlightlines=722]{}}
      \onslide*<3->{\providecommand\step{5}\input{diagrams/backtrace/backtrace.tex}}
    \end{column}
  \end{columns}
\end{frame}

\begin{frame}{Backtraces}{\funcname{caml\_probably\_raise}'s handler doesn't catch \typename{ExnC}}
  \begin{columns}[c]
    \begin{column}{0.45\textwidth}
      \minipage[c][0.75\textheight][s]{\columnwidth}
      \begin{itemize}
        \item<1-> \funcname{match \dots with}\footnotemark pattern matching can also match exceptions
        \item<1-> The CMM of \funcname{probably\_raise} shows that this \funcname{match \dots with} is implemented with a pattern matching and a \funcname{try \dots with}
        \item<1-> But this one only matches on \typename{ExnA} exception
        \item<2-> Since the pattern matching fails to catch the exception, \funcname{reraise} is called with our current \typename{ExnC} exception
        \item<3-> Disassembly shows that \funcname{reraise} is actually a call to \funcname{caml\_reraise\_exn}, which is also an assembly function provided by the runtime
      \end{itemize}
      \vfill
      \mintocaml[firstline=15,lastline=21,highlightlines={18-21}]{../src/backtrace/backtrace.ml}
      \endminipage
    \end{column}
    \begin{column}{0.55\textwidth}
      \centering
      \onslide*<1>{\mintcmm[highlightlines={10-18}]{../src/backtrace/camlBacktrace\_\_probably\_raise\_351.cmm}}
      \onslide*<2>{\listcmm[highlightlines=18]{../src/backtrace/camlBacktrace\_\_probably\_raise_351.cmm}{CMM of probably\_raise}}
      \onslide*<3>{\mintobjdump[firstline=5,lastline=35,highlightlines=31]{../src/backtrace/camlBacktrace\_\_probably\_raise_351.objdump}}
    \end{column}
  \end{columns}
  \only<1>{\footnotetext[1]{https://blog.janestreet.com/pattern-matching-and-exception-handling-unite/}}
\end{frame}

\newcommand\listCamlReraiseExn[1][]{
  \listasm[firstline=727,lastline=734,#1]{../ocaml/runtime/amd64.S}{runtime/amd64.S:727}}

\begin{frame}{Backtraces}{Introducing \funcname{caml\_reraise\_exn}}
  \begin{columns}[c]
    \begin{column}{0.5\textwidth}
      \minipage[c][0.66\textheight][s]{\columnwidth}
      \begin{itemize}
        \item<1-> The exception have to be reraised to the next handler
        \item<2-> First, checks if the backtrace should be collected with \funcarg{domain\_state->backtrace\_active}
        \only<3>{
        \begin{itemize}
            \item If not, behaves like a \funcname{raise\_notrace} with \funcname{RESTORE\_EXN\_HANDLER\_OCAML}
        \end{itemize}
        }
        \item<4-> Jumps into \funcname{caml\_raise\_exn}
        \item<5-> Calls \funcname{caml\_stash\_backtrace} again, to scan the next part of the stack: from the trap just pop-ed to the next trap
      \end{itemize}
      \vfill
      \mintocaml[firstline=15,lastline=21,highlightlines={18-21}]{../src/backtrace/backtrace.ml}
      \endminipage
    \end{column}
    \begin{column}{0.5\textwidth}
      \raggedleft
      \onslide*<1>{\listCamlReraiseExn{}}
      \onslide*<2>{\listCamlReraiseExn[highlightlines=729]{}}
      \onslide*<3>{
        \mintc[firstline=190,lastline=192,highlightlines={191-192}]{../ocaml/runtime/amd64.S}
        \mintc[firstline=234,lastline=237,highlightlines={235-237}]{../ocaml/runtime/amd64.S}
        \medskip
        \listCamlReraiseExn[highlightlines=731]{}
      }
      \onslide*<4-5>{
        % TODO refactor with \listCamlReraiseExn
        \mintasm[firstline=727,lastline=734,highlightlines=730]{../ocaml/runtime/amd64.S}
        \medskip
      }
      \onslide*<4>{\listCamlRaiseExn[highlightlines=711]{}}
      \onslide*<5>{\listCamlRaiseExn[highlightlines=720]{}}
    \end{column}
  \end{columns}
\end{frame}

\begin{frame}{Backtraces}{Backtrace collection continues}
  \begin{columns}[c]
    \begin{column}{0.55\textwidth}
      \minipage[c][0.75\textheight][s]{\columnwidth}
      \begin{itemize}
        \item<1-> \funcname{caml\_stash\_backtrace} scans the older part of the stack, up to the next trap
        \item<6-> Controls jumps to the nested exception handler in \funcname{maybe\_raise}, which matches only on \typename{ExnB}
        \item<7-> \funcname{caml\_reraise\_exn} is called again, backtrace collection resumes with \funcname{caml\_stash\_backtrace}
      \end{itemize}
      \medskip
      \begin{itemize}
        \item<2-> Constructed backtrace:
        \begin{itemize}
          \item <2-> \funcname{definitely\_raise}
          \item <2-> \funcname{probably\_raise}
          \item <3-> \funcname{probably\_raise} (re-raise)
          \item <4-> \funcname{maybe\_raise}
          \item <5-> \funcname{main}
          \item <8-> \funcname{main} (re-raise)
        \end{itemize}
      \end{itemize}
      \vfill
      \onslide*<1-5>{\mintocaml[firstline=15,lastline=21]{../src/backtrace/backtrace.ml}}
      \onslide*<6>{\mintocaml[firstline=28,lastline=34,highlightlines=31]{../src/backtrace/backtrace.ml}}
      \onslide*<7->{\mintocaml[firstline=28,lastline=34,highlightlines=33]{../src/backtrace/backtrace.ml}}
      \endminipage
    \end{column}
    \begin{column}{0.45\textwidth}
      \raggedleft
      \onslide*<1>{\providecommand\step{6}\input{diagrams/backtrace/backtrace.tex}}%
      \onslide*<2>{\providecommand\step{6}\input{diagrams/backtrace/backtrace.tex}}%
      \onslide*<3>{\providecommand\step{7}\input{diagrams/backtrace/backtrace.tex}}%
      \onslide*<4>{\providecommand\step{8}\input{diagrams/backtrace/backtrace.tex}}%
      \onslide*<5>{\providecommand\step{9}\input{diagrams/backtrace/backtrace.tex}}%
      \onslide*<6>{\providecommand\step{10}\input{diagrams/backtrace/backtrace.tex}}%
      \onslide*<7>{\providecommand\step{11}\input{diagrams/backtrace/backtrace.tex}}%
      \onslide*<8>{\providecommand\step{12}\input{diagrams/backtrace/backtrace.tex}}%
      \onslide*<9>{\providecommand\step{13}\input{diagrams/backtrace/backtrace.tex}}%
    \end{column}
  \end{columns}
\end{frame}

\begin{frame}{Backtraces}{Printing the backtrace}
  \begin{columns}[c]
    \begin{column}{0.45\textwidth}
      \minipage[c][0.66\textheight][s]{\columnwidth}
      \begin{itemize}
        \item<1-> The exception backtrace can now be printed to \funcarg{stdout} with \funcname{Printexc.print_backtrace}
        \item<2-> First the exception backtrace is retrieved into an OCaml array with \funcname{get_raw_backtrace }
        \item<3-> Which is implemented as the \funcname{caml_get_exception_raw_backtrace} C function in the runtime
        \item<4-> And finally printed with \funcname{print_exception_backtrace}
      \end{itemize}
      \vfill
      \mintocaml[firstline=28,lastline=34,highlightlines=33]{../src/backtrace/backtrace.ml}
      \endminipage
    \end{column}
    \begin{column}{0.55\textwidth}
      \onslide*<1,2>{
        \mintocaml[firstline=100,lastline=101]{../ocaml/stdlib/printexc.ml}
      }
      \only<1>{
        \listocaml[firstline=170,lastline=172]{../ocaml/stdlib/printexc.ml}{stdlib/printexc.ml:170}
      }
      \only<2>{
        \listocaml[firstline=170,lastline=172,highlightlines=172]{../ocaml/stdlib/printexc.ml}{stdlib/printexc.ml:170}
      }
      \only<3>{
        \mintc[firstline=154,lastline=156]{../ocaml/runtime/backtrace.c}
        \listc[firstline=171,lastline=192,highlightlines={184-187}]{../ocaml/runtime/backtrace.c}{runtime/backtrace.c:154}
      }
      \only<4>{
        \listocaml[firstline=155,lastline=165]{../ocaml/stdlib/printexc.ml}{stdlib/printexc.ml:55}
      }
    \end{column}
  \end{columns}
\end{frame}


\subsection{The multiple kinds of raise}
\frameSubsection{}{}

\begin{frame}{Backtraces}{When backtraces are not needed}
  \begin{columns}[c]
    \begin{column}{0.45\textwidth}
      \minipage[c][0.66\textheight][s]{\columnwidth}
      \begin{itemize}
        \item<1-> What if the backtrace for an exception is never needed?
        \item<2-> It's possible to raise the exception explicitly with \funcname{raise\_notrace} to make raising as cheap as possible
        \item<3-> An example of use in the \funcname{dynlink} library of the OCaml compiler
      \end{itemize}
      \vfill
      \onslide<3->{
      \listocaml[firstline=205,lastline=209,highlightlines=207]{../ocaml/otherlibs/dynlink/dynlink\_compilerlibs/misc.ml}{otherlibs/dynlink/dynlink\_compilerlibs/misc.ml:205}
      }
      \endminipage
    \end{column}
    \begin{column}{0.55\textwidth}
    \only<4>{
      \mintcmm[highlightlines=3]{../src/notrace/camlNotrace\_\_fun_347.cmm}
      \listcmm{../src/notrace/camlNotrace\_\_all\_somes\_327.cmm}{all\_somes}
    }
    \onslide*<5->{
      \listobjdump[highlightlines={6-9}]{../src/notrace/camlNotrace\_\_fun_347.objdump}{anonymous function from \funcname{all\_somes}}
    }
    \end{column}
  \end{columns}
\end{frame}

\begin{frame}{Backtraces}{Summary table of the multiple kinds of raise}
  \begin{itemize}
    \item When debugging information is enabled and if the backtrace of an exception isn't needed, it's possible to raise with \funcname{raise\_notrace} for performance
    \item When debugging information is \emph{not} enabled, the compiler optimises \funcname{raise} calls to inline assembly (same effect as using \funcname{raise\_notrace})
  \end{itemize}
  \bigskip
  \centering
  \begin{tabular}{| l | c | c | c | c |}
    \hline
    \parbox{10em}{Debug information \\ (\funcarg{-g} flag)}  & \multicolumn{2}{|c|}{Disabled}                                     & \multicolumn{2}{|c|}{Enabled} \\
    \hline
    Raise kind                                               & \funcname{raise} & \funcname{raise\_notrace}                       & \funcname{raise} & \funcname{raise\_notrace} \\
    \hline
    Compilation                                              & \parbox{8em}{inline assembly\\ (optimization)} & inline assembly   & \funcname{caml\_raise\_exn} & inline assembly \\
    \hline
  \end{tabular}
\end{frame}


%
% Takeaway #5
%
\subsection*{Takeaway \#5}
\frameSubsectionTakeaway{}
\againframe<5>{takeaway}
}%
      \onslide*<8>{\providecommand\step{12}%
% Backtraces
%
\subsection{One advantage of raising with debugging information}
\frameSubsection{}{}

\begin{frame}{Backtraces}{Debugging information \& source code}
  \begin{columns}[c]
    \begin{column}{0.5\textwidth}
      \begin{itemize}
        \item Raising an exception is fun and games, but how do we know where the exception originated? $\rightarrow$ With the backtrace associated with the exception!
        \item In order to get a backtrace from the point where an exception was raised, it's necessary to compile with debugging information enabled (\funcarg{-g} flag for the compiler)
        \item Same example as in the `Nested handlers' section
      \end{itemize}
    \end{column}
    \begin{column}{0.5\textwidth}
      \listocaml[firstline=9,lastline=34]{../src/backtrace/backtrace.ml}{backtrace/backtrace.ml}
    \end{column}
  \end{columns}
\end{frame}

\begin{frame}{Backtraces}{CMM and disassembly of \funcname{definitely\_raise}}
  \begin{columns}[c]
    \begin{column}{0.5\textwidth}
      \minipage[c][0.66\textheight][s]{\columnwidth}
      \begin{itemize}
        \item<1-> An effect of compiling with debugging information (\funcarg{-g}): the CMM no longer uses \funcname{raise\_notrace} to raise the exception but \funcname{raise} instead
        \item<2-> Disassembly shows that CMM \funcname{raise} is actually a call to \funcname{caml\_raise\_exn}, a function in assembler provided by the runtime
      \end{itemize}
      \vfill
      \mintocaml[firstline=9,lastline=13,highlightlines=12]{../src/backtrace/backtrace.ml}
      \endminipage
    \end{column}
    \begin{column}{0.5\textwidth}
      \onslide*<1>{
        \mintcmm[highlightlines=5]{../src/backtrace/camlBacktrace\_\_definitely\_raise\_347.cmm}
      }
      \onslide*<2>{
        \mintobjdump[firstline=2,highlightlines=14]{../src/backtrace/camlBacktrace\_\_definitely\_raise\_347.objdump}
      }
    \end{column}
  \end{columns}
\end{frame}

\begin{frame}{Backtraces}{Introducing \funcname{caml\_raise\_exn}}
  \begin{columns}[c]
    \begin{column}{0.5\textwidth}
      \minipage[c][0.66\textheight][s]{\columnwidth}
      \onslide*<1>{
        \begin{itemize}
          \item Raising the exception is performed using \funcname{caml\_raise\_exn} which is
            \begin{itemize}
              \item An assembly function provided by the runtime
              \item Architecture specific
            \end{itemize}
          \item Let's see what it does differently from \funcname{raise\_notrace}\dots
          \item We are about to raise \typename{ExnC} with \funcname{caml\_raise\_exn} from \funcname{definitely\_raise}
        \end{itemize}
      }
      \onslide*<2>{
        \mintobjdump[firstline=2,highlightlines=14]{../src/backtrace/camlBacktrace\_\_definitely\_raise\_347.objdump}
      }
      \vfill
      \mintocaml[firstline=9,lastline=13,highlightlines=12]{../src/backtrace/backtrace.ml}
      \endminipage
    \end{column}
    \begin{column}{0.5\textwidth}
      \centering
      \providecommand\step{1}\input{diagrams/backtrace/backtrace.tex}
    \end{column}
  \end{columns}
\end{frame}

\begin{frame}{Backtraces}{But before that, we need more fields from \typename{caml\_domain\_state}}
  \begin{columns}
    \begin{column}{0.5\textwidth}
      \begin{itemize}
        \item Remember \typename{caml\_domain\_state} the per-domain runtime state?
        \item Some other members are of interest to us now\dots
        \item \localname{backtrace\_active} is used to know if the runtime should collect backtraces
        \item \localname{backtrace\_active} can be set by
          \begin{itemize}
            \item The \funcarg{b} flag in the \funcname{OCAMLRUNPARAM} environment variable
            \item \mintinline{ocaml}{Printexc.record_backtrace : bool -> unit} \footnotemark
          \end{itemize}
      \end{itemize}
    \end{column}
    \begin{column}{0.5\textwidth}
      \begin{listing}
        \mintc[highlightlines={9-11}]{src/ocaml/domain\_state\_backtrace.h}
        \caption{runtime/caml/domain\_state.h}
      \end{listing}
    \end{column}
  \end{columns}
  \footnotetext[1]{https://v2.ocaml.org/api/Printexc.html}
\end{frame}

\newcommand\listCamlRaiseExn[1][]{
  \listasm[firstline=702,lastline=725,#1]{../ocaml/runtime/amd64.S}{runtime/amd64.S:702}}

\begin{frame}{Backtraces}{Walk through \funcname{caml\_raise\_exn}}
  \begin{columns}[T]
    \begin{column}{0.5\textwidth}
      \begin{itemize}
        \item<1-> First checks whether exceptions backtrace should be recorded
        \item<1-> By checking the \funcarg{domain\_state->backtrace\_active} value
        \item<2-> If not, acts as \funcname{raise\_notrace} with \funcname{RESTORE\_EXN\_HANDLER\_OCAML}
          \begin{itemize}
            \item Set \regname{RSP} to \localname{domain\_state->exn\_handler} value
            \item Restore \localname{domain\_state->exn\_handler} from trap
            \item Jump with \funcname{ret} (same effect as using \funcname{pop} and \funcname{jmp})
          \end{itemize}
        \item<3-> Otherwise, switches to the C stack to be able to execute C code
        \item<4-> Calls \funcname{caml\_stash\_backtrace}, a C function provided by the runtime\ignorespaces
          \onslide<6->{, with
            \begin{itemize}
              \item \funcarg{exn}: exception value
              \item \funcarg{pc}: code address of the raising point
              \item \funcarg{sp}: stack address of the raising function
              \item \funcarg{trapsp}: stack address of the next handler
            \end{itemize}
          }
      \end{itemize}
      \bigskip
      \mintocaml[firstline=9,lastline=13,highlightlines=12]{../src/backtrace/backtrace.ml}
    \end{column}
    \begin{column}{0.5\textwidth}
      \raggedleft
      \onslide*<1,3-4>{
        \mintc[firstline=190,lastline=192]{../ocaml/runtime/amd64.S}
        \mintc[firstline=234,lastline=237]{../ocaml/runtime/amd64.S}
      }
      \onslide*<2>{
        \mintc[firstline=190,lastline=192,highlightlines={191-192}]{../ocaml/runtime/amd64.S}
        \mintc[firstline=234,lastline=237,highlightlines={235-237}]{../ocaml/runtime/amd64.S}
      }
      \onslide*<1>{\listCamlRaiseExn[highlightlines=705]{}}%
      \onslide*<2>{\listCamlRaiseExn[highlightlines={707-708}]{}}%
      \onslide*<3>{\listCamlRaiseExn[highlightlines={712-714}]{}}%
      \onslide*<4>{\listCamlRaiseExn[highlightlines={715-720}]{}}%
      \onslide*<5>{\providecommand\step{1}\input{diagrams/backtrace/backtrace.tex}}%
      \onslide*<6>{\providecommand\step{2}\input{diagrams/backtrace/backtrace.tex}}%
    \end{column}
  \end{columns}
\end{frame}

\newcommand\listStashBacktrace[1][]{
  \listc[firstline=91,lastline=120,#1]{../ocaml/runtime/backtrace\_nat.c}{runtime/backtrace\_nat.c:91}}

\begin{frame}{Backtraces}{Introducing \funcname{caml\_stash\_backtrace}}
  \begin{columns}[c]
    \begin{column}{0.5\textwidth}
      \minipage[c][0.66\textheight][s]{\columnwidth}
      \begin{itemize}
        \item<1-> A C function provided by the runtime (specific to native code)
        \item<2-> It scans the stack, between the stack pointer at the raising point up to the next trap, in order to collect the names of the detected function frames
          \onslide*<2>{
            \begin{itemize}
              \item And more information, like line number, column number...
            \end{itemize}
          }
        \item<3-> It uses the same technique and information as the Garbage Collector (GC) uses to scan the values live on the stack
          \onslide*<4>{
            \begin{itemize}
              \item \typename{frame\_descr} are at the core of stack unwinding
            \end{itemize}
          }
      \end{itemize}
      \vfill
      \mintocaml[firstline=9,lastline=13,highlightlines=12]{../src/backtrace/backtrace.ml}
      \endminipage
    \end{column}
    \begin{column}{0.5\textwidth}
      \onslide*<1>{\listStashBacktrace[highlightlines=91]{}}
      \onslide*<2>{\listStashBacktrace[highlightlines={91,117-118}]{}}
      \onslide*<3->{\listStashBacktrace[highlightlines={94,106,109-110}]{}}
    \end{column}
  \end{columns}
\end{frame}

\newcommand\listFrameDescr[1][]{
  \listocaml[firstline=111,lastline=124,#1]{../ocaml/asmcomp/emitaux.ml}{asmcomp/emitaux.ml:111}}
\newcommand\listDebugInfo[1][]{
  \listocaml[firstline=38,lastline=49,#1]{../ocaml/lambda/debuginfo.mli}{lambda/debuginfo.mli:38}}

\begin{frame}{Backtraces}{\typename{frame\_descr} and \typename{frame\_debuginfo} in a nutshell}
  \begin{columns}[c]
    \begin{column}{0.5\textwidth}
      \begin{itemize}
        \item<1-> For every return address in an OCaml program, there is a associated \typename{frame\_descr}
        \item<2-> A \typename{frame\_descr} specifies
        \begin{itemize}
          \item<2-> The size of the function stack frame
          \item<3-> Where live values are on the stack (so that the GC can mark them as live and prevent from being swept)
        \end{itemize}
        \item<4-> When compiled with debug information, \typename{frame\_descr} will be linked to a \typename{frame\_debuginfo}
        \begin{itemize}
          \item<5-> Contains information about the position in the source code (file name, line and column number)
        \end{itemize}
      \end{itemize}
    \end{column}
    \begin{column}{0.5\textwidth}
        \onslide*<1>{\listFrameDescr[highlightlines={118-122}]{}}
        \onslide*<2>{\listFrameDescr[highlightlines={120}]{}}
        \onslide*<3>{\listFrameDescr[highlightlines={121}]{}}
        \onslide*<4->{\listFrameDescr[highlightlines={122}]{}}
        \medskip
        \onslide*<1-3>{\listDebugInfo{}}
        \onslide*<4>{\listDebugInfo[highlightlines={38,49}]{}}
        \onslide*<5>{\listDebugInfo[highlightlines={39-46}]{}}
    \end{column}
  \end{columns}
\end{frame}

\begin{frame}{Backtraces}{Scanning the stack with \typename{frame\_descr}}
  \begin{columns}[c]
    \begin{column}{0.5\textwidth}
      \begin{itemize}
        \item Given the return address of the code that raises the exception by calling \funcname{caml\_raise\_exn} (\funcarg{pc} argument of \funcname{caml\_stash\_backtrace})
        \item And the position of the stack pointer (\funcarg{sp} argument of \funcname{caml\_stash\_backtrace})
        \item The frame size given by \typename{frame\_descr}.\funcarg{frame\_size}, allows to find the end of the previous frame
        \item And the return address of the caller of this function
        \bigskip
        \onslide<3->{
        \item Note that traps (here the reddish \funcname{ExnA}, \funcname{ExnB} and \funcname{ExnC} blocks) belong to the frame that installed them, and so the frame size changes when traps are removed
        }
      \end{itemize}
    \end{column}
    \begin{column}{0.5\textwidth}
      \raggedleft
      \onslide*<1>{\providecommand\step{2}\input{diagrams/backtrace/backtrace.tex}}%
      \onslide*<2>{\providecommand\step{1}\input{diagrams/backtrace/scanstack.tex}}%
      \onslide*<3>{\providecommand\step{2}\input{diagrams/backtrace/scanstack.tex}}%
      \onslide*<4>{\providecommand\step{3}\input{diagrams/backtrace/scanstack.tex}}%
      \onslide*<5>{\providecommand\step{4}\input{diagrams/backtrace/scanstack.tex}}%
      \onslide*<6>{\providecommand\step{5}\input{diagrams/backtrace/scanstack.tex}}%
    \end{column}
  \end{columns}
\end{frame}

% Duplicate of previous-previous slide
\begin{frame}{Backtraces}{Continuing on \funcname{caml\_stash\_backtrace}}
  \begin{columns}[c]
    \begin{column}{0.5\textwidth}
      \minipage[c][0.75\textheight][s]{\columnwidth}
      \begin{itemize}
          \color{gray}
        \item A C function provided by the runtime (specific to native code)
        \item It scans the stack, between the stack pointer at the raising point up to the next trap, in order to collect the names of the detected function frames
        \item It uses the same technique and information as the Garbage Collector (GC) uses to scan the values live on the stack
      \end{itemize}
      \begin{itemize}
        \item For each function frame detected, store a pointer to the \typename{frame\_descr} in the \localname{domain\_state->backtrace\_buffer} array, effectively constructing the backtrace
      \end{itemize}
      \onslide<2->{
        \begin{itemize}
            \smallskip
          \item Constructed backtrace:
            \funcname{definitely\_raise},
            \onslide<3->{\funcname{probably\_raise}}
        \end{itemize}
      }
      \vfill
      \mintocaml[firstline=9,lastline=13,highlightlines=12]{../src/backtrace/backtrace.ml}
      \endminipage
    \end{column}
    \begin{column}{0.5\textwidth}
      \raggedleft
      \onslide*<1>{\listStashBacktrace[highlightlines={114-115}]{}}
      \onslide*<2>{\providecommand\step{3}\input{diagrams/backtrace/backtrace.tex}}%
      \onslide*<3>{\providecommand\step{4}\input{diagrams/backtrace/backtrace.tex}}%
    \end{column}
  \end{columns}
\end{frame}

\begin{frame}{Backtraces}{Back to \funcname{caml\_raise\_exn}}
  \begin{columns}[c]
    \begin{column}{0.5\textwidth}
      \minipage[c][0.66\textheight][s]{\columnwidth}
      \begin{itemize}\color{gray}
        \item Checks wheteher exceptions backtrace should be recorded, by checking the \funcarg{domain\_state->backtrace\_active} value
        \item Switches to the C stack to be able to execute C code
        \item Calls \funcname{caml\_stash\_backtrace}, to collect the backtrace up to the next trap
      \end{itemize}
      \onslide*<2->{
        \begin{itemize}
          \item<2-> Executes the exception handler in \funcname{probably\_raise} with \funcname{RESTORE\_EXN\_HANDLER\_OCAML}
            \begin{itemize}
              \item Set \regname{RSP} to \localname{domain\_state->exn\_handler} value
              \item Restore \localname{domain\_state->exn\_handler} from trap
              \item Jump to the handler code with \funcname{ret}
            \end{itemize}
        \end{itemize}
      }
      \vfill
      \onslide*<1>{\mintocaml[firstline=9,lastline=13,highlightlines=12]{../src/backtrace/backtrace.ml}}
      \onslide*<2->{\mintocaml[firstline=15,lastline=21,highlightlines={19-21}]{../src/backtrace/backtrace.ml}}
      \endminipage
    \end{column}
    \begin{column}{0.5\textwidth}
      \centering
      \onslide*<1>{
        \mintc[firstline=190,lastline=192]{../ocaml/runtime/amd64.S}
        \mintc[firstline=234,lastline=237]{../ocaml/runtime/amd64.S}
      }
      \onslide*<2>{
        \mintc[firstline=190,lastline=192,highlightlines={191-192}]{../ocaml/runtime/amd64.S}
        \mintc[firstline=234,lastline=237,highlightlines={235-237}]{../ocaml/runtime/amd64.S}
      }
      \onslide*<1>{\listCamlRaiseExn[highlightlines=720]{}}
      \onslide*<2>{\listCamlRaiseExn[highlightlines=722]{}}
      \onslide*<3->{\providecommand\step{5}\input{diagrams/backtrace/backtrace.tex}}
    \end{column}
  \end{columns}
\end{frame}

\begin{frame}{Backtraces}{\funcname{caml\_probably\_raise}'s handler doesn't catch \typename{ExnC}}
  \begin{columns}[c]
    \begin{column}{0.45\textwidth}
      \minipage[c][0.75\textheight][s]{\columnwidth}
      \begin{itemize}
        \item<1-> \funcname{match \dots with}\footnotemark pattern matching can also match exceptions
        \item<1-> The CMM of \funcname{probably\_raise} shows that this \funcname{match \dots with} is implemented with a pattern matching and a \funcname{try \dots with}
        \item<1-> But this one only matches on \typename{ExnA} exception
        \item<2-> Since the pattern matching fails to catch the exception, \funcname{reraise} is called with our current \typename{ExnC} exception
        \item<3-> Disassembly shows that \funcname{reraise} is actually a call to \funcname{caml\_reraise\_exn}, which is also an assembly function provided by the runtime
      \end{itemize}
      \vfill
      \mintocaml[firstline=15,lastline=21,highlightlines={18-21}]{../src/backtrace/backtrace.ml}
      \endminipage
    \end{column}
    \begin{column}{0.55\textwidth}
      \centering
      \onslide*<1>{\mintcmm[highlightlines={10-18}]{../src/backtrace/camlBacktrace\_\_probably\_raise\_351.cmm}}
      \onslide*<2>{\listcmm[highlightlines=18]{../src/backtrace/camlBacktrace\_\_probably\_raise_351.cmm}{CMM of probably\_raise}}
      \onslide*<3>{\mintobjdump[firstline=5,lastline=35,highlightlines=31]{../src/backtrace/camlBacktrace\_\_probably\_raise_351.objdump}}
    \end{column}
  \end{columns}
  \only<1>{\footnotetext[1]{https://blog.janestreet.com/pattern-matching-and-exception-handling-unite/}}
\end{frame}

\newcommand\listCamlReraiseExn[1][]{
  \listasm[firstline=727,lastline=734,#1]{../ocaml/runtime/amd64.S}{runtime/amd64.S:727}}

\begin{frame}{Backtraces}{Introducing \funcname{caml\_reraise\_exn}}
  \begin{columns}[c]
    \begin{column}{0.5\textwidth}
      \minipage[c][0.66\textheight][s]{\columnwidth}
      \begin{itemize}
        \item<1-> The exception have to be reraised to the next handler
        \item<2-> First, checks if the backtrace should be collected with \funcarg{domain\_state->backtrace\_active}
        \only<3>{
        \begin{itemize}
            \item If not, behaves like a \funcname{raise\_notrace} with \funcname{RESTORE\_EXN\_HANDLER\_OCAML}
        \end{itemize}
        }
        \item<4-> Jumps into \funcname{caml\_raise\_exn}
        \item<5-> Calls \funcname{caml\_stash\_backtrace} again, to scan the next part of the stack: from the trap just pop-ed to the next trap
      \end{itemize}
      \vfill
      \mintocaml[firstline=15,lastline=21,highlightlines={18-21}]{../src/backtrace/backtrace.ml}
      \endminipage
    \end{column}
    \begin{column}{0.5\textwidth}
      \raggedleft
      \onslide*<1>{\listCamlReraiseExn{}}
      \onslide*<2>{\listCamlReraiseExn[highlightlines=729]{}}
      \onslide*<3>{
        \mintc[firstline=190,lastline=192,highlightlines={191-192}]{../ocaml/runtime/amd64.S}
        \mintc[firstline=234,lastline=237,highlightlines={235-237}]{../ocaml/runtime/amd64.S}
        \medskip
        \listCamlReraiseExn[highlightlines=731]{}
      }
      \onslide*<4-5>{
        % TODO refactor with \listCamlReraiseExn
        \mintasm[firstline=727,lastline=734,highlightlines=730]{../ocaml/runtime/amd64.S}
        \medskip
      }
      \onslide*<4>{\listCamlRaiseExn[highlightlines=711]{}}
      \onslide*<5>{\listCamlRaiseExn[highlightlines=720]{}}
    \end{column}
  \end{columns}
\end{frame}

\begin{frame}{Backtraces}{Backtrace collection continues}
  \begin{columns}[c]
    \begin{column}{0.55\textwidth}
      \minipage[c][0.75\textheight][s]{\columnwidth}
      \begin{itemize}
        \item<1-> \funcname{caml\_stash\_backtrace} scans the older part of the stack, up to the next trap
        \item<6-> Controls jumps to the nested exception handler in \funcname{maybe\_raise}, which matches only on \typename{ExnB}
        \item<7-> \funcname{caml\_reraise\_exn} is called again, backtrace collection resumes with \funcname{caml\_stash\_backtrace}
      \end{itemize}
      \medskip
      \begin{itemize}
        \item<2-> Constructed backtrace:
        \begin{itemize}
          \item <2-> \funcname{definitely\_raise}
          \item <2-> \funcname{probably\_raise}
          \item <3-> \funcname{probably\_raise} (re-raise)
          \item <4-> \funcname{maybe\_raise}
          \item <5-> \funcname{main}
          \item <8-> \funcname{main} (re-raise)
        \end{itemize}
      \end{itemize}
      \vfill
      \onslide*<1-5>{\mintocaml[firstline=15,lastline=21]{../src/backtrace/backtrace.ml}}
      \onslide*<6>{\mintocaml[firstline=28,lastline=34,highlightlines=31]{../src/backtrace/backtrace.ml}}
      \onslide*<7->{\mintocaml[firstline=28,lastline=34,highlightlines=33]{../src/backtrace/backtrace.ml}}
      \endminipage
    \end{column}
    \begin{column}{0.45\textwidth}
      \raggedleft
      \onslide*<1>{\providecommand\step{6}\input{diagrams/backtrace/backtrace.tex}}%
      \onslide*<2>{\providecommand\step{6}\input{diagrams/backtrace/backtrace.tex}}%
      \onslide*<3>{\providecommand\step{7}\input{diagrams/backtrace/backtrace.tex}}%
      \onslide*<4>{\providecommand\step{8}\input{diagrams/backtrace/backtrace.tex}}%
      \onslide*<5>{\providecommand\step{9}\input{diagrams/backtrace/backtrace.tex}}%
      \onslide*<6>{\providecommand\step{10}\input{diagrams/backtrace/backtrace.tex}}%
      \onslide*<7>{\providecommand\step{11}\input{diagrams/backtrace/backtrace.tex}}%
      \onslide*<8>{\providecommand\step{12}\input{diagrams/backtrace/backtrace.tex}}%
      \onslide*<9>{\providecommand\step{13}\input{diagrams/backtrace/backtrace.tex}}%
    \end{column}
  \end{columns}
\end{frame}

\begin{frame}{Backtraces}{Printing the backtrace}
  \begin{columns}[c]
    \begin{column}{0.45\textwidth}
      \minipage[c][0.66\textheight][s]{\columnwidth}
      \begin{itemize}
        \item<1-> The exception backtrace can now be printed to \funcarg{stdout} with \funcname{Printexc.print_backtrace}
        \item<2-> First the exception backtrace is retrieved into an OCaml array with \funcname{get_raw_backtrace }
        \item<3-> Which is implemented as the \funcname{caml_get_exception_raw_backtrace} C function in the runtime
        \item<4-> And finally printed with \funcname{print_exception_backtrace}
      \end{itemize}
      \vfill
      \mintocaml[firstline=28,lastline=34,highlightlines=33]{../src/backtrace/backtrace.ml}
      \endminipage
    \end{column}
    \begin{column}{0.55\textwidth}
      \onslide*<1,2>{
        \mintocaml[firstline=100,lastline=101]{../ocaml/stdlib/printexc.ml}
      }
      \only<1>{
        \listocaml[firstline=170,lastline=172]{../ocaml/stdlib/printexc.ml}{stdlib/printexc.ml:170}
      }
      \only<2>{
        \listocaml[firstline=170,lastline=172,highlightlines=172]{../ocaml/stdlib/printexc.ml}{stdlib/printexc.ml:170}
      }
      \only<3>{
        \mintc[firstline=154,lastline=156]{../ocaml/runtime/backtrace.c}
        \listc[firstline=171,lastline=192,highlightlines={184-187}]{../ocaml/runtime/backtrace.c}{runtime/backtrace.c:154}
      }
      \only<4>{
        \listocaml[firstline=155,lastline=165]{../ocaml/stdlib/printexc.ml}{stdlib/printexc.ml:55}
      }
    \end{column}
  \end{columns}
\end{frame}


\subsection{The multiple kinds of raise}
\frameSubsection{}{}

\begin{frame}{Backtraces}{When backtraces are not needed}
  \begin{columns}[c]
    \begin{column}{0.45\textwidth}
      \minipage[c][0.66\textheight][s]{\columnwidth}
      \begin{itemize}
        \item<1-> What if the backtrace for an exception is never needed?
        \item<2-> It's possible to raise the exception explicitly with \funcname{raise\_notrace} to make raising as cheap as possible
        \item<3-> An example of use in the \funcname{dynlink} library of the OCaml compiler
      \end{itemize}
      \vfill
      \onslide<3->{
      \listocaml[firstline=205,lastline=209,highlightlines=207]{../ocaml/otherlibs/dynlink/dynlink\_compilerlibs/misc.ml}{otherlibs/dynlink/dynlink\_compilerlibs/misc.ml:205}
      }
      \endminipage
    \end{column}
    \begin{column}{0.55\textwidth}
    \only<4>{
      \mintcmm[highlightlines=3]{../src/notrace/camlNotrace\_\_fun_347.cmm}
      \listcmm{../src/notrace/camlNotrace\_\_all\_somes\_327.cmm}{all\_somes}
    }
    \onslide*<5->{
      \listobjdump[highlightlines={6-9}]{../src/notrace/camlNotrace\_\_fun_347.objdump}{anonymous function from \funcname{all\_somes}}
    }
    \end{column}
  \end{columns}
\end{frame}

\begin{frame}{Backtraces}{Summary table of the multiple kinds of raise}
  \begin{itemize}
    \item When debugging information is enabled and if the backtrace of an exception isn't needed, it's possible to raise with \funcname{raise\_notrace} for performance
    \item When debugging information is \emph{not} enabled, the compiler optimises \funcname{raise} calls to inline assembly (same effect as using \funcname{raise\_notrace})
  \end{itemize}
  \bigskip
  \centering
  \begin{tabular}{| l | c | c | c | c |}
    \hline
    \parbox{10em}{Debug information \\ (\funcarg{-g} flag)}  & \multicolumn{2}{|c|}{Disabled}                                     & \multicolumn{2}{|c|}{Enabled} \\
    \hline
    Raise kind                                               & \funcname{raise} & \funcname{raise\_notrace}                       & \funcname{raise} & \funcname{raise\_notrace} \\
    \hline
    Compilation                                              & \parbox{8em}{inline assembly\\ (optimization)} & inline assembly   & \funcname{caml\_raise\_exn} & inline assembly \\
    \hline
  \end{tabular}
\end{frame}


%
% Takeaway #5
%
\subsection*{Takeaway \#5}
\frameSubsectionTakeaway{}
\againframe<5>{takeaway}
}%
      \onslide*<9>{\providecommand\step{13}%
% Backtraces
%
\subsection{One advantage of raising with debugging information}
\frameSubsection{}{}

\begin{frame}{Backtraces}{Debugging information \& source code}
  \begin{columns}[c]
    \begin{column}{0.5\textwidth}
      \begin{itemize}
        \item Raising an exception is fun and games, but how do we know where the exception originated? $\rightarrow$ With the backtrace associated with the exception!
        \item In order to get a backtrace from the point where an exception was raised, it's necessary to compile with debugging information enabled (\funcarg{-g} flag for the compiler)
        \item Same example as in the `Nested handlers' section
      \end{itemize}
    \end{column}
    \begin{column}{0.5\textwidth}
      \listocaml[firstline=9,lastline=34]{../src/backtrace/backtrace.ml}{backtrace/backtrace.ml}
    \end{column}
  \end{columns}
\end{frame}

\begin{frame}{Backtraces}{CMM and disassembly of \funcname{definitely\_raise}}
  \begin{columns}[c]
    \begin{column}{0.5\textwidth}
      \minipage[c][0.66\textheight][s]{\columnwidth}
      \begin{itemize}
        \item<1-> An effect of compiling with debugging information (\funcarg{-g}): the CMM no longer uses \funcname{raise\_notrace} to raise the exception but \funcname{raise} instead
        \item<2-> Disassembly shows that CMM \funcname{raise} is actually a call to \funcname{caml\_raise\_exn}, a function in assembler provided by the runtime
      \end{itemize}
      \vfill
      \mintocaml[firstline=9,lastline=13,highlightlines=12]{../src/backtrace/backtrace.ml}
      \endminipage
    \end{column}
    \begin{column}{0.5\textwidth}
      \onslide*<1>{
        \mintcmm[highlightlines=5]{../src/backtrace/camlBacktrace\_\_definitely\_raise\_347.cmm}
      }
      \onslide*<2>{
        \mintobjdump[firstline=2,highlightlines=14]{../src/backtrace/camlBacktrace\_\_definitely\_raise\_347.objdump}
      }
    \end{column}
  \end{columns}
\end{frame}

\begin{frame}{Backtraces}{Introducing \funcname{caml\_raise\_exn}}
  \begin{columns}[c]
    \begin{column}{0.5\textwidth}
      \minipage[c][0.66\textheight][s]{\columnwidth}
      \onslide*<1>{
        \begin{itemize}
          \item Raising the exception is performed using \funcname{caml\_raise\_exn} which is
            \begin{itemize}
              \item An assembly function provided by the runtime
              \item Architecture specific
            \end{itemize}
          \item Let's see what it does differently from \funcname{raise\_notrace}\dots
          \item We are about to raise \typename{ExnC} with \funcname{caml\_raise\_exn} from \funcname{definitely\_raise}
        \end{itemize}
      }
      \onslide*<2>{
        \mintobjdump[firstline=2,highlightlines=14]{../src/backtrace/camlBacktrace\_\_definitely\_raise\_347.objdump}
      }
      \vfill
      \mintocaml[firstline=9,lastline=13,highlightlines=12]{../src/backtrace/backtrace.ml}
      \endminipage
    \end{column}
    \begin{column}{0.5\textwidth}
      \centering
      \providecommand\step{1}\input{diagrams/backtrace/backtrace.tex}
    \end{column}
  \end{columns}
\end{frame}

\begin{frame}{Backtraces}{But before that, we need more fields from \typename{caml\_domain\_state}}
  \begin{columns}
    \begin{column}{0.5\textwidth}
      \begin{itemize}
        \item Remember \typename{caml\_domain\_state} the per-domain runtime state?
        \item Some other members are of interest to us now\dots
        \item \localname{backtrace\_active} is used to know if the runtime should collect backtraces
        \item \localname{backtrace\_active} can be set by
          \begin{itemize}
            \item The \funcarg{b} flag in the \funcname{OCAMLRUNPARAM} environment variable
            \item \mintinline{ocaml}{Printexc.record_backtrace : bool -> unit} \footnotemark
          \end{itemize}
      \end{itemize}
    \end{column}
    \begin{column}{0.5\textwidth}
      \begin{listing}
        \mintc[highlightlines={9-11}]{src/ocaml/domain\_state\_backtrace.h}
        \caption{runtime/caml/domain\_state.h}
      \end{listing}
    \end{column}
  \end{columns}
  \footnotetext[1]{https://v2.ocaml.org/api/Printexc.html}
\end{frame}

\newcommand\listCamlRaiseExn[1][]{
  \listasm[firstline=702,lastline=725,#1]{../ocaml/runtime/amd64.S}{runtime/amd64.S:702}}

\begin{frame}{Backtraces}{Walk through \funcname{caml\_raise\_exn}}
  \begin{columns}[T]
    \begin{column}{0.5\textwidth}
      \begin{itemize}
        \item<1-> First checks whether exceptions backtrace should be recorded
        \item<1-> By checking the \funcarg{domain\_state->backtrace\_active} value
        \item<2-> If not, acts as \funcname{raise\_notrace} with \funcname{RESTORE\_EXN\_HANDLER\_OCAML}
          \begin{itemize}
            \item Set \regname{RSP} to \localname{domain\_state->exn\_handler} value
            \item Restore \localname{domain\_state->exn\_handler} from trap
            \item Jump with \funcname{ret} (same effect as using \funcname{pop} and \funcname{jmp})
          \end{itemize}
        \item<3-> Otherwise, switches to the C stack to be able to execute C code
        \item<4-> Calls \funcname{caml\_stash\_backtrace}, a C function provided by the runtime\ignorespaces
          \onslide<6->{, with
            \begin{itemize}
              \item \funcarg{exn}: exception value
              \item \funcarg{pc}: code address of the raising point
              \item \funcarg{sp}: stack address of the raising function
              \item \funcarg{trapsp}: stack address of the next handler
            \end{itemize}
          }
      \end{itemize}
      \bigskip
      \mintocaml[firstline=9,lastline=13,highlightlines=12]{../src/backtrace/backtrace.ml}
    \end{column}
    \begin{column}{0.5\textwidth}
      \raggedleft
      \onslide*<1,3-4>{
        \mintc[firstline=190,lastline=192]{../ocaml/runtime/amd64.S}
        \mintc[firstline=234,lastline=237]{../ocaml/runtime/amd64.S}
      }
      \onslide*<2>{
        \mintc[firstline=190,lastline=192,highlightlines={191-192}]{../ocaml/runtime/amd64.S}
        \mintc[firstline=234,lastline=237,highlightlines={235-237}]{../ocaml/runtime/amd64.S}
      }
      \onslide*<1>{\listCamlRaiseExn[highlightlines=705]{}}%
      \onslide*<2>{\listCamlRaiseExn[highlightlines={707-708}]{}}%
      \onslide*<3>{\listCamlRaiseExn[highlightlines={712-714}]{}}%
      \onslide*<4>{\listCamlRaiseExn[highlightlines={715-720}]{}}%
      \onslide*<5>{\providecommand\step{1}\input{diagrams/backtrace/backtrace.tex}}%
      \onslide*<6>{\providecommand\step{2}\input{diagrams/backtrace/backtrace.tex}}%
    \end{column}
  \end{columns}
\end{frame}

\newcommand\listStashBacktrace[1][]{
  \listc[firstline=91,lastline=120,#1]{../ocaml/runtime/backtrace\_nat.c}{runtime/backtrace\_nat.c:91}}

\begin{frame}{Backtraces}{Introducing \funcname{caml\_stash\_backtrace}}
  \begin{columns}[c]
    \begin{column}{0.5\textwidth}
      \minipage[c][0.66\textheight][s]{\columnwidth}
      \begin{itemize}
        \item<1-> A C function provided by the runtime (specific to native code)
        \item<2-> It scans the stack, between the stack pointer at the raising point up to the next trap, in order to collect the names of the detected function frames
          \onslide*<2>{
            \begin{itemize}
              \item And more information, like line number, column number...
            \end{itemize}
          }
        \item<3-> It uses the same technique and information as the Garbage Collector (GC) uses to scan the values live on the stack
          \onslide*<4>{
            \begin{itemize}
              \item \typename{frame\_descr} are at the core of stack unwinding
            \end{itemize}
          }
      \end{itemize}
      \vfill
      \mintocaml[firstline=9,lastline=13,highlightlines=12]{../src/backtrace/backtrace.ml}
      \endminipage
    \end{column}
    \begin{column}{0.5\textwidth}
      \onslide*<1>{\listStashBacktrace[highlightlines=91]{}}
      \onslide*<2>{\listStashBacktrace[highlightlines={91,117-118}]{}}
      \onslide*<3->{\listStashBacktrace[highlightlines={94,106,109-110}]{}}
    \end{column}
  \end{columns}
\end{frame}

\newcommand\listFrameDescr[1][]{
  \listocaml[firstline=111,lastline=124,#1]{../ocaml/asmcomp/emitaux.ml}{asmcomp/emitaux.ml:111}}
\newcommand\listDebugInfo[1][]{
  \listocaml[firstline=38,lastline=49,#1]{../ocaml/lambda/debuginfo.mli}{lambda/debuginfo.mli:38}}

\begin{frame}{Backtraces}{\typename{frame\_descr} and \typename{frame\_debuginfo} in a nutshell}
  \begin{columns}[c]
    \begin{column}{0.5\textwidth}
      \begin{itemize}
        \item<1-> For every return address in an OCaml program, there is a associated \typename{frame\_descr}
        \item<2-> A \typename{frame\_descr} specifies
        \begin{itemize}
          \item<2-> The size of the function stack frame
          \item<3-> Where live values are on the stack (so that the GC can mark them as live and prevent from being swept)
        \end{itemize}
        \item<4-> When compiled with debug information, \typename{frame\_descr} will be linked to a \typename{frame\_debuginfo}
        \begin{itemize}
          \item<5-> Contains information about the position in the source code (file name, line and column number)
        \end{itemize}
      \end{itemize}
    \end{column}
    \begin{column}{0.5\textwidth}
        \onslide*<1>{\listFrameDescr[highlightlines={118-122}]{}}
        \onslide*<2>{\listFrameDescr[highlightlines={120}]{}}
        \onslide*<3>{\listFrameDescr[highlightlines={121}]{}}
        \onslide*<4->{\listFrameDescr[highlightlines={122}]{}}
        \medskip
        \onslide*<1-3>{\listDebugInfo{}}
        \onslide*<4>{\listDebugInfo[highlightlines={38,49}]{}}
        \onslide*<5>{\listDebugInfo[highlightlines={39-46}]{}}
    \end{column}
  \end{columns}
\end{frame}

\begin{frame}{Backtraces}{Scanning the stack with \typename{frame\_descr}}
  \begin{columns}[c]
    \begin{column}{0.5\textwidth}
      \begin{itemize}
        \item Given the return address of the code that raises the exception by calling \funcname{caml\_raise\_exn} (\funcarg{pc} argument of \funcname{caml\_stash\_backtrace})
        \item And the position of the stack pointer (\funcarg{sp} argument of \funcname{caml\_stash\_backtrace})
        \item The frame size given by \typename{frame\_descr}.\funcarg{frame\_size}, allows to find the end of the previous frame
        \item And the return address of the caller of this function
        \bigskip
        \onslide<3->{
        \item Note that traps (here the reddish \funcname{ExnA}, \funcname{ExnB} and \funcname{ExnC} blocks) belong to the frame that installed them, and so the frame size changes when traps are removed
        }
      \end{itemize}
    \end{column}
    \begin{column}{0.5\textwidth}
      \raggedleft
      \onslide*<1>{\providecommand\step{2}\input{diagrams/backtrace/backtrace.tex}}%
      \onslide*<2>{\providecommand\step{1}\input{diagrams/backtrace/scanstack.tex}}%
      \onslide*<3>{\providecommand\step{2}\input{diagrams/backtrace/scanstack.tex}}%
      \onslide*<4>{\providecommand\step{3}\input{diagrams/backtrace/scanstack.tex}}%
      \onslide*<5>{\providecommand\step{4}\input{diagrams/backtrace/scanstack.tex}}%
      \onslide*<6>{\providecommand\step{5}\input{diagrams/backtrace/scanstack.tex}}%
    \end{column}
  \end{columns}
\end{frame}

% Duplicate of previous-previous slide
\begin{frame}{Backtraces}{Continuing on \funcname{caml\_stash\_backtrace}}
  \begin{columns}[c]
    \begin{column}{0.5\textwidth}
      \minipage[c][0.75\textheight][s]{\columnwidth}
      \begin{itemize}
          \color{gray}
        \item A C function provided by the runtime (specific to native code)
        \item It scans the stack, between the stack pointer at the raising point up to the next trap, in order to collect the names of the detected function frames
        \item It uses the same technique and information as the Garbage Collector (GC) uses to scan the values live on the stack
      \end{itemize}
      \begin{itemize}
        \item For each function frame detected, store a pointer to the \typename{frame\_descr} in the \localname{domain\_state->backtrace\_buffer} array, effectively constructing the backtrace
      \end{itemize}
      \onslide<2->{
        \begin{itemize}
            \smallskip
          \item Constructed backtrace:
            \funcname{definitely\_raise},
            \onslide<3->{\funcname{probably\_raise}}
        \end{itemize}
      }
      \vfill
      \mintocaml[firstline=9,lastline=13,highlightlines=12]{../src/backtrace/backtrace.ml}
      \endminipage
    \end{column}
    \begin{column}{0.5\textwidth}
      \raggedleft
      \onslide*<1>{\listStashBacktrace[highlightlines={114-115}]{}}
      \onslide*<2>{\providecommand\step{3}\input{diagrams/backtrace/backtrace.tex}}%
      \onslide*<3>{\providecommand\step{4}\input{diagrams/backtrace/backtrace.tex}}%
    \end{column}
  \end{columns}
\end{frame}

\begin{frame}{Backtraces}{Back to \funcname{caml\_raise\_exn}}
  \begin{columns}[c]
    \begin{column}{0.5\textwidth}
      \minipage[c][0.66\textheight][s]{\columnwidth}
      \begin{itemize}\color{gray}
        \item Checks wheteher exceptions backtrace should be recorded, by checking the \funcarg{domain\_state->backtrace\_active} value
        \item Switches to the C stack to be able to execute C code
        \item Calls \funcname{caml\_stash\_backtrace}, to collect the backtrace up to the next trap
      \end{itemize}
      \onslide*<2->{
        \begin{itemize}
          \item<2-> Executes the exception handler in \funcname{probably\_raise} with \funcname{RESTORE\_EXN\_HANDLER\_OCAML}
            \begin{itemize}
              \item Set \regname{RSP} to \localname{domain\_state->exn\_handler} value
              \item Restore \localname{domain\_state->exn\_handler} from trap
              \item Jump to the handler code with \funcname{ret}
            \end{itemize}
        \end{itemize}
      }
      \vfill
      \onslide*<1>{\mintocaml[firstline=9,lastline=13,highlightlines=12]{../src/backtrace/backtrace.ml}}
      \onslide*<2->{\mintocaml[firstline=15,lastline=21,highlightlines={19-21}]{../src/backtrace/backtrace.ml}}
      \endminipage
    \end{column}
    \begin{column}{0.5\textwidth}
      \centering
      \onslide*<1>{
        \mintc[firstline=190,lastline=192]{../ocaml/runtime/amd64.S}
        \mintc[firstline=234,lastline=237]{../ocaml/runtime/amd64.S}
      }
      \onslide*<2>{
        \mintc[firstline=190,lastline=192,highlightlines={191-192}]{../ocaml/runtime/amd64.S}
        \mintc[firstline=234,lastline=237,highlightlines={235-237}]{../ocaml/runtime/amd64.S}
      }
      \onslide*<1>{\listCamlRaiseExn[highlightlines=720]{}}
      \onslide*<2>{\listCamlRaiseExn[highlightlines=722]{}}
      \onslide*<3->{\providecommand\step{5}\input{diagrams/backtrace/backtrace.tex}}
    \end{column}
  \end{columns}
\end{frame}

\begin{frame}{Backtraces}{\funcname{caml\_probably\_raise}'s handler doesn't catch \typename{ExnC}}
  \begin{columns}[c]
    \begin{column}{0.45\textwidth}
      \minipage[c][0.75\textheight][s]{\columnwidth}
      \begin{itemize}
        \item<1-> \funcname{match \dots with}\footnotemark pattern matching can also match exceptions
        \item<1-> The CMM of \funcname{probably\_raise} shows that this \funcname{match \dots with} is implemented with a pattern matching and a \funcname{try \dots with}
        \item<1-> But this one only matches on \typename{ExnA} exception
        \item<2-> Since the pattern matching fails to catch the exception, \funcname{reraise} is called with our current \typename{ExnC} exception
        \item<3-> Disassembly shows that \funcname{reraise} is actually a call to \funcname{caml\_reraise\_exn}, which is also an assembly function provided by the runtime
      \end{itemize}
      \vfill
      \mintocaml[firstline=15,lastline=21,highlightlines={18-21}]{../src/backtrace/backtrace.ml}
      \endminipage
    \end{column}
    \begin{column}{0.55\textwidth}
      \centering
      \onslide*<1>{\mintcmm[highlightlines={10-18}]{../src/backtrace/camlBacktrace\_\_probably\_raise\_351.cmm}}
      \onslide*<2>{\listcmm[highlightlines=18]{../src/backtrace/camlBacktrace\_\_probably\_raise_351.cmm}{CMM of probably\_raise}}
      \onslide*<3>{\mintobjdump[firstline=5,lastline=35,highlightlines=31]{../src/backtrace/camlBacktrace\_\_probably\_raise_351.objdump}}
    \end{column}
  \end{columns}
  \only<1>{\footnotetext[1]{https://blog.janestreet.com/pattern-matching-and-exception-handling-unite/}}
\end{frame}

\newcommand\listCamlReraiseExn[1][]{
  \listasm[firstline=727,lastline=734,#1]{../ocaml/runtime/amd64.S}{runtime/amd64.S:727}}

\begin{frame}{Backtraces}{Introducing \funcname{caml\_reraise\_exn}}
  \begin{columns}[c]
    \begin{column}{0.5\textwidth}
      \minipage[c][0.66\textheight][s]{\columnwidth}
      \begin{itemize}
        \item<1-> The exception have to be reraised to the next handler
        \item<2-> First, checks if the backtrace should be collected with \funcarg{domain\_state->backtrace\_active}
        \only<3>{
        \begin{itemize}
            \item If not, behaves like a \funcname{raise\_notrace} with \funcname{RESTORE\_EXN\_HANDLER\_OCAML}
        \end{itemize}
        }
        \item<4-> Jumps into \funcname{caml\_raise\_exn}
        \item<5-> Calls \funcname{caml\_stash\_backtrace} again, to scan the next part of the stack: from the trap just pop-ed to the next trap
      \end{itemize}
      \vfill
      \mintocaml[firstline=15,lastline=21,highlightlines={18-21}]{../src/backtrace/backtrace.ml}
      \endminipage
    \end{column}
    \begin{column}{0.5\textwidth}
      \raggedleft
      \onslide*<1>{\listCamlReraiseExn{}}
      \onslide*<2>{\listCamlReraiseExn[highlightlines=729]{}}
      \onslide*<3>{
        \mintc[firstline=190,lastline=192,highlightlines={191-192}]{../ocaml/runtime/amd64.S}
        \mintc[firstline=234,lastline=237,highlightlines={235-237}]{../ocaml/runtime/amd64.S}
        \medskip
        \listCamlReraiseExn[highlightlines=731]{}
      }
      \onslide*<4-5>{
        % TODO refactor with \listCamlReraiseExn
        \mintasm[firstline=727,lastline=734,highlightlines=730]{../ocaml/runtime/amd64.S}
        \medskip
      }
      \onslide*<4>{\listCamlRaiseExn[highlightlines=711]{}}
      \onslide*<5>{\listCamlRaiseExn[highlightlines=720]{}}
    \end{column}
  \end{columns}
\end{frame}

\begin{frame}{Backtraces}{Backtrace collection continues}
  \begin{columns}[c]
    \begin{column}{0.55\textwidth}
      \minipage[c][0.75\textheight][s]{\columnwidth}
      \begin{itemize}
        \item<1-> \funcname{caml\_stash\_backtrace} scans the older part of the stack, up to the next trap
        \item<6-> Controls jumps to the nested exception handler in \funcname{maybe\_raise}, which matches only on \typename{ExnB}
        \item<7-> \funcname{caml\_reraise\_exn} is called again, backtrace collection resumes with \funcname{caml\_stash\_backtrace}
      \end{itemize}
      \medskip
      \begin{itemize}
        \item<2-> Constructed backtrace:
        \begin{itemize}
          \item <2-> \funcname{definitely\_raise}
          \item <2-> \funcname{probably\_raise}
          \item <3-> \funcname{probably\_raise} (re-raise)
          \item <4-> \funcname{maybe\_raise}
          \item <5-> \funcname{main}
          \item <8-> \funcname{main} (re-raise)
        \end{itemize}
      \end{itemize}
      \vfill
      \onslide*<1-5>{\mintocaml[firstline=15,lastline=21]{../src/backtrace/backtrace.ml}}
      \onslide*<6>{\mintocaml[firstline=28,lastline=34,highlightlines=31]{../src/backtrace/backtrace.ml}}
      \onslide*<7->{\mintocaml[firstline=28,lastline=34,highlightlines=33]{../src/backtrace/backtrace.ml}}
      \endminipage
    \end{column}
    \begin{column}{0.45\textwidth}
      \raggedleft
      \onslide*<1>{\providecommand\step{6}\input{diagrams/backtrace/backtrace.tex}}%
      \onslide*<2>{\providecommand\step{6}\input{diagrams/backtrace/backtrace.tex}}%
      \onslide*<3>{\providecommand\step{7}\input{diagrams/backtrace/backtrace.tex}}%
      \onslide*<4>{\providecommand\step{8}\input{diagrams/backtrace/backtrace.tex}}%
      \onslide*<5>{\providecommand\step{9}\input{diagrams/backtrace/backtrace.tex}}%
      \onslide*<6>{\providecommand\step{10}\input{diagrams/backtrace/backtrace.tex}}%
      \onslide*<7>{\providecommand\step{11}\input{diagrams/backtrace/backtrace.tex}}%
      \onslide*<8>{\providecommand\step{12}\input{diagrams/backtrace/backtrace.tex}}%
      \onslide*<9>{\providecommand\step{13}\input{diagrams/backtrace/backtrace.tex}}%
    \end{column}
  \end{columns}
\end{frame}

\begin{frame}{Backtraces}{Printing the backtrace}
  \begin{columns}[c]
    \begin{column}{0.45\textwidth}
      \minipage[c][0.66\textheight][s]{\columnwidth}
      \begin{itemize}
        \item<1-> The exception backtrace can now be printed to \funcarg{stdout} with \funcname{Printexc.print_backtrace}
        \item<2-> First the exception backtrace is retrieved into an OCaml array with \funcname{get_raw_backtrace }
        \item<3-> Which is implemented as the \funcname{caml_get_exception_raw_backtrace} C function in the runtime
        \item<4-> And finally printed with \funcname{print_exception_backtrace}
      \end{itemize}
      \vfill
      \mintocaml[firstline=28,lastline=34,highlightlines=33]{../src/backtrace/backtrace.ml}
      \endminipage
    \end{column}
    \begin{column}{0.55\textwidth}
      \onslide*<1,2>{
        \mintocaml[firstline=100,lastline=101]{../ocaml/stdlib/printexc.ml}
      }
      \only<1>{
        \listocaml[firstline=170,lastline=172]{../ocaml/stdlib/printexc.ml}{stdlib/printexc.ml:170}
      }
      \only<2>{
        \listocaml[firstline=170,lastline=172,highlightlines=172]{../ocaml/stdlib/printexc.ml}{stdlib/printexc.ml:170}
      }
      \only<3>{
        \mintc[firstline=154,lastline=156]{../ocaml/runtime/backtrace.c}
        \listc[firstline=171,lastline=192,highlightlines={184-187}]{../ocaml/runtime/backtrace.c}{runtime/backtrace.c:154}
      }
      \only<4>{
        \listocaml[firstline=155,lastline=165]{../ocaml/stdlib/printexc.ml}{stdlib/printexc.ml:55}
      }
    \end{column}
  \end{columns}
\end{frame}


\subsection{The multiple kinds of raise}
\frameSubsection{}{}

\begin{frame}{Backtraces}{When backtraces are not needed}
  \begin{columns}[c]
    \begin{column}{0.45\textwidth}
      \minipage[c][0.66\textheight][s]{\columnwidth}
      \begin{itemize}
        \item<1-> What if the backtrace for an exception is never needed?
        \item<2-> It's possible to raise the exception explicitly with \funcname{raise\_notrace} to make raising as cheap as possible
        \item<3-> An example of use in the \funcname{dynlink} library of the OCaml compiler
      \end{itemize}
      \vfill
      \onslide<3->{
      \listocaml[firstline=205,lastline=209,highlightlines=207]{../ocaml/otherlibs/dynlink/dynlink\_compilerlibs/misc.ml}{otherlibs/dynlink/dynlink\_compilerlibs/misc.ml:205}
      }
      \endminipage
    \end{column}
    \begin{column}{0.55\textwidth}
    \only<4>{
      \mintcmm[highlightlines=3]{../src/notrace/camlNotrace\_\_fun_347.cmm}
      \listcmm{../src/notrace/camlNotrace\_\_all\_somes\_327.cmm}{all\_somes}
    }
    \onslide*<5->{
      \listobjdump[highlightlines={6-9}]{../src/notrace/camlNotrace\_\_fun_347.objdump}{anonymous function from \funcname{all\_somes}}
    }
    \end{column}
  \end{columns}
\end{frame}

\begin{frame}{Backtraces}{Summary table of the multiple kinds of raise}
  \begin{itemize}
    \item When debugging information is enabled and if the backtrace of an exception isn't needed, it's possible to raise with \funcname{raise\_notrace} for performance
    \item When debugging information is \emph{not} enabled, the compiler optimises \funcname{raise} calls to inline assembly (same effect as using \funcname{raise\_notrace})
  \end{itemize}
  \bigskip
  \centering
  \begin{tabular}{| l | c | c | c | c |}
    \hline
    \parbox{10em}{Debug information \\ (\funcarg{-g} flag)}  & \multicolumn{2}{|c|}{Disabled}                                     & \multicolumn{2}{|c|}{Enabled} \\
    \hline
    Raise kind                                               & \funcname{raise} & \funcname{raise\_notrace}                       & \funcname{raise} & \funcname{raise\_notrace} \\
    \hline
    Compilation                                              & \parbox{8em}{inline assembly\\ (optimization)} & inline assembly   & \funcname{caml\_raise\_exn} & inline assembly \\
    \hline
  \end{tabular}
\end{frame}


%
% Takeaway #5
%
\subsection*{Takeaway \#5}
\frameSubsectionTakeaway{}
\againframe<5>{takeaway}
}%
    \end{column}
  \end{columns}
\end{frame}

\begin{frame}{Backtraces}{Printing the backtrace}
  \begin{columns}[c]
    \begin{column}{0.45\textwidth}
      \minipage[c][0.66\textheight][s]{\columnwidth}
      \begin{itemize}
        \item<1-> The exception backtrace can now be printed to \funcarg{stdout} with \funcname{Printexc.print_backtrace}
        \item<2-> First the exception backtrace is retrieved into an OCaml array with \funcname{get_raw_backtrace }
        \item<3-> Which is implemented as the \funcname{caml_get_exception_raw_backtrace} C function in the runtime
        \item<4-> And finally printed with \funcname{print_exception_backtrace}
      \end{itemize}
      \vfill
      \mintocaml[firstline=28,lastline=34,highlightlines=33]{../src/backtrace/backtrace.ml}
      \endminipage
    \end{column}
    \begin{column}{0.55\textwidth}
      \onslide*<1,2>{
        \mintocaml[firstline=100,lastline=101]{../ocaml/stdlib/printexc.ml}
      }
      \only<1>{
        \listocaml[firstline=170,lastline=172]{../ocaml/stdlib/printexc.ml}{stdlib/printexc.ml:170}
      }
      \only<2>{
        \listocaml[firstline=170,lastline=172,highlightlines=172]{../ocaml/stdlib/printexc.ml}{stdlib/printexc.ml:170}
      }
      \only<3>{
        \mintc[firstline=154,lastline=156]{../ocaml/runtime/backtrace.c}
        \listc[firstline=171,lastline=192,highlightlines={184-187}]{../ocaml/runtime/backtrace.c}{runtime/backtrace.c:154}
      }
      \only<4>{
        \listocaml[firstline=155,lastline=165]{../ocaml/stdlib/printexc.ml}{stdlib/printexc.ml:55}
      }
    \end{column}
  \end{columns}
\end{frame}


\subsection{The multiple kinds of raise}
\frameSubsection{}{}

\begin{frame}{Backtraces}{When backtraces are not needed}
  \begin{columns}[c]
    \begin{column}{0.45\textwidth}
      \minipage[c][0.66\textheight][s]{\columnwidth}
      \begin{itemize}
        \item<1-> What if the backtrace for an exception is never needed?
        \item<2-> It's possible to raise the exception explicitly with \funcname{raise\_notrace} to make raising as cheap as possible
        \item<3-> An example of use in the \funcname{dynlink} library of the OCaml compiler
      \end{itemize}
      \vfill
      \onslide<3->{
      \listocaml[firstline=205,lastline=209,highlightlines=207]{../ocaml/otherlibs/dynlink/dynlink\_compilerlibs/misc.ml}{otherlibs/dynlink/dynlink\_compilerlibs/misc.ml:205}
      }
      \endminipage
    \end{column}
    \begin{column}{0.55\textwidth}
    \only<4>{
      \mintcmm[highlightlines=3]{../src/notrace/camlNotrace\_\_fun_347.cmm}
      \listcmm{../src/notrace/camlNotrace\_\_all\_somes\_327.cmm}{all\_somes}
    }
    \onslide*<5->{
      \listobjdump[highlightlines={6-9}]{../src/notrace/camlNotrace\_\_fun_347.objdump}{anonymous function from \funcname{all\_somes}}
    }
    \end{column}
  \end{columns}
\end{frame}

\begin{frame}{Backtraces}{Summary table of the multiple kinds of raise}
  \begin{itemize}
    \item When debugging information is enabled and if the backtrace of an exception isn't needed, it's possible to raise with \funcname{raise\_notrace} for performance
    \item When debugging information is \emph{not} enabled, the compiler optimises \funcname{raise} calls to inline assembly (same effect as using \funcname{raise\_notrace})
  \end{itemize}
  \bigskip
  \centering
  \begin{tabular}{| l | c | c | c | c |}
    \hline
    \parbox{10em}{Debug information \\ (\funcarg{-g} flag)}  & \multicolumn{2}{|c|}{Disabled}                                     & \multicolumn{2}{|c|}{Enabled} \\
    \hline
    Raise kind                                               & \funcname{raise} & \funcname{raise\_notrace}                       & \funcname{raise} & \funcname{raise\_notrace} \\
    \hline
    Compilation                                              & \parbox{8em}{inline assembly\\ (optimization)} & inline assembly   & \funcname{caml\_raise\_exn} & inline assembly \\
    \hline
  \end{tabular}
\end{frame}


%
% Takeaway #5
%
\subsection*{Takeaway \#5}
\frameSubsectionTakeaway{}
\againframe<5>{takeaway}
}%
    \end{column}
  \end{columns}
\end{frame}

\begin{frame}{Backtraces}{Printing the backtrace}
  \begin{columns}[c]
    \begin{column}{0.45\textwidth}
      \minipage[c][0.66\textheight][s]{\columnwidth}
      \begin{itemize}
        \item<1-> The exception backtrace can now be printed to \funcarg{stdout} with \funcname{Printexc.print_backtrace}
        \item<2-> First the exception backtrace is retrieved into an OCaml array with \funcname{get_raw_backtrace }
        \item<3-> Which is implemented as the \funcname{caml_get_exception_raw_backtrace} C function in the runtime
        \item<4-> And finally printed with \funcname{print_exception_backtrace}
      \end{itemize}
      \vfill
      \mintocaml[firstline=28,lastline=34,highlightlines=33]{../src/backtrace/backtrace.ml}
      \endminipage
    \end{column}
    \begin{column}{0.55\textwidth}
      \onslide*<1,2>{
        \mintocaml[firstline=100,lastline=101]{../ocaml/stdlib/printexc.ml}
      }
      \only<1>{
        \listocaml[firstline=170,lastline=172]{../ocaml/stdlib/printexc.ml}{stdlib/printexc.ml:170}
      }
      \only<2>{
        \listocaml[firstline=170,lastline=172,highlightlines=172]{../ocaml/stdlib/printexc.ml}{stdlib/printexc.ml:170}
      }
      \only<3>{
        \mintc[firstline=154,lastline=156]{../ocaml/runtime/backtrace.c}
        \listc[firstline=171,lastline=192,highlightlines={184-187}]{../ocaml/runtime/backtrace.c}{runtime/backtrace.c:154}
      }
      \only<4>{
        \listocaml[firstline=155,lastline=165]{../ocaml/stdlib/printexc.ml}{stdlib/printexc.ml:55}
      }
    \end{column}
  \end{columns}
\end{frame}


\subsection{The multiple kinds of raise}
\frameSubsection{}{}

\begin{frame}{Backtraces}{When backtraces are not needed}
  \begin{columns}[c]
    \begin{column}{0.45\textwidth}
      \minipage[c][0.66\textheight][s]{\columnwidth}
      \begin{itemize}
        \item<1-> What if the backtrace for an exception is never needed?
        \item<2-> It's possible to raise the exception explicitly with \funcname{raise\_notrace} to make raising as cheap as possible
        \item<3-> An example of use in the \funcname{dynlink} library of the OCaml compiler
      \end{itemize}
      \vfill
      \onslide<3->{
      \listocaml[firstline=205,lastline=209,highlightlines=207]{../ocaml/otherlibs/dynlink/dynlink\_compilerlibs/misc.ml}{otherlibs/dynlink/dynlink\_compilerlibs/misc.ml:205}
      }
      \endminipage
    \end{column}
    \begin{column}{0.55\textwidth}
    \only<4>{
      \mintcmm[highlightlines=3]{../src/notrace/camlNotrace\_\_fun_347.cmm}
      \listcmm{../src/notrace/camlNotrace\_\_all\_somes\_327.cmm}{all\_somes}
    }
    \onslide*<5->{
      \listobjdump[highlightlines={6-9}]{../src/notrace/camlNotrace\_\_fun_347.objdump}{anonymous function from \funcname{all\_somes}}
    }
    \end{column}
  \end{columns}
\end{frame}

\begin{frame}{Backtraces}{Summary table of the multiple kinds of raise}
  \begin{itemize}
    \item When debugging information is enabled and if the backtrace of an exception isn't needed, it's possible to raise with \funcname{raise\_notrace} for performance
    \item When debugging information is \emph{not} enabled, the compiler optimises \funcname{raise} calls to inline assembly (same effect as using \funcname{raise\_notrace})
  \end{itemize}
  \bigskip
  \centering
  \begin{tabular}{| l | c | c | c | c |}
    \hline
    \parbox{10em}{Debug information \\ (\funcarg{-g} flag)}  & \multicolumn{2}{|c|}{Disabled}                                     & \multicolumn{2}{|c|}{Enabled} \\
    \hline
    Raise kind                                               & \funcname{raise} & \funcname{raise\_notrace}                       & \funcname{raise} & \funcname{raise\_notrace} \\
    \hline
    Compilation                                              & \parbox{8em}{inline assembly\\ (optimization)} & inline assembly   & \funcname{caml\_raise\_exn} & inline assembly \\
    \hline
  \end{tabular}
\end{frame}


%
% Takeaway #5
%
\subsection*{Takeaway \#5}
\frameSubsectionTakeaway{}
\againframe<5>{takeaway}
}%
      \onslide*<9>{\providecommand\step{13}%
% Backtraces
%
\subsection{One advantage of raising with debugging information}
\frameSubsection{}{}

\begin{frame}{Backtraces}{Debugging information \& source code}
  \begin{columns}[c]
    \begin{column}{0.5\textwidth}
      \begin{itemize}
        \item Raising an exception is fun and games, but how do we know where the exception originated? $\rightarrow$ With the backtrace associated with the exception!
        \item In order to get a backtrace from the point where an exception was raised, it's necessary to compile with debugging information enabled (\funcarg{-g} flag for the compiler)
        \item Same example as in the `Nested handlers' section
      \end{itemize}
    \end{column}
    \begin{column}{0.5\textwidth}
      \listocaml[firstline=9,lastline=34]{../src/backtrace/backtrace.ml}{backtrace/backtrace.ml}
    \end{column}
  \end{columns}
\end{frame}

\begin{frame}{Backtraces}{CMM and disassembly of \funcname{definitely\_raise}}
  \begin{columns}[c]
    \begin{column}{0.5\textwidth}
      \minipage[c][0.66\textheight][s]{\columnwidth}
      \begin{itemize}
        \item<1-> An effect of compiling with debugging information (\funcarg{-g}): the CMM no longer uses \funcname{raise\_notrace} to raise the exception but \funcname{raise} instead
        \item<2-> Disassembly shows that CMM \funcname{raise} is actually a call to \funcname{caml\_raise\_exn}, a function in assembler provided by the runtime
      \end{itemize}
      \vfill
      \mintocaml[firstline=9,lastline=13,highlightlines=12]{../src/backtrace/backtrace.ml}
      \endminipage
    \end{column}
    \begin{column}{0.5\textwidth}
      \onslide*<1>{
        \mintcmm[highlightlines=5]{../src/backtrace/camlBacktrace\_\_definitely\_raise\_347.cmm}
      }
      \onslide*<2>{
        \mintobjdump[firstline=2,highlightlines=14]{../src/backtrace/camlBacktrace\_\_definitely\_raise\_347.objdump}
      }
    \end{column}
  \end{columns}
\end{frame}

\begin{frame}{Backtraces}{Introducing \funcname{caml\_raise\_exn}}
  \begin{columns}[c]
    \begin{column}{0.5\textwidth}
      \minipage[c][0.66\textheight][s]{\columnwidth}
      \onslide*<1>{
        \begin{itemize}
          \item Raising the exception is performed using \funcname{caml\_raise\_exn} which is
            \begin{itemize}
              \item An assembly function provided by the runtime
              \item Architecture specific
            \end{itemize}
          \item Let's see what it does differently from \funcname{raise\_notrace}\dots
          \item We are about to raise \typename{ExnC} with \funcname{caml\_raise\_exn} from \funcname{definitely\_raise}
        \end{itemize}
      }
      \onslide*<2>{
        \mintobjdump[firstline=2,highlightlines=14]{../src/backtrace/camlBacktrace\_\_definitely\_raise\_347.objdump}
      }
      \vfill
      \mintocaml[firstline=9,lastline=13,highlightlines=12]{../src/backtrace/backtrace.ml}
      \endminipage
    \end{column}
    \begin{column}{0.5\textwidth}
      \centering
      \providecommand\step{1}%
% Backtraces
%
\subsection{One advantage of raising with debugging information}
\frameSubsection{}{}

\begin{frame}{Backtraces}{Debugging information \& source code}
  \begin{columns}[c]
    \begin{column}{0.5\textwidth}
      \begin{itemize}
        \item Raising an exception is fun and games, but how do we know where the exception originated? $\rightarrow$ With the backtrace associated with the exception!
        \item In order to get a backtrace from the point where an exception was raised, it's necessary to compile with debugging information enabled (\funcarg{-g} flag for the compiler)
        \item Same example as in the `Nested handlers' section
      \end{itemize}
    \end{column}
    \begin{column}{0.5\textwidth}
      \listocaml[firstline=9,lastline=34]{../src/backtrace/backtrace.ml}{backtrace/backtrace.ml}
    \end{column}
  \end{columns}
\end{frame}

\begin{frame}{Backtraces}{CMM and disassembly of \funcname{definitely\_raise}}
  \begin{columns}[c]
    \begin{column}{0.5\textwidth}
      \minipage[c][0.66\textheight][s]{\columnwidth}
      \begin{itemize}
        \item<1-> An effect of compiling with debugging information (\funcarg{-g}): the CMM no longer uses \funcname{raise\_notrace} to raise the exception but \funcname{raise} instead
        \item<2-> Disassembly shows that CMM \funcname{raise} is actually a call to \funcname{caml\_raise\_exn}, a function in assembler provided by the runtime
      \end{itemize}
      \vfill
      \mintocaml[firstline=9,lastline=13,highlightlines=12]{../src/backtrace/backtrace.ml}
      \endminipage
    \end{column}
    \begin{column}{0.5\textwidth}
      \onslide*<1>{
        \mintcmm[highlightlines=5]{../src/backtrace/camlBacktrace\_\_definitely\_raise\_347.cmm}
      }
      \onslide*<2>{
        \mintobjdump[firstline=2,highlightlines=14]{../src/backtrace/camlBacktrace\_\_definitely\_raise\_347.objdump}
      }
    \end{column}
  \end{columns}
\end{frame}

\begin{frame}{Backtraces}{Introducing \funcname{caml\_raise\_exn}}
  \begin{columns}[c]
    \begin{column}{0.5\textwidth}
      \minipage[c][0.66\textheight][s]{\columnwidth}
      \onslide*<1>{
        \begin{itemize}
          \item Raising the exception is performed using \funcname{caml\_raise\_exn} which is
            \begin{itemize}
              \item An assembly function provided by the runtime
              \item Architecture specific
            \end{itemize}
          \item Let's see what it does differently from \funcname{raise\_notrace}\dots
          \item We are about to raise \typename{ExnC} with \funcname{caml\_raise\_exn} from \funcname{definitely\_raise}
        \end{itemize}
      }
      \onslide*<2>{
        \mintobjdump[firstline=2,highlightlines=14]{../src/backtrace/camlBacktrace\_\_definitely\_raise\_347.objdump}
      }
      \vfill
      \mintocaml[firstline=9,lastline=13,highlightlines=12]{../src/backtrace/backtrace.ml}
      \endminipage
    \end{column}
    \begin{column}{0.5\textwidth}
      \centering
      \providecommand\step{1}%
% Backtraces
%
\subsection{One advantage of raising with debugging information}
\frameSubsection{}{}

\begin{frame}{Backtraces}{Debugging information \& source code}
  \begin{columns}[c]
    \begin{column}{0.5\textwidth}
      \begin{itemize}
        \item Raising an exception is fun and games, but how do we know where the exception originated? $\rightarrow$ With the backtrace associated with the exception!
        \item In order to get a backtrace from the point where an exception was raised, it's necessary to compile with debugging information enabled (\funcarg{-g} flag for the compiler)
        \item Same example as in the `Nested handlers' section
      \end{itemize}
    \end{column}
    \begin{column}{0.5\textwidth}
      \listocaml[firstline=9,lastline=34]{../src/backtrace/backtrace.ml}{backtrace/backtrace.ml}
    \end{column}
  \end{columns}
\end{frame}

\begin{frame}{Backtraces}{CMM and disassembly of \funcname{definitely\_raise}}
  \begin{columns}[c]
    \begin{column}{0.5\textwidth}
      \minipage[c][0.66\textheight][s]{\columnwidth}
      \begin{itemize}
        \item<1-> An effect of compiling with debugging information (\funcarg{-g}): the CMM no longer uses \funcname{raise\_notrace} to raise the exception but \funcname{raise} instead
        \item<2-> Disassembly shows that CMM \funcname{raise} is actually a call to \funcname{caml\_raise\_exn}, a function in assembler provided by the runtime
      \end{itemize}
      \vfill
      \mintocaml[firstline=9,lastline=13,highlightlines=12]{../src/backtrace/backtrace.ml}
      \endminipage
    \end{column}
    \begin{column}{0.5\textwidth}
      \onslide*<1>{
        \mintcmm[highlightlines=5]{../src/backtrace/camlBacktrace\_\_definitely\_raise\_347.cmm}
      }
      \onslide*<2>{
        \mintobjdump[firstline=2,highlightlines=14]{../src/backtrace/camlBacktrace\_\_definitely\_raise\_347.objdump}
      }
    \end{column}
  \end{columns}
\end{frame}

\begin{frame}{Backtraces}{Introducing \funcname{caml\_raise\_exn}}
  \begin{columns}[c]
    \begin{column}{0.5\textwidth}
      \minipage[c][0.66\textheight][s]{\columnwidth}
      \onslide*<1>{
        \begin{itemize}
          \item Raising the exception is performed using \funcname{caml\_raise\_exn} which is
            \begin{itemize}
              \item An assembly function provided by the runtime
              \item Architecture specific
            \end{itemize}
          \item Let's see what it does differently from \funcname{raise\_notrace}\dots
          \item We are about to raise \typename{ExnC} with \funcname{caml\_raise\_exn} from \funcname{definitely\_raise}
        \end{itemize}
      }
      \onslide*<2>{
        \mintobjdump[firstline=2,highlightlines=14]{../src/backtrace/camlBacktrace\_\_definitely\_raise\_347.objdump}
      }
      \vfill
      \mintocaml[firstline=9,lastline=13,highlightlines=12]{../src/backtrace/backtrace.ml}
      \endminipage
    \end{column}
    \begin{column}{0.5\textwidth}
      \centering
      \providecommand\step{1}\input{diagrams/backtrace/backtrace.tex}
    \end{column}
  \end{columns}
\end{frame}

\begin{frame}{Backtraces}{But before that, we need more fields from \typename{caml\_domain\_state}}
  \begin{columns}
    \begin{column}{0.5\textwidth}
      \begin{itemize}
        \item Remember \typename{caml\_domain\_state} the per-domain runtime state?
        \item Some other members are of interest to us now\dots
        \item \localname{backtrace\_active} is used to know if the runtime should collect backtraces
        \item \localname{backtrace\_active} can be set by
          \begin{itemize}
            \item The \funcarg{b} flag in the \funcname{OCAMLRUNPARAM} environment variable
            \item \mintinline{ocaml}{Printexc.record_backtrace : bool -> unit} \footnotemark
          \end{itemize}
      \end{itemize}
    \end{column}
    \begin{column}{0.5\textwidth}
      \begin{listing}
        \mintc[highlightlines={9-11}]{src/ocaml/domain\_state\_backtrace.h}
        \caption{runtime/caml/domain\_state.h}
      \end{listing}
    \end{column}
  \end{columns}
  \footnotetext[1]{https://v2.ocaml.org/api/Printexc.html}
\end{frame}

\newcommand\listCamlRaiseExn[1][]{
  \listasm[firstline=702,lastline=725,#1]{../ocaml/runtime/amd64.S}{runtime/amd64.S:702}}

\begin{frame}{Backtraces}{Walk through \funcname{caml\_raise\_exn}}
  \begin{columns}[T]
    \begin{column}{0.5\textwidth}
      \begin{itemize}
        \item<1-> First checks whether exceptions backtrace should be recorded
        \item<1-> By checking the \funcarg{domain\_state->backtrace\_active} value
        \item<2-> If not, acts as \funcname{raise\_notrace} with \funcname{RESTORE\_EXN\_HANDLER\_OCAML}
          \begin{itemize}
            \item Set \regname{RSP} to \localname{domain\_state->exn\_handler} value
            \item Restore \localname{domain\_state->exn\_handler} from trap
            \item Jump with \funcname{ret} (same effect as using \funcname{pop} and \funcname{jmp})
          \end{itemize}
        \item<3-> Otherwise, switches to the C stack to be able to execute C code
        \item<4-> Calls \funcname{caml\_stash\_backtrace}, a C function provided by the runtime\ignorespaces
          \onslide<6->{, with
            \begin{itemize}
              \item \funcarg{exn}: exception value
              \item \funcarg{pc}: code address of the raising point
              \item \funcarg{sp}: stack address of the raising function
              \item \funcarg{trapsp}: stack address of the next handler
            \end{itemize}
          }
      \end{itemize}
      \bigskip
      \mintocaml[firstline=9,lastline=13,highlightlines=12]{../src/backtrace/backtrace.ml}
    \end{column}
    \begin{column}{0.5\textwidth}
      \raggedleft
      \onslide*<1,3-4>{
        \mintc[firstline=190,lastline=192]{../ocaml/runtime/amd64.S}
        \mintc[firstline=234,lastline=237]{../ocaml/runtime/amd64.S}
      }
      \onslide*<2>{
        \mintc[firstline=190,lastline=192,highlightlines={191-192}]{../ocaml/runtime/amd64.S}
        \mintc[firstline=234,lastline=237,highlightlines={235-237}]{../ocaml/runtime/amd64.S}
      }
      \onslide*<1>{\listCamlRaiseExn[highlightlines=705]{}}%
      \onslide*<2>{\listCamlRaiseExn[highlightlines={707-708}]{}}%
      \onslide*<3>{\listCamlRaiseExn[highlightlines={712-714}]{}}%
      \onslide*<4>{\listCamlRaiseExn[highlightlines={715-720}]{}}%
      \onslide*<5>{\providecommand\step{1}\input{diagrams/backtrace/backtrace.tex}}%
      \onslide*<6>{\providecommand\step{2}\input{diagrams/backtrace/backtrace.tex}}%
    \end{column}
  \end{columns}
\end{frame}

\newcommand\listStashBacktrace[1][]{
  \listc[firstline=91,lastline=120,#1]{../ocaml/runtime/backtrace\_nat.c}{runtime/backtrace\_nat.c:91}}

\begin{frame}{Backtraces}{Introducing \funcname{caml\_stash\_backtrace}}
  \begin{columns}[c]
    \begin{column}{0.5\textwidth}
      \minipage[c][0.66\textheight][s]{\columnwidth}
      \begin{itemize}
        \item<1-> A C function provided by the runtime (specific to native code)
        \item<2-> It scans the stack, between the stack pointer at the raising point up to the next trap, in order to collect the names of the detected function frames
          \onslide*<2>{
            \begin{itemize}
              \item And more information, like line number, column number...
            \end{itemize}
          }
        \item<3-> It uses the same technique and information as the Garbage Collector (GC) uses to scan the values live on the stack
          \onslide*<4>{
            \begin{itemize}
              \item \typename{frame\_descr} are at the core of stack unwinding
            \end{itemize}
          }
      \end{itemize}
      \vfill
      \mintocaml[firstline=9,lastline=13,highlightlines=12]{../src/backtrace/backtrace.ml}
      \endminipage
    \end{column}
    \begin{column}{0.5\textwidth}
      \onslide*<1>{\listStashBacktrace[highlightlines=91]{}}
      \onslide*<2>{\listStashBacktrace[highlightlines={91,117-118}]{}}
      \onslide*<3->{\listStashBacktrace[highlightlines={94,106,109-110}]{}}
    \end{column}
  \end{columns}
\end{frame}

\newcommand\listFrameDescr[1][]{
  \listocaml[firstline=111,lastline=124,#1]{../ocaml/asmcomp/emitaux.ml}{asmcomp/emitaux.ml:111}}
\newcommand\listDebugInfo[1][]{
  \listocaml[firstline=38,lastline=49,#1]{../ocaml/lambda/debuginfo.mli}{lambda/debuginfo.mli:38}}

\begin{frame}{Backtraces}{\typename{frame\_descr} and \typename{frame\_debuginfo} in a nutshell}
  \begin{columns}[c]
    \begin{column}{0.5\textwidth}
      \begin{itemize}
        \item<1-> For every return address in an OCaml program, there is a associated \typename{frame\_descr}
        \item<2-> A \typename{frame\_descr} specifies
        \begin{itemize}
          \item<2-> The size of the function stack frame
          \item<3-> Where live values are on the stack (so that the GC can mark them as live and prevent from being swept)
        \end{itemize}
        \item<4-> When compiled with debug information, \typename{frame\_descr} will be linked to a \typename{frame\_debuginfo}
        \begin{itemize}
          \item<5-> Contains information about the position in the source code (file name, line and column number)
        \end{itemize}
      \end{itemize}
    \end{column}
    \begin{column}{0.5\textwidth}
        \onslide*<1>{\listFrameDescr[highlightlines={118-122}]{}}
        \onslide*<2>{\listFrameDescr[highlightlines={120}]{}}
        \onslide*<3>{\listFrameDescr[highlightlines={121}]{}}
        \onslide*<4->{\listFrameDescr[highlightlines={122}]{}}
        \medskip
        \onslide*<1-3>{\listDebugInfo{}}
        \onslide*<4>{\listDebugInfo[highlightlines={38,49}]{}}
        \onslide*<5>{\listDebugInfo[highlightlines={39-46}]{}}
    \end{column}
  \end{columns}
\end{frame}

\begin{frame}{Backtraces}{Scanning the stack with \typename{frame\_descr}}
  \begin{columns}[c]
    \begin{column}{0.5\textwidth}
      \begin{itemize}
        \item Given the return address of the code that raises the exception by calling \funcname{caml\_raise\_exn} (\funcarg{pc} argument of \funcname{caml\_stash\_backtrace})
        \item And the position of the stack pointer (\funcarg{sp} argument of \funcname{caml\_stash\_backtrace})
        \item The frame size given by \typename{frame\_descr}.\funcarg{frame\_size}, allows to find the end of the previous frame
        \item And the return address of the caller of this function
        \bigskip
        \onslide<3->{
        \item Note that traps (here the reddish \funcname{ExnA}, \funcname{ExnB} and \funcname{ExnC} blocks) belong to the frame that installed them, and so the frame size changes when traps are removed
        }
      \end{itemize}
    \end{column}
    \begin{column}{0.5\textwidth}
      \raggedleft
      \onslide*<1>{\providecommand\step{2}\input{diagrams/backtrace/backtrace.tex}}%
      \onslide*<2>{\providecommand\step{1}\input{diagrams/backtrace/scanstack.tex}}%
      \onslide*<3>{\providecommand\step{2}\input{diagrams/backtrace/scanstack.tex}}%
      \onslide*<4>{\providecommand\step{3}\input{diagrams/backtrace/scanstack.tex}}%
      \onslide*<5>{\providecommand\step{4}\input{diagrams/backtrace/scanstack.tex}}%
      \onslide*<6>{\providecommand\step{5}\input{diagrams/backtrace/scanstack.tex}}%
    \end{column}
  \end{columns}
\end{frame}

% Duplicate of previous-previous slide
\begin{frame}{Backtraces}{Continuing on \funcname{caml\_stash\_backtrace}}
  \begin{columns}[c]
    \begin{column}{0.5\textwidth}
      \minipage[c][0.75\textheight][s]{\columnwidth}
      \begin{itemize}
          \color{gray}
        \item A C function provided by the runtime (specific to native code)
        \item It scans the stack, between the stack pointer at the raising point up to the next trap, in order to collect the names of the detected function frames
        \item It uses the same technique and information as the Garbage Collector (GC) uses to scan the values live on the stack
      \end{itemize}
      \begin{itemize}
        \item For each function frame detected, store a pointer to the \typename{frame\_descr} in the \localname{domain\_state->backtrace\_buffer} array, effectively constructing the backtrace
      \end{itemize}
      \onslide<2->{
        \begin{itemize}
            \smallskip
          \item Constructed backtrace:
            \funcname{definitely\_raise},
            \onslide<3->{\funcname{probably\_raise}}
        \end{itemize}
      }
      \vfill
      \mintocaml[firstline=9,lastline=13,highlightlines=12]{../src/backtrace/backtrace.ml}
      \endminipage
    \end{column}
    \begin{column}{0.5\textwidth}
      \raggedleft
      \onslide*<1>{\listStashBacktrace[highlightlines={114-115}]{}}
      \onslide*<2>{\providecommand\step{3}\input{diagrams/backtrace/backtrace.tex}}%
      \onslide*<3>{\providecommand\step{4}\input{diagrams/backtrace/backtrace.tex}}%
    \end{column}
  \end{columns}
\end{frame}

\begin{frame}{Backtraces}{Back to \funcname{caml\_raise\_exn}}
  \begin{columns}[c]
    \begin{column}{0.5\textwidth}
      \minipage[c][0.66\textheight][s]{\columnwidth}
      \begin{itemize}\color{gray}
        \item Checks wheteher exceptions backtrace should be recorded, by checking the \funcarg{domain\_state->backtrace\_active} value
        \item Switches to the C stack to be able to execute C code
        \item Calls \funcname{caml\_stash\_backtrace}, to collect the backtrace up to the next trap
      \end{itemize}
      \onslide*<2->{
        \begin{itemize}
          \item<2-> Executes the exception handler in \funcname{probably\_raise} with \funcname{RESTORE\_EXN\_HANDLER\_OCAML}
            \begin{itemize}
              \item Set \regname{RSP} to \localname{domain\_state->exn\_handler} value
              \item Restore \localname{domain\_state->exn\_handler} from trap
              \item Jump to the handler code with \funcname{ret}
            \end{itemize}
        \end{itemize}
      }
      \vfill
      \onslide*<1>{\mintocaml[firstline=9,lastline=13,highlightlines=12]{../src/backtrace/backtrace.ml}}
      \onslide*<2->{\mintocaml[firstline=15,lastline=21,highlightlines={19-21}]{../src/backtrace/backtrace.ml}}
      \endminipage
    \end{column}
    \begin{column}{0.5\textwidth}
      \centering
      \onslide*<1>{
        \mintc[firstline=190,lastline=192]{../ocaml/runtime/amd64.S}
        \mintc[firstline=234,lastline=237]{../ocaml/runtime/amd64.S}
      }
      \onslide*<2>{
        \mintc[firstline=190,lastline=192,highlightlines={191-192}]{../ocaml/runtime/amd64.S}
        \mintc[firstline=234,lastline=237,highlightlines={235-237}]{../ocaml/runtime/amd64.S}
      }
      \onslide*<1>{\listCamlRaiseExn[highlightlines=720]{}}
      \onslide*<2>{\listCamlRaiseExn[highlightlines=722]{}}
      \onslide*<3->{\providecommand\step{5}\input{diagrams/backtrace/backtrace.tex}}
    \end{column}
  \end{columns}
\end{frame}

\begin{frame}{Backtraces}{\funcname{caml\_probably\_raise}'s handler doesn't catch \typename{ExnC}}
  \begin{columns}[c]
    \begin{column}{0.45\textwidth}
      \minipage[c][0.75\textheight][s]{\columnwidth}
      \begin{itemize}
        \item<1-> \funcname{match \dots with}\footnotemark pattern matching can also match exceptions
        \item<1-> The CMM of \funcname{probably\_raise} shows that this \funcname{match \dots with} is implemented with a pattern matching and a \funcname{try \dots with}
        \item<1-> But this one only matches on \typename{ExnA} exception
        \item<2-> Since the pattern matching fails to catch the exception, \funcname{reraise} is called with our current \typename{ExnC} exception
        \item<3-> Disassembly shows that \funcname{reraise} is actually a call to \funcname{caml\_reraise\_exn}, which is also an assembly function provided by the runtime
      \end{itemize}
      \vfill
      \mintocaml[firstline=15,lastline=21,highlightlines={18-21}]{../src/backtrace/backtrace.ml}
      \endminipage
    \end{column}
    \begin{column}{0.55\textwidth}
      \centering
      \onslide*<1>{\mintcmm[highlightlines={10-18}]{../src/backtrace/camlBacktrace\_\_probably\_raise\_351.cmm}}
      \onslide*<2>{\listcmm[highlightlines=18]{../src/backtrace/camlBacktrace\_\_probably\_raise_351.cmm}{CMM of probably\_raise}}
      \onslide*<3>{\mintobjdump[firstline=5,lastline=35,highlightlines=31]{../src/backtrace/camlBacktrace\_\_probably\_raise_351.objdump}}
    \end{column}
  \end{columns}
  \only<1>{\footnotetext[1]{https://blog.janestreet.com/pattern-matching-and-exception-handling-unite/}}
\end{frame}

\newcommand\listCamlReraiseExn[1][]{
  \listasm[firstline=727,lastline=734,#1]{../ocaml/runtime/amd64.S}{runtime/amd64.S:727}}

\begin{frame}{Backtraces}{Introducing \funcname{caml\_reraise\_exn}}
  \begin{columns}[c]
    \begin{column}{0.5\textwidth}
      \minipage[c][0.66\textheight][s]{\columnwidth}
      \begin{itemize}
        \item<1-> The exception have to be reraised to the next handler
        \item<2-> First, checks if the backtrace should be collected with \funcarg{domain\_state->backtrace\_active}
        \only<3>{
        \begin{itemize}
            \item If not, behaves like a \funcname{raise\_notrace} with \funcname{RESTORE\_EXN\_HANDLER\_OCAML}
        \end{itemize}
        }
        \item<4-> Jumps into \funcname{caml\_raise\_exn}
        \item<5-> Calls \funcname{caml\_stash\_backtrace} again, to scan the next part of the stack: from the trap just pop-ed to the next trap
      \end{itemize}
      \vfill
      \mintocaml[firstline=15,lastline=21,highlightlines={18-21}]{../src/backtrace/backtrace.ml}
      \endminipage
    \end{column}
    \begin{column}{0.5\textwidth}
      \raggedleft
      \onslide*<1>{\listCamlReraiseExn{}}
      \onslide*<2>{\listCamlReraiseExn[highlightlines=729]{}}
      \onslide*<3>{
        \mintc[firstline=190,lastline=192,highlightlines={191-192}]{../ocaml/runtime/amd64.S}
        \mintc[firstline=234,lastline=237,highlightlines={235-237}]{../ocaml/runtime/amd64.S}
        \medskip
        \listCamlReraiseExn[highlightlines=731]{}
      }
      \onslide*<4-5>{
        % TODO refactor with \listCamlReraiseExn
        \mintasm[firstline=727,lastline=734,highlightlines=730]{../ocaml/runtime/amd64.S}
        \medskip
      }
      \onslide*<4>{\listCamlRaiseExn[highlightlines=711]{}}
      \onslide*<5>{\listCamlRaiseExn[highlightlines=720]{}}
    \end{column}
  \end{columns}
\end{frame}

\begin{frame}{Backtraces}{Backtrace collection continues}
  \begin{columns}[c]
    \begin{column}{0.55\textwidth}
      \minipage[c][0.75\textheight][s]{\columnwidth}
      \begin{itemize}
        \item<1-> \funcname{caml\_stash\_backtrace} scans the older part of the stack, up to the next trap
        \item<6-> Controls jumps to the nested exception handler in \funcname{maybe\_raise}, which matches only on \typename{ExnB}
        \item<7-> \funcname{caml\_reraise\_exn} is called again, backtrace collection resumes with \funcname{caml\_stash\_backtrace}
      \end{itemize}
      \medskip
      \begin{itemize}
        \item<2-> Constructed backtrace:
        \begin{itemize}
          \item <2-> \funcname{definitely\_raise}
          \item <2-> \funcname{probably\_raise}
          \item <3-> \funcname{probably\_raise} (re-raise)
          \item <4-> \funcname{maybe\_raise}
          \item <5-> \funcname{main}
          \item <8-> \funcname{main} (re-raise)
        \end{itemize}
      \end{itemize}
      \vfill
      \onslide*<1-5>{\mintocaml[firstline=15,lastline=21]{../src/backtrace/backtrace.ml}}
      \onslide*<6>{\mintocaml[firstline=28,lastline=34,highlightlines=31]{../src/backtrace/backtrace.ml}}
      \onslide*<7->{\mintocaml[firstline=28,lastline=34,highlightlines=33]{../src/backtrace/backtrace.ml}}
      \endminipage
    \end{column}
    \begin{column}{0.45\textwidth}
      \raggedleft
      \onslide*<1>{\providecommand\step{6}\input{diagrams/backtrace/backtrace.tex}}%
      \onslide*<2>{\providecommand\step{6}\input{diagrams/backtrace/backtrace.tex}}%
      \onslide*<3>{\providecommand\step{7}\input{diagrams/backtrace/backtrace.tex}}%
      \onslide*<4>{\providecommand\step{8}\input{diagrams/backtrace/backtrace.tex}}%
      \onslide*<5>{\providecommand\step{9}\input{diagrams/backtrace/backtrace.tex}}%
      \onslide*<6>{\providecommand\step{10}\input{diagrams/backtrace/backtrace.tex}}%
      \onslide*<7>{\providecommand\step{11}\input{diagrams/backtrace/backtrace.tex}}%
      \onslide*<8>{\providecommand\step{12}\input{diagrams/backtrace/backtrace.tex}}%
      \onslide*<9>{\providecommand\step{13}\input{diagrams/backtrace/backtrace.tex}}%
    \end{column}
  \end{columns}
\end{frame}

\begin{frame}{Backtraces}{Printing the backtrace}
  \begin{columns}[c]
    \begin{column}{0.45\textwidth}
      \minipage[c][0.66\textheight][s]{\columnwidth}
      \begin{itemize}
        \item<1-> The exception backtrace can now be printed to \funcarg{stdout} with \funcname{Printexc.print_backtrace}
        \item<2-> First the exception backtrace is retrieved into an OCaml array with \funcname{get_raw_backtrace }
        \item<3-> Which is implemented as the \funcname{caml_get_exception_raw_backtrace} C function in the runtime
        \item<4-> And finally printed with \funcname{print_exception_backtrace}
      \end{itemize}
      \vfill
      \mintocaml[firstline=28,lastline=34,highlightlines=33]{../src/backtrace/backtrace.ml}
      \endminipage
    \end{column}
    \begin{column}{0.55\textwidth}
      \onslide*<1,2>{
        \mintocaml[firstline=100,lastline=101]{../ocaml/stdlib/printexc.ml}
      }
      \only<1>{
        \listocaml[firstline=170,lastline=172]{../ocaml/stdlib/printexc.ml}{stdlib/printexc.ml:170}
      }
      \only<2>{
        \listocaml[firstline=170,lastline=172,highlightlines=172]{../ocaml/stdlib/printexc.ml}{stdlib/printexc.ml:170}
      }
      \only<3>{
        \mintc[firstline=154,lastline=156]{../ocaml/runtime/backtrace.c}
        \listc[firstline=171,lastline=192,highlightlines={184-187}]{../ocaml/runtime/backtrace.c}{runtime/backtrace.c:154}
      }
      \only<4>{
        \listocaml[firstline=155,lastline=165]{../ocaml/stdlib/printexc.ml}{stdlib/printexc.ml:55}
      }
    \end{column}
  \end{columns}
\end{frame}


\subsection{The multiple kinds of raise}
\frameSubsection{}{}

\begin{frame}{Backtraces}{When backtraces are not needed}
  \begin{columns}[c]
    \begin{column}{0.45\textwidth}
      \minipage[c][0.66\textheight][s]{\columnwidth}
      \begin{itemize}
        \item<1-> What if the backtrace for an exception is never needed?
        \item<2-> It's possible to raise the exception explicitly with \funcname{raise\_notrace} to make raising as cheap as possible
        \item<3-> An example of use in the \funcname{dynlink} library of the OCaml compiler
      \end{itemize}
      \vfill
      \onslide<3->{
      \listocaml[firstline=205,lastline=209,highlightlines=207]{../ocaml/otherlibs/dynlink/dynlink\_compilerlibs/misc.ml}{otherlibs/dynlink/dynlink\_compilerlibs/misc.ml:205}
      }
      \endminipage
    \end{column}
    \begin{column}{0.55\textwidth}
    \only<4>{
      \mintcmm[highlightlines=3]{../src/notrace/camlNotrace\_\_fun_347.cmm}
      \listcmm{../src/notrace/camlNotrace\_\_all\_somes\_327.cmm}{all\_somes}
    }
    \onslide*<5->{
      \listobjdump[highlightlines={6-9}]{../src/notrace/camlNotrace\_\_fun_347.objdump}{anonymous function from \funcname{all\_somes}}
    }
    \end{column}
  \end{columns}
\end{frame}

\begin{frame}{Backtraces}{Summary table of the multiple kinds of raise}
  \begin{itemize}
    \item When debugging information is enabled and if the backtrace of an exception isn't needed, it's possible to raise with \funcname{raise\_notrace} for performance
    \item When debugging information is \emph{not} enabled, the compiler optimises \funcname{raise} calls to inline assembly (same effect as using \funcname{raise\_notrace})
  \end{itemize}
  \bigskip
  \centering
  \begin{tabular}{| l | c | c | c | c |}
    \hline
    \parbox{10em}{Debug information \\ (\funcarg{-g} flag)}  & \multicolumn{2}{|c|}{Disabled}                                     & \multicolumn{2}{|c|}{Enabled} \\
    \hline
    Raise kind                                               & \funcname{raise} & \funcname{raise\_notrace}                       & \funcname{raise} & \funcname{raise\_notrace} \\
    \hline
    Compilation                                              & \parbox{8em}{inline assembly\\ (optimization)} & inline assembly   & \funcname{caml\_raise\_exn} & inline assembly \\
    \hline
  \end{tabular}
\end{frame}


%
% Takeaway #5
%
\subsection*{Takeaway \#5}
\frameSubsectionTakeaway{}
\againframe<5>{takeaway}

    \end{column}
  \end{columns}
\end{frame}

\begin{frame}{Backtraces}{But before that, we need more fields from \typename{caml\_domain\_state}}
  \begin{columns}
    \begin{column}{0.5\textwidth}
      \begin{itemize}
        \item Remember \typename{caml\_domain\_state} the per-domain runtime state?
        \item Some other members are of interest to us now\dots
        \item \localname{backtrace\_active} is used to know if the runtime should collect backtraces
        \item \localname{backtrace\_active} can be set by
          \begin{itemize}
            \item The \funcarg{b} flag in the \funcname{OCAMLRUNPARAM} environment variable
            \item \mintinline{ocaml}{Printexc.record_backtrace : bool -> unit} \footnotemark
          \end{itemize}
      \end{itemize}
    \end{column}
    \begin{column}{0.5\textwidth}
      \begin{listing}
        \mintc[highlightlines={9-11}]{src/ocaml/domain\_state\_backtrace.h}
        \caption{runtime/caml/domain\_state.h}
      \end{listing}
    \end{column}
  \end{columns}
  \footnotetext[1]{https://v2.ocaml.org/api/Printexc.html}
\end{frame}

\newcommand\listCamlRaiseExn[1][]{
  \listasm[firstline=702,lastline=725,#1]{../ocaml/runtime/amd64.S}{runtime/amd64.S:702}}

\begin{frame}{Backtraces}{Walk through \funcname{caml\_raise\_exn}}
  \begin{columns}[T]
    \begin{column}{0.5\textwidth}
      \begin{itemize}
        \item<1-> First checks whether exceptions backtrace should be recorded
        \item<1-> By checking the \funcarg{domain\_state->backtrace\_active} value
        \item<2-> If not, acts as \funcname{raise\_notrace} with \funcname{RESTORE\_EXN\_HANDLER\_OCAML}
          \begin{itemize}
            \item Set \regname{RSP} to \localname{domain\_state->exn\_handler} value
            \item Restore \localname{domain\_state->exn\_handler} from trap
            \item Jump with \funcname{ret} (same effect as using \funcname{pop} and \funcname{jmp})
          \end{itemize}
        \item<3-> Otherwise, switches to the C stack to be able to execute C code
        \item<4-> Calls \funcname{caml\_stash\_backtrace}, a C function provided by the runtime\ignorespaces
          \onslide<6->{, with
            \begin{itemize}
              \item \funcarg{exn}: exception value
              \item \funcarg{pc}: code address of the raising point
              \item \funcarg{sp}: stack address of the raising function
              \item \funcarg{trapsp}: stack address of the next handler
            \end{itemize}
          }
      \end{itemize}
      \bigskip
      \mintocaml[firstline=9,lastline=13,highlightlines=12]{../src/backtrace/backtrace.ml}
    \end{column}
    \begin{column}{0.5\textwidth}
      \raggedleft
      \onslide*<1,3-4>{
        \mintc[firstline=190,lastline=192]{../ocaml/runtime/amd64.S}
        \mintc[firstline=234,lastline=237]{../ocaml/runtime/amd64.S}
      }
      \onslide*<2>{
        \mintc[firstline=190,lastline=192,highlightlines={191-192}]{../ocaml/runtime/amd64.S}
        \mintc[firstline=234,lastline=237,highlightlines={235-237}]{../ocaml/runtime/amd64.S}
      }
      \onslide*<1>{\listCamlRaiseExn[highlightlines=705]{}}%
      \onslide*<2>{\listCamlRaiseExn[highlightlines={707-708}]{}}%
      \onslide*<3>{\listCamlRaiseExn[highlightlines={712-714}]{}}%
      \onslide*<4>{\listCamlRaiseExn[highlightlines={715-720}]{}}%
      \onslide*<5>{\providecommand\step{1}%
% Backtraces
%
\subsection{One advantage of raising with debugging information}
\frameSubsection{}{}

\begin{frame}{Backtraces}{Debugging information \& source code}
  \begin{columns}[c]
    \begin{column}{0.5\textwidth}
      \begin{itemize}
        \item Raising an exception is fun and games, but how do we know where the exception originated? $\rightarrow$ With the backtrace associated with the exception!
        \item In order to get a backtrace from the point where an exception was raised, it's necessary to compile with debugging information enabled (\funcarg{-g} flag for the compiler)
        \item Same example as in the `Nested handlers' section
      \end{itemize}
    \end{column}
    \begin{column}{0.5\textwidth}
      \listocaml[firstline=9,lastline=34]{../src/backtrace/backtrace.ml}{backtrace/backtrace.ml}
    \end{column}
  \end{columns}
\end{frame}

\begin{frame}{Backtraces}{CMM and disassembly of \funcname{definitely\_raise}}
  \begin{columns}[c]
    \begin{column}{0.5\textwidth}
      \minipage[c][0.66\textheight][s]{\columnwidth}
      \begin{itemize}
        \item<1-> An effect of compiling with debugging information (\funcarg{-g}): the CMM no longer uses \funcname{raise\_notrace} to raise the exception but \funcname{raise} instead
        \item<2-> Disassembly shows that CMM \funcname{raise} is actually a call to \funcname{caml\_raise\_exn}, a function in assembler provided by the runtime
      \end{itemize}
      \vfill
      \mintocaml[firstline=9,lastline=13,highlightlines=12]{../src/backtrace/backtrace.ml}
      \endminipage
    \end{column}
    \begin{column}{0.5\textwidth}
      \onslide*<1>{
        \mintcmm[highlightlines=5]{../src/backtrace/camlBacktrace\_\_definitely\_raise\_347.cmm}
      }
      \onslide*<2>{
        \mintobjdump[firstline=2,highlightlines=14]{../src/backtrace/camlBacktrace\_\_definitely\_raise\_347.objdump}
      }
    \end{column}
  \end{columns}
\end{frame}

\begin{frame}{Backtraces}{Introducing \funcname{caml\_raise\_exn}}
  \begin{columns}[c]
    \begin{column}{0.5\textwidth}
      \minipage[c][0.66\textheight][s]{\columnwidth}
      \onslide*<1>{
        \begin{itemize}
          \item Raising the exception is performed using \funcname{caml\_raise\_exn} which is
            \begin{itemize}
              \item An assembly function provided by the runtime
              \item Architecture specific
            \end{itemize}
          \item Let's see what it does differently from \funcname{raise\_notrace}\dots
          \item We are about to raise \typename{ExnC} with \funcname{caml\_raise\_exn} from \funcname{definitely\_raise}
        \end{itemize}
      }
      \onslide*<2>{
        \mintobjdump[firstline=2,highlightlines=14]{../src/backtrace/camlBacktrace\_\_definitely\_raise\_347.objdump}
      }
      \vfill
      \mintocaml[firstline=9,lastline=13,highlightlines=12]{../src/backtrace/backtrace.ml}
      \endminipage
    \end{column}
    \begin{column}{0.5\textwidth}
      \centering
      \providecommand\step{1}\input{diagrams/backtrace/backtrace.tex}
    \end{column}
  \end{columns}
\end{frame}

\begin{frame}{Backtraces}{But before that, we need more fields from \typename{caml\_domain\_state}}
  \begin{columns}
    \begin{column}{0.5\textwidth}
      \begin{itemize}
        \item Remember \typename{caml\_domain\_state} the per-domain runtime state?
        \item Some other members are of interest to us now\dots
        \item \localname{backtrace\_active} is used to know if the runtime should collect backtraces
        \item \localname{backtrace\_active} can be set by
          \begin{itemize}
            \item The \funcarg{b} flag in the \funcname{OCAMLRUNPARAM} environment variable
            \item \mintinline{ocaml}{Printexc.record_backtrace : bool -> unit} \footnotemark
          \end{itemize}
      \end{itemize}
    \end{column}
    \begin{column}{0.5\textwidth}
      \begin{listing}
        \mintc[highlightlines={9-11}]{src/ocaml/domain\_state\_backtrace.h}
        \caption{runtime/caml/domain\_state.h}
      \end{listing}
    \end{column}
  \end{columns}
  \footnotetext[1]{https://v2.ocaml.org/api/Printexc.html}
\end{frame}

\newcommand\listCamlRaiseExn[1][]{
  \listasm[firstline=702,lastline=725,#1]{../ocaml/runtime/amd64.S}{runtime/amd64.S:702}}

\begin{frame}{Backtraces}{Walk through \funcname{caml\_raise\_exn}}
  \begin{columns}[T]
    \begin{column}{0.5\textwidth}
      \begin{itemize}
        \item<1-> First checks whether exceptions backtrace should be recorded
        \item<1-> By checking the \funcarg{domain\_state->backtrace\_active} value
        \item<2-> If not, acts as \funcname{raise\_notrace} with \funcname{RESTORE\_EXN\_HANDLER\_OCAML}
          \begin{itemize}
            \item Set \regname{RSP} to \localname{domain\_state->exn\_handler} value
            \item Restore \localname{domain\_state->exn\_handler} from trap
            \item Jump with \funcname{ret} (same effect as using \funcname{pop} and \funcname{jmp})
          \end{itemize}
        \item<3-> Otherwise, switches to the C stack to be able to execute C code
        \item<4-> Calls \funcname{caml\_stash\_backtrace}, a C function provided by the runtime\ignorespaces
          \onslide<6->{, with
            \begin{itemize}
              \item \funcarg{exn}: exception value
              \item \funcarg{pc}: code address of the raising point
              \item \funcarg{sp}: stack address of the raising function
              \item \funcarg{trapsp}: stack address of the next handler
            \end{itemize}
          }
      \end{itemize}
      \bigskip
      \mintocaml[firstline=9,lastline=13,highlightlines=12]{../src/backtrace/backtrace.ml}
    \end{column}
    \begin{column}{0.5\textwidth}
      \raggedleft
      \onslide*<1,3-4>{
        \mintc[firstline=190,lastline=192]{../ocaml/runtime/amd64.S}
        \mintc[firstline=234,lastline=237]{../ocaml/runtime/amd64.S}
      }
      \onslide*<2>{
        \mintc[firstline=190,lastline=192,highlightlines={191-192}]{../ocaml/runtime/amd64.S}
        \mintc[firstline=234,lastline=237,highlightlines={235-237}]{../ocaml/runtime/amd64.S}
      }
      \onslide*<1>{\listCamlRaiseExn[highlightlines=705]{}}%
      \onslide*<2>{\listCamlRaiseExn[highlightlines={707-708}]{}}%
      \onslide*<3>{\listCamlRaiseExn[highlightlines={712-714}]{}}%
      \onslide*<4>{\listCamlRaiseExn[highlightlines={715-720}]{}}%
      \onslide*<5>{\providecommand\step{1}\input{diagrams/backtrace/backtrace.tex}}%
      \onslide*<6>{\providecommand\step{2}\input{diagrams/backtrace/backtrace.tex}}%
    \end{column}
  \end{columns}
\end{frame}

\newcommand\listStashBacktrace[1][]{
  \listc[firstline=91,lastline=120,#1]{../ocaml/runtime/backtrace\_nat.c}{runtime/backtrace\_nat.c:91}}

\begin{frame}{Backtraces}{Introducing \funcname{caml\_stash\_backtrace}}
  \begin{columns}[c]
    \begin{column}{0.5\textwidth}
      \minipage[c][0.66\textheight][s]{\columnwidth}
      \begin{itemize}
        \item<1-> A C function provided by the runtime (specific to native code)
        \item<2-> It scans the stack, between the stack pointer at the raising point up to the next trap, in order to collect the names of the detected function frames
          \onslide*<2>{
            \begin{itemize}
              \item And more information, like line number, column number...
            \end{itemize}
          }
        \item<3-> It uses the same technique and information as the Garbage Collector (GC) uses to scan the values live on the stack
          \onslide*<4>{
            \begin{itemize}
              \item \typename{frame\_descr} are at the core of stack unwinding
            \end{itemize}
          }
      \end{itemize}
      \vfill
      \mintocaml[firstline=9,lastline=13,highlightlines=12]{../src/backtrace/backtrace.ml}
      \endminipage
    \end{column}
    \begin{column}{0.5\textwidth}
      \onslide*<1>{\listStashBacktrace[highlightlines=91]{}}
      \onslide*<2>{\listStashBacktrace[highlightlines={91,117-118}]{}}
      \onslide*<3->{\listStashBacktrace[highlightlines={94,106,109-110}]{}}
    \end{column}
  \end{columns}
\end{frame}

\newcommand\listFrameDescr[1][]{
  \listocaml[firstline=111,lastline=124,#1]{../ocaml/asmcomp/emitaux.ml}{asmcomp/emitaux.ml:111}}
\newcommand\listDebugInfo[1][]{
  \listocaml[firstline=38,lastline=49,#1]{../ocaml/lambda/debuginfo.mli}{lambda/debuginfo.mli:38}}

\begin{frame}{Backtraces}{\typename{frame\_descr} and \typename{frame\_debuginfo} in a nutshell}
  \begin{columns}[c]
    \begin{column}{0.5\textwidth}
      \begin{itemize}
        \item<1-> For every return address in an OCaml program, there is a associated \typename{frame\_descr}
        \item<2-> A \typename{frame\_descr} specifies
        \begin{itemize}
          \item<2-> The size of the function stack frame
          \item<3-> Where live values are on the stack (so that the GC can mark them as live and prevent from being swept)
        \end{itemize}
        \item<4-> When compiled with debug information, \typename{frame\_descr} will be linked to a \typename{frame\_debuginfo}
        \begin{itemize}
          \item<5-> Contains information about the position in the source code (file name, line and column number)
        \end{itemize}
      \end{itemize}
    \end{column}
    \begin{column}{0.5\textwidth}
        \onslide*<1>{\listFrameDescr[highlightlines={118-122}]{}}
        \onslide*<2>{\listFrameDescr[highlightlines={120}]{}}
        \onslide*<3>{\listFrameDescr[highlightlines={121}]{}}
        \onslide*<4->{\listFrameDescr[highlightlines={122}]{}}
        \medskip
        \onslide*<1-3>{\listDebugInfo{}}
        \onslide*<4>{\listDebugInfo[highlightlines={38,49}]{}}
        \onslide*<5>{\listDebugInfo[highlightlines={39-46}]{}}
    \end{column}
  \end{columns}
\end{frame}

\begin{frame}{Backtraces}{Scanning the stack with \typename{frame\_descr}}
  \begin{columns}[c]
    \begin{column}{0.5\textwidth}
      \begin{itemize}
        \item Given the return address of the code that raises the exception by calling \funcname{caml\_raise\_exn} (\funcarg{pc} argument of \funcname{caml\_stash\_backtrace})
        \item And the position of the stack pointer (\funcarg{sp} argument of \funcname{caml\_stash\_backtrace})
        \item The frame size given by \typename{frame\_descr}.\funcarg{frame\_size}, allows to find the end of the previous frame
        \item And the return address of the caller of this function
        \bigskip
        \onslide<3->{
        \item Note that traps (here the reddish \funcname{ExnA}, \funcname{ExnB} and \funcname{ExnC} blocks) belong to the frame that installed them, and so the frame size changes when traps are removed
        }
      \end{itemize}
    \end{column}
    \begin{column}{0.5\textwidth}
      \raggedleft
      \onslide*<1>{\providecommand\step{2}\input{diagrams/backtrace/backtrace.tex}}%
      \onslide*<2>{\providecommand\step{1}\input{diagrams/backtrace/scanstack.tex}}%
      \onslide*<3>{\providecommand\step{2}\input{diagrams/backtrace/scanstack.tex}}%
      \onslide*<4>{\providecommand\step{3}\input{diagrams/backtrace/scanstack.tex}}%
      \onslide*<5>{\providecommand\step{4}\input{diagrams/backtrace/scanstack.tex}}%
      \onslide*<6>{\providecommand\step{5}\input{diagrams/backtrace/scanstack.tex}}%
    \end{column}
  \end{columns}
\end{frame}

% Duplicate of previous-previous slide
\begin{frame}{Backtraces}{Continuing on \funcname{caml\_stash\_backtrace}}
  \begin{columns}[c]
    \begin{column}{0.5\textwidth}
      \minipage[c][0.75\textheight][s]{\columnwidth}
      \begin{itemize}
          \color{gray}
        \item A C function provided by the runtime (specific to native code)
        \item It scans the stack, between the stack pointer at the raising point up to the next trap, in order to collect the names of the detected function frames
        \item It uses the same technique and information as the Garbage Collector (GC) uses to scan the values live on the stack
      \end{itemize}
      \begin{itemize}
        \item For each function frame detected, store a pointer to the \typename{frame\_descr} in the \localname{domain\_state->backtrace\_buffer} array, effectively constructing the backtrace
      \end{itemize}
      \onslide<2->{
        \begin{itemize}
            \smallskip
          \item Constructed backtrace:
            \funcname{definitely\_raise},
            \onslide<3->{\funcname{probably\_raise}}
        \end{itemize}
      }
      \vfill
      \mintocaml[firstline=9,lastline=13,highlightlines=12]{../src/backtrace/backtrace.ml}
      \endminipage
    \end{column}
    \begin{column}{0.5\textwidth}
      \raggedleft
      \onslide*<1>{\listStashBacktrace[highlightlines={114-115}]{}}
      \onslide*<2>{\providecommand\step{3}\input{diagrams/backtrace/backtrace.tex}}%
      \onslide*<3>{\providecommand\step{4}\input{diagrams/backtrace/backtrace.tex}}%
    \end{column}
  \end{columns}
\end{frame}

\begin{frame}{Backtraces}{Back to \funcname{caml\_raise\_exn}}
  \begin{columns}[c]
    \begin{column}{0.5\textwidth}
      \minipage[c][0.66\textheight][s]{\columnwidth}
      \begin{itemize}\color{gray}
        \item Checks wheteher exceptions backtrace should be recorded, by checking the \funcarg{domain\_state->backtrace\_active} value
        \item Switches to the C stack to be able to execute C code
        \item Calls \funcname{caml\_stash\_backtrace}, to collect the backtrace up to the next trap
      \end{itemize}
      \onslide*<2->{
        \begin{itemize}
          \item<2-> Executes the exception handler in \funcname{probably\_raise} with \funcname{RESTORE\_EXN\_HANDLER\_OCAML}
            \begin{itemize}
              \item Set \regname{RSP} to \localname{domain\_state->exn\_handler} value
              \item Restore \localname{domain\_state->exn\_handler} from trap
              \item Jump to the handler code with \funcname{ret}
            \end{itemize}
        \end{itemize}
      }
      \vfill
      \onslide*<1>{\mintocaml[firstline=9,lastline=13,highlightlines=12]{../src/backtrace/backtrace.ml}}
      \onslide*<2->{\mintocaml[firstline=15,lastline=21,highlightlines={19-21}]{../src/backtrace/backtrace.ml}}
      \endminipage
    \end{column}
    \begin{column}{0.5\textwidth}
      \centering
      \onslide*<1>{
        \mintc[firstline=190,lastline=192]{../ocaml/runtime/amd64.S}
        \mintc[firstline=234,lastline=237]{../ocaml/runtime/amd64.S}
      }
      \onslide*<2>{
        \mintc[firstline=190,lastline=192,highlightlines={191-192}]{../ocaml/runtime/amd64.S}
        \mintc[firstline=234,lastline=237,highlightlines={235-237}]{../ocaml/runtime/amd64.S}
      }
      \onslide*<1>{\listCamlRaiseExn[highlightlines=720]{}}
      \onslide*<2>{\listCamlRaiseExn[highlightlines=722]{}}
      \onslide*<3->{\providecommand\step{5}\input{diagrams/backtrace/backtrace.tex}}
    \end{column}
  \end{columns}
\end{frame}

\begin{frame}{Backtraces}{\funcname{caml\_probably\_raise}'s handler doesn't catch \typename{ExnC}}
  \begin{columns}[c]
    \begin{column}{0.45\textwidth}
      \minipage[c][0.75\textheight][s]{\columnwidth}
      \begin{itemize}
        \item<1-> \funcname{match \dots with}\footnotemark pattern matching can also match exceptions
        \item<1-> The CMM of \funcname{probably\_raise} shows that this \funcname{match \dots with} is implemented with a pattern matching and a \funcname{try \dots with}
        \item<1-> But this one only matches on \typename{ExnA} exception
        \item<2-> Since the pattern matching fails to catch the exception, \funcname{reraise} is called with our current \typename{ExnC} exception
        \item<3-> Disassembly shows that \funcname{reraise} is actually a call to \funcname{caml\_reraise\_exn}, which is also an assembly function provided by the runtime
      \end{itemize}
      \vfill
      \mintocaml[firstline=15,lastline=21,highlightlines={18-21}]{../src/backtrace/backtrace.ml}
      \endminipage
    \end{column}
    \begin{column}{0.55\textwidth}
      \centering
      \onslide*<1>{\mintcmm[highlightlines={10-18}]{../src/backtrace/camlBacktrace\_\_probably\_raise\_351.cmm}}
      \onslide*<2>{\listcmm[highlightlines=18]{../src/backtrace/camlBacktrace\_\_probably\_raise_351.cmm}{CMM of probably\_raise}}
      \onslide*<3>{\mintobjdump[firstline=5,lastline=35,highlightlines=31]{../src/backtrace/camlBacktrace\_\_probably\_raise_351.objdump}}
    \end{column}
  \end{columns}
  \only<1>{\footnotetext[1]{https://blog.janestreet.com/pattern-matching-and-exception-handling-unite/}}
\end{frame}

\newcommand\listCamlReraiseExn[1][]{
  \listasm[firstline=727,lastline=734,#1]{../ocaml/runtime/amd64.S}{runtime/amd64.S:727}}

\begin{frame}{Backtraces}{Introducing \funcname{caml\_reraise\_exn}}
  \begin{columns}[c]
    \begin{column}{0.5\textwidth}
      \minipage[c][0.66\textheight][s]{\columnwidth}
      \begin{itemize}
        \item<1-> The exception have to be reraised to the next handler
        \item<2-> First, checks if the backtrace should be collected with \funcarg{domain\_state->backtrace\_active}
        \only<3>{
        \begin{itemize}
            \item If not, behaves like a \funcname{raise\_notrace} with \funcname{RESTORE\_EXN\_HANDLER\_OCAML}
        \end{itemize}
        }
        \item<4-> Jumps into \funcname{caml\_raise\_exn}
        \item<5-> Calls \funcname{caml\_stash\_backtrace} again, to scan the next part of the stack: from the trap just pop-ed to the next trap
      \end{itemize}
      \vfill
      \mintocaml[firstline=15,lastline=21,highlightlines={18-21}]{../src/backtrace/backtrace.ml}
      \endminipage
    \end{column}
    \begin{column}{0.5\textwidth}
      \raggedleft
      \onslide*<1>{\listCamlReraiseExn{}}
      \onslide*<2>{\listCamlReraiseExn[highlightlines=729]{}}
      \onslide*<3>{
        \mintc[firstline=190,lastline=192,highlightlines={191-192}]{../ocaml/runtime/amd64.S}
        \mintc[firstline=234,lastline=237,highlightlines={235-237}]{../ocaml/runtime/amd64.S}
        \medskip
        \listCamlReraiseExn[highlightlines=731]{}
      }
      \onslide*<4-5>{
        % TODO refactor with \listCamlReraiseExn
        \mintasm[firstline=727,lastline=734,highlightlines=730]{../ocaml/runtime/amd64.S}
        \medskip
      }
      \onslide*<4>{\listCamlRaiseExn[highlightlines=711]{}}
      \onslide*<5>{\listCamlRaiseExn[highlightlines=720]{}}
    \end{column}
  \end{columns}
\end{frame}

\begin{frame}{Backtraces}{Backtrace collection continues}
  \begin{columns}[c]
    \begin{column}{0.55\textwidth}
      \minipage[c][0.75\textheight][s]{\columnwidth}
      \begin{itemize}
        \item<1-> \funcname{caml\_stash\_backtrace} scans the older part of the stack, up to the next trap
        \item<6-> Controls jumps to the nested exception handler in \funcname{maybe\_raise}, which matches only on \typename{ExnB}
        \item<7-> \funcname{caml\_reraise\_exn} is called again, backtrace collection resumes with \funcname{caml\_stash\_backtrace}
      \end{itemize}
      \medskip
      \begin{itemize}
        \item<2-> Constructed backtrace:
        \begin{itemize}
          \item <2-> \funcname{definitely\_raise}
          \item <2-> \funcname{probably\_raise}
          \item <3-> \funcname{probably\_raise} (re-raise)
          \item <4-> \funcname{maybe\_raise}
          \item <5-> \funcname{main}
          \item <8-> \funcname{main} (re-raise)
        \end{itemize}
      \end{itemize}
      \vfill
      \onslide*<1-5>{\mintocaml[firstline=15,lastline=21]{../src/backtrace/backtrace.ml}}
      \onslide*<6>{\mintocaml[firstline=28,lastline=34,highlightlines=31]{../src/backtrace/backtrace.ml}}
      \onslide*<7->{\mintocaml[firstline=28,lastline=34,highlightlines=33]{../src/backtrace/backtrace.ml}}
      \endminipage
    \end{column}
    \begin{column}{0.45\textwidth}
      \raggedleft
      \onslide*<1>{\providecommand\step{6}\input{diagrams/backtrace/backtrace.tex}}%
      \onslide*<2>{\providecommand\step{6}\input{diagrams/backtrace/backtrace.tex}}%
      \onslide*<3>{\providecommand\step{7}\input{diagrams/backtrace/backtrace.tex}}%
      \onslide*<4>{\providecommand\step{8}\input{diagrams/backtrace/backtrace.tex}}%
      \onslide*<5>{\providecommand\step{9}\input{diagrams/backtrace/backtrace.tex}}%
      \onslide*<6>{\providecommand\step{10}\input{diagrams/backtrace/backtrace.tex}}%
      \onslide*<7>{\providecommand\step{11}\input{diagrams/backtrace/backtrace.tex}}%
      \onslide*<8>{\providecommand\step{12}\input{diagrams/backtrace/backtrace.tex}}%
      \onslide*<9>{\providecommand\step{13}\input{diagrams/backtrace/backtrace.tex}}%
    \end{column}
  \end{columns}
\end{frame}

\begin{frame}{Backtraces}{Printing the backtrace}
  \begin{columns}[c]
    \begin{column}{0.45\textwidth}
      \minipage[c][0.66\textheight][s]{\columnwidth}
      \begin{itemize}
        \item<1-> The exception backtrace can now be printed to \funcarg{stdout} with \funcname{Printexc.print_backtrace}
        \item<2-> First the exception backtrace is retrieved into an OCaml array with \funcname{get_raw_backtrace }
        \item<3-> Which is implemented as the \funcname{caml_get_exception_raw_backtrace} C function in the runtime
        \item<4-> And finally printed with \funcname{print_exception_backtrace}
      \end{itemize}
      \vfill
      \mintocaml[firstline=28,lastline=34,highlightlines=33]{../src/backtrace/backtrace.ml}
      \endminipage
    \end{column}
    \begin{column}{0.55\textwidth}
      \onslide*<1,2>{
        \mintocaml[firstline=100,lastline=101]{../ocaml/stdlib/printexc.ml}
      }
      \only<1>{
        \listocaml[firstline=170,lastline=172]{../ocaml/stdlib/printexc.ml}{stdlib/printexc.ml:170}
      }
      \only<2>{
        \listocaml[firstline=170,lastline=172,highlightlines=172]{../ocaml/stdlib/printexc.ml}{stdlib/printexc.ml:170}
      }
      \only<3>{
        \mintc[firstline=154,lastline=156]{../ocaml/runtime/backtrace.c}
        \listc[firstline=171,lastline=192,highlightlines={184-187}]{../ocaml/runtime/backtrace.c}{runtime/backtrace.c:154}
      }
      \only<4>{
        \listocaml[firstline=155,lastline=165]{../ocaml/stdlib/printexc.ml}{stdlib/printexc.ml:55}
      }
    \end{column}
  \end{columns}
\end{frame}


\subsection{The multiple kinds of raise}
\frameSubsection{}{}

\begin{frame}{Backtraces}{When backtraces are not needed}
  \begin{columns}[c]
    \begin{column}{0.45\textwidth}
      \minipage[c][0.66\textheight][s]{\columnwidth}
      \begin{itemize}
        \item<1-> What if the backtrace for an exception is never needed?
        \item<2-> It's possible to raise the exception explicitly with \funcname{raise\_notrace} to make raising as cheap as possible
        \item<3-> An example of use in the \funcname{dynlink} library of the OCaml compiler
      \end{itemize}
      \vfill
      \onslide<3->{
      \listocaml[firstline=205,lastline=209,highlightlines=207]{../ocaml/otherlibs/dynlink/dynlink\_compilerlibs/misc.ml}{otherlibs/dynlink/dynlink\_compilerlibs/misc.ml:205}
      }
      \endminipage
    \end{column}
    \begin{column}{0.55\textwidth}
    \only<4>{
      \mintcmm[highlightlines=3]{../src/notrace/camlNotrace\_\_fun_347.cmm}
      \listcmm{../src/notrace/camlNotrace\_\_all\_somes\_327.cmm}{all\_somes}
    }
    \onslide*<5->{
      \listobjdump[highlightlines={6-9}]{../src/notrace/camlNotrace\_\_fun_347.objdump}{anonymous function from \funcname{all\_somes}}
    }
    \end{column}
  \end{columns}
\end{frame}

\begin{frame}{Backtraces}{Summary table of the multiple kinds of raise}
  \begin{itemize}
    \item When debugging information is enabled and if the backtrace of an exception isn't needed, it's possible to raise with \funcname{raise\_notrace} for performance
    \item When debugging information is \emph{not} enabled, the compiler optimises \funcname{raise} calls to inline assembly (same effect as using \funcname{raise\_notrace})
  \end{itemize}
  \bigskip
  \centering
  \begin{tabular}{| l | c | c | c | c |}
    \hline
    \parbox{10em}{Debug information \\ (\funcarg{-g} flag)}  & \multicolumn{2}{|c|}{Disabled}                                     & \multicolumn{2}{|c|}{Enabled} \\
    \hline
    Raise kind                                               & \funcname{raise} & \funcname{raise\_notrace}                       & \funcname{raise} & \funcname{raise\_notrace} \\
    \hline
    Compilation                                              & \parbox{8em}{inline assembly\\ (optimization)} & inline assembly   & \funcname{caml\_raise\_exn} & inline assembly \\
    \hline
  \end{tabular}
\end{frame}


%
% Takeaway #5
%
\subsection*{Takeaway \#5}
\frameSubsectionTakeaway{}
\againframe<5>{takeaway}
}%
      \onslide*<6>{\providecommand\step{2}%
% Backtraces
%
\subsection{One advantage of raising with debugging information}
\frameSubsection{}{}

\begin{frame}{Backtraces}{Debugging information \& source code}
  \begin{columns}[c]
    \begin{column}{0.5\textwidth}
      \begin{itemize}
        \item Raising an exception is fun and games, but how do we know where the exception originated? $\rightarrow$ With the backtrace associated with the exception!
        \item In order to get a backtrace from the point where an exception was raised, it's necessary to compile with debugging information enabled (\funcarg{-g} flag for the compiler)
        \item Same example as in the `Nested handlers' section
      \end{itemize}
    \end{column}
    \begin{column}{0.5\textwidth}
      \listocaml[firstline=9,lastline=34]{../src/backtrace/backtrace.ml}{backtrace/backtrace.ml}
    \end{column}
  \end{columns}
\end{frame}

\begin{frame}{Backtraces}{CMM and disassembly of \funcname{definitely\_raise}}
  \begin{columns}[c]
    \begin{column}{0.5\textwidth}
      \minipage[c][0.66\textheight][s]{\columnwidth}
      \begin{itemize}
        \item<1-> An effect of compiling with debugging information (\funcarg{-g}): the CMM no longer uses \funcname{raise\_notrace} to raise the exception but \funcname{raise} instead
        \item<2-> Disassembly shows that CMM \funcname{raise} is actually a call to \funcname{caml\_raise\_exn}, a function in assembler provided by the runtime
      \end{itemize}
      \vfill
      \mintocaml[firstline=9,lastline=13,highlightlines=12]{../src/backtrace/backtrace.ml}
      \endminipage
    \end{column}
    \begin{column}{0.5\textwidth}
      \onslide*<1>{
        \mintcmm[highlightlines=5]{../src/backtrace/camlBacktrace\_\_definitely\_raise\_347.cmm}
      }
      \onslide*<2>{
        \mintobjdump[firstline=2,highlightlines=14]{../src/backtrace/camlBacktrace\_\_definitely\_raise\_347.objdump}
      }
    \end{column}
  \end{columns}
\end{frame}

\begin{frame}{Backtraces}{Introducing \funcname{caml\_raise\_exn}}
  \begin{columns}[c]
    \begin{column}{0.5\textwidth}
      \minipage[c][0.66\textheight][s]{\columnwidth}
      \onslide*<1>{
        \begin{itemize}
          \item Raising the exception is performed using \funcname{caml\_raise\_exn} which is
            \begin{itemize}
              \item An assembly function provided by the runtime
              \item Architecture specific
            \end{itemize}
          \item Let's see what it does differently from \funcname{raise\_notrace}\dots
          \item We are about to raise \typename{ExnC} with \funcname{caml\_raise\_exn} from \funcname{definitely\_raise}
        \end{itemize}
      }
      \onslide*<2>{
        \mintobjdump[firstline=2,highlightlines=14]{../src/backtrace/camlBacktrace\_\_definitely\_raise\_347.objdump}
      }
      \vfill
      \mintocaml[firstline=9,lastline=13,highlightlines=12]{../src/backtrace/backtrace.ml}
      \endminipage
    \end{column}
    \begin{column}{0.5\textwidth}
      \centering
      \providecommand\step{1}\input{diagrams/backtrace/backtrace.tex}
    \end{column}
  \end{columns}
\end{frame}

\begin{frame}{Backtraces}{But before that, we need more fields from \typename{caml\_domain\_state}}
  \begin{columns}
    \begin{column}{0.5\textwidth}
      \begin{itemize}
        \item Remember \typename{caml\_domain\_state} the per-domain runtime state?
        \item Some other members are of interest to us now\dots
        \item \localname{backtrace\_active} is used to know if the runtime should collect backtraces
        \item \localname{backtrace\_active} can be set by
          \begin{itemize}
            \item The \funcarg{b} flag in the \funcname{OCAMLRUNPARAM} environment variable
            \item \mintinline{ocaml}{Printexc.record_backtrace : bool -> unit} \footnotemark
          \end{itemize}
      \end{itemize}
    \end{column}
    \begin{column}{0.5\textwidth}
      \begin{listing}
        \mintc[highlightlines={9-11}]{src/ocaml/domain\_state\_backtrace.h}
        \caption{runtime/caml/domain\_state.h}
      \end{listing}
    \end{column}
  \end{columns}
  \footnotetext[1]{https://v2.ocaml.org/api/Printexc.html}
\end{frame}

\newcommand\listCamlRaiseExn[1][]{
  \listasm[firstline=702,lastline=725,#1]{../ocaml/runtime/amd64.S}{runtime/amd64.S:702}}

\begin{frame}{Backtraces}{Walk through \funcname{caml\_raise\_exn}}
  \begin{columns}[T]
    \begin{column}{0.5\textwidth}
      \begin{itemize}
        \item<1-> First checks whether exceptions backtrace should be recorded
        \item<1-> By checking the \funcarg{domain\_state->backtrace\_active} value
        \item<2-> If not, acts as \funcname{raise\_notrace} with \funcname{RESTORE\_EXN\_HANDLER\_OCAML}
          \begin{itemize}
            \item Set \regname{RSP} to \localname{domain\_state->exn\_handler} value
            \item Restore \localname{domain\_state->exn\_handler} from trap
            \item Jump with \funcname{ret} (same effect as using \funcname{pop} and \funcname{jmp})
          \end{itemize}
        \item<3-> Otherwise, switches to the C stack to be able to execute C code
        \item<4-> Calls \funcname{caml\_stash\_backtrace}, a C function provided by the runtime\ignorespaces
          \onslide<6->{, with
            \begin{itemize}
              \item \funcarg{exn}: exception value
              \item \funcarg{pc}: code address of the raising point
              \item \funcarg{sp}: stack address of the raising function
              \item \funcarg{trapsp}: stack address of the next handler
            \end{itemize}
          }
      \end{itemize}
      \bigskip
      \mintocaml[firstline=9,lastline=13,highlightlines=12]{../src/backtrace/backtrace.ml}
    \end{column}
    \begin{column}{0.5\textwidth}
      \raggedleft
      \onslide*<1,3-4>{
        \mintc[firstline=190,lastline=192]{../ocaml/runtime/amd64.S}
        \mintc[firstline=234,lastline=237]{../ocaml/runtime/amd64.S}
      }
      \onslide*<2>{
        \mintc[firstline=190,lastline=192,highlightlines={191-192}]{../ocaml/runtime/amd64.S}
        \mintc[firstline=234,lastline=237,highlightlines={235-237}]{../ocaml/runtime/amd64.S}
      }
      \onslide*<1>{\listCamlRaiseExn[highlightlines=705]{}}%
      \onslide*<2>{\listCamlRaiseExn[highlightlines={707-708}]{}}%
      \onslide*<3>{\listCamlRaiseExn[highlightlines={712-714}]{}}%
      \onslide*<4>{\listCamlRaiseExn[highlightlines={715-720}]{}}%
      \onslide*<5>{\providecommand\step{1}\input{diagrams/backtrace/backtrace.tex}}%
      \onslide*<6>{\providecommand\step{2}\input{diagrams/backtrace/backtrace.tex}}%
    \end{column}
  \end{columns}
\end{frame}

\newcommand\listStashBacktrace[1][]{
  \listc[firstline=91,lastline=120,#1]{../ocaml/runtime/backtrace\_nat.c}{runtime/backtrace\_nat.c:91}}

\begin{frame}{Backtraces}{Introducing \funcname{caml\_stash\_backtrace}}
  \begin{columns}[c]
    \begin{column}{0.5\textwidth}
      \minipage[c][0.66\textheight][s]{\columnwidth}
      \begin{itemize}
        \item<1-> A C function provided by the runtime (specific to native code)
        \item<2-> It scans the stack, between the stack pointer at the raising point up to the next trap, in order to collect the names of the detected function frames
          \onslide*<2>{
            \begin{itemize}
              \item And more information, like line number, column number...
            \end{itemize}
          }
        \item<3-> It uses the same technique and information as the Garbage Collector (GC) uses to scan the values live on the stack
          \onslide*<4>{
            \begin{itemize}
              \item \typename{frame\_descr} are at the core of stack unwinding
            \end{itemize}
          }
      \end{itemize}
      \vfill
      \mintocaml[firstline=9,lastline=13,highlightlines=12]{../src/backtrace/backtrace.ml}
      \endminipage
    \end{column}
    \begin{column}{0.5\textwidth}
      \onslide*<1>{\listStashBacktrace[highlightlines=91]{}}
      \onslide*<2>{\listStashBacktrace[highlightlines={91,117-118}]{}}
      \onslide*<3->{\listStashBacktrace[highlightlines={94,106,109-110}]{}}
    \end{column}
  \end{columns}
\end{frame}

\newcommand\listFrameDescr[1][]{
  \listocaml[firstline=111,lastline=124,#1]{../ocaml/asmcomp/emitaux.ml}{asmcomp/emitaux.ml:111}}
\newcommand\listDebugInfo[1][]{
  \listocaml[firstline=38,lastline=49,#1]{../ocaml/lambda/debuginfo.mli}{lambda/debuginfo.mli:38}}

\begin{frame}{Backtraces}{\typename{frame\_descr} and \typename{frame\_debuginfo} in a nutshell}
  \begin{columns}[c]
    \begin{column}{0.5\textwidth}
      \begin{itemize}
        \item<1-> For every return address in an OCaml program, there is a associated \typename{frame\_descr}
        \item<2-> A \typename{frame\_descr} specifies
        \begin{itemize}
          \item<2-> The size of the function stack frame
          \item<3-> Where live values are on the stack (so that the GC can mark them as live and prevent from being swept)
        \end{itemize}
        \item<4-> When compiled with debug information, \typename{frame\_descr} will be linked to a \typename{frame\_debuginfo}
        \begin{itemize}
          \item<5-> Contains information about the position in the source code (file name, line and column number)
        \end{itemize}
      \end{itemize}
    \end{column}
    \begin{column}{0.5\textwidth}
        \onslide*<1>{\listFrameDescr[highlightlines={118-122}]{}}
        \onslide*<2>{\listFrameDescr[highlightlines={120}]{}}
        \onslide*<3>{\listFrameDescr[highlightlines={121}]{}}
        \onslide*<4->{\listFrameDescr[highlightlines={122}]{}}
        \medskip
        \onslide*<1-3>{\listDebugInfo{}}
        \onslide*<4>{\listDebugInfo[highlightlines={38,49}]{}}
        \onslide*<5>{\listDebugInfo[highlightlines={39-46}]{}}
    \end{column}
  \end{columns}
\end{frame}

\begin{frame}{Backtraces}{Scanning the stack with \typename{frame\_descr}}
  \begin{columns}[c]
    \begin{column}{0.5\textwidth}
      \begin{itemize}
        \item Given the return address of the code that raises the exception by calling \funcname{caml\_raise\_exn} (\funcarg{pc} argument of \funcname{caml\_stash\_backtrace})
        \item And the position of the stack pointer (\funcarg{sp} argument of \funcname{caml\_stash\_backtrace})
        \item The frame size given by \typename{frame\_descr}.\funcarg{frame\_size}, allows to find the end of the previous frame
        \item And the return address of the caller of this function
        \bigskip
        \onslide<3->{
        \item Note that traps (here the reddish \funcname{ExnA}, \funcname{ExnB} and \funcname{ExnC} blocks) belong to the frame that installed them, and so the frame size changes when traps are removed
        }
      \end{itemize}
    \end{column}
    \begin{column}{0.5\textwidth}
      \raggedleft
      \onslide*<1>{\providecommand\step{2}\input{diagrams/backtrace/backtrace.tex}}%
      \onslide*<2>{\providecommand\step{1}\input{diagrams/backtrace/scanstack.tex}}%
      \onslide*<3>{\providecommand\step{2}\input{diagrams/backtrace/scanstack.tex}}%
      \onslide*<4>{\providecommand\step{3}\input{diagrams/backtrace/scanstack.tex}}%
      \onslide*<5>{\providecommand\step{4}\input{diagrams/backtrace/scanstack.tex}}%
      \onslide*<6>{\providecommand\step{5}\input{diagrams/backtrace/scanstack.tex}}%
    \end{column}
  \end{columns}
\end{frame}

% Duplicate of previous-previous slide
\begin{frame}{Backtraces}{Continuing on \funcname{caml\_stash\_backtrace}}
  \begin{columns}[c]
    \begin{column}{0.5\textwidth}
      \minipage[c][0.75\textheight][s]{\columnwidth}
      \begin{itemize}
          \color{gray}
        \item A C function provided by the runtime (specific to native code)
        \item It scans the stack, between the stack pointer at the raising point up to the next trap, in order to collect the names of the detected function frames
        \item It uses the same technique and information as the Garbage Collector (GC) uses to scan the values live on the stack
      \end{itemize}
      \begin{itemize}
        \item For each function frame detected, store a pointer to the \typename{frame\_descr} in the \localname{domain\_state->backtrace\_buffer} array, effectively constructing the backtrace
      \end{itemize}
      \onslide<2->{
        \begin{itemize}
            \smallskip
          \item Constructed backtrace:
            \funcname{definitely\_raise},
            \onslide<3->{\funcname{probably\_raise}}
        \end{itemize}
      }
      \vfill
      \mintocaml[firstline=9,lastline=13,highlightlines=12]{../src/backtrace/backtrace.ml}
      \endminipage
    \end{column}
    \begin{column}{0.5\textwidth}
      \raggedleft
      \onslide*<1>{\listStashBacktrace[highlightlines={114-115}]{}}
      \onslide*<2>{\providecommand\step{3}\input{diagrams/backtrace/backtrace.tex}}%
      \onslide*<3>{\providecommand\step{4}\input{diagrams/backtrace/backtrace.tex}}%
    \end{column}
  \end{columns}
\end{frame}

\begin{frame}{Backtraces}{Back to \funcname{caml\_raise\_exn}}
  \begin{columns}[c]
    \begin{column}{0.5\textwidth}
      \minipage[c][0.66\textheight][s]{\columnwidth}
      \begin{itemize}\color{gray}
        \item Checks wheteher exceptions backtrace should be recorded, by checking the \funcarg{domain\_state->backtrace\_active} value
        \item Switches to the C stack to be able to execute C code
        \item Calls \funcname{caml\_stash\_backtrace}, to collect the backtrace up to the next trap
      \end{itemize}
      \onslide*<2->{
        \begin{itemize}
          \item<2-> Executes the exception handler in \funcname{probably\_raise} with \funcname{RESTORE\_EXN\_HANDLER\_OCAML}
            \begin{itemize}
              \item Set \regname{RSP} to \localname{domain\_state->exn\_handler} value
              \item Restore \localname{domain\_state->exn\_handler} from trap
              \item Jump to the handler code with \funcname{ret}
            \end{itemize}
        \end{itemize}
      }
      \vfill
      \onslide*<1>{\mintocaml[firstline=9,lastline=13,highlightlines=12]{../src/backtrace/backtrace.ml}}
      \onslide*<2->{\mintocaml[firstline=15,lastline=21,highlightlines={19-21}]{../src/backtrace/backtrace.ml}}
      \endminipage
    \end{column}
    \begin{column}{0.5\textwidth}
      \centering
      \onslide*<1>{
        \mintc[firstline=190,lastline=192]{../ocaml/runtime/amd64.S}
        \mintc[firstline=234,lastline=237]{../ocaml/runtime/amd64.S}
      }
      \onslide*<2>{
        \mintc[firstline=190,lastline=192,highlightlines={191-192}]{../ocaml/runtime/amd64.S}
        \mintc[firstline=234,lastline=237,highlightlines={235-237}]{../ocaml/runtime/amd64.S}
      }
      \onslide*<1>{\listCamlRaiseExn[highlightlines=720]{}}
      \onslide*<2>{\listCamlRaiseExn[highlightlines=722]{}}
      \onslide*<3->{\providecommand\step{5}\input{diagrams/backtrace/backtrace.tex}}
    \end{column}
  \end{columns}
\end{frame}

\begin{frame}{Backtraces}{\funcname{caml\_probably\_raise}'s handler doesn't catch \typename{ExnC}}
  \begin{columns}[c]
    \begin{column}{0.45\textwidth}
      \minipage[c][0.75\textheight][s]{\columnwidth}
      \begin{itemize}
        \item<1-> \funcname{match \dots with}\footnotemark pattern matching can also match exceptions
        \item<1-> The CMM of \funcname{probably\_raise} shows that this \funcname{match \dots with} is implemented with a pattern matching and a \funcname{try \dots with}
        \item<1-> But this one only matches on \typename{ExnA} exception
        \item<2-> Since the pattern matching fails to catch the exception, \funcname{reraise} is called with our current \typename{ExnC} exception
        \item<3-> Disassembly shows that \funcname{reraise} is actually a call to \funcname{caml\_reraise\_exn}, which is also an assembly function provided by the runtime
      \end{itemize}
      \vfill
      \mintocaml[firstline=15,lastline=21,highlightlines={18-21}]{../src/backtrace/backtrace.ml}
      \endminipage
    \end{column}
    \begin{column}{0.55\textwidth}
      \centering
      \onslide*<1>{\mintcmm[highlightlines={10-18}]{../src/backtrace/camlBacktrace\_\_probably\_raise\_351.cmm}}
      \onslide*<2>{\listcmm[highlightlines=18]{../src/backtrace/camlBacktrace\_\_probably\_raise_351.cmm}{CMM of probably\_raise}}
      \onslide*<3>{\mintobjdump[firstline=5,lastline=35,highlightlines=31]{../src/backtrace/camlBacktrace\_\_probably\_raise_351.objdump}}
    \end{column}
  \end{columns}
  \only<1>{\footnotetext[1]{https://blog.janestreet.com/pattern-matching-and-exception-handling-unite/}}
\end{frame}

\newcommand\listCamlReraiseExn[1][]{
  \listasm[firstline=727,lastline=734,#1]{../ocaml/runtime/amd64.S}{runtime/amd64.S:727}}

\begin{frame}{Backtraces}{Introducing \funcname{caml\_reraise\_exn}}
  \begin{columns}[c]
    \begin{column}{0.5\textwidth}
      \minipage[c][0.66\textheight][s]{\columnwidth}
      \begin{itemize}
        \item<1-> The exception have to be reraised to the next handler
        \item<2-> First, checks if the backtrace should be collected with \funcarg{domain\_state->backtrace\_active}
        \only<3>{
        \begin{itemize}
            \item If not, behaves like a \funcname{raise\_notrace} with \funcname{RESTORE\_EXN\_HANDLER\_OCAML}
        \end{itemize}
        }
        \item<4-> Jumps into \funcname{caml\_raise\_exn}
        \item<5-> Calls \funcname{caml\_stash\_backtrace} again, to scan the next part of the stack: from the trap just pop-ed to the next trap
      \end{itemize}
      \vfill
      \mintocaml[firstline=15,lastline=21,highlightlines={18-21}]{../src/backtrace/backtrace.ml}
      \endminipage
    \end{column}
    \begin{column}{0.5\textwidth}
      \raggedleft
      \onslide*<1>{\listCamlReraiseExn{}}
      \onslide*<2>{\listCamlReraiseExn[highlightlines=729]{}}
      \onslide*<3>{
        \mintc[firstline=190,lastline=192,highlightlines={191-192}]{../ocaml/runtime/amd64.S}
        \mintc[firstline=234,lastline=237,highlightlines={235-237}]{../ocaml/runtime/amd64.S}
        \medskip
        \listCamlReraiseExn[highlightlines=731]{}
      }
      \onslide*<4-5>{
        % TODO refactor with \listCamlReraiseExn
        \mintasm[firstline=727,lastline=734,highlightlines=730]{../ocaml/runtime/amd64.S}
        \medskip
      }
      \onslide*<4>{\listCamlRaiseExn[highlightlines=711]{}}
      \onslide*<5>{\listCamlRaiseExn[highlightlines=720]{}}
    \end{column}
  \end{columns}
\end{frame}

\begin{frame}{Backtraces}{Backtrace collection continues}
  \begin{columns}[c]
    \begin{column}{0.55\textwidth}
      \minipage[c][0.75\textheight][s]{\columnwidth}
      \begin{itemize}
        \item<1-> \funcname{caml\_stash\_backtrace} scans the older part of the stack, up to the next trap
        \item<6-> Controls jumps to the nested exception handler in \funcname{maybe\_raise}, which matches only on \typename{ExnB}
        \item<7-> \funcname{caml\_reraise\_exn} is called again, backtrace collection resumes with \funcname{caml\_stash\_backtrace}
      \end{itemize}
      \medskip
      \begin{itemize}
        \item<2-> Constructed backtrace:
        \begin{itemize}
          \item <2-> \funcname{definitely\_raise}
          \item <2-> \funcname{probably\_raise}
          \item <3-> \funcname{probably\_raise} (re-raise)
          \item <4-> \funcname{maybe\_raise}
          \item <5-> \funcname{main}
          \item <8-> \funcname{main} (re-raise)
        \end{itemize}
      \end{itemize}
      \vfill
      \onslide*<1-5>{\mintocaml[firstline=15,lastline=21]{../src/backtrace/backtrace.ml}}
      \onslide*<6>{\mintocaml[firstline=28,lastline=34,highlightlines=31]{../src/backtrace/backtrace.ml}}
      \onslide*<7->{\mintocaml[firstline=28,lastline=34,highlightlines=33]{../src/backtrace/backtrace.ml}}
      \endminipage
    \end{column}
    \begin{column}{0.45\textwidth}
      \raggedleft
      \onslide*<1>{\providecommand\step{6}\input{diagrams/backtrace/backtrace.tex}}%
      \onslide*<2>{\providecommand\step{6}\input{diagrams/backtrace/backtrace.tex}}%
      \onslide*<3>{\providecommand\step{7}\input{diagrams/backtrace/backtrace.tex}}%
      \onslide*<4>{\providecommand\step{8}\input{diagrams/backtrace/backtrace.tex}}%
      \onslide*<5>{\providecommand\step{9}\input{diagrams/backtrace/backtrace.tex}}%
      \onslide*<6>{\providecommand\step{10}\input{diagrams/backtrace/backtrace.tex}}%
      \onslide*<7>{\providecommand\step{11}\input{diagrams/backtrace/backtrace.tex}}%
      \onslide*<8>{\providecommand\step{12}\input{diagrams/backtrace/backtrace.tex}}%
      \onslide*<9>{\providecommand\step{13}\input{diagrams/backtrace/backtrace.tex}}%
    \end{column}
  \end{columns}
\end{frame}

\begin{frame}{Backtraces}{Printing the backtrace}
  \begin{columns}[c]
    \begin{column}{0.45\textwidth}
      \minipage[c][0.66\textheight][s]{\columnwidth}
      \begin{itemize}
        \item<1-> The exception backtrace can now be printed to \funcarg{stdout} with \funcname{Printexc.print_backtrace}
        \item<2-> First the exception backtrace is retrieved into an OCaml array with \funcname{get_raw_backtrace }
        \item<3-> Which is implemented as the \funcname{caml_get_exception_raw_backtrace} C function in the runtime
        \item<4-> And finally printed with \funcname{print_exception_backtrace}
      \end{itemize}
      \vfill
      \mintocaml[firstline=28,lastline=34,highlightlines=33]{../src/backtrace/backtrace.ml}
      \endminipage
    \end{column}
    \begin{column}{0.55\textwidth}
      \onslide*<1,2>{
        \mintocaml[firstline=100,lastline=101]{../ocaml/stdlib/printexc.ml}
      }
      \only<1>{
        \listocaml[firstline=170,lastline=172]{../ocaml/stdlib/printexc.ml}{stdlib/printexc.ml:170}
      }
      \only<2>{
        \listocaml[firstline=170,lastline=172,highlightlines=172]{../ocaml/stdlib/printexc.ml}{stdlib/printexc.ml:170}
      }
      \only<3>{
        \mintc[firstline=154,lastline=156]{../ocaml/runtime/backtrace.c}
        \listc[firstline=171,lastline=192,highlightlines={184-187}]{../ocaml/runtime/backtrace.c}{runtime/backtrace.c:154}
      }
      \only<4>{
        \listocaml[firstline=155,lastline=165]{../ocaml/stdlib/printexc.ml}{stdlib/printexc.ml:55}
      }
    \end{column}
  \end{columns}
\end{frame}


\subsection{The multiple kinds of raise}
\frameSubsection{}{}

\begin{frame}{Backtraces}{When backtraces are not needed}
  \begin{columns}[c]
    \begin{column}{0.45\textwidth}
      \minipage[c][0.66\textheight][s]{\columnwidth}
      \begin{itemize}
        \item<1-> What if the backtrace for an exception is never needed?
        \item<2-> It's possible to raise the exception explicitly with \funcname{raise\_notrace} to make raising as cheap as possible
        \item<3-> An example of use in the \funcname{dynlink} library of the OCaml compiler
      \end{itemize}
      \vfill
      \onslide<3->{
      \listocaml[firstline=205,lastline=209,highlightlines=207]{../ocaml/otherlibs/dynlink/dynlink\_compilerlibs/misc.ml}{otherlibs/dynlink/dynlink\_compilerlibs/misc.ml:205}
      }
      \endminipage
    \end{column}
    \begin{column}{0.55\textwidth}
    \only<4>{
      \mintcmm[highlightlines=3]{../src/notrace/camlNotrace\_\_fun_347.cmm}
      \listcmm{../src/notrace/camlNotrace\_\_all\_somes\_327.cmm}{all\_somes}
    }
    \onslide*<5->{
      \listobjdump[highlightlines={6-9}]{../src/notrace/camlNotrace\_\_fun_347.objdump}{anonymous function from \funcname{all\_somes}}
    }
    \end{column}
  \end{columns}
\end{frame}

\begin{frame}{Backtraces}{Summary table of the multiple kinds of raise}
  \begin{itemize}
    \item When debugging information is enabled and if the backtrace of an exception isn't needed, it's possible to raise with \funcname{raise\_notrace} for performance
    \item When debugging information is \emph{not} enabled, the compiler optimises \funcname{raise} calls to inline assembly (same effect as using \funcname{raise\_notrace})
  \end{itemize}
  \bigskip
  \centering
  \begin{tabular}{| l | c | c | c | c |}
    \hline
    \parbox{10em}{Debug information \\ (\funcarg{-g} flag)}  & \multicolumn{2}{|c|}{Disabled}                                     & \multicolumn{2}{|c|}{Enabled} \\
    \hline
    Raise kind                                               & \funcname{raise} & \funcname{raise\_notrace}                       & \funcname{raise} & \funcname{raise\_notrace} \\
    \hline
    Compilation                                              & \parbox{8em}{inline assembly\\ (optimization)} & inline assembly   & \funcname{caml\_raise\_exn} & inline assembly \\
    \hline
  \end{tabular}
\end{frame}


%
% Takeaway #5
%
\subsection*{Takeaway \#5}
\frameSubsectionTakeaway{}
\againframe<5>{takeaway}
}%
    \end{column}
  \end{columns}
\end{frame}

\newcommand\listStashBacktrace[1][]{
  \listc[firstline=91,lastline=120,#1]{../ocaml/runtime/backtrace\_nat.c}{runtime/backtrace\_nat.c:91}}

\begin{frame}{Backtraces}{Introducing \funcname{caml\_stash\_backtrace}}
  \begin{columns}[c]
    \begin{column}{0.5\textwidth}
      \minipage[c][0.66\textheight][s]{\columnwidth}
      \begin{itemize}
        \item<1-> A C function provided by the runtime (specific to native code)
        \item<2-> It scans the stack, between the stack pointer at the raising point up to the next trap, in order to collect the names of the detected function frames
          \onslide*<2>{
            \begin{itemize}
              \item And more information, like line number, column number...
            \end{itemize}
          }
        \item<3-> It uses the same technique and information as the Garbage Collector (GC) uses to scan the values live on the stack
          \onslide*<4>{
            \begin{itemize}
              \item \typename{frame\_descr} are at the core of stack unwinding
            \end{itemize}
          }
      \end{itemize}
      \vfill
      \mintocaml[firstline=9,lastline=13,highlightlines=12]{../src/backtrace/backtrace.ml}
      \endminipage
    \end{column}
    \begin{column}{0.5\textwidth}
      \onslide*<1>{\listStashBacktrace[highlightlines=91]{}}
      \onslide*<2>{\listStashBacktrace[highlightlines={91,117-118}]{}}
      \onslide*<3->{\listStashBacktrace[highlightlines={94,106,109-110}]{}}
    \end{column}
  \end{columns}
\end{frame}

\newcommand\listFrameDescr[1][]{
  \listocaml[firstline=111,lastline=124,#1]{../ocaml/asmcomp/emitaux.ml}{asmcomp/emitaux.ml:111}}
\newcommand\listDebugInfo[1][]{
  \listocaml[firstline=38,lastline=49,#1]{../ocaml/lambda/debuginfo.mli}{lambda/debuginfo.mli:38}}

\begin{frame}{Backtraces}{\typename{frame\_descr} and \typename{frame\_debuginfo} in a nutshell}
  \begin{columns}[c]
    \begin{column}{0.5\textwidth}
      \begin{itemize}
        \item<1-> For every return address in an OCaml program, there is a associated \typename{frame\_descr}
        \item<2-> A \typename{frame\_descr} specifies
        \begin{itemize}
          \item<2-> The size of the function stack frame
          \item<3-> Where live values are on the stack (so that the GC can mark them as live and prevent from being swept)
        \end{itemize}
        \item<4-> When compiled with debug information, \typename{frame\_descr} will be linked to a \typename{frame\_debuginfo}
        \begin{itemize}
          \item<5-> Contains information about the position in the source code (file name, line and column number)
        \end{itemize}
      \end{itemize}
    \end{column}
    \begin{column}{0.5\textwidth}
        \onslide*<1>{\listFrameDescr[highlightlines={118-122}]{}}
        \onslide*<2>{\listFrameDescr[highlightlines={120}]{}}
        \onslide*<3>{\listFrameDescr[highlightlines={121}]{}}
        \onslide*<4->{\listFrameDescr[highlightlines={122}]{}}
        \medskip
        \onslide*<1-3>{\listDebugInfo{}}
        \onslide*<4>{\listDebugInfo[highlightlines={38,49}]{}}
        \onslide*<5>{\listDebugInfo[highlightlines={39-46}]{}}
    \end{column}
  \end{columns}
\end{frame}

\begin{frame}{Backtraces}{Scanning the stack with \typename{frame\_descr}}
  \begin{columns}[c]
    \begin{column}{0.5\textwidth}
      \begin{itemize}
        \item Given the return address of the code that raises the exception by calling \funcname{caml\_raise\_exn} (\funcarg{pc} argument of \funcname{caml\_stash\_backtrace})
        \item And the position of the stack pointer (\funcarg{sp} argument of \funcname{caml\_stash\_backtrace})
        \item The frame size given by \typename{frame\_descr}.\funcarg{frame\_size}, allows to find the end of the previous frame
        \item And the return address of the caller of this function
        \bigskip
        \onslide<3->{
        \item Note that traps (here the reddish \funcname{ExnA}, \funcname{ExnB} and \funcname{ExnC} blocks) belong to the frame that installed them, and so the frame size changes when traps are removed
        }
      \end{itemize}
    \end{column}
    \begin{column}{0.5\textwidth}
      \raggedleft
      \onslide*<1>{\providecommand\step{2}%
% Backtraces
%
\subsection{One advantage of raising with debugging information}
\frameSubsection{}{}

\begin{frame}{Backtraces}{Debugging information \& source code}
  \begin{columns}[c]
    \begin{column}{0.5\textwidth}
      \begin{itemize}
        \item Raising an exception is fun and games, but how do we know where the exception originated? $\rightarrow$ With the backtrace associated with the exception!
        \item In order to get a backtrace from the point where an exception was raised, it's necessary to compile with debugging information enabled (\funcarg{-g} flag for the compiler)
        \item Same example as in the `Nested handlers' section
      \end{itemize}
    \end{column}
    \begin{column}{0.5\textwidth}
      \listocaml[firstline=9,lastline=34]{../src/backtrace/backtrace.ml}{backtrace/backtrace.ml}
    \end{column}
  \end{columns}
\end{frame}

\begin{frame}{Backtraces}{CMM and disassembly of \funcname{definitely\_raise}}
  \begin{columns}[c]
    \begin{column}{0.5\textwidth}
      \minipage[c][0.66\textheight][s]{\columnwidth}
      \begin{itemize}
        \item<1-> An effect of compiling with debugging information (\funcarg{-g}): the CMM no longer uses \funcname{raise\_notrace} to raise the exception but \funcname{raise} instead
        \item<2-> Disassembly shows that CMM \funcname{raise} is actually a call to \funcname{caml\_raise\_exn}, a function in assembler provided by the runtime
      \end{itemize}
      \vfill
      \mintocaml[firstline=9,lastline=13,highlightlines=12]{../src/backtrace/backtrace.ml}
      \endminipage
    \end{column}
    \begin{column}{0.5\textwidth}
      \onslide*<1>{
        \mintcmm[highlightlines=5]{../src/backtrace/camlBacktrace\_\_definitely\_raise\_347.cmm}
      }
      \onslide*<2>{
        \mintobjdump[firstline=2,highlightlines=14]{../src/backtrace/camlBacktrace\_\_definitely\_raise\_347.objdump}
      }
    \end{column}
  \end{columns}
\end{frame}

\begin{frame}{Backtraces}{Introducing \funcname{caml\_raise\_exn}}
  \begin{columns}[c]
    \begin{column}{0.5\textwidth}
      \minipage[c][0.66\textheight][s]{\columnwidth}
      \onslide*<1>{
        \begin{itemize}
          \item Raising the exception is performed using \funcname{caml\_raise\_exn} which is
            \begin{itemize}
              \item An assembly function provided by the runtime
              \item Architecture specific
            \end{itemize}
          \item Let's see what it does differently from \funcname{raise\_notrace}\dots
          \item We are about to raise \typename{ExnC} with \funcname{caml\_raise\_exn} from \funcname{definitely\_raise}
        \end{itemize}
      }
      \onslide*<2>{
        \mintobjdump[firstline=2,highlightlines=14]{../src/backtrace/camlBacktrace\_\_definitely\_raise\_347.objdump}
      }
      \vfill
      \mintocaml[firstline=9,lastline=13,highlightlines=12]{../src/backtrace/backtrace.ml}
      \endminipage
    \end{column}
    \begin{column}{0.5\textwidth}
      \centering
      \providecommand\step{1}\input{diagrams/backtrace/backtrace.tex}
    \end{column}
  \end{columns}
\end{frame}

\begin{frame}{Backtraces}{But before that, we need more fields from \typename{caml\_domain\_state}}
  \begin{columns}
    \begin{column}{0.5\textwidth}
      \begin{itemize}
        \item Remember \typename{caml\_domain\_state} the per-domain runtime state?
        \item Some other members are of interest to us now\dots
        \item \localname{backtrace\_active} is used to know if the runtime should collect backtraces
        \item \localname{backtrace\_active} can be set by
          \begin{itemize}
            \item The \funcarg{b} flag in the \funcname{OCAMLRUNPARAM} environment variable
            \item \mintinline{ocaml}{Printexc.record_backtrace : bool -> unit} \footnotemark
          \end{itemize}
      \end{itemize}
    \end{column}
    \begin{column}{0.5\textwidth}
      \begin{listing}
        \mintc[highlightlines={9-11}]{src/ocaml/domain\_state\_backtrace.h}
        \caption{runtime/caml/domain\_state.h}
      \end{listing}
    \end{column}
  \end{columns}
  \footnotetext[1]{https://v2.ocaml.org/api/Printexc.html}
\end{frame}

\newcommand\listCamlRaiseExn[1][]{
  \listasm[firstline=702,lastline=725,#1]{../ocaml/runtime/amd64.S}{runtime/amd64.S:702}}

\begin{frame}{Backtraces}{Walk through \funcname{caml\_raise\_exn}}
  \begin{columns}[T]
    \begin{column}{0.5\textwidth}
      \begin{itemize}
        \item<1-> First checks whether exceptions backtrace should be recorded
        \item<1-> By checking the \funcarg{domain\_state->backtrace\_active} value
        \item<2-> If not, acts as \funcname{raise\_notrace} with \funcname{RESTORE\_EXN\_HANDLER\_OCAML}
          \begin{itemize}
            \item Set \regname{RSP} to \localname{domain\_state->exn\_handler} value
            \item Restore \localname{domain\_state->exn\_handler} from trap
            \item Jump with \funcname{ret} (same effect as using \funcname{pop} and \funcname{jmp})
          \end{itemize}
        \item<3-> Otherwise, switches to the C stack to be able to execute C code
        \item<4-> Calls \funcname{caml\_stash\_backtrace}, a C function provided by the runtime\ignorespaces
          \onslide<6->{, with
            \begin{itemize}
              \item \funcarg{exn}: exception value
              \item \funcarg{pc}: code address of the raising point
              \item \funcarg{sp}: stack address of the raising function
              \item \funcarg{trapsp}: stack address of the next handler
            \end{itemize}
          }
      \end{itemize}
      \bigskip
      \mintocaml[firstline=9,lastline=13,highlightlines=12]{../src/backtrace/backtrace.ml}
    \end{column}
    \begin{column}{0.5\textwidth}
      \raggedleft
      \onslide*<1,3-4>{
        \mintc[firstline=190,lastline=192]{../ocaml/runtime/amd64.S}
        \mintc[firstline=234,lastline=237]{../ocaml/runtime/amd64.S}
      }
      \onslide*<2>{
        \mintc[firstline=190,lastline=192,highlightlines={191-192}]{../ocaml/runtime/amd64.S}
        \mintc[firstline=234,lastline=237,highlightlines={235-237}]{../ocaml/runtime/amd64.S}
      }
      \onslide*<1>{\listCamlRaiseExn[highlightlines=705]{}}%
      \onslide*<2>{\listCamlRaiseExn[highlightlines={707-708}]{}}%
      \onslide*<3>{\listCamlRaiseExn[highlightlines={712-714}]{}}%
      \onslide*<4>{\listCamlRaiseExn[highlightlines={715-720}]{}}%
      \onslide*<5>{\providecommand\step{1}\input{diagrams/backtrace/backtrace.tex}}%
      \onslide*<6>{\providecommand\step{2}\input{diagrams/backtrace/backtrace.tex}}%
    \end{column}
  \end{columns}
\end{frame}

\newcommand\listStashBacktrace[1][]{
  \listc[firstline=91,lastline=120,#1]{../ocaml/runtime/backtrace\_nat.c}{runtime/backtrace\_nat.c:91}}

\begin{frame}{Backtraces}{Introducing \funcname{caml\_stash\_backtrace}}
  \begin{columns}[c]
    \begin{column}{0.5\textwidth}
      \minipage[c][0.66\textheight][s]{\columnwidth}
      \begin{itemize}
        \item<1-> A C function provided by the runtime (specific to native code)
        \item<2-> It scans the stack, between the stack pointer at the raising point up to the next trap, in order to collect the names of the detected function frames
          \onslide*<2>{
            \begin{itemize}
              \item And more information, like line number, column number...
            \end{itemize}
          }
        \item<3-> It uses the same technique and information as the Garbage Collector (GC) uses to scan the values live on the stack
          \onslide*<4>{
            \begin{itemize}
              \item \typename{frame\_descr} are at the core of stack unwinding
            \end{itemize}
          }
      \end{itemize}
      \vfill
      \mintocaml[firstline=9,lastline=13,highlightlines=12]{../src/backtrace/backtrace.ml}
      \endminipage
    \end{column}
    \begin{column}{0.5\textwidth}
      \onslide*<1>{\listStashBacktrace[highlightlines=91]{}}
      \onslide*<2>{\listStashBacktrace[highlightlines={91,117-118}]{}}
      \onslide*<3->{\listStashBacktrace[highlightlines={94,106,109-110}]{}}
    \end{column}
  \end{columns}
\end{frame}

\newcommand\listFrameDescr[1][]{
  \listocaml[firstline=111,lastline=124,#1]{../ocaml/asmcomp/emitaux.ml}{asmcomp/emitaux.ml:111}}
\newcommand\listDebugInfo[1][]{
  \listocaml[firstline=38,lastline=49,#1]{../ocaml/lambda/debuginfo.mli}{lambda/debuginfo.mli:38}}

\begin{frame}{Backtraces}{\typename{frame\_descr} and \typename{frame\_debuginfo} in a nutshell}
  \begin{columns}[c]
    \begin{column}{0.5\textwidth}
      \begin{itemize}
        \item<1-> For every return address in an OCaml program, there is a associated \typename{frame\_descr}
        \item<2-> A \typename{frame\_descr} specifies
        \begin{itemize}
          \item<2-> The size of the function stack frame
          \item<3-> Where live values are on the stack (so that the GC can mark them as live and prevent from being swept)
        \end{itemize}
        \item<4-> When compiled with debug information, \typename{frame\_descr} will be linked to a \typename{frame\_debuginfo}
        \begin{itemize}
          \item<5-> Contains information about the position in the source code (file name, line and column number)
        \end{itemize}
      \end{itemize}
    \end{column}
    \begin{column}{0.5\textwidth}
        \onslide*<1>{\listFrameDescr[highlightlines={118-122}]{}}
        \onslide*<2>{\listFrameDescr[highlightlines={120}]{}}
        \onslide*<3>{\listFrameDescr[highlightlines={121}]{}}
        \onslide*<4->{\listFrameDescr[highlightlines={122}]{}}
        \medskip
        \onslide*<1-3>{\listDebugInfo{}}
        \onslide*<4>{\listDebugInfo[highlightlines={38,49}]{}}
        \onslide*<5>{\listDebugInfo[highlightlines={39-46}]{}}
    \end{column}
  \end{columns}
\end{frame}

\begin{frame}{Backtraces}{Scanning the stack with \typename{frame\_descr}}
  \begin{columns}[c]
    \begin{column}{0.5\textwidth}
      \begin{itemize}
        \item Given the return address of the code that raises the exception by calling \funcname{caml\_raise\_exn} (\funcarg{pc} argument of \funcname{caml\_stash\_backtrace})
        \item And the position of the stack pointer (\funcarg{sp} argument of \funcname{caml\_stash\_backtrace})
        \item The frame size given by \typename{frame\_descr}.\funcarg{frame\_size}, allows to find the end of the previous frame
        \item And the return address of the caller of this function
        \bigskip
        \onslide<3->{
        \item Note that traps (here the reddish \funcname{ExnA}, \funcname{ExnB} and \funcname{ExnC} blocks) belong to the frame that installed them, and so the frame size changes when traps are removed
        }
      \end{itemize}
    \end{column}
    \begin{column}{0.5\textwidth}
      \raggedleft
      \onslide*<1>{\providecommand\step{2}\input{diagrams/backtrace/backtrace.tex}}%
      \onslide*<2>{\providecommand\step{1}\input{diagrams/backtrace/scanstack.tex}}%
      \onslide*<3>{\providecommand\step{2}\input{diagrams/backtrace/scanstack.tex}}%
      \onslide*<4>{\providecommand\step{3}\input{diagrams/backtrace/scanstack.tex}}%
      \onslide*<5>{\providecommand\step{4}\input{diagrams/backtrace/scanstack.tex}}%
      \onslide*<6>{\providecommand\step{5}\input{diagrams/backtrace/scanstack.tex}}%
    \end{column}
  \end{columns}
\end{frame}

% Duplicate of previous-previous slide
\begin{frame}{Backtraces}{Continuing on \funcname{caml\_stash\_backtrace}}
  \begin{columns}[c]
    \begin{column}{0.5\textwidth}
      \minipage[c][0.75\textheight][s]{\columnwidth}
      \begin{itemize}
          \color{gray}
        \item A C function provided by the runtime (specific to native code)
        \item It scans the stack, between the stack pointer at the raising point up to the next trap, in order to collect the names of the detected function frames
        \item It uses the same technique and information as the Garbage Collector (GC) uses to scan the values live on the stack
      \end{itemize}
      \begin{itemize}
        \item For each function frame detected, store a pointer to the \typename{frame\_descr} in the \localname{domain\_state->backtrace\_buffer} array, effectively constructing the backtrace
      \end{itemize}
      \onslide<2->{
        \begin{itemize}
            \smallskip
          \item Constructed backtrace:
            \funcname{definitely\_raise},
            \onslide<3->{\funcname{probably\_raise}}
        \end{itemize}
      }
      \vfill
      \mintocaml[firstline=9,lastline=13,highlightlines=12]{../src/backtrace/backtrace.ml}
      \endminipage
    \end{column}
    \begin{column}{0.5\textwidth}
      \raggedleft
      \onslide*<1>{\listStashBacktrace[highlightlines={114-115}]{}}
      \onslide*<2>{\providecommand\step{3}\input{diagrams/backtrace/backtrace.tex}}%
      \onslide*<3>{\providecommand\step{4}\input{diagrams/backtrace/backtrace.tex}}%
    \end{column}
  \end{columns}
\end{frame}

\begin{frame}{Backtraces}{Back to \funcname{caml\_raise\_exn}}
  \begin{columns}[c]
    \begin{column}{0.5\textwidth}
      \minipage[c][0.66\textheight][s]{\columnwidth}
      \begin{itemize}\color{gray}
        \item Checks wheteher exceptions backtrace should be recorded, by checking the \funcarg{domain\_state->backtrace\_active} value
        \item Switches to the C stack to be able to execute C code
        \item Calls \funcname{caml\_stash\_backtrace}, to collect the backtrace up to the next trap
      \end{itemize}
      \onslide*<2->{
        \begin{itemize}
          \item<2-> Executes the exception handler in \funcname{probably\_raise} with \funcname{RESTORE\_EXN\_HANDLER\_OCAML}
            \begin{itemize}
              \item Set \regname{RSP} to \localname{domain\_state->exn\_handler} value
              \item Restore \localname{domain\_state->exn\_handler} from trap
              \item Jump to the handler code with \funcname{ret}
            \end{itemize}
        \end{itemize}
      }
      \vfill
      \onslide*<1>{\mintocaml[firstline=9,lastline=13,highlightlines=12]{../src/backtrace/backtrace.ml}}
      \onslide*<2->{\mintocaml[firstline=15,lastline=21,highlightlines={19-21}]{../src/backtrace/backtrace.ml}}
      \endminipage
    \end{column}
    \begin{column}{0.5\textwidth}
      \centering
      \onslide*<1>{
        \mintc[firstline=190,lastline=192]{../ocaml/runtime/amd64.S}
        \mintc[firstline=234,lastline=237]{../ocaml/runtime/amd64.S}
      }
      \onslide*<2>{
        \mintc[firstline=190,lastline=192,highlightlines={191-192}]{../ocaml/runtime/amd64.S}
        \mintc[firstline=234,lastline=237,highlightlines={235-237}]{../ocaml/runtime/amd64.S}
      }
      \onslide*<1>{\listCamlRaiseExn[highlightlines=720]{}}
      \onslide*<2>{\listCamlRaiseExn[highlightlines=722]{}}
      \onslide*<3->{\providecommand\step{5}\input{diagrams/backtrace/backtrace.tex}}
    \end{column}
  \end{columns}
\end{frame}

\begin{frame}{Backtraces}{\funcname{caml\_probably\_raise}'s handler doesn't catch \typename{ExnC}}
  \begin{columns}[c]
    \begin{column}{0.45\textwidth}
      \minipage[c][0.75\textheight][s]{\columnwidth}
      \begin{itemize}
        \item<1-> \funcname{match \dots with}\footnotemark pattern matching can also match exceptions
        \item<1-> The CMM of \funcname{probably\_raise} shows that this \funcname{match \dots with} is implemented with a pattern matching and a \funcname{try \dots with}
        \item<1-> But this one only matches on \typename{ExnA} exception
        \item<2-> Since the pattern matching fails to catch the exception, \funcname{reraise} is called with our current \typename{ExnC} exception
        \item<3-> Disassembly shows that \funcname{reraise} is actually a call to \funcname{caml\_reraise\_exn}, which is also an assembly function provided by the runtime
      \end{itemize}
      \vfill
      \mintocaml[firstline=15,lastline=21,highlightlines={18-21}]{../src/backtrace/backtrace.ml}
      \endminipage
    \end{column}
    \begin{column}{0.55\textwidth}
      \centering
      \onslide*<1>{\mintcmm[highlightlines={10-18}]{../src/backtrace/camlBacktrace\_\_probably\_raise\_351.cmm}}
      \onslide*<2>{\listcmm[highlightlines=18]{../src/backtrace/camlBacktrace\_\_probably\_raise_351.cmm}{CMM of probably\_raise}}
      \onslide*<3>{\mintobjdump[firstline=5,lastline=35,highlightlines=31]{../src/backtrace/camlBacktrace\_\_probably\_raise_351.objdump}}
    \end{column}
  \end{columns}
  \only<1>{\footnotetext[1]{https://blog.janestreet.com/pattern-matching-and-exception-handling-unite/}}
\end{frame}

\newcommand\listCamlReraiseExn[1][]{
  \listasm[firstline=727,lastline=734,#1]{../ocaml/runtime/amd64.S}{runtime/amd64.S:727}}

\begin{frame}{Backtraces}{Introducing \funcname{caml\_reraise\_exn}}
  \begin{columns}[c]
    \begin{column}{0.5\textwidth}
      \minipage[c][0.66\textheight][s]{\columnwidth}
      \begin{itemize}
        \item<1-> The exception have to be reraised to the next handler
        \item<2-> First, checks if the backtrace should be collected with \funcarg{domain\_state->backtrace\_active}
        \only<3>{
        \begin{itemize}
            \item If not, behaves like a \funcname{raise\_notrace} with \funcname{RESTORE\_EXN\_HANDLER\_OCAML}
        \end{itemize}
        }
        \item<4-> Jumps into \funcname{caml\_raise\_exn}
        \item<5-> Calls \funcname{caml\_stash\_backtrace} again, to scan the next part of the stack: from the trap just pop-ed to the next trap
      \end{itemize}
      \vfill
      \mintocaml[firstline=15,lastline=21,highlightlines={18-21}]{../src/backtrace/backtrace.ml}
      \endminipage
    \end{column}
    \begin{column}{0.5\textwidth}
      \raggedleft
      \onslide*<1>{\listCamlReraiseExn{}}
      \onslide*<2>{\listCamlReraiseExn[highlightlines=729]{}}
      \onslide*<3>{
        \mintc[firstline=190,lastline=192,highlightlines={191-192}]{../ocaml/runtime/amd64.S}
        \mintc[firstline=234,lastline=237,highlightlines={235-237}]{../ocaml/runtime/amd64.S}
        \medskip
        \listCamlReraiseExn[highlightlines=731]{}
      }
      \onslide*<4-5>{
        % TODO refactor with \listCamlReraiseExn
        \mintasm[firstline=727,lastline=734,highlightlines=730]{../ocaml/runtime/amd64.S}
        \medskip
      }
      \onslide*<4>{\listCamlRaiseExn[highlightlines=711]{}}
      \onslide*<5>{\listCamlRaiseExn[highlightlines=720]{}}
    \end{column}
  \end{columns}
\end{frame}

\begin{frame}{Backtraces}{Backtrace collection continues}
  \begin{columns}[c]
    \begin{column}{0.55\textwidth}
      \minipage[c][0.75\textheight][s]{\columnwidth}
      \begin{itemize}
        \item<1-> \funcname{caml\_stash\_backtrace} scans the older part of the stack, up to the next trap
        \item<6-> Controls jumps to the nested exception handler in \funcname{maybe\_raise}, which matches only on \typename{ExnB}
        \item<7-> \funcname{caml\_reraise\_exn} is called again, backtrace collection resumes with \funcname{caml\_stash\_backtrace}
      \end{itemize}
      \medskip
      \begin{itemize}
        \item<2-> Constructed backtrace:
        \begin{itemize}
          \item <2-> \funcname{definitely\_raise}
          \item <2-> \funcname{probably\_raise}
          \item <3-> \funcname{probably\_raise} (re-raise)
          \item <4-> \funcname{maybe\_raise}
          \item <5-> \funcname{main}
          \item <8-> \funcname{main} (re-raise)
        \end{itemize}
      \end{itemize}
      \vfill
      \onslide*<1-5>{\mintocaml[firstline=15,lastline=21]{../src/backtrace/backtrace.ml}}
      \onslide*<6>{\mintocaml[firstline=28,lastline=34,highlightlines=31]{../src/backtrace/backtrace.ml}}
      \onslide*<7->{\mintocaml[firstline=28,lastline=34,highlightlines=33]{../src/backtrace/backtrace.ml}}
      \endminipage
    \end{column}
    \begin{column}{0.45\textwidth}
      \raggedleft
      \onslide*<1>{\providecommand\step{6}\input{diagrams/backtrace/backtrace.tex}}%
      \onslide*<2>{\providecommand\step{6}\input{diagrams/backtrace/backtrace.tex}}%
      \onslide*<3>{\providecommand\step{7}\input{diagrams/backtrace/backtrace.tex}}%
      \onslide*<4>{\providecommand\step{8}\input{diagrams/backtrace/backtrace.tex}}%
      \onslide*<5>{\providecommand\step{9}\input{diagrams/backtrace/backtrace.tex}}%
      \onslide*<6>{\providecommand\step{10}\input{diagrams/backtrace/backtrace.tex}}%
      \onslide*<7>{\providecommand\step{11}\input{diagrams/backtrace/backtrace.tex}}%
      \onslide*<8>{\providecommand\step{12}\input{diagrams/backtrace/backtrace.tex}}%
      \onslide*<9>{\providecommand\step{13}\input{diagrams/backtrace/backtrace.tex}}%
    \end{column}
  \end{columns}
\end{frame}

\begin{frame}{Backtraces}{Printing the backtrace}
  \begin{columns}[c]
    \begin{column}{0.45\textwidth}
      \minipage[c][0.66\textheight][s]{\columnwidth}
      \begin{itemize}
        \item<1-> The exception backtrace can now be printed to \funcarg{stdout} with \funcname{Printexc.print_backtrace}
        \item<2-> First the exception backtrace is retrieved into an OCaml array with \funcname{get_raw_backtrace }
        \item<3-> Which is implemented as the \funcname{caml_get_exception_raw_backtrace} C function in the runtime
        \item<4-> And finally printed with \funcname{print_exception_backtrace}
      \end{itemize}
      \vfill
      \mintocaml[firstline=28,lastline=34,highlightlines=33]{../src/backtrace/backtrace.ml}
      \endminipage
    \end{column}
    \begin{column}{0.55\textwidth}
      \onslide*<1,2>{
        \mintocaml[firstline=100,lastline=101]{../ocaml/stdlib/printexc.ml}
      }
      \only<1>{
        \listocaml[firstline=170,lastline=172]{../ocaml/stdlib/printexc.ml}{stdlib/printexc.ml:170}
      }
      \only<2>{
        \listocaml[firstline=170,lastline=172,highlightlines=172]{../ocaml/stdlib/printexc.ml}{stdlib/printexc.ml:170}
      }
      \only<3>{
        \mintc[firstline=154,lastline=156]{../ocaml/runtime/backtrace.c}
        \listc[firstline=171,lastline=192,highlightlines={184-187}]{../ocaml/runtime/backtrace.c}{runtime/backtrace.c:154}
      }
      \only<4>{
        \listocaml[firstline=155,lastline=165]{../ocaml/stdlib/printexc.ml}{stdlib/printexc.ml:55}
      }
    \end{column}
  \end{columns}
\end{frame}


\subsection{The multiple kinds of raise}
\frameSubsection{}{}

\begin{frame}{Backtraces}{When backtraces are not needed}
  \begin{columns}[c]
    \begin{column}{0.45\textwidth}
      \minipage[c][0.66\textheight][s]{\columnwidth}
      \begin{itemize}
        \item<1-> What if the backtrace for an exception is never needed?
        \item<2-> It's possible to raise the exception explicitly with \funcname{raise\_notrace} to make raising as cheap as possible
        \item<3-> An example of use in the \funcname{dynlink} library of the OCaml compiler
      \end{itemize}
      \vfill
      \onslide<3->{
      \listocaml[firstline=205,lastline=209,highlightlines=207]{../ocaml/otherlibs/dynlink/dynlink\_compilerlibs/misc.ml}{otherlibs/dynlink/dynlink\_compilerlibs/misc.ml:205}
      }
      \endminipage
    \end{column}
    \begin{column}{0.55\textwidth}
    \only<4>{
      \mintcmm[highlightlines=3]{../src/notrace/camlNotrace\_\_fun_347.cmm}
      \listcmm{../src/notrace/camlNotrace\_\_all\_somes\_327.cmm}{all\_somes}
    }
    \onslide*<5->{
      \listobjdump[highlightlines={6-9}]{../src/notrace/camlNotrace\_\_fun_347.objdump}{anonymous function from \funcname{all\_somes}}
    }
    \end{column}
  \end{columns}
\end{frame}

\begin{frame}{Backtraces}{Summary table of the multiple kinds of raise}
  \begin{itemize}
    \item When debugging information is enabled and if the backtrace of an exception isn't needed, it's possible to raise with \funcname{raise\_notrace} for performance
    \item When debugging information is \emph{not} enabled, the compiler optimises \funcname{raise} calls to inline assembly (same effect as using \funcname{raise\_notrace})
  \end{itemize}
  \bigskip
  \centering
  \begin{tabular}{| l | c | c | c | c |}
    \hline
    \parbox{10em}{Debug information \\ (\funcarg{-g} flag)}  & \multicolumn{2}{|c|}{Disabled}                                     & \multicolumn{2}{|c|}{Enabled} \\
    \hline
    Raise kind                                               & \funcname{raise} & \funcname{raise\_notrace}                       & \funcname{raise} & \funcname{raise\_notrace} \\
    \hline
    Compilation                                              & \parbox{8em}{inline assembly\\ (optimization)} & inline assembly   & \funcname{caml\_raise\_exn} & inline assembly \\
    \hline
  \end{tabular}
\end{frame}


%
% Takeaway #5
%
\subsection*{Takeaway \#5}
\frameSubsectionTakeaway{}
\againframe<5>{takeaway}
}%
      \onslide*<2>{\providecommand\step{1}\input{diagrams/backtrace/scanstack.tex}}%
      \onslide*<3>{\providecommand\step{2}\input{diagrams/backtrace/scanstack.tex}}%
      \onslide*<4>{\providecommand\step{3}\input{diagrams/backtrace/scanstack.tex}}%
      \onslide*<5>{\providecommand\step{4}\input{diagrams/backtrace/scanstack.tex}}%
      \onslide*<6>{\providecommand\step{5}\input{diagrams/backtrace/scanstack.tex}}%
    \end{column}
  \end{columns}
\end{frame}

% Duplicate of previous-previous slide
\begin{frame}{Backtraces}{Continuing on \funcname{caml\_stash\_backtrace}}
  \begin{columns}[c]
    \begin{column}{0.5\textwidth}
      \minipage[c][0.75\textheight][s]{\columnwidth}
      \begin{itemize}
          \color{gray}
        \item A C function provided by the runtime (specific to native code)
        \item It scans the stack, between the stack pointer at the raising point up to the next trap, in order to collect the names of the detected function frames
        \item It uses the same technique and information as the Garbage Collector (GC) uses to scan the values live on the stack
      \end{itemize}
      \begin{itemize}
        \item For each function frame detected, store a pointer to the \typename{frame\_descr} in the \localname{domain\_state->backtrace\_buffer} array, effectively constructing the backtrace
      \end{itemize}
      \onslide<2->{
        \begin{itemize}
            \smallskip
          \item Constructed backtrace:
            \funcname{definitely\_raise},
            \onslide<3->{\funcname{probably\_raise}}
        \end{itemize}
      }
      \vfill
      \mintocaml[firstline=9,lastline=13,highlightlines=12]{../src/backtrace/backtrace.ml}
      \endminipage
    \end{column}
    \begin{column}{0.5\textwidth}
      \raggedleft
      \onslide*<1>{\listStashBacktrace[highlightlines={114-115}]{}}
      \onslide*<2>{\providecommand\step{3}%
% Backtraces
%
\subsection{One advantage of raising with debugging information}
\frameSubsection{}{}

\begin{frame}{Backtraces}{Debugging information \& source code}
  \begin{columns}[c]
    \begin{column}{0.5\textwidth}
      \begin{itemize}
        \item Raising an exception is fun and games, but how do we know where the exception originated? $\rightarrow$ With the backtrace associated with the exception!
        \item In order to get a backtrace from the point where an exception was raised, it's necessary to compile with debugging information enabled (\funcarg{-g} flag for the compiler)
        \item Same example as in the `Nested handlers' section
      \end{itemize}
    \end{column}
    \begin{column}{0.5\textwidth}
      \listocaml[firstline=9,lastline=34]{../src/backtrace/backtrace.ml}{backtrace/backtrace.ml}
    \end{column}
  \end{columns}
\end{frame}

\begin{frame}{Backtraces}{CMM and disassembly of \funcname{definitely\_raise}}
  \begin{columns}[c]
    \begin{column}{0.5\textwidth}
      \minipage[c][0.66\textheight][s]{\columnwidth}
      \begin{itemize}
        \item<1-> An effect of compiling with debugging information (\funcarg{-g}): the CMM no longer uses \funcname{raise\_notrace} to raise the exception but \funcname{raise} instead
        \item<2-> Disassembly shows that CMM \funcname{raise} is actually a call to \funcname{caml\_raise\_exn}, a function in assembler provided by the runtime
      \end{itemize}
      \vfill
      \mintocaml[firstline=9,lastline=13,highlightlines=12]{../src/backtrace/backtrace.ml}
      \endminipage
    \end{column}
    \begin{column}{0.5\textwidth}
      \onslide*<1>{
        \mintcmm[highlightlines=5]{../src/backtrace/camlBacktrace\_\_definitely\_raise\_347.cmm}
      }
      \onslide*<2>{
        \mintobjdump[firstline=2,highlightlines=14]{../src/backtrace/camlBacktrace\_\_definitely\_raise\_347.objdump}
      }
    \end{column}
  \end{columns}
\end{frame}

\begin{frame}{Backtraces}{Introducing \funcname{caml\_raise\_exn}}
  \begin{columns}[c]
    \begin{column}{0.5\textwidth}
      \minipage[c][0.66\textheight][s]{\columnwidth}
      \onslide*<1>{
        \begin{itemize}
          \item Raising the exception is performed using \funcname{caml\_raise\_exn} which is
            \begin{itemize}
              \item An assembly function provided by the runtime
              \item Architecture specific
            \end{itemize}
          \item Let's see what it does differently from \funcname{raise\_notrace}\dots
          \item We are about to raise \typename{ExnC} with \funcname{caml\_raise\_exn} from \funcname{definitely\_raise}
        \end{itemize}
      }
      \onslide*<2>{
        \mintobjdump[firstline=2,highlightlines=14]{../src/backtrace/camlBacktrace\_\_definitely\_raise\_347.objdump}
      }
      \vfill
      \mintocaml[firstline=9,lastline=13,highlightlines=12]{../src/backtrace/backtrace.ml}
      \endminipage
    \end{column}
    \begin{column}{0.5\textwidth}
      \centering
      \providecommand\step{1}\input{diagrams/backtrace/backtrace.tex}
    \end{column}
  \end{columns}
\end{frame}

\begin{frame}{Backtraces}{But before that, we need more fields from \typename{caml\_domain\_state}}
  \begin{columns}
    \begin{column}{0.5\textwidth}
      \begin{itemize}
        \item Remember \typename{caml\_domain\_state} the per-domain runtime state?
        \item Some other members are of interest to us now\dots
        \item \localname{backtrace\_active} is used to know if the runtime should collect backtraces
        \item \localname{backtrace\_active} can be set by
          \begin{itemize}
            \item The \funcarg{b} flag in the \funcname{OCAMLRUNPARAM} environment variable
            \item \mintinline{ocaml}{Printexc.record_backtrace : bool -> unit} \footnotemark
          \end{itemize}
      \end{itemize}
    \end{column}
    \begin{column}{0.5\textwidth}
      \begin{listing}
        \mintc[highlightlines={9-11}]{src/ocaml/domain\_state\_backtrace.h}
        \caption{runtime/caml/domain\_state.h}
      \end{listing}
    \end{column}
  \end{columns}
  \footnotetext[1]{https://v2.ocaml.org/api/Printexc.html}
\end{frame}

\newcommand\listCamlRaiseExn[1][]{
  \listasm[firstline=702,lastline=725,#1]{../ocaml/runtime/amd64.S}{runtime/amd64.S:702}}

\begin{frame}{Backtraces}{Walk through \funcname{caml\_raise\_exn}}
  \begin{columns}[T]
    \begin{column}{0.5\textwidth}
      \begin{itemize}
        \item<1-> First checks whether exceptions backtrace should be recorded
        \item<1-> By checking the \funcarg{domain\_state->backtrace\_active} value
        \item<2-> If not, acts as \funcname{raise\_notrace} with \funcname{RESTORE\_EXN\_HANDLER\_OCAML}
          \begin{itemize}
            \item Set \regname{RSP} to \localname{domain\_state->exn\_handler} value
            \item Restore \localname{domain\_state->exn\_handler} from trap
            \item Jump with \funcname{ret} (same effect as using \funcname{pop} and \funcname{jmp})
          \end{itemize}
        \item<3-> Otherwise, switches to the C stack to be able to execute C code
        \item<4-> Calls \funcname{caml\_stash\_backtrace}, a C function provided by the runtime\ignorespaces
          \onslide<6->{, with
            \begin{itemize}
              \item \funcarg{exn}: exception value
              \item \funcarg{pc}: code address of the raising point
              \item \funcarg{sp}: stack address of the raising function
              \item \funcarg{trapsp}: stack address of the next handler
            \end{itemize}
          }
      \end{itemize}
      \bigskip
      \mintocaml[firstline=9,lastline=13,highlightlines=12]{../src/backtrace/backtrace.ml}
    \end{column}
    \begin{column}{0.5\textwidth}
      \raggedleft
      \onslide*<1,3-4>{
        \mintc[firstline=190,lastline=192]{../ocaml/runtime/amd64.S}
        \mintc[firstline=234,lastline=237]{../ocaml/runtime/amd64.S}
      }
      \onslide*<2>{
        \mintc[firstline=190,lastline=192,highlightlines={191-192}]{../ocaml/runtime/amd64.S}
        \mintc[firstline=234,lastline=237,highlightlines={235-237}]{../ocaml/runtime/amd64.S}
      }
      \onslide*<1>{\listCamlRaiseExn[highlightlines=705]{}}%
      \onslide*<2>{\listCamlRaiseExn[highlightlines={707-708}]{}}%
      \onslide*<3>{\listCamlRaiseExn[highlightlines={712-714}]{}}%
      \onslide*<4>{\listCamlRaiseExn[highlightlines={715-720}]{}}%
      \onslide*<5>{\providecommand\step{1}\input{diagrams/backtrace/backtrace.tex}}%
      \onslide*<6>{\providecommand\step{2}\input{diagrams/backtrace/backtrace.tex}}%
    \end{column}
  \end{columns}
\end{frame}

\newcommand\listStashBacktrace[1][]{
  \listc[firstline=91,lastline=120,#1]{../ocaml/runtime/backtrace\_nat.c}{runtime/backtrace\_nat.c:91}}

\begin{frame}{Backtraces}{Introducing \funcname{caml\_stash\_backtrace}}
  \begin{columns}[c]
    \begin{column}{0.5\textwidth}
      \minipage[c][0.66\textheight][s]{\columnwidth}
      \begin{itemize}
        \item<1-> A C function provided by the runtime (specific to native code)
        \item<2-> It scans the stack, between the stack pointer at the raising point up to the next trap, in order to collect the names of the detected function frames
          \onslide*<2>{
            \begin{itemize}
              \item And more information, like line number, column number...
            \end{itemize}
          }
        \item<3-> It uses the same technique and information as the Garbage Collector (GC) uses to scan the values live on the stack
          \onslide*<4>{
            \begin{itemize}
              \item \typename{frame\_descr} are at the core of stack unwinding
            \end{itemize}
          }
      \end{itemize}
      \vfill
      \mintocaml[firstline=9,lastline=13,highlightlines=12]{../src/backtrace/backtrace.ml}
      \endminipage
    \end{column}
    \begin{column}{0.5\textwidth}
      \onslide*<1>{\listStashBacktrace[highlightlines=91]{}}
      \onslide*<2>{\listStashBacktrace[highlightlines={91,117-118}]{}}
      \onslide*<3->{\listStashBacktrace[highlightlines={94,106,109-110}]{}}
    \end{column}
  \end{columns}
\end{frame}

\newcommand\listFrameDescr[1][]{
  \listocaml[firstline=111,lastline=124,#1]{../ocaml/asmcomp/emitaux.ml}{asmcomp/emitaux.ml:111}}
\newcommand\listDebugInfo[1][]{
  \listocaml[firstline=38,lastline=49,#1]{../ocaml/lambda/debuginfo.mli}{lambda/debuginfo.mli:38}}

\begin{frame}{Backtraces}{\typename{frame\_descr} and \typename{frame\_debuginfo} in a nutshell}
  \begin{columns}[c]
    \begin{column}{0.5\textwidth}
      \begin{itemize}
        \item<1-> For every return address in an OCaml program, there is a associated \typename{frame\_descr}
        \item<2-> A \typename{frame\_descr} specifies
        \begin{itemize}
          \item<2-> The size of the function stack frame
          \item<3-> Where live values are on the stack (so that the GC can mark them as live and prevent from being swept)
        \end{itemize}
        \item<4-> When compiled with debug information, \typename{frame\_descr} will be linked to a \typename{frame\_debuginfo}
        \begin{itemize}
          \item<5-> Contains information about the position in the source code (file name, line and column number)
        \end{itemize}
      \end{itemize}
    \end{column}
    \begin{column}{0.5\textwidth}
        \onslide*<1>{\listFrameDescr[highlightlines={118-122}]{}}
        \onslide*<2>{\listFrameDescr[highlightlines={120}]{}}
        \onslide*<3>{\listFrameDescr[highlightlines={121}]{}}
        \onslide*<4->{\listFrameDescr[highlightlines={122}]{}}
        \medskip
        \onslide*<1-3>{\listDebugInfo{}}
        \onslide*<4>{\listDebugInfo[highlightlines={38,49}]{}}
        \onslide*<5>{\listDebugInfo[highlightlines={39-46}]{}}
    \end{column}
  \end{columns}
\end{frame}

\begin{frame}{Backtraces}{Scanning the stack with \typename{frame\_descr}}
  \begin{columns}[c]
    \begin{column}{0.5\textwidth}
      \begin{itemize}
        \item Given the return address of the code that raises the exception by calling \funcname{caml\_raise\_exn} (\funcarg{pc} argument of \funcname{caml\_stash\_backtrace})
        \item And the position of the stack pointer (\funcarg{sp} argument of \funcname{caml\_stash\_backtrace})
        \item The frame size given by \typename{frame\_descr}.\funcarg{frame\_size}, allows to find the end of the previous frame
        \item And the return address of the caller of this function
        \bigskip
        \onslide<3->{
        \item Note that traps (here the reddish \funcname{ExnA}, \funcname{ExnB} and \funcname{ExnC} blocks) belong to the frame that installed them, and so the frame size changes when traps are removed
        }
      \end{itemize}
    \end{column}
    \begin{column}{0.5\textwidth}
      \raggedleft
      \onslide*<1>{\providecommand\step{2}\input{diagrams/backtrace/backtrace.tex}}%
      \onslide*<2>{\providecommand\step{1}\input{diagrams/backtrace/scanstack.tex}}%
      \onslide*<3>{\providecommand\step{2}\input{diagrams/backtrace/scanstack.tex}}%
      \onslide*<4>{\providecommand\step{3}\input{diagrams/backtrace/scanstack.tex}}%
      \onslide*<5>{\providecommand\step{4}\input{diagrams/backtrace/scanstack.tex}}%
      \onslide*<6>{\providecommand\step{5}\input{diagrams/backtrace/scanstack.tex}}%
    \end{column}
  \end{columns}
\end{frame}

% Duplicate of previous-previous slide
\begin{frame}{Backtraces}{Continuing on \funcname{caml\_stash\_backtrace}}
  \begin{columns}[c]
    \begin{column}{0.5\textwidth}
      \minipage[c][0.75\textheight][s]{\columnwidth}
      \begin{itemize}
          \color{gray}
        \item A C function provided by the runtime (specific to native code)
        \item It scans the stack, between the stack pointer at the raising point up to the next trap, in order to collect the names of the detected function frames
        \item It uses the same technique and information as the Garbage Collector (GC) uses to scan the values live on the stack
      \end{itemize}
      \begin{itemize}
        \item For each function frame detected, store a pointer to the \typename{frame\_descr} in the \localname{domain\_state->backtrace\_buffer} array, effectively constructing the backtrace
      \end{itemize}
      \onslide<2->{
        \begin{itemize}
            \smallskip
          \item Constructed backtrace:
            \funcname{definitely\_raise},
            \onslide<3->{\funcname{probably\_raise}}
        \end{itemize}
      }
      \vfill
      \mintocaml[firstline=9,lastline=13,highlightlines=12]{../src/backtrace/backtrace.ml}
      \endminipage
    \end{column}
    \begin{column}{0.5\textwidth}
      \raggedleft
      \onslide*<1>{\listStashBacktrace[highlightlines={114-115}]{}}
      \onslide*<2>{\providecommand\step{3}\input{diagrams/backtrace/backtrace.tex}}%
      \onslide*<3>{\providecommand\step{4}\input{diagrams/backtrace/backtrace.tex}}%
    \end{column}
  \end{columns}
\end{frame}

\begin{frame}{Backtraces}{Back to \funcname{caml\_raise\_exn}}
  \begin{columns}[c]
    \begin{column}{0.5\textwidth}
      \minipage[c][0.66\textheight][s]{\columnwidth}
      \begin{itemize}\color{gray}
        \item Checks wheteher exceptions backtrace should be recorded, by checking the \funcarg{domain\_state->backtrace\_active} value
        \item Switches to the C stack to be able to execute C code
        \item Calls \funcname{caml\_stash\_backtrace}, to collect the backtrace up to the next trap
      \end{itemize}
      \onslide*<2->{
        \begin{itemize}
          \item<2-> Executes the exception handler in \funcname{probably\_raise} with \funcname{RESTORE\_EXN\_HANDLER\_OCAML}
            \begin{itemize}
              \item Set \regname{RSP} to \localname{domain\_state->exn\_handler} value
              \item Restore \localname{domain\_state->exn\_handler} from trap
              \item Jump to the handler code with \funcname{ret}
            \end{itemize}
        \end{itemize}
      }
      \vfill
      \onslide*<1>{\mintocaml[firstline=9,lastline=13,highlightlines=12]{../src/backtrace/backtrace.ml}}
      \onslide*<2->{\mintocaml[firstline=15,lastline=21,highlightlines={19-21}]{../src/backtrace/backtrace.ml}}
      \endminipage
    \end{column}
    \begin{column}{0.5\textwidth}
      \centering
      \onslide*<1>{
        \mintc[firstline=190,lastline=192]{../ocaml/runtime/amd64.S}
        \mintc[firstline=234,lastline=237]{../ocaml/runtime/amd64.S}
      }
      \onslide*<2>{
        \mintc[firstline=190,lastline=192,highlightlines={191-192}]{../ocaml/runtime/amd64.S}
        \mintc[firstline=234,lastline=237,highlightlines={235-237}]{../ocaml/runtime/amd64.S}
      }
      \onslide*<1>{\listCamlRaiseExn[highlightlines=720]{}}
      \onslide*<2>{\listCamlRaiseExn[highlightlines=722]{}}
      \onslide*<3->{\providecommand\step{5}\input{diagrams/backtrace/backtrace.tex}}
    \end{column}
  \end{columns}
\end{frame}

\begin{frame}{Backtraces}{\funcname{caml\_probably\_raise}'s handler doesn't catch \typename{ExnC}}
  \begin{columns}[c]
    \begin{column}{0.45\textwidth}
      \minipage[c][0.75\textheight][s]{\columnwidth}
      \begin{itemize}
        \item<1-> \funcname{match \dots with}\footnotemark pattern matching can also match exceptions
        \item<1-> The CMM of \funcname{probably\_raise} shows that this \funcname{match \dots with} is implemented with a pattern matching and a \funcname{try \dots with}
        \item<1-> But this one only matches on \typename{ExnA} exception
        \item<2-> Since the pattern matching fails to catch the exception, \funcname{reraise} is called with our current \typename{ExnC} exception
        \item<3-> Disassembly shows that \funcname{reraise} is actually a call to \funcname{caml\_reraise\_exn}, which is also an assembly function provided by the runtime
      \end{itemize}
      \vfill
      \mintocaml[firstline=15,lastline=21,highlightlines={18-21}]{../src/backtrace/backtrace.ml}
      \endminipage
    \end{column}
    \begin{column}{0.55\textwidth}
      \centering
      \onslide*<1>{\mintcmm[highlightlines={10-18}]{../src/backtrace/camlBacktrace\_\_probably\_raise\_351.cmm}}
      \onslide*<2>{\listcmm[highlightlines=18]{../src/backtrace/camlBacktrace\_\_probably\_raise_351.cmm}{CMM of probably\_raise}}
      \onslide*<3>{\mintobjdump[firstline=5,lastline=35,highlightlines=31]{../src/backtrace/camlBacktrace\_\_probably\_raise_351.objdump}}
    \end{column}
  \end{columns}
  \only<1>{\footnotetext[1]{https://blog.janestreet.com/pattern-matching-and-exception-handling-unite/}}
\end{frame}

\newcommand\listCamlReraiseExn[1][]{
  \listasm[firstline=727,lastline=734,#1]{../ocaml/runtime/amd64.S}{runtime/amd64.S:727}}

\begin{frame}{Backtraces}{Introducing \funcname{caml\_reraise\_exn}}
  \begin{columns}[c]
    \begin{column}{0.5\textwidth}
      \minipage[c][0.66\textheight][s]{\columnwidth}
      \begin{itemize}
        \item<1-> The exception have to be reraised to the next handler
        \item<2-> First, checks if the backtrace should be collected with \funcarg{domain\_state->backtrace\_active}
        \only<3>{
        \begin{itemize}
            \item If not, behaves like a \funcname{raise\_notrace} with \funcname{RESTORE\_EXN\_HANDLER\_OCAML}
        \end{itemize}
        }
        \item<4-> Jumps into \funcname{caml\_raise\_exn}
        \item<5-> Calls \funcname{caml\_stash\_backtrace} again, to scan the next part of the stack: from the trap just pop-ed to the next trap
      \end{itemize}
      \vfill
      \mintocaml[firstline=15,lastline=21,highlightlines={18-21}]{../src/backtrace/backtrace.ml}
      \endminipage
    \end{column}
    \begin{column}{0.5\textwidth}
      \raggedleft
      \onslide*<1>{\listCamlReraiseExn{}}
      \onslide*<2>{\listCamlReraiseExn[highlightlines=729]{}}
      \onslide*<3>{
        \mintc[firstline=190,lastline=192,highlightlines={191-192}]{../ocaml/runtime/amd64.S}
        \mintc[firstline=234,lastline=237,highlightlines={235-237}]{../ocaml/runtime/amd64.S}
        \medskip
        \listCamlReraiseExn[highlightlines=731]{}
      }
      \onslide*<4-5>{
        % TODO refactor with \listCamlReraiseExn
        \mintasm[firstline=727,lastline=734,highlightlines=730]{../ocaml/runtime/amd64.S}
        \medskip
      }
      \onslide*<4>{\listCamlRaiseExn[highlightlines=711]{}}
      \onslide*<5>{\listCamlRaiseExn[highlightlines=720]{}}
    \end{column}
  \end{columns}
\end{frame}

\begin{frame}{Backtraces}{Backtrace collection continues}
  \begin{columns}[c]
    \begin{column}{0.55\textwidth}
      \minipage[c][0.75\textheight][s]{\columnwidth}
      \begin{itemize}
        \item<1-> \funcname{caml\_stash\_backtrace} scans the older part of the stack, up to the next trap
        \item<6-> Controls jumps to the nested exception handler in \funcname{maybe\_raise}, which matches only on \typename{ExnB}
        \item<7-> \funcname{caml\_reraise\_exn} is called again, backtrace collection resumes with \funcname{caml\_stash\_backtrace}
      \end{itemize}
      \medskip
      \begin{itemize}
        \item<2-> Constructed backtrace:
        \begin{itemize}
          \item <2-> \funcname{definitely\_raise}
          \item <2-> \funcname{probably\_raise}
          \item <3-> \funcname{probably\_raise} (re-raise)
          \item <4-> \funcname{maybe\_raise}
          \item <5-> \funcname{main}
          \item <8-> \funcname{main} (re-raise)
        \end{itemize}
      \end{itemize}
      \vfill
      \onslide*<1-5>{\mintocaml[firstline=15,lastline=21]{../src/backtrace/backtrace.ml}}
      \onslide*<6>{\mintocaml[firstline=28,lastline=34,highlightlines=31]{../src/backtrace/backtrace.ml}}
      \onslide*<7->{\mintocaml[firstline=28,lastline=34,highlightlines=33]{../src/backtrace/backtrace.ml}}
      \endminipage
    \end{column}
    \begin{column}{0.45\textwidth}
      \raggedleft
      \onslide*<1>{\providecommand\step{6}\input{diagrams/backtrace/backtrace.tex}}%
      \onslide*<2>{\providecommand\step{6}\input{diagrams/backtrace/backtrace.tex}}%
      \onslide*<3>{\providecommand\step{7}\input{diagrams/backtrace/backtrace.tex}}%
      \onslide*<4>{\providecommand\step{8}\input{diagrams/backtrace/backtrace.tex}}%
      \onslide*<5>{\providecommand\step{9}\input{diagrams/backtrace/backtrace.tex}}%
      \onslide*<6>{\providecommand\step{10}\input{diagrams/backtrace/backtrace.tex}}%
      \onslide*<7>{\providecommand\step{11}\input{diagrams/backtrace/backtrace.tex}}%
      \onslide*<8>{\providecommand\step{12}\input{diagrams/backtrace/backtrace.tex}}%
      \onslide*<9>{\providecommand\step{13}\input{diagrams/backtrace/backtrace.tex}}%
    \end{column}
  \end{columns}
\end{frame}

\begin{frame}{Backtraces}{Printing the backtrace}
  \begin{columns}[c]
    \begin{column}{0.45\textwidth}
      \minipage[c][0.66\textheight][s]{\columnwidth}
      \begin{itemize}
        \item<1-> The exception backtrace can now be printed to \funcarg{stdout} with \funcname{Printexc.print_backtrace}
        \item<2-> First the exception backtrace is retrieved into an OCaml array with \funcname{get_raw_backtrace }
        \item<3-> Which is implemented as the \funcname{caml_get_exception_raw_backtrace} C function in the runtime
        \item<4-> And finally printed with \funcname{print_exception_backtrace}
      \end{itemize}
      \vfill
      \mintocaml[firstline=28,lastline=34,highlightlines=33]{../src/backtrace/backtrace.ml}
      \endminipage
    \end{column}
    \begin{column}{0.55\textwidth}
      \onslide*<1,2>{
        \mintocaml[firstline=100,lastline=101]{../ocaml/stdlib/printexc.ml}
      }
      \only<1>{
        \listocaml[firstline=170,lastline=172]{../ocaml/stdlib/printexc.ml}{stdlib/printexc.ml:170}
      }
      \only<2>{
        \listocaml[firstline=170,lastline=172,highlightlines=172]{../ocaml/stdlib/printexc.ml}{stdlib/printexc.ml:170}
      }
      \only<3>{
        \mintc[firstline=154,lastline=156]{../ocaml/runtime/backtrace.c}
        \listc[firstline=171,lastline=192,highlightlines={184-187}]{../ocaml/runtime/backtrace.c}{runtime/backtrace.c:154}
      }
      \only<4>{
        \listocaml[firstline=155,lastline=165]{../ocaml/stdlib/printexc.ml}{stdlib/printexc.ml:55}
      }
    \end{column}
  \end{columns}
\end{frame}


\subsection{The multiple kinds of raise}
\frameSubsection{}{}

\begin{frame}{Backtraces}{When backtraces are not needed}
  \begin{columns}[c]
    \begin{column}{0.45\textwidth}
      \minipage[c][0.66\textheight][s]{\columnwidth}
      \begin{itemize}
        \item<1-> What if the backtrace for an exception is never needed?
        \item<2-> It's possible to raise the exception explicitly with \funcname{raise\_notrace} to make raising as cheap as possible
        \item<3-> An example of use in the \funcname{dynlink} library of the OCaml compiler
      \end{itemize}
      \vfill
      \onslide<3->{
      \listocaml[firstline=205,lastline=209,highlightlines=207]{../ocaml/otherlibs/dynlink/dynlink\_compilerlibs/misc.ml}{otherlibs/dynlink/dynlink\_compilerlibs/misc.ml:205}
      }
      \endminipage
    \end{column}
    \begin{column}{0.55\textwidth}
    \only<4>{
      \mintcmm[highlightlines=3]{../src/notrace/camlNotrace\_\_fun_347.cmm}
      \listcmm{../src/notrace/camlNotrace\_\_all\_somes\_327.cmm}{all\_somes}
    }
    \onslide*<5->{
      \listobjdump[highlightlines={6-9}]{../src/notrace/camlNotrace\_\_fun_347.objdump}{anonymous function from \funcname{all\_somes}}
    }
    \end{column}
  \end{columns}
\end{frame}

\begin{frame}{Backtraces}{Summary table of the multiple kinds of raise}
  \begin{itemize}
    \item When debugging information is enabled and if the backtrace of an exception isn't needed, it's possible to raise with \funcname{raise\_notrace} for performance
    \item When debugging information is \emph{not} enabled, the compiler optimises \funcname{raise} calls to inline assembly (same effect as using \funcname{raise\_notrace})
  \end{itemize}
  \bigskip
  \centering
  \begin{tabular}{| l | c | c | c | c |}
    \hline
    \parbox{10em}{Debug information \\ (\funcarg{-g} flag)}  & \multicolumn{2}{|c|}{Disabled}                                     & \multicolumn{2}{|c|}{Enabled} \\
    \hline
    Raise kind                                               & \funcname{raise} & \funcname{raise\_notrace}                       & \funcname{raise} & \funcname{raise\_notrace} \\
    \hline
    Compilation                                              & \parbox{8em}{inline assembly\\ (optimization)} & inline assembly   & \funcname{caml\_raise\_exn} & inline assembly \\
    \hline
  \end{tabular}
\end{frame}


%
% Takeaway #5
%
\subsection*{Takeaway \#5}
\frameSubsectionTakeaway{}
\againframe<5>{takeaway}
}%
      \onslide*<3>{\providecommand\step{4}%
% Backtraces
%
\subsection{One advantage of raising with debugging information}
\frameSubsection{}{}

\begin{frame}{Backtraces}{Debugging information \& source code}
  \begin{columns}[c]
    \begin{column}{0.5\textwidth}
      \begin{itemize}
        \item Raising an exception is fun and games, but how do we know where the exception originated? $\rightarrow$ With the backtrace associated with the exception!
        \item In order to get a backtrace from the point where an exception was raised, it's necessary to compile with debugging information enabled (\funcarg{-g} flag for the compiler)
        \item Same example as in the `Nested handlers' section
      \end{itemize}
    \end{column}
    \begin{column}{0.5\textwidth}
      \listocaml[firstline=9,lastline=34]{../src/backtrace/backtrace.ml}{backtrace/backtrace.ml}
    \end{column}
  \end{columns}
\end{frame}

\begin{frame}{Backtraces}{CMM and disassembly of \funcname{definitely\_raise}}
  \begin{columns}[c]
    \begin{column}{0.5\textwidth}
      \minipage[c][0.66\textheight][s]{\columnwidth}
      \begin{itemize}
        \item<1-> An effect of compiling with debugging information (\funcarg{-g}): the CMM no longer uses \funcname{raise\_notrace} to raise the exception but \funcname{raise} instead
        \item<2-> Disassembly shows that CMM \funcname{raise} is actually a call to \funcname{caml\_raise\_exn}, a function in assembler provided by the runtime
      \end{itemize}
      \vfill
      \mintocaml[firstline=9,lastline=13,highlightlines=12]{../src/backtrace/backtrace.ml}
      \endminipage
    \end{column}
    \begin{column}{0.5\textwidth}
      \onslide*<1>{
        \mintcmm[highlightlines=5]{../src/backtrace/camlBacktrace\_\_definitely\_raise\_347.cmm}
      }
      \onslide*<2>{
        \mintobjdump[firstline=2,highlightlines=14]{../src/backtrace/camlBacktrace\_\_definitely\_raise\_347.objdump}
      }
    \end{column}
  \end{columns}
\end{frame}

\begin{frame}{Backtraces}{Introducing \funcname{caml\_raise\_exn}}
  \begin{columns}[c]
    \begin{column}{0.5\textwidth}
      \minipage[c][0.66\textheight][s]{\columnwidth}
      \onslide*<1>{
        \begin{itemize}
          \item Raising the exception is performed using \funcname{caml\_raise\_exn} which is
            \begin{itemize}
              \item An assembly function provided by the runtime
              \item Architecture specific
            \end{itemize}
          \item Let's see what it does differently from \funcname{raise\_notrace}\dots
          \item We are about to raise \typename{ExnC} with \funcname{caml\_raise\_exn} from \funcname{definitely\_raise}
        \end{itemize}
      }
      \onslide*<2>{
        \mintobjdump[firstline=2,highlightlines=14]{../src/backtrace/camlBacktrace\_\_definitely\_raise\_347.objdump}
      }
      \vfill
      \mintocaml[firstline=9,lastline=13,highlightlines=12]{../src/backtrace/backtrace.ml}
      \endminipage
    \end{column}
    \begin{column}{0.5\textwidth}
      \centering
      \providecommand\step{1}\input{diagrams/backtrace/backtrace.tex}
    \end{column}
  \end{columns}
\end{frame}

\begin{frame}{Backtraces}{But before that, we need more fields from \typename{caml\_domain\_state}}
  \begin{columns}
    \begin{column}{0.5\textwidth}
      \begin{itemize}
        \item Remember \typename{caml\_domain\_state} the per-domain runtime state?
        \item Some other members are of interest to us now\dots
        \item \localname{backtrace\_active} is used to know if the runtime should collect backtraces
        \item \localname{backtrace\_active} can be set by
          \begin{itemize}
            \item The \funcarg{b} flag in the \funcname{OCAMLRUNPARAM} environment variable
            \item \mintinline{ocaml}{Printexc.record_backtrace : bool -> unit} \footnotemark
          \end{itemize}
      \end{itemize}
    \end{column}
    \begin{column}{0.5\textwidth}
      \begin{listing}
        \mintc[highlightlines={9-11}]{src/ocaml/domain\_state\_backtrace.h}
        \caption{runtime/caml/domain\_state.h}
      \end{listing}
    \end{column}
  \end{columns}
  \footnotetext[1]{https://v2.ocaml.org/api/Printexc.html}
\end{frame}

\newcommand\listCamlRaiseExn[1][]{
  \listasm[firstline=702,lastline=725,#1]{../ocaml/runtime/amd64.S}{runtime/amd64.S:702}}

\begin{frame}{Backtraces}{Walk through \funcname{caml\_raise\_exn}}
  \begin{columns}[T]
    \begin{column}{0.5\textwidth}
      \begin{itemize}
        \item<1-> First checks whether exceptions backtrace should be recorded
        \item<1-> By checking the \funcarg{domain\_state->backtrace\_active} value
        \item<2-> If not, acts as \funcname{raise\_notrace} with \funcname{RESTORE\_EXN\_HANDLER\_OCAML}
          \begin{itemize}
            \item Set \regname{RSP} to \localname{domain\_state->exn\_handler} value
            \item Restore \localname{domain\_state->exn\_handler} from trap
            \item Jump with \funcname{ret} (same effect as using \funcname{pop} and \funcname{jmp})
          \end{itemize}
        \item<3-> Otherwise, switches to the C stack to be able to execute C code
        \item<4-> Calls \funcname{caml\_stash\_backtrace}, a C function provided by the runtime\ignorespaces
          \onslide<6->{, with
            \begin{itemize}
              \item \funcarg{exn}: exception value
              \item \funcarg{pc}: code address of the raising point
              \item \funcarg{sp}: stack address of the raising function
              \item \funcarg{trapsp}: stack address of the next handler
            \end{itemize}
          }
      \end{itemize}
      \bigskip
      \mintocaml[firstline=9,lastline=13,highlightlines=12]{../src/backtrace/backtrace.ml}
    \end{column}
    \begin{column}{0.5\textwidth}
      \raggedleft
      \onslide*<1,3-4>{
        \mintc[firstline=190,lastline=192]{../ocaml/runtime/amd64.S}
        \mintc[firstline=234,lastline=237]{../ocaml/runtime/amd64.S}
      }
      \onslide*<2>{
        \mintc[firstline=190,lastline=192,highlightlines={191-192}]{../ocaml/runtime/amd64.S}
        \mintc[firstline=234,lastline=237,highlightlines={235-237}]{../ocaml/runtime/amd64.S}
      }
      \onslide*<1>{\listCamlRaiseExn[highlightlines=705]{}}%
      \onslide*<2>{\listCamlRaiseExn[highlightlines={707-708}]{}}%
      \onslide*<3>{\listCamlRaiseExn[highlightlines={712-714}]{}}%
      \onslide*<4>{\listCamlRaiseExn[highlightlines={715-720}]{}}%
      \onslide*<5>{\providecommand\step{1}\input{diagrams/backtrace/backtrace.tex}}%
      \onslide*<6>{\providecommand\step{2}\input{diagrams/backtrace/backtrace.tex}}%
    \end{column}
  \end{columns}
\end{frame}

\newcommand\listStashBacktrace[1][]{
  \listc[firstline=91,lastline=120,#1]{../ocaml/runtime/backtrace\_nat.c}{runtime/backtrace\_nat.c:91}}

\begin{frame}{Backtraces}{Introducing \funcname{caml\_stash\_backtrace}}
  \begin{columns}[c]
    \begin{column}{0.5\textwidth}
      \minipage[c][0.66\textheight][s]{\columnwidth}
      \begin{itemize}
        \item<1-> A C function provided by the runtime (specific to native code)
        \item<2-> It scans the stack, between the stack pointer at the raising point up to the next trap, in order to collect the names of the detected function frames
          \onslide*<2>{
            \begin{itemize}
              \item And more information, like line number, column number...
            \end{itemize}
          }
        \item<3-> It uses the same technique and information as the Garbage Collector (GC) uses to scan the values live on the stack
          \onslide*<4>{
            \begin{itemize}
              \item \typename{frame\_descr} are at the core of stack unwinding
            \end{itemize}
          }
      \end{itemize}
      \vfill
      \mintocaml[firstline=9,lastline=13,highlightlines=12]{../src/backtrace/backtrace.ml}
      \endminipage
    \end{column}
    \begin{column}{0.5\textwidth}
      \onslide*<1>{\listStashBacktrace[highlightlines=91]{}}
      \onslide*<2>{\listStashBacktrace[highlightlines={91,117-118}]{}}
      \onslide*<3->{\listStashBacktrace[highlightlines={94,106,109-110}]{}}
    \end{column}
  \end{columns}
\end{frame}

\newcommand\listFrameDescr[1][]{
  \listocaml[firstline=111,lastline=124,#1]{../ocaml/asmcomp/emitaux.ml}{asmcomp/emitaux.ml:111}}
\newcommand\listDebugInfo[1][]{
  \listocaml[firstline=38,lastline=49,#1]{../ocaml/lambda/debuginfo.mli}{lambda/debuginfo.mli:38}}

\begin{frame}{Backtraces}{\typename{frame\_descr} and \typename{frame\_debuginfo} in a nutshell}
  \begin{columns}[c]
    \begin{column}{0.5\textwidth}
      \begin{itemize}
        \item<1-> For every return address in an OCaml program, there is a associated \typename{frame\_descr}
        \item<2-> A \typename{frame\_descr} specifies
        \begin{itemize}
          \item<2-> The size of the function stack frame
          \item<3-> Where live values are on the stack (so that the GC can mark them as live and prevent from being swept)
        \end{itemize}
        \item<4-> When compiled with debug information, \typename{frame\_descr} will be linked to a \typename{frame\_debuginfo}
        \begin{itemize}
          \item<5-> Contains information about the position in the source code (file name, line and column number)
        \end{itemize}
      \end{itemize}
    \end{column}
    \begin{column}{0.5\textwidth}
        \onslide*<1>{\listFrameDescr[highlightlines={118-122}]{}}
        \onslide*<2>{\listFrameDescr[highlightlines={120}]{}}
        \onslide*<3>{\listFrameDescr[highlightlines={121}]{}}
        \onslide*<4->{\listFrameDescr[highlightlines={122}]{}}
        \medskip
        \onslide*<1-3>{\listDebugInfo{}}
        \onslide*<4>{\listDebugInfo[highlightlines={38,49}]{}}
        \onslide*<5>{\listDebugInfo[highlightlines={39-46}]{}}
    \end{column}
  \end{columns}
\end{frame}

\begin{frame}{Backtraces}{Scanning the stack with \typename{frame\_descr}}
  \begin{columns}[c]
    \begin{column}{0.5\textwidth}
      \begin{itemize}
        \item Given the return address of the code that raises the exception by calling \funcname{caml\_raise\_exn} (\funcarg{pc} argument of \funcname{caml\_stash\_backtrace})
        \item And the position of the stack pointer (\funcarg{sp} argument of \funcname{caml\_stash\_backtrace})
        \item The frame size given by \typename{frame\_descr}.\funcarg{frame\_size}, allows to find the end of the previous frame
        \item And the return address of the caller of this function
        \bigskip
        \onslide<3->{
        \item Note that traps (here the reddish \funcname{ExnA}, \funcname{ExnB} and \funcname{ExnC} blocks) belong to the frame that installed them, and so the frame size changes when traps are removed
        }
      \end{itemize}
    \end{column}
    \begin{column}{0.5\textwidth}
      \raggedleft
      \onslide*<1>{\providecommand\step{2}\input{diagrams/backtrace/backtrace.tex}}%
      \onslide*<2>{\providecommand\step{1}\input{diagrams/backtrace/scanstack.tex}}%
      \onslide*<3>{\providecommand\step{2}\input{diagrams/backtrace/scanstack.tex}}%
      \onslide*<4>{\providecommand\step{3}\input{diagrams/backtrace/scanstack.tex}}%
      \onslide*<5>{\providecommand\step{4}\input{diagrams/backtrace/scanstack.tex}}%
      \onslide*<6>{\providecommand\step{5}\input{diagrams/backtrace/scanstack.tex}}%
    \end{column}
  \end{columns}
\end{frame}

% Duplicate of previous-previous slide
\begin{frame}{Backtraces}{Continuing on \funcname{caml\_stash\_backtrace}}
  \begin{columns}[c]
    \begin{column}{0.5\textwidth}
      \minipage[c][0.75\textheight][s]{\columnwidth}
      \begin{itemize}
          \color{gray}
        \item A C function provided by the runtime (specific to native code)
        \item It scans the stack, between the stack pointer at the raising point up to the next trap, in order to collect the names of the detected function frames
        \item It uses the same technique and information as the Garbage Collector (GC) uses to scan the values live on the stack
      \end{itemize}
      \begin{itemize}
        \item For each function frame detected, store a pointer to the \typename{frame\_descr} in the \localname{domain\_state->backtrace\_buffer} array, effectively constructing the backtrace
      \end{itemize}
      \onslide<2->{
        \begin{itemize}
            \smallskip
          \item Constructed backtrace:
            \funcname{definitely\_raise},
            \onslide<3->{\funcname{probably\_raise}}
        \end{itemize}
      }
      \vfill
      \mintocaml[firstline=9,lastline=13,highlightlines=12]{../src/backtrace/backtrace.ml}
      \endminipage
    \end{column}
    \begin{column}{0.5\textwidth}
      \raggedleft
      \onslide*<1>{\listStashBacktrace[highlightlines={114-115}]{}}
      \onslide*<2>{\providecommand\step{3}\input{diagrams/backtrace/backtrace.tex}}%
      \onslide*<3>{\providecommand\step{4}\input{diagrams/backtrace/backtrace.tex}}%
    \end{column}
  \end{columns}
\end{frame}

\begin{frame}{Backtraces}{Back to \funcname{caml\_raise\_exn}}
  \begin{columns}[c]
    \begin{column}{0.5\textwidth}
      \minipage[c][0.66\textheight][s]{\columnwidth}
      \begin{itemize}\color{gray}
        \item Checks wheteher exceptions backtrace should be recorded, by checking the \funcarg{domain\_state->backtrace\_active} value
        \item Switches to the C stack to be able to execute C code
        \item Calls \funcname{caml\_stash\_backtrace}, to collect the backtrace up to the next trap
      \end{itemize}
      \onslide*<2->{
        \begin{itemize}
          \item<2-> Executes the exception handler in \funcname{probably\_raise} with \funcname{RESTORE\_EXN\_HANDLER\_OCAML}
            \begin{itemize}
              \item Set \regname{RSP} to \localname{domain\_state->exn\_handler} value
              \item Restore \localname{domain\_state->exn\_handler} from trap
              \item Jump to the handler code with \funcname{ret}
            \end{itemize}
        \end{itemize}
      }
      \vfill
      \onslide*<1>{\mintocaml[firstline=9,lastline=13,highlightlines=12]{../src/backtrace/backtrace.ml}}
      \onslide*<2->{\mintocaml[firstline=15,lastline=21,highlightlines={19-21}]{../src/backtrace/backtrace.ml}}
      \endminipage
    \end{column}
    \begin{column}{0.5\textwidth}
      \centering
      \onslide*<1>{
        \mintc[firstline=190,lastline=192]{../ocaml/runtime/amd64.S}
        \mintc[firstline=234,lastline=237]{../ocaml/runtime/amd64.S}
      }
      \onslide*<2>{
        \mintc[firstline=190,lastline=192,highlightlines={191-192}]{../ocaml/runtime/amd64.S}
        \mintc[firstline=234,lastline=237,highlightlines={235-237}]{../ocaml/runtime/amd64.S}
      }
      \onslide*<1>{\listCamlRaiseExn[highlightlines=720]{}}
      \onslide*<2>{\listCamlRaiseExn[highlightlines=722]{}}
      \onslide*<3->{\providecommand\step{5}\input{diagrams/backtrace/backtrace.tex}}
    \end{column}
  \end{columns}
\end{frame}

\begin{frame}{Backtraces}{\funcname{caml\_probably\_raise}'s handler doesn't catch \typename{ExnC}}
  \begin{columns}[c]
    \begin{column}{0.45\textwidth}
      \minipage[c][0.75\textheight][s]{\columnwidth}
      \begin{itemize}
        \item<1-> \funcname{match \dots with}\footnotemark pattern matching can also match exceptions
        \item<1-> The CMM of \funcname{probably\_raise} shows that this \funcname{match \dots with} is implemented with a pattern matching and a \funcname{try \dots with}
        \item<1-> But this one only matches on \typename{ExnA} exception
        \item<2-> Since the pattern matching fails to catch the exception, \funcname{reraise} is called with our current \typename{ExnC} exception
        \item<3-> Disassembly shows that \funcname{reraise} is actually a call to \funcname{caml\_reraise\_exn}, which is also an assembly function provided by the runtime
      \end{itemize}
      \vfill
      \mintocaml[firstline=15,lastline=21,highlightlines={18-21}]{../src/backtrace/backtrace.ml}
      \endminipage
    \end{column}
    \begin{column}{0.55\textwidth}
      \centering
      \onslide*<1>{\mintcmm[highlightlines={10-18}]{../src/backtrace/camlBacktrace\_\_probably\_raise\_351.cmm}}
      \onslide*<2>{\listcmm[highlightlines=18]{../src/backtrace/camlBacktrace\_\_probably\_raise_351.cmm}{CMM of probably\_raise}}
      \onslide*<3>{\mintobjdump[firstline=5,lastline=35,highlightlines=31]{../src/backtrace/camlBacktrace\_\_probably\_raise_351.objdump}}
    \end{column}
  \end{columns}
  \only<1>{\footnotetext[1]{https://blog.janestreet.com/pattern-matching-and-exception-handling-unite/}}
\end{frame}

\newcommand\listCamlReraiseExn[1][]{
  \listasm[firstline=727,lastline=734,#1]{../ocaml/runtime/amd64.S}{runtime/amd64.S:727}}

\begin{frame}{Backtraces}{Introducing \funcname{caml\_reraise\_exn}}
  \begin{columns}[c]
    \begin{column}{0.5\textwidth}
      \minipage[c][0.66\textheight][s]{\columnwidth}
      \begin{itemize}
        \item<1-> The exception have to be reraised to the next handler
        \item<2-> First, checks if the backtrace should be collected with \funcarg{domain\_state->backtrace\_active}
        \only<3>{
        \begin{itemize}
            \item If not, behaves like a \funcname{raise\_notrace} with \funcname{RESTORE\_EXN\_HANDLER\_OCAML}
        \end{itemize}
        }
        \item<4-> Jumps into \funcname{caml\_raise\_exn}
        \item<5-> Calls \funcname{caml\_stash\_backtrace} again, to scan the next part of the stack: from the trap just pop-ed to the next trap
      \end{itemize}
      \vfill
      \mintocaml[firstline=15,lastline=21,highlightlines={18-21}]{../src/backtrace/backtrace.ml}
      \endminipage
    \end{column}
    \begin{column}{0.5\textwidth}
      \raggedleft
      \onslide*<1>{\listCamlReraiseExn{}}
      \onslide*<2>{\listCamlReraiseExn[highlightlines=729]{}}
      \onslide*<3>{
        \mintc[firstline=190,lastline=192,highlightlines={191-192}]{../ocaml/runtime/amd64.S}
        \mintc[firstline=234,lastline=237,highlightlines={235-237}]{../ocaml/runtime/amd64.S}
        \medskip
        \listCamlReraiseExn[highlightlines=731]{}
      }
      \onslide*<4-5>{
        % TODO refactor with \listCamlReraiseExn
        \mintasm[firstline=727,lastline=734,highlightlines=730]{../ocaml/runtime/amd64.S}
        \medskip
      }
      \onslide*<4>{\listCamlRaiseExn[highlightlines=711]{}}
      \onslide*<5>{\listCamlRaiseExn[highlightlines=720]{}}
    \end{column}
  \end{columns}
\end{frame}

\begin{frame}{Backtraces}{Backtrace collection continues}
  \begin{columns}[c]
    \begin{column}{0.55\textwidth}
      \minipage[c][0.75\textheight][s]{\columnwidth}
      \begin{itemize}
        \item<1-> \funcname{caml\_stash\_backtrace} scans the older part of the stack, up to the next trap
        \item<6-> Controls jumps to the nested exception handler in \funcname{maybe\_raise}, which matches only on \typename{ExnB}
        \item<7-> \funcname{caml\_reraise\_exn} is called again, backtrace collection resumes with \funcname{caml\_stash\_backtrace}
      \end{itemize}
      \medskip
      \begin{itemize}
        \item<2-> Constructed backtrace:
        \begin{itemize}
          \item <2-> \funcname{definitely\_raise}
          \item <2-> \funcname{probably\_raise}
          \item <3-> \funcname{probably\_raise} (re-raise)
          \item <4-> \funcname{maybe\_raise}
          \item <5-> \funcname{main}
          \item <8-> \funcname{main} (re-raise)
        \end{itemize}
      \end{itemize}
      \vfill
      \onslide*<1-5>{\mintocaml[firstline=15,lastline=21]{../src/backtrace/backtrace.ml}}
      \onslide*<6>{\mintocaml[firstline=28,lastline=34,highlightlines=31]{../src/backtrace/backtrace.ml}}
      \onslide*<7->{\mintocaml[firstline=28,lastline=34,highlightlines=33]{../src/backtrace/backtrace.ml}}
      \endminipage
    \end{column}
    \begin{column}{0.45\textwidth}
      \raggedleft
      \onslide*<1>{\providecommand\step{6}\input{diagrams/backtrace/backtrace.tex}}%
      \onslide*<2>{\providecommand\step{6}\input{diagrams/backtrace/backtrace.tex}}%
      \onslide*<3>{\providecommand\step{7}\input{diagrams/backtrace/backtrace.tex}}%
      \onslide*<4>{\providecommand\step{8}\input{diagrams/backtrace/backtrace.tex}}%
      \onslide*<5>{\providecommand\step{9}\input{diagrams/backtrace/backtrace.tex}}%
      \onslide*<6>{\providecommand\step{10}\input{diagrams/backtrace/backtrace.tex}}%
      \onslide*<7>{\providecommand\step{11}\input{diagrams/backtrace/backtrace.tex}}%
      \onslide*<8>{\providecommand\step{12}\input{diagrams/backtrace/backtrace.tex}}%
      \onslide*<9>{\providecommand\step{13}\input{diagrams/backtrace/backtrace.tex}}%
    \end{column}
  \end{columns}
\end{frame}

\begin{frame}{Backtraces}{Printing the backtrace}
  \begin{columns}[c]
    \begin{column}{0.45\textwidth}
      \minipage[c][0.66\textheight][s]{\columnwidth}
      \begin{itemize}
        \item<1-> The exception backtrace can now be printed to \funcarg{stdout} with \funcname{Printexc.print_backtrace}
        \item<2-> First the exception backtrace is retrieved into an OCaml array with \funcname{get_raw_backtrace }
        \item<3-> Which is implemented as the \funcname{caml_get_exception_raw_backtrace} C function in the runtime
        \item<4-> And finally printed with \funcname{print_exception_backtrace}
      \end{itemize}
      \vfill
      \mintocaml[firstline=28,lastline=34,highlightlines=33]{../src/backtrace/backtrace.ml}
      \endminipage
    \end{column}
    \begin{column}{0.55\textwidth}
      \onslide*<1,2>{
        \mintocaml[firstline=100,lastline=101]{../ocaml/stdlib/printexc.ml}
      }
      \only<1>{
        \listocaml[firstline=170,lastline=172]{../ocaml/stdlib/printexc.ml}{stdlib/printexc.ml:170}
      }
      \only<2>{
        \listocaml[firstline=170,lastline=172,highlightlines=172]{../ocaml/stdlib/printexc.ml}{stdlib/printexc.ml:170}
      }
      \only<3>{
        \mintc[firstline=154,lastline=156]{../ocaml/runtime/backtrace.c}
        \listc[firstline=171,lastline=192,highlightlines={184-187}]{../ocaml/runtime/backtrace.c}{runtime/backtrace.c:154}
      }
      \only<4>{
        \listocaml[firstline=155,lastline=165]{../ocaml/stdlib/printexc.ml}{stdlib/printexc.ml:55}
      }
    \end{column}
  \end{columns}
\end{frame}


\subsection{The multiple kinds of raise}
\frameSubsection{}{}

\begin{frame}{Backtraces}{When backtraces are not needed}
  \begin{columns}[c]
    \begin{column}{0.45\textwidth}
      \minipage[c][0.66\textheight][s]{\columnwidth}
      \begin{itemize}
        \item<1-> What if the backtrace for an exception is never needed?
        \item<2-> It's possible to raise the exception explicitly with \funcname{raise\_notrace} to make raising as cheap as possible
        \item<3-> An example of use in the \funcname{dynlink} library of the OCaml compiler
      \end{itemize}
      \vfill
      \onslide<3->{
      \listocaml[firstline=205,lastline=209,highlightlines=207]{../ocaml/otherlibs/dynlink/dynlink\_compilerlibs/misc.ml}{otherlibs/dynlink/dynlink\_compilerlibs/misc.ml:205}
      }
      \endminipage
    \end{column}
    \begin{column}{0.55\textwidth}
    \only<4>{
      \mintcmm[highlightlines=3]{../src/notrace/camlNotrace\_\_fun_347.cmm}
      \listcmm{../src/notrace/camlNotrace\_\_all\_somes\_327.cmm}{all\_somes}
    }
    \onslide*<5->{
      \listobjdump[highlightlines={6-9}]{../src/notrace/camlNotrace\_\_fun_347.objdump}{anonymous function from \funcname{all\_somes}}
    }
    \end{column}
  \end{columns}
\end{frame}

\begin{frame}{Backtraces}{Summary table of the multiple kinds of raise}
  \begin{itemize}
    \item When debugging information is enabled and if the backtrace of an exception isn't needed, it's possible to raise with \funcname{raise\_notrace} for performance
    \item When debugging information is \emph{not} enabled, the compiler optimises \funcname{raise} calls to inline assembly (same effect as using \funcname{raise\_notrace})
  \end{itemize}
  \bigskip
  \centering
  \begin{tabular}{| l | c | c | c | c |}
    \hline
    \parbox{10em}{Debug information \\ (\funcarg{-g} flag)}  & \multicolumn{2}{|c|}{Disabled}                                     & \multicolumn{2}{|c|}{Enabled} \\
    \hline
    Raise kind                                               & \funcname{raise} & \funcname{raise\_notrace}                       & \funcname{raise} & \funcname{raise\_notrace} \\
    \hline
    Compilation                                              & \parbox{8em}{inline assembly\\ (optimization)} & inline assembly   & \funcname{caml\_raise\_exn} & inline assembly \\
    \hline
  \end{tabular}
\end{frame}


%
% Takeaway #5
%
\subsection*{Takeaway \#5}
\frameSubsectionTakeaway{}
\againframe<5>{takeaway}
}%
    \end{column}
  \end{columns}
\end{frame}

\begin{frame}{Backtraces}{Back to \funcname{caml\_raise\_exn}}
  \begin{columns}[c]
    \begin{column}{0.5\textwidth}
      \minipage[c][0.66\textheight][s]{\columnwidth}
      \begin{itemize}\color{gray}
        \item Checks wheteher exceptions backtrace should be recorded, by checking the \funcarg{domain\_state->backtrace\_active} value
        \item Switches to the C stack to be able to execute C code
        \item Calls \funcname{caml\_stash\_backtrace}, to collect the backtrace up to the next trap
      \end{itemize}
      \onslide*<2->{
        \begin{itemize}
          \item<2-> Executes the exception handler in \funcname{probably\_raise} with \funcname{RESTORE\_EXN\_HANDLER\_OCAML}
            \begin{itemize}
              \item Set \regname{RSP} to \localname{domain\_state->exn\_handler} value
              \item Restore \localname{domain\_state->exn\_handler} from trap
              \item Jump to the handler code with \funcname{ret}
            \end{itemize}
        \end{itemize}
      }
      \vfill
      \onslide*<1>{\mintocaml[firstline=9,lastline=13,highlightlines=12]{../src/backtrace/backtrace.ml}}
      \onslide*<2->{\mintocaml[firstline=15,lastline=21,highlightlines={19-21}]{../src/backtrace/backtrace.ml}}
      \endminipage
    \end{column}
    \begin{column}{0.5\textwidth}
      \centering
      \onslide*<1>{
        \mintc[firstline=190,lastline=192]{../ocaml/runtime/amd64.S}
        \mintc[firstline=234,lastline=237]{../ocaml/runtime/amd64.S}
      }
      \onslide*<2>{
        \mintc[firstline=190,lastline=192,highlightlines={191-192}]{../ocaml/runtime/amd64.S}
        \mintc[firstline=234,lastline=237,highlightlines={235-237}]{../ocaml/runtime/amd64.S}
      }
      \onslide*<1>{\listCamlRaiseExn[highlightlines=720]{}}
      \onslide*<2>{\listCamlRaiseExn[highlightlines=722]{}}
      \onslide*<3->{\providecommand\step{5}%
% Backtraces
%
\subsection{One advantage of raising with debugging information}
\frameSubsection{}{}

\begin{frame}{Backtraces}{Debugging information \& source code}
  \begin{columns}[c]
    \begin{column}{0.5\textwidth}
      \begin{itemize}
        \item Raising an exception is fun and games, but how do we know where the exception originated? $\rightarrow$ With the backtrace associated with the exception!
        \item In order to get a backtrace from the point where an exception was raised, it's necessary to compile with debugging information enabled (\funcarg{-g} flag for the compiler)
        \item Same example as in the `Nested handlers' section
      \end{itemize}
    \end{column}
    \begin{column}{0.5\textwidth}
      \listocaml[firstline=9,lastline=34]{../src/backtrace/backtrace.ml}{backtrace/backtrace.ml}
    \end{column}
  \end{columns}
\end{frame}

\begin{frame}{Backtraces}{CMM and disassembly of \funcname{definitely\_raise}}
  \begin{columns}[c]
    \begin{column}{0.5\textwidth}
      \minipage[c][0.66\textheight][s]{\columnwidth}
      \begin{itemize}
        \item<1-> An effect of compiling with debugging information (\funcarg{-g}): the CMM no longer uses \funcname{raise\_notrace} to raise the exception but \funcname{raise} instead
        \item<2-> Disassembly shows that CMM \funcname{raise} is actually a call to \funcname{caml\_raise\_exn}, a function in assembler provided by the runtime
      \end{itemize}
      \vfill
      \mintocaml[firstline=9,lastline=13,highlightlines=12]{../src/backtrace/backtrace.ml}
      \endminipage
    \end{column}
    \begin{column}{0.5\textwidth}
      \onslide*<1>{
        \mintcmm[highlightlines=5]{../src/backtrace/camlBacktrace\_\_definitely\_raise\_347.cmm}
      }
      \onslide*<2>{
        \mintobjdump[firstline=2,highlightlines=14]{../src/backtrace/camlBacktrace\_\_definitely\_raise\_347.objdump}
      }
    \end{column}
  \end{columns}
\end{frame}

\begin{frame}{Backtraces}{Introducing \funcname{caml\_raise\_exn}}
  \begin{columns}[c]
    \begin{column}{0.5\textwidth}
      \minipage[c][0.66\textheight][s]{\columnwidth}
      \onslide*<1>{
        \begin{itemize}
          \item Raising the exception is performed using \funcname{caml\_raise\_exn} which is
            \begin{itemize}
              \item An assembly function provided by the runtime
              \item Architecture specific
            \end{itemize}
          \item Let's see what it does differently from \funcname{raise\_notrace}\dots
          \item We are about to raise \typename{ExnC} with \funcname{caml\_raise\_exn} from \funcname{definitely\_raise}
        \end{itemize}
      }
      \onslide*<2>{
        \mintobjdump[firstline=2,highlightlines=14]{../src/backtrace/camlBacktrace\_\_definitely\_raise\_347.objdump}
      }
      \vfill
      \mintocaml[firstline=9,lastline=13,highlightlines=12]{../src/backtrace/backtrace.ml}
      \endminipage
    \end{column}
    \begin{column}{0.5\textwidth}
      \centering
      \providecommand\step{1}\input{diagrams/backtrace/backtrace.tex}
    \end{column}
  \end{columns}
\end{frame}

\begin{frame}{Backtraces}{But before that, we need more fields from \typename{caml\_domain\_state}}
  \begin{columns}
    \begin{column}{0.5\textwidth}
      \begin{itemize}
        \item Remember \typename{caml\_domain\_state} the per-domain runtime state?
        \item Some other members are of interest to us now\dots
        \item \localname{backtrace\_active} is used to know if the runtime should collect backtraces
        \item \localname{backtrace\_active} can be set by
          \begin{itemize}
            \item The \funcarg{b} flag in the \funcname{OCAMLRUNPARAM} environment variable
            \item \mintinline{ocaml}{Printexc.record_backtrace : bool -> unit} \footnotemark
          \end{itemize}
      \end{itemize}
    \end{column}
    \begin{column}{0.5\textwidth}
      \begin{listing}
        \mintc[highlightlines={9-11}]{src/ocaml/domain\_state\_backtrace.h}
        \caption{runtime/caml/domain\_state.h}
      \end{listing}
    \end{column}
  \end{columns}
  \footnotetext[1]{https://v2.ocaml.org/api/Printexc.html}
\end{frame}

\newcommand\listCamlRaiseExn[1][]{
  \listasm[firstline=702,lastline=725,#1]{../ocaml/runtime/amd64.S}{runtime/amd64.S:702}}

\begin{frame}{Backtraces}{Walk through \funcname{caml\_raise\_exn}}
  \begin{columns}[T]
    \begin{column}{0.5\textwidth}
      \begin{itemize}
        \item<1-> First checks whether exceptions backtrace should be recorded
        \item<1-> By checking the \funcarg{domain\_state->backtrace\_active} value
        \item<2-> If not, acts as \funcname{raise\_notrace} with \funcname{RESTORE\_EXN\_HANDLER\_OCAML}
          \begin{itemize}
            \item Set \regname{RSP} to \localname{domain\_state->exn\_handler} value
            \item Restore \localname{domain\_state->exn\_handler} from trap
            \item Jump with \funcname{ret} (same effect as using \funcname{pop} and \funcname{jmp})
          \end{itemize}
        \item<3-> Otherwise, switches to the C stack to be able to execute C code
        \item<4-> Calls \funcname{caml\_stash\_backtrace}, a C function provided by the runtime\ignorespaces
          \onslide<6->{, with
            \begin{itemize}
              \item \funcarg{exn}: exception value
              \item \funcarg{pc}: code address of the raising point
              \item \funcarg{sp}: stack address of the raising function
              \item \funcarg{trapsp}: stack address of the next handler
            \end{itemize}
          }
      \end{itemize}
      \bigskip
      \mintocaml[firstline=9,lastline=13,highlightlines=12]{../src/backtrace/backtrace.ml}
    \end{column}
    \begin{column}{0.5\textwidth}
      \raggedleft
      \onslide*<1,3-4>{
        \mintc[firstline=190,lastline=192]{../ocaml/runtime/amd64.S}
        \mintc[firstline=234,lastline=237]{../ocaml/runtime/amd64.S}
      }
      \onslide*<2>{
        \mintc[firstline=190,lastline=192,highlightlines={191-192}]{../ocaml/runtime/amd64.S}
        \mintc[firstline=234,lastline=237,highlightlines={235-237}]{../ocaml/runtime/amd64.S}
      }
      \onslide*<1>{\listCamlRaiseExn[highlightlines=705]{}}%
      \onslide*<2>{\listCamlRaiseExn[highlightlines={707-708}]{}}%
      \onslide*<3>{\listCamlRaiseExn[highlightlines={712-714}]{}}%
      \onslide*<4>{\listCamlRaiseExn[highlightlines={715-720}]{}}%
      \onslide*<5>{\providecommand\step{1}\input{diagrams/backtrace/backtrace.tex}}%
      \onslide*<6>{\providecommand\step{2}\input{diagrams/backtrace/backtrace.tex}}%
    \end{column}
  \end{columns}
\end{frame}

\newcommand\listStashBacktrace[1][]{
  \listc[firstline=91,lastline=120,#1]{../ocaml/runtime/backtrace\_nat.c}{runtime/backtrace\_nat.c:91}}

\begin{frame}{Backtraces}{Introducing \funcname{caml\_stash\_backtrace}}
  \begin{columns}[c]
    \begin{column}{0.5\textwidth}
      \minipage[c][0.66\textheight][s]{\columnwidth}
      \begin{itemize}
        \item<1-> A C function provided by the runtime (specific to native code)
        \item<2-> It scans the stack, between the stack pointer at the raising point up to the next trap, in order to collect the names of the detected function frames
          \onslide*<2>{
            \begin{itemize}
              \item And more information, like line number, column number...
            \end{itemize}
          }
        \item<3-> It uses the same technique and information as the Garbage Collector (GC) uses to scan the values live on the stack
          \onslide*<4>{
            \begin{itemize}
              \item \typename{frame\_descr} are at the core of stack unwinding
            \end{itemize}
          }
      \end{itemize}
      \vfill
      \mintocaml[firstline=9,lastline=13,highlightlines=12]{../src/backtrace/backtrace.ml}
      \endminipage
    \end{column}
    \begin{column}{0.5\textwidth}
      \onslide*<1>{\listStashBacktrace[highlightlines=91]{}}
      \onslide*<2>{\listStashBacktrace[highlightlines={91,117-118}]{}}
      \onslide*<3->{\listStashBacktrace[highlightlines={94,106,109-110}]{}}
    \end{column}
  \end{columns}
\end{frame}

\newcommand\listFrameDescr[1][]{
  \listocaml[firstline=111,lastline=124,#1]{../ocaml/asmcomp/emitaux.ml}{asmcomp/emitaux.ml:111}}
\newcommand\listDebugInfo[1][]{
  \listocaml[firstline=38,lastline=49,#1]{../ocaml/lambda/debuginfo.mli}{lambda/debuginfo.mli:38}}

\begin{frame}{Backtraces}{\typename{frame\_descr} and \typename{frame\_debuginfo} in a nutshell}
  \begin{columns}[c]
    \begin{column}{0.5\textwidth}
      \begin{itemize}
        \item<1-> For every return address in an OCaml program, there is a associated \typename{frame\_descr}
        \item<2-> A \typename{frame\_descr} specifies
        \begin{itemize}
          \item<2-> The size of the function stack frame
          \item<3-> Where live values are on the stack (so that the GC can mark them as live and prevent from being swept)
        \end{itemize}
        \item<4-> When compiled with debug information, \typename{frame\_descr} will be linked to a \typename{frame\_debuginfo}
        \begin{itemize}
          \item<5-> Contains information about the position in the source code (file name, line and column number)
        \end{itemize}
      \end{itemize}
    \end{column}
    \begin{column}{0.5\textwidth}
        \onslide*<1>{\listFrameDescr[highlightlines={118-122}]{}}
        \onslide*<2>{\listFrameDescr[highlightlines={120}]{}}
        \onslide*<3>{\listFrameDescr[highlightlines={121}]{}}
        \onslide*<4->{\listFrameDescr[highlightlines={122}]{}}
        \medskip
        \onslide*<1-3>{\listDebugInfo{}}
        \onslide*<4>{\listDebugInfo[highlightlines={38,49}]{}}
        \onslide*<5>{\listDebugInfo[highlightlines={39-46}]{}}
    \end{column}
  \end{columns}
\end{frame}

\begin{frame}{Backtraces}{Scanning the stack with \typename{frame\_descr}}
  \begin{columns}[c]
    \begin{column}{0.5\textwidth}
      \begin{itemize}
        \item Given the return address of the code that raises the exception by calling \funcname{caml\_raise\_exn} (\funcarg{pc} argument of \funcname{caml\_stash\_backtrace})
        \item And the position of the stack pointer (\funcarg{sp} argument of \funcname{caml\_stash\_backtrace})
        \item The frame size given by \typename{frame\_descr}.\funcarg{frame\_size}, allows to find the end of the previous frame
        \item And the return address of the caller of this function
        \bigskip
        \onslide<3->{
        \item Note that traps (here the reddish \funcname{ExnA}, \funcname{ExnB} and \funcname{ExnC} blocks) belong to the frame that installed them, and so the frame size changes when traps are removed
        }
      \end{itemize}
    \end{column}
    \begin{column}{0.5\textwidth}
      \raggedleft
      \onslide*<1>{\providecommand\step{2}\input{diagrams/backtrace/backtrace.tex}}%
      \onslide*<2>{\providecommand\step{1}\input{diagrams/backtrace/scanstack.tex}}%
      \onslide*<3>{\providecommand\step{2}\input{diagrams/backtrace/scanstack.tex}}%
      \onslide*<4>{\providecommand\step{3}\input{diagrams/backtrace/scanstack.tex}}%
      \onslide*<5>{\providecommand\step{4}\input{diagrams/backtrace/scanstack.tex}}%
      \onslide*<6>{\providecommand\step{5}\input{diagrams/backtrace/scanstack.tex}}%
    \end{column}
  \end{columns}
\end{frame}

% Duplicate of previous-previous slide
\begin{frame}{Backtraces}{Continuing on \funcname{caml\_stash\_backtrace}}
  \begin{columns}[c]
    \begin{column}{0.5\textwidth}
      \minipage[c][0.75\textheight][s]{\columnwidth}
      \begin{itemize}
          \color{gray}
        \item A C function provided by the runtime (specific to native code)
        \item It scans the stack, between the stack pointer at the raising point up to the next trap, in order to collect the names of the detected function frames
        \item It uses the same technique and information as the Garbage Collector (GC) uses to scan the values live on the stack
      \end{itemize}
      \begin{itemize}
        \item For each function frame detected, store a pointer to the \typename{frame\_descr} in the \localname{domain\_state->backtrace\_buffer} array, effectively constructing the backtrace
      \end{itemize}
      \onslide<2->{
        \begin{itemize}
            \smallskip
          \item Constructed backtrace:
            \funcname{definitely\_raise},
            \onslide<3->{\funcname{probably\_raise}}
        \end{itemize}
      }
      \vfill
      \mintocaml[firstline=9,lastline=13,highlightlines=12]{../src/backtrace/backtrace.ml}
      \endminipage
    \end{column}
    \begin{column}{0.5\textwidth}
      \raggedleft
      \onslide*<1>{\listStashBacktrace[highlightlines={114-115}]{}}
      \onslide*<2>{\providecommand\step{3}\input{diagrams/backtrace/backtrace.tex}}%
      \onslide*<3>{\providecommand\step{4}\input{diagrams/backtrace/backtrace.tex}}%
    \end{column}
  \end{columns}
\end{frame}

\begin{frame}{Backtraces}{Back to \funcname{caml\_raise\_exn}}
  \begin{columns}[c]
    \begin{column}{0.5\textwidth}
      \minipage[c][0.66\textheight][s]{\columnwidth}
      \begin{itemize}\color{gray}
        \item Checks wheteher exceptions backtrace should be recorded, by checking the \funcarg{domain\_state->backtrace\_active} value
        \item Switches to the C stack to be able to execute C code
        \item Calls \funcname{caml\_stash\_backtrace}, to collect the backtrace up to the next trap
      \end{itemize}
      \onslide*<2->{
        \begin{itemize}
          \item<2-> Executes the exception handler in \funcname{probably\_raise} with \funcname{RESTORE\_EXN\_HANDLER\_OCAML}
            \begin{itemize}
              \item Set \regname{RSP} to \localname{domain\_state->exn\_handler} value
              \item Restore \localname{domain\_state->exn\_handler} from trap
              \item Jump to the handler code with \funcname{ret}
            \end{itemize}
        \end{itemize}
      }
      \vfill
      \onslide*<1>{\mintocaml[firstline=9,lastline=13,highlightlines=12]{../src/backtrace/backtrace.ml}}
      \onslide*<2->{\mintocaml[firstline=15,lastline=21,highlightlines={19-21}]{../src/backtrace/backtrace.ml}}
      \endminipage
    \end{column}
    \begin{column}{0.5\textwidth}
      \centering
      \onslide*<1>{
        \mintc[firstline=190,lastline=192]{../ocaml/runtime/amd64.S}
        \mintc[firstline=234,lastline=237]{../ocaml/runtime/amd64.S}
      }
      \onslide*<2>{
        \mintc[firstline=190,lastline=192,highlightlines={191-192}]{../ocaml/runtime/amd64.S}
        \mintc[firstline=234,lastline=237,highlightlines={235-237}]{../ocaml/runtime/amd64.S}
      }
      \onslide*<1>{\listCamlRaiseExn[highlightlines=720]{}}
      \onslide*<2>{\listCamlRaiseExn[highlightlines=722]{}}
      \onslide*<3->{\providecommand\step{5}\input{diagrams/backtrace/backtrace.tex}}
    \end{column}
  \end{columns}
\end{frame}

\begin{frame}{Backtraces}{\funcname{caml\_probably\_raise}'s handler doesn't catch \typename{ExnC}}
  \begin{columns}[c]
    \begin{column}{0.45\textwidth}
      \minipage[c][0.75\textheight][s]{\columnwidth}
      \begin{itemize}
        \item<1-> \funcname{match \dots with}\footnotemark pattern matching can also match exceptions
        \item<1-> The CMM of \funcname{probably\_raise} shows that this \funcname{match \dots with} is implemented with a pattern matching and a \funcname{try \dots with}
        \item<1-> But this one only matches on \typename{ExnA} exception
        \item<2-> Since the pattern matching fails to catch the exception, \funcname{reraise} is called with our current \typename{ExnC} exception
        \item<3-> Disassembly shows that \funcname{reraise} is actually a call to \funcname{caml\_reraise\_exn}, which is also an assembly function provided by the runtime
      \end{itemize}
      \vfill
      \mintocaml[firstline=15,lastline=21,highlightlines={18-21}]{../src/backtrace/backtrace.ml}
      \endminipage
    \end{column}
    \begin{column}{0.55\textwidth}
      \centering
      \onslide*<1>{\mintcmm[highlightlines={10-18}]{../src/backtrace/camlBacktrace\_\_probably\_raise\_351.cmm}}
      \onslide*<2>{\listcmm[highlightlines=18]{../src/backtrace/camlBacktrace\_\_probably\_raise_351.cmm}{CMM of probably\_raise}}
      \onslide*<3>{\mintobjdump[firstline=5,lastline=35,highlightlines=31]{../src/backtrace/camlBacktrace\_\_probably\_raise_351.objdump}}
    \end{column}
  \end{columns}
  \only<1>{\footnotetext[1]{https://blog.janestreet.com/pattern-matching-and-exception-handling-unite/}}
\end{frame}

\newcommand\listCamlReraiseExn[1][]{
  \listasm[firstline=727,lastline=734,#1]{../ocaml/runtime/amd64.S}{runtime/amd64.S:727}}

\begin{frame}{Backtraces}{Introducing \funcname{caml\_reraise\_exn}}
  \begin{columns}[c]
    \begin{column}{0.5\textwidth}
      \minipage[c][0.66\textheight][s]{\columnwidth}
      \begin{itemize}
        \item<1-> The exception have to be reraised to the next handler
        \item<2-> First, checks if the backtrace should be collected with \funcarg{domain\_state->backtrace\_active}
        \only<3>{
        \begin{itemize}
            \item If not, behaves like a \funcname{raise\_notrace} with \funcname{RESTORE\_EXN\_HANDLER\_OCAML}
        \end{itemize}
        }
        \item<4-> Jumps into \funcname{caml\_raise\_exn}
        \item<5-> Calls \funcname{caml\_stash\_backtrace} again, to scan the next part of the stack: from the trap just pop-ed to the next trap
      \end{itemize}
      \vfill
      \mintocaml[firstline=15,lastline=21,highlightlines={18-21}]{../src/backtrace/backtrace.ml}
      \endminipage
    \end{column}
    \begin{column}{0.5\textwidth}
      \raggedleft
      \onslide*<1>{\listCamlReraiseExn{}}
      \onslide*<2>{\listCamlReraiseExn[highlightlines=729]{}}
      \onslide*<3>{
        \mintc[firstline=190,lastline=192,highlightlines={191-192}]{../ocaml/runtime/amd64.S}
        \mintc[firstline=234,lastline=237,highlightlines={235-237}]{../ocaml/runtime/amd64.S}
        \medskip
        \listCamlReraiseExn[highlightlines=731]{}
      }
      \onslide*<4-5>{
        % TODO refactor with \listCamlReraiseExn
        \mintasm[firstline=727,lastline=734,highlightlines=730]{../ocaml/runtime/amd64.S}
        \medskip
      }
      \onslide*<4>{\listCamlRaiseExn[highlightlines=711]{}}
      \onslide*<5>{\listCamlRaiseExn[highlightlines=720]{}}
    \end{column}
  \end{columns}
\end{frame}

\begin{frame}{Backtraces}{Backtrace collection continues}
  \begin{columns}[c]
    \begin{column}{0.55\textwidth}
      \minipage[c][0.75\textheight][s]{\columnwidth}
      \begin{itemize}
        \item<1-> \funcname{caml\_stash\_backtrace} scans the older part of the stack, up to the next trap
        \item<6-> Controls jumps to the nested exception handler in \funcname{maybe\_raise}, which matches only on \typename{ExnB}
        \item<7-> \funcname{caml\_reraise\_exn} is called again, backtrace collection resumes with \funcname{caml\_stash\_backtrace}
      \end{itemize}
      \medskip
      \begin{itemize}
        \item<2-> Constructed backtrace:
        \begin{itemize}
          \item <2-> \funcname{definitely\_raise}
          \item <2-> \funcname{probably\_raise}
          \item <3-> \funcname{probably\_raise} (re-raise)
          \item <4-> \funcname{maybe\_raise}
          \item <5-> \funcname{main}
          \item <8-> \funcname{main} (re-raise)
        \end{itemize}
      \end{itemize}
      \vfill
      \onslide*<1-5>{\mintocaml[firstline=15,lastline=21]{../src/backtrace/backtrace.ml}}
      \onslide*<6>{\mintocaml[firstline=28,lastline=34,highlightlines=31]{../src/backtrace/backtrace.ml}}
      \onslide*<7->{\mintocaml[firstline=28,lastline=34,highlightlines=33]{../src/backtrace/backtrace.ml}}
      \endminipage
    \end{column}
    \begin{column}{0.45\textwidth}
      \raggedleft
      \onslide*<1>{\providecommand\step{6}\input{diagrams/backtrace/backtrace.tex}}%
      \onslide*<2>{\providecommand\step{6}\input{diagrams/backtrace/backtrace.tex}}%
      \onslide*<3>{\providecommand\step{7}\input{diagrams/backtrace/backtrace.tex}}%
      \onslide*<4>{\providecommand\step{8}\input{diagrams/backtrace/backtrace.tex}}%
      \onslide*<5>{\providecommand\step{9}\input{diagrams/backtrace/backtrace.tex}}%
      \onslide*<6>{\providecommand\step{10}\input{diagrams/backtrace/backtrace.tex}}%
      \onslide*<7>{\providecommand\step{11}\input{diagrams/backtrace/backtrace.tex}}%
      \onslide*<8>{\providecommand\step{12}\input{diagrams/backtrace/backtrace.tex}}%
      \onslide*<9>{\providecommand\step{13}\input{diagrams/backtrace/backtrace.tex}}%
    \end{column}
  \end{columns}
\end{frame}

\begin{frame}{Backtraces}{Printing the backtrace}
  \begin{columns}[c]
    \begin{column}{0.45\textwidth}
      \minipage[c][0.66\textheight][s]{\columnwidth}
      \begin{itemize}
        \item<1-> The exception backtrace can now be printed to \funcarg{stdout} with \funcname{Printexc.print_backtrace}
        \item<2-> First the exception backtrace is retrieved into an OCaml array with \funcname{get_raw_backtrace }
        \item<3-> Which is implemented as the \funcname{caml_get_exception_raw_backtrace} C function in the runtime
        \item<4-> And finally printed with \funcname{print_exception_backtrace}
      \end{itemize}
      \vfill
      \mintocaml[firstline=28,lastline=34,highlightlines=33]{../src/backtrace/backtrace.ml}
      \endminipage
    \end{column}
    \begin{column}{0.55\textwidth}
      \onslide*<1,2>{
        \mintocaml[firstline=100,lastline=101]{../ocaml/stdlib/printexc.ml}
      }
      \only<1>{
        \listocaml[firstline=170,lastline=172]{../ocaml/stdlib/printexc.ml}{stdlib/printexc.ml:170}
      }
      \only<2>{
        \listocaml[firstline=170,lastline=172,highlightlines=172]{../ocaml/stdlib/printexc.ml}{stdlib/printexc.ml:170}
      }
      \only<3>{
        \mintc[firstline=154,lastline=156]{../ocaml/runtime/backtrace.c}
        \listc[firstline=171,lastline=192,highlightlines={184-187}]{../ocaml/runtime/backtrace.c}{runtime/backtrace.c:154}
      }
      \only<4>{
        \listocaml[firstline=155,lastline=165]{../ocaml/stdlib/printexc.ml}{stdlib/printexc.ml:55}
      }
    \end{column}
  \end{columns}
\end{frame}


\subsection{The multiple kinds of raise}
\frameSubsection{}{}

\begin{frame}{Backtraces}{When backtraces are not needed}
  \begin{columns}[c]
    \begin{column}{0.45\textwidth}
      \minipage[c][0.66\textheight][s]{\columnwidth}
      \begin{itemize}
        \item<1-> What if the backtrace for an exception is never needed?
        \item<2-> It's possible to raise the exception explicitly with \funcname{raise\_notrace} to make raising as cheap as possible
        \item<3-> An example of use in the \funcname{dynlink} library of the OCaml compiler
      \end{itemize}
      \vfill
      \onslide<3->{
      \listocaml[firstline=205,lastline=209,highlightlines=207]{../ocaml/otherlibs/dynlink/dynlink\_compilerlibs/misc.ml}{otherlibs/dynlink/dynlink\_compilerlibs/misc.ml:205}
      }
      \endminipage
    \end{column}
    \begin{column}{0.55\textwidth}
    \only<4>{
      \mintcmm[highlightlines=3]{../src/notrace/camlNotrace\_\_fun_347.cmm}
      \listcmm{../src/notrace/camlNotrace\_\_all\_somes\_327.cmm}{all\_somes}
    }
    \onslide*<5->{
      \listobjdump[highlightlines={6-9}]{../src/notrace/camlNotrace\_\_fun_347.objdump}{anonymous function from \funcname{all\_somes}}
    }
    \end{column}
  \end{columns}
\end{frame}

\begin{frame}{Backtraces}{Summary table of the multiple kinds of raise}
  \begin{itemize}
    \item When debugging information is enabled and if the backtrace of an exception isn't needed, it's possible to raise with \funcname{raise\_notrace} for performance
    \item When debugging information is \emph{not} enabled, the compiler optimises \funcname{raise} calls to inline assembly (same effect as using \funcname{raise\_notrace})
  \end{itemize}
  \bigskip
  \centering
  \begin{tabular}{| l | c | c | c | c |}
    \hline
    \parbox{10em}{Debug information \\ (\funcarg{-g} flag)}  & \multicolumn{2}{|c|}{Disabled}                                     & \multicolumn{2}{|c|}{Enabled} \\
    \hline
    Raise kind                                               & \funcname{raise} & \funcname{raise\_notrace}                       & \funcname{raise} & \funcname{raise\_notrace} \\
    \hline
    Compilation                                              & \parbox{8em}{inline assembly\\ (optimization)} & inline assembly   & \funcname{caml\_raise\_exn} & inline assembly \\
    \hline
  \end{tabular}
\end{frame}


%
% Takeaway #5
%
\subsection*{Takeaway \#5}
\frameSubsectionTakeaway{}
\againframe<5>{takeaway}
}
    \end{column}
  \end{columns}
\end{frame}

\begin{frame}{Backtraces}{\funcname{caml\_probably\_raise}'s handler doesn't catch \typename{ExnC}}
  \begin{columns}[c]
    \begin{column}{0.45\textwidth}
      \minipage[c][0.75\textheight][s]{\columnwidth}
      \begin{itemize}
        \item<1-> \funcname{match \dots with}\footnotemark pattern matching can also match exceptions
        \item<1-> The CMM of \funcname{probably\_raise} shows that this \funcname{match \dots with} is implemented with a pattern matching and a \funcname{try \dots with}
        \item<1-> But this one only matches on \typename{ExnA} exception
        \item<2-> Since the pattern matching fails to catch the exception, \funcname{reraise} is called with our current \typename{ExnC} exception
        \item<3-> Disassembly shows that \funcname{reraise} is actually a call to \funcname{caml\_reraise\_exn}, which is also an assembly function provided by the runtime
      \end{itemize}
      \vfill
      \mintocaml[firstline=15,lastline=21,highlightlines={18-21}]{../src/backtrace/backtrace.ml}
      \endminipage
    \end{column}
    \begin{column}{0.55\textwidth}
      \centering
      \onslide*<1>{\mintcmm[highlightlines={10-18}]{../src/backtrace/camlBacktrace\_\_probably\_raise\_351.cmm}}
      \onslide*<2>{\listcmm[highlightlines=18]{../src/backtrace/camlBacktrace\_\_probably\_raise_351.cmm}{CMM of probably\_raise}}
      \onslide*<3>{\mintobjdump[firstline=5,lastline=35,highlightlines=31]{../src/backtrace/camlBacktrace\_\_probably\_raise_351.objdump}}
    \end{column}
  \end{columns}
  \only<1>{\footnotetext[1]{https://blog.janestreet.com/pattern-matching-and-exception-handling-unite/}}
\end{frame}

\newcommand\listCamlReraiseExn[1][]{
  \listasm[firstline=727,lastline=734,#1]{../ocaml/runtime/amd64.S}{runtime/amd64.S:727}}

\begin{frame}{Backtraces}{Introducing \funcname{caml\_reraise\_exn}}
  \begin{columns}[c]
    \begin{column}{0.5\textwidth}
      \minipage[c][0.66\textheight][s]{\columnwidth}
      \begin{itemize}
        \item<1-> The exception have to be reraised to the next handler
        \item<2-> First, checks if the backtrace should be collected with \funcarg{domain\_state->backtrace\_active}
        \only<3>{
        \begin{itemize}
            \item If not, behaves like a \funcname{raise\_notrace} with \funcname{RESTORE\_EXN\_HANDLER\_OCAML}
        \end{itemize}
        }
        \item<4-> Jumps into \funcname{caml\_raise\_exn}
        \item<5-> Calls \funcname{caml\_stash\_backtrace} again, to scan the next part of the stack: from the trap just pop-ed to the next trap
      \end{itemize}
      \vfill
      \mintocaml[firstline=15,lastline=21,highlightlines={18-21}]{../src/backtrace/backtrace.ml}
      \endminipage
    \end{column}
    \begin{column}{0.5\textwidth}
      \raggedleft
      \onslide*<1>{\listCamlReraiseExn{}}
      \onslide*<2>{\listCamlReraiseExn[highlightlines=729]{}}
      \onslide*<3>{
        \mintc[firstline=190,lastline=192,highlightlines={191-192}]{../ocaml/runtime/amd64.S}
        \mintc[firstline=234,lastline=237,highlightlines={235-237}]{../ocaml/runtime/amd64.S}
        \medskip
        \listCamlReraiseExn[highlightlines=731]{}
      }
      \onslide*<4-5>{
        % TODO refactor with \listCamlReraiseExn
        \mintasm[firstline=727,lastline=734,highlightlines=730]{../ocaml/runtime/amd64.S}
        \medskip
      }
      \onslide*<4>{\listCamlRaiseExn[highlightlines=711]{}}
      \onslide*<5>{\listCamlRaiseExn[highlightlines=720]{}}
    \end{column}
  \end{columns}
\end{frame}

\begin{frame}{Backtraces}{Backtrace collection continues}
  \begin{columns}[c]
    \begin{column}{0.55\textwidth}
      \minipage[c][0.75\textheight][s]{\columnwidth}
      \begin{itemize}
        \item<1-> \funcname{caml\_stash\_backtrace} scans the older part of the stack, up to the next trap
        \item<6-> Controls jumps to the nested exception handler in \funcname{maybe\_raise}, which matches only on \typename{ExnB}
        \item<7-> \funcname{caml\_reraise\_exn} is called again, backtrace collection resumes with \funcname{caml\_stash\_backtrace}
      \end{itemize}
      \medskip
      \begin{itemize}
        \item<2-> Constructed backtrace:
        \begin{itemize}
          \item <2-> \funcname{definitely\_raise}
          \item <2-> \funcname{probably\_raise}
          \item <3-> \funcname{probably\_raise} (re-raise)
          \item <4-> \funcname{maybe\_raise}
          \item <5-> \funcname{main}
          \item <8-> \funcname{main} (re-raise)
        \end{itemize}
      \end{itemize}
      \vfill
      \onslide*<1-5>{\mintocaml[firstline=15,lastline=21]{../src/backtrace/backtrace.ml}}
      \onslide*<6>{\mintocaml[firstline=28,lastline=34,highlightlines=31]{../src/backtrace/backtrace.ml}}
      \onslide*<7->{\mintocaml[firstline=28,lastline=34,highlightlines=33]{../src/backtrace/backtrace.ml}}
      \endminipage
    \end{column}
    \begin{column}{0.45\textwidth}
      \raggedleft
      \onslide*<1>{\providecommand\step{6}%
% Backtraces
%
\subsection{One advantage of raising with debugging information}
\frameSubsection{}{}

\begin{frame}{Backtraces}{Debugging information \& source code}
  \begin{columns}[c]
    \begin{column}{0.5\textwidth}
      \begin{itemize}
        \item Raising an exception is fun and games, but how do we know where the exception originated? $\rightarrow$ With the backtrace associated with the exception!
        \item In order to get a backtrace from the point where an exception was raised, it's necessary to compile with debugging information enabled (\funcarg{-g} flag for the compiler)
        \item Same example as in the `Nested handlers' section
      \end{itemize}
    \end{column}
    \begin{column}{0.5\textwidth}
      \listocaml[firstline=9,lastline=34]{../src/backtrace/backtrace.ml}{backtrace/backtrace.ml}
    \end{column}
  \end{columns}
\end{frame}

\begin{frame}{Backtraces}{CMM and disassembly of \funcname{definitely\_raise}}
  \begin{columns}[c]
    \begin{column}{0.5\textwidth}
      \minipage[c][0.66\textheight][s]{\columnwidth}
      \begin{itemize}
        \item<1-> An effect of compiling with debugging information (\funcarg{-g}): the CMM no longer uses \funcname{raise\_notrace} to raise the exception but \funcname{raise} instead
        \item<2-> Disassembly shows that CMM \funcname{raise} is actually a call to \funcname{caml\_raise\_exn}, a function in assembler provided by the runtime
      \end{itemize}
      \vfill
      \mintocaml[firstline=9,lastline=13,highlightlines=12]{../src/backtrace/backtrace.ml}
      \endminipage
    \end{column}
    \begin{column}{0.5\textwidth}
      \onslide*<1>{
        \mintcmm[highlightlines=5]{../src/backtrace/camlBacktrace\_\_definitely\_raise\_347.cmm}
      }
      \onslide*<2>{
        \mintobjdump[firstline=2,highlightlines=14]{../src/backtrace/camlBacktrace\_\_definitely\_raise\_347.objdump}
      }
    \end{column}
  \end{columns}
\end{frame}

\begin{frame}{Backtraces}{Introducing \funcname{caml\_raise\_exn}}
  \begin{columns}[c]
    \begin{column}{0.5\textwidth}
      \minipage[c][0.66\textheight][s]{\columnwidth}
      \onslide*<1>{
        \begin{itemize}
          \item Raising the exception is performed using \funcname{caml\_raise\_exn} which is
            \begin{itemize}
              \item An assembly function provided by the runtime
              \item Architecture specific
            \end{itemize}
          \item Let's see what it does differently from \funcname{raise\_notrace}\dots
          \item We are about to raise \typename{ExnC} with \funcname{caml\_raise\_exn} from \funcname{definitely\_raise}
        \end{itemize}
      }
      \onslide*<2>{
        \mintobjdump[firstline=2,highlightlines=14]{../src/backtrace/camlBacktrace\_\_definitely\_raise\_347.objdump}
      }
      \vfill
      \mintocaml[firstline=9,lastline=13,highlightlines=12]{../src/backtrace/backtrace.ml}
      \endminipage
    \end{column}
    \begin{column}{0.5\textwidth}
      \centering
      \providecommand\step{1}\input{diagrams/backtrace/backtrace.tex}
    \end{column}
  \end{columns}
\end{frame}

\begin{frame}{Backtraces}{But before that, we need more fields from \typename{caml\_domain\_state}}
  \begin{columns}
    \begin{column}{0.5\textwidth}
      \begin{itemize}
        \item Remember \typename{caml\_domain\_state} the per-domain runtime state?
        \item Some other members are of interest to us now\dots
        \item \localname{backtrace\_active} is used to know if the runtime should collect backtraces
        \item \localname{backtrace\_active} can be set by
          \begin{itemize}
            \item The \funcarg{b} flag in the \funcname{OCAMLRUNPARAM} environment variable
            \item \mintinline{ocaml}{Printexc.record_backtrace : bool -> unit} \footnotemark
          \end{itemize}
      \end{itemize}
    \end{column}
    \begin{column}{0.5\textwidth}
      \begin{listing}
        \mintc[highlightlines={9-11}]{src/ocaml/domain\_state\_backtrace.h}
        \caption{runtime/caml/domain\_state.h}
      \end{listing}
    \end{column}
  \end{columns}
  \footnotetext[1]{https://v2.ocaml.org/api/Printexc.html}
\end{frame}

\newcommand\listCamlRaiseExn[1][]{
  \listasm[firstline=702,lastline=725,#1]{../ocaml/runtime/amd64.S}{runtime/amd64.S:702}}

\begin{frame}{Backtraces}{Walk through \funcname{caml\_raise\_exn}}
  \begin{columns}[T]
    \begin{column}{0.5\textwidth}
      \begin{itemize}
        \item<1-> First checks whether exceptions backtrace should be recorded
        \item<1-> By checking the \funcarg{domain\_state->backtrace\_active} value
        \item<2-> If not, acts as \funcname{raise\_notrace} with \funcname{RESTORE\_EXN\_HANDLER\_OCAML}
          \begin{itemize}
            \item Set \regname{RSP} to \localname{domain\_state->exn\_handler} value
            \item Restore \localname{domain\_state->exn\_handler} from trap
            \item Jump with \funcname{ret} (same effect as using \funcname{pop} and \funcname{jmp})
          \end{itemize}
        \item<3-> Otherwise, switches to the C stack to be able to execute C code
        \item<4-> Calls \funcname{caml\_stash\_backtrace}, a C function provided by the runtime\ignorespaces
          \onslide<6->{, with
            \begin{itemize}
              \item \funcarg{exn}: exception value
              \item \funcarg{pc}: code address of the raising point
              \item \funcarg{sp}: stack address of the raising function
              \item \funcarg{trapsp}: stack address of the next handler
            \end{itemize}
          }
      \end{itemize}
      \bigskip
      \mintocaml[firstline=9,lastline=13,highlightlines=12]{../src/backtrace/backtrace.ml}
    \end{column}
    \begin{column}{0.5\textwidth}
      \raggedleft
      \onslide*<1,3-4>{
        \mintc[firstline=190,lastline=192]{../ocaml/runtime/amd64.S}
        \mintc[firstline=234,lastline=237]{../ocaml/runtime/amd64.S}
      }
      \onslide*<2>{
        \mintc[firstline=190,lastline=192,highlightlines={191-192}]{../ocaml/runtime/amd64.S}
        \mintc[firstline=234,lastline=237,highlightlines={235-237}]{../ocaml/runtime/amd64.S}
      }
      \onslide*<1>{\listCamlRaiseExn[highlightlines=705]{}}%
      \onslide*<2>{\listCamlRaiseExn[highlightlines={707-708}]{}}%
      \onslide*<3>{\listCamlRaiseExn[highlightlines={712-714}]{}}%
      \onslide*<4>{\listCamlRaiseExn[highlightlines={715-720}]{}}%
      \onslide*<5>{\providecommand\step{1}\input{diagrams/backtrace/backtrace.tex}}%
      \onslide*<6>{\providecommand\step{2}\input{diagrams/backtrace/backtrace.tex}}%
    \end{column}
  \end{columns}
\end{frame}

\newcommand\listStashBacktrace[1][]{
  \listc[firstline=91,lastline=120,#1]{../ocaml/runtime/backtrace\_nat.c}{runtime/backtrace\_nat.c:91}}

\begin{frame}{Backtraces}{Introducing \funcname{caml\_stash\_backtrace}}
  \begin{columns}[c]
    \begin{column}{0.5\textwidth}
      \minipage[c][0.66\textheight][s]{\columnwidth}
      \begin{itemize}
        \item<1-> A C function provided by the runtime (specific to native code)
        \item<2-> It scans the stack, between the stack pointer at the raising point up to the next trap, in order to collect the names of the detected function frames
          \onslide*<2>{
            \begin{itemize}
              \item And more information, like line number, column number...
            \end{itemize}
          }
        \item<3-> It uses the same technique and information as the Garbage Collector (GC) uses to scan the values live on the stack
          \onslide*<4>{
            \begin{itemize}
              \item \typename{frame\_descr} are at the core of stack unwinding
            \end{itemize}
          }
      \end{itemize}
      \vfill
      \mintocaml[firstline=9,lastline=13,highlightlines=12]{../src/backtrace/backtrace.ml}
      \endminipage
    \end{column}
    \begin{column}{0.5\textwidth}
      \onslide*<1>{\listStashBacktrace[highlightlines=91]{}}
      \onslide*<2>{\listStashBacktrace[highlightlines={91,117-118}]{}}
      \onslide*<3->{\listStashBacktrace[highlightlines={94,106,109-110}]{}}
    \end{column}
  \end{columns}
\end{frame}

\newcommand\listFrameDescr[1][]{
  \listocaml[firstline=111,lastline=124,#1]{../ocaml/asmcomp/emitaux.ml}{asmcomp/emitaux.ml:111}}
\newcommand\listDebugInfo[1][]{
  \listocaml[firstline=38,lastline=49,#1]{../ocaml/lambda/debuginfo.mli}{lambda/debuginfo.mli:38}}

\begin{frame}{Backtraces}{\typename{frame\_descr} and \typename{frame\_debuginfo} in a nutshell}
  \begin{columns}[c]
    \begin{column}{0.5\textwidth}
      \begin{itemize}
        \item<1-> For every return address in an OCaml program, there is a associated \typename{frame\_descr}
        \item<2-> A \typename{frame\_descr} specifies
        \begin{itemize}
          \item<2-> The size of the function stack frame
          \item<3-> Where live values are on the stack (so that the GC can mark them as live and prevent from being swept)
        \end{itemize}
        \item<4-> When compiled with debug information, \typename{frame\_descr} will be linked to a \typename{frame\_debuginfo}
        \begin{itemize}
          \item<5-> Contains information about the position in the source code (file name, line and column number)
        \end{itemize}
      \end{itemize}
    \end{column}
    \begin{column}{0.5\textwidth}
        \onslide*<1>{\listFrameDescr[highlightlines={118-122}]{}}
        \onslide*<2>{\listFrameDescr[highlightlines={120}]{}}
        \onslide*<3>{\listFrameDescr[highlightlines={121}]{}}
        \onslide*<4->{\listFrameDescr[highlightlines={122}]{}}
        \medskip
        \onslide*<1-3>{\listDebugInfo{}}
        \onslide*<4>{\listDebugInfo[highlightlines={38,49}]{}}
        \onslide*<5>{\listDebugInfo[highlightlines={39-46}]{}}
    \end{column}
  \end{columns}
\end{frame}

\begin{frame}{Backtraces}{Scanning the stack with \typename{frame\_descr}}
  \begin{columns}[c]
    \begin{column}{0.5\textwidth}
      \begin{itemize}
        \item Given the return address of the code that raises the exception by calling \funcname{caml\_raise\_exn} (\funcarg{pc} argument of \funcname{caml\_stash\_backtrace})
        \item And the position of the stack pointer (\funcarg{sp} argument of \funcname{caml\_stash\_backtrace})
        \item The frame size given by \typename{frame\_descr}.\funcarg{frame\_size}, allows to find the end of the previous frame
        \item And the return address of the caller of this function
        \bigskip
        \onslide<3->{
        \item Note that traps (here the reddish \funcname{ExnA}, \funcname{ExnB} and \funcname{ExnC} blocks) belong to the frame that installed them, and so the frame size changes when traps are removed
        }
      \end{itemize}
    \end{column}
    \begin{column}{0.5\textwidth}
      \raggedleft
      \onslide*<1>{\providecommand\step{2}\input{diagrams/backtrace/backtrace.tex}}%
      \onslide*<2>{\providecommand\step{1}\input{diagrams/backtrace/scanstack.tex}}%
      \onslide*<3>{\providecommand\step{2}\input{diagrams/backtrace/scanstack.tex}}%
      \onslide*<4>{\providecommand\step{3}\input{diagrams/backtrace/scanstack.tex}}%
      \onslide*<5>{\providecommand\step{4}\input{diagrams/backtrace/scanstack.tex}}%
      \onslide*<6>{\providecommand\step{5}\input{diagrams/backtrace/scanstack.tex}}%
    \end{column}
  \end{columns}
\end{frame}

% Duplicate of previous-previous slide
\begin{frame}{Backtraces}{Continuing on \funcname{caml\_stash\_backtrace}}
  \begin{columns}[c]
    \begin{column}{0.5\textwidth}
      \minipage[c][0.75\textheight][s]{\columnwidth}
      \begin{itemize}
          \color{gray}
        \item A C function provided by the runtime (specific to native code)
        \item It scans the stack, between the stack pointer at the raising point up to the next trap, in order to collect the names of the detected function frames
        \item It uses the same technique and information as the Garbage Collector (GC) uses to scan the values live on the stack
      \end{itemize}
      \begin{itemize}
        \item For each function frame detected, store a pointer to the \typename{frame\_descr} in the \localname{domain\_state->backtrace\_buffer} array, effectively constructing the backtrace
      \end{itemize}
      \onslide<2->{
        \begin{itemize}
            \smallskip
          \item Constructed backtrace:
            \funcname{definitely\_raise},
            \onslide<3->{\funcname{probably\_raise}}
        \end{itemize}
      }
      \vfill
      \mintocaml[firstline=9,lastline=13,highlightlines=12]{../src/backtrace/backtrace.ml}
      \endminipage
    \end{column}
    \begin{column}{0.5\textwidth}
      \raggedleft
      \onslide*<1>{\listStashBacktrace[highlightlines={114-115}]{}}
      \onslide*<2>{\providecommand\step{3}\input{diagrams/backtrace/backtrace.tex}}%
      \onslide*<3>{\providecommand\step{4}\input{diagrams/backtrace/backtrace.tex}}%
    \end{column}
  \end{columns}
\end{frame}

\begin{frame}{Backtraces}{Back to \funcname{caml\_raise\_exn}}
  \begin{columns}[c]
    \begin{column}{0.5\textwidth}
      \minipage[c][0.66\textheight][s]{\columnwidth}
      \begin{itemize}\color{gray}
        \item Checks wheteher exceptions backtrace should be recorded, by checking the \funcarg{domain\_state->backtrace\_active} value
        \item Switches to the C stack to be able to execute C code
        \item Calls \funcname{caml\_stash\_backtrace}, to collect the backtrace up to the next trap
      \end{itemize}
      \onslide*<2->{
        \begin{itemize}
          \item<2-> Executes the exception handler in \funcname{probably\_raise} with \funcname{RESTORE\_EXN\_HANDLER\_OCAML}
            \begin{itemize}
              \item Set \regname{RSP} to \localname{domain\_state->exn\_handler} value
              \item Restore \localname{domain\_state->exn\_handler} from trap
              \item Jump to the handler code with \funcname{ret}
            \end{itemize}
        \end{itemize}
      }
      \vfill
      \onslide*<1>{\mintocaml[firstline=9,lastline=13,highlightlines=12]{../src/backtrace/backtrace.ml}}
      \onslide*<2->{\mintocaml[firstline=15,lastline=21,highlightlines={19-21}]{../src/backtrace/backtrace.ml}}
      \endminipage
    \end{column}
    \begin{column}{0.5\textwidth}
      \centering
      \onslide*<1>{
        \mintc[firstline=190,lastline=192]{../ocaml/runtime/amd64.S}
        \mintc[firstline=234,lastline=237]{../ocaml/runtime/amd64.S}
      }
      \onslide*<2>{
        \mintc[firstline=190,lastline=192,highlightlines={191-192}]{../ocaml/runtime/amd64.S}
        \mintc[firstline=234,lastline=237,highlightlines={235-237}]{../ocaml/runtime/amd64.S}
      }
      \onslide*<1>{\listCamlRaiseExn[highlightlines=720]{}}
      \onslide*<2>{\listCamlRaiseExn[highlightlines=722]{}}
      \onslide*<3->{\providecommand\step{5}\input{diagrams/backtrace/backtrace.tex}}
    \end{column}
  \end{columns}
\end{frame}

\begin{frame}{Backtraces}{\funcname{caml\_probably\_raise}'s handler doesn't catch \typename{ExnC}}
  \begin{columns}[c]
    \begin{column}{0.45\textwidth}
      \minipage[c][0.75\textheight][s]{\columnwidth}
      \begin{itemize}
        \item<1-> \funcname{match \dots with}\footnotemark pattern matching can also match exceptions
        \item<1-> The CMM of \funcname{probably\_raise} shows that this \funcname{match \dots with} is implemented with a pattern matching and a \funcname{try \dots with}
        \item<1-> But this one only matches on \typename{ExnA} exception
        \item<2-> Since the pattern matching fails to catch the exception, \funcname{reraise} is called with our current \typename{ExnC} exception
        \item<3-> Disassembly shows that \funcname{reraise} is actually a call to \funcname{caml\_reraise\_exn}, which is also an assembly function provided by the runtime
      \end{itemize}
      \vfill
      \mintocaml[firstline=15,lastline=21,highlightlines={18-21}]{../src/backtrace/backtrace.ml}
      \endminipage
    \end{column}
    \begin{column}{0.55\textwidth}
      \centering
      \onslide*<1>{\mintcmm[highlightlines={10-18}]{../src/backtrace/camlBacktrace\_\_probably\_raise\_351.cmm}}
      \onslide*<2>{\listcmm[highlightlines=18]{../src/backtrace/camlBacktrace\_\_probably\_raise_351.cmm}{CMM of probably\_raise}}
      \onslide*<3>{\mintobjdump[firstline=5,lastline=35,highlightlines=31]{../src/backtrace/camlBacktrace\_\_probably\_raise_351.objdump}}
    \end{column}
  \end{columns}
  \only<1>{\footnotetext[1]{https://blog.janestreet.com/pattern-matching-and-exception-handling-unite/}}
\end{frame}

\newcommand\listCamlReraiseExn[1][]{
  \listasm[firstline=727,lastline=734,#1]{../ocaml/runtime/amd64.S}{runtime/amd64.S:727}}

\begin{frame}{Backtraces}{Introducing \funcname{caml\_reraise\_exn}}
  \begin{columns}[c]
    \begin{column}{0.5\textwidth}
      \minipage[c][0.66\textheight][s]{\columnwidth}
      \begin{itemize}
        \item<1-> The exception have to be reraised to the next handler
        \item<2-> First, checks if the backtrace should be collected with \funcarg{domain\_state->backtrace\_active}
        \only<3>{
        \begin{itemize}
            \item If not, behaves like a \funcname{raise\_notrace} with \funcname{RESTORE\_EXN\_HANDLER\_OCAML}
        \end{itemize}
        }
        \item<4-> Jumps into \funcname{caml\_raise\_exn}
        \item<5-> Calls \funcname{caml\_stash\_backtrace} again, to scan the next part of the stack: from the trap just pop-ed to the next trap
      \end{itemize}
      \vfill
      \mintocaml[firstline=15,lastline=21,highlightlines={18-21}]{../src/backtrace/backtrace.ml}
      \endminipage
    \end{column}
    \begin{column}{0.5\textwidth}
      \raggedleft
      \onslide*<1>{\listCamlReraiseExn{}}
      \onslide*<2>{\listCamlReraiseExn[highlightlines=729]{}}
      \onslide*<3>{
        \mintc[firstline=190,lastline=192,highlightlines={191-192}]{../ocaml/runtime/amd64.S}
        \mintc[firstline=234,lastline=237,highlightlines={235-237}]{../ocaml/runtime/amd64.S}
        \medskip
        \listCamlReraiseExn[highlightlines=731]{}
      }
      \onslide*<4-5>{
        % TODO refactor with \listCamlReraiseExn
        \mintasm[firstline=727,lastline=734,highlightlines=730]{../ocaml/runtime/amd64.S}
        \medskip
      }
      \onslide*<4>{\listCamlRaiseExn[highlightlines=711]{}}
      \onslide*<5>{\listCamlRaiseExn[highlightlines=720]{}}
    \end{column}
  \end{columns}
\end{frame}

\begin{frame}{Backtraces}{Backtrace collection continues}
  \begin{columns}[c]
    \begin{column}{0.55\textwidth}
      \minipage[c][0.75\textheight][s]{\columnwidth}
      \begin{itemize}
        \item<1-> \funcname{caml\_stash\_backtrace} scans the older part of the stack, up to the next trap
        \item<6-> Controls jumps to the nested exception handler in \funcname{maybe\_raise}, which matches only on \typename{ExnB}
        \item<7-> \funcname{caml\_reraise\_exn} is called again, backtrace collection resumes with \funcname{caml\_stash\_backtrace}
      \end{itemize}
      \medskip
      \begin{itemize}
        \item<2-> Constructed backtrace:
        \begin{itemize}
          \item <2-> \funcname{definitely\_raise}
          \item <2-> \funcname{probably\_raise}
          \item <3-> \funcname{probably\_raise} (re-raise)
          \item <4-> \funcname{maybe\_raise}
          \item <5-> \funcname{main}
          \item <8-> \funcname{main} (re-raise)
        \end{itemize}
      \end{itemize}
      \vfill
      \onslide*<1-5>{\mintocaml[firstline=15,lastline=21]{../src/backtrace/backtrace.ml}}
      \onslide*<6>{\mintocaml[firstline=28,lastline=34,highlightlines=31]{../src/backtrace/backtrace.ml}}
      \onslide*<7->{\mintocaml[firstline=28,lastline=34,highlightlines=33]{../src/backtrace/backtrace.ml}}
      \endminipage
    \end{column}
    \begin{column}{0.45\textwidth}
      \raggedleft
      \onslide*<1>{\providecommand\step{6}\input{diagrams/backtrace/backtrace.tex}}%
      \onslide*<2>{\providecommand\step{6}\input{diagrams/backtrace/backtrace.tex}}%
      \onslide*<3>{\providecommand\step{7}\input{diagrams/backtrace/backtrace.tex}}%
      \onslide*<4>{\providecommand\step{8}\input{diagrams/backtrace/backtrace.tex}}%
      \onslide*<5>{\providecommand\step{9}\input{diagrams/backtrace/backtrace.tex}}%
      \onslide*<6>{\providecommand\step{10}\input{diagrams/backtrace/backtrace.tex}}%
      \onslide*<7>{\providecommand\step{11}\input{diagrams/backtrace/backtrace.tex}}%
      \onslide*<8>{\providecommand\step{12}\input{diagrams/backtrace/backtrace.tex}}%
      \onslide*<9>{\providecommand\step{13}\input{diagrams/backtrace/backtrace.tex}}%
    \end{column}
  \end{columns}
\end{frame}

\begin{frame}{Backtraces}{Printing the backtrace}
  \begin{columns}[c]
    \begin{column}{0.45\textwidth}
      \minipage[c][0.66\textheight][s]{\columnwidth}
      \begin{itemize}
        \item<1-> The exception backtrace can now be printed to \funcarg{stdout} with \funcname{Printexc.print_backtrace}
        \item<2-> First the exception backtrace is retrieved into an OCaml array with \funcname{get_raw_backtrace }
        \item<3-> Which is implemented as the \funcname{caml_get_exception_raw_backtrace} C function in the runtime
        \item<4-> And finally printed with \funcname{print_exception_backtrace}
      \end{itemize}
      \vfill
      \mintocaml[firstline=28,lastline=34,highlightlines=33]{../src/backtrace/backtrace.ml}
      \endminipage
    \end{column}
    \begin{column}{0.55\textwidth}
      \onslide*<1,2>{
        \mintocaml[firstline=100,lastline=101]{../ocaml/stdlib/printexc.ml}
      }
      \only<1>{
        \listocaml[firstline=170,lastline=172]{../ocaml/stdlib/printexc.ml}{stdlib/printexc.ml:170}
      }
      \only<2>{
        \listocaml[firstline=170,lastline=172,highlightlines=172]{../ocaml/stdlib/printexc.ml}{stdlib/printexc.ml:170}
      }
      \only<3>{
        \mintc[firstline=154,lastline=156]{../ocaml/runtime/backtrace.c}
        \listc[firstline=171,lastline=192,highlightlines={184-187}]{../ocaml/runtime/backtrace.c}{runtime/backtrace.c:154}
      }
      \only<4>{
        \listocaml[firstline=155,lastline=165]{../ocaml/stdlib/printexc.ml}{stdlib/printexc.ml:55}
      }
    \end{column}
  \end{columns}
\end{frame}


\subsection{The multiple kinds of raise}
\frameSubsection{}{}

\begin{frame}{Backtraces}{When backtraces are not needed}
  \begin{columns}[c]
    \begin{column}{0.45\textwidth}
      \minipage[c][0.66\textheight][s]{\columnwidth}
      \begin{itemize}
        \item<1-> What if the backtrace for an exception is never needed?
        \item<2-> It's possible to raise the exception explicitly with \funcname{raise\_notrace} to make raising as cheap as possible
        \item<3-> An example of use in the \funcname{dynlink} library of the OCaml compiler
      \end{itemize}
      \vfill
      \onslide<3->{
      \listocaml[firstline=205,lastline=209,highlightlines=207]{../ocaml/otherlibs/dynlink/dynlink\_compilerlibs/misc.ml}{otherlibs/dynlink/dynlink\_compilerlibs/misc.ml:205}
      }
      \endminipage
    \end{column}
    \begin{column}{0.55\textwidth}
    \only<4>{
      \mintcmm[highlightlines=3]{../src/notrace/camlNotrace\_\_fun_347.cmm}
      \listcmm{../src/notrace/camlNotrace\_\_all\_somes\_327.cmm}{all\_somes}
    }
    \onslide*<5->{
      \listobjdump[highlightlines={6-9}]{../src/notrace/camlNotrace\_\_fun_347.objdump}{anonymous function from \funcname{all\_somes}}
    }
    \end{column}
  \end{columns}
\end{frame}

\begin{frame}{Backtraces}{Summary table of the multiple kinds of raise}
  \begin{itemize}
    \item When debugging information is enabled and if the backtrace of an exception isn't needed, it's possible to raise with \funcname{raise\_notrace} for performance
    \item When debugging information is \emph{not} enabled, the compiler optimises \funcname{raise} calls to inline assembly (same effect as using \funcname{raise\_notrace})
  \end{itemize}
  \bigskip
  \centering
  \begin{tabular}{| l | c | c | c | c |}
    \hline
    \parbox{10em}{Debug information \\ (\funcarg{-g} flag)}  & \multicolumn{2}{|c|}{Disabled}                                     & \multicolumn{2}{|c|}{Enabled} \\
    \hline
    Raise kind                                               & \funcname{raise} & \funcname{raise\_notrace}                       & \funcname{raise} & \funcname{raise\_notrace} \\
    \hline
    Compilation                                              & \parbox{8em}{inline assembly\\ (optimization)} & inline assembly   & \funcname{caml\_raise\_exn} & inline assembly \\
    \hline
  \end{tabular}
\end{frame}


%
% Takeaway #5
%
\subsection*{Takeaway \#5}
\frameSubsectionTakeaway{}
\againframe<5>{takeaway}
}%
      \onslide*<2>{\providecommand\step{6}%
% Backtraces
%
\subsection{One advantage of raising with debugging information}
\frameSubsection{}{}

\begin{frame}{Backtraces}{Debugging information \& source code}
  \begin{columns}[c]
    \begin{column}{0.5\textwidth}
      \begin{itemize}
        \item Raising an exception is fun and games, but how do we know where the exception originated? $\rightarrow$ With the backtrace associated with the exception!
        \item In order to get a backtrace from the point where an exception was raised, it's necessary to compile with debugging information enabled (\funcarg{-g} flag for the compiler)
        \item Same example as in the `Nested handlers' section
      \end{itemize}
    \end{column}
    \begin{column}{0.5\textwidth}
      \listocaml[firstline=9,lastline=34]{../src/backtrace/backtrace.ml}{backtrace/backtrace.ml}
    \end{column}
  \end{columns}
\end{frame}

\begin{frame}{Backtraces}{CMM and disassembly of \funcname{definitely\_raise}}
  \begin{columns}[c]
    \begin{column}{0.5\textwidth}
      \minipage[c][0.66\textheight][s]{\columnwidth}
      \begin{itemize}
        \item<1-> An effect of compiling with debugging information (\funcarg{-g}): the CMM no longer uses \funcname{raise\_notrace} to raise the exception but \funcname{raise} instead
        \item<2-> Disassembly shows that CMM \funcname{raise} is actually a call to \funcname{caml\_raise\_exn}, a function in assembler provided by the runtime
      \end{itemize}
      \vfill
      \mintocaml[firstline=9,lastline=13,highlightlines=12]{../src/backtrace/backtrace.ml}
      \endminipage
    \end{column}
    \begin{column}{0.5\textwidth}
      \onslide*<1>{
        \mintcmm[highlightlines=5]{../src/backtrace/camlBacktrace\_\_definitely\_raise\_347.cmm}
      }
      \onslide*<2>{
        \mintobjdump[firstline=2,highlightlines=14]{../src/backtrace/camlBacktrace\_\_definitely\_raise\_347.objdump}
      }
    \end{column}
  \end{columns}
\end{frame}

\begin{frame}{Backtraces}{Introducing \funcname{caml\_raise\_exn}}
  \begin{columns}[c]
    \begin{column}{0.5\textwidth}
      \minipage[c][0.66\textheight][s]{\columnwidth}
      \onslide*<1>{
        \begin{itemize}
          \item Raising the exception is performed using \funcname{caml\_raise\_exn} which is
            \begin{itemize}
              \item An assembly function provided by the runtime
              \item Architecture specific
            \end{itemize}
          \item Let's see what it does differently from \funcname{raise\_notrace}\dots
          \item We are about to raise \typename{ExnC} with \funcname{caml\_raise\_exn} from \funcname{definitely\_raise}
        \end{itemize}
      }
      \onslide*<2>{
        \mintobjdump[firstline=2,highlightlines=14]{../src/backtrace/camlBacktrace\_\_definitely\_raise\_347.objdump}
      }
      \vfill
      \mintocaml[firstline=9,lastline=13,highlightlines=12]{../src/backtrace/backtrace.ml}
      \endminipage
    \end{column}
    \begin{column}{0.5\textwidth}
      \centering
      \providecommand\step{1}\input{diagrams/backtrace/backtrace.tex}
    \end{column}
  \end{columns}
\end{frame}

\begin{frame}{Backtraces}{But before that, we need more fields from \typename{caml\_domain\_state}}
  \begin{columns}
    \begin{column}{0.5\textwidth}
      \begin{itemize}
        \item Remember \typename{caml\_domain\_state} the per-domain runtime state?
        \item Some other members are of interest to us now\dots
        \item \localname{backtrace\_active} is used to know if the runtime should collect backtraces
        \item \localname{backtrace\_active} can be set by
          \begin{itemize}
            \item The \funcarg{b} flag in the \funcname{OCAMLRUNPARAM} environment variable
            \item \mintinline{ocaml}{Printexc.record_backtrace : bool -> unit} \footnotemark
          \end{itemize}
      \end{itemize}
    \end{column}
    \begin{column}{0.5\textwidth}
      \begin{listing}
        \mintc[highlightlines={9-11}]{src/ocaml/domain\_state\_backtrace.h}
        \caption{runtime/caml/domain\_state.h}
      \end{listing}
    \end{column}
  \end{columns}
  \footnotetext[1]{https://v2.ocaml.org/api/Printexc.html}
\end{frame}

\newcommand\listCamlRaiseExn[1][]{
  \listasm[firstline=702,lastline=725,#1]{../ocaml/runtime/amd64.S}{runtime/amd64.S:702}}

\begin{frame}{Backtraces}{Walk through \funcname{caml\_raise\_exn}}
  \begin{columns}[T]
    \begin{column}{0.5\textwidth}
      \begin{itemize}
        \item<1-> First checks whether exceptions backtrace should be recorded
        \item<1-> By checking the \funcarg{domain\_state->backtrace\_active} value
        \item<2-> If not, acts as \funcname{raise\_notrace} with \funcname{RESTORE\_EXN\_HANDLER\_OCAML}
          \begin{itemize}
            \item Set \regname{RSP} to \localname{domain\_state->exn\_handler} value
            \item Restore \localname{domain\_state->exn\_handler} from trap
            \item Jump with \funcname{ret} (same effect as using \funcname{pop} and \funcname{jmp})
          \end{itemize}
        \item<3-> Otherwise, switches to the C stack to be able to execute C code
        \item<4-> Calls \funcname{caml\_stash\_backtrace}, a C function provided by the runtime\ignorespaces
          \onslide<6->{, with
            \begin{itemize}
              \item \funcarg{exn}: exception value
              \item \funcarg{pc}: code address of the raising point
              \item \funcarg{sp}: stack address of the raising function
              \item \funcarg{trapsp}: stack address of the next handler
            \end{itemize}
          }
      \end{itemize}
      \bigskip
      \mintocaml[firstline=9,lastline=13,highlightlines=12]{../src/backtrace/backtrace.ml}
    \end{column}
    \begin{column}{0.5\textwidth}
      \raggedleft
      \onslide*<1,3-4>{
        \mintc[firstline=190,lastline=192]{../ocaml/runtime/amd64.S}
        \mintc[firstline=234,lastline=237]{../ocaml/runtime/amd64.S}
      }
      \onslide*<2>{
        \mintc[firstline=190,lastline=192,highlightlines={191-192}]{../ocaml/runtime/amd64.S}
        \mintc[firstline=234,lastline=237,highlightlines={235-237}]{../ocaml/runtime/amd64.S}
      }
      \onslide*<1>{\listCamlRaiseExn[highlightlines=705]{}}%
      \onslide*<2>{\listCamlRaiseExn[highlightlines={707-708}]{}}%
      \onslide*<3>{\listCamlRaiseExn[highlightlines={712-714}]{}}%
      \onslide*<4>{\listCamlRaiseExn[highlightlines={715-720}]{}}%
      \onslide*<5>{\providecommand\step{1}\input{diagrams/backtrace/backtrace.tex}}%
      \onslide*<6>{\providecommand\step{2}\input{diagrams/backtrace/backtrace.tex}}%
    \end{column}
  \end{columns}
\end{frame}

\newcommand\listStashBacktrace[1][]{
  \listc[firstline=91,lastline=120,#1]{../ocaml/runtime/backtrace\_nat.c}{runtime/backtrace\_nat.c:91}}

\begin{frame}{Backtraces}{Introducing \funcname{caml\_stash\_backtrace}}
  \begin{columns}[c]
    \begin{column}{0.5\textwidth}
      \minipage[c][0.66\textheight][s]{\columnwidth}
      \begin{itemize}
        \item<1-> A C function provided by the runtime (specific to native code)
        \item<2-> It scans the stack, between the stack pointer at the raising point up to the next trap, in order to collect the names of the detected function frames
          \onslide*<2>{
            \begin{itemize}
              \item And more information, like line number, column number...
            \end{itemize}
          }
        \item<3-> It uses the same technique and information as the Garbage Collector (GC) uses to scan the values live on the stack
          \onslide*<4>{
            \begin{itemize}
              \item \typename{frame\_descr} are at the core of stack unwinding
            \end{itemize}
          }
      \end{itemize}
      \vfill
      \mintocaml[firstline=9,lastline=13,highlightlines=12]{../src/backtrace/backtrace.ml}
      \endminipage
    \end{column}
    \begin{column}{0.5\textwidth}
      \onslide*<1>{\listStashBacktrace[highlightlines=91]{}}
      \onslide*<2>{\listStashBacktrace[highlightlines={91,117-118}]{}}
      \onslide*<3->{\listStashBacktrace[highlightlines={94,106,109-110}]{}}
    \end{column}
  \end{columns}
\end{frame}

\newcommand\listFrameDescr[1][]{
  \listocaml[firstline=111,lastline=124,#1]{../ocaml/asmcomp/emitaux.ml}{asmcomp/emitaux.ml:111}}
\newcommand\listDebugInfo[1][]{
  \listocaml[firstline=38,lastline=49,#1]{../ocaml/lambda/debuginfo.mli}{lambda/debuginfo.mli:38}}

\begin{frame}{Backtraces}{\typename{frame\_descr} and \typename{frame\_debuginfo} in a nutshell}
  \begin{columns}[c]
    \begin{column}{0.5\textwidth}
      \begin{itemize}
        \item<1-> For every return address in an OCaml program, there is a associated \typename{frame\_descr}
        \item<2-> A \typename{frame\_descr} specifies
        \begin{itemize}
          \item<2-> The size of the function stack frame
          \item<3-> Where live values are on the stack (so that the GC can mark them as live and prevent from being swept)
        \end{itemize}
        \item<4-> When compiled with debug information, \typename{frame\_descr} will be linked to a \typename{frame\_debuginfo}
        \begin{itemize}
          \item<5-> Contains information about the position in the source code (file name, line and column number)
        \end{itemize}
      \end{itemize}
    \end{column}
    \begin{column}{0.5\textwidth}
        \onslide*<1>{\listFrameDescr[highlightlines={118-122}]{}}
        \onslide*<2>{\listFrameDescr[highlightlines={120}]{}}
        \onslide*<3>{\listFrameDescr[highlightlines={121}]{}}
        \onslide*<4->{\listFrameDescr[highlightlines={122}]{}}
        \medskip
        \onslide*<1-3>{\listDebugInfo{}}
        \onslide*<4>{\listDebugInfo[highlightlines={38,49}]{}}
        \onslide*<5>{\listDebugInfo[highlightlines={39-46}]{}}
    \end{column}
  \end{columns}
\end{frame}

\begin{frame}{Backtraces}{Scanning the stack with \typename{frame\_descr}}
  \begin{columns}[c]
    \begin{column}{0.5\textwidth}
      \begin{itemize}
        \item Given the return address of the code that raises the exception by calling \funcname{caml\_raise\_exn} (\funcarg{pc} argument of \funcname{caml\_stash\_backtrace})
        \item And the position of the stack pointer (\funcarg{sp} argument of \funcname{caml\_stash\_backtrace})
        \item The frame size given by \typename{frame\_descr}.\funcarg{frame\_size}, allows to find the end of the previous frame
        \item And the return address of the caller of this function
        \bigskip
        \onslide<3->{
        \item Note that traps (here the reddish \funcname{ExnA}, \funcname{ExnB} and \funcname{ExnC} blocks) belong to the frame that installed them, and so the frame size changes when traps are removed
        }
      \end{itemize}
    \end{column}
    \begin{column}{0.5\textwidth}
      \raggedleft
      \onslide*<1>{\providecommand\step{2}\input{diagrams/backtrace/backtrace.tex}}%
      \onslide*<2>{\providecommand\step{1}\input{diagrams/backtrace/scanstack.tex}}%
      \onslide*<3>{\providecommand\step{2}\input{diagrams/backtrace/scanstack.tex}}%
      \onslide*<4>{\providecommand\step{3}\input{diagrams/backtrace/scanstack.tex}}%
      \onslide*<5>{\providecommand\step{4}\input{diagrams/backtrace/scanstack.tex}}%
      \onslide*<6>{\providecommand\step{5}\input{diagrams/backtrace/scanstack.tex}}%
    \end{column}
  \end{columns}
\end{frame}

% Duplicate of previous-previous slide
\begin{frame}{Backtraces}{Continuing on \funcname{caml\_stash\_backtrace}}
  \begin{columns}[c]
    \begin{column}{0.5\textwidth}
      \minipage[c][0.75\textheight][s]{\columnwidth}
      \begin{itemize}
          \color{gray}
        \item A C function provided by the runtime (specific to native code)
        \item It scans the stack, between the stack pointer at the raising point up to the next trap, in order to collect the names of the detected function frames
        \item It uses the same technique and information as the Garbage Collector (GC) uses to scan the values live on the stack
      \end{itemize}
      \begin{itemize}
        \item For each function frame detected, store a pointer to the \typename{frame\_descr} in the \localname{domain\_state->backtrace\_buffer} array, effectively constructing the backtrace
      \end{itemize}
      \onslide<2->{
        \begin{itemize}
            \smallskip
          \item Constructed backtrace:
            \funcname{definitely\_raise},
            \onslide<3->{\funcname{probably\_raise}}
        \end{itemize}
      }
      \vfill
      \mintocaml[firstline=9,lastline=13,highlightlines=12]{../src/backtrace/backtrace.ml}
      \endminipage
    \end{column}
    \begin{column}{0.5\textwidth}
      \raggedleft
      \onslide*<1>{\listStashBacktrace[highlightlines={114-115}]{}}
      \onslide*<2>{\providecommand\step{3}\input{diagrams/backtrace/backtrace.tex}}%
      \onslide*<3>{\providecommand\step{4}\input{diagrams/backtrace/backtrace.tex}}%
    \end{column}
  \end{columns}
\end{frame}

\begin{frame}{Backtraces}{Back to \funcname{caml\_raise\_exn}}
  \begin{columns}[c]
    \begin{column}{0.5\textwidth}
      \minipage[c][0.66\textheight][s]{\columnwidth}
      \begin{itemize}\color{gray}
        \item Checks wheteher exceptions backtrace should be recorded, by checking the \funcarg{domain\_state->backtrace\_active} value
        \item Switches to the C stack to be able to execute C code
        \item Calls \funcname{caml\_stash\_backtrace}, to collect the backtrace up to the next trap
      \end{itemize}
      \onslide*<2->{
        \begin{itemize}
          \item<2-> Executes the exception handler in \funcname{probably\_raise} with \funcname{RESTORE\_EXN\_HANDLER\_OCAML}
            \begin{itemize}
              \item Set \regname{RSP} to \localname{domain\_state->exn\_handler} value
              \item Restore \localname{domain\_state->exn\_handler} from trap
              \item Jump to the handler code with \funcname{ret}
            \end{itemize}
        \end{itemize}
      }
      \vfill
      \onslide*<1>{\mintocaml[firstline=9,lastline=13,highlightlines=12]{../src/backtrace/backtrace.ml}}
      \onslide*<2->{\mintocaml[firstline=15,lastline=21,highlightlines={19-21}]{../src/backtrace/backtrace.ml}}
      \endminipage
    \end{column}
    \begin{column}{0.5\textwidth}
      \centering
      \onslide*<1>{
        \mintc[firstline=190,lastline=192]{../ocaml/runtime/amd64.S}
        \mintc[firstline=234,lastline=237]{../ocaml/runtime/amd64.S}
      }
      \onslide*<2>{
        \mintc[firstline=190,lastline=192,highlightlines={191-192}]{../ocaml/runtime/amd64.S}
        \mintc[firstline=234,lastline=237,highlightlines={235-237}]{../ocaml/runtime/amd64.S}
      }
      \onslide*<1>{\listCamlRaiseExn[highlightlines=720]{}}
      \onslide*<2>{\listCamlRaiseExn[highlightlines=722]{}}
      \onslide*<3->{\providecommand\step{5}\input{diagrams/backtrace/backtrace.tex}}
    \end{column}
  \end{columns}
\end{frame}

\begin{frame}{Backtraces}{\funcname{caml\_probably\_raise}'s handler doesn't catch \typename{ExnC}}
  \begin{columns}[c]
    \begin{column}{0.45\textwidth}
      \minipage[c][0.75\textheight][s]{\columnwidth}
      \begin{itemize}
        \item<1-> \funcname{match \dots with}\footnotemark pattern matching can also match exceptions
        \item<1-> The CMM of \funcname{probably\_raise} shows that this \funcname{match \dots with} is implemented with a pattern matching and a \funcname{try \dots with}
        \item<1-> But this one only matches on \typename{ExnA} exception
        \item<2-> Since the pattern matching fails to catch the exception, \funcname{reraise} is called with our current \typename{ExnC} exception
        \item<3-> Disassembly shows that \funcname{reraise} is actually a call to \funcname{caml\_reraise\_exn}, which is also an assembly function provided by the runtime
      \end{itemize}
      \vfill
      \mintocaml[firstline=15,lastline=21,highlightlines={18-21}]{../src/backtrace/backtrace.ml}
      \endminipage
    \end{column}
    \begin{column}{0.55\textwidth}
      \centering
      \onslide*<1>{\mintcmm[highlightlines={10-18}]{../src/backtrace/camlBacktrace\_\_probably\_raise\_351.cmm}}
      \onslide*<2>{\listcmm[highlightlines=18]{../src/backtrace/camlBacktrace\_\_probably\_raise_351.cmm}{CMM of probably\_raise}}
      \onslide*<3>{\mintobjdump[firstline=5,lastline=35,highlightlines=31]{../src/backtrace/camlBacktrace\_\_probably\_raise_351.objdump}}
    \end{column}
  \end{columns}
  \only<1>{\footnotetext[1]{https://blog.janestreet.com/pattern-matching-and-exception-handling-unite/}}
\end{frame}

\newcommand\listCamlReraiseExn[1][]{
  \listasm[firstline=727,lastline=734,#1]{../ocaml/runtime/amd64.S}{runtime/amd64.S:727}}

\begin{frame}{Backtraces}{Introducing \funcname{caml\_reraise\_exn}}
  \begin{columns}[c]
    \begin{column}{0.5\textwidth}
      \minipage[c][0.66\textheight][s]{\columnwidth}
      \begin{itemize}
        \item<1-> The exception have to be reraised to the next handler
        \item<2-> First, checks if the backtrace should be collected with \funcarg{domain\_state->backtrace\_active}
        \only<3>{
        \begin{itemize}
            \item If not, behaves like a \funcname{raise\_notrace} with \funcname{RESTORE\_EXN\_HANDLER\_OCAML}
        \end{itemize}
        }
        \item<4-> Jumps into \funcname{caml\_raise\_exn}
        \item<5-> Calls \funcname{caml\_stash\_backtrace} again, to scan the next part of the stack: from the trap just pop-ed to the next trap
      \end{itemize}
      \vfill
      \mintocaml[firstline=15,lastline=21,highlightlines={18-21}]{../src/backtrace/backtrace.ml}
      \endminipage
    \end{column}
    \begin{column}{0.5\textwidth}
      \raggedleft
      \onslide*<1>{\listCamlReraiseExn{}}
      \onslide*<2>{\listCamlReraiseExn[highlightlines=729]{}}
      \onslide*<3>{
        \mintc[firstline=190,lastline=192,highlightlines={191-192}]{../ocaml/runtime/amd64.S}
        \mintc[firstline=234,lastline=237,highlightlines={235-237}]{../ocaml/runtime/amd64.S}
        \medskip
        \listCamlReraiseExn[highlightlines=731]{}
      }
      \onslide*<4-5>{
        % TODO refactor with \listCamlReraiseExn
        \mintasm[firstline=727,lastline=734,highlightlines=730]{../ocaml/runtime/amd64.S}
        \medskip
      }
      \onslide*<4>{\listCamlRaiseExn[highlightlines=711]{}}
      \onslide*<5>{\listCamlRaiseExn[highlightlines=720]{}}
    \end{column}
  \end{columns}
\end{frame}

\begin{frame}{Backtraces}{Backtrace collection continues}
  \begin{columns}[c]
    \begin{column}{0.55\textwidth}
      \minipage[c][0.75\textheight][s]{\columnwidth}
      \begin{itemize}
        \item<1-> \funcname{caml\_stash\_backtrace} scans the older part of the stack, up to the next trap
        \item<6-> Controls jumps to the nested exception handler in \funcname{maybe\_raise}, which matches only on \typename{ExnB}
        \item<7-> \funcname{caml\_reraise\_exn} is called again, backtrace collection resumes with \funcname{caml\_stash\_backtrace}
      \end{itemize}
      \medskip
      \begin{itemize}
        \item<2-> Constructed backtrace:
        \begin{itemize}
          \item <2-> \funcname{definitely\_raise}
          \item <2-> \funcname{probably\_raise}
          \item <3-> \funcname{probably\_raise} (re-raise)
          \item <4-> \funcname{maybe\_raise}
          \item <5-> \funcname{main}
          \item <8-> \funcname{main} (re-raise)
        \end{itemize}
      \end{itemize}
      \vfill
      \onslide*<1-5>{\mintocaml[firstline=15,lastline=21]{../src/backtrace/backtrace.ml}}
      \onslide*<6>{\mintocaml[firstline=28,lastline=34,highlightlines=31]{../src/backtrace/backtrace.ml}}
      \onslide*<7->{\mintocaml[firstline=28,lastline=34,highlightlines=33]{../src/backtrace/backtrace.ml}}
      \endminipage
    \end{column}
    \begin{column}{0.45\textwidth}
      \raggedleft
      \onslide*<1>{\providecommand\step{6}\input{diagrams/backtrace/backtrace.tex}}%
      \onslide*<2>{\providecommand\step{6}\input{diagrams/backtrace/backtrace.tex}}%
      \onslide*<3>{\providecommand\step{7}\input{diagrams/backtrace/backtrace.tex}}%
      \onslide*<4>{\providecommand\step{8}\input{diagrams/backtrace/backtrace.tex}}%
      \onslide*<5>{\providecommand\step{9}\input{diagrams/backtrace/backtrace.tex}}%
      \onslide*<6>{\providecommand\step{10}\input{diagrams/backtrace/backtrace.tex}}%
      \onslide*<7>{\providecommand\step{11}\input{diagrams/backtrace/backtrace.tex}}%
      \onslide*<8>{\providecommand\step{12}\input{diagrams/backtrace/backtrace.tex}}%
      \onslide*<9>{\providecommand\step{13}\input{diagrams/backtrace/backtrace.tex}}%
    \end{column}
  \end{columns}
\end{frame}

\begin{frame}{Backtraces}{Printing the backtrace}
  \begin{columns}[c]
    \begin{column}{0.45\textwidth}
      \minipage[c][0.66\textheight][s]{\columnwidth}
      \begin{itemize}
        \item<1-> The exception backtrace can now be printed to \funcarg{stdout} with \funcname{Printexc.print_backtrace}
        \item<2-> First the exception backtrace is retrieved into an OCaml array with \funcname{get_raw_backtrace }
        \item<3-> Which is implemented as the \funcname{caml_get_exception_raw_backtrace} C function in the runtime
        \item<4-> And finally printed with \funcname{print_exception_backtrace}
      \end{itemize}
      \vfill
      \mintocaml[firstline=28,lastline=34,highlightlines=33]{../src/backtrace/backtrace.ml}
      \endminipage
    \end{column}
    \begin{column}{0.55\textwidth}
      \onslide*<1,2>{
        \mintocaml[firstline=100,lastline=101]{../ocaml/stdlib/printexc.ml}
      }
      \only<1>{
        \listocaml[firstline=170,lastline=172]{../ocaml/stdlib/printexc.ml}{stdlib/printexc.ml:170}
      }
      \only<2>{
        \listocaml[firstline=170,lastline=172,highlightlines=172]{../ocaml/stdlib/printexc.ml}{stdlib/printexc.ml:170}
      }
      \only<3>{
        \mintc[firstline=154,lastline=156]{../ocaml/runtime/backtrace.c}
        \listc[firstline=171,lastline=192,highlightlines={184-187}]{../ocaml/runtime/backtrace.c}{runtime/backtrace.c:154}
      }
      \only<4>{
        \listocaml[firstline=155,lastline=165]{../ocaml/stdlib/printexc.ml}{stdlib/printexc.ml:55}
      }
    \end{column}
  \end{columns}
\end{frame}


\subsection{The multiple kinds of raise}
\frameSubsection{}{}

\begin{frame}{Backtraces}{When backtraces are not needed}
  \begin{columns}[c]
    \begin{column}{0.45\textwidth}
      \minipage[c][0.66\textheight][s]{\columnwidth}
      \begin{itemize}
        \item<1-> What if the backtrace for an exception is never needed?
        \item<2-> It's possible to raise the exception explicitly with \funcname{raise\_notrace} to make raising as cheap as possible
        \item<3-> An example of use in the \funcname{dynlink} library of the OCaml compiler
      \end{itemize}
      \vfill
      \onslide<3->{
      \listocaml[firstline=205,lastline=209,highlightlines=207]{../ocaml/otherlibs/dynlink/dynlink\_compilerlibs/misc.ml}{otherlibs/dynlink/dynlink\_compilerlibs/misc.ml:205}
      }
      \endminipage
    \end{column}
    \begin{column}{0.55\textwidth}
    \only<4>{
      \mintcmm[highlightlines=3]{../src/notrace/camlNotrace\_\_fun_347.cmm}
      \listcmm{../src/notrace/camlNotrace\_\_all\_somes\_327.cmm}{all\_somes}
    }
    \onslide*<5->{
      \listobjdump[highlightlines={6-9}]{../src/notrace/camlNotrace\_\_fun_347.objdump}{anonymous function from \funcname{all\_somes}}
    }
    \end{column}
  \end{columns}
\end{frame}

\begin{frame}{Backtraces}{Summary table of the multiple kinds of raise}
  \begin{itemize}
    \item When debugging information is enabled and if the backtrace of an exception isn't needed, it's possible to raise with \funcname{raise\_notrace} for performance
    \item When debugging information is \emph{not} enabled, the compiler optimises \funcname{raise} calls to inline assembly (same effect as using \funcname{raise\_notrace})
  \end{itemize}
  \bigskip
  \centering
  \begin{tabular}{| l | c | c | c | c |}
    \hline
    \parbox{10em}{Debug information \\ (\funcarg{-g} flag)}  & \multicolumn{2}{|c|}{Disabled}                                     & \multicolumn{2}{|c|}{Enabled} \\
    \hline
    Raise kind                                               & \funcname{raise} & \funcname{raise\_notrace}                       & \funcname{raise} & \funcname{raise\_notrace} \\
    \hline
    Compilation                                              & \parbox{8em}{inline assembly\\ (optimization)} & inline assembly   & \funcname{caml\_raise\_exn} & inline assembly \\
    \hline
  \end{tabular}
\end{frame}


%
% Takeaway #5
%
\subsection*{Takeaway \#5}
\frameSubsectionTakeaway{}
\againframe<5>{takeaway}
}%
      \onslide*<3>{\providecommand\step{7}%
% Backtraces
%
\subsection{One advantage of raising with debugging information}
\frameSubsection{}{}

\begin{frame}{Backtraces}{Debugging information \& source code}
  \begin{columns}[c]
    \begin{column}{0.5\textwidth}
      \begin{itemize}
        \item Raising an exception is fun and games, but how do we know where the exception originated? $\rightarrow$ With the backtrace associated with the exception!
        \item In order to get a backtrace from the point where an exception was raised, it's necessary to compile with debugging information enabled (\funcarg{-g} flag for the compiler)
        \item Same example as in the `Nested handlers' section
      \end{itemize}
    \end{column}
    \begin{column}{0.5\textwidth}
      \listocaml[firstline=9,lastline=34]{../src/backtrace/backtrace.ml}{backtrace/backtrace.ml}
    \end{column}
  \end{columns}
\end{frame}

\begin{frame}{Backtraces}{CMM and disassembly of \funcname{definitely\_raise}}
  \begin{columns}[c]
    \begin{column}{0.5\textwidth}
      \minipage[c][0.66\textheight][s]{\columnwidth}
      \begin{itemize}
        \item<1-> An effect of compiling with debugging information (\funcarg{-g}): the CMM no longer uses \funcname{raise\_notrace} to raise the exception but \funcname{raise} instead
        \item<2-> Disassembly shows that CMM \funcname{raise} is actually a call to \funcname{caml\_raise\_exn}, a function in assembler provided by the runtime
      \end{itemize}
      \vfill
      \mintocaml[firstline=9,lastline=13,highlightlines=12]{../src/backtrace/backtrace.ml}
      \endminipage
    \end{column}
    \begin{column}{0.5\textwidth}
      \onslide*<1>{
        \mintcmm[highlightlines=5]{../src/backtrace/camlBacktrace\_\_definitely\_raise\_347.cmm}
      }
      \onslide*<2>{
        \mintobjdump[firstline=2,highlightlines=14]{../src/backtrace/camlBacktrace\_\_definitely\_raise\_347.objdump}
      }
    \end{column}
  \end{columns}
\end{frame}

\begin{frame}{Backtraces}{Introducing \funcname{caml\_raise\_exn}}
  \begin{columns}[c]
    \begin{column}{0.5\textwidth}
      \minipage[c][0.66\textheight][s]{\columnwidth}
      \onslide*<1>{
        \begin{itemize}
          \item Raising the exception is performed using \funcname{caml\_raise\_exn} which is
            \begin{itemize}
              \item An assembly function provided by the runtime
              \item Architecture specific
            \end{itemize}
          \item Let's see what it does differently from \funcname{raise\_notrace}\dots
          \item We are about to raise \typename{ExnC} with \funcname{caml\_raise\_exn} from \funcname{definitely\_raise}
        \end{itemize}
      }
      \onslide*<2>{
        \mintobjdump[firstline=2,highlightlines=14]{../src/backtrace/camlBacktrace\_\_definitely\_raise\_347.objdump}
      }
      \vfill
      \mintocaml[firstline=9,lastline=13,highlightlines=12]{../src/backtrace/backtrace.ml}
      \endminipage
    \end{column}
    \begin{column}{0.5\textwidth}
      \centering
      \providecommand\step{1}\input{diagrams/backtrace/backtrace.tex}
    \end{column}
  \end{columns}
\end{frame}

\begin{frame}{Backtraces}{But before that, we need more fields from \typename{caml\_domain\_state}}
  \begin{columns}
    \begin{column}{0.5\textwidth}
      \begin{itemize}
        \item Remember \typename{caml\_domain\_state} the per-domain runtime state?
        \item Some other members are of interest to us now\dots
        \item \localname{backtrace\_active} is used to know if the runtime should collect backtraces
        \item \localname{backtrace\_active} can be set by
          \begin{itemize}
            \item The \funcarg{b} flag in the \funcname{OCAMLRUNPARAM} environment variable
            \item \mintinline{ocaml}{Printexc.record_backtrace : bool -> unit} \footnotemark
          \end{itemize}
      \end{itemize}
    \end{column}
    \begin{column}{0.5\textwidth}
      \begin{listing}
        \mintc[highlightlines={9-11}]{src/ocaml/domain\_state\_backtrace.h}
        \caption{runtime/caml/domain\_state.h}
      \end{listing}
    \end{column}
  \end{columns}
  \footnotetext[1]{https://v2.ocaml.org/api/Printexc.html}
\end{frame}

\newcommand\listCamlRaiseExn[1][]{
  \listasm[firstline=702,lastline=725,#1]{../ocaml/runtime/amd64.S}{runtime/amd64.S:702}}

\begin{frame}{Backtraces}{Walk through \funcname{caml\_raise\_exn}}
  \begin{columns}[T]
    \begin{column}{0.5\textwidth}
      \begin{itemize}
        \item<1-> First checks whether exceptions backtrace should be recorded
        \item<1-> By checking the \funcarg{domain\_state->backtrace\_active} value
        \item<2-> If not, acts as \funcname{raise\_notrace} with \funcname{RESTORE\_EXN\_HANDLER\_OCAML}
          \begin{itemize}
            \item Set \regname{RSP} to \localname{domain\_state->exn\_handler} value
            \item Restore \localname{domain\_state->exn\_handler} from trap
            \item Jump with \funcname{ret} (same effect as using \funcname{pop} and \funcname{jmp})
          \end{itemize}
        \item<3-> Otherwise, switches to the C stack to be able to execute C code
        \item<4-> Calls \funcname{caml\_stash\_backtrace}, a C function provided by the runtime\ignorespaces
          \onslide<6->{, with
            \begin{itemize}
              \item \funcarg{exn}: exception value
              \item \funcarg{pc}: code address of the raising point
              \item \funcarg{sp}: stack address of the raising function
              \item \funcarg{trapsp}: stack address of the next handler
            \end{itemize}
          }
      \end{itemize}
      \bigskip
      \mintocaml[firstline=9,lastline=13,highlightlines=12]{../src/backtrace/backtrace.ml}
    \end{column}
    \begin{column}{0.5\textwidth}
      \raggedleft
      \onslide*<1,3-4>{
        \mintc[firstline=190,lastline=192]{../ocaml/runtime/amd64.S}
        \mintc[firstline=234,lastline=237]{../ocaml/runtime/amd64.S}
      }
      \onslide*<2>{
        \mintc[firstline=190,lastline=192,highlightlines={191-192}]{../ocaml/runtime/amd64.S}
        \mintc[firstline=234,lastline=237,highlightlines={235-237}]{../ocaml/runtime/amd64.S}
      }
      \onslide*<1>{\listCamlRaiseExn[highlightlines=705]{}}%
      \onslide*<2>{\listCamlRaiseExn[highlightlines={707-708}]{}}%
      \onslide*<3>{\listCamlRaiseExn[highlightlines={712-714}]{}}%
      \onslide*<4>{\listCamlRaiseExn[highlightlines={715-720}]{}}%
      \onslide*<5>{\providecommand\step{1}\input{diagrams/backtrace/backtrace.tex}}%
      \onslide*<6>{\providecommand\step{2}\input{diagrams/backtrace/backtrace.tex}}%
    \end{column}
  \end{columns}
\end{frame}

\newcommand\listStashBacktrace[1][]{
  \listc[firstline=91,lastline=120,#1]{../ocaml/runtime/backtrace\_nat.c}{runtime/backtrace\_nat.c:91}}

\begin{frame}{Backtraces}{Introducing \funcname{caml\_stash\_backtrace}}
  \begin{columns}[c]
    \begin{column}{0.5\textwidth}
      \minipage[c][0.66\textheight][s]{\columnwidth}
      \begin{itemize}
        \item<1-> A C function provided by the runtime (specific to native code)
        \item<2-> It scans the stack, between the stack pointer at the raising point up to the next trap, in order to collect the names of the detected function frames
          \onslide*<2>{
            \begin{itemize}
              \item And more information, like line number, column number...
            \end{itemize}
          }
        \item<3-> It uses the same technique and information as the Garbage Collector (GC) uses to scan the values live on the stack
          \onslide*<4>{
            \begin{itemize}
              \item \typename{frame\_descr} are at the core of stack unwinding
            \end{itemize}
          }
      \end{itemize}
      \vfill
      \mintocaml[firstline=9,lastline=13,highlightlines=12]{../src/backtrace/backtrace.ml}
      \endminipage
    \end{column}
    \begin{column}{0.5\textwidth}
      \onslide*<1>{\listStashBacktrace[highlightlines=91]{}}
      \onslide*<2>{\listStashBacktrace[highlightlines={91,117-118}]{}}
      \onslide*<3->{\listStashBacktrace[highlightlines={94,106,109-110}]{}}
    \end{column}
  \end{columns}
\end{frame}

\newcommand\listFrameDescr[1][]{
  \listocaml[firstline=111,lastline=124,#1]{../ocaml/asmcomp/emitaux.ml}{asmcomp/emitaux.ml:111}}
\newcommand\listDebugInfo[1][]{
  \listocaml[firstline=38,lastline=49,#1]{../ocaml/lambda/debuginfo.mli}{lambda/debuginfo.mli:38}}

\begin{frame}{Backtraces}{\typename{frame\_descr} and \typename{frame\_debuginfo} in a nutshell}
  \begin{columns}[c]
    \begin{column}{0.5\textwidth}
      \begin{itemize}
        \item<1-> For every return address in an OCaml program, there is a associated \typename{frame\_descr}
        \item<2-> A \typename{frame\_descr} specifies
        \begin{itemize}
          \item<2-> The size of the function stack frame
          \item<3-> Where live values are on the stack (so that the GC can mark them as live and prevent from being swept)
        \end{itemize}
        \item<4-> When compiled with debug information, \typename{frame\_descr} will be linked to a \typename{frame\_debuginfo}
        \begin{itemize}
          \item<5-> Contains information about the position in the source code (file name, line and column number)
        \end{itemize}
      \end{itemize}
    \end{column}
    \begin{column}{0.5\textwidth}
        \onslide*<1>{\listFrameDescr[highlightlines={118-122}]{}}
        \onslide*<2>{\listFrameDescr[highlightlines={120}]{}}
        \onslide*<3>{\listFrameDescr[highlightlines={121}]{}}
        \onslide*<4->{\listFrameDescr[highlightlines={122}]{}}
        \medskip
        \onslide*<1-3>{\listDebugInfo{}}
        \onslide*<4>{\listDebugInfo[highlightlines={38,49}]{}}
        \onslide*<5>{\listDebugInfo[highlightlines={39-46}]{}}
    \end{column}
  \end{columns}
\end{frame}

\begin{frame}{Backtraces}{Scanning the stack with \typename{frame\_descr}}
  \begin{columns}[c]
    \begin{column}{0.5\textwidth}
      \begin{itemize}
        \item Given the return address of the code that raises the exception by calling \funcname{caml\_raise\_exn} (\funcarg{pc} argument of \funcname{caml\_stash\_backtrace})
        \item And the position of the stack pointer (\funcarg{sp} argument of \funcname{caml\_stash\_backtrace})
        \item The frame size given by \typename{frame\_descr}.\funcarg{frame\_size}, allows to find the end of the previous frame
        \item And the return address of the caller of this function
        \bigskip
        \onslide<3->{
        \item Note that traps (here the reddish \funcname{ExnA}, \funcname{ExnB} and \funcname{ExnC} blocks) belong to the frame that installed them, and so the frame size changes when traps are removed
        }
      \end{itemize}
    \end{column}
    \begin{column}{0.5\textwidth}
      \raggedleft
      \onslide*<1>{\providecommand\step{2}\input{diagrams/backtrace/backtrace.tex}}%
      \onslide*<2>{\providecommand\step{1}\input{diagrams/backtrace/scanstack.tex}}%
      \onslide*<3>{\providecommand\step{2}\input{diagrams/backtrace/scanstack.tex}}%
      \onslide*<4>{\providecommand\step{3}\input{diagrams/backtrace/scanstack.tex}}%
      \onslide*<5>{\providecommand\step{4}\input{diagrams/backtrace/scanstack.tex}}%
      \onslide*<6>{\providecommand\step{5}\input{diagrams/backtrace/scanstack.tex}}%
    \end{column}
  \end{columns}
\end{frame}

% Duplicate of previous-previous slide
\begin{frame}{Backtraces}{Continuing on \funcname{caml\_stash\_backtrace}}
  \begin{columns}[c]
    \begin{column}{0.5\textwidth}
      \minipage[c][0.75\textheight][s]{\columnwidth}
      \begin{itemize}
          \color{gray}
        \item A C function provided by the runtime (specific to native code)
        \item It scans the stack, between the stack pointer at the raising point up to the next trap, in order to collect the names of the detected function frames
        \item It uses the same technique and information as the Garbage Collector (GC) uses to scan the values live on the stack
      \end{itemize}
      \begin{itemize}
        \item For each function frame detected, store a pointer to the \typename{frame\_descr} in the \localname{domain\_state->backtrace\_buffer} array, effectively constructing the backtrace
      \end{itemize}
      \onslide<2->{
        \begin{itemize}
            \smallskip
          \item Constructed backtrace:
            \funcname{definitely\_raise},
            \onslide<3->{\funcname{probably\_raise}}
        \end{itemize}
      }
      \vfill
      \mintocaml[firstline=9,lastline=13,highlightlines=12]{../src/backtrace/backtrace.ml}
      \endminipage
    \end{column}
    \begin{column}{0.5\textwidth}
      \raggedleft
      \onslide*<1>{\listStashBacktrace[highlightlines={114-115}]{}}
      \onslide*<2>{\providecommand\step{3}\input{diagrams/backtrace/backtrace.tex}}%
      \onslide*<3>{\providecommand\step{4}\input{diagrams/backtrace/backtrace.tex}}%
    \end{column}
  \end{columns}
\end{frame}

\begin{frame}{Backtraces}{Back to \funcname{caml\_raise\_exn}}
  \begin{columns}[c]
    \begin{column}{0.5\textwidth}
      \minipage[c][0.66\textheight][s]{\columnwidth}
      \begin{itemize}\color{gray}
        \item Checks wheteher exceptions backtrace should be recorded, by checking the \funcarg{domain\_state->backtrace\_active} value
        \item Switches to the C stack to be able to execute C code
        \item Calls \funcname{caml\_stash\_backtrace}, to collect the backtrace up to the next trap
      \end{itemize}
      \onslide*<2->{
        \begin{itemize}
          \item<2-> Executes the exception handler in \funcname{probably\_raise} with \funcname{RESTORE\_EXN\_HANDLER\_OCAML}
            \begin{itemize}
              \item Set \regname{RSP} to \localname{domain\_state->exn\_handler} value
              \item Restore \localname{domain\_state->exn\_handler} from trap
              \item Jump to the handler code with \funcname{ret}
            \end{itemize}
        \end{itemize}
      }
      \vfill
      \onslide*<1>{\mintocaml[firstline=9,lastline=13,highlightlines=12]{../src/backtrace/backtrace.ml}}
      \onslide*<2->{\mintocaml[firstline=15,lastline=21,highlightlines={19-21}]{../src/backtrace/backtrace.ml}}
      \endminipage
    \end{column}
    \begin{column}{0.5\textwidth}
      \centering
      \onslide*<1>{
        \mintc[firstline=190,lastline=192]{../ocaml/runtime/amd64.S}
        \mintc[firstline=234,lastline=237]{../ocaml/runtime/amd64.S}
      }
      \onslide*<2>{
        \mintc[firstline=190,lastline=192,highlightlines={191-192}]{../ocaml/runtime/amd64.S}
        \mintc[firstline=234,lastline=237,highlightlines={235-237}]{../ocaml/runtime/amd64.S}
      }
      \onslide*<1>{\listCamlRaiseExn[highlightlines=720]{}}
      \onslide*<2>{\listCamlRaiseExn[highlightlines=722]{}}
      \onslide*<3->{\providecommand\step{5}\input{diagrams/backtrace/backtrace.tex}}
    \end{column}
  \end{columns}
\end{frame}

\begin{frame}{Backtraces}{\funcname{caml\_probably\_raise}'s handler doesn't catch \typename{ExnC}}
  \begin{columns}[c]
    \begin{column}{0.45\textwidth}
      \minipage[c][0.75\textheight][s]{\columnwidth}
      \begin{itemize}
        \item<1-> \funcname{match \dots with}\footnotemark pattern matching can also match exceptions
        \item<1-> The CMM of \funcname{probably\_raise} shows that this \funcname{match \dots with} is implemented with a pattern matching and a \funcname{try \dots with}
        \item<1-> But this one only matches on \typename{ExnA} exception
        \item<2-> Since the pattern matching fails to catch the exception, \funcname{reraise} is called with our current \typename{ExnC} exception
        \item<3-> Disassembly shows that \funcname{reraise} is actually a call to \funcname{caml\_reraise\_exn}, which is also an assembly function provided by the runtime
      \end{itemize}
      \vfill
      \mintocaml[firstline=15,lastline=21,highlightlines={18-21}]{../src/backtrace/backtrace.ml}
      \endminipage
    \end{column}
    \begin{column}{0.55\textwidth}
      \centering
      \onslide*<1>{\mintcmm[highlightlines={10-18}]{../src/backtrace/camlBacktrace\_\_probably\_raise\_351.cmm}}
      \onslide*<2>{\listcmm[highlightlines=18]{../src/backtrace/camlBacktrace\_\_probably\_raise_351.cmm}{CMM of probably\_raise}}
      \onslide*<3>{\mintobjdump[firstline=5,lastline=35,highlightlines=31]{../src/backtrace/camlBacktrace\_\_probably\_raise_351.objdump}}
    \end{column}
  \end{columns}
  \only<1>{\footnotetext[1]{https://blog.janestreet.com/pattern-matching-and-exception-handling-unite/}}
\end{frame}

\newcommand\listCamlReraiseExn[1][]{
  \listasm[firstline=727,lastline=734,#1]{../ocaml/runtime/amd64.S}{runtime/amd64.S:727}}

\begin{frame}{Backtraces}{Introducing \funcname{caml\_reraise\_exn}}
  \begin{columns}[c]
    \begin{column}{0.5\textwidth}
      \minipage[c][0.66\textheight][s]{\columnwidth}
      \begin{itemize}
        \item<1-> The exception have to be reraised to the next handler
        \item<2-> First, checks if the backtrace should be collected with \funcarg{domain\_state->backtrace\_active}
        \only<3>{
        \begin{itemize}
            \item If not, behaves like a \funcname{raise\_notrace} with \funcname{RESTORE\_EXN\_HANDLER\_OCAML}
        \end{itemize}
        }
        \item<4-> Jumps into \funcname{caml\_raise\_exn}
        \item<5-> Calls \funcname{caml\_stash\_backtrace} again, to scan the next part of the stack: from the trap just pop-ed to the next trap
      \end{itemize}
      \vfill
      \mintocaml[firstline=15,lastline=21,highlightlines={18-21}]{../src/backtrace/backtrace.ml}
      \endminipage
    \end{column}
    \begin{column}{0.5\textwidth}
      \raggedleft
      \onslide*<1>{\listCamlReraiseExn{}}
      \onslide*<2>{\listCamlReraiseExn[highlightlines=729]{}}
      \onslide*<3>{
        \mintc[firstline=190,lastline=192,highlightlines={191-192}]{../ocaml/runtime/amd64.S}
        \mintc[firstline=234,lastline=237,highlightlines={235-237}]{../ocaml/runtime/amd64.S}
        \medskip
        \listCamlReraiseExn[highlightlines=731]{}
      }
      \onslide*<4-5>{
        % TODO refactor with \listCamlReraiseExn
        \mintasm[firstline=727,lastline=734,highlightlines=730]{../ocaml/runtime/amd64.S}
        \medskip
      }
      \onslide*<4>{\listCamlRaiseExn[highlightlines=711]{}}
      \onslide*<5>{\listCamlRaiseExn[highlightlines=720]{}}
    \end{column}
  \end{columns}
\end{frame}

\begin{frame}{Backtraces}{Backtrace collection continues}
  \begin{columns}[c]
    \begin{column}{0.55\textwidth}
      \minipage[c][0.75\textheight][s]{\columnwidth}
      \begin{itemize}
        \item<1-> \funcname{caml\_stash\_backtrace} scans the older part of the stack, up to the next trap
        \item<6-> Controls jumps to the nested exception handler in \funcname{maybe\_raise}, which matches only on \typename{ExnB}
        \item<7-> \funcname{caml\_reraise\_exn} is called again, backtrace collection resumes with \funcname{caml\_stash\_backtrace}
      \end{itemize}
      \medskip
      \begin{itemize}
        \item<2-> Constructed backtrace:
        \begin{itemize}
          \item <2-> \funcname{definitely\_raise}
          \item <2-> \funcname{probably\_raise}
          \item <3-> \funcname{probably\_raise} (re-raise)
          \item <4-> \funcname{maybe\_raise}
          \item <5-> \funcname{main}
          \item <8-> \funcname{main} (re-raise)
        \end{itemize}
      \end{itemize}
      \vfill
      \onslide*<1-5>{\mintocaml[firstline=15,lastline=21]{../src/backtrace/backtrace.ml}}
      \onslide*<6>{\mintocaml[firstline=28,lastline=34,highlightlines=31]{../src/backtrace/backtrace.ml}}
      \onslide*<7->{\mintocaml[firstline=28,lastline=34,highlightlines=33]{../src/backtrace/backtrace.ml}}
      \endminipage
    \end{column}
    \begin{column}{0.45\textwidth}
      \raggedleft
      \onslide*<1>{\providecommand\step{6}\input{diagrams/backtrace/backtrace.tex}}%
      \onslide*<2>{\providecommand\step{6}\input{diagrams/backtrace/backtrace.tex}}%
      \onslide*<3>{\providecommand\step{7}\input{diagrams/backtrace/backtrace.tex}}%
      \onslide*<4>{\providecommand\step{8}\input{diagrams/backtrace/backtrace.tex}}%
      \onslide*<5>{\providecommand\step{9}\input{diagrams/backtrace/backtrace.tex}}%
      \onslide*<6>{\providecommand\step{10}\input{diagrams/backtrace/backtrace.tex}}%
      \onslide*<7>{\providecommand\step{11}\input{diagrams/backtrace/backtrace.tex}}%
      \onslide*<8>{\providecommand\step{12}\input{diagrams/backtrace/backtrace.tex}}%
      \onslide*<9>{\providecommand\step{13}\input{diagrams/backtrace/backtrace.tex}}%
    \end{column}
  \end{columns}
\end{frame}

\begin{frame}{Backtraces}{Printing the backtrace}
  \begin{columns}[c]
    \begin{column}{0.45\textwidth}
      \minipage[c][0.66\textheight][s]{\columnwidth}
      \begin{itemize}
        \item<1-> The exception backtrace can now be printed to \funcarg{stdout} with \funcname{Printexc.print_backtrace}
        \item<2-> First the exception backtrace is retrieved into an OCaml array with \funcname{get_raw_backtrace }
        \item<3-> Which is implemented as the \funcname{caml_get_exception_raw_backtrace} C function in the runtime
        \item<4-> And finally printed with \funcname{print_exception_backtrace}
      \end{itemize}
      \vfill
      \mintocaml[firstline=28,lastline=34,highlightlines=33]{../src/backtrace/backtrace.ml}
      \endminipage
    \end{column}
    \begin{column}{0.55\textwidth}
      \onslide*<1,2>{
        \mintocaml[firstline=100,lastline=101]{../ocaml/stdlib/printexc.ml}
      }
      \only<1>{
        \listocaml[firstline=170,lastline=172]{../ocaml/stdlib/printexc.ml}{stdlib/printexc.ml:170}
      }
      \only<2>{
        \listocaml[firstline=170,lastline=172,highlightlines=172]{../ocaml/stdlib/printexc.ml}{stdlib/printexc.ml:170}
      }
      \only<3>{
        \mintc[firstline=154,lastline=156]{../ocaml/runtime/backtrace.c}
        \listc[firstline=171,lastline=192,highlightlines={184-187}]{../ocaml/runtime/backtrace.c}{runtime/backtrace.c:154}
      }
      \only<4>{
        \listocaml[firstline=155,lastline=165]{../ocaml/stdlib/printexc.ml}{stdlib/printexc.ml:55}
      }
    \end{column}
  \end{columns}
\end{frame}


\subsection{The multiple kinds of raise}
\frameSubsection{}{}

\begin{frame}{Backtraces}{When backtraces are not needed}
  \begin{columns}[c]
    \begin{column}{0.45\textwidth}
      \minipage[c][0.66\textheight][s]{\columnwidth}
      \begin{itemize}
        \item<1-> What if the backtrace for an exception is never needed?
        \item<2-> It's possible to raise the exception explicitly with \funcname{raise\_notrace} to make raising as cheap as possible
        \item<3-> An example of use in the \funcname{dynlink} library of the OCaml compiler
      \end{itemize}
      \vfill
      \onslide<3->{
      \listocaml[firstline=205,lastline=209,highlightlines=207]{../ocaml/otherlibs/dynlink/dynlink\_compilerlibs/misc.ml}{otherlibs/dynlink/dynlink\_compilerlibs/misc.ml:205}
      }
      \endminipage
    \end{column}
    \begin{column}{0.55\textwidth}
    \only<4>{
      \mintcmm[highlightlines=3]{../src/notrace/camlNotrace\_\_fun_347.cmm}
      \listcmm{../src/notrace/camlNotrace\_\_all\_somes\_327.cmm}{all\_somes}
    }
    \onslide*<5->{
      \listobjdump[highlightlines={6-9}]{../src/notrace/camlNotrace\_\_fun_347.objdump}{anonymous function from \funcname{all\_somes}}
    }
    \end{column}
  \end{columns}
\end{frame}

\begin{frame}{Backtraces}{Summary table of the multiple kinds of raise}
  \begin{itemize}
    \item When debugging information is enabled and if the backtrace of an exception isn't needed, it's possible to raise with \funcname{raise\_notrace} for performance
    \item When debugging information is \emph{not} enabled, the compiler optimises \funcname{raise} calls to inline assembly (same effect as using \funcname{raise\_notrace})
  \end{itemize}
  \bigskip
  \centering
  \begin{tabular}{| l | c | c | c | c |}
    \hline
    \parbox{10em}{Debug information \\ (\funcarg{-g} flag)}  & \multicolumn{2}{|c|}{Disabled}                                     & \multicolumn{2}{|c|}{Enabled} \\
    \hline
    Raise kind                                               & \funcname{raise} & \funcname{raise\_notrace}                       & \funcname{raise} & \funcname{raise\_notrace} \\
    \hline
    Compilation                                              & \parbox{8em}{inline assembly\\ (optimization)} & inline assembly   & \funcname{caml\_raise\_exn} & inline assembly \\
    \hline
  \end{tabular}
\end{frame}


%
% Takeaway #5
%
\subsection*{Takeaway \#5}
\frameSubsectionTakeaway{}
\againframe<5>{takeaway}
}%
      \onslide*<4>{\providecommand\step{8}%
% Backtraces
%
\subsection{One advantage of raising with debugging information}
\frameSubsection{}{}

\begin{frame}{Backtraces}{Debugging information \& source code}
  \begin{columns}[c]
    \begin{column}{0.5\textwidth}
      \begin{itemize}
        \item Raising an exception is fun and games, but how do we know where the exception originated? $\rightarrow$ With the backtrace associated with the exception!
        \item In order to get a backtrace from the point where an exception was raised, it's necessary to compile with debugging information enabled (\funcarg{-g} flag for the compiler)
        \item Same example as in the `Nested handlers' section
      \end{itemize}
    \end{column}
    \begin{column}{0.5\textwidth}
      \listocaml[firstline=9,lastline=34]{../src/backtrace/backtrace.ml}{backtrace/backtrace.ml}
    \end{column}
  \end{columns}
\end{frame}

\begin{frame}{Backtraces}{CMM and disassembly of \funcname{definitely\_raise}}
  \begin{columns}[c]
    \begin{column}{0.5\textwidth}
      \minipage[c][0.66\textheight][s]{\columnwidth}
      \begin{itemize}
        \item<1-> An effect of compiling with debugging information (\funcarg{-g}): the CMM no longer uses \funcname{raise\_notrace} to raise the exception but \funcname{raise} instead
        \item<2-> Disassembly shows that CMM \funcname{raise} is actually a call to \funcname{caml\_raise\_exn}, a function in assembler provided by the runtime
      \end{itemize}
      \vfill
      \mintocaml[firstline=9,lastline=13,highlightlines=12]{../src/backtrace/backtrace.ml}
      \endminipage
    \end{column}
    \begin{column}{0.5\textwidth}
      \onslide*<1>{
        \mintcmm[highlightlines=5]{../src/backtrace/camlBacktrace\_\_definitely\_raise\_347.cmm}
      }
      \onslide*<2>{
        \mintobjdump[firstline=2,highlightlines=14]{../src/backtrace/camlBacktrace\_\_definitely\_raise\_347.objdump}
      }
    \end{column}
  \end{columns}
\end{frame}

\begin{frame}{Backtraces}{Introducing \funcname{caml\_raise\_exn}}
  \begin{columns}[c]
    \begin{column}{0.5\textwidth}
      \minipage[c][0.66\textheight][s]{\columnwidth}
      \onslide*<1>{
        \begin{itemize}
          \item Raising the exception is performed using \funcname{caml\_raise\_exn} which is
            \begin{itemize}
              \item An assembly function provided by the runtime
              \item Architecture specific
            \end{itemize}
          \item Let's see what it does differently from \funcname{raise\_notrace}\dots
          \item We are about to raise \typename{ExnC} with \funcname{caml\_raise\_exn} from \funcname{definitely\_raise}
        \end{itemize}
      }
      \onslide*<2>{
        \mintobjdump[firstline=2,highlightlines=14]{../src/backtrace/camlBacktrace\_\_definitely\_raise\_347.objdump}
      }
      \vfill
      \mintocaml[firstline=9,lastline=13,highlightlines=12]{../src/backtrace/backtrace.ml}
      \endminipage
    \end{column}
    \begin{column}{0.5\textwidth}
      \centering
      \providecommand\step{1}\input{diagrams/backtrace/backtrace.tex}
    \end{column}
  \end{columns}
\end{frame}

\begin{frame}{Backtraces}{But before that, we need more fields from \typename{caml\_domain\_state}}
  \begin{columns}
    \begin{column}{0.5\textwidth}
      \begin{itemize}
        \item Remember \typename{caml\_domain\_state} the per-domain runtime state?
        \item Some other members are of interest to us now\dots
        \item \localname{backtrace\_active} is used to know if the runtime should collect backtraces
        \item \localname{backtrace\_active} can be set by
          \begin{itemize}
            \item The \funcarg{b} flag in the \funcname{OCAMLRUNPARAM} environment variable
            \item \mintinline{ocaml}{Printexc.record_backtrace : bool -> unit} \footnotemark
          \end{itemize}
      \end{itemize}
    \end{column}
    \begin{column}{0.5\textwidth}
      \begin{listing}
        \mintc[highlightlines={9-11}]{src/ocaml/domain\_state\_backtrace.h}
        \caption{runtime/caml/domain\_state.h}
      \end{listing}
    \end{column}
  \end{columns}
  \footnotetext[1]{https://v2.ocaml.org/api/Printexc.html}
\end{frame}

\newcommand\listCamlRaiseExn[1][]{
  \listasm[firstline=702,lastline=725,#1]{../ocaml/runtime/amd64.S}{runtime/amd64.S:702}}

\begin{frame}{Backtraces}{Walk through \funcname{caml\_raise\_exn}}
  \begin{columns}[T]
    \begin{column}{0.5\textwidth}
      \begin{itemize}
        \item<1-> First checks whether exceptions backtrace should be recorded
        \item<1-> By checking the \funcarg{domain\_state->backtrace\_active} value
        \item<2-> If not, acts as \funcname{raise\_notrace} with \funcname{RESTORE\_EXN\_HANDLER\_OCAML}
          \begin{itemize}
            \item Set \regname{RSP} to \localname{domain\_state->exn\_handler} value
            \item Restore \localname{domain\_state->exn\_handler} from trap
            \item Jump with \funcname{ret} (same effect as using \funcname{pop} and \funcname{jmp})
          \end{itemize}
        \item<3-> Otherwise, switches to the C stack to be able to execute C code
        \item<4-> Calls \funcname{caml\_stash\_backtrace}, a C function provided by the runtime\ignorespaces
          \onslide<6->{, with
            \begin{itemize}
              \item \funcarg{exn}: exception value
              \item \funcarg{pc}: code address of the raising point
              \item \funcarg{sp}: stack address of the raising function
              \item \funcarg{trapsp}: stack address of the next handler
            \end{itemize}
          }
      \end{itemize}
      \bigskip
      \mintocaml[firstline=9,lastline=13,highlightlines=12]{../src/backtrace/backtrace.ml}
    \end{column}
    \begin{column}{0.5\textwidth}
      \raggedleft
      \onslide*<1,3-4>{
        \mintc[firstline=190,lastline=192]{../ocaml/runtime/amd64.S}
        \mintc[firstline=234,lastline=237]{../ocaml/runtime/amd64.S}
      }
      \onslide*<2>{
        \mintc[firstline=190,lastline=192,highlightlines={191-192}]{../ocaml/runtime/amd64.S}
        \mintc[firstline=234,lastline=237,highlightlines={235-237}]{../ocaml/runtime/amd64.S}
      }
      \onslide*<1>{\listCamlRaiseExn[highlightlines=705]{}}%
      \onslide*<2>{\listCamlRaiseExn[highlightlines={707-708}]{}}%
      \onslide*<3>{\listCamlRaiseExn[highlightlines={712-714}]{}}%
      \onslide*<4>{\listCamlRaiseExn[highlightlines={715-720}]{}}%
      \onslide*<5>{\providecommand\step{1}\input{diagrams/backtrace/backtrace.tex}}%
      \onslide*<6>{\providecommand\step{2}\input{diagrams/backtrace/backtrace.tex}}%
    \end{column}
  \end{columns}
\end{frame}

\newcommand\listStashBacktrace[1][]{
  \listc[firstline=91,lastline=120,#1]{../ocaml/runtime/backtrace\_nat.c}{runtime/backtrace\_nat.c:91}}

\begin{frame}{Backtraces}{Introducing \funcname{caml\_stash\_backtrace}}
  \begin{columns}[c]
    \begin{column}{0.5\textwidth}
      \minipage[c][0.66\textheight][s]{\columnwidth}
      \begin{itemize}
        \item<1-> A C function provided by the runtime (specific to native code)
        \item<2-> It scans the stack, between the stack pointer at the raising point up to the next trap, in order to collect the names of the detected function frames
          \onslide*<2>{
            \begin{itemize}
              \item And more information, like line number, column number...
            \end{itemize}
          }
        \item<3-> It uses the same technique and information as the Garbage Collector (GC) uses to scan the values live on the stack
          \onslide*<4>{
            \begin{itemize}
              \item \typename{frame\_descr} are at the core of stack unwinding
            \end{itemize}
          }
      \end{itemize}
      \vfill
      \mintocaml[firstline=9,lastline=13,highlightlines=12]{../src/backtrace/backtrace.ml}
      \endminipage
    \end{column}
    \begin{column}{0.5\textwidth}
      \onslide*<1>{\listStashBacktrace[highlightlines=91]{}}
      \onslide*<2>{\listStashBacktrace[highlightlines={91,117-118}]{}}
      \onslide*<3->{\listStashBacktrace[highlightlines={94,106,109-110}]{}}
    \end{column}
  \end{columns}
\end{frame}

\newcommand\listFrameDescr[1][]{
  \listocaml[firstline=111,lastline=124,#1]{../ocaml/asmcomp/emitaux.ml}{asmcomp/emitaux.ml:111}}
\newcommand\listDebugInfo[1][]{
  \listocaml[firstline=38,lastline=49,#1]{../ocaml/lambda/debuginfo.mli}{lambda/debuginfo.mli:38}}

\begin{frame}{Backtraces}{\typename{frame\_descr} and \typename{frame\_debuginfo} in a nutshell}
  \begin{columns}[c]
    \begin{column}{0.5\textwidth}
      \begin{itemize}
        \item<1-> For every return address in an OCaml program, there is a associated \typename{frame\_descr}
        \item<2-> A \typename{frame\_descr} specifies
        \begin{itemize}
          \item<2-> The size of the function stack frame
          \item<3-> Where live values are on the stack (so that the GC can mark them as live and prevent from being swept)
        \end{itemize}
        \item<4-> When compiled with debug information, \typename{frame\_descr} will be linked to a \typename{frame\_debuginfo}
        \begin{itemize}
          \item<5-> Contains information about the position in the source code (file name, line and column number)
        \end{itemize}
      \end{itemize}
    \end{column}
    \begin{column}{0.5\textwidth}
        \onslide*<1>{\listFrameDescr[highlightlines={118-122}]{}}
        \onslide*<2>{\listFrameDescr[highlightlines={120}]{}}
        \onslide*<3>{\listFrameDescr[highlightlines={121}]{}}
        \onslide*<4->{\listFrameDescr[highlightlines={122}]{}}
        \medskip
        \onslide*<1-3>{\listDebugInfo{}}
        \onslide*<4>{\listDebugInfo[highlightlines={38,49}]{}}
        \onslide*<5>{\listDebugInfo[highlightlines={39-46}]{}}
    \end{column}
  \end{columns}
\end{frame}

\begin{frame}{Backtraces}{Scanning the stack with \typename{frame\_descr}}
  \begin{columns}[c]
    \begin{column}{0.5\textwidth}
      \begin{itemize}
        \item Given the return address of the code that raises the exception by calling \funcname{caml\_raise\_exn} (\funcarg{pc} argument of \funcname{caml\_stash\_backtrace})
        \item And the position of the stack pointer (\funcarg{sp} argument of \funcname{caml\_stash\_backtrace})
        \item The frame size given by \typename{frame\_descr}.\funcarg{frame\_size}, allows to find the end of the previous frame
        \item And the return address of the caller of this function
        \bigskip
        \onslide<3->{
        \item Note that traps (here the reddish \funcname{ExnA}, \funcname{ExnB} and \funcname{ExnC} blocks) belong to the frame that installed them, and so the frame size changes when traps are removed
        }
      \end{itemize}
    \end{column}
    \begin{column}{0.5\textwidth}
      \raggedleft
      \onslide*<1>{\providecommand\step{2}\input{diagrams/backtrace/backtrace.tex}}%
      \onslide*<2>{\providecommand\step{1}\input{diagrams/backtrace/scanstack.tex}}%
      \onslide*<3>{\providecommand\step{2}\input{diagrams/backtrace/scanstack.tex}}%
      \onslide*<4>{\providecommand\step{3}\input{diagrams/backtrace/scanstack.tex}}%
      \onslide*<5>{\providecommand\step{4}\input{diagrams/backtrace/scanstack.tex}}%
      \onslide*<6>{\providecommand\step{5}\input{diagrams/backtrace/scanstack.tex}}%
    \end{column}
  \end{columns}
\end{frame}

% Duplicate of previous-previous slide
\begin{frame}{Backtraces}{Continuing on \funcname{caml\_stash\_backtrace}}
  \begin{columns}[c]
    \begin{column}{0.5\textwidth}
      \minipage[c][0.75\textheight][s]{\columnwidth}
      \begin{itemize}
          \color{gray}
        \item A C function provided by the runtime (specific to native code)
        \item It scans the stack, between the stack pointer at the raising point up to the next trap, in order to collect the names of the detected function frames
        \item It uses the same technique and information as the Garbage Collector (GC) uses to scan the values live on the stack
      \end{itemize}
      \begin{itemize}
        \item For each function frame detected, store a pointer to the \typename{frame\_descr} in the \localname{domain\_state->backtrace\_buffer} array, effectively constructing the backtrace
      \end{itemize}
      \onslide<2->{
        \begin{itemize}
            \smallskip
          \item Constructed backtrace:
            \funcname{definitely\_raise},
            \onslide<3->{\funcname{probably\_raise}}
        \end{itemize}
      }
      \vfill
      \mintocaml[firstline=9,lastline=13,highlightlines=12]{../src/backtrace/backtrace.ml}
      \endminipage
    \end{column}
    \begin{column}{0.5\textwidth}
      \raggedleft
      \onslide*<1>{\listStashBacktrace[highlightlines={114-115}]{}}
      \onslide*<2>{\providecommand\step{3}\input{diagrams/backtrace/backtrace.tex}}%
      \onslide*<3>{\providecommand\step{4}\input{diagrams/backtrace/backtrace.tex}}%
    \end{column}
  \end{columns}
\end{frame}

\begin{frame}{Backtraces}{Back to \funcname{caml\_raise\_exn}}
  \begin{columns}[c]
    \begin{column}{0.5\textwidth}
      \minipage[c][0.66\textheight][s]{\columnwidth}
      \begin{itemize}\color{gray}
        \item Checks wheteher exceptions backtrace should be recorded, by checking the \funcarg{domain\_state->backtrace\_active} value
        \item Switches to the C stack to be able to execute C code
        \item Calls \funcname{caml\_stash\_backtrace}, to collect the backtrace up to the next trap
      \end{itemize}
      \onslide*<2->{
        \begin{itemize}
          \item<2-> Executes the exception handler in \funcname{probably\_raise} with \funcname{RESTORE\_EXN\_HANDLER\_OCAML}
            \begin{itemize}
              \item Set \regname{RSP} to \localname{domain\_state->exn\_handler} value
              \item Restore \localname{domain\_state->exn\_handler} from trap
              \item Jump to the handler code with \funcname{ret}
            \end{itemize}
        \end{itemize}
      }
      \vfill
      \onslide*<1>{\mintocaml[firstline=9,lastline=13,highlightlines=12]{../src/backtrace/backtrace.ml}}
      \onslide*<2->{\mintocaml[firstline=15,lastline=21,highlightlines={19-21}]{../src/backtrace/backtrace.ml}}
      \endminipage
    \end{column}
    \begin{column}{0.5\textwidth}
      \centering
      \onslide*<1>{
        \mintc[firstline=190,lastline=192]{../ocaml/runtime/amd64.S}
        \mintc[firstline=234,lastline=237]{../ocaml/runtime/amd64.S}
      }
      \onslide*<2>{
        \mintc[firstline=190,lastline=192,highlightlines={191-192}]{../ocaml/runtime/amd64.S}
        \mintc[firstline=234,lastline=237,highlightlines={235-237}]{../ocaml/runtime/amd64.S}
      }
      \onslide*<1>{\listCamlRaiseExn[highlightlines=720]{}}
      \onslide*<2>{\listCamlRaiseExn[highlightlines=722]{}}
      \onslide*<3->{\providecommand\step{5}\input{diagrams/backtrace/backtrace.tex}}
    \end{column}
  \end{columns}
\end{frame}

\begin{frame}{Backtraces}{\funcname{caml\_probably\_raise}'s handler doesn't catch \typename{ExnC}}
  \begin{columns}[c]
    \begin{column}{0.45\textwidth}
      \minipage[c][0.75\textheight][s]{\columnwidth}
      \begin{itemize}
        \item<1-> \funcname{match \dots with}\footnotemark pattern matching can also match exceptions
        \item<1-> The CMM of \funcname{probably\_raise} shows that this \funcname{match \dots with} is implemented with a pattern matching and a \funcname{try \dots with}
        \item<1-> But this one only matches on \typename{ExnA} exception
        \item<2-> Since the pattern matching fails to catch the exception, \funcname{reraise} is called with our current \typename{ExnC} exception
        \item<3-> Disassembly shows that \funcname{reraise} is actually a call to \funcname{caml\_reraise\_exn}, which is also an assembly function provided by the runtime
      \end{itemize}
      \vfill
      \mintocaml[firstline=15,lastline=21,highlightlines={18-21}]{../src/backtrace/backtrace.ml}
      \endminipage
    \end{column}
    \begin{column}{0.55\textwidth}
      \centering
      \onslide*<1>{\mintcmm[highlightlines={10-18}]{../src/backtrace/camlBacktrace\_\_probably\_raise\_351.cmm}}
      \onslide*<2>{\listcmm[highlightlines=18]{../src/backtrace/camlBacktrace\_\_probably\_raise_351.cmm}{CMM of probably\_raise}}
      \onslide*<3>{\mintobjdump[firstline=5,lastline=35,highlightlines=31]{../src/backtrace/camlBacktrace\_\_probably\_raise_351.objdump}}
    \end{column}
  \end{columns}
  \only<1>{\footnotetext[1]{https://blog.janestreet.com/pattern-matching-and-exception-handling-unite/}}
\end{frame}

\newcommand\listCamlReraiseExn[1][]{
  \listasm[firstline=727,lastline=734,#1]{../ocaml/runtime/amd64.S}{runtime/amd64.S:727}}

\begin{frame}{Backtraces}{Introducing \funcname{caml\_reraise\_exn}}
  \begin{columns}[c]
    \begin{column}{0.5\textwidth}
      \minipage[c][0.66\textheight][s]{\columnwidth}
      \begin{itemize}
        \item<1-> The exception have to be reraised to the next handler
        \item<2-> First, checks if the backtrace should be collected with \funcarg{domain\_state->backtrace\_active}
        \only<3>{
        \begin{itemize}
            \item If not, behaves like a \funcname{raise\_notrace} with \funcname{RESTORE\_EXN\_HANDLER\_OCAML}
        \end{itemize}
        }
        \item<4-> Jumps into \funcname{caml\_raise\_exn}
        \item<5-> Calls \funcname{caml\_stash\_backtrace} again, to scan the next part of the stack: from the trap just pop-ed to the next trap
      \end{itemize}
      \vfill
      \mintocaml[firstline=15,lastline=21,highlightlines={18-21}]{../src/backtrace/backtrace.ml}
      \endminipage
    \end{column}
    \begin{column}{0.5\textwidth}
      \raggedleft
      \onslide*<1>{\listCamlReraiseExn{}}
      \onslide*<2>{\listCamlReraiseExn[highlightlines=729]{}}
      \onslide*<3>{
        \mintc[firstline=190,lastline=192,highlightlines={191-192}]{../ocaml/runtime/amd64.S}
        \mintc[firstline=234,lastline=237,highlightlines={235-237}]{../ocaml/runtime/amd64.S}
        \medskip
        \listCamlReraiseExn[highlightlines=731]{}
      }
      \onslide*<4-5>{
        % TODO refactor with \listCamlReraiseExn
        \mintasm[firstline=727,lastline=734,highlightlines=730]{../ocaml/runtime/amd64.S}
        \medskip
      }
      \onslide*<4>{\listCamlRaiseExn[highlightlines=711]{}}
      \onslide*<5>{\listCamlRaiseExn[highlightlines=720]{}}
    \end{column}
  \end{columns}
\end{frame}

\begin{frame}{Backtraces}{Backtrace collection continues}
  \begin{columns}[c]
    \begin{column}{0.55\textwidth}
      \minipage[c][0.75\textheight][s]{\columnwidth}
      \begin{itemize}
        \item<1-> \funcname{caml\_stash\_backtrace} scans the older part of the stack, up to the next trap
        \item<6-> Controls jumps to the nested exception handler in \funcname{maybe\_raise}, which matches only on \typename{ExnB}
        \item<7-> \funcname{caml\_reraise\_exn} is called again, backtrace collection resumes with \funcname{caml\_stash\_backtrace}
      \end{itemize}
      \medskip
      \begin{itemize}
        \item<2-> Constructed backtrace:
        \begin{itemize}
          \item <2-> \funcname{definitely\_raise}
          \item <2-> \funcname{probably\_raise}
          \item <3-> \funcname{probably\_raise} (re-raise)
          \item <4-> \funcname{maybe\_raise}
          \item <5-> \funcname{main}
          \item <8-> \funcname{main} (re-raise)
        \end{itemize}
      \end{itemize}
      \vfill
      \onslide*<1-5>{\mintocaml[firstline=15,lastline=21]{../src/backtrace/backtrace.ml}}
      \onslide*<6>{\mintocaml[firstline=28,lastline=34,highlightlines=31]{../src/backtrace/backtrace.ml}}
      \onslide*<7->{\mintocaml[firstline=28,lastline=34,highlightlines=33]{../src/backtrace/backtrace.ml}}
      \endminipage
    \end{column}
    \begin{column}{0.45\textwidth}
      \raggedleft
      \onslide*<1>{\providecommand\step{6}\input{diagrams/backtrace/backtrace.tex}}%
      \onslide*<2>{\providecommand\step{6}\input{diagrams/backtrace/backtrace.tex}}%
      \onslide*<3>{\providecommand\step{7}\input{diagrams/backtrace/backtrace.tex}}%
      \onslide*<4>{\providecommand\step{8}\input{diagrams/backtrace/backtrace.tex}}%
      \onslide*<5>{\providecommand\step{9}\input{diagrams/backtrace/backtrace.tex}}%
      \onslide*<6>{\providecommand\step{10}\input{diagrams/backtrace/backtrace.tex}}%
      \onslide*<7>{\providecommand\step{11}\input{diagrams/backtrace/backtrace.tex}}%
      \onslide*<8>{\providecommand\step{12}\input{diagrams/backtrace/backtrace.tex}}%
      \onslide*<9>{\providecommand\step{13}\input{diagrams/backtrace/backtrace.tex}}%
    \end{column}
  \end{columns}
\end{frame}

\begin{frame}{Backtraces}{Printing the backtrace}
  \begin{columns}[c]
    \begin{column}{0.45\textwidth}
      \minipage[c][0.66\textheight][s]{\columnwidth}
      \begin{itemize}
        \item<1-> The exception backtrace can now be printed to \funcarg{stdout} with \funcname{Printexc.print_backtrace}
        \item<2-> First the exception backtrace is retrieved into an OCaml array with \funcname{get_raw_backtrace }
        \item<3-> Which is implemented as the \funcname{caml_get_exception_raw_backtrace} C function in the runtime
        \item<4-> And finally printed with \funcname{print_exception_backtrace}
      \end{itemize}
      \vfill
      \mintocaml[firstline=28,lastline=34,highlightlines=33]{../src/backtrace/backtrace.ml}
      \endminipage
    \end{column}
    \begin{column}{0.55\textwidth}
      \onslide*<1,2>{
        \mintocaml[firstline=100,lastline=101]{../ocaml/stdlib/printexc.ml}
      }
      \only<1>{
        \listocaml[firstline=170,lastline=172]{../ocaml/stdlib/printexc.ml}{stdlib/printexc.ml:170}
      }
      \only<2>{
        \listocaml[firstline=170,lastline=172,highlightlines=172]{../ocaml/stdlib/printexc.ml}{stdlib/printexc.ml:170}
      }
      \only<3>{
        \mintc[firstline=154,lastline=156]{../ocaml/runtime/backtrace.c}
        \listc[firstline=171,lastline=192,highlightlines={184-187}]{../ocaml/runtime/backtrace.c}{runtime/backtrace.c:154}
      }
      \only<4>{
        \listocaml[firstline=155,lastline=165]{../ocaml/stdlib/printexc.ml}{stdlib/printexc.ml:55}
      }
    \end{column}
  \end{columns}
\end{frame}


\subsection{The multiple kinds of raise}
\frameSubsection{}{}

\begin{frame}{Backtraces}{When backtraces are not needed}
  \begin{columns}[c]
    \begin{column}{0.45\textwidth}
      \minipage[c][0.66\textheight][s]{\columnwidth}
      \begin{itemize}
        \item<1-> What if the backtrace for an exception is never needed?
        \item<2-> It's possible to raise the exception explicitly with \funcname{raise\_notrace} to make raising as cheap as possible
        \item<3-> An example of use in the \funcname{dynlink} library of the OCaml compiler
      \end{itemize}
      \vfill
      \onslide<3->{
      \listocaml[firstline=205,lastline=209,highlightlines=207]{../ocaml/otherlibs/dynlink/dynlink\_compilerlibs/misc.ml}{otherlibs/dynlink/dynlink\_compilerlibs/misc.ml:205}
      }
      \endminipage
    \end{column}
    \begin{column}{0.55\textwidth}
    \only<4>{
      \mintcmm[highlightlines=3]{../src/notrace/camlNotrace\_\_fun_347.cmm}
      \listcmm{../src/notrace/camlNotrace\_\_all\_somes\_327.cmm}{all\_somes}
    }
    \onslide*<5->{
      \listobjdump[highlightlines={6-9}]{../src/notrace/camlNotrace\_\_fun_347.objdump}{anonymous function from \funcname{all\_somes}}
    }
    \end{column}
  \end{columns}
\end{frame}

\begin{frame}{Backtraces}{Summary table of the multiple kinds of raise}
  \begin{itemize}
    \item When debugging information is enabled and if the backtrace of an exception isn't needed, it's possible to raise with \funcname{raise\_notrace} for performance
    \item When debugging information is \emph{not} enabled, the compiler optimises \funcname{raise} calls to inline assembly (same effect as using \funcname{raise\_notrace})
  \end{itemize}
  \bigskip
  \centering
  \begin{tabular}{| l | c | c | c | c |}
    \hline
    \parbox{10em}{Debug information \\ (\funcarg{-g} flag)}  & \multicolumn{2}{|c|}{Disabled}                                     & \multicolumn{2}{|c|}{Enabled} \\
    \hline
    Raise kind                                               & \funcname{raise} & \funcname{raise\_notrace}                       & \funcname{raise} & \funcname{raise\_notrace} \\
    \hline
    Compilation                                              & \parbox{8em}{inline assembly\\ (optimization)} & inline assembly   & \funcname{caml\_raise\_exn} & inline assembly \\
    \hline
  \end{tabular}
\end{frame}


%
% Takeaway #5
%
\subsection*{Takeaway \#5}
\frameSubsectionTakeaway{}
\againframe<5>{takeaway}
}%
      \onslide*<5>{\providecommand\step{9}%
% Backtraces
%
\subsection{One advantage of raising with debugging information}
\frameSubsection{}{}

\begin{frame}{Backtraces}{Debugging information \& source code}
  \begin{columns}[c]
    \begin{column}{0.5\textwidth}
      \begin{itemize}
        \item Raising an exception is fun and games, but how do we know where the exception originated? $\rightarrow$ With the backtrace associated with the exception!
        \item In order to get a backtrace from the point where an exception was raised, it's necessary to compile with debugging information enabled (\funcarg{-g} flag for the compiler)
        \item Same example as in the `Nested handlers' section
      \end{itemize}
    \end{column}
    \begin{column}{0.5\textwidth}
      \listocaml[firstline=9,lastline=34]{../src/backtrace/backtrace.ml}{backtrace/backtrace.ml}
    \end{column}
  \end{columns}
\end{frame}

\begin{frame}{Backtraces}{CMM and disassembly of \funcname{definitely\_raise}}
  \begin{columns}[c]
    \begin{column}{0.5\textwidth}
      \minipage[c][0.66\textheight][s]{\columnwidth}
      \begin{itemize}
        \item<1-> An effect of compiling with debugging information (\funcarg{-g}): the CMM no longer uses \funcname{raise\_notrace} to raise the exception but \funcname{raise} instead
        \item<2-> Disassembly shows that CMM \funcname{raise} is actually a call to \funcname{caml\_raise\_exn}, a function in assembler provided by the runtime
      \end{itemize}
      \vfill
      \mintocaml[firstline=9,lastline=13,highlightlines=12]{../src/backtrace/backtrace.ml}
      \endminipage
    \end{column}
    \begin{column}{0.5\textwidth}
      \onslide*<1>{
        \mintcmm[highlightlines=5]{../src/backtrace/camlBacktrace\_\_definitely\_raise\_347.cmm}
      }
      \onslide*<2>{
        \mintobjdump[firstline=2,highlightlines=14]{../src/backtrace/camlBacktrace\_\_definitely\_raise\_347.objdump}
      }
    \end{column}
  \end{columns}
\end{frame}

\begin{frame}{Backtraces}{Introducing \funcname{caml\_raise\_exn}}
  \begin{columns}[c]
    \begin{column}{0.5\textwidth}
      \minipage[c][0.66\textheight][s]{\columnwidth}
      \onslide*<1>{
        \begin{itemize}
          \item Raising the exception is performed using \funcname{caml\_raise\_exn} which is
            \begin{itemize}
              \item An assembly function provided by the runtime
              \item Architecture specific
            \end{itemize}
          \item Let's see what it does differently from \funcname{raise\_notrace}\dots
          \item We are about to raise \typename{ExnC} with \funcname{caml\_raise\_exn} from \funcname{definitely\_raise}
        \end{itemize}
      }
      \onslide*<2>{
        \mintobjdump[firstline=2,highlightlines=14]{../src/backtrace/camlBacktrace\_\_definitely\_raise\_347.objdump}
      }
      \vfill
      \mintocaml[firstline=9,lastline=13,highlightlines=12]{../src/backtrace/backtrace.ml}
      \endminipage
    \end{column}
    \begin{column}{0.5\textwidth}
      \centering
      \providecommand\step{1}\input{diagrams/backtrace/backtrace.tex}
    \end{column}
  \end{columns}
\end{frame}

\begin{frame}{Backtraces}{But before that, we need more fields from \typename{caml\_domain\_state}}
  \begin{columns}
    \begin{column}{0.5\textwidth}
      \begin{itemize}
        \item Remember \typename{caml\_domain\_state} the per-domain runtime state?
        \item Some other members are of interest to us now\dots
        \item \localname{backtrace\_active} is used to know if the runtime should collect backtraces
        \item \localname{backtrace\_active} can be set by
          \begin{itemize}
            \item The \funcarg{b} flag in the \funcname{OCAMLRUNPARAM} environment variable
            \item \mintinline{ocaml}{Printexc.record_backtrace : bool -> unit} \footnotemark
          \end{itemize}
      \end{itemize}
    \end{column}
    \begin{column}{0.5\textwidth}
      \begin{listing}
        \mintc[highlightlines={9-11}]{src/ocaml/domain\_state\_backtrace.h}
        \caption{runtime/caml/domain\_state.h}
      \end{listing}
    \end{column}
  \end{columns}
  \footnotetext[1]{https://v2.ocaml.org/api/Printexc.html}
\end{frame}

\newcommand\listCamlRaiseExn[1][]{
  \listasm[firstline=702,lastline=725,#1]{../ocaml/runtime/amd64.S}{runtime/amd64.S:702}}

\begin{frame}{Backtraces}{Walk through \funcname{caml\_raise\_exn}}
  \begin{columns}[T]
    \begin{column}{0.5\textwidth}
      \begin{itemize}
        \item<1-> First checks whether exceptions backtrace should be recorded
        \item<1-> By checking the \funcarg{domain\_state->backtrace\_active} value
        \item<2-> If not, acts as \funcname{raise\_notrace} with \funcname{RESTORE\_EXN\_HANDLER\_OCAML}
          \begin{itemize}
            \item Set \regname{RSP} to \localname{domain\_state->exn\_handler} value
            \item Restore \localname{domain\_state->exn\_handler} from trap
            \item Jump with \funcname{ret} (same effect as using \funcname{pop} and \funcname{jmp})
          \end{itemize}
        \item<3-> Otherwise, switches to the C stack to be able to execute C code
        \item<4-> Calls \funcname{caml\_stash\_backtrace}, a C function provided by the runtime\ignorespaces
          \onslide<6->{, with
            \begin{itemize}
              \item \funcarg{exn}: exception value
              \item \funcarg{pc}: code address of the raising point
              \item \funcarg{sp}: stack address of the raising function
              \item \funcarg{trapsp}: stack address of the next handler
            \end{itemize}
          }
      \end{itemize}
      \bigskip
      \mintocaml[firstline=9,lastline=13,highlightlines=12]{../src/backtrace/backtrace.ml}
    \end{column}
    \begin{column}{0.5\textwidth}
      \raggedleft
      \onslide*<1,3-4>{
        \mintc[firstline=190,lastline=192]{../ocaml/runtime/amd64.S}
        \mintc[firstline=234,lastline=237]{../ocaml/runtime/amd64.S}
      }
      \onslide*<2>{
        \mintc[firstline=190,lastline=192,highlightlines={191-192}]{../ocaml/runtime/amd64.S}
        \mintc[firstline=234,lastline=237,highlightlines={235-237}]{../ocaml/runtime/amd64.S}
      }
      \onslide*<1>{\listCamlRaiseExn[highlightlines=705]{}}%
      \onslide*<2>{\listCamlRaiseExn[highlightlines={707-708}]{}}%
      \onslide*<3>{\listCamlRaiseExn[highlightlines={712-714}]{}}%
      \onslide*<4>{\listCamlRaiseExn[highlightlines={715-720}]{}}%
      \onslide*<5>{\providecommand\step{1}\input{diagrams/backtrace/backtrace.tex}}%
      \onslide*<6>{\providecommand\step{2}\input{diagrams/backtrace/backtrace.tex}}%
    \end{column}
  \end{columns}
\end{frame}

\newcommand\listStashBacktrace[1][]{
  \listc[firstline=91,lastline=120,#1]{../ocaml/runtime/backtrace\_nat.c}{runtime/backtrace\_nat.c:91}}

\begin{frame}{Backtraces}{Introducing \funcname{caml\_stash\_backtrace}}
  \begin{columns}[c]
    \begin{column}{0.5\textwidth}
      \minipage[c][0.66\textheight][s]{\columnwidth}
      \begin{itemize}
        \item<1-> A C function provided by the runtime (specific to native code)
        \item<2-> It scans the stack, between the stack pointer at the raising point up to the next trap, in order to collect the names of the detected function frames
          \onslide*<2>{
            \begin{itemize}
              \item And more information, like line number, column number...
            \end{itemize}
          }
        \item<3-> It uses the same technique and information as the Garbage Collector (GC) uses to scan the values live on the stack
          \onslide*<4>{
            \begin{itemize}
              \item \typename{frame\_descr} are at the core of stack unwinding
            \end{itemize}
          }
      \end{itemize}
      \vfill
      \mintocaml[firstline=9,lastline=13,highlightlines=12]{../src/backtrace/backtrace.ml}
      \endminipage
    \end{column}
    \begin{column}{0.5\textwidth}
      \onslide*<1>{\listStashBacktrace[highlightlines=91]{}}
      \onslide*<2>{\listStashBacktrace[highlightlines={91,117-118}]{}}
      \onslide*<3->{\listStashBacktrace[highlightlines={94,106,109-110}]{}}
    \end{column}
  \end{columns}
\end{frame}

\newcommand\listFrameDescr[1][]{
  \listocaml[firstline=111,lastline=124,#1]{../ocaml/asmcomp/emitaux.ml}{asmcomp/emitaux.ml:111}}
\newcommand\listDebugInfo[1][]{
  \listocaml[firstline=38,lastline=49,#1]{../ocaml/lambda/debuginfo.mli}{lambda/debuginfo.mli:38}}

\begin{frame}{Backtraces}{\typename{frame\_descr} and \typename{frame\_debuginfo} in a nutshell}
  \begin{columns}[c]
    \begin{column}{0.5\textwidth}
      \begin{itemize}
        \item<1-> For every return address in an OCaml program, there is a associated \typename{frame\_descr}
        \item<2-> A \typename{frame\_descr} specifies
        \begin{itemize}
          \item<2-> The size of the function stack frame
          \item<3-> Where live values are on the stack (so that the GC can mark them as live and prevent from being swept)
        \end{itemize}
        \item<4-> When compiled with debug information, \typename{frame\_descr} will be linked to a \typename{frame\_debuginfo}
        \begin{itemize}
          \item<5-> Contains information about the position in the source code (file name, line and column number)
        \end{itemize}
      \end{itemize}
    \end{column}
    \begin{column}{0.5\textwidth}
        \onslide*<1>{\listFrameDescr[highlightlines={118-122}]{}}
        \onslide*<2>{\listFrameDescr[highlightlines={120}]{}}
        \onslide*<3>{\listFrameDescr[highlightlines={121}]{}}
        \onslide*<4->{\listFrameDescr[highlightlines={122}]{}}
        \medskip
        \onslide*<1-3>{\listDebugInfo{}}
        \onslide*<4>{\listDebugInfo[highlightlines={38,49}]{}}
        \onslide*<5>{\listDebugInfo[highlightlines={39-46}]{}}
    \end{column}
  \end{columns}
\end{frame}

\begin{frame}{Backtraces}{Scanning the stack with \typename{frame\_descr}}
  \begin{columns}[c]
    \begin{column}{0.5\textwidth}
      \begin{itemize}
        \item Given the return address of the code that raises the exception by calling \funcname{caml\_raise\_exn} (\funcarg{pc} argument of \funcname{caml\_stash\_backtrace})
        \item And the position of the stack pointer (\funcarg{sp} argument of \funcname{caml\_stash\_backtrace})
        \item The frame size given by \typename{frame\_descr}.\funcarg{frame\_size}, allows to find the end of the previous frame
        \item And the return address of the caller of this function
        \bigskip
        \onslide<3->{
        \item Note that traps (here the reddish \funcname{ExnA}, \funcname{ExnB} and \funcname{ExnC} blocks) belong to the frame that installed them, and so the frame size changes when traps are removed
        }
      \end{itemize}
    \end{column}
    \begin{column}{0.5\textwidth}
      \raggedleft
      \onslide*<1>{\providecommand\step{2}\input{diagrams/backtrace/backtrace.tex}}%
      \onslide*<2>{\providecommand\step{1}\input{diagrams/backtrace/scanstack.tex}}%
      \onslide*<3>{\providecommand\step{2}\input{diagrams/backtrace/scanstack.tex}}%
      \onslide*<4>{\providecommand\step{3}\input{diagrams/backtrace/scanstack.tex}}%
      \onslide*<5>{\providecommand\step{4}\input{diagrams/backtrace/scanstack.tex}}%
      \onslide*<6>{\providecommand\step{5}\input{diagrams/backtrace/scanstack.tex}}%
    \end{column}
  \end{columns}
\end{frame}

% Duplicate of previous-previous slide
\begin{frame}{Backtraces}{Continuing on \funcname{caml\_stash\_backtrace}}
  \begin{columns}[c]
    \begin{column}{0.5\textwidth}
      \minipage[c][0.75\textheight][s]{\columnwidth}
      \begin{itemize}
          \color{gray}
        \item A C function provided by the runtime (specific to native code)
        \item It scans the stack, between the stack pointer at the raising point up to the next trap, in order to collect the names of the detected function frames
        \item It uses the same technique and information as the Garbage Collector (GC) uses to scan the values live on the stack
      \end{itemize}
      \begin{itemize}
        \item For each function frame detected, store a pointer to the \typename{frame\_descr} in the \localname{domain\_state->backtrace\_buffer} array, effectively constructing the backtrace
      \end{itemize}
      \onslide<2->{
        \begin{itemize}
            \smallskip
          \item Constructed backtrace:
            \funcname{definitely\_raise},
            \onslide<3->{\funcname{probably\_raise}}
        \end{itemize}
      }
      \vfill
      \mintocaml[firstline=9,lastline=13,highlightlines=12]{../src/backtrace/backtrace.ml}
      \endminipage
    \end{column}
    \begin{column}{0.5\textwidth}
      \raggedleft
      \onslide*<1>{\listStashBacktrace[highlightlines={114-115}]{}}
      \onslide*<2>{\providecommand\step{3}\input{diagrams/backtrace/backtrace.tex}}%
      \onslide*<3>{\providecommand\step{4}\input{diagrams/backtrace/backtrace.tex}}%
    \end{column}
  \end{columns}
\end{frame}

\begin{frame}{Backtraces}{Back to \funcname{caml\_raise\_exn}}
  \begin{columns}[c]
    \begin{column}{0.5\textwidth}
      \minipage[c][0.66\textheight][s]{\columnwidth}
      \begin{itemize}\color{gray}
        \item Checks wheteher exceptions backtrace should be recorded, by checking the \funcarg{domain\_state->backtrace\_active} value
        \item Switches to the C stack to be able to execute C code
        \item Calls \funcname{caml\_stash\_backtrace}, to collect the backtrace up to the next trap
      \end{itemize}
      \onslide*<2->{
        \begin{itemize}
          \item<2-> Executes the exception handler in \funcname{probably\_raise} with \funcname{RESTORE\_EXN\_HANDLER\_OCAML}
            \begin{itemize}
              \item Set \regname{RSP} to \localname{domain\_state->exn\_handler} value
              \item Restore \localname{domain\_state->exn\_handler} from trap
              \item Jump to the handler code with \funcname{ret}
            \end{itemize}
        \end{itemize}
      }
      \vfill
      \onslide*<1>{\mintocaml[firstline=9,lastline=13,highlightlines=12]{../src/backtrace/backtrace.ml}}
      \onslide*<2->{\mintocaml[firstline=15,lastline=21,highlightlines={19-21}]{../src/backtrace/backtrace.ml}}
      \endminipage
    \end{column}
    \begin{column}{0.5\textwidth}
      \centering
      \onslide*<1>{
        \mintc[firstline=190,lastline=192]{../ocaml/runtime/amd64.S}
        \mintc[firstline=234,lastline=237]{../ocaml/runtime/amd64.S}
      }
      \onslide*<2>{
        \mintc[firstline=190,lastline=192,highlightlines={191-192}]{../ocaml/runtime/amd64.S}
        \mintc[firstline=234,lastline=237,highlightlines={235-237}]{../ocaml/runtime/amd64.S}
      }
      \onslide*<1>{\listCamlRaiseExn[highlightlines=720]{}}
      \onslide*<2>{\listCamlRaiseExn[highlightlines=722]{}}
      \onslide*<3->{\providecommand\step{5}\input{diagrams/backtrace/backtrace.tex}}
    \end{column}
  \end{columns}
\end{frame}

\begin{frame}{Backtraces}{\funcname{caml\_probably\_raise}'s handler doesn't catch \typename{ExnC}}
  \begin{columns}[c]
    \begin{column}{0.45\textwidth}
      \minipage[c][0.75\textheight][s]{\columnwidth}
      \begin{itemize}
        \item<1-> \funcname{match \dots with}\footnotemark pattern matching can also match exceptions
        \item<1-> The CMM of \funcname{probably\_raise} shows that this \funcname{match \dots with} is implemented with a pattern matching and a \funcname{try \dots with}
        \item<1-> But this one only matches on \typename{ExnA} exception
        \item<2-> Since the pattern matching fails to catch the exception, \funcname{reraise} is called with our current \typename{ExnC} exception
        \item<3-> Disassembly shows that \funcname{reraise} is actually a call to \funcname{caml\_reraise\_exn}, which is also an assembly function provided by the runtime
      \end{itemize}
      \vfill
      \mintocaml[firstline=15,lastline=21,highlightlines={18-21}]{../src/backtrace/backtrace.ml}
      \endminipage
    \end{column}
    \begin{column}{0.55\textwidth}
      \centering
      \onslide*<1>{\mintcmm[highlightlines={10-18}]{../src/backtrace/camlBacktrace\_\_probably\_raise\_351.cmm}}
      \onslide*<2>{\listcmm[highlightlines=18]{../src/backtrace/camlBacktrace\_\_probably\_raise_351.cmm}{CMM of probably\_raise}}
      \onslide*<3>{\mintobjdump[firstline=5,lastline=35,highlightlines=31]{../src/backtrace/camlBacktrace\_\_probably\_raise_351.objdump}}
    \end{column}
  \end{columns}
  \only<1>{\footnotetext[1]{https://blog.janestreet.com/pattern-matching-and-exception-handling-unite/}}
\end{frame}

\newcommand\listCamlReraiseExn[1][]{
  \listasm[firstline=727,lastline=734,#1]{../ocaml/runtime/amd64.S}{runtime/amd64.S:727}}

\begin{frame}{Backtraces}{Introducing \funcname{caml\_reraise\_exn}}
  \begin{columns}[c]
    \begin{column}{0.5\textwidth}
      \minipage[c][0.66\textheight][s]{\columnwidth}
      \begin{itemize}
        \item<1-> The exception have to be reraised to the next handler
        \item<2-> First, checks if the backtrace should be collected with \funcarg{domain\_state->backtrace\_active}
        \only<3>{
        \begin{itemize}
            \item If not, behaves like a \funcname{raise\_notrace} with \funcname{RESTORE\_EXN\_HANDLER\_OCAML}
        \end{itemize}
        }
        \item<4-> Jumps into \funcname{caml\_raise\_exn}
        \item<5-> Calls \funcname{caml\_stash\_backtrace} again, to scan the next part of the stack: from the trap just pop-ed to the next trap
      \end{itemize}
      \vfill
      \mintocaml[firstline=15,lastline=21,highlightlines={18-21}]{../src/backtrace/backtrace.ml}
      \endminipage
    \end{column}
    \begin{column}{0.5\textwidth}
      \raggedleft
      \onslide*<1>{\listCamlReraiseExn{}}
      \onslide*<2>{\listCamlReraiseExn[highlightlines=729]{}}
      \onslide*<3>{
        \mintc[firstline=190,lastline=192,highlightlines={191-192}]{../ocaml/runtime/amd64.S}
        \mintc[firstline=234,lastline=237,highlightlines={235-237}]{../ocaml/runtime/amd64.S}
        \medskip
        \listCamlReraiseExn[highlightlines=731]{}
      }
      \onslide*<4-5>{
        % TODO refactor with \listCamlReraiseExn
        \mintasm[firstline=727,lastline=734,highlightlines=730]{../ocaml/runtime/amd64.S}
        \medskip
      }
      \onslide*<4>{\listCamlRaiseExn[highlightlines=711]{}}
      \onslide*<5>{\listCamlRaiseExn[highlightlines=720]{}}
    \end{column}
  \end{columns}
\end{frame}

\begin{frame}{Backtraces}{Backtrace collection continues}
  \begin{columns}[c]
    \begin{column}{0.55\textwidth}
      \minipage[c][0.75\textheight][s]{\columnwidth}
      \begin{itemize}
        \item<1-> \funcname{caml\_stash\_backtrace} scans the older part of the stack, up to the next trap
        \item<6-> Controls jumps to the nested exception handler in \funcname{maybe\_raise}, which matches only on \typename{ExnB}
        \item<7-> \funcname{caml\_reraise\_exn} is called again, backtrace collection resumes with \funcname{caml\_stash\_backtrace}
      \end{itemize}
      \medskip
      \begin{itemize}
        \item<2-> Constructed backtrace:
        \begin{itemize}
          \item <2-> \funcname{definitely\_raise}
          \item <2-> \funcname{probably\_raise}
          \item <3-> \funcname{probably\_raise} (re-raise)
          \item <4-> \funcname{maybe\_raise}
          \item <5-> \funcname{main}
          \item <8-> \funcname{main} (re-raise)
        \end{itemize}
      \end{itemize}
      \vfill
      \onslide*<1-5>{\mintocaml[firstline=15,lastline=21]{../src/backtrace/backtrace.ml}}
      \onslide*<6>{\mintocaml[firstline=28,lastline=34,highlightlines=31]{../src/backtrace/backtrace.ml}}
      \onslide*<7->{\mintocaml[firstline=28,lastline=34,highlightlines=33]{../src/backtrace/backtrace.ml}}
      \endminipage
    \end{column}
    \begin{column}{0.45\textwidth}
      \raggedleft
      \onslide*<1>{\providecommand\step{6}\input{diagrams/backtrace/backtrace.tex}}%
      \onslide*<2>{\providecommand\step{6}\input{diagrams/backtrace/backtrace.tex}}%
      \onslide*<3>{\providecommand\step{7}\input{diagrams/backtrace/backtrace.tex}}%
      \onslide*<4>{\providecommand\step{8}\input{diagrams/backtrace/backtrace.tex}}%
      \onslide*<5>{\providecommand\step{9}\input{diagrams/backtrace/backtrace.tex}}%
      \onslide*<6>{\providecommand\step{10}\input{diagrams/backtrace/backtrace.tex}}%
      \onslide*<7>{\providecommand\step{11}\input{diagrams/backtrace/backtrace.tex}}%
      \onslide*<8>{\providecommand\step{12}\input{diagrams/backtrace/backtrace.tex}}%
      \onslide*<9>{\providecommand\step{13}\input{diagrams/backtrace/backtrace.tex}}%
    \end{column}
  \end{columns}
\end{frame}

\begin{frame}{Backtraces}{Printing the backtrace}
  \begin{columns}[c]
    \begin{column}{0.45\textwidth}
      \minipage[c][0.66\textheight][s]{\columnwidth}
      \begin{itemize}
        \item<1-> The exception backtrace can now be printed to \funcarg{stdout} with \funcname{Printexc.print_backtrace}
        \item<2-> First the exception backtrace is retrieved into an OCaml array with \funcname{get_raw_backtrace }
        \item<3-> Which is implemented as the \funcname{caml_get_exception_raw_backtrace} C function in the runtime
        \item<4-> And finally printed with \funcname{print_exception_backtrace}
      \end{itemize}
      \vfill
      \mintocaml[firstline=28,lastline=34,highlightlines=33]{../src/backtrace/backtrace.ml}
      \endminipage
    \end{column}
    \begin{column}{0.55\textwidth}
      \onslide*<1,2>{
        \mintocaml[firstline=100,lastline=101]{../ocaml/stdlib/printexc.ml}
      }
      \only<1>{
        \listocaml[firstline=170,lastline=172]{../ocaml/stdlib/printexc.ml}{stdlib/printexc.ml:170}
      }
      \only<2>{
        \listocaml[firstline=170,lastline=172,highlightlines=172]{../ocaml/stdlib/printexc.ml}{stdlib/printexc.ml:170}
      }
      \only<3>{
        \mintc[firstline=154,lastline=156]{../ocaml/runtime/backtrace.c}
        \listc[firstline=171,lastline=192,highlightlines={184-187}]{../ocaml/runtime/backtrace.c}{runtime/backtrace.c:154}
      }
      \only<4>{
        \listocaml[firstline=155,lastline=165]{../ocaml/stdlib/printexc.ml}{stdlib/printexc.ml:55}
      }
    \end{column}
  \end{columns}
\end{frame}


\subsection{The multiple kinds of raise}
\frameSubsection{}{}

\begin{frame}{Backtraces}{When backtraces are not needed}
  \begin{columns}[c]
    \begin{column}{0.45\textwidth}
      \minipage[c][0.66\textheight][s]{\columnwidth}
      \begin{itemize}
        \item<1-> What if the backtrace for an exception is never needed?
        \item<2-> It's possible to raise the exception explicitly with \funcname{raise\_notrace} to make raising as cheap as possible
        \item<3-> An example of use in the \funcname{dynlink} library of the OCaml compiler
      \end{itemize}
      \vfill
      \onslide<3->{
      \listocaml[firstline=205,lastline=209,highlightlines=207]{../ocaml/otherlibs/dynlink/dynlink\_compilerlibs/misc.ml}{otherlibs/dynlink/dynlink\_compilerlibs/misc.ml:205}
      }
      \endminipage
    \end{column}
    \begin{column}{0.55\textwidth}
    \only<4>{
      \mintcmm[highlightlines=3]{../src/notrace/camlNotrace\_\_fun_347.cmm}
      \listcmm{../src/notrace/camlNotrace\_\_all\_somes\_327.cmm}{all\_somes}
    }
    \onslide*<5->{
      \listobjdump[highlightlines={6-9}]{../src/notrace/camlNotrace\_\_fun_347.objdump}{anonymous function from \funcname{all\_somes}}
    }
    \end{column}
  \end{columns}
\end{frame}

\begin{frame}{Backtraces}{Summary table of the multiple kinds of raise}
  \begin{itemize}
    \item When debugging information is enabled and if the backtrace of an exception isn't needed, it's possible to raise with \funcname{raise\_notrace} for performance
    \item When debugging information is \emph{not} enabled, the compiler optimises \funcname{raise} calls to inline assembly (same effect as using \funcname{raise\_notrace})
  \end{itemize}
  \bigskip
  \centering
  \begin{tabular}{| l | c | c | c | c |}
    \hline
    \parbox{10em}{Debug information \\ (\funcarg{-g} flag)}  & \multicolumn{2}{|c|}{Disabled}                                     & \multicolumn{2}{|c|}{Enabled} \\
    \hline
    Raise kind                                               & \funcname{raise} & \funcname{raise\_notrace}                       & \funcname{raise} & \funcname{raise\_notrace} \\
    \hline
    Compilation                                              & \parbox{8em}{inline assembly\\ (optimization)} & inline assembly   & \funcname{caml\_raise\_exn} & inline assembly \\
    \hline
  \end{tabular}
\end{frame}


%
% Takeaway #5
%
\subsection*{Takeaway \#5}
\frameSubsectionTakeaway{}
\againframe<5>{takeaway}
}%
      \onslide*<6>{\providecommand\step{10}%
% Backtraces
%
\subsection{One advantage of raising with debugging information}
\frameSubsection{}{}

\begin{frame}{Backtraces}{Debugging information \& source code}
  \begin{columns}[c]
    \begin{column}{0.5\textwidth}
      \begin{itemize}
        \item Raising an exception is fun and games, but how do we know where the exception originated? $\rightarrow$ With the backtrace associated with the exception!
        \item In order to get a backtrace from the point where an exception was raised, it's necessary to compile with debugging information enabled (\funcarg{-g} flag for the compiler)
        \item Same example as in the `Nested handlers' section
      \end{itemize}
    \end{column}
    \begin{column}{0.5\textwidth}
      \listocaml[firstline=9,lastline=34]{../src/backtrace/backtrace.ml}{backtrace/backtrace.ml}
    \end{column}
  \end{columns}
\end{frame}

\begin{frame}{Backtraces}{CMM and disassembly of \funcname{definitely\_raise}}
  \begin{columns}[c]
    \begin{column}{0.5\textwidth}
      \minipage[c][0.66\textheight][s]{\columnwidth}
      \begin{itemize}
        \item<1-> An effect of compiling with debugging information (\funcarg{-g}): the CMM no longer uses \funcname{raise\_notrace} to raise the exception but \funcname{raise} instead
        \item<2-> Disassembly shows that CMM \funcname{raise} is actually a call to \funcname{caml\_raise\_exn}, a function in assembler provided by the runtime
      \end{itemize}
      \vfill
      \mintocaml[firstline=9,lastline=13,highlightlines=12]{../src/backtrace/backtrace.ml}
      \endminipage
    \end{column}
    \begin{column}{0.5\textwidth}
      \onslide*<1>{
        \mintcmm[highlightlines=5]{../src/backtrace/camlBacktrace\_\_definitely\_raise\_347.cmm}
      }
      \onslide*<2>{
        \mintobjdump[firstline=2,highlightlines=14]{../src/backtrace/camlBacktrace\_\_definitely\_raise\_347.objdump}
      }
    \end{column}
  \end{columns}
\end{frame}

\begin{frame}{Backtraces}{Introducing \funcname{caml\_raise\_exn}}
  \begin{columns}[c]
    \begin{column}{0.5\textwidth}
      \minipage[c][0.66\textheight][s]{\columnwidth}
      \onslide*<1>{
        \begin{itemize}
          \item Raising the exception is performed using \funcname{caml\_raise\_exn} which is
            \begin{itemize}
              \item An assembly function provided by the runtime
              \item Architecture specific
            \end{itemize}
          \item Let's see what it does differently from \funcname{raise\_notrace}\dots
          \item We are about to raise \typename{ExnC} with \funcname{caml\_raise\_exn} from \funcname{definitely\_raise}
        \end{itemize}
      }
      \onslide*<2>{
        \mintobjdump[firstline=2,highlightlines=14]{../src/backtrace/camlBacktrace\_\_definitely\_raise\_347.objdump}
      }
      \vfill
      \mintocaml[firstline=9,lastline=13,highlightlines=12]{../src/backtrace/backtrace.ml}
      \endminipage
    \end{column}
    \begin{column}{0.5\textwidth}
      \centering
      \providecommand\step{1}\input{diagrams/backtrace/backtrace.tex}
    \end{column}
  \end{columns}
\end{frame}

\begin{frame}{Backtraces}{But before that, we need more fields from \typename{caml\_domain\_state}}
  \begin{columns}
    \begin{column}{0.5\textwidth}
      \begin{itemize}
        \item Remember \typename{caml\_domain\_state} the per-domain runtime state?
        \item Some other members are of interest to us now\dots
        \item \localname{backtrace\_active} is used to know if the runtime should collect backtraces
        \item \localname{backtrace\_active} can be set by
          \begin{itemize}
            \item The \funcarg{b} flag in the \funcname{OCAMLRUNPARAM} environment variable
            \item \mintinline{ocaml}{Printexc.record_backtrace : bool -> unit} \footnotemark
          \end{itemize}
      \end{itemize}
    \end{column}
    \begin{column}{0.5\textwidth}
      \begin{listing}
        \mintc[highlightlines={9-11}]{src/ocaml/domain\_state\_backtrace.h}
        \caption{runtime/caml/domain\_state.h}
      \end{listing}
    \end{column}
  \end{columns}
  \footnotetext[1]{https://v2.ocaml.org/api/Printexc.html}
\end{frame}

\newcommand\listCamlRaiseExn[1][]{
  \listasm[firstline=702,lastline=725,#1]{../ocaml/runtime/amd64.S}{runtime/amd64.S:702}}

\begin{frame}{Backtraces}{Walk through \funcname{caml\_raise\_exn}}
  \begin{columns}[T]
    \begin{column}{0.5\textwidth}
      \begin{itemize}
        \item<1-> First checks whether exceptions backtrace should be recorded
        \item<1-> By checking the \funcarg{domain\_state->backtrace\_active} value
        \item<2-> If not, acts as \funcname{raise\_notrace} with \funcname{RESTORE\_EXN\_HANDLER\_OCAML}
          \begin{itemize}
            \item Set \regname{RSP} to \localname{domain\_state->exn\_handler} value
            \item Restore \localname{domain\_state->exn\_handler} from trap
            \item Jump with \funcname{ret} (same effect as using \funcname{pop} and \funcname{jmp})
          \end{itemize}
        \item<3-> Otherwise, switches to the C stack to be able to execute C code
        \item<4-> Calls \funcname{caml\_stash\_backtrace}, a C function provided by the runtime\ignorespaces
          \onslide<6->{, with
            \begin{itemize}
              \item \funcarg{exn}: exception value
              \item \funcarg{pc}: code address of the raising point
              \item \funcarg{sp}: stack address of the raising function
              \item \funcarg{trapsp}: stack address of the next handler
            \end{itemize}
          }
      \end{itemize}
      \bigskip
      \mintocaml[firstline=9,lastline=13,highlightlines=12]{../src/backtrace/backtrace.ml}
    \end{column}
    \begin{column}{0.5\textwidth}
      \raggedleft
      \onslide*<1,3-4>{
        \mintc[firstline=190,lastline=192]{../ocaml/runtime/amd64.S}
        \mintc[firstline=234,lastline=237]{../ocaml/runtime/amd64.S}
      }
      \onslide*<2>{
        \mintc[firstline=190,lastline=192,highlightlines={191-192}]{../ocaml/runtime/amd64.S}
        \mintc[firstline=234,lastline=237,highlightlines={235-237}]{../ocaml/runtime/amd64.S}
      }
      \onslide*<1>{\listCamlRaiseExn[highlightlines=705]{}}%
      \onslide*<2>{\listCamlRaiseExn[highlightlines={707-708}]{}}%
      \onslide*<3>{\listCamlRaiseExn[highlightlines={712-714}]{}}%
      \onslide*<4>{\listCamlRaiseExn[highlightlines={715-720}]{}}%
      \onslide*<5>{\providecommand\step{1}\input{diagrams/backtrace/backtrace.tex}}%
      \onslide*<6>{\providecommand\step{2}\input{diagrams/backtrace/backtrace.tex}}%
    \end{column}
  \end{columns}
\end{frame}

\newcommand\listStashBacktrace[1][]{
  \listc[firstline=91,lastline=120,#1]{../ocaml/runtime/backtrace\_nat.c}{runtime/backtrace\_nat.c:91}}

\begin{frame}{Backtraces}{Introducing \funcname{caml\_stash\_backtrace}}
  \begin{columns}[c]
    \begin{column}{0.5\textwidth}
      \minipage[c][0.66\textheight][s]{\columnwidth}
      \begin{itemize}
        \item<1-> A C function provided by the runtime (specific to native code)
        \item<2-> It scans the stack, between the stack pointer at the raising point up to the next trap, in order to collect the names of the detected function frames
          \onslide*<2>{
            \begin{itemize}
              \item And more information, like line number, column number...
            \end{itemize}
          }
        \item<3-> It uses the same technique and information as the Garbage Collector (GC) uses to scan the values live on the stack
          \onslide*<4>{
            \begin{itemize}
              \item \typename{frame\_descr} are at the core of stack unwinding
            \end{itemize}
          }
      \end{itemize}
      \vfill
      \mintocaml[firstline=9,lastline=13,highlightlines=12]{../src/backtrace/backtrace.ml}
      \endminipage
    \end{column}
    \begin{column}{0.5\textwidth}
      \onslide*<1>{\listStashBacktrace[highlightlines=91]{}}
      \onslide*<2>{\listStashBacktrace[highlightlines={91,117-118}]{}}
      \onslide*<3->{\listStashBacktrace[highlightlines={94,106,109-110}]{}}
    \end{column}
  \end{columns}
\end{frame}

\newcommand\listFrameDescr[1][]{
  \listocaml[firstline=111,lastline=124,#1]{../ocaml/asmcomp/emitaux.ml}{asmcomp/emitaux.ml:111}}
\newcommand\listDebugInfo[1][]{
  \listocaml[firstline=38,lastline=49,#1]{../ocaml/lambda/debuginfo.mli}{lambda/debuginfo.mli:38}}

\begin{frame}{Backtraces}{\typename{frame\_descr} and \typename{frame\_debuginfo} in a nutshell}
  \begin{columns}[c]
    \begin{column}{0.5\textwidth}
      \begin{itemize}
        \item<1-> For every return address in an OCaml program, there is a associated \typename{frame\_descr}
        \item<2-> A \typename{frame\_descr} specifies
        \begin{itemize}
          \item<2-> The size of the function stack frame
          \item<3-> Where live values are on the stack (so that the GC can mark them as live and prevent from being swept)
        \end{itemize}
        \item<4-> When compiled with debug information, \typename{frame\_descr} will be linked to a \typename{frame\_debuginfo}
        \begin{itemize}
          \item<5-> Contains information about the position in the source code (file name, line and column number)
        \end{itemize}
      \end{itemize}
    \end{column}
    \begin{column}{0.5\textwidth}
        \onslide*<1>{\listFrameDescr[highlightlines={118-122}]{}}
        \onslide*<2>{\listFrameDescr[highlightlines={120}]{}}
        \onslide*<3>{\listFrameDescr[highlightlines={121}]{}}
        \onslide*<4->{\listFrameDescr[highlightlines={122}]{}}
        \medskip
        \onslide*<1-3>{\listDebugInfo{}}
        \onslide*<4>{\listDebugInfo[highlightlines={38,49}]{}}
        \onslide*<5>{\listDebugInfo[highlightlines={39-46}]{}}
    \end{column}
  \end{columns}
\end{frame}

\begin{frame}{Backtraces}{Scanning the stack with \typename{frame\_descr}}
  \begin{columns}[c]
    \begin{column}{0.5\textwidth}
      \begin{itemize}
        \item Given the return address of the code that raises the exception by calling \funcname{caml\_raise\_exn} (\funcarg{pc} argument of \funcname{caml\_stash\_backtrace})
        \item And the position of the stack pointer (\funcarg{sp} argument of \funcname{caml\_stash\_backtrace})
        \item The frame size given by \typename{frame\_descr}.\funcarg{frame\_size}, allows to find the end of the previous frame
        \item And the return address of the caller of this function
        \bigskip
        \onslide<3->{
        \item Note that traps (here the reddish \funcname{ExnA}, \funcname{ExnB} and \funcname{ExnC} blocks) belong to the frame that installed them, and so the frame size changes when traps are removed
        }
      \end{itemize}
    \end{column}
    \begin{column}{0.5\textwidth}
      \raggedleft
      \onslide*<1>{\providecommand\step{2}\input{diagrams/backtrace/backtrace.tex}}%
      \onslide*<2>{\providecommand\step{1}\input{diagrams/backtrace/scanstack.tex}}%
      \onslide*<3>{\providecommand\step{2}\input{diagrams/backtrace/scanstack.tex}}%
      \onslide*<4>{\providecommand\step{3}\input{diagrams/backtrace/scanstack.tex}}%
      \onslide*<5>{\providecommand\step{4}\input{diagrams/backtrace/scanstack.tex}}%
      \onslide*<6>{\providecommand\step{5}\input{diagrams/backtrace/scanstack.tex}}%
    \end{column}
  \end{columns}
\end{frame}

% Duplicate of previous-previous slide
\begin{frame}{Backtraces}{Continuing on \funcname{caml\_stash\_backtrace}}
  \begin{columns}[c]
    \begin{column}{0.5\textwidth}
      \minipage[c][0.75\textheight][s]{\columnwidth}
      \begin{itemize}
          \color{gray}
        \item A C function provided by the runtime (specific to native code)
        \item It scans the stack, between the stack pointer at the raising point up to the next trap, in order to collect the names of the detected function frames
        \item It uses the same technique and information as the Garbage Collector (GC) uses to scan the values live on the stack
      \end{itemize}
      \begin{itemize}
        \item For each function frame detected, store a pointer to the \typename{frame\_descr} in the \localname{domain\_state->backtrace\_buffer} array, effectively constructing the backtrace
      \end{itemize}
      \onslide<2->{
        \begin{itemize}
            \smallskip
          \item Constructed backtrace:
            \funcname{definitely\_raise},
            \onslide<3->{\funcname{probably\_raise}}
        \end{itemize}
      }
      \vfill
      \mintocaml[firstline=9,lastline=13,highlightlines=12]{../src/backtrace/backtrace.ml}
      \endminipage
    \end{column}
    \begin{column}{0.5\textwidth}
      \raggedleft
      \onslide*<1>{\listStashBacktrace[highlightlines={114-115}]{}}
      \onslide*<2>{\providecommand\step{3}\input{diagrams/backtrace/backtrace.tex}}%
      \onslide*<3>{\providecommand\step{4}\input{diagrams/backtrace/backtrace.tex}}%
    \end{column}
  \end{columns}
\end{frame}

\begin{frame}{Backtraces}{Back to \funcname{caml\_raise\_exn}}
  \begin{columns}[c]
    \begin{column}{0.5\textwidth}
      \minipage[c][0.66\textheight][s]{\columnwidth}
      \begin{itemize}\color{gray}
        \item Checks wheteher exceptions backtrace should be recorded, by checking the \funcarg{domain\_state->backtrace\_active} value
        \item Switches to the C stack to be able to execute C code
        \item Calls \funcname{caml\_stash\_backtrace}, to collect the backtrace up to the next trap
      \end{itemize}
      \onslide*<2->{
        \begin{itemize}
          \item<2-> Executes the exception handler in \funcname{probably\_raise} with \funcname{RESTORE\_EXN\_HANDLER\_OCAML}
            \begin{itemize}
              \item Set \regname{RSP} to \localname{domain\_state->exn\_handler} value
              \item Restore \localname{domain\_state->exn\_handler} from trap
              \item Jump to the handler code with \funcname{ret}
            \end{itemize}
        \end{itemize}
      }
      \vfill
      \onslide*<1>{\mintocaml[firstline=9,lastline=13,highlightlines=12]{../src/backtrace/backtrace.ml}}
      \onslide*<2->{\mintocaml[firstline=15,lastline=21,highlightlines={19-21}]{../src/backtrace/backtrace.ml}}
      \endminipage
    \end{column}
    \begin{column}{0.5\textwidth}
      \centering
      \onslide*<1>{
        \mintc[firstline=190,lastline=192]{../ocaml/runtime/amd64.S}
        \mintc[firstline=234,lastline=237]{../ocaml/runtime/amd64.S}
      }
      \onslide*<2>{
        \mintc[firstline=190,lastline=192,highlightlines={191-192}]{../ocaml/runtime/amd64.S}
        \mintc[firstline=234,lastline=237,highlightlines={235-237}]{../ocaml/runtime/amd64.S}
      }
      \onslide*<1>{\listCamlRaiseExn[highlightlines=720]{}}
      \onslide*<2>{\listCamlRaiseExn[highlightlines=722]{}}
      \onslide*<3->{\providecommand\step{5}\input{diagrams/backtrace/backtrace.tex}}
    \end{column}
  \end{columns}
\end{frame}

\begin{frame}{Backtraces}{\funcname{caml\_probably\_raise}'s handler doesn't catch \typename{ExnC}}
  \begin{columns}[c]
    \begin{column}{0.45\textwidth}
      \minipage[c][0.75\textheight][s]{\columnwidth}
      \begin{itemize}
        \item<1-> \funcname{match \dots with}\footnotemark pattern matching can also match exceptions
        \item<1-> The CMM of \funcname{probably\_raise} shows that this \funcname{match \dots with} is implemented with a pattern matching and a \funcname{try \dots with}
        \item<1-> But this one only matches on \typename{ExnA} exception
        \item<2-> Since the pattern matching fails to catch the exception, \funcname{reraise} is called with our current \typename{ExnC} exception
        \item<3-> Disassembly shows that \funcname{reraise} is actually a call to \funcname{caml\_reraise\_exn}, which is also an assembly function provided by the runtime
      \end{itemize}
      \vfill
      \mintocaml[firstline=15,lastline=21,highlightlines={18-21}]{../src/backtrace/backtrace.ml}
      \endminipage
    \end{column}
    \begin{column}{0.55\textwidth}
      \centering
      \onslide*<1>{\mintcmm[highlightlines={10-18}]{../src/backtrace/camlBacktrace\_\_probably\_raise\_351.cmm}}
      \onslide*<2>{\listcmm[highlightlines=18]{../src/backtrace/camlBacktrace\_\_probably\_raise_351.cmm}{CMM of probably\_raise}}
      \onslide*<3>{\mintobjdump[firstline=5,lastline=35,highlightlines=31]{../src/backtrace/camlBacktrace\_\_probably\_raise_351.objdump}}
    \end{column}
  \end{columns}
  \only<1>{\footnotetext[1]{https://blog.janestreet.com/pattern-matching-and-exception-handling-unite/}}
\end{frame}

\newcommand\listCamlReraiseExn[1][]{
  \listasm[firstline=727,lastline=734,#1]{../ocaml/runtime/amd64.S}{runtime/amd64.S:727}}

\begin{frame}{Backtraces}{Introducing \funcname{caml\_reraise\_exn}}
  \begin{columns}[c]
    \begin{column}{0.5\textwidth}
      \minipage[c][0.66\textheight][s]{\columnwidth}
      \begin{itemize}
        \item<1-> The exception have to be reraised to the next handler
        \item<2-> First, checks if the backtrace should be collected with \funcarg{domain\_state->backtrace\_active}
        \only<3>{
        \begin{itemize}
            \item If not, behaves like a \funcname{raise\_notrace} with \funcname{RESTORE\_EXN\_HANDLER\_OCAML}
        \end{itemize}
        }
        \item<4-> Jumps into \funcname{caml\_raise\_exn}
        \item<5-> Calls \funcname{caml\_stash\_backtrace} again, to scan the next part of the stack: from the trap just pop-ed to the next trap
      \end{itemize}
      \vfill
      \mintocaml[firstline=15,lastline=21,highlightlines={18-21}]{../src/backtrace/backtrace.ml}
      \endminipage
    \end{column}
    \begin{column}{0.5\textwidth}
      \raggedleft
      \onslide*<1>{\listCamlReraiseExn{}}
      \onslide*<2>{\listCamlReraiseExn[highlightlines=729]{}}
      \onslide*<3>{
        \mintc[firstline=190,lastline=192,highlightlines={191-192}]{../ocaml/runtime/amd64.S}
        \mintc[firstline=234,lastline=237,highlightlines={235-237}]{../ocaml/runtime/amd64.S}
        \medskip
        \listCamlReraiseExn[highlightlines=731]{}
      }
      \onslide*<4-5>{
        % TODO refactor with \listCamlReraiseExn
        \mintasm[firstline=727,lastline=734,highlightlines=730]{../ocaml/runtime/amd64.S}
        \medskip
      }
      \onslide*<4>{\listCamlRaiseExn[highlightlines=711]{}}
      \onslide*<5>{\listCamlRaiseExn[highlightlines=720]{}}
    \end{column}
  \end{columns}
\end{frame}

\begin{frame}{Backtraces}{Backtrace collection continues}
  \begin{columns}[c]
    \begin{column}{0.55\textwidth}
      \minipage[c][0.75\textheight][s]{\columnwidth}
      \begin{itemize}
        \item<1-> \funcname{caml\_stash\_backtrace} scans the older part of the stack, up to the next trap
        \item<6-> Controls jumps to the nested exception handler in \funcname{maybe\_raise}, which matches only on \typename{ExnB}
        \item<7-> \funcname{caml\_reraise\_exn} is called again, backtrace collection resumes with \funcname{caml\_stash\_backtrace}
      \end{itemize}
      \medskip
      \begin{itemize}
        \item<2-> Constructed backtrace:
        \begin{itemize}
          \item <2-> \funcname{definitely\_raise}
          \item <2-> \funcname{probably\_raise}
          \item <3-> \funcname{probably\_raise} (re-raise)
          \item <4-> \funcname{maybe\_raise}
          \item <5-> \funcname{main}
          \item <8-> \funcname{main} (re-raise)
        \end{itemize}
      \end{itemize}
      \vfill
      \onslide*<1-5>{\mintocaml[firstline=15,lastline=21]{../src/backtrace/backtrace.ml}}
      \onslide*<6>{\mintocaml[firstline=28,lastline=34,highlightlines=31]{../src/backtrace/backtrace.ml}}
      \onslide*<7->{\mintocaml[firstline=28,lastline=34,highlightlines=33]{../src/backtrace/backtrace.ml}}
      \endminipage
    \end{column}
    \begin{column}{0.45\textwidth}
      \raggedleft
      \onslide*<1>{\providecommand\step{6}\input{diagrams/backtrace/backtrace.tex}}%
      \onslide*<2>{\providecommand\step{6}\input{diagrams/backtrace/backtrace.tex}}%
      \onslide*<3>{\providecommand\step{7}\input{diagrams/backtrace/backtrace.tex}}%
      \onslide*<4>{\providecommand\step{8}\input{diagrams/backtrace/backtrace.tex}}%
      \onslide*<5>{\providecommand\step{9}\input{diagrams/backtrace/backtrace.tex}}%
      \onslide*<6>{\providecommand\step{10}\input{diagrams/backtrace/backtrace.tex}}%
      \onslide*<7>{\providecommand\step{11}\input{diagrams/backtrace/backtrace.tex}}%
      \onslide*<8>{\providecommand\step{12}\input{diagrams/backtrace/backtrace.tex}}%
      \onslide*<9>{\providecommand\step{13}\input{diagrams/backtrace/backtrace.tex}}%
    \end{column}
  \end{columns}
\end{frame}

\begin{frame}{Backtraces}{Printing the backtrace}
  \begin{columns}[c]
    \begin{column}{0.45\textwidth}
      \minipage[c][0.66\textheight][s]{\columnwidth}
      \begin{itemize}
        \item<1-> The exception backtrace can now be printed to \funcarg{stdout} with \funcname{Printexc.print_backtrace}
        \item<2-> First the exception backtrace is retrieved into an OCaml array with \funcname{get_raw_backtrace }
        \item<3-> Which is implemented as the \funcname{caml_get_exception_raw_backtrace} C function in the runtime
        \item<4-> And finally printed with \funcname{print_exception_backtrace}
      \end{itemize}
      \vfill
      \mintocaml[firstline=28,lastline=34,highlightlines=33]{../src/backtrace/backtrace.ml}
      \endminipage
    \end{column}
    \begin{column}{0.55\textwidth}
      \onslide*<1,2>{
        \mintocaml[firstline=100,lastline=101]{../ocaml/stdlib/printexc.ml}
      }
      \only<1>{
        \listocaml[firstline=170,lastline=172]{../ocaml/stdlib/printexc.ml}{stdlib/printexc.ml:170}
      }
      \only<2>{
        \listocaml[firstline=170,lastline=172,highlightlines=172]{../ocaml/stdlib/printexc.ml}{stdlib/printexc.ml:170}
      }
      \only<3>{
        \mintc[firstline=154,lastline=156]{../ocaml/runtime/backtrace.c}
        \listc[firstline=171,lastline=192,highlightlines={184-187}]{../ocaml/runtime/backtrace.c}{runtime/backtrace.c:154}
      }
      \only<4>{
        \listocaml[firstline=155,lastline=165]{../ocaml/stdlib/printexc.ml}{stdlib/printexc.ml:55}
      }
    \end{column}
  \end{columns}
\end{frame}


\subsection{The multiple kinds of raise}
\frameSubsection{}{}

\begin{frame}{Backtraces}{When backtraces are not needed}
  \begin{columns}[c]
    \begin{column}{0.45\textwidth}
      \minipage[c][0.66\textheight][s]{\columnwidth}
      \begin{itemize}
        \item<1-> What if the backtrace for an exception is never needed?
        \item<2-> It's possible to raise the exception explicitly with \funcname{raise\_notrace} to make raising as cheap as possible
        \item<3-> An example of use in the \funcname{dynlink} library of the OCaml compiler
      \end{itemize}
      \vfill
      \onslide<3->{
      \listocaml[firstline=205,lastline=209,highlightlines=207]{../ocaml/otherlibs/dynlink/dynlink\_compilerlibs/misc.ml}{otherlibs/dynlink/dynlink\_compilerlibs/misc.ml:205}
      }
      \endminipage
    \end{column}
    \begin{column}{0.55\textwidth}
    \only<4>{
      \mintcmm[highlightlines=3]{../src/notrace/camlNotrace\_\_fun_347.cmm}
      \listcmm{../src/notrace/camlNotrace\_\_all\_somes\_327.cmm}{all\_somes}
    }
    \onslide*<5->{
      \listobjdump[highlightlines={6-9}]{../src/notrace/camlNotrace\_\_fun_347.objdump}{anonymous function from \funcname{all\_somes}}
    }
    \end{column}
  \end{columns}
\end{frame}

\begin{frame}{Backtraces}{Summary table of the multiple kinds of raise}
  \begin{itemize}
    \item When debugging information is enabled and if the backtrace of an exception isn't needed, it's possible to raise with \funcname{raise\_notrace} for performance
    \item When debugging information is \emph{not} enabled, the compiler optimises \funcname{raise} calls to inline assembly (same effect as using \funcname{raise\_notrace})
  \end{itemize}
  \bigskip
  \centering
  \begin{tabular}{| l | c | c | c | c |}
    \hline
    \parbox{10em}{Debug information \\ (\funcarg{-g} flag)}  & \multicolumn{2}{|c|}{Disabled}                                     & \multicolumn{2}{|c|}{Enabled} \\
    \hline
    Raise kind                                               & \funcname{raise} & \funcname{raise\_notrace}                       & \funcname{raise} & \funcname{raise\_notrace} \\
    \hline
    Compilation                                              & \parbox{8em}{inline assembly\\ (optimization)} & inline assembly   & \funcname{caml\_raise\_exn} & inline assembly \\
    \hline
  \end{tabular}
\end{frame}


%
% Takeaway #5
%
\subsection*{Takeaway \#5}
\frameSubsectionTakeaway{}
\againframe<5>{takeaway}
}%
      \onslide*<7>{\providecommand\step{11}%
% Backtraces
%
\subsection{One advantage of raising with debugging information}
\frameSubsection{}{}

\begin{frame}{Backtraces}{Debugging information \& source code}
  \begin{columns}[c]
    \begin{column}{0.5\textwidth}
      \begin{itemize}
        \item Raising an exception is fun and games, but how do we know where the exception originated? $\rightarrow$ With the backtrace associated with the exception!
        \item In order to get a backtrace from the point where an exception was raised, it's necessary to compile with debugging information enabled (\funcarg{-g} flag for the compiler)
        \item Same example as in the `Nested handlers' section
      \end{itemize}
    \end{column}
    \begin{column}{0.5\textwidth}
      \listocaml[firstline=9,lastline=34]{../src/backtrace/backtrace.ml}{backtrace/backtrace.ml}
    \end{column}
  \end{columns}
\end{frame}

\begin{frame}{Backtraces}{CMM and disassembly of \funcname{definitely\_raise}}
  \begin{columns}[c]
    \begin{column}{0.5\textwidth}
      \minipage[c][0.66\textheight][s]{\columnwidth}
      \begin{itemize}
        \item<1-> An effect of compiling with debugging information (\funcarg{-g}): the CMM no longer uses \funcname{raise\_notrace} to raise the exception but \funcname{raise} instead
        \item<2-> Disassembly shows that CMM \funcname{raise} is actually a call to \funcname{caml\_raise\_exn}, a function in assembler provided by the runtime
      \end{itemize}
      \vfill
      \mintocaml[firstline=9,lastline=13,highlightlines=12]{../src/backtrace/backtrace.ml}
      \endminipage
    \end{column}
    \begin{column}{0.5\textwidth}
      \onslide*<1>{
        \mintcmm[highlightlines=5]{../src/backtrace/camlBacktrace\_\_definitely\_raise\_347.cmm}
      }
      \onslide*<2>{
        \mintobjdump[firstline=2,highlightlines=14]{../src/backtrace/camlBacktrace\_\_definitely\_raise\_347.objdump}
      }
    \end{column}
  \end{columns}
\end{frame}

\begin{frame}{Backtraces}{Introducing \funcname{caml\_raise\_exn}}
  \begin{columns}[c]
    \begin{column}{0.5\textwidth}
      \minipage[c][0.66\textheight][s]{\columnwidth}
      \onslide*<1>{
        \begin{itemize}
          \item Raising the exception is performed using \funcname{caml\_raise\_exn} which is
            \begin{itemize}
              \item An assembly function provided by the runtime
              \item Architecture specific
            \end{itemize}
          \item Let's see what it does differently from \funcname{raise\_notrace}\dots
          \item We are about to raise \typename{ExnC} with \funcname{caml\_raise\_exn} from \funcname{definitely\_raise}
        \end{itemize}
      }
      \onslide*<2>{
        \mintobjdump[firstline=2,highlightlines=14]{../src/backtrace/camlBacktrace\_\_definitely\_raise\_347.objdump}
      }
      \vfill
      \mintocaml[firstline=9,lastline=13,highlightlines=12]{../src/backtrace/backtrace.ml}
      \endminipage
    \end{column}
    \begin{column}{0.5\textwidth}
      \centering
      \providecommand\step{1}\input{diagrams/backtrace/backtrace.tex}
    \end{column}
  \end{columns}
\end{frame}

\begin{frame}{Backtraces}{But before that, we need more fields from \typename{caml\_domain\_state}}
  \begin{columns}
    \begin{column}{0.5\textwidth}
      \begin{itemize}
        \item Remember \typename{caml\_domain\_state} the per-domain runtime state?
        \item Some other members are of interest to us now\dots
        \item \localname{backtrace\_active} is used to know if the runtime should collect backtraces
        \item \localname{backtrace\_active} can be set by
          \begin{itemize}
            \item The \funcarg{b} flag in the \funcname{OCAMLRUNPARAM} environment variable
            \item \mintinline{ocaml}{Printexc.record_backtrace : bool -> unit} \footnotemark
          \end{itemize}
      \end{itemize}
    \end{column}
    \begin{column}{0.5\textwidth}
      \begin{listing}
        \mintc[highlightlines={9-11}]{src/ocaml/domain\_state\_backtrace.h}
        \caption{runtime/caml/domain\_state.h}
      \end{listing}
    \end{column}
  \end{columns}
  \footnotetext[1]{https://v2.ocaml.org/api/Printexc.html}
\end{frame}

\newcommand\listCamlRaiseExn[1][]{
  \listasm[firstline=702,lastline=725,#1]{../ocaml/runtime/amd64.S}{runtime/amd64.S:702}}

\begin{frame}{Backtraces}{Walk through \funcname{caml\_raise\_exn}}
  \begin{columns}[T]
    \begin{column}{0.5\textwidth}
      \begin{itemize}
        \item<1-> First checks whether exceptions backtrace should be recorded
        \item<1-> By checking the \funcarg{domain\_state->backtrace\_active} value
        \item<2-> If not, acts as \funcname{raise\_notrace} with \funcname{RESTORE\_EXN\_HANDLER\_OCAML}
          \begin{itemize}
            \item Set \regname{RSP} to \localname{domain\_state->exn\_handler} value
            \item Restore \localname{domain\_state->exn\_handler} from trap
            \item Jump with \funcname{ret} (same effect as using \funcname{pop} and \funcname{jmp})
          \end{itemize}
        \item<3-> Otherwise, switches to the C stack to be able to execute C code
        \item<4-> Calls \funcname{caml\_stash\_backtrace}, a C function provided by the runtime\ignorespaces
          \onslide<6->{, with
            \begin{itemize}
              \item \funcarg{exn}: exception value
              \item \funcarg{pc}: code address of the raising point
              \item \funcarg{sp}: stack address of the raising function
              \item \funcarg{trapsp}: stack address of the next handler
            \end{itemize}
          }
      \end{itemize}
      \bigskip
      \mintocaml[firstline=9,lastline=13,highlightlines=12]{../src/backtrace/backtrace.ml}
    \end{column}
    \begin{column}{0.5\textwidth}
      \raggedleft
      \onslide*<1,3-4>{
        \mintc[firstline=190,lastline=192]{../ocaml/runtime/amd64.S}
        \mintc[firstline=234,lastline=237]{../ocaml/runtime/amd64.S}
      }
      \onslide*<2>{
        \mintc[firstline=190,lastline=192,highlightlines={191-192}]{../ocaml/runtime/amd64.S}
        \mintc[firstline=234,lastline=237,highlightlines={235-237}]{../ocaml/runtime/amd64.S}
      }
      \onslide*<1>{\listCamlRaiseExn[highlightlines=705]{}}%
      \onslide*<2>{\listCamlRaiseExn[highlightlines={707-708}]{}}%
      \onslide*<3>{\listCamlRaiseExn[highlightlines={712-714}]{}}%
      \onslide*<4>{\listCamlRaiseExn[highlightlines={715-720}]{}}%
      \onslide*<5>{\providecommand\step{1}\input{diagrams/backtrace/backtrace.tex}}%
      \onslide*<6>{\providecommand\step{2}\input{diagrams/backtrace/backtrace.tex}}%
    \end{column}
  \end{columns}
\end{frame}

\newcommand\listStashBacktrace[1][]{
  \listc[firstline=91,lastline=120,#1]{../ocaml/runtime/backtrace\_nat.c}{runtime/backtrace\_nat.c:91}}

\begin{frame}{Backtraces}{Introducing \funcname{caml\_stash\_backtrace}}
  \begin{columns}[c]
    \begin{column}{0.5\textwidth}
      \minipage[c][0.66\textheight][s]{\columnwidth}
      \begin{itemize}
        \item<1-> A C function provided by the runtime (specific to native code)
        \item<2-> It scans the stack, between the stack pointer at the raising point up to the next trap, in order to collect the names of the detected function frames
          \onslide*<2>{
            \begin{itemize}
              \item And more information, like line number, column number...
            \end{itemize}
          }
        \item<3-> It uses the same technique and information as the Garbage Collector (GC) uses to scan the values live on the stack
          \onslide*<4>{
            \begin{itemize}
              \item \typename{frame\_descr} are at the core of stack unwinding
            \end{itemize}
          }
      \end{itemize}
      \vfill
      \mintocaml[firstline=9,lastline=13,highlightlines=12]{../src/backtrace/backtrace.ml}
      \endminipage
    \end{column}
    \begin{column}{0.5\textwidth}
      \onslide*<1>{\listStashBacktrace[highlightlines=91]{}}
      \onslide*<2>{\listStashBacktrace[highlightlines={91,117-118}]{}}
      \onslide*<3->{\listStashBacktrace[highlightlines={94,106,109-110}]{}}
    \end{column}
  \end{columns}
\end{frame}

\newcommand\listFrameDescr[1][]{
  \listocaml[firstline=111,lastline=124,#1]{../ocaml/asmcomp/emitaux.ml}{asmcomp/emitaux.ml:111}}
\newcommand\listDebugInfo[1][]{
  \listocaml[firstline=38,lastline=49,#1]{../ocaml/lambda/debuginfo.mli}{lambda/debuginfo.mli:38}}

\begin{frame}{Backtraces}{\typename{frame\_descr} and \typename{frame\_debuginfo} in a nutshell}
  \begin{columns}[c]
    \begin{column}{0.5\textwidth}
      \begin{itemize}
        \item<1-> For every return address in an OCaml program, there is a associated \typename{frame\_descr}
        \item<2-> A \typename{frame\_descr} specifies
        \begin{itemize}
          \item<2-> The size of the function stack frame
          \item<3-> Where live values are on the stack (so that the GC can mark them as live and prevent from being swept)
        \end{itemize}
        \item<4-> When compiled with debug information, \typename{frame\_descr} will be linked to a \typename{frame\_debuginfo}
        \begin{itemize}
          \item<5-> Contains information about the position in the source code (file name, line and column number)
        \end{itemize}
      \end{itemize}
    \end{column}
    \begin{column}{0.5\textwidth}
        \onslide*<1>{\listFrameDescr[highlightlines={118-122}]{}}
        \onslide*<2>{\listFrameDescr[highlightlines={120}]{}}
        \onslide*<3>{\listFrameDescr[highlightlines={121}]{}}
        \onslide*<4->{\listFrameDescr[highlightlines={122}]{}}
        \medskip
        \onslide*<1-3>{\listDebugInfo{}}
        \onslide*<4>{\listDebugInfo[highlightlines={38,49}]{}}
        \onslide*<5>{\listDebugInfo[highlightlines={39-46}]{}}
    \end{column}
  \end{columns}
\end{frame}

\begin{frame}{Backtraces}{Scanning the stack with \typename{frame\_descr}}
  \begin{columns}[c]
    \begin{column}{0.5\textwidth}
      \begin{itemize}
        \item Given the return address of the code that raises the exception by calling \funcname{caml\_raise\_exn} (\funcarg{pc} argument of \funcname{caml\_stash\_backtrace})
        \item And the position of the stack pointer (\funcarg{sp} argument of \funcname{caml\_stash\_backtrace})
        \item The frame size given by \typename{frame\_descr}.\funcarg{frame\_size}, allows to find the end of the previous frame
        \item And the return address of the caller of this function
        \bigskip
        \onslide<3->{
        \item Note that traps (here the reddish \funcname{ExnA}, \funcname{ExnB} and \funcname{ExnC} blocks) belong to the frame that installed them, and so the frame size changes when traps are removed
        }
      \end{itemize}
    \end{column}
    \begin{column}{0.5\textwidth}
      \raggedleft
      \onslide*<1>{\providecommand\step{2}\input{diagrams/backtrace/backtrace.tex}}%
      \onslide*<2>{\providecommand\step{1}\input{diagrams/backtrace/scanstack.tex}}%
      \onslide*<3>{\providecommand\step{2}\input{diagrams/backtrace/scanstack.tex}}%
      \onslide*<4>{\providecommand\step{3}\input{diagrams/backtrace/scanstack.tex}}%
      \onslide*<5>{\providecommand\step{4}\input{diagrams/backtrace/scanstack.tex}}%
      \onslide*<6>{\providecommand\step{5}\input{diagrams/backtrace/scanstack.tex}}%
    \end{column}
  \end{columns}
\end{frame}

% Duplicate of previous-previous slide
\begin{frame}{Backtraces}{Continuing on \funcname{caml\_stash\_backtrace}}
  \begin{columns}[c]
    \begin{column}{0.5\textwidth}
      \minipage[c][0.75\textheight][s]{\columnwidth}
      \begin{itemize}
          \color{gray}
        \item A C function provided by the runtime (specific to native code)
        \item It scans the stack, between the stack pointer at the raising point up to the next trap, in order to collect the names of the detected function frames
        \item It uses the same technique and information as the Garbage Collector (GC) uses to scan the values live on the stack
      \end{itemize}
      \begin{itemize}
        \item For each function frame detected, store a pointer to the \typename{frame\_descr} in the \localname{domain\_state->backtrace\_buffer} array, effectively constructing the backtrace
      \end{itemize}
      \onslide<2->{
        \begin{itemize}
            \smallskip
          \item Constructed backtrace:
            \funcname{definitely\_raise},
            \onslide<3->{\funcname{probably\_raise}}
        \end{itemize}
      }
      \vfill
      \mintocaml[firstline=9,lastline=13,highlightlines=12]{../src/backtrace/backtrace.ml}
      \endminipage
    \end{column}
    \begin{column}{0.5\textwidth}
      \raggedleft
      \onslide*<1>{\listStashBacktrace[highlightlines={114-115}]{}}
      \onslide*<2>{\providecommand\step{3}\input{diagrams/backtrace/backtrace.tex}}%
      \onslide*<3>{\providecommand\step{4}\input{diagrams/backtrace/backtrace.tex}}%
    \end{column}
  \end{columns}
\end{frame}

\begin{frame}{Backtraces}{Back to \funcname{caml\_raise\_exn}}
  \begin{columns}[c]
    \begin{column}{0.5\textwidth}
      \minipage[c][0.66\textheight][s]{\columnwidth}
      \begin{itemize}\color{gray}
        \item Checks wheteher exceptions backtrace should be recorded, by checking the \funcarg{domain\_state->backtrace\_active} value
        \item Switches to the C stack to be able to execute C code
        \item Calls \funcname{caml\_stash\_backtrace}, to collect the backtrace up to the next trap
      \end{itemize}
      \onslide*<2->{
        \begin{itemize}
          \item<2-> Executes the exception handler in \funcname{probably\_raise} with \funcname{RESTORE\_EXN\_HANDLER\_OCAML}
            \begin{itemize}
              \item Set \regname{RSP} to \localname{domain\_state->exn\_handler} value
              \item Restore \localname{domain\_state->exn\_handler} from trap
              \item Jump to the handler code with \funcname{ret}
            \end{itemize}
        \end{itemize}
      }
      \vfill
      \onslide*<1>{\mintocaml[firstline=9,lastline=13,highlightlines=12]{../src/backtrace/backtrace.ml}}
      \onslide*<2->{\mintocaml[firstline=15,lastline=21,highlightlines={19-21}]{../src/backtrace/backtrace.ml}}
      \endminipage
    \end{column}
    \begin{column}{0.5\textwidth}
      \centering
      \onslide*<1>{
        \mintc[firstline=190,lastline=192]{../ocaml/runtime/amd64.S}
        \mintc[firstline=234,lastline=237]{../ocaml/runtime/amd64.S}
      }
      \onslide*<2>{
        \mintc[firstline=190,lastline=192,highlightlines={191-192}]{../ocaml/runtime/amd64.S}
        \mintc[firstline=234,lastline=237,highlightlines={235-237}]{../ocaml/runtime/amd64.S}
      }
      \onslide*<1>{\listCamlRaiseExn[highlightlines=720]{}}
      \onslide*<2>{\listCamlRaiseExn[highlightlines=722]{}}
      \onslide*<3->{\providecommand\step{5}\input{diagrams/backtrace/backtrace.tex}}
    \end{column}
  \end{columns}
\end{frame}

\begin{frame}{Backtraces}{\funcname{caml\_probably\_raise}'s handler doesn't catch \typename{ExnC}}
  \begin{columns}[c]
    \begin{column}{0.45\textwidth}
      \minipage[c][0.75\textheight][s]{\columnwidth}
      \begin{itemize}
        \item<1-> \funcname{match \dots with}\footnotemark pattern matching can also match exceptions
        \item<1-> The CMM of \funcname{probably\_raise} shows that this \funcname{match \dots with} is implemented with a pattern matching and a \funcname{try \dots with}
        \item<1-> But this one only matches on \typename{ExnA} exception
        \item<2-> Since the pattern matching fails to catch the exception, \funcname{reraise} is called with our current \typename{ExnC} exception
        \item<3-> Disassembly shows that \funcname{reraise} is actually a call to \funcname{caml\_reraise\_exn}, which is also an assembly function provided by the runtime
      \end{itemize}
      \vfill
      \mintocaml[firstline=15,lastline=21,highlightlines={18-21}]{../src/backtrace/backtrace.ml}
      \endminipage
    \end{column}
    \begin{column}{0.55\textwidth}
      \centering
      \onslide*<1>{\mintcmm[highlightlines={10-18}]{../src/backtrace/camlBacktrace\_\_probably\_raise\_351.cmm}}
      \onslide*<2>{\listcmm[highlightlines=18]{../src/backtrace/camlBacktrace\_\_probably\_raise_351.cmm}{CMM of probably\_raise}}
      \onslide*<3>{\mintobjdump[firstline=5,lastline=35,highlightlines=31]{../src/backtrace/camlBacktrace\_\_probably\_raise_351.objdump}}
    \end{column}
  \end{columns}
  \only<1>{\footnotetext[1]{https://blog.janestreet.com/pattern-matching-and-exception-handling-unite/}}
\end{frame}

\newcommand\listCamlReraiseExn[1][]{
  \listasm[firstline=727,lastline=734,#1]{../ocaml/runtime/amd64.S}{runtime/amd64.S:727}}

\begin{frame}{Backtraces}{Introducing \funcname{caml\_reraise\_exn}}
  \begin{columns}[c]
    \begin{column}{0.5\textwidth}
      \minipage[c][0.66\textheight][s]{\columnwidth}
      \begin{itemize}
        \item<1-> The exception have to be reraised to the next handler
        \item<2-> First, checks if the backtrace should be collected with \funcarg{domain\_state->backtrace\_active}
        \only<3>{
        \begin{itemize}
            \item If not, behaves like a \funcname{raise\_notrace} with \funcname{RESTORE\_EXN\_HANDLER\_OCAML}
        \end{itemize}
        }
        \item<4-> Jumps into \funcname{caml\_raise\_exn}
        \item<5-> Calls \funcname{caml\_stash\_backtrace} again, to scan the next part of the stack: from the trap just pop-ed to the next trap
      \end{itemize}
      \vfill
      \mintocaml[firstline=15,lastline=21,highlightlines={18-21}]{../src/backtrace/backtrace.ml}
      \endminipage
    \end{column}
    \begin{column}{0.5\textwidth}
      \raggedleft
      \onslide*<1>{\listCamlReraiseExn{}}
      \onslide*<2>{\listCamlReraiseExn[highlightlines=729]{}}
      \onslide*<3>{
        \mintc[firstline=190,lastline=192,highlightlines={191-192}]{../ocaml/runtime/amd64.S}
        \mintc[firstline=234,lastline=237,highlightlines={235-237}]{../ocaml/runtime/amd64.S}
        \medskip
        \listCamlReraiseExn[highlightlines=731]{}
      }
      \onslide*<4-5>{
        % TODO refactor with \listCamlReraiseExn
        \mintasm[firstline=727,lastline=734,highlightlines=730]{../ocaml/runtime/amd64.S}
        \medskip
      }
      \onslide*<4>{\listCamlRaiseExn[highlightlines=711]{}}
      \onslide*<5>{\listCamlRaiseExn[highlightlines=720]{}}
    \end{column}
  \end{columns}
\end{frame}

\begin{frame}{Backtraces}{Backtrace collection continues}
  \begin{columns}[c]
    \begin{column}{0.55\textwidth}
      \minipage[c][0.75\textheight][s]{\columnwidth}
      \begin{itemize}
        \item<1-> \funcname{caml\_stash\_backtrace} scans the older part of the stack, up to the next trap
        \item<6-> Controls jumps to the nested exception handler in \funcname{maybe\_raise}, which matches only on \typename{ExnB}
        \item<7-> \funcname{caml\_reraise\_exn} is called again, backtrace collection resumes with \funcname{caml\_stash\_backtrace}
      \end{itemize}
      \medskip
      \begin{itemize}
        \item<2-> Constructed backtrace:
        \begin{itemize}
          \item <2-> \funcname{definitely\_raise}
          \item <2-> \funcname{probably\_raise}
          \item <3-> \funcname{probably\_raise} (re-raise)
          \item <4-> \funcname{maybe\_raise}
          \item <5-> \funcname{main}
          \item <8-> \funcname{main} (re-raise)
        \end{itemize}
      \end{itemize}
      \vfill
      \onslide*<1-5>{\mintocaml[firstline=15,lastline=21]{../src/backtrace/backtrace.ml}}
      \onslide*<6>{\mintocaml[firstline=28,lastline=34,highlightlines=31]{../src/backtrace/backtrace.ml}}
      \onslide*<7->{\mintocaml[firstline=28,lastline=34,highlightlines=33]{../src/backtrace/backtrace.ml}}
      \endminipage
    \end{column}
    \begin{column}{0.45\textwidth}
      \raggedleft
      \onslide*<1>{\providecommand\step{6}\input{diagrams/backtrace/backtrace.tex}}%
      \onslide*<2>{\providecommand\step{6}\input{diagrams/backtrace/backtrace.tex}}%
      \onslide*<3>{\providecommand\step{7}\input{diagrams/backtrace/backtrace.tex}}%
      \onslide*<4>{\providecommand\step{8}\input{diagrams/backtrace/backtrace.tex}}%
      \onslide*<5>{\providecommand\step{9}\input{diagrams/backtrace/backtrace.tex}}%
      \onslide*<6>{\providecommand\step{10}\input{diagrams/backtrace/backtrace.tex}}%
      \onslide*<7>{\providecommand\step{11}\input{diagrams/backtrace/backtrace.tex}}%
      \onslide*<8>{\providecommand\step{12}\input{diagrams/backtrace/backtrace.tex}}%
      \onslide*<9>{\providecommand\step{13}\input{diagrams/backtrace/backtrace.tex}}%
    \end{column}
  \end{columns}
\end{frame}

\begin{frame}{Backtraces}{Printing the backtrace}
  \begin{columns}[c]
    \begin{column}{0.45\textwidth}
      \minipage[c][0.66\textheight][s]{\columnwidth}
      \begin{itemize}
        \item<1-> The exception backtrace can now be printed to \funcarg{stdout} with \funcname{Printexc.print_backtrace}
        \item<2-> First the exception backtrace is retrieved into an OCaml array with \funcname{get_raw_backtrace }
        \item<3-> Which is implemented as the \funcname{caml_get_exception_raw_backtrace} C function in the runtime
        \item<4-> And finally printed with \funcname{print_exception_backtrace}
      \end{itemize}
      \vfill
      \mintocaml[firstline=28,lastline=34,highlightlines=33]{../src/backtrace/backtrace.ml}
      \endminipage
    \end{column}
    \begin{column}{0.55\textwidth}
      \onslide*<1,2>{
        \mintocaml[firstline=100,lastline=101]{../ocaml/stdlib/printexc.ml}
      }
      \only<1>{
        \listocaml[firstline=170,lastline=172]{../ocaml/stdlib/printexc.ml}{stdlib/printexc.ml:170}
      }
      \only<2>{
        \listocaml[firstline=170,lastline=172,highlightlines=172]{../ocaml/stdlib/printexc.ml}{stdlib/printexc.ml:170}
      }
      \only<3>{
        \mintc[firstline=154,lastline=156]{../ocaml/runtime/backtrace.c}
        \listc[firstline=171,lastline=192,highlightlines={184-187}]{../ocaml/runtime/backtrace.c}{runtime/backtrace.c:154}
      }
      \only<4>{
        \listocaml[firstline=155,lastline=165]{../ocaml/stdlib/printexc.ml}{stdlib/printexc.ml:55}
      }
    \end{column}
  \end{columns}
\end{frame}


\subsection{The multiple kinds of raise}
\frameSubsection{}{}

\begin{frame}{Backtraces}{When backtraces are not needed}
  \begin{columns}[c]
    \begin{column}{0.45\textwidth}
      \minipage[c][0.66\textheight][s]{\columnwidth}
      \begin{itemize}
        \item<1-> What if the backtrace for an exception is never needed?
        \item<2-> It's possible to raise the exception explicitly with \funcname{raise\_notrace} to make raising as cheap as possible
        \item<3-> An example of use in the \funcname{dynlink} library of the OCaml compiler
      \end{itemize}
      \vfill
      \onslide<3->{
      \listocaml[firstline=205,lastline=209,highlightlines=207]{../ocaml/otherlibs/dynlink/dynlink\_compilerlibs/misc.ml}{otherlibs/dynlink/dynlink\_compilerlibs/misc.ml:205}
      }
      \endminipage
    \end{column}
    \begin{column}{0.55\textwidth}
    \only<4>{
      \mintcmm[highlightlines=3]{../src/notrace/camlNotrace\_\_fun_347.cmm}
      \listcmm{../src/notrace/camlNotrace\_\_all\_somes\_327.cmm}{all\_somes}
    }
    \onslide*<5->{
      \listobjdump[highlightlines={6-9}]{../src/notrace/camlNotrace\_\_fun_347.objdump}{anonymous function from \funcname{all\_somes}}
    }
    \end{column}
  \end{columns}
\end{frame}

\begin{frame}{Backtraces}{Summary table of the multiple kinds of raise}
  \begin{itemize}
    \item When debugging information is enabled and if the backtrace of an exception isn't needed, it's possible to raise with \funcname{raise\_notrace} for performance
    \item When debugging information is \emph{not} enabled, the compiler optimises \funcname{raise} calls to inline assembly (same effect as using \funcname{raise\_notrace})
  \end{itemize}
  \bigskip
  \centering
  \begin{tabular}{| l | c | c | c | c |}
    \hline
    \parbox{10em}{Debug information \\ (\funcarg{-g} flag)}  & \multicolumn{2}{|c|}{Disabled}                                     & \multicolumn{2}{|c|}{Enabled} \\
    \hline
    Raise kind                                               & \funcname{raise} & \funcname{raise\_notrace}                       & \funcname{raise} & \funcname{raise\_notrace} \\
    \hline
    Compilation                                              & \parbox{8em}{inline assembly\\ (optimization)} & inline assembly   & \funcname{caml\_raise\_exn} & inline assembly \\
    \hline
  \end{tabular}
\end{frame}


%
% Takeaway #5
%
\subsection*{Takeaway \#5}
\frameSubsectionTakeaway{}
\againframe<5>{takeaway}
}%
      \onslide*<8>{\providecommand\step{12}%
% Backtraces
%
\subsection{One advantage of raising with debugging information}
\frameSubsection{}{}

\begin{frame}{Backtraces}{Debugging information \& source code}
  \begin{columns}[c]
    \begin{column}{0.5\textwidth}
      \begin{itemize}
        \item Raising an exception is fun and games, but how do we know where the exception originated? $\rightarrow$ With the backtrace associated with the exception!
        \item In order to get a backtrace from the point where an exception was raised, it's necessary to compile with debugging information enabled (\funcarg{-g} flag for the compiler)
        \item Same example as in the `Nested handlers' section
      \end{itemize}
    \end{column}
    \begin{column}{0.5\textwidth}
      \listocaml[firstline=9,lastline=34]{../src/backtrace/backtrace.ml}{backtrace/backtrace.ml}
    \end{column}
  \end{columns}
\end{frame}

\begin{frame}{Backtraces}{CMM and disassembly of \funcname{definitely\_raise}}
  \begin{columns}[c]
    \begin{column}{0.5\textwidth}
      \minipage[c][0.66\textheight][s]{\columnwidth}
      \begin{itemize}
        \item<1-> An effect of compiling with debugging information (\funcarg{-g}): the CMM no longer uses \funcname{raise\_notrace} to raise the exception but \funcname{raise} instead
        \item<2-> Disassembly shows that CMM \funcname{raise} is actually a call to \funcname{caml\_raise\_exn}, a function in assembler provided by the runtime
      \end{itemize}
      \vfill
      \mintocaml[firstline=9,lastline=13,highlightlines=12]{../src/backtrace/backtrace.ml}
      \endminipage
    \end{column}
    \begin{column}{0.5\textwidth}
      \onslide*<1>{
        \mintcmm[highlightlines=5]{../src/backtrace/camlBacktrace\_\_definitely\_raise\_347.cmm}
      }
      \onslide*<2>{
        \mintobjdump[firstline=2,highlightlines=14]{../src/backtrace/camlBacktrace\_\_definitely\_raise\_347.objdump}
      }
    \end{column}
  \end{columns}
\end{frame}

\begin{frame}{Backtraces}{Introducing \funcname{caml\_raise\_exn}}
  \begin{columns}[c]
    \begin{column}{0.5\textwidth}
      \minipage[c][0.66\textheight][s]{\columnwidth}
      \onslide*<1>{
        \begin{itemize}
          \item Raising the exception is performed using \funcname{caml\_raise\_exn} which is
            \begin{itemize}
              \item An assembly function provided by the runtime
              \item Architecture specific
            \end{itemize}
          \item Let's see what it does differently from \funcname{raise\_notrace}\dots
          \item We are about to raise \typename{ExnC} with \funcname{caml\_raise\_exn} from \funcname{definitely\_raise}
        \end{itemize}
      }
      \onslide*<2>{
        \mintobjdump[firstline=2,highlightlines=14]{../src/backtrace/camlBacktrace\_\_definitely\_raise\_347.objdump}
      }
      \vfill
      \mintocaml[firstline=9,lastline=13,highlightlines=12]{../src/backtrace/backtrace.ml}
      \endminipage
    \end{column}
    \begin{column}{0.5\textwidth}
      \centering
      \providecommand\step{1}\input{diagrams/backtrace/backtrace.tex}
    \end{column}
  \end{columns}
\end{frame}

\begin{frame}{Backtraces}{But before that, we need more fields from \typename{caml\_domain\_state}}
  \begin{columns}
    \begin{column}{0.5\textwidth}
      \begin{itemize}
        \item Remember \typename{caml\_domain\_state} the per-domain runtime state?
        \item Some other members are of interest to us now\dots
        \item \localname{backtrace\_active} is used to know if the runtime should collect backtraces
        \item \localname{backtrace\_active} can be set by
          \begin{itemize}
            \item The \funcarg{b} flag in the \funcname{OCAMLRUNPARAM} environment variable
            \item \mintinline{ocaml}{Printexc.record_backtrace : bool -> unit} \footnotemark
          \end{itemize}
      \end{itemize}
    \end{column}
    \begin{column}{0.5\textwidth}
      \begin{listing}
        \mintc[highlightlines={9-11}]{src/ocaml/domain\_state\_backtrace.h}
        \caption{runtime/caml/domain\_state.h}
      \end{listing}
    \end{column}
  \end{columns}
  \footnotetext[1]{https://v2.ocaml.org/api/Printexc.html}
\end{frame}

\newcommand\listCamlRaiseExn[1][]{
  \listasm[firstline=702,lastline=725,#1]{../ocaml/runtime/amd64.S}{runtime/amd64.S:702}}

\begin{frame}{Backtraces}{Walk through \funcname{caml\_raise\_exn}}
  \begin{columns}[T]
    \begin{column}{0.5\textwidth}
      \begin{itemize}
        \item<1-> First checks whether exceptions backtrace should be recorded
        \item<1-> By checking the \funcarg{domain\_state->backtrace\_active} value
        \item<2-> If not, acts as \funcname{raise\_notrace} with \funcname{RESTORE\_EXN\_HANDLER\_OCAML}
          \begin{itemize}
            \item Set \regname{RSP} to \localname{domain\_state->exn\_handler} value
            \item Restore \localname{domain\_state->exn\_handler} from trap
            \item Jump with \funcname{ret} (same effect as using \funcname{pop} and \funcname{jmp})
          \end{itemize}
        \item<3-> Otherwise, switches to the C stack to be able to execute C code
        \item<4-> Calls \funcname{caml\_stash\_backtrace}, a C function provided by the runtime\ignorespaces
          \onslide<6->{, with
            \begin{itemize}
              \item \funcarg{exn}: exception value
              \item \funcarg{pc}: code address of the raising point
              \item \funcarg{sp}: stack address of the raising function
              \item \funcarg{trapsp}: stack address of the next handler
            \end{itemize}
          }
      \end{itemize}
      \bigskip
      \mintocaml[firstline=9,lastline=13,highlightlines=12]{../src/backtrace/backtrace.ml}
    \end{column}
    \begin{column}{0.5\textwidth}
      \raggedleft
      \onslide*<1,3-4>{
        \mintc[firstline=190,lastline=192]{../ocaml/runtime/amd64.S}
        \mintc[firstline=234,lastline=237]{../ocaml/runtime/amd64.S}
      }
      \onslide*<2>{
        \mintc[firstline=190,lastline=192,highlightlines={191-192}]{../ocaml/runtime/amd64.S}
        \mintc[firstline=234,lastline=237,highlightlines={235-237}]{../ocaml/runtime/amd64.S}
      }
      \onslide*<1>{\listCamlRaiseExn[highlightlines=705]{}}%
      \onslide*<2>{\listCamlRaiseExn[highlightlines={707-708}]{}}%
      \onslide*<3>{\listCamlRaiseExn[highlightlines={712-714}]{}}%
      \onslide*<4>{\listCamlRaiseExn[highlightlines={715-720}]{}}%
      \onslide*<5>{\providecommand\step{1}\input{diagrams/backtrace/backtrace.tex}}%
      \onslide*<6>{\providecommand\step{2}\input{diagrams/backtrace/backtrace.tex}}%
    \end{column}
  \end{columns}
\end{frame}

\newcommand\listStashBacktrace[1][]{
  \listc[firstline=91,lastline=120,#1]{../ocaml/runtime/backtrace\_nat.c}{runtime/backtrace\_nat.c:91}}

\begin{frame}{Backtraces}{Introducing \funcname{caml\_stash\_backtrace}}
  \begin{columns}[c]
    \begin{column}{0.5\textwidth}
      \minipage[c][0.66\textheight][s]{\columnwidth}
      \begin{itemize}
        \item<1-> A C function provided by the runtime (specific to native code)
        \item<2-> It scans the stack, between the stack pointer at the raising point up to the next trap, in order to collect the names of the detected function frames
          \onslide*<2>{
            \begin{itemize}
              \item And more information, like line number, column number...
            \end{itemize}
          }
        \item<3-> It uses the same technique and information as the Garbage Collector (GC) uses to scan the values live on the stack
          \onslide*<4>{
            \begin{itemize}
              \item \typename{frame\_descr} are at the core of stack unwinding
            \end{itemize}
          }
      \end{itemize}
      \vfill
      \mintocaml[firstline=9,lastline=13,highlightlines=12]{../src/backtrace/backtrace.ml}
      \endminipage
    \end{column}
    \begin{column}{0.5\textwidth}
      \onslide*<1>{\listStashBacktrace[highlightlines=91]{}}
      \onslide*<2>{\listStashBacktrace[highlightlines={91,117-118}]{}}
      \onslide*<3->{\listStashBacktrace[highlightlines={94,106,109-110}]{}}
    \end{column}
  \end{columns}
\end{frame}

\newcommand\listFrameDescr[1][]{
  \listocaml[firstline=111,lastline=124,#1]{../ocaml/asmcomp/emitaux.ml}{asmcomp/emitaux.ml:111}}
\newcommand\listDebugInfo[1][]{
  \listocaml[firstline=38,lastline=49,#1]{../ocaml/lambda/debuginfo.mli}{lambda/debuginfo.mli:38}}

\begin{frame}{Backtraces}{\typename{frame\_descr} and \typename{frame\_debuginfo} in a nutshell}
  \begin{columns}[c]
    \begin{column}{0.5\textwidth}
      \begin{itemize}
        \item<1-> For every return address in an OCaml program, there is a associated \typename{frame\_descr}
        \item<2-> A \typename{frame\_descr} specifies
        \begin{itemize}
          \item<2-> The size of the function stack frame
          \item<3-> Where live values are on the stack (so that the GC can mark them as live and prevent from being swept)
        \end{itemize}
        \item<4-> When compiled with debug information, \typename{frame\_descr} will be linked to a \typename{frame\_debuginfo}
        \begin{itemize}
          \item<5-> Contains information about the position in the source code (file name, line and column number)
        \end{itemize}
      \end{itemize}
    \end{column}
    \begin{column}{0.5\textwidth}
        \onslide*<1>{\listFrameDescr[highlightlines={118-122}]{}}
        \onslide*<2>{\listFrameDescr[highlightlines={120}]{}}
        \onslide*<3>{\listFrameDescr[highlightlines={121}]{}}
        \onslide*<4->{\listFrameDescr[highlightlines={122}]{}}
        \medskip
        \onslide*<1-3>{\listDebugInfo{}}
        \onslide*<4>{\listDebugInfo[highlightlines={38,49}]{}}
        \onslide*<5>{\listDebugInfo[highlightlines={39-46}]{}}
    \end{column}
  \end{columns}
\end{frame}

\begin{frame}{Backtraces}{Scanning the stack with \typename{frame\_descr}}
  \begin{columns}[c]
    \begin{column}{0.5\textwidth}
      \begin{itemize}
        \item Given the return address of the code that raises the exception by calling \funcname{caml\_raise\_exn} (\funcarg{pc} argument of \funcname{caml\_stash\_backtrace})
        \item And the position of the stack pointer (\funcarg{sp} argument of \funcname{caml\_stash\_backtrace})
        \item The frame size given by \typename{frame\_descr}.\funcarg{frame\_size}, allows to find the end of the previous frame
        \item And the return address of the caller of this function
        \bigskip
        \onslide<3->{
        \item Note that traps (here the reddish \funcname{ExnA}, \funcname{ExnB} and \funcname{ExnC} blocks) belong to the frame that installed them, and so the frame size changes when traps are removed
        }
      \end{itemize}
    \end{column}
    \begin{column}{0.5\textwidth}
      \raggedleft
      \onslide*<1>{\providecommand\step{2}\input{diagrams/backtrace/backtrace.tex}}%
      \onslide*<2>{\providecommand\step{1}\input{diagrams/backtrace/scanstack.tex}}%
      \onslide*<3>{\providecommand\step{2}\input{diagrams/backtrace/scanstack.tex}}%
      \onslide*<4>{\providecommand\step{3}\input{diagrams/backtrace/scanstack.tex}}%
      \onslide*<5>{\providecommand\step{4}\input{diagrams/backtrace/scanstack.tex}}%
      \onslide*<6>{\providecommand\step{5}\input{diagrams/backtrace/scanstack.tex}}%
    \end{column}
  \end{columns}
\end{frame}

% Duplicate of previous-previous slide
\begin{frame}{Backtraces}{Continuing on \funcname{caml\_stash\_backtrace}}
  \begin{columns}[c]
    \begin{column}{0.5\textwidth}
      \minipage[c][0.75\textheight][s]{\columnwidth}
      \begin{itemize}
          \color{gray}
        \item A C function provided by the runtime (specific to native code)
        \item It scans the stack, between the stack pointer at the raising point up to the next trap, in order to collect the names of the detected function frames
        \item It uses the same technique and information as the Garbage Collector (GC) uses to scan the values live on the stack
      \end{itemize}
      \begin{itemize}
        \item For each function frame detected, store a pointer to the \typename{frame\_descr} in the \localname{domain\_state->backtrace\_buffer} array, effectively constructing the backtrace
      \end{itemize}
      \onslide<2->{
        \begin{itemize}
            \smallskip
          \item Constructed backtrace:
            \funcname{definitely\_raise},
            \onslide<3->{\funcname{probably\_raise}}
        \end{itemize}
      }
      \vfill
      \mintocaml[firstline=9,lastline=13,highlightlines=12]{../src/backtrace/backtrace.ml}
      \endminipage
    \end{column}
    \begin{column}{0.5\textwidth}
      \raggedleft
      \onslide*<1>{\listStashBacktrace[highlightlines={114-115}]{}}
      \onslide*<2>{\providecommand\step{3}\input{diagrams/backtrace/backtrace.tex}}%
      \onslide*<3>{\providecommand\step{4}\input{diagrams/backtrace/backtrace.tex}}%
    \end{column}
  \end{columns}
\end{frame}

\begin{frame}{Backtraces}{Back to \funcname{caml\_raise\_exn}}
  \begin{columns}[c]
    \begin{column}{0.5\textwidth}
      \minipage[c][0.66\textheight][s]{\columnwidth}
      \begin{itemize}\color{gray}
        \item Checks wheteher exceptions backtrace should be recorded, by checking the \funcarg{domain\_state->backtrace\_active} value
        \item Switches to the C stack to be able to execute C code
        \item Calls \funcname{caml\_stash\_backtrace}, to collect the backtrace up to the next trap
      \end{itemize}
      \onslide*<2->{
        \begin{itemize}
          \item<2-> Executes the exception handler in \funcname{probably\_raise} with \funcname{RESTORE\_EXN\_HANDLER\_OCAML}
            \begin{itemize}
              \item Set \regname{RSP} to \localname{domain\_state->exn\_handler} value
              \item Restore \localname{domain\_state->exn\_handler} from trap
              \item Jump to the handler code with \funcname{ret}
            \end{itemize}
        \end{itemize}
      }
      \vfill
      \onslide*<1>{\mintocaml[firstline=9,lastline=13,highlightlines=12]{../src/backtrace/backtrace.ml}}
      \onslide*<2->{\mintocaml[firstline=15,lastline=21,highlightlines={19-21}]{../src/backtrace/backtrace.ml}}
      \endminipage
    \end{column}
    \begin{column}{0.5\textwidth}
      \centering
      \onslide*<1>{
        \mintc[firstline=190,lastline=192]{../ocaml/runtime/amd64.S}
        \mintc[firstline=234,lastline=237]{../ocaml/runtime/amd64.S}
      }
      \onslide*<2>{
        \mintc[firstline=190,lastline=192,highlightlines={191-192}]{../ocaml/runtime/amd64.S}
        \mintc[firstline=234,lastline=237,highlightlines={235-237}]{../ocaml/runtime/amd64.S}
      }
      \onslide*<1>{\listCamlRaiseExn[highlightlines=720]{}}
      \onslide*<2>{\listCamlRaiseExn[highlightlines=722]{}}
      \onslide*<3->{\providecommand\step{5}\input{diagrams/backtrace/backtrace.tex}}
    \end{column}
  \end{columns}
\end{frame}

\begin{frame}{Backtraces}{\funcname{caml\_probably\_raise}'s handler doesn't catch \typename{ExnC}}
  \begin{columns}[c]
    \begin{column}{0.45\textwidth}
      \minipage[c][0.75\textheight][s]{\columnwidth}
      \begin{itemize}
        \item<1-> \funcname{match \dots with}\footnotemark pattern matching can also match exceptions
        \item<1-> The CMM of \funcname{probably\_raise} shows that this \funcname{match \dots with} is implemented with a pattern matching and a \funcname{try \dots with}
        \item<1-> But this one only matches on \typename{ExnA} exception
        \item<2-> Since the pattern matching fails to catch the exception, \funcname{reraise} is called with our current \typename{ExnC} exception
        \item<3-> Disassembly shows that \funcname{reraise} is actually a call to \funcname{caml\_reraise\_exn}, which is also an assembly function provided by the runtime
      \end{itemize}
      \vfill
      \mintocaml[firstline=15,lastline=21,highlightlines={18-21}]{../src/backtrace/backtrace.ml}
      \endminipage
    \end{column}
    \begin{column}{0.55\textwidth}
      \centering
      \onslide*<1>{\mintcmm[highlightlines={10-18}]{../src/backtrace/camlBacktrace\_\_probably\_raise\_351.cmm}}
      \onslide*<2>{\listcmm[highlightlines=18]{../src/backtrace/camlBacktrace\_\_probably\_raise_351.cmm}{CMM of probably\_raise}}
      \onslide*<3>{\mintobjdump[firstline=5,lastline=35,highlightlines=31]{../src/backtrace/camlBacktrace\_\_probably\_raise_351.objdump}}
    \end{column}
  \end{columns}
  \only<1>{\footnotetext[1]{https://blog.janestreet.com/pattern-matching-and-exception-handling-unite/}}
\end{frame}

\newcommand\listCamlReraiseExn[1][]{
  \listasm[firstline=727,lastline=734,#1]{../ocaml/runtime/amd64.S}{runtime/amd64.S:727}}

\begin{frame}{Backtraces}{Introducing \funcname{caml\_reraise\_exn}}
  \begin{columns}[c]
    \begin{column}{0.5\textwidth}
      \minipage[c][0.66\textheight][s]{\columnwidth}
      \begin{itemize}
        \item<1-> The exception have to be reraised to the next handler
        \item<2-> First, checks if the backtrace should be collected with \funcarg{domain\_state->backtrace\_active}
        \only<3>{
        \begin{itemize}
            \item If not, behaves like a \funcname{raise\_notrace} with \funcname{RESTORE\_EXN\_HANDLER\_OCAML}
        \end{itemize}
        }
        \item<4-> Jumps into \funcname{caml\_raise\_exn}
        \item<5-> Calls \funcname{caml\_stash\_backtrace} again, to scan the next part of the stack: from the trap just pop-ed to the next trap
      \end{itemize}
      \vfill
      \mintocaml[firstline=15,lastline=21,highlightlines={18-21}]{../src/backtrace/backtrace.ml}
      \endminipage
    \end{column}
    \begin{column}{0.5\textwidth}
      \raggedleft
      \onslide*<1>{\listCamlReraiseExn{}}
      \onslide*<2>{\listCamlReraiseExn[highlightlines=729]{}}
      \onslide*<3>{
        \mintc[firstline=190,lastline=192,highlightlines={191-192}]{../ocaml/runtime/amd64.S}
        \mintc[firstline=234,lastline=237,highlightlines={235-237}]{../ocaml/runtime/amd64.S}
        \medskip
        \listCamlReraiseExn[highlightlines=731]{}
      }
      \onslide*<4-5>{
        % TODO refactor with \listCamlReraiseExn
        \mintasm[firstline=727,lastline=734,highlightlines=730]{../ocaml/runtime/amd64.S}
        \medskip
      }
      \onslide*<4>{\listCamlRaiseExn[highlightlines=711]{}}
      \onslide*<5>{\listCamlRaiseExn[highlightlines=720]{}}
    \end{column}
  \end{columns}
\end{frame}

\begin{frame}{Backtraces}{Backtrace collection continues}
  \begin{columns}[c]
    \begin{column}{0.55\textwidth}
      \minipage[c][0.75\textheight][s]{\columnwidth}
      \begin{itemize}
        \item<1-> \funcname{caml\_stash\_backtrace} scans the older part of the stack, up to the next trap
        \item<6-> Controls jumps to the nested exception handler in \funcname{maybe\_raise}, which matches only on \typename{ExnB}
        \item<7-> \funcname{caml\_reraise\_exn} is called again, backtrace collection resumes with \funcname{caml\_stash\_backtrace}
      \end{itemize}
      \medskip
      \begin{itemize}
        \item<2-> Constructed backtrace:
        \begin{itemize}
          \item <2-> \funcname{definitely\_raise}
          \item <2-> \funcname{probably\_raise}
          \item <3-> \funcname{probably\_raise} (re-raise)
          \item <4-> \funcname{maybe\_raise}
          \item <5-> \funcname{main}
          \item <8-> \funcname{main} (re-raise)
        \end{itemize}
      \end{itemize}
      \vfill
      \onslide*<1-5>{\mintocaml[firstline=15,lastline=21]{../src/backtrace/backtrace.ml}}
      \onslide*<6>{\mintocaml[firstline=28,lastline=34,highlightlines=31]{../src/backtrace/backtrace.ml}}
      \onslide*<7->{\mintocaml[firstline=28,lastline=34,highlightlines=33]{../src/backtrace/backtrace.ml}}
      \endminipage
    \end{column}
    \begin{column}{0.45\textwidth}
      \raggedleft
      \onslide*<1>{\providecommand\step{6}\input{diagrams/backtrace/backtrace.tex}}%
      \onslide*<2>{\providecommand\step{6}\input{diagrams/backtrace/backtrace.tex}}%
      \onslide*<3>{\providecommand\step{7}\input{diagrams/backtrace/backtrace.tex}}%
      \onslide*<4>{\providecommand\step{8}\input{diagrams/backtrace/backtrace.tex}}%
      \onslide*<5>{\providecommand\step{9}\input{diagrams/backtrace/backtrace.tex}}%
      \onslide*<6>{\providecommand\step{10}\input{diagrams/backtrace/backtrace.tex}}%
      \onslide*<7>{\providecommand\step{11}\input{diagrams/backtrace/backtrace.tex}}%
      \onslide*<8>{\providecommand\step{12}\input{diagrams/backtrace/backtrace.tex}}%
      \onslide*<9>{\providecommand\step{13}\input{diagrams/backtrace/backtrace.tex}}%
    \end{column}
  \end{columns}
\end{frame}

\begin{frame}{Backtraces}{Printing the backtrace}
  \begin{columns}[c]
    \begin{column}{0.45\textwidth}
      \minipage[c][0.66\textheight][s]{\columnwidth}
      \begin{itemize}
        \item<1-> The exception backtrace can now be printed to \funcarg{stdout} with \funcname{Printexc.print_backtrace}
        \item<2-> First the exception backtrace is retrieved into an OCaml array with \funcname{get_raw_backtrace }
        \item<3-> Which is implemented as the \funcname{caml_get_exception_raw_backtrace} C function in the runtime
        \item<4-> And finally printed with \funcname{print_exception_backtrace}
      \end{itemize}
      \vfill
      \mintocaml[firstline=28,lastline=34,highlightlines=33]{../src/backtrace/backtrace.ml}
      \endminipage
    \end{column}
    \begin{column}{0.55\textwidth}
      \onslide*<1,2>{
        \mintocaml[firstline=100,lastline=101]{../ocaml/stdlib/printexc.ml}
      }
      \only<1>{
        \listocaml[firstline=170,lastline=172]{../ocaml/stdlib/printexc.ml}{stdlib/printexc.ml:170}
      }
      \only<2>{
        \listocaml[firstline=170,lastline=172,highlightlines=172]{../ocaml/stdlib/printexc.ml}{stdlib/printexc.ml:170}
      }
      \only<3>{
        \mintc[firstline=154,lastline=156]{../ocaml/runtime/backtrace.c}
        \listc[firstline=171,lastline=192,highlightlines={184-187}]{../ocaml/runtime/backtrace.c}{runtime/backtrace.c:154}
      }
      \only<4>{
        \listocaml[firstline=155,lastline=165]{../ocaml/stdlib/printexc.ml}{stdlib/printexc.ml:55}
      }
    \end{column}
  \end{columns}
\end{frame}


\subsection{The multiple kinds of raise}
\frameSubsection{}{}

\begin{frame}{Backtraces}{When backtraces are not needed}
  \begin{columns}[c]
    \begin{column}{0.45\textwidth}
      \minipage[c][0.66\textheight][s]{\columnwidth}
      \begin{itemize}
        \item<1-> What if the backtrace for an exception is never needed?
        \item<2-> It's possible to raise the exception explicitly with \funcname{raise\_notrace} to make raising as cheap as possible
        \item<3-> An example of use in the \funcname{dynlink} library of the OCaml compiler
      \end{itemize}
      \vfill
      \onslide<3->{
      \listocaml[firstline=205,lastline=209,highlightlines=207]{../ocaml/otherlibs/dynlink/dynlink\_compilerlibs/misc.ml}{otherlibs/dynlink/dynlink\_compilerlibs/misc.ml:205}
      }
      \endminipage
    \end{column}
    \begin{column}{0.55\textwidth}
    \only<4>{
      \mintcmm[highlightlines=3]{../src/notrace/camlNotrace\_\_fun_347.cmm}
      \listcmm{../src/notrace/camlNotrace\_\_all\_somes\_327.cmm}{all\_somes}
    }
    \onslide*<5->{
      \listobjdump[highlightlines={6-9}]{../src/notrace/camlNotrace\_\_fun_347.objdump}{anonymous function from \funcname{all\_somes}}
    }
    \end{column}
  \end{columns}
\end{frame}

\begin{frame}{Backtraces}{Summary table of the multiple kinds of raise}
  \begin{itemize}
    \item When debugging information is enabled and if the backtrace of an exception isn't needed, it's possible to raise with \funcname{raise\_notrace} for performance
    \item When debugging information is \emph{not} enabled, the compiler optimises \funcname{raise} calls to inline assembly (same effect as using \funcname{raise\_notrace})
  \end{itemize}
  \bigskip
  \centering
  \begin{tabular}{| l | c | c | c | c |}
    \hline
    \parbox{10em}{Debug information \\ (\funcarg{-g} flag)}  & \multicolumn{2}{|c|}{Disabled}                                     & \multicolumn{2}{|c|}{Enabled} \\
    \hline
    Raise kind                                               & \funcname{raise} & \funcname{raise\_notrace}                       & \funcname{raise} & \funcname{raise\_notrace} \\
    \hline
    Compilation                                              & \parbox{8em}{inline assembly\\ (optimization)} & inline assembly   & \funcname{caml\_raise\_exn} & inline assembly \\
    \hline
  \end{tabular}
\end{frame}


%
% Takeaway #5
%
\subsection*{Takeaway \#5}
\frameSubsectionTakeaway{}
\againframe<5>{takeaway}
}%
      \onslide*<9>{\providecommand\step{13}%
% Backtraces
%
\subsection{One advantage of raising with debugging information}
\frameSubsection{}{}

\begin{frame}{Backtraces}{Debugging information \& source code}
  \begin{columns}[c]
    \begin{column}{0.5\textwidth}
      \begin{itemize}
        \item Raising an exception is fun and games, but how do we know where the exception originated? $\rightarrow$ With the backtrace associated with the exception!
        \item In order to get a backtrace from the point where an exception was raised, it's necessary to compile with debugging information enabled (\funcarg{-g} flag for the compiler)
        \item Same example as in the `Nested handlers' section
      \end{itemize}
    \end{column}
    \begin{column}{0.5\textwidth}
      \listocaml[firstline=9,lastline=34]{../src/backtrace/backtrace.ml}{backtrace/backtrace.ml}
    \end{column}
  \end{columns}
\end{frame}

\begin{frame}{Backtraces}{CMM and disassembly of \funcname{definitely\_raise}}
  \begin{columns}[c]
    \begin{column}{0.5\textwidth}
      \minipage[c][0.66\textheight][s]{\columnwidth}
      \begin{itemize}
        \item<1-> An effect of compiling with debugging information (\funcarg{-g}): the CMM no longer uses \funcname{raise\_notrace} to raise the exception but \funcname{raise} instead
        \item<2-> Disassembly shows that CMM \funcname{raise} is actually a call to \funcname{caml\_raise\_exn}, a function in assembler provided by the runtime
      \end{itemize}
      \vfill
      \mintocaml[firstline=9,lastline=13,highlightlines=12]{../src/backtrace/backtrace.ml}
      \endminipage
    \end{column}
    \begin{column}{0.5\textwidth}
      \onslide*<1>{
        \mintcmm[highlightlines=5]{../src/backtrace/camlBacktrace\_\_definitely\_raise\_347.cmm}
      }
      \onslide*<2>{
        \mintobjdump[firstline=2,highlightlines=14]{../src/backtrace/camlBacktrace\_\_definitely\_raise\_347.objdump}
      }
    \end{column}
  \end{columns}
\end{frame}

\begin{frame}{Backtraces}{Introducing \funcname{caml\_raise\_exn}}
  \begin{columns}[c]
    \begin{column}{0.5\textwidth}
      \minipage[c][0.66\textheight][s]{\columnwidth}
      \onslide*<1>{
        \begin{itemize}
          \item Raising the exception is performed using \funcname{caml\_raise\_exn} which is
            \begin{itemize}
              \item An assembly function provided by the runtime
              \item Architecture specific
            \end{itemize}
          \item Let's see what it does differently from \funcname{raise\_notrace}\dots
          \item We are about to raise \typename{ExnC} with \funcname{caml\_raise\_exn} from \funcname{definitely\_raise}
        \end{itemize}
      }
      \onslide*<2>{
        \mintobjdump[firstline=2,highlightlines=14]{../src/backtrace/camlBacktrace\_\_definitely\_raise\_347.objdump}
      }
      \vfill
      \mintocaml[firstline=9,lastline=13,highlightlines=12]{../src/backtrace/backtrace.ml}
      \endminipage
    \end{column}
    \begin{column}{0.5\textwidth}
      \centering
      \providecommand\step{1}\input{diagrams/backtrace/backtrace.tex}
    \end{column}
  \end{columns}
\end{frame}

\begin{frame}{Backtraces}{But before that, we need more fields from \typename{caml\_domain\_state}}
  \begin{columns}
    \begin{column}{0.5\textwidth}
      \begin{itemize}
        \item Remember \typename{caml\_domain\_state} the per-domain runtime state?
        \item Some other members are of interest to us now\dots
        \item \localname{backtrace\_active} is used to know if the runtime should collect backtraces
        \item \localname{backtrace\_active} can be set by
          \begin{itemize}
            \item The \funcarg{b} flag in the \funcname{OCAMLRUNPARAM} environment variable
            \item \mintinline{ocaml}{Printexc.record_backtrace : bool -> unit} \footnotemark
          \end{itemize}
      \end{itemize}
    \end{column}
    \begin{column}{0.5\textwidth}
      \begin{listing}
        \mintc[highlightlines={9-11}]{src/ocaml/domain\_state\_backtrace.h}
        \caption{runtime/caml/domain\_state.h}
      \end{listing}
    \end{column}
  \end{columns}
  \footnotetext[1]{https://v2.ocaml.org/api/Printexc.html}
\end{frame}

\newcommand\listCamlRaiseExn[1][]{
  \listasm[firstline=702,lastline=725,#1]{../ocaml/runtime/amd64.S}{runtime/amd64.S:702}}

\begin{frame}{Backtraces}{Walk through \funcname{caml\_raise\_exn}}
  \begin{columns}[T]
    \begin{column}{0.5\textwidth}
      \begin{itemize}
        \item<1-> First checks whether exceptions backtrace should be recorded
        \item<1-> By checking the \funcarg{domain\_state->backtrace\_active} value
        \item<2-> If not, acts as \funcname{raise\_notrace} with \funcname{RESTORE\_EXN\_HANDLER\_OCAML}
          \begin{itemize}
            \item Set \regname{RSP} to \localname{domain\_state->exn\_handler} value
            \item Restore \localname{domain\_state->exn\_handler} from trap
            \item Jump with \funcname{ret} (same effect as using \funcname{pop} and \funcname{jmp})
          \end{itemize}
        \item<3-> Otherwise, switches to the C stack to be able to execute C code
        \item<4-> Calls \funcname{caml\_stash\_backtrace}, a C function provided by the runtime\ignorespaces
          \onslide<6->{, with
            \begin{itemize}
              \item \funcarg{exn}: exception value
              \item \funcarg{pc}: code address of the raising point
              \item \funcarg{sp}: stack address of the raising function
              \item \funcarg{trapsp}: stack address of the next handler
            \end{itemize}
          }
      \end{itemize}
      \bigskip
      \mintocaml[firstline=9,lastline=13,highlightlines=12]{../src/backtrace/backtrace.ml}
    \end{column}
    \begin{column}{0.5\textwidth}
      \raggedleft
      \onslide*<1,3-4>{
        \mintc[firstline=190,lastline=192]{../ocaml/runtime/amd64.S}
        \mintc[firstline=234,lastline=237]{../ocaml/runtime/amd64.S}
      }
      \onslide*<2>{
        \mintc[firstline=190,lastline=192,highlightlines={191-192}]{../ocaml/runtime/amd64.S}
        \mintc[firstline=234,lastline=237,highlightlines={235-237}]{../ocaml/runtime/amd64.S}
      }
      \onslide*<1>{\listCamlRaiseExn[highlightlines=705]{}}%
      \onslide*<2>{\listCamlRaiseExn[highlightlines={707-708}]{}}%
      \onslide*<3>{\listCamlRaiseExn[highlightlines={712-714}]{}}%
      \onslide*<4>{\listCamlRaiseExn[highlightlines={715-720}]{}}%
      \onslide*<5>{\providecommand\step{1}\input{diagrams/backtrace/backtrace.tex}}%
      \onslide*<6>{\providecommand\step{2}\input{diagrams/backtrace/backtrace.tex}}%
    \end{column}
  \end{columns}
\end{frame}

\newcommand\listStashBacktrace[1][]{
  \listc[firstline=91,lastline=120,#1]{../ocaml/runtime/backtrace\_nat.c}{runtime/backtrace\_nat.c:91}}

\begin{frame}{Backtraces}{Introducing \funcname{caml\_stash\_backtrace}}
  \begin{columns}[c]
    \begin{column}{0.5\textwidth}
      \minipage[c][0.66\textheight][s]{\columnwidth}
      \begin{itemize}
        \item<1-> A C function provided by the runtime (specific to native code)
        \item<2-> It scans the stack, between the stack pointer at the raising point up to the next trap, in order to collect the names of the detected function frames
          \onslide*<2>{
            \begin{itemize}
              \item And more information, like line number, column number...
            \end{itemize}
          }
        \item<3-> It uses the same technique and information as the Garbage Collector (GC) uses to scan the values live on the stack
          \onslide*<4>{
            \begin{itemize}
              \item \typename{frame\_descr} are at the core of stack unwinding
            \end{itemize}
          }
      \end{itemize}
      \vfill
      \mintocaml[firstline=9,lastline=13,highlightlines=12]{../src/backtrace/backtrace.ml}
      \endminipage
    \end{column}
    \begin{column}{0.5\textwidth}
      \onslide*<1>{\listStashBacktrace[highlightlines=91]{}}
      \onslide*<2>{\listStashBacktrace[highlightlines={91,117-118}]{}}
      \onslide*<3->{\listStashBacktrace[highlightlines={94,106,109-110}]{}}
    \end{column}
  \end{columns}
\end{frame}

\newcommand\listFrameDescr[1][]{
  \listocaml[firstline=111,lastline=124,#1]{../ocaml/asmcomp/emitaux.ml}{asmcomp/emitaux.ml:111}}
\newcommand\listDebugInfo[1][]{
  \listocaml[firstline=38,lastline=49,#1]{../ocaml/lambda/debuginfo.mli}{lambda/debuginfo.mli:38}}

\begin{frame}{Backtraces}{\typename{frame\_descr} and \typename{frame\_debuginfo} in a nutshell}
  \begin{columns}[c]
    \begin{column}{0.5\textwidth}
      \begin{itemize}
        \item<1-> For every return address in an OCaml program, there is a associated \typename{frame\_descr}
        \item<2-> A \typename{frame\_descr} specifies
        \begin{itemize}
          \item<2-> The size of the function stack frame
          \item<3-> Where live values are on the stack (so that the GC can mark them as live and prevent from being swept)
        \end{itemize}
        \item<4-> When compiled with debug information, \typename{frame\_descr} will be linked to a \typename{frame\_debuginfo}
        \begin{itemize}
          \item<5-> Contains information about the position in the source code (file name, line and column number)
        \end{itemize}
      \end{itemize}
    \end{column}
    \begin{column}{0.5\textwidth}
        \onslide*<1>{\listFrameDescr[highlightlines={118-122}]{}}
        \onslide*<2>{\listFrameDescr[highlightlines={120}]{}}
        \onslide*<3>{\listFrameDescr[highlightlines={121}]{}}
        \onslide*<4->{\listFrameDescr[highlightlines={122}]{}}
        \medskip
        \onslide*<1-3>{\listDebugInfo{}}
        \onslide*<4>{\listDebugInfo[highlightlines={38,49}]{}}
        \onslide*<5>{\listDebugInfo[highlightlines={39-46}]{}}
    \end{column}
  \end{columns}
\end{frame}

\begin{frame}{Backtraces}{Scanning the stack with \typename{frame\_descr}}
  \begin{columns}[c]
    \begin{column}{0.5\textwidth}
      \begin{itemize}
        \item Given the return address of the code that raises the exception by calling \funcname{caml\_raise\_exn} (\funcarg{pc} argument of \funcname{caml\_stash\_backtrace})
        \item And the position of the stack pointer (\funcarg{sp} argument of \funcname{caml\_stash\_backtrace})
        \item The frame size given by \typename{frame\_descr}.\funcarg{frame\_size}, allows to find the end of the previous frame
        \item And the return address of the caller of this function
        \bigskip
        \onslide<3->{
        \item Note that traps (here the reddish \funcname{ExnA}, \funcname{ExnB} and \funcname{ExnC} blocks) belong to the frame that installed them, and so the frame size changes when traps are removed
        }
      \end{itemize}
    \end{column}
    \begin{column}{0.5\textwidth}
      \raggedleft
      \onslide*<1>{\providecommand\step{2}\input{diagrams/backtrace/backtrace.tex}}%
      \onslide*<2>{\providecommand\step{1}\input{diagrams/backtrace/scanstack.tex}}%
      \onslide*<3>{\providecommand\step{2}\input{diagrams/backtrace/scanstack.tex}}%
      \onslide*<4>{\providecommand\step{3}\input{diagrams/backtrace/scanstack.tex}}%
      \onslide*<5>{\providecommand\step{4}\input{diagrams/backtrace/scanstack.tex}}%
      \onslide*<6>{\providecommand\step{5}\input{diagrams/backtrace/scanstack.tex}}%
    \end{column}
  \end{columns}
\end{frame}

% Duplicate of previous-previous slide
\begin{frame}{Backtraces}{Continuing on \funcname{caml\_stash\_backtrace}}
  \begin{columns}[c]
    \begin{column}{0.5\textwidth}
      \minipage[c][0.75\textheight][s]{\columnwidth}
      \begin{itemize}
          \color{gray}
        \item A C function provided by the runtime (specific to native code)
        \item It scans the stack, between the stack pointer at the raising point up to the next trap, in order to collect the names of the detected function frames
        \item It uses the same technique and information as the Garbage Collector (GC) uses to scan the values live on the stack
      \end{itemize}
      \begin{itemize}
        \item For each function frame detected, store a pointer to the \typename{frame\_descr} in the \localname{domain\_state->backtrace\_buffer} array, effectively constructing the backtrace
      \end{itemize}
      \onslide<2->{
        \begin{itemize}
            \smallskip
          \item Constructed backtrace:
            \funcname{definitely\_raise},
            \onslide<3->{\funcname{probably\_raise}}
        \end{itemize}
      }
      \vfill
      \mintocaml[firstline=9,lastline=13,highlightlines=12]{../src/backtrace/backtrace.ml}
      \endminipage
    \end{column}
    \begin{column}{0.5\textwidth}
      \raggedleft
      \onslide*<1>{\listStashBacktrace[highlightlines={114-115}]{}}
      \onslide*<2>{\providecommand\step{3}\input{diagrams/backtrace/backtrace.tex}}%
      \onslide*<3>{\providecommand\step{4}\input{diagrams/backtrace/backtrace.tex}}%
    \end{column}
  \end{columns}
\end{frame}

\begin{frame}{Backtraces}{Back to \funcname{caml\_raise\_exn}}
  \begin{columns}[c]
    \begin{column}{0.5\textwidth}
      \minipage[c][0.66\textheight][s]{\columnwidth}
      \begin{itemize}\color{gray}
        \item Checks wheteher exceptions backtrace should be recorded, by checking the \funcarg{domain\_state->backtrace\_active} value
        \item Switches to the C stack to be able to execute C code
        \item Calls \funcname{caml\_stash\_backtrace}, to collect the backtrace up to the next trap
      \end{itemize}
      \onslide*<2->{
        \begin{itemize}
          \item<2-> Executes the exception handler in \funcname{probably\_raise} with \funcname{RESTORE\_EXN\_HANDLER\_OCAML}
            \begin{itemize}
              \item Set \regname{RSP} to \localname{domain\_state->exn\_handler} value
              \item Restore \localname{domain\_state->exn\_handler} from trap
              \item Jump to the handler code with \funcname{ret}
            \end{itemize}
        \end{itemize}
      }
      \vfill
      \onslide*<1>{\mintocaml[firstline=9,lastline=13,highlightlines=12]{../src/backtrace/backtrace.ml}}
      \onslide*<2->{\mintocaml[firstline=15,lastline=21,highlightlines={19-21}]{../src/backtrace/backtrace.ml}}
      \endminipage
    \end{column}
    \begin{column}{0.5\textwidth}
      \centering
      \onslide*<1>{
        \mintc[firstline=190,lastline=192]{../ocaml/runtime/amd64.S}
        \mintc[firstline=234,lastline=237]{../ocaml/runtime/amd64.S}
      }
      \onslide*<2>{
        \mintc[firstline=190,lastline=192,highlightlines={191-192}]{../ocaml/runtime/amd64.S}
        \mintc[firstline=234,lastline=237,highlightlines={235-237}]{../ocaml/runtime/amd64.S}
      }
      \onslide*<1>{\listCamlRaiseExn[highlightlines=720]{}}
      \onslide*<2>{\listCamlRaiseExn[highlightlines=722]{}}
      \onslide*<3->{\providecommand\step{5}\input{diagrams/backtrace/backtrace.tex}}
    \end{column}
  \end{columns}
\end{frame}

\begin{frame}{Backtraces}{\funcname{caml\_probably\_raise}'s handler doesn't catch \typename{ExnC}}
  \begin{columns}[c]
    \begin{column}{0.45\textwidth}
      \minipage[c][0.75\textheight][s]{\columnwidth}
      \begin{itemize}
        \item<1-> \funcname{match \dots with}\footnotemark pattern matching can also match exceptions
        \item<1-> The CMM of \funcname{probably\_raise} shows that this \funcname{match \dots with} is implemented with a pattern matching and a \funcname{try \dots with}
        \item<1-> But this one only matches on \typename{ExnA} exception
        \item<2-> Since the pattern matching fails to catch the exception, \funcname{reraise} is called with our current \typename{ExnC} exception
        \item<3-> Disassembly shows that \funcname{reraise} is actually a call to \funcname{caml\_reraise\_exn}, which is also an assembly function provided by the runtime
      \end{itemize}
      \vfill
      \mintocaml[firstline=15,lastline=21,highlightlines={18-21}]{../src/backtrace/backtrace.ml}
      \endminipage
    \end{column}
    \begin{column}{0.55\textwidth}
      \centering
      \onslide*<1>{\mintcmm[highlightlines={10-18}]{../src/backtrace/camlBacktrace\_\_probably\_raise\_351.cmm}}
      \onslide*<2>{\listcmm[highlightlines=18]{../src/backtrace/camlBacktrace\_\_probably\_raise_351.cmm}{CMM of probably\_raise}}
      \onslide*<3>{\mintobjdump[firstline=5,lastline=35,highlightlines=31]{../src/backtrace/camlBacktrace\_\_probably\_raise_351.objdump}}
    \end{column}
  \end{columns}
  \only<1>{\footnotetext[1]{https://blog.janestreet.com/pattern-matching-and-exception-handling-unite/}}
\end{frame}

\newcommand\listCamlReraiseExn[1][]{
  \listasm[firstline=727,lastline=734,#1]{../ocaml/runtime/amd64.S}{runtime/amd64.S:727}}

\begin{frame}{Backtraces}{Introducing \funcname{caml\_reraise\_exn}}
  \begin{columns}[c]
    \begin{column}{0.5\textwidth}
      \minipage[c][0.66\textheight][s]{\columnwidth}
      \begin{itemize}
        \item<1-> The exception have to be reraised to the next handler
        \item<2-> First, checks if the backtrace should be collected with \funcarg{domain\_state->backtrace\_active}
        \only<3>{
        \begin{itemize}
            \item If not, behaves like a \funcname{raise\_notrace} with \funcname{RESTORE\_EXN\_HANDLER\_OCAML}
        \end{itemize}
        }
        \item<4-> Jumps into \funcname{caml\_raise\_exn}
        \item<5-> Calls \funcname{caml\_stash\_backtrace} again, to scan the next part of the stack: from the trap just pop-ed to the next trap
      \end{itemize}
      \vfill
      \mintocaml[firstline=15,lastline=21,highlightlines={18-21}]{../src/backtrace/backtrace.ml}
      \endminipage
    \end{column}
    \begin{column}{0.5\textwidth}
      \raggedleft
      \onslide*<1>{\listCamlReraiseExn{}}
      \onslide*<2>{\listCamlReraiseExn[highlightlines=729]{}}
      \onslide*<3>{
        \mintc[firstline=190,lastline=192,highlightlines={191-192}]{../ocaml/runtime/amd64.S}
        \mintc[firstline=234,lastline=237,highlightlines={235-237}]{../ocaml/runtime/amd64.S}
        \medskip
        \listCamlReraiseExn[highlightlines=731]{}
      }
      \onslide*<4-5>{
        % TODO refactor with \listCamlReraiseExn
        \mintasm[firstline=727,lastline=734,highlightlines=730]{../ocaml/runtime/amd64.S}
        \medskip
      }
      \onslide*<4>{\listCamlRaiseExn[highlightlines=711]{}}
      \onslide*<5>{\listCamlRaiseExn[highlightlines=720]{}}
    \end{column}
  \end{columns}
\end{frame}

\begin{frame}{Backtraces}{Backtrace collection continues}
  \begin{columns}[c]
    \begin{column}{0.55\textwidth}
      \minipage[c][0.75\textheight][s]{\columnwidth}
      \begin{itemize}
        \item<1-> \funcname{caml\_stash\_backtrace} scans the older part of the stack, up to the next trap
        \item<6-> Controls jumps to the nested exception handler in \funcname{maybe\_raise}, which matches only on \typename{ExnB}
        \item<7-> \funcname{caml\_reraise\_exn} is called again, backtrace collection resumes with \funcname{caml\_stash\_backtrace}
      \end{itemize}
      \medskip
      \begin{itemize}
        \item<2-> Constructed backtrace:
        \begin{itemize}
          \item <2-> \funcname{definitely\_raise}
          \item <2-> \funcname{probably\_raise}
          \item <3-> \funcname{probably\_raise} (re-raise)
          \item <4-> \funcname{maybe\_raise}
          \item <5-> \funcname{main}
          \item <8-> \funcname{main} (re-raise)
        \end{itemize}
      \end{itemize}
      \vfill
      \onslide*<1-5>{\mintocaml[firstline=15,lastline=21]{../src/backtrace/backtrace.ml}}
      \onslide*<6>{\mintocaml[firstline=28,lastline=34,highlightlines=31]{../src/backtrace/backtrace.ml}}
      \onslide*<7->{\mintocaml[firstline=28,lastline=34,highlightlines=33]{../src/backtrace/backtrace.ml}}
      \endminipage
    \end{column}
    \begin{column}{0.45\textwidth}
      \raggedleft
      \onslide*<1>{\providecommand\step{6}\input{diagrams/backtrace/backtrace.tex}}%
      \onslide*<2>{\providecommand\step{6}\input{diagrams/backtrace/backtrace.tex}}%
      \onslide*<3>{\providecommand\step{7}\input{diagrams/backtrace/backtrace.tex}}%
      \onslide*<4>{\providecommand\step{8}\input{diagrams/backtrace/backtrace.tex}}%
      \onslide*<5>{\providecommand\step{9}\input{diagrams/backtrace/backtrace.tex}}%
      \onslide*<6>{\providecommand\step{10}\input{diagrams/backtrace/backtrace.tex}}%
      \onslide*<7>{\providecommand\step{11}\input{diagrams/backtrace/backtrace.tex}}%
      \onslide*<8>{\providecommand\step{12}\input{diagrams/backtrace/backtrace.tex}}%
      \onslide*<9>{\providecommand\step{13}\input{diagrams/backtrace/backtrace.tex}}%
    \end{column}
  \end{columns}
\end{frame}

\begin{frame}{Backtraces}{Printing the backtrace}
  \begin{columns}[c]
    \begin{column}{0.45\textwidth}
      \minipage[c][0.66\textheight][s]{\columnwidth}
      \begin{itemize}
        \item<1-> The exception backtrace can now be printed to \funcarg{stdout} with \funcname{Printexc.print_backtrace}
        \item<2-> First the exception backtrace is retrieved into an OCaml array with \funcname{get_raw_backtrace }
        \item<3-> Which is implemented as the \funcname{caml_get_exception_raw_backtrace} C function in the runtime
        \item<4-> And finally printed with \funcname{print_exception_backtrace}
      \end{itemize}
      \vfill
      \mintocaml[firstline=28,lastline=34,highlightlines=33]{../src/backtrace/backtrace.ml}
      \endminipage
    \end{column}
    \begin{column}{0.55\textwidth}
      \onslide*<1,2>{
        \mintocaml[firstline=100,lastline=101]{../ocaml/stdlib/printexc.ml}
      }
      \only<1>{
        \listocaml[firstline=170,lastline=172]{../ocaml/stdlib/printexc.ml}{stdlib/printexc.ml:170}
      }
      \only<2>{
        \listocaml[firstline=170,lastline=172,highlightlines=172]{../ocaml/stdlib/printexc.ml}{stdlib/printexc.ml:170}
      }
      \only<3>{
        \mintc[firstline=154,lastline=156]{../ocaml/runtime/backtrace.c}
        \listc[firstline=171,lastline=192,highlightlines={184-187}]{../ocaml/runtime/backtrace.c}{runtime/backtrace.c:154}
      }
      \only<4>{
        \listocaml[firstline=155,lastline=165]{../ocaml/stdlib/printexc.ml}{stdlib/printexc.ml:55}
      }
    \end{column}
  \end{columns}
\end{frame}


\subsection{The multiple kinds of raise}
\frameSubsection{}{}

\begin{frame}{Backtraces}{When backtraces are not needed}
  \begin{columns}[c]
    \begin{column}{0.45\textwidth}
      \minipage[c][0.66\textheight][s]{\columnwidth}
      \begin{itemize}
        \item<1-> What if the backtrace for an exception is never needed?
        \item<2-> It's possible to raise the exception explicitly with \funcname{raise\_notrace} to make raising as cheap as possible
        \item<3-> An example of use in the \funcname{dynlink} library of the OCaml compiler
      \end{itemize}
      \vfill
      \onslide<3->{
      \listocaml[firstline=205,lastline=209,highlightlines=207]{../ocaml/otherlibs/dynlink/dynlink\_compilerlibs/misc.ml}{otherlibs/dynlink/dynlink\_compilerlibs/misc.ml:205}
      }
      \endminipage
    \end{column}
    \begin{column}{0.55\textwidth}
    \only<4>{
      \mintcmm[highlightlines=3]{../src/notrace/camlNotrace\_\_fun_347.cmm}
      \listcmm{../src/notrace/camlNotrace\_\_all\_somes\_327.cmm}{all\_somes}
    }
    \onslide*<5->{
      \listobjdump[highlightlines={6-9}]{../src/notrace/camlNotrace\_\_fun_347.objdump}{anonymous function from \funcname{all\_somes}}
    }
    \end{column}
  \end{columns}
\end{frame}

\begin{frame}{Backtraces}{Summary table of the multiple kinds of raise}
  \begin{itemize}
    \item When debugging information is enabled and if the backtrace of an exception isn't needed, it's possible to raise with \funcname{raise\_notrace} for performance
    \item When debugging information is \emph{not} enabled, the compiler optimises \funcname{raise} calls to inline assembly (same effect as using \funcname{raise\_notrace})
  \end{itemize}
  \bigskip
  \centering
  \begin{tabular}{| l | c | c | c | c |}
    \hline
    \parbox{10em}{Debug information \\ (\funcarg{-g} flag)}  & \multicolumn{2}{|c|}{Disabled}                                     & \multicolumn{2}{|c|}{Enabled} \\
    \hline
    Raise kind                                               & \funcname{raise} & \funcname{raise\_notrace}                       & \funcname{raise} & \funcname{raise\_notrace} \\
    \hline
    Compilation                                              & \parbox{8em}{inline assembly\\ (optimization)} & inline assembly   & \funcname{caml\_raise\_exn} & inline assembly \\
    \hline
  \end{tabular}
\end{frame}


%
% Takeaway #5
%
\subsection*{Takeaway \#5}
\frameSubsectionTakeaway{}
\againframe<5>{takeaway}
}%
    \end{column}
  \end{columns}
\end{frame}

\begin{frame}{Backtraces}{Printing the backtrace}
  \begin{columns}[c]
    \begin{column}{0.45\textwidth}
      \minipage[c][0.66\textheight][s]{\columnwidth}
      \begin{itemize}
        \item<1-> The exception backtrace can now be printed to \funcarg{stdout} with \funcname{Printexc.print_backtrace}
        \item<2-> First the exception backtrace is retrieved into an OCaml array with \funcname{get_raw_backtrace }
        \item<3-> Which is implemented as the \funcname{caml_get_exception_raw_backtrace} C function in the runtime
        \item<4-> And finally printed with \funcname{print_exception_backtrace}
      \end{itemize}
      \vfill
      \mintocaml[firstline=28,lastline=34,highlightlines=33]{../src/backtrace/backtrace.ml}
      \endminipage
    \end{column}
    \begin{column}{0.55\textwidth}
      \onslide*<1,2>{
        \mintocaml[firstline=100,lastline=101]{../ocaml/stdlib/printexc.ml}
      }
      \only<1>{
        \listocaml[firstline=170,lastline=172]{../ocaml/stdlib/printexc.ml}{stdlib/printexc.ml:170}
      }
      \only<2>{
        \listocaml[firstline=170,lastline=172,highlightlines=172]{../ocaml/stdlib/printexc.ml}{stdlib/printexc.ml:170}
      }
      \only<3>{
        \mintc[firstline=154,lastline=156]{../ocaml/runtime/backtrace.c}
        \listc[firstline=171,lastline=192,highlightlines={184-187}]{../ocaml/runtime/backtrace.c}{runtime/backtrace.c:154}
      }
      \only<4>{
        \listocaml[firstline=155,lastline=165]{../ocaml/stdlib/printexc.ml}{stdlib/printexc.ml:55}
      }
    \end{column}
  \end{columns}
\end{frame}


\subsection{The multiple kinds of raise}
\frameSubsection{}{}

\begin{frame}{Backtraces}{When backtraces are not needed}
  \begin{columns}[c]
    \begin{column}{0.45\textwidth}
      \minipage[c][0.66\textheight][s]{\columnwidth}
      \begin{itemize}
        \item<1-> What if the backtrace for an exception is never needed?
        \item<2-> It's possible to raise the exception explicitly with \funcname{raise\_notrace} to make raising as cheap as possible
        \item<3-> An example of use in the \funcname{dynlink} library of the OCaml compiler
      \end{itemize}
      \vfill
      \onslide<3->{
      \listocaml[firstline=205,lastline=209,highlightlines=207]{../ocaml/otherlibs/dynlink/dynlink\_compilerlibs/misc.ml}{otherlibs/dynlink/dynlink\_compilerlibs/misc.ml:205}
      }
      \endminipage
    \end{column}
    \begin{column}{0.55\textwidth}
    \only<4>{
      \mintcmm[highlightlines=3]{../src/notrace/camlNotrace\_\_fun_347.cmm}
      \listcmm{../src/notrace/camlNotrace\_\_all\_somes\_327.cmm}{all\_somes}
    }
    \onslide*<5->{
      \listobjdump[highlightlines={6-9}]{../src/notrace/camlNotrace\_\_fun_347.objdump}{anonymous function from \funcname{all\_somes}}
    }
    \end{column}
  \end{columns}
\end{frame}

\begin{frame}{Backtraces}{Summary table of the multiple kinds of raise}
  \begin{itemize}
    \item When debugging information is enabled and if the backtrace of an exception isn't needed, it's possible to raise with \funcname{raise\_notrace} for performance
    \item When debugging information is \emph{not} enabled, the compiler optimises \funcname{raise} calls to inline assembly (same effect as using \funcname{raise\_notrace})
  \end{itemize}
  \bigskip
  \centering
  \begin{tabular}{| l | c | c | c | c |}
    \hline
    \parbox{10em}{Debug information \\ (\funcarg{-g} flag)}  & \multicolumn{2}{|c|}{Disabled}                                     & \multicolumn{2}{|c|}{Enabled} \\
    \hline
    Raise kind                                               & \funcname{raise} & \funcname{raise\_notrace}                       & \funcname{raise} & \funcname{raise\_notrace} \\
    \hline
    Compilation                                              & \parbox{8em}{inline assembly\\ (optimization)} & inline assembly   & \funcname{caml\_raise\_exn} & inline assembly \\
    \hline
  \end{tabular}
\end{frame}


%
% Takeaway #5
%
\subsection*{Takeaway \#5}
\frameSubsectionTakeaway{}
\againframe<5>{takeaway}

    \end{column}
  \end{columns}
\end{frame}

\begin{frame}{Backtraces}{But before that, we need more fields from \typename{caml\_domain\_state}}
  \begin{columns}
    \begin{column}{0.5\textwidth}
      \begin{itemize}
        \item Remember \typename{caml\_domain\_state} the per-domain runtime state?
        \item Some other members are of interest to us now\dots
        \item \localname{backtrace\_active} is used to know if the runtime should collect backtraces
        \item \localname{backtrace\_active} can be set by
          \begin{itemize}
            \item The \funcarg{b} flag in the \funcname{OCAMLRUNPARAM} environment variable
            \item \mintinline{ocaml}{Printexc.record_backtrace : bool -> unit} \footnotemark
          \end{itemize}
      \end{itemize}
    \end{column}
    \begin{column}{0.5\textwidth}
      \begin{listing}
        \mintc[highlightlines={9-11}]{src/ocaml/domain\_state\_backtrace.h}
        \caption{runtime/caml/domain\_state.h}
      \end{listing}
    \end{column}
  \end{columns}
  \footnotetext[1]{https://v2.ocaml.org/api/Printexc.html}
\end{frame}

\newcommand\listCamlRaiseExn[1][]{
  \listasm[firstline=702,lastline=725,#1]{../ocaml/runtime/amd64.S}{runtime/amd64.S:702}}

\begin{frame}{Backtraces}{Walk through \funcname{caml\_raise\_exn}}
  \begin{columns}[T]
    \begin{column}{0.5\textwidth}
      \begin{itemize}
        \item<1-> First checks whether exceptions backtrace should be recorded
        \item<1-> By checking the \funcarg{domain\_state->backtrace\_active} value
        \item<2-> If not, acts as \funcname{raise\_notrace} with \funcname{RESTORE\_EXN\_HANDLER\_OCAML}
          \begin{itemize}
            \item Set \regname{RSP} to \localname{domain\_state->exn\_handler} value
            \item Restore \localname{domain\_state->exn\_handler} from trap
            \item Jump with \funcname{ret} (same effect as using \funcname{pop} and \funcname{jmp})
          \end{itemize}
        \item<3-> Otherwise, switches to the C stack to be able to execute C code
        \item<4-> Calls \funcname{caml\_stash\_backtrace}, a C function provided by the runtime\ignorespaces
          \onslide<6->{, with
            \begin{itemize}
              \item \funcarg{exn}: exception value
              \item \funcarg{pc}: code address of the raising point
              \item \funcarg{sp}: stack address of the raising function
              \item \funcarg{trapsp}: stack address of the next handler
            \end{itemize}
          }
      \end{itemize}
      \bigskip
      \mintocaml[firstline=9,lastline=13,highlightlines=12]{../src/backtrace/backtrace.ml}
    \end{column}
    \begin{column}{0.5\textwidth}
      \raggedleft
      \onslide*<1,3-4>{
        \mintc[firstline=190,lastline=192]{../ocaml/runtime/amd64.S}
        \mintc[firstline=234,lastline=237]{../ocaml/runtime/amd64.S}
      }
      \onslide*<2>{
        \mintc[firstline=190,lastline=192,highlightlines={191-192}]{../ocaml/runtime/amd64.S}
        \mintc[firstline=234,lastline=237,highlightlines={235-237}]{../ocaml/runtime/amd64.S}
      }
      \onslide*<1>{\listCamlRaiseExn[highlightlines=705]{}}%
      \onslide*<2>{\listCamlRaiseExn[highlightlines={707-708}]{}}%
      \onslide*<3>{\listCamlRaiseExn[highlightlines={712-714}]{}}%
      \onslide*<4>{\listCamlRaiseExn[highlightlines={715-720}]{}}%
      \onslide*<5>{\providecommand\step{1}%
% Backtraces
%
\subsection{One advantage of raising with debugging information}
\frameSubsection{}{}

\begin{frame}{Backtraces}{Debugging information \& source code}
  \begin{columns}[c]
    \begin{column}{0.5\textwidth}
      \begin{itemize}
        \item Raising an exception is fun and games, but how do we know where the exception originated? $\rightarrow$ With the backtrace associated with the exception!
        \item In order to get a backtrace from the point where an exception was raised, it's necessary to compile with debugging information enabled (\funcarg{-g} flag for the compiler)
        \item Same example as in the `Nested handlers' section
      \end{itemize}
    \end{column}
    \begin{column}{0.5\textwidth}
      \listocaml[firstline=9,lastline=34]{../src/backtrace/backtrace.ml}{backtrace/backtrace.ml}
    \end{column}
  \end{columns}
\end{frame}

\begin{frame}{Backtraces}{CMM and disassembly of \funcname{definitely\_raise}}
  \begin{columns}[c]
    \begin{column}{0.5\textwidth}
      \minipage[c][0.66\textheight][s]{\columnwidth}
      \begin{itemize}
        \item<1-> An effect of compiling with debugging information (\funcarg{-g}): the CMM no longer uses \funcname{raise\_notrace} to raise the exception but \funcname{raise} instead
        \item<2-> Disassembly shows that CMM \funcname{raise} is actually a call to \funcname{caml\_raise\_exn}, a function in assembler provided by the runtime
      \end{itemize}
      \vfill
      \mintocaml[firstline=9,lastline=13,highlightlines=12]{../src/backtrace/backtrace.ml}
      \endminipage
    \end{column}
    \begin{column}{0.5\textwidth}
      \onslide*<1>{
        \mintcmm[highlightlines=5]{../src/backtrace/camlBacktrace\_\_definitely\_raise\_347.cmm}
      }
      \onslide*<2>{
        \mintobjdump[firstline=2,highlightlines=14]{../src/backtrace/camlBacktrace\_\_definitely\_raise\_347.objdump}
      }
    \end{column}
  \end{columns}
\end{frame}

\begin{frame}{Backtraces}{Introducing \funcname{caml\_raise\_exn}}
  \begin{columns}[c]
    \begin{column}{0.5\textwidth}
      \minipage[c][0.66\textheight][s]{\columnwidth}
      \onslide*<1>{
        \begin{itemize}
          \item Raising the exception is performed using \funcname{caml\_raise\_exn} which is
            \begin{itemize}
              \item An assembly function provided by the runtime
              \item Architecture specific
            \end{itemize}
          \item Let's see what it does differently from \funcname{raise\_notrace}\dots
          \item We are about to raise \typename{ExnC} with \funcname{caml\_raise\_exn} from \funcname{definitely\_raise}
        \end{itemize}
      }
      \onslide*<2>{
        \mintobjdump[firstline=2,highlightlines=14]{../src/backtrace/camlBacktrace\_\_definitely\_raise\_347.objdump}
      }
      \vfill
      \mintocaml[firstline=9,lastline=13,highlightlines=12]{../src/backtrace/backtrace.ml}
      \endminipage
    \end{column}
    \begin{column}{0.5\textwidth}
      \centering
      \providecommand\step{1}%
% Backtraces
%
\subsection{One advantage of raising with debugging information}
\frameSubsection{}{}

\begin{frame}{Backtraces}{Debugging information \& source code}
  \begin{columns}[c]
    \begin{column}{0.5\textwidth}
      \begin{itemize}
        \item Raising an exception is fun and games, but how do we know where the exception originated? $\rightarrow$ With the backtrace associated with the exception!
        \item In order to get a backtrace from the point where an exception was raised, it's necessary to compile with debugging information enabled (\funcarg{-g} flag for the compiler)
        \item Same example as in the `Nested handlers' section
      \end{itemize}
    \end{column}
    \begin{column}{0.5\textwidth}
      \listocaml[firstline=9,lastline=34]{../src/backtrace/backtrace.ml}{backtrace/backtrace.ml}
    \end{column}
  \end{columns}
\end{frame}

\begin{frame}{Backtraces}{CMM and disassembly of \funcname{definitely\_raise}}
  \begin{columns}[c]
    \begin{column}{0.5\textwidth}
      \minipage[c][0.66\textheight][s]{\columnwidth}
      \begin{itemize}
        \item<1-> An effect of compiling with debugging information (\funcarg{-g}): the CMM no longer uses \funcname{raise\_notrace} to raise the exception but \funcname{raise} instead
        \item<2-> Disassembly shows that CMM \funcname{raise} is actually a call to \funcname{caml\_raise\_exn}, a function in assembler provided by the runtime
      \end{itemize}
      \vfill
      \mintocaml[firstline=9,lastline=13,highlightlines=12]{../src/backtrace/backtrace.ml}
      \endminipage
    \end{column}
    \begin{column}{0.5\textwidth}
      \onslide*<1>{
        \mintcmm[highlightlines=5]{../src/backtrace/camlBacktrace\_\_definitely\_raise\_347.cmm}
      }
      \onslide*<2>{
        \mintobjdump[firstline=2,highlightlines=14]{../src/backtrace/camlBacktrace\_\_definitely\_raise\_347.objdump}
      }
    \end{column}
  \end{columns}
\end{frame}

\begin{frame}{Backtraces}{Introducing \funcname{caml\_raise\_exn}}
  \begin{columns}[c]
    \begin{column}{0.5\textwidth}
      \minipage[c][0.66\textheight][s]{\columnwidth}
      \onslide*<1>{
        \begin{itemize}
          \item Raising the exception is performed using \funcname{caml\_raise\_exn} which is
            \begin{itemize}
              \item An assembly function provided by the runtime
              \item Architecture specific
            \end{itemize}
          \item Let's see what it does differently from \funcname{raise\_notrace}\dots
          \item We are about to raise \typename{ExnC} with \funcname{caml\_raise\_exn} from \funcname{definitely\_raise}
        \end{itemize}
      }
      \onslide*<2>{
        \mintobjdump[firstline=2,highlightlines=14]{../src/backtrace/camlBacktrace\_\_definitely\_raise\_347.objdump}
      }
      \vfill
      \mintocaml[firstline=9,lastline=13,highlightlines=12]{../src/backtrace/backtrace.ml}
      \endminipage
    \end{column}
    \begin{column}{0.5\textwidth}
      \centering
      \providecommand\step{1}\input{diagrams/backtrace/backtrace.tex}
    \end{column}
  \end{columns}
\end{frame}

\begin{frame}{Backtraces}{But before that, we need more fields from \typename{caml\_domain\_state}}
  \begin{columns}
    \begin{column}{0.5\textwidth}
      \begin{itemize}
        \item Remember \typename{caml\_domain\_state} the per-domain runtime state?
        \item Some other members are of interest to us now\dots
        \item \localname{backtrace\_active} is used to know if the runtime should collect backtraces
        \item \localname{backtrace\_active} can be set by
          \begin{itemize}
            \item The \funcarg{b} flag in the \funcname{OCAMLRUNPARAM} environment variable
            \item \mintinline{ocaml}{Printexc.record_backtrace : bool -> unit} \footnotemark
          \end{itemize}
      \end{itemize}
    \end{column}
    \begin{column}{0.5\textwidth}
      \begin{listing}
        \mintc[highlightlines={9-11}]{src/ocaml/domain\_state\_backtrace.h}
        \caption{runtime/caml/domain\_state.h}
      \end{listing}
    \end{column}
  \end{columns}
  \footnotetext[1]{https://v2.ocaml.org/api/Printexc.html}
\end{frame}

\newcommand\listCamlRaiseExn[1][]{
  \listasm[firstline=702,lastline=725,#1]{../ocaml/runtime/amd64.S}{runtime/amd64.S:702}}

\begin{frame}{Backtraces}{Walk through \funcname{caml\_raise\_exn}}
  \begin{columns}[T]
    \begin{column}{0.5\textwidth}
      \begin{itemize}
        \item<1-> First checks whether exceptions backtrace should be recorded
        \item<1-> By checking the \funcarg{domain\_state->backtrace\_active} value
        \item<2-> If not, acts as \funcname{raise\_notrace} with \funcname{RESTORE\_EXN\_HANDLER\_OCAML}
          \begin{itemize}
            \item Set \regname{RSP} to \localname{domain\_state->exn\_handler} value
            \item Restore \localname{domain\_state->exn\_handler} from trap
            \item Jump with \funcname{ret} (same effect as using \funcname{pop} and \funcname{jmp})
          \end{itemize}
        \item<3-> Otherwise, switches to the C stack to be able to execute C code
        \item<4-> Calls \funcname{caml\_stash\_backtrace}, a C function provided by the runtime\ignorespaces
          \onslide<6->{, with
            \begin{itemize}
              \item \funcarg{exn}: exception value
              \item \funcarg{pc}: code address of the raising point
              \item \funcarg{sp}: stack address of the raising function
              \item \funcarg{trapsp}: stack address of the next handler
            \end{itemize}
          }
      \end{itemize}
      \bigskip
      \mintocaml[firstline=9,lastline=13,highlightlines=12]{../src/backtrace/backtrace.ml}
    \end{column}
    \begin{column}{0.5\textwidth}
      \raggedleft
      \onslide*<1,3-4>{
        \mintc[firstline=190,lastline=192]{../ocaml/runtime/amd64.S}
        \mintc[firstline=234,lastline=237]{../ocaml/runtime/amd64.S}
      }
      \onslide*<2>{
        \mintc[firstline=190,lastline=192,highlightlines={191-192}]{../ocaml/runtime/amd64.S}
        \mintc[firstline=234,lastline=237,highlightlines={235-237}]{../ocaml/runtime/amd64.S}
      }
      \onslide*<1>{\listCamlRaiseExn[highlightlines=705]{}}%
      \onslide*<2>{\listCamlRaiseExn[highlightlines={707-708}]{}}%
      \onslide*<3>{\listCamlRaiseExn[highlightlines={712-714}]{}}%
      \onslide*<4>{\listCamlRaiseExn[highlightlines={715-720}]{}}%
      \onslide*<5>{\providecommand\step{1}\input{diagrams/backtrace/backtrace.tex}}%
      \onslide*<6>{\providecommand\step{2}\input{diagrams/backtrace/backtrace.tex}}%
    \end{column}
  \end{columns}
\end{frame}

\newcommand\listStashBacktrace[1][]{
  \listc[firstline=91,lastline=120,#1]{../ocaml/runtime/backtrace\_nat.c}{runtime/backtrace\_nat.c:91}}

\begin{frame}{Backtraces}{Introducing \funcname{caml\_stash\_backtrace}}
  \begin{columns}[c]
    \begin{column}{0.5\textwidth}
      \minipage[c][0.66\textheight][s]{\columnwidth}
      \begin{itemize}
        \item<1-> A C function provided by the runtime (specific to native code)
        \item<2-> It scans the stack, between the stack pointer at the raising point up to the next trap, in order to collect the names of the detected function frames
          \onslide*<2>{
            \begin{itemize}
              \item And more information, like line number, column number...
            \end{itemize}
          }
        \item<3-> It uses the same technique and information as the Garbage Collector (GC) uses to scan the values live on the stack
          \onslide*<4>{
            \begin{itemize}
              \item \typename{frame\_descr} are at the core of stack unwinding
            \end{itemize}
          }
      \end{itemize}
      \vfill
      \mintocaml[firstline=9,lastline=13,highlightlines=12]{../src/backtrace/backtrace.ml}
      \endminipage
    \end{column}
    \begin{column}{0.5\textwidth}
      \onslide*<1>{\listStashBacktrace[highlightlines=91]{}}
      \onslide*<2>{\listStashBacktrace[highlightlines={91,117-118}]{}}
      \onslide*<3->{\listStashBacktrace[highlightlines={94,106,109-110}]{}}
    \end{column}
  \end{columns}
\end{frame}

\newcommand\listFrameDescr[1][]{
  \listocaml[firstline=111,lastline=124,#1]{../ocaml/asmcomp/emitaux.ml}{asmcomp/emitaux.ml:111}}
\newcommand\listDebugInfo[1][]{
  \listocaml[firstline=38,lastline=49,#1]{../ocaml/lambda/debuginfo.mli}{lambda/debuginfo.mli:38}}

\begin{frame}{Backtraces}{\typename{frame\_descr} and \typename{frame\_debuginfo} in a nutshell}
  \begin{columns}[c]
    \begin{column}{0.5\textwidth}
      \begin{itemize}
        \item<1-> For every return address in an OCaml program, there is a associated \typename{frame\_descr}
        \item<2-> A \typename{frame\_descr} specifies
        \begin{itemize}
          \item<2-> The size of the function stack frame
          \item<3-> Where live values are on the stack (so that the GC can mark them as live and prevent from being swept)
        \end{itemize}
        \item<4-> When compiled with debug information, \typename{frame\_descr} will be linked to a \typename{frame\_debuginfo}
        \begin{itemize}
          \item<5-> Contains information about the position in the source code (file name, line and column number)
        \end{itemize}
      \end{itemize}
    \end{column}
    \begin{column}{0.5\textwidth}
        \onslide*<1>{\listFrameDescr[highlightlines={118-122}]{}}
        \onslide*<2>{\listFrameDescr[highlightlines={120}]{}}
        \onslide*<3>{\listFrameDescr[highlightlines={121}]{}}
        \onslide*<4->{\listFrameDescr[highlightlines={122}]{}}
        \medskip
        \onslide*<1-3>{\listDebugInfo{}}
        \onslide*<4>{\listDebugInfo[highlightlines={38,49}]{}}
        \onslide*<5>{\listDebugInfo[highlightlines={39-46}]{}}
    \end{column}
  \end{columns}
\end{frame}

\begin{frame}{Backtraces}{Scanning the stack with \typename{frame\_descr}}
  \begin{columns}[c]
    \begin{column}{0.5\textwidth}
      \begin{itemize}
        \item Given the return address of the code that raises the exception by calling \funcname{caml\_raise\_exn} (\funcarg{pc} argument of \funcname{caml\_stash\_backtrace})
        \item And the position of the stack pointer (\funcarg{sp} argument of \funcname{caml\_stash\_backtrace})
        \item The frame size given by \typename{frame\_descr}.\funcarg{frame\_size}, allows to find the end of the previous frame
        \item And the return address of the caller of this function
        \bigskip
        \onslide<3->{
        \item Note that traps (here the reddish \funcname{ExnA}, \funcname{ExnB} and \funcname{ExnC} blocks) belong to the frame that installed them, and so the frame size changes when traps are removed
        }
      \end{itemize}
    \end{column}
    \begin{column}{0.5\textwidth}
      \raggedleft
      \onslide*<1>{\providecommand\step{2}\input{diagrams/backtrace/backtrace.tex}}%
      \onslide*<2>{\providecommand\step{1}\input{diagrams/backtrace/scanstack.tex}}%
      \onslide*<3>{\providecommand\step{2}\input{diagrams/backtrace/scanstack.tex}}%
      \onslide*<4>{\providecommand\step{3}\input{diagrams/backtrace/scanstack.tex}}%
      \onslide*<5>{\providecommand\step{4}\input{diagrams/backtrace/scanstack.tex}}%
      \onslide*<6>{\providecommand\step{5}\input{diagrams/backtrace/scanstack.tex}}%
    \end{column}
  \end{columns}
\end{frame}

% Duplicate of previous-previous slide
\begin{frame}{Backtraces}{Continuing on \funcname{caml\_stash\_backtrace}}
  \begin{columns}[c]
    \begin{column}{0.5\textwidth}
      \minipage[c][0.75\textheight][s]{\columnwidth}
      \begin{itemize}
          \color{gray}
        \item A C function provided by the runtime (specific to native code)
        \item It scans the stack, between the stack pointer at the raising point up to the next trap, in order to collect the names of the detected function frames
        \item It uses the same technique and information as the Garbage Collector (GC) uses to scan the values live on the stack
      \end{itemize}
      \begin{itemize}
        \item For each function frame detected, store a pointer to the \typename{frame\_descr} in the \localname{domain\_state->backtrace\_buffer} array, effectively constructing the backtrace
      \end{itemize}
      \onslide<2->{
        \begin{itemize}
            \smallskip
          \item Constructed backtrace:
            \funcname{definitely\_raise},
            \onslide<3->{\funcname{probably\_raise}}
        \end{itemize}
      }
      \vfill
      \mintocaml[firstline=9,lastline=13,highlightlines=12]{../src/backtrace/backtrace.ml}
      \endminipage
    \end{column}
    \begin{column}{0.5\textwidth}
      \raggedleft
      \onslide*<1>{\listStashBacktrace[highlightlines={114-115}]{}}
      \onslide*<2>{\providecommand\step{3}\input{diagrams/backtrace/backtrace.tex}}%
      \onslide*<3>{\providecommand\step{4}\input{diagrams/backtrace/backtrace.tex}}%
    \end{column}
  \end{columns}
\end{frame}

\begin{frame}{Backtraces}{Back to \funcname{caml\_raise\_exn}}
  \begin{columns}[c]
    \begin{column}{0.5\textwidth}
      \minipage[c][0.66\textheight][s]{\columnwidth}
      \begin{itemize}\color{gray}
        \item Checks wheteher exceptions backtrace should be recorded, by checking the \funcarg{domain\_state->backtrace\_active} value
        \item Switches to the C stack to be able to execute C code
        \item Calls \funcname{caml\_stash\_backtrace}, to collect the backtrace up to the next trap
      \end{itemize}
      \onslide*<2->{
        \begin{itemize}
          \item<2-> Executes the exception handler in \funcname{probably\_raise} with \funcname{RESTORE\_EXN\_HANDLER\_OCAML}
            \begin{itemize}
              \item Set \regname{RSP} to \localname{domain\_state->exn\_handler} value
              \item Restore \localname{domain\_state->exn\_handler} from trap
              \item Jump to the handler code with \funcname{ret}
            \end{itemize}
        \end{itemize}
      }
      \vfill
      \onslide*<1>{\mintocaml[firstline=9,lastline=13,highlightlines=12]{../src/backtrace/backtrace.ml}}
      \onslide*<2->{\mintocaml[firstline=15,lastline=21,highlightlines={19-21}]{../src/backtrace/backtrace.ml}}
      \endminipage
    \end{column}
    \begin{column}{0.5\textwidth}
      \centering
      \onslide*<1>{
        \mintc[firstline=190,lastline=192]{../ocaml/runtime/amd64.S}
        \mintc[firstline=234,lastline=237]{../ocaml/runtime/amd64.S}
      }
      \onslide*<2>{
        \mintc[firstline=190,lastline=192,highlightlines={191-192}]{../ocaml/runtime/amd64.S}
        \mintc[firstline=234,lastline=237,highlightlines={235-237}]{../ocaml/runtime/amd64.S}
      }
      \onslide*<1>{\listCamlRaiseExn[highlightlines=720]{}}
      \onslide*<2>{\listCamlRaiseExn[highlightlines=722]{}}
      \onslide*<3->{\providecommand\step{5}\input{diagrams/backtrace/backtrace.tex}}
    \end{column}
  \end{columns}
\end{frame}

\begin{frame}{Backtraces}{\funcname{caml\_probably\_raise}'s handler doesn't catch \typename{ExnC}}
  \begin{columns}[c]
    \begin{column}{0.45\textwidth}
      \minipage[c][0.75\textheight][s]{\columnwidth}
      \begin{itemize}
        \item<1-> \funcname{match \dots with}\footnotemark pattern matching can also match exceptions
        \item<1-> The CMM of \funcname{probably\_raise} shows that this \funcname{match \dots with} is implemented with a pattern matching and a \funcname{try \dots with}
        \item<1-> But this one only matches on \typename{ExnA} exception
        \item<2-> Since the pattern matching fails to catch the exception, \funcname{reraise} is called with our current \typename{ExnC} exception
        \item<3-> Disassembly shows that \funcname{reraise} is actually a call to \funcname{caml\_reraise\_exn}, which is also an assembly function provided by the runtime
      \end{itemize}
      \vfill
      \mintocaml[firstline=15,lastline=21,highlightlines={18-21}]{../src/backtrace/backtrace.ml}
      \endminipage
    \end{column}
    \begin{column}{0.55\textwidth}
      \centering
      \onslide*<1>{\mintcmm[highlightlines={10-18}]{../src/backtrace/camlBacktrace\_\_probably\_raise\_351.cmm}}
      \onslide*<2>{\listcmm[highlightlines=18]{../src/backtrace/camlBacktrace\_\_probably\_raise_351.cmm}{CMM of probably\_raise}}
      \onslide*<3>{\mintobjdump[firstline=5,lastline=35,highlightlines=31]{../src/backtrace/camlBacktrace\_\_probably\_raise_351.objdump}}
    \end{column}
  \end{columns}
  \only<1>{\footnotetext[1]{https://blog.janestreet.com/pattern-matching-and-exception-handling-unite/}}
\end{frame}

\newcommand\listCamlReraiseExn[1][]{
  \listasm[firstline=727,lastline=734,#1]{../ocaml/runtime/amd64.S}{runtime/amd64.S:727}}

\begin{frame}{Backtraces}{Introducing \funcname{caml\_reraise\_exn}}
  \begin{columns}[c]
    \begin{column}{0.5\textwidth}
      \minipage[c][0.66\textheight][s]{\columnwidth}
      \begin{itemize}
        \item<1-> The exception have to be reraised to the next handler
        \item<2-> First, checks if the backtrace should be collected with \funcarg{domain\_state->backtrace\_active}
        \only<3>{
        \begin{itemize}
            \item If not, behaves like a \funcname{raise\_notrace} with \funcname{RESTORE\_EXN\_HANDLER\_OCAML}
        \end{itemize}
        }
        \item<4-> Jumps into \funcname{caml\_raise\_exn}
        \item<5-> Calls \funcname{caml\_stash\_backtrace} again, to scan the next part of the stack: from the trap just pop-ed to the next trap
      \end{itemize}
      \vfill
      \mintocaml[firstline=15,lastline=21,highlightlines={18-21}]{../src/backtrace/backtrace.ml}
      \endminipage
    \end{column}
    \begin{column}{0.5\textwidth}
      \raggedleft
      \onslide*<1>{\listCamlReraiseExn{}}
      \onslide*<2>{\listCamlReraiseExn[highlightlines=729]{}}
      \onslide*<3>{
        \mintc[firstline=190,lastline=192,highlightlines={191-192}]{../ocaml/runtime/amd64.S}
        \mintc[firstline=234,lastline=237,highlightlines={235-237}]{../ocaml/runtime/amd64.S}
        \medskip
        \listCamlReraiseExn[highlightlines=731]{}
      }
      \onslide*<4-5>{
        % TODO refactor with \listCamlReraiseExn
        \mintasm[firstline=727,lastline=734,highlightlines=730]{../ocaml/runtime/amd64.S}
        \medskip
      }
      \onslide*<4>{\listCamlRaiseExn[highlightlines=711]{}}
      \onslide*<5>{\listCamlRaiseExn[highlightlines=720]{}}
    \end{column}
  \end{columns}
\end{frame}

\begin{frame}{Backtraces}{Backtrace collection continues}
  \begin{columns}[c]
    \begin{column}{0.55\textwidth}
      \minipage[c][0.75\textheight][s]{\columnwidth}
      \begin{itemize}
        \item<1-> \funcname{caml\_stash\_backtrace} scans the older part of the stack, up to the next trap
        \item<6-> Controls jumps to the nested exception handler in \funcname{maybe\_raise}, which matches only on \typename{ExnB}
        \item<7-> \funcname{caml\_reraise\_exn} is called again, backtrace collection resumes with \funcname{caml\_stash\_backtrace}
      \end{itemize}
      \medskip
      \begin{itemize}
        \item<2-> Constructed backtrace:
        \begin{itemize}
          \item <2-> \funcname{definitely\_raise}
          \item <2-> \funcname{probably\_raise}
          \item <3-> \funcname{probably\_raise} (re-raise)
          \item <4-> \funcname{maybe\_raise}
          \item <5-> \funcname{main}
          \item <8-> \funcname{main} (re-raise)
        \end{itemize}
      \end{itemize}
      \vfill
      \onslide*<1-5>{\mintocaml[firstline=15,lastline=21]{../src/backtrace/backtrace.ml}}
      \onslide*<6>{\mintocaml[firstline=28,lastline=34,highlightlines=31]{../src/backtrace/backtrace.ml}}
      \onslide*<7->{\mintocaml[firstline=28,lastline=34,highlightlines=33]{../src/backtrace/backtrace.ml}}
      \endminipage
    \end{column}
    \begin{column}{0.45\textwidth}
      \raggedleft
      \onslide*<1>{\providecommand\step{6}\input{diagrams/backtrace/backtrace.tex}}%
      \onslide*<2>{\providecommand\step{6}\input{diagrams/backtrace/backtrace.tex}}%
      \onslide*<3>{\providecommand\step{7}\input{diagrams/backtrace/backtrace.tex}}%
      \onslide*<4>{\providecommand\step{8}\input{diagrams/backtrace/backtrace.tex}}%
      \onslide*<5>{\providecommand\step{9}\input{diagrams/backtrace/backtrace.tex}}%
      \onslide*<6>{\providecommand\step{10}\input{diagrams/backtrace/backtrace.tex}}%
      \onslide*<7>{\providecommand\step{11}\input{diagrams/backtrace/backtrace.tex}}%
      \onslide*<8>{\providecommand\step{12}\input{diagrams/backtrace/backtrace.tex}}%
      \onslide*<9>{\providecommand\step{13}\input{diagrams/backtrace/backtrace.tex}}%
    \end{column}
  \end{columns}
\end{frame}

\begin{frame}{Backtraces}{Printing the backtrace}
  \begin{columns}[c]
    \begin{column}{0.45\textwidth}
      \minipage[c][0.66\textheight][s]{\columnwidth}
      \begin{itemize}
        \item<1-> The exception backtrace can now be printed to \funcarg{stdout} with \funcname{Printexc.print_backtrace}
        \item<2-> First the exception backtrace is retrieved into an OCaml array with \funcname{get_raw_backtrace }
        \item<3-> Which is implemented as the \funcname{caml_get_exception_raw_backtrace} C function in the runtime
        \item<4-> And finally printed with \funcname{print_exception_backtrace}
      \end{itemize}
      \vfill
      \mintocaml[firstline=28,lastline=34,highlightlines=33]{../src/backtrace/backtrace.ml}
      \endminipage
    \end{column}
    \begin{column}{0.55\textwidth}
      \onslide*<1,2>{
        \mintocaml[firstline=100,lastline=101]{../ocaml/stdlib/printexc.ml}
      }
      \only<1>{
        \listocaml[firstline=170,lastline=172]{../ocaml/stdlib/printexc.ml}{stdlib/printexc.ml:170}
      }
      \only<2>{
        \listocaml[firstline=170,lastline=172,highlightlines=172]{../ocaml/stdlib/printexc.ml}{stdlib/printexc.ml:170}
      }
      \only<3>{
        \mintc[firstline=154,lastline=156]{../ocaml/runtime/backtrace.c}
        \listc[firstline=171,lastline=192,highlightlines={184-187}]{../ocaml/runtime/backtrace.c}{runtime/backtrace.c:154}
      }
      \only<4>{
        \listocaml[firstline=155,lastline=165]{../ocaml/stdlib/printexc.ml}{stdlib/printexc.ml:55}
      }
    \end{column}
  \end{columns}
\end{frame}


\subsection{The multiple kinds of raise}
\frameSubsection{}{}

\begin{frame}{Backtraces}{When backtraces are not needed}
  \begin{columns}[c]
    \begin{column}{0.45\textwidth}
      \minipage[c][0.66\textheight][s]{\columnwidth}
      \begin{itemize}
        \item<1-> What if the backtrace for an exception is never needed?
        \item<2-> It's possible to raise the exception explicitly with \funcname{raise\_notrace} to make raising as cheap as possible
        \item<3-> An example of use in the \funcname{dynlink} library of the OCaml compiler
      \end{itemize}
      \vfill
      \onslide<3->{
      \listocaml[firstline=205,lastline=209,highlightlines=207]{../ocaml/otherlibs/dynlink/dynlink\_compilerlibs/misc.ml}{otherlibs/dynlink/dynlink\_compilerlibs/misc.ml:205}
      }
      \endminipage
    \end{column}
    \begin{column}{0.55\textwidth}
    \only<4>{
      \mintcmm[highlightlines=3]{../src/notrace/camlNotrace\_\_fun_347.cmm}
      \listcmm{../src/notrace/camlNotrace\_\_all\_somes\_327.cmm}{all\_somes}
    }
    \onslide*<5->{
      \listobjdump[highlightlines={6-9}]{../src/notrace/camlNotrace\_\_fun_347.objdump}{anonymous function from \funcname{all\_somes}}
    }
    \end{column}
  \end{columns}
\end{frame}

\begin{frame}{Backtraces}{Summary table of the multiple kinds of raise}
  \begin{itemize}
    \item When debugging information is enabled and if the backtrace of an exception isn't needed, it's possible to raise with \funcname{raise\_notrace} for performance
    \item When debugging information is \emph{not} enabled, the compiler optimises \funcname{raise} calls to inline assembly (same effect as using \funcname{raise\_notrace})
  \end{itemize}
  \bigskip
  \centering
  \begin{tabular}{| l | c | c | c | c |}
    \hline
    \parbox{10em}{Debug information \\ (\funcarg{-g} flag)}  & \multicolumn{2}{|c|}{Disabled}                                     & \multicolumn{2}{|c|}{Enabled} \\
    \hline
    Raise kind                                               & \funcname{raise} & \funcname{raise\_notrace}                       & \funcname{raise} & \funcname{raise\_notrace} \\
    \hline
    Compilation                                              & \parbox{8em}{inline assembly\\ (optimization)} & inline assembly   & \funcname{caml\_raise\_exn} & inline assembly \\
    \hline
  \end{tabular}
\end{frame}


%
% Takeaway #5
%
\subsection*{Takeaway \#5}
\frameSubsectionTakeaway{}
\againframe<5>{takeaway}

    \end{column}
  \end{columns}
\end{frame}

\begin{frame}{Backtraces}{But before that, we need more fields from \typename{caml\_domain\_state}}
  \begin{columns}
    \begin{column}{0.5\textwidth}
      \begin{itemize}
        \item Remember \typename{caml\_domain\_state} the per-domain runtime state?
        \item Some other members are of interest to us now\dots
        \item \localname{backtrace\_active} is used to know if the runtime should collect backtraces
        \item \localname{backtrace\_active} can be set by
          \begin{itemize}
            \item The \funcarg{b} flag in the \funcname{OCAMLRUNPARAM} environment variable
            \item \mintinline{ocaml}{Printexc.record_backtrace : bool -> unit} \footnotemark
          \end{itemize}
      \end{itemize}
    \end{column}
    \begin{column}{0.5\textwidth}
      \begin{listing}
        \mintc[highlightlines={9-11}]{src/ocaml/domain\_state\_backtrace.h}
        \caption{runtime/caml/domain\_state.h}
      \end{listing}
    \end{column}
  \end{columns}
  \footnotetext[1]{https://v2.ocaml.org/api/Printexc.html}
\end{frame}

\newcommand\listCamlRaiseExn[1][]{
  \listasm[firstline=702,lastline=725,#1]{../ocaml/runtime/amd64.S}{runtime/amd64.S:702}}

\begin{frame}{Backtraces}{Walk through \funcname{caml\_raise\_exn}}
  \begin{columns}[T]
    \begin{column}{0.5\textwidth}
      \begin{itemize}
        \item<1-> First checks whether exceptions backtrace should be recorded
        \item<1-> By checking the \funcarg{domain\_state->backtrace\_active} value
        \item<2-> If not, acts as \funcname{raise\_notrace} with \funcname{RESTORE\_EXN\_HANDLER\_OCAML}
          \begin{itemize}
            \item Set \regname{RSP} to \localname{domain\_state->exn\_handler} value
            \item Restore \localname{domain\_state->exn\_handler} from trap
            \item Jump with \funcname{ret} (same effect as using \funcname{pop} and \funcname{jmp})
          \end{itemize}
        \item<3-> Otherwise, switches to the C stack to be able to execute C code
        \item<4-> Calls \funcname{caml\_stash\_backtrace}, a C function provided by the runtime\ignorespaces
          \onslide<6->{, with
            \begin{itemize}
              \item \funcarg{exn}: exception value
              \item \funcarg{pc}: code address of the raising point
              \item \funcarg{sp}: stack address of the raising function
              \item \funcarg{trapsp}: stack address of the next handler
            \end{itemize}
          }
      \end{itemize}
      \bigskip
      \mintocaml[firstline=9,lastline=13,highlightlines=12]{../src/backtrace/backtrace.ml}
    \end{column}
    \begin{column}{0.5\textwidth}
      \raggedleft
      \onslide*<1,3-4>{
        \mintc[firstline=190,lastline=192]{../ocaml/runtime/amd64.S}
        \mintc[firstline=234,lastline=237]{../ocaml/runtime/amd64.S}
      }
      \onslide*<2>{
        \mintc[firstline=190,lastline=192,highlightlines={191-192}]{../ocaml/runtime/amd64.S}
        \mintc[firstline=234,lastline=237,highlightlines={235-237}]{../ocaml/runtime/amd64.S}
      }
      \onslide*<1>{\listCamlRaiseExn[highlightlines=705]{}}%
      \onslide*<2>{\listCamlRaiseExn[highlightlines={707-708}]{}}%
      \onslide*<3>{\listCamlRaiseExn[highlightlines={712-714}]{}}%
      \onslide*<4>{\listCamlRaiseExn[highlightlines={715-720}]{}}%
      \onslide*<5>{\providecommand\step{1}%
% Backtraces
%
\subsection{One advantage of raising with debugging information}
\frameSubsection{}{}

\begin{frame}{Backtraces}{Debugging information \& source code}
  \begin{columns}[c]
    \begin{column}{0.5\textwidth}
      \begin{itemize}
        \item Raising an exception is fun and games, but how do we know where the exception originated? $\rightarrow$ With the backtrace associated with the exception!
        \item In order to get a backtrace from the point where an exception was raised, it's necessary to compile with debugging information enabled (\funcarg{-g} flag for the compiler)
        \item Same example as in the `Nested handlers' section
      \end{itemize}
    \end{column}
    \begin{column}{0.5\textwidth}
      \listocaml[firstline=9,lastline=34]{../src/backtrace/backtrace.ml}{backtrace/backtrace.ml}
    \end{column}
  \end{columns}
\end{frame}

\begin{frame}{Backtraces}{CMM and disassembly of \funcname{definitely\_raise}}
  \begin{columns}[c]
    \begin{column}{0.5\textwidth}
      \minipage[c][0.66\textheight][s]{\columnwidth}
      \begin{itemize}
        \item<1-> An effect of compiling with debugging information (\funcarg{-g}): the CMM no longer uses \funcname{raise\_notrace} to raise the exception but \funcname{raise} instead
        \item<2-> Disassembly shows that CMM \funcname{raise} is actually a call to \funcname{caml\_raise\_exn}, a function in assembler provided by the runtime
      \end{itemize}
      \vfill
      \mintocaml[firstline=9,lastline=13,highlightlines=12]{../src/backtrace/backtrace.ml}
      \endminipage
    \end{column}
    \begin{column}{0.5\textwidth}
      \onslide*<1>{
        \mintcmm[highlightlines=5]{../src/backtrace/camlBacktrace\_\_definitely\_raise\_347.cmm}
      }
      \onslide*<2>{
        \mintobjdump[firstline=2,highlightlines=14]{../src/backtrace/camlBacktrace\_\_definitely\_raise\_347.objdump}
      }
    \end{column}
  \end{columns}
\end{frame}

\begin{frame}{Backtraces}{Introducing \funcname{caml\_raise\_exn}}
  \begin{columns}[c]
    \begin{column}{0.5\textwidth}
      \minipage[c][0.66\textheight][s]{\columnwidth}
      \onslide*<1>{
        \begin{itemize}
          \item Raising the exception is performed using \funcname{caml\_raise\_exn} which is
            \begin{itemize}
              \item An assembly function provided by the runtime
              \item Architecture specific
            \end{itemize}
          \item Let's see what it does differently from \funcname{raise\_notrace}\dots
          \item We are about to raise \typename{ExnC} with \funcname{caml\_raise\_exn} from \funcname{definitely\_raise}
        \end{itemize}
      }
      \onslide*<2>{
        \mintobjdump[firstline=2,highlightlines=14]{../src/backtrace/camlBacktrace\_\_definitely\_raise\_347.objdump}
      }
      \vfill
      \mintocaml[firstline=9,lastline=13,highlightlines=12]{../src/backtrace/backtrace.ml}
      \endminipage
    \end{column}
    \begin{column}{0.5\textwidth}
      \centering
      \providecommand\step{1}\input{diagrams/backtrace/backtrace.tex}
    \end{column}
  \end{columns}
\end{frame}

\begin{frame}{Backtraces}{But before that, we need more fields from \typename{caml\_domain\_state}}
  \begin{columns}
    \begin{column}{0.5\textwidth}
      \begin{itemize}
        \item Remember \typename{caml\_domain\_state} the per-domain runtime state?
        \item Some other members are of interest to us now\dots
        \item \localname{backtrace\_active} is used to know if the runtime should collect backtraces
        \item \localname{backtrace\_active} can be set by
          \begin{itemize}
            \item The \funcarg{b} flag in the \funcname{OCAMLRUNPARAM} environment variable
            \item \mintinline{ocaml}{Printexc.record_backtrace : bool -> unit} \footnotemark
          \end{itemize}
      \end{itemize}
    \end{column}
    \begin{column}{0.5\textwidth}
      \begin{listing}
        \mintc[highlightlines={9-11}]{src/ocaml/domain\_state\_backtrace.h}
        \caption{runtime/caml/domain\_state.h}
      \end{listing}
    \end{column}
  \end{columns}
  \footnotetext[1]{https://v2.ocaml.org/api/Printexc.html}
\end{frame}

\newcommand\listCamlRaiseExn[1][]{
  \listasm[firstline=702,lastline=725,#1]{../ocaml/runtime/amd64.S}{runtime/amd64.S:702}}

\begin{frame}{Backtraces}{Walk through \funcname{caml\_raise\_exn}}
  \begin{columns}[T]
    \begin{column}{0.5\textwidth}
      \begin{itemize}
        \item<1-> First checks whether exceptions backtrace should be recorded
        \item<1-> By checking the \funcarg{domain\_state->backtrace\_active} value
        \item<2-> If not, acts as \funcname{raise\_notrace} with \funcname{RESTORE\_EXN\_HANDLER\_OCAML}
          \begin{itemize}
            \item Set \regname{RSP} to \localname{domain\_state->exn\_handler} value
            \item Restore \localname{domain\_state->exn\_handler} from trap
            \item Jump with \funcname{ret} (same effect as using \funcname{pop} and \funcname{jmp})
          \end{itemize}
        \item<3-> Otherwise, switches to the C stack to be able to execute C code
        \item<4-> Calls \funcname{caml\_stash\_backtrace}, a C function provided by the runtime\ignorespaces
          \onslide<6->{, with
            \begin{itemize}
              \item \funcarg{exn}: exception value
              \item \funcarg{pc}: code address of the raising point
              \item \funcarg{sp}: stack address of the raising function
              \item \funcarg{trapsp}: stack address of the next handler
            \end{itemize}
          }
      \end{itemize}
      \bigskip
      \mintocaml[firstline=9,lastline=13,highlightlines=12]{../src/backtrace/backtrace.ml}
    \end{column}
    \begin{column}{0.5\textwidth}
      \raggedleft
      \onslide*<1,3-4>{
        \mintc[firstline=190,lastline=192]{../ocaml/runtime/amd64.S}
        \mintc[firstline=234,lastline=237]{../ocaml/runtime/amd64.S}
      }
      \onslide*<2>{
        \mintc[firstline=190,lastline=192,highlightlines={191-192}]{../ocaml/runtime/amd64.S}
        \mintc[firstline=234,lastline=237,highlightlines={235-237}]{../ocaml/runtime/amd64.S}
      }
      \onslide*<1>{\listCamlRaiseExn[highlightlines=705]{}}%
      \onslide*<2>{\listCamlRaiseExn[highlightlines={707-708}]{}}%
      \onslide*<3>{\listCamlRaiseExn[highlightlines={712-714}]{}}%
      \onslide*<4>{\listCamlRaiseExn[highlightlines={715-720}]{}}%
      \onslide*<5>{\providecommand\step{1}\input{diagrams/backtrace/backtrace.tex}}%
      \onslide*<6>{\providecommand\step{2}\input{diagrams/backtrace/backtrace.tex}}%
    \end{column}
  \end{columns}
\end{frame}

\newcommand\listStashBacktrace[1][]{
  \listc[firstline=91,lastline=120,#1]{../ocaml/runtime/backtrace\_nat.c}{runtime/backtrace\_nat.c:91}}

\begin{frame}{Backtraces}{Introducing \funcname{caml\_stash\_backtrace}}
  \begin{columns}[c]
    \begin{column}{0.5\textwidth}
      \minipage[c][0.66\textheight][s]{\columnwidth}
      \begin{itemize}
        \item<1-> A C function provided by the runtime (specific to native code)
        \item<2-> It scans the stack, between the stack pointer at the raising point up to the next trap, in order to collect the names of the detected function frames
          \onslide*<2>{
            \begin{itemize}
              \item And more information, like line number, column number...
            \end{itemize}
          }
        \item<3-> It uses the same technique and information as the Garbage Collector (GC) uses to scan the values live on the stack
          \onslide*<4>{
            \begin{itemize}
              \item \typename{frame\_descr} are at the core of stack unwinding
            \end{itemize}
          }
      \end{itemize}
      \vfill
      \mintocaml[firstline=9,lastline=13,highlightlines=12]{../src/backtrace/backtrace.ml}
      \endminipage
    \end{column}
    \begin{column}{0.5\textwidth}
      \onslide*<1>{\listStashBacktrace[highlightlines=91]{}}
      \onslide*<2>{\listStashBacktrace[highlightlines={91,117-118}]{}}
      \onslide*<3->{\listStashBacktrace[highlightlines={94,106,109-110}]{}}
    \end{column}
  \end{columns}
\end{frame}

\newcommand\listFrameDescr[1][]{
  \listocaml[firstline=111,lastline=124,#1]{../ocaml/asmcomp/emitaux.ml}{asmcomp/emitaux.ml:111}}
\newcommand\listDebugInfo[1][]{
  \listocaml[firstline=38,lastline=49,#1]{../ocaml/lambda/debuginfo.mli}{lambda/debuginfo.mli:38}}

\begin{frame}{Backtraces}{\typename{frame\_descr} and \typename{frame\_debuginfo} in a nutshell}
  \begin{columns}[c]
    \begin{column}{0.5\textwidth}
      \begin{itemize}
        \item<1-> For every return address in an OCaml program, there is a associated \typename{frame\_descr}
        \item<2-> A \typename{frame\_descr} specifies
        \begin{itemize}
          \item<2-> The size of the function stack frame
          \item<3-> Where live values are on the stack (so that the GC can mark them as live and prevent from being swept)
        \end{itemize}
        \item<4-> When compiled with debug information, \typename{frame\_descr} will be linked to a \typename{frame\_debuginfo}
        \begin{itemize}
          \item<5-> Contains information about the position in the source code (file name, line and column number)
        \end{itemize}
      \end{itemize}
    \end{column}
    \begin{column}{0.5\textwidth}
        \onslide*<1>{\listFrameDescr[highlightlines={118-122}]{}}
        \onslide*<2>{\listFrameDescr[highlightlines={120}]{}}
        \onslide*<3>{\listFrameDescr[highlightlines={121}]{}}
        \onslide*<4->{\listFrameDescr[highlightlines={122}]{}}
        \medskip
        \onslide*<1-3>{\listDebugInfo{}}
        \onslide*<4>{\listDebugInfo[highlightlines={38,49}]{}}
        \onslide*<5>{\listDebugInfo[highlightlines={39-46}]{}}
    \end{column}
  \end{columns}
\end{frame}

\begin{frame}{Backtraces}{Scanning the stack with \typename{frame\_descr}}
  \begin{columns}[c]
    \begin{column}{0.5\textwidth}
      \begin{itemize}
        \item Given the return address of the code that raises the exception by calling \funcname{caml\_raise\_exn} (\funcarg{pc} argument of \funcname{caml\_stash\_backtrace})
        \item And the position of the stack pointer (\funcarg{sp} argument of \funcname{caml\_stash\_backtrace})
        \item The frame size given by \typename{frame\_descr}.\funcarg{frame\_size}, allows to find the end of the previous frame
        \item And the return address of the caller of this function
        \bigskip
        \onslide<3->{
        \item Note that traps (here the reddish \funcname{ExnA}, \funcname{ExnB} and \funcname{ExnC} blocks) belong to the frame that installed them, and so the frame size changes when traps are removed
        }
      \end{itemize}
    \end{column}
    \begin{column}{0.5\textwidth}
      \raggedleft
      \onslide*<1>{\providecommand\step{2}\input{diagrams/backtrace/backtrace.tex}}%
      \onslide*<2>{\providecommand\step{1}\input{diagrams/backtrace/scanstack.tex}}%
      \onslide*<3>{\providecommand\step{2}\input{diagrams/backtrace/scanstack.tex}}%
      \onslide*<4>{\providecommand\step{3}\input{diagrams/backtrace/scanstack.tex}}%
      \onslide*<5>{\providecommand\step{4}\input{diagrams/backtrace/scanstack.tex}}%
      \onslide*<6>{\providecommand\step{5}\input{diagrams/backtrace/scanstack.tex}}%
    \end{column}
  \end{columns}
\end{frame}

% Duplicate of previous-previous slide
\begin{frame}{Backtraces}{Continuing on \funcname{caml\_stash\_backtrace}}
  \begin{columns}[c]
    \begin{column}{0.5\textwidth}
      \minipage[c][0.75\textheight][s]{\columnwidth}
      \begin{itemize}
          \color{gray}
        \item A C function provided by the runtime (specific to native code)
        \item It scans the stack, between the stack pointer at the raising point up to the next trap, in order to collect the names of the detected function frames
        \item It uses the same technique and information as the Garbage Collector (GC) uses to scan the values live on the stack
      \end{itemize}
      \begin{itemize}
        \item For each function frame detected, store a pointer to the \typename{frame\_descr} in the \localname{domain\_state->backtrace\_buffer} array, effectively constructing the backtrace
      \end{itemize}
      \onslide<2->{
        \begin{itemize}
            \smallskip
          \item Constructed backtrace:
            \funcname{definitely\_raise},
            \onslide<3->{\funcname{probably\_raise}}
        \end{itemize}
      }
      \vfill
      \mintocaml[firstline=9,lastline=13,highlightlines=12]{../src/backtrace/backtrace.ml}
      \endminipage
    \end{column}
    \begin{column}{0.5\textwidth}
      \raggedleft
      \onslide*<1>{\listStashBacktrace[highlightlines={114-115}]{}}
      \onslide*<2>{\providecommand\step{3}\input{diagrams/backtrace/backtrace.tex}}%
      \onslide*<3>{\providecommand\step{4}\input{diagrams/backtrace/backtrace.tex}}%
    \end{column}
  \end{columns}
\end{frame}

\begin{frame}{Backtraces}{Back to \funcname{caml\_raise\_exn}}
  \begin{columns}[c]
    \begin{column}{0.5\textwidth}
      \minipage[c][0.66\textheight][s]{\columnwidth}
      \begin{itemize}\color{gray}
        \item Checks wheteher exceptions backtrace should be recorded, by checking the \funcarg{domain\_state->backtrace\_active} value
        \item Switches to the C stack to be able to execute C code
        \item Calls \funcname{caml\_stash\_backtrace}, to collect the backtrace up to the next trap
      \end{itemize}
      \onslide*<2->{
        \begin{itemize}
          \item<2-> Executes the exception handler in \funcname{probably\_raise} with \funcname{RESTORE\_EXN\_HANDLER\_OCAML}
            \begin{itemize}
              \item Set \regname{RSP} to \localname{domain\_state->exn\_handler} value
              \item Restore \localname{domain\_state->exn\_handler} from trap
              \item Jump to the handler code with \funcname{ret}
            \end{itemize}
        \end{itemize}
      }
      \vfill
      \onslide*<1>{\mintocaml[firstline=9,lastline=13,highlightlines=12]{../src/backtrace/backtrace.ml}}
      \onslide*<2->{\mintocaml[firstline=15,lastline=21,highlightlines={19-21}]{../src/backtrace/backtrace.ml}}
      \endminipage
    \end{column}
    \begin{column}{0.5\textwidth}
      \centering
      \onslide*<1>{
        \mintc[firstline=190,lastline=192]{../ocaml/runtime/amd64.S}
        \mintc[firstline=234,lastline=237]{../ocaml/runtime/amd64.S}
      }
      \onslide*<2>{
        \mintc[firstline=190,lastline=192,highlightlines={191-192}]{../ocaml/runtime/amd64.S}
        \mintc[firstline=234,lastline=237,highlightlines={235-237}]{../ocaml/runtime/amd64.S}
      }
      \onslide*<1>{\listCamlRaiseExn[highlightlines=720]{}}
      \onslide*<2>{\listCamlRaiseExn[highlightlines=722]{}}
      \onslide*<3->{\providecommand\step{5}\input{diagrams/backtrace/backtrace.tex}}
    \end{column}
  \end{columns}
\end{frame}

\begin{frame}{Backtraces}{\funcname{caml\_probably\_raise}'s handler doesn't catch \typename{ExnC}}
  \begin{columns}[c]
    \begin{column}{0.45\textwidth}
      \minipage[c][0.75\textheight][s]{\columnwidth}
      \begin{itemize}
        \item<1-> \funcname{match \dots with}\footnotemark pattern matching can also match exceptions
        \item<1-> The CMM of \funcname{probably\_raise} shows that this \funcname{match \dots with} is implemented with a pattern matching and a \funcname{try \dots with}
        \item<1-> But this one only matches on \typename{ExnA} exception
        \item<2-> Since the pattern matching fails to catch the exception, \funcname{reraise} is called with our current \typename{ExnC} exception
        \item<3-> Disassembly shows that \funcname{reraise} is actually a call to \funcname{caml\_reraise\_exn}, which is also an assembly function provided by the runtime
      \end{itemize}
      \vfill
      \mintocaml[firstline=15,lastline=21,highlightlines={18-21}]{../src/backtrace/backtrace.ml}
      \endminipage
    \end{column}
    \begin{column}{0.55\textwidth}
      \centering
      \onslide*<1>{\mintcmm[highlightlines={10-18}]{../src/backtrace/camlBacktrace\_\_probably\_raise\_351.cmm}}
      \onslide*<2>{\listcmm[highlightlines=18]{../src/backtrace/camlBacktrace\_\_probably\_raise_351.cmm}{CMM of probably\_raise}}
      \onslide*<3>{\mintobjdump[firstline=5,lastline=35,highlightlines=31]{../src/backtrace/camlBacktrace\_\_probably\_raise_351.objdump}}
    \end{column}
  \end{columns}
  \only<1>{\footnotetext[1]{https://blog.janestreet.com/pattern-matching-and-exception-handling-unite/}}
\end{frame}

\newcommand\listCamlReraiseExn[1][]{
  \listasm[firstline=727,lastline=734,#1]{../ocaml/runtime/amd64.S}{runtime/amd64.S:727}}

\begin{frame}{Backtraces}{Introducing \funcname{caml\_reraise\_exn}}
  \begin{columns}[c]
    \begin{column}{0.5\textwidth}
      \minipage[c][0.66\textheight][s]{\columnwidth}
      \begin{itemize}
        \item<1-> The exception have to be reraised to the next handler
        \item<2-> First, checks if the backtrace should be collected with \funcarg{domain\_state->backtrace\_active}
        \only<3>{
        \begin{itemize}
            \item If not, behaves like a \funcname{raise\_notrace} with \funcname{RESTORE\_EXN\_HANDLER\_OCAML}
        \end{itemize}
        }
        \item<4-> Jumps into \funcname{caml\_raise\_exn}
        \item<5-> Calls \funcname{caml\_stash\_backtrace} again, to scan the next part of the stack: from the trap just pop-ed to the next trap
      \end{itemize}
      \vfill
      \mintocaml[firstline=15,lastline=21,highlightlines={18-21}]{../src/backtrace/backtrace.ml}
      \endminipage
    \end{column}
    \begin{column}{0.5\textwidth}
      \raggedleft
      \onslide*<1>{\listCamlReraiseExn{}}
      \onslide*<2>{\listCamlReraiseExn[highlightlines=729]{}}
      \onslide*<3>{
        \mintc[firstline=190,lastline=192,highlightlines={191-192}]{../ocaml/runtime/amd64.S}
        \mintc[firstline=234,lastline=237,highlightlines={235-237}]{../ocaml/runtime/amd64.S}
        \medskip
        \listCamlReraiseExn[highlightlines=731]{}
      }
      \onslide*<4-5>{
        % TODO refactor with \listCamlReraiseExn
        \mintasm[firstline=727,lastline=734,highlightlines=730]{../ocaml/runtime/amd64.S}
        \medskip
      }
      \onslide*<4>{\listCamlRaiseExn[highlightlines=711]{}}
      \onslide*<5>{\listCamlRaiseExn[highlightlines=720]{}}
    \end{column}
  \end{columns}
\end{frame}

\begin{frame}{Backtraces}{Backtrace collection continues}
  \begin{columns}[c]
    \begin{column}{0.55\textwidth}
      \minipage[c][0.75\textheight][s]{\columnwidth}
      \begin{itemize}
        \item<1-> \funcname{caml\_stash\_backtrace} scans the older part of the stack, up to the next trap
        \item<6-> Controls jumps to the nested exception handler in \funcname{maybe\_raise}, which matches only on \typename{ExnB}
        \item<7-> \funcname{caml\_reraise\_exn} is called again, backtrace collection resumes with \funcname{caml\_stash\_backtrace}
      \end{itemize}
      \medskip
      \begin{itemize}
        \item<2-> Constructed backtrace:
        \begin{itemize}
          \item <2-> \funcname{definitely\_raise}
          \item <2-> \funcname{probably\_raise}
          \item <3-> \funcname{probably\_raise} (re-raise)
          \item <4-> \funcname{maybe\_raise}
          \item <5-> \funcname{main}
          \item <8-> \funcname{main} (re-raise)
        \end{itemize}
      \end{itemize}
      \vfill
      \onslide*<1-5>{\mintocaml[firstline=15,lastline=21]{../src/backtrace/backtrace.ml}}
      \onslide*<6>{\mintocaml[firstline=28,lastline=34,highlightlines=31]{../src/backtrace/backtrace.ml}}
      \onslide*<7->{\mintocaml[firstline=28,lastline=34,highlightlines=33]{../src/backtrace/backtrace.ml}}
      \endminipage
    \end{column}
    \begin{column}{0.45\textwidth}
      \raggedleft
      \onslide*<1>{\providecommand\step{6}\input{diagrams/backtrace/backtrace.tex}}%
      \onslide*<2>{\providecommand\step{6}\input{diagrams/backtrace/backtrace.tex}}%
      \onslide*<3>{\providecommand\step{7}\input{diagrams/backtrace/backtrace.tex}}%
      \onslide*<4>{\providecommand\step{8}\input{diagrams/backtrace/backtrace.tex}}%
      \onslide*<5>{\providecommand\step{9}\input{diagrams/backtrace/backtrace.tex}}%
      \onslide*<6>{\providecommand\step{10}\input{diagrams/backtrace/backtrace.tex}}%
      \onslide*<7>{\providecommand\step{11}\input{diagrams/backtrace/backtrace.tex}}%
      \onslide*<8>{\providecommand\step{12}\input{diagrams/backtrace/backtrace.tex}}%
      \onslide*<9>{\providecommand\step{13}\input{diagrams/backtrace/backtrace.tex}}%
    \end{column}
  \end{columns}
\end{frame}

\begin{frame}{Backtraces}{Printing the backtrace}
  \begin{columns}[c]
    \begin{column}{0.45\textwidth}
      \minipage[c][0.66\textheight][s]{\columnwidth}
      \begin{itemize}
        \item<1-> The exception backtrace can now be printed to \funcarg{stdout} with \funcname{Printexc.print_backtrace}
        \item<2-> First the exception backtrace is retrieved into an OCaml array with \funcname{get_raw_backtrace }
        \item<3-> Which is implemented as the \funcname{caml_get_exception_raw_backtrace} C function in the runtime
        \item<4-> And finally printed with \funcname{print_exception_backtrace}
      \end{itemize}
      \vfill
      \mintocaml[firstline=28,lastline=34,highlightlines=33]{../src/backtrace/backtrace.ml}
      \endminipage
    \end{column}
    \begin{column}{0.55\textwidth}
      \onslide*<1,2>{
        \mintocaml[firstline=100,lastline=101]{../ocaml/stdlib/printexc.ml}
      }
      \only<1>{
        \listocaml[firstline=170,lastline=172]{../ocaml/stdlib/printexc.ml}{stdlib/printexc.ml:170}
      }
      \only<2>{
        \listocaml[firstline=170,lastline=172,highlightlines=172]{../ocaml/stdlib/printexc.ml}{stdlib/printexc.ml:170}
      }
      \only<3>{
        \mintc[firstline=154,lastline=156]{../ocaml/runtime/backtrace.c}
        \listc[firstline=171,lastline=192,highlightlines={184-187}]{../ocaml/runtime/backtrace.c}{runtime/backtrace.c:154}
      }
      \only<4>{
        \listocaml[firstline=155,lastline=165]{../ocaml/stdlib/printexc.ml}{stdlib/printexc.ml:55}
      }
    \end{column}
  \end{columns}
\end{frame}


\subsection{The multiple kinds of raise}
\frameSubsection{}{}

\begin{frame}{Backtraces}{When backtraces are not needed}
  \begin{columns}[c]
    \begin{column}{0.45\textwidth}
      \minipage[c][0.66\textheight][s]{\columnwidth}
      \begin{itemize}
        \item<1-> What if the backtrace for an exception is never needed?
        \item<2-> It's possible to raise the exception explicitly with \funcname{raise\_notrace} to make raising as cheap as possible
        \item<3-> An example of use in the \funcname{dynlink} library of the OCaml compiler
      \end{itemize}
      \vfill
      \onslide<3->{
      \listocaml[firstline=205,lastline=209,highlightlines=207]{../ocaml/otherlibs/dynlink/dynlink\_compilerlibs/misc.ml}{otherlibs/dynlink/dynlink\_compilerlibs/misc.ml:205}
      }
      \endminipage
    \end{column}
    \begin{column}{0.55\textwidth}
    \only<4>{
      \mintcmm[highlightlines=3]{../src/notrace/camlNotrace\_\_fun_347.cmm}
      \listcmm{../src/notrace/camlNotrace\_\_all\_somes\_327.cmm}{all\_somes}
    }
    \onslide*<5->{
      \listobjdump[highlightlines={6-9}]{../src/notrace/camlNotrace\_\_fun_347.objdump}{anonymous function from \funcname{all\_somes}}
    }
    \end{column}
  \end{columns}
\end{frame}

\begin{frame}{Backtraces}{Summary table of the multiple kinds of raise}
  \begin{itemize}
    \item When debugging information is enabled and if the backtrace of an exception isn't needed, it's possible to raise with \funcname{raise\_notrace} for performance
    \item When debugging information is \emph{not} enabled, the compiler optimises \funcname{raise} calls to inline assembly (same effect as using \funcname{raise\_notrace})
  \end{itemize}
  \bigskip
  \centering
  \begin{tabular}{| l | c | c | c | c |}
    \hline
    \parbox{10em}{Debug information \\ (\funcarg{-g} flag)}  & \multicolumn{2}{|c|}{Disabled}                                     & \multicolumn{2}{|c|}{Enabled} \\
    \hline
    Raise kind                                               & \funcname{raise} & \funcname{raise\_notrace}                       & \funcname{raise} & \funcname{raise\_notrace} \\
    \hline
    Compilation                                              & \parbox{8em}{inline assembly\\ (optimization)} & inline assembly   & \funcname{caml\_raise\_exn} & inline assembly \\
    \hline
  \end{tabular}
\end{frame}


%
% Takeaway #5
%
\subsection*{Takeaway \#5}
\frameSubsectionTakeaway{}
\againframe<5>{takeaway}
}%
      \onslide*<6>{\providecommand\step{2}%
% Backtraces
%
\subsection{One advantage of raising with debugging information}
\frameSubsection{}{}

\begin{frame}{Backtraces}{Debugging information \& source code}
  \begin{columns}[c]
    \begin{column}{0.5\textwidth}
      \begin{itemize}
        \item Raising an exception is fun and games, but how do we know where the exception originated? $\rightarrow$ With the backtrace associated with the exception!
        \item In order to get a backtrace from the point where an exception was raised, it's necessary to compile with debugging information enabled (\funcarg{-g} flag for the compiler)
        \item Same example as in the `Nested handlers' section
      \end{itemize}
    \end{column}
    \begin{column}{0.5\textwidth}
      \listocaml[firstline=9,lastline=34]{../src/backtrace/backtrace.ml}{backtrace/backtrace.ml}
    \end{column}
  \end{columns}
\end{frame}

\begin{frame}{Backtraces}{CMM and disassembly of \funcname{definitely\_raise}}
  \begin{columns}[c]
    \begin{column}{0.5\textwidth}
      \minipage[c][0.66\textheight][s]{\columnwidth}
      \begin{itemize}
        \item<1-> An effect of compiling with debugging information (\funcarg{-g}): the CMM no longer uses \funcname{raise\_notrace} to raise the exception but \funcname{raise} instead
        \item<2-> Disassembly shows that CMM \funcname{raise} is actually a call to \funcname{caml\_raise\_exn}, a function in assembler provided by the runtime
      \end{itemize}
      \vfill
      \mintocaml[firstline=9,lastline=13,highlightlines=12]{../src/backtrace/backtrace.ml}
      \endminipage
    \end{column}
    \begin{column}{0.5\textwidth}
      \onslide*<1>{
        \mintcmm[highlightlines=5]{../src/backtrace/camlBacktrace\_\_definitely\_raise\_347.cmm}
      }
      \onslide*<2>{
        \mintobjdump[firstline=2,highlightlines=14]{../src/backtrace/camlBacktrace\_\_definitely\_raise\_347.objdump}
      }
    \end{column}
  \end{columns}
\end{frame}

\begin{frame}{Backtraces}{Introducing \funcname{caml\_raise\_exn}}
  \begin{columns}[c]
    \begin{column}{0.5\textwidth}
      \minipage[c][0.66\textheight][s]{\columnwidth}
      \onslide*<1>{
        \begin{itemize}
          \item Raising the exception is performed using \funcname{caml\_raise\_exn} which is
            \begin{itemize}
              \item An assembly function provided by the runtime
              \item Architecture specific
            \end{itemize}
          \item Let's see what it does differently from \funcname{raise\_notrace}\dots
          \item We are about to raise \typename{ExnC} with \funcname{caml\_raise\_exn} from \funcname{definitely\_raise}
        \end{itemize}
      }
      \onslide*<2>{
        \mintobjdump[firstline=2,highlightlines=14]{../src/backtrace/camlBacktrace\_\_definitely\_raise\_347.objdump}
      }
      \vfill
      \mintocaml[firstline=9,lastline=13,highlightlines=12]{../src/backtrace/backtrace.ml}
      \endminipage
    \end{column}
    \begin{column}{0.5\textwidth}
      \centering
      \providecommand\step{1}\input{diagrams/backtrace/backtrace.tex}
    \end{column}
  \end{columns}
\end{frame}

\begin{frame}{Backtraces}{But before that, we need more fields from \typename{caml\_domain\_state}}
  \begin{columns}
    \begin{column}{0.5\textwidth}
      \begin{itemize}
        \item Remember \typename{caml\_domain\_state} the per-domain runtime state?
        \item Some other members are of interest to us now\dots
        \item \localname{backtrace\_active} is used to know if the runtime should collect backtraces
        \item \localname{backtrace\_active} can be set by
          \begin{itemize}
            \item The \funcarg{b} flag in the \funcname{OCAMLRUNPARAM} environment variable
            \item \mintinline{ocaml}{Printexc.record_backtrace : bool -> unit} \footnotemark
          \end{itemize}
      \end{itemize}
    \end{column}
    \begin{column}{0.5\textwidth}
      \begin{listing}
        \mintc[highlightlines={9-11}]{src/ocaml/domain\_state\_backtrace.h}
        \caption{runtime/caml/domain\_state.h}
      \end{listing}
    \end{column}
  \end{columns}
  \footnotetext[1]{https://v2.ocaml.org/api/Printexc.html}
\end{frame}

\newcommand\listCamlRaiseExn[1][]{
  \listasm[firstline=702,lastline=725,#1]{../ocaml/runtime/amd64.S}{runtime/amd64.S:702}}

\begin{frame}{Backtraces}{Walk through \funcname{caml\_raise\_exn}}
  \begin{columns}[T]
    \begin{column}{0.5\textwidth}
      \begin{itemize}
        \item<1-> First checks whether exceptions backtrace should be recorded
        \item<1-> By checking the \funcarg{domain\_state->backtrace\_active} value
        \item<2-> If not, acts as \funcname{raise\_notrace} with \funcname{RESTORE\_EXN\_HANDLER\_OCAML}
          \begin{itemize}
            \item Set \regname{RSP} to \localname{domain\_state->exn\_handler} value
            \item Restore \localname{domain\_state->exn\_handler} from trap
            \item Jump with \funcname{ret} (same effect as using \funcname{pop} and \funcname{jmp})
          \end{itemize}
        \item<3-> Otherwise, switches to the C stack to be able to execute C code
        \item<4-> Calls \funcname{caml\_stash\_backtrace}, a C function provided by the runtime\ignorespaces
          \onslide<6->{, with
            \begin{itemize}
              \item \funcarg{exn}: exception value
              \item \funcarg{pc}: code address of the raising point
              \item \funcarg{sp}: stack address of the raising function
              \item \funcarg{trapsp}: stack address of the next handler
            \end{itemize}
          }
      \end{itemize}
      \bigskip
      \mintocaml[firstline=9,lastline=13,highlightlines=12]{../src/backtrace/backtrace.ml}
    \end{column}
    \begin{column}{0.5\textwidth}
      \raggedleft
      \onslide*<1,3-4>{
        \mintc[firstline=190,lastline=192]{../ocaml/runtime/amd64.S}
        \mintc[firstline=234,lastline=237]{../ocaml/runtime/amd64.S}
      }
      \onslide*<2>{
        \mintc[firstline=190,lastline=192,highlightlines={191-192}]{../ocaml/runtime/amd64.S}
        \mintc[firstline=234,lastline=237,highlightlines={235-237}]{../ocaml/runtime/amd64.S}
      }
      \onslide*<1>{\listCamlRaiseExn[highlightlines=705]{}}%
      \onslide*<2>{\listCamlRaiseExn[highlightlines={707-708}]{}}%
      \onslide*<3>{\listCamlRaiseExn[highlightlines={712-714}]{}}%
      \onslide*<4>{\listCamlRaiseExn[highlightlines={715-720}]{}}%
      \onslide*<5>{\providecommand\step{1}\input{diagrams/backtrace/backtrace.tex}}%
      \onslide*<6>{\providecommand\step{2}\input{diagrams/backtrace/backtrace.tex}}%
    \end{column}
  \end{columns}
\end{frame}

\newcommand\listStashBacktrace[1][]{
  \listc[firstline=91,lastline=120,#1]{../ocaml/runtime/backtrace\_nat.c}{runtime/backtrace\_nat.c:91}}

\begin{frame}{Backtraces}{Introducing \funcname{caml\_stash\_backtrace}}
  \begin{columns}[c]
    \begin{column}{0.5\textwidth}
      \minipage[c][0.66\textheight][s]{\columnwidth}
      \begin{itemize}
        \item<1-> A C function provided by the runtime (specific to native code)
        \item<2-> It scans the stack, between the stack pointer at the raising point up to the next trap, in order to collect the names of the detected function frames
          \onslide*<2>{
            \begin{itemize}
              \item And more information, like line number, column number...
            \end{itemize}
          }
        \item<3-> It uses the same technique and information as the Garbage Collector (GC) uses to scan the values live on the stack
          \onslide*<4>{
            \begin{itemize}
              \item \typename{frame\_descr} are at the core of stack unwinding
            \end{itemize}
          }
      \end{itemize}
      \vfill
      \mintocaml[firstline=9,lastline=13,highlightlines=12]{../src/backtrace/backtrace.ml}
      \endminipage
    \end{column}
    \begin{column}{0.5\textwidth}
      \onslide*<1>{\listStashBacktrace[highlightlines=91]{}}
      \onslide*<2>{\listStashBacktrace[highlightlines={91,117-118}]{}}
      \onslide*<3->{\listStashBacktrace[highlightlines={94,106,109-110}]{}}
    \end{column}
  \end{columns}
\end{frame}

\newcommand\listFrameDescr[1][]{
  \listocaml[firstline=111,lastline=124,#1]{../ocaml/asmcomp/emitaux.ml}{asmcomp/emitaux.ml:111}}
\newcommand\listDebugInfo[1][]{
  \listocaml[firstline=38,lastline=49,#1]{../ocaml/lambda/debuginfo.mli}{lambda/debuginfo.mli:38}}

\begin{frame}{Backtraces}{\typename{frame\_descr} and \typename{frame\_debuginfo} in a nutshell}
  \begin{columns}[c]
    \begin{column}{0.5\textwidth}
      \begin{itemize}
        \item<1-> For every return address in an OCaml program, there is a associated \typename{frame\_descr}
        \item<2-> A \typename{frame\_descr} specifies
        \begin{itemize}
          \item<2-> The size of the function stack frame
          \item<3-> Where live values are on the stack (so that the GC can mark them as live and prevent from being swept)
        \end{itemize}
        \item<4-> When compiled with debug information, \typename{frame\_descr} will be linked to a \typename{frame\_debuginfo}
        \begin{itemize}
          \item<5-> Contains information about the position in the source code (file name, line and column number)
        \end{itemize}
      \end{itemize}
    \end{column}
    \begin{column}{0.5\textwidth}
        \onslide*<1>{\listFrameDescr[highlightlines={118-122}]{}}
        \onslide*<2>{\listFrameDescr[highlightlines={120}]{}}
        \onslide*<3>{\listFrameDescr[highlightlines={121}]{}}
        \onslide*<4->{\listFrameDescr[highlightlines={122}]{}}
        \medskip
        \onslide*<1-3>{\listDebugInfo{}}
        \onslide*<4>{\listDebugInfo[highlightlines={38,49}]{}}
        \onslide*<5>{\listDebugInfo[highlightlines={39-46}]{}}
    \end{column}
  \end{columns}
\end{frame}

\begin{frame}{Backtraces}{Scanning the stack with \typename{frame\_descr}}
  \begin{columns}[c]
    \begin{column}{0.5\textwidth}
      \begin{itemize}
        \item Given the return address of the code that raises the exception by calling \funcname{caml\_raise\_exn} (\funcarg{pc} argument of \funcname{caml\_stash\_backtrace})
        \item And the position of the stack pointer (\funcarg{sp} argument of \funcname{caml\_stash\_backtrace})
        \item The frame size given by \typename{frame\_descr}.\funcarg{frame\_size}, allows to find the end of the previous frame
        \item And the return address of the caller of this function
        \bigskip
        \onslide<3->{
        \item Note that traps (here the reddish \funcname{ExnA}, \funcname{ExnB} and \funcname{ExnC} blocks) belong to the frame that installed them, and so the frame size changes when traps are removed
        }
      \end{itemize}
    \end{column}
    \begin{column}{0.5\textwidth}
      \raggedleft
      \onslide*<1>{\providecommand\step{2}\input{diagrams/backtrace/backtrace.tex}}%
      \onslide*<2>{\providecommand\step{1}\input{diagrams/backtrace/scanstack.tex}}%
      \onslide*<3>{\providecommand\step{2}\input{diagrams/backtrace/scanstack.tex}}%
      \onslide*<4>{\providecommand\step{3}\input{diagrams/backtrace/scanstack.tex}}%
      \onslide*<5>{\providecommand\step{4}\input{diagrams/backtrace/scanstack.tex}}%
      \onslide*<6>{\providecommand\step{5}\input{diagrams/backtrace/scanstack.tex}}%
    \end{column}
  \end{columns}
\end{frame}

% Duplicate of previous-previous slide
\begin{frame}{Backtraces}{Continuing on \funcname{caml\_stash\_backtrace}}
  \begin{columns}[c]
    \begin{column}{0.5\textwidth}
      \minipage[c][0.75\textheight][s]{\columnwidth}
      \begin{itemize}
          \color{gray}
        \item A C function provided by the runtime (specific to native code)
        \item It scans the stack, between the stack pointer at the raising point up to the next trap, in order to collect the names of the detected function frames
        \item It uses the same technique and information as the Garbage Collector (GC) uses to scan the values live on the stack
      \end{itemize}
      \begin{itemize}
        \item For each function frame detected, store a pointer to the \typename{frame\_descr} in the \localname{domain\_state->backtrace\_buffer} array, effectively constructing the backtrace
      \end{itemize}
      \onslide<2->{
        \begin{itemize}
            \smallskip
          \item Constructed backtrace:
            \funcname{definitely\_raise},
            \onslide<3->{\funcname{probably\_raise}}
        \end{itemize}
      }
      \vfill
      \mintocaml[firstline=9,lastline=13,highlightlines=12]{../src/backtrace/backtrace.ml}
      \endminipage
    \end{column}
    \begin{column}{0.5\textwidth}
      \raggedleft
      \onslide*<1>{\listStashBacktrace[highlightlines={114-115}]{}}
      \onslide*<2>{\providecommand\step{3}\input{diagrams/backtrace/backtrace.tex}}%
      \onslide*<3>{\providecommand\step{4}\input{diagrams/backtrace/backtrace.tex}}%
    \end{column}
  \end{columns}
\end{frame}

\begin{frame}{Backtraces}{Back to \funcname{caml\_raise\_exn}}
  \begin{columns}[c]
    \begin{column}{0.5\textwidth}
      \minipage[c][0.66\textheight][s]{\columnwidth}
      \begin{itemize}\color{gray}
        \item Checks wheteher exceptions backtrace should be recorded, by checking the \funcarg{domain\_state->backtrace\_active} value
        \item Switches to the C stack to be able to execute C code
        \item Calls \funcname{caml\_stash\_backtrace}, to collect the backtrace up to the next trap
      \end{itemize}
      \onslide*<2->{
        \begin{itemize}
          \item<2-> Executes the exception handler in \funcname{probably\_raise} with \funcname{RESTORE\_EXN\_HANDLER\_OCAML}
            \begin{itemize}
              \item Set \regname{RSP} to \localname{domain\_state->exn\_handler} value
              \item Restore \localname{domain\_state->exn\_handler} from trap
              \item Jump to the handler code with \funcname{ret}
            \end{itemize}
        \end{itemize}
      }
      \vfill
      \onslide*<1>{\mintocaml[firstline=9,lastline=13,highlightlines=12]{../src/backtrace/backtrace.ml}}
      \onslide*<2->{\mintocaml[firstline=15,lastline=21,highlightlines={19-21}]{../src/backtrace/backtrace.ml}}
      \endminipage
    \end{column}
    \begin{column}{0.5\textwidth}
      \centering
      \onslide*<1>{
        \mintc[firstline=190,lastline=192]{../ocaml/runtime/amd64.S}
        \mintc[firstline=234,lastline=237]{../ocaml/runtime/amd64.S}
      }
      \onslide*<2>{
        \mintc[firstline=190,lastline=192,highlightlines={191-192}]{../ocaml/runtime/amd64.S}
        \mintc[firstline=234,lastline=237,highlightlines={235-237}]{../ocaml/runtime/amd64.S}
      }
      \onslide*<1>{\listCamlRaiseExn[highlightlines=720]{}}
      \onslide*<2>{\listCamlRaiseExn[highlightlines=722]{}}
      \onslide*<3->{\providecommand\step{5}\input{diagrams/backtrace/backtrace.tex}}
    \end{column}
  \end{columns}
\end{frame}

\begin{frame}{Backtraces}{\funcname{caml\_probably\_raise}'s handler doesn't catch \typename{ExnC}}
  \begin{columns}[c]
    \begin{column}{0.45\textwidth}
      \minipage[c][0.75\textheight][s]{\columnwidth}
      \begin{itemize}
        \item<1-> \funcname{match \dots with}\footnotemark pattern matching can also match exceptions
        \item<1-> The CMM of \funcname{probably\_raise} shows that this \funcname{match \dots with} is implemented with a pattern matching and a \funcname{try \dots with}
        \item<1-> But this one only matches on \typename{ExnA} exception
        \item<2-> Since the pattern matching fails to catch the exception, \funcname{reraise} is called with our current \typename{ExnC} exception
        \item<3-> Disassembly shows that \funcname{reraise} is actually a call to \funcname{caml\_reraise\_exn}, which is also an assembly function provided by the runtime
      \end{itemize}
      \vfill
      \mintocaml[firstline=15,lastline=21,highlightlines={18-21}]{../src/backtrace/backtrace.ml}
      \endminipage
    \end{column}
    \begin{column}{0.55\textwidth}
      \centering
      \onslide*<1>{\mintcmm[highlightlines={10-18}]{../src/backtrace/camlBacktrace\_\_probably\_raise\_351.cmm}}
      \onslide*<2>{\listcmm[highlightlines=18]{../src/backtrace/camlBacktrace\_\_probably\_raise_351.cmm}{CMM of probably\_raise}}
      \onslide*<3>{\mintobjdump[firstline=5,lastline=35,highlightlines=31]{../src/backtrace/camlBacktrace\_\_probably\_raise_351.objdump}}
    \end{column}
  \end{columns}
  \only<1>{\footnotetext[1]{https://blog.janestreet.com/pattern-matching-and-exception-handling-unite/}}
\end{frame}

\newcommand\listCamlReraiseExn[1][]{
  \listasm[firstline=727,lastline=734,#1]{../ocaml/runtime/amd64.S}{runtime/amd64.S:727}}

\begin{frame}{Backtraces}{Introducing \funcname{caml\_reraise\_exn}}
  \begin{columns}[c]
    \begin{column}{0.5\textwidth}
      \minipage[c][0.66\textheight][s]{\columnwidth}
      \begin{itemize}
        \item<1-> The exception have to be reraised to the next handler
        \item<2-> First, checks if the backtrace should be collected with \funcarg{domain\_state->backtrace\_active}
        \only<3>{
        \begin{itemize}
            \item If not, behaves like a \funcname{raise\_notrace} with \funcname{RESTORE\_EXN\_HANDLER\_OCAML}
        \end{itemize}
        }
        \item<4-> Jumps into \funcname{caml\_raise\_exn}
        \item<5-> Calls \funcname{caml\_stash\_backtrace} again, to scan the next part of the stack: from the trap just pop-ed to the next trap
      \end{itemize}
      \vfill
      \mintocaml[firstline=15,lastline=21,highlightlines={18-21}]{../src/backtrace/backtrace.ml}
      \endminipage
    \end{column}
    \begin{column}{0.5\textwidth}
      \raggedleft
      \onslide*<1>{\listCamlReraiseExn{}}
      \onslide*<2>{\listCamlReraiseExn[highlightlines=729]{}}
      \onslide*<3>{
        \mintc[firstline=190,lastline=192,highlightlines={191-192}]{../ocaml/runtime/amd64.S}
        \mintc[firstline=234,lastline=237,highlightlines={235-237}]{../ocaml/runtime/amd64.S}
        \medskip
        \listCamlReraiseExn[highlightlines=731]{}
      }
      \onslide*<4-5>{
        % TODO refactor with \listCamlReraiseExn
        \mintasm[firstline=727,lastline=734,highlightlines=730]{../ocaml/runtime/amd64.S}
        \medskip
      }
      \onslide*<4>{\listCamlRaiseExn[highlightlines=711]{}}
      \onslide*<5>{\listCamlRaiseExn[highlightlines=720]{}}
    \end{column}
  \end{columns}
\end{frame}

\begin{frame}{Backtraces}{Backtrace collection continues}
  \begin{columns}[c]
    \begin{column}{0.55\textwidth}
      \minipage[c][0.75\textheight][s]{\columnwidth}
      \begin{itemize}
        \item<1-> \funcname{caml\_stash\_backtrace} scans the older part of the stack, up to the next trap
        \item<6-> Controls jumps to the nested exception handler in \funcname{maybe\_raise}, which matches only on \typename{ExnB}
        \item<7-> \funcname{caml\_reraise\_exn} is called again, backtrace collection resumes with \funcname{caml\_stash\_backtrace}
      \end{itemize}
      \medskip
      \begin{itemize}
        \item<2-> Constructed backtrace:
        \begin{itemize}
          \item <2-> \funcname{definitely\_raise}
          \item <2-> \funcname{probably\_raise}
          \item <3-> \funcname{probably\_raise} (re-raise)
          \item <4-> \funcname{maybe\_raise}
          \item <5-> \funcname{main}
          \item <8-> \funcname{main} (re-raise)
        \end{itemize}
      \end{itemize}
      \vfill
      \onslide*<1-5>{\mintocaml[firstline=15,lastline=21]{../src/backtrace/backtrace.ml}}
      \onslide*<6>{\mintocaml[firstline=28,lastline=34,highlightlines=31]{../src/backtrace/backtrace.ml}}
      \onslide*<7->{\mintocaml[firstline=28,lastline=34,highlightlines=33]{../src/backtrace/backtrace.ml}}
      \endminipage
    \end{column}
    \begin{column}{0.45\textwidth}
      \raggedleft
      \onslide*<1>{\providecommand\step{6}\input{diagrams/backtrace/backtrace.tex}}%
      \onslide*<2>{\providecommand\step{6}\input{diagrams/backtrace/backtrace.tex}}%
      \onslide*<3>{\providecommand\step{7}\input{diagrams/backtrace/backtrace.tex}}%
      \onslide*<4>{\providecommand\step{8}\input{diagrams/backtrace/backtrace.tex}}%
      \onslide*<5>{\providecommand\step{9}\input{diagrams/backtrace/backtrace.tex}}%
      \onslide*<6>{\providecommand\step{10}\input{diagrams/backtrace/backtrace.tex}}%
      \onslide*<7>{\providecommand\step{11}\input{diagrams/backtrace/backtrace.tex}}%
      \onslide*<8>{\providecommand\step{12}\input{diagrams/backtrace/backtrace.tex}}%
      \onslide*<9>{\providecommand\step{13}\input{diagrams/backtrace/backtrace.tex}}%
    \end{column}
  \end{columns}
\end{frame}

\begin{frame}{Backtraces}{Printing the backtrace}
  \begin{columns}[c]
    \begin{column}{0.45\textwidth}
      \minipage[c][0.66\textheight][s]{\columnwidth}
      \begin{itemize}
        \item<1-> The exception backtrace can now be printed to \funcarg{stdout} with \funcname{Printexc.print_backtrace}
        \item<2-> First the exception backtrace is retrieved into an OCaml array with \funcname{get_raw_backtrace }
        \item<3-> Which is implemented as the \funcname{caml_get_exception_raw_backtrace} C function in the runtime
        \item<4-> And finally printed with \funcname{print_exception_backtrace}
      \end{itemize}
      \vfill
      \mintocaml[firstline=28,lastline=34,highlightlines=33]{../src/backtrace/backtrace.ml}
      \endminipage
    \end{column}
    \begin{column}{0.55\textwidth}
      \onslide*<1,2>{
        \mintocaml[firstline=100,lastline=101]{../ocaml/stdlib/printexc.ml}
      }
      \only<1>{
        \listocaml[firstline=170,lastline=172]{../ocaml/stdlib/printexc.ml}{stdlib/printexc.ml:170}
      }
      \only<2>{
        \listocaml[firstline=170,lastline=172,highlightlines=172]{../ocaml/stdlib/printexc.ml}{stdlib/printexc.ml:170}
      }
      \only<3>{
        \mintc[firstline=154,lastline=156]{../ocaml/runtime/backtrace.c}
        \listc[firstline=171,lastline=192,highlightlines={184-187}]{../ocaml/runtime/backtrace.c}{runtime/backtrace.c:154}
      }
      \only<4>{
        \listocaml[firstline=155,lastline=165]{../ocaml/stdlib/printexc.ml}{stdlib/printexc.ml:55}
      }
    \end{column}
  \end{columns}
\end{frame}


\subsection{The multiple kinds of raise}
\frameSubsection{}{}

\begin{frame}{Backtraces}{When backtraces are not needed}
  \begin{columns}[c]
    \begin{column}{0.45\textwidth}
      \minipage[c][0.66\textheight][s]{\columnwidth}
      \begin{itemize}
        \item<1-> What if the backtrace for an exception is never needed?
        \item<2-> It's possible to raise the exception explicitly with \funcname{raise\_notrace} to make raising as cheap as possible
        \item<3-> An example of use in the \funcname{dynlink} library of the OCaml compiler
      \end{itemize}
      \vfill
      \onslide<3->{
      \listocaml[firstline=205,lastline=209,highlightlines=207]{../ocaml/otherlibs/dynlink/dynlink\_compilerlibs/misc.ml}{otherlibs/dynlink/dynlink\_compilerlibs/misc.ml:205}
      }
      \endminipage
    \end{column}
    \begin{column}{0.55\textwidth}
    \only<4>{
      \mintcmm[highlightlines=3]{../src/notrace/camlNotrace\_\_fun_347.cmm}
      \listcmm{../src/notrace/camlNotrace\_\_all\_somes\_327.cmm}{all\_somes}
    }
    \onslide*<5->{
      \listobjdump[highlightlines={6-9}]{../src/notrace/camlNotrace\_\_fun_347.objdump}{anonymous function from \funcname{all\_somes}}
    }
    \end{column}
  \end{columns}
\end{frame}

\begin{frame}{Backtraces}{Summary table of the multiple kinds of raise}
  \begin{itemize}
    \item When debugging information is enabled and if the backtrace of an exception isn't needed, it's possible to raise with \funcname{raise\_notrace} for performance
    \item When debugging information is \emph{not} enabled, the compiler optimises \funcname{raise} calls to inline assembly (same effect as using \funcname{raise\_notrace})
  \end{itemize}
  \bigskip
  \centering
  \begin{tabular}{| l | c | c | c | c |}
    \hline
    \parbox{10em}{Debug information \\ (\funcarg{-g} flag)}  & \multicolumn{2}{|c|}{Disabled}                                     & \multicolumn{2}{|c|}{Enabled} \\
    \hline
    Raise kind                                               & \funcname{raise} & \funcname{raise\_notrace}                       & \funcname{raise} & \funcname{raise\_notrace} \\
    \hline
    Compilation                                              & \parbox{8em}{inline assembly\\ (optimization)} & inline assembly   & \funcname{caml\_raise\_exn} & inline assembly \\
    \hline
  \end{tabular}
\end{frame}


%
% Takeaway #5
%
\subsection*{Takeaway \#5}
\frameSubsectionTakeaway{}
\againframe<5>{takeaway}
}%
    \end{column}
  \end{columns}
\end{frame}

\newcommand\listStashBacktrace[1][]{
  \listc[firstline=91,lastline=120,#1]{../ocaml/runtime/backtrace\_nat.c}{runtime/backtrace\_nat.c:91}}

\begin{frame}{Backtraces}{Introducing \funcname{caml\_stash\_backtrace}}
  \begin{columns}[c]
    \begin{column}{0.5\textwidth}
      \minipage[c][0.66\textheight][s]{\columnwidth}
      \begin{itemize}
        \item<1-> A C function provided by the runtime (specific to native code)
        \item<2-> It scans the stack, between the stack pointer at the raising point up to the next trap, in order to collect the names of the detected function frames
          \onslide*<2>{
            \begin{itemize}
              \item And more information, like line number, column number...
            \end{itemize}
          }
        \item<3-> It uses the same technique and information as the Garbage Collector (GC) uses to scan the values live on the stack
          \onslide*<4>{
            \begin{itemize}
              \item \typename{frame\_descr} are at the core of stack unwinding
            \end{itemize}
          }
      \end{itemize}
      \vfill
      \mintocaml[firstline=9,lastline=13,highlightlines=12]{../src/backtrace/backtrace.ml}
      \endminipage
    \end{column}
    \begin{column}{0.5\textwidth}
      \onslide*<1>{\listStashBacktrace[highlightlines=91]{}}
      \onslide*<2>{\listStashBacktrace[highlightlines={91,117-118}]{}}
      \onslide*<3->{\listStashBacktrace[highlightlines={94,106,109-110}]{}}
    \end{column}
  \end{columns}
\end{frame}

\newcommand\listFrameDescr[1][]{
  \listocaml[firstline=111,lastline=124,#1]{../ocaml/asmcomp/emitaux.ml}{asmcomp/emitaux.ml:111}}
\newcommand\listDebugInfo[1][]{
  \listocaml[firstline=38,lastline=49,#1]{../ocaml/lambda/debuginfo.mli}{lambda/debuginfo.mli:38}}

\begin{frame}{Backtraces}{\typename{frame\_descr} and \typename{frame\_debuginfo} in a nutshell}
  \begin{columns}[c]
    \begin{column}{0.5\textwidth}
      \begin{itemize}
        \item<1-> For every return address in an OCaml program, there is a associated \typename{frame\_descr}
        \item<2-> A \typename{frame\_descr} specifies
        \begin{itemize}
          \item<2-> The size of the function stack frame
          \item<3-> Where live values are on the stack (so that the GC can mark them as live and prevent from being swept)
        \end{itemize}
        \item<4-> When compiled with debug information, \typename{frame\_descr} will be linked to a \typename{frame\_debuginfo}
        \begin{itemize}
          \item<5-> Contains information about the position in the source code (file name, line and column number)
        \end{itemize}
      \end{itemize}
    \end{column}
    \begin{column}{0.5\textwidth}
        \onslide*<1>{\listFrameDescr[highlightlines={118-122}]{}}
        \onslide*<2>{\listFrameDescr[highlightlines={120}]{}}
        \onslide*<3>{\listFrameDescr[highlightlines={121}]{}}
        \onslide*<4->{\listFrameDescr[highlightlines={122}]{}}
        \medskip
        \onslide*<1-3>{\listDebugInfo{}}
        \onslide*<4>{\listDebugInfo[highlightlines={38,49}]{}}
        \onslide*<5>{\listDebugInfo[highlightlines={39-46}]{}}
    \end{column}
  \end{columns}
\end{frame}

\begin{frame}{Backtraces}{Scanning the stack with \typename{frame\_descr}}
  \begin{columns}[c]
    \begin{column}{0.5\textwidth}
      \begin{itemize}
        \item Given the return address of the code that raises the exception by calling \funcname{caml\_raise\_exn} (\funcarg{pc} argument of \funcname{caml\_stash\_backtrace})
        \item And the position of the stack pointer (\funcarg{sp} argument of \funcname{caml\_stash\_backtrace})
        \item The frame size given by \typename{frame\_descr}.\funcarg{frame\_size}, allows to find the end of the previous frame
        \item And the return address of the caller of this function
        \bigskip
        \onslide<3->{
        \item Note that traps (here the reddish \funcname{ExnA}, \funcname{ExnB} and \funcname{ExnC} blocks) belong to the frame that installed them, and so the frame size changes when traps are removed
        }
      \end{itemize}
    \end{column}
    \begin{column}{0.5\textwidth}
      \raggedleft
      \onslide*<1>{\providecommand\step{2}%
% Backtraces
%
\subsection{One advantage of raising with debugging information}
\frameSubsection{}{}

\begin{frame}{Backtraces}{Debugging information \& source code}
  \begin{columns}[c]
    \begin{column}{0.5\textwidth}
      \begin{itemize}
        \item Raising an exception is fun and games, but how do we know where the exception originated? $\rightarrow$ With the backtrace associated with the exception!
        \item In order to get a backtrace from the point where an exception was raised, it's necessary to compile with debugging information enabled (\funcarg{-g} flag for the compiler)
        \item Same example as in the `Nested handlers' section
      \end{itemize}
    \end{column}
    \begin{column}{0.5\textwidth}
      \listocaml[firstline=9,lastline=34]{../src/backtrace/backtrace.ml}{backtrace/backtrace.ml}
    \end{column}
  \end{columns}
\end{frame}

\begin{frame}{Backtraces}{CMM and disassembly of \funcname{definitely\_raise}}
  \begin{columns}[c]
    \begin{column}{0.5\textwidth}
      \minipage[c][0.66\textheight][s]{\columnwidth}
      \begin{itemize}
        \item<1-> An effect of compiling with debugging information (\funcarg{-g}): the CMM no longer uses \funcname{raise\_notrace} to raise the exception but \funcname{raise} instead
        \item<2-> Disassembly shows that CMM \funcname{raise} is actually a call to \funcname{caml\_raise\_exn}, a function in assembler provided by the runtime
      \end{itemize}
      \vfill
      \mintocaml[firstline=9,lastline=13,highlightlines=12]{../src/backtrace/backtrace.ml}
      \endminipage
    \end{column}
    \begin{column}{0.5\textwidth}
      \onslide*<1>{
        \mintcmm[highlightlines=5]{../src/backtrace/camlBacktrace\_\_definitely\_raise\_347.cmm}
      }
      \onslide*<2>{
        \mintobjdump[firstline=2,highlightlines=14]{../src/backtrace/camlBacktrace\_\_definitely\_raise\_347.objdump}
      }
    \end{column}
  \end{columns}
\end{frame}

\begin{frame}{Backtraces}{Introducing \funcname{caml\_raise\_exn}}
  \begin{columns}[c]
    \begin{column}{0.5\textwidth}
      \minipage[c][0.66\textheight][s]{\columnwidth}
      \onslide*<1>{
        \begin{itemize}
          \item Raising the exception is performed using \funcname{caml\_raise\_exn} which is
            \begin{itemize}
              \item An assembly function provided by the runtime
              \item Architecture specific
            \end{itemize}
          \item Let's see what it does differently from \funcname{raise\_notrace}\dots
          \item We are about to raise \typename{ExnC} with \funcname{caml\_raise\_exn} from \funcname{definitely\_raise}
        \end{itemize}
      }
      \onslide*<2>{
        \mintobjdump[firstline=2,highlightlines=14]{../src/backtrace/camlBacktrace\_\_definitely\_raise\_347.objdump}
      }
      \vfill
      \mintocaml[firstline=9,lastline=13,highlightlines=12]{../src/backtrace/backtrace.ml}
      \endminipage
    \end{column}
    \begin{column}{0.5\textwidth}
      \centering
      \providecommand\step{1}\input{diagrams/backtrace/backtrace.tex}
    \end{column}
  \end{columns}
\end{frame}

\begin{frame}{Backtraces}{But before that, we need more fields from \typename{caml\_domain\_state}}
  \begin{columns}
    \begin{column}{0.5\textwidth}
      \begin{itemize}
        \item Remember \typename{caml\_domain\_state} the per-domain runtime state?
        \item Some other members are of interest to us now\dots
        \item \localname{backtrace\_active} is used to know if the runtime should collect backtraces
        \item \localname{backtrace\_active} can be set by
          \begin{itemize}
            \item The \funcarg{b} flag in the \funcname{OCAMLRUNPARAM} environment variable
            \item \mintinline{ocaml}{Printexc.record_backtrace : bool -> unit} \footnotemark
          \end{itemize}
      \end{itemize}
    \end{column}
    \begin{column}{0.5\textwidth}
      \begin{listing}
        \mintc[highlightlines={9-11}]{src/ocaml/domain\_state\_backtrace.h}
        \caption{runtime/caml/domain\_state.h}
      \end{listing}
    \end{column}
  \end{columns}
  \footnotetext[1]{https://v2.ocaml.org/api/Printexc.html}
\end{frame}

\newcommand\listCamlRaiseExn[1][]{
  \listasm[firstline=702,lastline=725,#1]{../ocaml/runtime/amd64.S}{runtime/amd64.S:702}}

\begin{frame}{Backtraces}{Walk through \funcname{caml\_raise\_exn}}
  \begin{columns}[T]
    \begin{column}{0.5\textwidth}
      \begin{itemize}
        \item<1-> First checks whether exceptions backtrace should be recorded
        \item<1-> By checking the \funcarg{domain\_state->backtrace\_active} value
        \item<2-> If not, acts as \funcname{raise\_notrace} with \funcname{RESTORE\_EXN\_HANDLER\_OCAML}
          \begin{itemize}
            \item Set \regname{RSP} to \localname{domain\_state->exn\_handler} value
            \item Restore \localname{domain\_state->exn\_handler} from trap
            \item Jump with \funcname{ret} (same effect as using \funcname{pop} and \funcname{jmp})
          \end{itemize}
        \item<3-> Otherwise, switches to the C stack to be able to execute C code
        \item<4-> Calls \funcname{caml\_stash\_backtrace}, a C function provided by the runtime\ignorespaces
          \onslide<6->{, with
            \begin{itemize}
              \item \funcarg{exn}: exception value
              \item \funcarg{pc}: code address of the raising point
              \item \funcarg{sp}: stack address of the raising function
              \item \funcarg{trapsp}: stack address of the next handler
            \end{itemize}
          }
      \end{itemize}
      \bigskip
      \mintocaml[firstline=9,lastline=13,highlightlines=12]{../src/backtrace/backtrace.ml}
    \end{column}
    \begin{column}{0.5\textwidth}
      \raggedleft
      \onslide*<1,3-4>{
        \mintc[firstline=190,lastline=192]{../ocaml/runtime/amd64.S}
        \mintc[firstline=234,lastline=237]{../ocaml/runtime/amd64.S}
      }
      \onslide*<2>{
        \mintc[firstline=190,lastline=192,highlightlines={191-192}]{../ocaml/runtime/amd64.S}
        \mintc[firstline=234,lastline=237,highlightlines={235-237}]{../ocaml/runtime/amd64.S}
      }
      \onslide*<1>{\listCamlRaiseExn[highlightlines=705]{}}%
      \onslide*<2>{\listCamlRaiseExn[highlightlines={707-708}]{}}%
      \onslide*<3>{\listCamlRaiseExn[highlightlines={712-714}]{}}%
      \onslide*<4>{\listCamlRaiseExn[highlightlines={715-720}]{}}%
      \onslide*<5>{\providecommand\step{1}\input{diagrams/backtrace/backtrace.tex}}%
      \onslide*<6>{\providecommand\step{2}\input{diagrams/backtrace/backtrace.tex}}%
    \end{column}
  \end{columns}
\end{frame}

\newcommand\listStashBacktrace[1][]{
  \listc[firstline=91,lastline=120,#1]{../ocaml/runtime/backtrace\_nat.c}{runtime/backtrace\_nat.c:91}}

\begin{frame}{Backtraces}{Introducing \funcname{caml\_stash\_backtrace}}
  \begin{columns}[c]
    \begin{column}{0.5\textwidth}
      \minipage[c][0.66\textheight][s]{\columnwidth}
      \begin{itemize}
        \item<1-> A C function provided by the runtime (specific to native code)
        \item<2-> It scans the stack, between the stack pointer at the raising point up to the next trap, in order to collect the names of the detected function frames
          \onslide*<2>{
            \begin{itemize}
              \item And more information, like line number, column number...
            \end{itemize}
          }
        \item<3-> It uses the same technique and information as the Garbage Collector (GC) uses to scan the values live on the stack
          \onslide*<4>{
            \begin{itemize}
              \item \typename{frame\_descr} are at the core of stack unwinding
            \end{itemize}
          }
      \end{itemize}
      \vfill
      \mintocaml[firstline=9,lastline=13,highlightlines=12]{../src/backtrace/backtrace.ml}
      \endminipage
    \end{column}
    \begin{column}{0.5\textwidth}
      \onslide*<1>{\listStashBacktrace[highlightlines=91]{}}
      \onslide*<2>{\listStashBacktrace[highlightlines={91,117-118}]{}}
      \onslide*<3->{\listStashBacktrace[highlightlines={94,106,109-110}]{}}
    \end{column}
  \end{columns}
\end{frame}

\newcommand\listFrameDescr[1][]{
  \listocaml[firstline=111,lastline=124,#1]{../ocaml/asmcomp/emitaux.ml}{asmcomp/emitaux.ml:111}}
\newcommand\listDebugInfo[1][]{
  \listocaml[firstline=38,lastline=49,#1]{../ocaml/lambda/debuginfo.mli}{lambda/debuginfo.mli:38}}

\begin{frame}{Backtraces}{\typename{frame\_descr} and \typename{frame\_debuginfo} in a nutshell}
  \begin{columns}[c]
    \begin{column}{0.5\textwidth}
      \begin{itemize}
        \item<1-> For every return address in an OCaml program, there is a associated \typename{frame\_descr}
        \item<2-> A \typename{frame\_descr} specifies
        \begin{itemize}
          \item<2-> The size of the function stack frame
          \item<3-> Where live values are on the stack (so that the GC can mark them as live and prevent from being swept)
        \end{itemize}
        \item<4-> When compiled with debug information, \typename{frame\_descr} will be linked to a \typename{frame\_debuginfo}
        \begin{itemize}
          \item<5-> Contains information about the position in the source code (file name, line and column number)
        \end{itemize}
      \end{itemize}
    \end{column}
    \begin{column}{0.5\textwidth}
        \onslide*<1>{\listFrameDescr[highlightlines={118-122}]{}}
        \onslide*<2>{\listFrameDescr[highlightlines={120}]{}}
        \onslide*<3>{\listFrameDescr[highlightlines={121}]{}}
        \onslide*<4->{\listFrameDescr[highlightlines={122}]{}}
        \medskip
        \onslide*<1-3>{\listDebugInfo{}}
        \onslide*<4>{\listDebugInfo[highlightlines={38,49}]{}}
        \onslide*<5>{\listDebugInfo[highlightlines={39-46}]{}}
    \end{column}
  \end{columns}
\end{frame}

\begin{frame}{Backtraces}{Scanning the stack with \typename{frame\_descr}}
  \begin{columns}[c]
    \begin{column}{0.5\textwidth}
      \begin{itemize}
        \item Given the return address of the code that raises the exception by calling \funcname{caml\_raise\_exn} (\funcarg{pc} argument of \funcname{caml\_stash\_backtrace})
        \item And the position of the stack pointer (\funcarg{sp} argument of \funcname{caml\_stash\_backtrace})
        \item The frame size given by \typename{frame\_descr}.\funcarg{frame\_size}, allows to find the end of the previous frame
        \item And the return address of the caller of this function
        \bigskip
        \onslide<3->{
        \item Note that traps (here the reddish \funcname{ExnA}, \funcname{ExnB} and \funcname{ExnC} blocks) belong to the frame that installed them, and so the frame size changes when traps are removed
        }
      \end{itemize}
    \end{column}
    \begin{column}{0.5\textwidth}
      \raggedleft
      \onslide*<1>{\providecommand\step{2}\input{diagrams/backtrace/backtrace.tex}}%
      \onslide*<2>{\providecommand\step{1}\input{diagrams/backtrace/scanstack.tex}}%
      \onslide*<3>{\providecommand\step{2}\input{diagrams/backtrace/scanstack.tex}}%
      \onslide*<4>{\providecommand\step{3}\input{diagrams/backtrace/scanstack.tex}}%
      \onslide*<5>{\providecommand\step{4}\input{diagrams/backtrace/scanstack.tex}}%
      \onslide*<6>{\providecommand\step{5}\input{diagrams/backtrace/scanstack.tex}}%
    \end{column}
  \end{columns}
\end{frame}

% Duplicate of previous-previous slide
\begin{frame}{Backtraces}{Continuing on \funcname{caml\_stash\_backtrace}}
  \begin{columns}[c]
    \begin{column}{0.5\textwidth}
      \minipage[c][0.75\textheight][s]{\columnwidth}
      \begin{itemize}
          \color{gray}
        \item A C function provided by the runtime (specific to native code)
        \item It scans the stack, between the stack pointer at the raising point up to the next trap, in order to collect the names of the detected function frames
        \item It uses the same technique and information as the Garbage Collector (GC) uses to scan the values live on the stack
      \end{itemize}
      \begin{itemize}
        \item For each function frame detected, store a pointer to the \typename{frame\_descr} in the \localname{domain\_state->backtrace\_buffer} array, effectively constructing the backtrace
      \end{itemize}
      \onslide<2->{
        \begin{itemize}
            \smallskip
          \item Constructed backtrace:
            \funcname{definitely\_raise},
            \onslide<3->{\funcname{probably\_raise}}
        \end{itemize}
      }
      \vfill
      \mintocaml[firstline=9,lastline=13,highlightlines=12]{../src/backtrace/backtrace.ml}
      \endminipage
    \end{column}
    \begin{column}{0.5\textwidth}
      \raggedleft
      \onslide*<1>{\listStashBacktrace[highlightlines={114-115}]{}}
      \onslide*<2>{\providecommand\step{3}\input{diagrams/backtrace/backtrace.tex}}%
      \onslide*<3>{\providecommand\step{4}\input{diagrams/backtrace/backtrace.tex}}%
    \end{column}
  \end{columns}
\end{frame}

\begin{frame}{Backtraces}{Back to \funcname{caml\_raise\_exn}}
  \begin{columns}[c]
    \begin{column}{0.5\textwidth}
      \minipage[c][0.66\textheight][s]{\columnwidth}
      \begin{itemize}\color{gray}
        \item Checks wheteher exceptions backtrace should be recorded, by checking the \funcarg{domain\_state->backtrace\_active} value
        \item Switches to the C stack to be able to execute C code
        \item Calls \funcname{caml\_stash\_backtrace}, to collect the backtrace up to the next trap
      \end{itemize}
      \onslide*<2->{
        \begin{itemize}
          \item<2-> Executes the exception handler in \funcname{probably\_raise} with \funcname{RESTORE\_EXN\_HANDLER\_OCAML}
            \begin{itemize}
              \item Set \regname{RSP} to \localname{domain\_state->exn\_handler} value
              \item Restore \localname{domain\_state->exn\_handler} from trap
              \item Jump to the handler code with \funcname{ret}
            \end{itemize}
        \end{itemize}
      }
      \vfill
      \onslide*<1>{\mintocaml[firstline=9,lastline=13,highlightlines=12]{../src/backtrace/backtrace.ml}}
      \onslide*<2->{\mintocaml[firstline=15,lastline=21,highlightlines={19-21}]{../src/backtrace/backtrace.ml}}
      \endminipage
    \end{column}
    \begin{column}{0.5\textwidth}
      \centering
      \onslide*<1>{
        \mintc[firstline=190,lastline=192]{../ocaml/runtime/amd64.S}
        \mintc[firstline=234,lastline=237]{../ocaml/runtime/amd64.S}
      }
      \onslide*<2>{
        \mintc[firstline=190,lastline=192,highlightlines={191-192}]{../ocaml/runtime/amd64.S}
        \mintc[firstline=234,lastline=237,highlightlines={235-237}]{../ocaml/runtime/amd64.S}
      }
      \onslide*<1>{\listCamlRaiseExn[highlightlines=720]{}}
      \onslide*<2>{\listCamlRaiseExn[highlightlines=722]{}}
      \onslide*<3->{\providecommand\step{5}\input{diagrams/backtrace/backtrace.tex}}
    \end{column}
  \end{columns}
\end{frame}

\begin{frame}{Backtraces}{\funcname{caml\_probably\_raise}'s handler doesn't catch \typename{ExnC}}
  \begin{columns}[c]
    \begin{column}{0.45\textwidth}
      \minipage[c][0.75\textheight][s]{\columnwidth}
      \begin{itemize}
        \item<1-> \funcname{match \dots with}\footnotemark pattern matching can also match exceptions
        \item<1-> The CMM of \funcname{probably\_raise} shows that this \funcname{match \dots with} is implemented with a pattern matching and a \funcname{try \dots with}
        \item<1-> But this one only matches on \typename{ExnA} exception
        \item<2-> Since the pattern matching fails to catch the exception, \funcname{reraise} is called with our current \typename{ExnC} exception
        \item<3-> Disassembly shows that \funcname{reraise} is actually a call to \funcname{caml\_reraise\_exn}, which is also an assembly function provided by the runtime
      \end{itemize}
      \vfill
      \mintocaml[firstline=15,lastline=21,highlightlines={18-21}]{../src/backtrace/backtrace.ml}
      \endminipage
    \end{column}
    \begin{column}{0.55\textwidth}
      \centering
      \onslide*<1>{\mintcmm[highlightlines={10-18}]{../src/backtrace/camlBacktrace\_\_probably\_raise\_351.cmm}}
      \onslide*<2>{\listcmm[highlightlines=18]{../src/backtrace/camlBacktrace\_\_probably\_raise_351.cmm}{CMM of probably\_raise}}
      \onslide*<3>{\mintobjdump[firstline=5,lastline=35,highlightlines=31]{../src/backtrace/camlBacktrace\_\_probably\_raise_351.objdump}}
    \end{column}
  \end{columns}
  \only<1>{\footnotetext[1]{https://blog.janestreet.com/pattern-matching-and-exception-handling-unite/}}
\end{frame}

\newcommand\listCamlReraiseExn[1][]{
  \listasm[firstline=727,lastline=734,#1]{../ocaml/runtime/amd64.S}{runtime/amd64.S:727}}

\begin{frame}{Backtraces}{Introducing \funcname{caml\_reraise\_exn}}
  \begin{columns}[c]
    \begin{column}{0.5\textwidth}
      \minipage[c][0.66\textheight][s]{\columnwidth}
      \begin{itemize}
        \item<1-> The exception have to be reraised to the next handler
        \item<2-> First, checks if the backtrace should be collected with \funcarg{domain\_state->backtrace\_active}
        \only<3>{
        \begin{itemize}
            \item If not, behaves like a \funcname{raise\_notrace} with \funcname{RESTORE\_EXN\_HANDLER\_OCAML}
        \end{itemize}
        }
        \item<4-> Jumps into \funcname{caml\_raise\_exn}
        \item<5-> Calls \funcname{caml\_stash\_backtrace} again, to scan the next part of the stack: from the trap just pop-ed to the next trap
      \end{itemize}
      \vfill
      \mintocaml[firstline=15,lastline=21,highlightlines={18-21}]{../src/backtrace/backtrace.ml}
      \endminipage
    \end{column}
    \begin{column}{0.5\textwidth}
      \raggedleft
      \onslide*<1>{\listCamlReraiseExn{}}
      \onslide*<2>{\listCamlReraiseExn[highlightlines=729]{}}
      \onslide*<3>{
        \mintc[firstline=190,lastline=192,highlightlines={191-192}]{../ocaml/runtime/amd64.S}
        \mintc[firstline=234,lastline=237,highlightlines={235-237}]{../ocaml/runtime/amd64.S}
        \medskip
        \listCamlReraiseExn[highlightlines=731]{}
      }
      \onslide*<4-5>{
        % TODO refactor with \listCamlReraiseExn
        \mintasm[firstline=727,lastline=734,highlightlines=730]{../ocaml/runtime/amd64.S}
        \medskip
      }
      \onslide*<4>{\listCamlRaiseExn[highlightlines=711]{}}
      \onslide*<5>{\listCamlRaiseExn[highlightlines=720]{}}
    \end{column}
  \end{columns}
\end{frame}

\begin{frame}{Backtraces}{Backtrace collection continues}
  \begin{columns}[c]
    \begin{column}{0.55\textwidth}
      \minipage[c][0.75\textheight][s]{\columnwidth}
      \begin{itemize}
        \item<1-> \funcname{caml\_stash\_backtrace} scans the older part of the stack, up to the next trap
        \item<6-> Controls jumps to the nested exception handler in \funcname{maybe\_raise}, which matches only on \typename{ExnB}
        \item<7-> \funcname{caml\_reraise\_exn} is called again, backtrace collection resumes with \funcname{caml\_stash\_backtrace}
      \end{itemize}
      \medskip
      \begin{itemize}
        \item<2-> Constructed backtrace:
        \begin{itemize}
          \item <2-> \funcname{definitely\_raise}
          \item <2-> \funcname{probably\_raise}
          \item <3-> \funcname{probably\_raise} (re-raise)
          \item <4-> \funcname{maybe\_raise}
          \item <5-> \funcname{main}
          \item <8-> \funcname{main} (re-raise)
        \end{itemize}
      \end{itemize}
      \vfill
      \onslide*<1-5>{\mintocaml[firstline=15,lastline=21]{../src/backtrace/backtrace.ml}}
      \onslide*<6>{\mintocaml[firstline=28,lastline=34,highlightlines=31]{../src/backtrace/backtrace.ml}}
      \onslide*<7->{\mintocaml[firstline=28,lastline=34,highlightlines=33]{../src/backtrace/backtrace.ml}}
      \endminipage
    \end{column}
    \begin{column}{0.45\textwidth}
      \raggedleft
      \onslide*<1>{\providecommand\step{6}\input{diagrams/backtrace/backtrace.tex}}%
      \onslide*<2>{\providecommand\step{6}\input{diagrams/backtrace/backtrace.tex}}%
      \onslide*<3>{\providecommand\step{7}\input{diagrams/backtrace/backtrace.tex}}%
      \onslide*<4>{\providecommand\step{8}\input{diagrams/backtrace/backtrace.tex}}%
      \onslide*<5>{\providecommand\step{9}\input{diagrams/backtrace/backtrace.tex}}%
      \onslide*<6>{\providecommand\step{10}\input{diagrams/backtrace/backtrace.tex}}%
      \onslide*<7>{\providecommand\step{11}\input{diagrams/backtrace/backtrace.tex}}%
      \onslide*<8>{\providecommand\step{12}\input{diagrams/backtrace/backtrace.tex}}%
      \onslide*<9>{\providecommand\step{13}\input{diagrams/backtrace/backtrace.tex}}%
    \end{column}
  \end{columns}
\end{frame}

\begin{frame}{Backtraces}{Printing the backtrace}
  \begin{columns}[c]
    \begin{column}{0.45\textwidth}
      \minipage[c][0.66\textheight][s]{\columnwidth}
      \begin{itemize}
        \item<1-> The exception backtrace can now be printed to \funcarg{stdout} with \funcname{Printexc.print_backtrace}
        \item<2-> First the exception backtrace is retrieved into an OCaml array with \funcname{get_raw_backtrace }
        \item<3-> Which is implemented as the \funcname{caml_get_exception_raw_backtrace} C function in the runtime
        \item<4-> And finally printed with \funcname{print_exception_backtrace}
      \end{itemize}
      \vfill
      \mintocaml[firstline=28,lastline=34,highlightlines=33]{../src/backtrace/backtrace.ml}
      \endminipage
    \end{column}
    \begin{column}{0.55\textwidth}
      \onslide*<1,2>{
        \mintocaml[firstline=100,lastline=101]{../ocaml/stdlib/printexc.ml}
      }
      \only<1>{
        \listocaml[firstline=170,lastline=172]{../ocaml/stdlib/printexc.ml}{stdlib/printexc.ml:170}
      }
      \only<2>{
        \listocaml[firstline=170,lastline=172,highlightlines=172]{../ocaml/stdlib/printexc.ml}{stdlib/printexc.ml:170}
      }
      \only<3>{
        \mintc[firstline=154,lastline=156]{../ocaml/runtime/backtrace.c}
        \listc[firstline=171,lastline=192,highlightlines={184-187}]{../ocaml/runtime/backtrace.c}{runtime/backtrace.c:154}
      }
      \only<4>{
        \listocaml[firstline=155,lastline=165]{../ocaml/stdlib/printexc.ml}{stdlib/printexc.ml:55}
      }
    \end{column}
  \end{columns}
\end{frame}


\subsection{The multiple kinds of raise}
\frameSubsection{}{}

\begin{frame}{Backtraces}{When backtraces are not needed}
  \begin{columns}[c]
    \begin{column}{0.45\textwidth}
      \minipage[c][0.66\textheight][s]{\columnwidth}
      \begin{itemize}
        \item<1-> What if the backtrace for an exception is never needed?
        \item<2-> It's possible to raise the exception explicitly with \funcname{raise\_notrace} to make raising as cheap as possible
        \item<3-> An example of use in the \funcname{dynlink} library of the OCaml compiler
      \end{itemize}
      \vfill
      \onslide<3->{
      \listocaml[firstline=205,lastline=209,highlightlines=207]{../ocaml/otherlibs/dynlink/dynlink\_compilerlibs/misc.ml}{otherlibs/dynlink/dynlink\_compilerlibs/misc.ml:205}
      }
      \endminipage
    \end{column}
    \begin{column}{0.55\textwidth}
    \only<4>{
      \mintcmm[highlightlines=3]{../src/notrace/camlNotrace\_\_fun_347.cmm}
      \listcmm{../src/notrace/camlNotrace\_\_all\_somes\_327.cmm}{all\_somes}
    }
    \onslide*<5->{
      \listobjdump[highlightlines={6-9}]{../src/notrace/camlNotrace\_\_fun_347.objdump}{anonymous function from \funcname{all\_somes}}
    }
    \end{column}
  \end{columns}
\end{frame}

\begin{frame}{Backtraces}{Summary table of the multiple kinds of raise}
  \begin{itemize}
    \item When debugging information is enabled and if the backtrace of an exception isn't needed, it's possible to raise with \funcname{raise\_notrace} for performance
    \item When debugging information is \emph{not} enabled, the compiler optimises \funcname{raise} calls to inline assembly (same effect as using \funcname{raise\_notrace})
  \end{itemize}
  \bigskip
  \centering
  \begin{tabular}{| l | c | c | c | c |}
    \hline
    \parbox{10em}{Debug information \\ (\funcarg{-g} flag)}  & \multicolumn{2}{|c|}{Disabled}                                     & \multicolumn{2}{|c|}{Enabled} \\
    \hline
    Raise kind                                               & \funcname{raise} & \funcname{raise\_notrace}                       & \funcname{raise} & \funcname{raise\_notrace} \\
    \hline
    Compilation                                              & \parbox{8em}{inline assembly\\ (optimization)} & inline assembly   & \funcname{caml\_raise\_exn} & inline assembly \\
    \hline
  \end{tabular}
\end{frame}


%
% Takeaway #5
%
\subsection*{Takeaway \#5}
\frameSubsectionTakeaway{}
\againframe<5>{takeaway}
}%
      \onslide*<2>{\providecommand\step{1}\input{diagrams/backtrace/scanstack.tex}}%
      \onslide*<3>{\providecommand\step{2}\input{diagrams/backtrace/scanstack.tex}}%
      \onslide*<4>{\providecommand\step{3}\input{diagrams/backtrace/scanstack.tex}}%
      \onslide*<5>{\providecommand\step{4}\input{diagrams/backtrace/scanstack.tex}}%
      \onslide*<6>{\providecommand\step{5}\input{diagrams/backtrace/scanstack.tex}}%
    \end{column}
  \end{columns}
\end{frame}

% Duplicate of previous-previous slide
\begin{frame}{Backtraces}{Continuing on \funcname{caml\_stash\_backtrace}}
  \begin{columns}[c]
    \begin{column}{0.5\textwidth}
      \minipage[c][0.75\textheight][s]{\columnwidth}
      \begin{itemize}
          \color{gray}
        \item A C function provided by the runtime (specific to native code)
        \item It scans the stack, between the stack pointer at the raising point up to the next trap, in order to collect the names of the detected function frames
        \item It uses the same technique and information as the Garbage Collector (GC) uses to scan the values live on the stack
      \end{itemize}
      \begin{itemize}
        \item For each function frame detected, store a pointer to the \typename{frame\_descr} in the \localname{domain\_state->backtrace\_buffer} array, effectively constructing the backtrace
      \end{itemize}
      \onslide<2->{
        \begin{itemize}
            \smallskip
          \item Constructed backtrace:
            \funcname{definitely\_raise},
            \onslide<3->{\funcname{probably\_raise}}
        \end{itemize}
      }
      \vfill
      \mintocaml[firstline=9,lastline=13,highlightlines=12]{../src/backtrace/backtrace.ml}
      \endminipage
    \end{column}
    \begin{column}{0.5\textwidth}
      \raggedleft
      \onslide*<1>{\listStashBacktrace[highlightlines={114-115}]{}}
      \onslide*<2>{\providecommand\step{3}%
% Backtraces
%
\subsection{One advantage of raising with debugging information}
\frameSubsection{}{}

\begin{frame}{Backtraces}{Debugging information \& source code}
  \begin{columns}[c]
    \begin{column}{0.5\textwidth}
      \begin{itemize}
        \item Raising an exception is fun and games, but how do we know where the exception originated? $\rightarrow$ With the backtrace associated with the exception!
        \item In order to get a backtrace from the point where an exception was raised, it's necessary to compile with debugging information enabled (\funcarg{-g} flag for the compiler)
        \item Same example as in the `Nested handlers' section
      \end{itemize}
    \end{column}
    \begin{column}{0.5\textwidth}
      \listocaml[firstline=9,lastline=34]{../src/backtrace/backtrace.ml}{backtrace/backtrace.ml}
    \end{column}
  \end{columns}
\end{frame}

\begin{frame}{Backtraces}{CMM and disassembly of \funcname{definitely\_raise}}
  \begin{columns}[c]
    \begin{column}{0.5\textwidth}
      \minipage[c][0.66\textheight][s]{\columnwidth}
      \begin{itemize}
        \item<1-> An effect of compiling with debugging information (\funcarg{-g}): the CMM no longer uses \funcname{raise\_notrace} to raise the exception but \funcname{raise} instead
        \item<2-> Disassembly shows that CMM \funcname{raise} is actually a call to \funcname{caml\_raise\_exn}, a function in assembler provided by the runtime
      \end{itemize}
      \vfill
      \mintocaml[firstline=9,lastline=13,highlightlines=12]{../src/backtrace/backtrace.ml}
      \endminipage
    \end{column}
    \begin{column}{0.5\textwidth}
      \onslide*<1>{
        \mintcmm[highlightlines=5]{../src/backtrace/camlBacktrace\_\_definitely\_raise\_347.cmm}
      }
      \onslide*<2>{
        \mintobjdump[firstline=2,highlightlines=14]{../src/backtrace/camlBacktrace\_\_definitely\_raise\_347.objdump}
      }
    \end{column}
  \end{columns}
\end{frame}

\begin{frame}{Backtraces}{Introducing \funcname{caml\_raise\_exn}}
  \begin{columns}[c]
    \begin{column}{0.5\textwidth}
      \minipage[c][0.66\textheight][s]{\columnwidth}
      \onslide*<1>{
        \begin{itemize}
          \item Raising the exception is performed using \funcname{caml\_raise\_exn} which is
            \begin{itemize}
              \item An assembly function provided by the runtime
              \item Architecture specific
            \end{itemize}
          \item Let's see what it does differently from \funcname{raise\_notrace}\dots
          \item We are about to raise \typename{ExnC} with \funcname{caml\_raise\_exn} from \funcname{definitely\_raise}
        \end{itemize}
      }
      \onslide*<2>{
        \mintobjdump[firstline=2,highlightlines=14]{../src/backtrace/camlBacktrace\_\_definitely\_raise\_347.objdump}
      }
      \vfill
      \mintocaml[firstline=9,lastline=13,highlightlines=12]{../src/backtrace/backtrace.ml}
      \endminipage
    \end{column}
    \begin{column}{0.5\textwidth}
      \centering
      \providecommand\step{1}\input{diagrams/backtrace/backtrace.tex}
    \end{column}
  \end{columns}
\end{frame}

\begin{frame}{Backtraces}{But before that, we need more fields from \typename{caml\_domain\_state}}
  \begin{columns}
    \begin{column}{0.5\textwidth}
      \begin{itemize}
        \item Remember \typename{caml\_domain\_state} the per-domain runtime state?
        \item Some other members are of interest to us now\dots
        \item \localname{backtrace\_active} is used to know if the runtime should collect backtraces
        \item \localname{backtrace\_active} can be set by
          \begin{itemize}
            \item The \funcarg{b} flag in the \funcname{OCAMLRUNPARAM} environment variable
            \item \mintinline{ocaml}{Printexc.record_backtrace : bool -> unit} \footnotemark
          \end{itemize}
      \end{itemize}
    \end{column}
    \begin{column}{0.5\textwidth}
      \begin{listing}
        \mintc[highlightlines={9-11}]{src/ocaml/domain\_state\_backtrace.h}
        \caption{runtime/caml/domain\_state.h}
      \end{listing}
    \end{column}
  \end{columns}
  \footnotetext[1]{https://v2.ocaml.org/api/Printexc.html}
\end{frame}

\newcommand\listCamlRaiseExn[1][]{
  \listasm[firstline=702,lastline=725,#1]{../ocaml/runtime/amd64.S}{runtime/amd64.S:702}}

\begin{frame}{Backtraces}{Walk through \funcname{caml\_raise\_exn}}
  \begin{columns}[T]
    \begin{column}{0.5\textwidth}
      \begin{itemize}
        \item<1-> First checks whether exceptions backtrace should be recorded
        \item<1-> By checking the \funcarg{domain\_state->backtrace\_active} value
        \item<2-> If not, acts as \funcname{raise\_notrace} with \funcname{RESTORE\_EXN\_HANDLER\_OCAML}
          \begin{itemize}
            \item Set \regname{RSP} to \localname{domain\_state->exn\_handler} value
            \item Restore \localname{domain\_state->exn\_handler} from trap
            \item Jump with \funcname{ret} (same effect as using \funcname{pop} and \funcname{jmp})
          \end{itemize}
        \item<3-> Otherwise, switches to the C stack to be able to execute C code
        \item<4-> Calls \funcname{caml\_stash\_backtrace}, a C function provided by the runtime\ignorespaces
          \onslide<6->{, with
            \begin{itemize}
              \item \funcarg{exn}: exception value
              \item \funcarg{pc}: code address of the raising point
              \item \funcarg{sp}: stack address of the raising function
              \item \funcarg{trapsp}: stack address of the next handler
            \end{itemize}
          }
      \end{itemize}
      \bigskip
      \mintocaml[firstline=9,lastline=13,highlightlines=12]{../src/backtrace/backtrace.ml}
    \end{column}
    \begin{column}{0.5\textwidth}
      \raggedleft
      \onslide*<1,3-4>{
        \mintc[firstline=190,lastline=192]{../ocaml/runtime/amd64.S}
        \mintc[firstline=234,lastline=237]{../ocaml/runtime/amd64.S}
      }
      \onslide*<2>{
        \mintc[firstline=190,lastline=192,highlightlines={191-192}]{../ocaml/runtime/amd64.S}
        \mintc[firstline=234,lastline=237,highlightlines={235-237}]{../ocaml/runtime/amd64.S}
      }
      \onslide*<1>{\listCamlRaiseExn[highlightlines=705]{}}%
      \onslide*<2>{\listCamlRaiseExn[highlightlines={707-708}]{}}%
      \onslide*<3>{\listCamlRaiseExn[highlightlines={712-714}]{}}%
      \onslide*<4>{\listCamlRaiseExn[highlightlines={715-720}]{}}%
      \onslide*<5>{\providecommand\step{1}\input{diagrams/backtrace/backtrace.tex}}%
      \onslide*<6>{\providecommand\step{2}\input{diagrams/backtrace/backtrace.tex}}%
    \end{column}
  \end{columns}
\end{frame}

\newcommand\listStashBacktrace[1][]{
  \listc[firstline=91,lastline=120,#1]{../ocaml/runtime/backtrace\_nat.c}{runtime/backtrace\_nat.c:91}}

\begin{frame}{Backtraces}{Introducing \funcname{caml\_stash\_backtrace}}
  \begin{columns}[c]
    \begin{column}{0.5\textwidth}
      \minipage[c][0.66\textheight][s]{\columnwidth}
      \begin{itemize}
        \item<1-> A C function provided by the runtime (specific to native code)
        \item<2-> It scans the stack, between the stack pointer at the raising point up to the next trap, in order to collect the names of the detected function frames
          \onslide*<2>{
            \begin{itemize}
              \item And more information, like line number, column number...
            \end{itemize}
          }
        \item<3-> It uses the same technique and information as the Garbage Collector (GC) uses to scan the values live on the stack
          \onslide*<4>{
            \begin{itemize}
              \item \typename{frame\_descr} are at the core of stack unwinding
            \end{itemize}
          }
      \end{itemize}
      \vfill
      \mintocaml[firstline=9,lastline=13,highlightlines=12]{../src/backtrace/backtrace.ml}
      \endminipage
    \end{column}
    \begin{column}{0.5\textwidth}
      \onslide*<1>{\listStashBacktrace[highlightlines=91]{}}
      \onslide*<2>{\listStashBacktrace[highlightlines={91,117-118}]{}}
      \onslide*<3->{\listStashBacktrace[highlightlines={94,106,109-110}]{}}
    \end{column}
  \end{columns}
\end{frame}

\newcommand\listFrameDescr[1][]{
  \listocaml[firstline=111,lastline=124,#1]{../ocaml/asmcomp/emitaux.ml}{asmcomp/emitaux.ml:111}}
\newcommand\listDebugInfo[1][]{
  \listocaml[firstline=38,lastline=49,#1]{../ocaml/lambda/debuginfo.mli}{lambda/debuginfo.mli:38}}

\begin{frame}{Backtraces}{\typename{frame\_descr} and \typename{frame\_debuginfo} in a nutshell}
  \begin{columns}[c]
    \begin{column}{0.5\textwidth}
      \begin{itemize}
        \item<1-> For every return address in an OCaml program, there is a associated \typename{frame\_descr}
        \item<2-> A \typename{frame\_descr} specifies
        \begin{itemize}
          \item<2-> The size of the function stack frame
          \item<3-> Where live values are on the stack (so that the GC can mark them as live and prevent from being swept)
        \end{itemize}
        \item<4-> When compiled with debug information, \typename{frame\_descr} will be linked to a \typename{frame\_debuginfo}
        \begin{itemize}
          \item<5-> Contains information about the position in the source code (file name, line and column number)
        \end{itemize}
      \end{itemize}
    \end{column}
    \begin{column}{0.5\textwidth}
        \onslide*<1>{\listFrameDescr[highlightlines={118-122}]{}}
        \onslide*<2>{\listFrameDescr[highlightlines={120}]{}}
        \onslide*<3>{\listFrameDescr[highlightlines={121}]{}}
        \onslide*<4->{\listFrameDescr[highlightlines={122}]{}}
        \medskip
        \onslide*<1-3>{\listDebugInfo{}}
        \onslide*<4>{\listDebugInfo[highlightlines={38,49}]{}}
        \onslide*<5>{\listDebugInfo[highlightlines={39-46}]{}}
    \end{column}
  \end{columns}
\end{frame}

\begin{frame}{Backtraces}{Scanning the stack with \typename{frame\_descr}}
  \begin{columns}[c]
    \begin{column}{0.5\textwidth}
      \begin{itemize}
        \item Given the return address of the code that raises the exception by calling \funcname{caml\_raise\_exn} (\funcarg{pc} argument of \funcname{caml\_stash\_backtrace})
        \item And the position of the stack pointer (\funcarg{sp} argument of \funcname{caml\_stash\_backtrace})
        \item The frame size given by \typename{frame\_descr}.\funcarg{frame\_size}, allows to find the end of the previous frame
        \item And the return address of the caller of this function
        \bigskip
        \onslide<3->{
        \item Note that traps (here the reddish \funcname{ExnA}, \funcname{ExnB} and \funcname{ExnC} blocks) belong to the frame that installed them, and so the frame size changes when traps are removed
        }
      \end{itemize}
    \end{column}
    \begin{column}{0.5\textwidth}
      \raggedleft
      \onslide*<1>{\providecommand\step{2}\input{diagrams/backtrace/backtrace.tex}}%
      \onslide*<2>{\providecommand\step{1}\input{diagrams/backtrace/scanstack.tex}}%
      \onslide*<3>{\providecommand\step{2}\input{diagrams/backtrace/scanstack.tex}}%
      \onslide*<4>{\providecommand\step{3}\input{diagrams/backtrace/scanstack.tex}}%
      \onslide*<5>{\providecommand\step{4}\input{diagrams/backtrace/scanstack.tex}}%
      \onslide*<6>{\providecommand\step{5}\input{diagrams/backtrace/scanstack.tex}}%
    \end{column}
  \end{columns}
\end{frame}

% Duplicate of previous-previous slide
\begin{frame}{Backtraces}{Continuing on \funcname{caml\_stash\_backtrace}}
  \begin{columns}[c]
    \begin{column}{0.5\textwidth}
      \minipage[c][0.75\textheight][s]{\columnwidth}
      \begin{itemize}
          \color{gray}
        \item A C function provided by the runtime (specific to native code)
        \item It scans the stack, between the stack pointer at the raising point up to the next trap, in order to collect the names of the detected function frames
        \item It uses the same technique and information as the Garbage Collector (GC) uses to scan the values live on the stack
      \end{itemize}
      \begin{itemize}
        \item For each function frame detected, store a pointer to the \typename{frame\_descr} in the \localname{domain\_state->backtrace\_buffer} array, effectively constructing the backtrace
      \end{itemize}
      \onslide<2->{
        \begin{itemize}
            \smallskip
          \item Constructed backtrace:
            \funcname{definitely\_raise},
            \onslide<3->{\funcname{probably\_raise}}
        \end{itemize}
      }
      \vfill
      \mintocaml[firstline=9,lastline=13,highlightlines=12]{../src/backtrace/backtrace.ml}
      \endminipage
    \end{column}
    \begin{column}{0.5\textwidth}
      \raggedleft
      \onslide*<1>{\listStashBacktrace[highlightlines={114-115}]{}}
      \onslide*<2>{\providecommand\step{3}\input{diagrams/backtrace/backtrace.tex}}%
      \onslide*<3>{\providecommand\step{4}\input{diagrams/backtrace/backtrace.tex}}%
    \end{column}
  \end{columns}
\end{frame}

\begin{frame}{Backtraces}{Back to \funcname{caml\_raise\_exn}}
  \begin{columns}[c]
    \begin{column}{0.5\textwidth}
      \minipage[c][0.66\textheight][s]{\columnwidth}
      \begin{itemize}\color{gray}
        \item Checks wheteher exceptions backtrace should be recorded, by checking the \funcarg{domain\_state->backtrace\_active} value
        \item Switches to the C stack to be able to execute C code
        \item Calls \funcname{caml\_stash\_backtrace}, to collect the backtrace up to the next trap
      \end{itemize}
      \onslide*<2->{
        \begin{itemize}
          \item<2-> Executes the exception handler in \funcname{probably\_raise} with \funcname{RESTORE\_EXN\_HANDLER\_OCAML}
            \begin{itemize}
              \item Set \regname{RSP} to \localname{domain\_state->exn\_handler} value
              \item Restore \localname{domain\_state->exn\_handler} from trap
              \item Jump to the handler code with \funcname{ret}
            \end{itemize}
        \end{itemize}
      }
      \vfill
      \onslide*<1>{\mintocaml[firstline=9,lastline=13,highlightlines=12]{../src/backtrace/backtrace.ml}}
      \onslide*<2->{\mintocaml[firstline=15,lastline=21,highlightlines={19-21}]{../src/backtrace/backtrace.ml}}
      \endminipage
    \end{column}
    \begin{column}{0.5\textwidth}
      \centering
      \onslide*<1>{
        \mintc[firstline=190,lastline=192]{../ocaml/runtime/amd64.S}
        \mintc[firstline=234,lastline=237]{../ocaml/runtime/amd64.S}
      }
      \onslide*<2>{
        \mintc[firstline=190,lastline=192,highlightlines={191-192}]{../ocaml/runtime/amd64.S}
        \mintc[firstline=234,lastline=237,highlightlines={235-237}]{../ocaml/runtime/amd64.S}
      }
      \onslide*<1>{\listCamlRaiseExn[highlightlines=720]{}}
      \onslide*<2>{\listCamlRaiseExn[highlightlines=722]{}}
      \onslide*<3->{\providecommand\step{5}\input{diagrams/backtrace/backtrace.tex}}
    \end{column}
  \end{columns}
\end{frame}

\begin{frame}{Backtraces}{\funcname{caml\_probably\_raise}'s handler doesn't catch \typename{ExnC}}
  \begin{columns}[c]
    \begin{column}{0.45\textwidth}
      \minipage[c][0.75\textheight][s]{\columnwidth}
      \begin{itemize}
        \item<1-> \funcname{match \dots with}\footnotemark pattern matching can also match exceptions
        \item<1-> The CMM of \funcname{probably\_raise} shows that this \funcname{match \dots with} is implemented with a pattern matching and a \funcname{try \dots with}
        \item<1-> But this one only matches on \typename{ExnA} exception
        \item<2-> Since the pattern matching fails to catch the exception, \funcname{reraise} is called with our current \typename{ExnC} exception
        \item<3-> Disassembly shows that \funcname{reraise} is actually a call to \funcname{caml\_reraise\_exn}, which is also an assembly function provided by the runtime
      \end{itemize}
      \vfill
      \mintocaml[firstline=15,lastline=21,highlightlines={18-21}]{../src/backtrace/backtrace.ml}
      \endminipage
    \end{column}
    \begin{column}{0.55\textwidth}
      \centering
      \onslide*<1>{\mintcmm[highlightlines={10-18}]{../src/backtrace/camlBacktrace\_\_probably\_raise\_351.cmm}}
      \onslide*<2>{\listcmm[highlightlines=18]{../src/backtrace/camlBacktrace\_\_probably\_raise_351.cmm}{CMM of probably\_raise}}
      \onslide*<3>{\mintobjdump[firstline=5,lastline=35,highlightlines=31]{../src/backtrace/camlBacktrace\_\_probably\_raise_351.objdump}}
    \end{column}
  \end{columns}
  \only<1>{\footnotetext[1]{https://blog.janestreet.com/pattern-matching-and-exception-handling-unite/}}
\end{frame}

\newcommand\listCamlReraiseExn[1][]{
  \listasm[firstline=727,lastline=734,#1]{../ocaml/runtime/amd64.S}{runtime/amd64.S:727}}

\begin{frame}{Backtraces}{Introducing \funcname{caml\_reraise\_exn}}
  \begin{columns}[c]
    \begin{column}{0.5\textwidth}
      \minipage[c][0.66\textheight][s]{\columnwidth}
      \begin{itemize}
        \item<1-> The exception have to be reraised to the next handler
        \item<2-> First, checks if the backtrace should be collected with \funcarg{domain\_state->backtrace\_active}
        \only<3>{
        \begin{itemize}
            \item If not, behaves like a \funcname{raise\_notrace} with \funcname{RESTORE\_EXN\_HANDLER\_OCAML}
        \end{itemize}
        }
        \item<4-> Jumps into \funcname{caml\_raise\_exn}
        \item<5-> Calls \funcname{caml\_stash\_backtrace} again, to scan the next part of the stack: from the trap just pop-ed to the next trap
      \end{itemize}
      \vfill
      \mintocaml[firstline=15,lastline=21,highlightlines={18-21}]{../src/backtrace/backtrace.ml}
      \endminipage
    \end{column}
    \begin{column}{0.5\textwidth}
      \raggedleft
      \onslide*<1>{\listCamlReraiseExn{}}
      \onslide*<2>{\listCamlReraiseExn[highlightlines=729]{}}
      \onslide*<3>{
        \mintc[firstline=190,lastline=192,highlightlines={191-192}]{../ocaml/runtime/amd64.S}
        \mintc[firstline=234,lastline=237,highlightlines={235-237}]{../ocaml/runtime/amd64.S}
        \medskip
        \listCamlReraiseExn[highlightlines=731]{}
      }
      \onslide*<4-5>{
        % TODO refactor with \listCamlReraiseExn
        \mintasm[firstline=727,lastline=734,highlightlines=730]{../ocaml/runtime/amd64.S}
        \medskip
      }
      \onslide*<4>{\listCamlRaiseExn[highlightlines=711]{}}
      \onslide*<5>{\listCamlRaiseExn[highlightlines=720]{}}
    \end{column}
  \end{columns}
\end{frame}

\begin{frame}{Backtraces}{Backtrace collection continues}
  \begin{columns}[c]
    \begin{column}{0.55\textwidth}
      \minipage[c][0.75\textheight][s]{\columnwidth}
      \begin{itemize}
        \item<1-> \funcname{caml\_stash\_backtrace} scans the older part of the stack, up to the next trap
        \item<6-> Controls jumps to the nested exception handler in \funcname{maybe\_raise}, which matches only on \typename{ExnB}
        \item<7-> \funcname{caml\_reraise\_exn} is called again, backtrace collection resumes with \funcname{caml\_stash\_backtrace}
      \end{itemize}
      \medskip
      \begin{itemize}
        \item<2-> Constructed backtrace:
        \begin{itemize}
          \item <2-> \funcname{definitely\_raise}
          \item <2-> \funcname{probably\_raise}
          \item <3-> \funcname{probably\_raise} (re-raise)
          \item <4-> \funcname{maybe\_raise}
          \item <5-> \funcname{main}
          \item <8-> \funcname{main} (re-raise)
        \end{itemize}
      \end{itemize}
      \vfill
      \onslide*<1-5>{\mintocaml[firstline=15,lastline=21]{../src/backtrace/backtrace.ml}}
      \onslide*<6>{\mintocaml[firstline=28,lastline=34,highlightlines=31]{../src/backtrace/backtrace.ml}}
      \onslide*<7->{\mintocaml[firstline=28,lastline=34,highlightlines=33]{../src/backtrace/backtrace.ml}}
      \endminipage
    \end{column}
    \begin{column}{0.45\textwidth}
      \raggedleft
      \onslide*<1>{\providecommand\step{6}\input{diagrams/backtrace/backtrace.tex}}%
      \onslide*<2>{\providecommand\step{6}\input{diagrams/backtrace/backtrace.tex}}%
      \onslide*<3>{\providecommand\step{7}\input{diagrams/backtrace/backtrace.tex}}%
      \onslide*<4>{\providecommand\step{8}\input{diagrams/backtrace/backtrace.tex}}%
      \onslide*<5>{\providecommand\step{9}\input{diagrams/backtrace/backtrace.tex}}%
      \onslide*<6>{\providecommand\step{10}\input{diagrams/backtrace/backtrace.tex}}%
      \onslide*<7>{\providecommand\step{11}\input{diagrams/backtrace/backtrace.tex}}%
      \onslide*<8>{\providecommand\step{12}\input{diagrams/backtrace/backtrace.tex}}%
      \onslide*<9>{\providecommand\step{13}\input{diagrams/backtrace/backtrace.tex}}%
    \end{column}
  \end{columns}
\end{frame}

\begin{frame}{Backtraces}{Printing the backtrace}
  \begin{columns}[c]
    \begin{column}{0.45\textwidth}
      \minipage[c][0.66\textheight][s]{\columnwidth}
      \begin{itemize}
        \item<1-> The exception backtrace can now be printed to \funcarg{stdout} with \funcname{Printexc.print_backtrace}
        \item<2-> First the exception backtrace is retrieved into an OCaml array with \funcname{get_raw_backtrace }
        \item<3-> Which is implemented as the \funcname{caml_get_exception_raw_backtrace} C function in the runtime
        \item<4-> And finally printed with \funcname{print_exception_backtrace}
      \end{itemize}
      \vfill
      \mintocaml[firstline=28,lastline=34,highlightlines=33]{../src/backtrace/backtrace.ml}
      \endminipage
    \end{column}
    \begin{column}{0.55\textwidth}
      \onslide*<1,2>{
        \mintocaml[firstline=100,lastline=101]{../ocaml/stdlib/printexc.ml}
      }
      \only<1>{
        \listocaml[firstline=170,lastline=172]{../ocaml/stdlib/printexc.ml}{stdlib/printexc.ml:170}
      }
      \only<2>{
        \listocaml[firstline=170,lastline=172,highlightlines=172]{../ocaml/stdlib/printexc.ml}{stdlib/printexc.ml:170}
      }
      \only<3>{
        \mintc[firstline=154,lastline=156]{../ocaml/runtime/backtrace.c}
        \listc[firstline=171,lastline=192,highlightlines={184-187}]{../ocaml/runtime/backtrace.c}{runtime/backtrace.c:154}
      }
      \only<4>{
        \listocaml[firstline=155,lastline=165]{../ocaml/stdlib/printexc.ml}{stdlib/printexc.ml:55}
      }
    \end{column}
  \end{columns}
\end{frame}


\subsection{The multiple kinds of raise}
\frameSubsection{}{}

\begin{frame}{Backtraces}{When backtraces are not needed}
  \begin{columns}[c]
    \begin{column}{0.45\textwidth}
      \minipage[c][0.66\textheight][s]{\columnwidth}
      \begin{itemize}
        \item<1-> What if the backtrace for an exception is never needed?
        \item<2-> It's possible to raise the exception explicitly with \funcname{raise\_notrace} to make raising as cheap as possible
        \item<3-> An example of use in the \funcname{dynlink} library of the OCaml compiler
      \end{itemize}
      \vfill
      \onslide<3->{
      \listocaml[firstline=205,lastline=209,highlightlines=207]{../ocaml/otherlibs/dynlink/dynlink\_compilerlibs/misc.ml}{otherlibs/dynlink/dynlink\_compilerlibs/misc.ml:205}
      }
      \endminipage
    \end{column}
    \begin{column}{0.55\textwidth}
    \only<4>{
      \mintcmm[highlightlines=3]{../src/notrace/camlNotrace\_\_fun_347.cmm}
      \listcmm{../src/notrace/camlNotrace\_\_all\_somes\_327.cmm}{all\_somes}
    }
    \onslide*<5->{
      \listobjdump[highlightlines={6-9}]{../src/notrace/camlNotrace\_\_fun_347.objdump}{anonymous function from \funcname{all\_somes}}
    }
    \end{column}
  \end{columns}
\end{frame}

\begin{frame}{Backtraces}{Summary table of the multiple kinds of raise}
  \begin{itemize}
    \item When debugging information is enabled and if the backtrace of an exception isn't needed, it's possible to raise with \funcname{raise\_notrace} for performance
    \item When debugging information is \emph{not} enabled, the compiler optimises \funcname{raise} calls to inline assembly (same effect as using \funcname{raise\_notrace})
  \end{itemize}
  \bigskip
  \centering
  \begin{tabular}{| l | c | c | c | c |}
    \hline
    \parbox{10em}{Debug information \\ (\funcarg{-g} flag)}  & \multicolumn{2}{|c|}{Disabled}                                     & \multicolumn{2}{|c|}{Enabled} \\
    \hline
    Raise kind                                               & \funcname{raise} & \funcname{raise\_notrace}                       & \funcname{raise} & \funcname{raise\_notrace} \\
    \hline
    Compilation                                              & \parbox{8em}{inline assembly\\ (optimization)} & inline assembly   & \funcname{caml\_raise\_exn} & inline assembly \\
    \hline
  \end{tabular}
\end{frame}


%
% Takeaway #5
%
\subsection*{Takeaway \#5}
\frameSubsectionTakeaway{}
\againframe<5>{takeaway}
}%
      \onslide*<3>{\providecommand\step{4}%
% Backtraces
%
\subsection{One advantage of raising with debugging information}
\frameSubsection{}{}

\begin{frame}{Backtraces}{Debugging information \& source code}
  \begin{columns}[c]
    \begin{column}{0.5\textwidth}
      \begin{itemize}
        \item Raising an exception is fun and games, but how do we know where the exception originated? $\rightarrow$ With the backtrace associated with the exception!
        \item In order to get a backtrace from the point where an exception was raised, it's necessary to compile with debugging information enabled (\funcarg{-g} flag for the compiler)
        \item Same example as in the `Nested handlers' section
      \end{itemize}
    \end{column}
    \begin{column}{0.5\textwidth}
      \listocaml[firstline=9,lastline=34]{../src/backtrace/backtrace.ml}{backtrace/backtrace.ml}
    \end{column}
  \end{columns}
\end{frame}

\begin{frame}{Backtraces}{CMM and disassembly of \funcname{definitely\_raise}}
  \begin{columns}[c]
    \begin{column}{0.5\textwidth}
      \minipage[c][0.66\textheight][s]{\columnwidth}
      \begin{itemize}
        \item<1-> An effect of compiling with debugging information (\funcarg{-g}): the CMM no longer uses \funcname{raise\_notrace} to raise the exception but \funcname{raise} instead
        \item<2-> Disassembly shows that CMM \funcname{raise} is actually a call to \funcname{caml\_raise\_exn}, a function in assembler provided by the runtime
      \end{itemize}
      \vfill
      \mintocaml[firstline=9,lastline=13,highlightlines=12]{../src/backtrace/backtrace.ml}
      \endminipage
    \end{column}
    \begin{column}{0.5\textwidth}
      \onslide*<1>{
        \mintcmm[highlightlines=5]{../src/backtrace/camlBacktrace\_\_definitely\_raise\_347.cmm}
      }
      \onslide*<2>{
        \mintobjdump[firstline=2,highlightlines=14]{../src/backtrace/camlBacktrace\_\_definitely\_raise\_347.objdump}
      }
    \end{column}
  \end{columns}
\end{frame}

\begin{frame}{Backtraces}{Introducing \funcname{caml\_raise\_exn}}
  \begin{columns}[c]
    \begin{column}{0.5\textwidth}
      \minipage[c][0.66\textheight][s]{\columnwidth}
      \onslide*<1>{
        \begin{itemize}
          \item Raising the exception is performed using \funcname{caml\_raise\_exn} which is
            \begin{itemize}
              \item An assembly function provided by the runtime
              \item Architecture specific
            \end{itemize}
          \item Let's see what it does differently from \funcname{raise\_notrace}\dots
          \item We are about to raise \typename{ExnC} with \funcname{caml\_raise\_exn} from \funcname{definitely\_raise}
        \end{itemize}
      }
      \onslide*<2>{
        \mintobjdump[firstline=2,highlightlines=14]{../src/backtrace/camlBacktrace\_\_definitely\_raise\_347.objdump}
      }
      \vfill
      \mintocaml[firstline=9,lastline=13,highlightlines=12]{../src/backtrace/backtrace.ml}
      \endminipage
    \end{column}
    \begin{column}{0.5\textwidth}
      \centering
      \providecommand\step{1}\input{diagrams/backtrace/backtrace.tex}
    \end{column}
  \end{columns}
\end{frame}

\begin{frame}{Backtraces}{But before that, we need more fields from \typename{caml\_domain\_state}}
  \begin{columns}
    \begin{column}{0.5\textwidth}
      \begin{itemize}
        \item Remember \typename{caml\_domain\_state} the per-domain runtime state?
        \item Some other members are of interest to us now\dots
        \item \localname{backtrace\_active} is used to know if the runtime should collect backtraces
        \item \localname{backtrace\_active} can be set by
          \begin{itemize}
            \item The \funcarg{b} flag in the \funcname{OCAMLRUNPARAM} environment variable
            \item \mintinline{ocaml}{Printexc.record_backtrace : bool -> unit} \footnotemark
          \end{itemize}
      \end{itemize}
    \end{column}
    \begin{column}{0.5\textwidth}
      \begin{listing}
        \mintc[highlightlines={9-11}]{src/ocaml/domain\_state\_backtrace.h}
        \caption{runtime/caml/domain\_state.h}
      \end{listing}
    \end{column}
  \end{columns}
  \footnotetext[1]{https://v2.ocaml.org/api/Printexc.html}
\end{frame}

\newcommand\listCamlRaiseExn[1][]{
  \listasm[firstline=702,lastline=725,#1]{../ocaml/runtime/amd64.S}{runtime/amd64.S:702}}

\begin{frame}{Backtraces}{Walk through \funcname{caml\_raise\_exn}}
  \begin{columns}[T]
    \begin{column}{0.5\textwidth}
      \begin{itemize}
        \item<1-> First checks whether exceptions backtrace should be recorded
        \item<1-> By checking the \funcarg{domain\_state->backtrace\_active} value
        \item<2-> If not, acts as \funcname{raise\_notrace} with \funcname{RESTORE\_EXN\_HANDLER\_OCAML}
          \begin{itemize}
            \item Set \regname{RSP} to \localname{domain\_state->exn\_handler} value
            \item Restore \localname{domain\_state->exn\_handler} from trap
            \item Jump with \funcname{ret} (same effect as using \funcname{pop} and \funcname{jmp})
          \end{itemize}
        \item<3-> Otherwise, switches to the C stack to be able to execute C code
        \item<4-> Calls \funcname{caml\_stash\_backtrace}, a C function provided by the runtime\ignorespaces
          \onslide<6->{, with
            \begin{itemize}
              \item \funcarg{exn}: exception value
              \item \funcarg{pc}: code address of the raising point
              \item \funcarg{sp}: stack address of the raising function
              \item \funcarg{trapsp}: stack address of the next handler
            \end{itemize}
          }
      \end{itemize}
      \bigskip
      \mintocaml[firstline=9,lastline=13,highlightlines=12]{../src/backtrace/backtrace.ml}
    \end{column}
    \begin{column}{0.5\textwidth}
      \raggedleft
      \onslide*<1,3-4>{
        \mintc[firstline=190,lastline=192]{../ocaml/runtime/amd64.S}
        \mintc[firstline=234,lastline=237]{../ocaml/runtime/amd64.S}
      }
      \onslide*<2>{
        \mintc[firstline=190,lastline=192,highlightlines={191-192}]{../ocaml/runtime/amd64.S}
        \mintc[firstline=234,lastline=237,highlightlines={235-237}]{../ocaml/runtime/amd64.S}
      }
      \onslide*<1>{\listCamlRaiseExn[highlightlines=705]{}}%
      \onslide*<2>{\listCamlRaiseExn[highlightlines={707-708}]{}}%
      \onslide*<3>{\listCamlRaiseExn[highlightlines={712-714}]{}}%
      \onslide*<4>{\listCamlRaiseExn[highlightlines={715-720}]{}}%
      \onslide*<5>{\providecommand\step{1}\input{diagrams/backtrace/backtrace.tex}}%
      \onslide*<6>{\providecommand\step{2}\input{diagrams/backtrace/backtrace.tex}}%
    \end{column}
  \end{columns}
\end{frame}

\newcommand\listStashBacktrace[1][]{
  \listc[firstline=91,lastline=120,#1]{../ocaml/runtime/backtrace\_nat.c}{runtime/backtrace\_nat.c:91}}

\begin{frame}{Backtraces}{Introducing \funcname{caml\_stash\_backtrace}}
  \begin{columns}[c]
    \begin{column}{0.5\textwidth}
      \minipage[c][0.66\textheight][s]{\columnwidth}
      \begin{itemize}
        \item<1-> A C function provided by the runtime (specific to native code)
        \item<2-> It scans the stack, between the stack pointer at the raising point up to the next trap, in order to collect the names of the detected function frames
          \onslide*<2>{
            \begin{itemize}
              \item And more information, like line number, column number...
            \end{itemize}
          }
        \item<3-> It uses the same technique and information as the Garbage Collector (GC) uses to scan the values live on the stack
          \onslide*<4>{
            \begin{itemize}
              \item \typename{frame\_descr} are at the core of stack unwinding
            \end{itemize}
          }
      \end{itemize}
      \vfill
      \mintocaml[firstline=9,lastline=13,highlightlines=12]{../src/backtrace/backtrace.ml}
      \endminipage
    \end{column}
    \begin{column}{0.5\textwidth}
      \onslide*<1>{\listStashBacktrace[highlightlines=91]{}}
      \onslide*<2>{\listStashBacktrace[highlightlines={91,117-118}]{}}
      \onslide*<3->{\listStashBacktrace[highlightlines={94,106,109-110}]{}}
    \end{column}
  \end{columns}
\end{frame}

\newcommand\listFrameDescr[1][]{
  \listocaml[firstline=111,lastline=124,#1]{../ocaml/asmcomp/emitaux.ml}{asmcomp/emitaux.ml:111}}
\newcommand\listDebugInfo[1][]{
  \listocaml[firstline=38,lastline=49,#1]{../ocaml/lambda/debuginfo.mli}{lambda/debuginfo.mli:38}}

\begin{frame}{Backtraces}{\typename{frame\_descr} and \typename{frame\_debuginfo} in a nutshell}
  \begin{columns}[c]
    \begin{column}{0.5\textwidth}
      \begin{itemize}
        \item<1-> For every return address in an OCaml program, there is a associated \typename{frame\_descr}
        \item<2-> A \typename{frame\_descr} specifies
        \begin{itemize}
          \item<2-> The size of the function stack frame
          \item<3-> Where live values are on the stack (so that the GC can mark them as live and prevent from being swept)
        \end{itemize}
        \item<4-> When compiled with debug information, \typename{frame\_descr} will be linked to a \typename{frame\_debuginfo}
        \begin{itemize}
          \item<5-> Contains information about the position in the source code (file name, line and column number)
        \end{itemize}
      \end{itemize}
    \end{column}
    \begin{column}{0.5\textwidth}
        \onslide*<1>{\listFrameDescr[highlightlines={118-122}]{}}
        \onslide*<2>{\listFrameDescr[highlightlines={120}]{}}
        \onslide*<3>{\listFrameDescr[highlightlines={121}]{}}
        \onslide*<4->{\listFrameDescr[highlightlines={122}]{}}
        \medskip
        \onslide*<1-3>{\listDebugInfo{}}
        \onslide*<4>{\listDebugInfo[highlightlines={38,49}]{}}
        \onslide*<5>{\listDebugInfo[highlightlines={39-46}]{}}
    \end{column}
  \end{columns}
\end{frame}

\begin{frame}{Backtraces}{Scanning the stack with \typename{frame\_descr}}
  \begin{columns}[c]
    \begin{column}{0.5\textwidth}
      \begin{itemize}
        \item Given the return address of the code that raises the exception by calling \funcname{caml\_raise\_exn} (\funcarg{pc} argument of \funcname{caml\_stash\_backtrace})
        \item And the position of the stack pointer (\funcarg{sp} argument of \funcname{caml\_stash\_backtrace})
        \item The frame size given by \typename{frame\_descr}.\funcarg{frame\_size}, allows to find the end of the previous frame
        \item And the return address of the caller of this function
        \bigskip
        \onslide<3->{
        \item Note that traps (here the reddish \funcname{ExnA}, \funcname{ExnB} and \funcname{ExnC} blocks) belong to the frame that installed them, and so the frame size changes when traps are removed
        }
      \end{itemize}
    \end{column}
    \begin{column}{0.5\textwidth}
      \raggedleft
      \onslide*<1>{\providecommand\step{2}\input{diagrams/backtrace/backtrace.tex}}%
      \onslide*<2>{\providecommand\step{1}\input{diagrams/backtrace/scanstack.tex}}%
      \onslide*<3>{\providecommand\step{2}\input{diagrams/backtrace/scanstack.tex}}%
      \onslide*<4>{\providecommand\step{3}\input{diagrams/backtrace/scanstack.tex}}%
      \onslide*<5>{\providecommand\step{4}\input{diagrams/backtrace/scanstack.tex}}%
      \onslide*<6>{\providecommand\step{5}\input{diagrams/backtrace/scanstack.tex}}%
    \end{column}
  \end{columns}
\end{frame}

% Duplicate of previous-previous slide
\begin{frame}{Backtraces}{Continuing on \funcname{caml\_stash\_backtrace}}
  \begin{columns}[c]
    \begin{column}{0.5\textwidth}
      \minipage[c][0.75\textheight][s]{\columnwidth}
      \begin{itemize}
          \color{gray}
        \item A C function provided by the runtime (specific to native code)
        \item It scans the stack, between the stack pointer at the raising point up to the next trap, in order to collect the names of the detected function frames
        \item It uses the same technique and information as the Garbage Collector (GC) uses to scan the values live on the stack
      \end{itemize}
      \begin{itemize}
        \item For each function frame detected, store a pointer to the \typename{frame\_descr} in the \localname{domain\_state->backtrace\_buffer} array, effectively constructing the backtrace
      \end{itemize}
      \onslide<2->{
        \begin{itemize}
            \smallskip
          \item Constructed backtrace:
            \funcname{definitely\_raise},
            \onslide<3->{\funcname{probably\_raise}}
        \end{itemize}
      }
      \vfill
      \mintocaml[firstline=9,lastline=13,highlightlines=12]{../src/backtrace/backtrace.ml}
      \endminipage
    \end{column}
    \begin{column}{0.5\textwidth}
      \raggedleft
      \onslide*<1>{\listStashBacktrace[highlightlines={114-115}]{}}
      \onslide*<2>{\providecommand\step{3}\input{diagrams/backtrace/backtrace.tex}}%
      \onslide*<3>{\providecommand\step{4}\input{diagrams/backtrace/backtrace.tex}}%
    \end{column}
  \end{columns}
\end{frame}

\begin{frame}{Backtraces}{Back to \funcname{caml\_raise\_exn}}
  \begin{columns}[c]
    \begin{column}{0.5\textwidth}
      \minipage[c][0.66\textheight][s]{\columnwidth}
      \begin{itemize}\color{gray}
        \item Checks wheteher exceptions backtrace should be recorded, by checking the \funcarg{domain\_state->backtrace\_active} value
        \item Switches to the C stack to be able to execute C code
        \item Calls \funcname{caml\_stash\_backtrace}, to collect the backtrace up to the next trap
      \end{itemize}
      \onslide*<2->{
        \begin{itemize}
          \item<2-> Executes the exception handler in \funcname{probably\_raise} with \funcname{RESTORE\_EXN\_HANDLER\_OCAML}
            \begin{itemize}
              \item Set \regname{RSP} to \localname{domain\_state->exn\_handler} value
              \item Restore \localname{domain\_state->exn\_handler} from trap
              \item Jump to the handler code with \funcname{ret}
            \end{itemize}
        \end{itemize}
      }
      \vfill
      \onslide*<1>{\mintocaml[firstline=9,lastline=13,highlightlines=12]{../src/backtrace/backtrace.ml}}
      \onslide*<2->{\mintocaml[firstline=15,lastline=21,highlightlines={19-21}]{../src/backtrace/backtrace.ml}}
      \endminipage
    \end{column}
    \begin{column}{0.5\textwidth}
      \centering
      \onslide*<1>{
        \mintc[firstline=190,lastline=192]{../ocaml/runtime/amd64.S}
        \mintc[firstline=234,lastline=237]{../ocaml/runtime/amd64.S}
      }
      \onslide*<2>{
        \mintc[firstline=190,lastline=192,highlightlines={191-192}]{../ocaml/runtime/amd64.S}
        \mintc[firstline=234,lastline=237,highlightlines={235-237}]{../ocaml/runtime/amd64.S}
      }
      \onslide*<1>{\listCamlRaiseExn[highlightlines=720]{}}
      \onslide*<2>{\listCamlRaiseExn[highlightlines=722]{}}
      \onslide*<3->{\providecommand\step{5}\input{diagrams/backtrace/backtrace.tex}}
    \end{column}
  \end{columns}
\end{frame}

\begin{frame}{Backtraces}{\funcname{caml\_probably\_raise}'s handler doesn't catch \typename{ExnC}}
  \begin{columns}[c]
    \begin{column}{0.45\textwidth}
      \minipage[c][0.75\textheight][s]{\columnwidth}
      \begin{itemize}
        \item<1-> \funcname{match \dots with}\footnotemark pattern matching can also match exceptions
        \item<1-> The CMM of \funcname{probably\_raise} shows that this \funcname{match \dots with} is implemented with a pattern matching and a \funcname{try \dots with}
        \item<1-> But this one only matches on \typename{ExnA} exception
        \item<2-> Since the pattern matching fails to catch the exception, \funcname{reraise} is called with our current \typename{ExnC} exception
        \item<3-> Disassembly shows that \funcname{reraise} is actually a call to \funcname{caml\_reraise\_exn}, which is also an assembly function provided by the runtime
      \end{itemize}
      \vfill
      \mintocaml[firstline=15,lastline=21,highlightlines={18-21}]{../src/backtrace/backtrace.ml}
      \endminipage
    \end{column}
    \begin{column}{0.55\textwidth}
      \centering
      \onslide*<1>{\mintcmm[highlightlines={10-18}]{../src/backtrace/camlBacktrace\_\_probably\_raise\_351.cmm}}
      \onslide*<2>{\listcmm[highlightlines=18]{../src/backtrace/camlBacktrace\_\_probably\_raise_351.cmm}{CMM of probably\_raise}}
      \onslide*<3>{\mintobjdump[firstline=5,lastline=35,highlightlines=31]{../src/backtrace/camlBacktrace\_\_probably\_raise_351.objdump}}
    \end{column}
  \end{columns}
  \only<1>{\footnotetext[1]{https://blog.janestreet.com/pattern-matching-and-exception-handling-unite/}}
\end{frame}

\newcommand\listCamlReraiseExn[1][]{
  \listasm[firstline=727,lastline=734,#1]{../ocaml/runtime/amd64.S}{runtime/amd64.S:727}}

\begin{frame}{Backtraces}{Introducing \funcname{caml\_reraise\_exn}}
  \begin{columns}[c]
    \begin{column}{0.5\textwidth}
      \minipage[c][0.66\textheight][s]{\columnwidth}
      \begin{itemize}
        \item<1-> The exception have to be reraised to the next handler
        \item<2-> First, checks if the backtrace should be collected with \funcarg{domain\_state->backtrace\_active}
        \only<3>{
        \begin{itemize}
            \item If not, behaves like a \funcname{raise\_notrace} with \funcname{RESTORE\_EXN\_HANDLER\_OCAML}
        \end{itemize}
        }
        \item<4-> Jumps into \funcname{caml\_raise\_exn}
        \item<5-> Calls \funcname{caml\_stash\_backtrace} again, to scan the next part of the stack: from the trap just pop-ed to the next trap
      \end{itemize}
      \vfill
      \mintocaml[firstline=15,lastline=21,highlightlines={18-21}]{../src/backtrace/backtrace.ml}
      \endminipage
    \end{column}
    \begin{column}{0.5\textwidth}
      \raggedleft
      \onslide*<1>{\listCamlReraiseExn{}}
      \onslide*<2>{\listCamlReraiseExn[highlightlines=729]{}}
      \onslide*<3>{
        \mintc[firstline=190,lastline=192,highlightlines={191-192}]{../ocaml/runtime/amd64.S}
        \mintc[firstline=234,lastline=237,highlightlines={235-237}]{../ocaml/runtime/amd64.S}
        \medskip
        \listCamlReraiseExn[highlightlines=731]{}
      }
      \onslide*<4-5>{
        % TODO refactor with \listCamlReraiseExn
        \mintasm[firstline=727,lastline=734,highlightlines=730]{../ocaml/runtime/amd64.S}
        \medskip
      }
      \onslide*<4>{\listCamlRaiseExn[highlightlines=711]{}}
      \onslide*<5>{\listCamlRaiseExn[highlightlines=720]{}}
    \end{column}
  \end{columns}
\end{frame}

\begin{frame}{Backtraces}{Backtrace collection continues}
  \begin{columns}[c]
    \begin{column}{0.55\textwidth}
      \minipage[c][0.75\textheight][s]{\columnwidth}
      \begin{itemize}
        \item<1-> \funcname{caml\_stash\_backtrace} scans the older part of the stack, up to the next trap
        \item<6-> Controls jumps to the nested exception handler in \funcname{maybe\_raise}, which matches only on \typename{ExnB}
        \item<7-> \funcname{caml\_reraise\_exn} is called again, backtrace collection resumes with \funcname{caml\_stash\_backtrace}
      \end{itemize}
      \medskip
      \begin{itemize}
        \item<2-> Constructed backtrace:
        \begin{itemize}
          \item <2-> \funcname{definitely\_raise}
          \item <2-> \funcname{probably\_raise}
          \item <3-> \funcname{probably\_raise} (re-raise)
          \item <4-> \funcname{maybe\_raise}
          \item <5-> \funcname{main}
          \item <8-> \funcname{main} (re-raise)
        \end{itemize}
      \end{itemize}
      \vfill
      \onslide*<1-5>{\mintocaml[firstline=15,lastline=21]{../src/backtrace/backtrace.ml}}
      \onslide*<6>{\mintocaml[firstline=28,lastline=34,highlightlines=31]{../src/backtrace/backtrace.ml}}
      \onslide*<7->{\mintocaml[firstline=28,lastline=34,highlightlines=33]{../src/backtrace/backtrace.ml}}
      \endminipage
    \end{column}
    \begin{column}{0.45\textwidth}
      \raggedleft
      \onslide*<1>{\providecommand\step{6}\input{diagrams/backtrace/backtrace.tex}}%
      \onslide*<2>{\providecommand\step{6}\input{diagrams/backtrace/backtrace.tex}}%
      \onslide*<3>{\providecommand\step{7}\input{diagrams/backtrace/backtrace.tex}}%
      \onslide*<4>{\providecommand\step{8}\input{diagrams/backtrace/backtrace.tex}}%
      \onslide*<5>{\providecommand\step{9}\input{diagrams/backtrace/backtrace.tex}}%
      \onslide*<6>{\providecommand\step{10}\input{diagrams/backtrace/backtrace.tex}}%
      \onslide*<7>{\providecommand\step{11}\input{diagrams/backtrace/backtrace.tex}}%
      \onslide*<8>{\providecommand\step{12}\input{diagrams/backtrace/backtrace.tex}}%
      \onslide*<9>{\providecommand\step{13}\input{diagrams/backtrace/backtrace.tex}}%
    \end{column}
  \end{columns}
\end{frame}

\begin{frame}{Backtraces}{Printing the backtrace}
  \begin{columns}[c]
    \begin{column}{0.45\textwidth}
      \minipage[c][0.66\textheight][s]{\columnwidth}
      \begin{itemize}
        \item<1-> The exception backtrace can now be printed to \funcarg{stdout} with \funcname{Printexc.print_backtrace}
        \item<2-> First the exception backtrace is retrieved into an OCaml array with \funcname{get_raw_backtrace }
        \item<3-> Which is implemented as the \funcname{caml_get_exception_raw_backtrace} C function in the runtime
        \item<4-> And finally printed with \funcname{print_exception_backtrace}
      \end{itemize}
      \vfill
      \mintocaml[firstline=28,lastline=34,highlightlines=33]{../src/backtrace/backtrace.ml}
      \endminipage
    \end{column}
    \begin{column}{0.55\textwidth}
      \onslide*<1,2>{
        \mintocaml[firstline=100,lastline=101]{../ocaml/stdlib/printexc.ml}
      }
      \only<1>{
        \listocaml[firstline=170,lastline=172]{../ocaml/stdlib/printexc.ml}{stdlib/printexc.ml:170}
      }
      \only<2>{
        \listocaml[firstline=170,lastline=172,highlightlines=172]{../ocaml/stdlib/printexc.ml}{stdlib/printexc.ml:170}
      }
      \only<3>{
        \mintc[firstline=154,lastline=156]{../ocaml/runtime/backtrace.c}
        \listc[firstline=171,lastline=192,highlightlines={184-187}]{../ocaml/runtime/backtrace.c}{runtime/backtrace.c:154}
      }
      \only<4>{
        \listocaml[firstline=155,lastline=165]{../ocaml/stdlib/printexc.ml}{stdlib/printexc.ml:55}
      }
    \end{column}
  \end{columns}
\end{frame}


\subsection{The multiple kinds of raise}
\frameSubsection{}{}

\begin{frame}{Backtraces}{When backtraces are not needed}
  \begin{columns}[c]
    \begin{column}{0.45\textwidth}
      \minipage[c][0.66\textheight][s]{\columnwidth}
      \begin{itemize}
        \item<1-> What if the backtrace for an exception is never needed?
        \item<2-> It's possible to raise the exception explicitly with \funcname{raise\_notrace} to make raising as cheap as possible
        \item<3-> An example of use in the \funcname{dynlink} library of the OCaml compiler
      \end{itemize}
      \vfill
      \onslide<3->{
      \listocaml[firstline=205,lastline=209,highlightlines=207]{../ocaml/otherlibs/dynlink/dynlink\_compilerlibs/misc.ml}{otherlibs/dynlink/dynlink\_compilerlibs/misc.ml:205}
      }
      \endminipage
    \end{column}
    \begin{column}{0.55\textwidth}
    \only<4>{
      \mintcmm[highlightlines=3]{../src/notrace/camlNotrace\_\_fun_347.cmm}
      \listcmm{../src/notrace/camlNotrace\_\_all\_somes\_327.cmm}{all\_somes}
    }
    \onslide*<5->{
      \listobjdump[highlightlines={6-9}]{../src/notrace/camlNotrace\_\_fun_347.objdump}{anonymous function from \funcname{all\_somes}}
    }
    \end{column}
  \end{columns}
\end{frame}

\begin{frame}{Backtraces}{Summary table of the multiple kinds of raise}
  \begin{itemize}
    \item When debugging information is enabled and if the backtrace of an exception isn't needed, it's possible to raise with \funcname{raise\_notrace} for performance
    \item When debugging information is \emph{not} enabled, the compiler optimises \funcname{raise} calls to inline assembly (same effect as using \funcname{raise\_notrace})
  \end{itemize}
  \bigskip
  \centering
  \begin{tabular}{| l | c | c | c | c |}
    \hline
    \parbox{10em}{Debug information \\ (\funcarg{-g} flag)}  & \multicolumn{2}{|c|}{Disabled}                                     & \multicolumn{2}{|c|}{Enabled} \\
    \hline
    Raise kind                                               & \funcname{raise} & \funcname{raise\_notrace}                       & \funcname{raise} & \funcname{raise\_notrace} \\
    \hline
    Compilation                                              & \parbox{8em}{inline assembly\\ (optimization)} & inline assembly   & \funcname{caml\_raise\_exn} & inline assembly \\
    \hline
  \end{tabular}
\end{frame}


%
% Takeaway #5
%
\subsection*{Takeaway \#5}
\frameSubsectionTakeaway{}
\againframe<5>{takeaway}
}%
    \end{column}
  \end{columns}
\end{frame}

\begin{frame}{Backtraces}{Back to \funcname{caml\_raise\_exn}}
  \begin{columns}[c]
    \begin{column}{0.5\textwidth}
      \minipage[c][0.66\textheight][s]{\columnwidth}
      \begin{itemize}\color{gray}
        \item Checks wheteher exceptions backtrace should be recorded, by checking the \funcarg{domain\_state->backtrace\_active} value
        \item Switches to the C stack to be able to execute C code
        \item Calls \funcname{caml\_stash\_backtrace}, to collect the backtrace up to the next trap
      \end{itemize}
      \onslide*<2->{
        \begin{itemize}
          \item<2-> Executes the exception handler in \funcname{probably\_raise} with \funcname{RESTORE\_EXN\_HANDLER\_OCAML}
            \begin{itemize}
              \item Set \regname{RSP} to \localname{domain\_state->exn\_handler} value
              \item Restore \localname{domain\_state->exn\_handler} from trap
              \item Jump to the handler code with \funcname{ret}
            \end{itemize}
        \end{itemize}
      }
      \vfill
      \onslide*<1>{\mintocaml[firstline=9,lastline=13,highlightlines=12]{../src/backtrace/backtrace.ml}}
      \onslide*<2->{\mintocaml[firstline=15,lastline=21,highlightlines={19-21}]{../src/backtrace/backtrace.ml}}
      \endminipage
    \end{column}
    \begin{column}{0.5\textwidth}
      \centering
      \onslide*<1>{
        \mintc[firstline=190,lastline=192]{../ocaml/runtime/amd64.S}
        \mintc[firstline=234,lastline=237]{../ocaml/runtime/amd64.S}
      }
      \onslide*<2>{
        \mintc[firstline=190,lastline=192,highlightlines={191-192}]{../ocaml/runtime/amd64.S}
        \mintc[firstline=234,lastline=237,highlightlines={235-237}]{../ocaml/runtime/amd64.S}
      }
      \onslide*<1>{\listCamlRaiseExn[highlightlines=720]{}}
      \onslide*<2>{\listCamlRaiseExn[highlightlines=722]{}}
      \onslide*<3->{\providecommand\step{5}%
% Backtraces
%
\subsection{One advantage of raising with debugging information}
\frameSubsection{}{}

\begin{frame}{Backtraces}{Debugging information \& source code}
  \begin{columns}[c]
    \begin{column}{0.5\textwidth}
      \begin{itemize}
        \item Raising an exception is fun and games, but how do we know where the exception originated? $\rightarrow$ With the backtrace associated with the exception!
        \item In order to get a backtrace from the point where an exception was raised, it's necessary to compile with debugging information enabled (\funcarg{-g} flag for the compiler)
        \item Same example as in the `Nested handlers' section
      \end{itemize}
    \end{column}
    \begin{column}{0.5\textwidth}
      \listocaml[firstline=9,lastline=34]{../src/backtrace/backtrace.ml}{backtrace/backtrace.ml}
    \end{column}
  \end{columns}
\end{frame}

\begin{frame}{Backtraces}{CMM and disassembly of \funcname{definitely\_raise}}
  \begin{columns}[c]
    \begin{column}{0.5\textwidth}
      \minipage[c][0.66\textheight][s]{\columnwidth}
      \begin{itemize}
        \item<1-> An effect of compiling with debugging information (\funcarg{-g}): the CMM no longer uses \funcname{raise\_notrace} to raise the exception but \funcname{raise} instead
        \item<2-> Disassembly shows that CMM \funcname{raise} is actually a call to \funcname{caml\_raise\_exn}, a function in assembler provided by the runtime
      \end{itemize}
      \vfill
      \mintocaml[firstline=9,lastline=13,highlightlines=12]{../src/backtrace/backtrace.ml}
      \endminipage
    \end{column}
    \begin{column}{0.5\textwidth}
      \onslide*<1>{
        \mintcmm[highlightlines=5]{../src/backtrace/camlBacktrace\_\_definitely\_raise\_347.cmm}
      }
      \onslide*<2>{
        \mintobjdump[firstline=2,highlightlines=14]{../src/backtrace/camlBacktrace\_\_definitely\_raise\_347.objdump}
      }
    \end{column}
  \end{columns}
\end{frame}

\begin{frame}{Backtraces}{Introducing \funcname{caml\_raise\_exn}}
  \begin{columns}[c]
    \begin{column}{0.5\textwidth}
      \minipage[c][0.66\textheight][s]{\columnwidth}
      \onslide*<1>{
        \begin{itemize}
          \item Raising the exception is performed using \funcname{caml\_raise\_exn} which is
            \begin{itemize}
              \item An assembly function provided by the runtime
              \item Architecture specific
            \end{itemize}
          \item Let's see what it does differently from \funcname{raise\_notrace}\dots
          \item We are about to raise \typename{ExnC} with \funcname{caml\_raise\_exn} from \funcname{definitely\_raise}
        \end{itemize}
      }
      \onslide*<2>{
        \mintobjdump[firstline=2,highlightlines=14]{../src/backtrace/camlBacktrace\_\_definitely\_raise\_347.objdump}
      }
      \vfill
      \mintocaml[firstline=9,lastline=13,highlightlines=12]{../src/backtrace/backtrace.ml}
      \endminipage
    \end{column}
    \begin{column}{0.5\textwidth}
      \centering
      \providecommand\step{1}\input{diagrams/backtrace/backtrace.tex}
    \end{column}
  \end{columns}
\end{frame}

\begin{frame}{Backtraces}{But before that, we need more fields from \typename{caml\_domain\_state}}
  \begin{columns}
    \begin{column}{0.5\textwidth}
      \begin{itemize}
        \item Remember \typename{caml\_domain\_state} the per-domain runtime state?
        \item Some other members are of interest to us now\dots
        \item \localname{backtrace\_active} is used to know if the runtime should collect backtraces
        \item \localname{backtrace\_active} can be set by
          \begin{itemize}
            \item The \funcarg{b} flag in the \funcname{OCAMLRUNPARAM} environment variable
            \item \mintinline{ocaml}{Printexc.record_backtrace : bool -> unit} \footnotemark
          \end{itemize}
      \end{itemize}
    \end{column}
    \begin{column}{0.5\textwidth}
      \begin{listing}
        \mintc[highlightlines={9-11}]{src/ocaml/domain\_state\_backtrace.h}
        \caption{runtime/caml/domain\_state.h}
      \end{listing}
    \end{column}
  \end{columns}
  \footnotetext[1]{https://v2.ocaml.org/api/Printexc.html}
\end{frame}

\newcommand\listCamlRaiseExn[1][]{
  \listasm[firstline=702,lastline=725,#1]{../ocaml/runtime/amd64.S}{runtime/amd64.S:702}}

\begin{frame}{Backtraces}{Walk through \funcname{caml\_raise\_exn}}
  \begin{columns}[T]
    \begin{column}{0.5\textwidth}
      \begin{itemize}
        \item<1-> First checks whether exceptions backtrace should be recorded
        \item<1-> By checking the \funcarg{domain\_state->backtrace\_active} value
        \item<2-> If not, acts as \funcname{raise\_notrace} with \funcname{RESTORE\_EXN\_HANDLER\_OCAML}
          \begin{itemize}
            \item Set \regname{RSP} to \localname{domain\_state->exn\_handler} value
            \item Restore \localname{domain\_state->exn\_handler} from trap
            \item Jump with \funcname{ret} (same effect as using \funcname{pop} and \funcname{jmp})
          \end{itemize}
        \item<3-> Otherwise, switches to the C stack to be able to execute C code
        \item<4-> Calls \funcname{caml\_stash\_backtrace}, a C function provided by the runtime\ignorespaces
          \onslide<6->{, with
            \begin{itemize}
              \item \funcarg{exn}: exception value
              \item \funcarg{pc}: code address of the raising point
              \item \funcarg{sp}: stack address of the raising function
              \item \funcarg{trapsp}: stack address of the next handler
            \end{itemize}
          }
      \end{itemize}
      \bigskip
      \mintocaml[firstline=9,lastline=13,highlightlines=12]{../src/backtrace/backtrace.ml}
    \end{column}
    \begin{column}{0.5\textwidth}
      \raggedleft
      \onslide*<1,3-4>{
        \mintc[firstline=190,lastline=192]{../ocaml/runtime/amd64.S}
        \mintc[firstline=234,lastline=237]{../ocaml/runtime/amd64.S}
      }
      \onslide*<2>{
        \mintc[firstline=190,lastline=192,highlightlines={191-192}]{../ocaml/runtime/amd64.S}
        \mintc[firstline=234,lastline=237,highlightlines={235-237}]{../ocaml/runtime/amd64.S}
      }
      \onslide*<1>{\listCamlRaiseExn[highlightlines=705]{}}%
      \onslide*<2>{\listCamlRaiseExn[highlightlines={707-708}]{}}%
      \onslide*<3>{\listCamlRaiseExn[highlightlines={712-714}]{}}%
      \onslide*<4>{\listCamlRaiseExn[highlightlines={715-720}]{}}%
      \onslide*<5>{\providecommand\step{1}\input{diagrams/backtrace/backtrace.tex}}%
      \onslide*<6>{\providecommand\step{2}\input{diagrams/backtrace/backtrace.tex}}%
    \end{column}
  \end{columns}
\end{frame}

\newcommand\listStashBacktrace[1][]{
  \listc[firstline=91,lastline=120,#1]{../ocaml/runtime/backtrace\_nat.c}{runtime/backtrace\_nat.c:91}}

\begin{frame}{Backtraces}{Introducing \funcname{caml\_stash\_backtrace}}
  \begin{columns}[c]
    \begin{column}{0.5\textwidth}
      \minipage[c][0.66\textheight][s]{\columnwidth}
      \begin{itemize}
        \item<1-> A C function provided by the runtime (specific to native code)
        \item<2-> It scans the stack, between the stack pointer at the raising point up to the next trap, in order to collect the names of the detected function frames
          \onslide*<2>{
            \begin{itemize}
              \item And more information, like line number, column number...
            \end{itemize}
          }
        \item<3-> It uses the same technique and information as the Garbage Collector (GC) uses to scan the values live on the stack
          \onslide*<4>{
            \begin{itemize}
              \item \typename{frame\_descr} are at the core of stack unwinding
            \end{itemize}
          }
      \end{itemize}
      \vfill
      \mintocaml[firstline=9,lastline=13,highlightlines=12]{../src/backtrace/backtrace.ml}
      \endminipage
    \end{column}
    \begin{column}{0.5\textwidth}
      \onslide*<1>{\listStashBacktrace[highlightlines=91]{}}
      \onslide*<2>{\listStashBacktrace[highlightlines={91,117-118}]{}}
      \onslide*<3->{\listStashBacktrace[highlightlines={94,106,109-110}]{}}
    \end{column}
  \end{columns}
\end{frame}

\newcommand\listFrameDescr[1][]{
  \listocaml[firstline=111,lastline=124,#1]{../ocaml/asmcomp/emitaux.ml}{asmcomp/emitaux.ml:111}}
\newcommand\listDebugInfo[1][]{
  \listocaml[firstline=38,lastline=49,#1]{../ocaml/lambda/debuginfo.mli}{lambda/debuginfo.mli:38}}

\begin{frame}{Backtraces}{\typename{frame\_descr} and \typename{frame\_debuginfo} in a nutshell}
  \begin{columns}[c]
    \begin{column}{0.5\textwidth}
      \begin{itemize}
        \item<1-> For every return address in an OCaml program, there is a associated \typename{frame\_descr}
        \item<2-> A \typename{frame\_descr} specifies
        \begin{itemize}
          \item<2-> The size of the function stack frame
          \item<3-> Where live values are on the stack (so that the GC can mark them as live and prevent from being swept)
        \end{itemize}
        \item<4-> When compiled with debug information, \typename{frame\_descr} will be linked to a \typename{frame\_debuginfo}
        \begin{itemize}
          \item<5-> Contains information about the position in the source code (file name, line and column number)
        \end{itemize}
      \end{itemize}
    \end{column}
    \begin{column}{0.5\textwidth}
        \onslide*<1>{\listFrameDescr[highlightlines={118-122}]{}}
        \onslide*<2>{\listFrameDescr[highlightlines={120}]{}}
        \onslide*<3>{\listFrameDescr[highlightlines={121}]{}}
        \onslide*<4->{\listFrameDescr[highlightlines={122}]{}}
        \medskip
        \onslide*<1-3>{\listDebugInfo{}}
        \onslide*<4>{\listDebugInfo[highlightlines={38,49}]{}}
        \onslide*<5>{\listDebugInfo[highlightlines={39-46}]{}}
    \end{column}
  \end{columns}
\end{frame}

\begin{frame}{Backtraces}{Scanning the stack with \typename{frame\_descr}}
  \begin{columns}[c]
    \begin{column}{0.5\textwidth}
      \begin{itemize}
        \item Given the return address of the code that raises the exception by calling \funcname{caml\_raise\_exn} (\funcarg{pc} argument of \funcname{caml\_stash\_backtrace})
        \item And the position of the stack pointer (\funcarg{sp} argument of \funcname{caml\_stash\_backtrace})
        \item The frame size given by \typename{frame\_descr}.\funcarg{frame\_size}, allows to find the end of the previous frame
        \item And the return address of the caller of this function
        \bigskip
        \onslide<3->{
        \item Note that traps (here the reddish \funcname{ExnA}, \funcname{ExnB} and \funcname{ExnC} blocks) belong to the frame that installed them, and so the frame size changes when traps are removed
        }
      \end{itemize}
    \end{column}
    \begin{column}{0.5\textwidth}
      \raggedleft
      \onslide*<1>{\providecommand\step{2}\input{diagrams/backtrace/backtrace.tex}}%
      \onslide*<2>{\providecommand\step{1}\input{diagrams/backtrace/scanstack.tex}}%
      \onslide*<3>{\providecommand\step{2}\input{diagrams/backtrace/scanstack.tex}}%
      \onslide*<4>{\providecommand\step{3}\input{diagrams/backtrace/scanstack.tex}}%
      \onslide*<5>{\providecommand\step{4}\input{diagrams/backtrace/scanstack.tex}}%
      \onslide*<6>{\providecommand\step{5}\input{diagrams/backtrace/scanstack.tex}}%
    \end{column}
  \end{columns}
\end{frame}

% Duplicate of previous-previous slide
\begin{frame}{Backtraces}{Continuing on \funcname{caml\_stash\_backtrace}}
  \begin{columns}[c]
    \begin{column}{0.5\textwidth}
      \minipage[c][0.75\textheight][s]{\columnwidth}
      \begin{itemize}
          \color{gray}
        \item A C function provided by the runtime (specific to native code)
        \item It scans the stack, between the stack pointer at the raising point up to the next trap, in order to collect the names of the detected function frames
        \item It uses the same technique and information as the Garbage Collector (GC) uses to scan the values live on the stack
      \end{itemize}
      \begin{itemize}
        \item For each function frame detected, store a pointer to the \typename{frame\_descr} in the \localname{domain\_state->backtrace\_buffer} array, effectively constructing the backtrace
      \end{itemize}
      \onslide<2->{
        \begin{itemize}
            \smallskip
          \item Constructed backtrace:
            \funcname{definitely\_raise},
            \onslide<3->{\funcname{probably\_raise}}
        \end{itemize}
      }
      \vfill
      \mintocaml[firstline=9,lastline=13,highlightlines=12]{../src/backtrace/backtrace.ml}
      \endminipage
    \end{column}
    \begin{column}{0.5\textwidth}
      \raggedleft
      \onslide*<1>{\listStashBacktrace[highlightlines={114-115}]{}}
      \onslide*<2>{\providecommand\step{3}\input{diagrams/backtrace/backtrace.tex}}%
      \onslide*<3>{\providecommand\step{4}\input{diagrams/backtrace/backtrace.tex}}%
    \end{column}
  \end{columns}
\end{frame}

\begin{frame}{Backtraces}{Back to \funcname{caml\_raise\_exn}}
  \begin{columns}[c]
    \begin{column}{0.5\textwidth}
      \minipage[c][0.66\textheight][s]{\columnwidth}
      \begin{itemize}\color{gray}
        \item Checks wheteher exceptions backtrace should be recorded, by checking the \funcarg{domain\_state->backtrace\_active} value
        \item Switches to the C stack to be able to execute C code
        \item Calls \funcname{caml\_stash\_backtrace}, to collect the backtrace up to the next trap
      \end{itemize}
      \onslide*<2->{
        \begin{itemize}
          \item<2-> Executes the exception handler in \funcname{probably\_raise} with \funcname{RESTORE\_EXN\_HANDLER\_OCAML}
            \begin{itemize}
              \item Set \regname{RSP} to \localname{domain\_state->exn\_handler} value
              \item Restore \localname{domain\_state->exn\_handler} from trap
              \item Jump to the handler code with \funcname{ret}
            \end{itemize}
        \end{itemize}
      }
      \vfill
      \onslide*<1>{\mintocaml[firstline=9,lastline=13,highlightlines=12]{../src/backtrace/backtrace.ml}}
      \onslide*<2->{\mintocaml[firstline=15,lastline=21,highlightlines={19-21}]{../src/backtrace/backtrace.ml}}
      \endminipage
    \end{column}
    \begin{column}{0.5\textwidth}
      \centering
      \onslide*<1>{
        \mintc[firstline=190,lastline=192]{../ocaml/runtime/amd64.S}
        \mintc[firstline=234,lastline=237]{../ocaml/runtime/amd64.S}
      }
      \onslide*<2>{
        \mintc[firstline=190,lastline=192,highlightlines={191-192}]{../ocaml/runtime/amd64.S}
        \mintc[firstline=234,lastline=237,highlightlines={235-237}]{../ocaml/runtime/amd64.S}
      }
      \onslide*<1>{\listCamlRaiseExn[highlightlines=720]{}}
      \onslide*<2>{\listCamlRaiseExn[highlightlines=722]{}}
      \onslide*<3->{\providecommand\step{5}\input{diagrams/backtrace/backtrace.tex}}
    \end{column}
  \end{columns}
\end{frame}

\begin{frame}{Backtraces}{\funcname{caml\_probably\_raise}'s handler doesn't catch \typename{ExnC}}
  \begin{columns}[c]
    \begin{column}{0.45\textwidth}
      \minipage[c][0.75\textheight][s]{\columnwidth}
      \begin{itemize}
        \item<1-> \funcname{match \dots with}\footnotemark pattern matching can also match exceptions
        \item<1-> The CMM of \funcname{probably\_raise} shows that this \funcname{match \dots with} is implemented with a pattern matching and a \funcname{try \dots with}
        \item<1-> But this one only matches on \typename{ExnA} exception
        \item<2-> Since the pattern matching fails to catch the exception, \funcname{reraise} is called with our current \typename{ExnC} exception
        \item<3-> Disassembly shows that \funcname{reraise} is actually a call to \funcname{caml\_reraise\_exn}, which is also an assembly function provided by the runtime
      \end{itemize}
      \vfill
      \mintocaml[firstline=15,lastline=21,highlightlines={18-21}]{../src/backtrace/backtrace.ml}
      \endminipage
    \end{column}
    \begin{column}{0.55\textwidth}
      \centering
      \onslide*<1>{\mintcmm[highlightlines={10-18}]{../src/backtrace/camlBacktrace\_\_probably\_raise\_351.cmm}}
      \onslide*<2>{\listcmm[highlightlines=18]{../src/backtrace/camlBacktrace\_\_probably\_raise_351.cmm}{CMM of probably\_raise}}
      \onslide*<3>{\mintobjdump[firstline=5,lastline=35,highlightlines=31]{../src/backtrace/camlBacktrace\_\_probably\_raise_351.objdump}}
    \end{column}
  \end{columns}
  \only<1>{\footnotetext[1]{https://blog.janestreet.com/pattern-matching-and-exception-handling-unite/}}
\end{frame}

\newcommand\listCamlReraiseExn[1][]{
  \listasm[firstline=727,lastline=734,#1]{../ocaml/runtime/amd64.S}{runtime/amd64.S:727}}

\begin{frame}{Backtraces}{Introducing \funcname{caml\_reraise\_exn}}
  \begin{columns}[c]
    \begin{column}{0.5\textwidth}
      \minipage[c][0.66\textheight][s]{\columnwidth}
      \begin{itemize}
        \item<1-> The exception have to be reraised to the next handler
        \item<2-> First, checks if the backtrace should be collected with \funcarg{domain\_state->backtrace\_active}
        \only<3>{
        \begin{itemize}
            \item If not, behaves like a \funcname{raise\_notrace} with \funcname{RESTORE\_EXN\_HANDLER\_OCAML}
        \end{itemize}
        }
        \item<4-> Jumps into \funcname{caml\_raise\_exn}
        \item<5-> Calls \funcname{caml\_stash\_backtrace} again, to scan the next part of the stack: from the trap just pop-ed to the next trap
      \end{itemize}
      \vfill
      \mintocaml[firstline=15,lastline=21,highlightlines={18-21}]{../src/backtrace/backtrace.ml}
      \endminipage
    \end{column}
    \begin{column}{0.5\textwidth}
      \raggedleft
      \onslide*<1>{\listCamlReraiseExn{}}
      \onslide*<2>{\listCamlReraiseExn[highlightlines=729]{}}
      \onslide*<3>{
        \mintc[firstline=190,lastline=192,highlightlines={191-192}]{../ocaml/runtime/amd64.S}
        \mintc[firstline=234,lastline=237,highlightlines={235-237}]{../ocaml/runtime/amd64.S}
        \medskip
        \listCamlReraiseExn[highlightlines=731]{}
      }
      \onslide*<4-5>{
        % TODO refactor with \listCamlReraiseExn
        \mintasm[firstline=727,lastline=734,highlightlines=730]{../ocaml/runtime/amd64.S}
        \medskip
      }
      \onslide*<4>{\listCamlRaiseExn[highlightlines=711]{}}
      \onslide*<5>{\listCamlRaiseExn[highlightlines=720]{}}
    \end{column}
  \end{columns}
\end{frame}

\begin{frame}{Backtraces}{Backtrace collection continues}
  \begin{columns}[c]
    \begin{column}{0.55\textwidth}
      \minipage[c][0.75\textheight][s]{\columnwidth}
      \begin{itemize}
        \item<1-> \funcname{caml\_stash\_backtrace} scans the older part of the stack, up to the next trap
        \item<6-> Controls jumps to the nested exception handler in \funcname{maybe\_raise}, which matches only on \typename{ExnB}
        \item<7-> \funcname{caml\_reraise\_exn} is called again, backtrace collection resumes with \funcname{caml\_stash\_backtrace}
      \end{itemize}
      \medskip
      \begin{itemize}
        \item<2-> Constructed backtrace:
        \begin{itemize}
          \item <2-> \funcname{definitely\_raise}
          \item <2-> \funcname{probably\_raise}
          \item <3-> \funcname{probably\_raise} (re-raise)
          \item <4-> \funcname{maybe\_raise}
          \item <5-> \funcname{main}
          \item <8-> \funcname{main} (re-raise)
        \end{itemize}
      \end{itemize}
      \vfill
      \onslide*<1-5>{\mintocaml[firstline=15,lastline=21]{../src/backtrace/backtrace.ml}}
      \onslide*<6>{\mintocaml[firstline=28,lastline=34,highlightlines=31]{../src/backtrace/backtrace.ml}}
      \onslide*<7->{\mintocaml[firstline=28,lastline=34,highlightlines=33]{../src/backtrace/backtrace.ml}}
      \endminipage
    \end{column}
    \begin{column}{0.45\textwidth}
      \raggedleft
      \onslide*<1>{\providecommand\step{6}\input{diagrams/backtrace/backtrace.tex}}%
      \onslide*<2>{\providecommand\step{6}\input{diagrams/backtrace/backtrace.tex}}%
      \onslide*<3>{\providecommand\step{7}\input{diagrams/backtrace/backtrace.tex}}%
      \onslide*<4>{\providecommand\step{8}\input{diagrams/backtrace/backtrace.tex}}%
      \onslide*<5>{\providecommand\step{9}\input{diagrams/backtrace/backtrace.tex}}%
      \onslide*<6>{\providecommand\step{10}\input{diagrams/backtrace/backtrace.tex}}%
      \onslide*<7>{\providecommand\step{11}\input{diagrams/backtrace/backtrace.tex}}%
      \onslide*<8>{\providecommand\step{12}\input{diagrams/backtrace/backtrace.tex}}%
      \onslide*<9>{\providecommand\step{13}\input{diagrams/backtrace/backtrace.tex}}%
    \end{column}
  \end{columns}
\end{frame}

\begin{frame}{Backtraces}{Printing the backtrace}
  \begin{columns}[c]
    \begin{column}{0.45\textwidth}
      \minipage[c][0.66\textheight][s]{\columnwidth}
      \begin{itemize}
        \item<1-> The exception backtrace can now be printed to \funcarg{stdout} with \funcname{Printexc.print_backtrace}
        \item<2-> First the exception backtrace is retrieved into an OCaml array with \funcname{get_raw_backtrace }
        \item<3-> Which is implemented as the \funcname{caml_get_exception_raw_backtrace} C function in the runtime
        \item<4-> And finally printed with \funcname{print_exception_backtrace}
      \end{itemize}
      \vfill
      \mintocaml[firstline=28,lastline=34,highlightlines=33]{../src/backtrace/backtrace.ml}
      \endminipage
    \end{column}
    \begin{column}{0.55\textwidth}
      \onslide*<1,2>{
        \mintocaml[firstline=100,lastline=101]{../ocaml/stdlib/printexc.ml}
      }
      \only<1>{
        \listocaml[firstline=170,lastline=172]{../ocaml/stdlib/printexc.ml}{stdlib/printexc.ml:170}
      }
      \only<2>{
        \listocaml[firstline=170,lastline=172,highlightlines=172]{../ocaml/stdlib/printexc.ml}{stdlib/printexc.ml:170}
      }
      \only<3>{
        \mintc[firstline=154,lastline=156]{../ocaml/runtime/backtrace.c}
        \listc[firstline=171,lastline=192,highlightlines={184-187}]{../ocaml/runtime/backtrace.c}{runtime/backtrace.c:154}
      }
      \only<4>{
        \listocaml[firstline=155,lastline=165]{../ocaml/stdlib/printexc.ml}{stdlib/printexc.ml:55}
      }
    \end{column}
  \end{columns}
\end{frame}


\subsection{The multiple kinds of raise}
\frameSubsection{}{}

\begin{frame}{Backtraces}{When backtraces are not needed}
  \begin{columns}[c]
    \begin{column}{0.45\textwidth}
      \minipage[c][0.66\textheight][s]{\columnwidth}
      \begin{itemize}
        \item<1-> What if the backtrace for an exception is never needed?
        \item<2-> It's possible to raise the exception explicitly with \funcname{raise\_notrace} to make raising as cheap as possible
        \item<3-> An example of use in the \funcname{dynlink} library of the OCaml compiler
      \end{itemize}
      \vfill
      \onslide<3->{
      \listocaml[firstline=205,lastline=209,highlightlines=207]{../ocaml/otherlibs/dynlink/dynlink\_compilerlibs/misc.ml}{otherlibs/dynlink/dynlink\_compilerlibs/misc.ml:205}
      }
      \endminipage
    \end{column}
    \begin{column}{0.55\textwidth}
    \only<4>{
      \mintcmm[highlightlines=3]{../src/notrace/camlNotrace\_\_fun_347.cmm}
      \listcmm{../src/notrace/camlNotrace\_\_all\_somes\_327.cmm}{all\_somes}
    }
    \onslide*<5->{
      \listobjdump[highlightlines={6-9}]{../src/notrace/camlNotrace\_\_fun_347.objdump}{anonymous function from \funcname{all\_somes}}
    }
    \end{column}
  \end{columns}
\end{frame}

\begin{frame}{Backtraces}{Summary table of the multiple kinds of raise}
  \begin{itemize}
    \item When debugging information is enabled and if the backtrace of an exception isn't needed, it's possible to raise with \funcname{raise\_notrace} for performance
    \item When debugging information is \emph{not} enabled, the compiler optimises \funcname{raise} calls to inline assembly (same effect as using \funcname{raise\_notrace})
  \end{itemize}
  \bigskip
  \centering
  \begin{tabular}{| l | c | c | c | c |}
    \hline
    \parbox{10em}{Debug information \\ (\funcarg{-g} flag)}  & \multicolumn{2}{|c|}{Disabled}                                     & \multicolumn{2}{|c|}{Enabled} \\
    \hline
    Raise kind                                               & \funcname{raise} & \funcname{raise\_notrace}                       & \funcname{raise} & \funcname{raise\_notrace} \\
    \hline
    Compilation                                              & \parbox{8em}{inline assembly\\ (optimization)} & inline assembly   & \funcname{caml\_raise\_exn} & inline assembly \\
    \hline
  \end{tabular}
\end{frame}


%
% Takeaway #5
%
\subsection*{Takeaway \#5}
\frameSubsectionTakeaway{}
\againframe<5>{takeaway}
}
    \end{column}
  \end{columns}
\end{frame}

\begin{frame}{Backtraces}{\funcname{caml\_probably\_raise}'s handler doesn't catch \typename{ExnC}}
  \begin{columns}[c]
    \begin{column}{0.45\textwidth}
      \minipage[c][0.75\textheight][s]{\columnwidth}
      \begin{itemize}
        \item<1-> \funcname{match \dots with}\footnotemark pattern matching can also match exceptions
        \item<1-> The CMM of \funcname{probably\_raise} shows that this \funcname{match \dots with} is implemented with a pattern matching and a \funcname{try \dots with}
        \item<1-> But this one only matches on \typename{ExnA} exception
        \item<2-> Since the pattern matching fails to catch the exception, \funcname{reraise} is called with our current \typename{ExnC} exception
        \item<3-> Disassembly shows that \funcname{reraise} is actually a call to \funcname{caml\_reraise\_exn}, which is also an assembly function provided by the runtime
      \end{itemize}
      \vfill
      \mintocaml[firstline=15,lastline=21,highlightlines={18-21}]{../src/backtrace/backtrace.ml}
      \endminipage
    \end{column}
    \begin{column}{0.55\textwidth}
      \centering
      \onslide*<1>{\mintcmm[highlightlines={10-18}]{../src/backtrace/camlBacktrace\_\_probably\_raise\_351.cmm}}
      \onslide*<2>{\listcmm[highlightlines=18]{../src/backtrace/camlBacktrace\_\_probably\_raise_351.cmm}{CMM of probably\_raise}}
      \onslide*<3>{\mintobjdump[firstline=5,lastline=35,highlightlines=31]{../src/backtrace/camlBacktrace\_\_probably\_raise_351.objdump}}
    \end{column}
  \end{columns}
  \only<1>{\footnotetext[1]{https://blog.janestreet.com/pattern-matching-and-exception-handling-unite/}}
\end{frame}

\newcommand\listCamlReraiseExn[1][]{
  \listasm[firstline=727,lastline=734,#1]{../ocaml/runtime/amd64.S}{runtime/amd64.S:727}}

\begin{frame}{Backtraces}{Introducing \funcname{caml\_reraise\_exn}}
  \begin{columns}[c]
    \begin{column}{0.5\textwidth}
      \minipage[c][0.66\textheight][s]{\columnwidth}
      \begin{itemize}
        \item<1-> The exception have to be reraised to the next handler
        \item<2-> First, checks if the backtrace should be collected with \funcarg{domain\_state->backtrace\_active}
        \only<3>{
        \begin{itemize}
            \item If not, behaves like a \funcname{raise\_notrace} with \funcname{RESTORE\_EXN\_HANDLER\_OCAML}
        \end{itemize}
        }
        \item<4-> Jumps into \funcname{caml\_raise\_exn}
        \item<5-> Calls \funcname{caml\_stash\_backtrace} again, to scan the next part of the stack: from the trap just pop-ed to the next trap
      \end{itemize}
      \vfill
      \mintocaml[firstline=15,lastline=21,highlightlines={18-21}]{../src/backtrace/backtrace.ml}
      \endminipage
    \end{column}
    \begin{column}{0.5\textwidth}
      \raggedleft
      \onslide*<1>{\listCamlReraiseExn{}}
      \onslide*<2>{\listCamlReraiseExn[highlightlines=729]{}}
      \onslide*<3>{
        \mintc[firstline=190,lastline=192,highlightlines={191-192}]{../ocaml/runtime/amd64.S}
        \mintc[firstline=234,lastline=237,highlightlines={235-237}]{../ocaml/runtime/amd64.S}
        \medskip
        \listCamlReraiseExn[highlightlines=731]{}
      }
      \onslide*<4-5>{
        % TODO refactor with \listCamlReraiseExn
        \mintasm[firstline=727,lastline=734,highlightlines=730]{../ocaml/runtime/amd64.S}
        \medskip
      }
      \onslide*<4>{\listCamlRaiseExn[highlightlines=711]{}}
      \onslide*<5>{\listCamlRaiseExn[highlightlines=720]{}}
    \end{column}
  \end{columns}
\end{frame}

\begin{frame}{Backtraces}{Backtrace collection continues}
  \begin{columns}[c]
    \begin{column}{0.55\textwidth}
      \minipage[c][0.75\textheight][s]{\columnwidth}
      \begin{itemize}
        \item<1-> \funcname{caml\_stash\_backtrace} scans the older part of the stack, up to the next trap
        \item<6-> Controls jumps to the nested exception handler in \funcname{maybe\_raise}, which matches only on \typename{ExnB}
        \item<7-> \funcname{caml\_reraise\_exn} is called again, backtrace collection resumes with \funcname{caml\_stash\_backtrace}
      \end{itemize}
      \medskip
      \begin{itemize}
        \item<2-> Constructed backtrace:
        \begin{itemize}
          \item <2-> \funcname{definitely\_raise}
          \item <2-> \funcname{probably\_raise}
          \item <3-> \funcname{probably\_raise} (re-raise)
          \item <4-> \funcname{maybe\_raise}
          \item <5-> \funcname{main}
          \item <8-> \funcname{main} (re-raise)
        \end{itemize}
      \end{itemize}
      \vfill
      \onslide*<1-5>{\mintocaml[firstline=15,lastline=21]{../src/backtrace/backtrace.ml}}
      \onslide*<6>{\mintocaml[firstline=28,lastline=34,highlightlines=31]{../src/backtrace/backtrace.ml}}
      \onslide*<7->{\mintocaml[firstline=28,lastline=34,highlightlines=33]{../src/backtrace/backtrace.ml}}
      \endminipage
    \end{column}
    \begin{column}{0.45\textwidth}
      \raggedleft
      \onslide*<1>{\providecommand\step{6}%
% Backtraces
%
\subsection{One advantage of raising with debugging information}
\frameSubsection{}{}

\begin{frame}{Backtraces}{Debugging information \& source code}
  \begin{columns}[c]
    \begin{column}{0.5\textwidth}
      \begin{itemize}
        \item Raising an exception is fun and games, but how do we know where the exception originated? $\rightarrow$ With the backtrace associated with the exception!
        \item In order to get a backtrace from the point where an exception was raised, it's necessary to compile with debugging information enabled (\funcarg{-g} flag for the compiler)
        \item Same example as in the `Nested handlers' section
      \end{itemize}
    \end{column}
    \begin{column}{0.5\textwidth}
      \listocaml[firstline=9,lastline=34]{../src/backtrace/backtrace.ml}{backtrace/backtrace.ml}
    \end{column}
  \end{columns}
\end{frame}

\begin{frame}{Backtraces}{CMM and disassembly of \funcname{definitely\_raise}}
  \begin{columns}[c]
    \begin{column}{0.5\textwidth}
      \minipage[c][0.66\textheight][s]{\columnwidth}
      \begin{itemize}
        \item<1-> An effect of compiling with debugging information (\funcarg{-g}): the CMM no longer uses \funcname{raise\_notrace} to raise the exception but \funcname{raise} instead
        \item<2-> Disassembly shows that CMM \funcname{raise} is actually a call to \funcname{caml\_raise\_exn}, a function in assembler provided by the runtime
      \end{itemize}
      \vfill
      \mintocaml[firstline=9,lastline=13,highlightlines=12]{../src/backtrace/backtrace.ml}
      \endminipage
    \end{column}
    \begin{column}{0.5\textwidth}
      \onslide*<1>{
        \mintcmm[highlightlines=5]{../src/backtrace/camlBacktrace\_\_definitely\_raise\_347.cmm}
      }
      \onslide*<2>{
        \mintobjdump[firstline=2,highlightlines=14]{../src/backtrace/camlBacktrace\_\_definitely\_raise\_347.objdump}
      }
    \end{column}
  \end{columns}
\end{frame}

\begin{frame}{Backtraces}{Introducing \funcname{caml\_raise\_exn}}
  \begin{columns}[c]
    \begin{column}{0.5\textwidth}
      \minipage[c][0.66\textheight][s]{\columnwidth}
      \onslide*<1>{
        \begin{itemize}
          \item Raising the exception is performed using \funcname{caml\_raise\_exn} which is
            \begin{itemize}
              \item An assembly function provided by the runtime
              \item Architecture specific
            \end{itemize}
          \item Let's see what it does differently from \funcname{raise\_notrace}\dots
          \item We are about to raise \typename{ExnC} with \funcname{caml\_raise\_exn} from \funcname{definitely\_raise}
        \end{itemize}
      }
      \onslide*<2>{
        \mintobjdump[firstline=2,highlightlines=14]{../src/backtrace/camlBacktrace\_\_definitely\_raise\_347.objdump}
      }
      \vfill
      \mintocaml[firstline=9,lastline=13,highlightlines=12]{../src/backtrace/backtrace.ml}
      \endminipage
    \end{column}
    \begin{column}{0.5\textwidth}
      \centering
      \providecommand\step{1}\input{diagrams/backtrace/backtrace.tex}
    \end{column}
  \end{columns}
\end{frame}

\begin{frame}{Backtraces}{But before that, we need more fields from \typename{caml\_domain\_state}}
  \begin{columns}
    \begin{column}{0.5\textwidth}
      \begin{itemize}
        \item Remember \typename{caml\_domain\_state} the per-domain runtime state?
        \item Some other members are of interest to us now\dots
        \item \localname{backtrace\_active} is used to know if the runtime should collect backtraces
        \item \localname{backtrace\_active} can be set by
          \begin{itemize}
            \item The \funcarg{b} flag in the \funcname{OCAMLRUNPARAM} environment variable
            \item \mintinline{ocaml}{Printexc.record_backtrace : bool -> unit} \footnotemark
          \end{itemize}
      \end{itemize}
    \end{column}
    \begin{column}{0.5\textwidth}
      \begin{listing}
        \mintc[highlightlines={9-11}]{src/ocaml/domain\_state\_backtrace.h}
        \caption{runtime/caml/domain\_state.h}
      \end{listing}
    \end{column}
  \end{columns}
  \footnotetext[1]{https://v2.ocaml.org/api/Printexc.html}
\end{frame}

\newcommand\listCamlRaiseExn[1][]{
  \listasm[firstline=702,lastline=725,#1]{../ocaml/runtime/amd64.S}{runtime/amd64.S:702}}

\begin{frame}{Backtraces}{Walk through \funcname{caml\_raise\_exn}}
  \begin{columns}[T]
    \begin{column}{0.5\textwidth}
      \begin{itemize}
        \item<1-> First checks whether exceptions backtrace should be recorded
        \item<1-> By checking the \funcarg{domain\_state->backtrace\_active} value
        \item<2-> If not, acts as \funcname{raise\_notrace} with \funcname{RESTORE\_EXN\_HANDLER\_OCAML}
          \begin{itemize}
            \item Set \regname{RSP} to \localname{domain\_state->exn\_handler} value
            \item Restore \localname{domain\_state->exn\_handler} from trap
            \item Jump with \funcname{ret} (same effect as using \funcname{pop} and \funcname{jmp})
          \end{itemize}
        \item<3-> Otherwise, switches to the C stack to be able to execute C code
        \item<4-> Calls \funcname{caml\_stash\_backtrace}, a C function provided by the runtime\ignorespaces
          \onslide<6->{, with
            \begin{itemize}
              \item \funcarg{exn}: exception value
              \item \funcarg{pc}: code address of the raising point
              \item \funcarg{sp}: stack address of the raising function
              \item \funcarg{trapsp}: stack address of the next handler
            \end{itemize}
          }
      \end{itemize}
      \bigskip
      \mintocaml[firstline=9,lastline=13,highlightlines=12]{../src/backtrace/backtrace.ml}
    \end{column}
    \begin{column}{0.5\textwidth}
      \raggedleft
      \onslide*<1,3-4>{
        \mintc[firstline=190,lastline=192]{../ocaml/runtime/amd64.S}
        \mintc[firstline=234,lastline=237]{../ocaml/runtime/amd64.S}
      }
      \onslide*<2>{
        \mintc[firstline=190,lastline=192,highlightlines={191-192}]{../ocaml/runtime/amd64.S}
        \mintc[firstline=234,lastline=237,highlightlines={235-237}]{../ocaml/runtime/amd64.S}
      }
      \onslide*<1>{\listCamlRaiseExn[highlightlines=705]{}}%
      \onslide*<2>{\listCamlRaiseExn[highlightlines={707-708}]{}}%
      \onslide*<3>{\listCamlRaiseExn[highlightlines={712-714}]{}}%
      \onslide*<4>{\listCamlRaiseExn[highlightlines={715-720}]{}}%
      \onslide*<5>{\providecommand\step{1}\input{diagrams/backtrace/backtrace.tex}}%
      \onslide*<6>{\providecommand\step{2}\input{diagrams/backtrace/backtrace.tex}}%
    \end{column}
  \end{columns}
\end{frame}

\newcommand\listStashBacktrace[1][]{
  \listc[firstline=91,lastline=120,#1]{../ocaml/runtime/backtrace\_nat.c}{runtime/backtrace\_nat.c:91}}

\begin{frame}{Backtraces}{Introducing \funcname{caml\_stash\_backtrace}}
  \begin{columns}[c]
    \begin{column}{0.5\textwidth}
      \minipage[c][0.66\textheight][s]{\columnwidth}
      \begin{itemize}
        \item<1-> A C function provided by the runtime (specific to native code)
        \item<2-> It scans the stack, between the stack pointer at the raising point up to the next trap, in order to collect the names of the detected function frames
          \onslide*<2>{
            \begin{itemize}
              \item And more information, like line number, column number...
            \end{itemize}
          }
        \item<3-> It uses the same technique and information as the Garbage Collector (GC) uses to scan the values live on the stack
          \onslide*<4>{
            \begin{itemize}
              \item \typename{frame\_descr} are at the core of stack unwinding
            \end{itemize}
          }
      \end{itemize}
      \vfill
      \mintocaml[firstline=9,lastline=13,highlightlines=12]{../src/backtrace/backtrace.ml}
      \endminipage
    \end{column}
    \begin{column}{0.5\textwidth}
      \onslide*<1>{\listStashBacktrace[highlightlines=91]{}}
      \onslide*<2>{\listStashBacktrace[highlightlines={91,117-118}]{}}
      \onslide*<3->{\listStashBacktrace[highlightlines={94,106,109-110}]{}}
    \end{column}
  \end{columns}
\end{frame}

\newcommand\listFrameDescr[1][]{
  \listocaml[firstline=111,lastline=124,#1]{../ocaml/asmcomp/emitaux.ml}{asmcomp/emitaux.ml:111}}
\newcommand\listDebugInfo[1][]{
  \listocaml[firstline=38,lastline=49,#1]{../ocaml/lambda/debuginfo.mli}{lambda/debuginfo.mli:38}}

\begin{frame}{Backtraces}{\typename{frame\_descr} and \typename{frame\_debuginfo} in a nutshell}
  \begin{columns}[c]
    \begin{column}{0.5\textwidth}
      \begin{itemize}
        \item<1-> For every return address in an OCaml program, there is a associated \typename{frame\_descr}
        \item<2-> A \typename{frame\_descr} specifies
        \begin{itemize}
          \item<2-> The size of the function stack frame
          \item<3-> Where live values are on the stack (so that the GC can mark them as live and prevent from being swept)
        \end{itemize}
        \item<4-> When compiled with debug information, \typename{frame\_descr} will be linked to a \typename{frame\_debuginfo}
        \begin{itemize}
          \item<5-> Contains information about the position in the source code (file name, line and column number)
        \end{itemize}
      \end{itemize}
    \end{column}
    \begin{column}{0.5\textwidth}
        \onslide*<1>{\listFrameDescr[highlightlines={118-122}]{}}
        \onslide*<2>{\listFrameDescr[highlightlines={120}]{}}
        \onslide*<3>{\listFrameDescr[highlightlines={121}]{}}
        \onslide*<4->{\listFrameDescr[highlightlines={122}]{}}
        \medskip
        \onslide*<1-3>{\listDebugInfo{}}
        \onslide*<4>{\listDebugInfo[highlightlines={38,49}]{}}
        \onslide*<5>{\listDebugInfo[highlightlines={39-46}]{}}
    \end{column}
  \end{columns}
\end{frame}

\begin{frame}{Backtraces}{Scanning the stack with \typename{frame\_descr}}
  \begin{columns}[c]
    \begin{column}{0.5\textwidth}
      \begin{itemize}
        \item Given the return address of the code that raises the exception by calling \funcname{caml\_raise\_exn} (\funcarg{pc} argument of \funcname{caml\_stash\_backtrace})
        \item And the position of the stack pointer (\funcarg{sp} argument of \funcname{caml\_stash\_backtrace})
        \item The frame size given by \typename{frame\_descr}.\funcarg{frame\_size}, allows to find the end of the previous frame
        \item And the return address of the caller of this function
        \bigskip
        \onslide<3->{
        \item Note that traps (here the reddish \funcname{ExnA}, \funcname{ExnB} and \funcname{ExnC} blocks) belong to the frame that installed them, and so the frame size changes when traps are removed
        }
      \end{itemize}
    \end{column}
    \begin{column}{0.5\textwidth}
      \raggedleft
      \onslide*<1>{\providecommand\step{2}\input{diagrams/backtrace/backtrace.tex}}%
      \onslide*<2>{\providecommand\step{1}\input{diagrams/backtrace/scanstack.tex}}%
      \onslide*<3>{\providecommand\step{2}\input{diagrams/backtrace/scanstack.tex}}%
      \onslide*<4>{\providecommand\step{3}\input{diagrams/backtrace/scanstack.tex}}%
      \onslide*<5>{\providecommand\step{4}\input{diagrams/backtrace/scanstack.tex}}%
      \onslide*<6>{\providecommand\step{5}\input{diagrams/backtrace/scanstack.tex}}%
    \end{column}
  \end{columns}
\end{frame}

% Duplicate of previous-previous slide
\begin{frame}{Backtraces}{Continuing on \funcname{caml\_stash\_backtrace}}
  \begin{columns}[c]
    \begin{column}{0.5\textwidth}
      \minipage[c][0.75\textheight][s]{\columnwidth}
      \begin{itemize}
          \color{gray}
        \item A C function provided by the runtime (specific to native code)
        \item It scans the stack, between the stack pointer at the raising point up to the next trap, in order to collect the names of the detected function frames
        \item It uses the same technique and information as the Garbage Collector (GC) uses to scan the values live on the stack
      \end{itemize}
      \begin{itemize}
        \item For each function frame detected, store a pointer to the \typename{frame\_descr} in the \localname{domain\_state->backtrace\_buffer} array, effectively constructing the backtrace
      \end{itemize}
      \onslide<2->{
        \begin{itemize}
            \smallskip
          \item Constructed backtrace:
            \funcname{definitely\_raise},
            \onslide<3->{\funcname{probably\_raise}}
        \end{itemize}
      }
      \vfill
      \mintocaml[firstline=9,lastline=13,highlightlines=12]{../src/backtrace/backtrace.ml}
      \endminipage
    \end{column}
    \begin{column}{0.5\textwidth}
      \raggedleft
      \onslide*<1>{\listStashBacktrace[highlightlines={114-115}]{}}
      \onslide*<2>{\providecommand\step{3}\input{diagrams/backtrace/backtrace.tex}}%
      \onslide*<3>{\providecommand\step{4}\input{diagrams/backtrace/backtrace.tex}}%
    \end{column}
  \end{columns}
\end{frame}

\begin{frame}{Backtraces}{Back to \funcname{caml\_raise\_exn}}
  \begin{columns}[c]
    \begin{column}{0.5\textwidth}
      \minipage[c][0.66\textheight][s]{\columnwidth}
      \begin{itemize}\color{gray}
        \item Checks wheteher exceptions backtrace should be recorded, by checking the \funcarg{domain\_state->backtrace\_active} value
        \item Switches to the C stack to be able to execute C code
        \item Calls \funcname{caml\_stash\_backtrace}, to collect the backtrace up to the next trap
      \end{itemize}
      \onslide*<2->{
        \begin{itemize}
          \item<2-> Executes the exception handler in \funcname{probably\_raise} with \funcname{RESTORE\_EXN\_HANDLER\_OCAML}
            \begin{itemize}
              \item Set \regname{RSP} to \localname{domain\_state->exn\_handler} value
              \item Restore \localname{domain\_state->exn\_handler} from trap
              \item Jump to the handler code with \funcname{ret}
            \end{itemize}
        \end{itemize}
      }
      \vfill
      \onslide*<1>{\mintocaml[firstline=9,lastline=13,highlightlines=12]{../src/backtrace/backtrace.ml}}
      \onslide*<2->{\mintocaml[firstline=15,lastline=21,highlightlines={19-21}]{../src/backtrace/backtrace.ml}}
      \endminipage
    \end{column}
    \begin{column}{0.5\textwidth}
      \centering
      \onslide*<1>{
        \mintc[firstline=190,lastline=192]{../ocaml/runtime/amd64.S}
        \mintc[firstline=234,lastline=237]{../ocaml/runtime/amd64.S}
      }
      \onslide*<2>{
        \mintc[firstline=190,lastline=192,highlightlines={191-192}]{../ocaml/runtime/amd64.S}
        \mintc[firstline=234,lastline=237,highlightlines={235-237}]{../ocaml/runtime/amd64.S}
      }
      \onslide*<1>{\listCamlRaiseExn[highlightlines=720]{}}
      \onslide*<2>{\listCamlRaiseExn[highlightlines=722]{}}
      \onslide*<3->{\providecommand\step{5}\input{diagrams/backtrace/backtrace.tex}}
    \end{column}
  \end{columns}
\end{frame}

\begin{frame}{Backtraces}{\funcname{caml\_probably\_raise}'s handler doesn't catch \typename{ExnC}}
  \begin{columns}[c]
    \begin{column}{0.45\textwidth}
      \minipage[c][0.75\textheight][s]{\columnwidth}
      \begin{itemize}
        \item<1-> \funcname{match \dots with}\footnotemark pattern matching can also match exceptions
        \item<1-> The CMM of \funcname{probably\_raise} shows that this \funcname{match \dots with} is implemented with a pattern matching and a \funcname{try \dots with}
        \item<1-> But this one only matches on \typename{ExnA} exception
        \item<2-> Since the pattern matching fails to catch the exception, \funcname{reraise} is called with our current \typename{ExnC} exception
        \item<3-> Disassembly shows that \funcname{reraise} is actually a call to \funcname{caml\_reraise\_exn}, which is also an assembly function provided by the runtime
      \end{itemize}
      \vfill
      \mintocaml[firstline=15,lastline=21,highlightlines={18-21}]{../src/backtrace/backtrace.ml}
      \endminipage
    \end{column}
    \begin{column}{0.55\textwidth}
      \centering
      \onslide*<1>{\mintcmm[highlightlines={10-18}]{../src/backtrace/camlBacktrace\_\_probably\_raise\_351.cmm}}
      \onslide*<2>{\listcmm[highlightlines=18]{../src/backtrace/camlBacktrace\_\_probably\_raise_351.cmm}{CMM of probably\_raise}}
      \onslide*<3>{\mintobjdump[firstline=5,lastline=35,highlightlines=31]{../src/backtrace/camlBacktrace\_\_probably\_raise_351.objdump}}
    \end{column}
  \end{columns}
  \only<1>{\footnotetext[1]{https://blog.janestreet.com/pattern-matching-and-exception-handling-unite/}}
\end{frame}

\newcommand\listCamlReraiseExn[1][]{
  \listasm[firstline=727,lastline=734,#1]{../ocaml/runtime/amd64.S}{runtime/amd64.S:727}}

\begin{frame}{Backtraces}{Introducing \funcname{caml\_reraise\_exn}}
  \begin{columns}[c]
    \begin{column}{0.5\textwidth}
      \minipage[c][0.66\textheight][s]{\columnwidth}
      \begin{itemize}
        \item<1-> The exception have to be reraised to the next handler
        \item<2-> First, checks if the backtrace should be collected with \funcarg{domain\_state->backtrace\_active}
        \only<3>{
        \begin{itemize}
            \item If not, behaves like a \funcname{raise\_notrace} with \funcname{RESTORE\_EXN\_HANDLER\_OCAML}
        \end{itemize}
        }
        \item<4-> Jumps into \funcname{caml\_raise\_exn}
        \item<5-> Calls \funcname{caml\_stash\_backtrace} again, to scan the next part of the stack: from the trap just pop-ed to the next trap
      \end{itemize}
      \vfill
      \mintocaml[firstline=15,lastline=21,highlightlines={18-21}]{../src/backtrace/backtrace.ml}
      \endminipage
    \end{column}
    \begin{column}{0.5\textwidth}
      \raggedleft
      \onslide*<1>{\listCamlReraiseExn{}}
      \onslide*<2>{\listCamlReraiseExn[highlightlines=729]{}}
      \onslide*<3>{
        \mintc[firstline=190,lastline=192,highlightlines={191-192}]{../ocaml/runtime/amd64.S}
        \mintc[firstline=234,lastline=237,highlightlines={235-237}]{../ocaml/runtime/amd64.S}
        \medskip
        \listCamlReraiseExn[highlightlines=731]{}
      }
      \onslide*<4-5>{
        % TODO refactor with \listCamlReraiseExn
        \mintasm[firstline=727,lastline=734,highlightlines=730]{../ocaml/runtime/amd64.S}
        \medskip
      }
      \onslide*<4>{\listCamlRaiseExn[highlightlines=711]{}}
      \onslide*<5>{\listCamlRaiseExn[highlightlines=720]{}}
    \end{column}
  \end{columns}
\end{frame}

\begin{frame}{Backtraces}{Backtrace collection continues}
  \begin{columns}[c]
    \begin{column}{0.55\textwidth}
      \minipage[c][0.75\textheight][s]{\columnwidth}
      \begin{itemize}
        \item<1-> \funcname{caml\_stash\_backtrace} scans the older part of the stack, up to the next trap
        \item<6-> Controls jumps to the nested exception handler in \funcname{maybe\_raise}, which matches only on \typename{ExnB}
        \item<7-> \funcname{caml\_reraise\_exn} is called again, backtrace collection resumes with \funcname{caml\_stash\_backtrace}
      \end{itemize}
      \medskip
      \begin{itemize}
        \item<2-> Constructed backtrace:
        \begin{itemize}
          \item <2-> \funcname{definitely\_raise}
          \item <2-> \funcname{probably\_raise}
          \item <3-> \funcname{probably\_raise} (re-raise)
          \item <4-> \funcname{maybe\_raise}
          \item <5-> \funcname{main}
          \item <8-> \funcname{main} (re-raise)
        \end{itemize}
      \end{itemize}
      \vfill
      \onslide*<1-5>{\mintocaml[firstline=15,lastline=21]{../src/backtrace/backtrace.ml}}
      \onslide*<6>{\mintocaml[firstline=28,lastline=34,highlightlines=31]{../src/backtrace/backtrace.ml}}
      \onslide*<7->{\mintocaml[firstline=28,lastline=34,highlightlines=33]{../src/backtrace/backtrace.ml}}
      \endminipage
    \end{column}
    \begin{column}{0.45\textwidth}
      \raggedleft
      \onslide*<1>{\providecommand\step{6}\input{diagrams/backtrace/backtrace.tex}}%
      \onslide*<2>{\providecommand\step{6}\input{diagrams/backtrace/backtrace.tex}}%
      \onslide*<3>{\providecommand\step{7}\input{diagrams/backtrace/backtrace.tex}}%
      \onslide*<4>{\providecommand\step{8}\input{diagrams/backtrace/backtrace.tex}}%
      \onslide*<5>{\providecommand\step{9}\input{diagrams/backtrace/backtrace.tex}}%
      \onslide*<6>{\providecommand\step{10}\input{diagrams/backtrace/backtrace.tex}}%
      \onslide*<7>{\providecommand\step{11}\input{diagrams/backtrace/backtrace.tex}}%
      \onslide*<8>{\providecommand\step{12}\input{diagrams/backtrace/backtrace.tex}}%
      \onslide*<9>{\providecommand\step{13}\input{diagrams/backtrace/backtrace.tex}}%
    \end{column}
  \end{columns}
\end{frame}

\begin{frame}{Backtraces}{Printing the backtrace}
  \begin{columns}[c]
    \begin{column}{0.45\textwidth}
      \minipage[c][0.66\textheight][s]{\columnwidth}
      \begin{itemize}
        \item<1-> The exception backtrace can now be printed to \funcarg{stdout} with \funcname{Printexc.print_backtrace}
        \item<2-> First the exception backtrace is retrieved into an OCaml array with \funcname{get_raw_backtrace }
        \item<3-> Which is implemented as the \funcname{caml_get_exception_raw_backtrace} C function in the runtime
        \item<4-> And finally printed with \funcname{print_exception_backtrace}
      \end{itemize}
      \vfill
      \mintocaml[firstline=28,lastline=34,highlightlines=33]{../src/backtrace/backtrace.ml}
      \endminipage
    \end{column}
    \begin{column}{0.55\textwidth}
      \onslide*<1,2>{
        \mintocaml[firstline=100,lastline=101]{../ocaml/stdlib/printexc.ml}
      }
      \only<1>{
        \listocaml[firstline=170,lastline=172]{../ocaml/stdlib/printexc.ml}{stdlib/printexc.ml:170}
      }
      \only<2>{
        \listocaml[firstline=170,lastline=172,highlightlines=172]{../ocaml/stdlib/printexc.ml}{stdlib/printexc.ml:170}
      }
      \only<3>{
        \mintc[firstline=154,lastline=156]{../ocaml/runtime/backtrace.c}
        \listc[firstline=171,lastline=192,highlightlines={184-187}]{../ocaml/runtime/backtrace.c}{runtime/backtrace.c:154}
      }
      \only<4>{
        \listocaml[firstline=155,lastline=165]{../ocaml/stdlib/printexc.ml}{stdlib/printexc.ml:55}
      }
    \end{column}
  \end{columns}
\end{frame}


\subsection{The multiple kinds of raise}
\frameSubsection{}{}

\begin{frame}{Backtraces}{When backtraces are not needed}
  \begin{columns}[c]
    \begin{column}{0.45\textwidth}
      \minipage[c][0.66\textheight][s]{\columnwidth}
      \begin{itemize}
        \item<1-> What if the backtrace for an exception is never needed?
        \item<2-> It's possible to raise the exception explicitly with \funcname{raise\_notrace} to make raising as cheap as possible
        \item<3-> An example of use in the \funcname{dynlink} library of the OCaml compiler
      \end{itemize}
      \vfill
      \onslide<3->{
      \listocaml[firstline=205,lastline=209,highlightlines=207]{../ocaml/otherlibs/dynlink/dynlink\_compilerlibs/misc.ml}{otherlibs/dynlink/dynlink\_compilerlibs/misc.ml:205}
      }
      \endminipage
    \end{column}
    \begin{column}{0.55\textwidth}
    \only<4>{
      \mintcmm[highlightlines=3]{../src/notrace/camlNotrace\_\_fun_347.cmm}
      \listcmm{../src/notrace/camlNotrace\_\_all\_somes\_327.cmm}{all\_somes}
    }
    \onslide*<5->{
      \listobjdump[highlightlines={6-9}]{../src/notrace/camlNotrace\_\_fun_347.objdump}{anonymous function from \funcname{all\_somes}}
    }
    \end{column}
  \end{columns}
\end{frame}

\begin{frame}{Backtraces}{Summary table of the multiple kinds of raise}
  \begin{itemize}
    \item When debugging information is enabled and if the backtrace of an exception isn't needed, it's possible to raise with \funcname{raise\_notrace} for performance
    \item When debugging information is \emph{not} enabled, the compiler optimises \funcname{raise} calls to inline assembly (same effect as using \funcname{raise\_notrace})
  \end{itemize}
  \bigskip
  \centering
  \begin{tabular}{| l | c | c | c | c |}
    \hline
    \parbox{10em}{Debug information \\ (\funcarg{-g} flag)}  & \multicolumn{2}{|c|}{Disabled}                                     & \multicolumn{2}{|c|}{Enabled} \\
    \hline
    Raise kind                                               & \funcname{raise} & \funcname{raise\_notrace}                       & \funcname{raise} & \funcname{raise\_notrace} \\
    \hline
    Compilation                                              & \parbox{8em}{inline assembly\\ (optimization)} & inline assembly   & \funcname{caml\_raise\_exn} & inline assembly \\
    \hline
  \end{tabular}
\end{frame}


%
% Takeaway #5
%
\subsection*{Takeaway \#5}
\frameSubsectionTakeaway{}
\againframe<5>{takeaway}
}%
      \onslide*<2>{\providecommand\step{6}%
% Backtraces
%
\subsection{One advantage of raising with debugging information}
\frameSubsection{}{}

\begin{frame}{Backtraces}{Debugging information \& source code}
  \begin{columns}[c]
    \begin{column}{0.5\textwidth}
      \begin{itemize}
        \item Raising an exception is fun and games, but how do we know where the exception originated? $\rightarrow$ With the backtrace associated with the exception!
        \item In order to get a backtrace from the point where an exception was raised, it's necessary to compile with debugging information enabled (\funcarg{-g} flag for the compiler)
        \item Same example as in the `Nested handlers' section
      \end{itemize}
    \end{column}
    \begin{column}{0.5\textwidth}
      \listocaml[firstline=9,lastline=34]{../src/backtrace/backtrace.ml}{backtrace/backtrace.ml}
    \end{column}
  \end{columns}
\end{frame}

\begin{frame}{Backtraces}{CMM and disassembly of \funcname{definitely\_raise}}
  \begin{columns}[c]
    \begin{column}{0.5\textwidth}
      \minipage[c][0.66\textheight][s]{\columnwidth}
      \begin{itemize}
        \item<1-> An effect of compiling with debugging information (\funcarg{-g}): the CMM no longer uses \funcname{raise\_notrace} to raise the exception but \funcname{raise} instead
        \item<2-> Disassembly shows that CMM \funcname{raise} is actually a call to \funcname{caml\_raise\_exn}, a function in assembler provided by the runtime
      \end{itemize}
      \vfill
      \mintocaml[firstline=9,lastline=13,highlightlines=12]{../src/backtrace/backtrace.ml}
      \endminipage
    \end{column}
    \begin{column}{0.5\textwidth}
      \onslide*<1>{
        \mintcmm[highlightlines=5]{../src/backtrace/camlBacktrace\_\_definitely\_raise\_347.cmm}
      }
      \onslide*<2>{
        \mintobjdump[firstline=2,highlightlines=14]{../src/backtrace/camlBacktrace\_\_definitely\_raise\_347.objdump}
      }
    \end{column}
  \end{columns}
\end{frame}

\begin{frame}{Backtraces}{Introducing \funcname{caml\_raise\_exn}}
  \begin{columns}[c]
    \begin{column}{0.5\textwidth}
      \minipage[c][0.66\textheight][s]{\columnwidth}
      \onslide*<1>{
        \begin{itemize}
          \item Raising the exception is performed using \funcname{caml\_raise\_exn} which is
            \begin{itemize}
              \item An assembly function provided by the runtime
              \item Architecture specific
            \end{itemize}
          \item Let's see what it does differently from \funcname{raise\_notrace}\dots
          \item We are about to raise \typename{ExnC} with \funcname{caml\_raise\_exn} from \funcname{definitely\_raise}
        \end{itemize}
      }
      \onslide*<2>{
        \mintobjdump[firstline=2,highlightlines=14]{../src/backtrace/camlBacktrace\_\_definitely\_raise\_347.objdump}
      }
      \vfill
      \mintocaml[firstline=9,lastline=13,highlightlines=12]{../src/backtrace/backtrace.ml}
      \endminipage
    \end{column}
    \begin{column}{0.5\textwidth}
      \centering
      \providecommand\step{1}\input{diagrams/backtrace/backtrace.tex}
    \end{column}
  \end{columns}
\end{frame}

\begin{frame}{Backtraces}{But before that, we need more fields from \typename{caml\_domain\_state}}
  \begin{columns}
    \begin{column}{0.5\textwidth}
      \begin{itemize}
        \item Remember \typename{caml\_domain\_state} the per-domain runtime state?
        \item Some other members are of interest to us now\dots
        \item \localname{backtrace\_active} is used to know if the runtime should collect backtraces
        \item \localname{backtrace\_active} can be set by
          \begin{itemize}
            \item The \funcarg{b} flag in the \funcname{OCAMLRUNPARAM} environment variable
            \item \mintinline{ocaml}{Printexc.record_backtrace : bool -> unit} \footnotemark
          \end{itemize}
      \end{itemize}
    \end{column}
    \begin{column}{0.5\textwidth}
      \begin{listing}
        \mintc[highlightlines={9-11}]{src/ocaml/domain\_state\_backtrace.h}
        \caption{runtime/caml/domain\_state.h}
      \end{listing}
    \end{column}
  \end{columns}
  \footnotetext[1]{https://v2.ocaml.org/api/Printexc.html}
\end{frame}

\newcommand\listCamlRaiseExn[1][]{
  \listasm[firstline=702,lastline=725,#1]{../ocaml/runtime/amd64.S}{runtime/amd64.S:702}}

\begin{frame}{Backtraces}{Walk through \funcname{caml\_raise\_exn}}
  \begin{columns}[T]
    \begin{column}{0.5\textwidth}
      \begin{itemize}
        \item<1-> First checks whether exceptions backtrace should be recorded
        \item<1-> By checking the \funcarg{domain\_state->backtrace\_active} value
        \item<2-> If not, acts as \funcname{raise\_notrace} with \funcname{RESTORE\_EXN\_HANDLER\_OCAML}
          \begin{itemize}
            \item Set \regname{RSP} to \localname{domain\_state->exn\_handler} value
            \item Restore \localname{domain\_state->exn\_handler} from trap
            \item Jump with \funcname{ret} (same effect as using \funcname{pop} and \funcname{jmp})
          \end{itemize}
        \item<3-> Otherwise, switches to the C stack to be able to execute C code
        \item<4-> Calls \funcname{caml\_stash\_backtrace}, a C function provided by the runtime\ignorespaces
          \onslide<6->{, with
            \begin{itemize}
              \item \funcarg{exn}: exception value
              \item \funcarg{pc}: code address of the raising point
              \item \funcarg{sp}: stack address of the raising function
              \item \funcarg{trapsp}: stack address of the next handler
            \end{itemize}
          }
      \end{itemize}
      \bigskip
      \mintocaml[firstline=9,lastline=13,highlightlines=12]{../src/backtrace/backtrace.ml}
    \end{column}
    \begin{column}{0.5\textwidth}
      \raggedleft
      \onslide*<1,3-4>{
        \mintc[firstline=190,lastline=192]{../ocaml/runtime/amd64.S}
        \mintc[firstline=234,lastline=237]{../ocaml/runtime/amd64.S}
      }
      \onslide*<2>{
        \mintc[firstline=190,lastline=192,highlightlines={191-192}]{../ocaml/runtime/amd64.S}
        \mintc[firstline=234,lastline=237,highlightlines={235-237}]{../ocaml/runtime/amd64.S}
      }
      \onslide*<1>{\listCamlRaiseExn[highlightlines=705]{}}%
      \onslide*<2>{\listCamlRaiseExn[highlightlines={707-708}]{}}%
      \onslide*<3>{\listCamlRaiseExn[highlightlines={712-714}]{}}%
      \onslide*<4>{\listCamlRaiseExn[highlightlines={715-720}]{}}%
      \onslide*<5>{\providecommand\step{1}\input{diagrams/backtrace/backtrace.tex}}%
      \onslide*<6>{\providecommand\step{2}\input{diagrams/backtrace/backtrace.tex}}%
    \end{column}
  \end{columns}
\end{frame}

\newcommand\listStashBacktrace[1][]{
  \listc[firstline=91,lastline=120,#1]{../ocaml/runtime/backtrace\_nat.c}{runtime/backtrace\_nat.c:91}}

\begin{frame}{Backtraces}{Introducing \funcname{caml\_stash\_backtrace}}
  \begin{columns}[c]
    \begin{column}{0.5\textwidth}
      \minipage[c][0.66\textheight][s]{\columnwidth}
      \begin{itemize}
        \item<1-> A C function provided by the runtime (specific to native code)
        \item<2-> It scans the stack, between the stack pointer at the raising point up to the next trap, in order to collect the names of the detected function frames
          \onslide*<2>{
            \begin{itemize}
              \item And more information, like line number, column number...
            \end{itemize}
          }
        \item<3-> It uses the same technique and information as the Garbage Collector (GC) uses to scan the values live on the stack
          \onslide*<4>{
            \begin{itemize}
              \item \typename{frame\_descr} are at the core of stack unwinding
            \end{itemize}
          }
      \end{itemize}
      \vfill
      \mintocaml[firstline=9,lastline=13,highlightlines=12]{../src/backtrace/backtrace.ml}
      \endminipage
    \end{column}
    \begin{column}{0.5\textwidth}
      \onslide*<1>{\listStashBacktrace[highlightlines=91]{}}
      \onslide*<2>{\listStashBacktrace[highlightlines={91,117-118}]{}}
      \onslide*<3->{\listStashBacktrace[highlightlines={94,106,109-110}]{}}
    \end{column}
  \end{columns}
\end{frame}

\newcommand\listFrameDescr[1][]{
  \listocaml[firstline=111,lastline=124,#1]{../ocaml/asmcomp/emitaux.ml}{asmcomp/emitaux.ml:111}}
\newcommand\listDebugInfo[1][]{
  \listocaml[firstline=38,lastline=49,#1]{../ocaml/lambda/debuginfo.mli}{lambda/debuginfo.mli:38}}

\begin{frame}{Backtraces}{\typename{frame\_descr} and \typename{frame\_debuginfo} in a nutshell}
  \begin{columns}[c]
    \begin{column}{0.5\textwidth}
      \begin{itemize}
        \item<1-> For every return address in an OCaml program, there is a associated \typename{frame\_descr}
        \item<2-> A \typename{frame\_descr} specifies
        \begin{itemize}
          \item<2-> The size of the function stack frame
          \item<3-> Where live values are on the stack (so that the GC can mark them as live and prevent from being swept)
        \end{itemize}
        \item<4-> When compiled with debug information, \typename{frame\_descr} will be linked to a \typename{frame\_debuginfo}
        \begin{itemize}
          \item<5-> Contains information about the position in the source code (file name, line and column number)
        \end{itemize}
      \end{itemize}
    \end{column}
    \begin{column}{0.5\textwidth}
        \onslide*<1>{\listFrameDescr[highlightlines={118-122}]{}}
        \onslide*<2>{\listFrameDescr[highlightlines={120}]{}}
        \onslide*<3>{\listFrameDescr[highlightlines={121}]{}}
        \onslide*<4->{\listFrameDescr[highlightlines={122}]{}}
        \medskip
        \onslide*<1-3>{\listDebugInfo{}}
        \onslide*<4>{\listDebugInfo[highlightlines={38,49}]{}}
        \onslide*<5>{\listDebugInfo[highlightlines={39-46}]{}}
    \end{column}
  \end{columns}
\end{frame}

\begin{frame}{Backtraces}{Scanning the stack with \typename{frame\_descr}}
  \begin{columns}[c]
    \begin{column}{0.5\textwidth}
      \begin{itemize}
        \item Given the return address of the code that raises the exception by calling \funcname{caml\_raise\_exn} (\funcarg{pc} argument of \funcname{caml\_stash\_backtrace})
        \item And the position of the stack pointer (\funcarg{sp} argument of \funcname{caml\_stash\_backtrace})
        \item The frame size given by \typename{frame\_descr}.\funcarg{frame\_size}, allows to find the end of the previous frame
        \item And the return address of the caller of this function
        \bigskip
        \onslide<3->{
        \item Note that traps (here the reddish \funcname{ExnA}, \funcname{ExnB} and \funcname{ExnC} blocks) belong to the frame that installed them, and so the frame size changes when traps are removed
        }
      \end{itemize}
    \end{column}
    \begin{column}{0.5\textwidth}
      \raggedleft
      \onslide*<1>{\providecommand\step{2}\input{diagrams/backtrace/backtrace.tex}}%
      \onslide*<2>{\providecommand\step{1}\input{diagrams/backtrace/scanstack.tex}}%
      \onslide*<3>{\providecommand\step{2}\input{diagrams/backtrace/scanstack.tex}}%
      \onslide*<4>{\providecommand\step{3}\input{diagrams/backtrace/scanstack.tex}}%
      \onslide*<5>{\providecommand\step{4}\input{diagrams/backtrace/scanstack.tex}}%
      \onslide*<6>{\providecommand\step{5}\input{diagrams/backtrace/scanstack.tex}}%
    \end{column}
  \end{columns}
\end{frame}

% Duplicate of previous-previous slide
\begin{frame}{Backtraces}{Continuing on \funcname{caml\_stash\_backtrace}}
  \begin{columns}[c]
    \begin{column}{0.5\textwidth}
      \minipage[c][0.75\textheight][s]{\columnwidth}
      \begin{itemize}
          \color{gray}
        \item A C function provided by the runtime (specific to native code)
        \item It scans the stack, between the stack pointer at the raising point up to the next trap, in order to collect the names of the detected function frames
        \item It uses the same technique and information as the Garbage Collector (GC) uses to scan the values live on the stack
      \end{itemize}
      \begin{itemize}
        \item For each function frame detected, store a pointer to the \typename{frame\_descr} in the \localname{domain\_state->backtrace\_buffer} array, effectively constructing the backtrace
      \end{itemize}
      \onslide<2->{
        \begin{itemize}
            \smallskip
          \item Constructed backtrace:
            \funcname{definitely\_raise},
            \onslide<3->{\funcname{probably\_raise}}
        \end{itemize}
      }
      \vfill
      \mintocaml[firstline=9,lastline=13,highlightlines=12]{../src/backtrace/backtrace.ml}
      \endminipage
    \end{column}
    \begin{column}{0.5\textwidth}
      \raggedleft
      \onslide*<1>{\listStashBacktrace[highlightlines={114-115}]{}}
      \onslide*<2>{\providecommand\step{3}\input{diagrams/backtrace/backtrace.tex}}%
      \onslide*<3>{\providecommand\step{4}\input{diagrams/backtrace/backtrace.tex}}%
    \end{column}
  \end{columns}
\end{frame}

\begin{frame}{Backtraces}{Back to \funcname{caml\_raise\_exn}}
  \begin{columns}[c]
    \begin{column}{0.5\textwidth}
      \minipage[c][0.66\textheight][s]{\columnwidth}
      \begin{itemize}\color{gray}
        \item Checks wheteher exceptions backtrace should be recorded, by checking the \funcarg{domain\_state->backtrace\_active} value
        \item Switches to the C stack to be able to execute C code
        \item Calls \funcname{caml\_stash\_backtrace}, to collect the backtrace up to the next trap
      \end{itemize}
      \onslide*<2->{
        \begin{itemize}
          \item<2-> Executes the exception handler in \funcname{probably\_raise} with \funcname{RESTORE\_EXN\_HANDLER\_OCAML}
            \begin{itemize}
              \item Set \regname{RSP} to \localname{domain\_state->exn\_handler} value
              \item Restore \localname{domain\_state->exn\_handler} from trap
              \item Jump to the handler code with \funcname{ret}
            \end{itemize}
        \end{itemize}
      }
      \vfill
      \onslide*<1>{\mintocaml[firstline=9,lastline=13,highlightlines=12]{../src/backtrace/backtrace.ml}}
      \onslide*<2->{\mintocaml[firstline=15,lastline=21,highlightlines={19-21}]{../src/backtrace/backtrace.ml}}
      \endminipage
    \end{column}
    \begin{column}{0.5\textwidth}
      \centering
      \onslide*<1>{
        \mintc[firstline=190,lastline=192]{../ocaml/runtime/amd64.S}
        \mintc[firstline=234,lastline=237]{../ocaml/runtime/amd64.S}
      }
      \onslide*<2>{
        \mintc[firstline=190,lastline=192,highlightlines={191-192}]{../ocaml/runtime/amd64.S}
        \mintc[firstline=234,lastline=237,highlightlines={235-237}]{../ocaml/runtime/amd64.S}
      }
      \onslide*<1>{\listCamlRaiseExn[highlightlines=720]{}}
      \onslide*<2>{\listCamlRaiseExn[highlightlines=722]{}}
      \onslide*<3->{\providecommand\step{5}\input{diagrams/backtrace/backtrace.tex}}
    \end{column}
  \end{columns}
\end{frame}

\begin{frame}{Backtraces}{\funcname{caml\_probably\_raise}'s handler doesn't catch \typename{ExnC}}
  \begin{columns}[c]
    \begin{column}{0.45\textwidth}
      \minipage[c][0.75\textheight][s]{\columnwidth}
      \begin{itemize}
        \item<1-> \funcname{match \dots with}\footnotemark pattern matching can also match exceptions
        \item<1-> The CMM of \funcname{probably\_raise} shows that this \funcname{match \dots with} is implemented with a pattern matching and a \funcname{try \dots with}
        \item<1-> But this one only matches on \typename{ExnA} exception
        \item<2-> Since the pattern matching fails to catch the exception, \funcname{reraise} is called with our current \typename{ExnC} exception
        \item<3-> Disassembly shows that \funcname{reraise} is actually a call to \funcname{caml\_reraise\_exn}, which is also an assembly function provided by the runtime
      \end{itemize}
      \vfill
      \mintocaml[firstline=15,lastline=21,highlightlines={18-21}]{../src/backtrace/backtrace.ml}
      \endminipage
    \end{column}
    \begin{column}{0.55\textwidth}
      \centering
      \onslide*<1>{\mintcmm[highlightlines={10-18}]{../src/backtrace/camlBacktrace\_\_probably\_raise\_351.cmm}}
      \onslide*<2>{\listcmm[highlightlines=18]{../src/backtrace/camlBacktrace\_\_probably\_raise_351.cmm}{CMM of probably\_raise}}
      \onslide*<3>{\mintobjdump[firstline=5,lastline=35,highlightlines=31]{../src/backtrace/camlBacktrace\_\_probably\_raise_351.objdump}}
    \end{column}
  \end{columns}
  \only<1>{\footnotetext[1]{https://blog.janestreet.com/pattern-matching-and-exception-handling-unite/}}
\end{frame}

\newcommand\listCamlReraiseExn[1][]{
  \listasm[firstline=727,lastline=734,#1]{../ocaml/runtime/amd64.S}{runtime/amd64.S:727}}

\begin{frame}{Backtraces}{Introducing \funcname{caml\_reraise\_exn}}
  \begin{columns}[c]
    \begin{column}{0.5\textwidth}
      \minipage[c][0.66\textheight][s]{\columnwidth}
      \begin{itemize}
        \item<1-> The exception have to be reraised to the next handler
        \item<2-> First, checks if the backtrace should be collected with \funcarg{domain\_state->backtrace\_active}
        \only<3>{
        \begin{itemize}
            \item If not, behaves like a \funcname{raise\_notrace} with \funcname{RESTORE\_EXN\_HANDLER\_OCAML}
        \end{itemize}
        }
        \item<4-> Jumps into \funcname{caml\_raise\_exn}
        \item<5-> Calls \funcname{caml\_stash\_backtrace} again, to scan the next part of the stack: from the trap just pop-ed to the next trap
      \end{itemize}
      \vfill
      \mintocaml[firstline=15,lastline=21,highlightlines={18-21}]{../src/backtrace/backtrace.ml}
      \endminipage
    \end{column}
    \begin{column}{0.5\textwidth}
      \raggedleft
      \onslide*<1>{\listCamlReraiseExn{}}
      \onslide*<2>{\listCamlReraiseExn[highlightlines=729]{}}
      \onslide*<3>{
        \mintc[firstline=190,lastline=192,highlightlines={191-192}]{../ocaml/runtime/amd64.S}
        \mintc[firstline=234,lastline=237,highlightlines={235-237}]{../ocaml/runtime/amd64.S}
        \medskip
        \listCamlReraiseExn[highlightlines=731]{}
      }
      \onslide*<4-5>{
        % TODO refactor with \listCamlReraiseExn
        \mintasm[firstline=727,lastline=734,highlightlines=730]{../ocaml/runtime/amd64.S}
        \medskip
      }
      \onslide*<4>{\listCamlRaiseExn[highlightlines=711]{}}
      \onslide*<5>{\listCamlRaiseExn[highlightlines=720]{}}
    \end{column}
  \end{columns}
\end{frame}

\begin{frame}{Backtraces}{Backtrace collection continues}
  \begin{columns}[c]
    \begin{column}{0.55\textwidth}
      \minipage[c][0.75\textheight][s]{\columnwidth}
      \begin{itemize}
        \item<1-> \funcname{caml\_stash\_backtrace} scans the older part of the stack, up to the next trap
        \item<6-> Controls jumps to the nested exception handler in \funcname{maybe\_raise}, which matches only on \typename{ExnB}
        \item<7-> \funcname{caml\_reraise\_exn} is called again, backtrace collection resumes with \funcname{caml\_stash\_backtrace}
      \end{itemize}
      \medskip
      \begin{itemize}
        \item<2-> Constructed backtrace:
        \begin{itemize}
          \item <2-> \funcname{definitely\_raise}
          \item <2-> \funcname{probably\_raise}
          \item <3-> \funcname{probably\_raise} (re-raise)
          \item <4-> \funcname{maybe\_raise}
          \item <5-> \funcname{main}
          \item <8-> \funcname{main} (re-raise)
        \end{itemize}
      \end{itemize}
      \vfill
      \onslide*<1-5>{\mintocaml[firstline=15,lastline=21]{../src/backtrace/backtrace.ml}}
      \onslide*<6>{\mintocaml[firstline=28,lastline=34,highlightlines=31]{../src/backtrace/backtrace.ml}}
      \onslide*<7->{\mintocaml[firstline=28,lastline=34,highlightlines=33]{../src/backtrace/backtrace.ml}}
      \endminipage
    \end{column}
    \begin{column}{0.45\textwidth}
      \raggedleft
      \onslide*<1>{\providecommand\step{6}\input{diagrams/backtrace/backtrace.tex}}%
      \onslide*<2>{\providecommand\step{6}\input{diagrams/backtrace/backtrace.tex}}%
      \onslide*<3>{\providecommand\step{7}\input{diagrams/backtrace/backtrace.tex}}%
      \onslide*<4>{\providecommand\step{8}\input{diagrams/backtrace/backtrace.tex}}%
      \onslide*<5>{\providecommand\step{9}\input{diagrams/backtrace/backtrace.tex}}%
      \onslide*<6>{\providecommand\step{10}\input{diagrams/backtrace/backtrace.tex}}%
      \onslide*<7>{\providecommand\step{11}\input{diagrams/backtrace/backtrace.tex}}%
      \onslide*<8>{\providecommand\step{12}\input{diagrams/backtrace/backtrace.tex}}%
      \onslide*<9>{\providecommand\step{13}\input{diagrams/backtrace/backtrace.tex}}%
    \end{column}
  \end{columns}
\end{frame}

\begin{frame}{Backtraces}{Printing the backtrace}
  \begin{columns}[c]
    \begin{column}{0.45\textwidth}
      \minipage[c][0.66\textheight][s]{\columnwidth}
      \begin{itemize}
        \item<1-> The exception backtrace can now be printed to \funcarg{stdout} with \funcname{Printexc.print_backtrace}
        \item<2-> First the exception backtrace is retrieved into an OCaml array with \funcname{get_raw_backtrace }
        \item<3-> Which is implemented as the \funcname{caml_get_exception_raw_backtrace} C function in the runtime
        \item<4-> And finally printed with \funcname{print_exception_backtrace}
      \end{itemize}
      \vfill
      \mintocaml[firstline=28,lastline=34,highlightlines=33]{../src/backtrace/backtrace.ml}
      \endminipage
    \end{column}
    \begin{column}{0.55\textwidth}
      \onslide*<1,2>{
        \mintocaml[firstline=100,lastline=101]{../ocaml/stdlib/printexc.ml}
      }
      \only<1>{
        \listocaml[firstline=170,lastline=172]{../ocaml/stdlib/printexc.ml}{stdlib/printexc.ml:170}
      }
      \only<2>{
        \listocaml[firstline=170,lastline=172,highlightlines=172]{../ocaml/stdlib/printexc.ml}{stdlib/printexc.ml:170}
      }
      \only<3>{
        \mintc[firstline=154,lastline=156]{../ocaml/runtime/backtrace.c}
        \listc[firstline=171,lastline=192,highlightlines={184-187}]{../ocaml/runtime/backtrace.c}{runtime/backtrace.c:154}
      }
      \only<4>{
        \listocaml[firstline=155,lastline=165]{../ocaml/stdlib/printexc.ml}{stdlib/printexc.ml:55}
      }
    \end{column}
  \end{columns}
\end{frame}


\subsection{The multiple kinds of raise}
\frameSubsection{}{}

\begin{frame}{Backtraces}{When backtraces are not needed}
  \begin{columns}[c]
    \begin{column}{0.45\textwidth}
      \minipage[c][0.66\textheight][s]{\columnwidth}
      \begin{itemize}
        \item<1-> What if the backtrace for an exception is never needed?
        \item<2-> It's possible to raise the exception explicitly with \funcname{raise\_notrace} to make raising as cheap as possible
        \item<3-> An example of use in the \funcname{dynlink} library of the OCaml compiler
      \end{itemize}
      \vfill
      \onslide<3->{
      \listocaml[firstline=205,lastline=209,highlightlines=207]{../ocaml/otherlibs/dynlink/dynlink\_compilerlibs/misc.ml}{otherlibs/dynlink/dynlink\_compilerlibs/misc.ml:205}
      }
      \endminipage
    \end{column}
    \begin{column}{0.55\textwidth}
    \only<4>{
      \mintcmm[highlightlines=3]{../src/notrace/camlNotrace\_\_fun_347.cmm}
      \listcmm{../src/notrace/camlNotrace\_\_all\_somes\_327.cmm}{all\_somes}
    }
    \onslide*<5->{
      \listobjdump[highlightlines={6-9}]{../src/notrace/camlNotrace\_\_fun_347.objdump}{anonymous function from \funcname{all\_somes}}
    }
    \end{column}
  \end{columns}
\end{frame}

\begin{frame}{Backtraces}{Summary table of the multiple kinds of raise}
  \begin{itemize}
    \item When debugging information is enabled and if the backtrace of an exception isn't needed, it's possible to raise with \funcname{raise\_notrace} for performance
    \item When debugging information is \emph{not} enabled, the compiler optimises \funcname{raise} calls to inline assembly (same effect as using \funcname{raise\_notrace})
  \end{itemize}
  \bigskip
  \centering
  \begin{tabular}{| l | c | c | c | c |}
    \hline
    \parbox{10em}{Debug information \\ (\funcarg{-g} flag)}  & \multicolumn{2}{|c|}{Disabled}                                     & \multicolumn{2}{|c|}{Enabled} \\
    \hline
    Raise kind                                               & \funcname{raise} & \funcname{raise\_notrace}                       & \funcname{raise} & \funcname{raise\_notrace} \\
    \hline
    Compilation                                              & \parbox{8em}{inline assembly\\ (optimization)} & inline assembly   & \funcname{caml\_raise\_exn} & inline assembly \\
    \hline
  \end{tabular}
\end{frame}


%
% Takeaway #5
%
\subsection*{Takeaway \#5}
\frameSubsectionTakeaway{}
\againframe<5>{takeaway}
}%
      \onslide*<3>{\providecommand\step{7}%
% Backtraces
%
\subsection{One advantage of raising with debugging information}
\frameSubsection{}{}

\begin{frame}{Backtraces}{Debugging information \& source code}
  \begin{columns}[c]
    \begin{column}{0.5\textwidth}
      \begin{itemize}
        \item Raising an exception is fun and games, but how do we know where the exception originated? $\rightarrow$ With the backtrace associated with the exception!
        \item In order to get a backtrace from the point where an exception was raised, it's necessary to compile with debugging information enabled (\funcarg{-g} flag for the compiler)
        \item Same example as in the `Nested handlers' section
      \end{itemize}
    \end{column}
    \begin{column}{0.5\textwidth}
      \listocaml[firstline=9,lastline=34]{../src/backtrace/backtrace.ml}{backtrace/backtrace.ml}
    \end{column}
  \end{columns}
\end{frame}

\begin{frame}{Backtraces}{CMM and disassembly of \funcname{definitely\_raise}}
  \begin{columns}[c]
    \begin{column}{0.5\textwidth}
      \minipage[c][0.66\textheight][s]{\columnwidth}
      \begin{itemize}
        \item<1-> An effect of compiling with debugging information (\funcarg{-g}): the CMM no longer uses \funcname{raise\_notrace} to raise the exception but \funcname{raise} instead
        \item<2-> Disassembly shows that CMM \funcname{raise} is actually a call to \funcname{caml\_raise\_exn}, a function in assembler provided by the runtime
      \end{itemize}
      \vfill
      \mintocaml[firstline=9,lastline=13,highlightlines=12]{../src/backtrace/backtrace.ml}
      \endminipage
    \end{column}
    \begin{column}{0.5\textwidth}
      \onslide*<1>{
        \mintcmm[highlightlines=5]{../src/backtrace/camlBacktrace\_\_definitely\_raise\_347.cmm}
      }
      \onslide*<2>{
        \mintobjdump[firstline=2,highlightlines=14]{../src/backtrace/camlBacktrace\_\_definitely\_raise\_347.objdump}
      }
    \end{column}
  \end{columns}
\end{frame}

\begin{frame}{Backtraces}{Introducing \funcname{caml\_raise\_exn}}
  \begin{columns}[c]
    \begin{column}{0.5\textwidth}
      \minipage[c][0.66\textheight][s]{\columnwidth}
      \onslide*<1>{
        \begin{itemize}
          \item Raising the exception is performed using \funcname{caml\_raise\_exn} which is
            \begin{itemize}
              \item An assembly function provided by the runtime
              \item Architecture specific
            \end{itemize}
          \item Let's see what it does differently from \funcname{raise\_notrace}\dots
          \item We are about to raise \typename{ExnC} with \funcname{caml\_raise\_exn} from \funcname{definitely\_raise}
        \end{itemize}
      }
      \onslide*<2>{
        \mintobjdump[firstline=2,highlightlines=14]{../src/backtrace/camlBacktrace\_\_definitely\_raise\_347.objdump}
      }
      \vfill
      \mintocaml[firstline=9,lastline=13,highlightlines=12]{../src/backtrace/backtrace.ml}
      \endminipage
    \end{column}
    \begin{column}{0.5\textwidth}
      \centering
      \providecommand\step{1}\input{diagrams/backtrace/backtrace.tex}
    \end{column}
  \end{columns}
\end{frame}

\begin{frame}{Backtraces}{But before that, we need more fields from \typename{caml\_domain\_state}}
  \begin{columns}
    \begin{column}{0.5\textwidth}
      \begin{itemize}
        \item Remember \typename{caml\_domain\_state} the per-domain runtime state?
        \item Some other members are of interest to us now\dots
        \item \localname{backtrace\_active} is used to know if the runtime should collect backtraces
        \item \localname{backtrace\_active} can be set by
          \begin{itemize}
            \item The \funcarg{b} flag in the \funcname{OCAMLRUNPARAM} environment variable
            \item \mintinline{ocaml}{Printexc.record_backtrace : bool -> unit} \footnotemark
          \end{itemize}
      \end{itemize}
    \end{column}
    \begin{column}{0.5\textwidth}
      \begin{listing}
        \mintc[highlightlines={9-11}]{src/ocaml/domain\_state\_backtrace.h}
        \caption{runtime/caml/domain\_state.h}
      \end{listing}
    \end{column}
  \end{columns}
  \footnotetext[1]{https://v2.ocaml.org/api/Printexc.html}
\end{frame}

\newcommand\listCamlRaiseExn[1][]{
  \listasm[firstline=702,lastline=725,#1]{../ocaml/runtime/amd64.S}{runtime/amd64.S:702}}

\begin{frame}{Backtraces}{Walk through \funcname{caml\_raise\_exn}}
  \begin{columns}[T]
    \begin{column}{0.5\textwidth}
      \begin{itemize}
        \item<1-> First checks whether exceptions backtrace should be recorded
        \item<1-> By checking the \funcarg{domain\_state->backtrace\_active} value
        \item<2-> If not, acts as \funcname{raise\_notrace} with \funcname{RESTORE\_EXN\_HANDLER\_OCAML}
          \begin{itemize}
            \item Set \regname{RSP} to \localname{domain\_state->exn\_handler} value
            \item Restore \localname{domain\_state->exn\_handler} from trap
            \item Jump with \funcname{ret} (same effect as using \funcname{pop} and \funcname{jmp})
          \end{itemize}
        \item<3-> Otherwise, switches to the C stack to be able to execute C code
        \item<4-> Calls \funcname{caml\_stash\_backtrace}, a C function provided by the runtime\ignorespaces
          \onslide<6->{, with
            \begin{itemize}
              \item \funcarg{exn}: exception value
              \item \funcarg{pc}: code address of the raising point
              \item \funcarg{sp}: stack address of the raising function
              \item \funcarg{trapsp}: stack address of the next handler
            \end{itemize}
          }
      \end{itemize}
      \bigskip
      \mintocaml[firstline=9,lastline=13,highlightlines=12]{../src/backtrace/backtrace.ml}
    \end{column}
    \begin{column}{0.5\textwidth}
      \raggedleft
      \onslide*<1,3-4>{
        \mintc[firstline=190,lastline=192]{../ocaml/runtime/amd64.S}
        \mintc[firstline=234,lastline=237]{../ocaml/runtime/amd64.S}
      }
      \onslide*<2>{
        \mintc[firstline=190,lastline=192,highlightlines={191-192}]{../ocaml/runtime/amd64.S}
        \mintc[firstline=234,lastline=237,highlightlines={235-237}]{../ocaml/runtime/amd64.S}
      }
      \onslide*<1>{\listCamlRaiseExn[highlightlines=705]{}}%
      \onslide*<2>{\listCamlRaiseExn[highlightlines={707-708}]{}}%
      \onslide*<3>{\listCamlRaiseExn[highlightlines={712-714}]{}}%
      \onslide*<4>{\listCamlRaiseExn[highlightlines={715-720}]{}}%
      \onslide*<5>{\providecommand\step{1}\input{diagrams/backtrace/backtrace.tex}}%
      \onslide*<6>{\providecommand\step{2}\input{diagrams/backtrace/backtrace.tex}}%
    \end{column}
  \end{columns}
\end{frame}

\newcommand\listStashBacktrace[1][]{
  \listc[firstline=91,lastline=120,#1]{../ocaml/runtime/backtrace\_nat.c}{runtime/backtrace\_nat.c:91}}

\begin{frame}{Backtraces}{Introducing \funcname{caml\_stash\_backtrace}}
  \begin{columns}[c]
    \begin{column}{0.5\textwidth}
      \minipage[c][0.66\textheight][s]{\columnwidth}
      \begin{itemize}
        \item<1-> A C function provided by the runtime (specific to native code)
        \item<2-> It scans the stack, between the stack pointer at the raising point up to the next trap, in order to collect the names of the detected function frames
          \onslide*<2>{
            \begin{itemize}
              \item And more information, like line number, column number...
            \end{itemize}
          }
        \item<3-> It uses the same technique and information as the Garbage Collector (GC) uses to scan the values live on the stack
          \onslide*<4>{
            \begin{itemize}
              \item \typename{frame\_descr} are at the core of stack unwinding
            \end{itemize}
          }
      \end{itemize}
      \vfill
      \mintocaml[firstline=9,lastline=13,highlightlines=12]{../src/backtrace/backtrace.ml}
      \endminipage
    \end{column}
    \begin{column}{0.5\textwidth}
      \onslide*<1>{\listStashBacktrace[highlightlines=91]{}}
      \onslide*<2>{\listStashBacktrace[highlightlines={91,117-118}]{}}
      \onslide*<3->{\listStashBacktrace[highlightlines={94,106,109-110}]{}}
    \end{column}
  \end{columns}
\end{frame}

\newcommand\listFrameDescr[1][]{
  \listocaml[firstline=111,lastline=124,#1]{../ocaml/asmcomp/emitaux.ml}{asmcomp/emitaux.ml:111}}
\newcommand\listDebugInfo[1][]{
  \listocaml[firstline=38,lastline=49,#1]{../ocaml/lambda/debuginfo.mli}{lambda/debuginfo.mli:38}}

\begin{frame}{Backtraces}{\typename{frame\_descr} and \typename{frame\_debuginfo} in a nutshell}
  \begin{columns}[c]
    \begin{column}{0.5\textwidth}
      \begin{itemize}
        \item<1-> For every return address in an OCaml program, there is a associated \typename{frame\_descr}
        \item<2-> A \typename{frame\_descr} specifies
        \begin{itemize}
          \item<2-> The size of the function stack frame
          \item<3-> Where live values are on the stack (so that the GC can mark them as live and prevent from being swept)
        \end{itemize}
        \item<4-> When compiled with debug information, \typename{frame\_descr} will be linked to a \typename{frame\_debuginfo}
        \begin{itemize}
          \item<5-> Contains information about the position in the source code (file name, line and column number)
        \end{itemize}
      \end{itemize}
    \end{column}
    \begin{column}{0.5\textwidth}
        \onslide*<1>{\listFrameDescr[highlightlines={118-122}]{}}
        \onslide*<2>{\listFrameDescr[highlightlines={120}]{}}
        \onslide*<3>{\listFrameDescr[highlightlines={121}]{}}
        \onslide*<4->{\listFrameDescr[highlightlines={122}]{}}
        \medskip
        \onslide*<1-3>{\listDebugInfo{}}
        \onslide*<4>{\listDebugInfo[highlightlines={38,49}]{}}
        \onslide*<5>{\listDebugInfo[highlightlines={39-46}]{}}
    \end{column}
  \end{columns}
\end{frame}

\begin{frame}{Backtraces}{Scanning the stack with \typename{frame\_descr}}
  \begin{columns}[c]
    \begin{column}{0.5\textwidth}
      \begin{itemize}
        \item Given the return address of the code that raises the exception by calling \funcname{caml\_raise\_exn} (\funcarg{pc} argument of \funcname{caml\_stash\_backtrace})
        \item And the position of the stack pointer (\funcarg{sp} argument of \funcname{caml\_stash\_backtrace})
        \item The frame size given by \typename{frame\_descr}.\funcarg{frame\_size}, allows to find the end of the previous frame
        \item And the return address of the caller of this function
        \bigskip
        \onslide<3->{
        \item Note that traps (here the reddish \funcname{ExnA}, \funcname{ExnB} and \funcname{ExnC} blocks) belong to the frame that installed them, and so the frame size changes when traps are removed
        }
      \end{itemize}
    \end{column}
    \begin{column}{0.5\textwidth}
      \raggedleft
      \onslide*<1>{\providecommand\step{2}\input{diagrams/backtrace/backtrace.tex}}%
      \onslide*<2>{\providecommand\step{1}\input{diagrams/backtrace/scanstack.tex}}%
      \onslide*<3>{\providecommand\step{2}\input{diagrams/backtrace/scanstack.tex}}%
      \onslide*<4>{\providecommand\step{3}\input{diagrams/backtrace/scanstack.tex}}%
      \onslide*<5>{\providecommand\step{4}\input{diagrams/backtrace/scanstack.tex}}%
      \onslide*<6>{\providecommand\step{5}\input{diagrams/backtrace/scanstack.tex}}%
    \end{column}
  \end{columns}
\end{frame}

% Duplicate of previous-previous slide
\begin{frame}{Backtraces}{Continuing on \funcname{caml\_stash\_backtrace}}
  \begin{columns}[c]
    \begin{column}{0.5\textwidth}
      \minipage[c][0.75\textheight][s]{\columnwidth}
      \begin{itemize}
          \color{gray}
        \item A C function provided by the runtime (specific to native code)
        \item It scans the stack, between the stack pointer at the raising point up to the next trap, in order to collect the names of the detected function frames
        \item It uses the same technique and information as the Garbage Collector (GC) uses to scan the values live on the stack
      \end{itemize}
      \begin{itemize}
        \item For each function frame detected, store a pointer to the \typename{frame\_descr} in the \localname{domain\_state->backtrace\_buffer} array, effectively constructing the backtrace
      \end{itemize}
      \onslide<2->{
        \begin{itemize}
            \smallskip
          \item Constructed backtrace:
            \funcname{definitely\_raise},
            \onslide<3->{\funcname{probably\_raise}}
        \end{itemize}
      }
      \vfill
      \mintocaml[firstline=9,lastline=13,highlightlines=12]{../src/backtrace/backtrace.ml}
      \endminipage
    \end{column}
    \begin{column}{0.5\textwidth}
      \raggedleft
      \onslide*<1>{\listStashBacktrace[highlightlines={114-115}]{}}
      \onslide*<2>{\providecommand\step{3}\input{diagrams/backtrace/backtrace.tex}}%
      \onslide*<3>{\providecommand\step{4}\input{diagrams/backtrace/backtrace.tex}}%
    \end{column}
  \end{columns}
\end{frame}

\begin{frame}{Backtraces}{Back to \funcname{caml\_raise\_exn}}
  \begin{columns}[c]
    \begin{column}{0.5\textwidth}
      \minipage[c][0.66\textheight][s]{\columnwidth}
      \begin{itemize}\color{gray}
        \item Checks wheteher exceptions backtrace should be recorded, by checking the \funcarg{domain\_state->backtrace\_active} value
        \item Switches to the C stack to be able to execute C code
        \item Calls \funcname{caml\_stash\_backtrace}, to collect the backtrace up to the next trap
      \end{itemize}
      \onslide*<2->{
        \begin{itemize}
          \item<2-> Executes the exception handler in \funcname{probably\_raise} with \funcname{RESTORE\_EXN\_HANDLER\_OCAML}
            \begin{itemize}
              \item Set \regname{RSP} to \localname{domain\_state->exn\_handler} value
              \item Restore \localname{domain\_state->exn\_handler} from trap
              \item Jump to the handler code with \funcname{ret}
            \end{itemize}
        \end{itemize}
      }
      \vfill
      \onslide*<1>{\mintocaml[firstline=9,lastline=13,highlightlines=12]{../src/backtrace/backtrace.ml}}
      \onslide*<2->{\mintocaml[firstline=15,lastline=21,highlightlines={19-21}]{../src/backtrace/backtrace.ml}}
      \endminipage
    \end{column}
    \begin{column}{0.5\textwidth}
      \centering
      \onslide*<1>{
        \mintc[firstline=190,lastline=192]{../ocaml/runtime/amd64.S}
        \mintc[firstline=234,lastline=237]{../ocaml/runtime/amd64.S}
      }
      \onslide*<2>{
        \mintc[firstline=190,lastline=192,highlightlines={191-192}]{../ocaml/runtime/amd64.S}
        \mintc[firstline=234,lastline=237,highlightlines={235-237}]{../ocaml/runtime/amd64.S}
      }
      \onslide*<1>{\listCamlRaiseExn[highlightlines=720]{}}
      \onslide*<2>{\listCamlRaiseExn[highlightlines=722]{}}
      \onslide*<3->{\providecommand\step{5}\input{diagrams/backtrace/backtrace.tex}}
    \end{column}
  \end{columns}
\end{frame}

\begin{frame}{Backtraces}{\funcname{caml\_probably\_raise}'s handler doesn't catch \typename{ExnC}}
  \begin{columns}[c]
    \begin{column}{0.45\textwidth}
      \minipage[c][0.75\textheight][s]{\columnwidth}
      \begin{itemize}
        \item<1-> \funcname{match \dots with}\footnotemark pattern matching can also match exceptions
        \item<1-> The CMM of \funcname{probably\_raise} shows that this \funcname{match \dots with} is implemented with a pattern matching and a \funcname{try \dots with}
        \item<1-> But this one only matches on \typename{ExnA} exception
        \item<2-> Since the pattern matching fails to catch the exception, \funcname{reraise} is called with our current \typename{ExnC} exception
        \item<3-> Disassembly shows that \funcname{reraise} is actually a call to \funcname{caml\_reraise\_exn}, which is also an assembly function provided by the runtime
      \end{itemize}
      \vfill
      \mintocaml[firstline=15,lastline=21,highlightlines={18-21}]{../src/backtrace/backtrace.ml}
      \endminipage
    \end{column}
    \begin{column}{0.55\textwidth}
      \centering
      \onslide*<1>{\mintcmm[highlightlines={10-18}]{../src/backtrace/camlBacktrace\_\_probably\_raise\_351.cmm}}
      \onslide*<2>{\listcmm[highlightlines=18]{../src/backtrace/camlBacktrace\_\_probably\_raise_351.cmm}{CMM of probably\_raise}}
      \onslide*<3>{\mintobjdump[firstline=5,lastline=35,highlightlines=31]{../src/backtrace/camlBacktrace\_\_probably\_raise_351.objdump}}
    \end{column}
  \end{columns}
  \only<1>{\footnotetext[1]{https://blog.janestreet.com/pattern-matching-and-exception-handling-unite/}}
\end{frame}

\newcommand\listCamlReraiseExn[1][]{
  \listasm[firstline=727,lastline=734,#1]{../ocaml/runtime/amd64.S}{runtime/amd64.S:727}}

\begin{frame}{Backtraces}{Introducing \funcname{caml\_reraise\_exn}}
  \begin{columns}[c]
    \begin{column}{0.5\textwidth}
      \minipage[c][0.66\textheight][s]{\columnwidth}
      \begin{itemize}
        \item<1-> The exception have to be reraised to the next handler
        \item<2-> First, checks if the backtrace should be collected with \funcarg{domain\_state->backtrace\_active}
        \only<3>{
        \begin{itemize}
            \item If not, behaves like a \funcname{raise\_notrace} with \funcname{RESTORE\_EXN\_HANDLER\_OCAML}
        \end{itemize}
        }
        \item<4-> Jumps into \funcname{caml\_raise\_exn}
        \item<5-> Calls \funcname{caml\_stash\_backtrace} again, to scan the next part of the stack: from the trap just pop-ed to the next trap
      \end{itemize}
      \vfill
      \mintocaml[firstline=15,lastline=21,highlightlines={18-21}]{../src/backtrace/backtrace.ml}
      \endminipage
    \end{column}
    \begin{column}{0.5\textwidth}
      \raggedleft
      \onslide*<1>{\listCamlReraiseExn{}}
      \onslide*<2>{\listCamlReraiseExn[highlightlines=729]{}}
      \onslide*<3>{
        \mintc[firstline=190,lastline=192,highlightlines={191-192}]{../ocaml/runtime/amd64.S}
        \mintc[firstline=234,lastline=237,highlightlines={235-237}]{../ocaml/runtime/amd64.S}
        \medskip
        \listCamlReraiseExn[highlightlines=731]{}
      }
      \onslide*<4-5>{
        % TODO refactor with \listCamlReraiseExn
        \mintasm[firstline=727,lastline=734,highlightlines=730]{../ocaml/runtime/amd64.S}
        \medskip
      }
      \onslide*<4>{\listCamlRaiseExn[highlightlines=711]{}}
      \onslide*<5>{\listCamlRaiseExn[highlightlines=720]{}}
    \end{column}
  \end{columns}
\end{frame}

\begin{frame}{Backtraces}{Backtrace collection continues}
  \begin{columns}[c]
    \begin{column}{0.55\textwidth}
      \minipage[c][0.75\textheight][s]{\columnwidth}
      \begin{itemize}
        \item<1-> \funcname{caml\_stash\_backtrace} scans the older part of the stack, up to the next trap
        \item<6-> Controls jumps to the nested exception handler in \funcname{maybe\_raise}, which matches only on \typename{ExnB}
        \item<7-> \funcname{caml\_reraise\_exn} is called again, backtrace collection resumes with \funcname{caml\_stash\_backtrace}
      \end{itemize}
      \medskip
      \begin{itemize}
        \item<2-> Constructed backtrace:
        \begin{itemize}
          \item <2-> \funcname{definitely\_raise}
          \item <2-> \funcname{probably\_raise}
          \item <3-> \funcname{probably\_raise} (re-raise)
          \item <4-> \funcname{maybe\_raise}
          \item <5-> \funcname{main}
          \item <8-> \funcname{main} (re-raise)
        \end{itemize}
      \end{itemize}
      \vfill
      \onslide*<1-5>{\mintocaml[firstline=15,lastline=21]{../src/backtrace/backtrace.ml}}
      \onslide*<6>{\mintocaml[firstline=28,lastline=34,highlightlines=31]{../src/backtrace/backtrace.ml}}
      \onslide*<7->{\mintocaml[firstline=28,lastline=34,highlightlines=33]{../src/backtrace/backtrace.ml}}
      \endminipage
    \end{column}
    \begin{column}{0.45\textwidth}
      \raggedleft
      \onslide*<1>{\providecommand\step{6}\input{diagrams/backtrace/backtrace.tex}}%
      \onslide*<2>{\providecommand\step{6}\input{diagrams/backtrace/backtrace.tex}}%
      \onslide*<3>{\providecommand\step{7}\input{diagrams/backtrace/backtrace.tex}}%
      \onslide*<4>{\providecommand\step{8}\input{diagrams/backtrace/backtrace.tex}}%
      \onslide*<5>{\providecommand\step{9}\input{diagrams/backtrace/backtrace.tex}}%
      \onslide*<6>{\providecommand\step{10}\input{diagrams/backtrace/backtrace.tex}}%
      \onslide*<7>{\providecommand\step{11}\input{diagrams/backtrace/backtrace.tex}}%
      \onslide*<8>{\providecommand\step{12}\input{diagrams/backtrace/backtrace.tex}}%
      \onslide*<9>{\providecommand\step{13}\input{diagrams/backtrace/backtrace.tex}}%
    \end{column}
  \end{columns}
\end{frame}

\begin{frame}{Backtraces}{Printing the backtrace}
  \begin{columns}[c]
    \begin{column}{0.45\textwidth}
      \minipage[c][0.66\textheight][s]{\columnwidth}
      \begin{itemize}
        \item<1-> The exception backtrace can now be printed to \funcarg{stdout} with \funcname{Printexc.print_backtrace}
        \item<2-> First the exception backtrace is retrieved into an OCaml array with \funcname{get_raw_backtrace }
        \item<3-> Which is implemented as the \funcname{caml_get_exception_raw_backtrace} C function in the runtime
        \item<4-> And finally printed with \funcname{print_exception_backtrace}
      \end{itemize}
      \vfill
      \mintocaml[firstline=28,lastline=34,highlightlines=33]{../src/backtrace/backtrace.ml}
      \endminipage
    \end{column}
    \begin{column}{0.55\textwidth}
      \onslide*<1,2>{
        \mintocaml[firstline=100,lastline=101]{../ocaml/stdlib/printexc.ml}
      }
      \only<1>{
        \listocaml[firstline=170,lastline=172]{../ocaml/stdlib/printexc.ml}{stdlib/printexc.ml:170}
      }
      \only<2>{
        \listocaml[firstline=170,lastline=172,highlightlines=172]{../ocaml/stdlib/printexc.ml}{stdlib/printexc.ml:170}
      }
      \only<3>{
        \mintc[firstline=154,lastline=156]{../ocaml/runtime/backtrace.c}
        \listc[firstline=171,lastline=192,highlightlines={184-187}]{../ocaml/runtime/backtrace.c}{runtime/backtrace.c:154}
      }
      \only<4>{
        \listocaml[firstline=155,lastline=165]{../ocaml/stdlib/printexc.ml}{stdlib/printexc.ml:55}
      }
    \end{column}
  \end{columns}
\end{frame}


\subsection{The multiple kinds of raise}
\frameSubsection{}{}

\begin{frame}{Backtraces}{When backtraces are not needed}
  \begin{columns}[c]
    \begin{column}{0.45\textwidth}
      \minipage[c][0.66\textheight][s]{\columnwidth}
      \begin{itemize}
        \item<1-> What if the backtrace for an exception is never needed?
        \item<2-> It's possible to raise the exception explicitly with \funcname{raise\_notrace} to make raising as cheap as possible
        \item<3-> An example of use in the \funcname{dynlink} library of the OCaml compiler
      \end{itemize}
      \vfill
      \onslide<3->{
      \listocaml[firstline=205,lastline=209,highlightlines=207]{../ocaml/otherlibs/dynlink/dynlink\_compilerlibs/misc.ml}{otherlibs/dynlink/dynlink\_compilerlibs/misc.ml:205}
      }
      \endminipage
    \end{column}
    \begin{column}{0.55\textwidth}
    \only<4>{
      \mintcmm[highlightlines=3]{../src/notrace/camlNotrace\_\_fun_347.cmm}
      \listcmm{../src/notrace/camlNotrace\_\_all\_somes\_327.cmm}{all\_somes}
    }
    \onslide*<5->{
      \listobjdump[highlightlines={6-9}]{../src/notrace/camlNotrace\_\_fun_347.objdump}{anonymous function from \funcname{all\_somes}}
    }
    \end{column}
  \end{columns}
\end{frame}

\begin{frame}{Backtraces}{Summary table of the multiple kinds of raise}
  \begin{itemize}
    \item When debugging information is enabled and if the backtrace of an exception isn't needed, it's possible to raise with \funcname{raise\_notrace} for performance
    \item When debugging information is \emph{not} enabled, the compiler optimises \funcname{raise} calls to inline assembly (same effect as using \funcname{raise\_notrace})
  \end{itemize}
  \bigskip
  \centering
  \begin{tabular}{| l | c | c | c | c |}
    \hline
    \parbox{10em}{Debug information \\ (\funcarg{-g} flag)}  & \multicolumn{2}{|c|}{Disabled}                                     & \multicolumn{2}{|c|}{Enabled} \\
    \hline
    Raise kind                                               & \funcname{raise} & \funcname{raise\_notrace}                       & \funcname{raise} & \funcname{raise\_notrace} \\
    \hline
    Compilation                                              & \parbox{8em}{inline assembly\\ (optimization)} & inline assembly   & \funcname{caml\_raise\_exn} & inline assembly \\
    \hline
  \end{tabular}
\end{frame}


%
% Takeaway #5
%
\subsection*{Takeaway \#5}
\frameSubsectionTakeaway{}
\againframe<5>{takeaway}
}%
      \onslide*<4>{\providecommand\step{8}%
% Backtraces
%
\subsection{One advantage of raising with debugging information}
\frameSubsection{}{}

\begin{frame}{Backtraces}{Debugging information \& source code}
  \begin{columns}[c]
    \begin{column}{0.5\textwidth}
      \begin{itemize}
        \item Raising an exception is fun and games, but how do we know where the exception originated? $\rightarrow$ With the backtrace associated with the exception!
        \item In order to get a backtrace from the point where an exception was raised, it's necessary to compile with debugging information enabled (\funcarg{-g} flag for the compiler)
        \item Same example as in the `Nested handlers' section
      \end{itemize}
    \end{column}
    \begin{column}{0.5\textwidth}
      \listocaml[firstline=9,lastline=34]{../src/backtrace/backtrace.ml}{backtrace/backtrace.ml}
    \end{column}
  \end{columns}
\end{frame}

\begin{frame}{Backtraces}{CMM and disassembly of \funcname{definitely\_raise}}
  \begin{columns}[c]
    \begin{column}{0.5\textwidth}
      \minipage[c][0.66\textheight][s]{\columnwidth}
      \begin{itemize}
        \item<1-> An effect of compiling with debugging information (\funcarg{-g}): the CMM no longer uses \funcname{raise\_notrace} to raise the exception but \funcname{raise} instead
        \item<2-> Disassembly shows that CMM \funcname{raise} is actually a call to \funcname{caml\_raise\_exn}, a function in assembler provided by the runtime
      \end{itemize}
      \vfill
      \mintocaml[firstline=9,lastline=13,highlightlines=12]{../src/backtrace/backtrace.ml}
      \endminipage
    \end{column}
    \begin{column}{0.5\textwidth}
      \onslide*<1>{
        \mintcmm[highlightlines=5]{../src/backtrace/camlBacktrace\_\_definitely\_raise\_347.cmm}
      }
      \onslide*<2>{
        \mintobjdump[firstline=2,highlightlines=14]{../src/backtrace/camlBacktrace\_\_definitely\_raise\_347.objdump}
      }
    \end{column}
  \end{columns}
\end{frame}

\begin{frame}{Backtraces}{Introducing \funcname{caml\_raise\_exn}}
  \begin{columns}[c]
    \begin{column}{0.5\textwidth}
      \minipage[c][0.66\textheight][s]{\columnwidth}
      \onslide*<1>{
        \begin{itemize}
          \item Raising the exception is performed using \funcname{caml\_raise\_exn} which is
            \begin{itemize}
              \item An assembly function provided by the runtime
              \item Architecture specific
            \end{itemize}
          \item Let's see what it does differently from \funcname{raise\_notrace}\dots
          \item We are about to raise \typename{ExnC} with \funcname{caml\_raise\_exn} from \funcname{definitely\_raise}
        \end{itemize}
      }
      \onslide*<2>{
        \mintobjdump[firstline=2,highlightlines=14]{../src/backtrace/camlBacktrace\_\_definitely\_raise\_347.objdump}
      }
      \vfill
      \mintocaml[firstline=9,lastline=13,highlightlines=12]{../src/backtrace/backtrace.ml}
      \endminipage
    \end{column}
    \begin{column}{0.5\textwidth}
      \centering
      \providecommand\step{1}\input{diagrams/backtrace/backtrace.tex}
    \end{column}
  \end{columns}
\end{frame}

\begin{frame}{Backtraces}{But before that, we need more fields from \typename{caml\_domain\_state}}
  \begin{columns}
    \begin{column}{0.5\textwidth}
      \begin{itemize}
        \item Remember \typename{caml\_domain\_state} the per-domain runtime state?
        \item Some other members are of interest to us now\dots
        \item \localname{backtrace\_active} is used to know if the runtime should collect backtraces
        \item \localname{backtrace\_active} can be set by
          \begin{itemize}
            \item The \funcarg{b} flag in the \funcname{OCAMLRUNPARAM} environment variable
            \item \mintinline{ocaml}{Printexc.record_backtrace : bool -> unit} \footnotemark
          \end{itemize}
      \end{itemize}
    \end{column}
    \begin{column}{0.5\textwidth}
      \begin{listing}
        \mintc[highlightlines={9-11}]{src/ocaml/domain\_state\_backtrace.h}
        \caption{runtime/caml/domain\_state.h}
      \end{listing}
    \end{column}
  \end{columns}
  \footnotetext[1]{https://v2.ocaml.org/api/Printexc.html}
\end{frame}

\newcommand\listCamlRaiseExn[1][]{
  \listasm[firstline=702,lastline=725,#1]{../ocaml/runtime/amd64.S}{runtime/amd64.S:702}}

\begin{frame}{Backtraces}{Walk through \funcname{caml\_raise\_exn}}
  \begin{columns}[T]
    \begin{column}{0.5\textwidth}
      \begin{itemize}
        \item<1-> First checks whether exceptions backtrace should be recorded
        \item<1-> By checking the \funcarg{domain\_state->backtrace\_active} value
        \item<2-> If not, acts as \funcname{raise\_notrace} with \funcname{RESTORE\_EXN\_HANDLER\_OCAML}
          \begin{itemize}
            \item Set \regname{RSP} to \localname{domain\_state->exn\_handler} value
            \item Restore \localname{domain\_state->exn\_handler} from trap
            \item Jump with \funcname{ret} (same effect as using \funcname{pop} and \funcname{jmp})
          \end{itemize}
        \item<3-> Otherwise, switches to the C stack to be able to execute C code
        \item<4-> Calls \funcname{caml\_stash\_backtrace}, a C function provided by the runtime\ignorespaces
          \onslide<6->{, with
            \begin{itemize}
              \item \funcarg{exn}: exception value
              \item \funcarg{pc}: code address of the raising point
              \item \funcarg{sp}: stack address of the raising function
              \item \funcarg{trapsp}: stack address of the next handler
            \end{itemize}
          }
      \end{itemize}
      \bigskip
      \mintocaml[firstline=9,lastline=13,highlightlines=12]{../src/backtrace/backtrace.ml}
    \end{column}
    \begin{column}{0.5\textwidth}
      \raggedleft
      \onslide*<1,3-4>{
        \mintc[firstline=190,lastline=192]{../ocaml/runtime/amd64.S}
        \mintc[firstline=234,lastline=237]{../ocaml/runtime/amd64.S}
      }
      \onslide*<2>{
        \mintc[firstline=190,lastline=192,highlightlines={191-192}]{../ocaml/runtime/amd64.S}
        \mintc[firstline=234,lastline=237,highlightlines={235-237}]{../ocaml/runtime/amd64.S}
      }
      \onslide*<1>{\listCamlRaiseExn[highlightlines=705]{}}%
      \onslide*<2>{\listCamlRaiseExn[highlightlines={707-708}]{}}%
      \onslide*<3>{\listCamlRaiseExn[highlightlines={712-714}]{}}%
      \onslide*<4>{\listCamlRaiseExn[highlightlines={715-720}]{}}%
      \onslide*<5>{\providecommand\step{1}\input{diagrams/backtrace/backtrace.tex}}%
      \onslide*<6>{\providecommand\step{2}\input{diagrams/backtrace/backtrace.tex}}%
    \end{column}
  \end{columns}
\end{frame}

\newcommand\listStashBacktrace[1][]{
  \listc[firstline=91,lastline=120,#1]{../ocaml/runtime/backtrace\_nat.c}{runtime/backtrace\_nat.c:91}}

\begin{frame}{Backtraces}{Introducing \funcname{caml\_stash\_backtrace}}
  \begin{columns}[c]
    \begin{column}{0.5\textwidth}
      \minipage[c][0.66\textheight][s]{\columnwidth}
      \begin{itemize}
        \item<1-> A C function provided by the runtime (specific to native code)
        \item<2-> It scans the stack, between the stack pointer at the raising point up to the next trap, in order to collect the names of the detected function frames
          \onslide*<2>{
            \begin{itemize}
              \item And more information, like line number, column number...
            \end{itemize}
          }
        \item<3-> It uses the same technique and information as the Garbage Collector (GC) uses to scan the values live on the stack
          \onslide*<4>{
            \begin{itemize}
              \item \typename{frame\_descr} are at the core of stack unwinding
            \end{itemize}
          }
      \end{itemize}
      \vfill
      \mintocaml[firstline=9,lastline=13,highlightlines=12]{../src/backtrace/backtrace.ml}
      \endminipage
    \end{column}
    \begin{column}{0.5\textwidth}
      \onslide*<1>{\listStashBacktrace[highlightlines=91]{}}
      \onslide*<2>{\listStashBacktrace[highlightlines={91,117-118}]{}}
      \onslide*<3->{\listStashBacktrace[highlightlines={94,106,109-110}]{}}
    \end{column}
  \end{columns}
\end{frame}

\newcommand\listFrameDescr[1][]{
  \listocaml[firstline=111,lastline=124,#1]{../ocaml/asmcomp/emitaux.ml}{asmcomp/emitaux.ml:111}}
\newcommand\listDebugInfo[1][]{
  \listocaml[firstline=38,lastline=49,#1]{../ocaml/lambda/debuginfo.mli}{lambda/debuginfo.mli:38}}

\begin{frame}{Backtraces}{\typename{frame\_descr} and \typename{frame\_debuginfo} in a nutshell}
  \begin{columns}[c]
    \begin{column}{0.5\textwidth}
      \begin{itemize}
        \item<1-> For every return address in an OCaml program, there is a associated \typename{frame\_descr}
        \item<2-> A \typename{frame\_descr} specifies
        \begin{itemize}
          \item<2-> The size of the function stack frame
          \item<3-> Where live values are on the stack (so that the GC can mark them as live and prevent from being swept)
        \end{itemize}
        \item<4-> When compiled with debug information, \typename{frame\_descr} will be linked to a \typename{frame\_debuginfo}
        \begin{itemize}
          \item<5-> Contains information about the position in the source code (file name, line and column number)
        \end{itemize}
      \end{itemize}
    \end{column}
    \begin{column}{0.5\textwidth}
        \onslide*<1>{\listFrameDescr[highlightlines={118-122}]{}}
        \onslide*<2>{\listFrameDescr[highlightlines={120}]{}}
        \onslide*<3>{\listFrameDescr[highlightlines={121}]{}}
        \onslide*<4->{\listFrameDescr[highlightlines={122}]{}}
        \medskip
        \onslide*<1-3>{\listDebugInfo{}}
        \onslide*<4>{\listDebugInfo[highlightlines={38,49}]{}}
        \onslide*<5>{\listDebugInfo[highlightlines={39-46}]{}}
    \end{column}
  \end{columns}
\end{frame}

\begin{frame}{Backtraces}{Scanning the stack with \typename{frame\_descr}}
  \begin{columns}[c]
    \begin{column}{0.5\textwidth}
      \begin{itemize}
        \item Given the return address of the code that raises the exception by calling \funcname{caml\_raise\_exn} (\funcarg{pc} argument of \funcname{caml\_stash\_backtrace})
        \item And the position of the stack pointer (\funcarg{sp} argument of \funcname{caml\_stash\_backtrace})
        \item The frame size given by \typename{frame\_descr}.\funcarg{frame\_size}, allows to find the end of the previous frame
        \item And the return address of the caller of this function
        \bigskip
        \onslide<3->{
        \item Note that traps (here the reddish \funcname{ExnA}, \funcname{ExnB} and \funcname{ExnC} blocks) belong to the frame that installed them, and so the frame size changes when traps are removed
        }
      \end{itemize}
    \end{column}
    \begin{column}{0.5\textwidth}
      \raggedleft
      \onslide*<1>{\providecommand\step{2}\input{diagrams/backtrace/backtrace.tex}}%
      \onslide*<2>{\providecommand\step{1}\input{diagrams/backtrace/scanstack.tex}}%
      \onslide*<3>{\providecommand\step{2}\input{diagrams/backtrace/scanstack.tex}}%
      \onslide*<4>{\providecommand\step{3}\input{diagrams/backtrace/scanstack.tex}}%
      \onslide*<5>{\providecommand\step{4}\input{diagrams/backtrace/scanstack.tex}}%
      \onslide*<6>{\providecommand\step{5}\input{diagrams/backtrace/scanstack.tex}}%
    \end{column}
  \end{columns}
\end{frame}

% Duplicate of previous-previous slide
\begin{frame}{Backtraces}{Continuing on \funcname{caml\_stash\_backtrace}}
  \begin{columns}[c]
    \begin{column}{0.5\textwidth}
      \minipage[c][0.75\textheight][s]{\columnwidth}
      \begin{itemize}
          \color{gray}
        \item A C function provided by the runtime (specific to native code)
        \item It scans the stack, between the stack pointer at the raising point up to the next trap, in order to collect the names of the detected function frames
        \item It uses the same technique and information as the Garbage Collector (GC) uses to scan the values live on the stack
      \end{itemize}
      \begin{itemize}
        \item For each function frame detected, store a pointer to the \typename{frame\_descr} in the \localname{domain\_state->backtrace\_buffer} array, effectively constructing the backtrace
      \end{itemize}
      \onslide<2->{
        \begin{itemize}
            \smallskip
          \item Constructed backtrace:
            \funcname{definitely\_raise},
            \onslide<3->{\funcname{probably\_raise}}
        \end{itemize}
      }
      \vfill
      \mintocaml[firstline=9,lastline=13,highlightlines=12]{../src/backtrace/backtrace.ml}
      \endminipage
    \end{column}
    \begin{column}{0.5\textwidth}
      \raggedleft
      \onslide*<1>{\listStashBacktrace[highlightlines={114-115}]{}}
      \onslide*<2>{\providecommand\step{3}\input{diagrams/backtrace/backtrace.tex}}%
      \onslide*<3>{\providecommand\step{4}\input{diagrams/backtrace/backtrace.tex}}%
    \end{column}
  \end{columns}
\end{frame}

\begin{frame}{Backtraces}{Back to \funcname{caml\_raise\_exn}}
  \begin{columns}[c]
    \begin{column}{0.5\textwidth}
      \minipage[c][0.66\textheight][s]{\columnwidth}
      \begin{itemize}\color{gray}
        \item Checks wheteher exceptions backtrace should be recorded, by checking the \funcarg{domain\_state->backtrace\_active} value
        \item Switches to the C stack to be able to execute C code
        \item Calls \funcname{caml\_stash\_backtrace}, to collect the backtrace up to the next trap
      \end{itemize}
      \onslide*<2->{
        \begin{itemize}
          \item<2-> Executes the exception handler in \funcname{probably\_raise} with \funcname{RESTORE\_EXN\_HANDLER\_OCAML}
            \begin{itemize}
              \item Set \regname{RSP} to \localname{domain\_state->exn\_handler} value
              \item Restore \localname{domain\_state->exn\_handler} from trap
              \item Jump to the handler code with \funcname{ret}
            \end{itemize}
        \end{itemize}
      }
      \vfill
      \onslide*<1>{\mintocaml[firstline=9,lastline=13,highlightlines=12]{../src/backtrace/backtrace.ml}}
      \onslide*<2->{\mintocaml[firstline=15,lastline=21,highlightlines={19-21}]{../src/backtrace/backtrace.ml}}
      \endminipage
    \end{column}
    \begin{column}{0.5\textwidth}
      \centering
      \onslide*<1>{
        \mintc[firstline=190,lastline=192]{../ocaml/runtime/amd64.S}
        \mintc[firstline=234,lastline=237]{../ocaml/runtime/amd64.S}
      }
      \onslide*<2>{
        \mintc[firstline=190,lastline=192,highlightlines={191-192}]{../ocaml/runtime/amd64.S}
        \mintc[firstline=234,lastline=237,highlightlines={235-237}]{../ocaml/runtime/amd64.S}
      }
      \onslide*<1>{\listCamlRaiseExn[highlightlines=720]{}}
      \onslide*<2>{\listCamlRaiseExn[highlightlines=722]{}}
      \onslide*<3->{\providecommand\step{5}\input{diagrams/backtrace/backtrace.tex}}
    \end{column}
  \end{columns}
\end{frame}

\begin{frame}{Backtraces}{\funcname{caml\_probably\_raise}'s handler doesn't catch \typename{ExnC}}
  \begin{columns}[c]
    \begin{column}{0.45\textwidth}
      \minipage[c][0.75\textheight][s]{\columnwidth}
      \begin{itemize}
        \item<1-> \funcname{match \dots with}\footnotemark pattern matching can also match exceptions
        \item<1-> The CMM of \funcname{probably\_raise} shows that this \funcname{match \dots with} is implemented with a pattern matching and a \funcname{try \dots with}
        \item<1-> But this one only matches on \typename{ExnA} exception
        \item<2-> Since the pattern matching fails to catch the exception, \funcname{reraise} is called with our current \typename{ExnC} exception
        \item<3-> Disassembly shows that \funcname{reraise} is actually a call to \funcname{caml\_reraise\_exn}, which is also an assembly function provided by the runtime
      \end{itemize}
      \vfill
      \mintocaml[firstline=15,lastline=21,highlightlines={18-21}]{../src/backtrace/backtrace.ml}
      \endminipage
    \end{column}
    \begin{column}{0.55\textwidth}
      \centering
      \onslide*<1>{\mintcmm[highlightlines={10-18}]{../src/backtrace/camlBacktrace\_\_probably\_raise\_351.cmm}}
      \onslide*<2>{\listcmm[highlightlines=18]{../src/backtrace/camlBacktrace\_\_probably\_raise_351.cmm}{CMM of probably\_raise}}
      \onslide*<3>{\mintobjdump[firstline=5,lastline=35,highlightlines=31]{../src/backtrace/camlBacktrace\_\_probably\_raise_351.objdump}}
    \end{column}
  \end{columns}
  \only<1>{\footnotetext[1]{https://blog.janestreet.com/pattern-matching-and-exception-handling-unite/}}
\end{frame}

\newcommand\listCamlReraiseExn[1][]{
  \listasm[firstline=727,lastline=734,#1]{../ocaml/runtime/amd64.S}{runtime/amd64.S:727}}

\begin{frame}{Backtraces}{Introducing \funcname{caml\_reraise\_exn}}
  \begin{columns}[c]
    \begin{column}{0.5\textwidth}
      \minipage[c][0.66\textheight][s]{\columnwidth}
      \begin{itemize}
        \item<1-> The exception have to be reraised to the next handler
        \item<2-> First, checks if the backtrace should be collected with \funcarg{domain\_state->backtrace\_active}
        \only<3>{
        \begin{itemize}
            \item If not, behaves like a \funcname{raise\_notrace} with \funcname{RESTORE\_EXN\_HANDLER\_OCAML}
        \end{itemize}
        }
        \item<4-> Jumps into \funcname{caml\_raise\_exn}
        \item<5-> Calls \funcname{caml\_stash\_backtrace} again, to scan the next part of the stack: from the trap just pop-ed to the next trap
      \end{itemize}
      \vfill
      \mintocaml[firstline=15,lastline=21,highlightlines={18-21}]{../src/backtrace/backtrace.ml}
      \endminipage
    \end{column}
    \begin{column}{0.5\textwidth}
      \raggedleft
      \onslide*<1>{\listCamlReraiseExn{}}
      \onslide*<2>{\listCamlReraiseExn[highlightlines=729]{}}
      \onslide*<3>{
        \mintc[firstline=190,lastline=192,highlightlines={191-192}]{../ocaml/runtime/amd64.S}
        \mintc[firstline=234,lastline=237,highlightlines={235-237}]{../ocaml/runtime/amd64.S}
        \medskip
        \listCamlReraiseExn[highlightlines=731]{}
      }
      \onslide*<4-5>{
        % TODO refactor with \listCamlReraiseExn
        \mintasm[firstline=727,lastline=734,highlightlines=730]{../ocaml/runtime/amd64.S}
        \medskip
      }
      \onslide*<4>{\listCamlRaiseExn[highlightlines=711]{}}
      \onslide*<5>{\listCamlRaiseExn[highlightlines=720]{}}
    \end{column}
  \end{columns}
\end{frame}

\begin{frame}{Backtraces}{Backtrace collection continues}
  \begin{columns}[c]
    \begin{column}{0.55\textwidth}
      \minipage[c][0.75\textheight][s]{\columnwidth}
      \begin{itemize}
        \item<1-> \funcname{caml\_stash\_backtrace} scans the older part of the stack, up to the next trap
        \item<6-> Controls jumps to the nested exception handler in \funcname{maybe\_raise}, which matches only on \typename{ExnB}
        \item<7-> \funcname{caml\_reraise\_exn} is called again, backtrace collection resumes with \funcname{caml\_stash\_backtrace}
      \end{itemize}
      \medskip
      \begin{itemize}
        \item<2-> Constructed backtrace:
        \begin{itemize}
          \item <2-> \funcname{definitely\_raise}
          \item <2-> \funcname{probably\_raise}
          \item <3-> \funcname{probably\_raise} (re-raise)
          \item <4-> \funcname{maybe\_raise}
          \item <5-> \funcname{main}
          \item <8-> \funcname{main} (re-raise)
        \end{itemize}
      \end{itemize}
      \vfill
      \onslide*<1-5>{\mintocaml[firstline=15,lastline=21]{../src/backtrace/backtrace.ml}}
      \onslide*<6>{\mintocaml[firstline=28,lastline=34,highlightlines=31]{../src/backtrace/backtrace.ml}}
      \onslide*<7->{\mintocaml[firstline=28,lastline=34,highlightlines=33]{../src/backtrace/backtrace.ml}}
      \endminipage
    \end{column}
    \begin{column}{0.45\textwidth}
      \raggedleft
      \onslide*<1>{\providecommand\step{6}\input{diagrams/backtrace/backtrace.tex}}%
      \onslide*<2>{\providecommand\step{6}\input{diagrams/backtrace/backtrace.tex}}%
      \onslide*<3>{\providecommand\step{7}\input{diagrams/backtrace/backtrace.tex}}%
      \onslide*<4>{\providecommand\step{8}\input{diagrams/backtrace/backtrace.tex}}%
      \onslide*<5>{\providecommand\step{9}\input{diagrams/backtrace/backtrace.tex}}%
      \onslide*<6>{\providecommand\step{10}\input{diagrams/backtrace/backtrace.tex}}%
      \onslide*<7>{\providecommand\step{11}\input{diagrams/backtrace/backtrace.tex}}%
      \onslide*<8>{\providecommand\step{12}\input{diagrams/backtrace/backtrace.tex}}%
      \onslide*<9>{\providecommand\step{13}\input{diagrams/backtrace/backtrace.tex}}%
    \end{column}
  \end{columns}
\end{frame}

\begin{frame}{Backtraces}{Printing the backtrace}
  \begin{columns}[c]
    \begin{column}{0.45\textwidth}
      \minipage[c][0.66\textheight][s]{\columnwidth}
      \begin{itemize}
        \item<1-> The exception backtrace can now be printed to \funcarg{stdout} with \funcname{Printexc.print_backtrace}
        \item<2-> First the exception backtrace is retrieved into an OCaml array with \funcname{get_raw_backtrace }
        \item<3-> Which is implemented as the \funcname{caml_get_exception_raw_backtrace} C function in the runtime
        \item<4-> And finally printed with \funcname{print_exception_backtrace}
      \end{itemize}
      \vfill
      \mintocaml[firstline=28,lastline=34,highlightlines=33]{../src/backtrace/backtrace.ml}
      \endminipage
    \end{column}
    \begin{column}{0.55\textwidth}
      \onslide*<1,2>{
        \mintocaml[firstline=100,lastline=101]{../ocaml/stdlib/printexc.ml}
      }
      \only<1>{
        \listocaml[firstline=170,lastline=172]{../ocaml/stdlib/printexc.ml}{stdlib/printexc.ml:170}
      }
      \only<2>{
        \listocaml[firstline=170,lastline=172,highlightlines=172]{../ocaml/stdlib/printexc.ml}{stdlib/printexc.ml:170}
      }
      \only<3>{
        \mintc[firstline=154,lastline=156]{../ocaml/runtime/backtrace.c}
        \listc[firstline=171,lastline=192,highlightlines={184-187}]{../ocaml/runtime/backtrace.c}{runtime/backtrace.c:154}
      }
      \only<4>{
        \listocaml[firstline=155,lastline=165]{../ocaml/stdlib/printexc.ml}{stdlib/printexc.ml:55}
      }
    \end{column}
  \end{columns}
\end{frame}


\subsection{The multiple kinds of raise}
\frameSubsection{}{}

\begin{frame}{Backtraces}{When backtraces are not needed}
  \begin{columns}[c]
    \begin{column}{0.45\textwidth}
      \minipage[c][0.66\textheight][s]{\columnwidth}
      \begin{itemize}
        \item<1-> What if the backtrace for an exception is never needed?
        \item<2-> It's possible to raise the exception explicitly with \funcname{raise\_notrace} to make raising as cheap as possible
        \item<3-> An example of use in the \funcname{dynlink} library of the OCaml compiler
      \end{itemize}
      \vfill
      \onslide<3->{
      \listocaml[firstline=205,lastline=209,highlightlines=207]{../ocaml/otherlibs/dynlink/dynlink\_compilerlibs/misc.ml}{otherlibs/dynlink/dynlink\_compilerlibs/misc.ml:205}
      }
      \endminipage
    \end{column}
    \begin{column}{0.55\textwidth}
    \only<4>{
      \mintcmm[highlightlines=3]{../src/notrace/camlNotrace\_\_fun_347.cmm}
      \listcmm{../src/notrace/camlNotrace\_\_all\_somes\_327.cmm}{all\_somes}
    }
    \onslide*<5->{
      \listobjdump[highlightlines={6-9}]{../src/notrace/camlNotrace\_\_fun_347.objdump}{anonymous function from \funcname{all\_somes}}
    }
    \end{column}
  \end{columns}
\end{frame}

\begin{frame}{Backtraces}{Summary table of the multiple kinds of raise}
  \begin{itemize}
    \item When debugging information is enabled and if the backtrace of an exception isn't needed, it's possible to raise with \funcname{raise\_notrace} for performance
    \item When debugging information is \emph{not} enabled, the compiler optimises \funcname{raise} calls to inline assembly (same effect as using \funcname{raise\_notrace})
  \end{itemize}
  \bigskip
  \centering
  \begin{tabular}{| l | c | c | c | c |}
    \hline
    \parbox{10em}{Debug information \\ (\funcarg{-g} flag)}  & \multicolumn{2}{|c|}{Disabled}                                     & \multicolumn{2}{|c|}{Enabled} \\
    \hline
    Raise kind                                               & \funcname{raise} & \funcname{raise\_notrace}                       & \funcname{raise} & \funcname{raise\_notrace} \\
    \hline
    Compilation                                              & \parbox{8em}{inline assembly\\ (optimization)} & inline assembly   & \funcname{caml\_raise\_exn} & inline assembly \\
    \hline
  \end{tabular}
\end{frame}


%
% Takeaway #5
%
\subsection*{Takeaway \#5}
\frameSubsectionTakeaway{}
\againframe<5>{takeaway}
}%
      \onslide*<5>{\providecommand\step{9}%
% Backtraces
%
\subsection{One advantage of raising with debugging information}
\frameSubsection{}{}

\begin{frame}{Backtraces}{Debugging information \& source code}
  \begin{columns}[c]
    \begin{column}{0.5\textwidth}
      \begin{itemize}
        \item Raising an exception is fun and games, but how do we know where the exception originated? $\rightarrow$ With the backtrace associated with the exception!
        \item In order to get a backtrace from the point where an exception was raised, it's necessary to compile with debugging information enabled (\funcarg{-g} flag for the compiler)
        \item Same example as in the `Nested handlers' section
      \end{itemize}
    \end{column}
    \begin{column}{0.5\textwidth}
      \listocaml[firstline=9,lastline=34]{../src/backtrace/backtrace.ml}{backtrace/backtrace.ml}
    \end{column}
  \end{columns}
\end{frame}

\begin{frame}{Backtraces}{CMM and disassembly of \funcname{definitely\_raise}}
  \begin{columns}[c]
    \begin{column}{0.5\textwidth}
      \minipage[c][0.66\textheight][s]{\columnwidth}
      \begin{itemize}
        \item<1-> An effect of compiling with debugging information (\funcarg{-g}): the CMM no longer uses \funcname{raise\_notrace} to raise the exception but \funcname{raise} instead
        \item<2-> Disassembly shows that CMM \funcname{raise} is actually a call to \funcname{caml\_raise\_exn}, a function in assembler provided by the runtime
      \end{itemize}
      \vfill
      \mintocaml[firstline=9,lastline=13,highlightlines=12]{../src/backtrace/backtrace.ml}
      \endminipage
    \end{column}
    \begin{column}{0.5\textwidth}
      \onslide*<1>{
        \mintcmm[highlightlines=5]{../src/backtrace/camlBacktrace\_\_definitely\_raise\_347.cmm}
      }
      \onslide*<2>{
        \mintobjdump[firstline=2,highlightlines=14]{../src/backtrace/camlBacktrace\_\_definitely\_raise\_347.objdump}
      }
    \end{column}
  \end{columns}
\end{frame}

\begin{frame}{Backtraces}{Introducing \funcname{caml\_raise\_exn}}
  \begin{columns}[c]
    \begin{column}{0.5\textwidth}
      \minipage[c][0.66\textheight][s]{\columnwidth}
      \onslide*<1>{
        \begin{itemize}
          \item Raising the exception is performed using \funcname{caml\_raise\_exn} which is
            \begin{itemize}
              \item An assembly function provided by the runtime
              \item Architecture specific
            \end{itemize}
          \item Let's see what it does differently from \funcname{raise\_notrace}\dots
          \item We are about to raise \typename{ExnC} with \funcname{caml\_raise\_exn} from \funcname{definitely\_raise}
        \end{itemize}
      }
      \onslide*<2>{
        \mintobjdump[firstline=2,highlightlines=14]{../src/backtrace/camlBacktrace\_\_definitely\_raise\_347.objdump}
      }
      \vfill
      \mintocaml[firstline=9,lastline=13,highlightlines=12]{../src/backtrace/backtrace.ml}
      \endminipage
    \end{column}
    \begin{column}{0.5\textwidth}
      \centering
      \providecommand\step{1}\input{diagrams/backtrace/backtrace.tex}
    \end{column}
  \end{columns}
\end{frame}

\begin{frame}{Backtraces}{But before that, we need more fields from \typename{caml\_domain\_state}}
  \begin{columns}
    \begin{column}{0.5\textwidth}
      \begin{itemize}
        \item Remember \typename{caml\_domain\_state} the per-domain runtime state?
        \item Some other members are of interest to us now\dots
        \item \localname{backtrace\_active} is used to know if the runtime should collect backtraces
        \item \localname{backtrace\_active} can be set by
          \begin{itemize}
            \item The \funcarg{b} flag in the \funcname{OCAMLRUNPARAM} environment variable
            \item \mintinline{ocaml}{Printexc.record_backtrace : bool -> unit} \footnotemark
          \end{itemize}
      \end{itemize}
    \end{column}
    \begin{column}{0.5\textwidth}
      \begin{listing}
        \mintc[highlightlines={9-11}]{src/ocaml/domain\_state\_backtrace.h}
        \caption{runtime/caml/domain\_state.h}
      \end{listing}
    \end{column}
  \end{columns}
  \footnotetext[1]{https://v2.ocaml.org/api/Printexc.html}
\end{frame}

\newcommand\listCamlRaiseExn[1][]{
  \listasm[firstline=702,lastline=725,#1]{../ocaml/runtime/amd64.S}{runtime/amd64.S:702}}

\begin{frame}{Backtraces}{Walk through \funcname{caml\_raise\_exn}}
  \begin{columns}[T]
    \begin{column}{0.5\textwidth}
      \begin{itemize}
        \item<1-> First checks whether exceptions backtrace should be recorded
        \item<1-> By checking the \funcarg{domain\_state->backtrace\_active} value
        \item<2-> If not, acts as \funcname{raise\_notrace} with \funcname{RESTORE\_EXN\_HANDLER\_OCAML}
          \begin{itemize}
            \item Set \regname{RSP} to \localname{domain\_state->exn\_handler} value
            \item Restore \localname{domain\_state->exn\_handler} from trap
            \item Jump with \funcname{ret} (same effect as using \funcname{pop} and \funcname{jmp})
          \end{itemize}
        \item<3-> Otherwise, switches to the C stack to be able to execute C code
        \item<4-> Calls \funcname{caml\_stash\_backtrace}, a C function provided by the runtime\ignorespaces
          \onslide<6->{, with
            \begin{itemize}
              \item \funcarg{exn}: exception value
              \item \funcarg{pc}: code address of the raising point
              \item \funcarg{sp}: stack address of the raising function
              \item \funcarg{trapsp}: stack address of the next handler
            \end{itemize}
          }
      \end{itemize}
      \bigskip
      \mintocaml[firstline=9,lastline=13,highlightlines=12]{../src/backtrace/backtrace.ml}
    \end{column}
    \begin{column}{0.5\textwidth}
      \raggedleft
      \onslide*<1,3-4>{
        \mintc[firstline=190,lastline=192]{../ocaml/runtime/amd64.S}
        \mintc[firstline=234,lastline=237]{../ocaml/runtime/amd64.S}
      }
      \onslide*<2>{
        \mintc[firstline=190,lastline=192,highlightlines={191-192}]{../ocaml/runtime/amd64.S}
        \mintc[firstline=234,lastline=237,highlightlines={235-237}]{../ocaml/runtime/amd64.S}
      }
      \onslide*<1>{\listCamlRaiseExn[highlightlines=705]{}}%
      \onslide*<2>{\listCamlRaiseExn[highlightlines={707-708}]{}}%
      \onslide*<3>{\listCamlRaiseExn[highlightlines={712-714}]{}}%
      \onslide*<4>{\listCamlRaiseExn[highlightlines={715-720}]{}}%
      \onslide*<5>{\providecommand\step{1}\input{diagrams/backtrace/backtrace.tex}}%
      \onslide*<6>{\providecommand\step{2}\input{diagrams/backtrace/backtrace.tex}}%
    \end{column}
  \end{columns}
\end{frame}

\newcommand\listStashBacktrace[1][]{
  \listc[firstline=91,lastline=120,#1]{../ocaml/runtime/backtrace\_nat.c}{runtime/backtrace\_nat.c:91}}

\begin{frame}{Backtraces}{Introducing \funcname{caml\_stash\_backtrace}}
  \begin{columns}[c]
    \begin{column}{0.5\textwidth}
      \minipage[c][0.66\textheight][s]{\columnwidth}
      \begin{itemize}
        \item<1-> A C function provided by the runtime (specific to native code)
        \item<2-> It scans the stack, between the stack pointer at the raising point up to the next trap, in order to collect the names of the detected function frames
          \onslide*<2>{
            \begin{itemize}
              \item And more information, like line number, column number...
            \end{itemize}
          }
        \item<3-> It uses the same technique and information as the Garbage Collector (GC) uses to scan the values live on the stack
          \onslide*<4>{
            \begin{itemize}
              \item \typename{frame\_descr} are at the core of stack unwinding
            \end{itemize}
          }
      \end{itemize}
      \vfill
      \mintocaml[firstline=9,lastline=13,highlightlines=12]{../src/backtrace/backtrace.ml}
      \endminipage
    \end{column}
    \begin{column}{0.5\textwidth}
      \onslide*<1>{\listStashBacktrace[highlightlines=91]{}}
      \onslide*<2>{\listStashBacktrace[highlightlines={91,117-118}]{}}
      \onslide*<3->{\listStashBacktrace[highlightlines={94,106,109-110}]{}}
    \end{column}
  \end{columns}
\end{frame}

\newcommand\listFrameDescr[1][]{
  \listocaml[firstline=111,lastline=124,#1]{../ocaml/asmcomp/emitaux.ml}{asmcomp/emitaux.ml:111}}
\newcommand\listDebugInfo[1][]{
  \listocaml[firstline=38,lastline=49,#1]{../ocaml/lambda/debuginfo.mli}{lambda/debuginfo.mli:38}}

\begin{frame}{Backtraces}{\typename{frame\_descr} and \typename{frame\_debuginfo} in a nutshell}
  \begin{columns}[c]
    \begin{column}{0.5\textwidth}
      \begin{itemize}
        \item<1-> For every return address in an OCaml program, there is a associated \typename{frame\_descr}
        \item<2-> A \typename{frame\_descr} specifies
        \begin{itemize}
          \item<2-> The size of the function stack frame
          \item<3-> Where live values are on the stack (so that the GC can mark them as live and prevent from being swept)
        \end{itemize}
        \item<4-> When compiled with debug information, \typename{frame\_descr} will be linked to a \typename{frame\_debuginfo}
        \begin{itemize}
          \item<5-> Contains information about the position in the source code (file name, line and column number)
        \end{itemize}
      \end{itemize}
    \end{column}
    \begin{column}{0.5\textwidth}
        \onslide*<1>{\listFrameDescr[highlightlines={118-122}]{}}
        \onslide*<2>{\listFrameDescr[highlightlines={120}]{}}
        \onslide*<3>{\listFrameDescr[highlightlines={121}]{}}
        \onslide*<4->{\listFrameDescr[highlightlines={122}]{}}
        \medskip
        \onslide*<1-3>{\listDebugInfo{}}
        \onslide*<4>{\listDebugInfo[highlightlines={38,49}]{}}
        \onslide*<5>{\listDebugInfo[highlightlines={39-46}]{}}
    \end{column}
  \end{columns}
\end{frame}

\begin{frame}{Backtraces}{Scanning the stack with \typename{frame\_descr}}
  \begin{columns}[c]
    \begin{column}{0.5\textwidth}
      \begin{itemize}
        \item Given the return address of the code that raises the exception by calling \funcname{caml\_raise\_exn} (\funcarg{pc} argument of \funcname{caml\_stash\_backtrace})
        \item And the position of the stack pointer (\funcarg{sp} argument of \funcname{caml\_stash\_backtrace})
        \item The frame size given by \typename{frame\_descr}.\funcarg{frame\_size}, allows to find the end of the previous frame
        \item And the return address of the caller of this function
        \bigskip
        \onslide<3->{
        \item Note that traps (here the reddish \funcname{ExnA}, \funcname{ExnB} and \funcname{ExnC} blocks) belong to the frame that installed them, and so the frame size changes when traps are removed
        }
      \end{itemize}
    \end{column}
    \begin{column}{0.5\textwidth}
      \raggedleft
      \onslide*<1>{\providecommand\step{2}\input{diagrams/backtrace/backtrace.tex}}%
      \onslide*<2>{\providecommand\step{1}\input{diagrams/backtrace/scanstack.tex}}%
      \onslide*<3>{\providecommand\step{2}\input{diagrams/backtrace/scanstack.tex}}%
      \onslide*<4>{\providecommand\step{3}\input{diagrams/backtrace/scanstack.tex}}%
      \onslide*<5>{\providecommand\step{4}\input{diagrams/backtrace/scanstack.tex}}%
      \onslide*<6>{\providecommand\step{5}\input{diagrams/backtrace/scanstack.tex}}%
    \end{column}
  \end{columns}
\end{frame}

% Duplicate of previous-previous slide
\begin{frame}{Backtraces}{Continuing on \funcname{caml\_stash\_backtrace}}
  \begin{columns}[c]
    \begin{column}{0.5\textwidth}
      \minipage[c][0.75\textheight][s]{\columnwidth}
      \begin{itemize}
          \color{gray}
        \item A C function provided by the runtime (specific to native code)
        \item It scans the stack, between the stack pointer at the raising point up to the next trap, in order to collect the names of the detected function frames
        \item It uses the same technique and information as the Garbage Collector (GC) uses to scan the values live on the stack
      \end{itemize}
      \begin{itemize}
        \item For each function frame detected, store a pointer to the \typename{frame\_descr} in the \localname{domain\_state->backtrace\_buffer} array, effectively constructing the backtrace
      \end{itemize}
      \onslide<2->{
        \begin{itemize}
            \smallskip
          \item Constructed backtrace:
            \funcname{definitely\_raise},
            \onslide<3->{\funcname{probably\_raise}}
        \end{itemize}
      }
      \vfill
      \mintocaml[firstline=9,lastline=13,highlightlines=12]{../src/backtrace/backtrace.ml}
      \endminipage
    \end{column}
    \begin{column}{0.5\textwidth}
      \raggedleft
      \onslide*<1>{\listStashBacktrace[highlightlines={114-115}]{}}
      \onslide*<2>{\providecommand\step{3}\input{diagrams/backtrace/backtrace.tex}}%
      \onslide*<3>{\providecommand\step{4}\input{diagrams/backtrace/backtrace.tex}}%
    \end{column}
  \end{columns}
\end{frame}

\begin{frame}{Backtraces}{Back to \funcname{caml\_raise\_exn}}
  \begin{columns}[c]
    \begin{column}{0.5\textwidth}
      \minipage[c][0.66\textheight][s]{\columnwidth}
      \begin{itemize}\color{gray}
        \item Checks wheteher exceptions backtrace should be recorded, by checking the \funcarg{domain\_state->backtrace\_active} value
        \item Switches to the C stack to be able to execute C code
        \item Calls \funcname{caml\_stash\_backtrace}, to collect the backtrace up to the next trap
      \end{itemize}
      \onslide*<2->{
        \begin{itemize}
          \item<2-> Executes the exception handler in \funcname{probably\_raise} with \funcname{RESTORE\_EXN\_HANDLER\_OCAML}
            \begin{itemize}
              \item Set \regname{RSP} to \localname{domain\_state->exn\_handler} value
              \item Restore \localname{domain\_state->exn\_handler} from trap
              \item Jump to the handler code with \funcname{ret}
            \end{itemize}
        \end{itemize}
      }
      \vfill
      \onslide*<1>{\mintocaml[firstline=9,lastline=13,highlightlines=12]{../src/backtrace/backtrace.ml}}
      \onslide*<2->{\mintocaml[firstline=15,lastline=21,highlightlines={19-21}]{../src/backtrace/backtrace.ml}}
      \endminipage
    \end{column}
    \begin{column}{0.5\textwidth}
      \centering
      \onslide*<1>{
        \mintc[firstline=190,lastline=192]{../ocaml/runtime/amd64.S}
        \mintc[firstline=234,lastline=237]{../ocaml/runtime/amd64.S}
      }
      \onslide*<2>{
        \mintc[firstline=190,lastline=192,highlightlines={191-192}]{../ocaml/runtime/amd64.S}
        \mintc[firstline=234,lastline=237,highlightlines={235-237}]{../ocaml/runtime/amd64.S}
      }
      \onslide*<1>{\listCamlRaiseExn[highlightlines=720]{}}
      \onslide*<2>{\listCamlRaiseExn[highlightlines=722]{}}
      \onslide*<3->{\providecommand\step{5}\input{diagrams/backtrace/backtrace.tex}}
    \end{column}
  \end{columns}
\end{frame}

\begin{frame}{Backtraces}{\funcname{caml\_probably\_raise}'s handler doesn't catch \typename{ExnC}}
  \begin{columns}[c]
    \begin{column}{0.45\textwidth}
      \minipage[c][0.75\textheight][s]{\columnwidth}
      \begin{itemize}
        \item<1-> \funcname{match \dots with}\footnotemark pattern matching can also match exceptions
        \item<1-> The CMM of \funcname{probably\_raise} shows that this \funcname{match \dots with} is implemented with a pattern matching and a \funcname{try \dots with}
        \item<1-> But this one only matches on \typename{ExnA} exception
        \item<2-> Since the pattern matching fails to catch the exception, \funcname{reraise} is called with our current \typename{ExnC} exception
        \item<3-> Disassembly shows that \funcname{reraise} is actually a call to \funcname{caml\_reraise\_exn}, which is also an assembly function provided by the runtime
      \end{itemize}
      \vfill
      \mintocaml[firstline=15,lastline=21,highlightlines={18-21}]{../src/backtrace/backtrace.ml}
      \endminipage
    \end{column}
    \begin{column}{0.55\textwidth}
      \centering
      \onslide*<1>{\mintcmm[highlightlines={10-18}]{../src/backtrace/camlBacktrace\_\_probably\_raise\_351.cmm}}
      \onslide*<2>{\listcmm[highlightlines=18]{../src/backtrace/camlBacktrace\_\_probably\_raise_351.cmm}{CMM of probably\_raise}}
      \onslide*<3>{\mintobjdump[firstline=5,lastline=35,highlightlines=31]{../src/backtrace/camlBacktrace\_\_probably\_raise_351.objdump}}
    \end{column}
  \end{columns}
  \only<1>{\footnotetext[1]{https://blog.janestreet.com/pattern-matching-and-exception-handling-unite/}}
\end{frame}

\newcommand\listCamlReraiseExn[1][]{
  \listasm[firstline=727,lastline=734,#1]{../ocaml/runtime/amd64.S}{runtime/amd64.S:727}}

\begin{frame}{Backtraces}{Introducing \funcname{caml\_reraise\_exn}}
  \begin{columns}[c]
    \begin{column}{0.5\textwidth}
      \minipage[c][0.66\textheight][s]{\columnwidth}
      \begin{itemize}
        \item<1-> The exception have to be reraised to the next handler
        \item<2-> First, checks if the backtrace should be collected with \funcarg{domain\_state->backtrace\_active}
        \only<3>{
        \begin{itemize}
            \item If not, behaves like a \funcname{raise\_notrace} with \funcname{RESTORE\_EXN\_HANDLER\_OCAML}
        \end{itemize}
        }
        \item<4-> Jumps into \funcname{caml\_raise\_exn}
        \item<5-> Calls \funcname{caml\_stash\_backtrace} again, to scan the next part of the stack: from the trap just pop-ed to the next trap
      \end{itemize}
      \vfill
      \mintocaml[firstline=15,lastline=21,highlightlines={18-21}]{../src/backtrace/backtrace.ml}
      \endminipage
    \end{column}
    \begin{column}{0.5\textwidth}
      \raggedleft
      \onslide*<1>{\listCamlReraiseExn{}}
      \onslide*<2>{\listCamlReraiseExn[highlightlines=729]{}}
      \onslide*<3>{
        \mintc[firstline=190,lastline=192,highlightlines={191-192}]{../ocaml/runtime/amd64.S}
        \mintc[firstline=234,lastline=237,highlightlines={235-237}]{../ocaml/runtime/amd64.S}
        \medskip
        \listCamlReraiseExn[highlightlines=731]{}
      }
      \onslide*<4-5>{
        % TODO refactor with \listCamlReraiseExn
        \mintasm[firstline=727,lastline=734,highlightlines=730]{../ocaml/runtime/amd64.S}
        \medskip
      }
      \onslide*<4>{\listCamlRaiseExn[highlightlines=711]{}}
      \onslide*<5>{\listCamlRaiseExn[highlightlines=720]{}}
    \end{column}
  \end{columns}
\end{frame}

\begin{frame}{Backtraces}{Backtrace collection continues}
  \begin{columns}[c]
    \begin{column}{0.55\textwidth}
      \minipage[c][0.75\textheight][s]{\columnwidth}
      \begin{itemize}
        \item<1-> \funcname{caml\_stash\_backtrace} scans the older part of the stack, up to the next trap
        \item<6-> Controls jumps to the nested exception handler in \funcname{maybe\_raise}, which matches only on \typename{ExnB}
        \item<7-> \funcname{caml\_reraise\_exn} is called again, backtrace collection resumes with \funcname{caml\_stash\_backtrace}
      \end{itemize}
      \medskip
      \begin{itemize}
        \item<2-> Constructed backtrace:
        \begin{itemize}
          \item <2-> \funcname{definitely\_raise}
          \item <2-> \funcname{probably\_raise}
          \item <3-> \funcname{probably\_raise} (re-raise)
          \item <4-> \funcname{maybe\_raise}
          \item <5-> \funcname{main}
          \item <8-> \funcname{main} (re-raise)
        \end{itemize}
      \end{itemize}
      \vfill
      \onslide*<1-5>{\mintocaml[firstline=15,lastline=21]{../src/backtrace/backtrace.ml}}
      \onslide*<6>{\mintocaml[firstline=28,lastline=34,highlightlines=31]{../src/backtrace/backtrace.ml}}
      \onslide*<7->{\mintocaml[firstline=28,lastline=34,highlightlines=33]{../src/backtrace/backtrace.ml}}
      \endminipage
    \end{column}
    \begin{column}{0.45\textwidth}
      \raggedleft
      \onslide*<1>{\providecommand\step{6}\input{diagrams/backtrace/backtrace.tex}}%
      \onslide*<2>{\providecommand\step{6}\input{diagrams/backtrace/backtrace.tex}}%
      \onslide*<3>{\providecommand\step{7}\input{diagrams/backtrace/backtrace.tex}}%
      \onslide*<4>{\providecommand\step{8}\input{diagrams/backtrace/backtrace.tex}}%
      \onslide*<5>{\providecommand\step{9}\input{diagrams/backtrace/backtrace.tex}}%
      \onslide*<6>{\providecommand\step{10}\input{diagrams/backtrace/backtrace.tex}}%
      \onslide*<7>{\providecommand\step{11}\input{diagrams/backtrace/backtrace.tex}}%
      \onslide*<8>{\providecommand\step{12}\input{diagrams/backtrace/backtrace.tex}}%
      \onslide*<9>{\providecommand\step{13}\input{diagrams/backtrace/backtrace.tex}}%
    \end{column}
  \end{columns}
\end{frame}

\begin{frame}{Backtraces}{Printing the backtrace}
  \begin{columns}[c]
    \begin{column}{0.45\textwidth}
      \minipage[c][0.66\textheight][s]{\columnwidth}
      \begin{itemize}
        \item<1-> The exception backtrace can now be printed to \funcarg{stdout} with \funcname{Printexc.print_backtrace}
        \item<2-> First the exception backtrace is retrieved into an OCaml array with \funcname{get_raw_backtrace }
        \item<3-> Which is implemented as the \funcname{caml_get_exception_raw_backtrace} C function in the runtime
        \item<4-> And finally printed with \funcname{print_exception_backtrace}
      \end{itemize}
      \vfill
      \mintocaml[firstline=28,lastline=34,highlightlines=33]{../src/backtrace/backtrace.ml}
      \endminipage
    \end{column}
    \begin{column}{0.55\textwidth}
      \onslide*<1,2>{
        \mintocaml[firstline=100,lastline=101]{../ocaml/stdlib/printexc.ml}
      }
      \only<1>{
        \listocaml[firstline=170,lastline=172]{../ocaml/stdlib/printexc.ml}{stdlib/printexc.ml:170}
      }
      \only<2>{
        \listocaml[firstline=170,lastline=172,highlightlines=172]{../ocaml/stdlib/printexc.ml}{stdlib/printexc.ml:170}
      }
      \only<3>{
        \mintc[firstline=154,lastline=156]{../ocaml/runtime/backtrace.c}
        \listc[firstline=171,lastline=192,highlightlines={184-187}]{../ocaml/runtime/backtrace.c}{runtime/backtrace.c:154}
      }
      \only<4>{
        \listocaml[firstline=155,lastline=165]{../ocaml/stdlib/printexc.ml}{stdlib/printexc.ml:55}
      }
    \end{column}
  \end{columns}
\end{frame}


\subsection{The multiple kinds of raise}
\frameSubsection{}{}

\begin{frame}{Backtraces}{When backtraces are not needed}
  \begin{columns}[c]
    \begin{column}{0.45\textwidth}
      \minipage[c][0.66\textheight][s]{\columnwidth}
      \begin{itemize}
        \item<1-> What if the backtrace for an exception is never needed?
        \item<2-> It's possible to raise the exception explicitly with \funcname{raise\_notrace} to make raising as cheap as possible
        \item<3-> An example of use in the \funcname{dynlink} library of the OCaml compiler
      \end{itemize}
      \vfill
      \onslide<3->{
      \listocaml[firstline=205,lastline=209,highlightlines=207]{../ocaml/otherlibs/dynlink/dynlink\_compilerlibs/misc.ml}{otherlibs/dynlink/dynlink\_compilerlibs/misc.ml:205}
      }
      \endminipage
    \end{column}
    \begin{column}{0.55\textwidth}
    \only<4>{
      \mintcmm[highlightlines=3]{../src/notrace/camlNotrace\_\_fun_347.cmm}
      \listcmm{../src/notrace/camlNotrace\_\_all\_somes\_327.cmm}{all\_somes}
    }
    \onslide*<5->{
      \listobjdump[highlightlines={6-9}]{../src/notrace/camlNotrace\_\_fun_347.objdump}{anonymous function from \funcname{all\_somes}}
    }
    \end{column}
  \end{columns}
\end{frame}

\begin{frame}{Backtraces}{Summary table of the multiple kinds of raise}
  \begin{itemize}
    \item When debugging information is enabled and if the backtrace of an exception isn't needed, it's possible to raise with \funcname{raise\_notrace} for performance
    \item When debugging information is \emph{not} enabled, the compiler optimises \funcname{raise} calls to inline assembly (same effect as using \funcname{raise\_notrace})
  \end{itemize}
  \bigskip
  \centering
  \begin{tabular}{| l | c | c | c | c |}
    \hline
    \parbox{10em}{Debug information \\ (\funcarg{-g} flag)}  & \multicolumn{2}{|c|}{Disabled}                                     & \multicolumn{2}{|c|}{Enabled} \\
    \hline
    Raise kind                                               & \funcname{raise} & \funcname{raise\_notrace}                       & \funcname{raise} & \funcname{raise\_notrace} \\
    \hline
    Compilation                                              & \parbox{8em}{inline assembly\\ (optimization)} & inline assembly   & \funcname{caml\_raise\_exn} & inline assembly \\
    \hline
  \end{tabular}
\end{frame}


%
% Takeaway #5
%
\subsection*{Takeaway \#5}
\frameSubsectionTakeaway{}
\againframe<5>{takeaway}
}%
      \onslide*<6>{\providecommand\step{10}%
% Backtraces
%
\subsection{One advantage of raising with debugging information}
\frameSubsection{}{}

\begin{frame}{Backtraces}{Debugging information \& source code}
  \begin{columns}[c]
    \begin{column}{0.5\textwidth}
      \begin{itemize}
        \item Raising an exception is fun and games, but how do we know where the exception originated? $\rightarrow$ With the backtrace associated with the exception!
        \item In order to get a backtrace from the point where an exception was raised, it's necessary to compile with debugging information enabled (\funcarg{-g} flag for the compiler)
        \item Same example as in the `Nested handlers' section
      \end{itemize}
    \end{column}
    \begin{column}{0.5\textwidth}
      \listocaml[firstline=9,lastline=34]{../src/backtrace/backtrace.ml}{backtrace/backtrace.ml}
    \end{column}
  \end{columns}
\end{frame}

\begin{frame}{Backtraces}{CMM and disassembly of \funcname{definitely\_raise}}
  \begin{columns}[c]
    \begin{column}{0.5\textwidth}
      \minipage[c][0.66\textheight][s]{\columnwidth}
      \begin{itemize}
        \item<1-> An effect of compiling with debugging information (\funcarg{-g}): the CMM no longer uses \funcname{raise\_notrace} to raise the exception but \funcname{raise} instead
        \item<2-> Disassembly shows that CMM \funcname{raise} is actually a call to \funcname{caml\_raise\_exn}, a function in assembler provided by the runtime
      \end{itemize}
      \vfill
      \mintocaml[firstline=9,lastline=13,highlightlines=12]{../src/backtrace/backtrace.ml}
      \endminipage
    \end{column}
    \begin{column}{0.5\textwidth}
      \onslide*<1>{
        \mintcmm[highlightlines=5]{../src/backtrace/camlBacktrace\_\_definitely\_raise\_347.cmm}
      }
      \onslide*<2>{
        \mintobjdump[firstline=2,highlightlines=14]{../src/backtrace/camlBacktrace\_\_definitely\_raise\_347.objdump}
      }
    \end{column}
  \end{columns}
\end{frame}

\begin{frame}{Backtraces}{Introducing \funcname{caml\_raise\_exn}}
  \begin{columns}[c]
    \begin{column}{0.5\textwidth}
      \minipage[c][0.66\textheight][s]{\columnwidth}
      \onslide*<1>{
        \begin{itemize}
          \item Raising the exception is performed using \funcname{caml\_raise\_exn} which is
            \begin{itemize}
              \item An assembly function provided by the runtime
              \item Architecture specific
            \end{itemize}
          \item Let's see what it does differently from \funcname{raise\_notrace}\dots
          \item We are about to raise \typename{ExnC} with \funcname{caml\_raise\_exn} from \funcname{definitely\_raise}
        \end{itemize}
      }
      \onslide*<2>{
        \mintobjdump[firstline=2,highlightlines=14]{../src/backtrace/camlBacktrace\_\_definitely\_raise\_347.objdump}
      }
      \vfill
      \mintocaml[firstline=9,lastline=13,highlightlines=12]{../src/backtrace/backtrace.ml}
      \endminipage
    \end{column}
    \begin{column}{0.5\textwidth}
      \centering
      \providecommand\step{1}\input{diagrams/backtrace/backtrace.tex}
    \end{column}
  \end{columns}
\end{frame}

\begin{frame}{Backtraces}{But before that, we need more fields from \typename{caml\_domain\_state}}
  \begin{columns}
    \begin{column}{0.5\textwidth}
      \begin{itemize}
        \item Remember \typename{caml\_domain\_state} the per-domain runtime state?
        \item Some other members are of interest to us now\dots
        \item \localname{backtrace\_active} is used to know if the runtime should collect backtraces
        \item \localname{backtrace\_active} can be set by
          \begin{itemize}
            \item The \funcarg{b} flag in the \funcname{OCAMLRUNPARAM} environment variable
            \item \mintinline{ocaml}{Printexc.record_backtrace : bool -> unit} \footnotemark
          \end{itemize}
      \end{itemize}
    \end{column}
    \begin{column}{0.5\textwidth}
      \begin{listing}
        \mintc[highlightlines={9-11}]{src/ocaml/domain\_state\_backtrace.h}
        \caption{runtime/caml/domain\_state.h}
      \end{listing}
    \end{column}
  \end{columns}
  \footnotetext[1]{https://v2.ocaml.org/api/Printexc.html}
\end{frame}

\newcommand\listCamlRaiseExn[1][]{
  \listasm[firstline=702,lastline=725,#1]{../ocaml/runtime/amd64.S}{runtime/amd64.S:702}}

\begin{frame}{Backtraces}{Walk through \funcname{caml\_raise\_exn}}
  \begin{columns}[T]
    \begin{column}{0.5\textwidth}
      \begin{itemize}
        \item<1-> First checks whether exceptions backtrace should be recorded
        \item<1-> By checking the \funcarg{domain\_state->backtrace\_active} value
        \item<2-> If not, acts as \funcname{raise\_notrace} with \funcname{RESTORE\_EXN\_HANDLER\_OCAML}
          \begin{itemize}
            \item Set \regname{RSP} to \localname{domain\_state->exn\_handler} value
            \item Restore \localname{domain\_state->exn\_handler} from trap
            \item Jump with \funcname{ret} (same effect as using \funcname{pop} and \funcname{jmp})
          \end{itemize}
        \item<3-> Otherwise, switches to the C stack to be able to execute C code
        \item<4-> Calls \funcname{caml\_stash\_backtrace}, a C function provided by the runtime\ignorespaces
          \onslide<6->{, with
            \begin{itemize}
              \item \funcarg{exn}: exception value
              \item \funcarg{pc}: code address of the raising point
              \item \funcarg{sp}: stack address of the raising function
              \item \funcarg{trapsp}: stack address of the next handler
            \end{itemize}
          }
      \end{itemize}
      \bigskip
      \mintocaml[firstline=9,lastline=13,highlightlines=12]{../src/backtrace/backtrace.ml}
    \end{column}
    \begin{column}{0.5\textwidth}
      \raggedleft
      \onslide*<1,3-4>{
        \mintc[firstline=190,lastline=192]{../ocaml/runtime/amd64.S}
        \mintc[firstline=234,lastline=237]{../ocaml/runtime/amd64.S}
      }
      \onslide*<2>{
        \mintc[firstline=190,lastline=192,highlightlines={191-192}]{../ocaml/runtime/amd64.S}
        \mintc[firstline=234,lastline=237,highlightlines={235-237}]{../ocaml/runtime/amd64.S}
      }
      \onslide*<1>{\listCamlRaiseExn[highlightlines=705]{}}%
      \onslide*<2>{\listCamlRaiseExn[highlightlines={707-708}]{}}%
      \onslide*<3>{\listCamlRaiseExn[highlightlines={712-714}]{}}%
      \onslide*<4>{\listCamlRaiseExn[highlightlines={715-720}]{}}%
      \onslide*<5>{\providecommand\step{1}\input{diagrams/backtrace/backtrace.tex}}%
      \onslide*<6>{\providecommand\step{2}\input{diagrams/backtrace/backtrace.tex}}%
    \end{column}
  \end{columns}
\end{frame}

\newcommand\listStashBacktrace[1][]{
  \listc[firstline=91,lastline=120,#1]{../ocaml/runtime/backtrace\_nat.c}{runtime/backtrace\_nat.c:91}}

\begin{frame}{Backtraces}{Introducing \funcname{caml\_stash\_backtrace}}
  \begin{columns}[c]
    \begin{column}{0.5\textwidth}
      \minipage[c][0.66\textheight][s]{\columnwidth}
      \begin{itemize}
        \item<1-> A C function provided by the runtime (specific to native code)
        \item<2-> It scans the stack, between the stack pointer at the raising point up to the next trap, in order to collect the names of the detected function frames
          \onslide*<2>{
            \begin{itemize}
              \item And more information, like line number, column number...
            \end{itemize}
          }
        \item<3-> It uses the same technique and information as the Garbage Collector (GC) uses to scan the values live on the stack
          \onslide*<4>{
            \begin{itemize}
              \item \typename{frame\_descr} are at the core of stack unwinding
            \end{itemize}
          }
      \end{itemize}
      \vfill
      \mintocaml[firstline=9,lastline=13,highlightlines=12]{../src/backtrace/backtrace.ml}
      \endminipage
    \end{column}
    \begin{column}{0.5\textwidth}
      \onslide*<1>{\listStashBacktrace[highlightlines=91]{}}
      \onslide*<2>{\listStashBacktrace[highlightlines={91,117-118}]{}}
      \onslide*<3->{\listStashBacktrace[highlightlines={94,106,109-110}]{}}
    \end{column}
  \end{columns}
\end{frame}

\newcommand\listFrameDescr[1][]{
  \listocaml[firstline=111,lastline=124,#1]{../ocaml/asmcomp/emitaux.ml}{asmcomp/emitaux.ml:111}}
\newcommand\listDebugInfo[1][]{
  \listocaml[firstline=38,lastline=49,#1]{../ocaml/lambda/debuginfo.mli}{lambda/debuginfo.mli:38}}

\begin{frame}{Backtraces}{\typename{frame\_descr} and \typename{frame\_debuginfo} in a nutshell}
  \begin{columns}[c]
    \begin{column}{0.5\textwidth}
      \begin{itemize}
        \item<1-> For every return address in an OCaml program, there is a associated \typename{frame\_descr}
        \item<2-> A \typename{frame\_descr} specifies
        \begin{itemize}
          \item<2-> The size of the function stack frame
          \item<3-> Where live values are on the stack (so that the GC can mark them as live and prevent from being swept)
        \end{itemize}
        \item<4-> When compiled with debug information, \typename{frame\_descr} will be linked to a \typename{frame\_debuginfo}
        \begin{itemize}
          \item<5-> Contains information about the position in the source code (file name, line and column number)
        \end{itemize}
      \end{itemize}
    \end{column}
    \begin{column}{0.5\textwidth}
        \onslide*<1>{\listFrameDescr[highlightlines={118-122}]{}}
        \onslide*<2>{\listFrameDescr[highlightlines={120}]{}}
        \onslide*<3>{\listFrameDescr[highlightlines={121}]{}}
        \onslide*<4->{\listFrameDescr[highlightlines={122}]{}}
        \medskip
        \onslide*<1-3>{\listDebugInfo{}}
        \onslide*<4>{\listDebugInfo[highlightlines={38,49}]{}}
        \onslide*<5>{\listDebugInfo[highlightlines={39-46}]{}}
    \end{column}
  \end{columns}
\end{frame}

\begin{frame}{Backtraces}{Scanning the stack with \typename{frame\_descr}}
  \begin{columns}[c]
    \begin{column}{0.5\textwidth}
      \begin{itemize}
        \item Given the return address of the code that raises the exception by calling \funcname{caml\_raise\_exn} (\funcarg{pc} argument of \funcname{caml\_stash\_backtrace})
        \item And the position of the stack pointer (\funcarg{sp} argument of \funcname{caml\_stash\_backtrace})
        \item The frame size given by \typename{frame\_descr}.\funcarg{frame\_size}, allows to find the end of the previous frame
        \item And the return address of the caller of this function
        \bigskip
        \onslide<3->{
        \item Note that traps (here the reddish \funcname{ExnA}, \funcname{ExnB} and \funcname{ExnC} blocks) belong to the frame that installed them, and so the frame size changes when traps are removed
        }
      \end{itemize}
    \end{column}
    \begin{column}{0.5\textwidth}
      \raggedleft
      \onslide*<1>{\providecommand\step{2}\input{diagrams/backtrace/backtrace.tex}}%
      \onslide*<2>{\providecommand\step{1}\input{diagrams/backtrace/scanstack.tex}}%
      \onslide*<3>{\providecommand\step{2}\input{diagrams/backtrace/scanstack.tex}}%
      \onslide*<4>{\providecommand\step{3}\input{diagrams/backtrace/scanstack.tex}}%
      \onslide*<5>{\providecommand\step{4}\input{diagrams/backtrace/scanstack.tex}}%
      \onslide*<6>{\providecommand\step{5}\input{diagrams/backtrace/scanstack.tex}}%
    \end{column}
  \end{columns}
\end{frame}

% Duplicate of previous-previous slide
\begin{frame}{Backtraces}{Continuing on \funcname{caml\_stash\_backtrace}}
  \begin{columns}[c]
    \begin{column}{0.5\textwidth}
      \minipage[c][0.75\textheight][s]{\columnwidth}
      \begin{itemize}
          \color{gray}
        \item A C function provided by the runtime (specific to native code)
        \item It scans the stack, between the stack pointer at the raising point up to the next trap, in order to collect the names of the detected function frames
        \item It uses the same technique and information as the Garbage Collector (GC) uses to scan the values live on the stack
      \end{itemize}
      \begin{itemize}
        \item For each function frame detected, store a pointer to the \typename{frame\_descr} in the \localname{domain\_state->backtrace\_buffer} array, effectively constructing the backtrace
      \end{itemize}
      \onslide<2->{
        \begin{itemize}
            \smallskip
          \item Constructed backtrace:
            \funcname{definitely\_raise},
            \onslide<3->{\funcname{probably\_raise}}
        \end{itemize}
      }
      \vfill
      \mintocaml[firstline=9,lastline=13,highlightlines=12]{../src/backtrace/backtrace.ml}
      \endminipage
    \end{column}
    \begin{column}{0.5\textwidth}
      \raggedleft
      \onslide*<1>{\listStashBacktrace[highlightlines={114-115}]{}}
      \onslide*<2>{\providecommand\step{3}\input{diagrams/backtrace/backtrace.tex}}%
      \onslide*<3>{\providecommand\step{4}\input{diagrams/backtrace/backtrace.tex}}%
    \end{column}
  \end{columns}
\end{frame}

\begin{frame}{Backtraces}{Back to \funcname{caml\_raise\_exn}}
  \begin{columns}[c]
    \begin{column}{0.5\textwidth}
      \minipage[c][0.66\textheight][s]{\columnwidth}
      \begin{itemize}\color{gray}
        \item Checks wheteher exceptions backtrace should be recorded, by checking the \funcarg{domain\_state->backtrace\_active} value
        \item Switches to the C stack to be able to execute C code
        \item Calls \funcname{caml\_stash\_backtrace}, to collect the backtrace up to the next trap
      \end{itemize}
      \onslide*<2->{
        \begin{itemize}
          \item<2-> Executes the exception handler in \funcname{probably\_raise} with \funcname{RESTORE\_EXN\_HANDLER\_OCAML}
            \begin{itemize}
              \item Set \regname{RSP} to \localname{domain\_state->exn\_handler} value
              \item Restore \localname{domain\_state->exn\_handler} from trap
              \item Jump to the handler code with \funcname{ret}
            \end{itemize}
        \end{itemize}
      }
      \vfill
      \onslide*<1>{\mintocaml[firstline=9,lastline=13,highlightlines=12]{../src/backtrace/backtrace.ml}}
      \onslide*<2->{\mintocaml[firstline=15,lastline=21,highlightlines={19-21}]{../src/backtrace/backtrace.ml}}
      \endminipage
    \end{column}
    \begin{column}{0.5\textwidth}
      \centering
      \onslide*<1>{
        \mintc[firstline=190,lastline=192]{../ocaml/runtime/amd64.S}
        \mintc[firstline=234,lastline=237]{../ocaml/runtime/amd64.S}
      }
      \onslide*<2>{
        \mintc[firstline=190,lastline=192,highlightlines={191-192}]{../ocaml/runtime/amd64.S}
        \mintc[firstline=234,lastline=237,highlightlines={235-237}]{../ocaml/runtime/amd64.S}
      }
      \onslide*<1>{\listCamlRaiseExn[highlightlines=720]{}}
      \onslide*<2>{\listCamlRaiseExn[highlightlines=722]{}}
      \onslide*<3->{\providecommand\step{5}\input{diagrams/backtrace/backtrace.tex}}
    \end{column}
  \end{columns}
\end{frame}

\begin{frame}{Backtraces}{\funcname{caml\_probably\_raise}'s handler doesn't catch \typename{ExnC}}
  \begin{columns}[c]
    \begin{column}{0.45\textwidth}
      \minipage[c][0.75\textheight][s]{\columnwidth}
      \begin{itemize}
        \item<1-> \funcname{match \dots with}\footnotemark pattern matching can also match exceptions
        \item<1-> The CMM of \funcname{probably\_raise} shows that this \funcname{match \dots with} is implemented with a pattern matching and a \funcname{try \dots with}
        \item<1-> But this one only matches on \typename{ExnA} exception
        \item<2-> Since the pattern matching fails to catch the exception, \funcname{reraise} is called with our current \typename{ExnC} exception
        \item<3-> Disassembly shows that \funcname{reraise} is actually a call to \funcname{caml\_reraise\_exn}, which is also an assembly function provided by the runtime
      \end{itemize}
      \vfill
      \mintocaml[firstline=15,lastline=21,highlightlines={18-21}]{../src/backtrace/backtrace.ml}
      \endminipage
    \end{column}
    \begin{column}{0.55\textwidth}
      \centering
      \onslide*<1>{\mintcmm[highlightlines={10-18}]{../src/backtrace/camlBacktrace\_\_probably\_raise\_351.cmm}}
      \onslide*<2>{\listcmm[highlightlines=18]{../src/backtrace/camlBacktrace\_\_probably\_raise_351.cmm}{CMM of probably\_raise}}
      \onslide*<3>{\mintobjdump[firstline=5,lastline=35,highlightlines=31]{../src/backtrace/camlBacktrace\_\_probably\_raise_351.objdump}}
    \end{column}
  \end{columns}
  \only<1>{\footnotetext[1]{https://blog.janestreet.com/pattern-matching-and-exception-handling-unite/}}
\end{frame}

\newcommand\listCamlReraiseExn[1][]{
  \listasm[firstline=727,lastline=734,#1]{../ocaml/runtime/amd64.S}{runtime/amd64.S:727}}

\begin{frame}{Backtraces}{Introducing \funcname{caml\_reraise\_exn}}
  \begin{columns}[c]
    \begin{column}{0.5\textwidth}
      \minipage[c][0.66\textheight][s]{\columnwidth}
      \begin{itemize}
        \item<1-> The exception have to be reraised to the next handler
        \item<2-> First, checks if the backtrace should be collected with \funcarg{domain\_state->backtrace\_active}
        \only<3>{
        \begin{itemize}
            \item If not, behaves like a \funcname{raise\_notrace} with \funcname{RESTORE\_EXN\_HANDLER\_OCAML}
        \end{itemize}
        }
        \item<4-> Jumps into \funcname{caml\_raise\_exn}
        \item<5-> Calls \funcname{caml\_stash\_backtrace} again, to scan the next part of the stack: from the trap just pop-ed to the next trap
      \end{itemize}
      \vfill
      \mintocaml[firstline=15,lastline=21,highlightlines={18-21}]{../src/backtrace/backtrace.ml}
      \endminipage
    \end{column}
    \begin{column}{0.5\textwidth}
      \raggedleft
      \onslide*<1>{\listCamlReraiseExn{}}
      \onslide*<2>{\listCamlReraiseExn[highlightlines=729]{}}
      \onslide*<3>{
        \mintc[firstline=190,lastline=192,highlightlines={191-192}]{../ocaml/runtime/amd64.S}
        \mintc[firstline=234,lastline=237,highlightlines={235-237}]{../ocaml/runtime/amd64.S}
        \medskip
        \listCamlReraiseExn[highlightlines=731]{}
      }
      \onslide*<4-5>{
        % TODO refactor with \listCamlReraiseExn
        \mintasm[firstline=727,lastline=734,highlightlines=730]{../ocaml/runtime/amd64.S}
        \medskip
      }
      \onslide*<4>{\listCamlRaiseExn[highlightlines=711]{}}
      \onslide*<5>{\listCamlRaiseExn[highlightlines=720]{}}
    \end{column}
  \end{columns}
\end{frame}

\begin{frame}{Backtraces}{Backtrace collection continues}
  \begin{columns}[c]
    \begin{column}{0.55\textwidth}
      \minipage[c][0.75\textheight][s]{\columnwidth}
      \begin{itemize}
        \item<1-> \funcname{caml\_stash\_backtrace} scans the older part of the stack, up to the next trap
        \item<6-> Controls jumps to the nested exception handler in \funcname{maybe\_raise}, which matches only on \typename{ExnB}
        \item<7-> \funcname{caml\_reraise\_exn} is called again, backtrace collection resumes with \funcname{caml\_stash\_backtrace}
      \end{itemize}
      \medskip
      \begin{itemize}
        \item<2-> Constructed backtrace:
        \begin{itemize}
          \item <2-> \funcname{definitely\_raise}
          \item <2-> \funcname{probably\_raise}
          \item <3-> \funcname{probably\_raise} (re-raise)
          \item <4-> \funcname{maybe\_raise}
          \item <5-> \funcname{main}
          \item <8-> \funcname{main} (re-raise)
        \end{itemize}
      \end{itemize}
      \vfill
      \onslide*<1-5>{\mintocaml[firstline=15,lastline=21]{../src/backtrace/backtrace.ml}}
      \onslide*<6>{\mintocaml[firstline=28,lastline=34,highlightlines=31]{../src/backtrace/backtrace.ml}}
      \onslide*<7->{\mintocaml[firstline=28,lastline=34,highlightlines=33]{../src/backtrace/backtrace.ml}}
      \endminipage
    \end{column}
    \begin{column}{0.45\textwidth}
      \raggedleft
      \onslide*<1>{\providecommand\step{6}\input{diagrams/backtrace/backtrace.tex}}%
      \onslide*<2>{\providecommand\step{6}\input{diagrams/backtrace/backtrace.tex}}%
      \onslide*<3>{\providecommand\step{7}\input{diagrams/backtrace/backtrace.tex}}%
      \onslide*<4>{\providecommand\step{8}\input{diagrams/backtrace/backtrace.tex}}%
      \onslide*<5>{\providecommand\step{9}\input{diagrams/backtrace/backtrace.tex}}%
      \onslide*<6>{\providecommand\step{10}\input{diagrams/backtrace/backtrace.tex}}%
      \onslide*<7>{\providecommand\step{11}\input{diagrams/backtrace/backtrace.tex}}%
      \onslide*<8>{\providecommand\step{12}\input{diagrams/backtrace/backtrace.tex}}%
      \onslide*<9>{\providecommand\step{13}\input{diagrams/backtrace/backtrace.tex}}%
    \end{column}
  \end{columns}
\end{frame}

\begin{frame}{Backtraces}{Printing the backtrace}
  \begin{columns}[c]
    \begin{column}{0.45\textwidth}
      \minipage[c][0.66\textheight][s]{\columnwidth}
      \begin{itemize}
        \item<1-> The exception backtrace can now be printed to \funcarg{stdout} with \funcname{Printexc.print_backtrace}
        \item<2-> First the exception backtrace is retrieved into an OCaml array with \funcname{get_raw_backtrace }
        \item<3-> Which is implemented as the \funcname{caml_get_exception_raw_backtrace} C function in the runtime
        \item<4-> And finally printed with \funcname{print_exception_backtrace}
      \end{itemize}
      \vfill
      \mintocaml[firstline=28,lastline=34,highlightlines=33]{../src/backtrace/backtrace.ml}
      \endminipage
    \end{column}
    \begin{column}{0.55\textwidth}
      \onslide*<1,2>{
        \mintocaml[firstline=100,lastline=101]{../ocaml/stdlib/printexc.ml}
      }
      \only<1>{
        \listocaml[firstline=170,lastline=172]{../ocaml/stdlib/printexc.ml}{stdlib/printexc.ml:170}
      }
      \only<2>{
        \listocaml[firstline=170,lastline=172,highlightlines=172]{../ocaml/stdlib/printexc.ml}{stdlib/printexc.ml:170}
      }
      \only<3>{
        \mintc[firstline=154,lastline=156]{../ocaml/runtime/backtrace.c}
        \listc[firstline=171,lastline=192,highlightlines={184-187}]{../ocaml/runtime/backtrace.c}{runtime/backtrace.c:154}
      }
      \only<4>{
        \listocaml[firstline=155,lastline=165]{../ocaml/stdlib/printexc.ml}{stdlib/printexc.ml:55}
      }
    \end{column}
  \end{columns}
\end{frame}


\subsection{The multiple kinds of raise}
\frameSubsection{}{}

\begin{frame}{Backtraces}{When backtraces are not needed}
  \begin{columns}[c]
    \begin{column}{0.45\textwidth}
      \minipage[c][0.66\textheight][s]{\columnwidth}
      \begin{itemize}
        \item<1-> What if the backtrace for an exception is never needed?
        \item<2-> It's possible to raise the exception explicitly with \funcname{raise\_notrace} to make raising as cheap as possible
        \item<3-> An example of use in the \funcname{dynlink} library of the OCaml compiler
      \end{itemize}
      \vfill
      \onslide<3->{
      \listocaml[firstline=205,lastline=209,highlightlines=207]{../ocaml/otherlibs/dynlink/dynlink\_compilerlibs/misc.ml}{otherlibs/dynlink/dynlink\_compilerlibs/misc.ml:205}
      }
      \endminipage
    \end{column}
    \begin{column}{0.55\textwidth}
    \only<4>{
      \mintcmm[highlightlines=3]{../src/notrace/camlNotrace\_\_fun_347.cmm}
      \listcmm{../src/notrace/camlNotrace\_\_all\_somes\_327.cmm}{all\_somes}
    }
    \onslide*<5->{
      \listobjdump[highlightlines={6-9}]{../src/notrace/camlNotrace\_\_fun_347.objdump}{anonymous function from \funcname{all\_somes}}
    }
    \end{column}
  \end{columns}
\end{frame}

\begin{frame}{Backtraces}{Summary table of the multiple kinds of raise}
  \begin{itemize}
    \item When debugging information is enabled and if the backtrace of an exception isn't needed, it's possible to raise with \funcname{raise\_notrace} for performance
    \item When debugging information is \emph{not} enabled, the compiler optimises \funcname{raise} calls to inline assembly (same effect as using \funcname{raise\_notrace})
  \end{itemize}
  \bigskip
  \centering
  \begin{tabular}{| l | c | c | c | c |}
    \hline
    \parbox{10em}{Debug information \\ (\funcarg{-g} flag)}  & \multicolumn{2}{|c|}{Disabled}                                     & \multicolumn{2}{|c|}{Enabled} \\
    \hline
    Raise kind                                               & \funcname{raise} & \funcname{raise\_notrace}                       & \funcname{raise} & \funcname{raise\_notrace} \\
    \hline
    Compilation                                              & \parbox{8em}{inline assembly\\ (optimization)} & inline assembly   & \funcname{caml\_raise\_exn} & inline assembly \\
    \hline
  \end{tabular}
\end{frame}


%
% Takeaway #5
%
\subsection*{Takeaway \#5}
\frameSubsectionTakeaway{}
\againframe<5>{takeaway}
}%
      \onslide*<7>{\providecommand\step{11}%
% Backtraces
%
\subsection{One advantage of raising with debugging information}
\frameSubsection{}{}

\begin{frame}{Backtraces}{Debugging information \& source code}
  \begin{columns}[c]
    \begin{column}{0.5\textwidth}
      \begin{itemize}
        \item Raising an exception is fun and games, but how do we know where the exception originated? $\rightarrow$ With the backtrace associated with the exception!
        \item In order to get a backtrace from the point where an exception was raised, it's necessary to compile with debugging information enabled (\funcarg{-g} flag for the compiler)
        \item Same example as in the `Nested handlers' section
      \end{itemize}
    \end{column}
    \begin{column}{0.5\textwidth}
      \listocaml[firstline=9,lastline=34]{../src/backtrace/backtrace.ml}{backtrace/backtrace.ml}
    \end{column}
  \end{columns}
\end{frame}

\begin{frame}{Backtraces}{CMM and disassembly of \funcname{definitely\_raise}}
  \begin{columns}[c]
    \begin{column}{0.5\textwidth}
      \minipage[c][0.66\textheight][s]{\columnwidth}
      \begin{itemize}
        \item<1-> An effect of compiling with debugging information (\funcarg{-g}): the CMM no longer uses \funcname{raise\_notrace} to raise the exception but \funcname{raise} instead
        \item<2-> Disassembly shows that CMM \funcname{raise} is actually a call to \funcname{caml\_raise\_exn}, a function in assembler provided by the runtime
      \end{itemize}
      \vfill
      \mintocaml[firstline=9,lastline=13,highlightlines=12]{../src/backtrace/backtrace.ml}
      \endminipage
    \end{column}
    \begin{column}{0.5\textwidth}
      \onslide*<1>{
        \mintcmm[highlightlines=5]{../src/backtrace/camlBacktrace\_\_definitely\_raise\_347.cmm}
      }
      \onslide*<2>{
        \mintobjdump[firstline=2,highlightlines=14]{../src/backtrace/camlBacktrace\_\_definitely\_raise\_347.objdump}
      }
    \end{column}
  \end{columns}
\end{frame}

\begin{frame}{Backtraces}{Introducing \funcname{caml\_raise\_exn}}
  \begin{columns}[c]
    \begin{column}{0.5\textwidth}
      \minipage[c][0.66\textheight][s]{\columnwidth}
      \onslide*<1>{
        \begin{itemize}
          \item Raising the exception is performed using \funcname{caml\_raise\_exn} which is
            \begin{itemize}
              \item An assembly function provided by the runtime
              \item Architecture specific
            \end{itemize}
          \item Let's see what it does differently from \funcname{raise\_notrace}\dots
          \item We are about to raise \typename{ExnC} with \funcname{caml\_raise\_exn} from \funcname{definitely\_raise}
        \end{itemize}
      }
      \onslide*<2>{
        \mintobjdump[firstline=2,highlightlines=14]{../src/backtrace/camlBacktrace\_\_definitely\_raise\_347.objdump}
      }
      \vfill
      \mintocaml[firstline=9,lastline=13,highlightlines=12]{../src/backtrace/backtrace.ml}
      \endminipage
    \end{column}
    \begin{column}{0.5\textwidth}
      \centering
      \providecommand\step{1}\input{diagrams/backtrace/backtrace.tex}
    \end{column}
  \end{columns}
\end{frame}

\begin{frame}{Backtraces}{But before that, we need more fields from \typename{caml\_domain\_state}}
  \begin{columns}
    \begin{column}{0.5\textwidth}
      \begin{itemize}
        \item Remember \typename{caml\_domain\_state} the per-domain runtime state?
        \item Some other members are of interest to us now\dots
        \item \localname{backtrace\_active} is used to know if the runtime should collect backtraces
        \item \localname{backtrace\_active} can be set by
          \begin{itemize}
            \item The \funcarg{b} flag in the \funcname{OCAMLRUNPARAM} environment variable
            \item \mintinline{ocaml}{Printexc.record_backtrace : bool -> unit} \footnotemark
          \end{itemize}
      \end{itemize}
    \end{column}
    \begin{column}{0.5\textwidth}
      \begin{listing}
        \mintc[highlightlines={9-11}]{src/ocaml/domain\_state\_backtrace.h}
        \caption{runtime/caml/domain\_state.h}
      \end{listing}
    \end{column}
  \end{columns}
  \footnotetext[1]{https://v2.ocaml.org/api/Printexc.html}
\end{frame}

\newcommand\listCamlRaiseExn[1][]{
  \listasm[firstline=702,lastline=725,#1]{../ocaml/runtime/amd64.S}{runtime/amd64.S:702}}

\begin{frame}{Backtraces}{Walk through \funcname{caml\_raise\_exn}}
  \begin{columns}[T]
    \begin{column}{0.5\textwidth}
      \begin{itemize}
        \item<1-> First checks whether exceptions backtrace should be recorded
        \item<1-> By checking the \funcarg{domain\_state->backtrace\_active} value
        \item<2-> If not, acts as \funcname{raise\_notrace} with \funcname{RESTORE\_EXN\_HANDLER\_OCAML}
          \begin{itemize}
            \item Set \regname{RSP} to \localname{domain\_state->exn\_handler} value
            \item Restore \localname{domain\_state->exn\_handler} from trap
            \item Jump with \funcname{ret} (same effect as using \funcname{pop} and \funcname{jmp})
          \end{itemize}
        \item<3-> Otherwise, switches to the C stack to be able to execute C code
        \item<4-> Calls \funcname{caml\_stash\_backtrace}, a C function provided by the runtime\ignorespaces
          \onslide<6->{, with
            \begin{itemize}
              \item \funcarg{exn}: exception value
              \item \funcarg{pc}: code address of the raising point
              \item \funcarg{sp}: stack address of the raising function
              \item \funcarg{trapsp}: stack address of the next handler
            \end{itemize}
          }
      \end{itemize}
      \bigskip
      \mintocaml[firstline=9,lastline=13,highlightlines=12]{../src/backtrace/backtrace.ml}
    \end{column}
    \begin{column}{0.5\textwidth}
      \raggedleft
      \onslide*<1,3-4>{
        \mintc[firstline=190,lastline=192]{../ocaml/runtime/amd64.S}
        \mintc[firstline=234,lastline=237]{../ocaml/runtime/amd64.S}
      }
      \onslide*<2>{
        \mintc[firstline=190,lastline=192,highlightlines={191-192}]{../ocaml/runtime/amd64.S}
        \mintc[firstline=234,lastline=237,highlightlines={235-237}]{../ocaml/runtime/amd64.S}
      }
      \onslide*<1>{\listCamlRaiseExn[highlightlines=705]{}}%
      \onslide*<2>{\listCamlRaiseExn[highlightlines={707-708}]{}}%
      \onslide*<3>{\listCamlRaiseExn[highlightlines={712-714}]{}}%
      \onslide*<4>{\listCamlRaiseExn[highlightlines={715-720}]{}}%
      \onslide*<5>{\providecommand\step{1}\input{diagrams/backtrace/backtrace.tex}}%
      \onslide*<6>{\providecommand\step{2}\input{diagrams/backtrace/backtrace.tex}}%
    \end{column}
  \end{columns}
\end{frame}

\newcommand\listStashBacktrace[1][]{
  \listc[firstline=91,lastline=120,#1]{../ocaml/runtime/backtrace\_nat.c}{runtime/backtrace\_nat.c:91}}

\begin{frame}{Backtraces}{Introducing \funcname{caml\_stash\_backtrace}}
  \begin{columns}[c]
    \begin{column}{0.5\textwidth}
      \minipage[c][0.66\textheight][s]{\columnwidth}
      \begin{itemize}
        \item<1-> A C function provided by the runtime (specific to native code)
        \item<2-> It scans the stack, between the stack pointer at the raising point up to the next trap, in order to collect the names of the detected function frames
          \onslide*<2>{
            \begin{itemize}
              \item And more information, like line number, column number...
            \end{itemize}
          }
        \item<3-> It uses the same technique and information as the Garbage Collector (GC) uses to scan the values live on the stack
          \onslide*<4>{
            \begin{itemize}
              \item \typename{frame\_descr} are at the core of stack unwinding
            \end{itemize}
          }
      \end{itemize}
      \vfill
      \mintocaml[firstline=9,lastline=13,highlightlines=12]{../src/backtrace/backtrace.ml}
      \endminipage
    \end{column}
    \begin{column}{0.5\textwidth}
      \onslide*<1>{\listStashBacktrace[highlightlines=91]{}}
      \onslide*<2>{\listStashBacktrace[highlightlines={91,117-118}]{}}
      \onslide*<3->{\listStashBacktrace[highlightlines={94,106,109-110}]{}}
    \end{column}
  \end{columns}
\end{frame}

\newcommand\listFrameDescr[1][]{
  \listocaml[firstline=111,lastline=124,#1]{../ocaml/asmcomp/emitaux.ml}{asmcomp/emitaux.ml:111}}
\newcommand\listDebugInfo[1][]{
  \listocaml[firstline=38,lastline=49,#1]{../ocaml/lambda/debuginfo.mli}{lambda/debuginfo.mli:38}}

\begin{frame}{Backtraces}{\typename{frame\_descr} and \typename{frame\_debuginfo} in a nutshell}
  \begin{columns}[c]
    \begin{column}{0.5\textwidth}
      \begin{itemize}
        \item<1-> For every return address in an OCaml program, there is a associated \typename{frame\_descr}
        \item<2-> A \typename{frame\_descr} specifies
        \begin{itemize}
          \item<2-> The size of the function stack frame
          \item<3-> Where live values are on the stack (so that the GC can mark them as live and prevent from being swept)
        \end{itemize}
        \item<4-> When compiled with debug information, \typename{frame\_descr} will be linked to a \typename{frame\_debuginfo}
        \begin{itemize}
          \item<5-> Contains information about the position in the source code (file name, line and column number)
        \end{itemize}
      \end{itemize}
    \end{column}
    \begin{column}{0.5\textwidth}
        \onslide*<1>{\listFrameDescr[highlightlines={118-122}]{}}
        \onslide*<2>{\listFrameDescr[highlightlines={120}]{}}
        \onslide*<3>{\listFrameDescr[highlightlines={121}]{}}
        \onslide*<4->{\listFrameDescr[highlightlines={122}]{}}
        \medskip
        \onslide*<1-3>{\listDebugInfo{}}
        \onslide*<4>{\listDebugInfo[highlightlines={38,49}]{}}
        \onslide*<5>{\listDebugInfo[highlightlines={39-46}]{}}
    \end{column}
  \end{columns}
\end{frame}

\begin{frame}{Backtraces}{Scanning the stack with \typename{frame\_descr}}
  \begin{columns}[c]
    \begin{column}{0.5\textwidth}
      \begin{itemize}
        \item Given the return address of the code that raises the exception by calling \funcname{caml\_raise\_exn} (\funcarg{pc} argument of \funcname{caml\_stash\_backtrace})
        \item And the position of the stack pointer (\funcarg{sp} argument of \funcname{caml\_stash\_backtrace})
        \item The frame size given by \typename{frame\_descr}.\funcarg{frame\_size}, allows to find the end of the previous frame
        \item And the return address of the caller of this function
        \bigskip
        \onslide<3->{
        \item Note that traps (here the reddish \funcname{ExnA}, \funcname{ExnB} and \funcname{ExnC} blocks) belong to the frame that installed them, and so the frame size changes when traps are removed
        }
      \end{itemize}
    \end{column}
    \begin{column}{0.5\textwidth}
      \raggedleft
      \onslide*<1>{\providecommand\step{2}\input{diagrams/backtrace/backtrace.tex}}%
      \onslide*<2>{\providecommand\step{1}\input{diagrams/backtrace/scanstack.tex}}%
      \onslide*<3>{\providecommand\step{2}\input{diagrams/backtrace/scanstack.tex}}%
      \onslide*<4>{\providecommand\step{3}\input{diagrams/backtrace/scanstack.tex}}%
      \onslide*<5>{\providecommand\step{4}\input{diagrams/backtrace/scanstack.tex}}%
      \onslide*<6>{\providecommand\step{5}\input{diagrams/backtrace/scanstack.tex}}%
    \end{column}
  \end{columns}
\end{frame}

% Duplicate of previous-previous slide
\begin{frame}{Backtraces}{Continuing on \funcname{caml\_stash\_backtrace}}
  \begin{columns}[c]
    \begin{column}{0.5\textwidth}
      \minipage[c][0.75\textheight][s]{\columnwidth}
      \begin{itemize}
          \color{gray}
        \item A C function provided by the runtime (specific to native code)
        \item It scans the stack, between the stack pointer at the raising point up to the next trap, in order to collect the names of the detected function frames
        \item It uses the same technique and information as the Garbage Collector (GC) uses to scan the values live on the stack
      \end{itemize}
      \begin{itemize}
        \item For each function frame detected, store a pointer to the \typename{frame\_descr} in the \localname{domain\_state->backtrace\_buffer} array, effectively constructing the backtrace
      \end{itemize}
      \onslide<2->{
        \begin{itemize}
            \smallskip
          \item Constructed backtrace:
            \funcname{definitely\_raise},
            \onslide<3->{\funcname{probably\_raise}}
        \end{itemize}
      }
      \vfill
      \mintocaml[firstline=9,lastline=13,highlightlines=12]{../src/backtrace/backtrace.ml}
      \endminipage
    \end{column}
    \begin{column}{0.5\textwidth}
      \raggedleft
      \onslide*<1>{\listStashBacktrace[highlightlines={114-115}]{}}
      \onslide*<2>{\providecommand\step{3}\input{diagrams/backtrace/backtrace.tex}}%
      \onslide*<3>{\providecommand\step{4}\input{diagrams/backtrace/backtrace.tex}}%
    \end{column}
  \end{columns}
\end{frame}

\begin{frame}{Backtraces}{Back to \funcname{caml\_raise\_exn}}
  \begin{columns}[c]
    \begin{column}{0.5\textwidth}
      \minipage[c][0.66\textheight][s]{\columnwidth}
      \begin{itemize}\color{gray}
        \item Checks wheteher exceptions backtrace should be recorded, by checking the \funcarg{domain\_state->backtrace\_active} value
        \item Switches to the C stack to be able to execute C code
        \item Calls \funcname{caml\_stash\_backtrace}, to collect the backtrace up to the next trap
      \end{itemize}
      \onslide*<2->{
        \begin{itemize}
          \item<2-> Executes the exception handler in \funcname{probably\_raise} with \funcname{RESTORE\_EXN\_HANDLER\_OCAML}
            \begin{itemize}
              \item Set \regname{RSP} to \localname{domain\_state->exn\_handler} value
              \item Restore \localname{domain\_state->exn\_handler} from trap
              \item Jump to the handler code with \funcname{ret}
            \end{itemize}
        \end{itemize}
      }
      \vfill
      \onslide*<1>{\mintocaml[firstline=9,lastline=13,highlightlines=12]{../src/backtrace/backtrace.ml}}
      \onslide*<2->{\mintocaml[firstline=15,lastline=21,highlightlines={19-21}]{../src/backtrace/backtrace.ml}}
      \endminipage
    \end{column}
    \begin{column}{0.5\textwidth}
      \centering
      \onslide*<1>{
        \mintc[firstline=190,lastline=192]{../ocaml/runtime/amd64.S}
        \mintc[firstline=234,lastline=237]{../ocaml/runtime/amd64.S}
      }
      \onslide*<2>{
        \mintc[firstline=190,lastline=192,highlightlines={191-192}]{../ocaml/runtime/amd64.S}
        \mintc[firstline=234,lastline=237,highlightlines={235-237}]{../ocaml/runtime/amd64.S}
      }
      \onslide*<1>{\listCamlRaiseExn[highlightlines=720]{}}
      \onslide*<2>{\listCamlRaiseExn[highlightlines=722]{}}
      \onslide*<3->{\providecommand\step{5}\input{diagrams/backtrace/backtrace.tex}}
    \end{column}
  \end{columns}
\end{frame}

\begin{frame}{Backtraces}{\funcname{caml\_probably\_raise}'s handler doesn't catch \typename{ExnC}}
  \begin{columns}[c]
    \begin{column}{0.45\textwidth}
      \minipage[c][0.75\textheight][s]{\columnwidth}
      \begin{itemize}
        \item<1-> \funcname{match \dots with}\footnotemark pattern matching can also match exceptions
        \item<1-> The CMM of \funcname{probably\_raise} shows that this \funcname{match \dots with} is implemented with a pattern matching and a \funcname{try \dots with}
        \item<1-> But this one only matches on \typename{ExnA} exception
        \item<2-> Since the pattern matching fails to catch the exception, \funcname{reraise} is called with our current \typename{ExnC} exception
        \item<3-> Disassembly shows that \funcname{reraise} is actually a call to \funcname{caml\_reraise\_exn}, which is also an assembly function provided by the runtime
      \end{itemize}
      \vfill
      \mintocaml[firstline=15,lastline=21,highlightlines={18-21}]{../src/backtrace/backtrace.ml}
      \endminipage
    \end{column}
    \begin{column}{0.55\textwidth}
      \centering
      \onslide*<1>{\mintcmm[highlightlines={10-18}]{../src/backtrace/camlBacktrace\_\_probably\_raise\_351.cmm}}
      \onslide*<2>{\listcmm[highlightlines=18]{../src/backtrace/camlBacktrace\_\_probably\_raise_351.cmm}{CMM of probably\_raise}}
      \onslide*<3>{\mintobjdump[firstline=5,lastline=35,highlightlines=31]{../src/backtrace/camlBacktrace\_\_probably\_raise_351.objdump}}
    \end{column}
  \end{columns}
  \only<1>{\footnotetext[1]{https://blog.janestreet.com/pattern-matching-and-exception-handling-unite/}}
\end{frame}

\newcommand\listCamlReraiseExn[1][]{
  \listasm[firstline=727,lastline=734,#1]{../ocaml/runtime/amd64.S}{runtime/amd64.S:727}}

\begin{frame}{Backtraces}{Introducing \funcname{caml\_reraise\_exn}}
  \begin{columns}[c]
    \begin{column}{0.5\textwidth}
      \minipage[c][0.66\textheight][s]{\columnwidth}
      \begin{itemize}
        \item<1-> The exception have to be reraised to the next handler
        \item<2-> First, checks if the backtrace should be collected with \funcarg{domain\_state->backtrace\_active}
        \only<3>{
        \begin{itemize}
            \item If not, behaves like a \funcname{raise\_notrace} with \funcname{RESTORE\_EXN\_HANDLER\_OCAML}
        \end{itemize}
        }
        \item<4-> Jumps into \funcname{caml\_raise\_exn}
        \item<5-> Calls \funcname{caml\_stash\_backtrace} again, to scan the next part of the stack: from the trap just pop-ed to the next trap
      \end{itemize}
      \vfill
      \mintocaml[firstline=15,lastline=21,highlightlines={18-21}]{../src/backtrace/backtrace.ml}
      \endminipage
    \end{column}
    \begin{column}{0.5\textwidth}
      \raggedleft
      \onslide*<1>{\listCamlReraiseExn{}}
      \onslide*<2>{\listCamlReraiseExn[highlightlines=729]{}}
      \onslide*<3>{
        \mintc[firstline=190,lastline=192,highlightlines={191-192}]{../ocaml/runtime/amd64.S}
        \mintc[firstline=234,lastline=237,highlightlines={235-237}]{../ocaml/runtime/amd64.S}
        \medskip
        \listCamlReraiseExn[highlightlines=731]{}
      }
      \onslide*<4-5>{
        % TODO refactor with \listCamlReraiseExn
        \mintasm[firstline=727,lastline=734,highlightlines=730]{../ocaml/runtime/amd64.S}
        \medskip
      }
      \onslide*<4>{\listCamlRaiseExn[highlightlines=711]{}}
      \onslide*<5>{\listCamlRaiseExn[highlightlines=720]{}}
    \end{column}
  \end{columns}
\end{frame}

\begin{frame}{Backtraces}{Backtrace collection continues}
  \begin{columns}[c]
    \begin{column}{0.55\textwidth}
      \minipage[c][0.75\textheight][s]{\columnwidth}
      \begin{itemize}
        \item<1-> \funcname{caml\_stash\_backtrace} scans the older part of the stack, up to the next trap
        \item<6-> Controls jumps to the nested exception handler in \funcname{maybe\_raise}, which matches only on \typename{ExnB}
        \item<7-> \funcname{caml\_reraise\_exn} is called again, backtrace collection resumes with \funcname{caml\_stash\_backtrace}
      \end{itemize}
      \medskip
      \begin{itemize}
        \item<2-> Constructed backtrace:
        \begin{itemize}
          \item <2-> \funcname{definitely\_raise}
          \item <2-> \funcname{probably\_raise}
          \item <3-> \funcname{probably\_raise} (re-raise)
          \item <4-> \funcname{maybe\_raise}
          \item <5-> \funcname{main}
          \item <8-> \funcname{main} (re-raise)
        \end{itemize}
      \end{itemize}
      \vfill
      \onslide*<1-5>{\mintocaml[firstline=15,lastline=21]{../src/backtrace/backtrace.ml}}
      \onslide*<6>{\mintocaml[firstline=28,lastline=34,highlightlines=31]{../src/backtrace/backtrace.ml}}
      \onslide*<7->{\mintocaml[firstline=28,lastline=34,highlightlines=33]{../src/backtrace/backtrace.ml}}
      \endminipage
    \end{column}
    \begin{column}{0.45\textwidth}
      \raggedleft
      \onslide*<1>{\providecommand\step{6}\input{diagrams/backtrace/backtrace.tex}}%
      \onslide*<2>{\providecommand\step{6}\input{diagrams/backtrace/backtrace.tex}}%
      \onslide*<3>{\providecommand\step{7}\input{diagrams/backtrace/backtrace.tex}}%
      \onslide*<4>{\providecommand\step{8}\input{diagrams/backtrace/backtrace.tex}}%
      \onslide*<5>{\providecommand\step{9}\input{diagrams/backtrace/backtrace.tex}}%
      \onslide*<6>{\providecommand\step{10}\input{diagrams/backtrace/backtrace.tex}}%
      \onslide*<7>{\providecommand\step{11}\input{diagrams/backtrace/backtrace.tex}}%
      \onslide*<8>{\providecommand\step{12}\input{diagrams/backtrace/backtrace.tex}}%
      \onslide*<9>{\providecommand\step{13}\input{diagrams/backtrace/backtrace.tex}}%
    \end{column}
  \end{columns}
\end{frame}

\begin{frame}{Backtraces}{Printing the backtrace}
  \begin{columns}[c]
    \begin{column}{0.45\textwidth}
      \minipage[c][0.66\textheight][s]{\columnwidth}
      \begin{itemize}
        \item<1-> The exception backtrace can now be printed to \funcarg{stdout} with \funcname{Printexc.print_backtrace}
        \item<2-> First the exception backtrace is retrieved into an OCaml array with \funcname{get_raw_backtrace }
        \item<3-> Which is implemented as the \funcname{caml_get_exception_raw_backtrace} C function in the runtime
        \item<4-> And finally printed with \funcname{print_exception_backtrace}
      \end{itemize}
      \vfill
      \mintocaml[firstline=28,lastline=34,highlightlines=33]{../src/backtrace/backtrace.ml}
      \endminipage
    \end{column}
    \begin{column}{0.55\textwidth}
      \onslide*<1,2>{
        \mintocaml[firstline=100,lastline=101]{../ocaml/stdlib/printexc.ml}
      }
      \only<1>{
        \listocaml[firstline=170,lastline=172]{../ocaml/stdlib/printexc.ml}{stdlib/printexc.ml:170}
      }
      \only<2>{
        \listocaml[firstline=170,lastline=172,highlightlines=172]{../ocaml/stdlib/printexc.ml}{stdlib/printexc.ml:170}
      }
      \only<3>{
        \mintc[firstline=154,lastline=156]{../ocaml/runtime/backtrace.c}
        \listc[firstline=171,lastline=192,highlightlines={184-187}]{../ocaml/runtime/backtrace.c}{runtime/backtrace.c:154}
      }
      \only<4>{
        \listocaml[firstline=155,lastline=165]{../ocaml/stdlib/printexc.ml}{stdlib/printexc.ml:55}
      }
    \end{column}
  \end{columns}
\end{frame}


\subsection{The multiple kinds of raise}
\frameSubsection{}{}

\begin{frame}{Backtraces}{When backtraces are not needed}
  \begin{columns}[c]
    \begin{column}{0.45\textwidth}
      \minipage[c][0.66\textheight][s]{\columnwidth}
      \begin{itemize}
        \item<1-> What if the backtrace for an exception is never needed?
        \item<2-> It's possible to raise the exception explicitly with \funcname{raise\_notrace} to make raising as cheap as possible
        \item<3-> An example of use in the \funcname{dynlink} library of the OCaml compiler
      \end{itemize}
      \vfill
      \onslide<3->{
      \listocaml[firstline=205,lastline=209,highlightlines=207]{../ocaml/otherlibs/dynlink/dynlink\_compilerlibs/misc.ml}{otherlibs/dynlink/dynlink\_compilerlibs/misc.ml:205}
      }
      \endminipage
    \end{column}
    \begin{column}{0.55\textwidth}
    \only<4>{
      \mintcmm[highlightlines=3]{../src/notrace/camlNotrace\_\_fun_347.cmm}
      \listcmm{../src/notrace/camlNotrace\_\_all\_somes\_327.cmm}{all\_somes}
    }
    \onslide*<5->{
      \listobjdump[highlightlines={6-9}]{../src/notrace/camlNotrace\_\_fun_347.objdump}{anonymous function from \funcname{all\_somes}}
    }
    \end{column}
  \end{columns}
\end{frame}

\begin{frame}{Backtraces}{Summary table of the multiple kinds of raise}
  \begin{itemize}
    \item When debugging information is enabled and if the backtrace of an exception isn't needed, it's possible to raise with \funcname{raise\_notrace} for performance
    \item When debugging information is \emph{not} enabled, the compiler optimises \funcname{raise} calls to inline assembly (same effect as using \funcname{raise\_notrace})
  \end{itemize}
  \bigskip
  \centering
  \begin{tabular}{| l | c | c | c | c |}
    \hline
    \parbox{10em}{Debug information \\ (\funcarg{-g} flag)}  & \multicolumn{2}{|c|}{Disabled}                                     & \multicolumn{2}{|c|}{Enabled} \\
    \hline
    Raise kind                                               & \funcname{raise} & \funcname{raise\_notrace}                       & \funcname{raise} & \funcname{raise\_notrace} \\
    \hline
    Compilation                                              & \parbox{8em}{inline assembly\\ (optimization)} & inline assembly   & \funcname{caml\_raise\_exn} & inline assembly \\
    \hline
  \end{tabular}
\end{frame}


%
% Takeaway #5
%
\subsection*{Takeaway \#5}
\frameSubsectionTakeaway{}
\againframe<5>{takeaway}
}%
      \onslide*<8>{\providecommand\step{12}%
% Backtraces
%
\subsection{One advantage of raising with debugging information}
\frameSubsection{}{}

\begin{frame}{Backtraces}{Debugging information \& source code}
  \begin{columns}[c]
    \begin{column}{0.5\textwidth}
      \begin{itemize}
        \item Raising an exception is fun and games, but how do we know where the exception originated? $\rightarrow$ With the backtrace associated with the exception!
        \item In order to get a backtrace from the point where an exception was raised, it's necessary to compile with debugging information enabled (\funcarg{-g} flag for the compiler)
        \item Same example as in the `Nested handlers' section
      \end{itemize}
    \end{column}
    \begin{column}{0.5\textwidth}
      \listocaml[firstline=9,lastline=34]{../src/backtrace/backtrace.ml}{backtrace/backtrace.ml}
    \end{column}
  \end{columns}
\end{frame}

\begin{frame}{Backtraces}{CMM and disassembly of \funcname{definitely\_raise}}
  \begin{columns}[c]
    \begin{column}{0.5\textwidth}
      \minipage[c][0.66\textheight][s]{\columnwidth}
      \begin{itemize}
        \item<1-> An effect of compiling with debugging information (\funcarg{-g}): the CMM no longer uses \funcname{raise\_notrace} to raise the exception but \funcname{raise} instead
        \item<2-> Disassembly shows that CMM \funcname{raise} is actually a call to \funcname{caml\_raise\_exn}, a function in assembler provided by the runtime
      \end{itemize}
      \vfill
      \mintocaml[firstline=9,lastline=13,highlightlines=12]{../src/backtrace/backtrace.ml}
      \endminipage
    \end{column}
    \begin{column}{0.5\textwidth}
      \onslide*<1>{
        \mintcmm[highlightlines=5]{../src/backtrace/camlBacktrace\_\_definitely\_raise\_347.cmm}
      }
      \onslide*<2>{
        \mintobjdump[firstline=2,highlightlines=14]{../src/backtrace/camlBacktrace\_\_definitely\_raise\_347.objdump}
      }
    \end{column}
  \end{columns}
\end{frame}

\begin{frame}{Backtraces}{Introducing \funcname{caml\_raise\_exn}}
  \begin{columns}[c]
    \begin{column}{0.5\textwidth}
      \minipage[c][0.66\textheight][s]{\columnwidth}
      \onslide*<1>{
        \begin{itemize}
          \item Raising the exception is performed using \funcname{caml\_raise\_exn} which is
            \begin{itemize}
              \item An assembly function provided by the runtime
              \item Architecture specific
            \end{itemize}
          \item Let's see what it does differently from \funcname{raise\_notrace}\dots
          \item We are about to raise \typename{ExnC} with \funcname{caml\_raise\_exn} from \funcname{definitely\_raise}
        \end{itemize}
      }
      \onslide*<2>{
        \mintobjdump[firstline=2,highlightlines=14]{../src/backtrace/camlBacktrace\_\_definitely\_raise\_347.objdump}
      }
      \vfill
      \mintocaml[firstline=9,lastline=13,highlightlines=12]{../src/backtrace/backtrace.ml}
      \endminipage
    \end{column}
    \begin{column}{0.5\textwidth}
      \centering
      \providecommand\step{1}\input{diagrams/backtrace/backtrace.tex}
    \end{column}
  \end{columns}
\end{frame}

\begin{frame}{Backtraces}{But before that, we need more fields from \typename{caml\_domain\_state}}
  \begin{columns}
    \begin{column}{0.5\textwidth}
      \begin{itemize}
        \item Remember \typename{caml\_domain\_state} the per-domain runtime state?
        \item Some other members are of interest to us now\dots
        \item \localname{backtrace\_active} is used to know if the runtime should collect backtraces
        \item \localname{backtrace\_active} can be set by
          \begin{itemize}
            \item The \funcarg{b} flag in the \funcname{OCAMLRUNPARAM} environment variable
            \item \mintinline{ocaml}{Printexc.record_backtrace : bool -> unit} \footnotemark
          \end{itemize}
      \end{itemize}
    \end{column}
    \begin{column}{0.5\textwidth}
      \begin{listing}
        \mintc[highlightlines={9-11}]{src/ocaml/domain\_state\_backtrace.h}
        \caption{runtime/caml/domain\_state.h}
      \end{listing}
    \end{column}
  \end{columns}
  \footnotetext[1]{https://v2.ocaml.org/api/Printexc.html}
\end{frame}

\newcommand\listCamlRaiseExn[1][]{
  \listasm[firstline=702,lastline=725,#1]{../ocaml/runtime/amd64.S}{runtime/amd64.S:702}}

\begin{frame}{Backtraces}{Walk through \funcname{caml\_raise\_exn}}
  \begin{columns}[T]
    \begin{column}{0.5\textwidth}
      \begin{itemize}
        \item<1-> First checks whether exceptions backtrace should be recorded
        \item<1-> By checking the \funcarg{domain\_state->backtrace\_active} value
        \item<2-> If not, acts as \funcname{raise\_notrace} with \funcname{RESTORE\_EXN\_HANDLER\_OCAML}
          \begin{itemize}
            \item Set \regname{RSP} to \localname{domain\_state->exn\_handler} value
            \item Restore \localname{domain\_state->exn\_handler} from trap
            \item Jump with \funcname{ret} (same effect as using \funcname{pop} and \funcname{jmp})
          \end{itemize}
        \item<3-> Otherwise, switches to the C stack to be able to execute C code
        \item<4-> Calls \funcname{caml\_stash\_backtrace}, a C function provided by the runtime\ignorespaces
          \onslide<6->{, with
            \begin{itemize}
              \item \funcarg{exn}: exception value
              \item \funcarg{pc}: code address of the raising point
              \item \funcarg{sp}: stack address of the raising function
              \item \funcarg{trapsp}: stack address of the next handler
            \end{itemize}
          }
      \end{itemize}
      \bigskip
      \mintocaml[firstline=9,lastline=13,highlightlines=12]{../src/backtrace/backtrace.ml}
    \end{column}
    \begin{column}{0.5\textwidth}
      \raggedleft
      \onslide*<1,3-4>{
        \mintc[firstline=190,lastline=192]{../ocaml/runtime/amd64.S}
        \mintc[firstline=234,lastline=237]{../ocaml/runtime/amd64.S}
      }
      \onslide*<2>{
        \mintc[firstline=190,lastline=192,highlightlines={191-192}]{../ocaml/runtime/amd64.S}
        \mintc[firstline=234,lastline=237,highlightlines={235-237}]{../ocaml/runtime/amd64.S}
      }
      \onslide*<1>{\listCamlRaiseExn[highlightlines=705]{}}%
      \onslide*<2>{\listCamlRaiseExn[highlightlines={707-708}]{}}%
      \onslide*<3>{\listCamlRaiseExn[highlightlines={712-714}]{}}%
      \onslide*<4>{\listCamlRaiseExn[highlightlines={715-720}]{}}%
      \onslide*<5>{\providecommand\step{1}\input{diagrams/backtrace/backtrace.tex}}%
      \onslide*<6>{\providecommand\step{2}\input{diagrams/backtrace/backtrace.tex}}%
    \end{column}
  \end{columns}
\end{frame}

\newcommand\listStashBacktrace[1][]{
  \listc[firstline=91,lastline=120,#1]{../ocaml/runtime/backtrace\_nat.c}{runtime/backtrace\_nat.c:91}}

\begin{frame}{Backtraces}{Introducing \funcname{caml\_stash\_backtrace}}
  \begin{columns}[c]
    \begin{column}{0.5\textwidth}
      \minipage[c][0.66\textheight][s]{\columnwidth}
      \begin{itemize}
        \item<1-> A C function provided by the runtime (specific to native code)
        \item<2-> It scans the stack, between the stack pointer at the raising point up to the next trap, in order to collect the names of the detected function frames
          \onslide*<2>{
            \begin{itemize}
              \item And more information, like line number, column number...
            \end{itemize}
          }
        \item<3-> It uses the same technique and information as the Garbage Collector (GC) uses to scan the values live on the stack
          \onslide*<4>{
            \begin{itemize}
              \item \typename{frame\_descr} are at the core of stack unwinding
            \end{itemize}
          }
      \end{itemize}
      \vfill
      \mintocaml[firstline=9,lastline=13,highlightlines=12]{../src/backtrace/backtrace.ml}
      \endminipage
    \end{column}
    \begin{column}{0.5\textwidth}
      \onslide*<1>{\listStashBacktrace[highlightlines=91]{}}
      \onslide*<2>{\listStashBacktrace[highlightlines={91,117-118}]{}}
      \onslide*<3->{\listStashBacktrace[highlightlines={94,106,109-110}]{}}
    \end{column}
  \end{columns}
\end{frame}

\newcommand\listFrameDescr[1][]{
  \listocaml[firstline=111,lastline=124,#1]{../ocaml/asmcomp/emitaux.ml}{asmcomp/emitaux.ml:111}}
\newcommand\listDebugInfo[1][]{
  \listocaml[firstline=38,lastline=49,#1]{../ocaml/lambda/debuginfo.mli}{lambda/debuginfo.mli:38}}

\begin{frame}{Backtraces}{\typename{frame\_descr} and \typename{frame\_debuginfo} in a nutshell}
  \begin{columns}[c]
    \begin{column}{0.5\textwidth}
      \begin{itemize}
        \item<1-> For every return address in an OCaml program, there is a associated \typename{frame\_descr}
        \item<2-> A \typename{frame\_descr} specifies
        \begin{itemize}
          \item<2-> The size of the function stack frame
          \item<3-> Where live values are on the stack (so that the GC can mark them as live and prevent from being swept)
        \end{itemize}
        \item<4-> When compiled with debug information, \typename{frame\_descr} will be linked to a \typename{frame\_debuginfo}
        \begin{itemize}
          \item<5-> Contains information about the position in the source code (file name, line and column number)
        \end{itemize}
      \end{itemize}
    \end{column}
    \begin{column}{0.5\textwidth}
        \onslide*<1>{\listFrameDescr[highlightlines={118-122}]{}}
        \onslide*<2>{\listFrameDescr[highlightlines={120}]{}}
        \onslide*<3>{\listFrameDescr[highlightlines={121}]{}}
        \onslide*<4->{\listFrameDescr[highlightlines={122}]{}}
        \medskip
        \onslide*<1-3>{\listDebugInfo{}}
        \onslide*<4>{\listDebugInfo[highlightlines={38,49}]{}}
        \onslide*<5>{\listDebugInfo[highlightlines={39-46}]{}}
    \end{column}
  \end{columns}
\end{frame}

\begin{frame}{Backtraces}{Scanning the stack with \typename{frame\_descr}}
  \begin{columns}[c]
    \begin{column}{0.5\textwidth}
      \begin{itemize}
        \item Given the return address of the code that raises the exception by calling \funcname{caml\_raise\_exn} (\funcarg{pc} argument of \funcname{caml\_stash\_backtrace})
        \item And the position of the stack pointer (\funcarg{sp} argument of \funcname{caml\_stash\_backtrace})
        \item The frame size given by \typename{frame\_descr}.\funcarg{frame\_size}, allows to find the end of the previous frame
        \item And the return address of the caller of this function
        \bigskip
        \onslide<3->{
        \item Note that traps (here the reddish \funcname{ExnA}, \funcname{ExnB} and \funcname{ExnC} blocks) belong to the frame that installed them, and so the frame size changes when traps are removed
        }
      \end{itemize}
    \end{column}
    \begin{column}{0.5\textwidth}
      \raggedleft
      \onslide*<1>{\providecommand\step{2}\input{diagrams/backtrace/backtrace.tex}}%
      \onslide*<2>{\providecommand\step{1}\input{diagrams/backtrace/scanstack.tex}}%
      \onslide*<3>{\providecommand\step{2}\input{diagrams/backtrace/scanstack.tex}}%
      \onslide*<4>{\providecommand\step{3}\input{diagrams/backtrace/scanstack.tex}}%
      \onslide*<5>{\providecommand\step{4}\input{diagrams/backtrace/scanstack.tex}}%
      \onslide*<6>{\providecommand\step{5}\input{diagrams/backtrace/scanstack.tex}}%
    \end{column}
  \end{columns}
\end{frame}

% Duplicate of previous-previous slide
\begin{frame}{Backtraces}{Continuing on \funcname{caml\_stash\_backtrace}}
  \begin{columns}[c]
    \begin{column}{0.5\textwidth}
      \minipage[c][0.75\textheight][s]{\columnwidth}
      \begin{itemize}
          \color{gray}
        \item A C function provided by the runtime (specific to native code)
        \item It scans the stack, between the stack pointer at the raising point up to the next trap, in order to collect the names of the detected function frames
        \item It uses the same technique and information as the Garbage Collector (GC) uses to scan the values live on the stack
      \end{itemize}
      \begin{itemize}
        \item For each function frame detected, store a pointer to the \typename{frame\_descr} in the \localname{domain\_state->backtrace\_buffer} array, effectively constructing the backtrace
      \end{itemize}
      \onslide<2->{
        \begin{itemize}
            \smallskip
          \item Constructed backtrace:
            \funcname{definitely\_raise},
            \onslide<3->{\funcname{probably\_raise}}
        \end{itemize}
      }
      \vfill
      \mintocaml[firstline=9,lastline=13,highlightlines=12]{../src/backtrace/backtrace.ml}
      \endminipage
    \end{column}
    \begin{column}{0.5\textwidth}
      \raggedleft
      \onslide*<1>{\listStashBacktrace[highlightlines={114-115}]{}}
      \onslide*<2>{\providecommand\step{3}\input{diagrams/backtrace/backtrace.tex}}%
      \onslide*<3>{\providecommand\step{4}\input{diagrams/backtrace/backtrace.tex}}%
    \end{column}
  \end{columns}
\end{frame}

\begin{frame}{Backtraces}{Back to \funcname{caml\_raise\_exn}}
  \begin{columns}[c]
    \begin{column}{0.5\textwidth}
      \minipage[c][0.66\textheight][s]{\columnwidth}
      \begin{itemize}\color{gray}
        \item Checks wheteher exceptions backtrace should be recorded, by checking the \funcarg{domain\_state->backtrace\_active} value
        \item Switches to the C stack to be able to execute C code
        \item Calls \funcname{caml\_stash\_backtrace}, to collect the backtrace up to the next trap
      \end{itemize}
      \onslide*<2->{
        \begin{itemize}
          \item<2-> Executes the exception handler in \funcname{probably\_raise} with \funcname{RESTORE\_EXN\_HANDLER\_OCAML}
            \begin{itemize}
              \item Set \regname{RSP} to \localname{domain\_state->exn\_handler} value
              \item Restore \localname{domain\_state->exn\_handler} from trap
              \item Jump to the handler code with \funcname{ret}
            \end{itemize}
        \end{itemize}
      }
      \vfill
      \onslide*<1>{\mintocaml[firstline=9,lastline=13,highlightlines=12]{../src/backtrace/backtrace.ml}}
      \onslide*<2->{\mintocaml[firstline=15,lastline=21,highlightlines={19-21}]{../src/backtrace/backtrace.ml}}
      \endminipage
    \end{column}
    \begin{column}{0.5\textwidth}
      \centering
      \onslide*<1>{
        \mintc[firstline=190,lastline=192]{../ocaml/runtime/amd64.S}
        \mintc[firstline=234,lastline=237]{../ocaml/runtime/amd64.S}
      }
      \onslide*<2>{
        \mintc[firstline=190,lastline=192,highlightlines={191-192}]{../ocaml/runtime/amd64.S}
        \mintc[firstline=234,lastline=237,highlightlines={235-237}]{../ocaml/runtime/amd64.S}
      }
      \onslide*<1>{\listCamlRaiseExn[highlightlines=720]{}}
      \onslide*<2>{\listCamlRaiseExn[highlightlines=722]{}}
      \onslide*<3->{\providecommand\step{5}\input{diagrams/backtrace/backtrace.tex}}
    \end{column}
  \end{columns}
\end{frame}

\begin{frame}{Backtraces}{\funcname{caml\_probably\_raise}'s handler doesn't catch \typename{ExnC}}
  \begin{columns}[c]
    \begin{column}{0.45\textwidth}
      \minipage[c][0.75\textheight][s]{\columnwidth}
      \begin{itemize}
        \item<1-> \funcname{match \dots with}\footnotemark pattern matching can also match exceptions
        \item<1-> The CMM of \funcname{probably\_raise} shows that this \funcname{match \dots with} is implemented with a pattern matching and a \funcname{try \dots with}
        \item<1-> But this one only matches on \typename{ExnA} exception
        \item<2-> Since the pattern matching fails to catch the exception, \funcname{reraise} is called with our current \typename{ExnC} exception
        \item<3-> Disassembly shows that \funcname{reraise} is actually a call to \funcname{caml\_reraise\_exn}, which is also an assembly function provided by the runtime
      \end{itemize}
      \vfill
      \mintocaml[firstline=15,lastline=21,highlightlines={18-21}]{../src/backtrace/backtrace.ml}
      \endminipage
    \end{column}
    \begin{column}{0.55\textwidth}
      \centering
      \onslide*<1>{\mintcmm[highlightlines={10-18}]{../src/backtrace/camlBacktrace\_\_probably\_raise\_351.cmm}}
      \onslide*<2>{\listcmm[highlightlines=18]{../src/backtrace/camlBacktrace\_\_probably\_raise_351.cmm}{CMM of probably\_raise}}
      \onslide*<3>{\mintobjdump[firstline=5,lastline=35,highlightlines=31]{../src/backtrace/camlBacktrace\_\_probably\_raise_351.objdump}}
    \end{column}
  \end{columns}
  \only<1>{\footnotetext[1]{https://blog.janestreet.com/pattern-matching-and-exception-handling-unite/}}
\end{frame}

\newcommand\listCamlReraiseExn[1][]{
  \listasm[firstline=727,lastline=734,#1]{../ocaml/runtime/amd64.S}{runtime/amd64.S:727}}

\begin{frame}{Backtraces}{Introducing \funcname{caml\_reraise\_exn}}
  \begin{columns}[c]
    \begin{column}{0.5\textwidth}
      \minipage[c][0.66\textheight][s]{\columnwidth}
      \begin{itemize}
        \item<1-> The exception have to be reraised to the next handler
        \item<2-> First, checks if the backtrace should be collected with \funcarg{domain\_state->backtrace\_active}
        \only<3>{
        \begin{itemize}
            \item If not, behaves like a \funcname{raise\_notrace} with \funcname{RESTORE\_EXN\_HANDLER\_OCAML}
        \end{itemize}
        }
        \item<4-> Jumps into \funcname{caml\_raise\_exn}
        \item<5-> Calls \funcname{caml\_stash\_backtrace} again, to scan the next part of the stack: from the trap just pop-ed to the next trap
      \end{itemize}
      \vfill
      \mintocaml[firstline=15,lastline=21,highlightlines={18-21}]{../src/backtrace/backtrace.ml}
      \endminipage
    \end{column}
    \begin{column}{0.5\textwidth}
      \raggedleft
      \onslide*<1>{\listCamlReraiseExn{}}
      \onslide*<2>{\listCamlReraiseExn[highlightlines=729]{}}
      \onslide*<3>{
        \mintc[firstline=190,lastline=192,highlightlines={191-192}]{../ocaml/runtime/amd64.S}
        \mintc[firstline=234,lastline=237,highlightlines={235-237}]{../ocaml/runtime/amd64.S}
        \medskip
        \listCamlReraiseExn[highlightlines=731]{}
      }
      \onslide*<4-5>{
        % TODO refactor with \listCamlReraiseExn
        \mintasm[firstline=727,lastline=734,highlightlines=730]{../ocaml/runtime/amd64.S}
        \medskip
      }
      \onslide*<4>{\listCamlRaiseExn[highlightlines=711]{}}
      \onslide*<5>{\listCamlRaiseExn[highlightlines=720]{}}
    \end{column}
  \end{columns}
\end{frame}

\begin{frame}{Backtraces}{Backtrace collection continues}
  \begin{columns}[c]
    \begin{column}{0.55\textwidth}
      \minipage[c][0.75\textheight][s]{\columnwidth}
      \begin{itemize}
        \item<1-> \funcname{caml\_stash\_backtrace} scans the older part of the stack, up to the next trap
        \item<6-> Controls jumps to the nested exception handler in \funcname{maybe\_raise}, which matches only on \typename{ExnB}
        \item<7-> \funcname{caml\_reraise\_exn} is called again, backtrace collection resumes with \funcname{caml\_stash\_backtrace}
      \end{itemize}
      \medskip
      \begin{itemize}
        \item<2-> Constructed backtrace:
        \begin{itemize}
          \item <2-> \funcname{definitely\_raise}
          \item <2-> \funcname{probably\_raise}
          \item <3-> \funcname{probably\_raise} (re-raise)
          \item <4-> \funcname{maybe\_raise}
          \item <5-> \funcname{main}
          \item <8-> \funcname{main} (re-raise)
        \end{itemize}
      \end{itemize}
      \vfill
      \onslide*<1-5>{\mintocaml[firstline=15,lastline=21]{../src/backtrace/backtrace.ml}}
      \onslide*<6>{\mintocaml[firstline=28,lastline=34,highlightlines=31]{../src/backtrace/backtrace.ml}}
      \onslide*<7->{\mintocaml[firstline=28,lastline=34,highlightlines=33]{../src/backtrace/backtrace.ml}}
      \endminipage
    \end{column}
    \begin{column}{0.45\textwidth}
      \raggedleft
      \onslide*<1>{\providecommand\step{6}\input{diagrams/backtrace/backtrace.tex}}%
      \onslide*<2>{\providecommand\step{6}\input{diagrams/backtrace/backtrace.tex}}%
      \onslide*<3>{\providecommand\step{7}\input{diagrams/backtrace/backtrace.tex}}%
      \onslide*<4>{\providecommand\step{8}\input{diagrams/backtrace/backtrace.tex}}%
      \onslide*<5>{\providecommand\step{9}\input{diagrams/backtrace/backtrace.tex}}%
      \onslide*<6>{\providecommand\step{10}\input{diagrams/backtrace/backtrace.tex}}%
      \onslide*<7>{\providecommand\step{11}\input{diagrams/backtrace/backtrace.tex}}%
      \onslide*<8>{\providecommand\step{12}\input{diagrams/backtrace/backtrace.tex}}%
      \onslide*<9>{\providecommand\step{13}\input{diagrams/backtrace/backtrace.tex}}%
    \end{column}
  \end{columns}
\end{frame}

\begin{frame}{Backtraces}{Printing the backtrace}
  \begin{columns}[c]
    \begin{column}{0.45\textwidth}
      \minipage[c][0.66\textheight][s]{\columnwidth}
      \begin{itemize}
        \item<1-> The exception backtrace can now be printed to \funcarg{stdout} with \funcname{Printexc.print_backtrace}
        \item<2-> First the exception backtrace is retrieved into an OCaml array with \funcname{get_raw_backtrace }
        \item<3-> Which is implemented as the \funcname{caml_get_exception_raw_backtrace} C function in the runtime
        \item<4-> And finally printed with \funcname{print_exception_backtrace}
      \end{itemize}
      \vfill
      \mintocaml[firstline=28,lastline=34,highlightlines=33]{../src/backtrace/backtrace.ml}
      \endminipage
    \end{column}
    \begin{column}{0.55\textwidth}
      \onslide*<1,2>{
        \mintocaml[firstline=100,lastline=101]{../ocaml/stdlib/printexc.ml}
      }
      \only<1>{
        \listocaml[firstline=170,lastline=172]{../ocaml/stdlib/printexc.ml}{stdlib/printexc.ml:170}
      }
      \only<2>{
        \listocaml[firstline=170,lastline=172,highlightlines=172]{../ocaml/stdlib/printexc.ml}{stdlib/printexc.ml:170}
      }
      \only<3>{
        \mintc[firstline=154,lastline=156]{../ocaml/runtime/backtrace.c}
        \listc[firstline=171,lastline=192,highlightlines={184-187}]{../ocaml/runtime/backtrace.c}{runtime/backtrace.c:154}
      }
      \only<4>{
        \listocaml[firstline=155,lastline=165]{../ocaml/stdlib/printexc.ml}{stdlib/printexc.ml:55}
      }
    \end{column}
  \end{columns}
\end{frame}


\subsection{The multiple kinds of raise}
\frameSubsection{}{}

\begin{frame}{Backtraces}{When backtraces are not needed}
  \begin{columns}[c]
    \begin{column}{0.45\textwidth}
      \minipage[c][0.66\textheight][s]{\columnwidth}
      \begin{itemize}
        \item<1-> What if the backtrace for an exception is never needed?
        \item<2-> It's possible to raise the exception explicitly with \funcname{raise\_notrace} to make raising as cheap as possible
        \item<3-> An example of use in the \funcname{dynlink} library of the OCaml compiler
      \end{itemize}
      \vfill
      \onslide<3->{
      \listocaml[firstline=205,lastline=209,highlightlines=207]{../ocaml/otherlibs/dynlink/dynlink\_compilerlibs/misc.ml}{otherlibs/dynlink/dynlink\_compilerlibs/misc.ml:205}
      }
      \endminipage
    \end{column}
    \begin{column}{0.55\textwidth}
    \only<4>{
      \mintcmm[highlightlines=3]{../src/notrace/camlNotrace\_\_fun_347.cmm}
      \listcmm{../src/notrace/camlNotrace\_\_all\_somes\_327.cmm}{all\_somes}
    }
    \onslide*<5->{
      \listobjdump[highlightlines={6-9}]{../src/notrace/camlNotrace\_\_fun_347.objdump}{anonymous function from \funcname{all\_somes}}
    }
    \end{column}
  \end{columns}
\end{frame}

\begin{frame}{Backtraces}{Summary table of the multiple kinds of raise}
  \begin{itemize}
    \item When debugging information is enabled and if the backtrace of an exception isn't needed, it's possible to raise with \funcname{raise\_notrace} for performance
    \item When debugging information is \emph{not} enabled, the compiler optimises \funcname{raise} calls to inline assembly (same effect as using \funcname{raise\_notrace})
  \end{itemize}
  \bigskip
  \centering
  \begin{tabular}{| l | c | c | c | c |}
    \hline
    \parbox{10em}{Debug information \\ (\funcarg{-g} flag)}  & \multicolumn{2}{|c|}{Disabled}                                     & \multicolumn{2}{|c|}{Enabled} \\
    \hline
    Raise kind                                               & \funcname{raise} & \funcname{raise\_notrace}                       & \funcname{raise} & \funcname{raise\_notrace} \\
    \hline
    Compilation                                              & \parbox{8em}{inline assembly\\ (optimization)} & inline assembly   & \funcname{caml\_raise\_exn} & inline assembly \\
    \hline
  \end{tabular}
\end{frame}


%
% Takeaway #5
%
\subsection*{Takeaway \#5}
\frameSubsectionTakeaway{}
\againframe<5>{takeaway}
}%
      \onslide*<9>{\providecommand\step{13}%
% Backtraces
%
\subsection{One advantage of raising with debugging information}
\frameSubsection{}{}

\begin{frame}{Backtraces}{Debugging information \& source code}
  \begin{columns}[c]
    \begin{column}{0.5\textwidth}
      \begin{itemize}
        \item Raising an exception is fun and games, but how do we know where the exception originated? $\rightarrow$ With the backtrace associated with the exception!
        \item In order to get a backtrace from the point where an exception was raised, it's necessary to compile with debugging information enabled (\funcarg{-g} flag for the compiler)
        \item Same example as in the `Nested handlers' section
      \end{itemize}
    \end{column}
    \begin{column}{0.5\textwidth}
      \listocaml[firstline=9,lastline=34]{../src/backtrace/backtrace.ml}{backtrace/backtrace.ml}
    \end{column}
  \end{columns}
\end{frame}

\begin{frame}{Backtraces}{CMM and disassembly of \funcname{definitely\_raise}}
  \begin{columns}[c]
    \begin{column}{0.5\textwidth}
      \minipage[c][0.66\textheight][s]{\columnwidth}
      \begin{itemize}
        \item<1-> An effect of compiling with debugging information (\funcarg{-g}): the CMM no longer uses \funcname{raise\_notrace} to raise the exception but \funcname{raise} instead
        \item<2-> Disassembly shows that CMM \funcname{raise} is actually a call to \funcname{caml\_raise\_exn}, a function in assembler provided by the runtime
      \end{itemize}
      \vfill
      \mintocaml[firstline=9,lastline=13,highlightlines=12]{../src/backtrace/backtrace.ml}
      \endminipage
    \end{column}
    \begin{column}{0.5\textwidth}
      \onslide*<1>{
        \mintcmm[highlightlines=5]{../src/backtrace/camlBacktrace\_\_definitely\_raise\_347.cmm}
      }
      \onslide*<2>{
        \mintobjdump[firstline=2,highlightlines=14]{../src/backtrace/camlBacktrace\_\_definitely\_raise\_347.objdump}
      }
    \end{column}
  \end{columns}
\end{frame}

\begin{frame}{Backtraces}{Introducing \funcname{caml\_raise\_exn}}
  \begin{columns}[c]
    \begin{column}{0.5\textwidth}
      \minipage[c][0.66\textheight][s]{\columnwidth}
      \onslide*<1>{
        \begin{itemize}
          \item Raising the exception is performed using \funcname{caml\_raise\_exn} which is
            \begin{itemize}
              \item An assembly function provided by the runtime
              \item Architecture specific
            \end{itemize}
          \item Let's see what it does differently from \funcname{raise\_notrace}\dots
          \item We are about to raise \typename{ExnC} with \funcname{caml\_raise\_exn} from \funcname{definitely\_raise}
        \end{itemize}
      }
      \onslide*<2>{
        \mintobjdump[firstline=2,highlightlines=14]{../src/backtrace/camlBacktrace\_\_definitely\_raise\_347.objdump}
      }
      \vfill
      \mintocaml[firstline=9,lastline=13,highlightlines=12]{../src/backtrace/backtrace.ml}
      \endminipage
    \end{column}
    \begin{column}{0.5\textwidth}
      \centering
      \providecommand\step{1}\input{diagrams/backtrace/backtrace.tex}
    \end{column}
  \end{columns}
\end{frame}

\begin{frame}{Backtraces}{But before that, we need more fields from \typename{caml\_domain\_state}}
  \begin{columns}
    \begin{column}{0.5\textwidth}
      \begin{itemize}
        \item Remember \typename{caml\_domain\_state} the per-domain runtime state?
        \item Some other members are of interest to us now\dots
        \item \localname{backtrace\_active} is used to know if the runtime should collect backtraces
        \item \localname{backtrace\_active} can be set by
          \begin{itemize}
            \item The \funcarg{b} flag in the \funcname{OCAMLRUNPARAM} environment variable
            \item \mintinline{ocaml}{Printexc.record_backtrace : bool -> unit} \footnotemark
          \end{itemize}
      \end{itemize}
    \end{column}
    \begin{column}{0.5\textwidth}
      \begin{listing}
        \mintc[highlightlines={9-11}]{src/ocaml/domain\_state\_backtrace.h}
        \caption{runtime/caml/domain\_state.h}
      \end{listing}
    \end{column}
  \end{columns}
  \footnotetext[1]{https://v2.ocaml.org/api/Printexc.html}
\end{frame}

\newcommand\listCamlRaiseExn[1][]{
  \listasm[firstline=702,lastline=725,#1]{../ocaml/runtime/amd64.S}{runtime/amd64.S:702}}

\begin{frame}{Backtraces}{Walk through \funcname{caml\_raise\_exn}}
  \begin{columns}[T]
    \begin{column}{0.5\textwidth}
      \begin{itemize}
        \item<1-> First checks whether exceptions backtrace should be recorded
        \item<1-> By checking the \funcarg{domain\_state->backtrace\_active} value
        \item<2-> If not, acts as \funcname{raise\_notrace} with \funcname{RESTORE\_EXN\_HANDLER\_OCAML}
          \begin{itemize}
            \item Set \regname{RSP} to \localname{domain\_state->exn\_handler} value
            \item Restore \localname{domain\_state->exn\_handler} from trap
            \item Jump with \funcname{ret} (same effect as using \funcname{pop} and \funcname{jmp})
          \end{itemize}
        \item<3-> Otherwise, switches to the C stack to be able to execute C code
        \item<4-> Calls \funcname{caml\_stash\_backtrace}, a C function provided by the runtime\ignorespaces
          \onslide<6->{, with
            \begin{itemize}
              \item \funcarg{exn}: exception value
              \item \funcarg{pc}: code address of the raising point
              \item \funcarg{sp}: stack address of the raising function
              \item \funcarg{trapsp}: stack address of the next handler
            \end{itemize}
          }
      \end{itemize}
      \bigskip
      \mintocaml[firstline=9,lastline=13,highlightlines=12]{../src/backtrace/backtrace.ml}
    \end{column}
    \begin{column}{0.5\textwidth}
      \raggedleft
      \onslide*<1,3-4>{
        \mintc[firstline=190,lastline=192]{../ocaml/runtime/amd64.S}
        \mintc[firstline=234,lastline=237]{../ocaml/runtime/amd64.S}
      }
      \onslide*<2>{
        \mintc[firstline=190,lastline=192,highlightlines={191-192}]{../ocaml/runtime/amd64.S}
        \mintc[firstline=234,lastline=237,highlightlines={235-237}]{../ocaml/runtime/amd64.S}
      }
      \onslide*<1>{\listCamlRaiseExn[highlightlines=705]{}}%
      \onslide*<2>{\listCamlRaiseExn[highlightlines={707-708}]{}}%
      \onslide*<3>{\listCamlRaiseExn[highlightlines={712-714}]{}}%
      \onslide*<4>{\listCamlRaiseExn[highlightlines={715-720}]{}}%
      \onslide*<5>{\providecommand\step{1}\input{diagrams/backtrace/backtrace.tex}}%
      \onslide*<6>{\providecommand\step{2}\input{diagrams/backtrace/backtrace.tex}}%
    \end{column}
  \end{columns}
\end{frame}

\newcommand\listStashBacktrace[1][]{
  \listc[firstline=91,lastline=120,#1]{../ocaml/runtime/backtrace\_nat.c}{runtime/backtrace\_nat.c:91}}

\begin{frame}{Backtraces}{Introducing \funcname{caml\_stash\_backtrace}}
  \begin{columns}[c]
    \begin{column}{0.5\textwidth}
      \minipage[c][0.66\textheight][s]{\columnwidth}
      \begin{itemize}
        \item<1-> A C function provided by the runtime (specific to native code)
        \item<2-> It scans the stack, between the stack pointer at the raising point up to the next trap, in order to collect the names of the detected function frames
          \onslide*<2>{
            \begin{itemize}
              \item And more information, like line number, column number...
            \end{itemize}
          }
        \item<3-> It uses the same technique and information as the Garbage Collector (GC) uses to scan the values live on the stack
          \onslide*<4>{
            \begin{itemize}
              \item \typename{frame\_descr} are at the core of stack unwinding
            \end{itemize}
          }
      \end{itemize}
      \vfill
      \mintocaml[firstline=9,lastline=13,highlightlines=12]{../src/backtrace/backtrace.ml}
      \endminipage
    \end{column}
    \begin{column}{0.5\textwidth}
      \onslide*<1>{\listStashBacktrace[highlightlines=91]{}}
      \onslide*<2>{\listStashBacktrace[highlightlines={91,117-118}]{}}
      \onslide*<3->{\listStashBacktrace[highlightlines={94,106,109-110}]{}}
    \end{column}
  \end{columns}
\end{frame}

\newcommand\listFrameDescr[1][]{
  \listocaml[firstline=111,lastline=124,#1]{../ocaml/asmcomp/emitaux.ml}{asmcomp/emitaux.ml:111}}
\newcommand\listDebugInfo[1][]{
  \listocaml[firstline=38,lastline=49,#1]{../ocaml/lambda/debuginfo.mli}{lambda/debuginfo.mli:38}}

\begin{frame}{Backtraces}{\typename{frame\_descr} and \typename{frame\_debuginfo} in a nutshell}
  \begin{columns}[c]
    \begin{column}{0.5\textwidth}
      \begin{itemize}
        \item<1-> For every return address in an OCaml program, there is a associated \typename{frame\_descr}
        \item<2-> A \typename{frame\_descr} specifies
        \begin{itemize}
          \item<2-> The size of the function stack frame
          \item<3-> Where live values are on the stack (so that the GC can mark them as live and prevent from being swept)
        \end{itemize}
        \item<4-> When compiled with debug information, \typename{frame\_descr} will be linked to a \typename{frame\_debuginfo}
        \begin{itemize}
          \item<5-> Contains information about the position in the source code (file name, line and column number)
        \end{itemize}
      \end{itemize}
    \end{column}
    \begin{column}{0.5\textwidth}
        \onslide*<1>{\listFrameDescr[highlightlines={118-122}]{}}
        \onslide*<2>{\listFrameDescr[highlightlines={120}]{}}
        \onslide*<3>{\listFrameDescr[highlightlines={121}]{}}
        \onslide*<4->{\listFrameDescr[highlightlines={122}]{}}
        \medskip
        \onslide*<1-3>{\listDebugInfo{}}
        \onslide*<4>{\listDebugInfo[highlightlines={38,49}]{}}
        \onslide*<5>{\listDebugInfo[highlightlines={39-46}]{}}
    \end{column}
  \end{columns}
\end{frame}

\begin{frame}{Backtraces}{Scanning the stack with \typename{frame\_descr}}
  \begin{columns}[c]
    \begin{column}{0.5\textwidth}
      \begin{itemize}
        \item Given the return address of the code that raises the exception by calling \funcname{caml\_raise\_exn} (\funcarg{pc} argument of \funcname{caml\_stash\_backtrace})
        \item And the position of the stack pointer (\funcarg{sp} argument of \funcname{caml\_stash\_backtrace})
        \item The frame size given by \typename{frame\_descr}.\funcarg{frame\_size}, allows to find the end of the previous frame
        \item And the return address of the caller of this function
        \bigskip
        \onslide<3->{
        \item Note that traps (here the reddish \funcname{ExnA}, \funcname{ExnB} and \funcname{ExnC} blocks) belong to the frame that installed them, and so the frame size changes when traps are removed
        }
      \end{itemize}
    \end{column}
    \begin{column}{0.5\textwidth}
      \raggedleft
      \onslide*<1>{\providecommand\step{2}\input{diagrams/backtrace/backtrace.tex}}%
      \onslide*<2>{\providecommand\step{1}\input{diagrams/backtrace/scanstack.tex}}%
      \onslide*<3>{\providecommand\step{2}\input{diagrams/backtrace/scanstack.tex}}%
      \onslide*<4>{\providecommand\step{3}\input{diagrams/backtrace/scanstack.tex}}%
      \onslide*<5>{\providecommand\step{4}\input{diagrams/backtrace/scanstack.tex}}%
      \onslide*<6>{\providecommand\step{5}\input{diagrams/backtrace/scanstack.tex}}%
    \end{column}
  \end{columns}
\end{frame}

% Duplicate of previous-previous slide
\begin{frame}{Backtraces}{Continuing on \funcname{caml\_stash\_backtrace}}
  \begin{columns}[c]
    \begin{column}{0.5\textwidth}
      \minipage[c][0.75\textheight][s]{\columnwidth}
      \begin{itemize}
          \color{gray}
        \item A C function provided by the runtime (specific to native code)
        \item It scans the stack, between the stack pointer at the raising point up to the next trap, in order to collect the names of the detected function frames
        \item It uses the same technique and information as the Garbage Collector (GC) uses to scan the values live on the stack
      \end{itemize}
      \begin{itemize}
        \item For each function frame detected, store a pointer to the \typename{frame\_descr} in the \localname{domain\_state->backtrace\_buffer} array, effectively constructing the backtrace
      \end{itemize}
      \onslide<2->{
        \begin{itemize}
            \smallskip
          \item Constructed backtrace:
            \funcname{definitely\_raise},
            \onslide<3->{\funcname{probably\_raise}}
        \end{itemize}
      }
      \vfill
      \mintocaml[firstline=9,lastline=13,highlightlines=12]{../src/backtrace/backtrace.ml}
      \endminipage
    \end{column}
    \begin{column}{0.5\textwidth}
      \raggedleft
      \onslide*<1>{\listStashBacktrace[highlightlines={114-115}]{}}
      \onslide*<2>{\providecommand\step{3}\input{diagrams/backtrace/backtrace.tex}}%
      \onslide*<3>{\providecommand\step{4}\input{diagrams/backtrace/backtrace.tex}}%
    \end{column}
  \end{columns}
\end{frame}

\begin{frame}{Backtraces}{Back to \funcname{caml\_raise\_exn}}
  \begin{columns}[c]
    \begin{column}{0.5\textwidth}
      \minipage[c][0.66\textheight][s]{\columnwidth}
      \begin{itemize}\color{gray}
        \item Checks wheteher exceptions backtrace should be recorded, by checking the \funcarg{domain\_state->backtrace\_active} value
        \item Switches to the C stack to be able to execute C code
        \item Calls \funcname{caml\_stash\_backtrace}, to collect the backtrace up to the next trap
      \end{itemize}
      \onslide*<2->{
        \begin{itemize}
          \item<2-> Executes the exception handler in \funcname{probably\_raise} with \funcname{RESTORE\_EXN\_HANDLER\_OCAML}
            \begin{itemize}
              \item Set \regname{RSP} to \localname{domain\_state->exn\_handler} value
              \item Restore \localname{domain\_state->exn\_handler} from trap
              \item Jump to the handler code with \funcname{ret}
            \end{itemize}
        \end{itemize}
      }
      \vfill
      \onslide*<1>{\mintocaml[firstline=9,lastline=13,highlightlines=12]{../src/backtrace/backtrace.ml}}
      \onslide*<2->{\mintocaml[firstline=15,lastline=21,highlightlines={19-21}]{../src/backtrace/backtrace.ml}}
      \endminipage
    \end{column}
    \begin{column}{0.5\textwidth}
      \centering
      \onslide*<1>{
        \mintc[firstline=190,lastline=192]{../ocaml/runtime/amd64.S}
        \mintc[firstline=234,lastline=237]{../ocaml/runtime/amd64.S}
      }
      \onslide*<2>{
        \mintc[firstline=190,lastline=192,highlightlines={191-192}]{../ocaml/runtime/amd64.S}
        \mintc[firstline=234,lastline=237,highlightlines={235-237}]{../ocaml/runtime/amd64.S}
      }
      \onslide*<1>{\listCamlRaiseExn[highlightlines=720]{}}
      \onslide*<2>{\listCamlRaiseExn[highlightlines=722]{}}
      \onslide*<3->{\providecommand\step{5}\input{diagrams/backtrace/backtrace.tex}}
    \end{column}
  \end{columns}
\end{frame}

\begin{frame}{Backtraces}{\funcname{caml\_probably\_raise}'s handler doesn't catch \typename{ExnC}}
  \begin{columns}[c]
    \begin{column}{0.45\textwidth}
      \minipage[c][0.75\textheight][s]{\columnwidth}
      \begin{itemize}
        \item<1-> \funcname{match \dots with}\footnotemark pattern matching can also match exceptions
        \item<1-> The CMM of \funcname{probably\_raise} shows that this \funcname{match \dots with} is implemented with a pattern matching and a \funcname{try \dots with}
        \item<1-> But this one only matches on \typename{ExnA} exception
        \item<2-> Since the pattern matching fails to catch the exception, \funcname{reraise} is called with our current \typename{ExnC} exception
        \item<3-> Disassembly shows that \funcname{reraise} is actually a call to \funcname{caml\_reraise\_exn}, which is also an assembly function provided by the runtime
      \end{itemize}
      \vfill
      \mintocaml[firstline=15,lastline=21,highlightlines={18-21}]{../src/backtrace/backtrace.ml}
      \endminipage
    \end{column}
    \begin{column}{0.55\textwidth}
      \centering
      \onslide*<1>{\mintcmm[highlightlines={10-18}]{../src/backtrace/camlBacktrace\_\_probably\_raise\_351.cmm}}
      \onslide*<2>{\listcmm[highlightlines=18]{../src/backtrace/camlBacktrace\_\_probably\_raise_351.cmm}{CMM of probably\_raise}}
      \onslide*<3>{\mintobjdump[firstline=5,lastline=35,highlightlines=31]{../src/backtrace/camlBacktrace\_\_probably\_raise_351.objdump}}
    \end{column}
  \end{columns}
  \only<1>{\footnotetext[1]{https://blog.janestreet.com/pattern-matching-and-exception-handling-unite/}}
\end{frame}

\newcommand\listCamlReraiseExn[1][]{
  \listasm[firstline=727,lastline=734,#1]{../ocaml/runtime/amd64.S}{runtime/amd64.S:727}}

\begin{frame}{Backtraces}{Introducing \funcname{caml\_reraise\_exn}}
  \begin{columns}[c]
    \begin{column}{0.5\textwidth}
      \minipage[c][0.66\textheight][s]{\columnwidth}
      \begin{itemize}
        \item<1-> The exception have to be reraised to the next handler
        \item<2-> First, checks if the backtrace should be collected with \funcarg{domain\_state->backtrace\_active}
        \only<3>{
        \begin{itemize}
            \item If not, behaves like a \funcname{raise\_notrace} with \funcname{RESTORE\_EXN\_HANDLER\_OCAML}
        \end{itemize}
        }
        \item<4-> Jumps into \funcname{caml\_raise\_exn}
        \item<5-> Calls \funcname{caml\_stash\_backtrace} again, to scan the next part of the stack: from the trap just pop-ed to the next trap
      \end{itemize}
      \vfill
      \mintocaml[firstline=15,lastline=21,highlightlines={18-21}]{../src/backtrace/backtrace.ml}
      \endminipage
    \end{column}
    \begin{column}{0.5\textwidth}
      \raggedleft
      \onslide*<1>{\listCamlReraiseExn{}}
      \onslide*<2>{\listCamlReraiseExn[highlightlines=729]{}}
      \onslide*<3>{
        \mintc[firstline=190,lastline=192,highlightlines={191-192}]{../ocaml/runtime/amd64.S}
        \mintc[firstline=234,lastline=237,highlightlines={235-237}]{../ocaml/runtime/amd64.S}
        \medskip
        \listCamlReraiseExn[highlightlines=731]{}
      }
      \onslide*<4-5>{
        % TODO refactor with \listCamlReraiseExn
        \mintasm[firstline=727,lastline=734,highlightlines=730]{../ocaml/runtime/amd64.S}
        \medskip
      }
      \onslide*<4>{\listCamlRaiseExn[highlightlines=711]{}}
      \onslide*<5>{\listCamlRaiseExn[highlightlines=720]{}}
    \end{column}
  \end{columns}
\end{frame}

\begin{frame}{Backtraces}{Backtrace collection continues}
  \begin{columns}[c]
    \begin{column}{0.55\textwidth}
      \minipage[c][0.75\textheight][s]{\columnwidth}
      \begin{itemize}
        \item<1-> \funcname{caml\_stash\_backtrace} scans the older part of the stack, up to the next trap
        \item<6-> Controls jumps to the nested exception handler in \funcname{maybe\_raise}, which matches only on \typename{ExnB}
        \item<7-> \funcname{caml\_reraise\_exn} is called again, backtrace collection resumes with \funcname{caml\_stash\_backtrace}
      \end{itemize}
      \medskip
      \begin{itemize}
        \item<2-> Constructed backtrace:
        \begin{itemize}
          \item <2-> \funcname{definitely\_raise}
          \item <2-> \funcname{probably\_raise}
          \item <3-> \funcname{probably\_raise} (re-raise)
          \item <4-> \funcname{maybe\_raise}
          \item <5-> \funcname{main}
          \item <8-> \funcname{main} (re-raise)
        \end{itemize}
      \end{itemize}
      \vfill
      \onslide*<1-5>{\mintocaml[firstline=15,lastline=21]{../src/backtrace/backtrace.ml}}
      \onslide*<6>{\mintocaml[firstline=28,lastline=34,highlightlines=31]{../src/backtrace/backtrace.ml}}
      \onslide*<7->{\mintocaml[firstline=28,lastline=34,highlightlines=33]{../src/backtrace/backtrace.ml}}
      \endminipage
    \end{column}
    \begin{column}{0.45\textwidth}
      \raggedleft
      \onslide*<1>{\providecommand\step{6}\input{diagrams/backtrace/backtrace.tex}}%
      \onslide*<2>{\providecommand\step{6}\input{diagrams/backtrace/backtrace.tex}}%
      \onslide*<3>{\providecommand\step{7}\input{diagrams/backtrace/backtrace.tex}}%
      \onslide*<4>{\providecommand\step{8}\input{diagrams/backtrace/backtrace.tex}}%
      \onslide*<5>{\providecommand\step{9}\input{diagrams/backtrace/backtrace.tex}}%
      \onslide*<6>{\providecommand\step{10}\input{diagrams/backtrace/backtrace.tex}}%
      \onslide*<7>{\providecommand\step{11}\input{diagrams/backtrace/backtrace.tex}}%
      \onslide*<8>{\providecommand\step{12}\input{diagrams/backtrace/backtrace.tex}}%
      \onslide*<9>{\providecommand\step{13}\input{diagrams/backtrace/backtrace.tex}}%
    \end{column}
  \end{columns}
\end{frame}

\begin{frame}{Backtraces}{Printing the backtrace}
  \begin{columns}[c]
    \begin{column}{0.45\textwidth}
      \minipage[c][0.66\textheight][s]{\columnwidth}
      \begin{itemize}
        \item<1-> The exception backtrace can now be printed to \funcarg{stdout} with \funcname{Printexc.print_backtrace}
        \item<2-> First the exception backtrace is retrieved into an OCaml array with \funcname{get_raw_backtrace }
        \item<3-> Which is implemented as the \funcname{caml_get_exception_raw_backtrace} C function in the runtime
        \item<4-> And finally printed with \funcname{print_exception_backtrace}
      \end{itemize}
      \vfill
      \mintocaml[firstline=28,lastline=34,highlightlines=33]{../src/backtrace/backtrace.ml}
      \endminipage
    \end{column}
    \begin{column}{0.55\textwidth}
      \onslide*<1,2>{
        \mintocaml[firstline=100,lastline=101]{../ocaml/stdlib/printexc.ml}
      }
      \only<1>{
        \listocaml[firstline=170,lastline=172]{../ocaml/stdlib/printexc.ml}{stdlib/printexc.ml:170}
      }
      \only<2>{
        \listocaml[firstline=170,lastline=172,highlightlines=172]{../ocaml/stdlib/printexc.ml}{stdlib/printexc.ml:170}
      }
      \only<3>{
        \mintc[firstline=154,lastline=156]{../ocaml/runtime/backtrace.c}
        \listc[firstline=171,lastline=192,highlightlines={184-187}]{../ocaml/runtime/backtrace.c}{runtime/backtrace.c:154}
      }
      \only<4>{
        \listocaml[firstline=155,lastline=165]{../ocaml/stdlib/printexc.ml}{stdlib/printexc.ml:55}
      }
    \end{column}
  \end{columns}
\end{frame}


\subsection{The multiple kinds of raise}
\frameSubsection{}{}

\begin{frame}{Backtraces}{When backtraces are not needed}
  \begin{columns}[c]
    \begin{column}{0.45\textwidth}
      \minipage[c][0.66\textheight][s]{\columnwidth}
      \begin{itemize}
        \item<1-> What if the backtrace for an exception is never needed?
        \item<2-> It's possible to raise the exception explicitly with \funcname{raise\_notrace} to make raising as cheap as possible
        \item<3-> An example of use in the \funcname{dynlink} library of the OCaml compiler
      \end{itemize}
      \vfill
      \onslide<3->{
      \listocaml[firstline=205,lastline=209,highlightlines=207]{../ocaml/otherlibs/dynlink/dynlink\_compilerlibs/misc.ml}{otherlibs/dynlink/dynlink\_compilerlibs/misc.ml:205}
      }
      \endminipage
    \end{column}
    \begin{column}{0.55\textwidth}
    \only<4>{
      \mintcmm[highlightlines=3]{../src/notrace/camlNotrace\_\_fun_347.cmm}
      \listcmm{../src/notrace/camlNotrace\_\_all\_somes\_327.cmm}{all\_somes}
    }
    \onslide*<5->{
      \listobjdump[highlightlines={6-9}]{../src/notrace/camlNotrace\_\_fun_347.objdump}{anonymous function from \funcname{all\_somes}}
    }
    \end{column}
  \end{columns}
\end{frame}

\begin{frame}{Backtraces}{Summary table of the multiple kinds of raise}
  \begin{itemize}
    \item When debugging information is enabled and if the backtrace of an exception isn't needed, it's possible to raise with \funcname{raise\_notrace} for performance
    \item When debugging information is \emph{not} enabled, the compiler optimises \funcname{raise} calls to inline assembly (same effect as using \funcname{raise\_notrace})
  \end{itemize}
  \bigskip
  \centering
  \begin{tabular}{| l | c | c | c | c |}
    \hline
    \parbox{10em}{Debug information \\ (\funcarg{-g} flag)}  & \multicolumn{2}{|c|}{Disabled}                                     & \multicolumn{2}{|c|}{Enabled} \\
    \hline
    Raise kind                                               & \funcname{raise} & \funcname{raise\_notrace}                       & \funcname{raise} & \funcname{raise\_notrace} \\
    \hline
    Compilation                                              & \parbox{8em}{inline assembly\\ (optimization)} & inline assembly   & \funcname{caml\_raise\_exn} & inline assembly \\
    \hline
  \end{tabular}
\end{frame}


%
% Takeaway #5
%
\subsection*{Takeaway \#5}
\frameSubsectionTakeaway{}
\againframe<5>{takeaway}
}%
    \end{column}
  \end{columns}
\end{frame}

\begin{frame}{Backtraces}{Printing the backtrace}
  \begin{columns}[c]
    \begin{column}{0.45\textwidth}
      \minipage[c][0.66\textheight][s]{\columnwidth}
      \begin{itemize}
        \item<1-> The exception backtrace can now be printed to \funcarg{stdout} with \funcname{Printexc.print_backtrace}
        \item<2-> First the exception backtrace is retrieved into an OCaml array with \funcname{get_raw_backtrace }
        \item<3-> Which is implemented as the \funcname{caml_get_exception_raw_backtrace} C function in the runtime
        \item<4-> And finally printed with \funcname{print_exception_backtrace}
      \end{itemize}
      \vfill
      \mintocaml[firstline=28,lastline=34,highlightlines=33]{../src/backtrace/backtrace.ml}
      \endminipage
    \end{column}
    \begin{column}{0.55\textwidth}
      \onslide*<1,2>{
        \mintocaml[firstline=100,lastline=101]{../ocaml/stdlib/printexc.ml}
      }
      \only<1>{
        \listocaml[firstline=170,lastline=172]{../ocaml/stdlib/printexc.ml}{stdlib/printexc.ml:170}
      }
      \only<2>{
        \listocaml[firstline=170,lastline=172,highlightlines=172]{../ocaml/stdlib/printexc.ml}{stdlib/printexc.ml:170}
      }
      \only<3>{
        \mintc[firstline=154,lastline=156]{../ocaml/runtime/backtrace.c}
        \listc[firstline=171,lastline=192,highlightlines={184-187}]{../ocaml/runtime/backtrace.c}{runtime/backtrace.c:154}
      }
      \only<4>{
        \listocaml[firstline=155,lastline=165]{../ocaml/stdlib/printexc.ml}{stdlib/printexc.ml:55}
      }
    \end{column}
  \end{columns}
\end{frame}


\subsection{The multiple kinds of raise}
\frameSubsection{}{}

\begin{frame}{Backtraces}{When backtraces are not needed}
  \begin{columns}[c]
    \begin{column}{0.45\textwidth}
      \minipage[c][0.66\textheight][s]{\columnwidth}
      \begin{itemize}
        \item<1-> What if the backtrace for an exception is never needed?
        \item<2-> It's possible to raise the exception explicitly with \funcname{raise\_notrace} to make raising as cheap as possible
        \item<3-> An example of use in the \funcname{dynlink} library of the OCaml compiler
      \end{itemize}
      \vfill
      \onslide<3->{
      \listocaml[firstline=205,lastline=209,highlightlines=207]{../ocaml/otherlibs/dynlink/dynlink\_compilerlibs/misc.ml}{otherlibs/dynlink/dynlink\_compilerlibs/misc.ml:205}
      }
      \endminipage
    \end{column}
    \begin{column}{0.55\textwidth}
    \only<4>{
      \mintcmm[highlightlines=3]{../src/notrace/camlNotrace\_\_fun_347.cmm}
      \listcmm{../src/notrace/camlNotrace\_\_all\_somes\_327.cmm}{all\_somes}
    }
    \onslide*<5->{
      \listobjdump[highlightlines={6-9}]{../src/notrace/camlNotrace\_\_fun_347.objdump}{anonymous function from \funcname{all\_somes}}
    }
    \end{column}
  \end{columns}
\end{frame}

\begin{frame}{Backtraces}{Summary table of the multiple kinds of raise}
  \begin{itemize}
    \item When debugging information is enabled and if the backtrace of an exception isn't needed, it's possible to raise with \funcname{raise\_notrace} for performance
    \item When debugging information is \emph{not} enabled, the compiler optimises \funcname{raise} calls to inline assembly (same effect as using \funcname{raise\_notrace})
  \end{itemize}
  \bigskip
  \centering
  \begin{tabular}{| l | c | c | c | c |}
    \hline
    \parbox{10em}{Debug information \\ (\funcarg{-g} flag)}  & \multicolumn{2}{|c|}{Disabled}                                     & \multicolumn{2}{|c|}{Enabled} \\
    \hline
    Raise kind                                               & \funcname{raise} & \funcname{raise\_notrace}                       & \funcname{raise} & \funcname{raise\_notrace} \\
    \hline
    Compilation                                              & \parbox{8em}{inline assembly\\ (optimization)} & inline assembly   & \funcname{caml\_raise\_exn} & inline assembly \\
    \hline
  \end{tabular}
\end{frame}


%
% Takeaway #5
%
\subsection*{Takeaway \#5}
\frameSubsectionTakeaway{}
\againframe<5>{takeaway}
}%
      \onslide*<6>{\providecommand\step{2}%
% Backtraces
%
\subsection{One advantage of raising with debugging information}
\frameSubsection{}{}

\begin{frame}{Backtraces}{Debugging information \& source code}
  \begin{columns}[c]
    \begin{column}{0.5\textwidth}
      \begin{itemize}
        \item Raising an exception is fun and games, but how do we know where the exception originated? $\rightarrow$ With the backtrace associated with the exception!
        \item In order to get a backtrace from the point where an exception was raised, it's necessary to compile with debugging information enabled (\funcarg{-g} flag for the compiler)
        \item Same example as in the `Nested handlers' section
      \end{itemize}
    \end{column}
    \begin{column}{0.5\textwidth}
      \listocaml[firstline=9,lastline=34]{../src/backtrace/backtrace.ml}{backtrace/backtrace.ml}
    \end{column}
  \end{columns}
\end{frame}

\begin{frame}{Backtraces}{CMM and disassembly of \funcname{definitely\_raise}}
  \begin{columns}[c]
    \begin{column}{0.5\textwidth}
      \minipage[c][0.66\textheight][s]{\columnwidth}
      \begin{itemize}
        \item<1-> An effect of compiling with debugging information (\funcarg{-g}): the CMM no longer uses \funcname{raise\_notrace} to raise the exception but \funcname{raise} instead
        \item<2-> Disassembly shows that CMM \funcname{raise} is actually a call to \funcname{caml\_raise\_exn}, a function in assembler provided by the runtime
      \end{itemize}
      \vfill
      \mintocaml[firstline=9,lastline=13,highlightlines=12]{../src/backtrace/backtrace.ml}
      \endminipage
    \end{column}
    \begin{column}{0.5\textwidth}
      \onslide*<1>{
        \mintcmm[highlightlines=5]{../src/backtrace/camlBacktrace\_\_definitely\_raise\_347.cmm}
      }
      \onslide*<2>{
        \mintobjdump[firstline=2,highlightlines=14]{../src/backtrace/camlBacktrace\_\_definitely\_raise\_347.objdump}
      }
    \end{column}
  \end{columns}
\end{frame}

\begin{frame}{Backtraces}{Introducing \funcname{caml\_raise\_exn}}
  \begin{columns}[c]
    \begin{column}{0.5\textwidth}
      \minipage[c][0.66\textheight][s]{\columnwidth}
      \onslide*<1>{
        \begin{itemize}
          \item Raising the exception is performed using \funcname{caml\_raise\_exn} which is
            \begin{itemize}
              \item An assembly function provided by the runtime
              \item Architecture specific
            \end{itemize}
          \item Let's see what it does differently from \funcname{raise\_notrace}\dots
          \item We are about to raise \typename{ExnC} with \funcname{caml\_raise\_exn} from \funcname{definitely\_raise}
        \end{itemize}
      }
      \onslide*<2>{
        \mintobjdump[firstline=2,highlightlines=14]{../src/backtrace/camlBacktrace\_\_definitely\_raise\_347.objdump}
      }
      \vfill
      \mintocaml[firstline=9,lastline=13,highlightlines=12]{../src/backtrace/backtrace.ml}
      \endminipage
    \end{column}
    \begin{column}{0.5\textwidth}
      \centering
      \providecommand\step{1}%
% Backtraces
%
\subsection{One advantage of raising with debugging information}
\frameSubsection{}{}

\begin{frame}{Backtraces}{Debugging information \& source code}
  \begin{columns}[c]
    \begin{column}{0.5\textwidth}
      \begin{itemize}
        \item Raising an exception is fun and games, but how do we know where the exception originated? $\rightarrow$ With the backtrace associated with the exception!
        \item In order to get a backtrace from the point where an exception was raised, it's necessary to compile with debugging information enabled (\funcarg{-g} flag for the compiler)
        \item Same example as in the `Nested handlers' section
      \end{itemize}
    \end{column}
    \begin{column}{0.5\textwidth}
      \listocaml[firstline=9,lastline=34]{../src/backtrace/backtrace.ml}{backtrace/backtrace.ml}
    \end{column}
  \end{columns}
\end{frame}

\begin{frame}{Backtraces}{CMM and disassembly of \funcname{definitely\_raise}}
  \begin{columns}[c]
    \begin{column}{0.5\textwidth}
      \minipage[c][0.66\textheight][s]{\columnwidth}
      \begin{itemize}
        \item<1-> An effect of compiling with debugging information (\funcarg{-g}): the CMM no longer uses \funcname{raise\_notrace} to raise the exception but \funcname{raise} instead
        \item<2-> Disassembly shows that CMM \funcname{raise} is actually a call to \funcname{caml\_raise\_exn}, a function in assembler provided by the runtime
      \end{itemize}
      \vfill
      \mintocaml[firstline=9,lastline=13,highlightlines=12]{../src/backtrace/backtrace.ml}
      \endminipage
    \end{column}
    \begin{column}{0.5\textwidth}
      \onslide*<1>{
        \mintcmm[highlightlines=5]{../src/backtrace/camlBacktrace\_\_definitely\_raise\_347.cmm}
      }
      \onslide*<2>{
        \mintobjdump[firstline=2,highlightlines=14]{../src/backtrace/camlBacktrace\_\_definitely\_raise\_347.objdump}
      }
    \end{column}
  \end{columns}
\end{frame}

\begin{frame}{Backtraces}{Introducing \funcname{caml\_raise\_exn}}
  \begin{columns}[c]
    \begin{column}{0.5\textwidth}
      \minipage[c][0.66\textheight][s]{\columnwidth}
      \onslide*<1>{
        \begin{itemize}
          \item Raising the exception is performed using \funcname{caml\_raise\_exn} which is
            \begin{itemize}
              \item An assembly function provided by the runtime
              \item Architecture specific
            \end{itemize}
          \item Let's see what it does differently from \funcname{raise\_notrace}\dots
          \item We are about to raise \typename{ExnC} with \funcname{caml\_raise\_exn} from \funcname{definitely\_raise}
        \end{itemize}
      }
      \onslide*<2>{
        \mintobjdump[firstline=2,highlightlines=14]{../src/backtrace/camlBacktrace\_\_definitely\_raise\_347.objdump}
      }
      \vfill
      \mintocaml[firstline=9,lastline=13,highlightlines=12]{../src/backtrace/backtrace.ml}
      \endminipage
    \end{column}
    \begin{column}{0.5\textwidth}
      \centering
      \providecommand\step{1}\input{diagrams/backtrace/backtrace.tex}
    \end{column}
  \end{columns}
\end{frame}

\begin{frame}{Backtraces}{But before that, we need more fields from \typename{caml\_domain\_state}}
  \begin{columns}
    \begin{column}{0.5\textwidth}
      \begin{itemize}
        \item Remember \typename{caml\_domain\_state} the per-domain runtime state?
        \item Some other members are of interest to us now\dots
        \item \localname{backtrace\_active} is used to know if the runtime should collect backtraces
        \item \localname{backtrace\_active} can be set by
          \begin{itemize}
            \item The \funcarg{b} flag in the \funcname{OCAMLRUNPARAM} environment variable
            \item \mintinline{ocaml}{Printexc.record_backtrace : bool -> unit} \footnotemark
          \end{itemize}
      \end{itemize}
    \end{column}
    \begin{column}{0.5\textwidth}
      \begin{listing}
        \mintc[highlightlines={9-11}]{src/ocaml/domain\_state\_backtrace.h}
        \caption{runtime/caml/domain\_state.h}
      \end{listing}
    \end{column}
  \end{columns}
  \footnotetext[1]{https://v2.ocaml.org/api/Printexc.html}
\end{frame}

\newcommand\listCamlRaiseExn[1][]{
  \listasm[firstline=702,lastline=725,#1]{../ocaml/runtime/amd64.S}{runtime/amd64.S:702}}

\begin{frame}{Backtraces}{Walk through \funcname{caml\_raise\_exn}}
  \begin{columns}[T]
    \begin{column}{0.5\textwidth}
      \begin{itemize}
        \item<1-> First checks whether exceptions backtrace should be recorded
        \item<1-> By checking the \funcarg{domain\_state->backtrace\_active} value
        \item<2-> If not, acts as \funcname{raise\_notrace} with \funcname{RESTORE\_EXN\_HANDLER\_OCAML}
          \begin{itemize}
            \item Set \regname{RSP} to \localname{domain\_state->exn\_handler} value
            \item Restore \localname{domain\_state->exn\_handler} from trap
            \item Jump with \funcname{ret} (same effect as using \funcname{pop} and \funcname{jmp})
          \end{itemize}
        \item<3-> Otherwise, switches to the C stack to be able to execute C code
        \item<4-> Calls \funcname{caml\_stash\_backtrace}, a C function provided by the runtime\ignorespaces
          \onslide<6->{, with
            \begin{itemize}
              \item \funcarg{exn}: exception value
              \item \funcarg{pc}: code address of the raising point
              \item \funcarg{sp}: stack address of the raising function
              \item \funcarg{trapsp}: stack address of the next handler
            \end{itemize}
          }
      \end{itemize}
      \bigskip
      \mintocaml[firstline=9,lastline=13,highlightlines=12]{../src/backtrace/backtrace.ml}
    \end{column}
    \begin{column}{0.5\textwidth}
      \raggedleft
      \onslide*<1,3-4>{
        \mintc[firstline=190,lastline=192]{../ocaml/runtime/amd64.S}
        \mintc[firstline=234,lastline=237]{../ocaml/runtime/amd64.S}
      }
      \onslide*<2>{
        \mintc[firstline=190,lastline=192,highlightlines={191-192}]{../ocaml/runtime/amd64.S}
        \mintc[firstline=234,lastline=237,highlightlines={235-237}]{../ocaml/runtime/amd64.S}
      }
      \onslide*<1>{\listCamlRaiseExn[highlightlines=705]{}}%
      \onslide*<2>{\listCamlRaiseExn[highlightlines={707-708}]{}}%
      \onslide*<3>{\listCamlRaiseExn[highlightlines={712-714}]{}}%
      \onslide*<4>{\listCamlRaiseExn[highlightlines={715-720}]{}}%
      \onslide*<5>{\providecommand\step{1}\input{diagrams/backtrace/backtrace.tex}}%
      \onslide*<6>{\providecommand\step{2}\input{diagrams/backtrace/backtrace.tex}}%
    \end{column}
  \end{columns}
\end{frame}

\newcommand\listStashBacktrace[1][]{
  \listc[firstline=91,lastline=120,#1]{../ocaml/runtime/backtrace\_nat.c}{runtime/backtrace\_nat.c:91}}

\begin{frame}{Backtraces}{Introducing \funcname{caml\_stash\_backtrace}}
  \begin{columns}[c]
    \begin{column}{0.5\textwidth}
      \minipage[c][0.66\textheight][s]{\columnwidth}
      \begin{itemize}
        \item<1-> A C function provided by the runtime (specific to native code)
        \item<2-> It scans the stack, between the stack pointer at the raising point up to the next trap, in order to collect the names of the detected function frames
          \onslide*<2>{
            \begin{itemize}
              \item And more information, like line number, column number...
            \end{itemize}
          }
        \item<3-> It uses the same technique and information as the Garbage Collector (GC) uses to scan the values live on the stack
          \onslide*<4>{
            \begin{itemize}
              \item \typename{frame\_descr} are at the core of stack unwinding
            \end{itemize}
          }
      \end{itemize}
      \vfill
      \mintocaml[firstline=9,lastline=13,highlightlines=12]{../src/backtrace/backtrace.ml}
      \endminipage
    \end{column}
    \begin{column}{0.5\textwidth}
      \onslide*<1>{\listStashBacktrace[highlightlines=91]{}}
      \onslide*<2>{\listStashBacktrace[highlightlines={91,117-118}]{}}
      \onslide*<3->{\listStashBacktrace[highlightlines={94,106,109-110}]{}}
    \end{column}
  \end{columns}
\end{frame}

\newcommand\listFrameDescr[1][]{
  \listocaml[firstline=111,lastline=124,#1]{../ocaml/asmcomp/emitaux.ml}{asmcomp/emitaux.ml:111}}
\newcommand\listDebugInfo[1][]{
  \listocaml[firstline=38,lastline=49,#1]{../ocaml/lambda/debuginfo.mli}{lambda/debuginfo.mli:38}}

\begin{frame}{Backtraces}{\typename{frame\_descr} and \typename{frame\_debuginfo} in a nutshell}
  \begin{columns}[c]
    \begin{column}{0.5\textwidth}
      \begin{itemize}
        \item<1-> For every return address in an OCaml program, there is a associated \typename{frame\_descr}
        \item<2-> A \typename{frame\_descr} specifies
        \begin{itemize}
          \item<2-> The size of the function stack frame
          \item<3-> Where live values are on the stack (so that the GC can mark them as live and prevent from being swept)
        \end{itemize}
        \item<4-> When compiled with debug information, \typename{frame\_descr} will be linked to a \typename{frame\_debuginfo}
        \begin{itemize}
          \item<5-> Contains information about the position in the source code (file name, line and column number)
        \end{itemize}
      \end{itemize}
    \end{column}
    \begin{column}{0.5\textwidth}
        \onslide*<1>{\listFrameDescr[highlightlines={118-122}]{}}
        \onslide*<2>{\listFrameDescr[highlightlines={120}]{}}
        \onslide*<3>{\listFrameDescr[highlightlines={121}]{}}
        \onslide*<4->{\listFrameDescr[highlightlines={122}]{}}
        \medskip
        \onslide*<1-3>{\listDebugInfo{}}
        \onslide*<4>{\listDebugInfo[highlightlines={38,49}]{}}
        \onslide*<5>{\listDebugInfo[highlightlines={39-46}]{}}
    \end{column}
  \end{columns}
\end{frame}

\begin{frame}{Backtraces}{Scanning the stack with \typename{frame\_descr}}
  \begin{columns}[c]
    \begin{column}{0.5\textwidth}
      \begin{itemize}
        \item Given the return address of the code that raises the exception by calling \funcname{caml\_raise\_exn} (\funcarg{pc} argument of \funcname{caml\_stash\_backtrace})
        \item And the position of the stack pointer (\funcarg{sp} argument of \funcname{caml\_stash\_backtrace})
        \item The frame size given by \typename{frame\_descr}.\funcarg{frame\_size}, allows to find the end of the previous frame
        \item And the return address of the caller of this function
        \bigskip
        \onslide<3->{
        \item Note that traps (here the reddish \funcname{ExnA}, \funcname{ExnB} and \funcname{ExnC} blocks) belong to the frame that installed them, and so the frame size changes when traps are removed
        }
      \end{itemize}
    \end{column}
    \begin{column}{0.5\textwidth}
      \raggedleft
      \onslide*<1>{\providecommand\step{2}\input{diagrams/backtrace/backtrace.tex}}%
      \onslide*<2>{\providecommand\step{1}\input{diagrams/backtrace/scanstack.tex}}%
      \onslide*<3>{\providecommand\step{2}\input{diagrams/backtrace/scanstack.tex}}%
      \onslide*<4>{\providecommand\step{3}\input{diagrams/backtrace/scanstack.tex}}%
      \onslide*<5>{\providecommand\step{4}\input{diagrams/backtrace/scanstack.tex}}%
      \onslide*<6>{\providecommand\step{5}\input{diagrams/backtrace/scanstack.tex}}%
    \end{column}
  \end{columns}
\end{frame}

% Duplicate of previous-previous slide
\begin{frame}{Backtraces}{Continuing on \funcname{caml\_stash\_backtrace}}
  \begin{columns}[c]
    \begin{column}{0.5\textwidth}
      \minipage[c][0.75\textheight][s]{\columnwidth}
      \begin{itemize}
          \color{gray}
        \item A C function provided by the runtime (specific to native code)
        \item It scans the stack, between the stack pointer at the raising point up to the next trap, in order to collect the names of the detected function frames
        \item It uses the same technique and information as the Garbage Collector (GC) uses to scan the values live on the stack
      \end{itemize}
      \begin{itemize}
        \item For each function frame detected, store a pointer to the \typename{frame\_descr} in the \localname{domain\_state->backtrace\_buffer} array, effectively constructing the backtrace
      \end{itemize}
      \onslide<2->{
        \begin{itemize}
            \smallskip
          \item Constructed backtrace:
            \funcname{definitely\_raise},
            \onslide<3->{\funcname{probably\_raise}}
        \end{itemize}
      }
      \vfill
      \mintocaml[firstline=9,lastline=13,highlightlines=12]{../src/backtrace/backtrace.ml}
      \endminipage
    \end{column}
    \begin{column}{0.5\textwidth}
      \raggedleft
      \onslide*<1>{\listStashBacktrace[highlightlines={114-115}]{}}
      \onslide*<2>{\providecommand\step{3}\input{diagrams/backtrace/backtrace.tex}}%
      \onslide*<3>{\providecommand\step{4}\input{diagrams/backtrace/backtrace.tex}}%
    \end{column}
  \end{columns}
\end{frame}

\begin{frame}{Backtraces}{Back to \funcname{caml\_raise\_exn}}
  \begin{columns}[c]
    \begin{column}{0.5\textwidth}
      \minipage[c][0.66\textheight][s]{\columnwidth}
      \begin{itemize}\color{gray}
        \item Checks wheteher exceptions backtrace should be recorded, by checking the \funcarg{domain\_state->backtrace\_active} value
        \item Switches to the C stack to be able to execute C code
        \item Calls \funcname{caml\_stash\_backtrace}, to collect the backtrace up to the next trap
      \end{itemize}
      \onslide*<2->{
        \begin{itemize}
          \item<2-> Executes the exception handler in \funcname{probably\_raise} with \funcname{RESTORE\_EXN\_HANDLER\_OCAML}
            \begin{itemize}
              \item Set \regname{RSP} to \localname{domain\_state->exn\_handler} value
              \item Restore \localname{domain\_state->exn\_handler} from trap
              \item Jump to the handler code with \funcname{ret}
            \end{itemize}
        \end{itemize}
      }
      \vfill
      \onslide*<1>{\mintocaml[firstline=9,lastline=13,highlightlines=12]{../src/backtrace/backtrace.ml}}
      \onslide*<2->{\mintocaml[firstline=15,lastline=21,highlightlines={19-21}]{../src/backtrace/backtrace.ml}}
      \endminipage
    \end{column}
    \begin{column}{0.5\textwidth}
      \centering
      \onslide*<1>{
        \mintc[firstline=190,lastline=192]{../ocaml/runtime/amd64.S}
        \mintc[firstline=234,lastline=237]{../ocaml/runtime/amd64.S}
      }
      \onslide*<2>{
        \mintc[firstline=190,lastline=192,highlightlines={191-192}]{../ocaml/runtime/amd64.S}
        \mintc[firstline=234,lastline=237,highlightlines={235-237}]{../ocaml/runtime/amd64.S}
      }
      \onslide*<1>{\listCamlRaiseExn[highlightlines=720]{}}
      \onslide*<2>{\listCamlRaiseExn[highlightlines=722]{}}
      \onslide*<3->{\providecommand\step{5}\input{diagrams/backtrace/backtrace.tex}}
    \end{column}
  \end{columns}
\end{frame}

\begin{frame}{Backtraces}{\funcname{caml\_probably\_raise}'s handler doesn't catch \typename{ExnC}}
  \begin{columns}[c]
    \begin{column}{0.45\textwidth}
      \minipage[c][0.75\textheight][s]{\columnwidth}
      \begin{itemize}
        \item<1-> \funcname{match \dots with}\footnotemark pattern matching can also match exceptions
        \item<1-> The CMM of \funcname{probably\_raise} shows that this \funcname{match \dots with} is implemented with a pattern matching and a \funcname{try \dots with}
        \item<1-> But this one only matches on \typename{ExnA} exception
        \item<2-> Since the pattern matching fails to catch the exception, \funcname{reraise} is called with our current \typename{ExnC} exception
        \item<3-> Disassembly shows that \funcname{reraise} is actually a call to \funcname{caml\_reraise\_exn}, which is also an assembly function provided by the runtime
      \end{itemize}
      \vfill
      \mintocaml[firstline=15,lastline=21,highlightlines={18-21}]{../src/backtrace/backtrace.ml}
      \endminipage
    \end{column}
    \begin{column}{0.55\textwidth}
      \centering
      \onslide*<1>{\mintcmm[highlightlines={10-18}]{../src/backtrace/camlBacktrace\_\_probably\_raise\_351.cmm}}
      \onslide*<2>{\listcmm[highlightlines=18]{../src/backtrace/camlBacktrace\_\_probably\_raise_351.cmm}{CMM of probably\_raise}}
      \onslide*<3>{\mintobjdump[firstline=5,lastline=35,highlightlines=31]{../src/backtrace/camlBacktrace\_\_probably\_raise_351.objdump}}
    \end{column}
  \end{columns}
  \only<1>{\footnotetext[1]{https://blog.janestreet.com/pattern-matching-and-exception-handling-unite/}}
\end{frame}

\newcommand\listCamlReraiseExn[1][]{
  \listasm[firstline=727,lastline=734,#1]{../ocaml/runtime/amd64.S}{runtime/amd64.S:727}}

\begin{frame}{Backtraces}{Introducing \funcname{caml\_reraise\_exn}}
  \begin{columns}[c]
    \begin{column}{0.5\textwidth}
      \minipage[c][0.66\textheight][s]{\columnwidth}
      \begin{itemize}
        \item<1-> The exception have to be reraised to the next handler
        \item<2-> First, checks if the backtrace should be collected with \funcarg{domain\_state->backtrace\_active}
        \only<3>{
        \begin{itemize}
            \item If not, behaves like a \funcname{raise\_notrace} with \funcname{RESTORE\_EXN\_HANDLER\_OCAML}
        \end{itemize}
        }
        \item<4-> Jumps into \funcname{caml\_raise\_exn}
        \item<5-> Calls \funcname{caml\_stash\_backtrace} again, to scan the next part of the stack: from the trap just pop-ed to the next trap
      \end{itemize}
      \vfill
      \mintocaml[firstline=15,lastline=21,highlightlines={18-21}]{../src/backtrace/backtrace.ml}
      \endminipage
    \end{column}
    \begin{column}{0.5\textwidth}
      \raggedleft
      \onslide*<1>{\listCamlReraiseExn{}}
      \onslide*<2>{\listCamlReraiseExn[highlightlines=729]{}}
      \onslide*<3>{
        \mintc[firstline=190,lastline=192,highlightlines={191-192}]{../ocaml/runtime/amd64.S}
        \mintc[firstline=234,lastline=237,highlightlines={235-237}]{../ocaml/runtime/amd64.S}
        \medskip
        \listCamlReraiseExn[highlightlines=731]{}
      }
      \onslide*<4-5>{
        % TODO refactor with \listCamlReraiseExn
        \mintasm[firstline=727,lastline=734,highlightlines=730]{../ocaml/runtime/amd64.S}
        \medskip
      }
      \onslide*<4>{\listCamlRaiseExn[highlightlines=711]{}}
      \onslide*<5>{\listCamlRaiseExn[highlightlines=720]{}}
    \end{column}
  \end{columns}
\end{frame}

\begin{frame}{Backtraces}{Backtrace collection continues}
  \begin{columns}[c]
    \begin{column}{0.55\textwidth}
      \minipage[c][0.75\textheight][s]{\columnwidth}
      \begin{itemize}
        \item<1-> \funcname{caml\_stash\_backtrace} scans the older part of the stack, up to the next trap
        \item<6-> Controls jumps to the nested exception handler in \funcname{maybe\_raise}, which matches only on \typename{ExnB}
        \item<7-> \funcname{caml\_reraise\_exn} is called again, backtrace collection resumes with \funcname{caml\_stash\_backtrace}
      \end{itemize}
      \medskip
      \begin{itemize}
        \item<2-> Constructed backtrace:
        \begin{itemize}
          \item <2-> \funcname{definitely\_raise}
          \item <2-> \funcname{probably\_raise}
          \item <3-> \funcname{probably\_raise} (re-raise)
          \item <4-> \funcname{maybe\_raise}
          \item <5-> \funcname{main}
          \item <8-> \funcname{main} (re-raise)
        \end{itemize}
      \end{itemize}
      \vfill
      \onslide*<1-5>{\mintocaml[firstline=15,lastline=21]{../src/backtrace/backtrace.ml}}
      \onslide*<6>{\mintocaml[firstline=28,lastline=34,highlightlines=31]{../src/backtrace/backtrace.ml}}
      \onslide*<7->{\mintocaml[firstline=28,lastline=34,highlightlines=33]{../src/backtrace/backtrace.ml}}
      \endminipage
    \end{column}
    \begin{column}{0.45\textwidth}
      \raggedleft
      \onslide*<1>{\providecommand\step{6}\input{diagrams/backtrace/backtrace.tex}}%
      \onslide*<2>{\providecommand\step{6}\input{diagrams/backtrace/backtrace.tex}}%
      \onslide*<3>{\providecommand\step{7}\input{diagrams/backtrace/backtrace.tex}}%
      \onslide*<4>{\providecommand\step{8}\input{diagrams/backtrace/backtrace.tex}}%
      \onslide*<5>{\providecommand\step{9}\input{diagrams/backtrace/backtrace.tex}}%
      \onslide*<6>{\providecommand\step{10}\input{diagrams/backtrace/backtrace.tex}}%
      \onslide*<7>{\providecommand\step{11}\input{diagrams/backtrace/backtrace.tex}}%
      \onslide*<8>{\providecommand\step{12}\input{diagrams/backtrace/backtrace.tex}}%
      \onslide*<9>{\providecommand\step{13}\input{diagrams/backtrace/backtrace.tex}}%
    \end{column}
  \end{columns}
\end{frame}

\begin{frame}{Backtraces}{Printing the backtrace}
  \begin{columns}[c]
    \begin{column}{0.45\textwidth}
      \minipage[c][0.66\textheight][s]{\columnwidth}
      \begin{itemize}
        \item<1-> The exception backtrace can now be printed to \funcarg{stdout} with \funcname{Printexc.print_backtrace}
        \item<2-> First the exception backtrace is retrieved into an OCaml array with \funcname{get_raw_backtrace }
        \item<3-> Which is implemented as the \funcname{caml_get_exception_raw_backtrace} C function in the runtime
        \item<4-> And finally printed with \funcname{print_exception_backtrace}
      \end{itemize}
      \vfill
      \mintocaml[firstline=28,lastline=34,highlightlines=33]{../src/backtrace/backtrace.ml}
      \endminipage
    \end{column}
    \begin{column}{0.55\textwidth}
      \onslide*<1,2>{
        \mintocaml[firstline=100,lastline=101]{../ocaml/stdlib/printexc.ml}
      }
      \only<1>{
        \listocaml[firstline=170,lastline=172]{../ocaml/stdlib/printexc.ml}{stdlib/printexc.ml:170}
      }
      \only<2>{
        \listocaml[firstline=170,lastline=172,highlightlines=172]{../ocaml/stdlib/printexc.ml}{stdlib/printexc.ml:170}
      }
      \only<3>{
        \mintc[firstline=154,lastline=156]{../ocaml/runtime/backtrace.c}
        \listc[firstline=171,lastline=192,highlightlines={184-187}]{../ocaml/runtime/backtrace.c}{runtime/backtrace.c:154}
      }
      \only<4>{
        \listocaml[firstline=155,lastline=165]{../ocaml/stdlib/printexc.ml}{stdlib/printexc.ml:55}
      }
    \end{column}
  \end{columns}
\end{frame}


\subsection{The multiple kinds of raise}
\frameSubsection{}{}

\begin{frame}{Backtraces}{When backtraces are not needed}
  \begin{columns}[c]
    \begin{column}{0.45\textwidth}
      \minipage[c][0.66\textheight][s]{\columnwidth}
      \begin{itemize}
        \item<1-> What if the backtrace for an exception is never needed?
        \item<2-> It's possible to raise the exception explicitly with \funcname{raise\_notrace} to make raising as cheap as possible
        \item<3-> An example of use in the \funcname{dynlink} library of the OCaml compiler
      \end{itemize}
      \vfill
      \onslide<3->{
      \listocaml[firstline=205,lastline=209,highlightlines=207]{../ocaml/otherlibs/dynlink/dynlink\_compilerlibs/misc.ml}{otherlibs/dynlink/dynlink\_compilerlibs/misc.ml:205}
      }
      \endminipage
    \end{column}
    \begin{column}{0.55\textwidth}
    \only<4>{
      \mintcmm[highlightlines=3]{../src/notrace/camlNotrace\_\_fun_347.cmm}
      \listcmm{../src/notrace/camlNotrace\_\_all\_somes\_327.cmm}{all\_somes}
    }
    \onslide*<5->{
      \listobjdump[highlightlines={6-9}]{../src/notrace/camlNotrace\_\_fun_347.objdump}{anonymous function from \funcname{all\_somes}}
    }
    \end{column}
  \end{columns}
\end{frame}

\begin{frame}{Backtraces}{Summary table of the multiple kinds of raise}
  \begin{itemize}
    \item When debugging information is enabled and if the backtrace of an exception isn't needed, it's possible to raise with \funcname{raise\_notrace} for performance
    \item When debugging information is \emph{not} enabled, the compiler optimises \funcname{raise} calls to inline assembly (same effect as using \funcname{raise\_notrace})
  \end{itemize}
  \bigskip
  \centering
  \begin{tabular}{| l | c | c | c | c |}
    \hline
    \parbox{10em}{Debug information \\ (\funcarg{-g} flag)}  & \multicolumn{2}{|c|}{Disabled}                                     & \multicolumn{2}{|c|}{Enabled} \\
    \hline
    Raise kind                                               & \funcname{raise} & \funcname{raise\_notrace}                       & \funcname{raise} & \funcname{raise\_notrace} \\
    \hline
    Compilation                                              & \parbox{8em}{inline assembly\\ (optimization)} & inline assembly   & \funcname{caml\_raise\_exn} & inline assembly \\
    \hline
  \end{tabular}
\end{frame}


%
% Takeaway #5
%
\subsection*{Takeaway \#5}
\frameSubsectionTakeaway{}
\againframe<5>{takeaway}

    \end{column}
  \end{columns}
\end{frame}

\begin{frame}{Backtraces}{But before that, we need more fields from \typename{caml\_domain\_state}}
  \begin{columns}
    \begin{column}{0.5\textwidth}
      \begin{itemize}
        \item Remember \typename{caml\_domain\_state} the per-domain runtime state?
        \item Some other members are of interest to us now\dots
        \item \localname{backtrace\_active} is used to know if the runtime should collect backtraces
        \item \localname{backtrace\_active} can be set by
          \begin{itemize}
            \item The \funcarg{b} flag in the \funcname{OCAMLRUNPARAM} environment variable
            \item \mintinline{ocaml}{Printexc.record_backtrace : bool -> unit} \footnotemark
          \end{itemize}
      \end{itemize}
    \end{column}
    \begin{column}{0.5\textwidth}
      \begin{listing}
        \mintc[highlightlines={9-11}]{src/ocaml/domain\_state\_backtrace.h}
        \caption{runtime/caml/domain\_state.h}
      \end{listing}
    \end{column}
  \end{columns}
  \footnotetext[1]{https://v2.ocaml.org/api/Printexc.html}
\end{frame}

\newcommand\listCamlRaiseExn[1][]{
  \listasm[firstline=702,lastline=725,#1]{../ocaml/runtime/amd64.S}{runtime/amd64.S:702}}

\begin{frame}{Backtraces}{Walk through \funcname{caml\_raise\_exn}}
  \begin{columns}[T]
    \begin{column}{0.5\textwidth}
      \begin{itemize}
        \item<1-> First checks whether exceptions backtrace should be recorded
        \item<1-> By checking the \funcarg{domain\_state->backtrace\_active} value
        \item<2-> If not, acts as \funcname{raise\_notrace} with \funcname{RESTORE\_EXN\_HANDLER\_OCAML}
          \begin{itemize}
            \item Set \regname{RSP} to \localname{domain\_state->exn\_handler} value
            \item Restore \localname{domain\_state->exn\_handler} from trap
            \item Jump with \funcname{ret} (same effect as using \funcname{pop} and \funcname{jmp})
          \end{itemize}
        \item<3-> Otherwise, switches to the C stack to be able to execute C code
        \item<4-> Calls \funcname{caml\_stash\_backtrace}, a C function provided by the runtime\ignorespaces
          \onslide<6->{, with
            \begin{itemize}
              \item \funcarg{exn}: exception value
              \item \funcarg{pc}: code address of the raising point
              \item \funcarg{sp}: stack address of the raising function
              \item \funcarg{trapsp}: stack address of the next handler
            \end{itemize}
          }
      \end{itemize}
      \bigskip
      \mintocaml[firstline=9,lastline=13,highlightlines=12]{../src/backtrace/backtrace.ml}
    \end{column}
    \begin{column}{0.5\textwidth}
      \raggedleft
      \onslide*<1,3-4>{
        \mintc[firstline=190,lastline=192]{../ocaml/runtime/amd64.S}
        \mintc[firstline=234,lastline=237]{../ocaml/runtime/amd64.S}
      }
      \onslide*<2>{
        \mintc[firstline=190,lastline=192,highlightlines={191-192}]{../ocaml/runtime/amd64.S}
        \mintc[firstline=234,lastline=237,highlightlines={235-237}]{../ocaml/runtime/amd64.S}
      }
      \onslide*<1>{\listCamlRaiseExn[highlightlines=705]{}}%
      \onslide*<2>{\listCamlRaiseExn[highlightlines={707-708}]{}}%
      \onslide*<3>{\listCamlRaiseExn[highlightlines={712-714}]{}}%
      \onslide*<4>{\listCamlRaiseExn[highlightlines={715-720}]{}}%
      \onslide*<5>{\providecommand\step{1}%
% Backtraces
%
\subsection{One advantage of raising with debugging information}
\frameSubsection{}{}

\begin{frame}{Backtraces}{Debugging information \& source code}
  \begin{columns}[c]
    \begin{column}{0.5\textwidth}
      \begin{itemize}
        \item Raising an exception is fun and games, but how do we know where the exception originated? $\rightarrow$ With the backtrace associated with the exception!
        \item In order to get a backtrace from the point where an exception was raised, it's necessary to compile with debugging information enabled (\funcarg{-g} flag for the compiler)
        \item Same example as in the `Nested handlers' section
      \end{itemize}
    \end{column}
    \begin{column}{0.5\textwidth}
      \listocaml[firstline=9,lastline=34]{../src/backtrace/backtrace.ml}{backtrace/backtrace.ml}
    \end{column}
  \end{columns}
\end{frame}

\begin{frame}{Backtraces}{CMM and disassembly of \funcname{definitely\_raise}}
  \begin{columns}[c]
    \begin{column}{0.5\textwidth}
      \minipage[c][0.66\textheight][s]{\columnwidth}
      \begin{itemize}
        \item<1-> An effect of compiling with debugging information (\funcarg{-g}): the CMM no longer uses \funcname{raise\_notrace} to raise the exception but \funcname{raise} instead
        \item<2-> Disassembly shows that CMM \funcname{raise} is actually a call to \funcname{caml\_raise\_exn}, a function in assembler provided by the runtime
      \end{itemize}
      \vfill
      \mintocaml[firstline=9,lastline=13,highlightlines=12]{../src/backtrace/backtrace.ml}
      \endminipage
    \end{column}
    \begin{column}{0.5\textwidth}
      \onslide*<1>{
        \mintcmm[highlightlines=5]{../src/backtrace/camlBacktrace\_\_definitely\_raise\_347.cmm}
      }
      \onslide*<2>{
        \mintobjdump[firstline=2,highlightlines=14]{../src/backtrace/camlBacktrace\_\_definitely\_raise\_347.objdump}
      }
    \end{column}
  \end{columns}
\end{frame}

\begin{frame}{Backtraces}{Introducing \funcname{caml\_raise\_exn}}
  \begin{columns}[c]
    \begin{column}{0.5\textwidth}
      \minipage[c][0.66\textheight][s]{\columnwidth}
      \onslide*<1>{
        \begin{itemize}
          \item Raising the exception is performed using \funcname{caml\_raise\_exn} which is
            \begin{itemize}
              \item An assembly function provided by the runtime
              \item Architecture specific
            \end{itemize}
          \item Let's see what it does differently from \funcname{raise\_notrace}\dots
          \item We are about to raise \typename{ExnC} with \funcname{caml\_raise\_exn} from \funcname{definitely\_raise}
        \end{itemize}
      }
      \onslide*<2>{
        \mintobjdump[firstline=2,highlightlines=14]{../src/backtrace/camlBacktrace\_\_definitely\_raise\_347.objdump}
      }
      \vfill
      \mintocaml[firstline=9,lastline=13,highlightlines=12]{../src/backtrace/backtrace.ml}
      \endminipage
    \end{column}
    \begin{column}{0.5\textwidth}
      \centering
      \providecommand\step{1}\input{diagrams/backtrace/backtrace.tex}
    \end{column}
  \end{columns}
\end{frame}

\begin{frame}{Backtraces}{But before that, we need more fields from \typename{caml\_domain\_state}}
  \begin{columns}
    \begin{column}{0.5\textwidth}
      \begin{itemize}
        \item Remember \typename{caml\_domain\_state} the per-domain runtime state?
        \item Some other members are of interest to us now\dots
        \item \localname{backtrace\_active} is used to know if the runtime should collect backtraces
        \item \localname{backtrace\_active} can be set by
          \begin{itemize}
            \item The \funcarg{b} flag in the \funcname{OCAMLRUNPARAM} environment variable
            \item \mintinline{ocaml}{Printexc.record_backtrace : bool -> unit} \footnotemark
          \end{itemize}
      \end{itemize}
    \end{column}
    \begin{column}{0.5\textwidth}
      \begin{listing}
        \mintc[highlightlines={9-11}]{src/ocaml/domain\_state\_backtrace.h}
        \caption{runtime/caml/domain\_state.h}
      \end{listing}
    \end{column}
  \end{columns}
  \footnotetext[1]{https://v2.ocaml.org/api/Printexc.html}
\end{frame}

\newcommand\listCamlRaiseExn[1][]{
  \listasm[firstline=702,lastline=725,#1]{../ocaml/runtime/amd64.S}{runtime/amd64.S:702}}

\begin{frame}{Backtraces}{Walk through \funcname{caml\_raise\_exn}}
  \begin{columns}[T]
    \begin{column}{0.5\textwidth}
      \begin{itemize}
        \item<1-> First checks whether exceptions backtrace should be recorded
        \item<1-> By checking the \funcarg{domain\_state->backtrace\_active} value
        \item<2-> If not, acts as \funcname{raise\_notrace} with \funcname{RESTORE\_EXN\_HANDLER\_OCAML}
          \begin{itemize}
            \item Set \regname{RSP} to \localname{domain\_state->exn\_handler} value
            \item Restore \localname{domain\_state->exn\_handler} from trap
            \item Jump with \funcname{ret} (same effect as using \funcname{pop} and \funcname{jmp})
          \end{itemize}
        \item<3-> Otherwise, switches to the C stack to be able to execute C code
        \item<4-> Calls \funcname{caml\_stash\_backtrace}, a C function provided by the runtime\ignorespaces
          \onslide<6->{, with
            \begin{itemize}
              \item \funcarg{exn}: exception value
              \item \funcarg{pc}: code address of the raising point
              \item \funcarg{sp}: stack address of the raising function
              \item \funcarg{trapsp}: stack address of the next handler
            \end{itemize}
          }
      \end{itemize}
      \bigskip
      \mintocaml[firstline=9,lastline=13,highlightlines=12]{../src/backtrace/backtrace.ml}
    \end{column}
    \begin{column}{0.5\textwidth}
      \raggedleft
      \onslide*<1,3-4>{
        \mintc[firstline=190,lastline=192]{../ocaml/runtime/amd64.S}
        \mintc[firstline=234,lastline=237]{../ocaml/runtime/amd64.S}
      }
      \onslide*<2>{
        \mintc[firstline=190,lastline=192,highlightlines={191-192}]{../ocaml/runtime/amd64.S}
        \mintc[firstline=234,lastline=237,highlightlines={235-237}]{../ocaml/runtime/amd64.S}
      }
      \onslide*<1>{\listCamlRaiseExn[highlightlines=705]{}}%
      \onslide*<2>{\listCamlRaiseExn[highlightlines={707-708}]{}}%
      \onslide*<3>{\listCamlRaiseExn[highlightlines={712-714}]{}}%
      \onslide*<4>{\listCamlRaiseExn[highlightlines={715-720}]{}}%
      \onslide*<5>{\providecommand\step{1}\input{diagrams/backtrace/backtrace.tex}}%
      \onslide*<6>{\providecommand\step{2}\input{diagrams/backtrace/backtrace.tex}}%
    \end{column}
  \end{columns}
\end{frame}

\newcommand\listStashBacktrace[1][]{
  \listc[firstline=91,lastline=120,#1]{../ocaml/runtime/backtrace\_nat.c}{runtime/backtrace\_nat.c:91}}

\begin{frame}{Backtraces}{Introducing \funcname{caml\_stash\_backtrace}}
  \begin{columns}[c]
    \begin{column}{0.5\textwidth}
      \minipage[c][0.66\textheight][s]{\columnwidth}
      \begin{itemize}
        \item<1-> A C function provided by the runtime (specific to native code)
        \item<2-> It scans the stack, between the stack pointer at the raising point up to the next trap, in order to collect the names of the detected function frames
          \onslide*<2>{
            \begin{itemize}
              \item And more information, like line number, column number...
            \end{itemize}
          }
        \item<3-> It uses the same technique and information as the Garbage Collector (GC) uses to scan the values live on the stack
          \onslide*<4>{
            \begin{itemize}
              \item \typename{frame\_descr} are at the core of stack unwinding
            \end{itemize}
          }
      \end{itemize}
      \vfill
      \mintocaml[firstline=9,lastline=13,highlightlines=12]{../src/backtrace/backtrace.ml}
      \endminipage
    \end{column}
    \begin{column}{0.5\textwidth}
      \onslide*<1>{\listStashBacktrace[highlightlines=91]{}}
      \onslide*<2>{\listStashBacktrace[highlightlines={91,117-118}]{}}
      \onslide*<3->{\listStashBacktrace[highlightlines={94,106,109-110}]{}}
    \end{column}
  \end{columns}
\end{frame}

\newcommand\listFrameDescr[1][]{
  \listocaml[firstline=111,lastline=124,#1]{../ocaml/asmcomp/emitaux.ml}{asmcomp/emitaux.ml:111}}
\newcommand\listDebugInfo[1][]{
  \listocaml[firstline=38,lastline=49,#1]{../ocaml/lambda/debuginfo.mli}{lambda/debuginfo.mli:38}}

\begin{frame}{Backtraces}{\typename{frame\_descr} and \typename{frame\_debuginfo} in a nutshell}
  \begin{columns}[c]
    \begin{column}{0.5\textwidth}
      \begin{itemize}
        \item<1-> For every return address in an OCaml program, there is a associated \typename{frame\_descr}
        \item<2-> A \typename{frame\_descr} specifies
        \begin{itemize}
          \item<2-> The size of the function stack frame
          \item<3-> Where live values are on the stack (so that the GC can mark them as live and prevent from being swept)
        \end{itemize}
        \item<4-> When compiled with debug information, \typename{frame\_descr} will be linked to a \typename{frame\_debuginfo}
        \begin{itemize}
          \item<5-> Contains information about the position in the source code (file name, line and column number)
        \end{itemize}
      \end{itemize}
    \end{column}
    \begin{column}{0.5\textwidth}
        \onslide*<1>{\listFrameDescr[highlightlines={118-122}]{}}
        \onslide*<2>{\listFrameDescr[highlightlines={120}]{}}
        \onslide*<3>{\listFrameDescr[highlightlines={121}]{}}
        \onslide*<4->{\listFrameDescr[highlightlines={122}]{}}
        \medskip
        \onslide*<1-3>{\listDebugInfo{}}
        \onslide*<4>{\listDebugInfo[highlightlines={38,49}]{}}
        \onslide*<5>{\listDebugInfo[highlightlines={39-46}]{}}
    \end{column}
  \end{columns}
\end{frame}

\begin{frame}{Backtraces}{Scanning the stack with \typename{frame\_descr}}
  \begin{columns}[c]
    \begin{column}{0.5\textwidth}
      \begin{itemize}
        \item Given the return address of the code that raises the exception by calling \funcname{caml\_raise\_exn} (\funcarg{pc} argument of \funcname{caml\_stash\_backtrace})
        \item And the position of the stack pointer (\funcarg{sp} argument of \funcname{caml\_stash\_backtrace})
        \item The frame size given by \typename{frame\_descr}.\funcarg{frame\_size}, allows to find the end of the previous frame
        \item And the return address of the caller of this function
        \bigskip
        \onslide<3->{
        \item Note that traps (here the reddish \funcname{ExnA}, \funcname{ExnB} and \funcname{ExnC} blocks) belong to the frame that installed them, and so the frame size changes when traps are removed
        }
      \end{itemize}
    \end{column}
    \begin{column}{0.5\textwidth}
      \raggedleft
      \onslide*<1>{\providecommand\step{2}\input{diagrams/backtrace/backtrace.tex}}%
      \onslide*<2>{\providecommand\step{1}\input{diagrams/backtrace/scanstack.tex}}%
      \onslide*<3>{\providecommand\step{2}\input{diagrams/backtrace/scanstack.tex}}%
      \onslide*<4>{\providecommand\step{3}\input{diagrams/backtrace/scanstack.tex}}%
      \onslide*<5>{\providecommand\step{4}\input{diagrams/backtrace/scanstack.tex}}%
      \onslide*<6>{\providecommand\step{5}\input{diagrams/backtrace/scanstack.tex}}%
    \end{column}
  \end{columns}
\end{frame}

% Duplicate of previous-previous slide
\begin{frame}{Backtraces}{Continuing on \funcname{caml\_stash\_backtrace}}
  \begin{columns}[c]
    \begin{column}{0.5\textwidth}
      \minipage[c][0.75\textheight][s]{\columnwidth}
      \begin{itemize}
          \color{gray}
        \item A C function provided by the runtime (specific to native code)
        \item It scans the stack, between the stack pointer at the raising point up to the next trap, in order to collect the names of the detected function frames
        \item It uses the same technique and information as the Garbage Collector (GC) uses to scan the values live on the stack
      \end{itemize}
      \begin{itemize}
        \item For each function frame detected, store a pointer to the \typename{frame\_descr} in the \localname{domain\_state->backtrace\_buffer} array, effectively constructing the backtrace
      \end{itemize}
      \onslide<2->{
        \begin{itemize}
            \smallskip
          \item Constructed backtrace:
            \funcname{definitely\_raise},
            \onslide<3->{\funcname{probably\_raise}}
        \end{itemize}
      }
      \vfill
      \mintocaml[firstline=9,lastline=13,highlightlines=12]{../src/backtrace/backtrace.ml}
      \endminipage
    \end{column}
    \begin{column}{0.5\textwidth}
      \raggedleft
      \onslide*<1>{\listStashBacktrace[highlightlines={114-115}]{}}
      \onslide*<2>{\providecommand\step{3}\input{diagrams/backtrace/backtrace.tex}}%
      \onslide*<3>{\providecommand\step{4}\input{diagrams/backtrace/backtrace.tex}}%
    \end{column}
  \end{columns}
\end{frame}

\begin{frame}{Backtraces}{Back to \funcname{caml\_raise\_exn}}
  \begin{columns}[c]
    \begin{column}{0.5\textwidth}
      \minipage[c][0.66\textheight][s]{\columnwidth}
      \begin{itemize}\color{gray}
        \item Checks wheteher exceptions backtrace should be recorded, by checking the \funcarg{domain\_state->backtrace\_active} value
        \item Switches to the C stack to be able to execute C code
        \item Calls \funcname{caml\_stash\_backtrace}, to collect the backtrace up to the next trap
      \end{itemize}
      \onslide*<2->{
        \begin{itemize}
          \item<2-> Executes the exception handler in \funcname{probably\_raise} with \funcname{RESTORE\_EXN\_HANDLER\_OCAML}
            \begin{itemize}
              \item Set \regname{RSP} to \localname{domain\_state->exn\_handler} value
              \item Restore \localname{domain\_state->exn\_handler} from trap
              \item Jump to the handler code with \funcname{ret}
            \end{itemize}
        \end{itemize}
      }
      \vfill
      \onslide*<1>{\mintocaml[firstline=9,lastline=13,highlightlines=12]{../src/backtrace/backtrace.ml}}
      \onslide*<2->{\mintocaml[firstline=15,lastline=21,highlightlines={19-21}]{../src/backtrace/backtrace.ml}}
      \endminipage
    \end{column}
    \begin{column}{0.5\textwidth}
      \centering
      \onslide*<1>{
        \mintc[firstline=190,lastline=192]{../ocaml/runtime/amd64.S}
        \mintc[firstline=234,lastline=237]{../ocaml/runtime/amd64.S}
      }
      \onslide*<2>{
        \mintc[firstline=190,lastline=192,highlightlines={191-192}]{../ocaml/runtime/amd64.S}
        \mintc[firstline=234,lastline=237,highlightlines={235-237}]{../ocaml/runtime/amd64.S}
      }
      \onslide*<1>{\listCamlRaiseExn[highlightlines=720]{}}
      \onslide*<2>{\listCamlRaiseExn[highlightlines=722]{}}
      \onslide*<3->{\providecommand\step{5}\input{diagrams/backtrace/backtrace.tex}}
    \end{column}
  \end{columns}
\end{frame}

\begin{frame}{Backtraces}{\funcname{caml\_probably\_raise}'s handler doesn't catch \typename{ExnC}}
  \begin{columns}[c]
    \begin{column}{0.45\textwidth}
      \minipage[c][0.75\textheight][s]{\columnwidth}
      \begin{itemize}
        \item<1-> \funcname{match \dots with}\footnotemark pattern matching can also match exceptions
        \item<1-> The CMM of \funcname{probably\_raise} shows that this \funcname{match \dots with} is implemented with a pattern matching and a \funcname{try \dots with}
        \item<1-> But this one only matches on \typename{ExnA} exception
        \item<2-> Since the pattern matching fails to catch the exception, \funcname{reraise} is called with our current \typename{ExnC} exception
        \item<3-> Disassembly shows that \funcname{reraise} is actually a call to \funcname{caml\_reraise\_exn}, which is also an assembly function provided by the runtime
      \end{itemize}
      \vfill
      \mintocaml[firstline=15,lastline=21,highlightlines={18-21}]{../src/backtrace/backtrace.ml}
      \endminipage
    \end{column}
    \begin{column}{0.55\textwidth}
      \centering
      \onslide*<1>{\mintcmm[highlightlines={10-18}]{../src/backtrace/camlBacktrace\_\_probably\_raise\_351.cmm}}
      \onslide*<2>{\listcmm[highlightlines=18]{../src/backtrace/camlBacktrace\_\_probably\_raise_351.cmm}{CMM of probably\_raise}}
      \onslide*<3>{\mintobjdump[firstline=5,lastline=35,highlightlines=31]{../src/backtrace/camlBacktrace\_\_probably\_raise_351.objdump}}
    \end{column}
  \end{columns}
  \only<1>{\footnotetext[1]{https://blog.janestreet.com/pattern-matching-and-exception-handling-unite/}}
\end{frame}

\newcommand\listCamlReraiseExn[1][]{
  \listasm[firstline=727,lastline=734,#1]{../ocaml/runtime/amd64.S}{runtime/amd64.S:727}}

\begin{frame}{Backtraces}{Introducing \funcname{caml\_reraise\_exn}}
  \begin{columns}[c]
    \begin{column}{0.5\textwidth}
      \minipage[c][0.66\textheight][s]{\columnwidth}
      \begin{itemize}
        \item<1-> The exception have to be reraised to the next handler
        \item<2-> First, checks if the backtrace should be collected with \funcarg{domain\_state->backtrace\_active}
        \only<3>{
        \begin{itemize}
            \item If not, behaves like a \funcname{raise\_notrace} with \funcname{RESTORE\_EXN\_HANDLER\_OCAML}
        \end{itemize}
        }
        \item<4-> Jumps into \funcname{caml\_raise\_exn}
        \item<5-> Calls \funcname{caml\_stash\_backtrace} again, to scan the next part of the stack: from the trap just pop-ed to the next trap
      \end{itemize}
      \vfill
      \mintocaml[firstline=15,lastline=21,highlightlines={18-21}]{../src/backtrace/backtrace.ml}
      \endminipage
    \end{column}
    \begin{column}{0.5\textwidth}
      \raggedleft
      \onslide*<1>{\listCamlReraiseExn{}}
      \onslide*<2>{\listCamlReraiseExn[highlightlines=729]{}}
      \onslide*<3>{
        \mintc[firstline=190,lastline=192,highlightlines={191-192}]{../ocaml/runtime/amd64.S}
        \mintc[firstline=234,lastline=237,highlightlines={235-237}]{../ocaml/runtime/amd64.S}
        \medskip
        \listCamlReraiseExn[highlightlines=731]{}
      }
      \onslide*<4-5>{
        % TODO refactor with \listCamlReraiseExn
        \mintasm[firstline=727,lastline=734,highlightlines=730]{../ocaml/runtime/amd64.S}
        \medskip
      }
      \onslide*<4>{\listCamlRaiseExn[highlightlines=711]{}}
      \onslide*<5>{\listCamlRaiseExn[highlightlines=720]{}}
    \end{column}
  \end{columns}
\end{frame}

\begin{frame}{Backtraces}{Backtrace collection continues}
  \begin{columns}[c]
    \begin{column}{0.55\textwidth}
      \minipage[c][0.75\textheight][s]{\columnwidth}
      \begin{itemize}
        \item<1-> \funcname{caml\_stash\_backtrace} scans the older part of the stack, up to the next trap
        \item<6-> Controls jumps to the nested exception handler in \funcname{maybe\_raise}, which matches only on \typename{ExnB}
        \item<7-> \funcname{caml\_reraise\_exn} is called again, backtrace collection resumes with \funcname{caml\_stash\_backtrace}
      \end{itemize}
      \medskip
      \begin{itemize}
        \item<2-> Constructed backtrace:
        \begin{itemize}
          \item <2-> \funcname{definitely\_raise}
          \item <2-> \funcname{probably\_raise}
          \item <3-> \funcname{probably\_raise} (re-raise)
          \item <4-> \funcname{maybe\_raise}
          \item <5-> \funcname{main}
          \item <8-> \funcname{main} (re-raise)
        \end{itemize}
      \end{itemize}
      \vfill
      \onslide*<1-5>{\mintocaml[firstline=15,lastline=21]{../src/backtrace/backtrace.ml}}
      \onslide*<6>{\mintocaml[firstline=28,lastline=34,highlightlines=31]{../src/backtrace/backtrace.ml}}
      \onslide*<7->{\mintocaml[firstline=28,lastline=34,highlightlines=33]{../src/backtrace/backtrace.ml}}
      \endminipage
    \end{column}
    \begin{column}{0.45\textwidth}
      \raggedleft
      \onslide*<1>{\providecommand\step{6}\input{diagrams/backtrace/backtrace.tex}}%
      \onslide*<2>{\providecommand\step{6}\input{diagrams/backtrace/backtrace.tex}}%
      \onslide*<3>{\providecommand\step{7}\input{diagrams/backtrace/backtrace.tex}}%
      \onslide*<4>{\providecommand\step{8}\input{diagrams/backtrace/backtrace.tex}}%
      \onslide*<5>{\providecommand\step{9}\input{diagrams/backtrace/backtrace.tex}}%
      \onslide*<6>{\providecommand\step{10}\input{diagrams/backtrace/backtrace.tex}}%
      \onslide*<7>{\providecommand\step{11}\input{diagrams/backtrace/backtrace.tex}}%
      \onslide*<8>{\providecommand\step{12}\input{diagrams/backtrace/backtrace.tex}}%
      \onslide*<9>{\providecommand\step{13}\input{diagrams/backtrace/backtrace.tex}}%
    \end{column}
  \end{columns}
\end{frame}

\begin{frame}{Backtraces}{Printing the backtrace}
  \begin{columns}[c]
    \begin{column}{0.45\textwidth}
      \minipage[c][0.66\textheight][s]{\columnwidth}
      \begin{itemize}
        \item<1-> The exception backtrace can now be printed to \funcarg{stdout} with \funcname{Printexc.print_backtrace}
        \item<2-> First the exception backtrace is retrieved into an OCaml array with \funcname{get_raw_backtrace }
        \item<3-> Which is implemented as the \funcname{caml_get_exception_raw_backtrace} C function in the runtime
        \item<4-> And finally printed with \funcname{print_exception_backtrace}
      \end{itemize}
      \vfill
      \mintocaml[firstline=28,lastline=34,highlightlines=33]{../src/backtrace/backtrace.ml}
      \endminipage
    \end{column}
    \begin{column}{0.55\textwidth}
      \onslide*<1,2>{
        \mintocaml[firstline=100,lastline=101]{../ocaml/stdlib/printexc.ml}
      }
      \only<1>{
        \listocaml[firstline=170,lastline=172]{../ocaml/stdlib/printexc.ml}{stdlib/printexc.ml:170}
      }
      \only<2>{
        \listocaml[firstline=170,lastline=172,highlightlines=172]{../ocaml/stdlib/printexc.ml}{stdlib/printexc.ml:170}
      }
      \only<3>{
        \mintc[firstline=154,lastline=156]{../ocaml/runtime/backtrace.c}
        \listc[firstline=171,lastline=192,highlightlines={184-187}]{../ocaml/runtime/backtrace.c}{runtime/backtrace.c:154}
      }
      \only<4>{
        \listocaml[firstline=155,lastline=165]{../ocaml/stdlib/printexc.ml}{stdlib/printexc.ml:55}
      }
    \end{column}
  \end{columns}
\end{frame}


\subsection{The multiple kinds of raise}
\frameSubsection{}{}

\begin{frame}{Backtraces}{When backtraces are not needed}
  \begin{columns}[c]
    \begin{column}{0.45\textwidth}
      \minipage[c][0.66\textheight][s]{\columnwidth}
      \begin{itemize}
        \item<1-> What if the backtrace for an exception is never needed?
        \item<2-> It's possible to raise the exception explicitly with \funcname{raise\_notrace} to make raising as cheap as possible
        \item<3-> An example of use in the \funcname{dynlink} library of the OCaml compiler
      \end{itemize}
      \vfill
      \onslide<3->{
      \listocaml[firstline=205,lastline=209,highlightlines=207]{../ocaml/otherlibs/dynlink/dynlink\_compilerlibs/misc.ml}{otherlibs/dynlink/dynlink\_compilerlibs/misc.ml:205}
      }
      \endminipage
    \end{column}
    \begin{column}{0.55\textwidth}
    \only<4>{
      \mintcmm[highlightlines=3]{../src/notrace/camlNotrace\_\_fun_347.cmm}
      \listcmm{../src/notrace/camlNotrace\_\_all\_somes\_327.cmm}{all\_somes}
    }
    \onslide*<5->{
      \listobjdump[highlightlines={6-9}]{../src/notrace/camlNotrace\_\_fun_347.objdump}{anonymous function from \funcname{all\_somes}}
    }
    \end{column}
  \end{columns}
\end{frame}

\begin{frame}{Backtraces}{Summary table of the multiple kinds of raise}
  \begin{itemize}
    \item When debugging information is enabled and if the backtrace of an exception isn't needed, it's possible to raise with \funcname{raise\_notrace} for performance
    \item When debugging information is \emph{not} enabled, the compiler optimises \funcname{raise} calls to inline assembly (same effect as using \funcname{raise\_notrace})
  \end{itemize}
  \bigskip
  \centering
  \begin{tabular}{| l | c | c | c | c |}
    \hline
    \parbox{10em}{Debug information \\ (\funcarg{-g} flag)}  & \multicolumn{2}{|c|}{Disabled}                                     & \multicolumn{2}{|c|}{Enabled} \\
    \hline
    Raise kind                                               & \funcname{raise} & \funcname{raise\_notrace}                       & \funcname{raise} & \funcname{raise\_notrace} \\
    \hline
    Compilation                                              & \parbox{8em}{inline assembly\\ (optimization)} & inline assembly   & \funcname{caml\_raise\_exn} & inline assembly \\
    \hline
  \end{tabular}
\end{frame}


%
% Takeaway #5
%
\subsection*{Takeaway \#5}
\frameSubsectionTakeaway{}
\againframe<5>{takeaway}
}%
      \onslide*<6>{\providecommand\step{2}%
% Backtraces
%
\subsection{One advantage of raising with debugging information}
\frameSubsection{}{}

\begin{frame}{Backtraces}{Debugging information \& source code}
  \begin{columns}[c]
    \begin{column}{0.5\textwidth}
      \begin{itemize}
        \item Raising an exception is fun and games, but how do we know where the exception originated? $\rightarrow$ With the backtrace associated with the exception!
        \item In order to get a backtrace from the point where an exception was raised, it's necessary to compile with debugging information enabled (\funcarg{-g} flag for the compiler)
        \item Same example as in the `Nested handlers' section
      \end{itemize}
    \end{column}
    \begin{column}{0.5\textwidth}
      \listocaml[firstline=9,lastline=34]{../src/backtrace/backtrace.ml}{backtrace/backtrace.ml}
    \end{column}
  \end{columns}
\end{frame}

\begin{frame}{Backtraces}{CMM and disassembly of \funcname{definitely\_raise}}
  \begin{columns}[c]
    \begin{column}{0.5\textwidth}
      \minipage[c][0.66\textheight][s]{\columnwidth}
      \begin{itemize}
        \item<1-> An effect of compiling with debugging information (\funcarg{-g}): the CMM no longer uses \funcname{raise\_notrace} to raise the exception but \funcname{raise} instead
        \item<2-> Disassembly shows that CMM \funcname{raise} is actually a call to \funcname{caml\_raise\_exn}, a function in assembler provided by the runtime
      \end{itemize}
      \vfill
      \mintocaml[firstline=9,lastline=13,highlightlines=12]{../src/backtrace/backtrace.ml}
      \endminipage
    \end{column}
    \begin{column}{0.5\textwidth}
      \onslide*<1>{
        \mintcmm[highlightlines=5]{../src/backtrace/camlBacktrace\_\_definitely\_raise\_347.cmm}
      }
      \onslide*<2>{
        \mintobjdump[firstline=2,highlightlines=14]{../src/backtrace/camlBacktrace\_\_definitely\_raise\_347.objdump}
      }
    \end{column}
  \end{columns}
\end{frame}

\begin{frame}{Backtraces}{Introducing \funcname{caml\_raise\_exn}}
  \begin{columns}[c]
    \begin{column}{0.5\textwidth}
      \minipage[c][0.66\textheight][s]{\columnwidth}
      \onslide*<1>{
        \begin{itemize}
          \item Raising the exception is performed using \funcname{caml\_raise\_exn} which is
            \begin{itemize}
              \item An assembly function provided by the runtime
              \item Architecture specific
            \end{itemize}
          \item Let's see what it does differently from \funcname{raise\_notrace}\dots
          \item We are about to raise \typename{ExnC} with \funcname{caml\_raise\_exn} from \funcname{definitely\_raise}
        \end{itemize}
      }
      \onslide*<2>{
        \mintobjdump[firstline=2,highlightlines=14]{../src/backtrace/camlBacktrace\_\_definitely\_raise\_347.objdump}
      }
      \vfill
      \mintocaml[firstline=9,lastline=13,highlightlines=12]{../src/backtrace/backtrace.ml}
      \endminipage
    \end{column}
    \begin{column}{0.5\textwidth}
      \centering
      \providecommand\step{1}\input{diagrams/backtrace/backtrace.tex}
    \end{column}
  \end{columns}
\end{frame}

\begin{frame}{Backtraces}{But before that, we need more fields from \typename{caml\_domain\_state}}
  \begin{columns}
    \begin{column}{0.5\textwidth}
      \begin{itemize}
        \item Remember \typename{caml\_domain\_state} the per-domain runtime state?
        \item Some other members are of interest to us now\dots
        \item \localname{backtrace\_active} is used to know if the runtime should collect backtraces
        \item \localname{backtrace\_active} can be set by
          \begin{itemize}
            \item The \funcarg{b} flag in the \funcname{OCAMLRUNPARAM} environment variable
            \item \mintinline{ocaml}{Printexc.record_backtrace : bool -> unit} \footnotemark
          \end{itemize}
      \end{itemize}
    \end{column}
    \begin{column}{0.5\textwidth}
      \begin{listing}
        \mintc[highlightlines={9-11}]{src/ocaml/domain\_state\_backtrace.h}
        \caption{runtime/caml/domain\_state.h}
      \end{listing}
    \end{column}
  \end{columns}
  \footnotetext[1]{https://v2.ocaml.org/api/Printexc.html}
\end{frame}

\newcommand\listCamlRaiseExn[1][]{
  \listasm[firstline=702,lastline=725,#1]{../ocaml/runtime/amd64.S}{runtime/amd64.S:702}}

\begin{frame}{Backtraces}{Walk through \funcname{caml\_raise\_exn}}
  \begin{columns}[T]
    \begin{column}{0.5\textwidth}
      \begin{itemize}
        \item<1-> First checks whether exceptions backtrace should be recorded
        \item<1-> By checking the \funcarg{domain\_state->backtrace\_active} value
        \item<2-> If not, acts as \funcname{raise\_notrace} with \funcname{RESTORE\_EXN\_HANDLER\_OCAML}
          \begin{itemize}
            \item Set \regname{RSP} to \localname{domain\_state->exn\_handler} value
            \item Restore \localname{domain\_state->exn\_handler} from trap
            \item Jump with \funcname{ret} (same effect as using \funcname{pop} and \funcname{jmp})
          \end{itemize}
        \item<3-> Otherwise, switches to the C stack to be able to execute C code
        \item<4-> Calls \funcname{caml\_stash\_backtrace}, a C function provided by the runtime\ignorespaces
          \onslide<6->{, with
            \begin{itemize}
              \item \funcarg{exn}: exception value
              \item \funcarg{pc}: code address of the raising point
              \item \funcarg{sp}: stack address of the raising function
              \item \funcarg{trapsp}: stack address of the next handler
            \end{itemize}
          }
      \end{itemize}
      \bigskip
      \mintocaml[firstline=9,lastline=13,highlightlines=12]{../src/backtrace/backtrace.ml}
    \end{column}
    \begin{column}{0.5\textwidth}
      \raggedleft
      \onslide*<1,3-4>{
        \mintc[firstline=190,lastline=192]{../ocaml/runtime/amd64.S}
        \mintc[firstline=234,lastline=237]{../ocaml/runtime/amd64.S}
      }
      \onslide*<2>{
        \mintc[firstline=190,lastline=192,highlightlines={191-192}]{../ocaml/runtime/amd64.S}
        \mintc[firstline=234,lastline=237,highlightlines={235-237}]{../ocaml/runtime/amd64.S}
      }
      \onslide*<1>{\listCamlRaiseExn[highlightlines=705]{}}%
      \onslide*<2>{\listCamlRaiseExn[highlightlines={707-708}]{}}%
      \onslide*<3>{\listCamlRaiseExn[highlightlines={712-714}]{}}%
      \onslide*<4>{\listCamlRaiseExn[highlightlines={715-720}]{}}%
      \onslide*<5>{\providecommand\step{1}\input{diagrams/backtrace/backtrace.tex}}%
      \onslide*<6>{\providecommand\step{2}\input{diagrams/backtrace/backtrace.tex}}%
    \end{column}
  \end{columns}
\end{frame}

\newcommand\listStashBacktrace[1][]{
  \listc[firstline=91,lastline=120,#1]{../ocaml/runtime/backtrace\_nat.c}{runtime/backtrace\_nat.c:91}}

\begin{frame}{Backtraces}{Introducing \funcname{caml\_stash\_backtrace}}
  \begin{columns}[c]
    \begin{column}{0.5\textwidth}
      \minipage[c][0.66\textheight][s]{\columnwidth}
      \begin{itemize}
        \item<1-> A C function provided by the runtime (specific to native code)
        \item<2-> It scans the stack, between the stack pointer at the raising point up to the next trap, in order to collect the names of the detected function frames
          \onslide*<2>{
            \begin{itemize}
              \item And more information, like line number, column number...
            \end{itemize}
          }
        \item<3-> It uses the same technique and information as the Garbage Collector (GC) uses to scan the values live on the stack
          \onslide*<4>{
            \begin{itemize}
              \item \typename{frame\_descr} are at the core of stack unwinding
            \end{itemize}
          }
      \end{itemize}
      \vfill
      \mintocaml[firstline=9,lastline=13,highlightlines=12]{../src/backtrace/backtrace.ml}
      \endminipage
    \end{column}
    \begin{column}{0.5\textwidth}
      \onslide*<1>{\listStashBacktrace[highlightlines=91]{}}
      \onslide*<2>{\listStashBacktrace[highlightlines={91,117-118}]{}}
      \onslide*<3->{\listStashBacktrace[highlightlines={94,106,109-110}]{}}
    \end{column}
  \end{columns}
\end{frame}

\newcommand\listFrameDescr[1][]{
  \listocaml[firstline=111,lastline=124,#1]{../ocaml/asmcomp/emitaux.ml}{asmcomp/emitaux.ml:111}}
\newcommand\listDebugInfo[1][]{
  \listocaml[firstline=38,lastline=49,#1]{../ocaml/lambda/debuginfo.mli}{lambda/debuginfo.mli:38}}

\begin{frame}{Backtraces}{\typename{frame\_descr} and \typename{frame\_debuginfo} in a nutshell}
  \begin{columns}[c]
    \begin{column}{0.5\textwidth}
      \begin{itemize}
        \item<1-> For every return address in an OCaml program, there is a associated \typename{frame\_descr}
        \item<2-> A \typename{frame\_descr} specifies
        \begin{itemize}
          \item<2-> The size of the function stack frame
          \item<3-> Where live values are on the stack (so that the GC can mark them as live and prevent from being swept)
        \end{itemize}
        \item<4-> When compiled with debug information, \typename{frame\_descr} will be linked to a \typename{frame\_debuginfo}
        \begin{itemize}
          \item<5-> Contains information about the position in the source code (file name, line and column number)
        \end{itemize}
      \end{itemize}
    \end{column}
    \begin{column}{0.5\textwidth}
        \onslide*<1>{\listFrameDescr[highlightlines={118-122}]{}}
        \onslide*<2>{\listFrameDescr[highlightlines={120}]{}}
        \onslide*<3>{\listFrameDescr[highlightlines={121}]{}}
        \onslide*<4->{\listFrameDescr[highlightlines={122}]{}}
        \medskip
        \onslide*<1-3>{\listDebugInfo{}}
        \onslide*<4>{\listDebugInfo[highlightlines={38,49}]{}}
        \onslide*<5>{\listDebugInfo[highlightlines={39-46}]{}}
    \end{column}
  \end{columns}
\end{frame}

\begin{frame}{Backtraces}{Scanning the stack with \typename{frame\_descr}}
  \begin{columns}[c]
    \begin{column}{0.5\textwidth}
      \begin{itemize}
        \item Given the return address of the code that raises the exception by calling \funcname{caml\_raise\_exn} (\funcarg{pc} argument of \funcname{caml\_stash\_backtrace})
        \item And the position of the stack pointer (\funcarg{sp} argument of \funcname{caml\_stash\_backtrace})
        \item The frame size given by \typename{frame\_descr}.\funcarg{frame\_size}, allows to find the end of the previous frame
        \item And the return address of the caller of this function
        \bigskip
        \onslide<3->{
        \item Note that traps (here the reddish \funcname{ExnA}, \funcname{ExnB} and \funcname{ExnC} blocks) belong to the frame that installed them, and so the frame size changes when traps are removed
        }
      \end{itemize}
    \end{column}
    \begin{column}{0.5\textwidth}
      \raggedleft
      \onslide*<1>{\providecommand\step{2}\input{diagrams/backtrace/backtrace.tex}}%
      \onslide*<2>{\providecommand\step{1}\input{diagrams/backtrace/scanstack.tex}}%
      \onslide*<3>{\providecommand\step{2}\input{diagrams/backtrace/scanstack.tex}}%
      \onslide*<4>{\providecommand\step{3}\input{diagrams/backtrace/scanstack.tex}}%
      \onslide*<5>{\providecommand\step{4}\input{diagrams/backtrace/scanstack.tex}}%
      \onslide*<6>{\providecommand\step{5}\input{diagrams/backtrace/scanstack.tex}}%
    \end{column}
  \end{columns}
\end{frame}

% Duplicate of previous-previous slide
\begin{frame}{Backtraces}{Continuing on \funcname{caml\_stash\_backtrace}}
  \begin{columns}[c]
    \begin{column}{0.5\textwidth}
      \minipage[c][0.75\textheight][s]{\columnwidth}
      \begin{itemize}
          \color{gray}
        \item A C function provided by the runtime (specific to native code)
        \item It scans the stack, between the stack pointer at the raising point up to the next trap, in order to collect the names of the detected function frames
        \item It uses the same technique and information as the Garbage Collector (GC) uses to scan the values live on the stack
      \end{itemize}
      \begin{itemize}
        \item For each function frame detected, store a pointer to the \typename{frame\_descr} in the \localname{domain\_state->backtrace\_buffer} array, effectively constructing the backtrace
      \end{itemize}
      \onslide<2->{
        \begin{itemize}
            \smallskip
          \item Constructed backtrace:
            \funcname{definitely\_raise},
            \onslide<3->{\funcname{probably\_raise}}
        \end{itemize}
      }
      \vfill
      \mintocaml[firstline=9,lastline=13,highlightlines=12]{../src/backtrace/backtrace.ml}
      \endminipage
    \end{column}
    \begin{column}{0.5\textwidth}
      \raggedleft
      \onslide*<1>{\listStashBacktrace[highlightlines={114-115}]{}}
      \onslide*<2>{\providecommand\step{3}\input{diagrams/backtrace/backtrace.tex}}%
      \onslide*<3>{\providecommand\step{4}\input{diagrams/backtrace/backtrace.tex}}%
    \end{column}
  \end{columns}
\end{frame}

\begin{frame}{Backtraces}{Back to \funcname{caml\_raise\_exn}}
  \begin{columns}[c]
    \begin{column}{0.5\textwidth}
      \minipage[c][0.66\textheight][s]{\columnwidth}
      \begin{itemize}\color{gray}
        \item Checks wheteher exceptions backtrace should be recorded, by checking the \funcarg{domain\_state->backtrace\_active} value
        \item Switches to the C stack to be able to execute C code
        \item Calls \funcname{caml\_stash\_backtrace}, to collect the backtrace up to the next trap
      \end{itemize}
      \onslide*<2->{
        \begin{itemize}
          \item<2-> Executes the exception handler in \funcname{probably\_raise} with \funcname{RESTORE\_EXN\_HANDLER\_OCAML}
            \begin{itemize}
              \item Set \regname{RSP} to \localname{domain\_state->exn\_handler} value
              \item Restore \localname{domain\_state->exn\_handler} from trap
              \item Jump to the handler code with \funcname{ret}
            \end{itemize}
        \end{itemize}
      }
      \vfill
      \onslide*<1>{\mintocaml[firstline=9,lastline=13,highlightlines=12]{../src/backtrace/backtrace.ml}}
      \onslide*<2->{\mintocaml[firstline=15,lastline=21,highlightlines={19-21}]{../src/backtrace/backtrace.ml}}
      \endminipage
    \end{column}
    \begin{column}{0.5\textwidth}
      \centering
      \onslide*<1>{
        \mintc[firstline=190,lastline=192]{../ocaml/runtime/amd64.S}
        \mintc[firstline=234,lastline=237]{../ocaml/runtime/amd64.S}
      }
      \onslide*<2>{
        \mintc[firstline=190,lastline=192,highlightlines={191-192}]{../ocaml/runtime/amd64.S}
        \mintc[firstline=234,lastline=237,highlightlines={235-237}]{../ocaml/runtime/amd64.S}
      }
      \onslide*<1>{\listCamlRaiseExn[highlightlines=720]{}}
      \onslide*<2>{\listCamlRaiseExn[highlightlines=722]{}}
      \onslide*<3->{\providecommand\step{5}\input{diagrams/backtrace/backtrace.tex}}
    \end{column}
  \end{columns}
\end{frame}

\begin{frame}{Backtraces}{\funcname{caml\_probably\_raise}'s handler doesn't catch \typename{ExnC}}
  \begin{columns}[c]
    \begin{column}{0.45\textwidth}
      \minipage[c][0.75\textheight][s]{\columnwidth}
      \begin{itemize}
        \item<1-> \funcname{match \dots with}\footnotemark pattern matching can also match exceptions
        \item<1-> The CMM of \funcname{probably\_raise} shows that this \funcname{match \dots with} is implemented with a pattern matching and a \funcname{try \dots with}
        \item<1-> But this one only matches on \typename{ExnA} exception
        \item<2-> Since the pattern matching fails to catch the exception, \funcname{reraise} is called with our current \typename{ExnC} exception
        \item<3-> Disassembly shows that \funcname{reraise} is actually a call to \funcname{caml\_reraise\_exn}, which is also an assembly function provided by the runtime
      \end{itemize}
      \vfill
      \mintocaml[firstline=15,lastline=21,highlightlines={18-21}]{../src/backtrace/backtrace.ml}
      \endminipage
    \end{column}
    \begin{column}{0.55\textwidth}
      \centering
      \onslide*<1>{\mintcmm[highlightlines={10-18}]{../src/backtrace/camlBacktrace\_\_probably\_raise\_351.cmm}}
      \onslide*<2>{\listcmm[highlightlines=18]{../src/backtrace/camlBacktrace\_\_probably\_raise_351.cmm}{CMM of probably\_raise}}
      \onslide*<3>{\mintobjdump[firstline=5,lastline=35,highlightlines=31]{../src/backtrace/camlBacktrace\_\_probably\_raise_351.objdump}}
    \end{column}
  \end{columns}
  \only<1>{\footnotetext[1]{https://blog.janestreet.com/pattern-matching-and-exception-handling-unite/}}
\end{frame}

\newcommand\listCamlReraiseExn[1][]{
  \listasm[firstline=727,lastline=734,#1]{../ocaml/runtime/amd64.S}{runtime/amd64.S:727}}

\begin{frame}{Backtraces}{Introducing \funcname{caml\_reraise\_exn}}
  \begin{columns}[c]
    \begin{column}{0.5\textwidth}
      \minipage[c][0.66\textheight][s]{\columnwidth}
      \begin{itemize}
        \item<1-> The exception have to be reraised to the next handler
        \item<2-> First, checks if the backtrace should be collected with \funcarg{domain\_state->backtrace\_active}
        \only<3>{
        \begin{itemize}
            \item If not, behaves like a \funcname{raise\_notrace} with \funcname{RESTORE\_EXN\_HANDLER\_OCAML}
        \end{itemize}
        }
        \item<4-> Jumps into \funcname{caml\_raise\_exn}
        \item<5-> Calls \funcname{caml\_stash\_backtrace} again, to scan the next part of the stack: from the trap just pop-ed to the next trap
      \end{itemize}
      \vfill
      \mintocaml[firstline=15,lastline=21,highlightlines={18-21}]{../src/backtrace/backtrace.ml}
      \endminipage
    \end{column}
    \begin{column}{0.5\textwidth}
      \raggedleft
      \onslide*<1>{\listCamlReraiseExn{}}
      \onslide*<2>{\listCamlReraiseExn[highlightlines=729]{}}
      \onslide*<3>{
        \mintc[firstline=190,lastline=192,highlightlines={191-192}]{../ocaml/runtime/amd64.S}
        \mintc[firstline=234,lastline=237,highlightlines={235-237}]{../ocaml/runtime/amd64.S}
        \medskip
        \listCamlReraiseExn[highlightlines=731]{}
      }
      \onslide*<4-5>{
        % TODO refactor with \listCamlReraiseExn
        \mintasm[firstline=727,lastline=734,highlightlines=730]{../ocaml/runtime/amd64.S}
        \medskip
      }
      \onslide*<4>{\listCamlRaiseExn[highlightlines=711]{}}
      \onslide*<5>{\listCamlRaiseExn[highlightlines=720]{}}
    \end{column}
  \end{columns}
\end{frame}

\begin{frame}{Backtraces}{Backtrace collection continues}
  \begin{columns}[c]
    \begin{column}{0.55\textwidth}
      \minipage[c][0.75\textheight][s]{\columnwidth}
      \begin{itemize}
        \item<1-> \funcname{caml\_stash\_backtrace} scans the older part of the stack, up to the next trap
        \item<6-> Controls jumps to the nested exception handler in \funcname{maybe\_raise}, which matches only on \typename{ExnB}
        \item<7-> \funcname{caml\_reraise\_exn} is called again, backtrace collection resumes with \funcname{caml\_stash\_backtrace}
      \end{itemize}
      \medskip
      \begin{itemize}
        \item<2-> Constructed backtrace:
        \begin{itemize}
          \item <2-> \funcname{definitely\_raise}
          \item <2-> \funcname{probably\_raise}
          \item <3-> \funcname{probably\_raise} (re-raise)
          \item <4-> \funcname{maybe\_raise}
          \item <5-> \funcname{main}
          \item <8-> \funcname{main} (re-raise)
        \end{itemize}
      \end{itemize}
      \vfill
      \onslide*<1-5>{\mintocaml[firstline=15,lastline=21]{../src/backtrace/backtrace.ml}}
      \onslide*<6>{\mintocaml[firstline=28,lastline=34,highlightlines=31]{../src/backtrace/backtrace.ml}}
      \onslide*<7->{\mintocaml[firstline=28,lastline=34,highlightlines=33]{../src/backtrace/backtrace.ml}}
      \endminipage
    \end{column}
    \begin{column}{0.45\textwidth}
      \raggedleft
      \onslide*<1>{\providecommand\step{6}\input{diagrams/backtrace/backtrace.tex}}%
      \onslide*<2>{\providecommand\step{6}\input{diagrams/backtrace/backtrace.tex}}%
      \onslide*<3>{\providecommand\step{7}\input{diagrams/backtrace/backtrace.tex}}%
      \onslide*<4>{\providecommand\step{8}\input{diagrams/backtrace/backtrace.tex}}%
      \onslide*<5>{\providecommand\step{9}\input{diagrams/backtrace/backtrace.tex}}%
      \onslide*<6>{\providecommand\step{10}\input{diagrams/backtrace/backtrace.tex}}%
      \onslide*<7>{\providecommand\step{11}\input{diagrams/backtrace/backtrace.tex}}%
      \onslide*<8>{\providecommand\step{12}\input{diagrams/backtrace/backtrace.tex}}%
      \onslide*<9>{\providecommand\step{13}\input{diagrams/backtrace/backtrace.tex}}%
    \end{column}
  \end{columns}
\end{frame}

\begin{frame}{Backtraces}{Printing the backtrace}
  \begin{columns}[c]
    \begin{column}{0.45\textwidth}
      \minipage[c][0.66\textheight][s]{\columnwidth}
      \begin{itemize}
        \item<1-> The exception backtrace can now be printed to \funcarg{stdout} with \funcname{Printexc.print_backtrace}
        \item<2-> First the exception backtrace is retrieved into an OCaml array with \funcname{get_raw_backtrace }
        \item<3-> Which is implemented as the \funcname{caml_get_exception_raw_backtrace} C function in the runtime
        \item<4-> And finally printed with \funcname{print_exception_backtrace}
      \end{itemize}
      \vfill
      \mintocaml[firstline=28,lastline=34,highlightlines=33]{../src/backtrace/backtrace.ml}
      \endminipage
    \end{column}
    \begin{column}{0.55\textwidth}
      \onslide*<1,2>{
        \mintocaml[firstline=100,lastline=101]{../ocaml/stdlib/printexc.ml}
      }
      \only<1>{
        \listocaml[firstline=170,lastline=172]{../ocaml/stdlib/printexc.ml}{stdlib/printexc.ml:170}
      }
      \only<2>{
        \listocaml[firstline=170,lastline=172,highlightlines=172]{../ocaml/stdlib/printexc.ml}{stdlib/printexc.ml:170}
      }
      \only<3>{
        \mintc[firstline=154,lastline=156]{../ocaml/runtime/backtrace.c}
        \listc[firstline=171,lastline=192,highlightlines={184-187}]{../ocaml/runtime/backtrace.c}{runtime/backtrace.c:154}
      }
      \only<4>{
        \listocaml[firstline=155,lastline=165]{../ocaml/stdlib/printexc.ml}{stdlib/printexc.ml:55}
      }
    \end{column}
  \end{columns}
\end{frame}


\subsection{The multiple kinds of raise}
\frameSubsection{}{}

\begin{frame}{Backtraces}{When backtraces are not needed}
  \begin{columns}[c]
    \begin{column}{0.45\textwidth}
      \minipage[c][0.66\textheight][s]{\columnwidth}
      \begin{itemize}
        \item<1-> What if the backtrace for an exception is never needed?
        \item<2-> It's possible to raise the exception explicitly with \funcname{raise\_notrace} to make raising as cheap as possible
        \item<3-> An example of use in the \funcname{dynlink} library of the OCaml compiler
      \end{itemize}
      \vfill
      \onslide<3->{
      \listocaml[firstline=205,lastline=209,highlightlines=207]{../ocaml/otherlibs/dynlink/dynlink\_compilerlibs/misc.ml}{otherlibs/dynlink/dynlink\_compilerlibs/misc.ml:205}
      }
      \endminipage
    \end{column}
    \begin{column}{0.55\textwidth}
    \only<4>{
      \mintcmm[highlightlines=3]{../src/notrace/camlNotrace\_\_fun_347.cmm}
      \listcmm{../src/notrace/camlNotrace\_\_all\_somes\_327.cmm}{all\_somes}
    }
    \onslide*<5->{
      \listobjdump[highlightlines={6-9}]{../src/notrace/camlNotrace\_\_fun_347.objdump}{anonymous function from \funcname{all\_somes}}
    }
    \end{column}
  \end{columns}
\end{frame}

\begin{frame}{Backtraces}{Summary table of the multiple kinds of raise}
  \begin{itemize}
    \item When debugging information is enabled and if the backtrace of an exception isn't needed, it's possible to raise with \funcname{raise\_notrace} for performance
    \item When debugging information is \emph{not} enabled, the compiler optimises \funcname{raise} calls to inline assembly (same effect as using \funcname{raise\_notrace})
  \end{itemize}
  \bigskip
  \centering
  \begin{tabular}{| l | c | c | c | c |}
    \hline
    \parbox{10em}{Debug information \\ (\funcarg{-g} flag)}  & \multicolumn{2}{|c|}{Disabled}                                     & \multicolumn{2}{|c|}{Enabled} \\
    \hline
    Raise kind                                               & \funcname{raise} & \funcname{raise\_notrace}                       & \funcname{raise} & \funcname{raise\_notrace} \\
    \hline
    Compilation                                              & \parbox{8em}{inline assembly\\ (optimization)} & inline assembly   & \funcname{caml\_raise\_exn} & inline assembly \\
    \hline
  \end{tabular}
\end{frame}


%
% Takeaway #5
%
\subsection*{Takeaway \#5}
\frameSubsectionTakeaway{}
\againframe<5>{takeaway}
}%
    \end{column}
  \end{columns}
\end{frame}

\newcommand\listStashBacktrace[1][]{
  \listc[firstline=91,lastline=120,#1]{../ocaml/runtime/backtrace\_nat.c}{runtime/backtrace\_nat.c:91}}

\begin{frame}{Backtraces}{Introducing \funcname{caml\_stash\_backtrace}}
  \begin{columns}[c]
    \begin{column}{0.5\textwidth}
      \minipage[c][0.66\textheight][s]{\columnwidth}
      \begin{itemize}
        \item<1-> A C function provided by the runtime (specific to native code)
        \item<2-> It scans the stack, between the stack pointer at the raising point up to the next trap, in order to collect the names of the detected function frames
          \onslide*<2>{
            \begin{itemize}
              \item And more information, like line number, column number...
            \end{itemize}
          }
        \item<3-> It uses the same technique and information as the Garbage Collector (GC) uses to scan the values live on the stack
          \onslide*<4>{
            \begin{itemize}
              \item \typename{frame\_descr} are at the core of stack unwinding
            \end{itemize}
          }
      \end{itemize}
      \vfill
      \mintocaml[firstline=9,lastline=13,highlightlines=12]{../src/backtrace/backtrace.ml}
      \endminipage
    \end{column}
    \begin{column}{0.5\textwidth}
      \onslide*<1>{\listStashBacktrace[highlightlines=91]{}}
      \onslide*<2>{\listStashBacktrace[highlightlines={91,117-118}]{}}
      \onslide*<3->{\listStashBacktrace[highlightlines={94,106,109-110}]{}}
    \end{column}
  \end{columns}
\end{frame}

\newcommand\listFrameDescr[1][]{
  \listocaml[firstline=111,lastline=124,#1]{../ocaml/asmcomp/emitaux.ml}{asmcomp/emitaux.ml:111}}
\newcommand\listDebugInfo[1][]{
  \listocaml[firstline=38,lastline=49,#1]{../ocaml/lambda/debuginfo.mli}{lambda/debuginfo.mli:38}}

\begin{frame}{Backtraces}{\typename{frame\_descr} and \typename{frame\_debuginfo} in a nutshell}
  \begin{columns}[c]
    \begin{column}{0.5\textwidth}
      \begin{itemize}
        \item<1-> For every return address in an OCaml program, there is a associated \typename{frame\_descr}
        \item<2-> A \typename{frame\_descr} specifies
        \begin{itemize}
          \item<2-> The size of the function stack frame
          \item<3-> Where live values are on the stack (so that the GC can mark them as live and prevent from being swept)
        \end{itemize}
        \item<4-> When compiled with debug information, \typename{frame\_descr} will be linked to a \typename{frame\_debuginfo}
        \begin{itemize}
          \item<5-> Contains information about the position in the source code (file name, line and column number)
        \end{itemize}
      \end{itemize}
    \end{column}
    \begin{column}{0.5\textwidth}
        \onslide*<1>{\listFrameDescr[highlightlines={118-122}]{}}
        \onslide*<2>{\listFrameDescr[highlightlines={120}]{}}
        \onslide*<3>{\listFrameDescr[highlightlines={121}]{}}
        \onslide*<4->{\listFrameDescr[highlightlines={122}]{}}
        \medskip
        \onslide*<1-3>{\listDebugInfo{}}
        \onslide*<4>{\listDebugInfo[highlightlines={38,49}]{}}
        \onslide*<5>{\listDebugInfo[highlightlines={39-46}]{}}
    \end{column}
  \end{columns}
\end{frame}

\begin{frame}{Backtraces}{Scanning the stack with \typename{frame\_descr}}
  \begin{columns}[c]
    \begin{column}{0.5\textwidth}
      \begin{itemize}
        \item Given the return address of the code that raises the exception by calling \funcname{caml\_raise\_exn} (\funcarg{pc} argument of \funcname{caml\_stash\_backtrace})
        \item And the position of the stack pointer (\funcarg{sp} argument of \funcname{caml\_stash\_backtrace})
        \item The frame size given by \typename{frame\_descr}.\funcarg{frame\_size}, allows to find the end of the previous frame
        \item And the return address of the caller of this function
        \bigskip
        \onslide<3->{
        \item Note that traps (here the reddish \funcname{ExnA}, \funcname{ExnB} and \funcname{ExnC} blocks) belong to the frame that installed them, and so the frame size changes when traps are removed
        }
      \end{itemize}
    \end{column}
    \begin{column}{0.5\textwidth}
      \raggedleft
      \onslide*<1>{\providecommand\step{2}%
% Backtraces
%
\subsection{One advantage of raising with debugging information}
\frameSubsection{}{}

\begin{frame}{Backtraces}{Debugging information \& source code}
  \begin{columns}[c]
    \begin{column}{0.5\textwidth}
      \begin{itemize}
        \item Raising an exception is fun and games, but how do we know where the exception originated? $\rightarrow$ With the backtrace associated with the exception!
        \item In order to get a backtrace from the point where an exception was raised, it's necessary to compile with debugging information enabled (\funcarg{-g} flag for the compiler)
        \item Same example as in the `Nested handlers' section
      \end{itemize}
    \end{column}
    \begin{column}{0.5\textwidth}
      \listocaml[firstline=9,lastline=34]{../src/backtrace/backtrace.ml}{backtrace/backtrace.ml}
    \end{column}
  \end{columns}
\end{frame}

\begin{frame}{Backtraces}{CMM and disassembly of \funcname{definitely\_raise}}
  \begin{columns}[c]
    \begin{column}{0.5\textwidth}
      \minipage[c][0.66\textheight][s]{\columnwidth}
      \begin{itemize}
        \item<1-> An effect of compiling with debugging information (\funcarg{-g}): the CMM no longer uses \funcname{raise\_notrace} to raise the exception but \funcname{raise} instead
        \item<2-> Disassembly shows that CMM \funcname{raise} is actually a call to \funcname{caml\_raise\_exn}, a function in assembler provided by the runtime
      \end{itemize}
      \vfill
      \mintocaml[firstline=9,lastline=13,highlightlines=12]{../src/backtrace/backtrace.ml}
      \endminipage
    \end{column}
    \begin{column}{0.5\textwidth}
      \onslide*<1>{
        \mintcmm[highlightlines=5]{../src/backtrace/camlBacktrace\_\_definitely\_raise\_347.cmm}
      }
      \onslide*<2>{
        \mintobjdump[firstline=2,highlightlines=14]{../src/backtrace/camlBacktrace\_\_definitely\_raise\_347.objdump}
      }
    \end{column}
  \end{columns}
\end{frame}

\begin{frame}{Backtraces}{Introducing \funcname{caml\_raise\_exn}}
  \begin{columns}[c]
    \begin{column}{0.5\textwidth}
      \minipage[c][0.66\textheight][s]{\columnwidth}
      \onslide*<1>{
        \begin{itemize}
          \item Raising the exception is performed using \funcname{caml\_raise\_exn} which is
            \begin{itemize}
              \item An assembly function provided by the runtime
              \item Architecture specific
            \end{itemize}
          \item Let's see what it does differently from \funcname{raise\_notrace}\dots
          \item We are about to raise \typename{ExnC} with \funcname{caml\_raise\_exn} from \funcname{definitely\_raise}
        \end{itemize}
      }
      \onslide*<2>{
        \mintobjdump[firstline=2,highlightlines=14]{../src/backtrace/camlBacktrace\_\_definitely\_raise\_347.objdump}
      }
      \vfill
      \mintocaml[firstline=9,lastline=13,highlightlines=12]{../src/backtrace/backtrace.ml}
      \endminipage
    \end{column}
    \begin{column}{0.5\textwidth}
      \centering
      \providecommand\step{1}\input{diagrams/backtrace/backtrace.tex}
    \end{column}
  \end{columns}
\end{frame}

\begin{frame}{Backtraces}{But before that, we need more fields from \typename{caml\_domain\_state}}
  \begin{columns}
    \begin{column}{0.5\textwidth}
      \begin{itemize}
        \item Remember \typename{caml\_domain\_state} the per-domain runtime state?
        \item Some other members are of interest to us now\dots
        \item \localname{backtrace\_active} is used to know if the runtime should collect backtraces
        \item \localname{backtrace\_active} can be set by
          \begin{itemize}
            \item The \funcarg{b} flag in the \funcname{OCAMLRUNPARAM} environment variable
            \item \mintinline{ocaml}{Printexc.record_backtrace : bool -> unit} \footnotemark
          \end{itemize}
      \end{itemize}
    \end{column}
    \begin{column}{0.5\textwidth}
      \begin{listing}
        \mintc[highlightlines={9-11}]{src/ocaml/domain\_state\_backtrace.h}
        \caption{runtime/caml/domain\_state.h}
      \end{listing}
    \end{column}
  \end{columns}
  \footnotetext[1]{https://v2.ocaml.org/api/Printexc.html}
\end{frame}

\newcommand\listCamlRaiseExn[1][]{
  \listasm[firstline=702,lastline=725,#1]{../ocaml/runtime/amd64.S}{runtime/amd64.S:702}}

\begin{frame}{Backtraces}{Walk through \funcname{caml\_raise\_exn}}
  \begin{columns}[T]
    \begin{column}{0.5\textwidth}
      \begin{itemize}
        \item<1-> First checks whether exceptions backtrace should be recorded
        \item<1-> By checking the \funcarg{domain\_state->backtrace\_active} value
        \item<2-> If not, acts as \funcname{raise\_notrace} with \funcname{RESTORE\_EXN\_HANDLER\_OCAML}
          \begin{itemize}
            \item Set \regname{RSP} to \localname{domain\_state->exn\_handler} value
            \item Restore \localname{domain\_state->exn\_handler} from trap
            \item Jump with \funcname{ret} (same effect as using \funcname{pop} and \funcname{jmp})
          \end{itemize}
        \item<3-> Otherwise, switches to the C stack to be able to execute C code
        \item<4-> Calls \funcname{caml\_stash\_backtrace}, a C function provided by the runtime\ignorespaces
          \onslide<6->{, with
            \begin{itemize}
              \item \funcarg{exn}: exception value
              \item \funcarg{pc}: code address of the raising point
              \item \funcarg{sp}: stack address of the raising function
              \item \funcarg{trapsp}: stack address of the next handler
            \end{itemize}
          }
      \end{itemize}
      \bigskip
      \mintocaml[firstline=9,lastline=13,highlightlines=12]{../src/backtrace/backtrace.ml}
    \end{column}
    \begin{column}{0.5\textwidth}
      \raggedleft
      \onslide*<1,3-4>{
        \mintc[firstline=190,lastline=192]{../ocaml/runtime/amd64.S}
        \mintc[firstline=234,lastline=237]{../ocaml/runtime/amd64.S}
      }
      \onslide*<2>{
        \mintc[firstline=190,lastline=192,highlightlines={191-192}]{../ocaml/runtime/amd64.S}
        \mintc[firstline=234,lastline=237,highlightlines={235-237}]{../ocaml/runtime/amd64.S}
      }
      \onslide*<1>{\listCamlRaiseExn[highlightlines=705]{}}%
      \onslide*<2>{\listCamlRaiseExn[highlightlines={707-708}]{}}%
      \onslide*<3>{\listCamlRaiseExn[highlightlines={712-714}]{}}%
      \onslide*<4>{\listCamlRaiseExn[highlightlines={715-720}]{}}%
      \onslide*<5>{\providecommand\step{1}\input{diagrams/backtrace/backtrace.tex}}%
      \onslide*<6>{\providecommand\step{2}\input{diagrams/backtrace/backtrace.tex}}%
    \end{column}
  \end{columns}
\end{frame}

\newcommand\listStashBacktrace[1][]{
  \listc[firstline=91,lastline=120,#1]{../ocaml/runtime/backtrace\_nat.c}{runtime/backtrace\_nat.c:91}}

\begin{frame}{Backtraces}{Introducing \funcname{caml\_stash\_backtrace}}
  \begin{columns}[c]
    \begin{column}{0.5\textwidth}
      \minipage[c][0.66\textheight][s]{\columnwidth}
      \begin{itemize}
        \item<1-> A C function provided by the runtime (specific to native code)
        \item<2-> It scans the stack, between the stack pointer at the raising point up to the next trap, in order to collect the names of the detected function frames
          \onslide*<2>{
            \begin{itemize}
              \item And more information, like line number, column number...
            \end{itemize}
          }
        \item<3-> It uses the same technique and information as the Garbage Collector (GC) uses to scan the values live on the stack
          \onslide*<4>{
            \begin{itemize}
              \item \typename{frame\_descr} are at the core of stack unwinding
            \end{itemize}
          }
      \end{itemize}
      \vfill
      \mintocaml[firstline=9,lastline=13,highlightlines=12]{../src/backtrace/backtrace.ml}
      \endminipage
    \end{column}
    \begin{column}{0.5\textwidth}
      \onslide*<1>{\listStashBacktrace[highlightlines=91]{}}
      \onslide*<2>{\listStashBacktrace[highlightlines={91,117-118}]{}}
      \onslide*<3->{\listStashBacktrace[highlightlines={94,106,109-110}]{}}
    \end{column}
  \end{columns}
\end{frame}

\newcommand\listFrameDescr[1][]{
  \listocaml[firstline=111,lastline=124,#1]{../ocaml/asmcomp/emitaux.ml}{asmcomp/emitaux.ml:111}}
\newcommand\listDebugInfo[1][]{
  \listocaml[firstline=38,lastline=49,#1]{../ocaml/lambda/debuginfo.mli}{lambda/debuginfo.mli:38}}

\begin{frame}{Backtraces}{\typename{frame\_descr} and \typename{frame\_debuginfo} in a nutshell}
  \begin{columns}[c]
    \begin{column}{0.5\textwidth}
      \begin{itemize}
        \item<1-> For every return address in an OCaml program, there is a associated \typename{frame\_descr}
        \item<2-> A \typename{frame\_descr} specifies
        \begin{itemize}
          \item<2-> The size of the function stack frame
          \item<3-> Where live values are on the stack (so that the GC can mark them as live and prevent from being swept)
        \end{itemize}
        \item<4-> When compiled with debug information, \typename{frame\_descr} will be linked to a \typename{frame\_debuginfo}
        \begin{itemize}
          \item<5-> Contains information about the position in the source code (file name, line and column number)
        \end{itemize}
      \end{itemize}
    \end{column}
    \begin{column}{0.5\textwidth}
        \onslide*<1>{\listFrameDescr[highlightlines={118-122}]{}}
        \onslide*<2>{\listFrameDescr[highlightlines={120}]{}}
        \onslide*<3>{\listFrameDescr[highlightlines={121}]{}}
        \onslide*<4->{\listFrameDescr[highlightlines={122}]{}}
        \medskip
        \onslide*<1-3>{\listDebugInfo{}}
        \onslide*<4>{\listDebugInfo[highlightlines={38,49}]{}}
        \onslide*<5>{\listDebugInfo[highlightlines={39-46}]{}}
    \end{column}
  \end{columns}
\end{frame}

\begin{frame}{Backtraces}{Scanning the stack with \typename{frame\_descr}}
  \begin{columns}[c]
    \begin{column}{0.5\textwidth}
      \begin{itemize}
        \item Given the return address of the code that raises the exception by calling \funcname{caml\_raise\_exn} (\funcarg{pc} argument of \funcname{caml\_stash\_backtrace})
        \item And the position of the stack pointer (\funcarg{sp} argument of \funcname{caml\_stash\_backtrace})
        \item The frame size given by \typename{frame\_descr}.\funcarg{frame\_size}, allows to find the end of the previous frame
        \item And the return address of the caller of this function
        \bigskip
        \onslide<3->{
        \item Note that traps (here the reddish \funcname{ExnA}, \funcname{ExnB} and \funcname{ExnC} blocks) belong to the frame that installed them, and so the frame size changes when traps are removed
        }
      \end{itemize}
    \end{column}
    \begin{column}{0.5\textwidth}
      \raggedleft
      \onslide*<1>{\providecommand\step{2}\input{diagrams/backtrace/backtrace.tex}}%
      \onslide*<2>{\providecommand\step{1}\input{diagrams/backtrace/scanstack.tex}}%
      \onslide*<3>{\providecommand\step{2}\input{diagrams/backtrace/scanstack.tex}}%
      \onslide*<4>{\providecommand\step{3}\input{diagrams/backtrace/scanstack.tex}}%
      \onslide*<5>{\providecommand\step{4}\input{diagrams/backtrace/scanstack.tex}}%
      \onslide*<6>{\providecommand\step{5}\input{diagrams/backtrace/scanstack.tex}}%
    \end{column}
  \end{columns}
\end{frame}

% Duplicate of previous-previous slide
\begin{frame}{Backtraces}{Continuing on \funcname{caml\_stash\_backtrace}}
  \begin{columns}[c]
    \begin{column}{0.5\textwidth}
      \minipage[c][0.75\textheight][s]{\columnwidth}
      \begin{itemize}
          \color{gray}
        \item A C function provided by the runtime (specific to native code)
        \item It scans the stack, between the stack pointer at the raising point up to the next trap, in order to collect the names of the detected function frames
        \item It uses the same technique and information as the Garbage Collector (GC) uses to scan the values live on the stack
      \end{itemize}
      \begin{itemize}
        \item For each function frame detected, store a pointer to the \typename{frame\_descr} in the \localname{domain\_state->backtrace\_buffer} array, effectively constructing the backtrace
      \end{itemize}
      \onslide<2->{
        \begin{itemize}
            \smallskip
          \item Constructed backtrace:
            \funcname{definitely\_raise},
            \onslide<3->{\funcname{probably\_raise}}
        \end{itemize}
      }
      \vfill
      \mintocaml[firstline=9,lastline=13,highlightlines=12]{../src/backtrace/backtrace.ml}
      \endminipage
    \end{column}
    \begin{column}{0.5\textwidth}
      \raggedleft
      \onslide*<1>{\listStashBacktrace[highlightlines={114-115}]{}}
      \onslide*<2>{\providecommand\step{3}\input{diagrams/backtrace/backtrace.tex}}%
      \onslide*<3>{\providecommand\step{4}\input{diagrams/backtrace/backtrace.tex}}%
    \end{column}
  \end{columns}
\end{frame}

\begin{frame}{Backtraces}{Back to \funcname{caml\_raise\_exn}}
  \begin{columns}[c]
    \begin{column}{0.5\textwidth}
      \minipage[c][0.66\textheight][s]{\columnwidth}
      \begin{itemize}\color{gray}
        \item Checks wheteher exceptions backtrace should be recorded, by checking the \funcarg{domain\_state->backtrace\_active} value
        \item Switches to the C stack to be able to execute C code
        \item Calls \funcname{caml\_stash\_backtrace}, to collect the backtrace up to the next trap
      \end{itemize}
      \onslide*<2->{
        \begin{itemize}
          \item<2-> Executes the exception handler in \funcname{probably\_raise} with \funcname{RESTORE\_EXN\_HANDLER\_OCAML}
            \begin{itemize}
              \item Set \regname{RSP} to \localname{domain\_state->exn\_handler} value
              \item Restore \localname{domain\_state->exn\_handler} from trap
              \item Jump to the handler code with \funcname{ret}
            \end{itemize}
        \end{itemize}
      }
      \vfill
      \onslide*<1>{\mintocaml[firstline=9,lastline=13,highlightlines=12]{../src/backtrace/backtrace.ml}}
      \onslide*<2->{\mintocaml[firstline=15,lastline=21,highlightlines={19-21}]{../src/backtrace/backtrace.ml}}
      \endminipage
    \end{column}
    \begin{column}{0.5\textwidth}
      \centering
      \onslide*<1>{
        \mintc[firstline=190,lastline=192]{../ocaml/runtime/amd64.S}
        \mintc[firstline=234,lastline=237]{../ocaml/runtime/amd64.S}
      }
      \onslide*<2>{
        \mintc[firstline=190,lastline=192,highlightlines={191-192}]{../ocaml/runtime/amd64.S}
        \mintc[firstline=234,lastline=237,highlightlines={235-237}]{../ocaml/runtime/amd64.S}
      }
      \onslide*<1>{\listCamlRaiseExn[highlightlines=720]{}}
      \onslide*<2>{\listCamlRaiseExn[highlightlines=722]{}}
      \onslide*<3->{\providecommand\step{5}\input{diagrams/backtrace/backtrace.tex}}
    \end{column}
  \end{columns}
\end{frame}

\begin{frame}{Backtraces}{\funcname{caml\_probably\_raise}'s handler doesn't catch \typename{ExnC}}
  \begin{columns}[c]
    \begin{column}{0.45\textwidth}
      \minipage[c][0.75\textheight][s]{\columnwidth}
      \begin{itemize}
        \item<1-> \funcname{match \dots with}\footnotemark pattern matching can also match exceptions
        \item<1-> The CMM of \funcname{probably\_raise} shows that this \funcname{match \dots with} is implemented with a pattern matching and a \funcname{try \dots with}
        \item<1-> But this one only matches on \typename{ExnA} exception
        \item<2-> Since the pattern matching fails to catch the exception, \funcname{reraise} is called with our current \typename{ExnC} exception
        \item<3-> Disassembly shows that \funcname{reraise} is actually a call to \funcname{caml\_reraise\_exn}, which is also an assembly function provided by the runtime
      \end{itemize}
      \vfill
      \mintocaml[firstline=15,lastline=21,highlightlines={18-21}]{../src/backtrace/backtrace.ml}
      \endminipage
    \end{column}
    \begin{column}{0.55\textwidth}
      \centering
      \onslide*<1>{\mintcmm[highlightlines={10-18}]{../src/backtrace/camlBacktrace\_\_probably\_raise\_351.cmm}}
      \onslide*<2>{\listcmm[highlightlines=18]{../src/backtrace/camlBacktrace\_\_probably\_raise_351.cmm}{CMM of probably\_raise}}
      \onslide*<3>{\mintobjdump[firstline=5,lastline=35,highlightlines=31]{../src/backtrace/camlBacktrace\_\_probably\_raise_351.objdump}}
    \end{column}
  \end{columns}
  \only<1>{\footnotetext[1]{https://blog.janestreet.com/pattern-matching-and-exception-handling-unite/}}
\end{frame}

\newcommand\listCamlReraiseExn[1][]{
  \listasm[firstline=727,lastline=734,#1]{../ocaml/runtime/amd64.S}{runtime/amd64.S:727}}

\begin{frame}{Backtraces}{Introducing \funcname{caml\_reraise\_exn}}
  \begin{columns}[c]
    \begin{column}{0.5\textwidth}
      \minipage[c][0.66\textheight][s]{\columnwidth}
      \begin{itemize}
        \item<1-> The exception have to be reraised to the next handler
        \item<2-> First, checks if the backtrace should be collected with \funcarg{domain\_state->backtrace\_active}
        \only<3>{
        \begin{itemize}
            \item If not, behaves like a \funcname{raise\_notrace} with \funcname{RESTORE\_EXN\_HANDLER\_OCAML}
        \end{itemize}
        }
        \item<4-> Jumps into \funcname{caml\_raise\_exn}
        \item<5-> Calls \funcname{caml\_stash\_backtrace} again, to scan the next part of the stack: from the trap just pop-ed to the next trap
      \end{itemize}
      \vfill
      \mintocaml[firstline=15,lastline=21,highlightlines={18-21}]{../src/backtrace/backtrace.ml}
      \endminipage
    \end{column}
    \begin{column}{0.5\textwidth}
      \raggedleft
      \onslide*<1>{\listCamlReraiseExn{}}
      \onslide*<2>{\listCamlReraiseExn[highlightlines=729]{}}
      \onslide*<3>{
        \mintc[firstline=190,lastline=192,highlightlines={191-192}]{../ocaml/runtime/amd64.S}
        \mintc[firstline=234,lastline=237,highlightlines={235-237}]{../ocaml/runtime/amd64.S}
        \medskip
        \listCamlReraiseExn[highlightlines=731]{}
      }
      \onslide*<4-5>{
        % TODO refactor with \listCamlReraiseExn
        \mintasm[firstline=727,lastline=734,highlightlines=730]{../ocaml/runtime/amd64.S}
        \medskip
      }
      \onslide*<4>{\listCamlRaiseExn[highlightlines=711]{}}
      \onslide*<5>{\listCamlRaiseExn[highlightlines=720]{}}
    \end{column}
  \end{columns}
\end{frame}

\begin{frame}{Backtraces}{Backtrace collection continues}
  \begin{columns}[c]
    \begin{column}{0.55\textwidth}
      \minipage[c][0.75\textheight][s]{\columnwidth}
      \begin{itemize}
        \item<1-> \funcname{caml\_stash\_backtrace} scans the older part of the stack, up to the next trap
        \item<6-> Controls jumps to the nested exception handler in \funcname{maybe\_raise}, which matches only on \typename{ExnB}
        \item<7-> \funcname{caml\_reraise\_exn} is called again, backtrace collection resumes with \funcname{caml\_stash\_backtrace}
      \end{itemize}
      \medskip
      \begin{itemize}
        \item<2-> Constructed backtrace:
        \begin{itemize}
          \item <2-> \funcname{definitely\_raise}
          \item <2-> \funcname{probably\_raise}
          \item <3-> \funcname{probably\_raise} (re-raise)
          \item <4-> \funcname{maybe\_raise}
          \item <5-> \funcname{main}
          \item <8-> \funcname{main} (re-raise)
        \end{itemize}
      \end{itemize}
      \vfill
      \onslide*<1-5>{\mintocaml[firstline=15,lastline=21]{../src/backtrace/backtrace.ml}}
      \onslide*<6>{\mintocaml[firstline=28,lastline=34,highlightlines=31]{../src/backtrace/backtrace.ml}}
      \onslide*<7->{\mintocaml[firstline=28,lastline=34,highlightlines=33]{../src/backtrace/backtrace.ml}}
      \endminipage
    \end{column}
    \begin{column}{0.45\textwidth}
      \raggedleft
      \onslide*<1>{\providecommand\step{6}\input{diagrams/backtrace/backtrace.tex}}%
      \onslide*<2>{\providecommand\step{6}\input{diagrams/backtrace/backtrace.tex}}%
      \onslide*<3>{\providecommand\step{7}\input{diagrams/backtrace/backtrace.tex}}%
      \onslide*<4>{\providecommand\step{8}\input{diagrams/backtrace/backtrace.tex}}%
      \onslide*<5>{\providecommand\step{9}\input{diagrams/backtrace/backtrace.tex}}%
      \onslide*<6>{\providecommand\step{10}\input{diagrams/backtrace/backtrace.tex}}%
      \onslide*<7>{\providecommand\step{11}\input{diagrams/backtrace/backtrace.tex}}%
      \onslide*<8>{\providecommand\step{12}\input{diagrams/backtrace/backtrace.tex}}%
      \onslide*<9>{\providecommand\step{13}\input{diagrams/backtrace/backtrace.tex}}%
    \end{column}
  \end{columns}
\end{frame}

\begin{frame}{Backtraces}{Printing the backtrace}
  \begin{columns}[c]
    \begin{column}{0.45\textwidth}
      \minipage[c][0.66\textheight][s]{\columnwidth}
      \begin{itemize}
        \item<1-> The exception backtrace can now be printed to \funcarg{stdout} with \funcname{Printexc.print_backtrace}
        \item<2-> First the exception backtrace is retrieved into an OCaml array with \funcname{get_raw_backtrace }
        \item<3-> Which is implemented as the \funcname{caml_get_exception_raw_backtrace} C function in the runtime
        \item<4-> And finally printed with \funcname{print_exception_backtrace}
      \end{itemize}
      \vfill
      \mintocaml[firstline=28,lastline=34,highlightlines=33]{../src/backtrace/backtrace.ml}
      \endminipage
    \end{column}
    \begin{column}{0.55\textwidth}
      \onslide*<1,2>{
        \mintocaml[firstline=100,lastline=101]{../ocaml/stdlib/printexc.ml}
      }
      \only<1>{
        \listocaml[firstline=170,lastline=172]{../ocaml/stdlib/printexc.ml}{stdlib/printexc.ml:170}
      }
      \only<2>{
        \listocaml[firstline=170,lastline=172,highlightlines=172]{../ocaml/stdlib/printexc.ml}{stdlib/printexc.ml:170}
      }
      \only<3>{
        \mintc[firstline=154,lastline=156]{../ocaml/runtime/backtrace.c}
        \listc[firstline=171,lastline=192,highlightlines={184-187}]{../ocaml/runtime/backtrace.c}{runtime/backtrace.c:154}
      }
      \only<4>{
        \listocaml[firstline=155,lastline=165]{../ocaml/stdlib/printexc.ml}{stdlib/printexc.ml:55}
      }
    \end{column}
  \end{columns}
\end{frame}


\subsection{The multiple kinds of raise}
\frameSubsection{}{}

\begin{frame}{Backtraces}{When backtraces are not needed}
  \begin{columns}[c]
    \begin{column}{0.45\textwidth}
      \minipage[c][0.66\textheight][s]{\columnwidth}
      \begin{itemize}
        \item<1-> What if the backtrace for an exception is never needed?
        \item<2-> It's possible to raise the exception explicitly with \funcname{raise\_notrace} to make raising as cheap as possible
        \item<3-> An example of use in the \funcname{dynlink} library of the OCaml compiler
      \end{itemize}
      \vfill
      \onslide<3->{
      \listocaml[firstline=205,lastline=209,highlightlines=207]{../ocaml/otherlibs/dynlink/dynlink\_compilerlibs/misc.ml}{otherlibs/dynlink/dynlink\_compilerlibs/misc.ml:205}
      }
      \endminipage
    \end{column}
    \begin{column}{0.55\textwidth}
    \only<4>{
      \mintcmm[highlightlines=3]{../src/notrace/camlNotrace\_\_fun_347.cmm}
      \listcmm{../src/notrace/camlNotrace\_\_all\_somes\_327.cmm}{all\_somes}
    }
    \onslide*<5->{
      \listobjdump[highlightlines={6-9}]{../src/notrace/camlNotrace\_\_fun_347.objdump}{anonymous function from \funcname{all\_somes}}
    }
    \end{column}
  \end{columns}
\end{frame}

\begin{frame}{Backtraces}{Summary table of the multiple kinds of raise}
  \begin{itemize}
    \item When debugging information is enabled and if the backtrace of an exception isn't needed, it's possible to raise with \funcname{raise\_notrace} for performance
    \item When debugging information is \emph{not} enabled, the compiler optimises \funcname{raise} calls to inline assembly (same effect as using \funcname{raise\_notrace})
  \end{itemize}
  \bigskip
  \centering
  \begin{tabular}{| l | c | c | c | c |}
    \hline
    \parbox{10em}{Debug information \\ (\funcarg{-g} flag)}  & \multicolumn{2}{|c|}{Disabled}                                     & \multicolumn{2}{|c|}{Enabled} \\
    \hline
    Raise kind                                               & \funcname{raise} & \funcname{raise\_notrace}                       & \funcname{raise} & \funcname{raise\_notrace} \\
    \hline
    Compilation                                              & \parbox{8em}{inline assembly\\ (optimization)} & inline assembly   & \funcname{caml\_raise\_exn} & inline assembly \\
    \hline
  \end{tabular}
\end{frame}


%
% Takeaway #5
%
\subsection*{Takeaway \#5}
\frameSubsectionTakeaway{}
\againframe<5>{takeaway}
}%
      \onslide*<2>{\providecommand\step{1}\input{diagrams/backtrace/scanstack.tex}}%
      \onslide*<3>{\providecommand\step{2}\input{diagrams/backtrace/scanstack.tex}}%
      \onslide*<4>{\providecommand\step{3}\input{diagrams/backtrace/scanstack.tex}}%
      \onslide*<5>{\providecommand\step{4}\input{diagrams/backtrace/scanstack.tex}}%
      \onslide*<6>{\providecommand\step{5}\input{diagrams/backtrace/scanstack.tex}}%
    \end{column}
  \end{columns}
\end{frame}

% Duplicate of previous-previous slide
\begin{frame}{Backtraces}{Continuing on \funcname{caml\_stash\_backtrace}}
  \begin{columns}[c]
    \begin{column}{0.5\textwidth}
      \minipage[c][0.75\textheight][s]{\columnwidth}
      \begin{itemize}
          \color{gray}
        \item A C function provided by the runtime (specific to native code)
        \item It scans the stack, between the stack pointer at the raising point up to the next trap, in order to collect the names of the detected function frames
        \item It uses the same technique and information as the Garbage Collector (GC) uses to scan the values live on the stack
      \end{itemize}
      \begin{itemize}
        \item For each function frame detected, store a pointer to the \typename{frame\_descr} in the \localname{domain\_state->backtrace\_buffer} array, effectively constructing the backtrace
      \end{itemize}
      \onslide<2->{
        \begin{itemize}
            \smallskip
          \item Constructed backtrace:
            \funcname{definitely\_raise},
            \onslide<3->{\funcname{probably\_raise}}
        \end{itemize}
      }
      \vfill
      \mintocaml[firstline=9,lastline=13,highlightlines=12]{../src/backtrace/backtrace.ml}
      \endminipage
    \end{column}
    \begin{column}{0.5\textwidth}
      \raggedleft
      \onslide*<1>{\listStashBacktrace[highlightlines={114-115}]{}}
      \onslide*<2>{\providecommand\step{3}%
% Backtraces
%
\subsection{One advantage of raising with debugging information}
\frameSubsection{}{}

\begin{frame}{Backtraces}{Debugging information \& source code}
  \begin{columns}[c]
    \begin{column}{0.5\textwidth}
      \begin{itemize}
        \item Raising an exception is fun and games, but how do we know where the exception originated? $\rightarrow$ With the backtrace associated with the exception!
        \item In order to get a backtrace from the point where an exception was raised, it's necessary to compile with debugging information enabled (\funcarg{-g} flag for the compiler)
        \item Same example as in the `Nested handlers' section
      \end{itemize}
    \end{column}
    \begin{column}{0.5\textwidth}
      \listocaml[firstline=9,lastline=34]{../src/backtrace/backtrace.ml}{backtrace/backtrace.ml}
    \end{column}
  \end{columns}
\end{frame}

\begin{frame}{Backtraces}{CMM and disassembly of \funcname{definitely\_raise}}
  \begin{columns}[c]
    \begin{column}{0.5\textwidth}
      \minipage[c][0.66\textheight][s]{\columnwidth}
      \begin{itemize}
        \item<1-> An effect of compiling with debugging information (\funcarg{-g}): the CMM no longer uses \funcname{raise\_notrace} to raise the exception but \funcname{raise} instead
        \item<2-> Disassembly shows that CMM \funcname{raise} is actually a call to \funcname{caml\_raise\_exn}, a function in assembler provided by the runtime
      \end{itemize}
      \vfill
      \mintocaml[firstline=9,lastline=13,highlightlines=12]{../src/backtrace/backtrace.ml}
      \endminipage
    \end{column}
    \begin{column}{0.5\textwidth}
      \onslide*<1>{
        \mintcmm[highlightlines=5]{../src/backtrace/camlBacktrace\_\_definitely\_raise\_347.cmm}
      }
      \onslide*<2>{
        \mintobjdump[firstline=2,highlightlines=14]{../src/backtrace/camlBacktrace\_\_definitely\_raise\_347.objdump}
      }
    \end{column}
  \end{columns}
\end{frame}

\begin{frame}{Backtraces}{Introducing \funcname{caml\_raise\_exn}}
  \begin{columns}[c]
    \begin{column}{0.5\textwidth}
      \minipage[c][0.66\textheight][s]{\columnwidth}
      \onslide*<1>{
        \begin{itemize}
          \item Raising the exception is performed using \funcname{caml\_raise\_exn} which is
            \begin{itemize}
              \item An assembly function provided by the runtime
              \item Architecture specific
            \end{itemize}
          \item Let's see what it does differently from \funcname{raise\_notrace}\dots
          \item We are about to raise \typename{ExnC} with \funcname{caml\_raise\_exn} from \funcname{definitely\_raise}
        \end{itemize}
      }
      \onslide*<2>{
        \mintobjdump[firstline=2,highlightlines=14]{../src/backtrace/camlBacktrace\_\_definitely\_raise\_347.objdump}
      }
      \vfill
      \mintocaml[firstline=9,lastline=13,highlightlines=12]{../src/backtrace/backtrace.ml}
      \endminipage
    \end{column}
    \begin{column}{0.5\textwidth}
      \centering
      \providecommand\step{1}\input{diagrams/backtrace/backtrace.tex}
    \end{column}
  \end{columns}
\end{frame}

\begin{frame}{Backtraces}{But before that, we need more fields from \typename{caml\_domain\_state}}
  \begin{columns}
    \begin{column}{0.5\textwidth}
      \begin{itemize}
        \item Remember \typename{caml\_domain\_state} the per-domain runtime state?
        \item Some other members are of interest to us now\dots
        \item \localname{backtrace\_active} is used to know if the runtime should collect backtraces
        \item \localname{backtrace\_active} can be set by
          \begin{itemize}
            \item The \funcarg{b} flag in the \funcname{OCAMLRUNPARAM} environment variable
            \item \mintinline{ocaml}{Printexc.record_backtrace : bool -> unit} \footnotemark
          \end{itemize}
      \end{itemize}
    \end{column}
    \begin{column}{0.5\textwidth}
      \begin{listing}
        \mintc[highlightlines={9-11}]{src/ocaml/domain\_state\_backtrace.h}
        \caption{runtime/caml/domain\_state.h}
      \end{listing}
    \end{column}
  \end{columns}
  \footnotetext[1]{https://v2.ocaml.org/api/Printexc.html}
\end{frame}

\newcommand\listCamlRaiseExn[1][]{
  \listasm[firstline=702,lastline=725,#1]{../ocaml/runtime/amd64.S}{runtime/amd64.S:702}}

\begin{frame}{Backtraces}{Walk through \funcname{caml\_raise\_exn}}
  \begin{columns}[T]
    \begin{column}{0.5\textwidth}
      \begin{itemize}
        \item<1-> First checks whether exceptions backtrace should be recorded
        \item<1-> By checking the \funcarg{domain\_state->backtrace\_active} value
        \item<2-> If not, acts as \funcname{raise\_notrace} with \funcname{RESTORE\_EXN\_HANDLER\_OCAML}
          \begin{itemize}
            \item Set \regname{RSP} to \localname{domain\_state->exn\_handler} value
            \item Restore \localname{domain\_state->exn\_handler} from trap
            \item Jump with \funcname{ret} (same effect as using \funcname{pop} and \funcname{jmp})
          \end{itemize}
        \item<3-> Otherwise, switches to the C stack to be able to execute C code
        \item<4-> Calls \funcname{caml\_stash\_backtrace}, a C function provided by the runtime\ignorespaces
          \onslide<6->{, with
            \begin{itemize}
              \item \funcarg{exn}: exception value
              \item \funcarg{pc}: code address of the raising point
              \item \funcarg{sp}: stack address of the raising function
              \item \funcarg{trapsp}: stack address of the next handler
            \end{itemize}
          }
      \end{itemize}
      \bigskip
      \mintocaml[firstline=9,lastline=13,highlightlines=12]{../src/backtrace/backtrace.ml}
    \end{column}
    \begin{column}{0.5\textwidth}
      \raggedleft
      \onslide*<1,3-4>{
        \mintc[firstline=190,lastline=192]{../ocaml/runtime/amd64.S}
        \mintc[firstline=234,lastline=237]{../ocaml/runtime/amd64.S}
      }
      \onslide*<2>{
        \mintc[firstline=190,lastline=192,highlightlines={191-192}]{../ocaml/runtime/amd64.S}
        \mintc[firstline=234,lastline=237,highlightlines={235-237}]{../ocaml/runtime/amd64.S}
      }
      \onslide*<1>{\listCamlRaiseExn[highlightlines=705]{}}%
      \onslide*<2>{\listCamlRaiseExn[highlightlines={707-708}]{}}%
      \onslide*<3>{\listCamlRaiseExn[highlightlines={712-714}]{}}%
      \onslide*<4>{\listCamlRaiseExn[highlightlines={715-720}]{}}%
      \onslide*<5>{\providecommand\step{1}\input{diagrams/backtrace/backtrace.tex}}%
      \onslide*<6>{\providecommand\step{2}\input{diagrams/backtrace/backtrace.tex}}%
    \end{column}
  \end{columns}
\end{frame}

\newcommand\listStashBacktrace[1][]{
  \listc[firstline=91,lastline=120,#1]{../ocaml/runtime/backtrace\_nat.c}{runtime/backtrace\_nat.c:91}}

\begin{frame}{Backtraces}{Introducing \funcname{caml\_stash\_backtrace}}
  \begin{columns}[c]
    \begin{column}{0.5\textwidth}
      \minipage[c][0.66\textheight][s]{\columnwidth}
      \begin{itemize}
        \item<1-> A C function provided by the runtime (specific to native code)
        \item<2-> It scans the stack, between the stack pointer at the raising point up to the next trap, in order to collect the names of the detected function frames
          \onslide*<2>{
            \begin{itemize}
              \item And more information, like line number, column number...
            \end{itemize}
          }
        \item<3-> It uses the same technique and information as the Garbage Collector (GC) uses to scan the values live on the stack
          \onslide*<4>{
            \begin{itemize}
              \item \typename{frame\_descr} are at the core of stack unwinding
            \end{itemize}
          }
      \end{itemize}
      \vfill
      \mintocaml[firstline=9,lastline=13,highlightlines=12]{../src/backtrace/backtrace.ml}
      \endminipage
    \end{column}
    \begin{column}{0.5\textwidth}
      \onslide*<1>{\listStashBacktrace[highlightlines=91]{}}
      \onslide*<2>{\listStashBacktrace[highlightlines={91,117-118}]{}}
      \onslide*<3->{\listStashBacktrace[highlightlines={94,106,109-110}]{}}
    \end{column}
  \end{columns}
\end{frame}

\newcommand\listFrameDescr[1][]{
  \listocaml[firstline=111,lastline=124,#1]{../ocaml/asmcomp/emitaux.ml}{asmcomp/emitaux.ml:111}}
\newcommand\listDebugInfo[1][]{
  \listocaml[firstline=38,lastline=49,#1]{../ocaml/lambda/debuginfo.mli}{lambda/debuginfo.mli:38}}

\begin{frame}{Backtraces}{\typename{frame\_descr} and \typename{frame\_debuginfo} in a nutshell}
  \begin{columns}[c]
    \begin{column}{0.5\textwidth}
      \begin{itemize}
        \item<1-> For every return address in an OCaml program, there is a associated \typename{frame\_descr}
        \item<2-> A \typename{frame\_descr} specifies
        \begin{itemize}
          \item<2-> The size of the function stack frame
          \item<3-> Where live values are on the stack (so that the GC can mark them as live and prevent from being swept)
        \end{itemize}
        \item<4-> When compiled with debug information, \typename{frame\_descr} will be linked to a \typename{frame\_debuginfo}
        \begin{itemize}
          \item<5-> Contains information about the position in the source code (file name, line and column number)
        \end{itemize}
      \end{itemize}
    \end{column}
    \begin{column}{0.5\textwidth}
        \onslide*<1>{\listFrameDescr[highlightlines={118-122}]{}}
        \onslide*<2>{\listFrameDescr[highlightlines={120}]{}}
        \onslide*<3>{\listFrameDescr[highlightlines={121}]{}}
        \onslide*<4->{\listFrameDescr[highlightlines={122}]{}}
        \medskip
        \onslide*<1-3>{\listDebugInfo{}}
        \onslide*<4>{\listDebugInfo[highlightlines={38,49}]{}}
        \onslide*<5>{\listDebugInfo[highlightlines={39-46}]{}}
    \end{column}
  \end{columns}
\end{frame}

\begin{frame}{Backtraces}{Scanning the stack with \typename{frame\_descr}}
  \begin{columns}[c]
    \begin{column}{0.5\textwidth}
      \begin{itemize}
        \item Given the return address of the code that raises the exception by calling \funcname{caml\_raise\_exn} (\funcarg{pc} argument of \funcname{caml\_stash\_backtrace})
        \item And the position of the stack pointer (\funcarg{sp} argument of \funcname{caml\_stash\_backtrace})
        \item The frame size given by \typename{frame\_descr}.\funcarg{frame\_size}, allows to find the end of the previous frame
        \item And the return address of the caller of this function
        \bigskip
        \onslide<3->{
        \item Note that traps (here the reddish \funcname{ExnA}, \funcname{ExnB} and \funcname{ExnC} blocks) belong to the frame that installed them, and so the frame size changes when traps are removed
        }
      \end{itemize}
    \end{column}
    \begin{column}{0.5\textwidth}
      \raggedleft
      \onslide*<1>{\providecommand\step{2}\input{diagrams/backtrace/backtrace.tex}}%
      \onslide*<2>{\providecommand\step{1}\input{diagrams/backtrace/scanstack.tex}}%
      \onslide*<3>{\providecommand\step{2}\input{diagrams/backtrace/scanstack.tex}}%
      \onslide*<4>{\providecommand\step{3}\input{diagrams/backtrace/scanstack.tex}}%
      \onslide*<5>{\providecommand\step{4}\input{diagrams/backtrace/scanstack.tex}}%
      \onslide*<6>{\providecommand\step{5}\input{diagrams/backtrace/scanstack.tex}}%
    \end{column}
  \end{columns}
\end{frame}

% Duplicate of previous-previous slide
\begin{frame}{Backtraces}{Continuing on \funcname{caml\_stash\_backtrace}}
  \begin{columns}[c]
    \begin{column}{0.5\textwidth}
      \minipage[c][0.75\textheight][s]{\columnwidth}
      \begin{itemize}
          \color{gray}
        \item A C function provided by the runtime (specific to native code)
        \item It scans the stack, between the stack pointer at the raising point up to the next trap, in order to collect the names of the detected function frames
        \item It uses the same technique and information as the Garbage Collector (GC) uses to scan the values live on the stack
      \end{itemize}
      \begin{itemize}
        \item For each function frame detected, store a pointer to the \typename{frame\_descr} in the \localname{domain\_state->backtrace\_buffer} array, effectively constructing the backtrace
      \end{itemize}
      \onslide<2->{
        \begin{itemize}
            \smallskip
          \item Constructed backtrace:
            \funcname{definitely\_raise},
            \onslide<3->{\funcname{probably\_raise}}
        \end{itemize}
      }
      \vfill
      \mintocaml[firstline=9,lastline=13,highlightlines=12]{../src/backtrace/backtrace.ml}
      \endminipage
    \end{column}
    \begin{column}{0.5\textwidth}
      \raggedleft
      \onslide*<1>{\listStashBacktrace[highlightlines={114-115}]{}}
      \onslide*<2>{\providecommand\step{3}\input{diagrams/backtrace/backtrace.tex}}%
      \onslide*<3>{\providecommand\step{4}\input{diagrams/backtrace/backtrace.tex}}%
    \end{column}
  \end{columns}
\end{frame}

\begin{frame}{Backtraces}{Back to \funcname{caml\_raise\_exn}}
  \begin{columns}[c]
    \begin{column}{0.5\textwidth}
      \minipage[c][0.66\textheight][s]{\columnwidth}
      \begin{itemize}\color{gray}
        \item Checks wheteher exceptions backtrace should be recorded, by checking the \funcarg{domain\_state->backtrace\_active} value
        \item Switches to the C stack to be able to execute C code
        \item Calls \funcname{caml\_stash\_backtrace}, to collect the backtrace up to the next trap
      \end{itemize}
      \onslide*<2->{
        \begin{itemize}
          \item<2-> Executes the exception handler in \funcname{probably\_raise} with \funcname{RESTORE\_EXN\_HANDLER\_OCAML}
            \begin{itemize}
              \item Set \regname{RSP} to \localname{domain\_state->exn\_handler} value
              \item Restore \localname{domain\_state->exn\_handler} from trap
              \item Jump to the handler code with \funcname{ret}
            \end{itemize}
        \end{itemize}
      }
      \vfill
      \onslide*<1>{\mintocaml[firstline=9,lastline=13,highlightlines=12]{../src/backtrace/backtrace.ml}}
      \onslide*<2->{\mintocaml[firstline=15,lastline=21,highlightlines={19-21}]{../src/backtrace/backtrace.ml}}
      \endminipage
    \end{column}
    \begin{column}{0.5\textwidth}
      \centering
      \onslide*<1>{
        \mintc[firstline=190,lastline=192]{../ocaml/runtime/amd64.S}
        \mintc[firstline=234,lastline=237]{../ocaml/runtime/amd64.S}
      }
      \onslide*<2>{
        \mintc[firstline=190,lastline=192,highlightlines={191-192}]{../ocaml/runtime/amd64.S}
        \mintc[firstline=234,lastline=237,highlightlines={235-237}]{../ocaml/runtime/amd64.S}
      }
      \onslide*<1>{\listCamlRaiseExn[highlightlines=720]{}}
      \onslide*<2>{\listCamlRaiseExn[highlightlines=722]{}}
      \onslide*<3->{\providecommand\step{5}\input{diagrams/backtrace/backtrace.tex}}
    \end{column}
  \end{columns}
\end{frame}

\begin{frame}{Backtraces}{\funcname{caml\_probably\_raise}'s handler doesn't catch \typename{ExnC}}
  \begin{columns}[c]
    \begin{column}{0.45\textwidth}
      \minipage[c][0.75\textheight][s]{\columnwidth}
      \begin{itemize}
        \item<1-> \funcname{match \dots with}\footnotemark pattern matching can also match exceptions
        \item<1-> The CMM of \funcname{probably\_raise} shows that this \funcname{match \dots with} is implemented with a pattern matching and a \funcname{try \dots with}
        \item<1-> But this one only matches on \typename{ExnA} exception
        \item<2-> Since the pattern matching fails to catch the exception, \funcname{reraise} is called with our current \typename{ExnC} exception
        \item<3-> Disassembly shows that \funcname{reraise} is actually a call to \funcname{caml\_reraise\_exn}, which is also an assembly function provided by the runtime
      \end{itemize}
      \vfill
      \mintocaml[firstline=15,lastline=21,highlightlines={18-21}]{../src/backtrace/backtrace.ml}
      \endminipage
    \end{column}
    \begin{column}{0.55\textwidth}
      \centering
      \onslide*<1>{\mintcmm[highlightlines={10-18}]{../src/backtrace/camlBacktrace\_\_probably\_raise\_351.cmm}}
      \onslide*<2>{\listcmm[highlightlines=18]{../src/backtrace/camlBacktrace\_\_probably\_raise_351.cmm}{CMM of probably\_raise}}
      \onslide*<3>{\mintobjdump[firstline=5,lastline=35,highlightlines=31]{../src/backtrace/camlBacktrace\_\_probably\_raise_351.objdump}}
    \end{column}
  \end{columns}
  \only<1>{\footnotetext[1]{https://blog.janestreet.com/pattern-matching-and-exception-handling-unite/}}
\end{frame}

\newcommand\listCamlReraiseExn[1][]{
  \listasm[firstline=727,lastline=734,#1]{../ocaml/runtime/amd64.S}{runtime/amd64.S:727}}

\begin{frame}{Backtraces}{Introducing \funcname{caml\_reraise\_exn}}
  \begin{columns}[c]
    \begin{column}{0.5\textwidth}
      \minipage[c][0.66\textheight][s]{\columnwidth}
      \begin{itemize}
        \item<1-> The exception have to be reraised to the next handler
        \item<2-> First, checks if the backtrace should be collected with \funcarg{domain\_state->backtrace\_active}
        \only<3>{
        \begin{itemize}
            \item If not, behaves like a \funcname{raise\_notrace} with \funcname{RESTORE\_EXN\_HANDLER\_OCAML}
        \end{itemize}
        }
        \item<4-> Jumps into \funcname{caml\_raise\_exn}
        \item<5-> Calls \funcname{caml\_stash\_backtrace} again, to scan the next part of the stack: from the trap just pop-ed to the next trap
      \end{itemize}
      \vfill
      \mintocaml[firstline=15,lastline=21,highlightlines={18-21}]{../src/backtrace/backtrace.ml}
      \endminipage
    \end{column}
    \begin{column}{0.5\textwidth}
      \raggedleft
      \onslide*<1>{\listCamlReraiseExn{}}
      \onslide*<2>{\listCamlReraiseExn[highlightlines=729]{}}
      \onslide*<3>{
        \mintc[firstline=190,lastline=192,highlightlines={191-192}]{../ocaml/runtime/amd64.S}
        \mintc[firstline=234,lastline=237,highlightlines={235-237}]{../ocaml/runtime/amd64.S}
        \medskip
        \listCamlReraiseExn[highlightlines=731]{}
      }
      \onslide*<4-5>{
        % TODO refactor with \listCamlReraiseExn
        \mintasm[firstline=727,lastline=734,highlightlines=730]{../ocaml/runtime/amd64.S}
        \medskip
      }
      \onslide*<4>{\listCamlRaiseExn[highlightlines=711]{}}
      \onslide*<5>{\listCamlRaiseExn[highlightlines=720]{}}
    \end{column}
  \end{columns}
\end{frame}

\begin{frame}{Backtraces}{Backtrace collection continues}
  \begin{columns}[c]
    \begin{column}{0.55\textwidth}
      \minipage[c][0.75\textheight][s]{\columnwidth}
      \begin{itemize}
        \item<1-> \funcname{caml\_stash\_backtrace} scans the older part of the stack, up to the next trap
        \item<6-> Controls jumps to the nested exception handler in \funcname{maybe\_raise}, which matches only on \typename{ExnB}
        \item<7-> \funcname{caml\_reraise\_exn} is called again, backtrace collection resumes with \funcname{caml\_stash\_backtrace}
      \end{itemize}
      \medskip
      \begin{itemize}
        \item<2-> Constructed backtrace:
        \begin{itemize}
          \item <2-> \funcname{definitely\_raise}
          \item <2-> \funcname{probably\_raise}
          \item <3-> \funcname{probably\_raise} (re-raise)
          \item <4-> \funcname{maybe\_raise}
          \item <5-> \funcname{main}
          \item <8-> \funcname{main} (re-raise)
        \end{itemize}
      \end{itemize}
      \vfill
      \onslide*<1-5>{\mintocaml[firstline=15,lastline=21]{../src/backtrace/backtrace.ml}}
      \onslide*<6>{\mintocaml[firstline=28,lastline=34,highlightlines=31]{../src/backtrace/backtrace.ml}}
      \onslide*<7->{\mintocaml[firstline=28,lastline=34,highlightlines=33]{../src/backtrace/backtrace.ml}}
      \endminipage
    \end{column}
    \begin{column}{0.45\textwidth}
      \raggedleft
      \onslide*<1>{\providecommand\step{6}\input{diagrams/backtrace/backtrace.tex}}%
      \onslide*<2>{\providecommand\step{6}\input{diagrams/backtrace/backtrace.tex}}%
      \onslide*<3>{\providecommand\step{7}\input{diagrams/backtrace/backtrace.tex}}%
      \onslide*<4>{\providecommand\step{8}\input{diagrams/backtrace/backtrace.tex}}%
      \onslide*<5>{\providecommand\step{9}\input{diagrams/backtrace/backtrace.tex}}%
      \onslide*<6>{\providecommand\step{10}\input{diagrams/backtrace/backtrace.tex}}%
      \onslide*<7>{\providecommand\step{11}\input{diagrams/backtrace/backtrace.tex}}%
      \onslide*<8>{\providecommand\step{12}\input{diagrams/backtrace/backtrace.tex}}%
      \onslide*<9>{\providecommand\step{13}\input{diagrams/backtrace/backtrace.tex}}%
    \end{column}
  \end{columns}
\end{frame}

\begin{frame}{Backtraces}{Printing the backtrace}
  \begin{columns}[c]
    \begin{column}{0.45\textwidth}
      \minipage[c][0.66\textheight][s]{\columnwidth}
      \begin{itemize}
        \item<1-> The exception backtrace can now be printed to \funcarg{stdout} with \funcname{Printexc.print_backtrace}
        \item<2-> First the exception backtrace is retrieved into an OCaml array with \funcname{get_raw_backtrace }
        \item<3-> Which is implemented as the \funcname{caml_get_exception_raw_backtrace} C function in the runtime
        \item<4-> And finally printed with \funcname{print_exception_backtrace}
      \end{itemize}
      \vfill
      \mintocaml[firstline=28,lastline=34,highlightlines=33]{../src/backtrace/backtrace.ml}
      \endminipage
    \end{column}
    \begin{column}{0.55\textwidth}
      \onslide*<1,2>{
        \mintocaml[firstline=100,lastline=101]{../ocaml/stdlib/printexc.ml}
      }
      \only<1>{
        \listocaml[firstline=170,lastline=172]{../ocaml/stdlib/printexc.ml}{stdlib/printexc.ml:170}
      }
      \only<2>{
        \listocaml[firstline=170,lastline=172,highlightlines=172]{../ocaml/stdlib/printexc.ml}{stdlib/printexc.ml:170}
      }
      \only<3>{
        \mintc[firstline=154,lastline=156]{../ocaml/runtime/backtrace.c}
        \listc[firstline=171,lastline=192,highlightlines={184-187}]{../ocaml/runtime/backtrace.c}{runtime/backtrace.c:154}
      }
      \only<4>{
        \listocaml[firstline=155,lastline=165]{../ocaml/stdlib/printexc.ml}{stdlib/printexc.ml:55}
      }
    \end{column}
  \end{columns}
\end{frame}


\subsection{The multiple kinds of raise}
\frameSubsection{}{}

\begin{frame}{Backtraces}{When backtraces are not needed}
  \begin{columns}[c]
    \begin{column}{0.45\textwidth}
      \minipage[c][0.66\textheight][s]{\columnwidth}
      \begin{itemize}
        \item<1-> What if the backtrace for an exception is never needed?
        \item<2-> It's possible to raise the exception explicitly with \funcname{raise\_notrace} to make raising as cheap as possible
        \item<3-> An example of use in the \funcname{dynlink} library of the OCaml compiler
      \end{itemize}
      \vfill
      \onslide<3->{
      \listocaml[firstline=205,lastline=209,highlightlines=207]{../ocaml/otherlibs/dynlink/dynlink\_compilerlibs/misc.ml}{otherlibs/dynlink/dynlink\_compilerlibs/misc.ml:205}
      }
      \endminipage
    \end{column}
    \begin{column}{0.55\textwidth}
    \only<4>{
      \mintcmm[highlightlines=3]{../src/notrace/camlNotrace\_\_fun_347.cmm}
      \listcmm{../src/notrace/camlNotrace\_\_all\_somes\_327.cmm}{all\_somes}
    }
    \onslide*<5->{
      \listobjdump[highlightlines={6-9}]{../src/notrace/camlNotrace\_\_fun_347.objdump}{anonymous function from \funcname{all\_somes}}
    }
    \end{column}
  \end{columns}
\end{frame}

\begin{frame}{Backtraces}{Summary table of the multiple kinds of raise}
  \begin{itemize}
    \item When debugging information is enabled and if the backtrace of an exception isn't needed, it's possible to raise with \funcname{raise\_notrace} for performance
    \item When debugging information is \emph{not} enabled, the compiler optimises \funcname{raise} calls to inline assembly (same effect as using \funcname{raise\_notrace})
  \end{itemize}
  \bigskip
  \centering
  \begin{tabular}{| l | c | c | c | c |}
    \hline
    \parbox{10em}{Debug information \\ (\funcarg{-g} flag)}  & \multicolumn{2}{|c|}{Disabled}                                     & \multicolumn{2}{|c|}{Enabled} \\
    \hline
    Raise kind                                               & \funcname{raise} & \funcname{raise\_notrace}                       & \funcname{raise} & \funcname{raise\_notrace} \\
    \hline
    Compilation                                              & \parbox{8em}{inline assembly\\ (optimization)} & inline assembly   & \funcname{caml\_raise\_exn} & inline assembly \\
    \hline
  \end{tabular}
\end{frame}


%
% Takeaway #5
%
\subsection*{Takeaway \#5}
\frameSubsectionTakeaway{}
\againframe<5>{takeaway}
}%
      \onslide*<3>{\providecommand\step{4}%
% Backtraces
%
\subsection{One advantage of raising with debugging information}
\frameSubsection{}{}

\begin{frame}{Backtraces}{Debugging information \& source code}
  \begin{columns}[c]
    \begin{column}{0.5\textwidth}
      \begin{itemize}
        \item Raising an exception is fun and games, but how do we know where the exception originated? $\rightarrow$ With the backtrace associated with the exception!
        \item In order to get a backtrace from the point where an exception was raised, it's necessary to compile with debugging information enabled (\funcarg{-g} flag for the compiler)
        \item Same example as in the `Nested handlers' section
      \end{itemize}
    \end{column}
    \begin{column}{0.5\textwidth}
      \listocaml[firstline=9,lastline=34]{../src/backtrace/backtrace.ml}{backtrace/backtrace.ml}
    \end{column}
  \end{columns}
\end{frame}

\begin{frame}{Backtraces}{CMM and disassembly of \funcname{definitely\_raise}}
  \begin{columns}[c]
    \begin{column}{0.5\textwidth}
      \minipage[c][0.66\textheight][s]{\columnwidth}
      \begin{itemize}
        \item<1-> An effect of compiling with debugging information (\funcarg{-g}): the CMM no longer uses \funcname{raise\_notrace} to raise the exception but \funcname{raise} instead
        \item<2-> Disassembly shows that CMM \funcname{raise} is actually a call to \funcname{caml\_raise\_exn}, a function in assembler provided by the runtime
      \end{itemize}
      \vfill
      \mintocaml[firstline=9,lastline=13,highlightlines=12]{../src/backtrace/backtrace.ml}
      \endminipage
    \end{column}
    \begin{column}{0.5\textwidth}
      \onslide*<1>{
        \mintcmm[highlightlines=5]{../src/backtrace/camlBacktrace\_\_definitely\_raise\_347.cmm}
      }
      \onslide*<2>{
        \mintobjdump[firstline=2,highlightlines=14]{../src/backtrace/camlBacktrace\_\_definitely\_raise\_347.objdump}
      }
    \end{column}
  \end{columns}
\end{frame}

\begin{frame}{Backtraces}{Introducing \funcname{caml\_raise\_exn}}
  \begin{columns}[c]
    \begin{column}{0.5\textwidth}
      \minipage[c][0.66\textheight][s]{\columnwidth}
      \onslide*<1>{
        \begin{itemize}
          \item Raising the exception is performed using \funcname{caml\_raise\_exn} which is
            \begin{itemize}
              \item An assembly function provided by the runtime
              \item Architecture specific
            \end{itemize}
          \item Let's see what it does differently from \funcname{raise\_notrace}\dots
          \item We are about to raise \typename{ExnC} with \funcname{caml\_raise\_exn} from \funcname{definitely\_raise}
        \end{itemize}
      }
      \onslide*<2>{
        \mintobjdump[firstline=2,highlightlines=14]{../src/backtrace/camlBacktrace\_\_definitely\_raise\_347.objdump}
      }
      \vfill
      \mintocaml[firstline=9,lastline=13,highlightlines=12]{../src/backtrace/backtrace.ml}
      \endminipage
    \end{column}
    \begin{column}{0.5\textwidth}
      \centering
      \providecommand\step{1}\input{diagrams/backtrace/backtrace.tex}
    \end{column}
  \end{columns}
\end{frame}

\begin{frame}{Backtraces}{But before that, we need more fields from \typename{caml\_domain\_state}}
  \begin{columns}
    \begin{column}{0.5\textwidth}
      \begin{itemize}
        \item Remember \typename{caml\_domain\_state} the per-domain runtime state?
        \item Some other members are of interest to us now\dots
        \item \localname{backtrace\_active} is used to know if the runtime should collect backtraces
        \item \localname{backtrace\_active} can be set by
          \begin{itemize}
            \item The \funcarg{b} flag in the \funcname{OCAMLRUNPARAM} environment variable
            \item \mintinline{ocaml}{Printexc.record_backtrace : bool -> unit} \footnotemark
          \end{itemize}
      \end{itemize}
    \end{column}
    \begin{column}{0.5\textwidth}
      \begin{listing}
        \mintc[highlightlines={9-11}]{src/ocaml/domain\_state\_backtrace.h}
        \caption{runtime/caml/domain\_state.h}
      \end{listing}
    \end{column}
  \end{columns}
  \footnotetext[1]{https://v2.ocaml.org/api/Printexc.html}
\end{frame}

\newcommand\listCamlRaiseExn[1][]{
  \listasm[firstline=702,lastline=725,#1]{../ocaml/runtime/amd64.S}{runtime/amd64.S:702}}

\begin{frame}{Backtraces}{Walk through \funcname{caml\_raise\_exn}}
  \begin{columns}[T]
    \begin{column}{0.5\textwidth}
      \begin{itemize}
        \item<1-> First checks whether exceptions backtrace should be recorded
        \item<1-> By checking the \funcarg{domain\_state->backtrace\_active} value
        \item<2-> If not, acts as \funcname{raise\_notrace} with \funcname{RESTORE\_EXN\_HANDLER\_OCAML}
          \begin{itemize}
            \item Set \regname{RSP} to \localname{domain\_state->exn\_handler} value
            \item Restore \localname{domain\_state->exn\_handler} from trap
            \item Jump with \funcname{ret} (same effect as using \funcname{pop} and \funcname{jmp})
          \end{itemize}
        \item<3-> Otherwise, switches to the C stack to be able to execute C code
        \item<4-> Calls \funcname{caml\_stash\_backtrace}, a C function provided by the runtime\ignorespaces
          \onslide<6->{, with
            \begin{itemize}
              \item \funcarg{exn}: exception value
              \item \funcarg{pc}: code address of the raising point
              \item \funcarg{sp}: stack address of the raising function
              \item \funcarg{trapsp}: stack address of the next handler
            \end{itemize}
          }
      \end{itemize}
      \bigskip
      \mintocaml[firstline=9,lastline=13,highlightlines=12]{../src/backtrace/backtrace.ml}
    \end{column}
    \begin{column}{0.5\textwidth}
      \raggedleft
      \onslide*<1,3-4>{
        \mintc[firstline=190,lastline=192]{../ocaml/runtime/amd64.S}
        \mintc[firstline=234,lastline=237]{../ocaml/runtime/amd64.S}
      }
      \onslide*<2>{
        \mintc[firstline=190,lastline=192,highlightlines={191-192}]{../ocaml/runtime/amd64.S}
        \mintc[firstline=234,lastline=237,highlightlines={235-237}]{../ocaml/runtime/amd64.S}
      }
      \onslide*<1>{\listCamlRaiseExn[highlightlines=705]{}}%
      \onslide*<2>{\listCamlRaiseExn[highlightlines={707-708}]{}}%
      \onslide*<3>{\listCamlRaiseExn[highlightlines={712-714}]{}}%
      \onslide*<4>{\listCamlRaiseExn[highlightlines={715-720}]{}}%
      \onslide*<5>{\providecommand\step{1}\input{diagrams/backtrace/backtrace.tex}}%
      \onslide*<6>{\providecommand\step{2}\input{diagrams/backtrace/backtrace.tex}}%
    \end{column}
  \end{columns}
\end{frame}

\newcommand\listStashBacktrace[1][]{
  \listc[firstline=91,lastline=120,#1]{../ocaml/runtime/backtrace\_nat.c}{runtime/backtrace\_nat.c:91}}

\begin{frame}{Backtraces}{Introducing \funcname{caml\_stash\_backtrace}}
  \begin{columns}[c]
    \begin{column}{0.5\textwidth}
      \minipage[c][0.66\textheight][s]{\columnwidth}
      \begin{itemize}
        \item<1-> A C function provided by the runtime (specific to native code)
        \item<2-> It scans the stack, between the stack pointer at the raising point up to the next trap, in order to collect the names of the detected function frames
          \onslide*<2>{
            \begin{itemize}
              \item And more information, like line number, column number...
            \end{itemize}
          }
        \item<3-> It uses the same technique and information as the Garbage Collector (GC) uses to scan the values live on the stack
          \onslide*<4>{
            \begin{itemize}
              \item \typename{frame\_descr} are at the core of stack unwinding
            \end{itemize}
          }
      \end{itemize}
      \vfill
      \mintocaml[firstline=9,lastline=13,highlightlines=12]{../src/backtrace/backtrace.ml}
      \endminipage
    \end{column}
    \begin{column}{0.5\textwidth}
      \onslide*<1>{\listStashBacktrace[highlightlines=91]{}}
      \onslide*<2>{\listStashBacktrace[highlightlines={91,117-118}]{}}
      \onslide*<3->{\listStashBacktrace[highlightlines={94,106,109-110}]{}}
    \end{column}
  \end{columns}
\end{frame}

\newcommand\listFrameDescr[1][]{
  \listocaml[firstline=111,lastline=124,#1]{../ocaml/asmcomp/emitaux.ml}{asmcomp/emitaux.ml:111}}
\newcommand\listDebugInfo[1][]{
  \listocaml[firstline=38,lastline=49,#1]{../ocaml/lambda/debuginfo.mli}{lambda/debuginfo.mli:38}}

\begin{frame}{Backtraces}{\typename{frame\_descr} and \typename{frame\_debuginfo} in a nutshell}
  \begin{columns}[c]
    \begin{column}{0.5\textwidth}
      \begin{itemize}
        \item<1-> For every return address in an OCaml program, there is a associated \typename{frame\_descr}
        \item<2-> A \typename{frame\_descr} specifies
        \begin{itemize}
          \item<2-> The size of the function stack frame
          \item<3-> Where live values are on the stack (so that the GC can mark them as live and prevent from being swept)
        \end{itemize}
        \item<4-> When compiled with debug information, \typename{frame\_descr} will be linked to a \typename{frame\_debuginfo}
        \begin{itemize}
          \item<5-> Contains information about the position in the source code (file name, line and column number)
        \end{itemize}
      \end{itemize}
    \end{column}
    \begin{column}{0.5\textwidth}
        \onslide*<1>{\listFrameDescr[highlightlines={118-122}]{}}
        \onslide*<2>{\listFrameDescr[highlightlines={120}]{}}
        \onslide*<3>{\listFrameDescr[highlightlines={121}]{}}
        \onslide*<4->{\listFrameDescr[highlightlines={122}]{}}
        \medskip
        \onslide*<1-3>{\listDebugInfo{}}
        \onslide*<4>{\listDebugInfo[highlightlines={38,49}]{}}
        \onslide*<5>{\listDebugInfo[highlightlines={39-46}]{}}
    \end{column}
  \end{columns}
\end{frame}

\begin{frame}{Backtraces}{Scanning the stack with \typename{frame\_descr}}
  \begin{columns}[c]
    \begin{column}{0.5\textwidth}
      \begin{itemize}
        \item Given the return address of the code that raises the exception by calling \funcname{caml\_raise\_exn} (\funcarg{pc} argument of \funcname{caml\_stash\_backtrace})
        \item And the position of the stack pointer (\funcarg{sp} argument of \funcname{caml\_stash\_backtrace})
        \item The frame size given by \typename{frame\_descr}.\funcarg{frame\_size}, allows to find the end of the previous frame
        \item And the return address of the caller of this function
        \bigskip
        \onslide<3->{
        \item Note that traps (here the reddish \funcname{ExnA}, \funcname{ExnB} and \funcname{ExnC} blocks) belong to the frame that installed them, and so the frame size changes when traps are removed
        }
      \end{itemize}
    \end{column}
    \begin{column}{0.5\textwidth}
      \raggedleft
      \onslide*<1>{\providecommand\step{2}\input{diagrams/backtrace/backtrace.tex}}%
      \onslide*<2>{\providecommand\step{1}\input{diagrams/backtrace/scanstack.tex}}%
      \onslide*<3>{\providecommand\step{2}\input{diagrams/backtrace/scanstack.tex}}%
      \onslide*<4>{\providecommand\step{3}\input{diagrams/backtrace/scanstack.tex}}%
      \onslide*<5>{\providecommand\step{4}\input{diagrams/backtrace/scanstack.tex}}%
      \onslide*<6>{\providecommand\step{5}\input{diagrams/backtrace/scanstack.tex}}%
    \end{column}
  \end{columns}
\end{frame}

% Duplicate of previous-previous slide
\begin{frame}{Backtraces}{Continuing on \funcname{caml\_stash\_backtrace}}
  \begin{columns}[c]
    \begin{column}{0.5\textwidth}
      \minipage[c][0.75\textheight][s]{\columnwidth}
      \begin{itemize}
          \color{gray}
        \item A C function provided by the runtime (specific to native code)
        \item It scans the stack, between the stack pointer at the raising point up to the next trap, in order to collect the names of the detected function frames
        \item It uses the same technique and information as the Garbage Collector (GC) uses to scan the values live on the stack
      \end{itemize}
      \begin{itemize}
        \item For each function frame detected, store a pointer to the \typename{frame\_descr} in the \localname{domain\_state->backtrace\_buffer} array, effectively constructing the backtrace
      \end{itemize}
      \onslide<2->{
        \begin{itemize}
            \smallskip
          \item Constructed backtrace:
            \funcname{definitely\_raise},
            \onslide<3->{\funcname{probably\_raise}}
        \end{itemize}
      }
      \vfill
      \mintocaml[firstline=9,lastline=13,highlightlines=12]{../src/backtrace/backtrace.ml}
      \endminipage
    \end{column}
    \begin{column}{0.5\textwidth}
      \raggedleft
      \onslide*<1>{\listStashBacktrace[highlightlines={114-115}]{}}
      \onslide*<2>{\providecommand\step{3}\input{diagrams/backtrace/backtrace.tex}}%
      \onslide*<3>{\providecommand\step{4}\input{diagrams/backtrace/backtrace.tex}}%
    \end{column}
  \end{columns}
\end{frame}

\begin{frame}{Backtraces}{Back to \funcname{caml\_raise\_exn}}
  \begin{columns}[c]
    \begin{column}{0.5\textwidth}
      \minipage[c][0.66\textheight][s]{\columnwidth}
      \begin{itemize}\color{gray}
        \item Checks wheteher exceptions backtrace should be recorded, by checking the \funcarg{domain\_state->backtrace\_active} value
        \item Switches to the C stack to be able to execute C code
        \item Calls \funcname{caml\_stash\_backtrace}, to collect the backtrace up to the next trap
      \end{itemize}
      \onslide*<2->{
        \begin{itemize}
          \item<2-> Executes the exception handler in \funcname{probably\_raise} with \funcname{RESTORE\_EXN\_HANDLER\_OCAML}
            \begin{itemize}
              \item Set \regname{RSP} to \localname{domain\_state->exn\_handler} value
              \item Restore \localname{domain\_state->exn\_handler} from trap
              \item Jump to the handler code with \funcname{ret}
            \end{itemize}
        \end{itemize}
      }
      \vfill
      \onslide*<1>{\mintocaml[firstline=9,lastline=13,highlightlines=12]{../src/backtrace/backtrace.ml}}
      \onslide*<2->{\mintocaml[firstline=15,lastline=21,highlightlines={19-21}]{../src/backtrace/backtrace.ml}}
      \endminipage
    \end{column}
    \begin{column}{0.5\textwidth}
      \centering
      \onslide*<1>{
        \mintc[firstline=190,lastline=192]{../ocaml/runtime/amd64.S}
        \mintc[firstline=234,lastline=237]{../ocaml/runtime/amd64.S}
      }
      \onslide*<2>{
        \mintc[firstline=190,lastline=192,highlightlines={191-192}]{../ocaml/runtime/amd64.S}
        \mintc[firstline=234,lastline=237,highlightlines={235-237}]{../ocaml/runtime/amd64.S}
      }
      \onslide*<1>{\listCamlRaiseExn[highlightlines=720]{}}
      \onslide*<2>{\listCamlRaiseExn[highlightlines=722]{}}
      \onslide*<3->{\providecommand\step{5}\input{diagrams/backtrace/backtrace.tex}}
    \end{column}
  \end{columns}
\end{frame}

\begin{frame}{Backtraces}{\funcname{caml\_probably\_raise}'s handler doesn't catch \typename{ExnC}}
  \begin{columns}[c]
    \begin{column}{0.45\textwidth}
      \minipage[c][0.75\textheight][s]{\columnwidth}
      \begin{itemize}
        \item<1-> \funcname{match \dots with}\footnotemark pattern matching can also match exceptions
        \item<1-> The CMM of \funcname{probably\_raise} shows that this \funcname{match \dots with} is implemented with a pattern matching and a \funcname{try \dots with}
        \item<1-> But this one only matches on \typename{ExnA} exception
        \item<2-> Since the pattern matching fails to catch the exception, \funcname{reraise} is called with our current \typename{ExnC} exception
        \item<3-> Disassembly shows that \funcname{reraise} is actually a call to \funcname{caml\_reraise\_exn}, which is also an assembly function provided by the runtime
      \end{itemize}
      \vfill
      \mintocaml[firstline=15,lastline=21,highlightlines={18-21}]{../src/backtrace/backtrace.ml}
      \endminipage
    \end{column}
    \begin{column}{0.55\textwidth}
      \centering
      \onslide*<1>{\mintcmm[highlightlines={10-18}]{../src/backtrace/camlBacktrace\_\_probably\_raise\_351.cmm}}
      \onslide*<2>{\listcmm[highlightlines=18]{../src/backtrace/camlBacktrace\_\_probably\_raise_351.cmm}{CMM of probably\_raise}}
      \onslide*<3>{\mintobjdump[firstline=5,lastline=35,highlightlines=31]{../src/backtrace/camlBacktrace\_\_probably\_raise_351.objdump}}
    \end{column}
  \end{columns}
  \only<1>{\footnotetext[1]{https://blog.janestreet.com/pattern-matching-and-exception-handling-unite/}}
\end{frame}

\newcommand\listCamlReraiseExn[1][]{
  \listasm[firstline=727,lastline=734,#1]{../ocaml/runtime/amd64.S}{runtime/amd64.S:727}}

\begin{frame}{Backtraces}{Introducing \funcname{caml\_reraise\_exn}}
  \begin{columns}[c]
    \begin{column}{0.5\textwidth}
      \minipage[c][0.66\textheight][s]{\columnwidth}
      \begin{itemize}
        \item<1-> The exception have to be reraised to the next handler
        \item<2-> First, checks if the backtrace should be collected with \funcarg{domain\_state->backtrace\_active}
        \only<3>{
        \begin{itemize}
            \item If not, behaves like a \funcname{raise\_notrace} with \funcname{RESTORE\_EXN\_HANDLER\_OCAML}
        \end{itemize}
        }
        \item<4-> Jumps into \funcname{caml\_raise\_exn}
        \item<5-> Calls \funcname{caml\_stash\_backtrace} again, to scan the next part of the stack: from the trap just pop-ed to the next trap
      \end{itemize}
      \vfill
      \mintocaml[firstline=15,lastline=21,highlightlines={18-21}]{../src/backtrace/backtrace.ml}
      \endminipage
    \end{column}
    \begin{column}{0.5\textwidth}
      \raggedleft
      \onslide*<1>{\listCamlReraiseExn{}}
      \onslide*<2>{\listCamlReraiseExn[highlightlines=729]{}}
      \onslide*<3>{
        \mintc[firstline=190,lastline=192,highlightlines={191-192}]{../ocaml/runtime/amd64.S}
        \mintc[firstline=234,lastline=237,highlightlines={235-237}]{../ocaml/runtime/amd64.S}
        \medskip
        \listCamlReraiseExn[highlightlines=731]{}
      }
      \onslide*<4-5>{
        % TODO refactor with \listCamlReraiseExn
        \mintasm[firstline=727,lastline=734,highlightlines=730]{../ocaml/runtime/amd64.S}
        \medskip
      }
      \onslide*<4>{\listCamlRaiseExn[highlightlines=711]{}}
      \onslide*<5>{\listCamlRaiseExn[highlightlines=720]{}}
    \end{column}
  \end{columns}
\end{frame}

\begin{frame}{Backtraces}{Backtrace collection continues}
  \begin{columns}[c]
    \begin{column}{0.55\textwidth}
      \minipage[c][0.75\textheight][s]{\columnwidth}
      \begin{itemize}
        \item<1-> \funcname{caml\_stash\_backtrace} scans the older part of the stack, up to the next trap
        \item<6-> Controls jumps to the nested exception handler in \funcname{maybe\_raise}, which matches only on \typename{ExnB}
        \item<7-> \funcname{caml\_reraise\_exn} is called again, backtrace collection resumes with \funcname{caml\_stash\_backtrace}
      \end{itemize}
      \medskip
      \begin{itemize}
        \item<2-> Constructed backtrace:
        \begin{itemize}
          \item <2-> \funcname{definitely\_raise}
          \item <2-> \funcname{probably\_raise}
          \item <3-> \funcname{probably\_raise} (re-raise)
          \item <4-> \funcname{maybe\_raise}
          \item <5-> \funcname{main}
          \item <8-> \funcname{main} (re-raise)
        \end{itemize}
      \end{itemize}
      \vfill
      \onslide*<1-5>{\mintocaml[firstline=15,lastline=21]{../src/backtrace/backtrace.ml}}
      \onslide*<6>{\mintocaml[firstline=28,lastline=34,highlightlines=31]{../src/backtrace/backtrace.ml}}
      \onslide*<7->{\mintocaml[firstline=28,lastline=34,highlightlines=33]{../src/backtrace/backtrace.ml}}
      \endminipage
    \end{column}
    \begin{column}{0.45\textwidth}
      \raggedleft
      \onslide*<1>{\providecommand\step{6}\input{diagrams/backtrace/backtrace.tex}}%
      \onslide*<2>{\providecommand\step{6}\input{diagrams/backtrace/backtrace.tex}}%
      \onslide*<3>{\providecommand\step{7}\input{diagrams/backtrace/backtrace.tex}}%
      \onslide*<4>{\providecommand\step{8}\input{diagrams/backtrace/backtrace.tex}}%
      \onslide*<5>{\providecommand\step{9}\input{diagrams/backtrace/backtrace.tex}}%
      \onslide*<6>{\providecommand\step{10}\input{diagrams/backtrace/backtrace.tex}}%
      \onslide*<7>{\providecommand\step{11}\input{diagrams/backtrace/backtrace.tex}}%
      \onslide*<8>{\providecommand\step{12}\input{diagrams/backtrace/backtrace.tex}}%
      \onslide*<9>{\providecommand\step{13}\input{diagrams/backtrace/backtrace.tex}}%
    \end{column}
  \end{columns}
\end{frame}

\begin{frame}{Backtraces}{Printing the backtrace}
  \begin{columns}[c]
    \begin{column}{0.45\textwidth}
      \minipage[c][0.66\textheight][s]{\columnwidth}
      \begin{itemize}
        \item<1-> The exception backtrace can now be printed to \funcarg{stdout} with \funcname{Printexc.print_backtrace}
        \item<2-> First the exception backtrace is retrieved into an OCaml array with \funcname{get_raw_backtrace }
        \item<3-> Which is implemented as the \funcname{caml_get_exception_raw_backtrace} C function in the runtime
        \item<4-> And finally printed with \funcname{print_exception_backtrace}
      \end{itemize}
      \vfill
      \mintocaml[firstline=28,lastline=34,highlightlines=33]{../src/backtrace/backtrace.ml}
      \endminipage
    \end{column}
    \begin{column}{0.55\textwidth}
      \onslide*<1,2>{
        \mintocaml[firstline=100,lastline=101]{../ocaml/stdlib/printexc.ml}
      }
      \only<1>{
        \listocaml[firstline=170,lastline=172]{../ocaml/stdlib/printexc.ml}{stdlib/printexc.ml:170}
      }
      \only<2>{
        \listocaml[firstline=170,lastline=172,highlightlines=172]{../ocaml/stdlib/printexc.ml}{stdlib/printexc.ml:170}
      }
      \only<3>{
        \mintc[firstline=154,lastline=156]{../ocaml/runtime/backtrace.c}
        \listc[firstline=171,lastline=192,highlightlines={184-187}]{../ocaml/runtime/backtrace.c}{runtime/backtrace.c:154}
      }
      \only<4>{
        \listocaml[firstline=155,lastline=165]{../ocaml/stdlib/printexc.ml}{stdlib/printexc.ml:55}
      }
    \end{column}
  \end{columns}
\end{frame}


\subsection{The multiple kinds of raise}
\frameSubsection{}{}

\begin{frame}{Backtraces}{When backtraces are not needed}
  \begin{columns}[c]
    \begin{column}{0.45\textwidth}
      \minipage[c][0.66\textheight][s]{\columnwidth}
      \begin{itemize}
        \item<1-> What if the backtrace for an exception is never needed?
        \item<2-> It's possible to raise the exception explicitly with \funcname{raise\_notrace} to make raising as cheap as possible
        \item<3-> An example of use in the \funcname{dynlink} library of the OCaml compiler
      \end{itemize}
      \vfill
      \onslide<3->{
      \listocaml[firstline=205,lastline=209,highlightlines=207]{../ocaml/otherlibs/dynlink/dynlink\_compilerlibs/misc.ml}{otherlibs/dynlink/dynlink\_compilerlibs/misc.ml:205}
      }
      \endminipage
    \end{column}
    \begin{column}{0.55\textwidth}
    \only<4>{
      \mintcmm[highlightlines=3]{../src/notrace/camlNotrace\_\_fun_347.cmm}
      \listcmm{../src/notrace/camlNotrace\_\_all\_somes\_327.cmm}{all\_somes}
    }
    \onslide*<5->{
      \listobjdump[highlightlines={6-9}]{../src/notrace/camlNotrace\_\_fun_347.objdump}{anonymous function from \funcname{all\_somes}}
    }
    \end{column}
  \end{columns}
\end{frame}

\begin{frame}{Backtraces}{Summary table of the multiple kinds of raise}
  \begin{itemize}
    \item When debugging information is enabled and if the backtrace of an exception isn't needed, it's possible to raise with \funcname{raise\_notrace} for performance
    \item When debugging information is \emph{not} enabled, the compiler optimises \funcname{raise} calls to inline assembly (same effect as using \funcname{raise\_notrace})
  \end{itemize}
  \bigskip
  \centering
  \begin{tabular}{| l | c | c | c | c |}
    \hline
    \parbox{10em}{Debug information \\ (\funcarg{-g} flag)}  & \multicolumn{2}{|c|}{Disabled}                                     & \multicolumn{2}{|c|}{Enabled} \\
    \hline
    Raise kind                                               & \funcname{raise} & \funcname{raise\_notrace}                       & \funcname{raise} & \funcname{raise\_notrace} \\
    \hline
    Compilation                                              & \parbox{8em}{inline assembly\\ (optimization)} & inline assembly   & \funcname{caml\_raise\_exn} & inline assembly \\
    \hline
  \end{tabular}
\end{frame}


%
% Takeaway #5
%
\subsection*{Takeaway \#5}
\frameSubsectionTakeaway{}
\againframe<5>{takeaway}
}%
    \end{column}
  \end{columns}
\end{frame}

\begin{frame}{Backtraces}{Back to \funcname{caml\_raise\_exn}}
  \begin{columns}[c]
    \begin{column}{0.5\textwidth}
      \minipage[c][0.66\textheight][s]{\columnwidth}
      \begin{itemize}\color{gray}
        \item Checks wheteher exceptions backtrace should be recorded, by checking the \funcarg{domain\_state->backtrace\_active} value
        \item Switches to the C stack to be able to execute C code
        \item Calls \funcname{caml\_stash\_backtrace}, to collect the backtrace up to the next trap
      \end{itemize}
      \onslide*<2->{
        \begin{itemize}
          \item<2-> Executes the exception handler in \funcname{probably\_raise} with \funcname{RESTORE\_EXN\_HANDLER\_OCAML}
            \begin{itemize}
              \item Set \regname{RSP} to \localname{domain\_state->exn\_handler} value
              \item Restore \localname{domain\_state->exn\_handler} from trap
              \item Jump to the handler code with \funcname{ret}
            \end{itemize}
        \end{itemize}
      }
      \vfill
      \onslide*<1>{\mintocaml[firstline=9,lastline=13,highlightlines=12]{../src/backtrace/backtrace.ml}}
      \onslide*<2->{\mintocaml[firstline=15,lastline=21,highlightlines={19-21}]{../src/backtrace/backtrace.ml}}
      \endminipage
    \end{column}
    \begin{column}{0.5\textwidth}
      \centering
      \onslide*<1>{
        \mintc[firstline=190,lastline=192]{../ocaml/runtime/amd64.S}
        \mintc[firstline=234,lastline=237]{../ocaml/runtime/amd64.S}
      }
      \onslide*<2>{
        \mintc[firstline=190,lastline=192,highlightlines={191-192}]{../ocaml/runtime/amd64.S}
        \mintc[firstline=234,lastline=237,highlightlines={235-237}]{../ocaml/runtime/amd64.S}
      }
      \onslide*<1>{\listCamlRaiseExn[highlightlines=720]{}}
      \onslide*<2>{\listCamlRaiseExn[highlightlines=722]{}}
      \onslide*<3->{\providecommand\step{5}%
% Backtraces
%
\subsection{One advantage of raising with debugging information}
\frameSubsection{}{}

\begin{frame}{Backtraces}{Debugging information \& source code}
  \begin{columns}[c]
    \begin{column}{0.5\textwidth}
      \begin{itemize}
        \item Raising an exception is fun and games, but how do we know where the exception originated? $\rightarrow$ With the backtrace associated with the exception!
        \item In order to get a backtrace from the point where an exception was raised, it's necessary to compile with debugging information enabled (\funcarg{-g} flag for the compiler)
        \item Same example as in the `Nested handlers' section
      \end{itemize}
    \end{column}
    \begin{column}{0.5\textwidth}
      \listocaml[firstline=9,lastline=34]{../src/backtrace/backtrace.ml}{backtrace/backtrace.ml}
    \end{column}
  \end{columns}
\end{frame}

\begin{frame}{Backtraces}{CMM and disassembly of \funcname{definitely\_raise}}
  \begin{columns}[c]
    \begin{column}{0.5\textwidth}
      \minipage[c][0.66\textheight][s]{\columnwidth}
      \begin{itemize}
        \item<1-> An effect of compiling with debugging information (\funcarg{-g}): the CMM no longer uses \funcname{raise\_notrace} to raise the exception but \funcname{raise} instead
        \item<2-> Disassembly shows that CMM \funcname{raise} is actually a call to \funcname{caml\_raise\_exn}, a function in assembler provided by the runtime
      \end{itemize}
      \vfill
      \mintocaml[firstline=9,lastline=13,highlightlines=12]{../src/backtrace/backtrace.ml}
      \endminipage
    \end{column}
    \begin{column}{0.5\textwidth}
      \onslide*<1>{
        \mintcmm[highlightlines=5]{../src/backtrace/camlBacktrace\_\_definitely\_raise\_347.cmm}
      }
      \onslide*<2>{
        \mintobjdump[firstline=2,highlightlines=14]{../src/backtrace/camlBacktrace\_\_definitely\_raise\_347.objdump}
      }
    \end{column}
  \end{columns}
\end{frame}

\begin{frame}{Backtraces}{Introducing \funcname{caml\_raise\_exn}}
  \begin{columns}[c]
    \begin{column}{0.5\textwidth}
      \minipage[c][0.66\textheight][s]{\columnwidth}
      \onslide*<1>{
        \begin{itemize}
          \item Raising the exception is performed using \funcname{caml\_raise\_exn} which is
            \begin{itemize}
              \item An assembly function provided by the runtime
              \item Architecture specific
            \end{itemize}
          \item Let's see what it does differently from \funcname{raise\_notrace}\dots
          \item We are about to raise \typename{ExnC} with \funcname{caml\_raise\_exn} from \funcname{definitely\_raise}
        \end{itemize}
      }
      \onslide*<2>{
        \mintobjdump[firstline=2,highlightlines=14]{../src/backtrace/camlBacktrace\_\_definitely\_raise\_347.objdump}
      }
      \vfill
      \mintocaml[firstline=9,lastline=13,highlightlines=12]{../src/backtrace/backtrace.ml}
      \endminipage
    \end{column}
    \begin{column}{0.5\textwidth}
      \centering
      \providecommand\step{1}\input{diagrams/backtrace/backtrace.tex}
    \end{column}
  \end{columns}
\end{frame}

\begin{frame}{Backtraces}{But before that, we need more fields from \typename{caml\_domain\_state}}
  \begin{columns}
    \begin{column}{0.5\textwidth}
      \begin{itemize}
        \item Remember \typename{caml\_domain\_state} the per-domain runtime state?
        \item Some other members are of interest to us now\dots
        \item \localname{backtrace\_active} is used to know if the runtime should collect backtraces
        \item \localname{backtrace\_active} can be set by
          \begin{itemize}
            \item The \funcarg{b} flag in the \funcname{OCAMLRUNPARAM} environment variable
            \item \mintinline{ocaml}{Printexc.record_backtrace : bool -> unit} \footnotemark
          \end{itemize}
      \end{itemize}
    \end{column}
    \begin{column}{0.5\textwidth}
      \begin{listing}
        \mintc[highlightlines={9-11}]{src/ocaml/domain\_state\_backtrace.h}
        \caption{runtime/caml/domain\_state.h}
      \end{listing}
    \end{column}
  \end{columns}
  \footnotetext[1]{https://v2.ocaml.org/api/Printexc.html}
\end{frame}

\newcommand\listCamlRaiseExn[1][]{
  \listasm[firstline=702,lastline=725,#1]{../ocaml/runtime/amd64.S}{runtime/amd64.S:702}}

\begin{frame}{Backtraces}{Walk through \funcname{caml\_raise\_exn}}
  \begin{columns}[T]
    \begin{column}{0.5\textwidth}
      \begin{itemize}
        \item<1-> First checks whether exceptions backtrace should be recorded
        \item<1-> By checking the \funcarg{domain\_state->backtrace\_active} value
        \item<2-> If not, acts as \funcname{raise\_notrace} with \funcname{RESTORE\_EXN\_HANDLER\_OCAML}
          \begin{itemize}
            \item Set \regname{RSP} to \localname{domain\_state->exn\_handler} value
            \item Restore \localname{domain\_state->exn\_handler} from trap
            \item Jump with \funcname{ret} (same effect as using \funcname{pop} and \funcname{jmp})
          \end{itemize}
        \item<3-> Otherwise, switches to the C stack to be able to execute C code
        \item<4-> Calls \funcname{caml\_stash\_backtrace}, a C function provided by the runtime\ignorespaces
          \onslide<6->{, with
            \begin{itemize}
              \item \funcarg{exn}: exception value
              \item \funcarg{pc}: code address of the raising point
              \item \funcarg{sp}: stack address of the raising function
              \item \funcarg{trapsp}: stack address of the next handler
            \end{itemize}
          }
      \end{itemize}
      \bigskip
      \mintocaml[firstline=9,lastline=13,highlightlines=12]{../src/backtrace/backtrace.ml}
    \end{column}
    \begin{column}{0.5\textwidth}
      \raggedleft
      \onslide*<1,3-4>{
        \mintc[firstline=190,lastline=192]{../ocaml/runtime/amd64.S}
        \mintc[firstline=234,lastline=237]{../ocaml/runtime/amd64.S}
      }
      \onslide*<2>{
        \mintc[firstline=190,lastline=192,highlightlines={191-192}]{../ocaml/runtime/amd64.S}
        \mintc[firstline=234,lastline=237,highlightlines={235-237}]{../ocaml/runtime/amd64.S}
      }
      \onslide*<1>{\listCamlRaiseExn[highlightlines=705]{}}%
      \onslide*<2>{\listCamlRaiseExn[highlightlines={707-708}]{}}%
      \onslide*<3>{\listCamlRaiseExn[highlightlines={712-714}]{}}%
      \onslide*<4>{\listCamlRaiseExn[highlightlines={715-720}]{}}%
      \onslide*<5>{\providecommand\step{1}\input{diagrams/backtrace/backtrace.tex}}%
      \onslide*<6>{\providecommand\step{2}\input{diagrams/backtrace/backtrace.tex}}%
    \end{column}
  \end{columns}
\end{frame}

\newcommand\listStashBacktrace[1][]{
  \listc[firstline=91,lastline=120,#1]{../ocaml/runtime/backtrace\_nat.c}{runtime/backtrace\_nat.c:91}}

\begin{frame}{Backtraces}{Introducing \funcname{caml\_stash\_backtrace}}
  \begin{columns}[c]
    \begin{column}{0.5\textwidth}
      \minipage[c][0.66\textheight][s]{\columnwidth}
      \begin{itemize}
        \item<1-> A C function provided by the runtime (specific to native code)
        \item<2-> It scans the stack, between the stack pointer at the raising point up to the next trap, in order to collect the names of the detected function frames
          \onslide*<2>{
            \begin{itemize}
              \item And more information, like line number, column number...
            \end{itemize}
          }
        \item<3-> It uses the same technique and information as the Garbage Collector (GC) uses to scan the values live on the stack
          \onslide*<4>{
            \begin{itemize}
              \item \typename{frame\_descr} are at the core of stack unwinding
            \end{itemize}
          }
      \end{itemize}
      \vfill
      \mintocaml[firstline=9,lastline=13,highlightlines=12]{../src/backtrace/backtrace.ml}
      \endminipage
    \end{column}
    \begin{column}{0.5\textwidth}
      \onslide*<1>{\listStashBacktrace[highlightlines=91]{}}
      \onslide*<2>{\listStashBacktrace[highlightlines={91,117-118}]{}}
      \onslide*<3->{\listStashBacktrace[highlightlines={94,106,109-110}]{}}
    \end{column}
  \end{columns}
\end{frame}

\newcommand\listFrameDescr[1][]{
  \listocaml[firstline=111,lastline=124,#1]{../ocaml/asmcomp/emitaux.ml}{asmcomp/emitaux.ml:111}}
\newcommand\listDebugInfo[1][]{
  \listocaml[firstline=38,lastline=49,#1]{../ocaml/lambda/debuginfo.mli}{lambda/debuginfo.mli:38}}

\begin{frame}{Backtraces}{\typename{frame\_descr} and \typename{frame\_debuginfo} in a nutshell}
  \begin{columns}[c]
    \begin{column}{0.5\textwidth}
      \begin{itemize}
        \item<1-> For every return address in an OCaml program, there is a associated \typename{frame\_descr}
        \item<2-> A \typename{frame\_descr} specifies
        \begin{itemize}
          \item<2-> The size of the function stack frame
          \item<3-> Where live values are on the stack (so that the GC can mark them as live and prevent from being swept)
        \end{itemize}
        \item<4-> When compiled with debug information, \typename{frame\_descr} will be linked to a \typename{frame\_debuginfo}
        \begin{itemize}
          \item<5-> Contains information about the position in the source code (file name, line and column number)
        \end{itemize}
      \end{itemize}
    \end{column}
    \begin{column}{0.5\textwidth}
        \onslide*<1>{\listFrameDescr[highlightlines={118-122}]{}}
        \onslide*<2>{\listFrameDescr[highlightlines={120}]{}}
        \onslide*<3>{\listFrameDescr[highlightlines={121}]{}}
        \onslide*<4->{\listFrameDescr[highlightlines={122}]{}}
        \medskip
        \onslide*<1-3>{\listDebugInfo{}}
        \onslide*<4>{\listDebugInfo[highlightlines={38,49}]{}}
        \onslide*<5>{\listDebugInfo[highlightlines={39-46}]{}}
    \end{column}
  \end{columns}
\end{frame}

\begin{frame}{Backtraces}{Scanning the stack with \typename{frame\_descr}}
  \begin{columns}[c]
    \begin{column}{0.5\textwidth}
      \begin{itemize}
        \item Given the return address of the code that raises the exception by calling \funcname{caml\_raise\_exn} (\funcarg{pc} argument of \funcname{caml\_stash\_backtrace})
        \item And the position of the stack pointer (\funcarg{sp} argument of \funcname{caml\_stash\_backtrace})
        \item The frame size given by \typename{frame\_descr}.\funcarg{frame\_size}, allows to find the end of the previous frame
        \item And the return address of the caller of this function
        \bigskip
        \onslide<3->{
        \item Note that traps (here the reddish \funcname{ExnA}, \funcname{ExnB} and \funcname{ExnC} blocks) belong to the frame that installed them, and so the frame size changes when traps are removed
        }
      \end{itemize}
    \end{column}
    \begin{column}{0.5\textwidth}
      \raggedleft
      \onslide*<1>{\providecommand\step{2}\input{diagrams/backtrace/backtrace.tex}}%
      \onslide*<2>{\providecommand\step{1}\input{diagrams/backtrace/scanstack.tex}}%
      \onslide*<3>{\providecommand\step{2}\input{diagrams/backtrace/scanstack.tex}}%
      \onslide*<4>{\providecommand\step{3}\input{diagrams/backtrace/scanstack.tex}}%
      \onslide*<5>{\providecommand\step{4}\input{diagrams/backtrace/scanstack.tex}}%
      \onslide*<6>{\providecommand\step{5}\input{diagrams/backtrace/scanstack.tex}}%
    \end{column}
  \end{columns}
\end{frame}

% Duplicate of previous-previous slide
\begin{frame}{Backtraces}{Continuing on \funcname{caml\_stash\_backtrace}}
  \begin{columns}[c]
    \begin{column}{0.5\textwidth}
      \minipage[c][0.75\textheight][s]{\columnwidth}
      \begin{itemize}
          \color{gray}
        \item A C function provided by the runtime (specific to native code)
        \item It scans the stack, between the stack pointer at the raising point up to the next trap, in order to collect the names of the detected function frames
        \item It uses the same technique and information as the Garbage Collector (GC) uses to scan the values live on the stack
      \end{itemize}
      \begin{itemize}
        \item For each function frame detected, store a pointer to the \typename{frame\_descr} in the \localname{domain\_state->backtrace\_buffer} array, effectively constructing the backtrace
      \end{itemize}
      \onslide<2->{
        \begin{itemize}
            \smallskip
          \item Constructed backtrace:
            \funcname{definitely\_raise},
            \onslide<3->{\funcname{probably\_raise}}
        \end{itemize}
      }
      \vfill
      \mintocaml[firstline=9,lastline=13,highlightlines=12]{../src/backtrace/backtrace.ml}
      \endminipage
    \end{column}
    \begin{column}{0.5\textwidth}
      \raggedleft
      \onslide*<1>{\listStashBacktrace[highlightlines={114-115}]{}}
      \onslide*<2>{\providecommand\step{3}\input{diagrams/backtrace/backtrace.tex}}%
      \onslide*<3>{\providecommand\step{4}\input{diagrams/backtrace/backtrace.tex}}%
    \end{column}
  \end{columns}
\end{frame}

\begin{frame}{Backtraces}{Back to \funcname{caml\_raise\_exn}}
  \begin{columns}[c]
    \begin{column}{0.5\textwidth}
      \minipage[c][0.66\textheight][s]{\columnwidth}
      \begin{itemize}\color{gray}
        \item Checks wheteher exceptions backtrace should be recorded, by checking the \funcarg{domain\_state->backtrace\_active} value
        \item Switches to the C stack to be able to execute C code
        \item Calls \funcname{caml\_stash\_backtrace}, to collect the backtrace up to the next trap
      \end{itemize}
      \onslide*<2->{
        \begin{itemize}
          \item<2-> Executes the exception handler in \funcname{probably\_raise} with \funcname{RESTORE\_EXN\_HANDLER\_OCAML}
            \begin{itemize}
              \item Set \regname{RSP} to \localname{domain\_state->exn\_handler} value
              \item Restore \localname{domain\_state->exn\_handler} from trap
              \item Jump to the handler code with \funcname{ret}
            \end{itemize}
        \end{itemize}
      }
      \vfill
      \onslide*<1>{\mintocaml[firstline=9,lastline=13,highlightlines=12]{../src/backtrace/backtrace.ml}}
      \onslide*<2->{\mintocaml[firstline=15,lastline=21,highlightlines={19-21}]{../src/backtrace/backtrace.ml}}
      \endminipage
    \end{column}
    \begin{column}{0.5\textwidth}
      \centering
      \onslide*<1>{
        \mintc[firstline=190,lastline=192]{../ocaml/runtime/amd64.S}
        \mintc[firstline=234,lastline=237]{../ocaml/runtime/amd64.S}
      }
      \onslide*<2>{
        \mintc[firstline=190,lastline=192,highlightlines={191-192}]{../ocaml/runtime/amd64.S}
        \mintc[firstline=234,lastline=237,highlightlines={235-237}]{../ocaml/runtime/amd64.S}
      }
      \onslide*<1>{\listCamlRaiseExn[highlightlines=720]{}}
      \onslide*<2>{\listCamlRaiseExn[highlightlines=722]{}}
      \onslide*<3->{\providecommand\step{5}\input{diagrams/backtrace/backtrace.tex}}
    \end{column}
  \end{columns}
\end{frame}

\begin{frame}{Backtraces}{\funcname{caml\_probably\_raise}'s handler doesn't catch \typename{ExnC}}
  \begin{columns}[c]
    \begin{column}{0.45\textwidth}
      \minipage[c][0.75\textheight][s]{\columnwidth}
      \begin{itemize}
        \item<1-> \funcname{match \dots with}\footnotemark pattern matching can also match exceptions
        \item<1-> The CMM of \funcname{probably\_raise} shows that this \funcname{match \dots with} is implemented with a pattern matching and a \funcname{try \dots with}
        \item<1-> But this one only matches on \typename{ExnA} exception
        \item<2-> Since the pattern matching fails to catch the exception, \funcname{reraise} is called with our current \typename{ExnC} exception
        \item<3-> Disassembly shows that \funcname{reraise} is actually a call to \funcname{caml\_reraise\_exn}, which is also an assembly function provided by the runtime
      \end{itemize}
      \vfill
      \mintocaml[firstline=15,lastline=21,highlightlines={18-21}]{../src/backtrace/backtrace.ml}
      \endminipage
    \end{column}
    \begin{column}{0.55\textwidth}
      \centering
      \onslide*<1>{\mintcmm[highlightlines={10-18}]{../src/backtrace/camlBacktrace\_\_probably\_raise\_351.cmm}}
      \onslide*<2>{\listcmm[highlightlines=18]{../src/backtrace/camlBacktrace\_\_probably\_raise_351.cmm}{CMM of probably\_raise}}
      \onslide*<3>{\mintobjdump[firstline=5,lastline=35,highlightlines=31]{../src/backtrace/camlBacktrace\_\_probably\_raise_351.objdump}}
    \end{column}
  \end{columns}
  \only<1>{\footnotetext[1]{https://blog.janestreet.com/pattern-matching-and-exception-handling-unite/}}
\end{frame}

\newcommand\listCamlReraiseExn[1][]{
  \listasm[firstline=727,lastline=734,#1]{../ocaml/runtime/amd64.S}{runtime/amd64.S:727}}

\begin{frame}{Backtraces}{Introducing \funcname{caml\_reraise\_exn}}
  \begin{columns}[c]
    \begin{column}{0.5\textwidth}
      \minipage[c][0.66\textheight][s]{\columnwidth}
      \begin{itemize}
        \item<1-> The exception have to be reraised to the next handler
        \item<2-> First, checks if the backtrace should be collected with \funcarg{domain\_state->backtrace\_active}
        \only<3>{
        \begin{itemize}
            \item If not, behaves like a \funcname{raise\_notrace} with \funcname{RESTORE\_EXN\_HANDLER\_OCAML}
        \end{itemize}
        }
        \item<4-> Jumps into \funcname{caml\_raise\_exn}
        \item<5-> Calls \funcname{caml\_stash\_backtrace} again, to scan the next part of the stack: from the trap just pop-ed to the next trap
      \end{itemize}
      \vfill
      \mintocaml[firstline=15,lastline=21,highlightlines={18-21}]{../src/backtrace/backtrace.ml}
      \endminipage
    \end{column}
    \begin{column}{0.5\textwidth}
      \raggedleft
      \onslide*<1>{\listCamlReraiseExn{}}
      \onslide*<2>{\listCamlReraiseExn[highlightlines=729]{}}
      \onslide*<3>{
        \mintc[firstline=190,lastline=192,highlightlines={191-192}]{../ocaml/runtime/amd64.S}
        \mintc[firstline=234,lastline=237,highlightlines={235-237}]{../ocaml/runtime/amd64.S}
        \medskip
        \listCamlReraiseExn[highlightlines=731]{}
      }
      \onslide*<4-5>{
        % TODO refactor with \listCamlReraiseExn
        \mintasm[firstline=727,lastline=734,highlightlines=730]{../ocaml/runtime/amd64.S}
        \medskip
      }
      \onslide*<4>{\listCamlRaiseExn[highlightlines=711]{}}
      \onslide*<5>{\listCamlRaiseExn[highlightlines=720]{}}
    \end{column}
  \end{columns}
\end{frame}

\begin{frame}{Backtraces}{Backtrace collection continues}
  \begin{columns}[c]
    \begin{column}{0.55\textwidth}
      \minipage[c][0.75\textheight][s]{\columnwidth}
      \begin{itemize}
        \item<1-> \funcname{caml\_stash\_backtrace} scans the older part of the stack, up to the next trap
        \item<6-> Controls jumps to the nested exception handler in \funcname{maybe\_raise}, which matches only on \typename{ExnB}
        \item<7-> \funcname{caml\_reraise\_exn} is called again, backtrace collection resumes with \funcname{caml\_stash\_backtrace}
      \end{itemize}
      \medskip
      \begin{itemize}
        \item<2-> Constructed backtrace:
        \begin{itemize}
          \item <2-> \funcname{definitely\_raise}
          \item <2-> \funcname{probably\_raise}
          \item <3-> \funcname{probably\_raise} (re-raise)
          \item <4-> \funcname{maybe\_raise}
          \item <5-> \funcname{main}
          \item <8-> \funcname{main} (re-raise)
        \end{itemize}
      \end{itemize}
      \vfill
      \onslide*<1-5>{\mintocaml[firstline=15,lastline=21]{../src/backtrace/backtrace.ml}}
      \onslide*<6>{\mintocaml[firstline=28,lastline=34,highlightlines=31]{../src/backtrace/backtrace.ml}}
      \onslide*<7->{\mintocaml[firstline=28,lastline=34,highlightlines=33]{../src/backtrace/backtrace.ml}}
      \endminipage
    \end{column}
    \begin{column}{0.45\textwidth}
      \raggedleft
      \onslide*<1>{\providecommand\step{6}\input{diagrams/backtrace/backtrace.tex}}%
      \onslide*<2>{\providecommand\step{6}\input{diagrams/backtrace/backtrace.tex}}%
      \onslide*<3>{\providecommand\step{7}\input{diagrams/backtrace/backtrace.tex}}%
      \onslide*<4>{\providecommand\step{8}\input{diagrams/backtrace/backtrace.tex}}%
      \onslide*<5>{\providecommand\step{9}\input{diagrams/backtrace/backtrace.tex}}%
      \onslide*<6>{\providecommand\step{10}\input{diagrams/backtrace/backtrace.tex}}%
      \onslide*<7>{\providecommand\step{11}\input{diagrams/backtrace/backtrace.tex}}%
      \onslide*<8>{\providecommand\step{12}\input{diagrams/backtrace/backtrace.tex}}%
      \onslide*<9>{\providecommand\step{13}\input{diagrams/backtrace/backtrace.tex}}%
    \end{column}
  \end{columns}
\end{frame}

\begin{frame}{Backtraces}{Printing the backtrace}
  \begin{columns}[c]
    \begin{column}{0.45\textwidth}
      \minipage[c][0.66\textheight][s]{\columnwidth}
      \begin{itemize}
        \item<1-> The exception backtrace can now be printed to \funcarg{stdout} with \funcname{Printexc.print_backtrace}
        \item<2-> First the exception backtrace is retrieved into an OCaml array with \funcname{get_raw_backtrace }
        \item<3-> Which is implemented as the \funcname{caml_get_exception_raw_backtrace} C function in the runtime
        \item<4-> And finally printed with \funcname{print_exception_backtrace}
      \end{itemize}
      \vfill
      \mintocaml[firstline=28,lastline=34,highlightlines=33]{../src/backtrace/backtrace.ml}
      \endminipage
    \end{column}
    \begin{column}{0.55\textwidth}
      \onslide*<1,2>{
        \mintocaml[firstline=100,lastline=101]{../ocaml/stdlib/printexc.ml}
      }
      \only<1>{
        \listocaml[firstline=170,lastline=172]{../ocaml/stdlib/printexc.ml}{stdlib/printexc.ml:170}
      }
      \only<2>{
        \listocaml[firstline=170,lastline=172,highlightlines=172]{../ocaml/stdlib/printexc.ml}{stdlib/printexc.ml:170}
      }
      \only<3>{
        \mintc[firstline=154,lastline=156]{../ocaml/runtime/backtrace.c}
        \listc[firstline=171,lastline=192,highlightlines={184-187}]{../ocaml/runtime/backtrace.c}{runtime/backtrace.c:154}
      }
      \only<4>{
        \listocaml[firstline=155,lastline=165]{../ocaml/stdlib/printexc.ml}{stdlib/printexc.ml:55}
      }
    \end{column}
  \end{columns}
\end{frame}


\subsection{The multiple kinds of raise}
\frameSubsection{}{}

\begin{frame}{Backtraces}{When backtraces are not needed}
  \begin{columns}[c]
    \begin{column}{0.45\textwidth}
      \minipage[c][0.66\textheight][s]{\columnwidth}
      \begin{itemize}
        \item<1-> What if the backtrace for an exception is never needed?
        \item<2-> It's possible to raise the exception explicitly with \funcname{raise\_notrace} to make raising as cheap as possible
        \item<3-> An example of use in the \funcname{dynlink} library of the OCaml compiler
      \end{itemize}
      \vfill
      \onslide<3->{
      \listocaml[firstline=205,lastline=209,highlightlines=207]{../ocaml/otherlibs/dynlink/dynlink\_compilerlibs/misc.ml}{otherlibs/dynlink/dynlink\_compilerlibs/misc.ml:205}
      }
      \endminipage
    \end{column}
    \begin{column}{0.55\textwidth}
    \only<4>{
      \mintcmm[highlightlines=3]{../src/notrace/camlNotrace\_\_fun_347.cmm}
      \listcmm{../src/notrace/camlNotrace\_\_all\_somes\_327.cmm}{all\_somes}
    }
    \onslide*<5->{
      \listobjdump[highlightlines={6-9}]{../src/notrace/camlNotrace\_\_fun_347.objdump}{anonymous function from \funcname{all\_somes}}
    }
    \end{column}
  \end{columns}
\end{frame}

\begin{frame}{Backtraces}{Summary table of the multiple kinds of raise}
  \begin{itemize}
    \item When debugging information is enabled and if the backtrace of an exception isn't needed, it's possible to raise with \funcname{raise\_notrace} for performance
    \item When debugging information is \emph{not} enabled, the compiler optimises \funcname{raise} calls to inline assembly (same effect as using \funcname{raise\_notrace})
  \end{itemize}
  \bigskip
  \centering
  \begin{tabular}{| l | c | c | c | c |}
    \hline
    \parbox{10em}{Debug information \\ (\funcarg{-g} flag)}  & \multicolumn{2}{|c|}{Disabled}                                     & \multicolumn{2}{|c|}{Enabled} \\
    \hline
    Raise kind                                               & \funcname{raise} & \funcname{raise\_notrace}                       & \funcname{raise} & \funcname{raise\_notrace} \\
    \hline
    Compilation                                              & \parbox{8em}{inline assembly\\ (optimization)} & inline assembly   & \funcname{caml\_raise\_exn} & inline assembly \\
    \hline
  \end{tabular}
\end{frame}


%
% Takeaway #5
%
\subsection*{Takeaway \#5}
\frameSubsectionTakeaway{}
\againframe<5>{takeaway}
}
    \end{column}
  \end{columns}
\end{frame}

\begin{frame}{Backtraces}{\funcname{caml\_probably\_raise}'s handler doesn't catch \typename{ExnC}}
  \begin{columns}[c]
    \begin{column}{0.45\textwidth}
      \minipage[c][0.75\textheight][s]{\columnwidth}
      \begin{itemize}
        \item<1-> \funcname{match \dots with}\footnotemark pattern matching can also match exceptions
        \item<1-> The CMM of \funcname{probably\_raise} shows that this \funcname{match \dots with} is implemented with a pattern matching and a \funcname{try \dots with}
        \item<1-> But this one only matches on \typename{ExnA} exception
        \item<2-> Since the pattern matching fails to catch the exception, \funcname{reraise} is called with our current \typename{ExnC} exception
        \item<3-> Disassembly shows that \funcname{reraise} is actually a call to \funcname{caml\_reraise\_exn}, which is also an assembly function provided by the runtime
      \end{itemize}
      \vfill
      \mintocaml[firstline=15,lastline=21,highlightlines={18-21}]{../src/backtrace/backtrace.ml}
      \endminipage
    \end{column}
    \begin{column}{0.55\textwidth}
      \centering
      \onslide*<1>{\mintcmm[highlightlines={10-18}]{../src/backtrace/camlBacktrace\_\_probably\_raise\_351.cmm}}
      \onslide*<2>{\listcmm[highlightlines=18]{../src/backtrace/camlBacktrace\_\_probably\_raise_351.cmm}{CMM of probably\_raise}}
      \onslide*<3>{\mintobjdump[firstline=5,lastline=35,highlightlines=31]{../src/backtrace/camlBacktrace\_\_probably\_raise_351.objdump}}
    \end{column}
  \end{columns}
  \only<1>{\footnotetext[1]{https://blog.janestreet.com/pattern-matching-and-exception-handling-unite/}}
\end{frame}

\newcommand\listCamlReraiseExn[1][]{
  \listasm[firstline=727,lastline=734,#1]{../ocaml/runtime/amd64.S}{runtime/amd64.S:727}}

\begin{frame}{Backtraces}{Introducing \funcname{caml\_reraise\_exn}}
  \begin{columns}[c]
    \begin{column}{0.5\textwidth}
      \minipage[c][0.66\textheight][s]{\columnwidth}
      \begin{itemize}
        \item<1-> The exception have to be reraised to the next handler
        \item<2-> First, checks if the backtrace should be collected with \funcarg{domain\_state->backtrace\_active}
        \only<3>{
        \begin{itemize}
            \item If not, behaves like a \funcname{raise\_notrace} with \funcname{RESTORE\_EXN\_HANDLER\_OCAML}
        \end{itemize}
        }
        \item<4-> Jumps into \funcname{caml\_raise\_exn}
        \item<5-> Calls \funcname{caml\_stash\_backtrace} again, to scan the next part of the stack: from the trap just pop-ed to the next trap
      \end{itemize}
      \vfill
      \mintocaml[firstline=15,lastline=21,highlightlines={18-21}]{../src/backtrace/backtrace.ml}
      \endminipage
    \end{column}
    \begin{column}{0.5\textwidth}
      \raggedleft
      \onslide*<1>{\listCamlReraiseExn{}}
      \onslide*<2>{\listCamlReraiseExn[highlightlines=729]{}}
      \onslide*<3>{
        \mintc[firstline=190,lastline=192,highlightlines={191-192}]{../ocaml/runtime/amd64.S}
        \mintc[firstline=234,lastline=237,highlightlines={235-237}]{../ocaml/runtime/amd64.S}
        \medskip
        \listCamlReraiseExn[highlightlines=731]{}
      }
      \onslide*<4-5>{
        % TODO refactor with \listCamlReraiseExn
        \mintasm[firstline=727,lastline=734,highlightlines=730]{../ocaml/runtime/amd64.S}
        \medskip
      }
      \onslide*<4>{\listCamlRaiseExn[highlightlines=711]{}}
      \onslide*<5>{\listCamlRaiseExn[highlightlines=720]{}}
    \end{column}
  \end{columns}
\end{frame}

\begin{frame}{Backtraces}{Backtrace collection continues}
  \begin{columns}[c]
    \begin{column}{0.55\textwidth}
      \minipage[c][0.75\textheight][s]{\columnwidth}
      \begin{itemize}
        \item<1-> \funcname{caml\_stash\_backtrace} scans the older part of the stack, up to the next trap
        \item<6-> Controls jumps to the nested exception handler in \funcname{maybe\_raise}, which matches only on \typename{ExnB}
        \item<7-> \funcname{caml\_reraise\_exn} is called again, backtrace collection resumes with \funcname{caml\_stash\_backtrace}
      \end{itemize}
      \medskip
      \begin{itemize}
        \item<2-> Constructed backtrace:
        \begin{itemize}
          \item <2-> \funcname{definitely\_raise}
          \item <2-> \funcname{probably\_raise}
          \item <3-> \funcname{probably\_raise} (re-raise)
          \item <4-> \funcname{maybe\_raise}
          \item <5-> \funcname{main}
          \item <8-> \funcname{main} (re-raise)
        \end{itemize}
      \end{itemize}
      \vfill
      \onslide*<1-5>{\mintocaml[firstline=15,lastline=21]{../src/backtrace/backtrace.ml}}
      \onslide*<6>{\mintocaml[firstline=28,lastline=34,highlightlines=31]{../src/backtrace/backtrace.ml}}
      \onslide*<7->{\mintocaml[firstline=28,lastline=34,highlightlines=33]{../src/backtrace/backtrace.ml}}
      \endminipage
    \end{column}
    \begin{column}{0.45\textwidth}
      \raggedleft
      \onslide*<1>{\providecommand\step{6}%
% Backtraces
%
\subsection{One advantage of raising with debugging information}
\frameSubsection{}{}

\begin{frame}{Backtraces}{Debugging information \& source code}
  \begin{columns}[c]
    \begin{column}{0.5\textwidth}
      \begin{itemize}
        \item Raising an exception is fun and games, but how do we know where the exception originated? $\rightarrow$ With the backtrace associated with the exception!
        \item In order to get a backtrace from the point where an exception was raised, it's necessary to compile with debugging information enabled (\funcarg{-g} flag for the compiler)
        \item Same example as in the `Nested handlers' section
      \end{itemize}
    \end{column}
    \begin{column}{0.5\textwidth}
      \listocaml[firstline=9,lastline=34]{../src/backtrace/backtrace.ml}{backtrace/backtrace.ml}
    \end{column}
  \end{columns}
\end{frame}

\begin{frame}{Backtraces}{CMM and disassembly of \funcname{definitely\_raise}}
  \begin{columns}[c]
    \begin{column}{0.5\textwidth}
      \minipage[c][0.66\textheight][s]{\columnwidth}
      \begin{itemize}
        \item<1-> An effect of compiling with debugging information (\funcarg{-g}): the CMM no longer uses \funcname{raise\_notrace} to raise the exception but \funcname{raise} instead
        \item<2-> Disassembly shows that CMM \funcname{raise} is actually a call to \funcname{caml\_raise\_exn}, a function in assembler provided by the runtime
      \end{itemize}
      \vfill
      \mintocaml[firstline=9,lastline=13,highlightlines=12]{../src/backtrace/backtrace.ml}
      \endminipage
    \end{column}
    \begin{column}{0.5\textwidth}
      \onslide*<1>{
        \mintcmm[highlightlines=5]{../src/backtrace/camlBacktrace\_\_definitely\_raise\_347.cmm}
      }
      \onslide*<2>{
        \mintobjdump[firstline=2,highlightlines=14]{../src/backtrace/camlBacktrace\_\_definitely\_raise\_347.objdump}
      }
    \end{column}
  \end{columns}
\end{frame}

\begin{frame}{Backtraces}{Introducing \funcname{caml\_raise\_exn}}
  \begin{columns}[c]
    \begin{column}{0.5\textwidth}
      \minipage[c][0.66\textheight][s]{\columnwidth}
      \onslide*<1>{
        \begin{itemize}
          \item Raising the exception is performed using \funcname{caml\_raise\_exn} which is
            \begin{itemize}
              \item An assembly function provided by the runtime
              \item Architecture specific
            \end{itemize}
          \item Let's see what it does differently from \funcname{raise\_notrace}\dots
          \item We are about to raise \typename{ExnC} with \funcname{caml\_raise\_exn} from \funcname{definitely\_raise}
        \end{itemize}
      }
      \onslide*<2>{
        \mintobjdump[firstline=2,highlightlines=14]{../src/backtrace/camlBacktrace\_\_definitely\_raise\_347.objdump}
      }
      \vfill
      \mintocaml[firstline=9,lastline=13,highlightlines=12]{../src/backtrace/backtrace.ml}
      \endminipage
    \end{column}
    \begin{column}{0.5\textwidth}
      \centering
      \providecommand\step{1}\input{diagrams/backtrace/backtrace.tex}
    \end{column}
  \end{columns}
\end{frame}

\begin{frame}{Backtraces}{But before that, we need more fields from \typename{caml\_domain\_state}}
  \begin{columns}
    \begin{column}{0.5\textwidth}
      \begin{itemize}
        \item Remember \typename{caml\_domain\_state} the per-domain runtime state?
        \item Some other members are of interest to us now\dots
        \item \localname{backtrace\_active} is used to know if the runtime should collect backtraces
        \item \localname{backtrace\_active} can be set by
          \begin{itemize}
            \item The \funcarg{b} flag in the \funcname{OCAMLRUNPARAM} environment variable
            \item \mintinline{ocaml}{Printexc.record_backtrace : bool -> unit} \footnotemark
          \end{itemize}
      \end{itemize}
    \end{column}
    \begin{column}{0.5\textwidth}
      \begin{listing}
        \mintc[highlightlines={9-11}]{src/ocaml/domain\_state\_backtrace.h}
        \caption{runtime/caml/domain\_state.h}
      \end{listing}
    \end{column}
  \end{columns}
  \footnotetext[1]{https://v2.ocaml.org/api/Printexc.html}
\end{frame}

\newcommand\listCamlRaiseExn[1][]{
  \listasm[firstline=702,lastline=725,#1]{../ocaml/runtime/amd64.S}{runtime/amd64.S:702}}

\begin{frame}{Backtraces}{Walk through \funcname{caml\_raise\_exn}}
  \begin{columns}[T]
    \begin{column}{0.5\textwidth}
      \begin{itemize}
        \item<1-> First checks whether exceptions backtrace should be recorded
        \item<1-> By checking the \funcarg{domain\_state->backtrace\_active} value
        \item<2-> If not, acts as \funcname{raise\_notrace} with \funcname{RESTORE\_EXN\_HANDLER\_OCAML}
          \begin{itemize}
            \item Set \regname{RSP} to \localname{domain\_state->exn\_handler} value
            \item Restore \localname{domain\_state->exn\_handler} from trap
            \item Jump with \funcname{ret} (same effect as using \funcname{pop} and \funcname{jmp})
          \end{itemize}
        \item<3-> Otherwise, switches to the C stack to be able to execute C code
        \item<4-> Calls \funcname{caml\_stash\_backtrace}, a C function provided by the runtime\ignorespaces
          \onslide<6->{, with
            \begin{itemize}
              \item \funcarg{exn}: exception value
              \item \funcarg{pc}: code address of the raising point
              \item \funcarg{sp}: stack address of the raising function
              \item \funcarg{trapsp}: stack address of the next handler
            \end{itemize}
          }
      \end{itemize}
      \bigskip
      \mintocaml[firstline=9,lastline=13,highlightlines=12]{../src/backtrace/backtrace.ml}
    \end{column}
    \begin{column}{0.5\textwidth}
      \raggedleft
      \onslide*<1,3-4>{
        \mintc[firstline=190,lastline=192]{../ocaml/runtime/amd64.S}
        \mintc[firstline=234,lastline=237]{../ocaml/runtime/amd64.S}
      }
      \onslide*<2>{
        \mintc[firstline=190,lastline=192,highlightlines={191-192}]{../ocaml/runtime/amd64.S}
        \mintc[firstline=234,lastline=237,highlightlines={235-237}]{../ocaml/runtime/amd64.S}
      }
      \onslide*<1>{\listCamlRaiseExn[highlightlines=705]{}}%
      \onslide*<2>{\listCamlRaiseExn[highlightlines={707-708}]{}}%
      \onslide*<3>{\listCamlRaiseExn[highlightlines={712-714}]{}}%
      \onslide*<4>{\listCamlRaiseExn[highlightlines={715-720}]{}}%
      \onslide*<5>{\providecommand\step{1}\input{diagrams/backtrace/backtrace.tex}}%
      \onslide*<6>{\providecommand\step{2}\input{diagrams/backtrace/backtrace.tex}}%
    \end{column}
  \end{columns}
\end{frame}

\newcommand\listStashBacktrace[1][]{
  \listc[firstline=91,lastline=120,#1]{../ocaml/runtime/backtrace\_nat.c}{runtime/backtrace\_nat.c:91}}

\begin{frame}{Backtraces}{Introducing \funcname{caml\_stash\_backtrace}}
  \begin{columns}[c]
    \begin{column}{0.5\textwidth}
      \minipage[c][0.66\textheight][s]{\columnwidth}
      \begin{itemize}
        \item<1-> A C function provided by the runtime (specific to native code)
        \item<2-> It scans the stack, between the stack pointer at the raising point up to the next trap, in order to collect the names of the detected function frames
          \onslide*<2>{
            \begin{itemize}
              \item And more information, like line number, column number...
            \end{itemize}
          }
        \item<3-> It uses the same technique and information as the Garbage Collector (GC) uses to scan the values live on the stack
          \onslide*<4>{
            \begin{itemize}
              \item \typename{frame\_descr} are at the core of stack unwinding
            \end{itemize}
          }
      \end{itemize}
      \vfill
      \mintocaml[firstline=9,lastline=13,highlightlines=12]{../src/backtrace/backtrace.ml}
      \endminipage
    \end{column}
    \begin{column}{0.5\textwidth}
      \onslide*<1>{\listStashBacktrace[highlightlines=91]{}}
      \onslide*<2>{\listStashBacktrace[highlightlines={91,117-118}]{}}
      \onslide*<3->{\listStashBacktrace[highlightlines={94,106,109-110}]{}}
    \end{column}
  \end{columns}
\end{frame}

\newcommand\listFrameDescr[1][]{
  \listocaml[firstline=111,lastline=124,#1]{../ocaml/asmcomp/emitaux.ml}{asmcomp/emitaux.ml:111}}
\newcommand\listDebugInfo[1][]{
  \listocaml[firstline=38,lastline=49,#1]{../ocaml/lambda/debuginfo.mli}{lambda/debuginfo.mli:38}}

\begin{frame}{Backtraces}{\typename{frame\_descr} and \typename{frame\_debuginfo} in a nutshell}
  \begin{columns}[c]
    \begin{column}{0.5\textwidth}
      \begin{itemize}
        \item<1-> For every return address in an OCaml program, there is a associated \typename{frame\_descr}
        \item<2-> A \typename{frame\_descr} specifies
        \begin{itemize}
          \item<2-> The size of the function stack frame
          \item<3-> Where live values are on the stack (so that the GC can mark them as live and prevent from being swept)
        \end{itemize}
        \item<4-> When compiled with debug information, \typename{frame\_descr} will be linked to a \typename{frame\_debuginfo}
        \begin{itemize}
          \item<5-> Contains information about the position in the source code (file name, line and column number)
        \end{itemize}
      \end{itemize}
    \end{column}
    \begin{column}{0.5\textwidth}
        \onslide*<1>{\listFrameDescr[highlightlines={118-122}]{}}
        \onslide*<2>{\listFrameDescr[highlightlines={120}]{}}
        \onslide*<3>{\listFrameDescr[highlightlines={121}]{}}
        \onslide*<4->{\listFrameDescr[highlightlines={122}]{}}
        \medskip
        \onslide*<1-3>{\listDebugInfo{}}
        \onslide*<4>{\listDebugInfo[highlightlines={38,49}]{}}
        \onslide*<5>{\listDebugInfo[highlightlines={39-46}]{}}
    \end{column}
  \end{columns}
\end{frame}

\begin{frame}{Backtraces}{Scanning the stack with \typename{frame\_descr}}
  \begin{columns}[c]
    \begin{column}{0.5\textwidth}
      \begin{itemize}
        \item Given the return address of the code that raises the exception by calling \funcname{caml\_raise\_exn} (\funcarg{pc} argument of \funcname{caml\_stash\_backtrace})
        \item And the position of the stack pointer (\funcarg{sp} argument of \funcname{caml\_stash\_backtrace})
        \item The frame size given by \typename{frame\_descr}.\funcarg{frame\_size}, allows to find the end of the previous frame
        \item And the return address of the caller of this function
        \bigskip
        \onslide<3->{
        \item Note that traps (here the reddish \funcname{ExnA}, \funcname{ExnB} and \funcname{ExnC} blocks) belong to the frame that installed them, and so the frame size changes when traps are removed
        }
      \end{itemize}
    \end{column}
    \begin{column}{0.5\textwidth}
      \raggedleft
      \onslide*<1>{\providecommand\step{2}\input{diagrams/backtrace/backtrace.tex}}%
      \onslide*<2>{\providecommand\step{1}\input{diagrams/backtrace/scanstack.tex}}%
      \onslide*<3>{\providecommand\step{2}\input{diagrams/backtrace/scanstack.tex}}%
      \onslide*<4>{\providecommand\step{3}\input{diagrams/backtrace/scanstack.tex}}%
      \onslide*<5>{\providecommand\step{4}\input{diagrams/backtrace/scanstack.tex}}%
      \onslide*<6>{\providecommand\step{5}\input{diagrams/backtrace/scanstack.tex}}%
    \end{column}
  \end{columns}
\end{frame}

% Duplicate of previous-previous slide
\begin{frame}{Backtraces}{Continuing on \funcname{caml\_stash\_backtrace}}
  \begin{columns}[c]
    \begin{column}{0.5\textwidth}
      \minipage[c][0.75\textheight][s]{\columnwidth}
      \begin{itemize}
          \color{gray}
        \item A C function provided by the runtime (specific to native code)
        \item It scans the stack, between the stack pointer at the raising point up to the next trap, in order to collect the names of the detected function frames
        \item It uses the same technique and information as the Garbage Collector (GC) uses to scan the values live on the stack
      \end{itemize}
      \begin{itemize}
        \item For each function frame detected, store a pointer to the \typename{frame\_descr} in the \localname{domain\_state->backtrace\_buffer} array, effectively constructing the backtrace
      \end{itemize}
      \onslide<2->{
        \begin{itemize}
            \smallskip
          \item Constructed backtrace:
            \funcname{definitely\_raise},
            \onslide<3->{\funcname{probably\_raise}}
        \end{itemize}
      }
      \vfill
      \mintocaml[firstline=9,lastline=13,highlightlines=12]{../src/backtrace/backtrace.ml}
      \endminipage
    \end{column}
    \begin{column}{0.5\textwidth}
      \raggedleft
      \onslide*<1>{\listStashBacktrace[highlightlines={114-115}]{}}
      \onslide*<2>{\providecommand\step{3}\input{diagrams/backtrace/backtrace.tex}}%
      \onslide*<3>{\providecommand\step{4}\input{diagrams/backtrace/backtrace.tex}}%
    \end{column}
  \end{columns}
\end{frame}

\begin{frame}{Backtraces}{Back to \funcname{caml\_raise\_exn}}
  \begin{columns}[c]
    \begin{column}{0.5\textwidth}
      \minipage[c][0.66\textheight][s]{\columnwidth}
      \begin{itemize}\color{gray}
        \item Checks wheteher exceptions backtrace should be recorded, by checking the \funcarg{domain\_state->backtrace\_active} value
        \item Switches to the C stack to be able to execute C code
        \item Calls \funcname{caml\_stash\_backtrace}, to collect the backtrace up to the next trap
      \end{itemize}
      \onslide*<2->{
        \begin{itemize}
          \item<2-> Executes the exception handler in \funcname{probably\_raise} with \funcname{RESTORE\_EXN\_HANDLER\_OCAML}
            \begin{itemize}
              \item Set \regname{RSP} to \localname{domain\_state->exn\_handler} value
              \item Restore \localname{domain\_state->exn\_handler} from trap
              \item Jump to the handler code with \funcname{ret}
            \end{itemize}
        \end{itemize}
      }
      \vfill
      \onslide*<1>{\mintocaml[firstline=9,lastline=13,highlightlines=12]{../src/backtrace/backtrace.ml}}
      \onslide*<2->{\mintocaml[firstline=15,lastline=21,highlightlines={19-21}]{../src/backtrace/backtrace.ml}}
      \endminipage
    \end{column}
    \begin{column}{0.5\textwidth}
      \centering
      \onslide*<1>{
        \mintc[firstline=190,lastline=192]{../ocaml/runtime/amd64.S}
        \mintc[firstline=234,lastline=237]{../ocaml/runtime/amd64.S}
      }
      \onslide*<2>{
        \mintc[firstline=190,lastline=192,highlightlines={191-192}]{../ocaml/runtime/amd64.S}
        \mintc[firstline=234,lastline=237,highlightlines={235-237}]{../ocaml/runtime/amd64.S}
      }
      \onslide*<1>{\listCamlRaiseExn[highlightlines=720]{}}
      \onslide*<2>{\listCamlRaiseExn[highlightlines=722]{}}
      \onslide*<3->{\providecommand\step{5}\input{diagrams/backtrace/backtrace.tex}}
    \end{column}
  \end{columns}
\end{frame}

\begin{frame}{Backtraces}{\funcname{caml\_probably\_raise}'s handler doesn't catch \typename{ExnC}}
  \begin{columns}[c]
    \begin{column}{0.45\textwidth}
      \minipage[c][0.75\textheight][s]{\columnwidth}
      \begin{itemize}
        \item<1-> \funcname{match \dots with}\footnotemark pattern matching can also match exceptions
        \item<1-> The CMM of \funcname{probably\_raise} shows that this \funcname{match \dots with} is implemented with a pattern matching and a \funcname{try \dots with}
        \item<1-> But this one only matches on \typename{ExnA} exception
        \item<2-> Since the pattern matching fails to catch the exception, \funcname{reraise} is called with our current \typename{ExnC} exception
        \item<3-> Disassembly shows that \funcname{reraise} is actually a call to \funcname{caml\_reraise\_exn}, which is also an assembly function provided by the runtime
      \end{itemize}
      \vfill
      \mintocaml[firstline=15,lastline=21,highlightlines={18-21}]{../src/backtrace/backtrace.ml}
      \endminipage
    \end{column}
    \begin{column}{0.55\textwidth}
      \centering
      \onslide*<1>{\mintcmm[highlightlines={10-18}]{../src/backtrace/camlBacktrace\_\_probably\_raise\_351.cmm}}
      \onslide*<2>{\listcmm[highlightlines=18]{../src/backtrace/camlBacktrace\_\_probably\_raise_351.cmm}{CMM of probably\_raise}}
      \onslide*<3>{\mintobjdump[firstline=5,lastline=35,highlightlines=31]{../src/backtrace/camlBacktrace\_\_probably\_raise_351.objdump}}
    \end{column}
  \end{columns}
  \only<1>{\footnotetext[1]{https://blog.janestreet.com/pattern-matching-and-exception-handling-unite/}}
\end{frame}

\newcommand\listCamlReraiseExn[1][]{
  \listasm[firstline=727,lastline=734,#1]{../ocaml/runtime/amd64.S}{runtime/amd64.S:727}}

\begin{frame}{Backtraces}{Introducing \funcname{caml\_reraise\_exn}}
  \begin{columns}[c]
    \begin{column}{0.5\textwidth}
      \minipage[c][0.66\textheight][s]{\columnwidth}
      \begin{itemize}
        \item<1-> The exception have to be reraised to the next handler
        \item<2-> First, checks if the backtrace should be collected with \funcarg{domain\_state->backtrace\_active}
        \only<3>{
        \begin{itemize}
            \item If not, behaves like a \funcname{raise\_notrace} with \funcname{RESTORE\_EXN\_HANDLER\_OCAML}
        \end{itemize}
        }
        \item<4-> Jumps into \funcname{caml\_raise\_exn}
        \item<5-> Calls \funcname{caml\_stash\_backtrace} again, to scan the next part of the stack: from the trap just pop-ed to the next trap
      \end{itemize}
      \vfill
      \mintocaml[firstline=15,lastline=21,highlightlines={18-21}]{../src/backtrace/backtrace.ml}
      \endminipage
    \end{column}
    \begin{column}{0.5\textwidth}
      \raggedleft
      \onslide*<1>{\listCamlReraiseExn{}}
      \onslide*<2>{\listCamlReraiseExn[highlightlines=729]{}}
      \onslide*<3>{
        \mintc[firstline=190,lastline=192,highlightlines={191-192}]{../ocaml/runtime/amd64.S}
        \mintc[firstline=234,lastline=237,highlightlines={235-237}]{../ocaml/runtime/amd64.S}
        \medskip
        \listCamlReraiseExn[highlightlines=731]{}
      }
      \onslide*<4-5>{
        % TODO refactor with \listCamlReraiseExn
        \mintasm[firstline=727,lastline=734,highlightlines=730]{../ocaml/runtime/amd64.S}
        \medskip
      }
      \onslide*<4>{\listCamlRaiseExn[highlightlines=711]{}}
      \onslide*<5>{\listCamlRaiseExn[highlightlines=720]{}}
    \end{column}
  \end{columns}
\end{frame}

\begin{frame}{Backtraces}{Backtrace collection continues}
  \begin{columns}[c]
    \begin{column}{0.55\textwidth}
      \minipage[c][0.75\textheight][s]{\columnwidth}
      \begin{itemize}
        \item<1-> \funcname{caml\_stash\_backtrace} scans the older part of the stack, up to the next trap
        \item<6-> Controls jumps to the nested exception handler in \funcname{maybe\_raise}, which matches only on \typename{ExnB}
        \item<7-> \funcname{caml\_reraise\_exn} is called again, backtrace collection resumes with \funcname{caml\_stash\_backtrace}
      \end{itemize}
      \medskip
      \begin{itemize}
        \item<2-> Constructed backtrace:
        \begin{itemize}
          \item <2-> \funcname{definitely\_raise}
          \item <2-> \funcname{probably\_raise}
          \item <3-> \funcname{probably\_raise} (re-raise)
          \item <4-> \funcname{maybe\_raise}
          \item <5-> \funcname{main}
          \item <8-> \funcname{main} (re-raise)
        \end{itemize}
      \end{itemize}
      \vfill
      \onslide*<1-5>{\mintocaml[firstline=15,lastline=21]{../src/backtrace/backtrace.ml}}
      \onslide*<6>{\mintocaml[firstline=28,lastline=34,highlightlines=31]{../src/backtrace/backtrace.ml}}
      \onslide*<7->{\mintocaml[firstline=28,lastline=34,highlightlines=33]{../src/backtrace/backtrace.ml}}
      \endminipage
    \end{column}
    \begin{column}{0.45\textwidth}
      \raggedleft
      \onslide*<1>{\providecommand\step{6}\input{diagrams/backtrace/backtrace.tex}}%
      \onslide*<2>{\providecommand\step{6}\input{diagrams/backtrace/backtrace.tex}}%
      \onslide*<3>{\providecommand\step{7}\input{diagrams/backtrace/backtrace.tex}}%
      \onslide*<4>{\providecommand\step{8}\input{diagrams/backtrace/backtrace.tex}}%
      \onslide*<5>{\providecommand\step{9}\input{diagrams/backtrace/backtrace.tex}}%
      \onslide*<6>{\providecommand\step{10}\input{diagrams/backtrace/backtrace.tex}}%
      \onslide*<7>{\providecommand\step{11}\input{diagrams/backtrace/backtrace.tex}}%
      \onslide*<8>{\providecommand\step{12}\input{diagrams/backtrace/backtrace.tex}}%
      \onslide*<9>{\providecommand\step{13}\input{diagrams/backtrace/backtrace.tex}}%
    \end{column}
  \end{columns}
\end{frame}

\begin{frame}{Backtraces}{Printing the backtrace}
  \begin{columns}[c]
    \begin{column}{0.45\textwidth}
      \minipage[c][0.66\textheight][s]{\columnwidth}
      \begin{itemize}
        \item<1-> The exception backtrace can now be printed to \funcarg{stdout} with \funcname{Printexc.print_backtrace}
        \item<2-> First the exception backtrace is retrieved into an OCaml array with \funcname{get_raw_backtrace }
        \item<3-> Which is implemented as the \funcname{caml_get_exception_raw_backtrace} C function in the runtime
        \item<4-> And finally printed with \funcname{print_exception_backtrace}
      \end{itemize}
      \vfill
      \mintocaml[firstline=28,lastline=34,highlightlines=33]{../src/backtrace/backtrace.ml}
      \endminipage
    \end{column}
    \begin{column}{0.55\textwidth}
      \onslide*<1,2>{
        \mintocaml[firstline=100,lastline=101]{../ocaml/stdlib/printexc.ml}
      }
      \only<1>{
        \listocaml[firstline=170,lastline=172]{../ocaml/stdlib/printexc.ml}{stdlib/printexc.ml:170}
      }
      \only<2>{
        \listocaml[firstline=170,lastline=172,highlightlines=172]{../ocaml/stdlib/printexc.ml}{stdlib/printexc.ml:170}
      }
      \only<3>{
        \mintc[firstline=154,lastline=156]{../ocaml/runtime/backtrace.c}
        \listc[firstline=171,lastline=192,highlightlines={184-187}]{../ocaml/runtime/backtrace.c}{runtime/backtrace.c:154}
      }
      \only<4>{
        \listocaml[firstline=155,lastline=165]{../ocaml/stdlib/printexc.ml}{stdlib/printexc.ml:55}
      }
    \end{column}
  \end{columns}
\end{frame}


\subsection{The multiple kinds of raise}
\frameSubsection{}{}

\begin{frame}{Backtraces}{When backtraces are not needed}
  \begin{columns}[c]
    \begin{column}{0.45\textwidth}
      \minipage[c][0.66\textheight][s]{\columnwidth}
      \begin{itemize}
        \item<1-> What if the backtrace for an exception is never needed?
        \item<2-> It's possible to raise the exception explicitly with \funcname{raise\_notrace} to make raising as cheap as possible
        \item<3-> An example of use in the \funcname{dynlink} library of the OCaml compiler
      \end{itemize}
      \vfill
      \onslide<3->{
      \listocaml[firstline=205,lastline=209,highlightlines=207]{../ocaml/otherlibs/dynlink/dynlink\_compilerlibs/misc.ml}{otherlibs/dynlink/dynlink\_compilerlibs/misc.ml:205}
      }
      \endminipage
    \end{column}
    \begin{column}{0.55\textwidth}
    \only<4>{
      \mintcmm[highlightlines=3]{../src/notrace/camlNotrace\_\_fun_347.cmm}
      \listcmm{../src/notrace/camlNotrace\_\_all\_somes\_327.cmm}{all\_somes}
    }
    \onslide*<5->{
      \listobjdump[highlightlines={6-9}]{../src/notrace/camlNotrace\_\_fun_347.objdump}{anonymous function from \funcname{all\_somes}}
    }
    \end{column}
  \end{columns}
\end{frame}

\begin{frame}{Backtraces}{Summary table of the multiple kinds of raise}
  \begin{itemize}
    \item When debugging information is enabled and if the backtrace of an exception isn't needed, it's possible to raise with \funcname{raise\_notrace} for performance
    \item When debugging information is \emph{not} enabled, the compiler optimises \funcname{raise} calls to inline assembly (same effect as using \funcname{raise\_notrace})
  \end{itemize}
  \bigskip
  \centering
  \begin{tabular}{| l | c | c | c | c |}
    \hline
    \parbox{10em}{Debug information \\ (\funcarg{-g} flag)}  & \multicolumn{2}{|c|}{Disabled}                                     & \multicolumn{2}{|c|}{Enabled} \\
    \hline
    Raise kind                                               & \funcname{raise} & \funcname{raise\_notrace}                       & \funcname{raise} & \funcname{raise\_notrace} \\
    \hline
    Compilation                                              & \parbox{8em}{inline assembly\\ (optimization)} & inline assembly   & \funcname{caml\_raise\_exn} & inline assembly \\
    \hline
  \end{tabular}
\end{frame}


%
% Takeaway #5
%
\subsection*{Takeaway \#5}
\frameSubsectionTakeaway{}
\againframe<5>{takeaway}
}%
      \onslide*<2>{\providecommand\step{6}%
% Backtraces
%
\subsection{One advantage of raising with debugging information}
\frameSubsection{}{}

\begin{frame}{Backtraces}{Debugging information \& source code}
  \begin{columns}[c]
    \begin{column}{0.5\textwidth}
      \begin{itemize}
        \item Raising an exception is fun and games, but how do we know where the exception originated? $\rightarrow$ With the backtrace associated with the exception!
        \item In order to get a backtrace from the point where an exception was raised, it's necessary to compile with debugging information enabled (\funcarg{-g} flag for the compiler)
        \item Same example as in the `Nested handlers' section
      \end{itemize}
    \end{column}
    \begin{column}{0.5\textwidth}
      \listocaml[firstline=9,lastline=34]{../src/backtrace/backtrace.ml}{backtrace/backtrace.ml}
    \end{column}
  \end{columns}
\end{frame}

\begin{frame}{Backtraces}{CMM and disassembly of \funcname{definitely\_raise}}
  \begin{columns}[c]
    \begin{column}{0.5\textwidth}
      \minipage[c][0.66\textheight][s]{\columnwidth}
      \begin{itemize}
        \item<1-> An effect of compiling with debugging information (\funcarg{-g}): the CMM no longer uses \funcname{raise\_notrace} to raise the exception but \funcname{raise} instead
        \item<2-> Disassembly shows that CMM \funcname{raise} is actually a call to \funcname{caml\_raise\_exn}, a function in assembler provided by the runtime
      \end{itemize}
      \vfill
      \mintocaml[firstline=9,lastline=13,highlightlines=12]{../src/backtrace/backtrace.ml}
      \endminipage
    \end{column}
    \begin{column}{0.5\textwidth}
      \onslide*<1>{
        \mintcmm[highlightlines=5]{../src/backtrace/camlBacktrace\_\_definitely\_raise\_347.cmm}
      }
      \onslide*<2>{
        \mintobjdump[firstline=2,highlightlines=14]{../src/backtrace/camlBacktrace\_\_definitely\_raise\_347.objdump}
      }
    \end{column}
  \end{columns}
\end{frame}

\begin{frame}{Backtraces}{Introducing \funcname{caml\_raise\_exn}}
  \begin{columns}[c]
    \begin{column}{0.5\textwidth}
      \minipage[c][0.66\textheight][s]{\columnwidth}
      \onslide*<1>{
        \begin{itemize}
          \item Raising the exception is performed using \funcname{caml\_raise\_exn} which is
            \begin{itemize}
              \item An assembly function provided by the runtime
              \item Architecture specific
            \end{itemize}
          \item Let's see what it does differently from \funcname{raise\_notrace}\dots
          \item We are about to raise \typename{ExnC} with \funcname{caml\_raise\_exn} from \funcname{definitely\_raise}
        \end{itemize}
      }
      \onslide*<2>{
        \mintobjdump[firstline=2,highlightlines=14]{../src/backtrace/camlBacktrace\_\_definitely\_raise\_347.objdump}
      }
      \vfill
      \mintocaml[firstline=9,lastline=13,highlightlines=12]{../src/backtrace/backtrace.ml}
      \endminipage
    \end{column}
    \begin{column}{0.5\textwidth}
      \centering
      \providecommand\step{1}\input{diagrams/backtrace/backtrace.tex}
    \end{column}
  \end{columns}
\end{frame}

\begin{frame}{Backtraces}{But before that, we need more fields from \typename{caml\_domain\_state}}
  \begin{columns}
    \begin{column}{0.5\textwidth}
      \begin{itemize}
        \item Remember \typename{caml\_domain\_state} the per-domain runtime state?
        \item Some other members are of interest to us now\dots
        \item \localname{backtrace\_active} is used to know if the runtime should collect backtraces
        \item \localname{backtrace\_active} can be set by
          \begin{itemize}
            \item The \funcarg{b} flag in the \funcname{OCAMLRUNPARAM} environment variable
            \item \mintinline{ocaml}{Printexc.record_backtrace : bool -> unit} \footnotemark
          \end{itemize}
      \end{itemize}
    \end{column}
    \begin{column}{0.5\textwidth}
      \begin{listing}
        \mintc[highlightlines={9-11}]{src/ocaml/domain\_state\_backtrace.h}
        \caption{runtime/caml/domain\_state.h}
      \end{listing}
    \end{column}
  \end{columns}
  \footnotetext[1]{https://v2.ocaml.org/api/Printexc.html}
\end{frame}

\newcommand\listCamlRaiseExn[1][]{
  \listasm[firstline=702,lastline=725,#1]{../ocaml/runtime/amd64.S}{runtime/amd64.S:702}}

\begin{frame}{Backtraces}{Walk through \funcname{caml\_raise\_exn}}
  \begin{columns}[T]
    \begin{column}{0.5\textwidth}
      \begin{itemize}
        \item<1-> First checks whether exceptions backtrace should be recorded
        \item<1-> By checking the \funcarg{domain\_state->backtrace\_active} value
        \item<2-> If not, acts as \funcname{raise\_notrace} with \funcname{RESTORE\_EXN\_HANDLER\_OCAML}
          \begin{itemize}
            \item Set \regname{RSP} to \localname{domain\_state->exn\_handler} value
            \item Restore \localname{domain\_state->exn\_handler} from trap
            \item Jump with \funcname{ret} (same effect as using \funcname{pop} and \funcname{jmp})
          \end{itemize}
        \item<3-> Otherwise, switches to the C stack to be able to execute C code
        \item<4-> Calls \funcname{caml\_stash\_backtrace}, a C function provided by the runtime\ignorespaces
          \onslide<6->{, with
            \begin{itemize}
              \item \funcarg{exn}: exception value
              \item \funcarg{pc}: code address of the raising point
              \item \funcarg{sp}: stack address of the raising function
              \item \funcarg{trapsp}: stack address of the next handler
            \end{itemize}
          }
      \end{itemize}
      \bigskip
      \mintocaml[firstline=9,lastline=13,highlightlines=12]{../src/backtrace/backtrace.ml}
    \end{column}
    \begin{column}{0.5\textwidth}
      \raggedleft
      \onslide*<1,3-4>{
        \mintc[firstline=190,lastline=192]{../ocaml/runtime/amd64.S}
        \mintc[firstline=234,lastline=237]{../ocaml/runtime/amd64.S}
      }
      \onslide*<2>{
        \mintc[firstline=190,lastline=192,highlightlines={191-192}]{../ocaml/runtime/amd64.S}
        \mintc[firstline=234,lastline=237,highlightlines={235-237}]{../ocaml/runtime/amd64.S}
      }
      \onslide*<1>{\listCamlRaiseExn[highlightlines=705]{}}%
      \onslide*<2>{\listCamlRaiseExn[highlightlines={707-708}]{}}%
      \onslide*<3>{\listCamlRaiseExn[highlightlines={712-714}]{}}%
      \onslide*<4>{\listCamlRaiseExn[highlightlines={715-720}]{}}%
      \onslide*<5>{\providecommand\step{1}\input{diagrams/backtrace/backtrace.tex}}%
      \onslide*<6>{\providecommand\step{2}\input{diagrams/backtrace/backtrace.tex}}%
    \end{column}
  \end{columns}
\end{frame}

\newcommand\listStashBacktrace[1][]{
  \listc[firstline=91,lastline=120,#1]{../ocaml/runtime/backtrace\_nat.c}{runtime/backtrace\_nat.c:91}}

\begin{frame}{Backtraces}{Introducing \funcname{caml\_stash\_backtrace}}
  \begin{columns}[c]
    \begin{column}{0.5\textwidth}
      \minipage[c][0.66\textheight][s]{\columnwidth}
      \begin{itemize}
        \item<1-> A C function provided by the runtime (specific to native code)
        \item<2-> It scans the stack, between the stack pointer at the raising point up to the next trap, in order to collect the names of the detected function frames
          \onslide*<2>{
            \begin{itemize}
              \item And more information, like line number, column number...
            \end{itemize}
          }
        \item<3-> It uses the same technique and information as the Garbage Collector (GC) uses to scan the values live on the stack
          \onslide*<4>{
            \begin{itemize}
              \item \typename{frame\_descr} are at the core of stack unwinding
            \end{itemize}
          }
      \end{itemize}
      \vfill
      \mintocaml[firstline=9,lastline=13,highlightlines=12]{../src/backtrace/backtrace.ml}
      \endminipage
    \end{column}
    \begin{column}{0.5\textwidth}
      \onslide*<1>{\listStashBacktrace[highlightlines=91]{}}
      \onslide*<2>{\listStashBacktrace[highlightlines={91,117-118}]{}}
      \onslide*<3->{\listStashBacktrace[highlightlines={94,106,109-110}]{}}
    \end{column}
  \end{columns}
\end{frame}

\newcommand\listFrameDescr[1][]{
  \listocaml[firstline=111,lastline=124,#1]{../ocaml/asmcomp/emitaux.ml}{asmcomp/emitaux.ml:111}}
\newcommand\listDebugInfo[1][]{
  \listocaml[firstline=38,lastline=49,#1]{../ocaml/lambda/debuginfo.mli}{lambda/debuginfo.mli:38}}

\begin{frame}{Backtraces}{\typename{frame\_descr} and \typename{frame\_debuginfo} in a nutshell}
  \begin{columns}[c]
    \begin{column}{0.5\textwidth}
      \begin{itemize}
        \item<1-> For every return address in an OCaml program, there is a associated \typename{frame\_descr}
        \item<2-> A \typename{frame\_descr} specifies
        \begin{itemize}
          \item<2-> The size of the function stack frame
          \item<3-> Where live values are on the stack (so that the GC can mark them as live and prevent from being swept)
        \end{itemize}
        \item<4-> When compiled with debug information, \typename{frame\_descr} will be linked to a \typename{frame\_debuginfo}
        \begin{itemize}
          \item<5-> Contains information about the position in the source code (file name, line and column number)
        \end{itemize}
      \end{itemize}
    \end{column}
    \begin{column}{0.5\textwidth}
        \onslide*<1>{\listFrameDescr[highlightlines={118-122}]{}}
        \onslide*<2>{\listFrameDescr[highlightlines={120}]{}}
        \onslide*<3>{\listFrameDescr[highlightlines={121}]{}}
        \onslide*<4->{\listFrameDescr[highlightlines={122}]{}}
        \medskip
        \onslide*<1-3>{\listDebugInfo{}}
        \onslide*<4>{\listDebugInfo[highlightlines={38,49}]{}}
        \onslide*<5>{\listDebugInfo[highlightlines={39-46}]{}}
    \end{column}
  \end{columns}
\end{frame}

\begin{frame}{Backtraces}{Scanning the stack with \typename{frame\_descr}}
  \begin{columns}[c]
    \begin{column}{0.5\textwidth}
      \begin{itemize}
        \item Given the return address of the code that raises the exception by calling \funcname{caml\_raise\_exn} (\funcarg{pc} argument of \funcname{caml\_stash\_backtrace})
        \item And the position of the stack pointer (\funcarg{sp} argument of \funcname{caml\_stash\_backtrace})
        \item The frame size given by \typename{frame\_descr}.\funcarg{frame\_size}, allows to find the end of the previous frame
        \item And the return address of the caller of this function
        \bigskip
        \onslide<3->{
        \item Note that traps (here the reddish \funcname{ExnA}, \funcname{ExnB} and \funcname{ExnC} blocks) belong to the frame that installed them, and so the frame size changes when traps are removed
        }
      \end{itemize}
    \end{column}
    \begin{column}{0.5\textwidth}
      \raggedleft
      \onslide*<1>{\providecommand\step{2}\input{diagrams/backtrace/backtrace.tex}}%
      \onslide*<2>{\providecommand\step{1}\input{diagrams/backtrace/scanstack.tex}}%
      \onslide*<3>{\providecommand\step{2}\input{diagrams/backtrace/scanstack.tex}}%
      \onslide*<4>{\providecommand\step{3}\input{diagrams/backtrace/scanstack.tex}}%
      \onslide*<5>{\providecommand\step{4}\input{diagrams/backtrace/scanstack.tex}}%
      \onslide*<6>{\providecommand\step{5}\input{diagrams/backtrace/scanstack.tex}}%
    \end{column}
  \end{columns}
\end{frame}

% Duplicate of previous-previous slide
\begin{frame}{Backtraces}{Continuing on \funcname{caml\_stash\_backtrace}}
  \begin{columns}[c]
    \begin{column}{0.5\textwidth}
      \minipage[c][0.75\textheight][s]{\columnwidth}
      \begin{itemize}
          \color{gray}
        \item A C function provided by the runtime (specific to native code)
        \item It scans the stack, between the stack pointer at the raising point up to the next trap, in order to collect the names of the detected function frames
        \item It uses the same technique and information as the Garbage Collector (GC) uses to scan the values live on the stack
      \end{itemize}
      \begin{itemize}
        \item For each function frame detected, store a pointer to the \typename{frame\_descr} in the \localname{domain\_state->backtrace\_buffer} array, effectively constructing the backtrace
      \end{itemize}
      \onslide<2->{
        \begin{itemize}
            \smallskip
          \item Constructed backtrace:
            \funcname{definitely\_raise},
            \onslide<3->{\funcname{probably\_raise}}
        \end{itemize}
      }
      \vfill
      \mintocaml[firstline=9,lastline=13,highlightlines=12]{../src/backtrace/backtrace.ml}
      \endminipage
    \end{column}
    \begin{column}{0.5\textwidth}
      \raggedleft
      \onslide*<1>{\listStashBacktrace[highlightlines={114-115}]{}}
      \onslide*<2>{\providecommand\step{3}\input{diagrams/backtrace/backtrace.tex}}%
      \onslide*<3>{\providecommand\step{4}\input{diagrams/backtrace/backtrace.tex}}%
    \end{column}
  \end{columns}
\end{frame}

\begin{frame}{Backtraces}{Back to \funcname{caml\_raise\_exn}}
  \begin{columns}[c]
    \begin{column}{0.5\textwidth}
      \minipage[c][0.66\textheight][s]{\columnwidth}
      \begin{itemize}\color{gray}
        \item Checks wheteher exceptions backtrace should be recorded, by checking the \funcarg{domain\_state->backtrace\_active} value
        \item Switches to the C stack to be able to execute C code
        \item Calls \funcname{caml\_stash\_backtrace}, to collect the backtrace up to the next trap
      \end{itemize}
      \onslide*<2->{
        \begin{itemize}
          \item<2-> Executes the exception handler in \funcname{probably\_raise} with \funcname{RESTORE\_EXN\_HANDLER\_OCAML}
            \begin{itemize}
              \item Set \regname{RSP} to \localname{domain\_state->exn\_handler} value
              \item Restore \localname{domain\_state->exn\_handler} from trap
              \item Jump to the handler code with \funcname{ret}
            \end{itemize}
        \end{itemize}
      }
      \vfill
      \onslide*<1>{\mintocaml[firstline=9,lastline=13,highlightlines=12]{../src/backtrace/backtrace.ml}}
      \onslide*<2->{\mintocaml[firstline=15,lastline=21,highlightlines={19-21}]{../src/backtrace/backtrace.ml}}
      \endminipage
    \end{column}
    \begin{column}{0.5\textwidth}
      \centering
      \onslide*<1>{
        \mintc[firstline=190,lastline=192]{../ocaml/runtime/amd64.S}
        \mintc[firstline=234,lastline=237]{../ocaml/runtime/amd64.S}
      }
      \onslide*<2>{
        \mintc[firstline=190,lastline=192,highlightlines={191-192}]{../ocaml/runtime/amd64.S}
        \mintc[firstline=234,lastline=237,highlightlines={235-237}]{../ocaml/runtime/amd64.S}
      }
      \onslide*<1>{\listCamlRaiseExn[highlightlines=720]{}}
      \onslide*<2>{\listCamlRaiseExn[highlightlines=722]{}}
      \onslide*<3->{\providecommand\step{5}\input{diagrams/backtrace/backtrace.tex}}
    \end{column}
  \end{columns}
\end{frame}

\begin{frame}{Backtraces}{\funcname{caml\_probably\_raise}'s handler doesn't catch \typename{ExnC}}
  \begin{columns}[c]
    \begin{column}{0.45\textwidth}
      \minipage[c][0.75\textheight][s]{\columnwidth}
      \begin{itemize}
        \item<1-> \funcname{match \dots with}\footnotemark pattern matching can also match exceptions
        \item<1-> The CMM of \funcname{probably\_raise} shows that this \funcname{match \dots with} is implemented with a pattern matching and a \funcname{try \dots with}
        \item<1-> But this one only matches on \typename{ExnA} exception
        \item<2-> Since the pattern matching fails to catch the exception, \funcname{reraise} is called with our current \typename{ExnC} exception
        \item<3-> Disassembly shows that \funcname{reraise} is actually a call to \funcname{caml\_reraise\_exn}, which is also an assembly function provided by the runtime
      \end{itemize}
      \vfill
      \mintocaml[firstline=15,lastline=21,highlightlines={18-21}]{../src/backtrace/backtrace.ml}
      \endminipage
    \end{column}
    \begin{column}{0.55\textwidth}
      \centering
      \onslide*<1>{\mintcmm[highlightlines={10-18}]{../src/backtrace/camlBacktrace\_\_probably\_raise\_351.cmm}}
      \onslide*<2>{\listcmm[highlightlines=18]{../src/backtrace/camlBacktrace\_\_probably\_raise_351.cmm}{CMM of probably\_raise}}
      \onslide*<3>{\mintobjdump[firstline=5,lastline=35,highlightlines=31]{../src/backtrace/camlBacktrace\_\_probably\_raise_351.objdump}}
    \end{column}
  \end{columns}
  \only<1>{\footnotetext[1]{https://blog.janestreet.com/pattern-matching-and-exception-handling-unite/}}
\end{frame}

\newcommand\listCamlReraiseExn[1][]{
  \listasm[firstline=727,lastline=734,#1]{../ocaml/runtime/amd64.S}{runtime/amd64.S:727}}

\begin{frame}{Backtraces}{Introducing \funcname{caml\_reraise\_exn}}
  \begin{columns}[c]
    \begin{column}{0.5\textwidth}
      \minipage[c][0.66\textheight][s]{\columnwidth}
      \begin{itemize}
        \item<1-> The exception have to be reraised to the next handler
        \item<2-> First, checks if the backtrace should be collected with \funcarg{domain\_state->backtrace\_active}
        \only<3>{
        \begin{itemize}
            \item If not, behaves like a \funcname{raise\_notrace} with \funcname{RESTORE\_EXN\_HANDLER\_OCAML}
        \end{itemize}
        }
        \item<4-> Jumps into \funcname{caml\_raise\_exn}
        \item<5-> Calls \funcname{caml\_stash\_backtrace} again, to scan the next part of the stack: from the trap just pop-ed to the next trap
      \end{itemize}
      \vfill
      \mintocaml[firstline=15,lastline=21,highlightlines={18-21}]{../src/backtrace/backtrace.ml}
      \endminipage
    \end{column}
    \begin{column}{0.5\textwidth}
      \raggedleft
      \onslide*<1>{\listCamlReraiseExn{}}
      \onslide*<2>{\listCamlReraiseExn[highlightlines=729]{}}
      \onslide*<3>{
        \mintc[firstline=190,lastline=192,highlightlines={191-192}]{../ocaml/runtime/amd64.S}
        \mintc[firstline=234,lastline=237,highlightlines={235-237}]{../ocaml/runtime/amd64.S}
        \medskip
        \listCamlReraiseExn[highlightlines=731]{}
      }
      \onslide*<4-5>{
        % TODO refactor with \listCamlReraiseExn
        \mintasm[firstline=727,lastline=734,highlightlines=730]{../ocaml/runtime/amd64.S}
        \medskip
      }
      \onslide*<4>{\listCamlRaiseExn[highlightlines=711]{}}
      \onslide*<5>{\listCamlRaiseExn[highlightlines=720]{}}
    \end{column}
  \end{columns}
\end{frame}

\begin{frame}{Backtraces}{Backtrace collection continues}
  \begin{columns}[c]
    \begin{column}{0.55\textwidth}
      \minipage[c][0.75\textheight][s]{\columnwidth}
      \begin{itemize}
        \item<1-> \funcname{caml\_stash\_backtrace} scans the older part of the stack, up to the next trap
        \item<6-> Controls jumps to the nested exception handler in \funcname{maybe\_raise}, which matches only on \typename{ExnB}
        \item<7-> \funcname{caml\_reraise\_exn} is called again, backtrace collection resumes with \funcname{caml\_stash\_backtrace}
      \end{itemize}
      \medskip
      \begin{itemize}
        \item<2-> Constructed backtrace:
        \begin{itemize}
          \item <2-> \funcname{definitely\_raise}
          \item <2-> \funcname{probably\_raise}
          \item <3-> \funcname{probably\_raise} (re-raise)
          \item <4-> \funcname{maybe\_raise}
          \item <5-> \funcname{main}
          \item <8-> \funcname{main} (re-raise)
        \end{itemize}
      \end{itemize}
      \vfill
      \onslide*<1-5>{\mintocaml[firstline=15,lastline=21]{../src/backtrace/backtrace.ml}}
      \onslide*<6>{\mintocaml[firstline=28,lastline=34,highlightlines=31]{../src/backtrace/backtrace.ml}}
      \onslide*<7->{\mintocaml[firstline=28,lastline=34,highlightlines=33]{../src/backtrace/backtrace.ml}}
      \endminipage
    \end{column}
    \begin{column}{0.45\textwidth}
      \raggedleft
      \onslide*<1>{\providecommand\step{6}\input{diagrams/backtrace/backtrace.tex}}%
      \onslide*<2>{\providecommand\step{6}\input{diagrams/backtrace/backtrace.tex}}%
      \onslide*<3>{\providecommand\step{7}\input{diagrams/backtrace/backtrace.tex}}%
      \onslide*<4>{\providecommand\step{8}\input{diagrams/backtrace/backtrace.tex}}%
      \onslide*<5>{\providecommand\step{9}\input{diagrams/backtrace/backtrace.tex}}%
      \onslide*<6>{\providecommand\step{10}\input{diagrams/backtrace/backtrace.tex}}%
      \onslide*<7>{\providecommand\step{11}\input{diagrams/backtrace/backtrace.tex}}%
      \onslide*<8>{\providecommand\step{12}\input{diagrams/backtrace/backtrace.tex}}%
      \onslide*<9>{\providecommand\step{13}\input{diagrams/backtrace/backtrace.tex}}%
    \end{column}
  \end{columns}
\end{frame}

\begin{frame}{Backtraces}{Printing the backtrace}
  \begin{columns}[c]
    \begin{column}{0.45\textwidth}
      \minipage[c][0.66\textheight][s]{\columnwidth}
      \begin{itemize}
        \item<1-> The exception backtrace can now be printed to \funcarg{stdout} with \funcname{Printexc.print_backtrace}
        \item<2-> First the exception backtrace is retrieved into an OCaml array with \funcname{get_raw_backtrace }
        \item<3-> Which is implemented as the \funcname{caml_get_exception_raw_backtrace} C function in the runtime
        \item<4-> And finally printed with \funcname{print_exception_backtrace}
      \end{itemize}
      \vfill
      \mintocaml[firstline=28,lastline=34,highlightlines=33]{../src/backtrace/backtrace.ml}
      \endminipage
    \end{column}
    \begin{column}{0.55\textwidth}
      \onslide*<1,2>{
        \mintocaml[firstline=100,lastline=101]{../ocaml/stdlib/printexc.ml}
      }
      \only<1>{
        \listocaml[firstline=170,lastline=172]{../ocaml/stdlib/printexc.ml}{stdlib/printexc.ml:170}
      }
      \only<2>{
        \listocaml[firstline=170,lastline=172,highlightlines=172]{../ocaml/stdlib/printexc.ml}{stdlib/printexc.ml:170}
      }
      \only<3>{
        \mintc[firstline=154,lastline=156]{../ocaml/runtime/backtrace.c}
        \listc[firstline=171,lastline=192,highlightlines={184-187}]{../ocaml/runtime/backtrace.c}{runtime/backtrace.c:154}
      }
      \only<4>{
        \listocaml[firstline=155,lastline=165]{../ocaml/stdlib/printexc.ml}{stdlib/printexc.ml:55}
      }
    \end{column}
  \end{columns}
\end{frame}


\subsection{The multiple kinds of raise}
\frameSubsection{}{}

\begin{frame}{Backtraces}{When backtraces are not needed}
  \begin{columns}[c]
    \begin{column}{0.45\textwidth}
      \minipage[c][0.66\textheight][s]{\columnwidth}
      \begin{itemize}
        \item<1-> What if the backtrace for an exception is never needed?
        \item<2-> It's possible to raise the exception explicitly with \funcname{raise\_notrace} to make raising as cheap as possible
        \item<3-> An example of use in the \funcname{dynlink} library of the OCaml compiler
      \end{itemize}
      \vfill
      \onslide<3->{
      \listocaml[firstline=205,lastline=209,highlightlines=207]{../ocaml/otherlibs/dynlink/dynlink\_compilerlibs/misc.ml}{otherlibs/dynlink/dynlink\_compilerlibs/misc.ml:205}
      }
      \endminipage
    \end{column}
    \begin{column}{0.55\textwidth}
    \only<4>{
      \mintcmm[highlightlines=3]{../src/notrace/camlNotrace\_\_fun_347.cmm}
      \listcmm{../src/notrace/camlNotrace\_\_all\_somes\_327.cmm}{all\_somes}
    }
    \onslide*<5->{
      \listobjdump[highlightlines={6-9}]{../src/notrace/camlNotrace\_\_fun_347.objdump}{anonymous function from \funcname{all\_somes}}
    }
    \end{column}
  \end{columns}
\end{frame}

\begin{frame}{Backtraces}{Summary table of the multiple kinds of raise}
  \begin{itemize}
    \item When debugging information is enabled and if the backtrace of an exception isn't needed, it's possible to raise with \funcname{raise\_notrace} for performance
    \item When debugging information is \emph{not} enabled, the compiler optimises \funcname{raise} calls to inline assembly (same effect as using \funcname{raise\_notrace})
  \end{itemize}
  \bigskip
  \centering
  \begin{tabular}{| l | c | c | c | c |}
    \hline
    \parbox{10em}{Debug information \\ (\funcarg{-g} flag)}  & \multicolumn{2}{|c|}{Disabled}                                     & \multicolumn{2}{|c|}{Enabled} \\
    \hline
    Raise kind                                               & \funcname{raise} & \funcname{raise\_notrace}                       & \funcname{raise} & \funcname{raise\_notrace} \\
    \hline
    Compilation                                              & \parbox{8em}{inline assembly\\ (optimization)} & inline assembly   & \funcname{caml\_raise\_exn} & inline assembly \\
    \hline
  \end{tabular}
\end{frame}


%
% Takeaway #5
%
\subsection*{Takeaway \#5}
\frameSubsectionTakeaway{}
\againframe<5>{takeaway}
}%
      \onslide*<3>{\providecommand\step{7}%
% Backtraces
%
\subsection{One advantage of raising with debugging information}
\frameSubsection{}{}

\begin{frame}{Backtraces}{Debugging information \& source code}
  \begin{columns}[c]
    \begin{column}{0.5\textwidth}
      \begin{itemize}
        \item Raising an exception is fun and games, but how do we know where the exception originated? $\rightarrow$ With the backtrace associated with the exception!
        \item In order to get a backtrace from the point where an exception was raised, it's necessary to compile with debugging information enabled (\funcarg{-g} flag for the compiler)
        \item Same example as in the `Nested handlers' section
      \end{itemize}
    \end{column}
    \begin{column}{0.5\textwidth}
      \listocaml[firstline=9,lastline=34]{../src/backtrace/backtrace.ml}{backtrace/backtrace.ml}
    \end{column}
  \end{columns}
\end{frame}

\begin{frame}{Backtraces}{CMM and disassembly of \funcname{definitely\_raise}}
  \begin{columns}[c]
    \begin{column}{0.5\textwidth}
      \minipage[c][0.66\textheight][s]{\columnwidth}
      \begin{itemize}
        \item<1-> An effect of compiling with debugging information (\funcarg{-g}): the CMM no longer uses \funcname{raise\_notrace} to raise the exception but \funcname{raise} instead
        \item<2-> Disassembly shows that CMM \funcname{raise} is actually a call to \funcname{caml\_raise\_exn}, a function in assembler provided by the runtime
      \end{itemize}
      \vfill
      \mintocaml[firstline=9,lastline=13,highlightlines=12]{../src/backtrace/backtrace.ml}
      \endminipage
    \end{column}
    \begin{column}{0.5\textwidth}
      \onslide*<1>{
        \mintcmm[highlightlines=5]{../src/backtrace/camlBacktrace\_\_definitely\_raise\_347.cmm}
      }
      \onslide*<2>{
        \mintobjdump[firstline=2,highlightlines=14]{../src/backtrace/camlBacktrace\_\_definitely\_raise\_347.objdump}
      }
    \end{column}
  \end{columns}
\end{frame}

\begin{frame}{Backtraces}{Introducing \funcname{caml\_raise\_exn}}
  \begin{columns}[c]
    \begin{column}{0.5\textwidth}
      \minipage[c][0.66\textheight][s]{\columnwidth}
      \onslide*<1>{
        \begin{itemize}
          \item Raising the exception is performed using \funcname{caml\_raise\_exn} which is
            \begin{itemize}
              \item An assembly function provided by the runtime
              \item Architecture specific
            \end{itemize}
          \item Let's see what it does differently from \funcname{raise\_notrace}\dots
          \item We are about to raise \typename{ExnC} with \funcname{caml\_raise\_exn} from \funcname{definitely\_raise}
        \end{itemize}
      }
      \onslide*<2>{
        \mintobjdump[firstline=2,highlightlines=14]{../src/backtrace/camlBacktrace\_\_definitely\_raise\_347.objdump}
      }
      \vfill
      \mintocaml[firstline=9,lastline=13,highlightlines=12]{../src/backtrace/backtrace.ml}
      \endminipage
    \end{column}
    \begin{column}{0.5\textwidth}
      \centering
      \providecommand\step{1}\input{diagrams/backtrace/backtrace.tex}
    \end{column}
  \end{columns}
\end{frame}

\begin{frame}{Backtraces}{But before that, we need more fields from \typename{caml\_domain\_state}}
  \begin{columns}
    \begin{column}{0.5\textwidth}
      \begin{itemize}
        \item Remember \typename{caml\_domain\_state} the per-domain runtime state?
        \item Some other members are of interest to us now\dots
        \item \localname{backtrace\_active} is used to know if the runtime should collect backtraces
        \item \localname{backtrace\_active} can be set by
          \begin{itemize}
            \item The \funcarg{b} flag in the \funcname{OCAMLRUNPARAM} environment variable
            \item \mintinline{ocaml}{Printexc.record_backtrace : bool -> unit} \footnotemark
          \end{itemize}
      \end{itemize}
    \end{column}
    \begin{column}{0.5\textwidth}
      \begin{listing}
        \mintc[highlightlines={9-11}]{src/ocaml/domain\_state\_backtrace.h}
        \caption{runtime/caml/domain\_state.h}
      \end{listing}
    \end{column}
  \end{columns}
  \footnotetext[1]{https://v2.ocaml.org/api/Printexc.html}
\end{frame}

\newcommand\listCamlRaiseExn[1][]{
  \listasm[firstline=702,lastline=725,#1]{../ocaml/runtime/amd64.S}{runtime/amd64.S:702}}

\begin{frame}{Backtraces}{Walk through \funcname{caml\_raise\_exn}}
  \begin{columns}[T]
    \begin{column}{0.5\textwidth}
      \begin{itemize}
        \item<1-> First checks whether exceptions backtrace should be recorded
        \item<1-> By checking the \funcarg{domain\_state->backtrace\_active} value
        \item<2-> If not, acts as \funcname{raise\_notrace} with \funcname{RESTORE\_EXN\_HANDLER\_OCAML}
          \begin{itemize}
            \item Set \regname{RSP} to \localname{domain\_state->exn\_handler} value
            \item Restore \localname{domain\_state->exn\_handler} from trap
            \item Jump with \funcname{ret} (same effect as using \funcname{pop} and \funcname{jmp})
          \end{itemize}
        \item<3-> Otherwise, switches to the C stack to be able to execute C code
        \item<4-> Calls \funcname{caml\_stash\_backtrace}, a C function provided by the runtime\ignorespaces
          \onslide<6->{, with
            \begin{itemize}
              \item \funcarg{exn}: exception value
              \item \funcarg{pc}: code address of the raising point
              \item \funcarg{sp}: stack address of the raising function
              \item \funcarg{trapsp}: stack address of the next handler
            \end{itemize}
          }
      \end{itemize}
      \bigskip
      \mintocaml[firstline=9,lastline=13,highlightlines=12]{../src/backtrace/backtrace.ml}
    \end{column}
    \begin{column}{0.5\textwidth}
      \raggedleft
      \onslide*<1,3-4>{
        \mintc[firstline=190,lastline=192]{../ocaml/runtime/amd64.S}
        \mintc[firstline=234,lastline=237]{../ocaml/runtime/amd64.S}
      }
      \onslide*<2>{
        \mintc[firstline=190,lastline=192,highlightlines={191-192}]{../ocaml/runtime/amd64.S}
        \mintc[firstline=234,lastline=237,highlightlines={235-237}]{../ocaml/runtime/amd64.S}
      }
      \onslide*<1>{\listCamlRaiseExn[highlightlines=705]{}}%
      \onslide*<2>{\listCamlRaiseExn[highlightlines={707-708}]{}}%
      \onslide*<3>{\listCamlRaiseExn[highlightlines={712-714}]{}}%
      \onslide*<4>{\listCamlRaiseExn[highlightlines={715-720}]{}}%
      \onslide*<5>{\providecommand\step{1}\input{diagrams/backtrace/backtrace.tex}}%
      \onslide*<6>{\providecommand\step{2}\input{diagrams/backtrace/backtrace.tex}}%
    \end{column}
  \end{columns}
\end{frame}

\newcommand\listStashBacktrace[1][]{
  \listc[firstline=91,lastline=120,#1]{../ocaml/runtime/backtrace\_nat.c}{runtime/backtrace\_nat.c:91}}

\begin{frame}{Backtraces}{Introducing \funcname{caml\_stash\_backtrace}}
  \begin{columns}[c]
    \begin{column}{0.5\textwidth}
      \minipage[c][0.66\textheight][s]{\columnwidth}
      \begin{itemize}
        \item<1-> A C function provided by the runtime (specific to native code)
        \item<2-> It scans the stack, between the stack pointer at the raising point up to the next trap, in order to collect the names of the detected function frames
          \onslide*<2>{
            \begin{itemize}
              \item And more information, like line number, column number...
            \end{itemize}
          }
        \item<3-> It uses the same technique and information as the Garbage Collector (GC) uses to scan the values live on the stack
          \onslide*<4>{
            \begin{itemize}
              \item \typename{frame\_descr} are at the core of stack unwinding
            \end{itemize}
          }
      \end{itemize}
      \vfill
      \mintocaml[firstline=9,lastline=13,highlightlines=12]{../src/backtrace/backtrace.ml}
      \endminipage
    \end{column}
    \begin{column}{0.5\textwidth}
      \onslide*<1>{\listStashBacktrace[highlightlines=91]{}}
      \onslide*<2>{\listStashBacktrace[highlightlines={91,117-118}]{}}
      \onslide*<3->{\listStashBacktrace[highlightlines={94,106,109-110}]{}}
    \end{column}
  \end{columns}
\end{frame}

\newcommand\listFrameDescr[1][]{
  \listocaml[firstline=111,lastline=124,#1]{../ocaml/asmcomp/emitaux.ml}{asmcomp/emitaux.ml:111}}
\newcommand\listDebugInfo[1][]{
  \listocaml[firstline=38,lastline=49,#1]{../ocaml/lambda/debuginfo.mli}{lambda/debuginfo.mli:38}}

\begin{frame}{Backtraces}{\typename{frame\_descr} and \typename{frame\_debuginfo} in a nutshell}
  \begin{columns}[c]
    \begin{column}{0.5\textwidth}
      \begin{itemize}
        \item<1-> For every return address in an OCaml program, there is a associated \typename{frame\_descr}
        \item<2-> A \typename{frame\_descr} specifies
        \begin{itemize}
          \item<2-> The size of the function stack frame
          \item<3-> Where live values are on the stack (so that the GC can mark them as live and prevent from being swept)
        \end{itemize}
        \item<4-> When compiled with debug information, \typename{frame\_descr} will be linked to a \typename{frame\_debuginfo}
        \begin{itemize}
          \item<5-> Contains information about the position in the source code (file name, line and column number)
        \end{itemize}
      \end{itemize}
    \end{column}
    \begin{column}{0.5\textwidth}
        \onslide*<1>{\listFrameDescr[highlightlines={118-122}]{}}
        \onslide*<2>{\listFrameDescr[highlightlines={120}]{}}
        \onslide*<3>{\listFrameDescr[highlightlines={121}]{}}
        \onslide*<4->{\listFrameDescr[highlightlines={122}]{}}
        \medskip
        \onslide*<1-3>{\listDebugInfo{}}
        \onslide*<4>{\listDebugInfo[highlightlines={38,49}]{}}
        \onslide*<5>{\listDebugInfo[highlightlines={39-46}]{}}
    \end{column}
  \end{columns}
\end{frame}

\begin{frame}{Backtraces}{Scanning the stack with \typename{frame\_descr}}
  \begin{columns}[c]
    \begin{column}{0.5\textwidth}
      \begin{itemize}
        \item Given the return address of the code that raises the exception by calling \funcname{caml\_raise\_exn} (\funcarg{pc} argument of \funcname{caml\_stash\_backtrace})
        \item And the position of the stack pointer (\funcarg{sp} argument of \funcname{caml\_stash\_backtrace})
        \item The frame size given by \typename{frame\_descr}.\funcarg{frame\_size}, allows to find the end of the previous frame
        \item And the return address of the caller of this function
        \bigskip
        \onslide<3->{
        \item Note that traps (here the reddish \funcname{ExnA}, \funcname{ExnB} and \funcname{ExnC} blocks) belong to the frame that installed them, and so the frame size changes when traps are removed
        }
      \end{itemize}
    \end{column}
    \begin{column}{0.5\textwidth}
      \raggedleft
      \onslide*<1>{\providecommand\step{2}\input{diagrams/backtrace/backtrace.tex}}%
      \onslide*<2>{\providecommand\step{1}\input{diagrams/backtrace/scanstack.tex}}%
      \onslide*<3>{\providecommand\step{2}\input{diagrams/backtrace/scanstack.tex}}%
      \onslide*<4>{\providecommand\step{3}\input{diagrams/backtrace/scanstack.tex}}%
      \onslide*<5>{\providecommand\step{4}\input{diagrams/backtrace/scanstack.tex}}%
      \onslide*<6>{\providecommand\step{5}\input{diagrams/backtrace/scanstack.tex}}%
    \end{column}
  \end{columns}
\end{frame}

% Duplicate of previous-previous slide
\begin{frame}{Backtraces}{Continuing on \funcname{caml\_stash\_backtrace}}
  \begin{columns}[c]
    \begin{column}{0.5\textwidth}
      \minipage[c][0.75\textheight][s]{\columnwidth}
      \begin{itemize}
          \color{gray}
        \item A C function provided by the runtime (specific to native code)
        \item It scans the stack, between the stack pointer at the raising point up to the next trap, in order to collect the names of the detected function frames
        \item It uses the same technique and information as the Garbage Collector (GC) uses to scan the values live on the stack
      \end{itemize}
      \begin{itemize}
        \item For each function frame detected, store a pointer to the \typename{frame\_descr} in the \localname{domain\_state->backtrace\_buffer} array, effectively constructing the backtrace
      \end{itemize}
      \onslide<2->{
        \begin{itemize}
            \smallskip
          \item Constructed backtrace:
            \funcname{definitely\_raise},
            \onslide<3->{\funcname{probably\_raise}}
        \end{itemize}
      }
      \vfill
      \mintocaml[firstline=9,lastline=13,highlightlines=12]{../src/backtrace/backtrace.ml}
      \endminipage
    \end{column}
    \begin{column}{0.5\textwidth}
      \raggedleft
      \onslide*<1>{\listStashBacktrace[highlightlines={114-115}]{}}
      \onslide*<2>{\providecommand\step{3}\input{diagrams/backtrace/backtrace.tex}}%
      \onslide*<3>{\providecommand\step{4}\input{diagrams/backtrace/backtrace.tex}}%
    \end{column}
  \end{columns}
\end{frame}

\begin{frame}{Backtraces}{Back to \funcname{caml\_raise\_exn}}
  \begin{columns}[c]
    \begin{column}{0.5\textwidth}
      \minipage[c][0.66\textheight][s]{\columnwidth}
      \begin{itemize}\color{gray}
        \item Checks wheteher exceptions backtrace should be recorded, by checking the \funcarg{domain\_state->backtrace\_active} value
        \item Switches to the C stack to be able to execute C code
        \item Calls \funcname{caml\_stash\_backtrace}, to collect the backtrace up to the next trap
      \end{itemize}
      \onslide*<2->{
        \begin{itemize}
          \item<2-> Executes the exception handler in \funcname{probably\_raise} with \funcname{RESTORE\_EXN\_HANDLER\_OCAML}
            \begin{itemize}
              \item Set \regname{RSP} to \localname{domain\_state->exn\_handler} value
              \item Restore \localname{domain\_state->exn\_handler} from trap
              \item Jump to the handler code with \funcname{ret}
            \end{itemize}
        \end{itemize}
      }
      \vfill
      \onslide*<1>{\mintocaml[firstline=9,lastline=13,highlightlines=12]{../src/backtrace/backtrace.ml}}
      \onslide*<2->{\mintocaml[firstline=15,lastline=21,highlightlines={19-21}]{../src/backtrace/backtrace.ml}}
      \endminipage
    \end{column}
    \begin{column}{0.5\textwidth}
      \centering
      \onslide*<1>{
        \mintc[firstline=190,lastline=192]{../ocaml/runtime/amd64.S}
        \mintc[firstline=234,lastline=237]{../ocaml/runtime/amd64.S}
      }
      \onslide*<2>{
        \mintc[firstline=190,lastline=192,highlightlines={191-192}]{../ocaml/runtime/amd64.S}
        \mintc[firstline=234,lastline=237,highlightlines={235-237}]{../ocaml/runtime/amd64.S}
      }
      \onslide*<1>{\listCamlRaiseExn[highlightlines=720]{}}
      \onslide*<2>{\listCamlRaiseExn[highlightlines=722]{}}
      \onslide*<3->{\providecommand\step{5}\input{diagrams/backtrace/backtrace.tex}}
    \end{column}
  \end{columns}
\end{frame}

\begin{frame}{Backtraces}{\funcname{caml\_probably\_raise}'s handler doesn't catch \typename{ExnC}}
  \begin{columns}[c]
    \begin{column}{0.45\textwidth}
      \minipage[c][0.75\textheight][s]{\columnwidth}
      \begin{itemize}
        \item<1-> \funcname{match \dots with}\footnotemark pattern matching can also match exceptions
        \item<1-> The CMM of \funcname{probably\_raise} shows that this \funcname{match \dots with} is implemented with a pattern matching and a \funcname{try \dots with}
        \item<1-> But this one only matches on \typename{ExnA} exception
        \item<2-> Since the pattern matching fails to catch the exception, \funcname{reraise} is called with our current \typename{ExnC} exception
        \item<3-> Disassembly shows that \funcname{reraise} is actually a call to \funcname{caml\_reraise\_exn}, which is also an assembly function provided by the runtime
      \end{itemize}
      \vfill
      \mintocaml[firstline=15,lastline=21,highlightlines={18-21}]{../src/backtrace/backtrace.ml}
      \endminipage
    \end{column}
    \begin{column}{0.55\textwidth}
      \centering
      \onslide*<1>{\mintcmm[highlightlines={10-18}]{../src/backtrace/camlBacktrace\_\_probably\_raise\_351.cmm}}
      \onslide*<2>{\listcmm[highlightlines=18]{../src/backtrace/camlBacktrace\_\_probably\_raise_351.cmm}{CMM of probably\_raise}}
      \onslide*<3>{\mintobjdump[firstline=5,lastline=35,highlightlines=31]{../src/backtrace/camlBacktrace\_\_probably\_raise_351.objdump}}
    \end{column}
  \end{columns}
  \only<1>{\footnotetext[1]{https://blog.janestreet.com/pattern-matching-and-exception-handling-unite/}}
\end{frame}

\newcommand\listCamlReraiseExn[1][]{
  \listasm[firstline=727,lastline=734,#1]{../ocaml/runtime/amd64.S}{runtime/amd64.S:727}}

\begin{frame}{Backtraces}{Introducing \funcname{caml\_reraise\_exn}}
  \begin{columns}[c]
    \begin{column}{0.5\textwidth}
      \minipage[c][0.66\textheight][s]{\columnwidth}
      \begin{itemize}
        \item<1-> The exception have to be reraised to the next handler
        \item<2-> First, checks if the backtrace should be collected with \funcarg{domain\_state->backtrace\_active}
        \only<3>{
        \begin{itemize}
            \item If not, behaves like a \funcname{raise\_notrace} with \funcname{RESTORE\_EXN\_HANDLER\_OCAML}
        \end{itemize}
        }
        \item<4-> Jumps into \funcname{caml\_raise\_exn}
        \item<5-> Calls \funcname{caml\_stash\_backtrace} again, to scan the next part of the stack: from the trap just pop-ed to the next trap
      \end{itemize}
      \vfill
      \mintocaml[firstline=15,lastline=21,highlightlines={18-21}]{../src/backtrace/backtrace.ml}
      \endminipage
    \end{column}
    \begin{column}{0.5\textwidth}
      \raggedleft
      \onslide*<1>{\listCamlReraiseExn{}}
      \onslide*<2>{\listCamlReraiseExn[highlightlines=729]{}}
      \onslide*<3>{
        \mintc[firstline=190,lastline=192,highlightlines={191-192}]{../ocaml/runtime/amd64.S}
        \mintc[firstline=234,lastline=237,highlightlines={235-237}]{../ocaml/runtime/amd64.S}
        \medskip
        \listCamlReraiseExn[highlightlines=731]{}
      }
      \onslide*<4-5>{
        % TODO refactor with \listCamlReraiseExn
        \mintasm[firstline=727,lastline=734,highlightlines=730]{../ocaml/runtime/amd64.S}
        \medskip
      }
      \onslide*<4>{\listCamlRaiseExn[highlightlines=711]{}}
      \onslide*<5>{\listCamlRaiseExn[highlightlines=720]{}}
    \end{column}
  \end{columns}
\end{frame}

\begin{frame}{Backtraces}{Backtrace collection continues}
  \begin{columns}[c]
    \begin{column}{0.55\textwidth}
      \minipage[c][0.75\textheight][s]{\columnwidth}
      \begin{itemize}
        \item<1-> \funcname{caml\_stash\_backtrace} scans the older part of the stack, up to the next trap
        \item<6-> Controls jumps to the nested exception handler in \funcname{maybe\_raise}, which matches only on \typename{ExnB}
        \item<7-> \funcname{caml\_reraise\_exn} is called again, backtrace collection resumes with \funcname{caml\_stash\_backtrace}
      \end{itemize}
      \medskip
      \begin{itemize}
        \item<2-> Constructed backtrace:
        \begin{itemize}
          \item <2-> \funcname{definitely\_raise}
          \item <2-> \funcname{probably\_raise}
          \item <3-> \funcname{probably\_raise} (re-raise)
          \item <4-> \funcname{maybe\_raise}
          \item <5-> \funcname{main}
          \item <8-> \funcname{main} (re-raise)
        \end{itemize}
      \end{itemize}
      \vfill
      \onslide*<1-5>{\mintocaml[firstline=15,lastline=21]{../src/backtrace/backtrace.ml}}
      \onslide*<6>{\mintocaml[firstline=28,lastline=34,highlightlines=31]{../src/backtrace/backtrace.ml}}
      \onslide*<7->{\mintocaml[firstline=28,lastline=34,highlightlines=33]{../src/backtrace/backtrace.ml}}
      \endminipage
    \end{column}
    \begin{column}{0.45\textwidth}
      \raggedleft
      \onslide*<1>{\providecommand\step{6}\input{diagrams/backtrace/backtrace.tex}}%
      \onslide*<2>{\providecommand\step{6}\input{diagrams/backtrace/backtrace.tex}}%
      \onslide*<3>{\providecommand\step{7}\input{diagrams/backtrace/backtrace.tex}}%
      \onslide*<4>{\providecommand\step{8}\input{diagrams/backtrace/backtrace.tex}}%
      \onslide*<5>{\providecommand\step{9}\input{diagrams/backtrace/backtrace.tex}}%
      \onslide*<6>{\providecommand\step{10}\input{diagrams/backtrace/backtrace.tex}}%
      \onslide*<7>{\providecommand\step{11}\input{diagrams/backtrace/backtrace.tex}}%
      \onslide*<8>{\providecommand\step{12}\input{diagrams/backtrace/backtrace.tex}}%
      \onslide*<9>{\providecommand\step{13}\input{diagrams/backtrace/backtrace.tex}}%
    \end{column}
  \end{columns}
\end{frame}

\begin{frame}{Backtraces}{Printing the backtrace}
  \begin{columns}[c]
    \begin{column}{0.45\textwidth}
      \minipage[c][0.66\textheight][s]{\columnwidth}
      \begin{itemize}
        \item<1-> The exception backtrace can now be printed to \funcarg{stdout} with \funcname{Printexc.print_backtrace}
        \item<2-> First the exception backtrace is retrieved into an OCaml array with \funcname{get_raw_backtrace }
        \item<3-> Which is implemented as the \funcname{caml_get_exception_raw_backtrace} C function in the runtime
        \item<4-> And finally printed with \funcname{print_exception_backtrace}
      \end{itemize}
      \vfill
      \mintocaml[firstline=28,lastline=34,highlightlines=33]{../src/backtrace/backtrace.ml}
      \endminipage
    \end{column}
    \begin{column}{0.55\textwidth}
      \onslide*<1,2>{
        \mintocaml[firstline=100,lastline=101]{../ocaml/stdlib/printexc.ml}
      }
      \only<1>{
        \listocaml[firstline=170,lastline=172]{../ocaml/stdlib/printexc.ml}{stdlib/printexc.ml:170}
      }
      \only<2>{
        \listocaml[firstline=170,lastline=172,highlightlines=172]{../ocaml/stdlib/printexc.ml}{stdlib/printexc.ml:170}
      }
      \only<3>{
        \mintc[firstline=154,lastline=156]{../ocaml/runtime/backtrace.c}
        \listc[firstline=171,lastline=192,highlightlines={184-187}]{../ocaml/runtime/backtrace.c}{runtime/backtrace.c:154}
      }
      \only<4>{
        \listocaml[firstline=155,lastline=165]{../ocaml/stdlib/printexc.ml}{stdlib/printexc.ml:55}
      }
    \end{column}
  \end{columns}
\end{frame}


\subsection{The multiple kinds of raise}
\frameSubsection{}{}

\begin{frame}{Backtraces}{When backtraces are not needed}
  \begin{columns}[c]
    \begin{column}{0.45\textwidth}
      \minipage[c][0.66\textheight][s]{\columnwidth}
      \begin{itemize}
        \item<1-> What if the backtrace for an exception is never needed?
        \item<2-> It's possible to raise the exception explicitly with \funcname{raise\_notrace} to make raising as cheap as possible
        \item<3-> An example of use in the \funcname{dynlink} library of the OCaml compiler
      \end{itemize}
      \vfill
      \onslide<3->{
      \listocaml[firstline=205,lastline=209,highlightlines=207]{../ocaml/otherlibs/dynlink/dynlink\_compilerlibs/misc.ml}{otherlibs/dynlink/dynlink\_compilerlibs/misc.ml:205}
      }
      \endminipage
    \end{column}
    \begin{column}{0.55\textwidth}
    \only<4>{
      \mintcmm[highlightlines=3]{../src/notrace/camlNotrace\_\_fun_347.cmm}
      \listcmm{../src/notrace/camlNotrace\_\_all\_somes\_327.cmm}{all\_somes}
    }
    \onslide*<5->{
      \listobjdump[highlightlines={6-9}]{../src/notrace/camlNotrace\_\_fun_347.objdump}{anonymous function from \funcname{all\_somes}}
    }
    \end{column}
  \end{columns}
\end{frame}

\begin{frame}{Backtraces}{Summary table of the multiple kinds of raise}
  \begin{itemize}
    \item When debugging information is enabled and if the backtrace of an exception isn't needed, it's possible to raise with \funcname{raise\_notrace} for performance
    \item When debugging information is \emph{not} enabled, the compiler optimises \funcname{raise} calls to inline assembly (same effect as using \funcname{raise\_notrace})
  \end{itemize}
  \bigskip
  \centering
  \begin{tabular}{| l | c | c | c | c |}
    \hline
    \parbox{10em}{Debug information \\ (\funcarg{-g} flag)}  & \multicolumn{2}{|c|}{Disabled}                                     & \multicolumn{2}{|c|}{Enabled} \\
    \hline
    Raise kind                                               & \funcname{raise} & \funcname{raise\_notrace}                       & \funcname{raise} & \funcname{raise\_notrace} \\
    \hline
    Compilation                                              & \parbox{8em}{inline assembly\\ (optimization)} & inline assembly   & \funcname{caml\_raise\_exn} & inline assembly \\
    \hline
  \end{tabular}
\end{frame}


%
% Takeaway #5
%
\subsection*{Takeaway \#5}
\frameSubsectionTakeaway{}
\againframe<5>{takeaway}
}%
      \onslide*<4>{\providecommand\step{8}%
% Backtraces
%
\subsection{One advantage of raising with debugging information}
\frameSubsection{}{}

\begin{frame}{Backtraces}{Debugging information \& source code}
  \begin{columns}[c]
    \begin{column}{0.5\textwidth}
      \begin{itemize}
        \item Raising an exception is fun and games, but how do we know where the exception originated? $\rightarrow$ With the backtrace associated with the exception!
        \item In order to get a backtrace from the point where an exception was raised, it's necessary to compile with debugging information enabled (\funcarg{-g} flag for the compiler)
        \item Same example as in the `Nested handlers' section
      \end{itemize}
    \end{column}
    \begin{column}{0.5\textwidth}
      \listocaml[firstline=9,lastline=34]{../src/backtrace/backtrace.ml}{backtrace/backtrace.ml}
    \end{column}
  \end{columns}
\end{frame}

\begin{frame}{Backtraces}{CMM and disassembly of \funcname{definitely\_raise}}
  \begin{columns}[c]
    \begin{column}{0.5\textwidth}
      \minipage[c][0.66\textheight][s]{\columnwidth}
      \begin{itemize}
        \item<1-> An effect of compiling with debugging information (\funcarg{-g}): the CMM no longer uses \funcname{raise\_notrace} to raise the exception but \funcname{raise} instead
        \item<2-> Disassembly shows that CMM \funcname{raise} is actually a call to \funcname{caml\_raise\_exn}, a function in assembler provided by the runtime
      \end{itemize}
      \vfill
      \mintocaml[firstline=9,lastline=13,highlightlines=12]{../src/backtrace/backtrace.ml}
      \endminipage
    \end{column}
    \begin{column}{0.5\textwidth}
      \onslide*<1>{
        \mintcmm[highlightlines=5]{../src/backtrace/camlBacktrace\_\_definitely\_raise\_347.cmm}
      }
      \onslide*<2>{
        \mintobjdump[firstline=2,highlightlines=14]{../src/backtrace/camlBacktrace\_\_definitely\_raise\_347.objdump}
      }
    \end{column}
  \end{columns}
\end{frame}

\begin{frame}{Backtraces}{Introducing \funcname{caml\_raise\_exn}}
  \begin{columns}[c]
    \begin{column}{0.5\textwidth}
      \minipage[c][0.66\textheight][s]{\columnwidth}
      \onslide*<1>{
        \begin{itemize}
          \item Raising the exception is performed using \funcname{caml\_raise\_exn} which is
            \begin{itemize}
              \item An assembly function provided by the runtime
              \item Architecture specific
            \end{itemize}
          \item Let's see what it does differently from \funcname{raise\_notrace}\dots
          \item We are about to raise \typename{ExnC} with \funcname{caml\_raise\_exn} from \funcname{definitely\_raise}
        \end{itemize}
      }
      \onslide*<2>{
        \mintobjdump[firstline=2,highlightlines=14]{../src/backtrace/camlBacktrace\_\_definitely\_raise\_347.objdump}
      }
      \vfill
      \mintocaml[firstline=9,lastline=13,highlightlines=12]{../src/backtrace/backtrace.ml}
      \endminipage
    \end{column}
    \begin{column}{0.5\textwidth}
      \centering
      \providecommand\step{1}\input{diagrams/backtrace/backtrace.tex}
    \end{column}
  \end{columns}
\end{frame}

\begin{frame}{Backtraces}{But before that, we need more fields from \typename{caml\_domain\_state}}
  \begin{columns}
    \begin{column}{0.5\textwidth}
      \begin{itemize}
        \item Remember \typename{caml\_domain\_state} the per-domain runtime state?
        \item Some other members are of interest to us now\dots
        \item \localname{backtrace\_active} is used to know if the runtime should collect backtraces
        \item \localname{backtrace\_active} can be set by
          \begin{itemize}
            \item The \funcarg{b} flag in the \funcname{OCAMLRUNPARAM} environment variable
            \item \mintinline{ocaml}{Printexc.record_backtrace : bool -> unit} \footnotemark
          \end{itemize}
      \end{itemize}
    \end{column}
    \begin{column}{0.5\textwidth}
      \begin{listing}
        \mintc[highlightlines={9-11}]{src/ocaml/domain\_state\_backtrace.h}
        \caption{runtime/caml/domain\_state.h}
      \end{listing}
    \end{column}
  \end{columns}
  \footnotetext[1]{https://v2.ocaml.org/api/Printexc.html}
\end{frame}

\newcommand\listCamlRaiseExn[1][]{
  \listasm[firstline=702,lastline=725,#1]{../ocaml/runtime/amd64.S}{runtime/amd64.S:702}}

\begin{frame}{Backtraces}{Walk through \funcname{caml\_raise\_exn}}
  \begin{columns}[T]
    \begin{column}{0.5\textwidth}
      \begin{itemize}
        \item<1-> First checks whether exceptions backtrace should be recorded
        \item<1-> By checking the \funcarg{domain\_state->backtrace\_active} value
        \item<2-> If not, acts as \funcname{raise\_notrace} with \funcname{RESTORE\_EXN\_HANDLER\_OCAML}
          \begin{itemize}
            \item Set \regname{RSP} to \localname{domain\_state->exn\_handler} value
            \item Restore \localname{domain\_state->exn\_handler} from trap
            \item Jump with \funcname{ret} (same effect as using \funcname{pop} and \funcname{jmp})
          \end{itemize}
        \item<3-> Otherwise, switches to the C stack to be able to execute C code
        \item<4-> Calls \funcname{caml\_stash\_backtrace}, a C function provided by the runtime\ignorespaces
          \onslide<6->{, with
            \begin{itemize}
              \item \funcarg{exn}: exception value
              \item \funcarg{pc}: code address of the raising point
              \item \funcarg{sp}: stack address of the raising function
              \item \funcarg{trapsp}: stack address of the next handler
            \end{itemize}
          }
      \end{itemize}
      \bigskip
      \mintocaml[firstline=9,lastline=13,highlightlines=12]{../src/backtrace/backtrace.ml}
    \end{column}
    \begin{column}{0.5\textwidth}
      \raggedleft
      \onslide*<1,3-4>{
        \mintc[firstline=190,lastline=192]{../ocaml/runtime/amd64.S}
        \mintc[firstline=234,lastline=237]{../ocaml/runtime/amd64.S}
      }
      \onslide*<2>{
        \mintc[firstline=190,lastline=192,highlightlines={191-192}]{../ocaml/runtime/amd64.S}
        \mintc[firstline=234,lastline=237,highlightlines={235-237}]{../ocaml/runtime/amd64.S}
      }
      \onslide*<1>{\listCamlRaiseExn[highlightlines=705]{}}%
      \onslide*<2>{\listCamlRaiseExn[highlightlines={707-708}]{}}%
      \onslide*<3>{\listCamlRaiseExn[highlightlines={712-714}]{}}%
      \onslide*<4>{\listCamlRaiseExn[highlightlines={715-720}]{}}%
      \onslide*<5>{\providecommand\step{1}\input{diagrams/backtrace/backtrace.tex}}%
      \onslide*<6>{\providecommand\step{2}\input{diagrams/backtrace/backtrace.tex}}%
    \end{column}
  \end{columns}
\end{frame}

\newcommand\listStashBacktrace[1][]{
  \listc[firstline=91,lastline=120,#1]{../ocaml/runtime/backtrace\_nat.c}{runtime/backtrace\_nat.c:91}}

\begin{frame}{Backtraces}{Introducing \funcname{caml\_stash\_backtrace}}
  \begin{columns}[c]
    \begin{column}{0.5\textwidth}
      \minipage[c][0.66\textheight][s]{\columnwidth}
      \begin{itemize}
        \item<1-> A C function provided by the runtime (specific to native code)
        \item<2-> It scans the stack, between the stack pointer at the raising point up to the next trap, in order to collect the names of the detected function frames
          \onslide*<2>{
            \begin{itemize}
              \item And more information, like line number, column number...
            \end{itemize}
          }
        \item<3-> It uses the same technique and information as the Garbage Collector (GC) uses to scan the values live on the stack
          \onslide*<4>{
            \begin{itemize}
              \item \typename{frame\_descr} are at the core of stack unwinding
            \end{itemize}
          }
      \end{itemize}
      \vfill
      \mintocaml[firstline=9,lastline=13,highlightlines=12]{../src/backtrace/backtrace.ml}
      \endminipage
    \end{column}
    \begin{column}{0.5\textwidth}
      \onslide*<1>{\listStashBacktrace[highlightlines=91]{}}
      \onslide*<2>{\listStashBacktrace[highlightlines={91,117-118}]{}}
      \onslide*<3->{\listStashBacktrace[highlightlines={94,106,109-110}]{}}
    \end{column}
  \end{columns}
\end{frame}

\newcommand\listFrameDescr[1][]{
  \listocaml[firstline=111,lastline=124,#1]{../ocaml/asmcomp/emitaux.ml}{asmcomp/emitaux.ml:111}}
\newcommand\listDebugInfo[1][]{
  \listocaml[firstline=38,lastline=49,#1]{../ocaml/lambda/debuginfo.mli}{lambda/debuginfo.mli:38}}

\begin{frame}{Backtraces}{\typename{frame\_descr} and \typename{frame\_debuginfo} in a nutshell}
  \begin{columns}[c]
    \begin{column}{0.5\textwidth}
      \begin{itemize}
        \item<1-> For every return address in an OCaml program, there is a associated \typename{frame\_descr}
        \item<2-> A \typename{frame\_descr} specifies
        \begin{itemize}
          \item<2-> The size of the function stack frame
          \item<3-> Where live values are on the stack (so that the GC can mark them as live and prevent from being swept)
        \end{itemize}
        \item<4-> When compiled with debug information, \typename{frame\_descr} will be linked to a \typename{frame\_debuginfo}
        \begin{itemize}
          \item<5-> Contains information about the position in the source code (file name, line and column number)
        \end{itemize}
      \end{itemize}
    \end{column}
    \begin{column}{0.5\textwidth}
        \onslide*<1>{\listFrameDescr[highlightlines={118-122}]{}}
        \onslide*<2>{\listFrameDescr[highlightlines={120}]{}}
        \onslide*<3>{\listFrameDescr[highlightlines={121}]{}}
        \onslide*<4->{\listFrameDescr[highlightlines={122}]{}}
        \medskip
        \onslide*<1-3>{\listDebugInfo{}}
        \onslide*<4>{\listDebugInfo[highlightlines={38,49}]{}}
        \onslide*<5>{\listDebugInfo[highlightlines={39-46}]{}}
    \end{column}
  \end{columns}
\end{frame}

\begin{frame}{Backtraces}{Scanning the stack with \typename{frame\_descr}}
  \begin{columns}[c]
    \begin{column}{0.5\textwidth}
      \begin{itemize}
        \item Given the return address of the code that raises the exception by calling \funcname{caml\_raise\_exn} (\funcarg{pc} argument of \funcname{caml\_stash\_backtrace})
        \item And the position of the stack pointer (\funcarg{sp} argument of \funcname{caml\_stash\_backtrace})
        \item The frame size given by \typename{frame\_descr}.\funcarg{frame\_size}, allows to find the end of the previous frame
        \item And the return address of the caller of this function
        \bigskip
        \onslide<3->{
        \item Note that traps (here the reddish \funcname{ExnA}, \funcname{ExnB} and \funcname{ExnC} blocks) belong to the frame that installed them, and so the frame size changes when traps are removed
        }
      \end{itemize}
    \end{column}
    \begin{column}{0.5\textwidth}
      \raggedleft
      \onslide*<1>{\providecommand\step{2}\input{diagrams/backtrace/backtrace.tex}}%
      \onslide*<2>{\providecommand\step{1}\input{diagrams/backtrace/scanstack.tex}}%
      \onslide*<3>{\providecommand\step{2}\input{diagrams/backtrace/scanstack.tex}}%
      \onslide*<4>{\providecommand\step{3}\input{diagrams/backtrace/scanstack.tex}}%
      \onslide*<5>{\providecommand\step{4}\input{diagrams/backtrace/scanstack.tex}}%
      \onslide*<6>{\providecommand\step{5}\input{diagrams/backtrace/scanstack.tex}}%
    \end{column}
  \end{columns}
\end{frame}

% Duplicate of previous-previous slide
\begin{frame}{Backtraces}{Continuing on \funcname{caml\_stash\_backtrace}}
  \begin{columns}[c]
    \begin{column}{0.5\textwidth}
      \minipage[c][0.75\textheight][s]{\columnwidth}
      \begin{itemize}
          \color{gray}
        \item A C function provided by the runtime (specific to native code)
        \item It scans the stack, between the stack pointer at the raising point up to the next trap, in order to collect the names of the detected function frames
        \item It uses the same technique and information as the Garbage Collector (GC) uses to scan the values live on the stack
      \end{itemize}
      \begin{itemize}
        \item For each function frame detected, store a pointer to the \typename{frame\_descr} in the \localname{domain\_state->backtrace\_buffer} array, effectively constructing the backtrace
      \end{itemize}
      \onslide<2->{
        \begin{itemize}
            \smallskip
          \item Constructed backtrace:
            \funcname{definitely\_raise},
            \onslide<3->{\funcname{probably\_raise}}
        \end{itemize}
      }
      \vfill
      \mintocaml[firstline=9,lastline=13,highlightlines=12]{../src/backtrace/backtrace.ml}
      \endminipage
    \end{column}
    \begin{column}{0.5\textwidth}
      \raggedleft
      \onslide*<1>{\listStashBacktrace[highlightlines={114-115}]{}}
      \onslide*<2>{\providecommand\step{3}\input{diagrams/backtrace/backtrace.tex}}%
      \onslide*<3>{\providecommand\step{4}\input{diagrams/backtrace/backtrace.tex}}%
    \end{column}
  \end{columns}
\end{frame}

\begin{frame}{Backtraces}{Back to \funcname{caml\_raise\_exn}}
  \begin{columns}[c]
    \begin{column}{0.5\textwidth}
      \minipage[c][0.66\textheight][s]{\columnwidth}
      \begin{itemize}\color{gray}
        \item Checks wheteher exceptions backtrace should be recorded, by checking the \funcarg{domain\_state->backtrace\_active} value
        \item Switches to the C stack to be able to execute C code
        \item Calls \funcname{caml\_stash\_backtrace}, to collect the backtrace up to the next trap
      \end{itemize}
      \onslide*<2->{
        \begin{itemize}
          \item<2-> Executes the exception handler in \funcname{probably\_raise} with \funcname{RESTORE\_EXN\_HANDLER\_OCAML}
            \begin{itemize}
              \item Set \regname{RSP} to \localname{domain\_state->exn\_handler} value
              \item Restore \localname{domain\_state->exn\_handler} from trap
              \item Jump to the handler code with \funcname{ret}
            \end{itemize}
        \end{itemize}
      }
      \vfill
      \onslide*<1>{\mintocaml[firstline=9,lastline=13,highlightlines=12]{../src/backtrace/backtrace.ml}}
      \onslide*<2->{\mintocaml[firstline=15,lastline=21,highlightlines={19-21}]{../src/backtrace/backtrace.ml}}
      \endminipage
    \end{column}
    \begin{column}{0.5\textwidth}
      \centering
      \onslide*<1>{
        \mintc[firstline=190,lastline=192]{../ocaml/runtime/amd64.S}
        \mintc[firstline=234,lastline=237]{../ocaml/runtime/amd64.S}
      }
      \onslide*<2>{
        \mintc[firstline=190,lastline=192,highlightlines={191-192}]{../ocaml/runtime/amd64.S}
        \mintc[firstline=234,lastline=237,highlightlines={235-237}]{../ocaml/runtime/amd64.S}
      }
      \onslide*<1>{\listCamlRaiseExn[highlightlines=720]{}}
      \onslide*<2>{\listCamlRaiseExn[highlightlines=722]{}}
      \onslide*<3->{\providecommand\step{5}\input{diagrams/backtrace/backtrace.tex}}
    \end{column}
  \end{columns}
\end{frame}

\begin{frame}{Backtraces}{\funcname{caml\_probably\_raise}'s handler doesn't catch \typename{ExnC}}
  \begin{columns}[c]
    \begin{column}{0.45\textwidth}
      \minipage[c][0.75\textheight][s]{\columnwidth}
      \begin{itemize}
        \item<1-> \funcname{match \dots with}\footnotemark pattern matching can also match exceptions
        \item<1-> The CMM of \funcname{probably\_raise} shows that this \funcname{match \dots with} is implemented with a pattern matching and a \funcname{try \dots with}
        \item<1-> But this one only matches on \typename{ExnA} exception
        \item<2-> Since the pattern matching fails to catch the exception, \funcname{reraise} is called with our current \typename{ExnC} exception
        \item<3-> Disassembly shows that \funcname{reraise} is actually a call to \funcname{caml\_reraise\_exn}, which is also an assembly function provided by the runtime
      \end{itemize}
      \vfill
      \mintocaml[firstline=15,lastline=21,highlightlines={18-21}]{../src/backtrace/backtrace.ml}
      \endminipage
    \end{column}
    \begin{column}{0.55\textwidth}
      \centering
      \onslide*<1>{\mintcmm[highlightlines={10-18}]{../src/backtrace/camlBacktrace\_\_probably\_raise\_351.cmm}}
      \onslide*<2>{\listcmm[highlightlines=18]{../src/backtrace/camlBacktrace\_\_probably\_raise_351.cmm}{CMM of probably\_raise}}
      \onslide*<3>{\mintobjdump[firstline=5,lastline=35,highlightlines=31]{../src/backtrace/camlBacktrace\_\_probably\_raise_351.objdump}}
    \end{column}
  \end{columns}
  \only<1>{\footnotetext[1]{https://blog.janestreet.com/pattern-matching-and-exception-handling-unite/}}
\end{frame}

\newcommand\listCamlReraiseExn[1][]{
  \listasm[firstline=727,lastline=734,#1]{../ocaml/runtime/amd64.S}{runtime/amd64.S:727}}

\begin{frame}{Backtraces}{Introducing \funcname{caml\_reraise\_exn}}
  \begin{columns}[c]
    \begin{column}{0.5\textwidth}
      \minipage[c][0.66\textheight][s]{\columnwidth}
      \begin{itemize}
        \item<1-> The exception have to be reraised to the next handler
        \item<2-> First, checks if the backtrace should be collected with \funcarg{domain\_state->backtrace\_active}
        \only<3>{
        \begin{itemize}
            \item If not, behaves like a \funcname{raise\_notrace} with \funcname{RESTORE\_EXN\_HANDLER\_OCAML}
        \end{itemize}
        }
        \item<4-> Jumps into \funcname{caml\_raise\_exn}
        \item<5-> Calls \funcname{caml\_stash\_backtrace} again, to scan the next part of the stack: from the trap just pop-ed to the next trap
      \end{itemize}
      \vfill
      \mintocaml[firstline=15,lastline=21,highlightlines={18-21}]{../src/backtrace/backtrace.ml}
      \endminipage
    \end{column}
    \begin{column}{0.5\textwidth}
      \raggedleft
      \onslide*<1>{\listCamlReraiseExn{}}
      \onslide*<2>{\listCamlReraiseExn[highlightlines=729]{}}
      \onslide*<3>{
        \mintc[firstline=190,lastline=192,highlightlines={191-192}]{../ocaml/runtime/amd64.S}
        \mintc[firstline=234,lastline=237,highlightlines={235-237}]{../ocaml/runtime/amd64.S}
        \medskip
        \listCamlReraiseExn[highlightlines=731]{}
      }
      \onslide*<4-5>{
        % TODO refactor with \listCamlReraiseExn
        \mintasm[firstline=727,lastline=734,highlightlines=730]{../ocaml/runtime/amd64.S}
        \medskip
      }
      \onslide*<4>{\listCamlRaiseExn[highlightlines=711]{}}
      \onslide*<5>{\listCamlRaiseExn[highlightlines=720]{}}
    \end{column}
  \end{columns}
\end{frame}

\begin{frame}{Backtraces}{Backtrace collection continues}
  \begin{columns}[c]
    \begin{column}{0.55\textwidth}
      \minipage[c][0.75\textheight][s]{\columnwidth}
      \begin{itemize}
        \item<1-> \funcname{caml\_stash\_backtrace} scans the older part of the stack, up to the next trap
        \item<6-> Controls jumps to the nested exception handler in \funcname{maybe\_raise}, which matches only on \typename{ExnB}
        \item<7-> \funcname{caml\_reraise\_exn} is called again, backtrace collection resumes with \funcname{caml\_stash\_backtrace}
      \end{itemize}
      \medskip
      \begin{itemize}
        \item<2-> Constructed backtrace:
        \begin{itemize}
          \item <2-> \funcname{definitely\_raise}
          \item <2-> \funcname{probably\_raise}
          \item <3-> \funcname{probably\_raise} (re-raise)
          \item <4-> \funcname{maybe\_raise}
          \item <5-> \funcname{main}
          \item <8-> \funcname{main} (re-raise)
        \end{itemize}
      \end{itemize}
      \vfill
      \onslide*<1-5>{\mintocaml[firstline=15,lastline=21]{../src/backtrace/backtrace.ml}}
      \onslide*<6>{\mintocaml[firstline=28,lastline=34,highlightlines=31]{../src/backtrace/backtrace.ml}}
      \onslide*<7->{\mintocaml[firstline=28,lastline=34,highlightlines=33]{../src/backtrace/backtrace.ml}}
      \endminipage
    \end{column}
    \begin{column}{0.45\textwidth}
      \raggedleft
      \onslide*<1>{\providecommand\step{6}\input{diagrams/backtrace/backtrace.tex}}%
      \onslide*<2>{\providecommand\step{6}\input{diagrams/backtrace/backtrace.tex}}%
      \onslide*<3>{\providecommand\step{7}\input{diagrams/backtrace/backtrace.tex}}%
      \onslide*<4>{\providecommand\step{8}\input{diagrams/backtrace/backtrace.tex}}%
      \onslide*<5>{\providecommand\step{9}\input{diagrams/backtrace/backtrace.tex}}%
      \onslide*<6>{\providecommand\step{10}\input{diagrams/backtrace/backtrace.tex}}%
      \onslide*<7>{\providecommand\step{11}\input{diagrams/backtrace/backtrace.tex}}%
      \onslide*<8>{\providecommand\step{12}\input{diagrams/backtrace/backtrace.tex}}%
      \onslide*<9>{\providecommand\step{13}\input{diagrams/backtrace/backtrace.tex}}%
    \end{column}
  \end{columns}
\end{frame}

\begin{frame}{Backtraces}{Printing the backtrace}
  \begin{columns}[c]
    \begin{column}{0.45\textwidth}
      \minipage[c][0.66\textheight][s]{\columnwidth}
      \begin{itemize}
        \item<1-> The exception backtrace can now be printed to \funcarg{stdout} with \funcname{Printexc.print_backtrace}
        \item<2-> First the exception backtrace is retrieved into an OCaml array with \funcname{get_raw_backtrace }
        \item<3-> Which is implemented as the \funcname{caml_get_exception_raw_backtrace} C function in the runtime
        \item<4-> And finally printed with \funcname{print_exception_backtrace}
      \end{itemize}
      \vfill
      \mintocaml[firstline=28,lastline=34,highlightlines=33]{../src/backtrace/backtrace.ml}
      \endminipage
    \end{column}
    \begin{column}{0.55\textwidth}
      \onslide*<1,2>{
        \mintocaml[firstline=100,lastline=101]{../ocaml/stdlib/printexc.ml}
      }
      \only<1>{
        \listocaml[firstline=170,lastline=172]{../ocaml/stdlib/printexc.ml}{stdlib/printexc.ml:170}
      }
      \only<2>{
        \listocaml[firstline=170,lastline=172,highlightlines=172]{../ocaml/stdlib/printexc.ml}{stdlib/printexc.ml:170}
      }
      \only<3>{
        \mintc[firstline=154,lastline=156]{../ocaml/runtime/backtrace.c}
        \listc[firstline=171,lastline=192,highlightlines={184-187}]{../ocaml/runtime/backtrace.c}{runtime/backtrace.c:154}
      }
      \only<4>{
        \listocaml[firstline=155,lastline=165]{../ocaml/stdlib/printexc.ml}{stdlib/printexc.ml:55}
      }
    \end{column}
  \end{columns}
\end{frame}


\subsection{The multiple kinds of raise}
\frameSubsection{}{}

\begin{frame}{Backtraces}{When backtraces are not needed}
  \begin{columns}[c]
    \begin{column}{0.45\textwidth}
      \minipage[c][0.66\textheight][s]{\columnwidth}
      \begin{itemize}
        \item<1-> What if the backtrace for an exception is never needed?
        \item<2-> It's possible to raise the exception explicitly with \funcname{raise\_notrace} to make raising as cheap as possible
        \item<3-> An example of use in the \funcname{dynlink} library of the OCaml compiler
      \end{itemize}
      \vfill
      \onslide<3->{
      \listocaml[firstline=205,lastline=209,highlightlines=207]{../ocaml/otherlibs/dynlink/dynlink\_compilerlibs/misc.ml}{otherlibs/dynlink/dynlink\_compilerlibs/misc.ml:205}
      }
      \endminipage
    \end{column}
    \begin{column}{0.55\textwidth}
    \only<4>{
      \mintcmm[highlightlines=3]{../src/notrace/camlNotrace\_\_fun_347.cmm}
      \listcmm{../src/notrace/camlNotrace\_\_all\_somes\_327.cmm}{all\_somes}
    }
    \onslide*<5->{
      \listobjdump[highlightlines={6-9}]{../src/notrace/camlNotrace\_\_fun_347.objdump}{anonymous function from \funcname{all\_somes}}
    }
    \end{column}
  \end{columns}
\end{frame}

\begin{frame}{Backtraces}{Summary table of the multiple kinds of raise}
  \begin{itemize}
    \item When debugging information is enabled and if the backtrace of an exception isn't needed, it's possible to raise with \funcname{raise\_notrace} for performance
    \item When debugging information is \emph{not} enabled, the compiler optimises \funcname{raise} calls to inline assembly (same effect as using \funcname{raise\_notrace})
  \end{itemize}
  \bigskip
  \centering
  \begin{tabular}{| l | c | c | c | c |}
    \hline
    \parbox{10em}{Debug information \\ (\funcarg{-g} flag)}  & \multicolumn{2}{|c|}{Disabled}                                     & \multicolumn{2}{|c|}{Enabled} \\
    \hline
    Raise kind                                               & \funcname{raise} & \funcname{raise\_notrace}                       & \funcname{raise} & \funcname{raise\_notrace} \\
    \hline
    Compilation                                              & \parbox{8em}{inline assembly\\ (optimization)} & inline assembly   & \funcname{caml\_raise\_exn} & inline assembly \\
    \hline
  \end{tabular}
\end{frame}


%
% Takeaway #5
%
\subsection*{Takeaway \#5}
\frameSubsectionTakeaway{}
\againframe<5>{takeaway}
}%
      \onslide*<5>{\providecommand\step{9}%
% Backtraces
%
\subsection{One advantage of raising with debugging information}
\frameSubsection{}{}

\begin{frame}{Backtraces}{Debugging information \& source code}
  \begin{columns}[c]
    \begin{column}{0.5\textwidth}
      \begin{itemize}
        \item Raising an exception is fun and games, but how do we know where the exception originated? $\rightarrow$ With the backtrace associated with the exception!
        \item In order to get a backtrace from the point where an exception was raised, it's necessary to compile with debugging information enabled (\funcarg{-g} flag for the compiler)
        \item Same example as in the `Nested handlers' section
      \end{itemize}
    \end{column}
    \begin{column}{0.5\textwidth}
      \listocaml[firstline=9,lastline=34]{../src/backtrace/backtrace.ml}{backtrace/backtrace.ml}
    \end{column}
  \end{columns}
\end{frame}

\begin{frame}{Backtraces}{CMM and disassembly of \funcname{definitely\_raise}}
  \begin{columns}[c]
    \begin{column}{0.5\textwidth}
      \minipage[c][0.66\textheight][s]{\columnwidth}
      \begin{itemize}
        \item<1-> An effect of compiling with debugging information (\funcarg{-g}): the CMM no longer uses \funcname{raise\_notrace} to raise the exception but \funcname{raise} instead
        \item<2-> Disassembly shows that CMM \funcname{raise} is actually a call to \funcname{caml\_raise\_exn}, a function in assembler provided by the runtime
      \end{itemize}
      \vfill
      \mintocaml[firstline=9,lastline=13,highlightlines=12]{../src/backtrace/backtrace.ml}
      \endminipage
    \end{column}
    \begin{column}{0.5\textwidth}
      \onslide*<1>{
        \mintcmm[highlightlines=5]{../src/backtrace/camlBacktrace\_\_definitely\_raise\_347.cmm}
      }
      \onslide*<2>{
        \mintobjdump[firstline=2,highlightlines=14]{../src/backtrace/camlBacktrace\_\_definitely\_raise\_347.objdump}
      }
    \end{column}
  \end{columns}
\end{frame}

\begin{frame}{Backtraces}{Introducing \funcname{caml\_raise\_exn}}
  \begin{columns}[c]
    \begin{column}{0.5\textwidth}
      \minipage[c][0.66\textheight][s]{\columnwidth}
      \onslide*<1>{
        \begin{itemize}
          \item Raising the exception is performed using \funcname{caml\_raise\_exn} which is
            \begin{itemize}
              \item An assembly function provided by the runtime
              \item Architecture specific
            \end{itemize}
          \item Let's see what it does differently from \funcname{raise\_notrace}\dots
          \item We are about to raise \typename{ExnC} with \funcname{caml\_raise\_exn} from \funcname{definitely\_raise}
        \end{itemize}
      }
      \onslide*<2>{
        \mintobjdump[firstline=2,highlightlines=14]{../src/backtrace/camlBacktrace\_\_definitely\_raise\_347.objdump}
      }
      \vfill
      \mintocaml[firstline=9,lastline=13,highlightlines=12]{../src/backtrace/backtrace.ml}
      \endminipage
    \end{column}
    \begin{column}{0.5\textwidth}
      \centering
      \providecommand\step{1}\input{diagrams/backtrace/backtrace.tex}
    \end{column}
  \end{columns}
\end{frame}

\begin{frame}{Backtraces}{But before that, we need more fields from \typename{caml\_domain\_state}}
  \begin{columns}
    \begin{column}{0.5\textwidth}
      \begin{itemize}
        \item Remember \typename{caml\_domain\_state} the per-domain runtime state?
        \item Some other members are of interest to us now\dots
        \item \localname{backtrace\_active} is used to know if the runtime should collect backtraces
        \item \localname{backtrace\_active} can be set by
          \begin{itemize}
            \item The \funcarg{b} flag in the \funcname{OCAMLRUNPARAM} environment variable
            \item \mintinline{ocaml}{Printexc.record_backtrace : bool -> unit} \footnotemark
          \end{itemize}
      \end{itemize}
    \end{column}
    \begin{column}{0.5\textwidth}
      \begin{listing}
        \mintc[highlightlines={9-11}]{src/ocaml/domain\_state\_backtrace.h}
        \caption{runtime/caml/domain\_state.h}
      \end{listing}
    \end{column}
  \end{columns}
  \footnotetext[1]{https://v2.ocaml.org/api/Printexc.html}
\end{frame}

\newcommand\listCamlRaiseExn[1][]{
  \listasm[firstline=702,lastline=725,#1]{../ocaml/runtime/amd64.S}{runtime/amd64.S:702}}

\begin{frame}{Backtraces}{Walk through \funcname{caml\_raise\_exn}}
  \begin{columns}[T]
    \begin{column}{0.5\textwidth}
      \begin{itemize}
        \item<1-> First checks whether exceptions backtrace should be recorded
        \item<1-> By checking the \funcarg{domain\_state->backtrace\_active} value
        \item<2-> If not, acts as \funcname{raise\_notrace} with \funcname{RESTORE\_EXN\_HANDLER\_OCAML}
          \begin{itemize}
            \item Set \regname{RSP} to \localname{domain\_state->exn\_handler} value
            \item Restore \localname{domain\_state->exn\_handler} from trap
            \item Jump with \funcname{ret} (same effect as using \funcname{pop} and \funcname{jmp})
          \end{itemize}
        \item<3-> Otherwise, switches to the C stack to be able to execute C code
        \item<4-> Calls \funcname{caml\_stash\_backtrace}, a C function provided by the runtime\ignorespaces
          \onslide<6->{, with
            \begin{itemize}
              \item \funcarg{exn}: exception value
              \item \funcarg{pc}: code address of the raising point
              \item \funcarg{sp}: stack address of the raising function
              \item \funcarg{trapsp}: stack address of the next handler
            \end{itemize}
          }
      \end{itemize}
      \bigskip
      \mintocaml[firstline=9,lastline=13,highlightlines=12]{../src/backtrace/backtrace.ml}
    \end{column}
    \begin{column}{0.5\textwidth}
      \raggedleft
      \onslide*<1,3-4>{
        \mintc[firstline=190,lastline=192]{../ocaml/runtime/amd64.S}
        \mintc[firstline=234,lastline=237]{../ocaml/runtime/amd64.S}
      }
      \onslide*<2>{
        \mintc[firstline=190,lastline=192,highlightlines={191-192}]{../ocaml/runtime/amd64.S}
        \mintc[firstline=234,lastline=237,highlightlines={235-237}]{../ocaml/runtime/amd64.S}
      }
      \onslide*<1>{\listCamlRaiseExn[highlightlines=705]{}}%
      \onslide*<2>{\listCamlRaiseExn[highlightlines={707-708}]{}}%
      \onslide*<3>{\listCamlRaiseExn[highlightlines={712-714}]{}}%
      \onslide*<4>{\listCamlRaiseExn[highlightlines={715-720}]{}}%
      \onslide*<5>{\providecommand\step{1}\input{diagrams/backtrace/backtrace.tex}}%
      \onslide*<6>{\providecommand\step{2}\input{diagrams/backtrace/backtrace.tex}}%
    \end{column}
  \end{columns}
\end{frame}

\newcommand\listStashBacktrace[1][]{
  \listc[firstline=91,lastline=120,#1]{../ocaml/runtime/backtrace\_nat.c}{runtime/backtrace\_nat.c:91}}

\begin{frame}{Backtraces}{Introducing \funcname{caml\_stash\_backtrace}}
  \begin{columns}[c]
    \begin{column}{0.5\textwidth}
      \minipage[c][0.66\textheight][s]{\columnwidth}
      \begin{itemize}
        \item<1-> A C function provided by the runtime (specific to native code)
        \item<2-> It scans the stack, between the stack pointer at the raising point up to the next trap, in order to collect the names of the detected function frames
          \onslide*<2>{
            \begin{itemize}
              \item And more information, like line number, column number...
            \end{itemize}
          }
        \item<3-> It uses the same technique and information as the Garbage Collector (GC) uses to scan the values live on the stack
          \onslide*<4>{
            \begin{itemize}
              \item \typename{frame\_descr} are at the core of stack unwinding
            \end{itemize}
          }
      \end{itemize}
      \vfill
      \mintocaml[firstline=9,lastline=13,highlightlines=12]{../src/backtrace/backtrace.ml}
      \endminipage
    \end{column}
    \begin{column}{0.5\textwidth}
      \onslide*<1>{\listStashBacktrace[highlightlines=91]{}}
      \onslide*<2>{\listStashBacktrace[highlightlines={91,117-118}]{}}
      \onslide*<3->{\listStashBacktrace[highlightlines={94,106,109-110}]{}}
    \end{column}
  \end{columns}
\end{frame}

\newcommand\listFrameDescr[1][]{
  \listocaml[firstline=111,lastline=124,#1]{../ocaml/asmcomp/emitaux.ml}{asmcomp/emitaux.ml:111}}
\newcommand\listDebugInfo[1][]{
  \listocaml[firstline=38,lastline=49,#1]{../ocaml/lambda/debuginfo.mli}{lambda/debuginfo.mli:38}}

\begin{frame}{Backtraces}{\typename{frame\_descr} and \typename{frame\_debuginfo} in a nutshell}
  \begin{columns}[c]
    \begin{column}{0.5\textwidth}
      \begin{itemize}
        \item<1-> For every return address in an OCaml program, there is a associated \typename{frame\_descr}
        \item<2-> A \typename{frame\_descr} specifies
        \begin{itemize}
          \item<2-> The size of the function stack frame
          \item<3-> Where live values are on the stack (so that the GC can mark them as live and prevent from being swept)
        \end{itemize}
        \item<4-> When compiled with debug information, \typename{frame\_descr} will be linked to a \typename{frame\_debuginfo}
        \begin{itemize}
          \item<5-> Contains information about the position in the source code (file name, line and column number)
        \end{itemize}
      \end{itemize}
    \end{column}
    \begin{column}{0.5\textwidth}
        \onslide*<1>{\listFrameDescr[highlightlines={118-122}]{}}
        \onslide*<2>{\listFrameDescr[highlightlines={120}]{}}
        \onslide*<3>{\listFrameDescr[highlightlines={121}]{}}
        \onslide*<4->{\listFrameDescr[highlightlines={122}]{}}
        \medskip
        \onslide*<1-3>{\listDebugInfo{}}
        \onslide*<4>{\listDebugInfo[highlightlines={38,49}]{}}
        \onslide*<5>{\listDebugInfo[highlightlines={39-46}]{}}
    \end{column}
  \end{columns}
\end{frame}

\begin{frame}{Backtraces}{Scanning the stack with \typename{frame\_descr}}
  \begin{columns}[c]
    \begin{column}{0.5\textwidth}
      \begin{itemize}
        \item Given the return address of the code that raises the exception by calling \funcname{caml\_raise\_exn} (\funcarg{pc} argument of \funcname{caml\_stash\_backtrace})
        \item And the position of the stack pointer (\funcarg{sp} argument of \funcname{caml\_stash\_backtrace})
        \item The frame size given by \typename{frame\_descr}.\funcarg{frame\_size}, allows to find the end of the previous frame
        \item And the return address of the caller of this function
        \bigskip
        \onslide<3->{
        \item Note that traps (here the reddish \funcname{ExnA}, \funcname{ExnB} and \funcname{ExnC} blocks) belong to the frame that installed them, and so the frame size changes when traps are removed
        }
      \end{itemize}
    \end{column}
    \begin{column}{0.5\textwidth}
      \raggedleft
      \onslide*<1>{\providecommand\step{2}\input{diagrams/backtrace/backtrace.tex}}%
      \onslide*<2>{\providecommand\step{1}\input{diagrams/backtrace/scanstack.tex}}%
      \onslide*<3>{\providecommand\step{2}\input{diagrams/backtrace/scanstack.tex}}%
      \onslide*<4>{\providecommand\step{3}\input{diagrams/backtrace/scanstack.tex}}%
      \onslide*<5>{\providecommand\step{4}\input{diagrams/backtrace/scanstack.tex}}%
      \onslide*<6>{\providecommand\step{5}\input{diagrams/backtrace/scanstack.tex}}%
    \end{column}
  \end{columns}
\end{frame}

% Duplicate of previous-previous slide
\begin{frame}{Backtraces}{Continuing on \funcname{caml\_stash\_backtrace}}
  \begin{columns}[c]
    \begin{column}{0.5\textwidth}
      \minipage[c][0.75\textheight][s]{\columnwidth}
      \begin{itemize}
          \color{gray}
        \item A C function provided by the runtime (specific to native code)
        \item It scans the stack, between the stack pointer at the raising point up to the next trap, in order to collect the names of the detected function frames
        \item It uses the same technique and information as the Garbage Collector (GC) uses to scan the values live on the stack
      \end{itemize}
      \begin{itemize}
        \item For each function frame detected, store a pointer to the \typename{frame\_descr} in the \localname{domain\_state->backtrace\_buffer} array, effectively constructing the backtrace
      \end{itemize}
      \onslide<2->{
        \begin{itemize}
            \smallskip
          \item Constructed backtrace:
            \funcname{definitely\_raise},
            \onslide<3->{\funcname{probably\_raise}}
        \end{itemize}
      }
      \vfill
      \mintocaml[firstline=9,lastline=13,highlightlines=12]{../src/backtrace/backtrace.ml}
      \endminipage
    \end{column}
    \begin{column}{0.5\textwidth}
      \raggedleft
      \onslide*<1>{\listStashBacktrace[highlightlines={114-115}]{}}
      \onslide*<2>{\providecommand\step{3}\input{diagrams/backtrace/backtrace.tex}}%
      \onslide*<3>{\providecommand\step{4}\input{diagrams/backtrace/backtrace.tex}}%
    \end{column}
  \end{columns}
\end{frame}

\begin{frame}{Backtraces}{Back to \funcname{caml\_raise\_exn}}
  \begin{columns}[c]
    \begin{column}{0.5\textwidth}
      \minipage[c][0.66\textheight][s]{\columnwidth}
      \begin{itemize}\color{gray}
        \item Checks wheteher exceptions backtrace should be recorded, by checking the \funcarg{domain\_state->backtrace\_active} value
        \item Switches to the C stack to be able to execute C code
        \item Calls \funcname{caml\_stash\_backtrace}, to collect the backtrace up to the next trap
      \end{itemize}
      \onslide*<2->{
        \begin{itemize}
          \item<2-> Executes the exception handler in \funcname{probably\_raise} with \funcname{RESTORE\_EXN\_HANDLER\_OCAML}
            \begin{itemize}
              \item Set \regname{RSP} to \localname{domain\_state->exn\_handler} value
              \item Restore \localname{domain\_state->exn\_handler} from trap
              \item Jump to the handler code with \funcname{ret}
            \end{itemize}
        \end{itemize}
      }
      \vfill
      \onslide*<1>{\mintocaml[firstline=9,lastline=13,highlightlines=12]{../src/backtrace/backtrace.ml}}
      \onslide*<2->{\mintocaml[firstline=15,lastline=21,highlightlines={19-21}]{../src/backtrace/backtrace.ml}}
      \endminipage
    \end{column}
    \begin{column}{0.5\textwidth}
      \centering
      \onslide*<1>{
        \mintc[firstline=190,lastline=192]{../ocaml/runtime/amd64.S}
        \mintc[firstline=234,lastline=237]{../ocaml/runtime/amd64.S}
      }
      \onslide*<2>{
        \mintc[firstline=190,lastline=192,highlightlines={191-192}]{../ocaml/runtime/amd64.S}
        \mintc[firstline=234,lastline=237,highlightlines={235-237}]{../ocaml/runtime/amd64.S}
      }
      \onslide*<1>{\listCamlRaiseExn[highlightlines=720]{}}
      \onslide*<2>{\listCamlRaiseExn[highlightlines=722]{}}
      \onslide*<3->{\providecommand\step{5}\input{diagrams/backtrace/backtrace.tex}}
    \end{column}
  \end{columns}
\end{frame}

\begin{frame}{Backtraces}{\funcname{caml\_probably\_raise}'s handler doesn't catch \typename{ExnC}}
  \begin{columns}[c]
    \begin{column}{0.45\textwidth}
      \minipage[c][0.75\textheight][s]{\columnwidth}
      \begin{itemize}
        \item<1-> \funcname{match \dots with}\footnotemark pattern matching can also match exceptions
        \item<1-> The CMM of \funcname{probably\_raise} shows that this \funcname{match \dots with} is implemented with a pattern matching and a \funcname{try \dots with}
        \item<1-> But this one only matches on \typename{ExnA} exception
        \item<2-> Since the pattern matching fails to catch the exception, \funcname{reraise} is called with our current \typename{ExnC} exception
        \item<3-> Disassembly shows that \funcname{reraise} is actually a call to \funcname{caml\_reraise\_exn}, which is also an assembly function provided by the runtime
      \end{itemize}
      \vfill
      \mintocaml[firstline=15,lastline=21,highlightlines={18-21}]{../src/backtrace/backtrace.ml}
      \endminipage
    \end{column}
    \begin{column}{0.55\textwidth}
      \centering
      \onslide*<1>{\mintcmm[highlightlines={10-18}]{../src/backtrace/camlBacktrace\_\_probably\_raise\_351.cmm}}
      \onslide*<2>{\listcmm[highlightlines=18]{../src/backtrace/camlBacktrace\_\_probably\_raise_351.cmm}{CMM of probably\_raise}}
      \onslide*<3>{\mintobjdump[firstline=5,lastline=35,highlightlines=31]{../src/backtrace/camlBacktrace\_\_probably\_raise_351.objdump}}
    \end{column}
  \end{columns}
  \only<1>{\footnotetext[1]{https://blog.janestreet.com/pattern-matching-and-exception-handling-unite/}}
\end{frame}

\newcommand\listCamlReraiseExn[1][]{
  \listasm[firstline=727,lastline=734,#1]{../ocaml/runtime/amd64.S}{runtime/amd64.S:727}}

\begin{frame}{Backtraces}{Introducing \funcname{caml\_reraise\_exn}}
  \begin{columns}[c]
    \begin{column}{0.5\textwidth}
      \minipage[c][0.66\textheight][s]{\columnwidth}
      \begin{itemize}
        \item<1-> The exception have to be reraised to the next handler
        \item<2-> First, checks if the backtrace should be collected with \funcarg{domain\_state->backtrace\_active}
        \only<3>{
        \begin{itemize}
            \item If not, behaves like a \funcname{raise\_notrace} with \funcname{RESTORE\_EXN\_HANDLER\_OCAML}
        \end{itemize}
        }
        \item<4-> Jumps into \funcname{caml\_raise\_exn}
        \item<5-> Calls \funcname{caml\_stash\_backtrace} again, to scan the next part of the stack: from the trap just pop-ed to the next trap
      \end{itemize}
      \vfill
      \mintocaml[firstline=15,lastline=21,highlightlines={18-21}]{../src/backtrace/backtrace.ml}
      \endminipage
    \end{column}
    \begin{column}{0.5\textwidth}
      \raggedleft
      \onslide*<1>{\listCamlReraiseExn{}}
      \onslide*<2>{\listCamlReraiseExn[highlightlines=729]{}}
      \onslide*<3>{
        \mintc[firstline=190,lastline=192,highlightlines={191-192}]{../ocaml/runtime/amd64.S}
        \mintc[firstline=234,lastline=237,highlightlines={235-237}]{../ocaml/runtime/amd64.S}
        \medskip
        \listCamlReraiseExn[highlightlines=731]{}
      }
      \onslide*<4-5>{
        % TODO refactor with \listCamlReraiseExn
        \mintasm[firstline=727,lastline=734,highlightlines=730]{../ocaml/runtime/amd64.S}
        \medskip
      }
      \onslide*<4>{\listCamlRaiseExn[highlightlines=711]{}}
      \onslide*<5>{\listCamlRaiseExn[highlightlines=720]{}}
    \end{column}
  \end{columns}
\end{frame}

\begin{frame}{Backtraces}{Backtrace collection continues}
  \begin{columns}[c]
    \begin{column}{0.55\textwidth}
      \minipage[c][0.75\textheight][s]{\columnwidth}
      \begin{itemize}
        \item<1-> \funcname{caml\_stash\_backtrace} scans the older part of the stack, up to the next trap
        \item<6-> Controls jumps to the nested exception handler in \funcname{maybe\_raise}, which matches only on \typename{ExnB}
        \item<7-> \funcname{caml\_reraise\_exn} is called again, backtrace collection resumes with \funcname{caml\_stash\_backtrace}
      \end{itemize}
      \medskip
      \begin{itemize}
        \item<2-> Constructed backtrace:
        \begin{itemize}
          \item <2-> \funcname{definitely\_raise}
          \item <2-> \funcname{probably\_raise}
          \item <3-> \funcname{probably\_raise} (re-raise)
          \item <4-> \funcname{maybe\_raise}
          \item <5-> \funcname{main}
          \item <8-> \funcname{main} (re-raise)
        \end{itemize}
      \end{itemize}
      \vfill
      \onslide*<1-5>{\mintocaml[firstline=15,lastline=21]{../src/backtrace/backtrace.ml}}
      \onslide*<6>{\mintocaml[firstline=28,lastline=34,highlightlines=31]{../src/backtrace/backtrace.ml}}
      \onslide*<7->{\mintocaml[firstline=28,lastline=34,highlightlines=33]{../src/backtrace/backtrace.ml}}
      \endminipage
    \end{column}
    \begin{column}{0.45\textwidth}
      \raggedleft
      \onslide*<1>{\providecommand\step{6}\input{diagrams/backtrace/backtrace.tex}}%
      \onslide*<2>{\providecommand\step{6}\input{diagrams/backtrace/backtrace.tex}}%
      \onslide*<3>{\providecommand\step{7}\input{diagrams/backtrace/backtrace.tex}}%
      \onslide*<4>{\providecommand\step{8}\input{diagrams/backtrace/backtrace.tex}}%
      \onslide*<5>{\providecommand\step{9}\input{diagrams/backtrace/backtrace.tex}}%
      \onslide*<6>{\providecommand\step{10}\input{diagrams/backtrace/backtrace.tex}}%
      \onslide*<7>{\providecommand\step{11}\input{diagrams/backtrace/backtrace.tex}}%
      \onslide*<8>{\providecommand\step{12}\input{diagrams/backtrace/backtrace.tex}}%
      \onslide*<9>{\providecommand\step{13}\input{diagrams/backtrace/backtrace.tex}}%
    \end{column}
  \end{columns}
\end{frame}

\begin{frame}{Backtraces}{Printing the backtrace}
  \begin{columns}[c]
    \begin{column}{0.45\textwidth}
      \minipage[c][0.66\textheight][s]{\columnwidth}
      \begin{itemize}
        \item<1-> The exception backtrace can now be printed to \funcarg{stdout} with \funcname{Printexc.print_backtrace}
        \item<2-> First the exception backtrace is retrieved into an OCaml array with \funcname{get_raw_backtrace }
        \item<3-> Which is implemented as the \funcname{caml_get_exception_raw_backtrace} C function in the runtime
        \item<4-> And finally printed with \funcname{print_exception_backtrace}
      \end{itemize}
      \vfill
      \mintocaml[firstline=28,lastline=34,highlightlines=33]{../src/backtrace/backtrace.ml}
      \endminipage
    \end{column}
    \begin{column}{0.55\textwidth}
      \onslide*<1,2>{
        \mintocaml[firstline=100,lastline=101]{../ocaml/stdlib/printexc.ml}
      }
      \only<1>{
        \listocaml[firstline=170,lastline=172]{../ocaml/stdlib/printexc.ml}{stdlib/printexc.ml:170}
      }
      \only<2>{
        \listocaml[firstline=170,lastline=172,highlightlines=172]{../ocaml/stdlib/printexc.ml}{stdlib/printexc.ml:170}
      }
      \only<3>{
        \mintc[firstline=154,lastline=156]{../ocaml/runtime/backtrace.c}
        \listc[firstline=171,lastline=192,highlightlines={184-187}]{../ocaml/runtime/backtrace.c}{runtime/backtrace.c:154}
      }
      \only<4>{
        \listocaml[firstline=155,lastline=165]{../ocaml/stdlib/printexc.ml}{stdlib/printexc.ml:55}
      }
    \end{column}
  \end{columns}
\end{frame}


\subsection{The multiple kinds of raise}
\frameSubsection{}{}

\begin{frame}{Backtraces}{When backtraces are not needed}
  \begin{columns}[c]
    \begin{column}{0.45\textwidth}
      \minipage[c][0.66\textheight][s]{\columnwidth}
      \begin{itemize}
        \item<1-> What if the backtrace for an exception is never needed?
        \item<2-> It's possible to raise the exception explicitly with \funcname{raise\_notrace} to make raising as cheap as possible
        \item<3-> An example of use in the \funcname{dynlink} library of the OCaml compiler
      \end{itemize}
      \vfill
      \onslide<3->{
      \listocaml[firstline=205,lastline=209,highlightlines=207]{../ocaml/otherlibs/dynlink/dynlink\_compilerlibs/misc.ml}{otherlibs/dynlink/dynlink\_compilerlibs/misc.ml:205}
      }
      \endminipage
    \end{column}
    \begin{column}{0.55\textwidth}
    \only<4>{
      \mintcmm[highlightlines=3]{../src/notrace/camlNotrace\_\_fun_347.cmm}
      \listcmm{../src/notrace/camlNotrace\_\_all\_somes\_327.cmm}{all\_somes}
    }
    \onslide*<5->{
      \listobjdump[highlightlines={6-9}]{../src/notrace/camlNotrace\_\_fun_347.objdump}{anonymous function from \funcname{all\_somes}}
    }
    \end{column}
  \end{columns}
\end{frame}

\begin{frame}{Backtraces}{Summary table of the multiple kinds of raise}
  \begin{itemize}
    \item When debugging information is enabled and if the backtrace of an exception isn't needed, it's possible to raise with \funcname{raise\_notrace} for performance
    \item When debugging information is \emph{not} enabled, the compiler optimises \funcname{raise} calls to inline assembly (same effect as using \funcname{raise\_notrace})
  \end{itemize}
  \bigskip
  \centering
  \begin{tabular}{| l | c | c | c | c |}
    \hline
    \parbox{10em}{Debug information \\ (\funcarg{-g} flag)}  & \multicolumn{2}{|c|}{Disabled}                                     & \multicolumn{2}{|c|}{Enabled} \\
    \hline
    Raise kind                                               & \funcname{raise} & \funcname{raise\_notrace}                       & \funcname{raise} & \funcname{raise\_notrace} \\
    \hline
    Compilation                                              & \parbox{8em}{inline assembly\\ (optimization)} & inline assembly   & \funcname{caml\_raise\_exn} & inline assembly \\
    \hline
  \end{tabular}
\end{frame}


%
% Takeaway #5
%
\subsection*{Takeaway \#5}
\frameSubsectionTakeaway{}
\againframe<5>{takeaway}
}%
      \onslide*<6>{\providecommand\step{10}%
% Backtraces
%
\subsection{One advantage of raising with debugging information}
\frameSubsection{}{}

\begin{frame}{Backtraces}{Debugging information \& source code}
  \begin{columns}[c]
    \begin{column}{0.5\textwidth}
      \begin{itemize}
        \item Raising an exception is fun and games, but how do we know where the exception originated? $\rightarrow$ With the backtrace associated with the exception!
        \item In order to get a backtrace from the point where an exception was raised, it's necessary to compile with debugging information enabled (\funcarg{-g} flag for the compiler)
        \item Same example as in the `Nested handlers' section
      \end{itemize}
    \end{column}
    \begin{column}{0.5\textwidth}
      \listocaml[firstline=9,lastline=34]{../src/backtrace/backtrace.ml}{backtrace/backtrace.ml}
    \end{column}
  \end{columns}
\end{frame}

\begin{frame}{Backtraces}{CMM and disassembly of \funcname{definitely\_raise}}
  \begin{columns}[c]
    \begin{column}{0.5\textwidth}
      \minipage[c][0.66\textheight][s]{\columnwidth}
      \begin{itemize}
        \item<1-> An effect of compiling with debugging information (\funcarg{-g}): the CMM no longer uses \funcname{raise\_notrace} to raise the exception but \funcname{raise} instead
        \item<2-> Disassembly shows that CMM \funcname{raise} is actually a call to \funcname{caml\_raise\_exn}, a function in assembler provided by the runtime
      \end{itemize}
      \vfill
      \mintocaml[firstline=9,lastline=13,highlightlines=12]{../src/backtrace/backtrace.ml}
      \endminipage
    \end{column}
    \begin{column}{0.5\textwidth}
      \onslide*<1>{
        \mintcmm[highlightlines=5]{../src/backtrace/camlBacktrace\_\_definitely\_raise\_347.cmm}
      }
      \onslide*<2>{
        \mintobjdump[firstline=2,highlightlines=14]{../src/backtrace/camlBacktrace\_\_definitely\_raise\_347.objdump}
      }
    \end{column}
  \end{columns}
\end{frame}

\begin{frame}{Backtraces}{Introducing \funcname{caml\_raise\_exn}}
  \begin{columns}[c]
    \begin{column}{0.5\textwidth}
      \minipage[c][0.66\textheight][s]{\columnwidth}
      \onslide*<1>{
        \begin{itemize}
          \item Raising the exception is performed using \funcname{caml\_raise\_exn} which is
            \begin{itemize}
              \item An assembly function provided by the runtime
              \item Architecture specific
            \end{itemize}
          \item Let's see what it does differently from \funcname{raise\_notrace}\dots
          \item We are about to raise \typename{ExnC} with \funcname{caml\_raise\_exn} from \funcname{definitely\_raise}
        \end{itemize}
      }
      \onslide*<2>{
        \mintobjdump[firstline=2,highlightlines=14]{../src/backtrace/camlBacktrace\_\_definitely\_raise\_347.objdump}
      }
      \vfill
      \mintocaml[firstline=9,lastline=13,highlightlines=12]{../src/backtrace/backtrace.ml}
      \endminipage
    \end{column}
    \begin{column}{0.5\textwidth}
      \centering
      \providecommand\step{1}\input{diagrams/backtrace/backtrace.tex}
    \end{column}
  \end{columns}
\end{frame}

\begin{frame}{Backtraces}{But before that, we need more fields from \typename{caml\_domain\_state}}
  \begin{columns}
    \begin{column}{0.5\textwidth}
      \begin{itemize}
        \item Remember \typename{caml\_domain\_state} the per-domain runtime state?
        \item Some other members are of interest to us now\dots
        \item \localname{backtrace\_active} is used to know if the runtime should collect backtraces
        \item \localname{backtrace\_active} can be set by
          \begin{itemize}
            \item The \funcarg{b} flag in the \funcname{OCAMLRUNPARAM} environment variable
            \item \mintinline{ocaml}{Printexc.record_backtrace : bool -> unit} \footnotemark
          \end{itemize}
      \end{itemize}
    \end{column}
    \begin{column}{0.5\textwidth}
      \begin{listing}
        \mintc[highlightlines={9-11}]{src/ocaml/domain\_state\_backtrace.h}
        \caption{runtime/caml/domain\_state.h}
      \end{listing}
    \end{column}
  \end{columns}
  \footnotetext[1]{https://v2.ocaml.org/api/Printexc.html}
\end{frame}

\newcommand\listCamlRaiseExn[1][]{
  \listasm[firstline=702,lastline=725,#1]{../ocaml/runtime/amd64.S}{runtime/amd64.S:702}}

\begin{frame}{Backtraces}{Walk through \funcname{caml\_raise\_exn}}
  \begin{columns}[T]
    \begin{column}{0.5\textwidth}
      \begin{itemize}
        \item<1-> First checks whether exceptions backtrace should be recorded
        \item<1-> By checking the \funcarg{domain\_state->backtrace\_active} value
        \item<2-> If not, acts as \funcname{raise\_notrace} with \funcname{RESTORE\_EXN\_HANDLER\_OCAML}
          \begin{itemize}
            \item Set \regname{RSP} to \localname{domain\_state->exn\_handler} value
            \item Restore \localname{domain\_state->exn\_handler} from trap
            \item Jump with \funcname{ret} (same effect as using \funcname{pop} and \funcname{jmp})
          \end{itemize}
        \item<3-> Otherwise, switches to the C stack to be able to execute C code
        \item<4-> Calls \funcname{caml\_stash\_backtrace}, a C function provided by the runtime\ignorespaces
          \onslide<6->{, with
            \begin{itemize}
              \item \funcarg{exn}: exception value
              \item \funcarg{pc}: code address of the raising point
              \item \funcarg{sp}: stack address of the raising function
              \item \funcarg{trapsp}: stack address of the next handler
            \end{itemize}
          }
      \end{itemize}
      \bigskip
      \mintocaml[firstline=9,lastline=13,highlightlines=12]{../src/backtrace/backtrace.ml}
    \end{column}
    \begin{column}{0.5\textwidth}
      \raggedleft
      \onslide*<1,3-4>{
        \mintc[firstline=190,lastline=192]{../ocaml/runtime/amd64.S}
        \mintc[firstline=234,lastline=237]{../ocaml/runtime/amd64.S}
      }
      \onslide*<2>{
        \mintc[firstline=190,lastline=192,highlightlines={191-192}]{../ocaml/runtime/amd64.S}
        \mintc[firstline=234,lastline=237,highlightlines={235-237}]{../ocaml/runtime/amd64.S}
      }
      \onslide*<1>{\listCamlRaiseExn[highlightlines=705]{}}%
      \onslide*<2>{\listCamlRaiseExn[highlightlines={707-708}]{}}%
      \onslide*<3>{\listCamlRaiseExn[highlightlines={712-714}]{}}%
      \onslide*<4>{\listCamlRaiseExn[highlightlines={715-720}]{}}%
      \onslide*<5>{\providecommand\step{1}\input{diagrams/backtrace/backtrace.tex}}%
      \onslide*<6>{\providecommand\step{2}\input{diagrams/backtrace/backtrace.tex}}%
    \end{column}
  \end{columns}
\end{frame}

\newcommand\listStashBacktrace[1][]{
  \listc[firstline=91,lastline=120,#1]{../ocaml/runtime/backtrace\_nat.c}{runtime/backtrace\_nat.c:91}}

\begin{frame}{Backtraces}{Introducing \funcname{caml\_stash\_backtrace}}
  \begin{columns}[c]
    \begin{column}{0.5\textwidth}
      \minipage[c][0.66\textheight][s]{\columnwidth}
      \begin{itemize}
        \item<1-> A C function provided by the runtime (specific to native code)
        \item<2-> It scans the stack, between the stack pointer at the raising point up to the next trap, in order to collect the names of the detected function frames
          \onslide*<2>{
            \begin{itemize}
              \item And more information, like line number, column number...
            \end{itemize}
          }
        \item<3-> It uses the same technique and information as the Garbage Collector (GC) uses to scan the values live on the stack
          \onslide*<4>{
            \begin{itemize}
              \item \typename{frame\_descr} are at the core of stack unwinding
            \end{itemize}
          }
      \end{itemize}
      \vfill
      \mintocaml[firstline=9,lastline=13,highlightlines=12]{../src/backtrace/backtrace.ml}
      \endminipage
    \end{column}
    \begin{column}{0.5\textwidth}
      \onslide*<1>{\listStashBacktrace[highlightlines=91]{}}
      \onslide*<2>{\listStashBacktrace[highlightlines={91,117-118}]{}}
      \onslide*<3->{\listStashBacktrace[highlightlines={94,106,109-110}]{}}
    \end{column}
  \end{columns}
\end{frame}

\newcommand\listFrameDescr[1][]{
  \listocaml[firstline=111,lastline=124,#1]{../ocaml/asmcomp/emitaux.ml}{asmcomp/emitaux.ml:111}}
\newcommand\listDebugInfo[1][]{
  \listocaml[firstline=38,lastline=49,#1]{../ocaml/lambda/debuginfo.mli}{lambda/debuginfo.mli:38}}

\begin{frame}{Backtraces}{\typename{frame\_descr} and \typename{frame\_debuginfo} in a nutshell}
  \begin{columns}[c]
    \begin{column}{0.5\textwidth}
      \begin{itemize}
        \item<1-> For every return address in an OCaml program, there is a associated \typename{frame\_descr}
        \item<2-> A \typename{frame\_descr} specifies
        \begin{itemize}
          \item<2-> The size of the function stack frame
          \item<3-> Where live values are on the stack (so that the GC can mark them as live and prevent from being swept)
        \end{itemize}
        \item<4-> When compiled with debug information, \typename{frame\_descr} will be linked to a \typename{frame\_debuginfo}
        \begin{itemize}
          \item<5-> Contains information about the position in the source code (file name, line and column number)
        \end{itemize}
      \end{itemize}
    \end{column}
    \begin{column}{0.5\textwidth}
        \onslide*<1>{\listFrameDescr[highlightlines={118-122}]{}}
        \onslide*<2>{\listFrameDescr[highlightlines={120}]{}}
        \onslide*<3>{\listFrameDescr[highlightlines={121}]{}}
        \onslide*<4->{\listFrameDescr[highlightlines={122}]{}}
        \medskip
        \onslide*<1-3>{\listDebugInfo{}}
        \onslide*<4>{\listDebugInfo[highlightlines={38,49}]{}}
        \onslide*<5>{\listDebugInfo[highlightlines={39-46}]{}}
    \end{column}
  \end{columns}
\end{frame}

\begin{frame}{Backtraces}{Scanning the stack with \typename{frame\_descr}}
  \begin{columns}[c]
    \begin{column}{0.5\textwidth}
      \begin{itemize}
        \item Given the return address of the code that raises the exception by calling \funcname{caml\_raise\_exn} (\funcarg{pc} argument of \funcname{caml\_stash\_backtrace})
        \item And the position of the stack pointer (\funcarg{sp} argument of \funcname{caml\_stash\_backtrace})
        \item The frame size given by \typename{frame\_descr}.\funcarg{frame\_size}, allows to find the end of the previous frame
        \item And the return address of the caller of this function
        \bigskip
        \onslide<3->{
        \item Note that traps (here the reddish \funcname{ExnA}, \funcname{ExnB} and \funcname{ExnC} blocks) belong to the frame that installed them, and so the frame size changes when traps are removed
        }
      \end{itemize}
    \end{column}
    \begin{column}{0.5\textwidth}
      \raggedleft
      \onslide*<1>{\providecommand\step{2}\input{diagrams/backtrace/backtrace.tex}}%
      \onslide*<2>{\providecommand\step{1}\input{diagrams/backtrace/scanstack.tex}}%
      \onslide*<3>{\providecommand\step{2}\input{diagrams/backtrace/scanstack.tex}}%
      \onslide*<4>{\providecommand\step{3}\input{diagrams/backtrace/scanstack.tex}}%
      \onslide*<5>{\providecommand\step{4}\input{diagrams/backtrace/scanstack.tex}}%
      \onslide*<6>{\providecommand\step{5}\input{diagrams/backtrace/scanstack.tex}}%
    \end{column}
  \end{columns}
\end{frame}

% Duplicate of previous-previous slide
\begin{frame}{Backtraces}{Continuing on \funcname{caml\_stash\_backtrace}}
  \begin{columns}[c]
    \begin{column}{0.5\textwidth}
      \minipage[c][0.75\textheight][s]{\columnwidth}
      \begin{itemize}
          \color{gray}
        \item A C function provided by the runtime (specific to native code)
        \item It scans the stack, between the stack pointer at the raising point up to the next trap, in order to collect the names of the detected function frames
        \item It uses the same technique and information as the Garbage Collector (GC) uses to scan the values live on the stack
      \end{itemize}
      \begin{itemize}
        \item For each function frame detected, store a pointer to the \typename{frame\_descr} in the \localname{domain\_state->backtrace\_buffer} array, effectively constructing the backtrace
      \end{itemize}
      \onslide<2->{
        \begin{itemize}
            \smallskip
          \item Constructed backtrace:
            \funcname{definitely\_raise},
            \onslide<3->{\funcname{probably\_raise}}
        \end{itemize}
      }
      \vfill
      \mintocaml[firstline=9,lastline=13,highlightlines=12]{../src/backtrace/backtrace.ml}
      \endminipage
    \end{column}
    \begin{column}{0.5\textwidth}
      \raggedleft
      \onslide*<1>{\listStashBacktrace[highlightlines={114-115}]{}}
      \onslide*<2>{\providecommand\step{3}\input{diagrams/backtrace/backtrace.tex}}%
      \onslide*<3>{\providecommand\step{4}\input{diagrams/backtrace/backtrace.tex}}%
    \end{column}
  \end{columns}
\end{frame}

\begin{frame}{Backtraces}{Back to \funcname{caml\_raise\_exn}}
  \begin{columns}[c]
    \begin{column}{0.5\textwidth}
      \minipage[c][0.66\textheight][s]{\columnwidth}
      \begin{itemize}\color{gray}
        \item Checks wheteher exceptions backtrace should be recorded, by checking the \funcarg{domain\_state->backtrace\_active} value
        \item Switches to the C stack to be able to execute C code
        \item Calls \funcname{caml\_stash\_backtrace}, to collect the backtrace up to the next trap
      \end{itemize}
      \onslide*<2->{
        \begin{itemize}
          \item<2-> Executes the exception handler in \funcname{probably\_raise} with \funcname{RESTORE\_EXN\_HANDLER\_OCAML}
            \begin{itemize}
              \item Set \regname{RSP} to \localname{domain\_state->exn\_handler} value
              \item Restore \localname{domain\_state->exn\_handler} from trap
              \item Jump to the handler code with \funcname{ret}
            \end{itemize}
        \end{itemize}
      }
      \vfill
      \onslide*<1>{\mintocaml[firstline=9,lastline=13,highlightlines=12]{../src/backtrace/backtrace.ml}}
      \onslide*<2->{\mintocaml[firstline=15,lastline=21,highlightlines={19-21}]{../src/backtrace/backtrace.ml}}
      \endminipage
    \end{column}
    \begin{column}{0.5\textwidth}
      \centering
      \onslide*<1>{
        \mintc[firstline=190,lastline=192]{../ocaml/runtime/amd64.S}
        \mintc[firstline=234,lastline=237]{../ocaml/runtime/amd64.S}
      }
      \onslide*<2>{
        \mintc[firstline=190,lastline=192,highlightlines={191-192}]{../ocaml/runtime/amd64.S}
        \mintc[firstline=234,lastline=237,highlightlines={235-237}]{../ocaml/runtime/amd64.S}
      }
      \onslide*<1>{\listCamlRaiseExn[highlightlines=720]{}}
      \onslide*<2>{\listCamlRaiseExn[highlightlines=722]{}}
      \onslide*<3->{\providecommand\step{5}\input{diagrams/backtrace/backtrace.tex}}
    \end{column}
  \end{columns}
\end{frame}

\begin{frame}{Backtraces}{\funcname{caml\_probably\_raise}'s handler doesn't catch \typename{ExnC}}
  \begin{columns}[c]
    \begin{column}{0.45\textwidth}
      \minipage[c][0.75\textheight][s]{\columnwidth}
      \begin{itemize}
        \item<1-> \funcname{match \dots with}\footnotemark pattern matching can also match exceptions
        \item<1-> The CMM of \funcname{probably\_raise} shows that this \funcname{match \dots with} is implemented with a pattern matching and a \funcname{try \dots with}
        \item<1-> But this one only matches on \typename{ExnA} exception
        \item<2-> Since the pattern matching fails to catch the exception, \funcname{reraise} is called with our current \typename{ExnC} exception
        \item<3-> Disassembly shows that \funcname{reraise} is actually a call to \funcname{caml\_reraise\_exn}, which is also an assembly function provided by the runtime
      \end{itemize}
      \vfill
      \mintocaml[firstline=15,lastline=21,highlightlines={18-21}]{../src/backtrace/backtrace.ml}
      \endminipage
    \end{column}
    \begin{column}{0.55\textwidth}
      \centering
      \onslide*<1>{\mintcmm[highlightlines={10-18}]{../src/backtrace/camlBacktrace\_\_probably\_raise\_351.cmm}}
      \onslide*<2>{\listcmm[highlightlines=18]{../src/backtrace/camlBacktrace\_\_probably\_raise_351.cmm}{CMM of probably\_raise}}
      \onslide*<3>{\mintobjdump[firstline=5,lastline=35,highlightlines=31]{../src/backtrace/camlBacktrace\_\_probably\_raise_351.objdump}}
    \end{column}
  \end{columns}
  \only<1>{\footnotetext[1]{https://blog.janestreet.com/pattern-matching-and-exception-handling-unite/}}
\end{frame}

\newcommand\listCamlReraiseExn[1][]{
  \listasm[firstline=727,lastline=734,#1]{../ocaml/runtime/amd64.S}{runtime/amd64.S:727}}

\begin{frame}{Backtraces}{Introducing \funcname{caml\_reraise\_exn}}
  \begin{columns}[c]
    \begin{column}{0.5\textwidth}
      \minipage[c][0.66\textheight][s]{\columnwidth}
      \begin{itemize}
        \item<1-> The exception have to be reraised to the next handler
        \item<2-> First, checks if the backtrace should be collected with \funcarg{domain\_state->backtrace\_active}
        \only<3>{
        \begin{itemize}
            \item If not, behaves like a \funcname{raise\_notrace} with \funcname{RESTORE\_EXN\_HANDLER\_OCAML}
        \end{itemize}
        }
        \item<4-> Jumps into \funcname{caml\_raise\_exn}
        \item<5-> Calls \funcname{caml\_stash\_backtrace} again, to scan the next part of the stack: from the trap just pop-ed to the next trap
      \end{itemize}
      \vfill
      \mintocaml[firstline=15,lastline=21,highlightlines={18-21}]{../src/backtrace/backtrace.ml}
      \endminipage
    \end{column}
    \begin{column}{0.5\textwidth}
      \raggedleft
      \onslide*<1>{\listCamlReraiseExn{}}
      \onslide*<2>{\listCamlReraiseExn[highlightlines=729]{}}
      \onslide*<3>{
        \mintc[firstline=190,lastline=192,highlightlines={191-192}]{../ocaml/runtime/amd64.S}
        \mintc[firstline=234,lastline=237,highlightlines={235-237}]{../ocaml/runtime/amd64.S}
        \medskip
        \listCamlReraiseExn[highlightlines=731]{}
      }
      \onslide*<4-5>{
        % TODO refactor with \listCamlReraiseExn
        \mintasm[firstline=727,lastline=734,highlightlines=730]{../ocaml/runtime/amd64.S}
        \medskip
      }
      \onslide*<4>{\listCamlRaiseExn[highlightlines=711]{}}
      \onslide*<5>{\listCamlRaiseExn[highlightlines=720]{}}
    \end{column}
  \end{columns}
\end{frame}

\begin{frame}{Backtraces}{Backtrace collection continues}
  \begin{columns}[c]
    \begin{column}{0.55\textwidth}
      \minipage[c][0.75\textheight][s]{\columnwidth}
      \begin{itemize}
        \item<1-> \funcname{caml\_stash\_backtrace} scans the older part of the stack, up to the next trap
        \item<6-> Controls jumps to the nested exception handler in \funcname{maybe\_raise}, which matches only on \typename{ExnB}
        \item<7-> \funcname{caml\_reraise\_exn} is called again, backtrace collection resumes with \funcname{caml\_stash\_backtrace}
      \end{itemize}
      \medskip
      \begin{itemize}
        \item<2-> Constructed backtrace:
        \begin{itemize}
          \item <2-> \funcname{definitely\_raise}
          \item <2-> \funcname{probably\_raise}
          \item <3-> \funcname{probably\_raise} (re-raise)
          \item <4-> \funcname{maybe\_raise}
          \item <5-> \funcname{main}
          \item <8-> \funcname{main} (re-raise)
        \end{itemize}
      \end{itemize}
      \vfill
      \onslide*<1-5>{\mintocaml[firstline=15,lastline=21]{../src/backtrace/backtrace.ml}}
      \onslide*<6>{\mintocaml[firstline=28,lastline=34,highlightlines=31]{../src/backtrace/backtrace.ml}}
      \onslide*<7->{\mintocaml[firstline=28,lastline=34,highlightlines=33]{../src/backtrace/backtrace.ml}}
      \endminipage
    \end{column}
    \begin{column}{0.45\textwidth}
      \raggedleft
      \onslide*<1>{\providecommand\step{6}\input{diagrams/backtrace/backtrace.tex}}%
      \onslide*<2>{\providecommand\step{6}\input{diagrams/backtrace/backtrace.tex}}%
      \onslide*<3>{\providecommand\step{7}\input{diagrams/backtrace/backtrace.tex}}%
      \onslide*<4>{\providecommand\step{8}\input{diagrams/backtrace/backtrace.tex}}%
      \onslide*<5>{\providecommand\step{9}\input{diagrams/backtrace/backtrace.tex}}%
      \onslide*<6>{\providecommand\step{10}\input{diagrams/backtrace/backtrace.tex}}%
      \onslide*<7>{\providecommand\step{11}\input{diagrams/backtrace/backtrace.tex}}%
      \onslide*<8>{\providecommand\step{12}\input{diagrams/backtrace/backtrace.tex}}%
      \onslide*<9>{\providecommand\step{13}\input{diagrams/backtrace/backtrace.tex}}%
    \end{column}
  \end{columns}
\end{frame}

\begin{frame}{Backtraces}{Printing the backtrace}
  \begin{columns}[c]
    \begin{column}{0.45\textwidth}
      \minipage[c][0.66\textheight][s]{\columnwidth}
      \begin{itemize}
        \item<1-> The exception backtrace can now be printed to \funcarg{stdout} with \funcname{Printexc.print_backtrace}
        \item<2-> First the exception backtrace is retrieved into an OCaml array with \funcname{get_raw_backtrace }
        \item<3-> Which is implemented as the \funcname{caml_get_exception_raw_backtrace} C function in the runtime
        \item<4-> And finally printed with \funcname{print_exception_backtrace}
      \end{itemize}
      \vfill
      \mintocaml[firstline=28,lastline=34,highlightlines=33]{../src/backtrace/backtrace.ml}
      \endminipage
    \end{column}
    \begin{column}{0.55\textwidth}
      \onslide*<1,2>{
        \mintocaml[firstline=100,lastline=101]{../ocaml/stdlib/printexc.ml}
      }
      \only<1>{
        \listocaml[firstline=170,lastline=172]{../ocaml/stdlib/printexc.ml}{stdlib/printexc.ml:170}
      }
      \only<2>{
        \listocaml[firstline=170,lastline=172,highlightlines=172]{../ocaml/stdlib/printexc.ml}{stdlib/printexc.ml:170}
      }
      \only<3>{
        \mintc[firstline=154,lastline=156]{../ocaml/runtime/backtrace.c}
        \listc[firstline=171,lastline=192,highlightlines={184-187}]{../ocaml/runtime/backtrace.c}{runtime/backtrace.c:154}
      }
      \only<4>{
        \listocaml[firstline=155,lastline=165]{../ocaml/stdlib/printexc.ml}{stdlib/printexc.ml:55}
      }
    \end{column}
  \end{columns}
\end{frame}


\subsection{The multiple kinds of raise}
\frameSubsection{}{}

\begin{frame}{Backtraces}{When backtraces are not needed}
  \begin{columns}[c]
    \begin{column}{0.45\textwidth}
      \minipage[c][0.66\textheight][s]{\columnwidth}
      \begin{itemize}
        \item<1-> What if the backtrace for an exception is never needed?
        \item<2-> It's possible to raise the exception explicitly with \funcname{raise\_notrace} to make raising as cheap as possible
        \item<3-> An example of use in the \funcname{dynlink} library of the OCaml compiler
      \end{itemize}
      \vfill
      \onslide<3->{
      \listocaml[firstline=205,lastline=209,highlightlines=207]{../ocaml/otherlibs/dynlink/dynlink\_compilerlibs/misc.ml}{otherlibs/dynlink/dynlink\_compilerlibs/misc.ml:205}
      }
      \endminipage
    \end{column}
    \begin{column}{0.55\textwidth}
    \only<4>{
      \mintcmm[highlightlines=3]{../src/notrace/camlNotrace\_\_fun_347.cmm}
      \listcmm{../src/notrace/camlNotrace\_\_all\_somes\_327.cmm}{all\_somes}
    }
    \onslide*<5->{
      \listobjdump[highlightlines={6-9}]{../src/notrace/camlNotrace\_\_fun_347.objdump}{anonymous function from \funcname{all\_somes}}
    }
    \end{column}
  \end{columns}
\end{frame}

\begin{frame}{Backtraces}{Summary table of the multiple kinds of raise}
  \begin{itemize}
    \item When debugging information is enabled and if the backtrace of an exception isn't needed, it's possible to raise with \funcname{raise\_notrace} for performance
    \item When debugging information is \emph{not} enabled, the compiler optimises \funcname{raise} calls to inline assembly (same effect as using \funcname{raise\_notrace})
  \end{itemize}
  \bigskip
  \centering
  \begin{tabular}{| l | c | c | c | c |}
    \hline
    \parbox{10em}{Debug information \\ (\funcarg{-g} flag)}  & \multicolumn{2}{|c|}{Disabled}                                     & \multicolumn{2}{|c|}{Enabled} \\
    \hline
    Raise kind                                               & \funcname{raise} & \funcname{raise\_notrace}                       & \funcname{raise} & \funcname{raise\_notrace} \\
    \hline
    Compilation                                              & \parbox{8em}{inline assembly\\ (optimization)} & inline assembly   & \funcname{caml\_raise\_exn} & inline assembly \\
    \hline
  \end{tabular}
\end{frame}


%
% Takeaway #5
%
\subsection*{Takeaway \#5}
\frameSubsectionTakeaway{}
\againframe<5>{takeaway}
}%
      \onslide*<7>{\providecommand\step{11}%
% Backtraces
%
\subsection{One advantage of raising with debugging information}
\frameSubsection{}{}

\begin{frame}{Backtraces}{Debugging information \& source code}
  \begin{columns}[c]
    \begin{column}{0.5\textwidth}
      \begin{itemize}
        \item Raising an exception is fun and games, but how do we know where the exception originated? $\rightarrow$ With the backtrace associated with the exception!
        \item In order to get a backtrace from the point where an exception was raised, it's necessary to compile with debugging information enabled (\funcarg{-g} flag for the compiler)
        \item Same example as in the `Nested handlers' section
      \end{itemize}
    \end{column}
    \begin{column}{0.5\textwidth}
      \listocaml[firstline=9,lastline=34]{../src/backtrace/backtrace.ml}{backtrace/backtrace.ml}
    \end{column}
  \end{columns}
\end{frame}

\begin{frame}{Backtraces}{CMM and disassembly of \funcname{definitely\_raise}}
  \begin{columns}[c]
    \begin{column}{0.5\textwidth}
      \minipage[c][0.66\textheight][s]{\columnwidth}
      \begin{itemize}
        \item<1-> An effect of compiling with debugging information (\funcarg{-g}): the CMM no longer uses \funcname{raise\_notrace} to raise the exception but \funcname{raise} instead
        \item<2-> Disassembly shows that CMM \funcname{raise} is actually a call to \funcname{caml\_raise\_exn}, a function in assembler provided by the runtime
      \end{itemize}
      \vfill
      \mintocaml[firstline=9,lastline=13,highlightlines=12]{../src/backtrace/backtrace.ml}
      \endminipage
    \end{column}
    \begin{column}{0.5\textwidth}
      \onslide*<1>{
        \mintcmm[highlightlines=5]{../src/backtrace/camlBacktrace\_\_definitely\_raise\_347.cmm}
      }
      \onslide*<2>{
        \mintobjdump[firstline=2,highlightlines=14]{../src/backtrace/camlBacktrace\_\_definitely\_raise\_347.objdump}
      }
    \end{column}
  \end{columns}
\end{frame}

\begin{frame}{Backtraces}{Introducing \funcname{caml\_raise\_exn}}
  \begin{columns}[c]
    \begin{column}{0.5\textwidth}
      \minipage[c][0.66\textheight][s]{\columnwidth}
      \onslide*<1>{
        \begin{itemize}
          \item Raising the exception is performed using \funcname{caml\_raise\_exn} which is
            \begin{itemize}
              \item An assembly function provided by the runtime
              \item Architecture specific
            \end{itemize}
          \item Let's see what it does differently from \funcname{raise\_notrace}\dots
          \item We are about to raise \typename{ExnC} with \funcname{caml\_raise\_exn} from \funcname{definitely\_raise}
        \end{itemize}
      }
      \onslide*<2>{
        \mintobjdump[firstline=2,highlightlines=14]{../src/backtrace/camlBacktrace\_\_definitely\_raise\_347.objdump}
      }
      \vfill
      \mintocaml[firstline=9,lastline=13,highlightlines=12]{../src/backtrace/backtrace.ml}
      \endminipage
    \end{column}
    \begin{column}{0.5\textwidth}
      \centering
      \providecommand\step{1}\input{diagrams/backtrace/backtrace.tex}
    \end{column}
  \end{columns}
\end{frame}

\begin{frame}{Backtraces}{But before that, we need more fields from \typename{caml\_domain\_state}}
  \begin{columns}
    \begin{column}{0.5\textwidth}
      \begin{itemize}
        \item Remember \typename{caml\_domain\_state} the per-domain runtime state?
        \item Some other members are of interest to us now\dots
        \item \localname{backtrace\_active} is used to know if the runtime should collect backtraces
        \item \localname{backtrace\_active} can be set by
          \begin{itemize}
            \item The \funcarg{b} flag in the \funcname{OCAMLRUNPARAM} environment variable
            \item \mintinline{ocaml}{Printexc.record_backtrace : bool -> unit} \footnotemark
          \end{itemize}
      \end{itemize}
    \end{column}
    \begin{column}{0.5\textwidth}
      \begin{listing}
        \mintc[highlightlines={9-11}]{src/ocaml/domain\_state\_backtrace.h}
        \caption{runtime/caml/domain\_state.h}
      \end{listing}
    \end{column}
  \end{columns}
  \footnotetext[1]{https://v2.ocaml.org/api/Printexc.html}
\end{frame}

\newcommand\listCamlRaiseExn[1][]{
  \listasm[firstline=702,lastline=725,#1]{../ocaml/runtime/amd64.S}{runtime/amd64.S:702}}

\begin{frame}{Backtraces}{Walk through \funcname{caml\_raise\_exn}}
  \begin{columns}[T]
    \begin{column}{0.5\textwidth}
      \begin{itemize}
        \item<1-> First checks whether exceptions backtrace should be recorded
        \item<1-> By checking the \funcarg{domain\_state->backtrace\_active} value
        \item<2-> If not, acts as \funcname{raise\_notrace} with \funcname{RESTORE\_EXN\_HANDLER\_OCAML}
          \begin{itemize}
            \item Set \regname{RSP} to \localname{domain\_state->exn\_handler} value
            \item Restore \localname{domain\_state->exn\_handler} from trap
            \item Jump with \funcname{ret} (same effect as using \funcname{pop} and \funcname{jmp})
          \end{itemize}
        \item<3-> Otherwise, switches to the C stack to be able to execute C code
        \item<4-> Calls \funcname{caml\_stash\_backtrace}, a C function provided by the runtime\ignorespaces
          \onslide<6->{, with
            \begin{itemize}
              \item \funcarg{exn}: exception value
              \item \funcarg{pc}: code address of the raising point
              \item \funcarg{sp}: stack address of the raising function
              \item \funcarg{trapsp}: stack address of the next handler
            \end{itemize}
          }
      \end{itemize}
      \bigskip
      \mintocaml[firstline=9,lastline=13,highlightlines=12]{../src/backtrace/backtrace.ml}
    \end{column}
    \begin{column}{0.5\textwidth}
      \raggedleft
      \onslide*<1,3-4>{
        \mintc[firstline=190,lastline=192]{../ocaml/runtime/amd64.S}
        \mintc[firstline=234,lastline=237]{../ocaml/runtime/amd64.S}
      }
      \onslide*<2>{
        \mintc[firstline=190,lastline=192,highlightlines={191-192}]{../ocaml/runtime/amd64.S}
        \mintc[firstline=234,lastline=237,highlightlines={235-237}]{../ocaml/runtime/amd64.S}
      }
      \onslide*<1>{\listCamlRaiseExn[highlightlines=705]{}}%
      \onslide*<2>{\listCamlRaiseExn[highlightlines={707-708}]{}}%
      \onslide*<3>{\listCamlRaiseExn[highlightlines={712-714}]{}}%
      \onslide*<4>{\listCamlRaiseExn[highlightlines={715-720}]{}}%
      \onslide*<5>{\providecommand\step{1}\input{diagrams/backtrace/backtrace.tex}}%
      \onslide*<6>{\providecommand\step{2}\input{diagrams/backtrace/backtrace.tex}}%
    \end{column}
  \end{columns}
\end{frame}

\newcommand\listStashBacktrace[1][]{
  \listc[firstline=91,lastline=120,#1]{../ocaml/runtime/backtrace\_nat.c}{runtime/backtrace\_nat.c:91}}

\begin{frame}{Backtraces}{Introducing \funcname{caml\_stash\_backtrace}}
  \begin{columns}[c]
    \begin{column}{0.5\textwidth}
      \minipage[c][0.66\textheight][s]{\columnwidth}
      \begin{itemize}
        \item<1-> A C function provided by the runtime (specific to native code)
        \item<2-> It scans the stack, between the stack pointer at the raising point up to the next trap, in order to collect the names of the detected function frames
          \onslide*<2>{
            \begin{itemize}
              \item And more information, like line number, column number...
            \end{itemize}
          }
        \item<3-> It uses the same technique and information as the Garbage Collector (GC) uses to scan the values live on the stack
          \onslide*<4>{
            \begin{itemize}
              \item \typename{frame\_descr} are at the core of stack unwinding
            \end{itemize}
          }
      \end{itemize}
      \vfill
      \mintocaml[firstline=9,lastline=13,highlightlines=12]{../src/backtrace/backtrace.ml}
      \endminipage
    \end{column}
    \begin{column}{0.5\textwidth}
      \onslide*<1>{\listStashBacktrace[highlightlines=91]{}}
      \onslide*<2>{\listStashBacktrace[highlightlines={91,117-118}]{}}
      \onslide*<3->{\listStashBacktrace[highlightlines={94,106,109-110}]{}}
    \end{column}
  \end{columns}
\end{frame}

\newcommand\listFrameDescr[1][]{
  \listocaml[firstline=111,lastline=124,#1]{../ocaml/asmcomp/emitaux.ml}{asmcomp/emitaux.ml:111}}
\newcommand\listDebugInfo[1][]{
  \listocaml[firstline=38,lastline=49,#1]{../ocaml/lambda/debuginfo.mli}{lambda/debuginfo.mli:38}}

\begin{frame}{Backtraces}{\typename{frame\_descr} and \typename{frame\_debuginfo} in a nutshell}
  \begin{columns}[c]
    \begin{column}{0.5\textwidth}
      \begin{itemize}
        \item<1-> For every return address in an OCaml program, there is a associated \typename{frame\_descr}
        \item<2-> A \typename{frame\_descr} specifies
        \begin{itemize}
          \item<2-> The size of the function stack frame
          \item<3-> Where live values are on the stack (so that the GC can mark them as live and prevent from being swept)
        \end{itemize}
        \item<4-> When compiled with debug information, \typename{frame\_descr} will be linked to a \typename{frame\_debuginfo}
        \begin{itemize}
          \item<5-> Contains information about the position in the source code (file name, line and column number)
        \end{itemize}
      \end{itemize}
    \end{column}
    \begin{column}{0.5\textwidth}
        \onslide*<1>{\listFrameDescr[highlightlines={118-122}]{}}
        \onslide*<2>{\listFrameDescr[highlightlines={120}]{}}
        \onslide*<3>{\listFrameDescr[highlightlines={121}]{}}
        \onslide*<4->{\listFrameDescr[highlightlines={122}]{}}
        \medskip
        \onslide*<1-3>{\listDebugInfo{}}
        \onslide*<4>{\listDebugInfo[highlightlines={38,49}]{}}
        \onslide*<5>{\listDebugInfo[highlightlines={39-46}]{}}
    \end{column}
  \end{columns}
\end{frame}

\begin{frame}{Backtraces}{Scanning the stack with \typename{frame\_descr}}
  \begin{columns}[c]
    \begin{column}{0.5\textwidth}
      \begin{itemize}
        \item Given the return address of the code that raises the exception by calling \funcname{caml\_raise\_exn} (\funcarg{pc} argument of \funcname{caml\_stash\_backtrace})
        \item And the position of the stack pointer (\funcarg{sp} argument of \funcname{caml\_stash\_backtrace})
        \item The frame size given by \typename{frame\_descr}.\funcarg{frame\_size}, allows to find the end of the previous frame
        \item And the return address of the caller of this function
        \bigskip
        \onslide<3->{
        \item Note that traps (here the reddish \funcname{ExnA}, \funcname{ExnB} and \funcname{ExnC} blocks) belong to the frame that installed them, and so the frame size changes when traps are removed
        }
      \end{itemize}
    \end{column}
    \begin{column}{0.5\textwidth}
      \raggedleft
      \onslide*<1>{\providecommand\step{2}\input{diagrams/backtrace/backtrace.tex}}%
      \onslide*<2>{\providecommand\step{1}\input{diagrams/backtrace/scanstack.tex}}%
      \onslide*<3>{\providecommand\step{2}\input{diagrams/backtrace/scanstack.tex}}%
      \onslide*<4>{\providecommand\step{3}\input{diagrams/backtrace/scanstack.tex}}%
      \onslide*<5>{\providecommand\step{4}\input{diagrams/backtrace/scanstack.tex}}%
      \onslide*<6>{\providecommand\step{5}\input{diagrams/backtrace/scanstack.tex}}%
    \end{column}
  \end{columns}
\end{frame}

% Duplicate of previous-previous slide
\begin{frame}{Backtraces}{Continuing on \funcname{caml\_stash\_backtrace}}
  \begin{columns}[c]
    \begin{column}{0.5\textwidth}
      \minipage[c][0.75\textheight][s]{\columnwidth}
      \begin{itemize}
          \color{gray}
        \item A C function provided by the runtime (specific to native code)
        \item It scans the stack, between the stack pointer at the raising point up to the next trap, in order to collect the names of the detected function frames
        \item It uses the same technique and information as the Garbage Collector (GC) uses to scan the values live on the stack
      \end{itemize}
      \begin{itemize}
        \item For each function frame detected, store a pointer to the \typename{frame\_descr} in the \localname{domain\_state->backtrace\_buffer} array, effectively constructing the backtrace
      \end{itemize}
      \onslide<2->{
        \begin{itemize}
            \smallskip
          \item Constructed backtrace:
            \funcname{definitely\_raise},
            \onslide<3->{\funcname{probably\_raise}}
        \end{itemize}
      }
      \vfill
      \mintocaml[firstline=9,lastline=13,highlightlines=12]{../src/backtrace/backtrace.ml}
      \endminipage
    \end{column}
    \begin{column}{0.5\textwidth}
      \raggedleft
      \onslide*<1>{\listStashBacktrace[highlightlines={114-115}]{}}
      \onslide*<2>{\providecommand\step{3}\input{diagrams/backtrace/backtrace.tex}}%
      \onslide*<3>{\providecommand\step{4}\input{diagrams/backtrace/backtrace.tex}}%
    \end{column}
  \end{columns}
\end{frame}

\begin{frame}{Backtraces}{Back to \funcname{caml\_raise\_exn}}
  \begin{columns}[c]
    \begin{column}{0.5\textwidth}
      \minipage[c][0.66\textheight][s]{\columnwidth}
      \begin{itemize}\color{gray}
        \item Checks wheteher exceptions backtrace should be recorded, by checking the \funcarg{domain\_state->backtrace\_active} value
        \item Switches to the C stack to be able to execute C code
        \item Calls \funcname{caml\_stash\_backtrace}, to collect the backtrace up to the next trap
      \end{itemize}
      \onslide*<2->{
        \begin{itemize}
          \item<2-> Executes the exception handler in \funcname{probably\_raise} with \funcname{RESTORE\_EXN\_HANDLER\_OCAML}
            \begin{itemize}
              \item Set \regname{RSP} to \localname{domain\_state->exn\_handler} value
              \item Restore \localname{domain\_state->exn\_handler} from trap
              \item Jump to the handler code with \funcname{ret}
            \end{itemize}
        \end{itemize}
      }
      \vfill
      \onslide*<1>{\mintocaml[firstline=9,lastline=13,highlightlines=12]{../src/backtrace/backtrace.ml}}
      \onslide*<2->{\mintocaml[firstline=15,lastline=21,highlightlines={19-21}]{../src/backtrace/backtrace.ml}}
      \endminipage
    \end{column}
    \begin{column}{0.5\textwidth}
      \centering
      \onslide*<1>{
        \mintc[firstline=190,lastline=192]{../ocaml/runtime/amd64.S}
        \mintc[firstline=234,lastline=237]{../ocaml/runtime/amd64.S}
      }
      \onslide*<2>{
        \mintc[firstline=190,lastline=192,highlightlines={191-192}]{../ocaml/runtime/amd64.S}
        \mintc[firstline=234,lastline=237,highlightlines={235-237}]{../ocaml/runtime/amd64.S}
      }
      \onslide*<1>{\listCamlRaiseExn[highlightlines=720]{}}
      \onslide*<2>{\listCamlRaiseExn[highlightlines=722]{}}
      \onslide*<3->{\providecommand\step{5}\input{diagrams/backtrace/backtrace.tex}}
    \end{column}
  \end{columns}
\end{frame}

\begin{frame}{Backtraces}{\funcname{caml\_probably\_raise}'s handler doesn't catch \typename{ExnC}}
  \begin{columns}[c]
    \begin{column}{0.45\textwidth}
      \minipage[c][0.75\textheight][s]{\columnwidth}
      \begin{itemize}
        \item<1-> \funcname{match \dots with}\footnotemark pattern matching can also match exceptions
        \item<1-> The CMM of \funcname{probably\_raise} shows that this \funcname{match \dots with} is implemented with a pattern matching and a \funcname{try \dots with}
        \item<1-> But this one only matches on \typename{ExnA} exception
        \item<2-> Since the pattern matching fails to catch the exception, \funcname{reraise} is called with our current \typename{ExnC} exception
        \item<3-> Disassembly shows that \funcname{reraise} is actually a call to \funcname{caml\_reraise\_exn}, which is also an assembly function provided by the runtime
      \end{itemize}
      \vfill
      \mintocaml[firstline=15,lastline=21,highlightlines={18-21}]{../src/backtrace/backtrace.ml}
      \endminipage
    \end{column}
    \begin{column}{0.55\textwidth}
      \centering
      \onslide*<1>{\mintcmm[highlightlines={10-18}]{../src/backtrace/camlBacktrace\_\_probably\_raise\_351.cmm}}
      \onslide*<2>{\listcmm[highlightlines=18]{../src/backtrace/camlBacktrace\_\_probably\_raise_351.cmm}{CMM of probably\_raise}}
      \onslide*<3>{\mintobjdump[firstline=5,lastline=35,highlightlines=31]{../src/backtrace/camlBacktrace\_\_probably\_raise_351.objdump}}
    \end{column}
  \end{columns}
  \only<1>{\footnotetext[1]{https://blog.janestreet.com/pattern-matching-and-exception-handling-unite/}}
\end{frame}

\newcommand\listCamlReraiseExn[1][]{
  \listasm[firstline=727,lastline=734,#1]{../ocaml/runtime/amd64.S}{runtime/amd64.S:727}}

\begin{frame}{Backtraces}{Introducing \funcname{caml\_reraise\_exn}}
  \begin{columns}[c]
    \begin{column}{0.5\textwidth}
      \minipage[c][0.66\textheight][s]{\columnwidth}
      \begin{itemize}
        \item<1-> The exception have to be reraised to the next handler
        \item<2-> First, checks if the backtrace should be collected with \funcarg{domain\_state->backtrace\_active}
        \only<3>{
        \begin{itemize}
            \item If not, behaves like a \funcname{raise\_notrace} with \funcname{RESTORE\_EXN\_HANDLER\_OCAML}
        \end{itemize}
        }
        \item<4-> Jumps into \funcname{caml\_raise\_exn}
        \item<5-> Calls \funcname{caml\_stash\_backtrace} again, to scan the next part of the stack: from the trap just pop-ed to the next trap
      \end{itemize}
      \vfill
      \mintocaml[firstline=15,lastline=21,highlightlines={18-21}]{../src/backtrace/backtrace.ml}
      \endminipage
    \end{column}
    \begin{column}{0.5\textwidth}
      \raggedleft
      \onslide*<1>{\listCamlReraiseExn{}}
      \onslide*<2>{\listCamlReraiseExn[highlightlines=729]{}}
      \onslide*<3>{
        \mintc[firstline=190,lastline=192,highlightlines={191-192}]{../ocaml/runtime/amd64.S}
        \mintc[firstline=234,lastline=237,highlightlines={235-237}]{../ocaml/runtime/amd64.S}
        \medskip
        \listCamlReraiseExn[highlightlines=731]{}
      }
      \onslide*<4-5>{
        % TODO refactor with \listCamlReraiseExn
        \mintasm[firstline=727,lastline=734,highlightlines=730]{../ocaml/runtime/amd64.S}
        \medskip
      }
      \onslide*<4>{\listCamlRaiseExn[highlightlines=711]{}}
      \onslide*<5>{\listCamlRaiseExn[highlightlines=720]{}}
    \end{column}
  \end{columns}
\end{frame}

\begin{frame}{Backtraces}{Backtrace collection continues}
  \begin{columns}[c]
    \begin{column}{0.55\textwidth}
      \minipage[c][0.75\textheight][s]{\columnwidth}
      \begin{itemize}
        \item<1-> \funcname{caml\_stash\_backtrace} scans the older part of the stack, up to the next trap
        \item<6-> Controls jumps to the nested exception handler in \funcname{maybe\_raise}, which matches only on \typename{ExnB}
        \item<7-> \funcname{caml\_reraise\_exn} is called again, backtrace collection resumes with \funcname{caml\_stash\_backtrace}
      \end{itemize}
      \medskip
      \begin{itemize}
        \item<2-> Constructed backtrace:
        \begin{itemize}
          \item <2-> \funcname{definitely\_raise}
          \item <2-> \funcname{probably\_raise}
          \item <3-> \funcname{probably\_raise} (re-raise)
          \item <4-> \funcname{maybe\_raise}
          \item <5-> \funcname{main}
          \item <8-> \funcname{main} (re-raise)
        \end{itemize}
      \end{itemize}
      \vfill
      \onslide*<1-5>{\mintocaml[firstline=15,lastline=21]{../src/backtrace/backtrace.ml}}
      \onslide*<6>{\mintocaml[firstline=28,lastline=34,highlightlines=31]{../src/backtrace/backtrace.ml}}
      \onslide*<7->{\mintocaml[firstline=28,lastline=34,highlightlines=33]{../src/backtrace/backtrace.ml}}
      \endminipage
    \end{column}
    \begin{column}{0.45\textwidth}
      \raggedleft
      \onslide*<1>{\providecommand\step{6}\input{diagrams/backtrace/backtrace.tex}}%
      \onslide*<2>{\providecommand\step{6}\input{diagrams/backtrace/backtrace.tex}}%
      \onslide*<3>{\providecommand\step{7}\input{diagrams/backtrace/backtrace.tex}}%
      \onslide*<4>{\providecommand\step{8}\input{diagrams/backtrace/backtrace.tex}}%
      \onslide*<5>{\providecommand\step{9}\input{diagrams/backtrace/backtrace.tex}}%
      \onslide*<6>{\providecommand\step{10}\input{diagrams/backtrace/backtrace.tex}}%
      \onslide*<7>{\providecommand\step{11}\input{diagrams/backtrace/backtrace.tex}}%
      \onslide*<8>{\providecommand\step{12}\input{diagrams/backtrace/backtrace.tex}}%
      \onslide*<9>{\providecommand\step{13}\input{diagrams/backtrace/backtrace.tex}}%
    \end{column}
  \end{columns}
\end{frame}

\begin{frame}{Backtraces}{Printing the backtrace}
  \begin{columns}[c]
    \begin{column}{0.45\textwidth}
      \minipage[c][0.66\textheight][s]{\columnwidth}
      \begin{itemize}
        \item<1-> The exception backtrace can now be printed to \funcarg{stdout} with \funcname{Printexc.print_backtrace}
        \item<2-> First the exception backtrace is retrieved into an OCaml array with \funcname{get_raw_backtrace }
        \item<3-> Which is implemented as the \funcname{caml_get_exception_raw_backtrace} C function in the runtime
        \item<4-> And finally printed with \funcname{print_exception_backtrace}
      \end{itemize}
      \vfill
      \mintocaml[firstline=28,lastline=34,highlightlines=33]{../src/backtrace/backtrace.ml}
      \endminipage
    \end{column}
    \begin{column}{0.55\textwidth}
      \onslide*<1,2>{
        \mintocaml[firstline=100,lastline=101]{../ocaml/stdlib/printexc.ml}
      }
      \only<1>{
        \listocaml[firstline=170,lastline=172]{../ocaml/stdlib/printexc.ml}{stdlib/printexc.ml:170}
      }
      \only<2>{
        \listocaml[firstline=170,lastline=172,highlightlines=172]{../ocaml/stdlib/printexc.ml}{stdlib/printexc.ml:170}
      }
      \only<3>{
        \mintc[firstline=154,lastline=156]{../ocaml/runtime/backtrace.c}
        \listc[firstline=171,lastline=192,highlightlines={184-187}]{../ocaml/runtime/backtrace.c}{runtime/backtrace.c:154}
      }
      \only<4>{
        \listocaml[firstline=155,lastline=165]{../ocaml/stdlib/printexc.ml}{stdlib/printexc.ml:55}
      }
    \end{column}
  \end{columns}
\end{frame}


\subsection{The multiple kinds of raise}
\frameSubsection{}{}

\begin{frame}{Backtraces}{When backtraces are not needed}
  \begin{columns}[c]
    \begin{column}{0.45\textwidth}
      \minipage[c][0.66\textheight][s]{\columnwidth}
      \begin{itemize}
        \item<1-> What if the backtrace for an exception is never needed?
        \item<2-> It's possible to raise the exception explicitly with \funcname{raise\_notrace} to make raising as cheap as possible
        \item<3-> An example of use in the \funcname{dynlink} library of the OCaml compiler
      \end{itemize}
      \vfill
      \onslide<3->{
      \listocaml[firstline=205,lastline=209,highlightlines=207]{../ocaml/otherlibs/dynlink/dynlink\_compilerlibs/misc.ml}{otherlibs/dynlink/dynlink\_compilerlibs/misc.ml:205}
      }
      \endminipage
    \end{column}
    \begin{column}{0.55\textwidth}
    \only<4>{
      \mintcmm[highlightlines=3]{../src/notrace/camlNotrace\_\_fun_347.cmm}
      \listcmm{../src/notrace/camlNotrace\_\_all\_somes\_327.cmm}{all\_somes}
    }
    \onslide*<5->{
      \listobjdump[highlightlines={6-9}]{../src/notrace/camlNotrace\_\_fun_347.objdump}{anonymous function from \funcname{all\_somes}}
    }
    \end{column}
  \end{columns}
\end{frame}

\begin{frame}{Backtraces}{Summary table of the multiple kinds of raise}
  \begin{itemize}
    \item When debugging information is enabled and if the backtrace of an exception isn't needed, it's possible to raise with \funcname{raise\_notrace} for performance
    \item When debugging information is \emph{not} enabled, the compiler optimises \funcname{raise} calls to inline assembly (same effect as using \funcname{raise\_notrace})
  \end{itemize}
  \bigskip
  \centering
  \begin{tabular}{| l | c | c | c | c |}
    \hline
    \parbox{10em}{Debug information \\ (\funcarg{-g} flag)}  & \multicolumn{2}{|c|}{Disabled}                                     & \multicolumn{2}{|c|}{Enabled} \\
    \hline
    Raise kind                                               & \funcname{raise} & \funcname{raise\_notrace}                       & \funcname{raise} & \funcname{raise\_notrace} \\
    \hline
    Compilation                                              & \parbox{8em}{inline assembly\\ (optimization)} & inline assembly   & \funcname{caml\_raise\_exn} & inline assembly \\
    \hline
  \end{tabular}
\end{frame}


%
% Takeaway #5
%
\subsection*{Takeaway \#5}
\frameSubsectionTakeaway{}
\againframe<5>{takeaway}
}%
      \onslide*<8>{\providecommand\step{12}%
% Backtraces
%
\subsection{One advantage of raising with debugging information}
\frameSubsection{}{}

\begin{frame}{Backtraces}{Debugging information \& source code}
  \begin{columns}[c]
    \begin{column}{0.5\textwidth}
      \begin{itemize}
        \item Raising an exception is fun and games, but how do we know where the exception originated? $\rightarrow$ With the backtrace associated with the exception!
        \item In order to get a backtrace from the point where an exception was raised, it's necessary to compile with debugging information enabled (\funcarg{-g} flag for the compiler)
        \item Same example as in the `Nested handlers' section
      \end{itemize}
    \end{column}
    \begin{column}{0.5\textwidth}
      \listocaml[firstline=9,lastline=34]{../src/backtrace/backtrace.ml}{backtrace/backtrace.ml}
    \end{column}
  \end{columns}
\end{frame}

\begin{frame}{Backtraces}{CMM and disassembly of \funcname{definitely\_raise}}
  \begin{columns}[c]
    \begin{column}{0.5\textwidth}
      \minipage[c][0.66\textheight][s]{\columnwidth}
      \begin{itemize}
        \item<1-> An effect of compiling with debugging information (\funcarg{-g}): the CMM no longer uses \funcname{raise\_notrace} to raise the exception but \funcname{raise} instead
        \item<2-> Disassembly shows that CMM \funcname{raise} is actually a call to \funcname{caml\_raise\_exn}, a function in assembler provided by the runtime
      \end{itemize}
      \vfill
      \mintocaml[firstline=9,lastline=13,highlightlines=12]{../src/backtrace/backtrace.ml}
      \endminipage
    \end{column}
    \begin{column}{0.5\textwidth}
      \onslide*<1>{
        \mintcmm[highlightlines=5]{../src/backtrace/camlBacktrace\_\_definitely\_raise\_347.cmm}
      }
      \onslide*<2>{
        \mintobjdump[firstline=2,highlightlines=14]{../src/backtrace/camlBacktrace\_\_definitely\_raise\_347.objdump}
      }
    \end{column}
  \end{columns}
\end{frame}

\begin{frame}{Backtraces}{Introducing \funcname{caml\_raise\_exn}}
  \begin{columns}[c]
    \begin{column}{0.5\textwidth}
      \minipage[c][0.66\textheight][s]{\columnwidth}
      \onslide*<1>{
        \begin{itemize}
          \item Raising the exception is performed using \funcname{caml\_raise\_exn} which is
            \begin{itemize}
              \item An assembly function provided by the runtime
              \item Architecture specific
            \end{itemize}
          \item Let's see what it does differently from \funcname{raise\_notrace}\dots
          \item We are about to raise \typename{ExnC} with \funcname{caml\_raise\_exn} from \funcname{definitely\_raise}
        \end{itemize}
      }
      \onslide*<2>{
        \mintobjdump[firstline=2,highlightlines=14]{../src/backtrace/camlBacktrace\_\_definitely\_raise\_347.objdump}
      }
      \vfill
      \mintocaml[firstline=9,lastline=13,highlightlines=12]{../src/backtrace/backtrace.ml}
      \endminipage
    \end{column}
    \begin{column}{0.5\textwidth}
      \centering
      \providecommand\step{1}\input{diagrams/backtrace/backtrace.tex}
    \end{column}
  \end{columns}
\end{frame}

\begin{frame}{Backtraces}{But before that, we need more fields from \typename{caml\_domain\_state}}
  \begin{columns}
    \begin{column}{0.5\textwidth}
      \begin{itemize}
        \item Remember \typename{caml\_domain\_state} the per-domain runtime state?
        \item Some other members are of interest to us now\dots
        \item \localname{backtrace\_active} is used to know if the runtime should collect backtraces
        \item \localname{backtrace\_active} can be set by
          \begin{itemize}
            \item The \funcarg{b} flag in the \funcname{OCAMLRUNPARAM} environment variable
            \item \mintinline{ocaml}{Printexc.record_backtrace : bool -> unit} \footnotemark
          \end{itemize}
      \end{itemize}
    \end{column}
    \begin{column}{0.5\textwidth}
      \begin{listing}
        \mintc[highlightlines={9-11}]{src/ocaml/domain\_state\_backtrace.h}
        \caption{runtime/caml/domain\_state.h}
      \end{listing}
    \end{column}
  \end{columns}
  \footnotetext[1]{https://v2.ocaml.org/api/Printexc.html}
\end{frame}

\newcommand\listCamlRaiseExn[1][]{
  \listasm[firstline=702,lastline=725,#1]{../ocaml/runtime/amd64.S}{runtime/amd64.S:702}}

\begin{frame}{Backtraces}{Walk through \funcname{caml\_raise\_exn}}
  \begin{columns}[T]
    \begin{column}{0.5\textwidth}
      \begin{itemize}
        \item<1-> First checks whether exceptions backtrace should be recorded
        \item<1-> By checking the \funcarg{domain\_state->backtrace\_active} value
        \item<2-> If not, acts as \funcname{raise\_notrace} with \funcname{RESTORE\_EXN\_HANDLER\_OCAML}
          \begin{itemize}
            \item Set \regname{RSP} to \localname{domain\_state->exn\_handler} value
            \item Restore \localname{domain\_state->exn\_handler} from trap
            \item Jump with \funcname{ret} (same effect as using \funcname{pop} and \funcname{jmp})
          \end{itemize}
        \item<3-> Otherwise, switches to the C stack to be able to execute C code
        \item<4-> Calls \funcname{caml\_stash\_backtrace}, a C function provided by the runtime\ignorespaces
          \onslide<6->{, with
            \begin{itemize}
              \item \funcarg{exn}: exception value
              \item \funcarg{pc}: code address of the raising point
              \item \funcarg{sp}: stack address of the raising function
              \item \funcarg{trapsp}: stack address of the next handler
            \end{itemize}
          }
      \end{itemize}
      \bigskip
      \mintocaml[firstline=9,lastline=13,highlightlines=12]{../src/backtrace/backtrace.ml}
    \end{column}
    \begin{column}{0.5\textwidth}
      \raggedleft
      \onslide*<1,3-4>{
        \mintc[firstline=190,lastline=192]{../ocaml/runtime/amd64.S}
        \mintc[firstline=234,lastline=237]{../ocaml/runtime/amd64.S}
      }
      \onslide*<2>{
        \mintc[firstline=190,lastline=192,highlightlines={191-192}]{../ocaml/runtime/amd64.S}
        \mintc[firstline=234,lastline=237,highlightlines={235-237}]{../ocaml/runtime/amd64.S}
      }
      \onslide*<1>{\listCamlRaiseExn[highlightlines=705]{}}%
      \onslide*<2>{\listCamlRaiseExn[highlightlines={707-708}]{}}%
      \onslide*<3>{\listCamlRaiseExn[highlightlines={712-714}]{}}%
      \onslide*<4>{\listCamlRaiseExn[highlightlines={715-720}]{}}%
      \onslide*<5>{\providecommand\step{1}\input{diagrams/backtrace/backtrace.tex}}%
      \onslide*<6>{\providecommand\step{2}\input{diagrams/backtrace/backtrace.tex}}%
    \end{column}
  \end{columns}
\end{frame}

\newcommand\listStashBacktrace[1][]{
  \listc[firstline=91,lastline=120,#1]{../ocaml/runtime/backtrace\_nat.c}{runtime/backtrace\_nat.c:91}}

\begin{frame}{Backtraces}{Introducing \funcname{caml\_stash\_backtrace}}
  \begin{columns}[c]
    \begin{column}{0.5\textwidth}
      \minipage[c][0.66\textheight][s]{\columnwidth}
      \begin{itemize}
        \item<1-> A C function provided by the runtime (specific to native code)
        \item<2-> It scans the stack, between the stack pointer at the raising point up to the next trap, in order to collect the names of the detected function frames
          \onslide*<2>{
            \begin{itemize}
              \item And more information, like line number, column number...
            \end{itemize}
          }
        \item<3-> It uses the same technique and information as the Garbage Collector (GC) uses to scan the values live on the stack
          \onslide*<4>{
            \begin{itemize}
              \item \typename{frame\_descr} are at the core of stack unwinding
            \end{itemize}
          }
      \end{itemize}
      \vfill
      \mintocaml[firstline=9,lastline=13,highlightlines=12]{../src/backtrace/backtrace.ml}
      \endminipage
    \end{column}
    \begin{column}{0.5\textwidth}
      \onslide*<1>{\listStashBacktrace[highlightlines=91]{}}
      \onslide*<2>{\listStashBacktrace[highlightlines={91,117-118}]{}}
      \onslide*<3->{\listStashBacktrace[highlightlines={94,106,109-110}]{}}
    \end{column}
  \end{columns}
\end{frame}

\newcommand\listFrameDescr[1][]{
  \listocaml[firstline=111,lastline=124,#1]{../ocaml/asmcomp/emitaux.ml}{asmcomp/emitaux.ml:111}}
\newcommand\listDebugInfo[1][]{
  \listocaml[firstline=38,lastline=49,#1]{../ocaml/lambda/debuginfo.mli}{lambda/debuginfo.mli:38}}

\begin{frame}{Backtraces}{\typename{frame\_descr} and \typename{frame\_debuginfo} in a nutshell}
  \begin{columns}[c]
    \begin{column}{0.5\textwidth}
      \begin{itemize}
        \item<1-> For every return address in an OCaml program, there is a associated \typename{frame\_descr}
        \item<2-> A \typename{frame\_descr} specifies
        \begin{itemize}
          \item<2-> The size of the function stack frame
          \item<3-> Where live values are on the stack (so that the GC can mark them as live and prevent from being swept)
        \end{itemize}
        \item<4-> When compiled with debug information, \typename{frame\_descr} will be linked to a \typename{frame\_debuginfo}
        \begin{itemize}
          \item<5-> Contains information about the position in the source code (file name, line and column number)
        \end{itemize}
      \end{itemize}
    \end{column}
    \begin{column}{0.5\textwidth}
        \onslide*<1>{\listFrameDescr[highlightlines={118-122}]{}}
        \onslide*<2>{\listFrameDescr[highlightlines={120}]{}}
        \onslide*<3>{\listFrameDescr[highlightlines={121}]{}}
        \onslide*<4->{\listFrameDescr[highlightlines={122}]{}}
        \medskip
        \onslide*<1-3>{\listDebugInfo{}}
        \onslide*<4>{\listDebugInfo[highlightlines={38,49}]{}}
        \onslide*<5>{\listDebugInfo[highlightlines={39-46}]{}}
    \end{column}
  \end{columns}
\end{frame}

\begin{frame}{Backtraces}{Scanning the stack with \typename{frame\_descr}}
  \begin{columns}[c]
    \begin{column}{0.5\textwidth}
      \begin{itemize}
        \item Given the return address of the code that raises the exception by calling \funcname{caml\_raise\_exn} (\funcarg{pc} argument of \funcname{caml\_stash\_backtrace})
        \item And the position of the stack pointer (\funcarg{sp} argument of \funcname{caml\_stash\_backtrace})
        \item The frame size given by \typename{frame\_descr}.\funcarg{frame\_size}, allows to find the end of the previous frame
        \item And the return address of the caller of this function
        \bigskip
        \onslide<3->{
        \item Note that traps (here the reddish \funcname{ExnA}, \funcname{ExnB} and \funcname{ExnC} blocks) belong to the frame that installed them, and so the frame size changes when traps are removed
        }
      \end{itemize}
    \end{column}
    \begin{column}{0.5\textwidth}
      \raggedleft
      \onslide*<1>{\providecommand\step{2}\input{diagrams/backtrace/backtrace.tex}}%
      \onslide*<2>{\providecommand\step{1}\input{diagrams/backtrace/scanstack.tex}}%
      \onslide*<3>{\providecommand\step{2}\input{diagrams/backtrace/scanstack.tex}}%
      \onslide*<4>{\providecommand\step{3}\input{diagrams/backtrace/scanstack.tex}}%
      \onslide*<5>{\providecommand\step{4}\input{diagrams/backtrace/scanstack.tex}}%
      \onslide*<6>{\providecommand\step{5}\input{diagrams/backtrace/scanstack.tex}}%
    \end{column}
  \end{columns}
\end{frame}

% Duplicate of previous-previous slide
\begin{frame}{Backtraces}{Continuing on \funcname{caml\_stash\_backtrace}}
  \begin{columns}[c]
    \begin{column}{0.5\textwidth}
      \minipage[c][0.75\textheight][s]{\columnwidth}
      \begin{itemize}
          \color{gray}
        \item A C function provided by the runtime (specific to native code)
        \item It scans the stack, between the stack pointer at the raising point up to the next trap, in order to collect the names of the detected function frames
        \item It uses the same technique and information as the Garbage Collector (GC) uses to scan the values live on the stack
      \end{itemize}
      \begin{itemize}
        \item For each function frame detected, store a pointer to the \typename{frame\_descr} in the \localname{domain\_state->backtrace\_buffer} array, effectively constructing the backtrace
      \end{itemize}
      \onslide<2->{
        \begin{itemize}
            \smallskip
          \item Constructed backtrace:
            \funcname{definitely\_raise},
            \onslide<3->{\funcname{probably\_raise}}
        \end{itemize}
      }
      \vfill
      \mintocaml[firstline=9,lastline=13,highlightlines=12]{../src/backtrace/backtrace.ml}
      \endminipage
    \end{column}
    \begin{column}{0.5\textwidth}
      \raggedleft
      \onslide*<1>{\listStashBacktrace[highlightlines={114-115}]{}}
      \onslide*<2>{\providecommand\step{3}\input{diagrams/backtrace/backtrace.tex}}%
      \onslide*<3>{\providecommand\step{4}\input{diagrams/backtrace/backtrace.tex}}%
    \end{column}
  \end{columns}
\end{frame}

\begin{frame}{Backtraces}{Back to \funcname{caml\_raise\_exn}}
  \begin{columns}[c]
    \begin{column}{0.5\textwidth}
      \minipage[c][0.66\textheight][s]{\columnwidth}
      \begin{itemize}\color{gray}
        \item Checks wheteher exceptions backtrace should be recorded, by checking the \funcarg{domain\_state->backtrace\_active} value
        \item Switches to the C stack to be able to execute C code
        \item Calls \funcname{caml\_stash\_backtrace}, to collect the backtrace up to the next trap
      \end{itemize}
      \onslide*<2->{
        \begin{itemize}
          \item<2-> Executes the exception handler in \funcname{probably\_raise} with \funcname{RESTORE\_EXN\_HANDLER\_OCAML}
            \begin{itemize}
              \item Set \regname{RSP} to \localname{domain\_state->exn\_handler} value
              \item Restore \localname{domain\_state->exn\_handler} from trap
              \item Jump to the handler code with \funcname{ret}
            \end{itemize}
        \end{itemize}
      }
      \vfill
      \onslide*<1>{\mintocaml[firstline=9,lastline=13,highlightlines=12]{../src/backtrace/backtrace.ml}}
      \onslide*<2->{\mintocaml[firstline=15,lastline=21,highlightlines={19-21}]{../src/backtrace/backtrace.ml}}
      \endminipage
    \end{column}
    \begin{column}{0.5\textwidth}
      \centering
      \onslide*<1>{
        \mintc[firstline=190,lastline=192]{../ocaml/runtime/amd64.S}
        \mintc[firstline=234,lastline=237]{../ocaml/runtime/amd64.S}
      }
      \onslide*<2>{
        \mintc[firstline=190,lastline=192,highlightlines={191-192}]{../ocaml/runtime/amd64.S}
        \mintc[firstline=234,lastline=237,highlightlines={235-237}]{../ocaml/runtime/amd64.S}
      }
      \onslide*<1>{\listCamlRaiseExn[highlightlines=720]{}}
      \onslide*<2>{\listCamlRaiseExn[highlightlines=722]{}}
      \onslide*<3->{\providecommand\step{5}\input{diagrams/backtrace/backtrace.tex}}
    \end{column}
  \end{columns}
\end{frame}

\begin{frame}{Backtraces}{\funcname{caml\_probably\_raise}'s handler doesn't catch \typename{ExnC}}
  \begin{columns}[c]
    \begin{column}{0.45\textwidth}
      \minipage[c][0.75\textheight][s]{\columnwidth}
      \begin{itemize}
        \item<1-> \funcname{match \dots with}\footnotemark pattern matching can also match exceptions
        \item<1-> The CMM of \funcname{probably\_raise} shows that this \funcname{match \dots with} is implemented with a pattern matching and a \funcname{try \dots with}
        \item<1-> But this one only matches on \typename{ExnA} exception
        \item<2-> Since the pattern matching fails to catch the exception, \funcname{reraise} is called with our current \typename{ExnC} exception
        \item<3-> Disassembly shows that \funcname{reraise} is actually a call to \funcname{caml\_reraise\_exn}, which is also an assembly function provided by the runtime
      \end{itemize}
      \vfill
      \mintocaml[firstline=15,lastline=21,highlightlines={18-21}]{../src/backtrace/backtrace.ml}
      \endminipage
    \end{column}
    \begin{column}{0.55\textwidth}
      \centering
      \onslide*<1>{\mintcmm[highlightlines={10-18}]{../src/backtrace/camlBacktrace\_\_probably\_raise\_351.cmm}}
      \onslide*<2>{\listcmm[highlightlines=18]{../src/backtrace/camlBacktrace\_\_probably\_raise_351.cmm}{CMM of probably\_raise}}
      \onslide*<3>{\mintobjdump[firstline=5,lastline=35,highlightlines=31]{../src/backtrace/camlBacktrace\_\_probably\_raise_351.objdump}}
    \end{column}
  \end{columns}
  \only<1>{\footnotetext[1]{https://blog.janestreet.com/pattern-matching-and-exception-handling-unite/}}
\end{frame}

\newcommand\listCamlReraiseExn[1][]{
  \listasm[firstline=727,lastline=734,#1]{../ocaml/runtime/amd64.S}{runtime/amd64.S:727}}

\begin{frame}{Backtraces}{Introducing \funcname{caml\_reraise\_exn}}
  \begin{columns}[c]
    \begin{column}{0.5\textwidth}
      \minipage[c][0.66\textheight][s]{\columnwidth}
      \begin{itemize}
        \item<1-> The exception have to be reraised to the next handler
        \item<2-> First, checks if the backtrace should be collected with \funcarg{domain\_state->backtrace\_active}
        \only<3>{
        \begin{itemize}
            \item If not, behaves like a \funcname{raise\_notrace} with \funcname{RESTORE\_EXN\_HANDLER\_OCAML}
        \end{itemize}
        }
        \item<4-> Jumps into \funcname{caml\_raise\_exn}
        \item<5-> Calls \funcname{caml\_stash\_backtrace} again, to scan the next part of the stack: from the trap just pop-ed to the next trap
      \end{itemize}
      \vfill
      \mintocaml[firstline=15,lastline=21,highlightlines={18-21}]{../src/backtrace/backtrace.ml}
      \endminipage
    \end{column}
    \begin{column}{0.5\textwidth}
      \raggedleft
      \onslide*<1>{\listCamlReraiseExn{}}
      \onslide*<2>{\listCamlReraiseExn[highlightlines=729]{}}
      \onslide*<3>{
        \mintc[firstline=190,lastline=192,highlightlines={191-192}]{../ocaml/runtime/amd64.S}
        \mintc[firstline=234,lastline=237,highlightlines={235-237}]{../ocaml/runtime/amd64.S}
        \medskip
        \listCamlReraiseExn[highlightlines=731]{}
      }
      \onslide*<4-5>{
        % TODO refactor with \listCamlReraiseExn
        \mintasm[firstline=727,lastline=734,highlightlines=730]{../ocaml/runtime/amd64.S}
        \medskip
      }
      \onslide*<4>{\listCamlRaiseExn[highlightlines=711]{}}
      \onslide*<5>{\listCamlRaiseExn[highlightlines=720]{}}
    \end{column}
  \end{columns}
\end{frame}

\begin{frame}{Backtraces}{Backtrace collection continues}
  \begin{columns}[c]
    \begin{column}{0.55\textwidth}
      \minipage[c][0.75\textheight][s]{\columnwidth}
      \begin{itemize}
        \item<1-> \funcname{caml\_stash\_backtrace} scans the older part of the stack, up to the next trap
        \item<6-> Controls jumps to the nested exception handler in \funcname{maybe\_raise}, which matches only on \typename{ExnB}
        \item<7-> \funcname{caml\_reraise\_exn} is called again, backtrace collection resumes with \funcname{caml\_stash\_backtrace}
      \end{itemize}
      \medskip
      \begin{itemize}
        \item<2-> Constructed backtrace:
        \begin{itemize}
          \item <2-> \funcname{definitely\_raise}
          \item <2-> \funcname{probably\_raise}
          \item <3-> \funcname{probably\_raise} (re-raise)
          \item <4-> \funcname{maybe\_raise}
          \item <5-> \funcname{main}
          \item <8-> \funcname{main} (re-raise)
        \end{itemize}
      \end{itemize}
      \vfill
      \onslide*<1-5>{\mintocaml[firstline=15,lastline=21]{../src/backtrace/backtrace.ml}}
      \onslide*<6>{\mintocaml[firstline=28,lastline=34,highlightlines=31]{../src/backtrace/backtrace.ml}}
      \onslide*<7->{\mintocaml[firstline=28,lastline=34,highlightlines=33]{../src/backtrace/backtrace.ml}}
      \endminipage
    \end{column}
    \begin{column}{0.45\textwidth}
      \raggedleft
      \onslide*<1>{\providecommand\step{6}\input{diagrams/backtrace/backtrace.tex}}%
      \onslide*<2>{\providecommand\step{6}\input{diagrams/backtrace/backtrace.tex}}%
      \onslide*<3>{\providecommand\step{7}\input{diagrams/backtrace/backtrace.tex}}%
      \onslide*<4>{\providecommand\step{8}\input{diagrams/backtrace/backtrace.tex}}%
      \onslide*<5>{\providecommand\step{9}\input{diagrams/backtrace/backtrace.tex}}%
      \onslide*<6>{\providecommand\step{10}\input{diagrams/backtrace/backtrace.tex}}%
      \onslide*<7>{\providecommand\step{11}\input{diagrams/backtrace/backtrace.tex}}%
      \onslide*<8>{\providecommand\step{12}\input{diagrams/backtrace/backtrace.tex}}%
      \onslide*<9>{\providecommand\step{13}\input{diagrams/backtrace/backtrace.tex}}%
    \end{column}
  \end{columns}
\end{frame}

\begin{frame}{Backtraces}{Printing the backtrace}
  \begin{columns}[c]
    \begin{column}{0.45\textwidth}
      \minipage[c][0.66\textheight][s]{\columnwidth}
      \begin{itemize}
        \item<1-> The exception backtrace can now be printed to \funcarg{stdout} with \funcname{Printexc.print_backtrace}
        \item<2-> First the exception backtrace is retrieved into an OCaml array with \funcname{get_raw_backtrace }
        \item<3-> Which is implemented as the \funcname{caml_get_exception_raw_backtrace} C function in the runtime
        \item<4-> And finally printed with \funcname{print_exception_backtrace}
      \end{itemize}
      \vfill
      \mintocaml[firstline=28,lastline=34,highlightlines=33]{../src/backtrace/backtrace.ml}
      \endminipage
    \end{column}
    \begin{column}{0.55\textwidth}
      \onslide*<1,2>{
        \mintocaml[firstline=100,lastline=101]{../ocaml/stdlib/printexc.ml}
      }
      \only<1>{
        \listocaml[firstline=170,lastline=172]{../ocaml/stdlib/printexc.ml}{stdlib/printexc.ml:170}
      }
      \only<2>{
        \listocaml[firstline=170,lastline=172,highlightlines=172]{../ocaml/stdlib/printexc.ml}{stdlib/printexc.ml:170}
      }
      \only<3>{
        \mintc[firstline=154,lastline=156]{../ocaml/runtime/backtrace.c}
        \listc[firstline=171,lastline=192,highlightlines={184-187}]{../ocaml/runtime/backtrace.c}{runtime/backtrace.c:154}
      }
      \only<4>{
        \listocaml[firstline=155,lastline=165]{../ocaml/stdlib/printexc.ml}{stdlib/printexc.ml:55}
      }
    \end{column}
  \end{columns}
\end{frame}


\subsection{The multiple kinds of raise}
\frameSubsection{}{}

\begin{frame}{Backtraces}{When backtraces are not needed}
  \begin{columns}[c]
    \begin{column}{0.45\textwidth}
      \minipage[c][0.66\textheight][s]{\columnwidth}
      \begin{itemize}
        \item<1-> What if the backtrace for an exception is never needed?
        \item<2-> It's possible to raise the exception explicitly with \funcname{raise\_notrace} to make raising as cheap as possible
        \item<3-> An example of use in the \funcname{dynlink} library of the OCaml compiler
      \end{itemize}
      \vfill
      \onslide<3->{
      \listocaml[firstline=205,lastline=209,highlightlines=207]{../ocaml/otherlibs/dynlink/dynlink\_compilerlibs/misc.ml}{otherlibs/dynlink/dynlink\_compilerlibs/misc.ml:205}
      }
      \endminipage
    \end{column}
    \begin{column}{0.55\textwidth}
    \only<4>{
      \mintcmm[highlightlines=3]{../src/notrace/camlNotrace\_\_fun_347.cmm}
      \listcmm{../src/notrace/camlNotrace\_\_all\_somes\_327.cmm}{all\_somes}
    }
    \onslide*<5->{
      \listobjdump[highlightlines={6-9}]{../src/notrace/camlNotrace\_\_fun_347.objdump}{anonymous function from \funcname{all\_somes}}
    }
    \end{column}
  \end{columns}
\end{frame}

\begin{frame}{Backtraces}{Summary table of the multiple kinds of raise}
  \begin{itemize}
    \item When debugging information is enabled and if the backtrace of an exception isn't needed, it's possible to raise with \funcname{raise\_notrace} for performance
    \item When debugging information is \emph{not} enabled, the compiler optimises \funcname{raise} calls to inline assembly (same effect as using \funcname{raise\_notrace})
  \end{itemize}
  \bigskip
  \centering
  \begin{tabular}{| l | c | c | c | c |}
    \hline
    \parbox{10em}{Debug information \\ (\funcarg{-g} flag)}  & \multicolumn{2}{|c|}{Disabled}                                     & \multicolumn{2}{|c|}{Enabled} \\
    \hline
    Raise kind                                               & \funcname{raise} & \funcname{raise\_notrace}                       & \funcname{raise} & \funcname{raise\_notrace} \\
    \hline
    Compilation                                              & \parbox{8em}{inline assembly\\ (optimization)} & inline assembly   & \funcname{caml\_raise\_exn} & inline assembly \\
    \hline
  \end{tabular}
\end{frame}


%
% Takeaway #5
%
\subsection*{Takeaway \#5}
\frameSubsectionTakeaway{}
\againframe<5>{takeaway}
}%
      \onslide*<9>{\providecommand\step{13}%
% Backtraces
%
\subsection{One advantage of raising with debugging information}
\frameSubsection{}{}

\begin{frame}{Backtraces}{Debugging information \& source code}
  \begin{columns}[c]
    \begin{column}{0.5\textwidth}
      \begin{itemize}
        \item Raising an exception is fun and games, but how do we know where the exception originated? $\rightarrow$ With the backtrace associated with the exception!
        \item In order to get a backtrace from the point where an exception was raised, it's necessary to compile with debugging information enabled (\funcarg{-g} flag for the compiler)
        \item Same example as in the `Nested handlers' section
      \end{itemize}
    \end{column}
    \begin{column}{0.5\textwidth}
      \listocaml[firstline=9,lastline=34]{../src/backtrace/backtrace.ml}{backtrace/backtrace.ml}
    \end{column}
  \end{columns}
\end{frame}

\begin{frame}{Backtraces}{CMM and disassembly of \funcname{definitely\_raise}}
  \begin{columns}[c]
    \begin{column}{0.5\textwidth}
      \minipage[c][0.66\textheight][s]{\columnwidth}
      \begin{itemize}
        \item<1-> An effect of compiling with debugging information (\funcarg{-g}): the CMM no longer uses \funcname{raise\_notrace} to raise the exception but \funcname{raise} instead
        \item<2-> Disassembly shows that CMM \funcname{raise} is actually a call to \funcname{caml\_raise\_exn}, a function in assembler provided by the runtime
      \end{itemize}
      \vfill
      \mintocaml[firstline=9,lastline=13,highlightlines=12]{../src/backtrace/backtrace.ml}
      \endminipage
    \end{column}
    \begin{column}{0.5\textwidth}
      \onslide*<1>{
        \mintcmm[highlightlines=5]{../src/backtrace/camlBacktrace\_\_definitely\_raise\_347.cmm}
      }
      \onslide*<2>{
        \mintobjdump[firstline=2,highlightlines=14]{../src/backtrace/camlBacktrace\_\_definitely\_raise\_347.objdump}
      }
    \end{column}
  \end{columns}
\end{frame}

\begin{frame}{Backtraces}{Introducing \funcname{caml\_raise\_exn}}
  \begin{columns}[c]
    \begin{column}{0.5\textwidth}
      \minipage[c][0.66\textheight][s]{\columnwidth}
      \onslide*<1>{
        \begin{itemize}
          \item Raising the exception is performed using \funcname{caml\_raise\_exn} which is
            \begin{itemize}
              \item An assembly function provided by the runtime
              \item Architecture specific
            \end{itemize}
          \item Let's see what it does differently from \funcname{raise\_notrace}\dots
          \item We are about to raise \typename{ExnC} with \funcname{caml\_raise\_exn} from \funcname{definitely\_raise}
        \end{itemize}
      }
      \onslide*<2>{
        \mintobjdump[firstline=2,highlightlines=14]{../src/backtrace/camlBacktrace\_\_definitely\_raise\_347.objdump}
      }
      \vfill
      \mintocaml[firstline=9,lastline=13,highlightlines=12]{../src/backtrace/backtrace.ml}
      \endminipage
    \end{column}
    \begin{column}{0.5\textwidth}
      \centering
      \providecommand\step{1}\input{diagrams/backtrace/backtrace.tex}
    \end{column}
  \end{columns}
\end{frame}

\begin{frame}{Backtraces}{But before that, we need more fields from \typename{caml\_domain\_state}}
  \begin{columns}
    \begin{column}{0.5\textwidth}
      \begin{itemize}
        \item Remember \typename{caml\_domain\_state} the per-domain runtime state?
        \item Some other members are of interest to us now\dots
        \item \localname{backtrace\_active} is used to know if the runtime should collect backtraces
        \item \localname{backtrace\_active} can be set by
          \begin{itemize}
            \item The \funcarg{b} flag in the \funcname{OCAMLRUNPARAM} environment variable
            \item \mintinline{ocaml}{Printexc.record_backtrace : bool -> unit} \footnotemark
          \end{itemize}
      \end{itemize}
    \end{column}
    \begin{column}{0.5\textwidth}
      \begin{listing}
        \mintc[highlightlines={9-11}]{src/ocaml/domain\_state\_backtrace.h}
        \caption{runtime/caml/domain\_state.h}
      \end{listing}
    \end{column}
  \end{columns}
  \footnotetext[1]{https://v2.ocaml.org/api/Printexc.html}
\end{frame}

\newcommand\listCamlRaiseExn[1][]{
  \listasm[firstline=702,lastline=725,#1]{../ocaml/runtime/amd64.S}{runtime/amd64.S:702}}

\begin{frame}{Backtraces}{Walk through \funcname{caml\_raise\_exn}}
  \begin{columns}[T]
    \begin{column}{0.5\textwidth}
      \begin{itemize}
        \item<1-> First checks whether exceptions backtrace should be recorded
        \item<1-> By checking the \funcarg{domain\_state->backtrace\_active} value
        \item<2-> If not, acts as \funcname{raise\_notrace} with \funcname{RESTORE\_EXN\_HANDLER\_OCAML}
          \begin{itemize}
            \item Set \regname{RSP} to \localname{domain\_state->exn\_handler} value
            \item Restore \localname{domain\_state->exn\_handler} from trap
            \item Jump with \funcname{ret} (same effect as using \funcname{pop} and \funcname{jmp})
          \end{itemize}
        \item<3-> Otherwise, switches to the C stack to be able to execute C code
        \item<4-> Calls \funcname{caml\_stash\_backtrace}, a C function provided by the runtime\ignorespaces
          \onslide<6->{, with
            \begin{itemize}
              \item \funcarg{exn}: exception value
              \item \funcarg{pc}: code address of the raising point
              \item \funcarg{sp}: stack address of the raising function
              \item \funcarg{trapsp}: stack address of the next handler
            \end{itemize}
          }
      \end{itemize}
      \bigskip
      \mintocaml[firstline=9,lastline=13,highlightlines=12]{../src/backtrace/backtrace.ml}
    \end{column}
    \begin{column}{0.5\textwidth}
      \raggedleft
      \onslide*<1,3-4>{
        \mintc[firstline=190,lastline=192]{../ocaml/runtime/amd64.S}
        \mintc[firstline=234,lastline=237]{../ocaml/runtime/amd64.S}
      }
      \onslide*<2>{
        \mintc[firstline=190,lastline=192,highlightlines={191-192}]{../ocaml/runtime/amd64.S}
        \mintc[firstline=234,lastline=237,highlightlines={235-237}]{../ocaml/runtime/amd64.S}
      }
      \onslide*<1>{\listCamlRaiseExn[highlightlines=705]{}}%
      \onslide*<2>{\listCamlRaiseExn[highlightlines={707-708}]{}}%
      \onslide*<3>{\listCamlRaiseExn[highlightlines={712-714}]{}}%
      \onslide*<4>{\listCamlRaiseExn[highlightlines={715-720}]{}}%
      \onslide*<5>{\providecommand\step{1}\input{diagrams/backtrace/backtrace.tex}}%
      \onslide*<6>{\providecommand\step{2}\input{diagrams/backtrace/backtrace.tex}}%
    \end{column}
  \end{columns}
\end{frame}

\newcommand\listStashBacktrace[1][]{
  \listc[firstline=91,lastline=120,#1]{../ocaml/runtime/backtrace\_nat.c}{runtime/backtrace\_nat.c:91}}

\begin{frame}{Backtraces}{Introducing \funcname{caml\_stash\_backtrace}}
  \begin{columns}[c]
    \begin{column}{0.5\textwidth}
      \minipage[c][0.66\textheight][s]{\columnwidth}
      \begin{itemize}
        \item<1-> A C function provided by the runtime (specific to native code)
        \item<2-> It scans the stack, between the stack pointer at the raising point up to the next trap, in order to collect the names of the detected function frames
          \onslide*<2>{
            \begin{itemize}
              \item And more information, like line number, column number...
            \end{itemize}
          }
        \item<3-> It uses the same technique and information as the Garbage Collector (GC) uses to scan the values live on the stack
          \onslide*<4>{
            \begin{itemize}
              \item \typename{frame\_descr} are at the core of stack unwinding
            \end{itemize}
          }
      \end{itemize}
      \vfill
      \mintocaml[firstline=9,lastline=13,highlightlines=12]{../src/backtrace/backtrace.ml}
      \endminipage
    \end{column}
    \begin{column}{0.5\textwidth}
      \onslide*<1>{\listStashBacktrace[highlightlines=91]{}}
      \onslide*<2>{\listStashBacktrace[highlightlines={91,117-118}]{}}
      \onslide*<3->{\listStashBacktrace[highlightlines={94,106,109-110}]{}}
    \end{column}
  \end{columns}
\end{frame}

\newcommand\listFrameDescr[1][]{
  \listocaml[firstline=111,lastline=124,#1]{../ocaml/asmcomp/emitaux.ml}{asmcomp/emitaux.ml:111}}
\newcommand\listDebugInfo[1][]{
  \listocaml[firstline=38,lastline=49,#1]{../ocaml/lambda/debuginfo.mli}{lambda/debuginfo.mli:38}}

\begin{frame}{Backtraces}{\typename{frame\_descr} and \typename{frame\_debuginfo} in a nutshell}
  \begin{columns}[c]
    \begin{column}{0.5\textwidth}
      \begin{itemize}
        \item<1-> For every return address in an OCaml program, there is a associated \typename{frame\_descr}
        \item<2-> A \typename{frame\_descr} specifies
        \begin{itemize}
          \item<2-> The size of the function stack frame
          \item<3-> Where live values are on the stack (so that the GC can mark them as live and prevent from being swept)
        \end{itemize}
        \item<4-> When compiled with debug information, \typename{frame\_descr} will be linked to a \typename{frame\_debuginfo}
        \begin{itemize}
          \item<5-> Contains information about the position in the source code (file name, line and column number)
        \end{itemize}
      \end{itemize}
    \end{column}
    \begin{column}{0.5\textwidth}
        \onslide*<1>{\listFrameDescr[highlightlines={118-122}]{}}
        \onslide*<2>{\listFrameDescr[highlightlines={120}]{}}
        \onslide*<3>{\listFrameDescr[highlightlines={121}]{}}
        \onslide*<4->{\listFrameDescr[highlightlines={122}]{}}
        \medskip
        \onslide*<1-3>{\listDebugInfo{}}
        \onslide*<4>{\listDebugInfo[highlightlines={38,49}]{}}
        \onslide*<5>{\listDebugInfo[highlightlines={39-46}]{}}
    \end{column}
  \end{columns}
\end{frame}

\begin{frame}{Backtraces}{Scanning the stack with \typename{frame\_descr}}
  \begin{columns}[c]
    \begin{column}{0.5\textwidth}
      \begin{itemize}
        \item Given the return address of the code that raises the exception by calling \funcname{caml\_raise\_exn} (\funcarg{pc} argument of \funcname{caml\_stash\_backtrace})
        \item And the position of the stack pointer (\funcarg{sp} argument of \funcname{caml\_stash\_backtrace})
        \item The frame size given by \typename{frame\_descr}.\funcarg{frame\_size}, allows to find the end of the previous frame
        \item And the return address of the caller of this function
        \bigskip
        \onslide<3->{
        \item Note that traps (here the reddish \funcname{ExnA}, \funcname{ExnB} and \funcname{ExnC} blocks) belong to the frame that installed them, and so the frame size changes when traps are removed
        }
      \end{itemize}
    \end{column}
    \begin{column}{0.5\textwidth}
      \raggedleft
      \onslide*<1>{\providecommand\step{2}\input{diagrams/backtrace/backtrace.tex}}%
      \onslide*<2>{\providecommand\step{1}\input{diagrams/backtrace/scanstack.tex}}%
      \onslide*<3>{\providecommand\step{2}\input{diagrams/backtrace/scanstack.tex}}%
      \onslide*<4>{\providecommand\step{3}\input{diagrams/backtrace/scanstack.tex}}%
      \onslide*<5>{\providecommand\step{4}\input{diagrams/backtrace/scanstack.tex}}%
      \onslide*<6>{\providecommand\step{5}\input{diagrams/backtrace/scanstack.tex}}%
    \end{column}
  \end{columns}
\end{frame}

% Duplicate of previous-previous slide
\begin{frame}{Backtraces}{Continuing on \funcname{caml\_stash\_backtrace}}
  \begin{columns}[c]
    \begin{column}{0.5\textwidth}
      \minipage[c][0.75\textheight][s]{\columnwidth}
      \begin{itemize}
          \color{gray}
        \item A C function provided by the runtime (specific to native code)
        \item It scans the stack, between the stack pointer at the raising point up to the next trap, in order to collect the names of the detected function frames
        \item It uses the same technique and information as the Garbage Collector (GC) uses to scan the values live on the stack
      \end{itemize}
      \begin{itemize}
        \item For each function frame detected, store a pointer to the \typename{frame\_descr} in the \localname{domain\_state->backtrace\_buffer} array, effectively constructing the backtrace
      \end{itemize}
      \onslide<2->{
        \begin{itemize}
            \smallskip
          \item Constructed backtrace:
            \funcname{definitely\_raise},
            \onslide<3->{\funcname{probably\_raise}}
        \end{itemize}
      }
      \vfill
      \mintocaml[firstline=9,lastline=13,highlightlines=12]{../src/backtrace/backtrace.ml}
      \endminipage
    \end{column}
    \begin{column}{0.5\textwidth}
      \raggedleft
      \onslide*<1>{\listStashBacktrace[highlightlines={114-115}]{}}
      \onslide*<2>{\providecommand\step{3}\input{diagrams/backtrace/backtrace.tex}}%
      \onslide*<3>{\providecommand\step{4}\input{diagrams/backtrace/backtrace.tex}}%
    \end{column}
  \end{columns}
\end{frame}

\begin{frame}{Backtraces}{Back to \funcname{caml\_raise\_exn}}
  \begin{columns}[c]
    \begin{column}{0.5\textwidth}
      \minipage[c][0.66\textheight][s]{\columnwidth}
      \begin{itemize}\color{gray}
        \item Checks wheteher exceptions backtrace should be recorded, by checking the \funcarg{domain\_state->backtrace\_active} value
        \item Switches to the C stack to be able to execute C code
        \item Calls \funcname{caml\_stash\_backtrace}, to collect the backtrace up to the next trap
      \end{itemize}
      \onslide*<2->{
        \begin{itemize}
          \item<2-> Executes the exception handler in \funcname{probably\_raise} with \funcname{RESTORE\_EXN\_HANDLER\_OCAML}
            \begin{itemize}
              \item Set \regname{RSP} to \localname{domain\_state->exn\_handler} value
              \item Restore \localname{domain\_state->exn\_handler} from trap
              \item Jump to the handler code with \funcname{ret}
            \end{itemize}
        \end{itemize}
      }
      \vfill
      \onslide*<1>{\mintocaml[firstline=9,lastline=13,highlightlines=12]{../src/backtrace/backtrace.ml}}
      \onslide*<2->{\mintocaml[firstline=15,lastline=21,highlightlines={19-21}]{../src/backtrace/backtrace.ml}}
      \endminipage
    \end{column}
    \begin{column}{0.5\textwidth}
      \centering
      \onslide*<1>{
        \mintc[firstline=190,lastline=192]{../ocaml/runtime/amd64.S}
        \mintc[firstline=234,lastline=237]{../ocaml/runtime/amd64.S}
      }
      \onslide*<2>{
        \mintc[firstline=190,lastline=192,highlightlines={191-192}]{../ocaml/runtime/amd64.S}
        \mintc[firstline=234,lastline=237,highlightlines={235-237}]{../ocaml/runtime/amd64.S}
      }
      \onslide*<1>{\listCamlRaiseExn[highlightlines=720]{}}
      \onslide*<2>{\listCamlRaiseExn[highlightlines=722]{}}
      \onslide*<3->{\providecommand\step{5}\input{diagrams/backtrace/backtrace.tex}}
    \end{column}
  \end{columns}
\end{frame}

\begin{frame}{Backtraces}{\funcname{caml\_probably\_raise}'s handler doesn't catch \typename{ExnC}}
  \begin{columns}[c]
    \begin{column}{0.45\textwidth}
      \minipage[c][0.75\textheight][s]{\columnwidth}
      \begin{itemize}
        \item<1-> \funcname{match \dots with}\footnotemark pattern matching can also match exceptions
        \item<1-> The CMM of \funcname{probably\_raise} shows that this \funcname{match \dots with} is implemented with a pattern matching and a \funcname{try \dots with}
        \item<1-> But this one only matches on \typename{ExnA} exception
        \item<2-> Since the pattern matching fails to catch the exception, \funcname{reraise} is called with our current \typename{ExnC} exception
        \item<3-> Disassembly shows that \funcname{reraise} is actually a call to \funcname{caml\_reraise\_exn}, which is also an assembly function provided by the runtime
      \end{itemize}
      \vfill
      \mintocaml[firstline=15,lastline=21,highlightlines={18-21}]{../src/backtrace/backtrace.ml}
      \endminipage
    \end{column}
    \begin{column}{0.55\textwidth}
      \centering
      \onslide*<1>{\mintcmm[highlightlines={10-18}]{../src/backtrace/camlBacktrace\_\_probably\_raise\_351.cmm}}
      \onslide*<2>{\listcmm[highlightlines=18]{../src/backtrace/camlBacktrace\_\_probably\_raise_351.cmm}{CMM of probably\_raise}}
      \onslide*<3>{\mintobjdump[firstline=5,lastline=35,highlightlines=31]{../src/backtrace/camlBacktrace\_\_probably\_raise_351.objdump}}
    \end{column}
  \end{columns}
  \only<1>{\footnotetext[1]{https://blog.janestreet.com/pattern-matching-and-exception-handling-unite/}}
\end{frame}

\newcommand\listCamlReraiseExn[1][]{
  \listasm[firstline=727,lastline=734,#1]{../ocaml/runtime/amd64.S}{runtime/amd64.S:727}}

\begin{frame}{Backtraces}{Introducing \funcname{caml\_reraise\_exn}}
  \begin{columns}[c]
    \begin{column}{0.5\textwidth}
      \minipage[c][0.66\textheight][s]{\columnwidth}
      \begin{itemize}
        \item<1-> The exception have to be reraised to the next handler
        \item<2-> First, checks if the backtrace should be collected with \funcarg{domain\_state->backtrace\_active}
        \only<3>{
        \begin{itemize}
            \item If not, behaves like a \funcname{raise\_notrace} with \funcname{RESTORE\_EXN\_HANDLER\_OCAML}
        \end{itemize}
        }
        \item<4-> Jumps into \funcname{caml\_raise\_exn}
        \item<5-> Calls \funcname{caml\_stash\_backtrace} again, to scan the next part of the stack: from the trap just pop-ed to the next trap
      \end{itemize}
      \vfill
      \mintocaml[firstline=15,lastline=21,highlightlines={18-21}]{../src/backtrace/backtrace.ml}
      \endminipage
    \end{column}
    \begin{column}{0.5\textwidth}
      \raggedleft
      \onslide*<1>{\listCamlReraiseExn{}}
      \onslide*<2>{\listCamlReraiseExn[highlightlines=729]{}}
      \onslide*<3>{
        \mintc[firstline=190,lastline=192,highlightlines={191-192}]{../ocaml/runtime/amd64.S}
        \mintc[firstline=234,lastline=237,highlightlines={235-237}]{../ocaml/runtime/amd64.S}
        \medskip
        \listCamlReraiseExn[highlightlines=731]{}
      }
      \onslide*<4-5>{
        % TODO refactor with \listCamlReraiseExn
        \mintasm[firstline=727,lastline=734,highlightlines=730]{../ocaml/runtime/amd64.S}
        \medskip
      }
      \onslide*<4>{\listCamlRaiseExn[highlightlines=711]{}}
      \onslide*<5>{\listCamlRaiseExn[highlightlines=720]{}}
    \end{column}
  \end{columns}
\end{frame}

\begin{frame}{Backtraces}{Backtrace collection continues}
  \begin{columns}[c]
    \begin{column}{0.55\textwidth}
      \minipage[c][0.75\textheight][s]{\columnwidth}
      \begin{itemize}
        \item<1-> \funcname{caml\_stash\_backtrace} scans the older part of the stack, up to the next trap
        \item<6-> Controls jumps to the nested exception handler in \funcname{maybe\_raise}, which matches only on \typename{ExnB}
        \item<7-> \funcname{caml\_reraise\_exn} is called again, backtrace collection resumes with \funcname{caml\_stash\_backtrace}
      \end{itemize}
      \medskip
      \begin{itemize}
        \item<2-> Constructed backtrace:
        \begin{itemize}
          \item <2-> \funcname{definitely\_raise}
          \item <2-> \funcname{probably\_raise}
          \item <3-> \funcname{probably\_raise} (re-raise)
          \item <4-> \funcname{maybe\_raise}
          \item <5-> \funcname{main}
          \item <8-> \funcname{main} (re-raise)
        \end{itemize}
      \end{itemize}
      \vfill
      \onslide*<1-5>{\mintocaml[firstline=15,lastline=21]{../src/backtrace/backtrace.ml}}
      \onslide*<6>{\mintocaml[firstline=28,lastline=34,highlightlines=31]{../src/backtrace/backtrace.ml}}
      \onslide*<7->{\mintocaml[firstline=28,lastline=34,highlightlines=33]{../src/backtrace/backtrace.ml}}
      \endminipage
    \end{column}
    \begin{column}{0.45\textwidth}
      \raggedleft
      \onslide*<1>{\providecommand\step{6}\input{diagrams/backtrace/backtrace.tex}}%
      \onslide*<2>{\providecommand\step{6}\input{diagrams/backtrace/backtrace.tex}}%
      \onslide*<3>{\providecommand\step{7}\input{diagrams/backtrace/backtrace.tex}}%
      \onslide*<4>{\providecommand\step{8}\input{diagrams/backtrace/backtrace.tex}}%
      \onslide*<5>{\providecommand\step{9}\input{diagrams/backtrace/backtrace.tex}}%
      \onslide*<6>{\providecommand\step{10}\input{diagrams/backtrace/backtrace.tex}}%
      \onslide*<7>{\providecommand\step{11}\input{diagrams/backtrace/backtrace.tex}}%
      \onslide*<8>{\providecommand\step{12}\input{diagrams/backtrace/backtrace.tex}}%
      \onslide*<9>{\providecommand\step{13}\input{diagrams/backtrace/backtrace.tex}}%
    \end{column}
  \end{columns}
\end{frame}

\begin{frame}{Backtraces}{Printing the backtrace}
  \begin{columns}[c]
    \begin{column}{0.45\textwidth}
      \minipage[c][0.66\textheight][s]{\columnwidth}
      \begin{itemize}
        \item<1-> The exception backtrace can now be printed to \funcarg{stdout} with \funcname{Printexc.print_backtrace}
        \item<2-> First the exception backtrace is retrieved into an OCaml array with \funcname{get_raw_backtrace }
        \item<3-> Which is implemented as the \funcname{caml_get_exception_raw_backtrace} C function in the runtime
        \item<4-> And finally printed with \funcname{print_exception_backtrace}
      \end{itemize}
      \vfill
      \mintocaml[firstline=28,lastline=34,highlightlines=33]{../src/backtrace/backtrace.ml}
      \endminipage
    \end{column}
    \begin{column}{0.55\textwidth}
      \onslide*<1,2>{
        \mintocaml[firstline=100,lastline=101]{../ocaml/stdlib/printexc.ml}
      }
      \only<1>{
        \listocaml[firstline=170,lastline=172]{../ocaml/stdlib/printexc.ml}{stdlib/printexc.ml:170}
      }
      \only<2>{
        \listocaml[firstline=170,lastline=172,highlightlines=172]{../ocaml/stdlib/printexc.ml}{stdlib/printexc.ml:170}
      }
      \only<3>{
        \mintc[firstline=154,lastline=156]{../ocaml/runtime/backtrace.c}
        \listc[firstline=171,lastline=192,highlightlines={184-187}]{../ocaml/runtime/backtrace.c}{runtime/backtrace.c:154}
      }
      \only<4>{
        \listocaml[firstline=155,lastline=165]{../ocaml/stdlib/printexc.ml}{stdlib/printexc.ml:55}
      }
    \end{column}
  \end{columns}
\end{frame}


\subsection{The multiple kinds of raise}
\frameSubsection{}{}

\begin{frame}{Backtraces}{When backtraces are not needed}
  \begin{columns}[c]
    \begin{column}{0.45\textwidth}
      \minipage[c][0.66\textheight][s]{\columnwidth}
      \begin{itemize}
        \item<1-> What if the backtrace for an exception is never needed?
        \item<2-> It's possible to raise the exception explicitly with \funcname{raise\_notrace} to make raising as cheap as possible
        \item<3-> An example of use in the \funcname{dynlink} library of the OCaml compiler
      \end{itemize}
      \vfill
      \onslide<3->{
      \listocaml[firstline=205,lastline=209,highlightlines=207]{../ocaml/otherlibs/dynlink/dynlink\_compilerlibs/misc.ml}{otherlibs/dynlink/dynlink\_compilerlibs/misc.ml:205}
      }
      \endminipage
    \end{column}
    \begin{column}{0.55\textwidth}
    \only<4>{
      \mintcmm[highlightlines=3]{../src/notrace/camlNotrace\_\_fun_347.cmm}
      \listcmm{../src/notrace/camlNotrace\_\_all\_somes\_327.cmm}{all\_somes}
    }
    \onslide*<5->{
      \listobjdump[highlightlines={6-9}]{../src/notrace/camlNotrace\_\_fun_347.objdump}{anonymous function from \funcname{all\_somes}}
    }
    \end{column}
  \end{columns}
\end{frame}

\begin{frame}{Backtraces}{Summary table of the multiple kinds of raise}
  \begin{itemize}
    \item When debugging information is enabled and if the backtrace of an exception isn't needed, it's possible to raise with \funcname{raise\_notrace} for performance
    \item When debugging information is \emph{not} enabled, the compiler optimises \funcname{raise} calls to inline assembly (same effect as using \funcname{raise\_notrace})
  \end{itemize}
  \bigskip
  \centering
  \begin{tabular}{| l | c | c | c | c |}
    \hline
    \parbox{10em}{Debug information \\ (\funcarg{-g} flag)}  & \multicolumn{2}{|c|}{Disabled}                                     & \multicolumn{2}{|c|}{Enabled} \\
    \hline
    Raise kind                                               & \funcname{raise} & \funcname{raise\_notrace}                       & \funcname{raise} & \funcname{raise\_notrace} \\
    \hline
    Compilation                                              & \parbox{8em}{inline assembly\\ (optimization)} & inline assembly   & \funcname{caml\_raise\_exn} & inline assembly \\
    \hline
  \end{tabular}
\end{frame}


%
% Takeaway #5
%
\subsection*{Takeaway \#5}
\frameSubsectionTakeaway{}
\againframe<5>{takeaway}
}%
    \end{column}
  \end{columns}
\end{frame}

\begin{frame}{Backtraces}{Printing the backtrace}
  \begin{columns}[c]
    \begin{column}{0.45\textwidth}
      \minipage[c][0.66\textheight][s]{\columnwidth}
      \begin{itemize}
        \item<1-> The exception backtrace can now be printed to \funcarg{stdout} with \funcname{Printexc.print_backtrace}
        \item<2-> First the exception backtrace is retrieved into an OCaml array with \funcname{get_raw_backtrace }
        \item<3-> Which is implemented as the \funcname{caml_get_exception_raw_backtrace} C function in the runtime
        \item<4-> And finally printed with \funcname{print_exception_backtrace}
      \end{itemize}
      \vfill
      \mintocaml[firstline=28,lastline=34,highlightlines=33]{../src/backtrace/backtrace.ml}
      \endminipage
    \end{column}
    \begin{column}{0.55\textwidth}
      \onslide*<1,2>{
        \mintocaml[firstline=100,lastline=101]{../ocaml/stdlib/printexc.ml}
      }
      \only<1>{
        \listocaml[firstline=170,lastline=172]{../ocaml/stdlib/printexc.ml}{stdlib/printexc.ml:170}
      }
      \only<2>{
        \listocaml[firstline=170,lastline=172,highlightlines=172]{../ocaml/stdlib/printexc.ml}{stdlib/printexc.ml:170}
      }
      \only<3>{
        \mintc[firstline=154,lastline=156]{../ocaml/runtime/backtrace.c}
        \listc[firstline=171,lastline=192,highlightlines={184-187}]{../ocaml/runtime/backtrace.c}{runtime/backtrace.c:154}
      }
      \only<4>{
        \listocaml[firstline=155,lastline=165]{../ocaml/stdlib/printexc.ml}{stdlib/printexc.ml:55}
      }
    \end{column}
  \end{columns}
\end{frame}


\subsection{The multiple kinds of raise}
\frameSubsection{}{}

\begin{frame}{Backtraces}{When backtraces are not needed}
  \begin{columns}[c]
    \begin{column}{0.45\textwidth}
      \minipage[c][0.66\textheight][s]{\columnwidth}
      \begin{itemize}
        \item<1-> What if the backtrace for an exception is never needed?
        \item<2-> It's possible to raise the exception explicitly with \funcname{raise\_notrace} to make raising as cheap as possible
        \item<3-> An example of use in the \funcname{dynlink} library of the OCaml compiler
      \end{itemize}
      \vfill
      \onslide<3->{
      \listocaml[firstline=205,lastline=209,highlightlines=207]{../ocaml/otherlibs/dynlink/dynlink\_compilerlibs/misc.ml}{otherlibs/dynlink/dynlink\_compilerlibs/misc.ml:205}
      }
      \endminipage
    \end{column}
    \begin{column}{0.55\textwidth}
    \only<4>{
      \mintcmm[highlightlines=3]{../src/notrace/camlNotrace\_\_fun_347.cmm}
      \listcmm{../src/notrace/camlNotrace\_\_all\_somes\_327.cmm}{all\_somes}
    }
    \onslide*<5->{
      \listobjdump[highlightlines={6-9}]{../src/notrace/camlNotrace\_\_fun_347.objdump}{anonymous function from \funcname{all\_somes}}
    }
    \end{column}
  \end{columns}
\end{frame}

\begin{frame}{Backtraces}{Summary table of the multiple kinds of raise}
  \begin{itemize}
    \item When debugging information is enabled and if the backtrace of an exception isn't needed, it's possible to raise with \funcname{raise\_notrace} for performance
    \item When debugging information is \emph{not} enabled, the compiler optimises \funcname{raise} calls to inline assembly (same effect as using \funcname{raise\_notrace})
  \end{itemize}
  \bigskip
  \centering
  \begin{tabular}{| l | c | c | c | c |}
    \hline
    \parbox{10em}{Debug information \\ (\funcarg{-g} flag)}  & \multicolumn{2}{|c|}{Disabled}                                     & \multicolumn{2}{|c|}{Enabled} \\
    \hline
    Raise kind                                               & \funcname{raise} & \funcname{raise\_notrace}                       & \funcname{raise} & \funcname{raise\_notrace} \\
    \hline
    Compilation                                              & \parbox{8em}{inline assembly\\ (optimization)} & inline assembly   & \funcname{caml\_raise\_exn} & inline assembly \\
    \hline
  \end{tabular}
\end{frame}


%
% Takeaway #5
%
\subsection*{Takeaway \#5}
\frameSubsectionTakeaway{}
\againframe<5>{takeaway}
}%
    \end{column}
  \end{columns}
\end{frame}

\newcommand\listStashBacktrace[1][]{
  \listc[firstline=91,lastline=120,#1]{../ocaml/runtime/backtrace\_nat.c}{runtime/backtrace\_nat.c:91}}

\begin{frame}{Backtraces}{Introducing \funcname{caml\_stash\_backtrace}}
  \begin{columns}[c]
    \begin{column}{0.5\textwidth}
      \minipage[c][0.66\textheight][s]{\columnwidth}
      \begin{itemize}
        \item<1-> A C function provided by the runtime (specific to native code)
        \item<2-> It scans the stack, between the stack pointer at the raising point up to the next trap, in order to collect the names of the detected function frames
          \onslide*<2>{
            \begin{itemize}
              \item And more information, like line number, column number...
            \end{itemize}
          }
        \item<3-> It uses the same technique and information as the Garbage Collector (GC) uses to scan the values live on the stack
          \onslide*<4>{
            \begin{itemize}
              \item \typename{frame\_descr} are at the core of stack unwinding
            \end{itemize}
          }
      \end{itemize}
      \vfill
      \mintocaml[firstline=9,lastline=13,highlightlines=12]{../src/backtrace/backtrace.ml}
      \endminipage
    \end{column}
    \begin{column}{0.5\textwidth}
      \onslide*<1>{\listStashBacktrace[highlightlines=91]{}}
      \onslide*<2>{\listStashBacktrace[highlightlines={91,117-118}]{}}
      \onslide*<3->{\listStashBacktrace[highlightlines={94,106,109-110}]{}}
    \end{column}
  \end{columns}
\end{frame}

\newcommand\listFrameDescr[1][]{
  \listocaml[firstline=111,lastline=124,#1]{../ocaml/asmcomp/emitaux.ml}{asmcomp/emitaux.ml:111}}
\newcommand\listDebugInfo[1][]{
  \listocaml[firstline=38,lastline=49,#1]{../ocaml/lambda/debuginfo.mli}{lambda/debuginfo.mli:38}}

\begin{frame}{Backtraces}{\typename{frame\_descr} and \typename{frame\_debuginfo} in a nutshell}
  \begin{columns}[c]
    \begin{column}{0.5\textwidth}
      \begin{itemize}
        \item<1-> For every return address in an OCaml program, there is a associated \typename{frame\_descr}
        \item<2-> A \typename{frame\_descr} specifies
        \begin{itemize}
          \item<2-> The size of the function stack frame
          \item<3-> Where live values are on the stack (so that the GC can mark them as live and prevent from being swept)
        \end{itemize}
        \item<4-> When compiled with debug information, \typename{frame\_descr} will be linked to a \typename{frame\_debuginfo}
        \begin{itemize}
          \item<5-> Contains information about the position in the source code (file name, line and column number)
        \end{itemize}
      \end{itemize}
    \end{column}
    \begin{column}{0.5\textwidth}
        \onslide*<1>{\listFrameDescr[highlightlines={118-122}]{}}
        \onslide*<2>{\listFrameDescr[highlightlines={120}]{}}
        \onslide*<3>{\listFrameDescr[highlightlines={121}]{}}
        \onslide*<4->{\listFrameDescr[highlightlines={122}]{}}
        \medskip
        \onslide*<1-3>{\listDebugInfo{}}
        \onslide*<4>{\listDebugInfo[highlightlines={38,49}]{}}
        \onslide*<5>{\listDebugInfo[highlightlines={39-46}]{}}
    \end{column}
  \end{columns}
\end{frame}

\begin{frame}{Backtraces}{Scanning the stack with \typename{frame\_descr}}
  \begin{columns}[c]
    \begin{column}{0.5\textwidth}
      \begin{itemize}
        \item Given the return address of the code that raises the exception by calling \funcname{caml\_raise\_exn} (\funcarg{pc} argument of \funcname{caml\_stash\_backtrace})
        \item And the position of the stack pointer (\funcarg{sp} argument of \funcname{caml\_stash\_backtrace})
        \item The frame size given by \typename{frame\_descr}.\funcarg{frame\_size}, allows to find the end of the previous frame
        \item And the return address of the caller of this function
        \bigskip
        \onslide<3->{
        \item Note that traps (here the reddish \funcname{ExnA}, \funcname{ExnB} and \funcname{ExnC} blocks) belong to the frame that installed them, and so the frame size changes when traps are removed
        }
      \end{itemize}
    \end{column}
    \begin{column}{0.5\textwidth}
      \raggedleft
      \onslide*<1>{\providecommand\step{2}%
% Backtraces
%
\subsection{One advantage of raising with debugging information}
\frameSubsection{}{}

\begin{frame}{Backtraces}{Debugging information \& source code}
  \begin{columns}[c]
    \begin{column}{0.5\textwidth}
      \begin{itemize}
        \item Raising an exception is fun and games, but how do we know where the exception originated? $\rightarrow$ With the backtrace associated with the exception!
        \item In order to get a backtrace from the point where an exception was raised, it's necessary to compile with debugging information enabled (\funcarg{-g} flag for the compiler)
        \item Same example as in the `Nested handlers' section
      \end{itemize}
    \end{column}
    \begin{column}{0.5\textwidth}
      \listocaml[firstline=9,lastline=34]{../src/backtrace/backtrace.ml}{backtrace/backtrace.ml}
    \end{column}
  \end{columns}
\end{frame}

\begin{frame}{Backtraces}{CMM and disassembly of \funcname{definitely\_raise}}
  \begin{columns}[c]
    \begin{column}{0.5\textwidth}
      \minipage[c][0.66\textheight][s]{\columnwidth}
      \begin{itemize}
        \item<1-> An effect of compiling with debugging information (\funcarg{-g}): the CMM no longer uses \funcname{raise\_notrace} to raise the exception but \funcname{raise} instead
        \item<2-> Disassembly shows that CMM \funcname{raise} is actually a call to \funcname{caml\_raise\_exn}, a function in assembler provided by the runtime
      \end{itemize}
      \vfill
      \mintocaml[firstline=9,lastline=13,highlightlines=12]{../src/backtrace/backtrace.ml}
      \endminipage
    \end{column}
    \begin{column}{0.5\textwidth}
      \onslide*<1>{
        \mintcmm[highlightlines=5]{../src/backtrace/camlBacktrace\_\_definitely\_raise\_347.cmm}
      }
      \onslide*<2>{
        \mintobjdump[firstline=2,highlightlines=14]{../src/backtrace/camlBacktrace\_\_definitely\_raise\_347.objdump}
      }
    \end{column}
  \end{columns}
\end{frame}

\begin{frame}{Backtraces}{Introducing \funcname{caml\_raise\_exn}}
  \begin{columns}[c]
    \begin{column}{0.5\textwidth}
      \minipage[c][0.66\textheight][s]{\columnwidth}
      \onslide*<1>{
        \begin{itemize}
          \item Raising the exception is performed using \funcname{caml\_raise\_exn} which is
            \begin{itemize}
              \item An assembly function provided by the runtime
              \item Architecture specific
            \end{itemize}
          \item Let's see what it does differently from \funcname{raise\_notrace}\dots
          \item We are about to raise \typename{ExnC} with \funcname{caml\_raise\_exn} from \funcname{definitely\_raise}
        \end{itemize}
      }
      \onslide*<2>{
        \mintobjdump[firstline=2,highlightlines=14]{../src/backtrace/camlBacktrace\_\_definitely\_raise\_347.objdump}
      }
      \vfill
      \mintocaml[firstline=9,lastline=13,highlightlines=12]{../src/backtrace/backtrace.ml}
      \endminipage
    \end{column}
    \begin{column}{0.5\textwidth}
      \centering
      \providecommand\step{1}%
% Backtraces
%
\subsection{One advantage of raising with debugging information}
\frameSubsection{}{}

\begin{frame}{Backtraces}{Debugging information \& source code}
  \begin{columns}[c]
    \begin{column}{0.5\textwidth}
      \begin{itemize}
        \item Raising an exception is fun and games, but how do we know where the exception originated? $\rightarrow$ With the backtrace associated with the exception!
        \item In order to get a backtrace from the point where an exception was raised, it's necessary to compile with debugging information enabled (\funcarg{-g} flag for the compiler)
        \item Same example as in the `Nested handlers' section
      \end{itemize}
    \end{column}
    \begin{column}{0.5\textwidth}
      \listocaml[firstline=9,lastline=34]{../src/backtrace/backtrace.ml}{backtrace/backtrace.ml}
    \end{column}
  \end{columns}
\end{frame}

\begin{frame}{Backtraces}{CMM and disassembly of \funcname{definitely\_raise}}
  \begin{columns}[c]
    \begin{column}{0.5\textwidth}
      \minipage[c][0.66\textheight][s]{\columnwidth}
      \begin{itemize}
        \item<1-> An effect of compiling with debugging information (\funcarg{-g}): the CMM no longer uses \funcname{raise\_notrace} to raise the exception but \funcname{raise} instead
        \item<2-> Disassembly shows that CMM \funcname{raise} is actually a call to \funcname{caml\_raise\_exn}, a function in assembler provided by the runtime
      \end{itemize}
      \vfill
      \mintocaml[firstline=9,lastline=13,highlightlines=12]{../src/backtrace/backtrace.ml}
      \endminipage
    \end{column}
    \begin{column}{0.5\textwidth}
      \onslide*<1>{
        \mintcmm[highlightlines=5]{../src/backtrace/camlBacktrace\_\_definitely\_raise\_347.cmm}
      }
      \onslide*<2>{
        \mintobjdump[firstline=2,highlightlines=14]{../src/backtrace/camlBacktrace\_\_definitely\_raise\_347.objdump}
      }
    \end{column}
  \end{columns}
\end{frame}

\begin{frame}{Backtraces}{Introducing \funcname{caml\_raise\_exn}}
  \begin{columns}[c]
    \begin{column}{0.5\textwidth}
      \minipage[c][0.66\textheight][s]{\columnwidth}
      \onslide*<1>{
        \begin{itemize}
          \item Raising the exception is performed using \funcname{caml\_raise\_exn} which is
            \begin{itemize}
              \item An assembly function provided by the runtime
              \item Architecture specific
            \end{itemize}
          \item Let's see what it does differently from \funcname{raise\_notrace}\dots
          \item We are about to raise \typename{ExnC} with \funcname{caml\_raise\_exn} from \funcname{definitely\_raise}
        \end{itemize}
      }
      \onslide*<2>{
        \mintobjdump[firstline=2,highlightlines=14]{../src/backtrace/camlBacktrace\_\_definitely\_raise\_347.objdump}
      }
      \vfill
      \mintocaml[firstline=9,lastline=13,highlightlines=12]{../src/backtrace/backtrace.ml}
      \endminipage
    \end{column}
    \begin{column}{0.5\textwidth}
      \centering
      \providecommand\step{1}\input{diagrams/backtrace/backtrace.tex}
    \end{column}
  \end{columns}
\end{frame}

\begin{frame}{Backtraces}{But before that, we need more fields from \typename{caml\_domain\_state}}
  \begin{columns}
    \begin{column}{0.5\textwidth}
      \begin{itemize}
        \item Remember \typename{caml\_domain\_state} the per-domain runtime state?
        \item Some other members are of interest to us now\dots
        \item \localname{backtrace\_active} is used to know if the runtime should collect backtraces
        \item \localname{backtrace\_active} can be set by
          \begin{itemize}
            \item The \funcarg{b} flag in the \funcname{OCAMLRUNPARAM} environment variable
            \item \mintinline{ocaml}{Printexc.record_backtrace : bool -> unit} \footnotemark
          \end{itemize}
      \end{itemize}
    \end{column}
    \begin{column}{0.5\textwidth}
      \begin{listing}
        \mintc[highlightlines={9-11}]{src/ocaml/domain\_state\_backtrace.h}
        \caption{runtime/caml/domain\_state.h}
      \end{listing}
    \end{column}
  \end{columns}
  \footnotetext[1]{https://v2.ocaml.org/api/Printexc.html}
\end{frame}

\newcommand\listCamlRaiseExn[1][]{
  \listasm[firstline=702,lastline=725,#1]{../ocaml/runtime/amd64.S}{runtime/amd64.S:702}}

\begin{frame}{Backtraces}{Walk through \funcname{caml\_raise\_exn}}
  \begin{columns}[T]
    \begin{column}{0.5\textwidth}
      \begin{itemize}
        \item<1-> First checks whether exceptions backtrace should be recorded
        \item<1-> By checking the \funcarg{domain\_state->backtrace\_active} value
        \item<2-> If not, acts as \funcname{raise\_notrace} with \funcname{RESTORE\_EXN\_HANDLER\_OCAML}
          \begin{itemize}
            \item Set \regname{RSP} to \localname{domain\_state->exn\_handler} value
            \item Restore \localname{domain\_state->exn\_handler} from trap
            \item Jump with \funcname{ret} (same effect as using \funcname{pop} and \funcname{jmp})
          \end{itemize}
        \item<3-> Otherwise, switches to the C stack to be able to execute C code
        \item<4-> Calls \funcname{caml\_stash\_backtrace}, a C function provided by the runtime\ignorespaces
          \onslide<6->{, with
            \begin{itemize}
              \item \funcarg{exn}: exception value
              \item \funcarg{pc}: code address of the raising point
              \item \funcarg{sp}: stack address of the raising function
              \item \funcarg{trapsp}: stack address of the next handler
            \end{itemize}
          }
      \end{itemize}
      \bigskip
      \mintocaml[firstline=9,lastline=13,highlightlines=12]{../src/backtrace/backtrace.ml}
    \end{column}
    \begin{column}{0.5\textwidth}
      \raggedleft
      \onslide*<1,3-4>{
        \mintc[firstline=190,lastline=192]{../ocaml/runtime/amd64.S}
        \mintc[firstline=234,lastline=237]{../ocaml/runtime/amd64.S}
      }
      \onslide*<2>{
        \mintc[firstline=190,lastline=192,highlightlines={191-192}]{../ocaml/runtime/amd64.S}
        \mintc[firstline=234,lastline=237,highlightlines={235-237}]{../ocaml/runtime/amd64.S}
      }
      \onslide*<1>{\listCamlRaiseExn[highlightlines=705]{}}%
      \onslide*<2>{\listCamlRaiseExn[highlightlines={707-708}]{}}%
      \onslide*<3>{\listCamlRaiseExn[highlightlines={712-714}]{}}%
      \onslide*<4>{\listCamlRaiseExn[highlightlines={715-720}]{}}%
      \onslide*<5>{\providecommand\step{1}\input{diagrams/backtrace/backtrace.tex}}%
      \onslide*<6>{\providecommand\step{2}\input{diagrams/backtrace/backtrace.tex}}%
    \end{column}
  \end{columns}
\end{frame}

\newcommand\listStashBacktrace[1][]{
  \listc[firstline=91,lastline=120,#1]{../ocaml/runtime/backtrace\_nat.c}{runtime/backtrace\_nat.c:91}}

\begin{frame}{Backtraces}{Introducing \funcname{caml\_stash\_backtrace}}
  \begin{columns}[c]
    \begin{column}{0.5\textwidth}
      \minipage[c][0.66\textheight][s]{\columnwidth}
      \begin{itemize}
        \item<1-> A C function provided by the runtime (specific to native code)
        \item<2-> It scans the stack, between the stack pointer at the raising point up to the next trap, in order to collect the names of the detected function frames
          \onslide*<2>{
            \begin{itemize}
              \item And more information, like line number, column number...
            \end{itemize}
          }
        \item<3-> It uses the same technique and information as the Garbage Collector (GC) uses to scan the values live on the stack
          \onslide*<4>{
            \begin{itemize}
              \item \typename{frame\_descr} are at the core of stack unwinding
            \end{itemize}
          }
      \end{itemize}
      \vfill
      \mintocaml[firstline=9,lastline=13,highlightlines=12]{../src/backtrace/backtrace.ml}
      \endminipage
    \end{column}
    \begin{column}{0.5\textwidth}
      \onslide*<1>{\listStashBacktrace[highlightlines=91]{}}
      \onslide*<2>{\listStashBacktrace[highlightlines={91,117-118}]{}}
      \onslide*<3->{\listStashBacktrace[highlightlines={94,106,109-110}]{}}
    \end{column}
  \end{columns}
\end{frame}

\newcommand\listFrameDescr[1][]{
  \listocaml[firstline=111,lastline=124,#1]{../ocaml/asmcomp/emitaux.ml}{asmcomp/emitaux.ml:111}}
\newcommand\listDebugInfo[1][]{
  \listocaml[firstline=38,lastline=49,#1]{../ocaml/lambda/debuginfo.mli}{lambda/debuginfo.mli:38}}

\begin{frame}{Backtraces}{\typename{frame\_descr} and \typename{frame\_debuginfo} in a nutshell}
  \begin{columns}[c]
    \begin{column}{0.5\textwidth}
      \begin{itemize}
        \item<1-> For every return address in an OCaml program, there is a associated \typename{frame\_descr}
        \item<2-> A \typename{frame\_descr} specifies
        \begin{itemize}
          \item<2-> The size of the function stack frame
          \item<3-> Where live values are on the stack (so that the GC can mark them as live and prevent from being swept)
        \end{itemize}
        \item<4-> When compiled with debug information, \typename{frame\_descr} will be linked to a \typename{frame\_debuginfo}
        \begin{itemize}
          \item<5-> Contains information about the position in the source code (file name, line and column number)
        \end{itemize}
      \end{itemize}
    \end{column}
    \begin{column}{0.5\textwidth}
        \onslide*<1>{\listFrameDescr[highlightlines={118-122}]{}}
        \onslide*<2>{\listFrameDescr[highlightlines={120}]{}}
        \onslide*<3>{\listFrameDescr[highlightlines={121}]{}}
        \onslide*<4->{\listFrameDescr[highlightlines={122}]{}}
        \medskip
        \onslide*<1-3>{\listDebugInfo{}}
        \onslide*<4>{\listDebugInfo[highlightlines={38,49}]{}}
        \onslide*<5>{\listDebugInfo[highlightlines={39-46}]{}}
    \end{column}
  \end{columns}
\end{frame}

\begin{frame}{Backtraces}{Scanning the stack with \typename{frame\_descr}}
  \begin{columns}[c]
    \begin{column}{0.5\textwidth}
      \begin{itemize}
        \item Given the return address of the code that raises the exception by calling \funcname{caml\_raise\_exn} (\funcarg{pc} argument of \funcname{caml\_stash\_backtrace})
        \item And the position of the stack pointer (\funcarg{sp} argument of \funcname{caml\_stash\_backtrace})
        \item The frame size given by \typename{frame\_descr}.\funcarg{frame\_size}, allows to find the end of the previous frame
        \item And the return address of the caller of this function
        \bigskip
        \onslide<3->{
        \item Note that traps (here the reddish \funcname{ExnA}, \funcname{ExnB} and \funcname{ExnC} blocks) belong to the frame that installed them, and so the frame size changes when traps are removed
        }
      \end{itemize}
    \end{column}
    \begin{column}{0.5\textwidth}
      \raggedleft
      \onslide*<1>{\providecommand\step{2}\input{diagrams/backtrace/backtrace.tex}}%
      \onslide*<2>{\providecommand\step{1}\input{diagrams/backtrace/scanstack.tex}}%
      \onslide*<3>{\providecommand\step{2}\input{diagrams/backtrace/scanstack.tex}}%
      \onslide*<4>{\providecommand\step{3}\input{diagrams/backtrace/scanstack.tex}}%
      \onslide*<5>{\providecommand\step{4}\input{diagrams/backtrace/scanstack.tex}}%
      \onslide*<6>{\providecommand\step{5}\input{diagrams/backtrace/scanstack.tex}}%
    \end{column}
  \end{columns}
\end{frame}

% Duplicate of previous-previous slide
\begin{frame}{Backtraces}{Continuing on \funcname{caml\_stash\_backtrace}}
  \begin{columns}[c]
    \begin{column}{0.5\textwidth}
      \minipage[c][0.75\textheight][s]{\columnwidth}
      \begin{itemize}
          \color{gray}
        \item A C function provided by the runtime (specific to native code)
        \item It scans the stack, between the stack pointer at the raising point up to the next trap, in order to collect the names of the detected function frames
        \item It uses the same technique and information as the Garbage Collector (GC) uses to scan the values live on the stack
      \end{itemize}
      \begin{itemize}
        \item For each function frame detected, store a pointer to the \typename{frame\_descr} in the \localname{domain\_state->backtrace\_buffer} array, effectively constructing the backtrace
      \end{itemize}
      \onslide<2->{
        \begin{itemize}
            \smallskip
          \item Constructed backtrace:
            \funcname{definitely\_raise},
            \onslide<3->{\funcname{probably\_raise}}
        \end{itemize}
      }
      \vfill
      \mintocaml[firstline=9,lastline=13,highlightlines=12]{../src/backtrace/backtrace.ml}
      \endminipage
    \end{column}
    \begin{column}{0.5\textwidth}
      \raggedleft
      \onslide*<1>{\listStashBacktrace[highlightlines={114-115}]{}}
      \onslide*<2>{\providecommand\step{3}\input{diagrams/backtrace/backtrace.tex}}%
      \onslide*<3>{\providecommand\step{4}\input{diagrams/backtrace/backtrace.tex}}%
    \end{column}
  \end{columns}
\end{frame}

\begin{frame}{Backtraces}{Back to \funcname{caml\_raise\_exn}}
  \begin{columns}[c]
    \begin{column}{0.5\textwidth}
      \minipage[c][0.66\textheight][s]{\columnwidth}
      \begin{itemize}\color{gray}
        \item Checks wheteher exceptions backtrace should be recorded, by checking the \funcarg{domain\_state->backtrace\_active} value
        \item Switches to the C stack to be able to execute C code
        \item Calls \funcname{caml\_stash\_backtrace}, to collect the backtrace up to the next trap
      \end{itemize}
      \onslide*<2->{
        \begin{itemize}
          \item<2-> Executes the exception handler in \funcname{probably\_raise} with \funcname{RESTORE\_EXN\_HANDLER\_OCAML}
            \begin{itemize}
              \item Set \regname{RSP} to \localname{domain\_state->exn\_handler} value
              \item Restore \localname{domain\_state->exn\_handler} from trap
              \item Jump to the handler code with \funcname{ret}
            \end{itemize}
        \end{itemize}
      }
      \vfill
      \onslide*<1>{\mintocaml[firstline=9,lastline=13,highlightlines=12]{../src/backtrace/backtrace.ml}}
      \onslide*<2->{\mintocaml[firstline=15,lastline=21,highlightlines={19-21}]{../src/backtrace/backtrace.ml}}
      \endminipage
    \end{column}
    \begin{column}{0.5\textwidth}
      \centering
      \onslide*<1>{
        \mintc[firstline=190,lastline=192]{../ocaml/runtime/amd64.S}
        \mintc[firstline=234,lastline=237]{../ocaml/runtime/amd64.S}
      }
      \onslide*<2>{
        \mintc[firstline=190,lastline=192,highlightlines={191-192}]{../ocaml/runtime/amd64.S}
        \mintc[firstline=234,lastline=237,highlightlines={235-237}]{../ocaml/runtime/amd64.S}
      }
      \onslide*<1>{\listCamlRaiseExn[highlightlines=720]{}}
      \onslide*<2>{\listCamlRaiseExn[highlightlines=722]{}}
      \onslide*<3->{\providecommand\step{5}\input{diagrams/backtrace/backtrace.tex}}
    \end{column}
  \end{columns}
\end{frame}

\begin{frame}{Backtraces}{\funcname{caml\_probably\_raise}'s handler doesn't catch \typename{ExnC}}
  \begin{columns}[c]
    \begin{column}{0.45\textwidth}
      \minipage[c][0.75\textheight][s]{\columnwidth}
      \begin{itemize}
        \item<1-> \funcname{match \dots with}\footnotemark pattern matching can also match exceptions
        \item<1-> The CMM of \funcname{probably\_raise} shows that this \funcname{match \dots with} is implemented with a pattern matching and a \funcname{try \dots with}
        \item<1-> But this one only matches on \typename{ExnA} exception
        \item<2-> Since the pattern matching fails to catch the exception, \funcname{reraise} is called with our current \typename{ExnC} exception
        \item<3-> Disassembly shows that \funcname{reraise} is actually a call to \funcname{caml\_reraise\_exn}, which is also an assembly function provided by the runtime
      \end{itemize}
      \vfill
      \mintocaml[firstline=15,lastline=21,highlightlines={18-21}]{../src/backtrace/backtrace.ml}
      \endminipage
    \end{column}
    \begin{column}{0.55\textwidth}
      \centering
      \onslide*<1>{\mintcmm[highlightlines={10-18}]{../src/backtrace/camlBacktrace\_\_probably\_raise\_351.cmm}}
      \onslide*<2>{\listcmm[highlightlines=18]{../src/backtrace/camlBacktrace\_\_probably\_raise_351.cmm}{CMM of probably\_raise}}
      \onslide*<3>{\mintobjdump[firstline=5,lastline=35,highlightlines=31]{../src/backtrace/camlBacktrace\_\_probably\_raise_351.objdump}}
    \end{column}
  \end{columns}
  \only<1>{\footnotetext[1]{https://blog.janestreet.com/pattern-matching-and-exception-handling-unite/}}
\end{frame}

\newcommand\listCamlReraiseExn[1][]{
  \listasm[firstline=727,lastline=734,#1]{../ocaml/runtime/amd64.S}{runtime/amd64.S:727}}

\begin{frame}{Backtraces}{Introducing \funcname{caml\_reraise\_exn}}
  \begin{columns}[c]
    \begin{column}{0.5\textwidth}
      \minipage[c][0.66\textheight][s]{\columnwidth}
      \begin{itemize}
        \item<1-> The exception have to be reraised to the next handler
        \item<2-> First, checks if the backtrace should be collected with \funcarg{domain\_state->backtrace\_active}
        \only<3>{
        \begin{itemize}
            \item If not, behaves like a \funcname{raise\_notrace} with \funcname{RESTORE\_EXN\_HANDLER\_OCAML}
        \end{itemize}
        }
        \item<4-> Jumps into \funcname{caml\_raise\_exn}
        \item<5-> Calls \funcname{caml\_stash\_backtrace} again, to scan the next part of the stack: from the trap just pop-ed to the next trap
      \end{itemize}
      \vfill
      \mintocaml[firstline=15,lastline=21,highlightlines={18-21}]{../src/backtrace/backtrace.ml}
      \endminipage
    \end{column}
    \begin{column}{0.5\textwidth}
      \raggedleft
      \onslide*<1>{\listCamlReraiseExn{}}
      \onslide*<2>{\listCamlReraiseExn[highlightlines=729]{}}
      \onslide*<3>{
        \mintc[firstline=190,lastline=192,highlightlines={191-192}]{../ocaml/runtime/amd64.S}
        \mintc[firstline=234,lastline=237,highlightlines={235-237}]{../ocaml/runtime/amd64.S}
        \medskip
        \listCamlReraiseExn[highlightlines=731]{}
      }
      \onslide*<4-5>{
        % TODO refactor with \listCamlReraiseExn
        \mintasm[firstline=727,lastline=734,highlightlines=730]{../ocaml/runtime/amd64.S}
        \medskip
      }
      \onslide*<4>{\listCamlRaiseExn[highlightlines=711]{}}
      \onslide*<5>{\listCamlRaiseExn[highlightlines=720]{}}
    \end{column}
  \end{columns}
\end{frame}

\begin{frame}{Backtraces}{Backtrace collection continues}
  \begin{columns}[c]
    \begin{column}{0.55\textwidth}
      \minipage[c][0.75\textheight][s]{\columnwidth}
      \begin{itemize}
        \item<1-> \funcname{caml\_stash\_backtrace} scans the older part of the stack, up to the next trap
        \item<6-> Controls jumps to the nested exception handler in \funcname{maybe\_raise}, which matches only on \typename{ExnB}
        \item<7-> \funcname{caml\_reraise\_exn} is called again, backtrace collection resumes with \funcname{caml\_stash\_backtrace}
      \end{itemize}
      \medskip
      \begin{itemize}
        \item<2-> Constructed backtrace:
        \begin{itemize}
          \item <2-> \funcname{definitely\_raise}
          \item <2-> \funcname{probably\_raise}
          \item <3-> \funcname{probably\_raise} (re-raise)
          \item <4-> \funcname{maybe\_raise}
          \item <5-> \funcname{main}
          \item <8-> \funcname{main} (re-raise)
        \end{itemize}
      \end{itemize}
      \vfill
      \onslide*<1-5>{\mintocaml[firstline=15,lastline=21]{../src/backtrace/backtrace.ml}}
      \onslide*<6>{\mintocaml[firstline=28,lastline=34,highlightlines=31]{../src/backtrace/backtrace.ml}}
      \onslide*<7->{\mintocaml[firstline=28,lastline=34,highlightlines=33]{../src/backtrace/backtrace.ml}}
      \endminipage
    \end{column}
    \begin{column}{0.45\textwidth}
      \raggedleft
      \onslide*<1>{\providecommand\step{6}\input{diagrams/backtrace/backtrace.tex}}%
      \onslide*<2>{\providecommand\step{6}\input{diagrams/backtrace/backtrace.tex}}%
      \onslide*<3>{\providecommand\step{7}\input{diagrams/backtrace/backtrace.tex}}%
      \onslide*<4>{\providecommand\step{8}\input{diagrams/backtrace/backtrace.tex}}%
      \onslide*<5>{\providecommand\step{9}\input{diagrams/backtrace/backtrace.tex}}%
      \onslide*<6>{\providecommand\step{10}\input{diagrams/backtrace/backtrace.tex}}%
      \onslide*<7>{\providecommand\step{11}\input{diagrams/backtrace/backtrace.tex}}%
      \onslide*<8>{\providecommand\step{12}\input{diagrams/backtrace/backtrace.tex}}%
      \onslide*<9>{\providecommand\step{13}\input{diagrams/backtrace/backtrace.tex}}%
    \end{column}
  \end{columns}
\end{frame}

\begin{frame}{Backtraces}{Printing the backtrace}
  \begin{columns}[c]
    \begin{column}{0.45\textwidth}
      \minipage[c][0.66\textheight][s]{\columnwidth}
      \begin{itemize}
        \item<1-> The exception backtrace can now be printed to \funcarg{stdout} with \funcname{Printexc.print_backtrace}
        \item<2-> First the exception backtrace is retrieved into an OCaml array with \funcname{get_raw_backtrace }
        \item<3-> Which is implemented as the \funcname{caml_get_exception_raw_backtrace} C function in the runtime
        \item<4-> And finally printed with \funcname{print_exception_backtrace}
      \end{itemize}
      \vfill
      \mintocaml[firstline=28,lastline=34,highlightlines=33]{../src/backtrace/backtrace.ml}
      \endminipage
    \end{column}
    \begin{column}{0.55\textwidth}
      \onslide*<1,2>{
        \mintocaml[firstline=100,lastline=101]{../ocaml/stdlib/printexc.ml}
      }
      \only<1>{
        \listocaml[firstline=170,lastline=172]{../ocaml/stdlib/printexc.ml}{stdlib/printexc.ml:170}
      }
      \only<2>{
        \listocaml[firstline=170,lastline=172,highlightlines=172]{../ocaml/stdlib/printexc.ml}{stdlib/printexc.ml:170}
      }
      \only<3>{
        \mintc[firstline=154,lastline=156]{../ocaml/runtime/backtrace.c}
        \listc[firstline=171,lastline=192,highlightlines={184-187}]{../ocaml/runtime/backtrace.c}{runtime/backtrace.c:154}
      }
      \only<4>{
        \listocaml[firstline=155,lastline=165]{../ocaml/stdlib/printexc.ml}{stdlib/printexc.ml:55}
      }
    \end{column}
  \end{columns}
\end{frame}


\subsection{The multiple kinds of raise}
\frameSubsection{}{}

\begin{frame}{Backtraces}{When backtraces are not needed}
  \begin{columns}[c]
    \begin{column}{0.45\textwidth}
      \minipage[c][0.66\textheight][s]{\columnwidth}
      \begin{itemize}
        \item<1-> What if the backtrace for an exception is never needed?
        \item<2-> It's possible to raise the exception explicitly with \funcname{raise\_notrace} to make raising as cheap as possible
        \item<3-> An example of use in the \funcname{dynlink} library of the OCaml compiler
      \end{itemize}
      \vfill
      \onslide<3->{
      \listocaml[firstline=205,lastline=209,highlightlines=207]{../ocaml/otherlibs/dynlink/dynlink\_compilerlibs/misc.ml}{otherlibs/dynlink/dynlink\_compilerlibs/misc.ml:205}
      }
      \endminipage
    \end{column}
    \begin{column}{0.55\textwidth}
    \only<4>{
      \mintcmm[highlightlines=3]{../src/notrace/camlNotrace\_\_fun_347.cmm}
      \listcmm{../src/notrace/camlNotrace\_\_all\_somes\_327.cmm}{all\_somes}
    }
    \onslide*<5->{
      \listobjdump[highlightlines={6-9}]{../src/notrace/camlNotrace\_\_fun_347.objdump}{anonymous function from \funcname{all\_somes}}
    }
    \end{column}
  \end{columns}
\end{frame}

\begin{frame}{Backtraces}{Summary table of the multiple kinds of raise}
  \begin{itemize}
    \item When debugging information is enabled and if the backtrace of an exception isn't needed, it's possible to raise with \funcname{raise\_notrace} for performance
    \item When debugging information is \emph{not} enabled, the compiler optimises \funcname{raise} calls to inline assembly (same effect as using \funcname{raise\_notrace})
  \end{itemize}
  \bigskip
  \centering
  \begin{tabular}{| l | c | c | c | c |}
    \hline
    \parbox{10em}{Debug information \\ (\funcarg{-g} flag)}  & \multicolumn{2}{|c|}{Disabled}                                     & \multicolumn{2}{|c|}{Enabled} \\
    \hline
    Raise kind                                               & \funcname{raise} & \funcname{raise\_notrace}                       & \funcname{raise} & \funcname{raise\_notrace} \\
    \hline
    Compilation                                              & \parbox{8em}{inline assembly\\ (optimization)} & inline assembly   & \funcname{caml\_raise\_exn} & inline assembly \\
    \hline
  \end{tabular}
\end{frame}


%
% Takeaway #5
%
\subsection*{Takeaway \#5}
\frameSubsectionTakeaway{}
\againframe<5>{takeaway}

    \end{column}
  \end{columns}
\end{frame}

\begin{frame}{Backtraces}{But before that, we need more fields from \typename{caml\_domain\_state}}
  \begin{columns}
    \begin{column}{0.5\textwidth}
      \begin{itemize}
        \item Remember \typename{caml\_domain\_state} the per-domain runtime state?
        \item Some other members are of interest to us now\dots
        \item \localname{backtrace\_active} is used to know if the runtime should collect backtraces
        \item \localname{backtrace\_active} can be set by
          \begin{itemize}
            \item The \funcarg{b} flag in the \funcname{OCAMLRUNPARAM} environment variable
            \item \mintinline{ocaml}{Printexc.record_backtrace : bool -> unit} \footnotemark
          \end{itemize}
      \end{itemize}
    \end{column}
    \begin{column}{0.5\textwidth}
      \begin{listing}
        \mintc[highlightlines={9-11}]{src/ocaml/domain\_state\_backtrace.h}
        \caption{runtime/caml/domain\_state.h}
      \end{listing}
    \end{column}
  \end{columns}
  \footnotetext[1]{https://v2.ocaml.org/api/Printexc.html}
\end{frame}

\newcommand\listCamlRaiseExn[1][]{
  \listasm[firstline=702,lastline=725,#1]{../ocaml/runtime/amd64.S}{runtime/amd64.S:702}}

\begin{frame}{Backtraces}{Walk through \funcname{caml\_raise\_exn}}
  \begin{columns}[T]
    \begin{column}{0.5\textwidth}
      \begin{itemize}
        \item<1-> First checks whether exceptions backtrace should be recorded
        \item<1-> By checking the \funcarg{domain\_state->backtrace\_active} value
        \item<2-> If not, acts as \funcname{raise\_notrace} with \funcname{RESTORE\_EXN\_HANDLER\_OCAML}
          \begin{itemize}
            \item Set \regname{RSP} to \localname{domain\_state->exn\_handler} value
            \item Restore \localname{domain\_state->exn\_handler} from trap
            \item Jump with \funcname{ret} (same effect as using \funcname{pop} and \funcname{jmp})
          \end{itemize}
        \item<3-> Otherwise, switches to the C stack to be able to execute C code
        \item<4-> Calls \funcname{caml\_stash\_backtrace}, a C function provided by the runtime\ignorespaces
          \onslide<6->{, with
            \begin{itemize}
              \item \funcarg{exn}: exception value
              \item \funcarg{pc}: code address of the raising point
              \item \funcarg{sp}: stack address of the raising function
              \item \funcarg{trapsp}: stack address of the next handler
            \end{itemize}
          }
      \end{itemize}
      \bigskip
      \mintocaml[firstline=9,lastline=13,highlightlines=12]{../src/backtrace/backtrace.ml}
    \end{column}
    \begin{column}{0.5\textwidth}
      \raggedleft
      \onslide*<1,3-4>{
        \mintc[firstline=190,lastline=192]{../ocaml/runtime/amd64.S}
        \mintc[firstline=234,lastline=237]{../ocaml/runtime/amd64.S}
      }
      \onslide*<2>{
        \mintc[firstline=190,lastline=192,highlightlines={191-192}]{../ocaml/runtime/amd64.S}
        \mintc[firstline=234,lastline=237,highlightlines={235-237}]{../ocaml/runtime/amd64.S}
      }
      \onslide*<1>{\listCamlRaiseExn[highlightlines=705]{}}%
      \onslide*<2>{\listCamlRaiseExn[highlightlines={707-708}]{}}%
      \onslide*<3>{\listCamlRaiseExn[highlightlines={712-714}]{}}%
      \onslide*<4>{\listCamlRaiseExn[highlightlines={715-720}]{}}%
      \onslide*<5>{\providecommand\step{1}%
% Backtraces
%
\subsection{One advantage of raising with debugging information}
\frameSubsection{}{}

\begin{frame}{Backtraces}{Debugging information \& source code}
  \begin{columns}[c]
    \begin{column}{0.5\textwidth}
      \begin{itemize}
        \item Raising an exception is fun and games, but how do we know where the exception originated? $\rightarrow$ With the backtrace associated with the exception!
        \item In order to get a backtrace from the point where an exception was raised, it's necessary to compile with debugging information enabled (\funcarg{-g} flag for the compiler)
        \item Same example as in the `Nested handlers' section
      \end{itemize}
    \end{column}
    \begin{column}{0.5\textwidth}
      \listocaml[firstline=9,lastline=34]{../src/backtrace/backtrace.ml}{backtrace/backtrace.ml}
    \end{column}
  \end{columns}
\end{frame}

\begin{frame}{Backtraces}{CMM and disassembly of \funcname{definitely\_raise}}
  \begin{columns}[c]
    \begin{column}{0.5\textwidth}
      \minipage[c][0.66\textheight][s]{\columnwidth}
      \begin{itemize}
        \item<1-> An effect of compiling with debugging information (\funcarg{-g}): the CMM no longer uses \funcname{raise\_notrace} to raise the exception but \funcname{raise} instead
        \item<2-> Disassembly shows that CMM \funcname{raise} is actually a call to \funcname{caml\_raise\_exn}, a function in assembler provided by the runtime
      \end{itemize}
      \vfill
      \mintocaml[firstline=9,lastline=13,highlightlines=12]{../src/backtrace/backtrace.ml}
      \endminipage
    \end{column}
    \begin{column}{0.5\textwidth}
      \onslide*<1>{
        \mintcmm[highlightlines=5]{../src/backtrace/camlBacktrace\_\_definitely\_raise\_347.cmm}
      }
      \onslide*<2>{
        \mintobjdump[firstline=2,highlightlines=14]{../src/backtrace/camlBacktrace\_\_definitely\_raise\_347.objdump}
      }
    \end{column}
  \end{columns}
\end{frame}

\begin{frame}{Backtraces}{Introducing \funcname{caml\_raise\_exn}}
  \begin{columns}[c]
    \begin{column}{0.5\textwidth}
      \minipage[c][0.66\textheight][s]{\columnwidth}
      \onslide*<1>{
        \begin{itemize}
          \item Raising the exception is performed using \funcname{caml\_raise\_exn} which is
            \begin{itemize}
              \item An assembly function provided by the runtime
              \item Architecture specific
            \end{itemize}
          \item Let's see what it does differently from \funcname{raise\_notrace}\dots
          \item We are about to raise \typename{ExnC} with \funcname{caml\_raise\_exn} from \funcname{definitely\_raise}
        \end{itemize}
      }
      \onslide*<2>{
        \mintobjdump[firstline=2,highlightlines=14]{../src/backtrace/camlBacktrace\_\_definitely\_raise\_347.objdump}
      }
      \vfill
      \mintocaml[firstline=9,lastline=13,highlightlines=12]{../src/backtrace/backtrace.ml}
      \endminipage
    \end{column}
    \begin{column}{0.5\textwidth}
      \centering
      \providecommand\step{1}\input{diagrams/backtrace/backtrace.tex}
    \end{column}
  \end{columns}
\end{frame}

\begin{frame}{Backtraces}{But before that, we need more fields from \typename{caml\_domain\_state}}
  \begin{columns}
    \begin{column}{0.5\textwidth}
      \begin{itemize}
        \item Remember \typename{caml\_domain\_state} the per-domain runtime state?
        \item Some other members are of interest to us now\dots
        \item \localname{backtrace\_active} is used to know if the runtime should collect backtraces
        \item \localname{backtrace\_active} can be set by
          \begin{itemize}
            \item The \funcarg{b} flag in the \funcname{OCAMLRUNPARAM} environment variable
            \item \mintinline{ocaml}{Printexc.record_backtrace : bool -> unit} \footnotemark
          \end{itemize}
      \end{itemize}
    \end{column}
    \begin{column}{0.5\textwidth}
      \begin{listing}
        \mintc[highlightlines={9-11}]{src/ocaml/domain\_state\_backtrace.h}
        \caption{runtime/caml/domain\_state.h}
      \end{listing}
    \end{column}
  \end{columns}
  \footnotetext[1]{https://v2.ocaml.org/api/Printexc.html}
\end{frame}

\newcommand\listCamlRaiseExn[1][]{
  \listasm[firstline=702,lastline=725,#1]{../ocaml/runtime/amd64.S}{runtime/amd64.S:702}}

\begin{frame}{Backtraces}{Walk through \funcname{caml\_raise\_exn}}
  \begin{columns}[T]
    \begin{column}{0.5\textwidth}
      \begin{itemize}
        \item<1-> First checks whether exceptions backtrace should be recorded
        \item<1-> By checking the \funcarg{domain\_state->backtrace\_active} value
        \item<2-> If not, acts as \funcname{raise\_notrace} with \funcname{RESTORE\_EXN\_HANDLER\_OCAML}
          \begin{itemize}
            \item Set \regname{RSP} to \localname{domain\_state->exn\_handler} value
            \item Restore \localname{domain\_state->exn\_handler} from trap
            \item Jump with \funcname{ret} (same effect as using \funcname{pop} and \funcname{jmp})
          \end{itemize}
        \item<3-> Otherwise, switches to the C stack to be able to execute C code
        \item<4-> Calls \funcname{caml\_stash\_backtrace}, a C function provided by the runtime\ignorespaces
          \onslide<6->{, with
            \begin{itemize}
              \item \funcarg{exn}: exception value
              \item \funcarg{pc}: code address of the raising point
              \item \funcarg{sp}: stack address of the raising function
              \item \funcarg{trapsp}: stack address of the next handler
            \end{itemize}
          }
      \end{itemize}
      \bigskip
      \mintocaml[firstline=9,lastline=13,highlightlines=12]{../src/backtrace/backtrace.ml}
    \end{column}
    \begin{column}{0.5\textwidth}
      \raggedleft
      \onslide*<1,3-4>{
        \mintc[firstline=190,lastline=192]{../ocaml/runtime/amd64.S}
        \mintc[firstline=234,lastline=237]{../ocaml/runtime/amd64.S}
      }
      \onslide*<2>{
        \mintc[firstline=190,lastline=192,highlightlines={191-192}]{../ocaml/runtime/amd64.S}
        \mintc[firstline=234,lastline=237,highlightlines={235-237}]{../ocaml/runtime/amd64.S}
      }
      \onslide*<1>{\listCamlRaiseExn[highlightlines=705]{}}%
      \onslide*<2>{\listCamlRaiseExn[highlightlines={707-708}]{}}%
      \onslide*<3>{\listCamlRaiseExn[highlightlines={712-714}]{}}%
      \onslide*<4>{\listCamlRaiseExn[highlightlines={715-720}]{}}%
      \onslide*<5>{\providecommand\step{1}\input{diagrams/backtrace/backtrace.tex}}%
      \onslide*<6>{\providecommand\step{2}\input{diagrams/backtrace/backtrace.tex}}%
    \end{column}
  \end{columns}
\end{frame}

\newcommand\listStashBacktrace[1][]{
  \listc[firstline=91,lastline=120,#1]{../ocaml/runtime/backtrace\_nat.c}{runtime/backtrace\_nat.c:91}}

\begin{frame}{Backtraces}{Introducing \funcname{caml\_stash\_backtrace}}
  \begin{columns}[c]
    \begin{column}{0.5\textwidth}
      \minipage[c][0.66\textheight][s]{\columnwidth}
      \begin{itemize}
        \item<1-> A C function provided by the runtime (specific to native code)
        \item<2-> It scans the stack, between the stack pointer at the raising point up to the next trap, in order to collect the names of the detected function frames
          \onslide*<2>{
            \begin{itemize}
              \item And more information, like line number, column number...
            \end{itemize}
          }
        \item<3-> It uses the same technique and information as the Garbage Collector (GC) uses to scan the values live on the stack
          \onslide*<4>{
            \begin{itemize}
              \item \typename{frame\_descr} are at the core of stack unwinding
            \end{itemize}
          }
      \end{itemize}
      \vfill
      \mintocaml[firstline=9,lastline=13,highlightlines=12]{../src/backtrace/backtrace.ml}
      \endminipage
    \end{column}
    \begin{column}{0.5\textwidth}
      \onslide*<1>{\listStashBacktrace[highlightlines=91]{}}
      \onslide*<2>{\listStashBacktrace[highlightlines={91,117-118}]{}}
      \onslide*<3->{\listStashBacktrace[highlightlines={94,106,109-110}]{}}
    \end{column}
  \end{columns}
\end{frame}

\newcommand\listFrameDescr[1][]{
  \listocaml[firstline=111,lastline=124,#1]{../ocaml/asmcomp/emitaux.ml}{asmcomp/emitaux.ml:111}}
\newcommand\listDebugInfo[1][]{
  \listocaml[firstline=38,lastline=49,#1]{../ocaml/lambda/debuginfo.mli}{lambda/debuginfo.mli:38}}

\begin{frame}{Backtraces}{\typename{frame\_descr} and \typename{frame\_debuginfo} in a nutshell}
  \begin{columns}[c]
    \begin{column}{0.5\textwidth}
      \begin{itemize}
        \item<1-> For every return address in an OCaml program, there is a associated \typename{frame\_descr}
        \item<2-> A \typename{frame\_descr} specifies
        \begin{itemize}
          \item<2-> The size of the function stack frame
          \item<3-> Where live values are on the stack (so that the GC can mark them as live and prevent from being swept)
        \end{itemize}
        \item<4-> When compiled with debug information, \typename{frame\_descr} will be linked to a \typename{frame\_debuginfo}
        \begin{itemize}
          \item<5-> Contains information about the position in the source code (file name, line and column number)
        \end{itemize}
      \end{itemize}
    \end{column}
    \begin{column}{0.5\textwidth}
        \onslide*<1>{\listFrameDescr[highlightlines={118-122}]{}}
        \onslide*<2>{\listFrameDescr[highlightlines={120}]{}}
        \onslide*<3>{\listFrameDescr[highlightlines={121}]{}}
        \onslide*<4->{\listFrameDescr[highlightlines={122}]{}}
        \medskip
        \onslide*<1-3>{\listDebugInfo{}}
        \onslide*<4>{\listDebugInfo[highlightlines={38,49}]{}}
        \onslide*<5>{\listDebugInfo[highlightlines={39-46}]{}}
    \end{column}
  \end{columns}
\end{frame}

\begin{frame}{Backtraces}{Scanning the stack with \typename{frame\_descr}}
  \begin{columns}[c]
    \begin{column}{0.5\textwidth}
      \begin{itemize}
        \item Given the return address of the code that raises the exception by calling \funcname{caml\_raise\_exn} (\funcarg{pc} argument of \funcname{caml\_stash\_backtrace})
        \item And the position of the stack pointer (\funcarg{sp} argument of \funcname{caml\_stash\_backtrace})
        \item The frame size given by \typename{frame\_descr}.\funcarg{frame\_size}, allows to find the end of the previous frame
        \item And the return address of the caller of this function
        \bigskip
        \onslide<3->{
        \item Note that traps (here the reddish \funcname{ExnA}, \funcname{ExnB} and \funcname{ExnC} blocks) belong to the frame that installed them, and so the frame size changes when traps are removed
        }
      \end{itemize}
    \end{column}
    \begin{column}{0.5\textwidth}
      \raggedleft
      \onslide*<1>{\providecommand\step{2}\input{diagrams/backtrace/backtrace.tex}}%
      \onslide*<2>{\providecommand\step{1}\input{diagrams/backtrace/scanstack.tex}}%
      \onslide*<3>{\providecommand\step{2}\input{diagrams/backtrace/scanstack.tex}}%
      \onslide*<4>{\providecommand\step{3}\input{diagrams/backtrace/scanstack.tex}}%
      \onslide*<5>{\providecommand\step{4}\input{diagrams/backtrace/scanstack.tex}}%
      \onslide*<6>{\providecommand\step{5}\input{diagrams/backtrace/scanstack.tex}}%
    \end{column}
  \end{columns}
\end{frame}

% Duplicate of previous-previous slide
\begin{frame}{Backtraces}{Continuing on \funcname{caml\_stash\_backtrace}}
  \begin{columns}[c]
    \begin{column}{0.5\textwidth}
      \minipage[c][0.75\textheight][s]{\columnwidth}
      \begin{itemize}
          \color{gray}
        \item A C function provided by the runtime (specific to native code)
        \item It scans the stack, between the stack pointer at the raising point up to the next trap, in order to collect the names of the detected function frames
        \item It uses the same technique and information as the Garbage Collector (GC) uses to scan the values live on the stack
      \end{itemize}
      \begin{itemize}
        \item For each function frame detected, store a pointer to the \typename{frame\_descr} in the \localname{domain\_state->backtrace\_buffer} array, effectively constructing the backtrace
      \end{itemize}
      \onslide<2->{
        \begin{itemize}
            \smallskip
          \item Constructed backtrace:
            \funcname{definitely\_raise},
            \onslide<3->{\funcname{probably\_raise}}
        \end{itemize}
      }
      \vfill
      \mintocaml[firstline=9,lastline=13,highlightlines=12]{../src/backtrace/backtrace.ml}
      \endminipage
    \end{column}
    \begin{column}{0.5\textwidth}
      \raggedleft
      \onslide*<1>{\listStashBacktrace[highlightlines={114-115}]{}}
      \onslide*<2>{\providecommand\step{3}\input{diagrams/backtrace/backtrace.tex}}%
      \onslide*<3>{\providecommand\step{4}\input{diagrams/backtrace/backtrace.tex}}%
    \end{column}
  \end{columns}
\end{frame}

\begin{frame}{Backtraces}{Back to \funcname{caml\_raise\_exn}}
  \begin{columns}[c]
    \begin{column}{0.5\textwidth}
      \minipage[c][0.66\textheight][s]{\columnwidth}
      \begin{itemize}\color{gray}
        \item Checks wheteher exceptions backtrace should be recorded, by checking the \funcarg{domain\_state->backtrace\_active} value
        \item Switches to the C stack to be able to execute C code
        \item Calls \funcname{caml\_stash\_backtrace}, to collect the backtrace up to the next trap
      \end{itemize}
      \onslide*<2->{
        \begin{itemize}
          \item<2-> Executes the exception handler in \funcname{probably\_raise} with \funcname{RESTORE\_EXN\_HANDLER\_OCAML}
            \begin{itemize}
              \item Set \regname{RSP} to \localname{domain\_state->exn\_handler} value
              \item Restore \localname{domain\_state->exn\_handler} from trap
              \item Jump to the handler code with \funcname{ret}
            \end{itemize}
        \end{itemize}
      }
      \vfill
      \onslide*<1>{\mintocaml[firstline=9,lastline=13,highlightlines=12]{../src/backtrace/backtrace.ml}}
      \onslide*<2->{\mintocaml[firstline=15,lastline=21,highlightlines={19-21}]{../src/backtrace/backtrace.ml}}
      \endminipage
    \end{column}
    \begin{column}{0.5\textwidth}
      \centering
      \onslide*<1>{
        \mintc[firstline=190,lastline=192]{../ocaml/runtime/amd64.S}
        \mintc[firstline=234,lastline=237]{../ocaml/runtime/amd64.S}
      }
      \onslide*<2>{
        \mintc[firstline=190,lastline=192,highlightlines={191-192}]{../ocaml/runtime/amd64.S}
        \mintc[firstline=234,lastline=237,highlightlines={235-237}]{../ocaml/runtime/amd64.S}
      }
      \onslide*<1>{\listCamlRaiseExn[highlightlines=720]{}}
      \onslide*<2>{\listCamlRaiseExn[highlightlines=722]{}}
      \onslide*<3->{\providecommand\step{5}\input{diagrams/backtrace/backtrace.tex}}
    \end{column}
  \end{columns}
\end{frame}

\begin{frame}{Backtraces}{\funcname{caml\_probably\_raise}'s handler doesn't catch \typename{ExnC}}
  \begin{columns}[c]
    \begin{column}{0.45\textwidth}
      \minipage[c][0.75\textheight][s]{\columnwidth}
      \begin{itemize}
        \item<1-> \funcname{match \dots with}\footnotemark pattern matching can also match exceptions
        \item<1-> The CMM of \funcname{probably\_raise} shows that this \funcname{match \dots with} is implemented with a pattern matching and a \funcname{try \dots with}
        \item<1-> But this one only matches on \typename{ExnA} exception
        \item<2-> Since the pattern matching fails to catch the exception, \funcname{reraise} is called with our current \typename{ExnC} exception
        \item<3-> Disassembly shows that \funcname{reraise} is actually a call to \funcname{caml\_reraise\_exn}, which is also an assembly function provided by the runtime
      \end{itemize}
      \vfill
      \mintocaml[firstline=15,lastline=21,highlightlines={18-21}]{../src/backtrace/backtrace.ml}
      \endminipage
    \end{column}
    \begin{column}{0.55\textwidth}
      \centering
      \onslide*<1>{\mintcmm[highlightlines={10-18}]{../src/backtrace/camlBacktrace\_\_probably\_raise\_351.cmm}}
      \onslide*<2>{\listcmm[highlightlines=18]{../src/backtrace/camlBacktrace\_\_probably\_raise_351.cmm}{CMM of probably\_raise}}
      \onslide*<3>{\mintobjdump[firstline=5,lastline=35,highlightlines=31]{../src/backtrace/camlBacktrace\_\_probably\_raise_351.objdump}}
    \end{column}
  \end{columns}
  \only<1>{\footnotetext[1]{https://blog.janestreet.com/pattern-matching-and-exception-handling-unite/}}
\end{frame}

\newcommand\listCamlReraiseExn[1][]{
  \listasm[firstline=727,lastline=734,#1]{../ocaml/runtime/amd64.S}{runtime/amd64.S:727}}

\begin{frame}{Backtraces}{Introducing \funcname{caml\_reraise\_exn}}
  \begin{columns}[c]
    \begin{column}{0.5\textwidth}
      \minipage[c][0.66\textheight][s]{\columnwidth}
      \begin{itemize}
        \item<1-> The exception have to be reraised to the next handler
        \item<2-> First, checks if the backtrace should be collected with \funcarg{domain\_state->backtrace\_active}
        \only<3>{
        \begin{itemize}
            \item If not, behaves like a \funcname{raise\_notrace} with \funcname{RESTORE\_EXN\_HANDLER\_OCAML}
        \end{itemize}
        }
        \item<4-> Jumps into \funcname{caml\_raise\_exn}
        \item<5-> Calls \funcname{caml\_stash\_backtrace} again, to scan the next part of the stack: from the trap just pop-ed to the next trap
      \end{itemize}
      \vfill
      \mintocaml[firstline=15,lastline=21,highlightlines={18-21}]{../src/backtrace/backtrace.ml}
      \endminipage
    \end{column}
    \begin{column}{0.5\textwidth}
      \raggedleft
      \onslide*<1>{\listCamlReraiseExn{}}
      \onslide*<2>{\listCamlReraiseExn[highlightlines=729]{}}
      \onslide*<3>{
        \mintc[firstline=190,lastline=192,highlightlines={191-192}]{../ocaml/runtime/amd64.S}
        \mintc[firstline=234,lastline=237,highlightlines={235-237}]{../ocaml/runtime/amd64.S}
        \medskip
        \listCamlReraiseExn[highlightlines=731]{}
      }
      \onslide*<4-5>{
        % TODO refactor with \listCamlReraiseExn
        \mintasm[firstline=727,lastline=734,highlightlines=730]{../ocaml/runtime/amd64.S}
        \medskip
      }
      \onslide*<4>{\listCamlRaiseExn[highlightlines=711]{}}
      \onslide*<5>{\listCamlRaiseExn[highlightlines=720]{}}
    \end{column}
  \end{columns}
\end{frame}

\begin{frame}{Backtraces}{Backtrace collection continues}
  \begin{columns}[c]
    \begin{column}{0.55\textwidth}
      \minipage[c][0.75\textheight][s]{\columnwidth}
      \begin{itemize}
        \item<1-> \funcname{caml\_stash\_backtrace} scans the older part of the stack, up to the next trap
        \item<6-> Controls jumps to the nested exception handler in \funcname{maybe\_raise}, which matches only on \typename{ExnB}
        \item<7-> \funcname{caml\_reraise\_exn} is called again, backtrace collection resumes with \funcname{caml\_stash\_backtrace}
      \end{itemize}
      \medskip
      \begin{itemize}
        \item<2-> Constructed backtrace:
        \begin{itemize}
          \item <2-> \funcname{definitely\_raise}
          \item <2-> \funcname{probably\_raise}
          \item <3-> \funcname{probably\_raise} (re-raise)
          \item <4-> \funcname{maybe\_raise}
          \item <5-> \funcname{main}
          \item <8-> \funcname{main} (re-raise)
        \end{itemize}
      \end{itemize}
      \vfill
      \onslide*<1-5>{\mintocaml[firstline=15,lastline=21]{../src/backtrace/backtrace.ml}}
      \onslide*<6>{\mintocaml[firstline=28,lastline=34,highlightlines=31]{../src/backtrace/backtrace.ml}}
      \onslide*<7->{\mintocaml[firstline=28,lastline=34,highlightlines=33]{../src/backtrace/backtrace.ml}}
      \endminipage
    \end{column}
    \begin{column}{0.45\textwidth}
      \raggedleft
      \onslide*<1>{\providecommand\step{6}\input{diagrams/backtrace/backtrace.tex}}%
      \onslide*<2>{\providecommand\step{6}\input{diagrams/backtrace/backtrace.tex}}%
      \onslide*<3>{\providecommand\step{7}\input{diagrams/backtrace/backtrace.tex}}%
      \onslide*<4>{\providecommand\step{8}\input{diagrams/backtrace/backtrace.tex}}%
      \onslide*<5>{\providecommand\step{9}\input{diagrams/backtrace/backtrace.tex}}%
      \onslide*<6>{\providecommand\step{10}\input{diagrams/backtrace/backtrace.tex}}%
      \onslide*<7>{\providecommand\step{11}\input{diagrams/backtrace/backtrace.tex}}%
      \onslide*<8>{\providecommand\step{12}\input{diagrams/backtrace/backtrace.tex}}%
      \onslide*<9>{\providecommand\step{13}\input{diagrams/backtrace/backtrace.tex}}%
    \end{column}
  \end{columns}
\end{frame}

\begin{frame}{Backtraces}{Printing the backtrace}
  \begin{columns}[c]
    \begin{column}{0.45\textwidth}
      \minipage[c][0.66\textheight][s]{\columnwidth}
      \begin{itemize}
        \item<1-> The exception backtrace can now be printed to \funcarg{stdout} with \funcname{Printexc.print_backtrace}
        \item<2-> First the exception backtrace is retrieved into an OCaml array with \funcname{get_raw_backtrace }
        \item<3-> Which is implemented as the \funcname{caml_get_exception_raw_backtrace} C function in the runtime
        \item<4-> And finally printed with \funcname{print_exception_backtrace}
      \end{itemize}
      \vfill
      \mintocaml[firstline=28,lastline=34,highlightlines=33]{../src/backtrace/backtrace.ml}
      \endminipage
    \end{column}
    \begin{column}{0.55\textwidth}
      \onslide*<1,2>{
        \mintocaml[firstline=100,lastline=101]{../ocaml/stdlib/printexc.ml}
      }
      \only<1>{
        \listocaml[firstline=170,lastline=172]{../ocaml/stdlib/printexc.ml}{stdlib/printexc.ml:170}
      }
      \only<2>{
        \listocaml[firstline=170,lastline=172,highlightlines=172]{../ocaml/stdlib/printexc.ml}{stdlib/printexc.ml:170}
      }
      \only<3>{
        \mintc[firstline=154,lastline=156]{../ocaml/runtime/backtrace.c}
        \listc[firstline=171,lastline=192,highlightlines={184-187}]{../ocaml/runtime/backtrace.c}{runtime/backtrace.c:154}
      }
      \only<4>{
        \listocaml[firstline=155,lastline=165]{../ocaml/stdlib/printexc.ml}{stdlib/printexc.ml:55}
      }
    \end{column}
  \end{columns}
\end{frame}


\subsection{The multiple kinds of raise}
\frameSubsection{}{}

\begin{frame}{Backtraces}{When backtraces are not needed}
  \begin{columns}[c]
    \begin{column}{0.45\textwidth}
      \minipage[c][0.66\textheight][s]{\columnwidth}
      \begin{itemize}
        \item<1-> What if the backtrace for an exception is never needed?
        \item<2-> It's possible to raise the exception explicitly with \funcname{raise\_notrace} to make raising as cheap as possible
        \item<3-> An example of use in the \funcname{dynlink} library of the OCaml compiler
      \end{itemize}
      \vfill
      \onslide<3->{
      \listocaml[firstline=205,lastline=209,highlightlines=207]{../ocaml/otherlibs/dynlink/dynlink\_compilerlibs/misc.ml}{otherlibs/dynlink/dynlink\_compilerlibs/misc.ml:205}
      }
      \endminipage
    \end{column}
    \begin{column}{0.55\textwidth}
    \only<4>{
      \mintcmm[highlightlines=3]{../src/notrace/camlNotrace\_\_fun_347.cmm}
      \listcmm{../src/notrace/camlNotrace\_\_all\_somes\_327.cmm}{all\_somes}
    }
    \onslide*<5->{
      \listobjdump[highlightlines={6-9}]{../src/notrace/camlNotrace\_\_fun_347.objdump}{anonymous function from \funcname{all\_somes}}
    }
    \end{column}
  \end{columns}
\end{frame}

\begin{frame}{Backtraces}{Summary table of the multiple kinds of raise}
  \begin{itemize}
    \item When debugging information is enabled and if the backtrace of an exception isn't needed, it's possible to raise with \funcname{raise\_notrace} for performance
    \item When debugging information is \emph{not} enabled, the compiler optimises \funcname{raise} calls to inline assembly (same effect as using \funcname{raise\_notrace})
  \end{itemize}
  \bigskip
  \centering
  \begin{tabular}{| l | c | c | c | c |}
    \hline
    \parbox{10em}{Debug information \\ (\funcarg{-g} flag)}  & \multicolumn{2}{|c|}{Disabled}                                     & \multicolumn{2}{|c|}{Enabled} \\
    \hline
    Raise kind                                               & \funcname{raise} & \funcname{raise\_notrace}                       & \funcname{raise} & \funcname{raise\_notrace} \\
    \hline
    Compilation                                              & \parbox{8em}{inline assembly\\ (optimization)} & inline assembly   & \funcname{caml\_raise\_exn} & inline assembly \\
    \hline
  \end{tabular}
\end{frame}


%
% Takeaway #5
%
\subsection*{Takeaway \#5}
\frameSubsectionTakeaway{}
\againframe<5>{takeaway}
}%
      \onslide*<6>{\providecommand\step{2}%
% Backtraces
%
\subsection{One advantage of raising with debugging information}
\frameSubsection{}{}

\begin{frame}{Backtraces}{Debugging information \& source code}
  \begin{columns}[c]
    \begin{column}{0.5\textwidth}
      \begin{itemize}
        \item Raising an exception is fun and games, but how do we know where the exception originated? $\rightarrow$ With the backtrace associated with the exception!
        \item In order to get a backtrace from the point where an exception was raised, it's necessary to compile with debugging information enabled (\funcarg{-g} flag for the compiler)
        \item Same example as in the `Nested handlers' section
      \end{itemize}
    \end{column}
    \begin{column}{0.5\textwidth}
      \listocaml[firstline=9,lastline=34]{../src/backtrace/backtrace.ml}{backtrace/backtrace.ml}
    \end{column}
  \end{columns}
\end{frame}

\begin{frame}{Backtraces}{CMM and disassembly of \funcname{definitely\_raise}}
  \begin{columns}[c]
    \begin{column}{0.5\textwidth}
      \minipage[c][0.66\textheight][s]{\columnwidth}
      \begin{itemize}
        \item<1-> An effect of compiling with debugging information (\funcarg{-g}): the CMM no longer uses \funcname{raise\_notrace} to raise the exception but \funcname{raise} instead
        \item<2-> Disassembly shows that CMM \funcname{raise} is actually a call to \funcname{caml\_raise\_exn}, a function in assembler provided by the runtime
      \end{itemize}
      \vfill
      \mintocaml[firstline=9,lastline=13,highlightlines=12]{../src/backtrace/backtrace.ml}
      \endminipage
    \end{column}
    \begin{column}{0.5\textwidth}
      \onslide*<1>{
        \mintcmm[highlightlines=5]{../src/backtrace/camlBacktrace\_\_definitely\_raise\_347.cmm}
      }
      \onslide*<2>{
        \mintobjdump[firstline=2,highlightlines=14]{../src/backtrace/camlBacktrace\_\_definitely\_raise\_347.objdump}
      }
    \end{column}
  \end{columns}
\end{frame}

\begin{frame}{Backtraces}{Introducing \funcname{caml\_raise\_exn}}
  \begin{columns}[c]
    \begin{column}{0.5\textwidth}
      \minipage[c][0.66\textheight][s]{\columnwidth}
      \onslide*<1>{
        \begin{itemize}
          \item Raising the exception is performed using \funcname{caml\_raise\_exn} which is
            \begin{itemize}
              \item An assembly function provided by the runtime
              \item Architecture specific
            \end{itemize}
          \item Let's see what it does differently from \funcname{raise\_notrace}\dots
          \item We are about to raise \typename{ExnC} with \funcname{caml\_raise\_exn} from \funcname{definitely\_raise}
        \end{itemize}
      }
      \onslide*<2>{
        \mintobjdump[firstline=2,highlightlines=14]{../src/backtrace/camlBacktrace\_\_definitely\_raise\_347.objdump}
      }
      \vfill
      \mintocaml[firstline=9,lastline=13,highlightlines=12]{../src/backtrace/backtrace.ml}
      \endminipage
    \end{column}
    \begin{column}{0.5\textwidth}
      \centering
      \providecommand\step{1}\input{diagrams/backtrace/backtrace.tex}
    \end{column}
  \end{columns}
\end{frame}

\begin{frame}{Backtraces}{But before that, we need more fields from \typename{caml\_domain\_state}}
  \begin{columns}
    \begin{column}{0.5\textwidth}
      \begin{itemize}
        \item Remember \typename{caml\_domain\_state} the per-domain runtime state?
        \item Some other members are of interest to us now\dots
        \item \localname{backtrace\_active} is used to know if the runtime should collect backtraces
        \item \localname{backtrace\_active} can be set by
          \begin{itemize}
            \item The \funcarg{b} flag in the \funcname{OCAMLRUNPARAM} environment variable
            \item \mintinline{ocaml}{Printexc.record_backtrace : bool -> unit} \footnotemark
          \end{itemize}
      \end{itemize}
    \end{column}
    \begin{column}{0.5\textwidth}
      \begin{listing}
        \mintc[highlightlines={9-11}]{src/ocaml/domain\_state\_backtrace.h}
        \caption{runtime/caml/domain\_state.h}
      \end{listing}
    \end{column}
  \end{columns}
  \footnotetext[1]{https://v2.ocaml.org/api/Printexc.html}
\end{frame}

\newcommand\listCamlRaiseExn[1][]{
  \listasm[firstline=702,lastline=725,#1]{../ocaml/runtime/amd64.S}{runtime/amd64.S:702}}

\begin{frame}{Backtraces}{Walk through \funcname{caml\_raise\_exn}}
  \begin{columns}[T]
    \begin{column}{0.5\textwidth}
      \begin{itemize}
        \item<1-> First checks whether exceptions backtrace should be recorded
        \item<1-> By checking the \funcarg{domain\_state->backtrace\_active} value
        \item<2-> If not, acts as \funcname{raise\_notrace} with \funcname{RESTORE\_EXN\_HANDLER\_OCAML}
          \begin{itemize}
            \item Set \regname{RSP} to \localname{domain\_state->exn\_handler} value
            \item Restore \localname{domain\_state->exn\_handler} from trap
            \item Jump with \funcname{ret} (same effect as using \funcname{pop} and \funcname{jmp})
          \end{itemize}
        \item<3-> Otherwise, switches to the C stack to be able to execute C code
        \item<4-> Calls \funcname{caml\_stash\_backtrace}, a C function provided by the runtime\ignorespaces
          \onslide<6->{, with
            \begin{itemize}
              \item \funcarg{exn}: exception value
              \item \funcarg{pc}: code address of the raising point
              \item \funcarg{sp}: stack address of the raising function
              \item \funcarg{trapsp}: stack address of the next handler
            \end{itemize}
          }
      \end{itemize}
      \bigskip
      \mintocaml[firstline=9,lastline=13,highlightlines=12]{../src/backtrace/backtrace.ml}
    \end{column}
    \begin{column}{0.5\textwidth}
      \raggedleft
      \onslide*<1,3-4>{
        \mintc[firstline=190,lastline=192]{../ocaml/runtime/amd64.S}
        \mintc[firstline=234,lastline=237]{../ocaml/runtime/amd64.S}
      }
      \onslide*<2>{
        \mintc[firstline=190,lastline=192,highlightlines={191-192}]{../ocaml/runtime/amd64.S}
        \mintc[firstline=234,lastline=237,highlightlines={235-237}]{../ocaml/runtime/amd64.S}
      }
      \onslide*<1>{\listCamlRaiseExn[highlightlines=705]{}}%
      \onslide*<2>{\listCamlRaiseExn[highlightlines={707-708}]{}}%
      \onslide*<3>{\listCamlRaiseExn[highlightlines={712-714}]{}}%
      \onslide*<4>{\listCamlRaiseExn[highlightlines={715-720}]{}}%
      \onslide*<5>{\providecommand\step{1}\input{diagrams/backtrace/backtrace.tex}}%
      \onslide*<6>{\providecommand\step{2}\input{diagrams/backtrace/backtrace.tex}}%
    \end{column}
  \end{columns}
\end{frame}

\newcommand\listStashBacktrace[1][]{
  \listc[firstline=91,lastline=120,#1]{../ocaml/runtime/backtrace\_nat.c}{runtime/backtrace\_nat.c:91}}

\begin{frame}{Backtraces}{Introducing \funcname{caml\_stash\_backtrace}}
  \begin{columns}[c]
    \begin{column}{0.5\textwidth}
      \minipage[c][0.66\textheight][s]{\columnwidth}
      \begin{itemize}
        \item<1-> A C function provided by the runtime (specific to native code)
        \item<2-> It scans the stack, between the stack pointer at the raising point up to the next trap, in order to collect the names of the detected function frames
          \onslide*<2>{
            \begin{itemize}
              \item And more information, like line number, column number...
            \end{itemize}
          }
        \item<3-> It uses the same technique and information as the Garbage Collector (GC) uses to scan the values live on the stack
          \onslide*<4>{
            \begin{itemize}
              \item \typename{frame\_descr} are at the core of stack unwinding
            \end{itemize}
          }
      \end{itemize}
      \vfill
      \mintocaml[firstline=9,lastline=13,highlightlines=12]{../src/backtrace/backtrace.ml}
      \endminipage
    \end{column}
    \begin{column}{0.5\textwidth}
      \onslide*<1>{\listStashBacktrace[highlightlines=91]{}}
      \onslide*<2>{\listStashBacktrace[highlightlines={91,117-118}]{}}
      \onslide*<3->{\listStashBacktrace[highlightlines={94,106,109-110}]{}}
    \end{column}
  \end{columns}
\end{frame}

\newcommand\listFrameDescr[1][]{
  \listocaml[firstline=111,lastline=124,#1]{../ocaml/asmcomp/emitaux.ml}{asmcomp/emitaux.ml:111}}
\newcommand\listDebugInfo[1][]{
  \listocaml[firstline=38,lastline=49,#1]{../ocaml/lambda/debuginfo.mli}{lambda/debuginfo.mli:38}}

\begin{frame}{Backtraces}{\typename{frame\_descr} and \typename{frame\_debuginfo} in a nutshell}
  \begin{columns}[c]
    \begin{column}{0.5\textwidth}
      \begin{itemize}
        \item<1-> For every return address in an OCaml program, there is a associated \typename{frame\_descr}
        \item<2-> A \typename{frame\_descr} specifies
        \begin{itemize}
          \item<2-> The size of the function stack frame
          \item<3-> Where live values are on the stack (so that the GC can mark them as live and prevent from being swept)
        \end{itemize}
        \item<4-> When compiled with debug information, \typename{frame\_descr} will be linked to a \typename{frame\_debuginfo}
        \begin{itemize}
          \item<5-> Contains information about the position in the source code (file name, line and column number)
        \end{itemize}
      \end{itemize}
    \end{column}
    \begin{column}{0.5\textwidth}
        \onslide*<1>{\listFrameDescr[highlightlines={118-122}]{}}
        \onslide*<2>{\listFrameDescr[highlightlines={120}]{}}
        \onslide*<3>{\listFrameDescr[highlightlines={121}]{}}
        \onslide*<4->{\listFrameDescr[highlightlines={122}]{}}
        \medskip
        \onslide*<1-3>{\listDebugInfo{}}
        \onslide*<4>{\listDebugInfo[highlightlines={38,49}]{}}
        \onslide*<5>{\listDebugInfo[highlightlines={39-46}]{}}
    \end{column}
  \end{columns}
\end{frame}

\begin{frame}{Backtraces}{Scanning the stack with \typename{frame\_descr}}
  \begin{columns}[c]
    \begin{column}{0.5\textwidth}
      \begin{itemize}
        \item Given the return address of the code that raises the exception by calling \funcname{caml\_raise\_exn} (\funcarg{pc} argument of \funcname{caml\_stash\_backtrace})
        \item And the position of the stack pointer (\funcarg{sp} argument of \funcname{caml\_stash\_backtrace})
        \item The frame size given by \typename{frame\_descr}.\funcarg{frame\_size}, allows to find the end of the previous frame
        \item And the return address of the caller of this function
        \bigskip
        \onslide<3->{
        \item Note that traps (here the reddish \funcname{ExnA}, \funcname{ExnB} and \funcname{ExnC} blocks) belong to the frame that installed them, and so the frame size changes when traps are removed
        }
      \end{itemize}
    \end{column}
    \begin{column}{0.5\textwidth}
      \raggedleft
      \onslide*<1>{\providecommand\step{2}\input{diagrams/backtrace/backtrace.tex}}%
      \onslide*<2>{\providecommand\step{1}\input{diagrams/backtrace/scanstack.tex}}%
      \onslide*<3>{\providecommand\step{2}\input{diagrams/backtrace/scanstack.tex}}%
      \onslide*<4>{\providecommand\step{3}\input{diagrams/backtrace/scanstack.tex}}%
      \onslide*<5>{\providecommand\step{4}\input{diagrams/backtrace/scanstack.tex}}%
      \onslide*<6>{\providecommand\step{5}\input{diagrams/backtrace/scanstack.tex}}%
    \end{column}
  \end{columns}
\end{frame}

% Duplicate of previous-previous slide
\begin{frame}{Backtraces}{Continuing on \funcname{caml\_stash\_backtrace}}
  \begin{columns}[c]
    \begin{column}{0.5\textwidth}
      \minipage[c][0.75\textheight][s]{\columnwidth}
      \begin{itemize}
          \color{gray}
        \item A C function provided by the runtime (specific to native code)
        \item It scans the stack, between the stack pointer at the raising point up to the next trap, in order to collect the names of the detected function frames
        \item It uses the same technique and information as the Garbage Collector (GC) uses to scan the values live on the stack
      \end{itemize}
      \begin{itemize}
        \item For each function frame detected, store a pointer to the \typename{frame\_descr} in the \localname{domain\_state->backtrace\_buffer} array, effectively constructing the backtrace
      \end{itemize}
      \onslide<2->{
        \begin{itemize}
            \smallskip
          \item Constructed backtrace:
            \funcname{definitely\_raise},
            \onslide<3->{\funcname{probably\_raise}}
        \end{itemize}
      }
      \vfill
      \mintocaml[firstline=9,lastline=13,highlightlines=12]{../src/backtrace/backtrace.ml}
      \endminipage
    \end{column}
    \begin{column}{0.5\textwidth}
      \raggedleft
      \onslide*<1>{\listStashBacktrace[highlightlines={114-115}]{}}
      \onslide*<2>{\providecommand\step{3}\input{diagrams/backtrace/backtrace.tex}}%
      \onslide*<3>{\providecommand\step{4}\input{diagrams/backtrace/backtrace.tex}}%
    \end{column}
  \end{columns}
\end{frame}

\begin{frame}{Backtraces}{Back to \funcname{caml\_raise\_exn}}
  \begin{columns}[c]
    \begin{column}{0.5\textwidth}
      \minipage[c][0.66\textheight][s]{\columnwidth}
      \begin{itemize}\color{gray}
        \item Checks wheteher exceptions backtrace should be recorded, by checking the \funcarg{domain\_state->backtrace\_active} value
        \item Switches to the C stack to be able to execute C code
        \item Calls \funcname{caml\_stash\_backtrace}, to collect the backtrace up to the next trap
      \end{itemize}
      \onslide*<2->{
        \begin{itemize}
          \item<2-> Executes the exception handler in \funcname{probably\_raise} with \funcname{RESTORE\_EXN\_HANDLER\_OCAML}
            \begin{itemize}
              \item Set \regname{RSP} to \localname{domain\_state->exn\_handler} value
              \item Restore \localname{domain\_state->exn\_handler} from trap
              \item Jump to the handler code with \funcname{ret}
            \end{itemize}
        \end{itemize}
      }
      \vfill
      \onslide*<1>{\mintocaml[firstline=9,lastline=13,highlightlines=12]{../src/backtrace/backtrace.ml}}
      \onslide*<2->{\mintocaml[firstline=15,lastline=21,highlightlines={19-21}]{../src/backtrace/backtrace.ml}}
      \endminipage
    \end{column}
    \begin{column}{0.5\textwidth}
      \centering
      \onslide*<1>{
        \mintc[firstline=190,lastline=192]{../ocaml/runtime/amd64.S}
        \mintc[firstline=234,lastline=237]{../ocaml/runtime/amd64.S}
      }
      \onslide*<2>{
        \mintc[firstline=190,lastline=192,highlightlines={191-192}]{../ocaml/runtime/amd64.S}
        \mintc[firstline=234,lastline=237,highlightlines={235-237}]{../ocaml/runtime/amd64.S}
      }
      \onslide*<1>{\listCamlRaiseExn[highlightlines=720]{}}
      \onslide*<2>{\listCamlRaiseExn[highlightlines=722]{}}
      \onslide*<3->{\providecommand\step{5}\input{diagrams/backtrace/backtrace.tex}}
    \end{column}
  \end{columns}
\end{frame}

\begin{frame}{Backtraces}{\funcname{caml\_probably\_raise}'s handler doesn't catch \typename{ExnC}}
  \begin{columns}[c]
    \begin{column}{0.45\textwidth}
      \minipage[c][0.75\textheight][s]{\columnwidth}
      \begin{itemize}
        \item<1-> \funcname{match \dots with}\footnotemark pattern matching can also match exceptions
        \item<1-> The CMM of \funcname{probably\_raise} shows that this \funcname{match \dots with} is implemented with a pattern matching and a \funcname{try \dots with}
        \item<1-> But this one only matches on \typename{ExnA} exception
        \item<2-> Since the pattern matching fails to catch the exception, \funcname{reraise} is called with our current \typename{ExnC} exception
        \item<3-> Disassembly shows that \funcname{reraise} is actually a call to \funcname{caml\_reraise\_exn}, which is also an assembly function provided by the runtime
      \end{itemize}
      \vfill
      \mintocaml[firstline=15,lastline=21,highlightlines={18-21}]{../src/backtrace/backtrace.ml}
      \endminipage
    \end{column}
    \begin{column}{0.55\textwidth}
      \centering
      \onslide*<1>{\mintcmm[highlightlines={10-18}]{../src/backtrace/camlBacktrace\_\_probably\_raise\_351.cmm}}
      \onslide*<2>{\listcmm[highlightlines=18]{../src/backtrace/camlBacktrace\_\_probably\_raise_351.cmm}{CMM of probably\_raise}}
      \onslide*<3>{\mintobjdump[firstline=5,lastline=35,highlightlines=31]{../src/backtrace/camlBacktrace\_\_probably\_raise_351.objdump}}
    \end{column}
  \end{columns}
  \only<1>{\footnotetext[1]{https://blog.janestreet.com/pattern-matching-and-exception-handling-unite/}}
\end{frame}

\newcommand\listCamlReraiseExn[1][]{
  \listasm[firstline=727,lastline=734,#1]{../ocaml/runtime/amd64.S}{runtime/amd64.S:727}}

\begin{frame}{Backtraces}{Introducing \funcname{caml\_reraise\_exn}}
  \begin{columns}[c]
    \begin{column}{0.5\textwidth}
      \minipage[c][0.66\textheight][s]{\columnwidth}
      \begin{itemize}
        \item<1-> The exception have to be reraised to the next handler
        \item<2-> First, checks if the backtrace should be collected with \funcarg{domain\_state->backtrace\_active}
        \only<3>{
        \begin{itemize}
            \item If not, behaves like a \funcname{raise\_notrace} with \funcname{RESTORE\_EXN\_HANDLER\_OCAML}
        \end{itemize}
        }
        \item<4-> Jumps into \funcname{caml\_raise\_exn}
        \item<5-> Calls \funcname{caml\_stash\_backtrace} again, to scan the next part of the stack: from the trap just pop-ed to the next trap
      \end{itemize}
      \vfill
      \mintocaml[firstline=15,lastline=21,highlightlines={18-21}]{../src/backtrace/backtrace.ml}
      \endminipage
    \end{column}
    \begin{column}{0.5\textwidth}
      \raggedleft
      \onslide*<1>{\listCamlReraiseExn{}}
      \onslide*<2>{\listCamlReraiseExn[highlightlines=729]{}}
      \onslide*<3>{
        \mintc[firstline=190,lastline=192,highlightlines={191-192}]{../ocaml/runtime/amd64.S}
        \mintc[firstline=234,lastline=237,highlightlines={235-237}]{../ocaml/runtime/amd64.S}
        \medskip
        \listCamlReraiseExn[highlightlines=731]{}
      }
      \onslide*<4-5>{
        % TODO refactor with \listCamlReraiseExn
        \mintasm[firstline=727,lastline=734,highlightlines=730]{../ocaml/runtime/amd64.S}
        \medskip
      }
      \onslide*<4>{\listCamlRaiseExn[highlightlines=711]{}}
      \onslide*<5>{\listCamlRaiseExn[highlightlines=720]{}}
    \end{column}
  \end{columns}
\end{frame}

\begin{frame}{Backtraces}{Backtrace collection continues}
  \begin{columns}[c]
    \begin{column}{0.55\textwidth}
      \minipage[c][0.75\textheight][s]{\columnwidth}
      \begin{itemize}
        \item<1-> \funcname{caml\_stash\_backtrace} scans the older part of the stack, up to the next trap
        \item<6-> Controls jumps to the nested exception handler in \funcname{maybe\_raise}, which matches only on \typename{ExnB}
        \item<7-> \funcname{caml\_reraise\_exn} is called again, backtrace collection resumes with \funcname{caml\_stash\_backtrace}
      \end{itemize}
      \medskip
      \begin{itemize}
        \item<2-> Constructed backtrace:
        \begin{itemize}
          \item <2-> \funcname{definitely\_raise}
          \item <2-> \funcname{probably\_raise}
          \item <3-> \funcname{probably\_raise} (re-raise)
          \item <4-> \funcname{maybe\_raise}
          \item <5-> \funcname{main}
          \item <8-> \funcname{main} (re-raise)
        \end{itemize}
      \end{itemize}
      \vfill
      \onslide*<1-5>{\mintocaml[firstline=15,lastline=21]{../src/backtrace/backtrace.ml}}
      \onslide*<6>{\mintocaml[firstline=28,lastline=34,highlightlines=31]{../src/backtrace/backtrace.ml}}
      \onslide*<7->{\mintocaml[firstline=28,lastline=34,highlightlines=33]{../src/backtrace/backtrace.ml}}
      \endminipage
    \end{column}
    \begin{column}{0.45\textwidth}
      \raggedleft
      \onslide*<1>{\providecommand\step{6}\input{diagrams/backtrace/backtrace.tex}}%
      \onslide*<2>{\providecommand\step{6}\input{diagrams/backtrace/backtrace.tex}}%
      \onslide*<3>{\providecommand\step{7}\input{diagrams/backtrace/backtrace.tex}}%
      \onslide*<4>{\providecommand\step{8}\input{diagrams/backtrace/backtrace.tex}}%
      \onslide*<5>{\providecommand\step{9}\input{diagrams/backtrace/backtrace.tex}}%
      \onslide*<6>{\providecommand\step{10}\input{diagrams/backtrace/backtrace.tex}}%
      \onslide*<7>{\providecommand\step{11}\input{diagrams/backtrace/backtrace.tex}}%
      \onslide*<8>{\providecommand\step{12}\input{diagrams/backtrace/backtrace.tex}}%
      \onslide*<9>{\providecommand\step{13}\input{diagrams/backtrace/backtrace.tex}}%
    \end{column}
  \end{columns}
\end{frame}

\begin{frame}{Backtraces}{Printing the backtrace}
  \begin{columns}[c]
    \begin{column}{0.45\textwidth}
      \minipage[c][0.66\textheight][s]{\columnwidth}
      \begin{itemize}
        \item<1-> The exception backtrace can now be printed to \funcarg{stdout} with \funcname{Printexc.print_backtrace}
        \item<2-> First the exception backtrace is retrieved into an OCaml array with \funcname{get_raw_backtrace }
        \item<3-> Which is implemented as the \funcname{caml_get_exception_raw_backtrace} C function in the runtime
        \item<4-> And finally printed with \funcname{print_exception_backtrace}
      \end{itemize}
      \vfill
      \mintocaml[firstline=28,lastline=34,highlightlines=33]{../src/backtrace/backtrace.ml}
      \endminipage
    \end{column}
    \begin{column}{0.55\textwidth}
      \onslide*<1,2>{
        \mintocaml[firstline=100,lastline=101]{../ocaml/stdlib/printexc.ml}
      }
      \only<1>{
        \listocaml[firstline=170,lastline=172]{../ocaml/stdlib/printexc.ml}{stdlib/printexc.ml:170}
      }
      \only<2>{
        \listocaml[firstline=170,lastline=172,highlightlines=172]{../ocaml/stdlib/printexc.ml}{stdlib/printexc.ml:170}
      }
      \only<3>{
        \mintc[firstline=154,lastline=156]{../ocaml/runtime/backtrace.c}
        \listc[firstline=171,lastline=192,highlightlines={184-187}]{../ocaml/runtime/backtrace.c}{runtime/backtrace.c:154}
      }
      \only<4>{
        \listocaml[firstline=155,lastline=165]{../ocaml/stdlib/printexc.ml}{stdlib/printexc.ml:55}
      }
    \end{column}
  \end{columns}
\end{frame}


\subsection{The multiple kinds of raise}
\frameSubsection{}{}

\begin{frame}{Backtraces}{When backtraces are not needed}
  \begin{columns}[c]
    \begin{column}{0.45\textwidth}
      \minipage[c][0.66\textheight][s]{\columnwidth}
      \begin{itemize}
        \item<1-> What if the backtrace for an exception is never needed?
        \item<2-> It's possible to raise the exception explicitly with \funcname{raise\_notrace} to make raising as cheap as possible
        \item<3-> An example of use in the \funcname{dynlink} library of the OCaml compiler
      \end{itemize}
      \vfill
      \onslide<3->{
      \listocaml[firstline=205,lastline=209,highlightlines=207]{../ocaml/otherlibs/dynlink/dynlink\_compilerlibs/misc.ml}{otherlibs/dynlink/dynlink\_compilerlibs/misc.ml:205}
      }
      \endminipage
    \end{column}
    \begin{column}{0.55\textwidth}
    \only<4>{
      \mintcmm[highlightlines=3]{../src/notrace/camlNotrace\_\_fun_347.cmm}
      \listcmm{../src/notrace/camlNotrace\_\_all\_somes\_327.cmm}{all\_somes}
    }
    \onslide*<5->{
      \listobjdump[highlightlines={6-9}]{../src/notrace/camlNotrace\_\_fun_347.objdump}{anonymous function from \funcname{all\_somes}}
    }
    \end{column}
  \end{columns}
\end{frame}

\begin{frame}{Backtraces}{Summary table of the multiple kinds of raise}
  \begin{itemize}
    \item When debugging information is enabled and if the backtrace of an exception isn't needed, it's possible to raise with \funcname{raise\_notrace} for performance
    \item When debugging information is \emph{not} enabled, the compiler optimises \funcname{raise} calls to inline assembly (same effect as using \funcname{raise\_notrace})
  \end{itemize}
  \bigskip
  \centering
  \begin{tabular}{| l | c | c | c | c |}
    \hline
    \parbox{10em}{Debug information \\ (\funcarg{-g} flag)}  & \multicolumn{2}{|c|}{Disabled}                                     & \multicolumn{2}{|c|}{Enabled} \\
    \hline
    Raise kind                                               & \funcname{raise} & \funcname{raise\_notrace}                       & \funcname{raise} & \funcname{raise\_notrace} \\
    \hline
    Compilation                                              & \parbox{8em}{inline assembly\\ (optimization)} & inline assembly   & \funcname{caml\_raise\_exn} & inline assembly \\
    \hline
  \end{tabular}
\end{frame}


%
% Takeaway #5
%
\subsection*{Takeaway \#5}
\frameSubsectionTakeaway{}
\againframe<5>{takeaway}
}%
    \end{column}
  \end{columns}
\end{frame}

\newcommand\listStashBacktrace[1][]{
  \listc[firstline=91,lastline=120,#1]{../ocaml/runtime/backtrace\_nat.c}{runtime/backtrace\_nat.c:91}}

\begin{frame}{Backtraces}{Introducing \funcname{caml\_stash\_backtrace}}
  \begin{columns}[c]
    \begin{column}{0.5\textwidth}
      \minipage[c][0.66\textheight][s]{\columnwidth}
      \begin{itemize}
        \item<1-> A C function provided by the runtime (specific to native code)
        \item<2-> It scans the stack, between the stack pointer at the raising point up to the next trap, in order to collect the names of the detected function frames
          \onslide*<2>{
            \begin{itemize}
              \item And more information, like line number, column number...
            \end{itemize}
          }
        \item<3-> It uses the same technique and information as the Garbage Collector (GC) uses to scan the values live on the stack
          \onslide*<4>{
            \begin{itemize}
              \item \typename{frame\_descr} are at the core of stack unwinding
            \end{itemize}
          }
      \end{itemize}
      \vfill
      \mintocaml[firstline=9,lastline=13,highlightlines=12]{../src/backtrace/backtrace.ml}
      \endminipage
    \end{column}
    \begin{column}{0.5\textwidth}
      \onslide*<1>{\listStashBacktrace[highlightlines=91]{}}
      \onslide*<2>{\listStashBacktrace[highlightlines={91,117-118}]{}}
      \onslide*<3->{\listStashBacktrace[highlightlines={94,106,109-110}]{}}
    \end{column}
  \end{columns}
\end{frame}

\newcommand\listFrameDescr[1][]{
  \listocaml[firstline=111,lastline=124,#1]{../ocaml/asmcomp/emitaux.ml}{asmcomp/emitaux.ml:111}}
\newcommand\listDebugInfo[1][]{
  \listocaml[firstline=38,lastline=49,#1]{../ocaml/lambda/debuginfo.mli}{lambda/debuginfo.mli:38}}

\begin{frame}{Backtraces}{\typename{frame\_descr} and \typename{frame\_debuginfo} in a nutshell}
  \begin{columns}[c]
    \begin{column}{0.5\textwidth}
      \begin{itemize}
        \item<1-> For every return address in an OCaml program, there is a associated \typename{frame\_descr}
        \item<2-> A \typename{frame\_descr} specifies
        \begin{itemize}
          \item<2-> The size of the function stack frame
          \item<3-> Where live values are on the stack (so that the GC can mark them as live and prevent from being swept)
        \end{itemize}
        \item<4-> When compiled with debug information, \typename{frame\_descr} will be linked to a \typename{frame\_debuginfo}
        \begin{itemize}
          \item<5-> Contains information about the position in the source code (file name, line and column number)
        \end{itemize}
      \end{itemize}
    \end{column}
    \begin{column}{0.5\textwidth}
        \onslide*<1>{\listFrameDescr[highlightlines={118-122}]{}}
        \onslide*<2>{\listFrameDescr[highlightlines={120}]{}}
        \onslide*<3>{\listFrameDescr[highlightlines={121}]{}}
        \onslide*<4->{\listFrameDescr[highlightlines={122}]{}}
        \medskip
        \onslide*<1-3>{\listDebugInfo{}}
        \onslide*<4>{\listDebugInfo[highlightlines={38,49}]{}}
        \onslide*<5>{\listDebugInfo[highlightlines={39-46}]{}}
    \end{column}
  \end{columns}
\end{frame}

\begin{frame}{Backtraces}{Scanning the stack with \typename{frame\_descr}}
  \begin{columns}[c]
    \begin{column}{0.5\textwidth}
      \begin{itemize}
        \item Given the return address of the code that raises the exception by calling \funcname{caml\_raise\_exn} (\funcarg{pc} argument of \funcname{caml\_stash\_backtrace})
        \item And the position of the stack pointer (\funcarg{sp} argument of \funcname{caml\_stash\_backtrace})
        \item The frame size given by \typename{frame\_descr}.\funcarg{frame\_size}, allows to find the end of the previous frame
        \item And the return address of the caller of this function
        \bigskip
        \onslide<3->{
        \item Note that traps (here the reddish \funcname{ExnA}, \funcname{ExnB} and \funcname{ExnC} blocks) belong to the frame that installed them, and so the frame size changes when traps are removed
        }
      \end{itemize}
    \end{column}
    \begin{column}{0.5\textwidth}
      \raggedleft
      \onslide*<1>{\providecommand\step{2}%
% Backtraces
%
\subsection{One advantage of raising with debugging information}
\frameSubsection{}{}

\begin{frame}{Backtraces}{Debugging information \& source code}
  \begin{columns}[c]
    \begin{column}{0.5\textwidth}
      \begin{itemize}
        \item Raising an exception is fun and games, but how do we know where the exception originated? $\rightarrow$ With the backtrace associated with the exception!
        \item In order to get a backtrace from the point where an exception was raised, it's necessary to compile with debugging information enabled (\funcarg{-g} flag for the compiler)
        \item Same example as in the `Nested handlers' section
      \end{itemize}
    \end{column}
    \begin{column}{0.5\textwidth}
      \listocaml[firstline=9,lastline=34]{../src/backtrace/backtrace.ml}{backtrace/backtrace.ml}
    \end{column}
  \end{columns}
\end{frame}

\begin{frame}{Backtraces}{CMM and disassembly of \funcname{definitely\_raise}}
  \begin{columns}[c]
    \begin{column}{0.5\textwidth}
      \minipage[c][0.66\textheight][s]{\columnwidth}
      \begin{itemize}
        \item<1-> An effect of compiling with debugging information (\funcarg{-g}): the CMM no longer uses \funcname{raise\_notrace} to raise the exception but \funcname{raise} instead
        \item<2-> Disassembly shows that CMM \funcname{raise} is actually a call to \funcname{caml\_raise\_exn}, a function in assembler provided by the runtime
      \end{itemize}
      \vfill
      \mintocaml[firstline=9,lastline=13,highlightlines=12]{../src/backtrace/backtrace.ml}
      \endminipage
    \end{column}
    \begin{column}{0.5\textwidth}
      \onslide*<1>{
        \mintcmm[highlightlines=5]{../src/backtrace/camlBacktrace\_\_definitely\_raise\_347.cmm}
      }
      \onslide*<2>{
        \mintobjdump[firstline=2,highlightlines=14]{../src/backtrace/camlBacktrace\_\_definitely\_raise\_347.objdump}
      }
    \end{column}
  \end{columns}
\end{frame}

\begin{frame}{Backtraces}{Introducing \funcname{caml\_raise\_exn}}
  \begin{columns}[c]
    \begin{column}{0.5\textwidth}
      \minipage[c][0.66\textheight][s]{\columnwidth}
      \onslide*<1>{
        \begin{itemize}
          \item Raising the exception is performed using \funcname{caml\_raise\_exn} which is
            \begin{itemize}
              \item An assembly function provided by the runtime
              \item Architecture specific
            \end{itemize}
          \item Let's see what it does differently from \funcname{raise\_notrace}\dots
          \item We are about to raise \typename{ExnC} with \funcname{caml\_raise\_exn} from \funcname{definitely\_raise}
        \end{itemize}
      }
      \onslide*<2>{
        \mintobjdump[firstline=2,highlightlines=14]{../src/backtrace/camlBacktrace\_\_definitely\_raise\_347.objdump}
      }
      \vfill
      \mintocaml[firstline=9,lastline=13,highlightlines=12]{../src/backtrace/backtrace.ml}
      \endminipage
    \end{column}
    \begin{column}{0.5\textwidth}
      \centering
      \providecommand\step{1}\input{diagrams/backtrace/backtrace.tex}
    \end{column}
  \end{columns}
\end{frame}

\begin{frame}{Backtraces}{But before that, we need more fields from \typename{caml\_domain\_state}}
  \begin{columns}
    \begin{column}{0.5\textwidth}
      \begin{itemize}
        \item Remember \typename{caml\_domain\_state} the per-domain runtime state?
        \item Some other members are of interest to us now\dots
        \item \localname{backtrace\_active} is used to know if the runtime should collect backtraces
        \item \localname{backtrace\_active} can be set by
          \begin{itemize}
            \item The \funcarg{b} flag in the \funcname{OCAMLRUNPARAM} environment variable
            \item \mintinline{ocaml}{Printexc.record_backtrace : bool -> unit} \footnotemark
          \end{itemize}
      \end{itemize}
    \end{column}
    \begin{column}{0.5\textwidth}
      \begin{listing}
        \mintc[highlightlines={9-11}]{src/ocaml/domain\_state\_backtrace.h}
        \caption{runtime/caml/domain\_state.h}
      \end{listing}
    \end{column}
  \end{columns}
  \footnotetext[1]{https://v2.ocaml.org/api/Printexc.html}
\end{frame}

\newcommand\listCamlRaiseExn[1][]{
  \listasm[firstline=702,lastline=725,#1]{../ocaml/runtime/amd64.S}{runtime/amd64.S:702}}

\begin{frame}{Backtraces}{Walk through \funcname{caml\_raise\_exn}}
  \begin{columns}[T]
    \begin{column}{0.5\textwidth}
      \begin{itemize}
        \item<1-> First checks whether exceptions backtrace should be recorded
        \item<1-> By checking the \funcarg{domain\_state->backtrace\_active} value
        \item<2-> If not, acts as \funcname{raise\_notrace} with \funcname{RESTORE\_EXN\_HANDLER\_OCAML}
          \begin{itemize}
            \item Set \regname{RSP} to \localname{domain\_state->exn\_handler} value
            \item Restore \localname{domain\_state->exn\_handler} from trap
            \item Jump with \funcname{ret} (same effect as using \funcname{pop} and \funcname{jmp})
          \end{itemize}
        \item<3-> Otherwise, switches to the C stack to be able to execute C code
        \item<4-> Calls \funcname{caml\_stash\_backtrace}, a C function provided by the runtime\ignorespaces
          \onslide<6->{, with
            \begin{itemize}
              \item \funcarg{exn}: exception value
              \item \funcarg{pc}: code address of the raising point
              \item \funcarg{sp}: stack address of the raising function
              \item \funcarg{trapsp}: stack address of the next handler
            \end{itemize}
          }
      \end{itemize}
      \bigskip
      \mintocaml[firstline=9,lastline=13,highlightlines=12]{../src/backtrace/backtrace.ml}
    \end{column}
    \begin{column}{0.5\textwidth}
      \raggedleft
      \onslide*<1,3-4>{
        \mintc[firstline=190,lastline=192]{../ocaml/runtime/amd64.S}
        \mintc[firstline=234,lastline=237]{../ocaml/runtime/amd64.S}
      }
      \onslide*<2>{
        \mintc[firstline=190,lastline=192,highlightlines={191-192}]{../ocaml/runtime/amd64.S}
        \mintc[firstline=234,lastline=237,highlightlines={235-237}]{../ocaml/runtime/amd64.S}
      }
      \onslide*<1>{\listCamlRaiseExn[highlightlines=705]{}}%
      \onslide*<2>{\listCamlRaiseExn[highlightlines={707-708}]{}}%
      \onslide*<3>{\listCamlRaiseExn[highlightlines={712-714}]{}}%
      \onslide*<4>{\listCamlRaiseExn[highlightlines={715-720}]{}}%
      \onslide*<5>{\providecommand\step{1}\input{diagrams/backtrace/backtrace.tex}}%
      \onslide*<6>{\providecommand\step{2}\input{diagrams/backtrace/backtrace.tex}}%
    \end{column}
  \end{columns}
\end{frame}

\newcommand\listStashBacktrace[1][]{
  \listc[firstline=91,lastline=120,#1]{../ocaml/runtime/backtrace\_nat.c}{runtime/backtrace\_nat.c:91}}

\begin{frame}{Backtraces}{Introducing \funcname{caml\_stash\_backtrace}}
  \begin{columns}[c]
    \begin{column}{0.5\textwidth}
      \minipage[c][0.66\textheight][s]{\columnwidth}
      \begin{itemize}
        \item<1-> A C function provided by the runtime (specific to native code)
        \item<2-> It scans the stack, between the stack pointer at the raising point up to the next trap, in order to collect the names of the detected function frames
          \onslide*<2>{
            \begin{itemize}
              \item And more information, like line number, column number...
            \end{itemize}
          }
        \item<3-> It uses the same technique and information as the Garbage Collector (GC) uses to scan the values live on the stack
          \onslide*<4>{
            \begin{itemize}
              \item \typename{frame\_descr} are at the core of stack unwinding
            \end{itemize}
          }
      \end{itemize}
      \vfill
      \mintocaml[firstline=9,lastline=13,highlightlines=12]{../src/backtrace/backtrace.ml}
      \endminipage
    \end{column}
    \begin{column}{0.5\textwidth}
      \onslide*<1>{\listStashBacktrace[highlightlines=91]{}}
      \onslide*<2>{\listStashBacktrace[highlightlines={91,117-118}]{}}
      \onslide*<3->{\listStashBacktrace[highlightlines={94,106,109-110}]{}}
    \end{column}
  \end{columns}
\end{frame}

\newcommand\listFrameDescr[1][]{
  \listocaml[firstline=111,lastline=124,#1]{../ocaml/asmcomp/emitaux.ml}{asmcomp/emitaux.ml:111}}
\newcommand\listDebugInfo[1][]{
  \listocaml[firstline=38,lastline=49,#1]{../ocaml/lambda/debuginfo.mli}{lambda/debuginfo.mli:38}}

\begin{frame}{Backtraces}{\typename{frame\_descr} and \typename{frame\_debuginfo} in a nutshell}
  \begin{columns}[c]
    \begin{column}{0.5\textwidth}
      \begin{itemize}
        \item<1-> For every return address in an OCaml program, there is a associated \typename{frame\_descr}
        \item<2-> A \typename{frame\_descr} specifies
        \begin{itemize}
          \item<2-> The size of the function stack frame
          \item<3-> Where live values are on the stack (so that the GC can mark them as live and prevent from being swept)
        \end{itemize}
        \item<4-> When compiled with debug information, \typename{frame\_descr} will be linked to a \typename{frame\_debuginfo}
        \begin{itemize}
          \item<5-> Contains information about the position in the source code (file name, line and column number)
        \end{itemize}
      \end{itemize}
    \end{column}
    \begin{column}{0.5\textwidth}
        \onslide*<1>{\listFrameDescr[highlightlines={118-122}]{}}
        \onslide*<2>{\listFrameDescr[highlightlines={120}]{}}
        \onslide*<3>{\listFrameDescr[highlightlines={121}]{}}
        \onslide*<4->{\listFrameDescr[highlightlines={122}]{}}
        \medskip
        \onslide*<1-3>{\listDebugInfo{}}
        \onslide*<4>{\listDebugInfo[highlightlines={38,49}]{}}
        \onslide*<5>{\listDebugInfo[highlightlines={39-46}]{}}
    \end{column}
  \end{columns}
\end{frame}

\begin{frame}{Backtraces}{Scanning the stack with \typename{frame\_descr}}
  \begin{columns}[c]
    \begin{column}{0.5\textwidth}
      \begin{itemize}
        \item Given the return address of the code that raises the exception by calling \funcname{caml\_raise\_exn} (\funcarg{pc} argument of \funcname{caml\_stash\_backtrace})
        \item And the position of the stack pointer (\funcarg{sp} argument of \funcname{caml\_stash\_backtrace})
        \item The frame size given by \typename{frame\_descr}.\funcarg{frame\_size}, allows to find the end of the previous frame
        \item And the return address of the caller of this function
        \bigskip
        \onslide<3->{
        \item Note that traps (here the reddish \funcname{ExnA}, \funcname{ExnB} and \funcname{ExnC} blocks) belong to the frame that installed them, and so the frame size changes when traps are removed
        }
      \end{itemize}
    \end{column}
    \begin{column}{0.5\textwidth}
      \raggedleft
      \onslide*<1>{\providecommand\step{2}\input{diagrams/backtrace/backtrace.tex}}%
      \onslide*<2>{\providecommand\step{1}\input{diagrams/backtrace/scanstack.tex}}%
      \onslide*<3>{\providecommand\step{2}\input{diagrams/backtrace/scanstack.tex}}%
      \onslide*<4>{\providecommand\step{3}\input{diagrams/backtrace/scanstack.tex}}%
      \onslide*<5>{\providecommand\step{4}\input{diagrams/backtrace/scanstack.tex}}%
      \onslide*<6>{\providecommand\step{5}\input{diagrams/backtrace/scanstack.tex}}%
    \end{column}
  \end{columns}
\end{frame}

% Duplicate of previous-previous slide
\begin{frame}{Backtraces}{Continuing on \funcname{caml\_stash\_backtrace}}
  \begin{columns}[c]
    \begin{column}{0.5\textwidth}
      \minipage[c][0.75\textheight][s]{\columnwidth}
      \begin{itemize}
          \color{gray}
        \item A C function provided by the runtime (specific to native code)
        \item It scans the stack, between the stack pointer at the raising point up to the next trap, in order to collect the names of the detected function frames
        \item It uses the same technique and information as the Garbage Collector (GC) uses to scan the values live on the stack
      \end{itemize}
      \begin{itemize}
        \item For each function frame detected, store a pointer to the \typename{frame\_descr} in the \localname{domain\_state->backtrace\_buffer} array, effectively constructing the backtrace
      \end{itemize}
      \onslide<2->{
        \begin{itemize}
            \smallskip
          \item Constructed backtrace:
            \funcname{definitely\_raise},
            \onslide<3->{\funcname{probably\_raise}}
        \end{itemize}
      }
      \vfill
      \mintocaml[firstline=9,lastline=13,highlightlines=12]{../src/backtrace/backtrace.ml}
      \endminipage
    \end{column}
    \begin{column}{0.5\textwidth}
      \raggedleft
      \onslide*<1>{\listStashBacktrace[highlightlines={114-115}]{}}
      \onslide*<2>{\providecommand\step{3}\input{diagrams/backtrace/backtrace.tex}}%
      \onslide*<3>{\providecommand\step{4}\input{diagrams/backtrace/backtrace.tex}}%
    \end{column}
  \end{columns}
\end{frame}

\begin{frame}{Backtraces}{Back to \funcname{caml\_raise\_exn}}
  \begin{columns}[c]
    \begin{column}{0.5\textwidth}
      \minipage[c][0.66\textheight][s]{\columnwidth}
      \begin{itemize}\color{gray}
        \item Checks wheteher exceptions backtrace should be recorded, by checking the \funcarg{domain\_state->backtrace\_active} value
        \item Switches to the C stack to be able to execute C code
        \item Calls \funcname{caml\_stash\_backtrace}, to collect the backtrace up to the next trap
      \end{itemize}
      \onslide*<2->{
        \begin{itemize}
          \item<2-> Executes the exception handler in \funcname{probably\_raise} with \funcname{RESTORE\_EXN\_HANDLER\_OCAML}
            \begin{itemize}
              \item Set \regname{RSP} to \localname{domain\_state->exn\_handler} value
              \item Restore \localname{domain\_state->exn\_handler} from trap
              \item Jump to the handler code with \funcname{ret}
            \end{itemize}
        \end{itemize}
      }
      \vfill
      \onslide*<1>{\mintocaml[firstline=9,lastline=13,highlightlines=12]{../src/backtrace/backtrace.ml}}
      \onslide*<2->{\mintocaml[firstline=15,lastline=21,highlightlines={19-21}]{../src/backtrace/backtrace.ml}}
      \endminipage
    \end{column}
    \begin{column}{0.5\textwidth}
      \centering
      \onslide*<1>{
        \mintc[firstline=190,lastline=192]{../ocaml/runtime/amd64.S}
        \mintc[firstline=234,lastline=237]{../ocaml/runtime/amd64.S}
      }
      \onslide*<2>{
        \mintc[firstline=190,lastline=192,highlightlines={191-192}]{../ocaml/runtime/amd64.S}
        \mintc[firstline=234,lastline=237,highlightlines={235-237}]{../ocaml/runtime/amd64.S}
      }
      \onslide*<1>{\listCamlRaiseExn[highlightlines=720]{}}
      \onslide*<2>{\listCamlRaiseExn[highlightlines=722]{}}
      \onslide*<3->{\providecommand\step{5}\input{diagrams/backtrace/backtrace.tex}}
    \end{column}
  \end{columns}
\end{frame}

\begin{frame}{Backtraces}{\funcname{caml\_probably\_raise}'s handler doesn't catch \typename{ExnC}}
  \begin{columns}[c]
    \begin{column}{0.45\textwidth}
      \minipage[c][0.75\textheight][s]{\columnwidth}
      \begin{itemize}
        \item<1-> \funcname{match \dots with}\footnotemark pattern matching can also match exceptions
        \item<1-> The CMM of \funcname{probably\_raise} shows that this \funcname{match \dots with} is implemented with a pattern matching and a \funcname{try \dots with}
        \item<1-> But this one only matches on \typename{ExnA} exception
        \item<2-> Since the pattern matching fails to catch the exception, \funcname{reraise} is called with our current \typename{ExnC} exception
        \item<3-> Disassembly shows that \funcname{reraise} is actually a call to \funcname{caml\_reraise\_exn}, which is also an assembly function provided by the runtime
      \end{itemize}
      \vfill
      \mintocaml[firstline=15,lastline=21,highlightlines={18-21}]{../src/backtrace/backtrace.ml}
      \endminipage
    \end{column}
    \begin{column}{0.55\textwidth}
      \centering
      \onslide*<1>{\mintcmm[highlightlines={10-18}]{../src/backtrace/camlBacktrace\_\_probably\_raise\_351.cmm}}
      \onslide*<2>{\listcmm[highlightlines=18]{../src/backtrace/camlBacktrace\_\_probably\_raise_351.cmm}{CMM of probably\_raise}}
      \onslide*<3>{\mintobjdump[firstline=5,lastline=35,highlightlines=31]{../src/backtrace/camlBacktrace\_\_probably\_raise_351.objdump}}
    \end{column}
  \end{columns}
  \only<1>{\footnotetext[1]{https://blog.janestreet.com/pattern-matching-and-exception-handling-unite/}}
\end{frame}

\newcommand\listCamlReraiseExn[1][]{
  \listasm[firstline=727,lastline=734,#1]{../ocaml/runtime/amd64.S}{runtime/amd64.S:727}}

\begin{frame}{Backtraces}{Introducing \funcname{caml\_reraise\_exn}}
  \begin{columns}[c]
    \begin{column}{0.5\textwidth}
      \minipage[c][0.66\textheight][s]{\columnwidth}
      \begin{itemize}
        \item<1-> The exception have to be reraised to the next handler
        \item<2-> First, checks if the backtrace should be collected with \funcarg{domain\_state->backtrace\_active}
        \only<3>{
        \begin{itemize}
            \item If not, behaves like a \funcname{raise\_notrace} with \funcname{RESTORE\_EXN\_HANDLER\_OCAML}
        \end{itemize}
        }
        \item<4-> Jumps into \funcname{caml\_raise\_exn}
        \item<5-> Calls \funcname{caml\_stash\_backtrace} again, to scan the next part of the stack: from the trap just pop-ed to the next trap
      \end{itemize}
      \vfill
      \mintocaml[firstline=15,lastline=21,highlightlines={18-21}]{../src/backtrace/backtrace.ml}
      \endminipage
    \end{column}
    \begin{column}{0.5\textwidth}
      \raggedleft
      \onslide*<1>{\listCamlReraiseExn{}}
      \onslide*<2>{\listCamlReraiseExn[highlightlines=729]{}}
      \onslide*<3>{
        \mintc[firstline=190,lastline=192,highlightlines={191-192}]{../ocaml/runtime/amd64.S}
        \mintc[firstline=234,lastline=237,highlightlines={235-237}]{../ocaml/runtime/amd64.S}
        \medskip
        \listCamlReraiseExn[highlightlines=731]{}
      }
      \onslide*<4-5>{
        % TODO refactor with \listCamlReraiseExn
        \mintasm[firstline=727,lastline=734,highlightlines=730]{../ocaml/runtime/amd64.S}
        \medskip
      }
      \onslide*<4>{\listCamlRaiseExn[highlightlines=711]{}}
      \onslide*<5>{\listCamlRaiseExn[highlightlines=720]{}}
    \end{column}
  \end{columns}
\end{frame}

\begin{frame}{Backtraces}{Backtrace collection continues}
  \begin{columns}[c]
    \begin{column}{0.55\textwidth}
      \minipage[c][0.75\textheight][s]{\columnwidth}
      \begin{itemize}
        \item<1-> \funcname{caml\_stash\_backtrace} scans the older part of the stack, up to the next trap
        \item<6-> Controls jumps to the nested exception handler in \funcname{maybe\_raise}, which matches only on \typename{ExnB}
        \item<7-> \funcname{caml\_reraise\_exn} is called again, backtrace collection resumes with \funcname{caml\_stash\_backtrace}
      \end{itemize}
      \medskip
      \begin{itemize}
        \item<2-> Constructed backtrace:
        \begin{itemize}
          \item <2-> \funcname{definitely\_raise}
          \item <2-> \funcname{probably\_raise}
          \item <3-> \funcname{probably\_raise} (re-raise)
          \item <4-> \funcname{maybe\_raise}
          \item <5-> \funcname{main}
          \item <8-> \funcname{main} (re-raise)
        \end{itemize}
      \end{itemize}
      \vfill
      \onslide*<1-5>{\mintocaml[firstline=15,lastline=21]{../src/backtrace/backtrace.ml}}
      \onslide*<6>{\mintocaml[firstline=28,lastline=34,highlightlines=31]{../src/backtrace/backtrace.ml}}
      \onslide*<7->{\mintocaml[firstline=28,lastline=34,highlightlines=33]{../src/backtrace/backtrace.ml}}
      \endminipage
    \end{column}
    \begin{column}{0.45\textwidth}
      \raggedleft
      \onslide*<1>{\providecommand\step{6}\input{diagrams/backtrace/backtrace.tex}}%
      \onslide*<2>{\providecommand\step{6}\input{diagrams/backtrace/backtrace.tex}}%
      \onslide*<3>{\providecommand\step{7}\input{diagrams/backtrace/backtrace.tex}}%
      \onslide*<4>{\providecommand\step{8}\input{diagrams/backtrace/backtrace.tex}}%
      \onslide*<5>{\providecommand\step{9}\input{diagrams/backtrace/backtrace.tex}}%
      \onslide*<6>{\providecommand\step{10}\input{diagrams/backtrace/backtrace.tex}}%
      \onslide*<7>{\providecommand\step{11}\input{diagrams/backtrace/backtrace.tex}}%
      \onslide*<8>{\providecommand\step{12}\input{diagrams/backtrace/backtrace.tex}}%
      \onslide*<9>{\providecommand\step{13}\input{diagrams/backtrace/backtrace.tex}}%
    \end{column}
  \end{columns}
\end{frame}

\begin{frame}{Backtraces}{Printing the backtrace}
  \begin{columns}[c]
    \begin{column}{0.45\textwidth}
      \minipage[c][0.66\textheight][s]{\columnwidth}
      \begin{itemize}
        \item<1-> The exception backtrace can now be printed to \funcarg{stdout} with \funcname{Printexc.print_backtrace}
        \item<2-> First the exception backtrace is retrieved into an OCaml array with \funcname{get_raw_backtrace }
        \item<3-> Which is implemented as the \funcname{caml_get_exception_raw_backtrace} C function in the runtime
        \item<4-> And finally printed with \funcname{print_exception_backtrace}
      \end{itemize}
      \vfill
      \mintocaml[firstline=28,lastline=34,highlightlines=33]{../src/backtrace/backtrace.ml}
      \endminipage
    \end{column}
    \begin{column}{0.55\textwidth}
      \onslide*<1,2>{
        \mintocaml[firstline=100,lastline=101]{../ocaml/stdlib/printexc.ml}
      }
      \only<1>{
        \listocaml[firstline=170,lastline=172]{../ocaml/stdlib/printexc.ml}{stdlib/printexc.ml:170}
      }
      \only<2>{
        \listocaml[firstline=170,lastline=172,highlightlines=172]{../ocaml/stdlib/printexc.ml}{stdlib/printexc.ml:170}
      }
      \only<3>{
        \mintc[firstline=154,lastline=156]{../ocaml/runtime/backtrace.c}
        \listc[firstline=171,lastline=192,highlightlines={184-187}]{../ocaml/runtime/backtrace.c}{runtime/backtrace.c:154}
      }
      \only<4>{
        \listocaml[firstline=155,lastline=165]{../ocaml/stdlib/printexc.ml}{stdlib/printexc.ml:55}
      }
    \end{column}
  \end{columns}
\end{frame}


\subsection{The multiple kinds of raise}
\frameSubsection{}{}

\begin{frame}{Backtraces}{When backtraces are not needed}
  \begin{columns}[c]
    \begin{column}{0.45\textwidth}
      \minipage[c][0.66\textheight][s]{\columnwidth}
      \begin{itemize}
        \item<1-> What if the backtrace for an exception is never needed?
        \item<2-> It's possible to raise the exception explicitly with \funcname{raise\_notrace} to make raising as cheap as possible
        \item<3-> An example of use in the \funcname{dynlink} library of the OCaml compiler
      \end{itemize}
      \vfill
      \onslide<3->{
      \listocaml[firstline=205,lastline=209,highlightlines=207]{../ocaml/otherlibs/dynlink/dynlink\_compilerlibs/misc.ml}{otherlibs/dynlink/dynlink\_compilerlibs/misc.ml:205}
      }
      \endminipage
    \end{column}
    \begin{column}{0.55\textwidth}
    \only<4>{
      \mintcmm[highlightlines=3]{../src/notrace/camlNotrace\_\_fun_347.cmm}
      \listcmm{../src/notrace/camlNotrace\_\_all\_somes\_327.cmm}{all\_somes}
    }
    \onslide*<5->{
      \listobjdump[highlightlines={6-9}]{../src/notrace/camlNotrace\_\_fun_347.objdump}{anonymous function from \funcname{all\_somes}}
    }
    \end{column}
  \end{columns}
\end{frame}

\begin{frame}{Backtraces}{Summary table of the multiple kinds of raise}
  \begin{itemize}
    \item When debugging information is enabled and if the backtrace of an exception isn't needed, it's possible to raise with \funcname{raise\_notrace} for performance
    \item When debugging information is \emph{not} enabled, the compiler optimises \funcname{raise} calls to inline assembly (same effect as using \funcname{raise\_notrace})
  \end{itemize}
  \bigskip
  \centering
  \begin{tabular}{| l | c | c | c | c |}
    \hline
    \parbox{10em}{Debug information \\ (\funcarg{-g} flag)}  & \multicolumn{2}{|c|}{Disabled}                                     & \multicolumn{2}{|c|}{Enabled} \\
    \hline
    Raise kind                                               & \funcname{raise} & \funcname{raise\_notrace}                       & \funcname{raise} & \funcname{raise\_notrace} \\
    \hline
    Compilation                                              & \parbox{8em}{inline assembly\\ (optimization)} & inline assembly   & \funcname{caml\_raise\_exn} & inline assembly \\
    \hline
  \end{tabular}
\end{frame}


%
% Takeaway #5
%
\subsection*{Takeaway \#5}
\frameSubsectionTakeaway{}
\againframe<5>{takeaway}
}%
      \onslide*<2>{\providecommand\step{1}\input{diagrams/backtrace/scanstack.tex}}%
      \onslide*<3>{\providecommand\step{2}\input{diagrams/backtrace/scanstack.tex}}%
      \onslide*<4>{\providecommand\step{3}\input{diagrams/backtrace/scanstack.tex}}%
      \onslide*<5>{\providecommand\step{4}\input{diagrams/backtrace/scanstack.tex}}%
      \onslide*<6>{\providecommand\step{5}\input{diagrams/backtrace/scanstack.tex}}%
    \end{column}
  \end{columns}
\end{frame}

% Duplicate of previous-previous slide
\begin{frame}{Backtraces}{Continuing on \funcname{caml\_stash\_backtrace}}
  \begin{columns}[c]
    \begin{column}{0.5\textwidth}
      \minipage[c][0.75\textheight][s]{\columnwidth}
      \begin{itemize}
          \color{gray}
        \item A C function provided by the runtime (specific to native code)
        \item It scans the stack, between the stack pointer at the raising point up to the next trap, in order to collect the names of the detected function frames
        \item It uses the same technique and information as the Garbage Collector (GC) uses to scan the values live on the stack
      \end{itemize}
      \begin{itemize}
        \item For each function frame detected, store a pointer to the \typename{frame\_descr} in the \localname{domain\_state->backtrace\_buffer} array, effectively constructing the backtrace
      \end{itemize}
      \onslide<2->{
        \begin{itemize}
            \smallskip
          \item Constructed backtrace:
            \funcname{definitely\_raise},
            \onslide<3->{\funcname{probably\_raise}}
        \end{itemize}
      }
      \vfill
      \mintocaml[firstline=9,lastline=13,highlightlines=12]{../src/backtrace/backtrace.ml}
      \endminipage
    \end{column}
    \begin{column}{0.5\textwidth}
      \raggedleft
      \onslide*<1>{\listStashBacktrace[highlightlines={114-115}]{}}
      \onslide*<2>{\providecommand\step{3}%
% Backtraces
%
\subsection{One advantage of raising with debugging information}
\frameSubsection{}{}

\begin{frame}{Backtraces}{Debugging information \& source code}
  \begin{columns}[c]
    \begin{column}{0.5\textwidth}
      \begin{itemize}
        \item Raising an exception is fun and games, but how do we know where the exception originated? $\rightarrow$ With the backtrace associated with the exception!
        \item In order to get a backtrace from the point where an exception was raised, it's necessary to compile with debugging information enabled (\funcarg{-g} flag for the compiler)
        \item Same example as in the `Nested handlers' section
      \end{itemize}
    \end{column}
    \begin{column}{0.5\textwidth}
      \listocaml[firstline=9,lastline=34]{../src/backtrace/backtrace.ml}{backtrace/backtrace.ml}
    \end{column}
  \end{columns}
\end{frame}

\begin{frame}{Backtraces}{CMM and disassembly of \funcname{definitely\_raise}}
  \begin{columns}[c]
    \begin{column}{0.5\textwidth}
      \minipage[c][0.66\textheight][s]{\columnwidth}
      \begin{itemize}
        \item<1-> An effect of compiling with debugging information (\funcarg{-g}): the CMM no longer uses \funcname{raise\_notrace} to raise the exception but \funcname{raise} instead
        \item<2-> Disassembly shows that CMM \funcname{raise} is actually a call to \funcname{caml\_raise\_exn}, a function in assembler provided by the runtime
      \end{itemize}
      \vfill
      \mintocaml[firstline=9,lastline=13,highlightlines=12]{../src/backtrace/backtrace.ml}
      \endminipage
    \end{column}
    \begin{column}{0.5\textwidth}
      \onslide*<1>{
        \mintcmm[highlightlines=5]{../src/backtrace/camlBacktrace\_\_definitely\_raise\_347.cmm}
      }
      \onslide*<2>{
        \mintobjdump[firstline=2,highlightlines=14]{../src/backtrace/camlBacktrace\_\_definitely\_raise\_347.objdump}
      }
    \end{column}
  \end{columns}
\end{frame}

\begin{frame}{Backtraces}{Introducing \funcname{caml\_raise\_exn}}
  \begin{columns}[c]
    \begin{column}{0.5\textwidth}
      \minipage[c][0.66\textheight][s]{\columnwidth}
      \onslide*<1>{
        \begin{itemize}
          \item Raising the exception is performed using \funcname{caml\_raise\_exn} which is
            \begin{itemize}
              \item An assembly function provided by the runtime
              \item Architecture specific
            \end{itemize}
          \item Let's see what it does differently from \funcname{raise\_notrace}\dots
          \item We are about to raise \typename{ExnC} with \funcname{caml\_raise\_exn} from \funcname{definitely\_raise}
        \end{itemize}
      }
      \onslide*<2>{
        \mintobjdump[firstline=2,highlightlines=14]{../src/backtrace/camlBacktrace\_\_definitely\_raise\_347.objdump}
      }
      \vfill
      \mintocaml[firstline=9,lastline=13,highlightlines=12]{../src/backtrace/backtrace.ml}
      \endminipage
    \end{column}
    \begin{column}{0.5\textwidth}
      \centering
      \providecommand\step{1}\input{diagrams/backtrace/backtrace.tex}
    \end{column}
  \end{columns}
\end{frame}

\begin{frame}{Backtraces}{But before that, we need more fields from \typename{caml\_domain\_state}}
  \begin{columns}
    \begin{column}{0.5\textwidth}
      \begin{itemize}
        \item Remember \typename{caml\_domain\_state} the per-domain runtime state?
        \item Some other members are of interest to us now\dots
        \item \localname{backtrace\_active} is used to know if the runtime should collect backtraces
        \item \localname{backtrace\_active} can be set by
          \begin{itemize}
            \item The \funcarg{b} flag in the \funcname{OCAMLRUNPARAM} environment variable
            \item \mintinline{ocaml}{Printexc.record_backtrace : bool -> unit} \footnotemark
          \end{itemize}
      \end{itemize}
    \end{column}
    \begin{column}{0.5\textwidth}
      \begin{listing}
        \mintc[highlightlines={9-11}]{src/ocaml/domain\_state\_backtrace.h}
        \caption{runtime/caml/domain\_state.h}
      \end{listing}
    \end{column}
  \end{columns}
  \footnotetext[1]{https://v2.ocaml.org/api/Printexc.html}
\end{frame}

\newcommand\listCamlRaiseExn[1][]{
  \listasm[firstline=702,lastline=725,#1]{../ocaml/runtime/amd64.S}{runtime/amd64.S:702}}

\begin{frame}{Backtraces}{Walk through \funcname{caml\_raise\_exn}}
  \begin{columns}[T]
    \begin{column}{0.5\textwidth}
      \begin{itemize}
        \item<1-> First checks whether exceptions backtrace should be recorded
        \item<1-> By checking the \funcarg{domain\_state->backtrace\_active} value
        \item<2-> If not, acts as \funcname{raise\_notrace} with \funcname{RESTORE\_EXN\_HANDLER\_OCAML}
          \begin{itemize}
            \item Set \regname{RSP} to \localname{domain\_state->exn\_handler} value
            \item Restore \localname{domain\_state->exn\_handler} from trap
            \item Jump with \funcname{ret} (same effect as using \funcname{pop} and \funcname{jmp})
          \end{itemize}
        \item<3-> Otherwise, switches to the C stack to be able to execute C code
        \item<4-> Calls \funcname{caml\_stash\_backtrace}, a C function provided by the runtime\ignorespaces
          \onslide<6->{, with
            \begin{itemize}
              \item \funcarg{exn}: exception value
              \item \funcarg{pc}: code address of the raising point
              \item \funcarg{sp}: stack address of the raising function
              \item \funcarg{trapsp}: stack address of the next handler
            \end{itemize}
          }
      \end{itemize}
      \bigskip
      \mintocaml[firstline=9,lastline=13,highlightlines=12]{../src/backtrace/backtrace.ml}
    \end{column}
    \begin{column}{0.5\textwidth}
      \raggedleft
      \onslide*<1,3-4>{
        \mintc[firstline=190,lastline=192]{../ocaml/runtime/amd64.S}
        \mintc[firstline=234,lastline=237]{../ocaml/runtime/amd64.S}
      }
      \onslide*<2>{
        \mintc[firstline=190,lastline=192,highlightlines={191-192}]{../ocaml/runtime/amd64.S}
        \mintc[firstline=234,lastline=237,highlightlines={235-237}]{../ocaml/runtime/amd64.S}
      }
      \onslide*<1>{\listCamlRaiseExn[highlightlines=705]{}}%
      \onslide*<2>{\listCamlRaiseExn[highlightlines={707-708}]{}}%
      \onslide*<3>{\listCamlRaiseExn[highlightlines={712-714}]{}}%
      \onslide*<4>{\listCamlRaiseExn[highlightlines={715-720}]{}}%
      \onslide*<5>{\providecommand\step{1}\input{diagrams/backtrace/backtrace.tex}}%
      \onslide*<6>{\providecommand\step{2}\input{diagrams/backtrace/backtrace.tex}}%
    \end{column}
  \end{columns}
\end{frame}

\newcommand\listStashBacktrace[1][]{
  \listc[firstline=91,lastline=120,#1]{../ocaml/runtime/backtrace\_nat.c}{runtime/backtrace\_nat.c:91}}

\begin{frame}{Backtraces}{Introducing \funcname{caml\_stash\_backtrace}}
  \begin{columns}[c]
    \begin{column}{0.5\textwidth}
      \minipage[c][0.66\textheight][s]{\columnwidth}
      \begin{itemize}
        \item<1-> A C function provided by the runtime (specific to native code)
        \item<2-> It scans the stack, between the stack pointer at the raising point up to the next trap, in order to collect the names of the detected function frames
          \onslide*<2>{
            \begin{itemize}
              \item And more information, like line number, column number...
            \end{itemize}
          }
        \item<3-> It uses the same technique and information as the Garbage Collector (GC) uses to scan the values live on the stack
          \onslide*<4>{
            \begin{itemize}
              \item \typename{frame\_descr} are at the core of stack unwinding
            \end{itemize}
          }
      \end{itemize}
      \vfill
      \mintocaml[firstline=9,lastline=13,highlightlines=12]{../src/backtrace/backtrace.ml}
      \endminipage
    \end{column}
    \begin{column}{0.5\textwidth}
      \onslide*<1>{\listStashBacktrace[highlightlines=91]{}}
      \onslide*<2>{\listStashBacktrace[highlightlines={91,117-118}]{}}
      \onslide*<3->{\listStashBacktrace[highlightlines={94,106,109-110}]{}}
    \end{column}
  \end{columns}
\end{frame}

\newcommand\listFrameDescr[1][]{
  \listocaml[firstline=111,lastline=124,#1]{../ocaml/asmcomp/emitaux.ml}{asmcomp/emitaux.ml:111}}
\newcommand\listDebugInfo[1][]{
  \listocaml[firstline=38,lastline=49,#1]{../ocaml/lambda/debuginfo.mli}{lambda/debuginfo.mli:38}}

\begin{frame}{Backtraces}{\typename{frame\_descr} and \typename{frame\_debuginfo} in a nutshell}
  \begin{columns}[c]
    \begin{column}{0.5\textwidth}
      \begin{itemize}
        \item<1-> For every return address in an OCaml program, there is a associated \typename{frame\_descr}
        \item<2-> A \typename{frame\_descr} specifies
        \begin{itemize}
          \item<2-> The size of the function stack frame
          \item<3-> Where live values are on the stack (so that the GC can mark them as live and prevent from being swept)
        \end{itemize}
        \item<4-> When compiled with debug information, \typename{frame\_descr} will be linked to a \typename{frame\_debuginfo}
        \begin{itemize}
          \item<5-> Contains information about the position in the source code (file name, line and column number)
        \end{itemize}
      \end{itemize}
    \end{column}
    \begin{column}{0.5\textwidth}
        \onslide*<1>{\listFrameDescr[highlightlines={118-122}]{}}
        \onslide*<2>{\listFrameDescr[highlightlines={120}]{}}
        \onslide*<3>{\listFrameDescr[highlightlines={121}]{}}
        \onslide*<4->{\listFrameDescr[highlightlines={122}]{}}
        \medskip
        \onslide*<1-3>{\listDebugInfo{}}
        \onslide*<4>{\listDebugInfo[highlightlines={38,49}]{}}
        \onslide*<5>{\listDebugInfo[highlightlines={39-46}]{}}
    \end{column}
  \end{columns}
\end{frame}

\begin{frame}{Backtraces}{Scanning the stack with \typename{frame\_descr}}
  \begin{columns}[c]
    \begin{column}{0.5\textwidth}
      \begin{itemize}
        \item Given the return address of the code that raises the exception by calling \funcname{caml\_raise\_exn} (\funcarg{pc} argument of \funcname{caml\_stash\_backtrace})
        \item And the position of the stack pointer (\funcarg{sp} argument of \funcname{caml\_stash\_backtrace})
        \item The frame size given by \typename{frame\_descr}.\funcarg{frame\_size}, allows to find the end of the previous frame
        \item And the return address of the caller of this function
        \bigskip
        \onslide<3->{
        \item Note that traps (here the reddish \funcname{ExnA}, \funcname{ExnB} and \funcname{ExnC} blocks) belong to the frame that installed them, and so the frame size changes when traps are removed
        }
      \end{itemize}
    \end{column}
    \begin{column}{0.5\textwidth}
      \raggedleft
      \onslide*<1>{\providecommand\step{2}\input{diagrams/backtrace/backtrace.tex}}%
      \onslide*<2>{\providecommand\step{1}\input{diagrams/backtrace/scanstack.tex}}%
      \onslide*<3>{\providecommand\step{2}\input{diagrams/backtrace/scanstack.tex}}%
      \onslide*<4>{\providecommand\step{3}\input{diagrams/backtrace/scanstack.tex}}%
      \onslide*<5>{\providecommand\step{4}\input{diagrams/backtrace/scanstack.tex}}%
      \onslide*<6>{\providecommand\step{5}\input{diagrams/backtrace/scanstack.tex}}%
    \end{column}
  \end{columns}
\end{frame}

% Duplicate of previous-previous slide
\begin{frame}{Backtraces}{Continuing on \funcname{caml\_stash\_backtrace}}
  \begin{columns}[c]
    \begin{column}{0.5\textwidth}
      \minipage[c][0.75\textheight][s]{\columnwidth}
      \begin{itemize}
          \color{gray}
        \item A C function provided by the runtime (specific to native code)
        \item It scans the stack, between the stack pointer at the raising point up to the next trap, in order to collect the names of the detected function frames
        \item It uses the same technique and information as the Garbage Collector (GC) uses to scan the values live on the stack
      \end{itemize}
      \begin{itemize}
        \item For each function frame detected, store a pointer to the \typename{frame\_descr} in the \localname{domain\_state->backtrace\_buffer} array, effectively constructing the backtrace
      \end{itemize}
      \onslide<2->{
        \begin{itemize}
            \smallskip
          \item Constructed backtrace:
            \funcname{definitely\_raise},
            \onslide<3->{\funcname{probably\_raise}}
        \end{itemize}
      }
      \vfill
      \mintocaml[firstline=9,lastline=13,highlightlines=12]{../src/backtrace/backtrace.ml}
      \endminipage
    \end{column}
    \begin{column}{0.5\textwidth}
      \raggedleft
      \onslide*<1>{\listStashBacktrace[highlightlines={114-115}]{}}
      \onslide*<2>{\providecommand\step{3}\input{diagrams/backtrace/backtrace.tex}}%
      \onslide*<3>{\providecommand\step{4}\input{diagrams/backtrace/backtrace.tex}}%
    \end{column}
  \end{columns}
\end{frame}

\begin{frame}{Backtraces}{Back to \funcname{caml\_raise\_exn}}
  \begin{columns}[c]
    \begin{column}{0.5\textwidth}
      \minipage[c][0.66\textheight][s]{\columnwidth}
      \begin{itemize}\color{gray}
        \item Checks wheteher exceptions backtrace should be recorded, by checking the \funcarg{domain\_state->backtrace\_active} value
        \item Switches to the C stack to be able to execute C code
        \item Calls \funcname{caml\_stash\_backtrace}, to collect the backtrace up to the next trap
      \end{itemize}
      \onslide*<2->{
        \begin{itemize}
          \item<2-> Executes the exception handler in \funcname{probably\_raise} with \funcname{RESTORE\_EXN\_HANDLER\_OCAML}
            \begin{itemize}
              \item Set \regname{RSP} to \localname{domain\_state->exn\_handler} value
              \item Restore \localname{domain\_state->exn\_handler} from trap
              \item Jump to the handler code with \funcname{ret}
            \end{itemize}
        \end{itemize}
      }
      \vfill
      \onslide*<1>{\mintocaml[firstline=9,lastline=13,highlightlines=12]{../src/backtrace/backtrace.ml}}
      \onslide*<2->{\mintocaml[firstline=15,lastline=21,highlightlines={19-21}]{../src/backtrace/backtrace.ml}}
      \endminipage
    \end{column}
    \begin{column}{0.5\textwidth}
      \centering
      \onslide*<1>{
        \mintc[firstline=190,lastline=192]{../ocaml/runtime/amd64.S}
        \mintc[firstline=234,lastline=237]{../ocaml/runtime/amd64.S}
      }
      \onslide*<2>{
        \mintc[firstline=190,lastline=192,highlightlines={191-192}]{../ocaml/runtime/amd64.S}
        \mintc[firstline=234,lastline=237,highlightlines={235-237}]{../ocaml/runtime/amd64.S}
      }
      \onslide*<1>{\listCamlRaiseExn[highlightlines=720]{}}
      \onslide*<2>{\listCamlRaiseExn[highlightlines=722]{}}
      \onslide*<3->{\providecommand\step{5}\input{diagrams/backtrace/backtrace.tex}}
    \end{column}
  \end{columns}
\end{frame}

\begin{frame}{Backtraces}{\funcname{caml\_probably\_raise}'s handler doesn't catch \typename{ExnC}}
  \begin{columns}[c]
    \begin{column}{0.45\textwidth}
      \minipage[c][0.75\textheight][s]{\columnwidth}
      \begin{itemize}
        \item<1-> \funcname{match \dots with}\footnotemark pattern matching can also match exceptions
        \item<1-> The CMM of \funcname{probably\_raise} shows that this \funcname{match \dots with} is implemented with a pattern matching and a \funcname{try \dots with}
        \item<1-> But this one only matches on \typename{ExnA} exception
        \item<2-> Since the pattern matching fails to catch the exception, \funcname{reraise} is called with our current \typename{ExnC} exception
        \item<3-> Disassembly shows that \funcname{reraise} is actually a call to \funcname{caml\_reraise\_exn}, which is also an assembly function provided by the runtime
      \end{itemize}
      \vfill
      \mintocaml[firstline=15,lastline=21,highlightlines={18-21}]{../src/backtrace/backtrace.ml}
      \endminipage
    \end{column}
    \begin{column}{0.55\textwidth}
      \centering
      \onslide*<1>{\mintcmm[highlightlines={10-18}]{../src/backtrace/camlBacktrace\_\_probably\_raise\_351.cmm}}
      \onslide*<2>{\listcmm[highlightlines=18]{../src/backtrace/camlBacktrace\_\_probably\_raise_351.cmm}{CMM of probably\_raise}}
      \onslide*<3>{\mintobjdump[firstline=5,lastline=35,highlightlines=31]{../src/backtrace/camlBacktrace\_\_probably\_raise_351.objdump}}
    \end{column}
  \end{columns}
  \only<1>{\footnotetext[1]{https://blog.janestreet.com/pattern-matching-and-exception-handling-unite/}}
\end{frame}

\newcommand\listCamlReraiseExn[1][]{
  \listasm[firstline=727,lastline=734,#1]{../ocaml/runtime/amd64.S}{runtime/amd64.S:727}}

\begin{frame}{Backtraces}{Introducing \funcname{caml\_reraise\_exn}}
  \begin{columns}[c]
    \begin{column}{0.5\textwidth}
      \minipage[c][0.66\textheight][s]{\columnwidth}
      \begin{itemize}
        \item<1-> The exception have to be reraised to the next handler
        \item<2-> First, checks if the backtrace should be collected with \funcarg{domain\_state->backtrace\_active}
        \only<3>{
        \begin{itemize}
            \item If not, behaves like a \funcname{raise\_notrace} with \funcname{RESTORE\_EXN\_HANDLER\_OCAML}
        \end{itemize}
        }
        \item<4-> Jumps into \funcname{caml\_raise\_exn}
        \item<5-> Calls \funcname{caml\_stash\_backtrace} again, to scan the next part of the stack: from the trap just pop-ed to the next trap
      \end{itemize}
      \vfill
      \mintocaml[firstline=15,lastline=21,highlightlines={18-21}]{../src/backtrace/backtrace.ml}
      \endminipage
    \end{column}
    \begin{column}{0.5\textwidth}
      \raggedleft
      \onslide*<1>{\listCamlReraiseExn{}}
      \onslide*<2>{\listCamlReraiseExn[highlightlines=729]{}}
      \onslide*<3>{
        \mintc[firstline=190,lastline=192,highlightlines={191-192}]{../ocaml/runtime/amd64.S}
        \mintc[firstline=234,lastline=237,highlightlines={235-237}]{../ocaml/runtime/amd64.S}
        \medskip
        \listCamlReraiseExn[highlightlines=731]{}
      }
      \onslide*<4-5>{
        % TODO refactor with \listCamlReraiseExn
        \mintasm[firstline=727,lastline=734,highlightlines=730]{../ocaml/runtime/amd64.S}
        \medskip
      }
      \onslide*<4>{\listCamlRaiseExn[highlightlines=711]{}}
      \onslide*<5>{\listCamlRaiseExn[highlightlines=720]{}}
    \end{column}
  \end{columns}
\end{frame}

\begin{frame}{Backtraces}{Backtrace collection continues}
  \begin{columns}[c]
    \begin{column}{0.55\textwidth}
      \minipage[c][0.75\textheight][s]{\columnwidth}
      \begin{itemize}
        \item<1-> \funcname{caml\_stash\_backtrace} scans the older part of the stack, up to the next trap
        \item<6-> Controls jumps to the nested exception handler in \funcname{maybe\_raise}, which matches only on \typename{ExnB}
        \item<7-> \funcname{caml\_reraise\_exn} is called again, backtrace collection resumes with \funcname{caml\_stash\_backtrace}
      \end{itemize}
      \medskip
      \begin{itemize}
        \item<2-> Constructed backtrace:
        \begin{itemize}
          \item <2-> \funcname{definitely\_raise}
          \item <2-> \funcname{probably\_raise}
          \item <3-> \funcname{probably\_raise} (re-raise)
          \item <4-> \funcname{maybe\_raise}
          \item <5-> \funcname{main}
          \item <8-> \funcname{main} (re-raise)
        \end{itemize}
      \end{itemize}
      \vfill
      \onslide*<1-5>{\mintocaml[firstline=15,lastline=21]{../src/backtrace/backtrace.ml}}
      \onslide*<6>{\mintocaml[firstline=28,lastline=34,highlightlines=31]{../src/backtrace/backtrace.ml}}
      \onslide*<7->{\mintocaml[firstline=28,lastline=34,highlightlines=33]{../src/backtrace/backtrace.ml}}
      \endminipage
    \end{column}
    \begin{column}{0.45\textwidth}
      \raggedleft
      \onslide*<1>{\providecommand\step{6}\input{diagrams/backtrace/backtrace.tex}}%
      \onslide*<2>{\providecommand\step{6}\input{diagrams/backtrace/backtrace.tex}}%
      \onslide*<3>{\providecommand\step{7}\input{diagrams/backtrace/backtrace.tex}}%
      \onslide*<4>{\providecommand\step{8}\input{diagrams/backtrace/backtrace.tex}}%
      \onslide*<5>{\providecommand\step{9}\input{diagrams/backtrace/backtrace.tex}}%
      \onslide*<6>{\providecommand\step{10}\input{diagrams/backtrace/backtrace.tex}}%
      \onslide*<7>{\providecommand\step{11}\input{diagrams/backtrace/backtrace.tex}}%
      \onslide*<8>{\providecommand\step{12}\input{diagrams/backtrace/backtrace.tex}}%
      \onslide*<9>{\providecommand\step{13}\input{diagrams/backtrace/backtrace.tex}}%
    \end{column}
  \end{columns}
\end{frame}

\begin{frame}{Backtraces}{Printing the backtrace}
  \begin{columns}[c]
    \begin{column}{0.45\textwidth}
      \minipage[c][0.66\textheight][s]{\columnwidth}
      \begin{itemize}
        \item<1-> The exception backtrace can now be printed to \funcarg{stdout} with \funcname{Printexc.print_backtrace}
        \item<2-> First the exception backtrace is retrieved into an OCaml array with \funcname{get_raw_backtrace }
        \item<3-> Which is implemented as the \funcname{caml_get_exception_raw_backtrace} C function in the runtime
        \item<4-> And finally printed with \funcname{print_exception_backtrace}
      \end{itemize}
      \vfill
      \mintocaml[firstline=28,lastline=34,highlightlines=33]{../src/backtrace/backtrace.ml}
      \endminipage
    \end{column}
    \begin{column}{0.55\textwidth}
      \onslide*<1,2>{
        \mintocaml[firstline=100,lastline=101]{../ocaml/stdlib/printexc.ml}
      }
      \only<1>{
        \listocaml[firstline=170,lastline=172]{../ocaml/stdlib/printexc.ml}{stdlib/printexc.ml:170}
      }
      \only<2>{
        \listocaml[firstline=170,lastline=172,highlightlines=172]{../ocaml/stdlib/printexc.ml}{stdlib/printexc.ml:170}
      }
      \only<3>{
        \mintc[firstline=154,lastline=156]{../ocaml/runtime/backtrace.c}
        \listc[firstline=171,lastline=192,highlightlines={184-187}]{../ocaml/runtime/backtrace.c}{runtime/backtrace.c:154}
      }
      \only<4>{
        \listocaml[firstline=155,lastline=165]{../ocaml/stdlib/printexc.ml}{stdlib/printexc.ml:55}
      }
    \end{column}
  \end{columns}
\end{frame}


\subsection{The multiple kinds of raise}
\frameSubsection{}{}

\begin{frame}{Backtraces}{When backtraces are not needed}
  \begin{columns}[c]
    \begin{column}{0.45\textwidth}
      \minipage[c][0.66\textheight][s]{\columnwidth}
      \begin{itemize}
        \item<1-> What if the backtrace for an exception is never needed?
        \item<2-> It's possible to raise the exception explicitly with \funcname{raise\_notrace} to make raising as cheap as possible
        \item<3-> An example of use in the \funcname{dynlink} library of the OCaml compiler
      \end{itemize}
      \vfill
      \onslide<3->{
      \listocaml[firstline=205,lastline=209,highlightlines=207]{../ocaml/otherlibs/dynlink/dynlink\_compilerlibs/misc.ml}{otherlibs/dynlink/dynlink\_compilerlibs/misc.ml:205}
      }
      \endminipage
    \end{column}
    \begin{column}{0.55\textwidth}
    \only<4>{
      \mintcmm[highlightlines=3]{../src/notrace/camlNotrace\_\_fun_347.cmm}
      \listcmm{../src/notrace/camlNotrace\_\_all\_somes\_327.cmm}{all\_somes}
    }
    \onslide*<5->{
      \listobjdump[highlightlines={6-9}]{../src/notrace/camlNotrace\_\_fun_347.objdump}{anonymous function from \funcname{all\_somes}}
    }
    \end{column}
  \end{columns}
\end{frame}

\begin{frame}{Backtraces}{Summary table of the multiple kinds of raise}
  \begin{itemize}
    \item When debugging information is enabled and if the backtrace of an exception isn't needed, it's possible to raise with \funcname{raise\_notrace} for performance
    \item When debugging information is \emph{not} enabled, the compiler optimises \funcname{raise} calls to inline assembly (same effect as using \funcname{raise\_notrace})
  \end{itemize}
  \bigskip
  \centering
  \begin{tabular}{| l | c | c | c | c |}
    \hline
    \parbox{10em}{Debug information \\ (\funcarg{-g} flag)}  & \multicolumn{2}{|c|}{Disabled}                                     & \multicolumn{2}{|c|}{Enabled} \\
    \hline
    Raise kind                                               & \funcname{raise} & \funcname{raise\_notrace}                       & \funcname{raise} & \funcname{raise\_notrace} \\
    \hline
    Compilation                                              & \parbox{8em}{inline assembly\\ (optimization)} & inline assembly   & \funcname{caml\_raise\_exn} & inline assembly \\
    \hline
  \end{tabular}
\end{frame}


%
% Takeaway #5
%
\subsection*{Takeaway \#5}
\frameSubsectionTakeaway{}
\againframe<5>{takeaway}
}%
      \onslide*<3>{\providecommand\step{4}%
% Backtraces
%
\subsection{One advantage of raising with debugging information}
\frameSubsection{}{}

\begin{frame}{Backtraces}{Debugging information \& source code}
  \begin{columns}[c]
    \begin{column}{0.5\textwidth}
      \begin{itemize}
        \item Raising an exception is fun and games, but how do we know where the exception originated? $\rightarrow$ With the backtrace associated with the exception!
        \item In order to get a backtrace from the point where an exception was raised, it's necessary to compile with debugging information enabled (\funcarg{-g} flag for the compiler)
        \item Same example as in the `Nested handlers' section
      \end{itemize}
    \end{column}
    \begin{column}{0.5\textwidth}
      \listocaml[firstline=9,lastline=34]{../src/backtrace/backtrace.ml}{backtrace/backtrace.ml}
    \end{column}
  \end{columns}
\end{frame}

\begin{frame}{Backtraces}{CMM and disassembly of \funcname{definitely\_raise}}
  \begin{columns}[c]
    \begin{column}{0.5\textwidth}
      \minipage[c][0.66\textheight][s]{\columnwidth}
      \begin{itemize}
        \item<1-> An effect of compiling with debugging information (\funcarg{-g}): the CMM no longer uses \funcname{raise\_notrace} to raise the exception but \funcname{raise} instead
        \item<2-> Disassembly shows that CMM \funcname{raise} is actually a call to \funcname{caml\_raise\_exn}, a function in assembler provided by the runtime
      \end{itemize}
      \vfill
      \mintocaml[firstline=9,lastline=13,highlightlines=12]{../src/backtrace/backtrace.ml}
      \endminipage
    \end{column}
    \begin{column}{0.5\textwidth}
      \onslide*<1>{
        \mintcmm[highlightlines=5]{../src/backtrace/camlBacktrace\_\_definitely\_raise\_347.cmm}
      }
      \onslide*<2>{
        \mintobjdump[firstline=2,highlightlines=14]{../src/backtrace/camlBacktrace\_\_definitely\_raise\_347.objdump}
      }
    \end{column}
  \end{columns}
\end{frame}

\begin{frame}{Backtraces}{Introducing \funcname{caml\_raise\_exn}}
  \begin{columns}[c]
    \begin{column}{0.5\textwidth}
      \minipage[c][0.66\textheight][s]{\columnwidth}
      \onslide*<1>{
        \begin{itemize}
          \item Raising the exception is performed using \funcname{caml\_raise\_exn} which is
            \begin{itemize}
              \item An assembly function provided by the runtime
              \item Architecture specific
            \end{itemize}
          \item Let's see what it does differently from \funcname{raise\_notrace}\dots
          \item We are about to raise \typename{ExnC} with \funcname{caml\_raise\_exn} from \funcname{definitely\_raise}
        \end{itemize}
      }
      \onslide*<2>{
        \mintobjdump[firstline=2,highlightlines=14]{../src/backtrace/camlBacktrace\_\_definitely\_raise\_347.objdump}
      }
      \vfill
      \mintocaml[firstline=9,lastline=13,highlightlines=12]{../src/backtrace/backtrace.ml}
      \endminipage
    \end{column}
    \begin{column}{0.5\textwidth}
      \centering
      \providecommand\step{1}\input{diagrams/backtrace/backtrace.tex}
    \end{column}
  \end{columns}
\end{frame}

\begin{frame}{Backtraces}{But before that, we need more fields from \typename{caml\_domain\_state}}
  \begin{columns}
    \begin{column}{0.5\textwidth}
      \begin{itemize}
        \item Remember \typename{caml\_domain\_state} the per-domain runtime state?
        \item Some other members are of interest to us now\dots
        \item \localname{backtrace\_active} is used to know if the runtime should collect backtraces
        \item \localname{backtrace\_active} can be set by
          \begin{itemize}
            \item The \funcarg{b} flag in the \funcname{OCAMLRUNPARAM} environment variable
            \item \mintinline{ocaml}{Printexc.record_backtrace : bool -> unit} \footnotemark
          \end{itemize}
      \end{itemize}
    \end{column}
    \begin{column}{0.5\textwidth}
      \begin{listing}
        \mintc[highlightlines={9-11}]{src/ocaml/domain\_state\_backtrace.h}
        \caption{runtime/caml/domain\_state.h}
      \end{listing}
    \end{column}
  \end{columns}
  \footnotetext[1]{https://v2.ocaml.org/api/Printexc.html}
\end{frame}

\newcommand\listCamlRaiseExn[1][]{
  \listasm[firstline=702,lastline=725,#1]{../ocaml/runtime/amd64.S}{runtime/amd64.S:702}}

\begin{frame}{Backtraces}{Walk through \funcname{caml\_raise\_exn}}
  \begin{columns}[T]
    \begin{column}{0.5\textwidth}
      \begin{itemize}
        \item<1-> First checks whether exceptions backtrace should be recorded
        \item<1-> By checking the \funcarg{domain\_state->backtrace\_active} value
        \item<2-> If not, acts as \funcname{raise\_notrace} with \funcname{RESTORE\_EXN\_HANDLER\_OCAML}
          \begin{itemize}
            \item Set \regname{RSP} to \localname{domain\_state->exn\_handler} value
            \item Restore \localname{domain\_state->exn\_handler} from trap
            \item Jump with \funcname{ret} (same effect as using \funcname{pop} and \funcname{jmp})
          \end{itemize}
        \item<3-> Otherwise, switches to the C stack to be able to execute C code
        \item<4-> Calls \funcname{caml\_stash\_backtrace}, a C function provided by the runtime\ignorespaces
          \onslide<6->{, with
            \begin{itemize}
              \item \funcarg{exn}: exception value
              \item \funcarg{pc}: code address of the raising point
              \item \funcarg{sp}: stack address of the raising function
              \item \funcarg{trapsp}: stack address of the next handler
            \end{itemize}
          }
      \end{itemize}
      \bigskip
      \mintocaml[firstline=9,lastline=13,highlightlines=12]{../src/backtrace/backtrace.ml}
    \end{column}
    \begin{column}{0.5\textwidth}
      \raggedleft
      \onslide*<1,3-4>{
        \mintc[firstline=190,lastline=192]{../ocaml/runtime/amd64.S}
        \mintc[firstline=234,lastline=237]{../ocaml/runtime/amd64.S}
      }
      \onslide*<2>{
        \mintc[firstline=190,lastline=192,highlightlines={191-192}]{../ocaml/runtime/amd64.S}
        \mintc[firstline=234,lastline=237,highlightlines={235-237}]{../ocaml/runtime/amd64.S}
      }
      \onslide*<1>{\listCamlRaiseExn[highlightlines=705]{}}%
      \onslide*<2>{\listCamlRaiseExn[highlightlines={707-708}]{}}%
      \onslide*<3>{\listCamlRaiseExn[highlightlines={712-714}]{}}%
      \onslide*<4>{\listCamlRaiseExn[highlightlines={715-720}]{}}%
      \onslide*<5>{\providecommand\step{1}\input{diagrams/backtrace/backtrace.tex}}%
      \onslide*<6>{\providecommand\step{2}\input{diagrams/backtrace/backtrace.tex}}%
    \end{column}
  \end{columns}
\end{frame}

\newcommand\listStashBacktrace[1][]{
  \listc[firstline=91,lastline=120,#1]{../ocaml/runtime/backtrace\_nat.c}{runtime/backtrace\_nat.c:91}}

\begin{frame}{Backtraces}{Introducing \funcname{caml\_stash\_backtrace}}
  \begin{columns}[c]
    \begin{column}{0.5\textwidth}
      \minipage[c][0.66\textheight][s]{\columnwidth}
      \begin{itemize}
        \item<1-> A C function provided by the runtime (specific to native code)
        \item<2-> It scans the stack, between the stack pointer at the raising point up to the next trap, in order to collect the names of the detected function frames
          \onslide*<2>{
            \begin{itemize}
              \item And more information, like line number, column number...
            \end{itemize}
          }
        \item<3-> It uses the same technique and information as the Garbage Collector (GC) uses to scan the values live on the stack
          \onslide*<4>{
            \begin{itemize}
              \item \typename{frame\_descr} are at the core of stack unwinding
            \end{itemize}
          }
      \end{itemize}
      \vfill
      \mintocaml[firstline=9,lastline=13,highlightlines=12]{../src/backtrace/backtrace.ml}
      \endminipage
    \end{column}
    \begin{column}{0.5\textwidth}
      \onslide*<1>{\listStashBacktrace[highlightlines=91]{}}
      \onslide*<2>{\listStashBacktrace[highlightlines={91,117-118}]{}}
      \onslide*<3->{\listStashBacktrace[highlightlines={94,106,109-110}]{}}
    \end{column}
  \end{columns}
\end{frame}

\newcommand\listFrameDescr[1][]{
  \listocaml[firstline=111,lastline=124,#1]{../ocaml/asmcomp/emitaux.ml}{asmcomp/emitaux.ml:111}}
\newcommand\listDebugInfo[1][]{
  \listocaml[firstline=38,lastline=49,#1]{../ocaml/lambda/debuginfo.mli}{lambda/debuginfo.mli:38}}

\begin{frame}{Backtraces}{\typename{frame\_descr} and \typename{frame\_debuginfo} in a nutshell}
  \begin{columns}[c]
    \begin{column}{0.5\textwidth}
      \begin{itemize}
        \item<1-> For every return address in an OCaml program, there is a associated \typename{frame\_descr}
        \item<2-> A \typename{frame\_descr} specifies
        \begin{itemize}
          \item<2-> The size of the function stack frame
          \item<3-> Where live values are on the stack (so that the GC can mark them as live and prevent from being swept)
        \end{itemize}
        \item<4-> When compiled with debug information, \typename{frame\_descr} will be linked to a \typename{frame\_debuginfo}
        \begin{itemize}
          \item<5-> Contains information about the position in the source code (file name, line and column number)
        \end{itemize}
      \end{itemize}
    \end{column}
    \begin{column}{0.5\textwidth}
        \onslide*<1>{\listFrameDescr[highlightlines={118-122}]{}}
        \onslide*<2>{\listFrameDescr[highlightlines={120}]{}}
        \onslide*<3>{\listFrameDescr[highlightlines={121}]{}}
        \onslide*<4->{\listFrameDescr[highlightlines={122}]{}}
        \medskip
        \onslide*<1-3>{\listDebugInfo{}}
        \onslide*<4>{\listDebugInfo[highlightlines={38,49}]{}}
        \onslide*<5>{\listDebugInfo[highlightlines={39-46}]{}}
    \end{column}
  \end{columns}
\end{frame}

\begin{frame}{Backtraces}{Scanning the stack with \typename{frame\_descr}}
  \begin{columns}[c]
    \begin{column}{0.5\textwidth}
      \begin{itemize}
        \item Given the return address of the code that raises the exception by calling \funcname{caml\_raise\_exn} (\funcarg{pc} argument of \funcname{caml\_stash\_backtrace})
        \item And the position of the stack pointer (\funcarg{sp} argument of \funcname{caml\_stash\_backtrace})
        \item The frame size given by \typename{frame\_descr}.\funcarg{frame\_size}, allows to find the end of the previous frame
        \item And the return address of the caller of this function
        \bigskip
        \onslide<3->{
        \item Note that traps (here the reddish \funcname{ExnA}, \funcname{ExnB} and \funcname{ExnC} blocks) belong to the frame that installed them, and so the frame size changes when traps are removed
        }
      \end{itemize}
    \end{column}
    \begin{column}{0.5\textwidth}
      \raggedleft
      \onslide*<1>{\providecommand\step{2}\input{diagrams/backtrace/backtrace.tex}}%
      \onslide*<2>{\providecommand\step{1}\input{diagrams/backtrace/scanstack.tex}}%
      \onslide*<3>{\providecommand\step{2}\input{diagrams/backtrace/scanstack.tex}}%
      \onslide*<4>{\providecommand\step{3}\input{diagrams/backtrace/scanstack.tex}}%
      \onslide*<5>{\providecommand\step{4}\input{diagrams/backtrace/scanstack.tex}}%
      \onslide*<6>{\providecommand\step{5}\input{diagrams/backtrace/scanstack.tex}}%
    \end{column}
  \end{columns}
\end{frame}

% Duplicate of previous-previous slide
\begin{frame}{Backtraces}{Continuing on \funcname{caml\_stash\_backtrace}}
  \begin{columns}[c]
    \begin{column}{0.5\textwidth}
      \minipage[c][0.75\textheight][s]{\columnwidth}
      \begin{itemize}
          \color{gray}
        \item A C function provided by the runtime (specific to native code)
        \item It scans the stack, between the stack pointer at the raising point up to the next trap, in order to collect the names of the detected function frames
        \item It uses the same technique and information as the Garbage Collector (GC) uses to scan the values live on the stack
      \end{itemize}
      \begin{itemize}
        \item For each function frame detected, store a pointer to the \typename{frame\_descr} in the \localname{domain\_state->backtrace\_buffer} array, effectively constructing the backtrace
      \end{itemize}
      \onslide<2->{
        \begin{itemize}
            \smallskip
          \item Constructed backtrace:
            \funcname{definitely\_raise},
            \onslide<3->{\funcname{probably\_raise}}
        \end{itemize}
      }
      \vfill
      \mintocaml[firstline=9,lastline=13,highlightlines=12]{../src/backtrace/backtrace.ml}
      \endminipage
    \end{column}
    \begin{column}{0.5\textwidth}
      \raggedleft
      \onslide*<1>{\listStashBacktrace[highlightlines={114-115}]{}}
      \onslide*<2>{\providecommand\step{3}\input{diagrams/backtrace/backtrace.tex}}%
      \onslide*<3>{\providecommand\step{4}\input{diagrams/backtrace/backtrace.tex}}%
    \end{column}
  \end{columns}
\end{frame}

\begin{frame}{Backtraces}{Back to \funcname{caml\_raise\_exn}}
  \begin{columns}[c]
    \begin{column}{0.5\textwidth}
      \minipage[c][0.66\textheight][s]{\columnwidth}
      \begin{itemize}\color{gray}
        \item Checks wheteher exceptions backtrace should be recorded, by checking the \funcarg{domain\_state->backtrace\_active} value
        \item Switches to the C stack to be able to execute C code
        \item Calls \funcname{caml\_stash\_backtrace}, to collect the backtrace up to the next trap
      \end{itemize}
      \onslide*<2->{
        \begin{itemize}
          \item<2-> Executes the exception handler in \funcname{probably\_raise} with \funcname{RESTORE\_EXN\_HANDLER\_OCAML}
            \begin{itemize}
              \item Set \regname{RSP} to \localname{domain\_state->exn\_handler} value
              \item Restore \localname{domain\_state->exn\_handler} from trap
              \item Jump to the handler code with \funcname{ret}
            \end{itemize}
        \end{itemize}
      }
      \vfill
      \onslide*<1>{\mintocaml[firstline=9,lastline=13,highlightlines=12]{../src/backtrace/backtrace.ml}}
      \onslide*<2->{\mintocaml[firstline=15,lastline=21,highlightlines={19-21}]{../src/backtrace/backtrace.ml}}
      \endminipage
    \end{column}
    \begin{column}{0.5\textwidth}
      \centering
      \onslide*<1>{
        \mintc[firstline=190,lastline=192]{../ocaml/runtime/amd64.S}
        \mintc[firstline=234,lastline=237]{../ocaml/runtime/amd64.S}
      }
      \onslide*<2>{
        \mintc[firstline=190,lastline=192,highlightlines={191-192}]{../ocaml/runtime/amd64.S}
        \mintc[firstline=234,lastline=237,highlightlines={235-237}]{../ocaml/runtime/amd64.S}
      }
      \onslide*<1>{\listCamlRaiseExn[highlightlines=720]{}}
      \onslide*<2>{\listCamlRaiseExn[highlightlines=722]{}}
      \onslide*<3->{\providecommand\step{5}\input{diagrams/backtrace/backtrace.tex}}
    \end{column}
  \end{columns}
\end{frame}

\begin{frame}{Backtraces}{\funcname{caml\_probably\_raise}'s handler doesn't catch \typename{ExnC}}
  \begin{columns}[c]
    \begin{column}{0.45\textwidth}
      \minipage[c][0.75\textheight][s]{\columnwidth}
      \begin{itemize}
        \item<1-> \funcname{match \dots with}\footnotemark pattern matching can also match exceptions
        \item<1-> The CMM of \funcname{probably\_raise} shows that this \funcname{match \dots with} is implemented with a pattern matching and a \funcname{try \dots with}
        \item<1-> But this one only matches on \typename{ExnA} exception
        \item<2-> Since the pattern matching fails to catch the exception, \funcname{reraise} is called with our current \typename{ExnC} exception
        \item<3-> Disassembly shows that \funcname{reraise} is actually a call to \funcname{caml\_reraise\_exn}, which is also an assembly function provided by the runtime
      \end{itemize}
      \vfill
      \mintocaml[firstline=15,lastline=21,highlightlines={18-21}]{../src/backtrace/backtrace.ml}
      \endminipage
    \end{column}
    \begin{column}{0.55\textwidth}
      \centering
      \onslide*<1>{\mintcmm[highlightlines={10-18}]{../src/backtrace/camlBacktrace\_\_probably\_raise\_351.cmm}}
      \onslide*<2>{\listcmm[highlightlines=18]{../src/backtrace/camlBacktrace\_\_probably\_raise_351.cmm}{CMM of probably\_raise}}
      \onslide*<3>{\mintobjdump[firstline=5,lastline=35,highlightlines=31]{../src/backtrace/camlBacktrace\_\_probably\_raise_351.objdump}}
    \end{column}
  \end{columns}
  \only<1>{\footnotetext[1]{https://blog.janestreet.com/pattern-matching-and-exception-handling-unite/}}
\end{frame}

\newcommand\listCamlReraiseExn[1][]{
  \listasm[firstline=727,lastline=734,#1]{../ocaml/runtime/amd64.S}{runtime/amd64.S:727}}

\begin{frame}{Backtraces}{Introducing \funcname{caml\_reraise\_exn}}
  \begin{columns}[c]
    \begin{column}{0.5\textwidth}
      \minipage[c][0.66\textheight][s]{\columnwidth}
      \begin{itemize}
        \item<1-> The exception have to be reraised to the next handler
        \item<2-> First, checks if the backtrace should be collected with \funcarg{domain\_state->backtrace\_active}
        \only<3>{
        \begin{itemize}
            \item If not, behaves like a \funcname{raise\_notrace} with \funcname{RESTORE\_EXN\_HANDLER\_OCAML}
        \end{itemize}
        }
        \item<4-> Jumps into \funcname{caml\_raise\_exn}
        \item<5-> Calls \funcname{caml\_stash\_backtrace} again, to scan the next part of the stack: from the trap just pop-ed to the next trap
      \end{itemize}
      \vfill
      \mintocaml[firstline=15,lastline=21,highlightlines={18-21}]{../src/backtrace/backtrace.ml}
      \endminipage
    \end{column}
    \begin{column}{0.5\textwidth}
      \raggedleft
      \onslide*<1>{\listCamlReraiseExn{}}
      \onslide*<2>{\listCamlReraiseExn[highlightlines=729]{}}
      \onslide*<3>{
        \mintc[firstline=190,lastline=192,highlightlines={191-192}]{../ocaml/runtime/amd64.S}
        \mintc[firstline=234,lastline=237,highlightlines={235-237}]{../ocaml/runtime/amd64.S}
        \medskip
        \listCamlReraiseExn[highlightlines=731]{}
      }
      \onslide*<4-5>{
        % TODO refactor with \listCamlReraiseExn
        \mintasm[firstline=727,lastline=734,highlightlines=730]{../ocaml/runtime/amd64.S}
        \medskip
      }
      \onslide*<4>{\listCamlRaiseExn[highlightlines=711]{}}
      \onslide*<5>{\listCamlRaiseExn[highlightlines=720]{}}
    \end{column}
  \end{columns}
\end{frame}

\begin{frame}{Backtraces}{Backtrace collection continues}
  \begin{columns}[c]
    \begin{column}{0.55\textwidth}
      \minipage[c][0.75\textheight][s]{\columnwidth}
      \begin{itemize}
        \item<1-> \funcname{caml\_stash\_backtrace} scans the older part of the stack, up to the next trap
        \item<6-> Controls jumps to the nested exception handler in \funcname{maybe\_raise}, which matches only on \typename{ExnB}
        \item<7-> \funcname{caml\_reraise\_exn} is called again, backtrace collection resumes with \funcname{caml\_stash\_backtrace}
      \end{itemize}
      \medskip
      \begin{itemize}
        \item<2-> Constructed backtrace:
        \begin{itemize}
          \item <2-> \funcname{definitely\_raise}
          \item <2-> \funcname{probably\_raise}
          \item <3-> \funcname{probably\_raise} (re-raise)
          \item <4-> \funcname{maybe\_raise}
          \item <5-> \funcname{main}
          \item <8-> \funcname{main} (re-raise)
        \end{itemize}
      \end{itemize}
      \vfill
      \onslide*<1-5>{\mintocaml[firstline=15,lastline=21]{../src/backtrace/backtrace.ml}}
      \onslide*<6>{\mintocaml[firstline=28,lastline=34,highlightlines=31]{../src/backtrace/backtrace.ml}}
      \onslide*<7->{\mintocaml[firstline=28,lastline=34,highlightlines=33]{../src/backtrace/backtrace.ml}}
      \endminipage
    \end{column}
    \begin{column}{0.45\textwidth}
      \raggedleft
      \onslide*<1>{\providecommand\step{6}\input{diagrams/backtrace/backtrace.tex}}%
      \onslide*<2>{\providecommand\step{6}\input{diagrams/backtrace/backtrace.tex}}%
      \onslide*<3>{\providecommand\step{7}\input{diagrams/backtrace/backtrace.tex}}%
      \onslide*<4>{\providecommand\step{8}\input{diagrams/backtrace/backtrace.tex}}%
      \onslide*<5>{\providecommand\step{9}\input{diagrams/backtrace/backtrace.tex}}%
      \onslide*<6>{\providecommand\step{10}\input{diagrams/backtrace/backtrace.tex}}%
      \onslide*<7>{\providecommand\step{11}\input{diagrams/backtrace/backtrace.tex}}%
      \onslide*<8>{\providecommand\step{12}\input{diagrams/backtrace/backtrace.tex}}%
      \onslide*<9>{\providecommand\step{13}\input{diagrams/backtrace/backtrace.tex}}%
    \end{column}
  \end{columns}
\end{frame}

\begin{frame}{Backtraces}{Printing the backtrace}
  \begin{columns}[c]
    \begin{column}{0.45\textwidth}
      \minipage[c][0.66\textheight][s]{\columnwidth}
      \begin{itemize}
        \item<1-> The exception backtrace can now be printed to \funcarg{stdout} with \funcname{Printexc.print_backtrace}
        \item<2-> First the exception backtrace is retrieved into an OCaml array with \funcname{get_raw_backtrace }
        \item<3-> Which is implemented as the \funcname{caml_get_exception_raw_backtrace} C function in the runtime
        \item<4-> And finally printed with \funcname{print_exception_backtrace}
      \end{itemize}
      \vfill
      \mintocaml[firstline=28,lastline=34,highlightlines=33]{../src/backtrace/backtrace.ml}
      \endminipage
    \end{column}
    \begin{column}{0.55\textwidth}
      \onslide*<1,2>{
        \mintocaml[firstline=100,lastline=101]{../ocaml/stdlib/printexc.ml}
      }
      \only<1>{
        \listocaml[firstline=170,lastline=172]{../ocaml/stdlib/printexc.ml}{stdlib/printexc.ml:170}
      }
      \only<2>{
        \listocaml[firstline=170,lastline=172,highlightlines=172]{../ocaml/stdlib/printexc.ml}{stdlib/printexc.ml:170}
      }
      \only<3>{
        \mintc[firstline=154,lastline=156]{../ocaml/runtime/backtrace.c}
        \listc[firstline=171,lastline=192,highlightlines={184-187}]{../ocaml/runtime/backtrace.c}{runtime/backtrace.c:154}
      }
      \only<4>{
        \listocaml[firstline=155,lastline=165]{../ocaml/stdlib/printexc.ml}{stdlib/printexc.ml:55}
      }
    \end{column}
  \end{columns}
\end{frame}


\subsection{The multiple kinds of raise}
\frameSubsection{}{}

\begin{frame}{Backtraces}{When backtraces are not needed}
  \begin{columns}[c]
    \begin{column}{0.45\textwidth}
      \minipage[c][0.66\textheight][s]{\columnwidth}
      \begin{itemize}
        \item<1-> What if the backtrace for an exception is never needed?
        \item<2-> It's possible to raise the exception explicitly with \funcname{raise\_notrace} to make raising as cheap as possible
        \item<3-> An example of use in the \funcname{dynlink} library of the OCaml compiler
      \end{itemize}
      \vfill
      \onslide<3->{
      \listocaml[firstline=205,lastline=209,highlightlines=207]{../ocaml/otherlibs/dynlink/dynlink\_compilerlibs/misc.ml}{otherlibs/dynlink/dynlink\_compilerlibs/misc.ml:205}
      }
      \endminipage
    \end{column}
    \begin{column}{0.55\textwidth}
    \only<4>{
      \mintcmm[highlightlines=3]{../src/notrace/camlNotrace\_\_fun_347.cmm}
      \listcmm{../src/notrace/camlNotrace\_\_all\_somes\_327.cmm}{all\_somes}
    }
    \onslide*<5->{
      \listobjdump[highlightlines={6-9}]{../src/notrace/camlNotrace\_\_fun_347.objdump}{anonymous function from \funcname{all\_somes}}
    }
    \end{column}
  \end{columns}
\end{frame}

\begin{frame}{Backtraces}{Summary table of the multiple kinds of raise}
  \begin{itemize}
    \item When debugging information is enabled and if the backtrace of an exception isn't needed, it's possible to raise with \funcname{raise\_notrace} for performance
    \item When debugging information is \emph{not} enabled, the compiler optimises \funcname{raise} calls to inline assembly (same effect as using \funcname{raise\_notrace})
  \end{itemize}
  \bigskip
  \centering
  \begin{tabular}{| l | c | c | c | c |}
    \hline
    \parbox{10em}{Debug information \\ (\funcarg{-g} flag)}  & \multicolumn{2}{|c|}{Disabled}                                     & \multicolumn{2}{|c|}{Enabled} \\
    \hline
    Raise kind                                               & \funcname{raise} & \funcname{raise\_notrace}                       & \funcname{raise} & \funcname{raise\_notrace} \\
    \hline
    Compilation                                              & \parbox{8em}{inline assembly\\ (optimization)} & inline assembly   & \funcname{caml\_raise\_exn} & inline assembly \\
    \hline
  \end{tabular}
\end{frame}


%
% Takeaway #5
%
\subsection*{Takeaway \#5}
\frameSubsectionTakeaway{}
\againframe<5>{takeaway}
}%
    \end{column}
  \end{columns}
\end{frame}

\begin{frame}{Backtraces}{Back to \funcname{caml\_raise\_exn}}
  \begin{columns}[c]
    \begin{column}{0.5\textwidth}
      \minipage[c][0.66\textheight][s]{\columnwidth}
      \begin{itemize}\color{gray}
        \item Checks wheteher exceptions backtrace should be recorded, by checking the \funcarg{domain\_state->backtrace\_active} value
        \item Switches to the C stack to be able to execute C code
        \item Calls \funcname{caml\_stash\_backtrace}, to collect the backtrace up to the next trap
      \end{itemize}
      \onslide*<2->{
        \begin{itemize}
          \item<2-> Executes the exception handler in \funcname{probably\_raise} with \funcname{RESTORE\_EXN\_HANDLER\_OCAML}
            \begin{itemize}
              \item Set \regname{RSP} to \localname{domain\_state->exn\_handler} value
              \item Restore \localname{domain\_state->exn\_handler} from trap
              \item Jump to the handler code with \funcname{ret}
            \end{itemize}
        \end{itemize}
      }
      \vfill
      \onslide*<1>{\mintocaml[firstline=9,lastline=13,highlightlines=12]{../src/backtrace/backtrace.ml}}
      \onslide*<2->{\mintocaml[firstline=15,lastline=21,highlightlines={19-21}]{../src/backtrace/backtrace.ml}}
      \endminipage
    \end{column}
    \begin{column}{0.5\textwidth}
      \centering
      \onslide*<1>{
        \mintc[firstline=190,lastline=192]{../ocaml/runtime/amd64.S}
        \mintc[firstline=234,lastline=237]{../ocaml/runtime/amd64.S}
      }
      \onslide*<2>{
        \mintc[firstline=190,lastline=192,highlightlines={191-192}]{../ocaml/runtime/amd64.S}
        \mintc[firstline=234,lastline=237,highlightlines={235-237}]{../ocaml/runtime/amd64.S}
      }
      \onslide*<1>{\listCamlRaiseExn[highlightlines=720]{}}
      \onslide*<2>{\listCamlRaiseExn[highlightlines=722]{}}
      \onslide*<3->{\providecommand\step{5}%
% Backtraces
%
\subsection{One advantage of raising with debugging information}
\frameSubsection{}{}

\begin{frame}{Backtraces}{Debugging information \& source code}
  \begin{columns}[c]
    \begin{column}{0.5\textwidth}
      \begin{itemize}
        \item Raising an exception is fun and games, but how do we know where the exception originated? $\rightarrow$ With the backtrace associated with the exception!
        \item In order to get a backtrace from the point where an exception was raised, it's necessary to compile with debugging information enabled (\funcarg{-g} flag for the compiler)
        \item Same example as in the `Nested handlers' section
      \end{itemize}
    \end{column}
    \begin{column}{0.5\textwidth}
      \listocaml[firstline=9,lastline=34]{../src/backtrace/backtrace.ml}{backtrace/backtrace.ml}
    \end{column}
  \end{columns}
\end{frame}

\begin{frame}{Backtraces}{CMM and disassembly of \funcname{definitely\_raise}}
  \begin{columns}[c]
    \begin{column}{0.5\textwidth}
      \minipage[c][0.66\textheight][s]{\columnwidth}
      \begin{itemize}
        \item<1-> An effect of compiling with debugging information (\funcarg{-g}): the CMM no longer uses \funcname{raise\_notrace} to raise the exception but \funcname{raise} instead
        \item<2-> Disassembly shows that CMM \funcname{raise} is actually a call to \funcname{caml\_raise\_exn}, a function in assembler provided by the runtime
      \end{itemize}
      \vfill
      \mintocaml[firstline=9,lastline=13,highlightlines=12]{../src/backtrace/backtrace.ml}
      \endminipage
    \end{column}
    \begin{column}{0.5\textwidth}
      \onslide*<1>{
        \mintcmm[highlightlines=5]{../src/backtrace/camlBacktrace\_\_definitely\_raise\_347.cmm}
      }
      \onslide*<2>{
        \mintobjdump[firstline=2,highlightlines=14]{../src/backtrace/camlBacktrace\_\_definitely\_raise\_347.objdump}
      }
    \end{column}
  \end{columns}
\end{frame}

\begin{frame}{Backtraces}{Introducing \funcname{caml\_raise\_exn}}
  \begin{columns}[c]
    \begin{column}{0.5\textwidth}
      \minipage[c][0.66\textheight][s]{\columnwidth}
      \onslide*<1>{
        \begin{itemize}
          \item Raising the exception is performed using \funcname{caml\_raise\_exn} which is
            \begin{itemize}
              \item An assembly function provided by the runtime
              \item Architecture specific
            \end{itemize}
          \item Let's see what it does differently from \funcname{raise\_notrace}\dots
          \item We are about to raise \typename{ExnC} with \funcname{caml\_raise\_exn} from \funcname{definitely\_raise}
        \end{itemize}
      }
      \onslide*<2>{
        \mintobjdump[firstline=2,highlightlines=14]{../src/backtrace/camlBacktrace\_\_definitely\_raise\_347.objdump}
      }
      \vfill
      \mintocaml[firstline=9,lastline=13,highlightlines=12]{../src/backtrace/backtrace.ml}
      \endminipage
    \end{column}
    \begin{column}{0.5\textwidth}
      \centering
      \providecommand\step{1}\input{diagrams/backtrace/backtrace.tex}
    \end{column}
  \end{columns}
\end{frame}

\begin{frame}{Backtraces}{But before that, we need more fields from \typename{caml\_domain\_state}}
  \begin{columns}
    \begin{column}{0.5\textwidth}
      \begin{itemize}
        \item Remember \typename{caml\_domain\_state} the per-domain runtime state?
        \item Some other members are of interest to us now\dots
        \item \localname{backtrace\_active} is used to know if the runtime should collect backtraces
        \item \localname{backtrace\_active} can be set by
          \begin{itemize}
            \item The \funcarg{b} flag in the \funcname{OCAMLRUNPARAM} environment variable
            \item \mintinline{ocaml}{Printexc.record_backtrace : bool -> unit} \footnotemark
          \end{itemize}
      \end{itemize}
    \end{column}
    \begin{column}{0.5\textwidth}
      \begin{listing}
        \mintc[highlightlines={9-11}]{src/ocaml/domain\_state\_backtrace.h}
        \caption{runtime/caml/domain\_state.h}
      \end{listing}
    \end{column}
  \end{columns}
  \footnotetext[1]{https://v2.ocaml.org/api/Printexc.html}
\end{frame}

\newcommand\listCamlRaiseExn[1][]{
  \listasm[firstline=702,lastline=725,#1]{../ocaml/runtime/amd64.S}{runtime/amd64.S:702}}

\begin{frame}{Backtraces}{Walk through \funcname{caml\_raise\_exn}}
  \begin{columns}[T]
    \begin{column}{0.5\textwidth}
      \begin{itemize}
        \item<1-> First checks whether exceptions backtrace should be recorded
        \item<1-> By checking the \funcarg{domain\_state->backtrace\_active} value
        \item<2-> If not, acts as \funcname{raise\_notrace} with \funcname{RESTORE\_EXN\_HANDLER\_OCAML}
          \begin{itemize}
            \item Set \regname{RSP} to \localname{domain\_state->exn\_handler} value
            \item Restore \localname{domain\_state->exn\_handler} from trap
            \item Jump with \funcname{ret} (same effect as using \funcname{pop} and \funcname{jmp})
          \end{itemize}
        \item<3-> Otherwise, switches to the C stack to be able to execute C code
        \item<4-> Calls \funcname{caml\_stash\_backtrace}, a C function provided by the runtime\ignorespaces
          \onslide<6->{, with
            \begin{itemize}
              \item \funcarg{exn}: exception value
              \item \funcarg{pc}: code address of the raising point
              \item \funcarg{sp}: stack address of the raising function
              \item \funcarg{trapsp}: stack address of the next handler
            \end{itemize}
          }
      \end{itemize}
      \bigskip
      \mintocaml[firstline=9,lastline=13,highlightlines=12]{../src/backtrace/backtrace.ml}
    \end{column}
    \begin{column}{0.5\textwidth}
      \raggedleft
      \onslide*<1,3-4>{
        \mintc[firstline=190,lastline=192]{../ocaml/runtime/amd64.S}
        \mintc[firstline=234,lastline=237]{../ocaml/runtime/amd64.S}
      }
      \onslide*<2>{
        \mintc[firstline=190,lastline=192,highlightlines={191-192}]{../ocaml/runtime/amd64.S}
        \mintc[firstline=234,lastline=237,highlightlines={235-237}]{../ocaml/runtime/amd64.S}
      }
      \onslide*<1>{\listCamlRaiseExn[highlightlines=705]{}}%
      \onslide*<2>{\listCamlRaiseExn[highlightlines={707-708}]{}}%
      \onslide*<3>{\listCamlRaiseExn[highlightlines={712-714}]{}}%
      \onslide*<4>{\listCamlRaiseExn[highlightlines={715-720}]{}}%
      \onslide*<5>{\providecommand\step{1}\input{diagrams/backtrace/backtrace.tex}}%
      \onslide*<6>{\providecommand\step{2}\input{diagrams/backtrace/backtrace.tex}}%
    \end{column}
  \end{columns}
\end{frame}

\newcommand\listStashBacktrace[1][]{
  \listc[firstline=91,lastline=120,#1]{../ocaml/runtime/backtrace\_nat.c}{runtime/backtrace\_nat.c:91}}

\begin{frame}{Backtraces}{Introducing \funcname{caml\_stash\_backtrace}}
  \begin{columns}[c]
    \begin{column}{0.5\textwidth}
      \minipage[c][0.66\textheight][s]{\columnwidth}
      \begin{itemize}
        \item<1-> A C function provided by the runtime (specific to native code)
        \item<2-> It scans the stack, between the stack pointer at the raising point up to the next trap, in order to collect the names of the detected function frames
          \onslide*<2>{
            \begin{itemize}
              \item And more information, like line number, column number...
            \end{itemize}
          }
        \item<3-> It uses the same technique and information as the Garbage Collector (GC) uses to scan the values live on the stack
          \onslide*<4>{
            \begin{itemize}
              \item \typename{frame\_descr} are at the core of stack unwinding
            \end{itemize}
          }
      \end{itemize}
      \vfill
      \mintocaml[firstline=9,lastline=13,highlightlines=12]{../src/backtrace/backtrace.ml}
      \endminipage
    \end{column}
    \begin{column}{0.5\textwidth}
      \onslide*<1>{\listStashBacktrace[highlightlines=91]{}}
      \onslide*<2>{\listStashBacktrace[highlightlines={91,117-118}]{}}
      \onslide*<3->{\listStashBacktrace[highlightlines={94,106,109-110}]{}}
    \end{column}
  \end{columns}
\end{frame}

\newcommand\listFrameDescr[1][]{
  \listocaml[firstline=111,lastline=124,#1]{../ocaml/asmcomp/emitaux.ml}{asmcomp/emitaux.ml:111}}
\newcommand\listDebugInfo[1][]{
  \listocaml[firstline=38,lastline=49,#1]{../ocaml/lambda/debuginfo.mli}{lambda/debuginfo.mli:38}}

\begin{frame}{Backtraces}{\typename{frame\_descr} and \typename{frame\_debuginfo} in a nutshell}
  \begin{columns}[c]
    \begin{column}{0.5\textwidth}
      \begin{itemize}
        \item<1-> For every return address in an OCaml program, there is a associated \typename{frame\_descr}
        \item<2-> A \typename{frame\_descr} specifies
        \begin{itemize}
          \item<2-> The size of the function stack frame
          \item<3-> Where live values are on the stack (so that the GC can mark them as live and prevent from being swept)
        \end{itemize}
        \item<4-> When compiled with debug information, \typename{frame\_descr} will be linked to a \typename{frame\_debuginfo}
        \begin{itemize}
          \item<5-> Contains information about the position in the source code (file name, line and column number)
        \end{itemize}
      \end{itemize}
    \end{column}
    \begin{column}{0.5\textwidth}
        \onslide*<1>{\listFrameDescr[highlightlines={118-122}]{}}
        \onslide*<2>{\listFrameDescr[highlightlines={120}]{}}
        \onslide*<3>{\listFrameDescr[highlightlines={121}]{}}
        \onslide*<4->{\listFrameDescr[highlightlines={122}]{}}
        \medskip
        \onslide*<1-3>{\listDebugInfo{}}
        \onslide*<4>{\listDebugInfo[highlightlines={38,49}]{}}
        \onslide*<5>{\listDebugInfo[highlightlines={39-46}]{}}
    \end{column}
  \end{columns}
\end{frame}

\begin{frame}{Backtraces}{Scanning the stack with \typename{frame\_descr}}
  \begin{columns}[c]
    \begin{column}{0.5\textwidth}
      \begin{itemize}
        \item Given the return address of the code that raises the exception by calling \funcname{caml\_raise\_exn} (\funcarg{pc} argument of \funcname{caml\_stash\_backtrace})
        \item And the position of the stack pointer (\funcarg{sp} argument of \funcname{caml\_stash\_backtrace})
        \item The frame size given by \typename{frame\_descr}.\funcarg{frame\_size}, allows to find the end of the previous frame
        \item And the return address of the caller of this function
        \bigskip
        \onslide<3->{
        \item Note that traps (here the reddish \funcname{ExnA}, \funcname{ExnB} and \funcname{ExnC} blocks) belong to the frame that installed them, and so the frame size changes when traps are removed
        }
      \end{itemize}
    \end{column}
    \begin{column}{0.5\textwidth}
      \raggedleft
      \onslide*<1>{\providecommand\step{2}\input{diagrams/backtrace/backtrace.tex}}%
      \onslide*<2>{\providecommand\step{1}\input{diagrams/backtrace/scanstack.tex}}%
      \onslide*<3>{\providecommand\step{2}\input{diagrams/backtrace/scanstack.tex}}%
      \onslide*<4>{\providecommand\step{3}\input{diagrams/backtrace/scanstack.tex}}%
      \onslide*<5>{\providecommand\step{4}\input{diagrams/backtrace/scanstack.tex}}%
      \onslide*<6>{\providecommand\step{5}\input{diagrams/backtrace/scanstack.tex}}%
    \end{column}
  \end{columns}
\end{frame}

% Duplicate of previous-previous slide
\begin{frame}{Backtraces}{Continuing on \funcname{caml\_stash\_backtrace}}
  \begin{columns}[c]
    \begin{column}{0.5\textwidth}
      \minipage[c][0.75\textheight][s]{\columnwidth}
      \begin{itemize}
          \color{gray}
        \item A C function provided by the runtime (specific to native code)
        \item It scans the stack, between the stack pointer at the raising point up to the next trap, in order to collect the names of the detected function frames
        \item It uses the same technique and information as the Garbage Collector (GC) uses to scan the values live on the stack
      \end{itemize}
      \begin{itemize}
        \item For each function frame detected, store a pointer to the \typename{frame\_descr} in the \localname{domain\_state->backtrace\_buffer} array, effectively constructing the backtrace
      \end{itemize}
      \onslide<2->{
        \begin{itemize}
            \smallskip
          \item Constructed backtrace:
            \funcname{definitely\_raise},
            \onslide<3->{\funcname{probably\_raise}}
        \end{itemize}
      }
      \vfill
      \mintocaml[firstline=9,lastline=13,highlightlines=12]{../src/backtrace/backtrace.ml}
      \endminipage
    \end{column}
    \begin{column}{0.5\textwidth}
      \raggedleft
      \onslide*<1>{\listStashBacktrace[highlightlines={114-115}]{}}
      \onslide*<2>{\providecommand\step{3}\input{diagrams/backtrace/backtrace.tex}}%
      \onslide*<3>{\providecommand\step{4}\input{diagrams/backtrace/backtrace.tex}}%
    \end{column}
  \end{columns}
\end{frame}

\begin{frame}{Backtraces}{Back to \funcname{caml\_raise\_exn}}
  \begin{columns}[c]
    \begin{column}{0.5\textwidth}
      \minipage[c][0.66\textheight][s]{\columnwidth}
      \begin{itemize}\color{gray}
        \item Checks wheteher exceptions backtrace should be recorded, by checking the \funcarg{domain\_state->backtrace\_active} value
        \item Switches to the C stack to be able to execute C code
        \item Calls \funcname{caml\_stash\_backtrace}, to collect the backtrace up to the next trap
      \end{itemize}
      \onslide*<2->{
        \begin{itemize}
          \item<2-> Executes the exception handler in \funcname{probably\_raise} with \funcname{RESTORE\_EXN\_HANDLER\_OCAML}
            \begin{itemize}
              \item Set \regname{RSP} to \localname{domain\_state->exn\_handler} value
              \item Restore \localname{domain\_state->exn\_handler} from trap
              \item Jump to the handler code with \funcname{ret}
            \end{itemize}
        \end{itemize}
      }
      \vfill
      \onslide*<1>{\mintocaml[firstline=9,lastline=13,highlightlines=12]{../src/backtrace/backtrace.ml}}
      \onslide*<2->{\mintocaml[firstline=15,lastline=21,highlightlines={19-21}]{../src/backtrace/backtrace.ml}}
      \endminipage
    \end{column}
    \begin{column}{0.5\textwidth}
      \centering
      \onslide*<1>{
        \mintc[firstline=190,lastline=192]{../ocaml/runtime/amd64.S}
        \mintc[firstline=234,lastline=237]{../ocaml/runtime/amd64.S}
      }
      \onslide*<2>{
        \mintc[firstline=190,lastline=192,highlightlines={191-192}]{../ocaml/runtime/amd64.S}
        \mintc[firstline=234,lastline=237,highlightlines={235-237}]{../ocaml/runtime/amd64.S}
      }
      \onslide*<1>{\listCamlRaiseExn[highlightlines=720]{}}
      \onslide*<2>{\listCamlRaiseExn[highlightlines=722]{}}
      \onslide*<3->{\providecommand\step{5}\input{diagrams/backtrace/backtrace.tex}}
    \end{column}
  \end{columns}
\end{frame}

\begin{frame}{Backtraces}{\funcname{caml\_probably\_raise}'s handler doesn't catch \typename{ExnC}}
  \begin{columns}[c]
    \begin{column}{0.45\textwidth}
      \minipage[c][0.75\textheight][s]{\columnwidth}
      \begin{itemize}
        \item<1-> \funcname{match \dots with}\footnotemark pattern matching can also match exceptions
        \item<1-> The CMM of \funcname{probably\_raise} shows that this \funcname{match \dots with} is implemented with a pattern matching and a \funcname{try \dots with}
        \item<1-> But this one only matches on \typename{ExnA} exception
        \item<2-> Since the pattern matching fails to catch the exception, \funcname{reraise} is called with our current \typename{ExnC} exception
        \item<3-> Disassembly shows that \funcname{reraise} is actually a call to \funcname{caml\_reraise\_exn}, which is also an assembly function provided by the runtime
      \end{itemize}
      \vfill
      \mintocaml[firstline=15,lastline=21,highlightlines={18-21}]{../src/backtrace/backtrace.ml}
      \endminipage
    \end{column}
    \begin{column}{0.55\textwidth}
      \centering
      \onslide*<1>{\mintcmm[highlightlines={10-18}]{../src/backtrace/camlBacktrace\_\_probably\_raise\_351.cmm}}
      \onslide*<2>{\listcmm[highlightlines=18]{../src/backtrace/camlBacktrace\_\_probably\_raise_351.cmm}{CMM of probably\_raise}}
      \onslide*<3>{\mintobjdump[firstline=5,lastline=35,highlightlines=31]{../src/backtrace/camlBacktrace\_\_probably\_raise_351.objdump}}
    \end{column}
  \end{columns}
  \only<1>{\footnotetext[1]{https://blog.janestreet.com/pattern-matching-and-exception-handling-unite/}}
\end{frame}

\newcommand\listCamlReraiseExn[1][]{
  \listasm[firstline=727,lastline=734,#1]{../ocaml/runtime/amd64.S}{runtime/amd64.S:727}}

\begin{frame}{Backtraces}{Introducing \funcname{caml\_reraise\_exn}}
  \begin{columns}[c]
    \begin{column}{0.5\textwidth}
      \minipage[c][0.66\textheight][s]{\columnwidth}
      \begin{itemize}
        \item<1-> The exception have to be reraised to the next handler
        \item<2-> First, checks if the backtrace should be collected with \funcarg{domain\_state->backtrace\_active}
        \only<3>{
        \begin{itemize}
            \item If not, behaves like a \funcname{raise\_notrace} with \funcname{RESTORE\_EXN\_HANDLER\_OCAML}
        \end{itemize}
        }
        \item<4-> Jumps into \funcname{caml\_raise\_exn}
        \item<5-> Calls \funcname{caml\_stash\_backtrace} again, to scan the next part of the stack: from the trap just pop-ed to the next trap
      \end{itemize}
      \vfill
      \mintocaml[firstline=15,lastline=21,highlightlines={18-21}]{../src/backtrace/backtrace.ml}
      \endminipage
    \end{column}
    \begin{column}{0.5\textwidth}
      \raggedleft
      \onslide*<1>{\listCamlReraiseExn{}}
      \onslide*<2>{\listCamlReraiseExn[highlightlines=729]{}}
      \onslide*<3>{
        \mintc[firstline=190,lastline=192,highlightlines={191-192}]{../ocaml/runtime/amd64.S}
        \mintc[firstline=234,lastline=237,highlightlines={235-237}]{../ocaml/runtime/amd64.S}
        \medskip
        \listCamlReraiseExn[highlightlines=731]{}
      }
      \onslide*<4-5>{
        % TODO refactor with \listCamlReraiseExn
        \mintasm[firstline=727,lastline=734,highlightlines=730]{../ocaml/runtime/amd64.S}
        \medskip
      }
      \onslide*<4>{\listCamlRaiseExn[highlightlines=711]{}}
      \onslide*<5>{\listCamlRaiseExn[highlightlines=720]{}}
    \end{column}
  \end{columns}
\end{frame}

\begin{frame}{Backtraces}{Backtrace collection continues}
  \begin{columns}[c]
    \begin{column}{0.55\textwidth}
      \minipage[c][0.75\textheight][s]{\columnwidth}
      \begin{itemize}
        \item<1-> \funcname{caml\_stash\_backtrace} scans the older part of the stack, up to the next trap
        \item<6-> Controls jumps to the nested exception handler in \funcname{maybe\_raise}, which matches only on \typename{ExnB}
        \item<7-> \funcname{caml\_reraise\_exn} is called again, backtrace collection resumes with \funcname{caml\_stash\_backtrace}
      \end{itemize}
      \medskip
      \begin{itemize}
        \item<2-> Constructed backtrace:
        \begin{itemize}
          \item <2-> \funcname{definitely\_raise}
          \item <2-> \funcname{probably\_raise}
          \item <3-> \funcname{probably\_raise} (re-raise)
          \item <4-> \funcname{maybe\_raise}
          \item <5-> \funcname{main}
          \item <8-> \funcname{main} (re-raise)
        \end{itemize}
      \end{itemize}
      \vfill
      \onslide*<1-5>{\mintocaml[firstline=15,lastline=21]{../src/backtrace/backtrace.ml}}
      \onslide*<6>{\mintocaml[firstline=28,lastline=34,highlightlines=31]{../src/backtrace/backtrace.ml}}
      \onslide*<7->{\mintocaml[firstline=28,lastline=34,highlightlines=33]{../src/backtrace/backtrace.ml}}
      \endminipage
    \end{column}
    \begin{column}{0.45\textwidth}
      \raggedleft
      \onslide*<1>{\providecommand\step{6}\input{diagrams/backtrace/backtrace.tex}}%
      \onslide*<2>{\providecommand\step{6}\input{diagrams/backtrace/backtrace.tex}}%
      \onslide*<3>{\providecommand\step{7}\input{diagrams/backtrace/backtrace.tex}}%
      \onslide*<4>{\providecommand\step{8}\input{diagrams/backtrace/backtrace.tex}}%
      \onslide*<5>{\providecommand\step{9}\input{diagrams/backtrace/backtrace.tex}}%
      \onslide*<6>{\providecommand\step{10}\input{diagrams/backtrace/backtrace.tex}}%
      \onslide*<7>{\providecommand\step{11}\input{diagrams/backtrace/backtrace.tex}}%
      \onslide*<8>{\providecommand\step{12}\input{diagrams/backtrace/backtrace.tex}}%
      \onslide*<9>{\providecommand\step{13}\input{diagrams/backtrace/backtrace.tex}}%
    \end{column}
  \end{columns}
\end{frame}

\begin{frame}{Backtraces}{Printing the backtrace}
  \begin{columns}[c]
    \begin{column}{0.45\textwidth}
      \minipage[c][0.66\textheight][s]{\columnwidth}
      \begin{itemize}
        \item<1-> The exception backtrace can now be printed to \funcarg{stdout} with \funcname{Printexc.print_backtrace}
        \item<2-> First the exception backtrace is retrieved into an OCaml array with \funcname{get_raw_backtrace }
        \item<3-> Which is implemented as the \funcname{caml_get_exception_raw_backtrace} C function in the runtime
        \item<4-> And finally printed with \funcname{print_exception_backtrace}
      \end{itemize}
      \vfill
      \mintocaml[firstline=28,lastline=34,highlightlines=33]{../src/backtrace/backtrace.ml}
      \endminipage
    \end{column}
    \begin{column}{0.55\textwidth}
      \onslide*<1,2>{
        \mintocaml[firstline=100,lastline=101]{../ocaml/stdlib/printexc.ml}
      }
      \only<1>{
        \listocaml[firstline=170,lastline=172]{../ocaml/stdlib/printexc.ml}{stdlib/printexc.ml:170}
      }
      \only<2>{
        \listocaml[firstline=170,lastline=172,highlightlines=172]{../ocaml/stdlib/printexc.ml}{stdlib/printexc.ml:170}
      }
      \only<3>{
        \mintc[firstline=154,lastline=156]{../ocaml/runtime/backtrace.c}
        \listc[firstline=171,lastline=192,highlightlines={184-187}]{../ocaml/runtime/backtrace.c}{runtime/backtrace.c:154}
      }
      \only<4>{
        \listocaml[firstline=155,lastline=165]{../ocaml/stdlib/printexc.ml}{stdlib/printexc.ml:55}
      }
    \end{column}
  \end{columns}
\end{frame}


\subsection{The multiple kinds of raise}
\frameSubsection{}{}

\begin{frame}{Backtraces}{When backtraces are not needed}
  \begin{columns}[c]
    \begin{column}{0.45\textwidth}
      \minipage[c][0.66\textheight][s]{\columnwidth}
      \begin{itemize}
        \item<1-> What if the backtrace for an exception is never needed?
        \item<2-> It's possible to raise the exception explicitly with \funcname{raise\_notrace} to make raising as cheap as possible
        \item<3-> An example of use in the \funcname{dynlink} library of the OCaml compiler
      \end{itemize}
      \vfill
      \onslide<3->{
      \listocaml[firstline=205,lastline=209,highlightlines=207]{../ocaml/otherlibs/dynlink/dynlink\_compilerlibs/misc.ml}{otherlibs/dynlink/dynlink\_compilerlibs/misc.ml:205}
      }
      \endminipage
    \end{column}
    \begin{column}{0.55\textwidth}
    \only<4>{
      \mintcmm[highlightlines=3]{../src/notrace/camlNotrace\_\_fun_347.cmm}
      \listcmm{../src/notrace/camlNotrace\_\_all\_somes\_327.cmm}{all\_somes}
    }
    \onslide*<5->{
      \listobjdump[highlightlines={6-9}]{../src/notrace/camlNotrace\_\_fun_347.objdump}{anonymous function from \funcname{all\_somes}}
    }
    \end{column}
  \end{columns}
\end{frame}

\begin{frame}{Backtraces}{Summary table of the multiple kinds of raise}
  \begin{itemize}
    \item When debugging information is enabled and if the backtrace of an exception isn't needed, it's possible to raise with \funcname{raise\_notrace} for performance
    \item When debugging information is \emph{not} enabled, the compiler optimises \funcname{raise} calls to inline assembly (same effect as using \funcname{raise\_notrace})
  \end{itemize}
  \bigskip
  \centering
  \begin{tabular}{| l | c | c | c | c |}
    \hline
    \parbox{10em}{Debug information \\ (\funcarg{-g} flag)}  & \multicolumn{2}{|c|}{Disabled}                                     & \multicolumn{2}{|c|}{Enabled} \\
    \hline
    Raise kind                                               & \funcname{raise} & \funcname{raise\_notrace}                       & \funcname{raise} & \funcname{raise\_notrace} \\
    \hline
    Compilation                                              & \parbox{8em}{inline assembly\\ (optimization)} & inline assembly   & \funcname{caml\_raise\_exn} & inline assembly \\
    \hline
  \end{tabular}
\end{frame}


%
% Takeaway #5
%
\subsection*{Takeaway \#5}
\frameSubsectionTakeaway{}
\againframe<5>{takeaway}
}
    \end{column}
  \end{columns}
\end{frame}

\begin{frame}{Backtraces}{\funcname{caml\_probably\_raise}'s handler doesn't catch \typename{ExnC}}
  \begin{columns}[c]
    \begin{column}{0.45\textwidth}
      \minipage[c][0.75\textheight][s]{\columnwidth}
      \begin{itemize}
        \item<1-> \funcname{match \dots with}\footnotemark pattern matching can also match exceptions
        \item<1-> The CMM of \funcname{probably\_raise} shows that this \funcname{match \dots with} is implemented with a pattern matching and a \funcname{try \dots with}
        \item<1-> But this one only matches on \typename{ExnA} exception
        \item<2-> Since the pattern matching fails to catch the exception, \funcname{reraise} is called with our current \typename{ExnC} exception
        \item<3-> Disassembly shows that \funcname{reraise} is actually a call to \funcname{caml\_reraise\_exn}, which is also an assembly function provided by the runtime
      \end{itemize}
      \vfill
      \mintocaml[firstline=15,lastline=21,highlightlines={18-21}]{../src/backtrace/backtrace.ml}
      \endminipage
    \end{column}
    \begin{column}{0.55\textwidth}
      \centering
      \onslide*<1>{\mintcmm[highlightlines={10-18}]{../src/backtrace/camlBacktrace\_\_probably\_raise\_351.cmm}}
      \onslide*<2>{\listcmm[highlightlines=18]{../src/backtrace/camlBacktrace\_\_probably\_raise_351.cmm}{CMM of probably\_raise}}
      \onslide*<3>{\mintobjdump[firstline=5,lastline=35,highlightlines=31]{../src/backtrace/camlBacktrace\_\_probably\_raise_351.objdump}}
    \end{column}
  \end{columns}
  \only<1>{\footnotetext[1]{https://blog.janestreet.com/pattern-matching-and-exception-handling-unite/}}
\end{frame}

\newcommand\listCamlReraiseExn[1][]{
  \listasm[firstline=727,lastline=734,#1]{../ocaml/runtime/amd64.S}{runtime/amd64.S:727}}

\begin{frame}{Backtraces}{Introducing \funcname{caml\_reraise\_exn}}
  \begin{columns}[c]
    \begin{column}{0.5\textwidth}
      \minipage[c][0.66\textheight][s]{\columnwidth}
      \begin{itemize}
        \item<1-> The exception have to be reraised to the next handler
        \item<2-> First, checks if the backtrace should be collected with \funcarg{domain\_state->backtrace\_active}
        \only<3>{
        \begin{itemize}
            \item If not, behaves like a \funcname{raise\_notrace} with \funcname{RESTORE\_EXN\_HANDLER\_OCAML}
        \end{itemize}
        }
        \item<4-> Jumps into \funcname{caml\_raise\_exn}
        \item<5-> Calls \funcname{caml\_stash\_backtrace} again, to scan the next part of the stack: from the trap just pop-ed to the next trap
      \end{itemize}
      \vfill
      \mintocaml[firstline=15,lastline=21,highlightlines={18-21}]{../src/backtrace/backtrace.ml}
      \endminipage
    \end{column}
    \begin{column}{0.5\textwidth}
      \raggedleft
      \onslide*<1>{\listCamlReraiseExn{}}
      \onslide*<2>{\listCamlReraiseExn[highlightlines=729]{}}
      \onslide*<3>{
        \mintc[firstline=190,lastline=192,highlightlines={191-192}]{../ocaml/runtime/amd64.S}
        \mintc[firstline=234,lastline=237,highlightlines={235-237}]{../ocaml/runtime/amd64.S}
        \medskip
        \listCamlReraiseExn[highlightlines=731]{}
      }
      \onslide*<4-5>{
        % TODO refactor with \listCamlReraiseExn
        \mintasm[firstline=727,lastline=734,highlightlines=730]{../ocaml/runtime/amd64.S}
        \medskip
      }
      \onslide*<4>{\listCamlRaiseExn[highlightlines=711]{}}
      \onslide*<5>{\listCamlRaiseExn[highlightlines=720]{}}
    \end{column}
  \end{columns}
\end{frame}

\begin{frame}{Backtraces}{Backtrace collection continues}
  \begin{columns}[c]
    \begin{column}{0.55\textwidth}
      \minipage[c][0.75\textheight][s]{\columnwidth}
      \begin{itemize}
        \item<1-> \funcname{caml\_stash\_backtrace} scans the older part of the stack, up to the next trap
        \item<6-> Controls jumps to the nested exception handler in \funcname{maybe\_raise}, which matches only on \typename{ExnB}
        \item<7-> \funcname{caml\_reraise\_exn} is called again, backtrace collection resumes with \funcname{caml\_stash\_backtrace}
      \end{itemize}
      \medskip
      \begin{itemize}
        \item<2-> Constructed backtrace:
        \begin{itemize}
          \item <2-> \funcname{definitely\_raise}
          \item <2-> \funcname{probably\_raise}
          \item <3-> \funcname{probably\_raise} (re-raise)
          \item <4-> \funcname{maybe\_raise}
          \item <5-> \funcname{main}
          \item <8-> \funcname{main} (re-raise)
        \end{itemize}
      \end{itemize}
      \vfill
      \onslide*<1-5>{\mintocaml[firstline=15,lastline=21]{../src/backtrace/backtrace.ml}}
      \onslide*<6>{\mintocaml[firstline=28,lastline=34,highlightlines=31]{../src/backtrace/backtrace.ml}}
      \onslide*<7->{\mintocaml[firstline=28,lastline=34,highlightlines=33]{../src/backtrace/backtrace.ml}}
      \endminipage
    \end{column}
    \begin{column}{0.45\textwidth}
      \raggedleft
      \onslide*<1>{\providecommand\step{6}%
% Backtraces
%
\subsection{One advantage of raising with debugging information}
\frameSubsection{}{}

\begin{frame}{Backtraces}{Debugging information \& source code}
  \begin{columns}[c]
    \begin{column}{0.5\textwidth}
      \begin{itemize}
        \item Raising an exception is fun and games, but how do we know where the exception originated? $\rightarrow$ With the backtrace associated with the exception!
        \item In order to get a backtrace from the point where an exception was raised, it's necessary to compile with debugging information enabled (\funcarg{-g} flag for the compiler)
        \item Same example as in the `Nested handlers' section
      \end{itemize}
    \end{column}
    \begin{column}{0.5\textwidth}
      \listocaml[firstline=9,lastline=34]{../src/backtrace/backtrace.ml}{backtrace/backtrace.ml}
    \end{column}
  \end{columns}
\end{frame}

\begin{frame}{Backtraces}{CMM and disassembly of \funcname{definitely\_raise}}
  \begin{columns}[c]
    \begin{column}{0.5\textwidth}
      \minipage[c][0.66\textheight][s]{\columnwidth}
      \begin{itemize}
        \item<1-> An effect of compiling with debugging information (\funcarg{-g}): the CMM no longer uses \funcname{raise\_notrace} to raise the exception but \funcname{raise} instead
        \item<2-> Disassembly shows that CMM \funcname{raise} is actually a call to \funcname{caml\_raise\_exn}, a function in assembler provided by the runtime
      \end{itemize}
      \vfill
      \mintocaml[firstline=9,lastline=13,highlightlines=12]{../src/backtrace/backtrace.ml}
      \endminipage
    \end{column}
    \begin{column}{0.5\textwidth}
      \onslide*<1>{
        \mintcmm[highlightlines=5]{../src/backtrace/camlBacktrace\_\_definitely\_raise\_347.cmm}
      }
      \onslide*<2>{
        \mintobjdump[firstline=2,highlightlines=14]{../src/backtrace/camlBacktrace\_\_definitely\_raise\_347.objdump}
      }
    \end{column}
  \end{columns}
\end{frame}

\begin{frame}{Backtraces}{Introducing \funcname{caml\_raise\_exn}}
  \begin{columns}[c]
    \begin{column}{0.5\textwidth}
      \minipage[c][0.66\textheight][s]{\columnwidth}
      \onslide*<1>{
        \begin{itemize}
          \item Raising the exception is performed using \funcname{caml\_raise\_exn} which is
            \begin{itemize}
              \item An assembly function provided by the runtime
              \item Architecture specific
            \end{itemize}
          \item Let's see what it does differently from \funcname{raise\_notrace}\dots
          \item We are about to raise \typename{ExnC} with \funcname{caml\_raise\_exn} from \funcname{definitely\_raise}
        \end{itemize}
      }
      \onslide*<2>{
        \mintobjdump[firstline=2,highlightlines=14]{../src/backtrace/camlBacktrace\_\_definitely\_raise\_347.objdump}
      }
      \vfill
      \mintocaml[firstline=9,lastline=13,highlightlines=12]{../src/backtrace/backtrace.ml}
      \endminipage
    \end{column}
    \begin{column}{0.5\textwidth}
      \centering
      \providecommand\step{1}\input{diagrams/backtrace/backtrace.tex}
    \end{column}
  \end{columns}
\end{frame}

\begin{frame}{Backtraces}{But before that, we need more fields from \typename{caml\_domain\_state}}
  \begin{columns}
    \begin{column}{0.5\textwidth}
      \begin{itemize}
        \item Remember \typename{caml\_domain\_state} the per-domain runtime state?
        \item Some other members are of interest to us now\dots
        \item \localname{backtrace\_active} is used to know if the runtime should collect backtraces
        \item \localname{backtrace\_active} can be set by
          \begin{itemize}
            \item The \funcarg{b} flag in the \funcname{OCAMLRUNPARAM} environment variable
            \item \mintinline{ocaml}{Printexc.record_backtrace : bool -> unit} \footnotemark
          \end{itemize}
      \end{itemize}
    \end{column}
    \begin{column}{0.5\textwidth}
      \begin{listing}
        \mintc[highlightlines={9-11}]{src/ocaml/domain\_state\_backtrace.h}
        \caption{runtime/caml/domain\_state.h}
      \end{listing}
    \end{column}
  \end{columns}
  \footnotetext[1]{https://v2.ocaml.org/api/Printexc.html}
\end{frame}

\newcommand\listCamlRaiseExn[1][]{
  \listasm[firstline=702,lastline=725,#1]{../ocaml/runtime/amd64.S}{runtime/amd64.S:702}}

\begin{frame}{Backtraces}{Walk through \funcname{caml\_raise\_exn}}
  \begin{columns}[T]
    \begin{column}{0.5\textwidth}
      \begin{itemize}
        \item<1-> First checks whether exceptions backtrace should be recorded
        \item<1-> By checking the \funcarg{domain\_state->backtrace\_active} value
        \item<2-> If not, acts as \funcname{raise\_notrace} with \funcname{RESTORE\_EXN\_HANDLER\_OCAML}
          \begin{itemize}
            \item Set \regname{RSP} to \localname{domain\_state->exn\_handler} value
            \item Restore \localname{domain\_state->exn\_handler} from trap
            \item Jump with \funcname{ret} (same effect as using \funcname{pop} and \funcname{jmp})
          \end{itemize}
        \item<3-> Otherwise, switches to the C stack to be able to execute C code
        \item<4-> Calls \funcname{caml\_stash\_backtrace}, a C function provided by the runtime\ignorespaces
          \onslide<6->{, with
            \begin{itemize}
              \item \funcarg{exn}: exception value
              \item \funcarg{pc}: code address of the raising point
              \item \funcarg{sp}: stack address of the raising function
              \item \funcarg{trapsp}: stack address of the next handler
            \end{itemize}
          }
      \end{itemize}
      \bigskip
      \mintocaml[firstline=9,lastline=13,highlightlines=12]{../src/backtrace/backtrace.ml}
    \end{column}
    \begin{column}{0.5\textwidth}
      \raggedleft
      \onslide*<1,3-4>{
        \mintc[firstline=190,lastline=192]{../ocaml/runtime/amd64.S}
        \mintc[firstline=234,lastline=237]{../ocaml/runtime/amd64.S}
      }
      \onslide*<2>{
        \mintc[firstline=190,lastline=192,highlightlines={191-192}]{../ocaml/runtime/amd64.S}
        \mintc[firstline=234,lastline=237,highlightlines={235-237}]{../ocaml/runtime/amd64.S}
      }
      \onslide*<1>{\listCamlRaiseExn[highlightlines=705]{}}%
      \onslide*<2>{\listCamlRaiseExn[highlightlines={707-708}]{}}%
      \onslide*<3>{\listCamlRaiseExn[highlightlines={712-714}]{}}%
      \onslide*<4>{\listCamlRaiseExn[highlightlines={715-720}]{}}%
      \onslide*<5>{\providecommand\step{1}\input{diagrams/backtrace/backtrace.tex}}%
      \onslide*<6>{\providecommand\step{2}\input{diagrams/backtrace/backtrace.tex}}%
    \end{column}
  \end{columns}
\end{frame}

\newcommand\listStashBacktrace[1][]{
  \listc[firstline=91,lastline=120,#1]{../ocaml/runtime/backtrace\_nat.c}{runtime/backtrace\_nat.c:91}}

\begin{frame}{Backtraces}{Introducing \funcname{caml\_stash\_backtrace}}
  \begin{columns}[c]
    \begin{column}{0.5\textwidth}
      \minipage[c][0.66\textheight][s]{\columnwidth}
      \begin{itemize}
        \item<1-> A C function provided by the runtime (specific to native code)
        \item<2-> It scans the stack, between the stack pointer at the raising point up to the next trap, in order to collect the names of the detected function frames
          \onslide*<2>{
            \begin{itemize}
              \item And more information, like line number, column number...
            \end{itemize}
          }
        \item<3-> It uses the same technique and information as the Garbage Collector (GC) uses to scan the values live on the stack
          \onslide*<4>{
            \begin{itemize}
              \item \typename{frame\_descr} are at the core of stack unwinding
            \end{itemize}
          }
      \end{itemize}
      \vfill
      \mintocaml[firstline=9,lastline=13,highlightlines=12]{../src/backtrace/backtrace.ml}
      \endminipage
    \end{column}
    \begin{column}{0.5\textwidth}
      \onslide*<1>{\listStashBacktrace[highlightlines=91]{}}
      \onslide*<2>{\listStashBacktrace[highlightlines={91,117-118}]{}}
      \onslide*<3->{\listStashBacktrace[highlightlines={94,106,109-110}]{}}
    \end{column}
  \end{columns}
\end{frame}

\newcommand\listFrameDescr[1][]{
  \listocaml[firstline=111,lastline=124,#1]{../ocaml/asmcomp/emitaux.ml}{asmcomp/emitaux.ml:111}}
\newcommand\listDebugInfo[1][]{
  \listocaml[firstline=38,lastline=49,#1]{../ocaml/lambda/debuginfo.mli}{lambda/debuginfo.mli:38}}

\begin{frame}{Backtraces}{\typename{frame\_descr} and \typename{frame\_debuginfo} in a nutshell}
  \begin{columns}[c]
    \begin{column}{0.5\textwidth}
      \begin{itemize}
        \item<1-> For every return address in an OCaml program, there is a associated \typename{frame\_descr}
        \item<2-> A \typename{frame\_descr} specifies
        \begin{itemize}
          \item<2-> The size of the function stack frame
          \item<3-> Where live values are on the stack (so that the GC can mark them as live and prevent from being swept)
        \end{itemize}
        \item<4-> When compiled with debug information, \typename{frame\_descr} will be linked to a \typename{frame\_debuginfo}
        \begin{itemize}
          \item<5-> Contains information about the position in the source code (file name, line and column number)
        \end{itemize}
      \end{itemize}
    \end{column}
    \begin{column}{0.5\textwidth}
        \onslide*<1>{\listFrameDescr[highlightlines={118-122}]{}}
        \onslide*<2>{\listFrameDescr[highlightlines={120}]{}}
        \onslide*<3>{\listFrameDescr[highlightlines={121}]{}}
        \onslide*<4->{\listFrameDescr[highlightlines={122}]{}}
        \medskip
        \onslide*<1-3>{\listDebugInfo{}}
        \onslide*<4>{\listDebugInfo[highlightlines={38,49}]{}}
        \onslide*<5>{\listDebugInfo[highlightlines={39-46}]{}}
    \end{column}
  \end{columns}
\end{frame}

\begin{frame}{Backtraces}{Scanning the stack with \typename{frame\_descr}}
  \begin{columns}[c]
    \begin{column}{0.5\textwidth}
      \begin{itemize}
        \item Given the return address of the code that raises the exception by calling \funcname{caml\_raise\_exn} (\funcarg{pc} argument of \funcname{caml\_stash\_backtrace})
        \item And the position of the stack pointer (\funcarg{sp} argument of \funcname{caml\_stash\_backtrace})
        \item The frame size given by \typename{frame\_descr}.\funcarg{frame\_size}, allows to find the end of the previous frame
        \item And the return address of the caller of this function
        \bigskip
        \onslide<3->{
        \item Note that traps (here the reddish \funcname{ExnA}, \funcname{ExnB} and \funcname{ExnC} blocks) belong to the frame that installed them, and so the frame size changes when traps are removed
        }
      \end{itemize}
    \end{column}
    \begin{column}{0.5\textwidth}
      \raggedleft
      \onslide*<1>{\providecommand\step{2}\input{diagrams/backtrace/backtrace.tex}}%
      \onslide*<2>{\providecommand\step{1}\input{diagrams/backtrace/scanstack.tex}}%
      \onslide*<3>{\providecommand\step{2}\input{diagrams/backtrace/scanstack.tex}}%
      \onslide*<4>{\providecommand\step{3}\input{diagrams/backtrace/scanstack.tex}}%
      \onslide*<5>{\providecommand\step{4}\input{diagrams/backtrace/scanstack.tex}}%
      \onslide*<6>{\providecommand\step{5}\input{diagrams/backtrace/scanstack.tex}}%
    \end{column}
  \end{columns}
\end{frame}

% Duplicate of previous-previous slide
\begin{frame}{Backtraces}{Continuing on \funcname{caml\_stash\_backtrace}}
  \begin{columns}[c]
    \begin{column}{0.5\textwidth}
      \minipage[c][0.75\textheight][s]{\columnwidth}
      \begin{itemize}
          \color{gray}
        \item A C function provided by the runtime (specific to native code)
        \item It scans the stack, between the stack pointer at the raising point up to the next trap, in order to collect the names of the detected function frames
        \item It uses the same technique and information as the Garbage Collector (GC) uses to scan the values live on the stack
      \end{itemize}
      \begin{itemize}
        \item For each function frame detected, store a pointer to the \typename{frame\_descr} in the \localname{domain\_state->backtrace\_buffer} array, effectively constructing the backtrace
      \end{itemize}
      \onslide<2->{
        \begin{itemize}
            \smallskip
          \item Constructed backtrace:
            \funcname{definitely\_raise},
            \onslide<3->{\funcname{probably\_raise}}
        \end{itemize}
      }
      \vfill
      \mintocaml[firstline=9,lastline=13,highlightlines=12]{../src/backtrace/backtrace.ml}
      \endminipage
    \end{column}
    \begin{column}{0.5\textwidth}
      \raggedleft
      \onslide*<1>{\listStashBacktrace[highlightlines={114-115}]{}}
      \onslide*<2>{\providecommand\step{3}\input{diagrams/backtrace/backtrace.tex}}%
      \onslide*<3>{\providecommand\step{4}\input{diagrams/backtrace/backtrace.tex}}%
    \end{column}
  \end{columns}
\end{frame}

\begin{frame}{Backtraces}{Back to \funcname{caml\_raise\_exn}}
  \begin{columns}[c]
    \begin{column}{0.5\textwidth}
      \minipage[c][0.66\textheight][s]{\columnwidth}
      \begin{itemize}\color{gray}
        \item Checks wheteher exceptions backtrace should be recorded, by checking the \funcarg{domain\_state->backtrace\_active} value
        \item Switches to the C stack to be able to execute C code
        \item Calls \funcname{caml\_stash\_backtrace}, to collect the backtrace up to the next trap
      \end{itemize}
      \onslide*<2->{
        \begin{itemize}
          \item<2-> Executes the exception handler in \funcname{probably\_raise} with \funcname{RESTORE\_EXN\_HANDLER\_OCAML}
            \begin{itemize}
              \item Set \regname{RSP} to \localname{domain\_state->exn\_handler} value
              \item Restore \localname{domain\_state->exn\_handler} from trap
              \item Jump to the handler code with \funcname{ret}
            \end{itemize}
        \end{itemize}
      }
      \vfill
      \onslide*<1>{\mintocaml[firstline=9,lastline=13,highlightlines=12]{../src/backtrace/backtrace.ml}}
      \onslide*<2->{\mintocaml[firstline=15,lastline=21,highlightlines={19-21}]{../src/backtrace/backtrace.ml}}
      \endminipage
    \end{column}
    \begin{column}{0.5\textwidth}
      \centering
      \onslide*<1>{
        \mintc[firstline=190,lastline=192]{../ocaml/runtime/amd64.S}
        \mintc[firstline=234,lastline=237]{../ocaml/runtime/amd64.S}
      }
      \onslide*<2>{
        \mintc[firstline=190,lastline=192,highlightlines={191-192}]{../ocaml/runtime/amd64.S}
        \mintc[firstline=234,lastline=237,highlightlines={235-237}]{../ocaml/runtime/amd64.S}
      }
      \onslide*<1>{\listCamlRaiseExn[highlightlines=720]{}}
      \onslide*<2>{\listCamlRaiseExn[highlightlines=722]{}}
      \onslide*<3->{\providecommand\step{5}\input{diagrams/backtrace/backtrace.tex}}
    \end{column}
  \end{columns}
\end{frame}

\begin{frame}{Backtraces}{\funcname{caml\_probably\_raise}'s handler doesn't catch \typename{ExnC}}
  \begin{columns}[c]
    \begin{column}{0.45\textwidth}
      \minipage[c][0.75\textheight][s]{\columnwidth}
      \begin{itemize}
        \item<1-> \funcname{match \dots with}\footnotemark pattern matching can also match exceptions
        \item<1-> The CMM of \funcname{probably\_raise} shows that this \funcname{match \dots with} is implemented with a pattern matching and a \funcname{try \dots with}
        \item<1-> But this one only matches on \typename{ExnA} exception
        \item<2-> Since the pattern matching fails to catch the exception, \funcname{reraise} is called with our current \typename{ExnC} exception
        \item<3-> Disassembly shows that \funcname{reraise} is actually a call to \funcname{caml\_reraise\_exn}, which is also an assembly function provided by the runtime
      \end{itemize}
      \vfill
      \mintocaml[firstline=15,lastline=21,highlightlines={18-21}]{../src/backtrace/backtrace.ml}
      \endminipage
    \end{column}
    \begin{column}{0.55\textwidth}
      \centering
      \onslide*<1>{\mintcmm[highlightlines={10-18}]{../src/backtrace/camlBacktrace\_\_probably\_raise\_351.cmm}}
      \onslide*<2>{\listcmm[highlightlines=18]{../src/backtrace/camlBacktrace\_\_probably\_raise_351.cmm}{CMM of probably\_raise}}
      \onslide*<3>{\mintobjdump[firstline=5,lastline=35,highlightlines=31]{../src/backtrace/camlBacktrace\_\_probably\_raise_351.objdump}}
    \end{column}
  \end{columns}
  \only<1>{\footnotetext[1]{https://blog.janestreet.com/pattern-matching-and-exception-handling-unite/}}
\end{frame}

\newcommand\listCamlReraiseExn[1][]{
  \listasm[firstline=727,lastline=734,#1]{../ocaml/runtime/amd64.S}{runtime/amd64.S:727}}

\begin{frame}{Backtraces}{Introducing \funcname{caml\_reraise\_exn}}
  \begin{columns}[c]
    \begin{column}{0.5\textwidth}
      \minipage[c][0.66\textheight][s]{\columnwidth}
      \begin{itemize}
        \item<1-> The exception have to be reraised to the next handler
        \item<2-> First, checks if the backtrace should be collected with \funcarg{domain\_state->backtrace\_active}
        \only<3>{
        \begin{itemize}
            \item If not, behaves like a \funcname{raise\_notrace} with \funcname{RESTORE\_EXN\_HANDLER\_OCAML}
        \end{itemize}
        }
        \item<4-> Jumps into \funcname{caml\_raise\_exn}
        \item<5-> Calls \funcname{caml\_stash\_backtrace} again, to scan the next part of the stack: from the trap just pop-ed to the next trap
      \end{itemize}
      \vfill
      \mintocaml[firstline=15,lastline=21,highlightlines={18-21}]{../src/backtrace/backtrace.ml}
      \endminipage
    \end{column}
    \begin{column}{0.5\textwidth}
      \raggedleft
      \onslide*<1>{\listCamlReraiseExn{}}
      \onslide*<2>{\listCamlReraiseExn[highlightlines=729]{}}
      \onslide*<3>{
        \mintc[firstline=190,lastline=192,highlightlines={191-192}]{../ocaml/runtime/amd64.S}
        \mintc[firstline=234,lastline=237,highlightlines={235-237}]{../ocaml/runtime/amd64.S}
        \medskip
        \listCamlReraiseExn[highlightlines=731]{}
      }
      \onslide*<4-5>{
        % TODO refactor with \listCamlReraiseExn
        \mintasm[firstline=727,lastline=734,highlightlines=730]{../ocaml/runtime/amd64.S}
        \medskip
      }
      \onslide*<4>{\listCamlRaiseExn[highlightlines=711]{}}
      \onslide*<5>{\listCamlRaiseExn[highlightlines=720]{}}
    \end{column}
  \end{columns}
\end{frame}

\begin{frame}{Backtraces}{Backtrace collection continues}
  \begin{columns}[c]
    \begin{column}{0.55\textwidth}
      \minipage[c][0.75\textheight][s]{\columnwidth}
      \begin{itemize}
        \item<1-> \funcname{caml\_stash\_backtrace} scans the older part of the stack, up to the next trap
        \item<6-> Controls jumps to the nested exception handler in \funcname{maybe\_raise}, which matches only on \typename{ExnB}
        \item<7-> \funcname{caml\_reraise\_exn} is called again, backtrace collection resumes with \funcname{caml\_stash\_backtrace}
      \end{itemize}
      \medskip
      \begin{itemize}
        \item<2-> Constructed backtrace:
        \begin{itemize}
          \item <2-> \funcname{definitely\_raise}
          \item <2-> \funcname{probably\_raise}
          \item <3-> \funcname{probably\_raise} (re-raise)
          \item <4-> \funcname{maybe\_raise}
          \item <5-> \funcname{main}
          \item <8-> \funcname{main} (re-raise)
        \end{itemize}
      \end{itemize}
      \vfill
      \onslide*<1-5>{\mintocaml[firstline=15,lastline=21]{../src/backtrace/backtrace.ml}}
      \onslide*<6>{\mintocaml[firstline=28,lastline=34,highlightlines=31]{../src/backtrace/backtrace.ml}}
      \onslide*<7->{\mintocaml[firstline=28,lastline=34,highlightlines=33]{../src/backtrace/backtrace.ml}}
      \endminipage
    \end{column}
    \begin{column}{0.45\textwidth}
      \raggedleft
      \onslide*<1>{\providecommand\step{6}\input{diagrams/backtrace/backtrace.tex}}%
      \onslide*<2>{\providecommand\step{6}\input{diagrams/backtrace/backtrace.tex}}%
      \onslide*<3>{\providecommand\step{7}\input{diagrams/backtrace/backtrace.tex}}%
      \onslide*<4>{\providecommand\step{8}\input{diagrams/backtrace/backtrace.tex}}%
      \onslide*<5>{\providecommand\step{9}\input{diagrams/backtrace/backtrace.tex}}%
      \onslide*<6>{\providecommand\step{10}\input{diagrams/backtrace/backtrace.tex}}%
      \onslide*<7>{\providecommand\step{11}\input{diagrams/backtrace/backtrace.tex}}%
      \onslide*<8>{\providecommand\step{12}\input{diagrams/backtrace/backtrace.tex}}%
      \onslide*<9>{\providecommand\step{13}\input{diagrams/backtrace/backtrace.tex}}%
    \end{column}
  \end{columns}
\end{frame}

\begin{frame}{Backtraces}{Printing the backtrace}
  \begin{columns}[c]
    \begin{column}{0.45\textwidth}
      \minipage[c][0.66\textheight][s]{\columnwidth}
      \begin{itemize}
        \item<1-> The exception backtrace can now be printed to \funcarg{stdout} with \funcname{Printexc.print_backtrace}
        \item<2-> First the exception backtrace is retrieved into an OCaml array with \funcname{get_raw_backtrace }
        \item<3-> Which is implemented as the \funcname{caml_get_exception_raw_backtrace} C function in the runtime
        \item<4-> And finally printed with \funcname{print_exception_backtrace}
      \end{itemize}
      \vfill
      \mintocaml[firstline=28,lastline=34,highlightlines=33]{../src/backtrace/backtrace.ml}
      \endminipage
    \end{column}
    \begin{column}{0.55\textwidth}
      \onslide*<1,2>{
        \mintocaml[firstline=100,lastline=101]{../ocaml/stdlib/printexc.ml}
      }
      \only<1>{
        \listocaml[firstline=170,lastline=172]{../ocaml/stdlib/printexc.ml}{stdlib/printexc.ml:170}
      }
      \only<2>{
        \listocaml[firstline=170,lastline=172,highlightlines=172]{../ocaml/stdlib/printexc.ml}{stdlib/printexc.ml:170}
      }
      \only<3>{
        \mintc[firstline=154,lastline=156]{../ocaml/runtime/backtrace.c}
        \listc[firstline=171,lastline=192,highlightlines={184-187}]{../ocaml/runtime/backtrace.c}{runtime/backtrace.c:154}
      }
      \only<4>{
        \listocaml[firstline=155,lastline=165]{../ocaml/stdlib/printexc.ml}{stdlib/printexc.ml:55}
      }
    \end{column}
  \end{columns}
\end{frame}


\subsection{The multiple kinds of raise}
\frameSubsection{}{}

\begin{frame}{Backtraces}{When backtraces are not needed}
  \begin{columns}[c]
    \begin{column}{0.45\textwidth}
      \minipage[c][0.66\textheight][s]{\columnwidth}
      \begin{itemize}
        \item<1-> What if the backtrace for an exception is never needed?
        \item<2-> It's possible to raise the exception explicitly with \funcname{raise\_notrace} to make raising as cheap as possible
        \item<3-> An example of use in the \funcname{dynlink} library of the OCaml compiler
      \end{itemize}
      \vfill
      \onslide<3->{
      \listocaml[firstline=205,lastline=209,highlightlines=207]{../ocaml/otherlibs/dynlink/dynlink\_compilerlibs/misc.ml}{otherlibs/dynlink/dynlink\_compilerlibs/misc.ml:205}
      }
      \endminipage
    \end{column}
    \begin{column}{0.55\textwidth}
    \only<4>{
      \mintcmm[highlightlines=3]{../src/notrace/camlNotrace\_\_fun_347.cmm}
      \listcmm{../src/notrace/camlNotrace\_\_all\_somes\_327.cmm}{all\_somes}
    }
    \onslide*<5->{
      \listobjdump[highlightlines={6-9}]{../src/notrace/camlNotrace\_\_fun_347.objdump}{anonymous function from \funcname{all\_somes}}
    }
    \end{column}
  \end{columns}
\end{frame}

\begin{frame}{Backtraces}{Summary table of the multiple kinds of raise}
  \begin{itemize}
    \item When debugging information is enabled and if the backtrace of an exception isn't needed, it's possible to raise with \funcname{raise\_notrace} for performance
    \item When debugging information is \emph{not} enabled, the compiler optimises \funcname{raise} calls to inline assembly (same effect as using \funcname{raise\_notrace})
  \end{itemize}
  \bigskip
  \centering
  \begin{tabular}{| l | c | c | c | c |}
    \hline
    \parbox{10em}{Debug information \\ (\funcarg{-g} flag)}  & \multicolumn{2}{|c|}{Disabled}                                     & \multicolumn{2}{|c|}{Enabled} \\
    \hline
    Raise kind                                               & \funcname{raise} & \funcname{raise\_notrace}                       & \funcname{raise} & \funcname{raise\_notrace} \\
    \hline
    Compilation                                              & \parbox{8em}{inline assembly\\ (optimization)} & inline assembly   & \funcname{caml\_raise\_exn} & inline assembly \\
    \hline
  \end{tabular}
\end{frame}


%
% Takeaway #5
%
\subsection*{Takeaway \#5}
\frameSubsectionTakeaway{}
\againframe<5>{takeaway}
}%
      \onslide*<2>{\providecommand\step{6}%
% Backtraces
%
\subsection{One advantage of raising with debugging information}
\frameSubsection{}{}

\begin{frame}{Backtraces}{Debugging information \& source code}
  \begin{columns}[c]
    \begin{column}{0.5\textwidth}
      \begin{itemize}
        \item Raising an exception is fun and games, but how do we know where the exception originated? $\rightarrow$ With the backtrace associated with the exception!
        \item In order to get a backtrace from the point where an exception was raised, it's necessary to compile with debugging information enabled (\funcarg{-g} flag for the compiler)
        \item Same example as in the `Nested handlers' section
      \end{itemize}
    \end{column}
    \begin{column}{0.5\textwidth}
      \listocaml[firstline=9,lastline=34]{../src/backtrace/backtrace.ml}{backtrace/backtrace.ml}
    \end{column}
  \end{columns}
\end{frame}

\begin{frame}{Backtraces}{CMM and disassembly of \funcname{definitely\_raise}}
  \begin{columns}[c]
    \begin{column}{0.5\textwidth}
      \minipage[c][0.66\textheight][s]{\columnwidth}
      \begin{itemize}
        \item<1-> An effect of compiling with debugging information (\funcarg{-g}): the CMM no longer uses \funcname{raise\_notrace} to raise the exception but \funcname{raise} instead
        \item<2-> Disassembly shows that CMM \funcname{raise} is actually a call to \funcname{caml\_raise\_exn}, a function in assembler provided by the runtime
      \end{itemize}
      \vfill
      \mintocaml[firstline=9,lastline=13,highlightlines=12]{../src/backtrace/backtrace.ml}
      \endminipage
    \end{column}
    \begin{column}{0.5\textwidth}
      \onslide*<1>{
        \mintcmm[highlightlines=5]{../src/backtrace/camlBacktrace\_\_definitely\_raise\_347.cmm}
      }
      \onslide*<2>{
        \mintobjdump[firstline=2,highlightlines=14]{../src/backtrace/camlBacktrace\_\_definitely\_raise\_347.objdump}
      }
    \end{column}
  \end{columns}
\end{frame}

\begin{frame}{Backtraces}{Introducing \funcname{caml\_raise\_exn}}
  \begin{columns}[c]
    \begin{column}{0.5\textwidth}
      \minipage[c][0.66\textheight][s]{\columnwidth}
      \onslide*<1>{
        \begin{itemize}
          \item Raising the exception is performed using \funcname{caml\_raise\_exn} which is
            \begin{itemize}
              \item An assembly function provided by the runtime
              \item Architecture specific
            \end{itemize}
          \item Let's see what it does differently from \funcname{raise\_notrace}\dots
          \item We are about to raise \typename{ExnC} with \funcname{caml\_raise\_exn} from \funcname{definitely\_raise}
        \end{itemize}
      }
      \onslide*<2>{
        \mintobjdump[firstline=2,highlightlines=14]{../src/backtrace/camlBacktrace\_\_definitely\_raise\_347.objdump}
      }
      \vfill
      \mintocaml[firstline=9,lastline=13,highlightlines=12]{../src/backtrace/backtrace.ml}
      \endminipage
    \end{column}
    \begin{column}{0.5\textwidth}
      \centering
      \providecommand\step{1}\input{diagrams/backtrace/backtrace.tex}
    \end{column}
  \end{columns}
\end{frame}

\begin{frame}{Backtraces}{But before that, we need more fields from \typename{caml\_domain\_state}}
  \begin{columns}
    \begin{column}{0.5\textwidth}
      \begin{itemize}
        \item Remember \typename{caml\_domain\_state} the per-domain runtime state?
        \item Some other members are of interest to us now\dots
        \item \localname{backtrace\_active} is used to know if the runtime should collect backtraces
        \item \localname{backtrace\_active} can be set by
          \begin{itemize}
            \item The \funcarg{b} flag in the \funcname{OCAMLRUNPARAM} environment variable
            \item \mintinline{ocaml}{Printexc.record_backtrace : bool -> unit} \footnotemark
          \end{itemize}
      \end{itemize}
    \end{column}
    \begin{column}{0.5\textwidth}
      \begin{listing}
        \mintc[highlightlines={9-11}]{src/ocaml/domain\_state\_backtrace.h}
        \caption{runtime/caml/domain\_state.h}
      \end{listing}
    \end{column}
  \end{columns}
  \footnotetext[1]{https://v2.ocaml.org/api/Printexc.html}
\end{frame}

\newcommand\listCamlRaiseExn[1][]{
  \listasm[firstline=702,lastline=725,#1]{../ocaml/runtime/amd64.S}{runtime/amd64.S:702}}

\begin{frame}{Backtraces}{Walk through \funcname{caml\_raise\_exn}}
  \begin{columns}[T]
    \begin{column}{0.5\textwidth}
      \begin{itemize}
        \item<1-> First checks whether exceptions backtrace should be recorded
        \item<1-> By checking the \funcarg{domain\_state->backtrace\_active} value
        \item<2-> If not, acts as \funcname{raise\_notrace} with \funcname{RESTORE\_EXN\_HANDLER\_OCAML}
          \begin{itemize}
            \item Set \regname{RSP} to \localname{domain\_state->exn\_handler} value
            \item Restore \localname{domain\_state->exn\_handler} from trap
            \item Jump with \funcname{ret} (same effect as using \funcname{pop} and \funcname{jmp})
          \end{itemize}
        \item<3-> Otherwise, switches to the C stack to be able to execute C code
        \item<4-> Calls \funcname{caml\_stash\_backtrace}, a C function provided by the runtime\ignorespaces
          \onslide<6->{, with
            \begin{itemize}
              \item \funcarg{exn}: exception value
              \item \funcarg{pc}: code address of the raising point
              \item \funcarg{sp}: stack address of the raising function
              \item \funcarg{trapsp}: stack address of the next handler
            \end{itemize}
          }
      \end{itemize}
      \bigskip
      \mintocaml[firstline=9,lastline=13,highlightlines=12]{../src/backtrace/backtrace.ml}
    \end{column}
    \begin{column}{0.5\textwidth}
      \raggedleft
      \onslide*<1,3-4>{
        \mintc[firstline=190,lastline=192]{../ocaml/runtime/amd64.S}
        \mintc[firstline=234,lastline=237]{../ocaml/runtime/amd64.S}
      }
      \onslide*<2>{
        \mintc[firstline=190,lastline=192,highlightlines={191-192}]{../ocaml/runtime/amd64.S}
        \mintc[firstline=234,lastline=237,highlightlines={235-237}]{../ocaml/runtime/amd64.S}
      }
      \onslide*<1>{\listCamlRaiseExn[highlightlines=705]{}}%
      \onslide*<2>{\listCamlRaiseExn[highlightlines={707-708}]{}}%
      \onslide*<3>{\listCamlRaiseExn[highlightlines={712-714}]{}}%
      \onslide*<4>{\listCamlRaiseExn[highlightlines={715-720}]{}}%
      \onslide*<5>{\providecommand\step{1}\input{diagrams/backtrace/backtrace.tex}}%
      \onslide*<6>{\providecommand\step{2}\input{diagrams/backtrace/backtrace.tex}}%
    \end{column}
  \end{columns}
\end{frame}

\newcommand\listStashBacktrace[1][]{
  \listc[firstline=91,lastline=120,#1]{../ocaml/runtime/backtrace\_nat.c}{runtime/backtrace\_nat.c:91}}

\begin{frame}{Backtraces}{Introducing \funcname{caml\_stash\_backtrace}}
  \begin{columns}[c]
    \begin{column}{0.5\textwidth}
      \minipage[c][0.66\textheight][s]{\columnwidth}
      \begin{itemize}
        \item<1-> A C function provided by the runtime (specific to native code)
        \item<2-> It scans the stack, between the stack pointer at the raising point up to the next trap, in order to collect the names of the detected function frames
          \onslide*<2>{
            \begin{itemize}
              \item And more information, like line number, column number...
            \end{itemize}
          }
        \item<3-> It uses the same technique and information as the Garbage Collector (GC) uses to scan the values live on the stack
          \onslide*<4>{
            \begin{itemize}
              \item \typename{frame\_descr} are at the core of stack unwinding
            \end{itemize}
          }
      \end{itemize}
      \vfill
      \mintocaml[firstline=9,lastline=13,highlightlines=12]{../src/backtrace/backtrace.ml}
      \endminipage
    \end{column}
    \begin{column}{0.5\textwidth}
      \onslide*<1>{\listStashBacktrace[highlightlines=91]{}}
      \onslide*<2>{\listStashBacktrace[highlightlines={91,117-118}]{}}
      \onslide*<3->{\listStashBacktrace[highlightlines={94,106,109-110}]{}}
    \end{column}
  \end{columns}
\end{frame}

\newcommand\listFrameDescr[1][]{
  \listocaml[firstline=111,lastline=124,#1]{../ocaml/asmcomp/emitaux.ml}{asmcomp/emitaux.ml:111}}
\newcommand\listDebugInfo[1][]{
  \listocaml[firstline=38,lastline=49,#1]{../ocaml/lambda/debuginfo.mli}{lambda/debuginfo.mli:38}}

\begin{frame}{Backtraces}{\typename{frame\_descr} and \typename{frame\_debuginfo} in a nutshell}
  \begin{columns}[c]
    \begin{column}{0.5\textwidth}
      \begin{itemize}
        \item<1-> For every return address in an OCaml program, there is a associated \typename{frame\_descr}
        \item<2-> A \typename{frame\_descr} specifies
        \begin{itemize}
          \item<2-> The size of the function stack frame
          \item<3-> Where live values are on the stack (so that the GC can mark them as live and prevent from being swept)
        \end{itemize}
        \item<4-> When compiled with debug information, \typename{frame\_descr} will be linked to a \typename{frame\_debuginfo}
        \begin{itemize}
          \item<5-> Contains information about the position in the source code (file name, line and column number)
        \end{itemize}
      \end{itemize}
    \end{column}
    \begin{column}{0.5\textwidth}
        \onslide*<1>{\listFrameDescr[highlightlines={118-122}]{}}
        \onslide*<2>{\listFrameDescr[highlightlines={120}]{}}
        \onslide*<3>{\listFrameDescr[highlightlines={121}]{}}
        \onslide*<4->{\listFrameDescr[highlightlines={122}]{}}
        \medskip
        \onslide*<1-3>{\listDebugInfo{}}
        \onslide*<4>{\listDebugInfo[highlightlines={38,49}]{}}
        \onslide*<5>{\listDebugInfo[highlightlines={39-46}]{}}
    \end{column}
  \end{columns}
\end{frame}

\begin{frame}{Backtraces}{Scanning the stack with \typename{frame\_descr}}
  \begin{columns}[c]
    \begin{column}{0.5\textwidth}
      \begin{itemize}
        \item Given the return address of the code that raises the exception by calling \funcname{caml\_raise\_exn} (\funcarg{pc} argument of \funcname{caml\_stash\_backtrace})
        \item And the position of the stack pointer (\funcarg{sp} argument of \funcname{caml\_stash\_backtrace})
        \item The frame size given by \typename{frame\_descr}.\funcarg{frame\_size}, allows to find the end of the previous frame
        \item And the return address of the caller of this function
        \bigskip
        \onslide<3->{
        \item Note that traps (here the reddish \funcname{ExnA}, \funcname{ExnB} and \funcname{ExnC} blocks) belong to the frame that installed them, and so the frame size changes when traps are removed
        }
      \end{itemize}
    \end{column}
    \begin{column}{0.5\textwidth}
      \raggedleft
      \onslide*<1>{\providecommand\step{2}\input{diagrams/backtrace/backtrace.tex}}%
      \onslide*<2>{\providecommand\step{1}\input{diagrams/backtrace/scanstack.tex}}%
      \onslide*<3>{\providecommand\step{2}\input{diagrams/backtrace/scanstack.tex}}%
      \onslide*<4>{\providecommand\step{3}\input{diagrams/backtrace/scanstack.tex}}%
      \onslide*<5>{\providecommand\step{4}\input{diagrams/backtrace/scanstack.tex}}%
      \onslide*<6>{\providecommand\step{5}\input{diagrams/backtrace/scanstack.tex}}%
    \end{column}
  \end{columns}
\end{frame}

% Duplicate of previous-previous slide
\begin{frame}{Backtraces}{Continuing on \funcname{caml\_stash\_backtrace}}
  \begin{columns}[c]
    \begin{column}{0.5\textwidth}
      \minipage[c][0.75\textheight][s]{\columnwidth}
      \begin{itemize}
          \color{gray}
        \item A C function provided by the runtime (specific to native code)
        \item It scans the stack, between the stack pointer at the raising point up to the next trap, in order to collect the names of the detected function frames
        \item It uses the same technique and information as the Garbage Collector (GC) uses to scan the values live on the stack
      \end{itemize}
      \begin{itemize}
        \item For each function frame detected, store a pointer to the \typename{frame\_descr} in the \localname{domain\_state->backtrace\_buffer} array, effectively constructing the backtrace
      \end{itemize}
      \onslide<2->{
        \begin{itemize}
            \smallskip
          \item Constructed backtrace:
            \funcname{definitely\_raise},
            \onslide<3->{\funcname{probably\_raise}}
        \end{itemize}
      }
      \vfill
      \mintocaml[firstline=9,lastline=13,highlightlines=12]{../src/backtrace/backtrace.ml}
      \endminipage
    \end{column}
    \begin{column}{0.5\textwidth}
      \raggedleft
      \onslide*<1>{\listStashBacktrace[highlightlines={114-115}]{}}
      \onslide*<2>{\providecommand\step{3}\input{diagrams/backtrace/backtrace.tex}}%
      \onslide*<3>{\providecommand\step{4}\input{diagrams/backtrace/backtrace.tex}}%
    \end{column}
  \end{columns}
\end{frame}

\begin{frame}{Backtraces}{Back to \funcname{caml\_raise\_exn}}
  \begin{columns}[c]
    \begin{column}{0.5\textwidth}
      \minipage[c][0.66\textheight][s]{\columnwidth}
      \begin{itemize}\color{gray}
        \item Checks wheteher exceptions backtrace should be recorded, by checking the \funcarg{domain\_state->backtrace\_active} value
        \item Switches to the C stack to be able to execute C code
        \item Calls \funcname{caml\_stash\_backtrace}, to collect the backtrace up to the next trap
      \end{itemize}
      \onslide*<2->{
        \begin{itemize}
          \item<2-> Executes the exception handler in \funcname{probably\_raise} with \funcname{RESTORE\_EXN\_HANDLER\_OCAML}
            \begin{itemize}
              \item Set \regname{RSP} to \localname{domain\_state->exn\_handler} value
              \item Restore \localname{domain\_state->exn\_handler} from trap
              \item Jump to the handler code with \funcname{ret}
            \end{itemize}
        \end{itemize}
      }
      \vfill
      \onslide*<1>{\mintocaml[firstline=9,lastline=13,highlightlines=12]{../src/backtrace/backtrace.ml}}
      \onslide*<2->{\mintocaml[firstline=15,lastline=21,highlightlines={19-21}]{../src/backtrace/backtrace.ml}}
      \endminipage
    \end{column}
    \begin{column}{0.5\textwidth}
      \centering
      \onslide*<1>{
        \mintc[firstline=190,lastline=192]{../ocaml/runtime/amd64.S}
        \mintc[firstline=234,lastline=237]{../ocaml/runtime/amd64.S}
      }
      \onslide*<2>{
        \mintc[firstline=190,lastline=192,highlightlines={191-192}]{../ocaml/runtime/amd64.S}
        \mintc[firstline=234,lastline=237,highlightlines={235-237}]{../ocaml/runtime/amd64.S}
      }
      \onslide*<1>{\listCamlRaiseExn[highlightlines=720]{}}
      \onslide*<2>{\listCamlRaiseExn[highlightlines=722]{}}
      \onslide*<3->{\providecommand\step{5}\input{diagrams/backtrace/backtrace.tex}}
    \end{column}
  \end{columns}
\end{frame}

\begin{frame}{Backtraces}{\funcname{caml\_probably\_raise}'s handler doesn't catch \typename{ExnC}}
  \begin{columns}[c]
    \begin{column}{0.45\textwidth}
      \minipage[c][0.75\textheight][s]{\columnwidth}
      \begin{itemize}
        \item<1-> \funcname{match \dots with}\footnotemark pattern matching can also match exceptions
        \item<1-> The CMM of \funcname{probably\_raise} shows that this \funcname{match \dots with} is implemented with a pattern matching and a \funcname{try \dots with}
        \item<1-> But this one only matches on \typename{ExnA} exception
        \item<2-> Since the pattern matching fails to catch the exception, \funcname{reraise} is called with our current \typename{ExnC} exception
        \item<3-> Disassembly shows that \funcname{reraise} is actually a call to \funcname{caml\_reraise\_exn}, which is also an assembly function provided by the runtime
      \end{itemize}
      \vfill
      \mintocaml[firstline=15,lastline=21,highlightlines={18-21}]{../src/backtrace/backtrace.ml}
      \endminipage
    \end{column}
    \begin{column}{0.55\textwidth}
      \centering
      \onslide*<1>{\mintcmm[highlightlines={10-18}]{../src/backtrace/camlBacktrace\_\_probably\_raise\_351.cmm}}
      \onslide*<2>{\listcmm[highlightlines=18]{../src/backtrace/camlBacktrace\_\_probably\_raise_351.cmm}{CMM of probably\_raise}}
      \onslide*<3>{\mintobjdump[firstline=5,lastline=35,highlightlines=31]{../src/backtrace/camlBacktrace\_\_probably\_raise_351.objdump}}
    \end{column}
  \end{columns}
  \only<1>{\footnotetext[1]{https://blog.janestreet.com/pattern-matching-and-exception-handling-unite/}}
\end{frame}

\newcommand\listCamlReraiseExn[1][]{
  \listasm[firstline=727,lastline=734,#1]{../ocaml/runtime/amd64.S}{runtime/amd64.S:727}}

\begin{frame}{Backtraces}{Introducing \funcname{caml\_reraise\_exn}}
  \begin{columns}[c]
    \begin{column}{0.5\textwidth}
      \minipage[c][0.66\textheight][s]{\columnwidth}
      \begin{itemize}
        \item<1-> The exception have to be reraised to the next handler
        \item<2-> First, checks if the backtrace should be collected with \funcarg{domain\_state->backtrace\_active}
        \only<3>{
        \begin{itemize}
            \item If not, behaves like a \funcname{raise\_notrace} with \funcname{RESTORE\_EXN\_HANDLER\_OCAML}
        \end{itemize}
        }
        \item<4-> Jumps into \funcname{caml\_raise\_exn}
        \item<5-> Calls \funcname{caml\_stash\_backtrace} again, to scan the next part of the stack: from the trap just pop-ed to the next trap
      \end{itemize}
      \vfill
      \mintocaml[firstline=15,lastline=21,highlightlines={18-21}]{../src/backtrace/backtrace.ml}
      \endminipage
    \end{column}
    \begin{column}{0.5\textwidth}
      \raggedleft
      \onslide*<1>{\listCamlReraiseExn{}}
      \onslide*<2>{\listCamlReraiseExn[highlightlines=729]{}}
      \onslide*<3>{
        \mintc[firstline=190,lastline=192,highlightlines={191-192}]{../ocaml/runtime/amd64.S}
        \mintc[firstline=234,lastline=237,highlightlines={235-237}]{../ocaml/runtime/amd64.S}
        \medskip
        \listCamlReraiseExn[highlightlines=731]{}
      }
      \onslide*<4-5>{
        % TODO refactor with \listCamlReraiseExn
        \mintasm[firstline=727,lastline=734,highlightlines=730]{../ocaml/runtime/amd64.S}
        \medskip
      }
      \onslide*<4>{\listCamlRaiseExn[highlightlines=711]{}}
      \onslide*<5>{\listCamlRaiseExn[highlightlines=720]{}}
    \end{column}
  \end{columns}
\end{frame}

\begin{frame}{Backtraces}{Backtrace collection continues}
  \begin{columns}[c]
    \begin{column}{0.55\textwidth}
      \minipage[c][0.75\textheight][s]{\columnwidth}
      \begin{itemize}
        \item<1-> \funcname{caml\_stash\_backtrace} scans the older part of the stack, up to the next trap
        \item<6-> Controls jumps to the nested exception handler in \funcname{maybe\_raise}, which matches only on \typename{ExnB}
        \item<7-> \funcname{caml\_reraise\_exn} is called again, backtrace collection resumes with \funcname{caml\_stash\_backtrace}
      \end{itemize}
      \medskip
      \begin{itemize}
        \item<2-> Constructed backtrace:
        \begin{itemize}
          \item <2-> \funcname{definitely\_raise}
          \item <2-> \funcname{probably\_raise}
          \item <3-> \funcname{probably\_raise} (re-raise)
          \item <4-> \funcname{maybe\_raise}
          \item <5-> \funcname{main}
          \item <8-> \funcname{main} (re-raise)
        \end{itemize}
      \end{itemize}
      \vfill
      \onslide*<1-5>{\mintocaml[firstline=15,lastline=21]{../src/backtrace/backtrace.ml}}
      \onslide*<6>{\mintocaml[firstline=28,lastline=34,highlightlines=31]{../src/backtrace/backtrace.ml}}
      \onslide*<7->{\mintocaml[firstline=28,lastline=34,highlightlines=33]{../src/backtrace/backtrace.ml}}
      \endminipage
    \end{column}
    \begin{column}{0.45\textwidth}
      \raggedleft
      \onslide*<1>{\providecommand\step{6}\input{diagrams/backtrace/backtrace.tex}}%
      \onslide*<2>{\providecommand\step{6}\input{diagrams/backtrace/backtrace.tex}}%
      \onslide*<3>{\providecommand\step{7}\input{diagrams/backtrace/backtrace.tex}}%
      \onslide*<4>{\providecommand\step{8}\input{diagrams/backtrace/backtrace.tex}}%
      \onslide*<5>{\providecommand\step{9}\input{diagrams/backtrace/backtrace.tex}}%
      \onslide*<6>{\providecommand\step{10}\input{diagrams/backtrace/backtrace.tex}}%
      \onslide*<7>{\providecommand\step{11}\input{diagrams/backtrace/backtrace.tex}}%
      \onslide*<8>{\providecommand\step{12}\input{diagrams/backtrace/backtrace.tex}}%
      \onslide*<9>{\providecommand\step{13}\input{diagrams/backtrace/backtrace.tex}}%
    \end{column}
  \end{columns}
\end{frame}

\begin{frame}{Backtraces}{Printing the backtrace}
  \begin{columns}[c]
    \begin{column}{0.45\textwidth}
      \minipage[c][0.66\textheight][s]{\columnwidth}
      \begin{itemize}
        \item<1-> The exception backtrace can now be printed to \funcarg{stdout} with \funcname{Printexc.print_backtrace}
        \item<2-> First the exception backtrace is retrieved into an OCaml array with \funcname{get_raw_backtrace }
        \item<3-> Which is implemented as the \funcname{caml_get_exception_raw_backtrace} C function in the runtime
        \item<4-> And finally printed with \funcname{print_exception_backtrace}
      \end{itemize}
      \vfill
      \mintocaml[firstline=28,lastline=34,highlightlines=33]{../src/backtrace/backtrace.ml}
      \endminipage
    \end{column}
    \begin{column}{0.55\textwidth}
      \onslide*<1,2>{
        \mintocaml[firstline=100,lastline=101]{../ocaml/stdlib/printexc.ml}
      }
      \only<1>{
        \listocaml[firstline=170,lastline=172]{../ocaml/stdlib/printexc.ml}{stdlib/printexc.ml:170}
      }
      \only<2>{
        \listocaml[firstline=170,lastline=172,highlightlines=172]{../ocaml/stdlib/printexc.ml}{stdlib/printexc.ml:170}
      }
      \only<3>{
        \mintc[firstline=154,lastline=156]{../ocaml/runtime/backtrace.c}
        \listc[firstline=171,lastline=192,highlightlines={184-187}]{../ocaml/runtime/backtrace.c}{runtime/backtrace.c:154}
      }
      \only<4>{
        \listocaml[firstline=155,lastline=165]{../ocaml/stdlib/printexc.ml}{stdlib/printexc.ml:55}
      }
    \end{column}
  \end{columns}
\end{frame}


\subsection{The multiple kinds of raise}
\frameSubsection{}{}

\begin{frame}{Backtraces}{When backtraces are not needed}
  \begin{columns}[c]
    \begin{column}{0.45\textwidth}
      \minipage[c][0.66\textheight][s]{\columnwidth}
      \begin{itemize}
        \item<1-> What if the backtrace for an exception is never needed?
        \item<2-> It's possible to raise the exception explicitly with \funcname{raise\_notrace} to make raising as cheap as possible
        \item<3-> An example of use in the \funcname{dynlink} library of the OCaml compiler
      \end{itemize}
      \vfill
      \onslide<3->{
      \listocaml[firstline=205,lastline=209,highlightlines=207]{../ocaml/otherlibs/dynlink/dynlink\_compilerlibs/misc.ml}{otherlibs/dynlink/dynlink\_compilerlibs/misc.ml:205}
      }
      \endminipage
    \end{column}
    \begin{column}{0.55\textwidth}
    \only<4>{
      \mintcmm[highlightlines=3]{../src/notrace/camlNotrace\_\_fun_347.cmm}
      \listcmm{../src/notrace/camlNotrace\_\_all\_somes\_327.cmm}{all\_somes}
    }
    \onslide*<5->{
      \listobjdump[highlightlines={6-9}]{../src/notrace/camlNotrace\_\_fun_347.objdump}{anonymous function from \funcname{all\_somes}}
    }
    \end{column}
  \end{columns}
\end{frame}

\begin{frame}{Backtraces}{Summary table of the multiple kinds of raise}
  \begin{itemize}
    \item When debugging information is enabled and if the backtrace of an exception isn't needed, it's possible to raise with \funcname{raise\_notrace} for performance
    \item When debugging information is \emph{not} enabled, the compiler optimises \funcname{raise} calls to inline assembly (same effect as using \funcname{raise\_notrace})
  \end{itemize}
  \bigskip
  \centering
  \begin{tabular}{| l | c | c | c | c |}
    \hline
    \parbox{10em}{Debug information \\ (\funcarg{-g} flag)}  & \multicolumn{2}{|c|}{Disabled}                                     & \multicolumn{2}{|c|}{Enabled} \\
    \hline
    Raise kind                                               & \funcname{raise} & \funcname{raise\_notrace}                       & \funcname{raise} & \funcname{raise\_notrace} \\
    \hline
    Compilation                                              & \parbox{8em}{inline assembly\\ (optimization)} & inline assembly   & \funcname{caml\_raise\_exn} & inline assembly \\
    \hline
  \end{tabular}
\end{frame}


%
% Takeaway #5
%
\subsection*{Takeaway \#5}
\frameSubsectionTakeaway{}
\againframe<5>{takeaway}
}%
      \onslide*<3>{\providecommand\step{7}%
% Backtraces
%
\subsection{One advantage of raising with debugging information}
\frameSubsection{}{}

\begin{frame}{Backtraces}{Debugging information \& source code}
  \begin{columns}[c]
    \begin{column}{0.5\textwidth}
      \begin{itemize}
        \item Raising an exception is fun and games, but how do we know where the exception originated? $\rightarrow$ With the backtrace associated with the exception!
        \item In order to get a backtrace from the point where an exception was raised, it's necessary to compile with debugging information enabled (\funcarg{-g} flag for the compiler)
        \item Same example as in the `Nested handlers' section
      \end{itemize}
    \end{column}
    \begin{column}{0.5\textwidth}
      \listocaml[firstline=9,lastline=34]{../src/backtrace/backtrace.ml}{backtrace/backtrace.ml}
    \end{column}
  \end{columns}
\end{frame}

\begin{frame}{Backtraces}{CMM and disassembly of \funcname{definitely\_raise}}
  \begin{columns}[c]
    \begin{column}{0.5\textwidth}
      \minipage[c][0.66\textheight][s]{\columnwidth}
      \begin{itemize}
        \item<1-> An effect of compiling with debugging information (\funcarg{-g}): the CMM no longer uses \funcname{raise\_notrace} to raise the exception but \funcname{raise} instead
        \item<2-> Disassembly shows that CMM \funcname{raise} is actually a call to \funcname{caml\_raise\_exn}, a function in assembler provided by the runtime
      \end{itemize}
      \vfill
      \mintocaml[firstline=9,lastline=13,highlightlines=12]{../src/backtrace/backtrace.ml}
      \endminipage
    \end{column}
    \begin{column}{0.5\textwidth}
      \onslide*<1>{
        \mintcmm[highlightlines=5]{../src/backtrace/camlBacktrace\_\_definitely\_raise\_347.cmm}
      }
      \onslide*<2>{
        \mintobjdump[firstline=2,highlightlines=14]{../src/backtrace/camlBacktrace\_\_definitely\_raise\_347.objdump}
      }
    \end{column}
  \end{columns}
\end{frame}

\begin{frame}{Backtraces}{Introducing \funcname{caml\_raise\_exn}}
  \begin{columns}[c]
    \begin{column}{0.5\textwidth}
      \minipage[c][0.66\textheight][s]{\columnwidth}
      \onslide*<1>{
        \begin{itemize}
          \item Raising the exception is performed using \funcname{caml\_raise\_exn} which is
            \begin{itemize}
              \item An assembly function provided by the runtime
              \item Architecture specific
            \end{itemize}
          \item Let's see what it does differently from \funcname{raise\_notrace}\dots
          \item We are about to raise \typename{ExnC} with \funcname{caml\_raise\_exn} from \funcname{definitely\_raise}
        \end{itemize}
      }
      \onslide*<2>{
        \mintobjdump[firstline=2,highlightlines=14]{../src/backtrace/camlBacktrace\_\_definitely\_raise\_347.objdump}
      }
      \vfill
      \mintocaml[firstline=9,lastline=13,highlightlines=12]{../src/backtrace/backtrace.ml}
      \endminipage
    \end{column}
    \begin{column}{0.5\textwidth}
      \centering
      \providecommand\step{1}\input{diagrams/backtrace/backtrace.tex}
    \end{column}
  \end{columns}
\end{frame}

\begin{frame}{Backtraces}{But before that, we need more fields from \typename{caml\_domain\_state}}
  \begin{columns}
    \begin{column}{0.5\textwidth}
      \begin{itemize}
        \item Remember \typename{caml\_domain\_state} the per-domain runtime state?
        \item Some other members are of interest to us now\dots
        \item \localname{backtrace\_active} is used to know if the runtime should collect backtraces
        \item \localname{backtrace\_active} can be set by
          \begin{itemize}
            \item The \funcarg{b} flag in the \funcname{OCAMLRUNPARAM} environment variable
            \item \mintinline{ocaml}{Printexc.record_backtrace : bool -> unit} \footnotemark
          \end{itemize}
      \end{itemize}
    \end{column}
    \begin{column}{0.5\textwidth}
      \begin{listing}
        \mintc[highlightlines={9-11}]{src/ocaml/domain\_state\_backtrace.h}
        \caption{runtime/caml/domain\_state.h}
      \end{listing}
    \end{column}
  \end{columns}
  \footnotetext[1]{https://v2.ocaml.org/api/Printexc.html}
\end{frame}

\newcommand\listCamlRaiseExn[1][]{
  \listasm[firstline=702,lastline=725,#1]{../ocaml/runtime/amd64.S}{runtime/amd64.S:702}}

\begin{frame}{Backtraces}{Walk through \funcname{caml\_raise\_exn}}
  \begin{columns}[T]
    \begin{column}{0.5\textwidth}
      \begin{itemize}
        \item<1-> First checks whether exceptions backtrace should be recorded
        \item<1-> By checking the \funcarg{domain\_state->backtrace\_active} value
        \item<2-> If not, acts as \funcname{raise\_notrace} with \funcname{RESTORE\_EXN\_HANDLER\_OCAML}
          \begin{itemize}
            \item Set \regname{RSP} to \localname{domain\_state->exn\_handler} value
            \item Restore \localname{domain\_state->exn\_handler} from trap
            \item Jump with \funcname{ret} (same effect as using \funcname{pop} and \funcname{jmp})
          \end{itemize}
        \item<3-> Otherwise, switches to the C stack to be able to execute C code
        \item<4-> Calls \funcname{caml\_stash\_backtrace}, a C function provided by the runtime\ignorespaces
          \onslide<6->{, with
            \begin{itemize}
              \item \funcarg{exn}: exception value
              \item \funcarg{pc}: code address of the raising point
              \item \funcarg{sp}: stack address of the raising function
              \item \funcarg{trapsp}: stack address of the next handler
            \end{itemize}
          }
      \end{itemize}
      \bigskip
      \mintocaml[firstline=9,lastline=13,highlightlines=12]{../src/backtrace/backtrace.ml}
    \end{column}
    \begin{column}{0.5\textwidth}
      \raggedleft
      \onslide*<1,3-4>{
        \mintc[firstline=190,lastline=192]{../ocaml/runtime/amd64.S}
        \mintc[firstline=234,lastline=237]{../ocaml/runtime/amd64.S}
      }
      \onslide*<2>{
        \mintc[firstline=190,lastline=192,highlightlines={191-192}]{../ocaml/runtime/amd64.S}
        \mintc[firstline=234,lastline=237,highlightlines={235-237}]{../ocaml/runtime/amd64.S}
      }
      \onslide*<1>{\listCamlRaiseExn[highlightlines=705]{}}%
      \onslide*<2>{\listCamlRaiseExn[highlightlines={707-708}]{}}%
      \onslide*<3>{\listCamlRaiseExn[highlightlines={712-714}]{}}%
      \onslide*<4>{\listCamlRaiseExn[highlightlines={715-720}]{}}%
      \onslide*<5>{\providecommand\step{1}\input{diagrams/backtrace/backtrace.tex}}%
      \onslide*<6>{\providecommand\step{2}\input{diagrams/backtrace/backtrace.tex}}%
    \end{column}
  \end{columns}
\end{frame}

\newcommand\listStashBacktrace[1][]{
  \listc[firstline=91,lastline=120,#1]{../ocaml/runtime/backtrace\_nat.c}{runtime/backtrace\_nat.c:91}}

\begin{frame}{Backtraces}{Introducing \funcname{caml\_stash\_backtrace}}
  \begin{columns}[c]
    \begin{column}{0.5\textwidth}
      \minipage[c][0.66\textheight][s]{\columnwidth}
      \begin{itemize}
        \item<1-> A C function provided by the runtime (specific to native code)
        \item<2-> It scans the stack, between the stack pointer at the raising point up to the next trap, in order to collect the names of the detected function frames
          \onslide*<2>{
            \begin{itemize}
              \item And more information, like line number, column number...
            \end{itemize}
          }
        \item<3-> It uses the same technique and information as the Garbage Collector (GC) uses to scan the values live on the stack
          \onslide*<4>{
            \begin{itemize}
              \item \typename{frame\_descr} are at the core of stack unwinding
            \end{itemize}
          }
      \end{itemize}
      \vfill
      \mintocaml[firstline=9,lastline=13,highlightlines=12]{../src/backtrace/backtrace.ml}
      \endminipage
    \end{column}
    \begin{column}{0.5\textwidth}
      \onslide*<1>{\listStashBacktrace[highlightlines=91]{}}
      \onslide*<2>{\listStashBacktrace[highlightlines={91,117-118}]{}}
      \onslide*<3->{\listStashBacktrace[highlightlines={94,106,109-110}]{}}
    \end{column}
  \end{columns}
\end{frame}

\newcommand\listFrameDescr[1][]{
  \listocaml[firstline=111,lastline=124,#1]{../ocaml/asmcomp/emitaux.ml}{asmcomp/emitaux.ml:111}}
\newcommand\listDebugInfo[1][]{
  \listocaml[firstline=38,lastline=49,#1]{../ocaml/lambda/debuginfo.mli}{lambda/debuginfo.mli:38}}

\begin{frame}{Backtraces}{\typename{frame\_descr} and \typename{frame\_debuginfo} in a nutshell}
  \begin{columns}[c]
    \begin{column}{0.5\textwidth}
      \begin{itemize}
        \item<1-> For every return address in an OCaml program, there is a associated \typename{frame\_descr}
        \item<2-> A \typename{frame\_descr} specifies
        \begin{itemize}
          \item<2-> The size of the function stack frame
          \item<3-> Where live values are on the stack (so that the GC can mark them as live and prevent from being swept)
        \end{itemize}
        \item<4-> When compiled with debug information, \typename{frame\_descr} will be linked to a \typename{frame\_debuginfo}
        \begin{itemize}
          \item<5-> Contains information about the position in the source code (file name, line and column number)
        \end{itemize}
      \end{itemize}
    \end{column}
    \begin{column}{0.5\textwidth}
        \onslide*<1>{\listFrameDescr[highlightlines={118-122}]{}}
        \onslide*<2>{\listFrameDescr[highlightlines={120}]{}}
        \onslide*<3>{\listFrameDescr[highlightlines={121}]{}}
        \onslide*<4->{\listFrameDescr[highlightlines={122}]{}}
        \medskip
        \onslide*<1-3>{\listDebugInfo{}}
        \onslide*<4>{\listDebugInfo[highlightlines={38,49}]{}}
        \onslide*<5>{\listDebugInfo[highlightlines={39-46}]{}}
    \end{column}
  \end{columns}
\end{frame}

\begin{frame}{Backtraces}{Scanning the stack with \typename{frame\_descr}}
  \begin{columns}[c]
    \begin{column}{0.5\textwidth}
      \begin{itemize}
        \item Given the return address of the code that raises the exception by calling \funcname{caml\_raise\_exn} (\funcarg{pc} argument of \funcname{caml\_stash\_backtrace})
        \item And the position of the stack pointer (\funcarg{sp} argument of \funcname{caml\_stash\_backtrace})
        \item The frame size given by \typename{frame\_descr}.\funcarg{frame\_size}, allows to find the end of the previous frame
        \item And the return address of the caller of this function
        \bigskip
        \onslide<3->{
        \item Note that traps (here the reddish \funcname{ExnA}, \funcname{ExnB} and \funcname{ExnC} blocks) belong to the frame that installed them, and so the frame size changes when traps are removed
        }
      \end{itemize}
    \end{column}
    \begin{column}{0.5\textwidth}
      \raggedleft
      \onslide*<1>{\providecommand\step{2}\input{diagrams/backtrace/backtrace.tex}}%
      \onslide*<2>{\providecommand\step{1}\input{diagrams/backtrace/scanstack.tex}}%
      \onslide*<3>{\providecommand\step{2}\input{diagrams/backtrace/scanstack.tex}}%
      \onslide*<4>{\providecommand\step{3}\input{diagrams/backtrace/scanstack.tex}}%
      \onslide*<5>{\providecommand\step{4}\input{diagrams/backtrace/scanstack.tex}}%
      \onslide*<6>{\providecommand\step{5}\input{diagrams/backtrace/scanstack.tex}}%
    \end{column}
  \end{columns}
\end{frame}

% Duplicate of previous-previous slide
\begin{frame}{Backtraces}{Continuing on \funcname{caml\_stash\_backtrace}}
  \begin{columns}[c]
    \begin{column}{0.5\textwidth}
      \minipage[c][0.75\textheight][s]{\columnwidth}
      \begin{itemize}
          \color{gray}
        \item A C function provided by the runtime (specific to native code)
        \item It scans the stack, between the stack pointer at the raising point up to the next trap, in order to collect the names of the detected function frames
        \item It uses the same technique and information as the Garbage Collector (GC) uses to scan the values live on the stack
      \end{itemize}
      \begin{itemize}
        \item For each function frame detected, store a pointer to the \typename{frame\_descr} in the \localname{domain\_state->backtrace\_buffer} array, effectively constructing the backtrace
      \end{itemize}
      \onslide<2->{
        \begin{itemize}
            \smallskip
          \item Constructed backtrace:
            \funcname{definitely\_raise},
            \onslide<3->{\funcname{probably\_raise}}
        \end{itemize}
      }
      \vfill
      \mintocaml[firstline=9,lastline=13,highlightlines=12]{../src/backtrace/backtrace.ml}
      \endminipage
    \end{column}
    \begin{column}{0.5\textwidth}
      \raggedleft
      \onslide*<1>{\listStashBacktrace[highlightlines={114-115}]{}}
      \onslide*<2>{\providecommand\step{3}\input{diagrams/backtrace/backtrace.tex}}%
      \onslide*<3>{\providecommand\step{4}\input{diagrams/backtrace/backtrace.tex}}%
    \end{column}
  \end{columns}
\end{frame}

\begin{frame}{Backtraces}{Back to \funcname{caml\_raise\_exn}}
  \begin{columns}[c]
    \begin{column}{0.5\textwidth}
      \minipage[c][0.66\textheight][s]{\columnwidth}
      \begin{itemize}\color{gray}
        \item Checks wheteher exceptions backtrace should be recorded, by checking the \funcarg{domain\_state->backtrace\_active} value
        \item Switches to the C stack to be able to execute C code
        \item Calls \funcname{caml\_stash\_backtrace}, to collect the backtrace up to the next trap
      \end{itemize}
      \onslide*<2->{
        \begin{itemize}
          \item<2-> Executes the exception handler in \funcname{probably\_raise} with \funcname{RESTORE\_EXN\_HANDLER\_OCAML}
            \begin{itemize}
              \item Set \regname{RSP} to \localname{domain\_state->exn\_handler} value
              \item Restore \localname{domain\_state->exn\_handler} from trap
              \item Jump to the handler code with \funcname{ret}
            \end{itemize}
        \end{itemize}
      }
      \vfill
      \onslide*<1>{\mintocaml[firstline=9,lastline=13,highlightlines=12]{../src/backtrace/backtrace.ml}}
      \onslide*<2->{\mintocaml[firstline=15,lastline=21,highlightlines={19-21}]{../src/backtrace/backtrace.ml}}
      \endminipage
    \end{column}
    \begin{column}{0.5\textwidth}
      \centering
      \onslide*<1>{
        \mintc[firstline=190,lastline=192]{../ocaml/runtime/amd64.S}
        \mintc[firstline=234,lastline=237]{../ocaml/runtime/amd64.S}
      }
      \onslide*<2>{
        \mintc[firstline=190,lastline=192,highlightlines={191-192}]{../ocaml/runtime/amd64.S}
        \mintc[firstline=234,lastline=237,highlightlines={235-237}]{../ocaml/runtime/amd64.S}
      }
      \onslide*<1>{\listCamlRaiseExn[highlightlines=720]{}}
      \onslide*<2>{\listCamlRaiseExn[highlightlines=722]{}}
      \onslide*<3->{\providecommand\step{5}\input{diagrams/backtrace/backtrace.tex}}
    \end{column}
  \end{columns}
\end{frame}

\begin{frame}{Backtraces}{\funcname{caml\_probably\_raise}'s handler doesn't catch \typename{ExnC}}
  \begin{columns}[c]
    \begin{column}{0.45\textwidth}
      \minipage[c][0.75\textheight][s]{\columnwidth}
      \begin{itemize}
        \item<1-> \funcname{match \dots with}\footnotemark pattern matching can also match exceptions
        \item<1-> The CMM of \funcname{probably\_raise} shows that this \funcname{match \dots with} is implemented with a pattern matching and a \funcname{try \dots with}
        \item<1-> But this one only matches on \typename{ExnA} exception
        \item<2-> Since the pattern matching fails to catch the exception, \funcname{reraise} is called with our current \typename{ExnC} exception
        \item<3-> Disassembly shows that \funcname{reraise} is actually a call to \funcname{caml\_reraise\_exn}, which is also an assembly function provided by the runtime
      \end{itemize}
      \vfill
      \mintocaml[firstline=15,lastline=21,highlightlines={18-21}]{../src/backtrace/backtrace.ml}
      \endminipage
    \end{column}
    \begin{column}{0.55\textwidth}
      \centering
      \onslide*<1>{\mintcmm[highlightlines={10-18}]{../src/backtrace/camlBacktrace\_\_probably\_raise\_351.cmm}}
      \onslide*<2>{\listcmm[highlightlines=18]{../src/backtrace/camlBacktrace\_\_probably\_raise_351.cmm}{CMM of probably\_raise}}
      \onslide*<3>{\mintobjdump[firstline=5,lastline=35,highlightlines=31]{../src/backtrace/camlBacktrace\_\_probably\_raise_351.objdump}}
    \end{column}
  \end{columns}
  \only<1>{\footnotetext[1]{https://blog.janestreet.com/pattern-matching-and-exception-handling-unite/}}
\end{frame}

\newcommand\listCamlReraiseExn[1][]{
  \listasm[firstline=727,lastline=734,#1]{../ocaml/runtime/amd64.S}{runtime/amd64.S:727}}

\begin{frame}{Backtraces}{Introducing \funcname{caml\_reraise\_exn}}
  \begin{columns}[c]
    \begin{column}{0.5\textwidth}
      \minipage[c][0.66\textheight][s]{\columnwidth}
      \begin{itemize}
        \item<1-> The exception have to be reraised to the next handler
        \item<2-> First, checks if the backtrace should be collected with \funcarg{domain\_state->backtrace\_active}
        \only<3>{
        \begin{itemize}
            \item If not, behaves like a \funcname{raise\_notrace} with \funcname{RESTORE\_EXN\_HANDLER\_OCAML}
        \end{itemize}
        }
        \item<4-> Jumps into \funcname{caml\_raise\_exn}
        \item<5-> Calls \funcname{caml\_stash\_backtrace} again, to scan the next part of the stack: from the trap just pop-ed to the next trap
      \end{itemize}
      \vfill
      \mintocaml[firstline=15,lastline=21,highlightlines={18-21}]{../src/backtrace/backtrace.ml}
      \endminipage
    \end{column}
    \begin{column}{0.5\textwidth}
      \raggedleft
      \onslide*<1>{\listCamlReraiseExn{}}
      \onslide*<2>{\listCamlReraiseExn[highlightlines=729]{}}
      \onslide*<3>{
        \mintc[firstline=190,lastline=192,highlightlines={191-192}]{../ocaml/runtime/amd64.S}
        \mintc[firstline=234,lastline=237,highlightlines={235-237}]{../ocaml/runtime/amd64.S}
        \medskip
        \listCamlReraiseExn[highlightlines=731]{}
      }
      \onslide*<4-5>{
        % TODO refactor with \listCamlReraiseExn
        \mintasm[firstline=727,lastline=734,highlightlines=730]{../ocaml/runtime/amd64.S}
        \medskip
      }
      \onslide*<4>{\listCamlRaiseExn[highlightlines=711]{}}
      \onslide*<5>{\listCamlRaiseExn[highlightlines=720]{}}
    \end{column}
  \end{columns}
\end{frame}

\begin{frame}{Backtraces}{Backtrace collection continues}
  \begin{columns}[c]
    \begin{column}{0.55\textwidth}
      \minipage[c][0.75\textheight][s]{\columnwidth}
      \begin{itemize}
        \item<1-> \funcname{caml\_stash\_backtrace} scans the older part of the stack, up to the next trap
        \item<6-> Controls jumps to the nested exception handler in \funcname{maybe\_raise}, which matches only on \typename{ExnB}
        \item<7-> \funcname{caml\_reraise\_exn} is called again, backtrace collection resumes with \funcname{caml\_stash\_backtrace}
      \end{itemize}
      \medskip
      \begin{itemize}
        \item<2-> Constructed backtrace:
        \begin{itemize}
          \item <2-> \funcname{definitely\_raise}
          \item <2-> \funcname{probably\_raise}
          \item <3-> \funcname{probably\_raise} (re-raise)
          \item <4-> \funcname{maybe\_raise}
          \item <5-> \funcname{main}
          \item <8-> \funcname{main} (re-raise)
        \end{itemize}
      \end{itemize}
      \vfill
      \onslide*<1-5>{\mintocaml[firstline=15,lastline=21]{../src/backtrace/backtrace.ml}}
      \onslide*<6>{\mintocaml[firstline=28,lastline=34,highlightlines=31]{../src/backtrace/backtrace.ml}}
      \onslide*<7->{\mintocaml[firstline=28,lastline=34,highlightlines=33]{../src/backtrace/backtrace.ml}}
      \endminipage
    \end{column}
    \begin{column}{0.45\textwidth}
      \raggedleft
      \onslide*<1>{\providecommand\step{6}\input{diagrams/backtrace/backtrace.tex}}%
      \onslide*<2>{\providecommand\step{6}\input{diagrams/backtrace/backtrace.tex}}%
      \onslide*<3>{\providecommand\step{7}\input{diagrams/backtrace/backtrace.tex}}%
      \onslide*<4>{\providecommand\step{8}\input{diagrams/backtrace/backtrace.tex}}%
      \onslide*<5>{\providecommand\step{9}\input{diagrams/backtrace/backtrace.tex}}%
      \onslide*<6>{\providecommand\step{10}\input{diagrams/backtrace/backtrace.tex}}%
      \onslide*<7>{\providecommand\step{11}\input{diagrams/backtrace/backtrace.tex}}%
      \onslide*<8>{\providecommand\step{12}\input{diagrams/backtrace/backtrace.tex}}%
      \onslide*<9>{\providecommand\step{13}\input{diagrams/backtrace/backtrace.tex}}%
    \end{column}
  \end{columns}
\end{frame}

\begin{frame}{Backtraces}{Printing the backtrace}
  \begin{columns}[c]
    \begin{column}{0.45\textwidth}
      \minipage[c][0.66\textheight][s]{\columnwidth}
      \begin{itemize}
        \item<1-> The exception backtrace can now be printed to \funcarg{stdout} with \funcname{Printexc.print_backtrace}
        \item<2-> First the exception backtrace is retrieved into an OCaml array with \funcname{get_raw_backtrace }
        \item<3-> Which is implemented as the \funcname{caml_get_exception_raw_backtrace} C function in the runtime
        \item<4-> And finally printed with \funcname{print_exception_backtrace}
      \end{itemize}
      \vfill
      \mintocaml[firstline=28,lastline=34,highlightlines=33]{../src/backtrace/backtrace.ml}
      \endminipage
    \end{column}
    \begin{column}{0.55\textwidth}
      \onslide*<1,2>{
        \mintocaml[firstline=100,lastline=101]{../ocaml/stdlib/printexc.ml}
      }
      \only<1>{
        \listocaml[firstline=170,lastline=172]{../ocaml/stdlib/printexc.ml}{stdlib/printexc.ml:170}
      }
      \only<2>{
        \listocaml[firstline=170,lastline=172,highlightlines=172]{../ocaml/stdlib/printexc.ml}{stdlib/printexc.ml:170}
      }
      \only<3>{
        \mintc[firstline=154,lastline=156]{../ocaml/runtime/backtrace.c}
        \listc[firstline=171,lastline=192,highlightlines={184-187}]{../ocaml/runtime/backtrace.c}{runtime/backtrace.c:154}
      }
      \only<4>{
        \listocaml[firstline=155,lastline=165]{../ocaml/stdlib/printexc.ml}{stdlib/printexc.ml:55}
      }
    \end{column}
  \end{columns}
\end{frame}


\subsection{The multiple kinds of raise}
\frameSubsection{}{}

\begin{frame}{Backtraces}{When backtraces are not needed}
  \begin{columns}[c]
    \begin{column}{0.45\textwidth}
      \minipage[c][0.66\textheight][s]{\columnwidth}
      \begin{itemize}
        \item<1-> What if the backtrace for an exception is never needed?
        \item<2-> It's possible to raise the exception explicitly with \funcname{raise\_notrace} to make raising as cheap as possible
        \item<3-> An example of use in the \funcname{dynlink} library of the OCaml compiler
      \end{itemize}
      \vfill
      \onslide<3->{
      \listocaml[firstline=205,lastline=209,highlightlines=207]{../ocaml/otherlibs/dynlink/dynlink\_compilerlibs/misc.ml}{otherlibs/dynlink/dynlink\_compilerlibs/misc.ml:205}
      }
      \endminipage
    \end{column}
    \begin{column}{0.55\textwidth}
    \only<4>{
      \mintcmm[highlightlines=3]{../src/notrace/camlNotrace\_\_fun_347.cmm}
      \listcmm{../src/notrace/camlNotrace\_\_all\_somes\_327.cmm}{all\_somes}
    }
    \onslide*<5->{
      \listobjdump[highlightlines={6-9}]{../src/notrace/camlNotrace\_\_fun_347.objdump}{anonymous function from \funcname{all\_somes}}
    }
    \end{column}
  \end{columns}
\end{frame}

\begin{frame}{Backtraces}{Summary table of the multiple kinds of raise}
  \begin{itemize}
    \item When debugging information is enabled and if the backtrace of an exception isn't needed, it's possible to raise with \funcname{raise\_notrace} for performance
    \item When debugging information is \emph{not} enabled, the compiler optimises \funcname{raise} calls to inline assembly (same effect as using \funcname{raise\_notrace})
  \end{itemize}
  \bigskip
  \centering
  \begin{tabular}{| l | c | c | c | c |}
    \hline
    \parbox{10em}{Debug information \\ (\funcarg{-g} flag)}  & \multicolumn{2}{|c|}{Disabled}                                     & \multicolumn{2}{|c|}{Enabled} \\
    \hline
    Raise kind                                               & \funcname{raise} & \funcname{raise\_notrace}                       & \funcname{raise} & \funcname{raise\_notrace} \\
    \hline
    Compilation                                              & \parbox{8em}{inline assembly\\ (optimization)} & inline assembly   & \funcname{caml\_raise\_exn} & inline assembly \\
    \hline
  \end{tabular}
\end{frame}


%
% Takeaway #5
%
\subsection*{Takeaway \#5}
\frameSubsectionTakeaway{}
\againframe<5>{takeaway}
}%
      \onslide*<4>{\providecommand\step{8}%
% Backtraces
%
\subsection{One advantage of raising with debugging information}
\frameSubsection{}{}

\begin{frame}{Backtraces}{Debugging information \& source code}
  \begin{columns}[c]
    \begin{column}{0.5\textwidth}
      \begin{itemize}
        \item Raising an exception is fun and games, but how do we know where the exception originated? $\rightarrow$ With the backtrace associated with the exception!
        \item In order to get a backtrace from the point where an exception was raised, it's necessary to compile with debugging information enabled (\funcarg{-g} flag for the compiler)
        \item Same example as in the `Nested handlers' section
      \end{itemize}
    \end{column}
    \begin{column}{0.5\textwidth}
      \listocaml[firstline=9,lastline=34]{../src/backtrace/backtrace.ml}{backtrace/backtrace.ml}
    \end{column}
  \end{columns}
\end{frame}

\begin{frame}{Backtraces}{CMM and disassembly of \funcname{definitely\_raise}}
  \begin{columns}[c]
    \begin{column}{0.5\textwidth}
      \minipage[c][0.66\textheight][s]{\columnwidth}
      \begin{itemize}
        \item<1-> An effect of compiling with debugging information (\funcarg{-g}): the CMM no longer uses \funcname{raise\_notrace} to raise the exception but \funcname{raise} instead
        \item<2-> Disassembly shows that CMM \funcname{raise} is actually a call to \funcname{caml\_raise\_exn}, a function in assembler provided by the runtime
      \end{itemize}
      \vfill
      \mintocaml[firstline=9,lastline=13,highlightlines=12]{../src/backtrace/backtrace.ml}
      \endminipage
    \end{column}
    \begin{column}{0.5\textwidth}
      \onslide*<1>{
        \mintcmm[highlightlines=5]{../src/backtrace/camlBacktrace\_\_definitely\_raise\_347.cmm}
      }
      \onslide*<2>{
        \mintobjdump[firstline=2,highlightlines=14]{../src/backtrace/camlBacktrace\_\_definitely\_raise\_347.objdump}
      }
    \end{column}
  \end{columns}
\end{frame}

\begin{frame}{Backtraces}{Introducing \funcname{caml\_raise\_exn}}
  \begin{columns}[c]
    \begin{column}{0.5\textwidth}
      \minipage[c][0.66\textheight][s]{\columnwidth}
      \onslide*<1>{
        \begin{itemize}
          \item Raising the exception is performed using \funcname{caml\_raise\_exn} which is
            \begin{itemize}
              \item An assembly function provided by the runtime
              \item Architecture specific
            \end{itemize}
          \item Let's see what it does differently from \funcname{raise\_notrace}\dots
          \item We are about to raise \typename{ExnC} with \funcname{caml\_raise\_exn} from \funcname{definitely\_raise}
        \end{itemize}
      }
      \onslide*<2>{
        \mintobjdump[firstline=2,highlightlines=14]{../src/backtrace/camlBacktrace\_\_definitely\_raise\_347.objdump}
      }
      \vfill
      \mintocaml[firstline=9,lastline=13,highlightlines=12]{../src/backtrace/backtrace.ml}
      \endminipage
    \end{column}
    \begin{column}{0.5\textwidth}
      \centering
      \providecommand\step{1}\input{diagrams/backtrace/backtrace.tex}
    \end{column}
  \end{columns}
\end{frame}

\begin{frame}{Backtraces}{But before that, we need more fields from \typename{caml\_domain\_state}}
  \begin{columns}
    \begin{column}{0.5\textwidth}
      \begin{itemize}
        \item Remember \typename{caml\_domain\_state} the per-domain runtime state?
        \item Some other members are of interest to us now\dots
        \item \localname{backtrace\_active} is used to know if the runtime should collect backtraces
        \item \localname{backtrace\_active} can be set by
          \begin{itemize}
            \item The \funcarg{b} flag in the \funcname{OCAMLRUNPARAM} environment variable
            \item \mintinline{ocaml}{Printexc.record_backtrace : bool -> unit} \footnotemark
          \end{itemize}
      \end{itemize}
    \end{column}
    \begin{column}{0.5\textwidth}
      \begin{listing}
        \mintc[highlightlines={9-11}]{src/ocaml/domain\_state\_backtrace.h}
        \caption{runtime/caml/domain\_state.h}
      \end{listing}
    \end{column}
  \end{columns}
  \footnotetext[1]{https://v2.ocaml.org/api/Printexc.html}
\end{frame}

\newcommand\listCamlRaiseExn[1][]{
  \listasm[firstline=702,lastline=725,#1]{../ocaml/runtime/amd64.S}{runtime/amd64.S:702}}

\begin{frame}{Backtraces}{Walk through \funcname{caml\_raise\_exn}}
  \begin{columns}[T]
    \begin{column}{0.5\textwidth}
      \begin{itemize}
        \item<1-> First checks whether exceptions backtrace should be recorded
        \item<1-> By checking the \funcarg{domain\_state->backtrace\_active} value
        \item<2-> If not, acts as \funcname{raise\_notrace} with \funcname{RESTORE\_EXN\_HANDLER\_OCAML}
          \begin{itemize}
            \item Set \regname{RSP} to \localname{domain\_state->exn\_handler} value
            \item Restore \localname{domain\_state->exn\_handler} from trap
            \item Jump with \funcname{ret} (same effect as using \funcname{pop} and \funcname{jmp})
          \end{itemize}
        \item<3-> Otherwise, switches to the C stack to be able to execute C code
        \item<4-> Calls \funcname{caml\_stash\_backtrace}, a C function provided by the runtime\ignorespaces
          \onslide<6->{, with
            \begin{itemize}
              \item \funcarg{exn}: exception value
              \item \funcarg{pc}: code address of the raising point
              \item \funcarg{sp}: stack address of the raising function
              \item \funcarg{trapsp}: stack address of the next handler
            \end{itemize}
          }
      \end{itemize}
      \bigskip
      \mintocaml[firstline=9,lastline=13,highlightlines=12]{../src/backtrace/backtrace.ml}
    \end{column}
    \begin{column}{0.5\textwidth}
      \raggedleft
      \onslide*<1,3-4>{
        \mintc[firstline=190,lastline=192]{../ocaml/runtime/amd64.S}
        \mintc[firstline=234,lastline=237]{../ocaml/runtime/amd64.S}
      }
      \onslide*<2>{
        \mintc[firstline=190,lastline=192,highlightlines={191-192}]{../ocaml/runtime/amd64.S}
        \mintc[firstline=234,lastline=237,highlightlines={235-237}]{../ocaml/runtime/amd64.S}
      }
      \onslide*<1>{\listCamlRaiseExn[highlightlines=705]{}}%
      \onslide*<2>{\listCamlRaiseExn[highlightlines={707-708}]{}}%
      \onslide*<3>{\listCamlRaiseExn[highlightlines={712-714}]{}}%
      \onslide*<4>{\listCamlRaiseExn[highlightlines={715-720}]{}}%
      \onslide*<5>{\providecommand\step{1}\input{diagrams/backtrace/backtrace.tex}}%
      \onslide*<6>{\providecommand\step{2}\input{diagrams/backtrace/backtrace.tex}}%
    \end{column}
  \end{columns}
\end{frame}

\newcommand\listStashBacktrace[1][]{
  \listc[firstline=91,lastline=120,#1]{../ocaml/runtime/backtrace\_nat.c}{runtime/backtrace\_nat.c:91}}

\begin{frame}{Backtraces}{Introducing \funcname{caml\_stash\_backtrace}}
  \begin{columns}[c]
    \begin{column}{0.5\textwidth}
      \minipage[c][0.66\textheight][s]{\columnwidth}
      \begin{itemize}
        \item<1-> A C function provided by the runtime (specific to native code)
        \item<2-> It scans the stack, between the stack pointer at the raising point up to the next trap, in order to collect the names of the detected function frames
          \onslide*<2>{
            \begin{itemize}
              \item And more information, like line number, column number...
            \end{itemize}
          }
        \item<3-> It uses the same technique and information as the Garbage Collector (GC) uses to scan the values live on the stack
          \onslide*<4>{
            \begin{itemize}
              \item \typename{frame\_descr} are at the core of stack unwinding
            \end{itemize}
          }
      \end{itemize}
      \vfill
      \mintocaml[firstline=9,lastline=13,highlightlines=12]{../src/backtrace/backtrace.ml}
      \endminipage
    \end{column}
    \begin{column}{0.5\textwidth}
      \onslide*<1>{\listStashBacktrace[highlightlines=91]{}}
      \onslide*<2>{\listStashBacktrace[highlightlines={91,117-118}]{}}
      \onslide*<3->{\listStashBacktrace[highlightlines={94,106,109-110}]{}}
    \end{column}
  \end{columns}
\end{frame}

\newcommand\listFrameDescr[1][]{
  \listocaml[firstline=111,lastline=124,#1]{../ocaml/asmcomp/emitaux.ml}{asmcomp/emitaux.ml:111}}
\newcommand\listDebugInfo[1][]{
  \listocaml[firstline=38,lastline=49,#1]{../ocaml/lambda/debuginfo.mli}{lambda/debuginfo.mli:38}}

\begin{frame}{Backtraces}{\typename{frame\_descr} and \typename{frame\_debuginfo} in a nutshell}
  \begin{columns}[c]
    \begin{column}{0.5\textwidth}
      \begin{itemize}
        \item<1-> For every return address in an OCaml program, there is a associated \typename{frame\_descr}
        \item<2-> A \typename{frame\_descr} specifies
        \begin{itemize}
          \item<2-> The size of the function stack frame
          \item<3-> Where live values are on the stack (so that the GC can mark them as live and prevent from being swept)
        \end{itemize}
        \item<4-> When compiled with debug information, \typename{frame\_descr} will be linked to a \typename{frame\_debuginfo}
        \begin{itemize}
          \item<5-> Contains information about the position in the source code (file name, line and column number)
        \end{itemize}
      \end{itemize}
    \end{column}
    \begin{column}{0.5\textwidth}
        \onslide*<1>{\listFrameDescr[highlightlines={118-122}]{}}
        \onslide*<2>{\listFrameDescr[highlightlines={120}]{}}
        \onslide*<3>{\listFrameDescr[highlightlines={121}]{}}
        \onslide*<4->{\listFrameDescr[highlightlines={122}]{}}
        \medskip
        \onslide*<1-3>{\listDebugInfo{}}
        \onslide*<4>{\listDebugInfo[highlightlines={38,49}]{}}
        \onslide*<5>{\listDebugInfo[highlightlines={39-46}]{}}
    \end{column}
  \end{columns}
\end{frame}

\begin{frame}{Backtraces}{Scanning the stack with \typename{frame\_descr}}
  \begin{columns}[c]
    \begin{column}{0.5\textwidth}
      \begin{itemize}
        \item Given the return address of the code that raises the exception by calling \funcname{caml\_raise\_exn} (\funcarg{pc} argument of \funcname{caml\_stash\_backtrace})
        \item And the position of the stack pointer (\funcarg{sp} argument of \funcname{caml\_stash\_backtrace})
        \item The frame size given by \typename{frame\_descr}.\funcarg{frame\_size}, allows to find the end of the previous frame
        \item And the return address of the caller of this function
        \bigskip
        \onslide<3->{
        \item Note that traps (here the reddish \funcname{ExnA}, \funcname{ExnB} and \funcname{ExnC} blocks) belong to the frame that installed them, and so the frame size changes when traps are removed
        }
      \end{itemize}
    \end{column}
    \begin{column}{0.5\textwidth}
      \raggedleft
      \onslide*<1>{\providecommand\step{2}\input{diagrams/backtrace/backtrace.tex}}%
      \onslide*<2>{\providecommand\step{1}\input{diagrams/backtrace/scanstack.tex}}%
      \onslide*<3>{\providecommand\step{2}\input{diagrams/backtrace/scanstack.tex}}%
      \onslide*<4>{\providecommand\step{3}\input{diagrams/backtrace/scanstack.tex}}%
      \onslide*<5>{\providecommand\step{4}\input{diagrams/backtrace/scanstack.tex}}%
      \onslide*<6>{\providecommand\step{5}\input{diagrams/backtrace/scanstack.tex}}%
    \end{column}
  \end{columns}
\end{frame}

% Duplicate of previous-previous slide
\begin{frame}{Backtraces}{Continuing on \funcname{caml\_stash\_backtrace}}
  \begin{columns}[c]
    \begin{column}{0.5\textwidth}
      \minipage[c][0.75\textheight][s]{\columnwidth}
      \begin{itemize}
          \color{gray}
        \item A C function provided by the runtime (specific to native code)
        \item It scans the stack, between the stack pointer at the raising point up to the next trap, in order to collect the names of the detected function frames
        \item It uses the same technique and information as the Garbage Collector (GC) uses to scan the values live on the stack
      \end{itemize}
      \begin{itemize}
        \item For each function frame detected, store a pointer to the \typename{frame\_descr} in the \localname{domain\_state->backtrace\_buffer} array, effectively constructing the backtrace
      \end{itemize}
      \onslide<2->{
        \begin{itemize}
            \smallskip
          \item Constructed backtrace:
            \funcname{definitely\_raise},
            \onslide<3->{\funcname{probably\_raise}}
        \end{itemize}
      }
      \vfill
      \mintocaml[firstline=9,lastline=13,highlightlines=12]{../src/backtrace/backtrace.ml}
      \endminipage
    \end{column}
    \begin{column}{0.5\textwidth}
      \raggedleft
      \onslide*<1>{\listStashBacktrace[highlightlines={114-115}]{}}
      \onslide*<2>{\providecommand\step{3}\input{diagrams/backtrace/backtrace.tex}}%
      \onslide*<3>{\providecommand\step{4}\input{diagrams/backtrace/backtrace.tex}}%
    \end{column}
  \end{columns}
\end{frame}

\begin{frame}{Backtraces}{Back to \funcname{caml\_raise\_exn}}
  \begin{columns}[c]
    \begin{column}{0.5\textwidth}
      \minipage[c][0.66\textheight][s]{\columnwidth}
      \begin{itemize}\color{gray}
        \item Checks wheteher exceptions backtrace should be recorded, by checking the \funcarg{domain\_state->backtrace\_active} value
        \item Switches to the C stack to be able to execute C code
        \item Calls \funcname{caml\_stash\_backtrace}, to collect the backtrace up to the next trap
      \end{itemize}
      \onslide*<2->{
        \begin{itemize}
          \item<2-> Executes the exception handler in \funcname{probably\_raise} with \funcname{RESTORE\_EXN\_HANDLER\_OCAML}
            \begin{itemize}
              \item Set \regname{RSP} to \localname{domain\_state->exn\_handler} value
              \item Restore \localname{domain\_state->exn\_handler} from trap
              \item Jump to the handler code with \funcname{ret}
            \end{itemize}
        \end{itemize}
      }
      \vfill
      \onslide*<1>{\mintocaml[firstline=9,lastline=13,highlightlines=12]{../src/backtrace/backtrace.ml}}
      \onslide*<2->{\mintocaml[firstline=15,lastline=21,highlightlines={19-21}]{../src/backtrace/backtrace.ml}}
      \endminipage
    \end{column}
    \begin{column}{0.5\textwidth}
      \centering
      \onslide*<1>{
        \mintc[firstline=190,lastline=192]{../ocaml/runtime/amd64.S}
        \mintc[firstline=234,lastline=237]{../ocaml/runtime/amd64.S}
      }
      \onslide*<2>{
        \mintc[firstline=190,lastline=192,highlightlines={191-192}]{../ocaml/runtime/amd64.S}
        \mintc[firstline=234,lastline=237,highlightlines={235-237}]{../ocaml/runtime/amd64.S}
      }
      \onslide*<1>{\listCamlRaiseExn[highlightlines=720]{}}
      \onslide*<2>{\listCamlRaiseExn[highlightlines=722]{}}
      \onslide*<3->{\providecommand\step{5}\input{diagrams/backtrace/backtrace.tex}}
    \end{column}
  \end{columns}
\end{frame}

\begin{frame}{Backtraces}{\funcname{caml\_probably\_raise}'s handler doesn't catch \typename{ExnC}}
  \begin{columns}[c]
    \begin{column}{0.45\textwidth}
      \minipage[c][0.75\textheight][s]{\columnwidth}
      \begin{itemize}
        \item<1-> \funcname{match \dots with}\footnotemark pattern matching can also match exceptions
        \item<1-> The CMM of \funcname{probably\_raise} shows that this \funcname{match \dots with} is implemented with a pattern matching and a \funcname{try \dots with}
        \item<1-> But this one only matches on \typename{ExnA} exception
        \item<2-> Since the pattern matching fails to catch the exception, \funcname{reraise} is called with our current \typename{ExnC} exception
        \item<3-> Disassembly shows that \funcname{reraise} is actually a call to \funcname{caml\_reraise\_exn}, which is also an assembly function provided by the runtime
      \end{itemize}
      \vfill
      \mintocaml[firstline=15,lastline=21,highlightlines={18-21}]{../src/backtrace/backtrace.ml}
      \endminipage
    \end{column}
    \begin{column}{0.55\textwidth}
      \centering
      \onslide*<1>{\mintcmm[highlightlines={10-18}]{../src/backtrace/camlBacktrace\_\_probably\_raise\_351.cmm}}
      \onslide*<2>{\listcmm[highlightlines=18]{../src/backtrace/camlBacktrace\_\_probably\_raise_351.cmm}{CMM of probably\_raise}}
      \onslide*<3>{\mintobjdump[firstline=5,lastline=35,highlightlines=31]{../src/backtrace/camlBacktrace\_\_probably\_raise_351.objdump}}
    \end{column}
  \end{columns}
  \only<1>{\footnotetext[1]{https://blog.janestreet.com/pattern-matching-and-exception-handling-unite/}}
\end{frame}

\newcommand\listCamlReraiseExn[1][]{
  \listasm[firstline=727,lastline=734,#1]{../ocaml/runtime/amd64.S}{runtime/amd64.S:727}}

\begin{frame}{Backtraces}{Introducing \funcname{caml\_reraise\_exn}}
  \begin{columns}[c]
    \begin{column}{0.5\textwidth}
      \minipage[c][0.66\textheight][s]{\columnwidth}
      \begin{itemize}
        \item<1-> The exception have to be reraised to the next handler
        \item<2-> First, checks if the backtrace should be collected with \funcarg{domain\_state->backtrace\_active}
        \only<3>{
        \begin{itemize}
            \item If not, behaves like a \funcname{raise\_notrace} with \funcname{RESTORE\_EXN\_HANDLER\_OCAML}
        \end{itemize}
        }
        \item<4-> Jumps into \funcname{caml\_raise\_exn}
        \item<5-> Calls \funcname{caml\_stash\_backtrace} again, to scan the next part of the stack: from the trap just pop-ed to the next trap
      \end{itemize}
      \vfill
      \mintocaml[firstline=15,lastline=21,highlightlines={18-21}]{../src/backtrace/backtrace.ml}
      \endminipage
    \end{column}
    \begin{column}{0.5\textwidth}
      \raggedleft
      \onslide*<1>{\listCamlReraiseExn{}}
      \onslide*<2>{\listCamlReraiseExn[highlightlines=729]{}}
      \onslide*<3>{
        \mintc[firstline=190,lastline=192,highlightlines={191-192}]{../ocaml/runtime/amd64.S}
        \mintc[firstline=234,lastline=237,highlightlines={235-237}]{../ocaml/runtime/amd64.S}
        \medskip
        \listCamlReraiseExn[highlightlines=731]{}
      }
      \onslide*<4-5>{
        % TODO refactor with \listCamlReraiseExn
        \mintasm[firstline=727,lastline=734,highlightlines=730]{../ocaml/runtime/amd64.S}
        \medskip
      }
      \onslide*<4>{\listCamlRaiseExn[highlightlines=711]{}}
      \onslide*<5>{\listCamlRaiseExn[highlightlines=720]{}}
    \end{column}
  \end{columns}
\end{frame}

\begin{frame}{Backtraces}{Backtrace collection continues}
  \begin{columns}[c]
    \begin{column}{0.55\textwidth}
      \minipage[c][0.75\textheight][s]{\columnwidth}
      \begin{itemize}
        \item<1-> \funcname{caml\_stash\_backtrace} scans the older part of the stack, up to the next trap
        \item<6-> Controls jumps to the nested exception handler in \funcname{maybe\_raise}, which matches only on \typename{ExnB}
        \item<7-> \funcname{caml\_reraise\_exn} is called again, backtrace collection resumes with \funcname{caml\_stash\_backtrace}
      \end{itemize}
      \medskip
      \begin{itemize}
        \item<2-> Constructed backtrace:
        \begin{itemize}
          \item <2-> \funcname{definitely\_raise}
          \item <2-> \funcname{probably\_raise}
          \item <3-> \funcname{probably\_raise} (re-raise)
          \item <4-> \funcname{maybe\_raise}
          \item <5-> \funcname{main}
          \item <8-> \funcname{main} (re-raise)
        \end{itemize}
      \end{itemize}
      \vfill
      \onslide*<1-5>{\mintocaml[firstline=15,lastline=21]{../src/backtrace/backtrace.ml}}
      \onslide*<6>{\mintocaml[firstline=28,lastline=34,highlightlines=31]{../src/backtrace/backtrace.ml}}
      \onslide*<7->{\mintocaml[firstline=28,lastline=34,highlightlines=33]{../src/backtrace/backtrace.ml}}
      \endminipage
    \end{column}
    \begin{column}{0.45\textwidth}
      \raggedleft
      \onslide*<1>{\providecommand\step{6}\input{diagrams/backtrace/backtrace.tex}}%
      \onslide*<2>{\providecommand\step{6}\input{diagrams/backtrace/backtrace.tex}}%
      \onslide*<3>{\providecommand\step{7}\input{diagrams/backtrace/backtrace.tex}}%
      \onslide*<4>{\providecommand\step{8}\input{diagrams/backtrace/backtrace.tex}}%
      \onslide*<5>{\providecommand\step{9}\input{diagrams/backtrace/backtrace.tex}}%
      \onslide*<6>{\providecommand\step{10}\input{diagrams/backtrace/backtrace.tex}}%
      \onslide*<7>{\providecommand\step{11}\input{diagrams/backtrace/backtrace.tex}}%
      \onslide*<8>{\providecommand\step{12}\input{diagrams/backtrace/backtrace.tex}}%
      \onslide*<9>{\providecommand\step{13}\input{diagrams/backtrace/backtrace.tex}}%
    \end{column}
  \end{columns}
\end{frame}

\begin{frame}{Backtraces}{Printing the backtrace}
  \begin{columns}[c]
    \begin{column}{0.45\textwidth}
      \minipage[c][0.66\textheight][s]{\columnwidth}
      \begin{itemize}
        \item<1-> The exception backtrace can now be printed to \funcarg{stdout} with \funcname{Printexc.print_backtrace}
        \item<2-> First the exception backtrace is retrieved into an OCaml array with \funcname{get_raw_backtrace }
        \item<3-> Which is implemented as the \funcname{caml_get_exception_raw_backtrace} C function in the runtime
        \item<4-> And finally printed with \funcname{print_exception_backtrace}
      \end{itemize}
      \vfill
      \mintocaml[firstline=28,lastline=34,highlightlines=33]{../src/backtrace/backtrace.ml}
      \endminipage
    \end{column}
    \begin{column}{0.55\textwidth}
      \onslide*<1,2>{
        \mintocaml[firstline=100,lastline=101]{../ocaml/stdlib/printexc.ml}
      }
      \only<1>{
        \listocaml[firstline=170,lastline=172]{../ocaml/stdlib/printexc.ml}{stdlib/printexc.ml:170}
      }
      \only<2>{
        \listocaml[firstline=170,lastline=172,highlightlines=172]{../ocaml/stdlib/printexc.ml}{stdlib/printexc.ml:170}
      }
      \only<3>{
        \mintc[firstline=154,lastline=156]{../ocaml/runtime/backtrace.c}
        \listc[firstline=171,lastline=192,highlightlines={184-187}]{../ocaml/runtime/backtrace.c}{runtime/backtrace.c:154}
      }
      \only<4>{
        \listocaml[firstline=155,lastline=165]{../ocaml/stdlib/printexc.ml}{stdlib/printexc.ml:55}
      }
    \end{column}
  \end{columns}
\end{frame}


\subsection{The multiple kinds of raise}
\frameSubsection{}{}

\begin{frame}{Backtraces}{When backtraces are not needed}
  \begin{columns}[c]
    \begin{column}{0.45\textwidth}
      \minipage[c][0.66\textheight][s]{\columnwidth}
      \begin{itemize}
        \item<1-> What if the backtrace for an exception is never needed?
        \item<2-> It's possible to raise the exception explicitly with \funcname{raise\_notrace} to make raising as cheap as possible
        \item<3-> An example of use in the \funcname{dynlink} library of the OCaml compiler
      \end{itemize}
      \vfill
      \onslide<3->{
      \listocaml[firstline=205,lastline=209,highlightlines=207]{../ocaml/otherlibs/dynlink/dynlink\_compilerlibs/misc.ml}{otherlibs/dynlink/dynlink\_compilerlibs/misc.ml:205}
      }
      \endminipage
    \end{column}
    \begin{column}{0.55\textwidth}
    \only<4>{
      \mintcmm[highlightlines=3]{../src/notrace/camlNotrace\_\_fun_347.cmm}
      \listcmm{../src/notrace/camlNotrace\_\_all\_somes\_327.cmm}{all\_somes}
    }
    \onslide*<5->{
      \listobjdump[highlightlines={6-9}]{../src/notrace/camlNotrace\_\_fun_347.objdump}{anonymous function from \funcname{all\_somes}}
    }
    \end{column}
  \end{columns}
\end{frame}

\begin{frame}{Backtraces}{Summary table of the multiple kinds of raise}
  \begin{itemize}
    \item When debugging information is enabled and if the backtrace of an exception isn't needed, it's possible to raise with \funcname{raise\_notrace} for performance
    \item When debugging information is \emph{not} enabled, the compiler optimises \funcname{raise} calls to inline assembly (same effect as using \funcname{raise\_notrace})
  \end{itemize}
  \bigskip
  \centering
  \begin{tabular}{| l | c | c | c | c |}
    \hline
    \parbox{10em}{Debug information \\ (\funcarg{-g} flag)}  & \multicolumn{2}{|c|}{Disabled}                                     & \multicolumn{2}{|c|}{Enabled} \\
    \hline
    Raise kind                                               & \funcname{raise} & \funcname{raise\_notrace}                       & \funcname{raise} & \funcname{raise\_notrace} \\
    \hline
    Compilation                                              & \parbox{8em}{inline assembly\\ (optimization)} & inline assembly   & \funcname{caml\_raise\_exn} & inline assembly \\
    \hline
  \end{tabular}
\end{frame}


%
% Takeaway #5
%
\subsection*{Takeaway \#5}
\frameSubsectionTakeaway{}
\againframe<5>{takeaway}
}%
      \onslide*<5>{\providecommand\step{9}%
% Backtraces
%
\subsection{One advantage of raising with debugging information}
\frameSubsection{}{}

\begin{frame}{Backtraces}{Debugging information \& source code}
  \begin{columns}[c]
    \begin{column}{0.5\textwidth}
      \begin{itemize}
        \item Raising an exception is fun and games, but how do we know where the exception originated? $\rightarrow$ With the backtrace associated with the exception!
        \item In order to get a backtrace from the point where an exception was raised, it's necessary to compile with debugging information enabled (\funcarg{-g} flag for the compiler)
        \item Same example as in the `Nested handlers' section
      \end{itemize}
    \end{column}
    \begin{column}{0.5\textwidth}
      \listocaml[firstline=9,lastline=34]{../src/backtrace/backtrace.ml}{backtrace/backtrace.ml}
    \end{column}
  \end{columns}
\end{frame}

\begin{frame}{Backtraces}{CMM and disassembly of \funcname{definitely\_raise}}
  \begin{columns}[c]
    \begin{column}{0.5\textwidth}
      \minipage[c][0.66\textheight][s]{\columnwidth}
      \begin{itemize}
        \item<1-> An effect of compiling with debugging information (\funcarg{-g}): the CMM no longer uses \funcname{raise\_notrace} to raise the exception but \funcname{raise} instead
        \item<2-> Disassembly shows that CMM \funcname{raise} is actually a call to \funcname{caml\_raise\_exn}, a function in assembler provided by the runtime
      \end{itemize}
      \vfill
      \mintocaml[firstline=9,lastline=13,highlightlines=12]{../src/backtrace/backtrace.ml}
      \endminipage
    \end{column}
    \begin{column}{0.5\textwidth}
      \onslide*<1>{
        \mintcmm[highlightlines=5]{../src/backtrace/camlBacktrace\_\_definitely\_raise\_347.cmm}
      }
      \onslide*<2>{
        \mintobjdump[firstline=2,highlightlines=14]{../src/backtrace/camlBacktrace\_\_definitely\_raise\_347.objdump}
      }
    \end{column}
  \end{columns}
\end{frame}

\begin{frame}{Backtraces}{Introducing \funcname{caml\_raise\_exn}}
  \begin{columns}[c]
    \begin{column}{0.5\textwidth}
      \minipage[c][0.66\textheight][s]{\columnwidth}
      \onslide*<1>{
        \begin{itemize}
          \item Raising the exception is performed using \funcname{caml\_raise\_exn} which is
            \begin{itemize}
              \item An assembly function provided by the runtime
              \item Architecture specific
            \end{itemize}
          \item Let's see what it does differently from \funcname{raise\_notrace}\dots
          \item We are about to raise \typename{ExnC} with \funcname{caml\_raise\_exn} from \funcname{definitely\_raise}
        \end{itemize}
      }
      \onslide*<2>{
        \mintobjdump[firstline=2,highlightlines=14]{../src/backtrace/camlBacktrace\_\_definitely\_raise\_347.objdump}
      }
      \vfill
      \mintocaml[firstline=9,lastline=13,highlightlines=12]{../src/backtrace/backtrace.ml}
      \endminipage
    \end{column}
    \begin{column}{0.5\textwidth}
      \centering
      \providecommand\step{1}\input{diagrams/backtrace/backtrace.tex}
    \end{column}
  \end{columns}
\end{frame}

\begin{frame}{Backtraces}{But before that, we need more fields from \typename{caml\_domain\_state}}
  \begin{columns}
    \begin{column}{0.5\textwidth}
      \begin{itemize}
        \item Remember \typename{caml\_domain\_state} the per-domain runtime state?
        \item Some other members are of interest to us now\dots
        \item \localname{backtrace\_active} is used to know if the runtime should collect backtraces
        \item \localname{backtrace\_active} can be set by
          \begin{itemize}
            \item The \funcarg{b} flag in the \funcname{OCAMLRUNPARAM} environment variable
            \item \mintinline{ocaml}{Printexc.record_backtrace : bool -> unit} \footnotemark
          \end{itemize}
      \end{itemize}
    \end{column}
    \begin{column}{0.5\textwidth}
      \begin{listing}
        \mintc[highlightlines={9-11}]{src/ocaml/domain\_state\_backtrace.h}
        \caption{runtime/caml/domain\_state.h}
      \end{listing}
    \end{column}
  \end{columns}
  \footnotetext[1]{https://v2.ocaml.org/api/Printexc.html}
\end{frame}

\newcommand\listCamlRaiseExn[1][]{
  \listasm[firstline=702,lastline=725,#1]{../ocaml/runtime/amd64.S}{runtime/amd64.S:702}}

\begin{frame}{Backtraces}{Walk through \funcname{caml\_raise\_exn}}
  \begin{columns}[T]
    \begin{column}{0.5\textwidth}
      \begin{itemize}
        \item<1-> First checks whether exceptions backtrace should be recorded
        \item<1-> By checking the \funcarg{domain\_state->backtrace\_active} value
        \item<2-> If not, acts as \funcname{raise\_notrace} with \funcname{RESTORE\_EXN\_HANDLER\_OCAML}
          \begin{itemize}
            \item Set \regname{RSP} to \localname{domain\_state->exn\_handler} value
            \item Restore \localname{domain\_state->exn\_handler} from trap
            \item Jump with \funcname{ret} (same effect as using \funcname{pop} and \funcname{jmp})
          \end{itemize}
        \item<3-> Otherwise, switches to the C stack to be able to execute C code
        \item<4-> Calls \funcname{caml\_stash\_backtrace}, a C function provided by the runtime\ignorespaces
          \onslide<6->{, with
            \begin{itemize}
              \item \funcarg{exn}: exception value
              \item \funcarg{pc}: code address of the raising point
              \item \funcarg{sp}: stack address of the raising function
              \item \funcarg{trapsp}: stack address of the next handler
            \end{itemize}
          }
      \end{itemize}
      \bigskip
      \mintocaml[firstline=9,lastline=13,highlightlines=12]{../src/backtrace/backtrace.ml}
    \end{column}
    \begin{column}{0.5\textwidth}
      \raggedleft
      \onslide*<1,3-4>{
        \mintc[firstline=190,lastline=192]{../ocaml/runtime/amd64.S}
        \mintc[firstline=234,lastline=237]{../ocaml/runtime/amd64.S}
      }
      \onslide*<2>{
        \mintc[firstline=190,lastline=192,highlightlines={191-192}]{../ocaml/runtime/amd64.S}
        \mintc[firstline=234,lastline=237,highlightlines={235-237}]{../ocaml/runtime/amd64.S}
      }
      \onslide*<1>{\listCamlRaiseExn[highlightlines=705]{}}%
      \onslide*<2>{\listCamlRaiseExn[highlightlines={707-708}]{}}%
      \onslide*<3>{\listCamlRaiseExn[highlightlines={712-714}]{}}%
      \onslide*<4>{\listCamlRaiseExn[highlightlines={715-720}]{}}%
      \onslide*<5>{\providecommand\step{1}\input{diagrams/backtrace/backtrace.tex}}%
      \onslide*<6>{\providecommand\step{2}\input{diagrams/backtrace/backtrace.tex}}%
    \end{column}
  \end{columns}
\end{frame}

\newcommand\listStashBacktrace[1][]{
  \listc[firstline=91,lastline=120,#1]{../ocaml/runtime/backtrace\_nat.c}{runtime/backtrace\_nat.c:91}}

\begin{frame}{Backtraces}{Introducing \funcname{caml\_stash\_backtrace}}
  \begin{columns}[c]
    \begin{column}{0.5\textwidth}
      \minipage[c][0.66\textheight][s]{\columnwidth}
      \begin{itemize}
        \item<1-> A C function provided by the runtime (specific to native code)
        \item<2-> It scans the stack, between the stack pointer at the raising point up to the next trap, in order to collect the names of the detected function frames
          \onslide*<2>{
            \begin{itemize}
              \item And more information, like line number, column number...
            \end{itemize}
          }
        \item<3-> It uses the same technique and information as the Garbage Collector (GC) uses to scan the values live on the stack
          \onslide*<4>{
            \begin{itemize}
              \item \typename{frame\_descr} are at the core of stack unwinding
            \end{itemize}
          }
      \end{itemize}
      \vfill
      \mintocaml[firstline=9,lastline=13,highlightlines=12]{../src/backtrace/backtrace.ml}
      \endminipage
    \end{column}
    \begin{column}{0.5\textwidth}
      \onslide*<1>{\listStashBacktrace[highlightlines=91]{}}
      \onslide*<2>{\listStashBacktrace[highlightlines={91,117-118}]{}}
      \onslide*<3->{\listStashBacktrace[highlightlines={94,106,109-110}]{}}
    \end{column}
  \end{columns}
\end{frame}

\newcommand\listFrameDescr[1][]{
  \listocaml[firstline=111,lastline=124,#1]{../ocaml/asmcomp/emitaux.ml}{asmcomp/emitaux.ml:111}}
\newcommand\listDebugInfo[1][]{
  \listocaml[firstline=38,lastline=49,#1]{../ocaml/lambda/debuginfo.mli}{lambda/debuginfo.mli:38}}

\begin{frame}{Backtraces}{\typename{frame\_descr} and \typename{frame\_debuginfo} in a nutshell}
  \begin{columns}[c]
    \begin{column}{0.5\textwidth}
      \begin{itemize}
        \item<1-> For every return address in an OCaml program, there is a associated \typename{frame\_descr}
        \item<2-> A \typename{frame\_descr} specifies
        \begin{itemize}
          \item<2-> The size of the function stack frame
          \item<3-> Where live values are on the stack (so that the GC can mark them as live and prevent from being swept)
        \end{itemize}
        \item<4-> When compiled with debug information, \typename{frame\_descr} will be linked to a \typename{frame\_debuginfo}
        \begin{itemize}
          \item<5-> Contains information about the position in the source code (file name, line and column number)
        \end{itemize}
      \end{itemize}
    \end{column}
    \begin{column}{0.5\textwidth}
        \onslide*<1>{\listFrameDescr[highlightlines={118-122}]{}}
        \onslide*<2>{\listFrameDescr[highlightlines={120}]{}}
        \onslide*<3>{\listFrameDescr[highlightlines={121}]{}}
        \onslide*<4->{\listFrameDescr[highlightlines={122}]{}}
        \medskip
        \onslide*<1-3>{\listDebugInfo{}}
        \onslide*<4>{\listDebugInfo[highlightlines={38,49}]{}}
        \onslide*<5>{\listDebugInfo[highlightlines={39-46}]{}}
    \end{column}
  \end{columns}
\end{frame}

\begin{frame}{Backtraces}{Scanning the stack with \typename{frame\_descr}}
  \begin{columns}[c]
    \begin{column}{0.5\textwidth}
      \begin{itemize}
        \item Given the return address of the code that raises the exception by calling \funcname{caml\_raise\_exn} (\funcarg{pc} argument of \funcname{caml\_stash\_backtrace})
        \item And the position of the stack pointer (\funcarg{sp} argument of \funcname{caml\_stash\_backtrace})
        \item The frame size given by \typename{frame\_descr}.\funcarg{frame\_size}, allows to find the end of the previous frame
        \item And the return address of the caller of this function
        \bigskip
        \onslide<3->{
        \item Note that traps (here the reddish \funcname{ExnA}, \funcname{ExnB} and \funcname{ExnC} blocks) belong to the frame that installed them, and so the frame size changes when traps are removed
        }
      \end{itemize}
    \end{column}
    \begin{column}{0.5\textwidth}
      \raggedleft
      \onslide*<1>{\providecommand\step{2}\input{diagrams/backtrace/backtrace.tex}}%
      \onslide*<2>{\providecommand\step{1}\input{diagrams/backtrace/scanstack.tex}}%
      \onslide*<3>{\providecommand\step{2}\input{diagrams/backtrace/scanstack.tex}}%
      \onslide*<4>{\providecommand\step{3}\input{diagrams/backtrace/scanstack.tex}}%
      \onslide*<5>{\providecommand\step{4}\input{diagrams/backtrace/scanstack.tex}}%
      \onslide*<6>{\providecommand\step{5}\input{diagrams/backtrace/scanstack.tex}}%
    \end{column}
  \end{columns}
\end{frame}

% Duplicate of previous-previous slide
\begin{frame}{Backtraces}{Continuing on \funcname{caml\_stash\_backtrace}}
  \begin{columns}[c]
    \begin{column}{0.5\textwidth}
      \minipage[c][0.75\textheight][s]{\columnwidth}
      \begin{itemize}
          \color{gray}
        \item A C function provided by the runtime (specific to native code)
        \item It scans the stack, between the stack pointer at the raising point up to the next trap, in order to collect the names of the detected function frames
        \item It uses the same technique and information as the Garbage Collector (GC) uses to scan the values live on the stack
      \end{itemize}
      \begin{itemize}
        \item For each function frame detected, store a pointer to the \typename{frame\_descr} in the \localname{domain\_state->backtrace\_buffer} array, effectively constructing the backtrace
      \end{itemize}
      \onslide<2->{
        \begin{itemize}
            \smallskip
          \item Constructed backtrace:
            \funcname{definitely\_raise},
            \onslide<3->{\funcname{probably\_raise}}
        \end{itemize}
      }
      \vfill
      \mintocaml[firstline=9,lastline=13,highlightlines=12]{../src/backtrace/backtrace.ml}
      \endminipage
    \end{column}
    \begin{column}{0.5\textwidth}
      \raggedleft
      \onslide*<1>{\listStashBacktrace[highlightlines={114-115}]{}}
      \onslide*<2>{\providecommand\step{3}\input{diagrams/backtrace/backtrace.tex}}%
      \onslide*<3>{\providecommand\step{4}\input{diagrams/backtrace/backtrace.tex}}%
    \end{column}
  \end{columns}
\end{frame}

\begin{frame}{Backtraces}{Back to \funcname{caml\_raise\_exn}}
  \begin{columns}[c]
    \begin{column}{0.5\textwidth}
      \minipage[c][0.66\textheight][s]{\columnwidth}
      \begin{itemize}\color{gray}
        \item Checks wheteher exceptions backtrace should be recorded, by checking the \funcarg{domain\_state->backtrace\_active} value
        \item Switches to the C stack to be able to execute C code
        \item Calls \funcname{caml\_stash\_backtrace}, to collect the backtrace up to the next trap
      \end{itemize}
      \onslide*<2->{
        \begin{itemize}
          \item<2-> Executes the exception handler in \funcname{probably\_raise} with \funcname{RESTORE\_EXN\_HANDLER\_OCAML}
            \begin{itemize}
              \item Set \regname{RSP} to \localname{domain\_state->exn\_handler} value
              \item Restore \localname{domain\_state->exn\_handler} from trap
              \item Jump to the handler code with \funcname{ret}
            \end{itemize}
        \end{itemize}
      }
      \vfill
      \onslide*<1>{\mintocaml[firstline=9,lastline=13,highlightlines=12]{../src/backtrace/backtrace.ml}}
      \onslide*<2->{\mintocaml[firstline=15,lastline=21,highlightlines={19-21}]{../src/backtrace/backtrace.ml}}
      \endminipage
    \end{column}
    \begin{column}{0.5\textwidth}
      \centering
      \onslide*<1>{
        \mintc[firstline=190,lastline=192]{../ocaml/runtime/amd64.S}
        \mintc[firstline=234,lastline=237]{../ocaml/runtime/amd64.S}
      }
      \onslide*<2>{
        \mintc[firstline=190,lastline=192,highlightlines={191-192}]{../ocaml/runtime/amd64.S}
        \mintc[firstline=234,lastline=237,highlightlines={235-237}]{../ocaml/runtime/amd64.S}
      }
      \onslide*<1>{\listCamlRaiseExn[highlightlines=720]{}}
      \onslide*<2>{\listCamlRaiseExn[highlightlines=722]{}}
      \onslide*<3->{\providecommand\step{5}\input{diagrams/backtrace/backtrace.tex}}
    \end{column}
  \end{columns}
\end{frame}

\begin{frame}{Backtraces}{\funcname{caml\_probably\_raise}'s handler doesn't catch \typename{ExnC}}
  \begin{columns}[c]
    \begin{column}{0.45\textwidth}
      \minipage[c][0.75\textheight][s]{\columnwidth}
      \begin{itemize}
        \item<1-> \funcname{match \dots with}\footnotemark pattern matching can also match exceptions
        \item<1-> The CMM of \funcname{probably\_raise} shows that this \funcname{match \dots with} is implemented with a pattern matching and a \funcname{try \dots with}
        \item<1-> But this one only matches on \typename{ExnA} exception
        \item<2-> Since the pattern matching fails to catch the exception, \funcname{reraise} is called with our current \typename{ExnC} exception
        \item<3-> Disassembly shows that \funcname{reraise} is actually a call to \funcname{caml\_reraise\_exn}, which is also an assembly function provided by the runtime
      \end{itemize}
      \vfill
      \mintocaml[firstline=15,lastline=21,highlightlines={18-21}]{../src/backtrace/backtrace.ml}
      \endminipage
    \end{column}
    \begin{column}{0.55\textwidth}
      \centering
      \onslide*<1>{\mintcmm[highlightlines={10-18}]{../src/backtrace/camlBacktrace\_\_probably\_raise\_351.cmm}}
      \onslide*<2>{\listcmm[highlightlines=18]{../src/backtrace/camlBacktrace\_\_probably\_raise_351.cmm}{CMM of probably\_raise}}
      \onslide*<3>{\mintobjdump[firstline=5,lastline=35,highlightlines=31]{../src/backtrace/camlBacktrace\_\_probably\_raise_351.objdump}}
    \end{column}
  \end{columns}
  \only<1>{\footnotetext[1]{https://blog.janestreet.com/pattern-matching-and-exception-handling-unite/}}
\end{frame}

\newcommand\listCamlReraiseExn[1][]{
  \listasm[firstline=727,lastline=734,#1]{../ocaml/runtime/amd64.S}{runtime/amd64.S:727}}

\begin{frame}{Backtraces}{Introducing \funcname{caml\_reraise\_exn}}
  \begin{columns}[c]
    \begin{column}{0.5\textwidth}
      \minipage[c][0.66\textheight][s]{\columnwidth}
      \begin{itemize}
        \item<1-> The exception have to be reraised to the next handler
        \item<2-> First, checks if the backtrace should be collected with \funcarg{domain\_state->backtrace\_active}
        \only<3>{
        \begin{itemize}
            \item If not, behaves like a \funcname{raise\_notrace} with \funcname{RESTORE\_EXN\_HANDLER\_OCAML}
        \end{itemize}
        }
        \item<4-> Jumps into \funcname{caml\_raise\_exn}
        \item<5-> Calls \funcname{caml\_stash\_backtrace} again, to scan the next part of the stack: from the trap just pop-ed to the next trap
      \end{itemize}
      \vfill
      \mintocaml[firstline=15,lastline=21,highlightlines={18-21}]{../src/backtrace/backtrace.ml}
      \endminipage
    \end{column}
    \begin{column}{0.5\textwidth}
      \raggedleft
      \onslide*<1>{\listCamlReraiseExn{}}
      \onslide*<2>{\listCamlReraiseExn[highlightlines=729]{}}
      \onslide*<3>{
        \mintc[firstline=190,lastline=192,highlightlines={191-192}]{../ocaml/runtime/amd64.S}
        \mintc[firstline=234,lastline=237,highlightlines={235-237}]{../ocaml/runtime/amd64.S}
        \medskip
        \listCamlReraiseExn[highlightlines=731]{}
      }
      \onslide*<4-5>{
        % TODO refactor with \listCamlReraiseExn
        \mintasm[firstline=727,lastline=734,highlightlines=730]{../ocaml/runtime/amd64.S}
        \medskip
      }
      \onslide*<4>{\listCamlRaiseExn[highlightlines=711]{}}
      \onslide*<5>{\listCamlRaiseExn[highlightlines=720]{}}
    \end{column}
  \end{columns}
\end{frame}

\begin{frame}{Backtraces}{Backtrace collection continues}
  \begin{columns}[c]
    \begin{column}{0.55\textwidth}
      \minipage[c][0.75\textheight][s]{\columnwidth}
      \begin{itemize}
        \item<1-> \funcname{caml\_stash\_backtrace} scans the older part of the stack, up to the next trap
        \item<6-> Controls jumps to the nested exception handler in \funcname{maybe\_raise}, which matches only on \typename{ExnB}
        \item<7-> \funcname{caml\_reraise\_exn} is called again, backtrace collection resumes with \funcname{caml\_stash\_backtrace}
      \end{itemize}
      \medskip
      \begin{itemize}
        \item<2-> Constructed backtrace:
        \begin{itemize}
          \item <2-> \funcname{definitely\_raise}
          \item <2-> \funcname{probably\_raise}
          \item <3-> \funcname{probably\_raise} (re-raise)
          \item <4-> \funcname{maybe\_raise}
          \item <5-> \funcname{main}
          \item <8-> \funcname{main} (re-raise)
        \end{itemize}
      \end{itemize}
      \vfill
      \onslide*<1-5>{\mintocaml[firstline=15,lastline=21]{../src/backtrace/backtrace.ml}}
      \onslide*<6>{\mintocaml[firstline=28,lastline=34,highlightlines=31]{../src/backtrace/backtrace.ml}}
      \onslide*<7->{\mintocaml[firstline=28,lastline=34,highlightlines=33]{../src/backtrace/backtrace.ml}}
      \endminipage
    \end{column}
    \begin{column}{0.45\textwidth}
      \raggedleft
      \onslide*<1>{\providecommand\step{6}\input{diagrams/backtrace/backtrace.tex}}%
      \onslide*<2>{\providecommand\step{6}\input{diagrams/backtrace/backtrace.tex}}%
      \onslide*<3>{\providecommand\step{7}\input{diagrams/backtrace/backtrace.tex}}%
      \onslide*<4>{\providecommand\step{8}\input{diagrams/backtrace/backtrace.tex}}%
      \onslide*<5>{\providecommand\step{9}\input{diagrams/backtrace/backtrace.tex}}%
      \onslide*<6>{\providecommand\step{10}\input{diagrams/backtrace/backtrace.tex}}%
      \onslide*<7>{\providecommand\step{11}\input{diagrams/backtrace/backtrace.tex}}%
      \onslide*<8>{\providecommand\step{12}\input{diagrams/backtrace/backtrace.tex}}%
      \onslide*<9>{\providecommand\step{13}\input{diagrams/backtrace/backtrace.tex}}%
    \end{column}
  \end{columns}
\end{frame}

\begin{frame}{Backtraces}{Printing the backtrace}
  \begin{columns}[c]
    \begin{column}{0.45\textwidth}
      \minipage[c][0.66\textheight][s]{\columnwidth}
      \begin{itemize}
        \item<1-> The exception backtrace can now be printed to \funcarg{stdout} with \funcname{Printexc.print_backtrace}
        \item<2-> First the exception backtrace is retrieved into an OCaml array with \funcname{get_raw_backtrace }
        \item<3-> Which is implemented as the \funcname{caml_get_exception_raw_backtrace} C function in the runtime
        \item<4-> And finally printed with \funcname{print_exception_backtrace}
      \end{itemize}
      \vfill
      \mintocaml[firstline=28,lastline=34,highlightlines=33]{../src/backtrace/backtrace.ml}
      \endminipage
    \end{column}
    \begin{column}{0.55\textwidth}
      \onslide*<1,2>{
        \mintocaml[firstline=100,lastline=101]{../ocaml/stdlib/printexc.ml}
      }
      \only<1>{
        \listocaml[firstline=170,lastline=172]{../ocaml/stdlib/printexc.ml}{stdlib/printexc.ml:170}
      }
      \only<2>{
        \listocaml[firstline=170,lastline=172,highlightlines=172]{../ocaml/stdlib/printexc.ml}{stdlib/printexc.ml:170}
      }
      \only<3>{
        \mintc[firstline=154,lastline=156]{../ocaml/runtime/backtrace.c}
        \listc[firstline=171,lastline=192,highlightlines={184-187}]{../ocaml/runtime/backtrace.c}{runtime/backtrace.c:154}
      }
      \only<4>{
        \listocaml[firstline=155,lastline=165]{../ocaml/stdlib/printexc.ml}{stdlib/printexc.ml:55}
      }
    \end{column}
  \end{columns}
\end{frame}


\subsection{The multiple kinds of raise}
\frameSubsection{}{}

\begin{frame}{Backtraces}{When backtraces are not needed}
  \begin{columns}[c]
    \begin{column}{0.45\textwidth}
      \minipage[c][0.66\textheight][s]{\columnwidth}
      \begin{itemize}
        \item<1-> What if the backtrace for an exception is never needed?
        \item<2-> It's possible to raise the exception explicitly with \funcname{raise\_notrace} to make raising as cheap as possible
        \item<3-> An example of use in the \funcname{dynlink} library of the OCaml compiler
      \end{itemize}
      \vfill
      \onslide<3->{
      \listocaml[firstline=205,lastline=209,highlightlines=207]{../ocaml/otherlibs/dynlink/dynlink\_compilerlibs/misc.ml}{otherlibs/dynlink/dynlink\_compilerlibs/misc.ml:205}
      }
      \endminipage
    \end{column}
    \begin{column}{0.55\textwidth}
    \only<4>{
      \mintcmm[highlightlines=3]{../src/notrace/camlNotrace\_\_fun_347.cmm}
      \listcmm{../src/notrace/camlNotrace\_\_all\_somes\_327.cmm}{all\_somes}
    }
    \onslide*<5->{
      \listobjdump[highlightlines={6-9}]{../src/notrace/camlNotrace\_\_fun_347.objdump}{anonymous function from \funcname{all\_somes}}
    }
    \end{column}
  \end{columns}
\end{frame}

\begin{frame}{Backtraces}{Summary table of the multiple kinds of raise}
  \begin{itemize}
    \item When debugging information is enabled and if the backtrace of an exception isn't needed, it's possible to raise with \funcname{raise\_notrace} for performance
    \item When debugging information is \emph{not} enabled, the compiler optimises \funcname{raise} calls to inline assembly (same effect as using \funcname{raise\_notrace})
  \end{itemize}
  \bigskip
  \centering
  \begin{tabular}{| l | c | c | c | c |}
    \hline
    \parbox{10em}{Debug information \\ (\funcarg{-g} flag)}  & \multicolumn{2}{|c|}{Disabled}                                     & \multicolumn{2}{|c|}{Enabled} \\
    \hline
    Raise kind                                               & \funcname{raise} & \funcname{raise\_notrace}                       & \funcname{raise} & \funcname{raise\_notrace} \\
    \hline
    Compilation                                              & \parbox{8em}{inline assembly\\ (optimization)} & inline assembly   & \funcname{caml\_raise\_exn} & inline assembly \\
    \hline
  \end{tabular}
\end{frame}


%
% Takeaway #5
%
\subsection*{Takeaway \#5}
\frameSubsectionTakeaway{}
\againframe<5>{takeaway}
}%
      \onslide*<6>{\providecommand\step{10}%
% Backtraces
%
\subsection{One advantage of raising with debugging information}
\frameSubsection{}{}

\begin{frame}{Backtraces}{Debugging information \& source code}
  \begin{columns}[c]
    \begin{column}{0.5\textwidth}
      \begin{itemize}
        \item Raising an exception is fun and games, but how do we know where the exception originated? $\rightarrow$ With the backtrace associated with the exception!
        \item In order to get a backtrace from the point where an exception was raised, it's necessary to compile with debugging information enabled (\funcarg{-g} flag for the compiler)
        \item Same example as in the `Nested handlers' section
      \end{itemize}
    \end{column}
    \begin{column}{0.5\textwidth}
      \listocaml[firstline=9,lastline=34]{../src/backtrace/backtrace.ml}{backtrace/backtrace.ml}
    \end{column}
  \end{columns}
\end{frame}

\begin{frame}{Backtraces}{CMM and disassembly of \funcname{definitely\_raise}}
  \begin{columns}[c]
    \begin{column}{0.5\textwidth}
      \minipage[c][0.66\textheight][s]{\columnwidth}
      \begin{itemize}
        \item<1-> An effect of compiling with debugging information (\funcarg{-g}): the CMM no longer uses \funcname{raise\_notrace} to raise the exception but \funcname{raise} instead
        \item<2-> Disassembly shows that CMM \funcname{raise} is actually a call to \funcname{caml\_raise\_exn}, a function in assembler provided by the runtime
      \end{itemize}
      \vfill
      \mintocaml[firstline=9,lastline=13,highlightlines=12]{../src/backtrace/backtrace.ml}
      \endminipage
    \end{column}
    \begin{column}{0.5\textwidth}
      \onslide*<1>{
        \mintcmm[highlightlines=5]{../src/backtrace/camlBacktrace\_\_definitely\_raise\_347.cmm}
      }
      \onslide*<2>{
        \mintobjdump[firstline=2,highlightlines=14]{../src/backtrace/camlBacktrace\_\_definitely\_raise\_347.objdump}
      }
    \end{column}
  \end{columns}
\end{frame}

\begin{frame}{Backtraces}{Introducing \funcname{caml\_raise\_exn}}
  \begin{columns}[c]
    \begin{column}{0.5\textwidth}
      \minipage[c][0.66\textheight][s]{\columnwidth}
      \onslide*<1>{
        \begin{itemize}
          \item Raising the exception is performed using \funcname{caml\_raise\_exn} which is
            \begin{itemize}
              \item An assembly function provided by the runtime
              \item Architecture specific
            \end{itemize}
          \item Let's see what it does differently from \funcname{raise\_notrace}\dots
          \item We are about to raise \typename{ExnC} with \funcname{caml\_raise\_exn} from \funcname{definitely\_raise}
        \end{itemize}
      }
      \onslide*<2>{
        \mintobjdump[firstline=2,highlightlines=14]{../src/backtrace/camlBacktrace\_\_definitely\_raise\_347.objdump}
      }
      \vfill
      \mintocaml[firstline=9,lastline=13,highlightlines=12]{../src/backtrace/backtrace.ml}
      \endminipage
    \end{column}
    \begin{column}{0.5\textwidth}
      \centering
      \providecommand\step{1}\input{diagrams/backtrace/backtrace.tex}
    \end{column}
  \end{columns}
\end{frame}

\begin{frame}{Backtraces}{But before that, we need more fields from \typename{caml\_domain\_state}}
  \begin{columns}
    \begin{column}{0.5\textwidth}
      \begin{itemize}
        \item Remember \typename{caml\_domain\_state} the per-domain runtime state?
        \item Some other members are of interest to us now\dots
        \item \localname{backtrace\_active} is used to know if the runtime should collect backtraces
        \item \localname{backtrace\_active} can be set by
          \begin{itemize}
            \item The \funcarg{b} flag in the \funcname{OCAMLRUNPARAM} environment variable
            \item \mintinline{ocaml}{Printexc.record_backtrace : bool -> unit} \footnotemark
          \end{itemize}
      \end{itemize}
    \end{column}
    \begin{column}{0.5\textwidth}
      \begin{listing}
        \mintc[highlightlines={9-11}]{src/ocaml/domain\_state\_backtrace.h}
        \caption{runtime/caml/domain\_state.h}
      \end{listing}
    \end{column}
  \end{columns}
  \footnotetext[1]{https://v2.ocaml.org/api/Printexc.html}
\end{frame}

\newcommand\listCamlRaiseExn[1][]{
  \listasm[firstline=702,lastline=725,#1]{../ocaml/runtime/amd64.S}{runtime/amd64.S:702}}

\begin{frame}{Backtraces}{Walk through \funcname{caml\_raise\_exn}}
  \begin{columns}[T]
    \begin{column}{0.5\textwidth}
      \begin{itemize}
        \item<1-> First checks whether exceptions backtrace should be recorded
        \item<1-> By checking the \funcarg{domain\_state->backtrace\_active} value
        \item<2-> If not, acts as \funcname{raise\_notrace} with \funcname{RESTORE\_EXN\_HANDLER\_OCAML}
          \begin{itemize}
            \item Set \regname{RSP} to \localname{domain\_state->exn\_handler} value
            \item Restore \localname{domain\_state->exn\_handler} from trap
            \item Jump with \funcname{ret} (same effect as using \funcname{pop} and \funcname{jmp})
          \end{itemize}
        \item<3-> Otherwise, switches to the C stack to be able to execute C code
        \item<4-> Calls \funcname{caml\_stash\_backtrace}, a C function provided by the runtime\ignorespaces
          \onslide<6->{, with
            \begin{itemize}
              \item \funcarg{exn}: exception value
              \item \funcarg{pc}: code address of the raising point
              \item \funcarg{sp}: stack address of the raising function
              \item \funcarg{trapsp}: stack address of the next handler
            \end{itemize}
          }
      \end{itemize}
      \bigskip
      \mintocaml[firstline=9,lastline=13,highlightlines=12]{../src/backtrace/backtrace.ml}
    \end{column}
    \begin{column}{0.5\textwidth}
      \raggedleft
      \onslide*<1,3-4>{
        \mintc[firstline=190,lastline=192]{../ocaml/runtime/amd64.S}
        \mintc[firstline=234,lastline=237]{../ocaml/runtime/amd64.S}
      }
      \onslide*<2>{
        \mintc[firstline=190,lastline=192,highlightlines={191-192}]{../ocaml/runtime/amd64.S}
        \mintc[firstline=234,lastline=237,highlightlines={235-237}]{../ocaml/runtime/amd64.S}
      }
      \onslide*<1>{\listCamlRaiseExn[highlightlines=705]{}}%
      \onslide*<2>{\listCamlRaiseExn[highlightlines={707-708}]{}}%
      \onslide*<3>{\listCamlRaiseExn[highlightlines={712-714}]{}}%
      \onslide*<4>{\listCamlRaiseExn[highlightlines={715-720}]{}}%
      \onslide*<5>{\providecommand\step{1}\input{diagrams/backtrace/backtrace.tex}}%
      \onslide*<6>{\providecommand\step{2}\input{diagrams/backtrace/backtrace.tex}}%
    \end{column}
  \end{columns}
\end{frame}

\newcommand\listStashBacktrace[1][]{
  \listc[firstline=91,lastline=120,#1]{../ocaml/runtime/backtrace\_nat.c}{runtime/backtrace\_nat.c:91}}

\begin{frame}{Backtraces}{Introducing \funcname{caml\_stash\_backtrace}}
  \begin{columns}[c]
    \begin{column}{0.5\textwidth}
      \minipage[c][0.66\textheight][s]{\columnwidth}
      \begin{itemize}
        \item<1-> A C function provided by the runtime (specific to native code)
        \item<2-> It scans the stack, between the stack pointer at the raising point up to the next trap, in order to collect the names of the detected function frames
          \onslide*<2>{
            \begin{itemize}
              \item And more information, like line number, column number...
            \end{itemize}
          }
        \item<3-> It uses the same technique and information as the Garbage Collector (GC) uses to scan the values live on the stack
          \onslide*<4>{
            \begin{itemize}
              \item \typename{frame\_descr} are at the core of stack unwinding
            \end{itemize}
          }
      \end{itemize}
      \vfill
      \mintocaml[firstline=9,lastline=13,highlightlines=12]{../src/backtrace/backtrace.ml}
      \endminipage
    \end{column}
    \begin{column}{0.5\textwidth}
      \onslide*<1>{\listStashBacktrace[highlightlines=91]{}}
      \onslide*<2>{\listStashBacktrace[highlightlines={91,117-118}]{}}
      \onslide*<3->{\listStashBacktrace[highlightlines={94,106,109-110}]{}}
    \end{column}
  \end{columns}
\end{frame}

\newcommand\listFrameDescr[1][]{
  \listocaml[firstline=111,lastline=124,#1]{../ocaml/asmcomp/emitaux.ml}{asmcomp/emitaux.ml:111}}
\newcommand\listDebugInfo[1][]{
  \listocaml[firstline=38,lastline=49,#1]{../ocaml/lambda/debuginfo.mli}{lambda/debuginfo.mli:38}}

\begin{frame}{Backtraces}{\typename{frame\_descr} and \typename{frame\_debuginfo} in a nutshell}
  \begin{columns}[c]
    \begin{column}{0.5\textwidth}
      \begin{itemize}
        \item<1-> For every return address in an OCaml program, there is a associated \typename{frame\_descr}
        \item<2-> A \typename{frame\_descr} specifies
        \begin{itemize}
          \item<2-> The size of the function stack frame
          \item<3-> Where live values are on the stack (so that the GC can mark them as live and prevent from being swept)
        \end{itemize}
        \item<4-> When compiled with debug information, \typename{frame\_descr} will be linked to a \typename{frame\_debuginfo}
        \begin{itemize}
          \item<5-> Contains information about the position in the source code (file name, line and column number)
        \end{itemize}
      \end{itemize}
    \end{column}
    \begin{column}{0.5\textwidth}
        \onslide*<1>{\listFrameDescr[highlightlines={118-122}]{}}
        \onslide*<2>{\listFrameDescr[highlightlines={120}]{}}
        \onslide*<3>{\listFrameDescr[highlightlines={121}]{}}
        \onslide*<4->{\listFrameDescr[highlightlines={122}]{}}
        \medskip
        \onslide*<1-3>{\listDebugInfo{}}
        \onslide*<4>{\listDebugInfo[highlightlines={38,49}]{}}
        \onslide*<5>{\listDebugInfo[highlightlines={39-46}]{}}
    \end{column}
  \end{columns}
\end{frame}

\begin{frame}{Backtraces}{Scanning the stack with \typename{frame\_descr}}
  \begin{columns}[c]
    \begin{column}{0.5\textwidth}
      \begin{itemize}
        \item Given the return address of the code that raises the exception by calling \funcname{caml\_raise\_exn} (\funcarg{pc} argument of \funcname{caml\_stash\_backtrace})
        \item And the position of the stack pointer (\funcarg{sp} argument of \funcname{caml\_stash\_backtrace})
        \item The frame size given by \typename{frame\_descr}.\funcarg{frame\_size}, allows to find the end of the previous frame
        \item And the return address of the caller of this function
        \bigskip
        \onslide<3->{
        \item Note that traps (here the reddish \funcname{ExnA}, \funcname{ExnB} and \funcname{ExnC} blocks) belong to the frame that installed them, and so the frame size changes when traps are removed
        }
      \end{itemize}
    \end{column}
    \begin{column}{0.5\textwidth}
      \raggedleft
      \onslide*<1>{\providecommand\step{2}\input{diagrams/backtrace/backtrace.tex}}%
      \onslide*<2>{\providecommand\step{1}\input{diagrams/backtrace/scanstack.tex}}%
      \onslide*<3>{\providecommand\step{2}\input{diagrams/backtrace/scanstack.tex}}%
      \onslide*<4>{\providecommand\step{3}\input{diagrams/backtrace/scanstack.tex}}%
      \onslide*<5>{\providecommand\step{4}\input{diagrams/backtrace/scanstack.tex}}%
      \onslide*<6>{\providecommand\step{5}\input{diagrams/backtrace/scanstack.tex}}%
    \end{column}
  \end{columns}
\end{frame}

% Duplicate of previous-previous slide
\begin{frame}{Backtraces}{Continuing on \funcname{caml\_stash\_backtrace}}
  \begin{columns}[c]
    \begin{column}{0.5\textwidth}
      \minipage[c][0.75\textheight][s]{\columnwidth}
      \begin{itemize}
          \color{gray}
        \item A C function provided by the runtime (specific to native code)
        \item It scans the stack, between the stack pointer at the raising point up to the next trap, in order to collect the names of the detected function frames
        \item It uses the same technique and information as the Garbage Collector (GC) uses to scan the values live on the stack
      \end{itemize}
      \begin{itemize}
        \item For each function frame detected, store a pointer to the \typename{frame\_descr} in the \localname{domain\_state->backtrace\_buffer} array, effectively constructing the backtrace
      \end{itemize}
      \onslide<2->{
        \begin{itemize}
            \smallskip
          \item Constructed backtrace:
            \funcname{definitely\_raise},
            \onslide<3->{\funcname{probably\_raise}}
        \end{itemize}
      }
      \vfill
      \mintocaml[firstline=9,lastline=13,highlightlines=12]{../src/backtrace/backtrace.ml}
      \endminipage
    \end{column}
    \begin{column}{0.5\textwidth}
      \raggedleft
      \onslide*<1>{\listStashBacktrace[highlightlines={114-115}]{}}
      \onslide*<2>{\providecommand\step{3}\input{diagrams/backtrace/backtrace.tex}}%
      \onslide*<3>{\providecommand\step{4}\input{diagrams/backtrace/backtrace.tex}}%
    \end{column}
  \end{columns}
\end{frame}

\begin{frame}{Backtraces}{Back to \funcname{caml\_raise\_exn}}
  \begin{columns}[c]
    \begin{column}{0.5\textwidth}
      \minipage[c][0.66\textheight][s]{\columnwidth}
      \begin{itemize}\color{gray}
        \item Checks wheteher exceptions backtrace should be recorded, by checking the \funcarg{domain\_state->backtrace\_active} value
        \item Switches to the C stack to be able to execute C code
        \item Calls \funcname{caml\_stash\_backtrace}, to collect the backtrace up to the next trap
      \end{itemize}
      \onslide*<2->{
        \begin{itemize}
          \item<2-> Executes the exception handler in \funcname{probably\_raise} with \funcname{RESTORE\_EXN\_HANDLER\_OCAML}
            \begin{itemize}
              \item Set \regname{RSP} to \localname{domain\_state->exn\_handler} value
              \item Restore \localname{domain\_state->exn\_handler} from trap
              \item Jump to the handler code with \funcname{ret}
            \end{itemize}
        \end{itemize}
      }
      \vfill
      \onslide*<1>{\mintocaml[firstline=9,lastline=13,highlightlines=12]{../src/backtrace/backtrace.ml}}
      \onslide*<2->{\mintocaml[firstline=15,lastline=21,highlightlines={19-21}]{../src/backtrace/backtrace.ml}}
      \endminipage
    \end{column}
    \begin{column}{0.5\textwidth}
      \centering
      \onslide*<1>{
        \mintc[firstline=190,lastline=192]{../ocaml/runtime/amd64.S}
        \mintc[firstline=234,lastline=237]{../ocaml/runtime/amd64.S}
      }
      \onslide*<2>{
        \mintc[firstline=190,lastline=192,highlightlines={191-192}]{../ocaml/runtime/amd64.S}
        \mintc[firstline=234,lastline=237,highlightlines={235-237}]{../ocaml/runtime/amd64.S}
      }
      \onslide*<1>{\listCamlRaiseExn[highlightlines=720]{}}
      \onslide*<2>{\listCamlRaiseExn[highlightlines=722]{}}
      \onslide*<3->{\providecommand\step{5}\input{diagrams/backtrace/backtrace.tex}}
    \end{column}
  \end{columns}
\end{frame}

\begin{frame}{Backtraces}{\funcname{caml\_probably\_raise}'s handler doesn't catch \typename{ExnC}}
  \begin{columns}[c]
    \begin{column}{0.45\textwidth}
      \minipage[c][0.75\textheight][s]{\columnwidth}
      \begin{itemize}
        \item<1-> \funcname{match \dots with}\footnotemark pattern matching can also match exceptions
        \item<1-> The CMM of \funcname{probably\_raise} shows that this \funcname{match \dots with} is implemented with a pattern matching and a \funcname{try \dots with}
        \item<1-> But this one only matches on \typename{ExnA} exception
        \item<2-> Since the pattern matching fails to catch the exception, \funcname{reraise} is called with our current \typename{ExnC} exception
        \item<3-> Disassembly shows that \funcname{reraise} is actually a call to \funcname{caml\_reraise\_exn}, which is also an assembly function provided by the runtime
      \end{itemize}
      \vfill
      \mintocaml[firstline=15,lastline=21,highlightlines={18-21}]{../src/backtrace/backtrace.ml}
      \endminipage
    \end{column}
    \begin{column}{0.55\textwidth}
      \centering
      \onslide*<1>{\mintcmm[highlightlines={10-18}]{../src/backtrace/camlBacktrace\_\_probably\_raise\_351.cmm}}
      \onslide*<2>{\listcmm[highlightlines=18]{../src/backtrace/camlBacktrace\_\_probably\_raise_351.cmm}{CMM of probably\_raise}}
      \onslide*<3>{\mintobjdump[firstline=5,lastline=35,highlightlines=31]{../src/backtrace/camlBacktrace\_\_probably\_raise_351.objdump}}
    \end{column}
  \end{columns}
  \only<1>{\footnotetext[1]{https://blog.janestreet.com/pattern-matching-and-exception-handling-unite/}}
\end{frame}

\newcommand\listCamlReraiseExn[1][]{
  \listasm[firstline=727,lastline=734,#1]{../ocaml/runtime/amd64.S}{runtime/amd64.S:727}}

\begin{frame}{Backtraces}{Introducing \funcname{caml\_reraise\_exn}}
  \begin{columns}[c]
    \begin{column}{0.5\textwidth}
      \minipage[c][0.66\textheight][s]{\columnwidth}
      \begin{itemize}
        \item<1-> The exception have to be reraised to the next handler
        \item<2-> First, checks if the backtrace should be collected with \funcarg{domain\_state->backtrace\_active}
        \only<3>{
        \begin{itemize}
            \item If not, behaves like a \funcname{raise\_notrace} with \funcname{RESTORE\_EXN\_HANDLER\_OCAML}
        \end{itemize}
        }
        \item<4-> Jumps into \funcname{caml\_raise\_exn}
        \item<5-> Calls \funcname{caml\_stash\_backtrace} again, to scan the next part of the stack: from the trap just pop-ed to the next trap
      \end{itemize}
      \vfill
      \mintocaml[firstline=15,lastline=21,highlightlines={18-21}]{../src/backtrace/backtrace.ml}
      \endminipage
    \end{column}
    \begin{column}{0.5\textwidth}
      \raggedleft
      \onslide*<1>{\listCamlReraiseExn{}}
      \onslide*<2>{\listCamlReraiseExn[highlightlines=729]{}}
      \onslide*<3>{
        \mintc[firstline=190,lastline=192,highlightlines={191-192}]{../ocaml/runtime/amd64.S}
        \mintc[firstline=234,lastline=237,highlightlines={235-237}]{../ocaml/runtime/amd64.S}
        \medskip
        \listCamlReraiseExn[highlightlines=731]{}
      }
      \onslide*<4-5>{
        % TODO refactor with \listCamlReraiseExn
        \mintasm[firstline=727,lastline=734,highlightlines=730]{../ocaml/runtime/amd64.S}
        \medskip
      }
      \onslide*<4>{\listCamlRaiseExn[highlightlines=711]{}}
      \onslide*<5>{\listCamlRaiseExn[highlightlines=720]{}}
    \end{column}
  \end{columns}
\end{frame}

\begin{frame}{Backtraces}{Backtrace collection continues}
  \begin{columns}[c]
    \begin{column}{0.55\textwidth}
      \minipage[c][0.75\textheight][s]{\columnwidth}
      \begin{itemize}
        \item<1-> \funcname{caml\_stash\_backtrace} scans the older part of the stack, up to the next trap
        \item<6-> Controls jumps to the nested exception handler in \funcname{maybe\_raise}, which matches only on \typename{ExnB}
        \item<7-> \funcname{caml\_reraise\_exn} is called again, backtrace collection resumes with \funcname{caml\_stash\_backtrace}
      \end{itemize}
      \medskip
      \begin{itemize}
        \item<2-> Constructed backtrace:
        \begin{itemize}
          \item <2-> \funcname{definitely\_raise}
          \item <2-> \funcname{probably\_raise}
          \item <3-> \funcname{probably\_raise} (re-raise)
          \item <4-> \funcname{maybe\_raise}
          \item <5-> \funcname{main}
          \item <8-> \funcname{main} (re-raise)
        \end{itemize}
      \end{itemize}
      \vfill
      \onslide*<1-5>{\mintocaml[firstline=15,lastline=21]{../src/backtrace/backtrace.ml}}
      \onslide*<6>{\mintocaml[firstline=28,lastline=34,highlightlines=31]{../src/backtrace/backtrace.ml}}
      \onslide*<7->{\mintocaml[firstline=28,lastline=34,highlightlines=33]{../src/backtrace/backtrace.ml}}
      \endminipage
    \end{column}
    \begin{column}{0.45\textwidth}
      \raggedleft
      \onslide*<1>{\providecommand\step{6}\input{diagrams/backtrace/backtrace.tex}}%
      \onslide*<2>{\providecommand\step{6}\input{diagrams/backtrace/backtrace.tex}}%
      \onslide*<3>{\providecommand\step{7}\input{diagrams/backtrace/backtrace.tex}}%
      \onslide*<4>{\providecommand\step{8}\input{diagrams/backtrace/backtrace.tex}}%
      \onslide*<5>{\providecommand\step{9}\input{diagrams/backtrace/backtrace.tex}}%
      \onslide*<6>{\providecommand\step{10}\input{diagrams/backtrace/backtrace.tex}}%
      \onslide*<7>{\providecommand\step{11}\input{diagrams/backtrace/backtrace.tex}}%
      \onslide*<8>{\providecommand\step{12}\input{diagrams/backtrace/backtrace.tex}}%
      \onslide*<9>{\providecommand\step{13}\input{diagrams/backtrace/backtrace.tex}}%
    \end{column}
  \end{columns}
\end{frame}

\begin{frame}{Backtraces}{Printing the backtrace}
  \begin{columns}[c]
    \begin{column}{0.45\textwidth}
      \minipage[c][0.66\textheight][s]{\columnwidth}
      \begin{itemize}
        \item<1-> The exception backtrace can now be printed to \funcarg{stdout} with \funcname{Printexc.print_backtrace}
        \item<2-> First the exception backtrace is retrieved into an OCaml array with \funcname{get_raw_backtrace }
        \item<3-> Which is implemented as the \funcname{caml_get_exception_raw_backtrace} C function in the runtime
        \item<4-> And finally printed with \funcname{print_exception_backtrace}
      \end{itemize}
      \vfill
      \mintocaml[firstline=28,lastline=34,highlightlines=33]{../src/backtrace/backtrace.ml}
      \endminipage
    \end{column}
    \begin{column}{0.55\textwidth}
      \onslide*<1,2>{
        \mintocaml[firstline=100,lastline=101]{../ocaml/stdlib/printexc.ml}
      }
      \only<1>{
        \listocaml[firstline=170,lastline=172]{../ocaml/stdlib/printexc.ml}{stdlib/printexc.ml:170}
      }
      \only<2>{
        \listocaml[firstline=170,lastline=172,highlightlines=172]{../ocaml/stdlib/printexc.ml}{stdlib/printexc.ml:170}
      }
      \only<3>{
        \mintc[firstline=154,lastline=156]{../ocaml/runtime/backtrace.c}
        \listc[firstline=171,lastline=192,highlightlines={184-187}]{../ocaml/runtime/backtrace.c}{runtime/backtrace.c:154}
      }
      \only<4>{
        \listocaml[firstline=155,lastline=165]{../ocaml/stdlib/printexc.ml}{stdlib/printexc.ml:55}
      }
    \end{column}
  \end{columns}
\end{frame}


\subsection{The multiple kinds of raise}
\frameSubsection{}{}

\begin{frame}{Backtraces}{When backtraces are not needed}
  \begin{columns}[c]
    \begin{column}{0.45\textwidth}
      \minipage[c][0.66\textheight][s]{\columnwidth}
      \begin{itemize}
        \item<1-> What if the backtrace for an exception is never needed?
        \item<2-> It's possible to raise the exception explicitly with \funcname{raise\_notrace} to make raising as cheap as possible
        \item<3-> An example of use in the \funcname{dynlink} library of the OCaml compiler
      \end{itemize}
      \vfill
      \onslide<3->{
      \listocaml[firstline=205,lastline=209,highlightlines=207]{../ocaml/otherlibs/dynlink/dynlink\_compilerlibs/misc.ml}{otherlibs/dynlink/dynlink\_compilerlibs/misc.ml:205}
      }
      \endminipage
    \end{column}
    \begin{column}{0.55\textwidth}
    \only<4>{
      \mintcmm[highlightlines=3]{../src/notrace/camlNotrace\_\_fun_347.cmm}
      \listcmm{../src/notrace/camlNotrace\_\_all\_somes\_327.cmm}{all\_somes}
    }
    \onslide*<5->{
      \listobjdump[highlightlines={6-9}]{../src/notrace/camlNotrace\_\_fun_347.objdump}{anonymous function from \funcname{all\_somes}}
    }
    \end{column}
  \end{columns}
\end{frame}

\begin{frame}{Backtraces}{Summary table of the multiple kinds of raise}
  \begin{itemize}
    \item When debugging information is enabled and if the backtrace of an exception isn't needed, it's possible to raise with \funcname{raise\_notrace} for performance
    \item When debugging information is \emph{not} enabled, the compiler optimises \funcname{raise} calls to inline assembly (same effect as using \funcname{raise\_notrace})
  \end{itemize}
  \bigskip
  \centering
  \begin{tabular}{| l | c | c | c | c |}
    \hline
    \parbox{10em}{Debug information \\ (\funcarg{-g} flag)}  & \multicolumn{2}{|c|}{Disabled}                                     & \multicolumn{2}{|c|}{Enabled} \\
    \hline
    Raise kind                                               & \funcname{raise} & \funcname{raise\_notrace}                       & \funcname{raise} & \funcname{raise\_notrace} \\
    \hline
    Compilation                                              & \parbox{8em}{inline assembly\\ (optimization)} & inline assembly   & \funcname{caml\_raise\_exn} & inline assembly \\
    \hline
  \end{tabular}
\end{frame}


%
% Takeaway #5
%
\subsection*{Takeaway \#5}
\frameSubsectionTakeaway{}
\againframe<5>{takeaway}
}%
      \onslide*<7>{\providecommand\step{11}%
% Backtraces
%
\subsection{One advantage of raising with debugging information}
\frameSubsection{}{}

\begin{frame}{Backtraces}{Debugging information \& source code}
  \begin{columns}[c]
    \begin{column}{0.5\textwidth}
      \begin{itemize}
        \item Raising an exception is fun and games, but how do we know where the exception originated? $\rightarrow$ With the backtrace associated with the exception!
        \item In order to get a backtrace from the point where an exception was raised, it's necessary to compile with debugging information enabled (\funcarg{-g} flag for the compiler)
        \item Same example as in the `Nested handlers' section
      \end{itemize}
    \end{column}
    \begin{column}{0.5\textwidth}
      \listocaml[firstline=9,lastline=34]{../src/backtrace/backtrace.ml}{backtrace/backtrace.ml}
    \end{column}
  \end{columns}
\end{frame}

\begin{frame}{Backtraces}{CMM and disassembly of \funcname{definitely\_raise}}
  \begin{columns}[c]
    \begin{column}{0.5\textwidth}
      \minipage[c][0.66\textheight][s]{\columnwidth}
      \begin{itemize}
        \item<1-> An effect of compiling with debugging information (\funcarg{-g}): the CMM no longer uses \funcname{raise\_notrace} to raise the exception but \funcname{raise} instead
        \item<2-> Disassembly shows that CMM \funcname{raise} is actually a call to \funcname{caml\_raise\_exn}, a function in assembler provided by the runtime
      \end{itemize}
      \vfill
      \mintocaml[firstline=9,lastline=13,highlightlines=12]{../src/backtrace/backtrace.ml}
      \endminipage
    \end{column}
    \begin{column}{0.5\textwidth}
      \onslide*<1>{
        \mintcmm[highlightlines=5]{../src/backtrace/camlBacktrace\_\_definitely\_raise\_347.cmm}
      }
      \onslide*<2>{
        \mintobjdump[firstline=2,highlightlines=14]{../src/backtrace/camlBacktrace\_\_definitely\_raise\_347.objdump}
      }
    \end{column}
  \end{columns}
\end{frame}

\begin{frame}{Backtraces}{Introducing \funcname{caml\_raise\_exn}}
  \begin{columns}[c]
    \begin{column}{0.5\textwidth}
      \minipage[c][0.66\textheight][s]{\columnwidth}
      \onslide*<1>{
        \begin{itemize}
          \item Raising the exception is performed using \funcname{caml\_raise\_exn} which is
            \begin{itemize}
              \item An assembly function provided by the runtime
              \item Architecture specific
            \end{itemize}
          \item Let's see what it does differently from \funcname{raise\_notrace}\dots
          \item We are about to raise \typename{ExnC} with \funcname{caml\_raise\_exn} from \funcname{definitely\_raise}
        \end{itemize}
      }
      \onslide*<2>{
        \mintobjdump[firstline=2,highlightlines=14]{../src/backtrace/camlBacktrace\_\_definitely\_raise\_347.objdump}
      }
      \vfill
      \mintocaml[firstline=9,lastline=13,highlightlines=12]{../src/backtrace/backtrace.ml}
      \endminipage
    \end{column}
    \begin{column}{0.5\textwidth}
      \centering
      \providecommand\step{1}\input{diagrams/backtrace/backtrace.tex}
    \end{column}
  \end{columns}
\end{frame}

\begin{frame}{Backtraces}{But before that, we need more fields from \typename{caml\_domain\_state}}
  \begin{columns}
    \begin{column}{0.5\textwidth}
      \begin{itemize}
        \item Remember \typename{caml\_domain\_state} the per-domain runtime state?
        \item Some other members are of interest to us now\dots
        \item \localname{backtrace\_active} is used to know if the runtime should collect backtraces
        \item \localname{backtrace\_active} can be set by
          \begin{itemize}
            \item The \funcarg{b} flag in the \funcname{OCAMLRUNPARAM} environment variable
            \item \mintinline{ocaml}{Printexc.record_backtrace : bool -> unit} \footnotemark
          \end{itemize}
      \end{itemize}
    \end{column}
    \begin{column}{0.5\textwidth}
      \begin{listing}
        \mintc[highlightlines={9-11}]{src/ocaml/domain\_state\_backtrace.h}
        \caption{runtime/caml/domain\_state.h}
      \end{listing}
    \end{column}
  \end{columns}
  \footnotetext[1]{https://v2.ocaml.org/api/Printexc.html}
\end{frame}

\newcommand\listCamlRaiseExn[1][]{
  \listasm[firstline=702,lastline=725,#1]{../ocaml/runtime/amd64.S}{runtime/amd64.S:702}}

\begin{frame}{Backtraces}{Walk through \funcname{caml\_raise\_exn}}
  \begin{columns}[T]
    \begin{column}{0.5\textwidth}
      \begin{itemize}
        \item<1-> First checks whether exceptions backtrace should be recorded
        \item<1-> By checking the \funcarg{domain\_state->backtrace\_active} value
        \item<2-> If not, acts as \funcname{raise\_notrace} with \funcname{RESTORE\_EXN\_HANDLER\_OCAML}
          \begin{itemize}
            \item Set \regname{RSP} to \localname{domain\_state->exn\_handler} value
            \item Restore \localname{domain\_state->exn\_handler} from trap
            \item Jump with \funcname{ret} (same effect as using \funcname{pop} and \funcname{jmp})
          \end{itemize}
        \item<3-> Otherwise, switches to the C stack to be able to execute C code
        \item<4-> Calls \funcname{caml\_stash\_backtrace}, a C function provided by the runtime\ignorespaces
          \onslide<6->{, with
            \begin{itemize}
              \item \funcarg{exn}: exception value
              \item \funcarg{pc}: code address of the raising point
              \item \funcarg{sp}: stack address of the raising function
              \item \funcarg{trapsp}: stack address of the next handler
            \end{itemize}
          }
      \end{itemize}
      \bigskip
      \mintocaml[firstline=9,lastline=13,highlightlines=12]{../src/backtrace/backtrace.ml}
    \end{column}
    \begin{column}{0.5\textwidth}
      \raggedleft
      \onslide*<1,3-4>{
        \mintc[firstline=190,lastline=192]{../ocaml/runtime/amd64.S}
        \mintc[firstline=234,lastline=237]{../ocaml/runtime/amd64.S}
      }
      \onslide*<2>{
        \mintc[firstline=190,lastline=192,highlightlines={191-192}]{../ocaml/runtime/amd64.S}
        \mintc[firstline=234,lastline=237,highlightlines={235-237}]{../ocaml/runtime/amd64.S}
      }
      \onslide*<1>{\listCamlRaiseExn[highlightlines=705]{}}%
      \onslide*<2>{\listCamlRaiseExn[highlightlines={707-708}]{}}%
      \onslide*<3>{\listCamlRaiseExn[highlightlines={712-714}]{}}%
      \onslide*<4>{\listCamlRaiseExn[highlightlines={715-720}]{}}%
      \onslide*<5>{\providecommand\step{1}\input{diagrams/backtrace/backtrace.tex}}%
      \onslide*<6>{\providecommand\step{2}\input{diagrams/backtrace/backtrace.tex}}%
    \end{column}
  \end{columns}
\end{frame}

\newcommand\listStashBacktrace[1][]{
  \listc[firstline=91,lastline=120,#1]{../ocaml/runtime/backtrace\_nat.c}{runtime/backtrace\_nat.c:91}}

\begin{frame}{Backtraces}{Introducing \funcname{caml\_stash\_backtrace}}
  \begin{columns}[c]
    \begin{column}{0.5\textwidth}
      \minipage[c][0.66\textheight][s]{\columnwidth}
      \begin{itemize}
        \item<1-> A C function provided by the runtime (specific to native code)
        \item<2-> It scans the stack, between the stack pointer at the raising point up to the next trap, in order to collect the names of the detected function frames
          \onslide*<2>{
            \begin{itemize}
              \item And more information, like line number, column number...
            \end{itemize}
          }
        \item<3-> It uses the same technique and information as the Garbage Collector (GC) uses to scan the values live on the stack
          \onslide*<4>{
            \begin{itemize}
              \item \typename{frame\_descr} are at the core of stack unwinding
            \end{itemize}
          }
      \end{itemize}
      \vfill
      \mintocaml[firstline=9,lastline=13,highlightlines=12]{../src/backtrace/backtrace.ml}
      \endminipage
    \end{column}
    \begin{column}{0.5\textwidth}
      \onslide*<1>{\listStashBacktrace[highlightlines=91]{}}
      \onslide*<2>{\listStashBacktrace[highlightlines={91,117-118}]{}}
      \onslide*<3->{\listStashBacktrace[highlightlines={94,106,109-110}]{}}
    \end{column}
  \end{columns}
\end{frame}

\newcommand\listFrameDescr[1][]{
  \listocaml[firstline=111,lastline=124,#1]{../ocaml/asmcomp/emitaux.ml}{asmcomp/emitaux.ml:111}}
\newcommand\listDebugInfo[1][]{
  \listocaml[firstline=38,lastline=49,#1]{../ocaml/lambda/debuginfo.mli}{lambda/debuginfo.mli:38}}

\begin{frame}{Backtraces}{\typename{frame\_descr} and \typename{frame\_debuginfo} in a nutshell}
  \begin{columns}[c]
    \begin{column}{0.5\textwidth}
      \begin{itemize}
        \item<1-> For every return address in an OCaml program, there is a associated \typename{frame\_descr}
        \item<2-> A \typename{frame\_descr} specifies
        \begin{itemize}
          \item<2-> The size of the function stack frame
          \item<3-> Where live values are on the stack (so that the GC can mark them as live and prevent from being swept)
        \end{itemize}
        \item<4-> When compiled with debug information, \typename{frame\_descr} will be linked to a \typename{frame\_debuginfo}
        \begin{itemize}
          \item<5-> Contains information about the position in the source code (file name, line and column number)
        \end{itemize}
      \end{itemize}
    \end{column}
    \begin{column}{0.5\textwidth}
        \onslide*<1>{\listFrameDescr[highlightlines={118-122}]{}}
        \onslide*<2>{\listFrameDescr[highlightlines={120}]{}}
        \onslide*<3>{\listFrameDescr[highlightlines={121}]{}}
        \onslide*<4->{\listFrameDescr[highlightlines={122}]{}}
        \medskip
        \onslide*<1-3>{\listDebugInfo{}}
        \onslide*<4>{\listDebugInfo[highlightlines={38,49}]{}}
        \onslide*<5>{\listDebugInfo[highlightlines={39-46}]{}}
    \end{column}
  \end{columns}
\end{frame}

\begin{frame}{Backtraces}{Scanning the stack with \typename{frame\_descr}}
  \begin{columns}[c]
    \begin{column}{0.5\textwidth}
      \begin{itemize}
        \item Given the return address of the code that raises the exception by calling \funcname{caml\_raise\_exn} (\funcarg{pc} argument of \funcname{caml\_stash\_backtrace})
        \item And the position of the stack pointer (\funcarg{sp} argument of \funcname{caml\_stash\_backtrace})
        \item The frame size given by \typename{frame\_descr}.\funcarg{frame\_size}, allows to find the end of the previous frame
        \item And the return address of the caller of this function
        \bigskip
        \onslide<3->{
        \item Note that traps (here the reddish \funcname{ExnA}, \funcname{ExnB} and \funcname{ExnC} blocks) belong to the frame that installed them, and so the frame size changes when traps are removed
        }
      \end{itemize}
    \end{column}
    \begin{column}{0.5\textwidth}
      \raggedleft
      \onslide*<1>{\providecommand\step{2}\input{diagrams/backtrace/backtrace.tex}}%
      \onslide*<2>{\providecommand\step{1}\input{diagrams/backtrace/scanstack.tex}}%
      \onslide*<3>{\providecommand\step{2}\input{diagrams/backtrace/scanstack.tex}}%
      \onslide*<4>{\providecommand\step{3}\input{diagrams/backtrace/scanstack.tex}}%
      \onslide*<5>{\providecommand\step{4}\input{diagrams/backtrace/scanstack.tex}}%
      \onslide*<6>{\providecommand\step{5}\input{diagrams/backtrace/scanstack.tex}}%
    \end{column}
  \end{columns}
\end{frame}

% Duplicate of previous-previous slide
\begin{frame}{Backtraces}{Continuing on \funcname{caml\_stash\_backtrace}}
  \begin{columns}[c]
    \begin{column}{0.5\textwidth}
      \minipage[c][0.75\textheight][s]{\columnwidth}
      \begin{itemize}
          \color{gray}
        \item A C function provided by the runtime (specific to native code)
        \item It scans the stack, between the stack pointer at the raising point up to the next trap, in order to collect the names of the detected function frames
        \item It uses the same technique and information as the Garbage Collector (GC) uses to scan the values live on the stack
      \end{itemize}
      \begin{itemize}
        \item For each function frame detected, store a pointer to the \typename{frame\_descr} in the \localname{domain\_state->backtrace\_buffer} array, effectively constructing the backtrace
      \end{itemize}
      \onslide<2->{
        \begin{itemize}
            \smallskip
          \item Constructed backtrace:
            \funcname{definitely\_raise},
            \onslide<3->{\funcname{probably\_raise}}
        \end{itemize}
      }
      \vfill
      \mintocaml[firstline=9,lastline=13,highlightlines=12]{../src/backtrace/backtrace.ml}
      \endminipage
    \end{column}
    \begin{column}{0.5\textwidth}
      \raggedleft
      \onslide*<1>{\listStashBacktrace[highlightlines={114-115}]{}}
      \onslide*<2>{\providecommand\step{3}\input{diagrams/backtrace/backtrace.tex}}%
      \onslide*<3>{\providecommand\step{4}\input{diagrams/backtrace/backtrace.tex}}%
    \end{column}
  \end{columns}
\end{frame}

\begin{frame}{Backtraces}{Back to \funcname{caml\_raise\_exn}}
  \begin{columns}[c]
    \begin{column}{0.5\textwidth}
      \minipage[c][0.66\textheight][s]{\columnwidth}
      \begin{itemize}\color{gray}
        \item Checks wheteher exceptions backtrace should be recorded, by checking the \funcarg{domain\_state->backtrace\_active} value
        \item Switches to the C stack to be able to execute C code
        \item Calls \funcname{caml\_stash\_backtrace}, to collect the backtrace up to the next trap
      \end{itemize}
      \onslide*<2->{
        \begin{itemize}
          \item<2-> Executes the exception handler in \funcname{probably\_raise} with \funcname{RESTORE\_EXN\_HANDLER\_OCAML}
            \begin{itemize}
              \item Set \regname{RSP} to \localname{domain\_state->exn\_handler} value
              \item Restore \localname{domain\_state->exn\_handler} from trap
              \item Jump to the handler code with \funcname{ret}
            \end{itemize}
        \end{itemize}
      }
      \vfill
      \onslide*<1>{\mintocaml[firstline=9,lastline=13,highlightlines=12]{../src/backtrace/backtrace.ml}}
      \onslide*<2->{\mintocaml[firstline=15,lastline=21,highlightlines={19-21}]{../src/backtrace/backtrace.ml}}
      \endminipage
    \end{column}
    \begin{column}{0.5\textwidth}
      \centering
      \onslide*<1>{
        \mintc[firstline=190,lastline=192]{../ocaml/runtime/amd64.S}
        \mintc[firstline=234,lastline=237]{../ocaml/runtime/amd64.S}
      }
      \onslide*<2>{
        \mintc[firstline=190,lastline=192,highlightlines={191-192}]{../ocaml/runtime/amd64.S}
        \mintc[firstline=234,lastline=237,highlightlines={235-237}]{../ocaml/runtime/amd64.S}
      }
      \onslide*<1>{\listCamlRaiseExn[highlightlines=720]{}}
      \onslide*<2>{\listCamlRaiseExn[highlightlines=722]{}}
      \onslide*<3->{\providecommand\step{5}\input{diagrams/backtrace/backtrace.tex}}
    \end{column}
  \end{columns}
\end{frame}

\begin{frame}{Backtraces}{\funcname{caml\_probably\_raise}'s handler doesn't catch \typename{ExnC}}
  \begin{columns}[c]
    \begin{column}{0.45\textwidth}
      \minipage[c][0.75\textheight][s]{\columnwidth}
      \begin{itemize}
        \item<1-> \funcname{match \dots with}\footnotemark pattern matching can also match exceptions
        \item<1-> The CMM of \funcname{probably\_raise} shows that this \funcname{match \dots with} is implemented with a pattern matching and a \funcname{try \dots with}
        \item<1-> But this one only matches on \typename{ExnA} exception
        \item<2-> Since the pattern matching fails to catch the exception, \funcname{reraise} is called with our current \typename{ExnC} exception
        \item<3-> Disassembly shows that \funcname{reraise} is actually a call to \funcname{caml\_reraise\_exn}, which is also an assembly function provided by the runtime
      \end{itemize}
      \vfill
      \mintocaml[firstline=15,lastline=21,highlightlines={18-21}]{../src/backtrace/backtrace.ml}
      \endminipage
    \end{column}
    \begin{column}{0.55\textwidth}
      \centering
      \onslide*<1>{\mintcmm[highlightlines={10-18}]{../src/backtrace/camlBacktrace\_\_probably\_raise\_351.cmm}}
      \onslide*<2>{\listcmm[highlightlines=18]{../src/backtrace/camlBacktrace\_\_probably\_raise_351.cmm}{CMM of probably\_raise}}
      \onslide*<3>{\mintobjdump[firstline=5,lastline=35,highlightlines=31]{../src/backtrace/camlBacktrace\_\_probably\_raise_351.objdump}}
    \end{column}
  \end{columns}
  \only<1>{\footnotetext[1]{https://blog.janestreet.com/pattern-matching-and-exception-handling-unite/}}
\end{frame}

\newcommand\listCamlReraiseExn[1][]{
  \listasm[firstline=727,lastline=734,#1]{../ocaml/runtime/amd64.S}{runtime/amd64.S:727}}

\begin{frame}{Backtraces}{Introducing \funcname{caml\_reraise\_exn}}
  \begin{columns}[c]
    \begin{column}{0.5\textwidth}
      \minipage[c][0.66\textheight][s]{\columnwidth}
      \begin{itemize}
        \item<1-> The exception have to be reraised to the next handler
        \item<2-> First, checks if the backtrace should be collected with \funcarg{domain\_state->backtrace\_active}
        \only<3>{
        \begin{itemize}
            \item If not, behaves like a \funcname{raise\_notrace} with \funcname{RESTORE\_EXN\_HANDLER\_OCAML}
        \end{itemize}
        }
        \item<4-> Jumps into \funcname{caml\_raise\_exn}
        \item<5-> Calls \funcname{caml\_stash\_backtrace} again, to scan the next part of the stack: from the trap just pop-ed to the next trap
      \end{itemize}
      \vfill
      \mintocaml[firstline=15,lastline=21,highlightlines={18-21}]{../src/backtrace/backtrace.ml}
      \endminipage
    \end{column}
    \begin{column}{0.5\textwidth}
      \raggedleft
      \onslide*<1>{\listCamlReraiseExn{}}
      \onslide*<2>{\listCamlReraiseExn[highlightlines=729]{}}
      \onslide*<3>{
        \mintc[firstline=190,lastline=192,highlightlines={191-192}]{../ocaml/runtime/amd64.S}
        \mintc[firstline=234,lastline=237,highlightlines={235-237}]{../ocaml/runtime/amd64.S}
        \medskip
        \listCamlReraiseExn[highlightlines=731]{}
      }
      \onslide*<4-5>{
        % TODO refactor with \listCamlReraiseExn
        \mintasm[firstline=727,lastline=734,highlightlines=730]{../ocaml/runtime/amd64.S}
        \medskip
      }
      \onslide*<4>{\listCamlRaiseExn[highlightlines=711]{}}
      \onslide*<5>{\listCamlRaiseExn[highlightlines=720]{}}
    \end{column}
  \end{columns}
\end{frame}

\begin{frame}{Backtraces}{Backtrace collection continues}
  \begin{columns}[c]
    \begin{column}{0.55\textwidth}
      \minipage[c][0.75\textheight][s]{\columnwidth}
      \begin{itemize}
        \item<1-> \funcname{caml\_stash\_backtrace} scans the older part of the stack, up to the next trap
        \item<6-> Controls jumps to the nested exception handler in \funcname{maybe\_raise}, which matches only on \typename{ExnB}
        \item<7-> \funcname{caml\_reraise\_exn} is called again, backtrace collection resumes with \funcname{caml\_stash\_backtrace}
      \end{itemize}
      \medskip
      \begin{itemize}
        \item<2-> Constructed backtrace:
        \begin{itemize}
          \item <2-> \funcname{definitely\_raise}
          \item <2-> \funcname{probably\_raise}
          \item <3-> \funcname{probably\_raise} (re-raise)
          \item <4-> \funcname{maybe\_raise}
          \item <5-> \funcname{main}
          \item <8-> \funcname{main} (re-raise)
        \end{itemize}
      \end{itemize}
      \vfill
      \onslide*<1-5>{\mintocaml[firstline=15,lastline=21]{../src/backtrace/backtrace.ml}}
      \onslide*<6>{\mintocaml[firstline=28,lastline=34,highlightlines=31]{../src/backtrace/backtrace.ml}}
      \onslide*<7->{\mintocaml[firstline=28,lastline=34,highlightlines=33]{../src/backtrace/backtrace.ml}}
      \endminipage
    \end{column}
    \begin{column}{0.45\textwidth}
      \raggedleft
      \onslide*<1>{\providecommand\step{6}\input{diagrams/backtrace/backtrace.tex}}%
      \onslide*<2>{\providecommand\step{6}\input{diagrams/backtrace/backtrace.tex}}%
      \onslide*<3>{\providecommand\step{7}\input{diagrams/backtrace/backtrace.tex}}%
      \onslide*<4>{\providecommand\step{8}\input{diagrams/backtrace/backtrace.tex}}%
      \onslide*<5>{\providecommand\step{9}\input{diagrams/backtrace/backtrace.tex}}%
      \onslide*<6>{\providecommand\step{10}\input{diagrams/backtrace/backtrace.tex}}%
      \onslide*<7>{\providecommand\step{11}\input{diagrams/backtrace/backtrace.tex}}%
      \onslide*<8>{\providecommand\step{12}\input{diagrams/backtrace/backtrace.tex}}%
      \onslide*<9>{\providecommand\step{13}\input{diagrams/backtrace/backtrace.tex}}%
    \end{column}
  \end{columns}
\end{frame}

\begin{frame}{Backtraces}{Printing the backtrace}
  \begin{columns}[c]
    \begin{column}{0.45\textwidth}
      \minipage[c][0.66\textheight][s]{\columnwidth}
      \begin{itemize}
        \item<1-> The exception backtrace can now be printed to \funcarg{stdout} with \funcname{Printexc.print_backtrace}
        \item<2-> First the exception backtrace is retrieved into an OCaml array with \funcname{get_raw_backtrace }
        \item<3-> Which is implemented as the \funcname{caml_get_exception_raw_backtrace} C function in the runtime
        \item<4-> And finally printed with \funcname{print_exception_backtrace}
      \end{itemize}
      \vfill
      \mintocaml[firstline=28,lastline=34,highlightlines=33]{../src/backtrace/backtrace.ml}
      \endminipage
    \end{column}
    \begin{column}{0.55\textwidth}
      \onslide*<1,2>{
        \mintocaml[firstline=100,lastline=101]{../ocaml/stdlib/printexc.ml}
      }
      \only<1>{
        \listocaml[firstline=170,lastline=172]{../ocaml/stdlib/printexc.ml}{stdlib/printexc.ml:170}
      }
      \only<2>{
        \listocaml[firstline=170,lastline=172,highlightlines=172]{../ocaml/stdlib/printexc.ml}{stdlib/printexc.ml:170}
      }
      \only<3>{
        \mintc[firstline=154,lastline=156]{../ocaml/runtime/backtrace.c}
        \listc[firstline=171,lastline=192,highlightlines={184-187}]{../ocaml/runtime/backtrace.c}{runtime/backtrace.c:154}
      }
      \only<4>{
        \listocaml[firstline=155,lastline=165]{../ocaml/stdlib/printexc.ml}{stdlib/printexc.ml:55}
      }
    \end{column}
  \end{columns}
\end{frame}


\subsection{The multiple kinds of raise}
\frameSubsection{}{}

\begin{frame}{Backtraces}{When backtraces are not needed}
  \begin{columns}[c]
    \begin{column}{0.45\textwidth}
      \minipage[c][0.66\textheight][s]{\columnwidth}
      \begin{itemize}
        \item<1-> What if the backtrace for an exception is never needed?
        \item<2-> It's possible to raise the exception explicitly with \funcname{raise\_notrace} to make raising as cheap as possible
        \item<3-> An example of use in the \funcname{dynlink} library of the OCaml compiler
      \end{itemize}
      \vfill
      \onslide<3->{
      \listocaml[firstline=205,lastline=209,highlightlines=207]{../ocaml/otherlibs/dynlink/dynlink\_compilerlibs/misc.ml}{otherlibs/dynlink/dynlink\_compilerlibs/misc.ml:205}
      }
      \endminipage
    \end{column}
    \begin{column}{0.55\textwidth}
    \only<4>{
      \mintcmm[highlightlines=3]{../src/notrace/camlNotrace\_\_fun_347.cmm}
      \listcmm{../src/notrace/camlNotrace\_\_all\_somes\_327.cmm}{all\_somes}
    }
    \onslide*<5->{
      \listobjdump[highlightlines={6-9}]{../src/notrace/camlNotrace\_\_fun_347.objdump}{anonymous function from \funcname{all\_somes}}
    }
    \end{column}
  \end{columns}
\end{frame}

\begin{frame}{Backtraces}{Summary table of the multiple kinds of raise}
  \begin{itemize}
    \item When debugging information is enabled and if the backtrace of an exception isn't needed, it's possible to raise with \funcname{raise\_notrace} for performance
    \item When debugging information is \emph{not} enabled, the compiler optimises \funcname{raise} calls to inline assembly (same effect as using \funcname{raise\_notrace})
  \end{itemize}
  \bigskip
  \centering
  \begin{tabular}{| l | c | c | c | c |}
    \hline
    \parbox{10em}{Debug information \\ (\funcarg{-g} flag)}  & \multicolumn{2}{|c|}{Disabled}                                     & \multicolumn{2}{|c|}{Enabled} \\
    \hline
    Raise kind                                               & \funcname{raise} & \funcname{raise\_notrace}                       & \funcname{raise} & \funcname{raise\_notrace} \\
    \hline
    Compilation                                              & \parbox{8em}{inline assembly\\ (optimization)} & inline assembly   & \funcname{caml\_raise\_exn} & inline assembly \\
    \hline
  \end{tabular}
\end{frame}


%
% Takeaway #5
%
\subsection*{Takeaway \#5}
\frameSubsectionTakeaway{}
\againframe<5>{takeaway}
}%
      \onslide*<8>{\providecommand\step{12}%
% Backtraces
%
\subsection{One advantage of raising with debugging information}
\frameSubsection{}{}

\begin{frame}{Backtraces}{Debugging information \& source code}
  \begin{columns}[c]
    \begin{column}{0.5\textwidth}
      \begin{itemize}
        \item Raising an exception is fun and games, but how do we know where the exception originated? $\rightarrow$ With the backtrace associated with the exception!
        \item In order to get a backtrace from the point where an exception was raised, it's necessary to compile with debugging information enabled (\funcarg{-g} flag for the compiler)
        \item Same example as in the `Nested handlers' section
      \end{itemize}
    \end{column}
    \begin{column}{0.5\textwidth}
      \listocaml[firstline=9,lastline=34]{../src/backtrace/backtrace.ml}{backtrace/backtrace.ml}
    \end{column}
  \end{columns}
\end{frame}

\begin{frame}{Backtraces}{CMM and disassembly of \funcname{definitely\_raise}}
  \begin{columns}[c]
    \begin{column}{0.5\textwidth}
      \minipage[c][0.66\textheight][s]{\columnwidth}
      \begin{itemize}
        \item<1-> An effect of compiling with debugging information (\funcarg{-g}): the CMM no longer uses \funcname{raise\_notrace} to raise the exception but \funcname{raise} instead
        \item<2-> Disassembly shows that CMM \funcname{raise} is actually a call to \funcname{caml\_raise\_exn}, a function in assembler provided by the runtime
      \end{itemize}
      \vfill
      \mintocaml[firstline=9,lastline=13,highlightlines=12]{../src/backtrace/backtrace.ml}
      \endminipage
    \end{column}
    \begin{column}{0.5\textwidth}
      \onslide*<1>{
        \mintcmm[highlightlines=5]{../src/backtrace/camlBacktrace\_\_definitely\_raise\_347.cmm}
      }
      \onslide*<2>{
        \mintobjdump[firstline=2,highlightlines=14]{../src/backtrace/camlBacktrace\_\_definitely\_raise\_347.objdump}
      }
    \end{column}
  \end{columns}
\end{frame}

\begin{frame}{Backtraces}{Introducing \funcname{caml\_raise\_exn}}
  \begin{columns}[c]
    \begin{column}{0.5\textwidth}
      \minipage[c][0.66\textheight][s]{\columnwidth}
      \onslide*<1>{
        \begin{itemize}
          \item Raising the exception is performed using \funcname{caml\_raise\_exn} which is
            \begin{itemize}
              \item An assembly function provided by the runtime
              \item Architecture specific
            \end{itemize}
          \item Let's see what it does differently from \funcname{raise\_notrace}\dots
          \item We are about to raise \typename{ExnC} with \funcname{caml\_raise\_exn} from \funcname{definitely\_raise}
        \end{itemize}
      }
      \onslide*<2>{
        \mintobjdump[firstline=2,highlightlines=14]{../src/backtrace/camlBacktrace\_\_definitely\_raise\_347.objdump}
      }
      \vfill
      \mintocaml[firstline=9,lastline=13,highlightlines=12]{../src/backtrace/backtrace.ml}
      \endminipage
    \end{column}
    \begin{column}{0.5\textwidth}
      \centering
      \providecommand\step{1}\input{diagrams/backtrace/backtrace.tex}
    \end{column}
  \end{columns}
\end{frame}

\begin{frame}{Backtraces}{But before that, we need more fields from \typename{caml\_domain\_state}}
  \begin{columns}
    \begin{column}{0.5\textwidth}
      \begin{itemize}
        \item Remember \typename{caml\_domain\_state} the per-domain runtime state?
        \item Some other members are of interest to us now\dots
        \item \localname{backtrace\_active} is used to know if the runtime should collect backtraces
        \item \localname{backtrace\_active} can be set by
          \begin{itemize}
            \item The \funcarg{b} flag in the \funcname{OCAMLRUNPARAM} environment variable
            \item \mintinline{ocaml}{Printexc.record_backtrace : bool -> unit} \footnotemark
          \end{itemize}
      \end{itemize}
    \end{column}
    \begin{column}{0.5\textwidth}
      \begin{listing}
        \mintc[highlightlines={9-11}]{src/ocaml/domain\_state\_backtrace.h}
        \caption{runtime/caml/domain\_state.h}
      \end{listing}
    \end{column}
  \end{columns}
  \footnotetext[1]{https://v2.ocaml.org/api/Printexc.html}
\end{frame}

\newcommand\listCamlRaiseExn[1][]{
  \listasm[firstline=702,lastline=725,#1]{../ocaml/runtime/amd64.S}{runtime/amd64.S:702}}

\begin{frame}{Backtraces}{Walk through \funcname{caml\_raise\_exn}}
  \begin{columns}[T]
    \begin{column}{0.5\textwidth}
      \begin{itemize}
        \item<1-> First checks whether exceptions backtrace should be recorded
        \item<1-> By checking the \funcarg{domain\_state->backtrace\_active} value
        \item<2-> If not, acts as \funcname{raise\_notrace} with \funcname{RESTORE\_EXN\_HANDLER\_OCAML}
          \begin{itemize}
            \item Set \regname{RSP} to \localname{domain\_state->exn\_handler} value
            \item Restore \localname{domain\_state->exn\_handler} from trap
            \item Jump with \funcname{ret} (same effect as using \funcname{pop} and \funcname{jmp})
          \end{itemize}
        \item<3-> Otherwise, switches to the C stack to be able to execute C code
        \item<4-> Calls \funcname{caml\_stash\_backtrace}, a C function provided by the runtime\ignorespaces
          \onslide<6->{, with
            \begin{itemize}
              \item \funcarg{exn}: exception value
              \item \funcarg{pc}: code address of the raising point
              \item \funcarg{sp}: stack address of the raising function
              \item \funcarg{trapsp}: stack address of the next handler
            \end{itemize}
          }
      \end{itemize}
      \bigskip
      \mintocaml[firstline=9,lastline=13,highlightlines=12]{../src/backtrace/backtrace.ml}
    \end{column}
    \begin{column}{0.5\textwidth}
      \raggedleft
      \onslide*<1,3-4>{
        \mintc[firstline=190,lastline=192]{../ocaml/runtime/amd64.S}
        \mintc[firstline=234,lastline=237]{../ocaml/runtime/amd64.S}
      }
      \onslide*<2>{
        \mintc[firstline=190,lastline=192,highlightlines={191-192}]{../ocaml/runtime/amd64.S}
        \mintc[firstline=234,lastline=237,highlightlines={235-237}]{../ocaml/runtime/amd64.S}
      }
      \onslide*<1>{\listCamlRaiseExn[highlightlines=705]{}}%
      \onslide*<2>{\listCamlRaiseExn[highlightlines={707-708}]{}}%
      \onslide*<3>{\listCamlRaiseExn[highlightlines={712-714}]{}}%
      \onslide*<4>{\listCamlRaiseExn[highlightlines={715-720}]{}}%
      \onslide*<5>{\providecommand\step{1}\input{diagrams/backtrace/backtrace.tex}}%
      \onslide*<6>{\providecommand\step{2}\input{diagrams/backtrace/backtrace.tex}}%
    \end{column}
  \end{columns}
\end{frame}

\newcommand\listStashBacktrace[1][]{
  \listc[firstline=91,lastline=120,#1]{../ocaml/runtime/backtrace\_nat.c}{runtime/backtrace\_nat.c:91}}

\begin{frame}{Backtraces}{Introducing \funcname{caml\_stash\_backtrace}}
  \begin{columns}[c]
    \begin{column}{0.5\textwidth}
      \minipage[c][0.66\textheight][s]{\columnwidth}
      \begin{itemize}
        \item<1-> A C function provided by the runtime (specific to native code)
        \item<2-> It scans the stack, between the stack pointer at the raising point up to the next trap, in order to collect the names of the detected function frames
          \onslide*<2>{
            \begin{itemize}
              \item And more information, like line number, column number...
            \end{itemize}
          }
        \item<3-> It uses the same technique and information as the Garbage Collector (GC) uses to scan the values live on the stack
          \onslide*<4>{
            \begin{itemize}
              \item \typename{frame\_descr} are at the core of stack unwinding
            \end{itemize}
          }
      \end{itemize}
      \vfill
      \mintocaml[firstline=9,lastline=13,highlightlines=12]{../src/backtrace/backtrace.ml}
      \endminipage
    \end{column}
    \begin{column}{0.5\textwidth}
      \onslide*<1>{\listStashBacktrace[highlightlines=91]{}}
      \onslide*<2>{\listStashBacktrace[highlightlines={91,117-118}]{}}
      \onslide*<3->{\listStashBacktrace[highlightlines={94,106,109-110}]{}}
    \end{column}
  \end{columns}
\end{frame}

\newcommand\listFrameDescr[1][]{
  \listocaml[firstline=111,lastline=124,#1]{../ocaml/asmcomp/emitaux.ml}{asmcomp/emitaux.ml:111}}
\newcommand\listDebugInfo[1][]{
  \listocaml[firstline=38,lastline=49,#1]{../ocaml/lambda/debuginfo.mli}{lambda/debuginfo.mli:38}}

\begin{frame}{Backtraces}{\typename{frame\_descr} and \typename{frame\_debuginfo} in a nutshell}
  \begin{columns}[c]
    \begin{column}{0.5\textwidth}
      \begin{itemize}
        \item<1-> For every return address in an OCaml program, there is a associated \typename{frame\_descr}
        \item<2-> A \typename{frame\_descr} specifies
        \begin{itemize}
          \item<2-> The size of the function stack frame
          \item<3-> Where live values are on the stack (so that the GC can mark them as live and prevent from being swept)
        \end{itemize}
        \item<4-> When compiled with debug information, \typename{frame\_descr} will be linked to a \typename{frame\_debuginfo}
        \begin{itemize}
          \item<5-> Contains information about the position in the source code (file name, line and column number)
        \end{itemize}
      \end{itemize}
    \end{column}
    \begin{column}{0.5\textwidth}
        \onslide*<1>{\listFrameDescr[highlightlines={118-122}]{}}
        \onslide*<2>{\listFrameDescr[highlightlines={120}]{}}
        \onslide*<3>{\listFrameDescr[highlightlines={121}]{}}
        \onslide*<4->{\listFrameDescr[highlightlines={122}]{}}
        \medskip
        \onslide*<1-3>{\listDebugInfo{}}
        \onslide*<4>{\listDebugInfo[highlightlines={38,49}]{}}
        \onslide*<5>{\listDebugInfo[highlightlines={39-46}]{}}
    \end{column}
  \end{columns}
\end{frame}

\begin{frame}{Backtraces}{Scanning the stack with \typename{frame\_descr}}
  \begin{columns}[c]
    \begin{column}{0.5\textwidth}
      \begin{itemize}
        \item Given the return address of the code that raises the exception by calling \funcname{caml\_raise\_exn} (\funcarg{pc} argument of \funcname{caml\_stash\_backtrace})
        \item And the position of the stack pointer (\funcarg{sp} argument of \funcname{caml\_stash\_backtrace})
        \item The frame size given by \typename{frame\_descr}.\funcarg{frame\_size}, allows to find the end of the previous frame
        \item And the return address of the caller of this function
        \bigskip
        \onslide<3->{
        \item Note that traps (here the reddish \funcname{ExnA}, \funcname{ExnB} and \funcname{ExnC} blocks) belong to the frame that installed them, and so the frame size changes when traps are removed
        }
      \end{itemize}
    \end{column}
    \begin{column}{0.5\textwidth}
      \raggedleft
      \onslide*<1>{\providecommand\step{2}\input{diagrams/backtrace/backtrace.tex}}%
      \onslide*<2>{\providecommand\step{1}\input{diagrams/backtrace/scanstack.tex}}%
      \onslide*<3>{\providecommand\step{2}\input{diagrams/backtrace/scanstack.tex}}%
      \onslide*<4>{\providecommand\step{3}\input{diagrams/backtrace/scanstack.tex}}%
      \onslide*<5>{\providecommand\step{4}\input{diagrams/backtrace/scanstack.tex}}%
      \onslide*<6>{\providecommand\step{5}\input{diagrams/backtrace/scanstack.tex}}%
    \end{column}
  \end{columns}
\end{frame}

% Duplicate of previous-previous slide
\begin{frame}{Backtraces}{Continuing on \funcname{caml\_stash\_backtrace}}
  \begin{columns}[c]
    \begin{column}{0.5\textwidth}
      \minipage[c][0.75\textheight][s]{\columnwidth}
      \begin{itemize}
          \color{gray}
        \item A C function provided by the runtime (specific to native code)
        \item It scans the stack, between the stack pointer at the raising point up to the next trap, in order to collect the names of the detected function frames
        \item It uses the same technique and information as the Garbage Collector (GC) uses to scan the values live on the stack
      \end{itemize}
      \begin{itemize}
        \item For each function frame detected, store a pointer to the \typename{frame\_descr} in the \localname{domain\_state->backtrace\_buffer} array, effectively constructing the backtrace
      \end{itemize}
      \onslide<2->{
        \begin{itemize}
            \smallskip
          \item Constructed backtrace:
            \funcname{definitely\_raise},
            \onslide<3->{\funcname{probably\_raise}}
        \end{itemize}
      }
      \vfill
      \mintocaml[firstline=9,lastline=13,highlightlines=12]{../src/backtrace/backtrace.ml}
      \endminipage
    \end{column}
    \begin{column}{0.5\textwidth}
      \raggedleft
      \onslide*<1>{\listStashBacktrace[highlightlines={114-115}]{}}
      \onslide*<2>{\providecommand\step{3}\input{diagrams/backtrace/backtrace.tex}}%
      \onslide*<3>{\providecommand\step{4}\input{diagrams/backtrace/backtrace.tex}}%
    \end{column}
  \end{columns}
\end{frame}

\begin{frame}{Backtraces}{Back to \funcname{caml\_raise\_exn}}
  \begin{columns}[c]
    \begin{column}{0.5\textwidth}
      \minipage[c][0.66\textheight][s]{\columnwidth}
      \begin{itemize}\color{gray}
        \item Checks wheteher exceptions backtrace should be recorded, by checking the \funcarg{domain\_state->backtrace\_active} value
        \item Switches to the C stack to be able to execute C code
        \item Calls \funcname{caml\_stash\_backtrace}, to collect the backtrace up to the next trap
      \end{itemize}
      \onslide*<2->{
        \begin{itemize}
          \item<2-> Executes the exception handler in \funcname{probably\_raise} with \funcname{RESTORE\_EXN\_HANDLER\_OCAML}
            \begin{itemize}
              \item Set \regname{RSP} to \localname{domain\_state->exn\_handler} value
              \item Restore \localname{domain\_state->exn\_handler} from trap
              \item Jump to the handler code with \funcname{ret}
            \end{itemize}
        \end{itemize}
      }
      \vfill
      \onslide*<1>{\mintocaml[firstline=9,lastline=13,highlightlines=12]{../src/backtrace/backtrace.ml}}
      \onslide*<2->{\mintocaml[firstline=15,lastline=21,highlightlines={19-21}]{../src/backtrace/backtrace.ml}}
      \endminipage
    \end{column}
    \begin{column}{0.5\textwidth}
      \centering
      \onslide*<1>{
        \mintc[firstline=190,lastline=192]{../ocaml/runtime/amd64.S}
        \mintc[firstline=234,lastline=237]{../ocaml/runtime/amd64.S}
      }
      \onslide*<2>{
        \mintc[firstline=190,lastline=192,highlightlines={191-192}]{../ocaml/runtime/amd64.S}
        \mintc[firstline=234,lastline=237,highlightlines={235-237}]{../ocaml/runtime/amd64.S}
      }
      \onslide*<1>{\listCamlRaiseExn[highlightlines=720]{}}
      \onslide*<2>{\listCamlRaiseExn[highlightlines=722]{}}
      \onslide*<3->{\providecommand\step{5}\input{diagrams/backtrace/backtrace.tex}}
    \end{column}
  \end{columns}
\end{frame}

\begin{frame}{Backtraces}{\funcname{caml\_probably\_raise}'s handler doesn't catch \typename{ExnC}}
  \begin{columns}[c]
    \begin{column}{0.45\textwidth}
      \minipage[c][0.75\textheight][s]{\columnwidth}
      \begin{itemize}
        \item<1-> \funcname{match \dots with}\footnotemark pattern matching can also match exceptions
        \item<1-> The CMM of \funcname{probably\_raise} shows that this \funcname{match \dots with} is implemented with a pattern matching and a \funcname{try \dots with}
        \item<1-> But this one only matches on \typename{ExnA} exception
        \item<2-> Since the pattern matching fails to catch the exception, \funcname{reraise} is called with our current \typename{ExnC} exception
        \item<3-> Disassembly shows that \funcname{reraise} is actually a call to \funcname{caml\_reraise\_exn}, which is also an assembly function provided by the runtime
      \end{itemize}
      \vfill
      \mintocaml[firstline=15,lastline=21,highlightlines={18-21}]{../src/backtrace/backtrace.ml}
      \endminipage
    \end{column}
    \begin{column}{0.55\textwidth}
      \centering
      \onslide*<1>{\mintcmm[highlightlines={10-18}]{../src/backtrace/camlBacktrace\_\_probably\_raise\_351.cmm}}
      \onslide*<2>{\listcmm[highlightlines=18]{../src/backtrace/camlBacktrace\_\_probably\_raise_351.cmm}{CMM of probably\_raise}}
      \onslide*<3>{\mintobjdump[firstline=5,lastline=35,highlightlines=31]{../src/backtrace/camlBacktrace\_\_probably\_raise_351.objdump}}
    \end{column}
  \end{columns}
  \only<1>{\footnotetext[1]{https://blog.janestreet.com/pattern-matching-and-exception-handling-unite/}}
\end{frame}

\newcommand\listCamlReraiseExn[1][]{
  \listasm[firstline=727,lastline=734,#1]{../ocaml/runtime/amd64.S}{runtime/amd64.S:727}}

\begin{frame}{Backtraces}{Introducing \funcname{caml\_reraise\_exn}}
  \begin{columns}[c]
    \begin{column}{0.5\textwidth}
      \minipage[c][0.66\textheight][s]{\columnwidth}
      \begin{itemize}
        \item<1-> The exception have to be reraised to the next handler
        \item<2-> First, checks if the backtrace should be collected with \funcarg{domain\_state->backtrace\_active}
        \only<3>{
        \begin{itemize}
            \item If not, behaves like a \funcname{raise\_notrace} with \funcname{RESTORE\_EXN\_HANDLER\_OCAML}
        \end{itemize}
        }
        \item<4-> Jumps into \funcname{caml\_raise\_exn}
        \item<5-> Calls \funcname{caml\_stash\_backtrace} again, to scan the next part of the stack: from the trap just pop-ed to the next trap
      \end{itemize}
      \vfill
      \mintocaml[firstline=15,lastline=21,highlightlines={18-21}]{../src/backtrace/backtrace.ml}
      \endminipage
    \end{column}
    \begin{column}{0.5\textwidth}
      \raggedleft
      \onslide*<1>{\listCamlReraiseExn{}}
      \onslide*<2>{\listCamlReraiseExn[highlightlines=729]{}}
      \onslide*<3>{
        \mintc[firstline=190,lastline=192,highlightlines={191-192}]{../ocaml/runtime/amd64.S}
        \mintc[firstline=234,lastline=237,highlightlines={235-237}]{../ocaml/runtime/amd64.S}
        \medskip
        \listCamlReraiseExn[highlightlines=731]{}
      }
      \onslide*<4-5>{
        % TODO refactor with \listCamlReraiseExn
        \mintasm[firstline=727,lastline=734,highlightlines=730]{../ocaml/runtime/amd64.S}
        \medskip
      }
      \onslide*<4>{\listCamlRaiseExn[highlightlines=711]{}}
      \onslide*<5>{\listCamlRaiseExn[highlightlines=720]{}}
    \end{column}
  \end{columns}
\end{frame}

\begin{frame}{Backtraces}{Backtrace collection continues}
  \begin{columns}[c]
    \begin{column}{0.55\textwidth}
      \minipage[c][0.75\textheight][s]{\columnwidth}
      \begin{itemize}
        \item<1-> \funcname{caml\_stash\_backtrace} scans the older part of the stack, up to the next trap
        \item<6-> Controls jumps to the nested exception handler in \funcname{maybe\_raise}, which matches only on \typename{ExnB}
        \item<7-> \funcname{caml\_reraise\_exn} is called again, backtrace collection resumes with \funcname{caml\_stash\_backtrace}
      \end{itemize}
      \medskip
      \begin{itemize}
        \item<2-> Constructed backtrace:
        \begin{itemize}
          \item <2-> \funcname{definitely\_raise}
          \item <2-> \funcname{probably\_raise}
          \item <3-> \funcname{probably\_raise} (re-raise)
          \item <4-> \funcname{maybe\_raise}
          \item <5-> \funcname{main}
          \item <8-> \funcname{main} (re-raise)
        \end{itemize}
      \end{itemize}
      \vfill
      \onslide*<1-5>{\mintocaml[firstline=15,lastline=21]{../src/backtrace/backtrace.ml}}
      \onslide*<6>{\mintocaml[firstline=28,lastline=34,highlightlines=31]{../src/backtrace/backtrace.ml}}
      \onslide*<7->{\mintocaml[firstline=28,lastline=34,highlightlines=33]{../src/backtrace/backtrace.ml}}
      \endminipage
    \end{column}
    \begin{column}{0.45\textwidth}
      \raggedleft
      \onslide*<1>{\providecommand\step{6}\input{diagrams/backtrace/backtrace.tex}}%
      \onslide*<2>{\providecommand\step{6}\input{diagrams/backtrace/backtrace.tex}}%
      \onslide*<3>{\providecommand\step{7}\input{diagrams/backtrace/backtrace.tex}}%
      \onslide*<4>{\providecommand\step{8}\input{diagrams/backtrace/backtrace.tex}}%
      \onslide*<5>{\providecommand\step{9}\input{diagrams/backtrace/backtrace.tex}}%
      \onslide*<6>{\providecommand\step{10}\input{diagrams/backtrace/backtrace.tex}}%
      \onslide*<7>{\providecommand\step{11}\input{diagrams/backtrace/backtrace.tex}}%
      \onslide*<8>{\providecommand\step{12}\input{diagrams/backtrace/backtrace.tex}}%
      \onslide*<9>{\providecommand\step{13}\input{diagrams/backtrace/backtrace.tex}}%
    \end{column}
  \end{columns}
\end{frame}

\begin{frame}{Backtraces}{Printing the backtrace}
  \begin{columns}[c]
    \begin{column}{0.45\textwidth}
      \minipage[c][0.66\textheight][s]{\columnwidth}
      \begin{itemize}
        \item<1-> The exception backtrace can now be printed to \funcarg{stdout} with \funcname{Printexc.print_backtrace}
        \item<2-> First the exception backtrace is retrieved into an OCaml array with \funcname{get_raw_backtrace }
        \item<3-> Which is implemented as the \funcname{caml_get_exception_raw_backtrace} C function in the runtime
        \item<4-> And finally printed with \funcname{print_exception_backtrace}
      \end{itemize}
      \vfill
      \mintocaml[firstline=28,lastline=34,highlightlines=33]{../src/backtrace/backtrace.ml}
      \endminipage
    \end{column}
    \begin{column}{0.55\textwidth}
      \onslide*<1,2>{
        \mintocaml[firstline=100,lastline=101]{../ocaml/stdlib/printexc.ml}
      }
      \only<1>{
        \listocaml[firstline=170,lastline=172]{../ocaml/stdlib/printexc.ml}{stdlib/printexc.ml:170}
      }
      \only<2>{
        \listocaml[firstline=170,lastline=172,highlightlines=172]{../ocaml/stdlib/printexc.ml}{stdlib/printexc.ml:170}
      }
      \only<3>{
        \mintc[firstline=154,lastline=156]{../ocaml/runtime/backtrace.c}
        \listc[firstline=171,lastline=192,highlightlines={184-187}]{../ocaml/runtime/backtrace.c}{runtime/backtrace.c:154}
      }
      \only<4>{
        \listocaml[firstline=155,lastline=165]{../ocaml/stdlib/printexc.ml}{stdlib/printexc.ml:55}
      }
    \end{column}
  \end{columns}
\end{frame}


\subsection{The multiple kinds of raise}
\frameSubsection{}{}

\begin{frame}{Backtraces}{When backtraces are not needed}
  \begin{columns}[c]
    \begin{column}{0.45\textwidth}
      \minipage[c][0.66\textheight][s]{\columnwidth}
      \begin{itemize}
        \item<1-> What if the backtrace for an exception is never needed?
        \item<2-> It's possible to raise the exception explicitly with \funcname{raise\_notrace} to make raising as cheap as possible
        \item<3-> An example of use in the \funcname{dynlink} library of the OCaml compiler
      \end{itemize}
      \vfill
      \onslide<3->{
      \listocaml[firstline=205,lastline=209,highlightlines=207]{../ocaml/otherlibs/dynlink/dynlink\_compilerlibs/misc.ml}{otherlibs/dynlink/dynlink\_compilerlibs/misc.ml:205}
      }
      \endminipage
    \end{column}
    \begin{column}{0.55\textwidth}
    \only<4>{
      \mintcmm[highlightlines=3]{../src/notrace/camlNotrace\_\_fun_347.cmm}
      \listcmm{../src/notrace/camlNotrace\_\_all\_somes\_327.cmm}{all\_somes}
    }
    \onslide*<5->{
      \listobjdump[highlightlines={6-9}]{../src/notrace/camlNotrace\_\_fun_347.objdump}{anonymous function from \funcname{all\_somes}}
    }
    \end{column}
  \end{columns}
\end{frame}

\begin{frame}{Backtraces}{Summary table of the multiple kinds of raise}
  \begin{itemize}
    \item When debugging information is enabled and if the backtrace of an exception isn't needed, it's possible to raise with \funcname{raise\_notrace} for performance
    \item When debugging information is \emph{not} enabled, the compiler optimises \funcname{raise} calls to inline assembly (same effect as using \funcname{raise\_notrace})
  \end{itemize}
  \bigskip
  \centering
  \begin{tabular}{| l | c | c | c | c |}
    \hline
    \parbox{10em}{Debug information \\ (\funcarg{-g} flag)}  & \multicolumn{2}{|c|}{Disabled}                                     & \multicolumn{2}{|c|}{Enabled} \\
    \hline
    Raise kind                                               & \funcname{raise} & \funcname{raise\_notrace}                       & \funcname{raise} & \funcname{raise\_notrace} \\
    \hline
    Compilation                                              & \parbox{8em}{inline assembly\\ (optimization)} & inline assembly   & \funcname{caml\_raise\_exn} & inline assembly \\
    \hline
  \end{tabular}
\end{frame}


%
% Takeaway #5
%
\subsection*{Takeaway \#5}
\frameSubsectionTakeaway{}
\againframe<5>{takeaway}
}%
      \onslide*<9>{\providecommand\step{13}%
% Backtraces
%
\subsection{One advantage of raising with debugging information}
\frameSubsection{}{}

\begin{frame}{Backtraces}{Debugging information \& source code}
  \begin{columns}[c]
    \begin{column}{0.5\textwidth}
      \begin{itemize}
        \item Raising an exception is fun and games, but how do we know where the exception originated? $\rightarrow$ With the backtrace associated with the exception!
        \item In order to get a backtrace from the point where an exception was raised, it's necessary to compile with debugging information enabled (\funcarg{-g} flag for the compiler)
        \item Same example as in the `Nested handlers' section
      \end{itemize}
    \end{column}
    \begin{column}{0.5\textwidth}
      \listocaml[firstline=9,lastline=34]{../src/backtrace/backtrace.ml}{backtrace/backtrace.ml}
    \end{column}
  \end{columns}
\end{frame}

\begin{frame}{Backtraces}{CMM and disassembly of \funcname{definitely\_raise}}
  \begin{columns}[c]
    \begin{column}{0.5\textwidth}
      \minipage[c][0.66\textheight][s]{\columnwidth}
      \begin{itemize}
        \item<1-> An effect of compiling with debugging information (\funcarg{-g}): the CMM no longer uses \funcname{raise\_notrace} to raise the exception but \funcname{raise} instead
        \item<2-> Disassembly shows that CMM \funcname{raise} is actually a call to \funcname{caml\_raise\_exn}, a function in assembler provided by the runtime
      \end{itemize}
      \vfill
      \mintocaml[firstline=9,lastline=13,highlightlines=12]{../src/backtrace/backtrace.ml}
      \endminipage
    \end{column}
    \begin{column}{0.5\textwidth}
      \onslide*<1>{
        \mintcmm[highlightlines=5]{../src/backtrace/camlBacktrace\_\_definitely\_raise\_347.cmm}
      }
      \onslide*<2>{
        \mintobjdump[firstline=2,highlightlines=14]{../src/backtrace/camlBacktrace\_\_definitely\_raise\_347.objdump}
      }
    \end{column}
  \end{columns}
\end{frame}

\begin{frame}{Backtraces}{Introducing \funcname{caml\_raise\_exn}}
  \begin{columns}[c]
    \begin{column}{0.5\textwidth}
      \minipage[c][0.66\textheight][s]{\columnwidth}
      \onslide*<1>{
        \begin{itemize}
          \item Raising the exception is performed using \funcname{caml\_raise\_exn} which is
            \begin{itemize}
              \item An assembly function provided by the runtime
              \item Architecture specific
            \end{itemize}
          \item Let's see what it does differently from \funcname{raise\_notrace}\dots
          \item We are about to raise \typename{ExnC} with \funcname{caml\_raise\_exn} from \funcname{definitely\_raise}
        \end{itemize}
      }
      \onslide*<2>{
        \mintobjdump[firstline=2,highlightlines=14]{../src/backtrace/camlBacktrace\_\_definitely\_raise\_347.objdump}
      }
      \vfill
      \mintocaml[firstline=9,lastline=13,highlightlines=12]{../src/backtrace/backtrace.ml}
      \endminipage
    \end{column}
    \begin{column}{0.5\textwidth}
      \centering
      \providecommand\step{1}\input{diagrams/backtrace/backtrace.tex}
    \end{column}
  \end{columns}
\end{frame}

\begin{frame}{Backtraces}{But before that, we need more fields from \typename{caml\_domain\_state}}
  \begin{columns}
    \begin{column}{0.5\textwidth}
      \begin{itemize}
        \item Remember \typename{caml\_domain\_state} the per-domain runtime state?
        \item Some other members are of interest to us now\dots
        \item \localname{backtrace\_active} is used to know if the runtime should collect backtraces
        \item \localname{backtrace\_active} can be set by
          \begin{itemize}
            \item The \funcarg{b} flag in the \funcname{OCAMLRUNPARAM} environment variable
            \item \mintinline{ocaml}{Printexc.record_backtrace : bool -> unit} \footnotemark
          \end{itemize}
      \end{itemize}
    \end{column}
    \begin{column}{0.5\textwidth}
      \begin{listing}
        \mintc[highlightlines={9-11}]{src/ocaml/domain\_state\_backtrace.h}
        \caption{runtime/caml/domain\_state.h}
      \end{listing}
    \end{column}
  \end{columns}
  \footnotetext[1]{https://v2.ocaml.org/api/Printexc.html}
\end{frame}

\newcommand\listCamlRaiseExn[1][]{
  \listasm[firstline=702,lastline=725,#1]{../ocaml/runtime/amd64.S}{runtime/amd64.S:702}}

\begin{frame}{Backtraces}{Walk through \funcname{caml\_raise\_exn}}
  \begin{columns}[T]
    \begin{column}{0.5\textwidth}
      \begin{itemize}
        \item<1-> First checks whether exceptions backtrace should be recorded
        \item<1-> By checking the \funcarg{domain\_state->backtrace\_active} value
        \item<2-> If not, acts as \funcname{raise\_notrace} with \funcname{RESTORE\_EXN\_HANDLER\_OCAML}
          \begin{itemize}
            \item Set \regname{RSP} to \localname{domain\_state->exn\_handler} value
            \item Restore \localname{domain\_state->exn\_handler} from trap
            \item Jump with \funcname{ret} (same effect as using \funcname{pop} and \funcname{jmp})
          \end{itemize}
        \item<3-> Otherwise, switches to the C stack to be able to execute C code
        \item<4-> Calls \funcname{caml\_stash\_backtrace}, a C function provided by the runtime\ignorespaces
          \onslide<6->{, with
            \begin{itemize}
              \item \funcarg{exn}: exception value
              \item \funcarg{pc}: code address of the raising point
              \item \funcarg{sp}: stack address of the raising function
              \item \funcarg{trapsp}: stack address of the next handler
            \end{itemize}
          }
      \end{itemize}
      \bigskip
      \mintocaml[firstline=9,lastline=13,highlightlines=12]{../src/backtrace/backtrace.ml}
    \end{column}
    \begin{column}{0.5\textwidth}
      \raggedleft
      \onslide*<1,3-4>{
        \mintc[firstline=190,lastline=192]{../ocaml/runtime/amd64.S}
        \mintc[firstline=234,lastline=237]{../ocaml/runtime/amd64.S}
      }
      \onslide*<2>{
        \mintc[firstline=190,lastline=192,highlightlines={191-192}]{../ocaml/runtime/amd64.S}
        \mintc[firstline=234,lastline=237,highlightlines={235-237}]{../ocaml/runtime/amd64.S}
      }
      \onslide*<1>{\listCamlRaiseExn[highlightlines=705]{}}%
      \onslide*<2>{\listCamlRaiseExn[highlightlines={707-708}]{}}%
      \onslide*<3>{\listCamlRaiseExn[highlightlines={712-714}]{}}%
      \onslide*<4>{\listCamlRaiseExn[highlightlines={715-720}]{}}%
      \onslide*<5>{\providecommand\step{1}\input{diagrams/backtrace/backtrace.tex}}%
      \onslide*<6>{\providecommand\step{2}\input{diagrams/backtrace/backtrace.tex}}%
    \end{column}
  \end{columns}
\end{frame}

\newcommand\listStashBacktrace[1][]{
  \listc[firstline=91,lastline=120,#1]{../ocaml/runtime/backtrace\_nat.c}{runtime/backtrace\_nat.c:91}}

\begin{frame}{Backtraces}{Introducing \funcname{caml\_stash\_backtrace}}
  \begin{columns}[c]
    \begin{column}{0.5\textwidth}
      \minipage[c][0.66\textheight][s]{\columnwidth}
      \begin{itemize}
        \item<1-> A C function provided by the runtime (specific to native code)
        \item<2-> It scans the stack, between the stack pointer at the raising point up to the next trap, in order to collect the names of the detected function frames
          \onslide*<2>{
            \begin{itemize}
              \item And more information, like line number, column number...
            \end{itemize}
          }
        \item<3-> It uses the same technique and information as the Garbage Collector (GC) uses to scan the values live on the stack
          \onslide*<4>{
            \begin{itemize}
              \item \typename{frame\_descr} are at the core of stack unwinding
            \end{itemize}
          }
      \end{itemize}
      \vfill
      \mintocaml[firstline=9,lastline=13,highlightlines=12]{../src/backtrace/backtrace.ml}
      \endminipage
    \end{column}
    \begin{column}{0.5\textwidth}
      \onslide*<1>{\listStashBacktrace[highlightlines=91]{}}
      \onslide*<2>{\listStashBacktrace[highlightlines={91,117-118}]{}}
      \onslide*<3->{\listStashBacktrace[highlightlines={94,106,109-110}]{}}
    \end{column}
  \end{columns}
\end{frame}

\newcommand\listFrameDescr[1][]{
  \listocaml[firstline=111,lastline=124,#1]{../ocaml/asmcomp/emitaux.ml}{asmcomp/emitaux.ml:111}}
\newcommand\listDebugInfo[1][]{
  \listocaml[firstline=38,lastline=49,#1]{../ocaml/lambda/debuginfo.mli}{lambda/debuginfo.mli:38}}

\begin{frame}{Backtraces}{\typename{frame\_descr} and \typename{frame\_debuginfo} in a nutshell}
  \begin{columns}[c]
    \begin{column}{0.5\textwidth}
      \begin{itemize}
        \item<1-> For every return address in an OCaml program, there is a associated \typename{frame\_descr}
        \item<2-> A \typename{frame\_descr} specifies
        \begin{itemize}
          \item<2-> The size of the function stack frame
          \item<3-> Where live values are on the stack (so that the GC can mark them as live and prevent from being swept)
        \end{itemize}
        \item<4-> When compiled with debug information, \typename{frame\_descr} will be linked to a \typename{frame\_debuginfo}
        \begin{itemize}
          \item<5-> Contains information about the position in the source code (file name, line and column number)
        \end{itemize}
      \end{itemize}
    \end{column}
    \begin{column}{0.5\textwidth}
        \onslide*<1>{\listFrameDescr[highlightlines={118-122}]{}}
        \onslide*<2>{\listFrameDescr[highlightlines={120}]{}}
        \onslide*<3>{\listFrameDescr[highlightlines={121}]{}}
        \onslide*<4->{\listFrameDescr[highlightlines={122}]{}}
        \medskip
        \onslide*<1-3>{\listDebugInfo{}}
        \onslide*<4>{\listDebugInfo[highlightlines={38,49}]{}}
        \onslide*<5>{\listDebugInfo[highlightlines={39-46}]{}}
    \end{column}
  \end{columns}
\end{frame}

\begin{frame}{Backtraces}{Scanning the stack with \typename{frame\_descr}}
  \begin{columns}[c]
    \begin{column}{0.5\textwidth}
      \begin{itemize}
        \item Given the return address of the code that raises the exception by calling \funcname{caml\_raise\_exn} (\funcarg{pc} argument of \funcname{caml\_stash\_backtrace})
        \item And the position of the stack pointer (\funcarg{sp} argument of \funcname{caml\_stash\_backtrace})
        \item The frame size given by \typename{frame\_descr}.\funcarg{frame\_size}, allows to find the end of the previous frame
        \item And the return address of the caller of this function
        \bigskip
        \onslide<3->{
        \item Note that traps (here the reddish \funcname{ExnA}, \funcname{ExnB} and \funcname{ExnC} blocks) belong to the frame that installed them, and so the frame size changes when traps are removed
        }
      \end{itemize}
    \end{column}
    \begin{column}{0.5\textwidth}
      \raggedleft
      \onslide*<1>{\providecommand\step{2}\input{diagrams/backtrace/backtrace.tex}}%
      \onslide*<2>{\providecommand\step{1}\input{diagrams/backtrace/scanstack.tex}}%
      \onslide*<3>{\providecommand\step{2}\input{diagrams/backtrace/scanstack.tex}}%
      \onslide*<4>{\providecommand\step{3}\input{diagrams/backtrace/scanstack.tex}}%
      \onslide*<5>{\providecommand\step{4}\input{diagrams/backtrace/scanstack.tex}}%
      \onslide*<6>{\providecommand\step{5}\input{diagrams/backtrace/scanstack.tex}}%
    \end{column}
  \end{columns}
\end{frame}

% Duplicate of previous-previous slide
\begin{frame}{Backtraces}{Continuing on \funcname{caml\_stash\_backtrace}}
  \begin{columns}[c]
    \begin{column}{0.5\textwidth}
      \minipage[c][0.75\textheight][s]{\columnwidth}
      \begin{itemize}
          \color{gray}
        \item A C function provided by the runtime (specific to native code)
        \item It scans the stack, between the stack pointer at the raising point up to the next trap, in order to collect the names of the detected function frames
        \item It uses the same technique and information as the Garbage Collector (GC) uses to scan the values live on the stack
      \end{itemize}
      \begin{itemize}
        \item For each function frame detected, store a pointer to the \typename{frame\_descr} in the \localname{domain\_state->backtrace\_buffer} array, effectively constructing the backtrace
      \end{itemize}
      \onslide<2->{
        \begin{itemize}
            \smallskip
          \item Constructed backtrace:
            \funcname{definitely\_raise},
            \onslide<3->{\funcname{probably\_raise}}
        \end{itemize}
      }
      \vfill
      \mintocaml[firstline=9,lastline=13,highlightlines=12]{../src/backtrace/backtrace.ml}
      \endminipage
    \end{column}
    \begin{column}{0.5\textwidth}
      \raggedleft
      \onslide*<1>{\listStashBacktrace[highlightlines={114-115}]{}}
      \onslide*<2>{\providecommand\step{3}\input{diagrams/backtrace/backtrace.tex}}%
      \onslide*<3>{\providecommand\step{4}\input{diagrams/backtrace/backtrace.tex}}%
    \end{column}
  \end{columns}
\end{frame}

\begin{frame}{Backtraces}{Back to \funcname{caml\_raise\_exn}}
  \begin{columns}[c]
    \begin{column}{0.5\textwidth}
      \minipage[c][0.66\textheight][s]{\columnwidth}
      \begin{itemize}\color{gray}
        \item Checks wheteher exceptions backtrace should be recorded, by checking the \funcarg{domain\_state->backtrace\_active} value
        \item Switches to the C stack to be able to execute C code
        \item Calls \funcname{caml\_stash\_backtrace}, to collect the backtrace up to the next trap
      \end{itemize}
      \onslide*<2->{
        \begin{itemize}
          \item<2-> Executes the exception handler in \funcname{probably\_raise} with \funcname{RESTORE\_EXN\_HANDLER\_OCAML}
            \begin{itemize}
              \item Set \regname{RSP} to \localname{domain\_state->exn\_handler} value
              \item Restore \localname{domain\_state->exn\_handler} from trap
              \item Jump to the handler code with \funcname{ret}
            \end{itemize}
        \end{itemize}
      }
      \vfill
      \onslide*<1>{\mintocaml[firstline=9,lastline=13,highlightlines=12]{../src/backtrace/backtrace.ml}}
      \onslide*<2->{\mintocaml[firstline=15,lastline=21,highlightlines={19-21}]{../src/backtrace/backtrace.ml}}
      \endminipage
    \end{column}
    \begin{column}{0.5\textwidth}
      \centering
      \onslide*<1>{
        \mintc[firstline=190,lastline=192]{../ocaml/runtime/amd64.S}
        \mintc[firstline=234,lastline=237]{../ocaml/runtime/amd64.S}
      }
      \onslide*<2>{
        \mintc[firstline=190,lastline=192,highlightlines={191-192}]{../ocaml/runtime/amd64.S}
        \mintc[firstline=234,lastline=237,highlightlines={235-237}]{../ocaml/runtime/amd64.S}
      }
      \onslide*<1>{\listCamlRaiseExn[highlightlines=720]{}}
      \onslide*<2>{\listCamlRaiseExn[highlightlines=722]{}}
      \onslide*<3->{\providecommand\step{5}\input{diagrams/backtrace/backtrace.tex}}
    \end{column}
  \end{columns}
\end{frame}

\begin{frame}{Backtraces}{\funcname{caml\_probably\_raise}'s handler doesn't catch \typename{ExnC}}
  \begin{columns}[c]
    \begin{column}{0.45\textwidth}
      \minipage[c][0.75\textheight][s]{\columnwidth}
      \begin{itemize}
        \item<1-> \funcname{match \dots with}\footnotemark pattern matching can also match exceptions
        \item<1-> The CMM of \funcname{probably\_raise} shows that this \funcname{match \dots with} is implemented with a pattern matching and a \funcname{try \dots with}
        \item<1-> But this one only matches on \typename{ExnA} exception
        \item<2-> Since the pattern matching fails to catch the exception, \funcname{reraise} is called with our current \typename{ExnC} exception
        \item<3-> Disassembly shows that \funcname{reraise} is actually a call to \funcname{caml\_reraise\_exn}, which is also an assembly function provided by the runtime
      \end{itemize}
      \vfill
      \mintocaml[firstline=15,lastline=21,highlightlines={18-21}]{../src/backtrace/backtrace.ml}
      \endminipage
    \end{column}
    \begin{column}{0.55\textwidth}
      \centering
      \onslide*<1>{\mintcmm[highlightlines={10-18}]{../src/backtrace/camlBacktrace\_\_probably\_raise\_351.cmm}}
      \onslide*<2>{\listcmm[highlightlines=18]{../src/backtrace/camlBacktrace\_\_probably\_raise_351.cmm}{CMM of probably\_raise}}
      \onslide*<3>{\mintobjdump[firstline=5,lastline=35,highlightlines=31]{../src/backtrace/camlBacktrace\_\_probably\_raise_351.objdump}}
    \end{column}
  \end{columns}
  \only<1>{\footnotetext[1]{https://blog.janestreet.com/pattern-matching-and-exception-handling-unite/}}
\end{frame}

\newcommand\listCamlReraiseExn[1][]{
  \listasm[firstline=727,lastline=734,#1]{../ocaml/runtime/amd64.S}{runtime/amd64.S:727}}

\begin{frame}{Backtraces}{Introducing \funcname{caml\_reraise\_exn}}
  \begin{columns}[c]
    \begin{column}{0.5\textwidth}
      \minipage[c][0.66\textheight][s]{\columnwidth}
      \begin{itemize}
        \item<1-> The exception have to be reraised to the next handler
        \item<2-> First, checks if the backtrace should be collected with \funcarg{domain\_state->backtrace\_active}
        \only<3>{
        \begin{itemize}
            \item If not, behaves like a \funcname{raise\_notrace} with \funcname{RESTORE\_EXN\_HANDLER\_OCAML}
        \end{itemize}
        }
        \item<4-> Jumps into \funcname{caml\_raise\_exn}
        \item<5-> Calls \funcname{caml\_stash\_backtrace} again, to scan the next part of the stack: from the trap just pop-ed to the next trap
      \end{itemize}
      \vfill
      \mintocaml[firstline=15,lastline=21,highlightlines={18-21}]{../src/backtrace/backtrace.ml}
      \endminipage
    \end{column}
    \begin{column}{0.5\textwidth}
      \raggedleft
      \onslide*<1>{\listCamlReraiseExn{}}
      \onslide*<2>{\listCamlReraiseExn[highlightlines=729]{}}
      \onslide*<3>{
        \mintc[firstline=190,lastline=192,highlightlines={191-192}]{../ocaml/runtime/amd64.S}
        \mintc[firstline=234,lastline=237,highlightlines={235-237}]{../ocaml/runtime/amd64.S}
        \medskip
        \listCamlReraiseExn[highlightlines=731]{}
      }
      \onslide*<4-5>{
        % TODO refactor with \listCamlReraiseExn
        \mintasm[firstline=727,lastline=734,highlightlines=730]{../ocaml/runtime/amd64.S}
        \medskip
      }
      \onslide*<4>{\listCamlRaiseExn[highlightlines=711]{}}
      \onslide*<5>{\listCamlRaiseExn[highlightlines=720]{}}
    \end{column}
  \end{columns}
\end{frame}

\begin{frame}{Backtraces}{Backtrace collection continues}
  \begin{columns}[c]
    \begin{column}{0.55\textwidth}
      \minipage[c][0.75\textheight][s]{\columnwidth}
      \begin{itemize}
        \item<1-> \funcname{caml\_stash\_backtrace} scans the older part of the stack, up to the next trap
        \item<6-> Controls jumps to the nested exception handler in \funcname{maybe\_raise}, which matches only on \typename{ExnB}
        \item<7-> \funcname{caml\_reraise\_exn} is called again, backtrace collection resumes with \funcname{caml\_stash\_backtrace}
      \end{itemize}
      \medskip
      \begin{itemize}
        \item<2-> Constructed backtrace:
        \begin{itemize}
          \item <2-> \funcname{definitely\_raise}
          \item <2-> \funcname{probably\_raise}
          \item <3-> \funcname{probably\_raise} (re-raise)
          \item <4-> \funcname{maybe\_raise}
          \item <5-> \funcname{main}
          \item <8-> \funcname{main} (re-raise)
        \end{itemize}
      \end{itemize}
      \vfill
      \onslide*<1-5>{\mintocaml[firstline=15,lastline=21]{../src/backtrace/backtrace.ml}}
      \onslide*<6>{\mintocaml[firstline=28,lastline=34,highlightlines=31]{../src/backtrace/backtrace.ml}}
      \onslide*<7->{\mintocaml[firstline=28,lastline=34,highlightlines=33]{../src/backtrace/backtrace.ml}}
      \endminipage
    \end{column}
    \begin{column}{0.45\textwidth}
      \raggedleft
      \onslide*<1>{\providecommand\step{6}\input{diagrams/backtrace/backtrace.tex}}%
      \onslide*<2>{\providecommand\step{6}\input{diagrams/backtrace/backtrace.tex}}%
      \onslide*<3>{\providecommand\step{7}\input{diagrams/backtrace/backtrace.tex}}%
      \onslide*<4>{\providecommand\step{8}\input{diagrams/backtrace/backtrace.tex}}%
      \onslide*<5>{\providecommand\step{9}\input{diagrams/backtrace/backtrace.tex}}%
      \onslide*<6>{\providecommand\step{10}\input{diagrams/backtrace/backtrace.tex}}%
      \onslide*<7>{\providecommand\step{11}\input{diagrams/backtrace/backtrace.tex}}%
      \onslide*<8>{\providecommand\step{12}\input{diagrams/backtrace/backtrace.tex}}%
      \onslide*<9>{\providecommand\step{13}\input{diagrams/backtrace/backtrace.tex}}%
    \end{column}
  \end{columns}
\end{frame}

\begin{frame}{Backtraces}{Printing the backtrace}
  \begin{columns}[c]
    \begin{column}{0.45\textwidth}
      \minipage[c][0.66\textheight][s]{\columnwidth}
      \begin{itemize}
        \item<1-> The exception backtrace can now be printed to \funcarg{stdout} with \funcname{Printexc.print_backtrace}
        \item<2-> First the exception backtrace is retrieved into an OCaml array with \funcname{get_raw_backtrace }
        \item<3-> Which is implemented as the \funcname{caml_get_exception_raw_backtrace} C function in the runtime
        \item<4-> And finally printed with \funcname{print_exception_backtrace}
      \end{itemize}
      \vfill
      \mintocaml[firstline=28,lastline=34,highlightlines=33]{../src/backtrace/backtrace.ml}
      \endminipage
    \end{column}
    \begin{column}{0.55\textwidth}
      \onslide*<1,2>{
        \mintocaml[firstline=100,lastline=101]{../ocaml/stdlib/printexc.ml}
      }
      \only<1>{
        \listocaml[firstline=170,lastline=172]{../ocaml/stdlib/printexc.ml}{stdlib/printexc.ml:170}
      }
      \only<2>{
        \listocaml[firstline=170,lastline=172,highlightlines=172]{../ocaml/stdlib/printexc.ml}{stdlib/printexc.ml:170}
      }
      \only<3>{
        \mintc[firstline=154,lastline=156]{../ocaml/runtime/backtrace.c}
        \listc[firstline=171,lastline=192,highlightlines={184-187}]{../ocaml/runtime/backtrace.c}{runtime/backtrace.c:154}
      }
      \only<4>{
        \listocaml[firstline=155,lastline=165]{../ocaml/stdlib/printexc.ml}{stdlib/printexc.ml:55}
      }
    \end{column}
  \end{columns}
\end{frame}


\subsection{The multiple kinds of raise}
\frameSubsection{}{}

\begin{frame}{Backtraces}{When backtraces are not needed}
  \begin{columns}[c]
    \begin{column}{0.45\textwidth}
      \minipage[c][0.66\textheight][s]{\columnwidth}
      \begin{itemize}
        \item<1-> What if the backtrace for an exception is never needed?
        \item<2-> It's possible to raise the exception explicitly with \funcname{raise\_notrace} to make raising as cheap as possible
        \item<3-> An example of use in the \funcname{dynlink} library of the OCaml compiler
      \end{itemize}
      \vfill
      \onslide<3->{
      \listocaml[firstline=205,lastline=209,highlightlines=207]{../ocaml/otherlibs/dynlink/dynlink\_compilerlibs/misc.ml}{otherlibs/dynlink/dynlink\_compilerlibs/misc.ml:205}
      }
      \endminipage
    \end{column}
    \begin{column}{0.55\textwidth}
    \only<4>{
      \mintcmm[highlightlines=3]{../src/notrace/camlNotrace\_\_fun_347.cmm}
      \listcmm{../src/notrace/camlNotrace\_\_all\_somes\_327.cmm}{all\_somes}
    }
    \onslide*<5->{
      \listobjdump[highlightlines={6-9}]{../src/notrace/camlNotrace\_\_fun_347.objdump}{anonymous function from \funcname{all\_somes}}
    }
    \end{column}
  \end{columns}
\end{frame}

\begin{frame}{Backtraces}{Summary table of the multiple kinds of raise}
  \begin{itemize}
    \item When debugging information is enabled and if the backtrace of an exception isn't needed, it's possible to raise with \funcname{raise\_notrace} for performance
    \item When debugging information is \emph{not} enabled, the compiler optimises \funcname{raise} calls to inline assembly (same effect as using \funcname{raise\_notrace})
  \end{itemize}
  \bigskip
  \centering
  \begin{tabular}{| l | c | c | c | c |}
    \hline
    \parbox{10em}{Debug information \\ (\funcarg{-g} flag)}  & \multicolumn{2}{|c|}{Disabled}                                     & \multicolumn{2}{|c|}{Enabled} \\
    \hline
    Raise kind                                               & \funcname{raise} & \funcname{raise\_notrace}                       & \funcname{raise} & \funcname{raise\_notrace} \\
    \hline
    Compilation                                              & \parbox{8em}{inline assembly\\ (optimization)} & inline assembly   & \funcname{caml\_raise\_exn} & inline assembly \\
    \hline
  \end{tabular}
\end{frame}


%
% Takeaway #5
%
\subsection*{Takeaway \#5}
\frameSubsectionTakeaway{}
\againframe<5>{takeaway}
}%
    \end{column}
  \end{columns}
\end{frame}

\begin{frame}{Backtraces}{Printing the backtrace}
  \begin{columns}[c]
    \begin{column}{0.45\textwidth}
      \minipage[c][0.66\textheight][s]{\columnwidth}
      \begin{itemize}
        \item<1-> The exception backtrace can now be printed to \funcarg{stdout} with \funcname{Printexc.print_backtrace}
        \item<2-> First the exception backtrace is retrieved into an OCaml array with \funcname{get_raw_backtrace }
        \item<3-> Which is implemented as the \funcname{caml_get_exception_raw_backtrace} C function in the runtime
        \item<4-> And finally printed with \funcname{print_exception_backtrace}
      \end{itemize}
      \vfill
      \mintocaml[firstline=28,lastline=34,highlightlines=33]{../src/backtrace/backtrace.ml}
      \endminipage
    \end{column}
    \begin{column}{0.55\textwidth}
      \onslide*<1,2>{
        \mintocaml[firstline=100,lastline=101]{../ocaml/stdlib/printexc.ml}
      }
      \only<1>{
        \listocaml[firstline=170,lastline=172]{../ocaml/stdlib/printexc.ml}{stdlib/printexc.ml:170}
      }
      \only<2>{
        \listocaml[firstline=170,lastline=172,highlightlines=172]{../ocaml/stdlib/printexc.ml}{stdlib/printexc.ml:170}
      }
      \only<3>{
        \mintc[firstline=154,lastline=156]{../ocaml/runtime/backtrace.c}
        \listc[firstline=171,lastline=192,highlightlines={184-187}]{../ocaml/runtime/backtrace.c}{runtime/backtrace.c:154}
      }
      \only<4>{
        \listocaml[firstline=155,lastline=165]{../ocaml/stdlib/printexc.ml}{stdlib/printexc.ml:55}
      }
    \end{column}
  \end{columns}
\end{frame}


\subsection{The multiple kinds of raise}
\frameSubsection{}{}

\begin{frame}{Backtraces}{When backtraces are not needed}
  \begin{columns}[c]
    \begin{column}{0.45\textwidth}
      \minipage[c][0.66\textheight][s]{\columnwidth}
      \begin{itemize}
        \item<1-> What if the backtrace for an exception is never needed?
        \item<2-> It's possible to raise the exception explicitly with \funcname{raise\_notrace} to make raising as cheap as possible
        \item<3-> An example of use in the \funcname{dynlink} library of the OCaml compiler
      \end{itemize}
      \vfill
      \onslide<3->{
      \listocaml[firstline=205,lastline=209,highlightlines=207]{../ocaml/otherlibs/dynlink/dynlink\_compilerlibs/misc.ml}{otherlibs/dynlink/dynlink\_compilerlibs/misc.ml:205}
      }
      \endminipage
    \end{column}
    \begin{column}{0.55\textwidth}
    \only<4>{
      \mintcmm[highlightlines=3]{../src/notrace/camlNotrace\_\_fun_347.cmm}
      \listcmm{../src/notrace/camlNotrace\_\_all\_somes\_327.cmm}{all\_somes}
    }
    \onslide*<5->{
      \listobjdump[highlightlines={6-9}]{../src/notrace/camlNotrace\_\_fun_347.objdump}{anonymous function from \funcname{all\_somes}}
    }
    \end{column}
  \end{columns}
\end{frame}

\begin{frame}{Backtraces}{Summary table of the multiple kinds of raise}
  \begin{itemize}
    \item When debugging information is enabled and if the backtrace of an exception isn't needed, it's possible to raise with \funcname{raise\_notrace} for performance
    \item When debugging information is \emph{not} enabled, the compiler optimises \funcname{raise} calls to inline assembly (same effect as using \funcname{raise\_notrace})
  \end{itemize}
  \bigskip
  \centering
  \begin{tabular}{| l | c | c | c | c |}
    \hline
    \parbox{10em}{Debug information \\ (\funcarg{-g} flag)}  & \multicolumn{2}{|c|}{Disabled}                                     & \multicolumn{2}{|c|}{Enabled} \\
    \hline
    Raise kind                                               & \funcname{raise} & \funcname{raise\_notrace}                       & \funcname{raise} & \funcname{raise\_notrace} \\
    \hline
    Compilation                                              & \parbox{8em}{inline assembly\\ (optimization)} & inline assembly   & \funcname{caml\_raise\_exn} & inline assembly \\
    \hline
  \end{tabular}
\end{frame}


%
% Takeaway #5
%
\subsection*{Takeaway \#5}
\frameSubsectionTakeaway{}
\againframe<5>{takeaway}
}%
      \onslide*<2>{\providecommand\step{1}\input{diagrams/backtrace/scanstack.tex}}%
      \onslide*<3>{\providecommand\step{2}\input{diagrams/backtrace/scanstack.tex}}%
      \onslide*<4>{\providecommand\step{3}\input{diagrams/backtrace/scanstack.tex}}%
      \onslide*<5>{\providecommand\step{4}\input{diagrams/backtrace/scanstack.tex}}%
      \onslide*<6>{\providecommand\step{5}\input{diagrams/backtrace/scanstack.tex}}%
    \end{column}
  \end{columns}
\end{frame}

% Duplicate of previous-previous slide
\begin{frame}{Backtraces}{Continuing on \funcname{caml\_stash\_backtrace}}
  \begin{columns}[c]
    \begin{column}{0.5\textwidth}
      \minipage[c][0.75\textheight][s]{\columnwidth}
      \begin{itemize}
          \color{gray}
        \item A C function provided by the runtime (specific to native code)
        \item It scans the stack, between the stack pointer at the raising point up to the next trap, in order to collect the names of the detected function frames
        \item It uses the same technique and information as the Garbage Collector (GC) uses to scan the values live on the stack
      \end{itemize}
      \begin{itemize}
        \item For each function frame detected, store a pointer to the \typename{frame\_descr} in the \localname{domain\_state->backtrace\_buffer} array, effectively constructing the backtrace
      \end{itemize}
      \onslide<2->{
        \begin{itemize}
            \smallskip
          \item Constructed backtrace:
            \funcname{definitely\_raise},
            \onslide<3->{\funcname{probably\_raise}}
        \end{itemize}
      }
      \vfill
      \mintocaml[firstline=9,lastline=13,highlightlines=12]{../src/backtrace/backtrace.ml}
      \endminipage
    \end{column}
    \begin{column}{0.5\textwidth}
      \raggedleft
      \onslide*<1>{\listStashBacktrace[highlightlines={114-115}]{}}
      \onslide*<2>{\providecommand\step{3}%
% Backtraces
%
\subsection{One advantage of raising with debugging information}
\frameSubsection{}{}

\begin{frame}{Backtraces}{Debugging information \& source code}
  \begin{columns}[c]
    \begin{column}{0.5\textwidth}
      \begin{itemize}
        \item Raising an exception is fun and games, but how do we know where the exception originated? $\rightarrow$ With the backtrace associated with the exception!
        \item In order to get a backtrace from the point where an exception was raised, it's necessary to compile with debugging information enabled (\funcarg{-g} flag for the compiler)
        \item Same example as in the `Nested handlers' section
      \end{itemize}
    \end{column}
    \begin{column}{0.5\textwidth}
      \listocaml[firstline=9,lastline=34]{../src/backtrace/backtrace.ml}{backtrace/backtrace.ml}
    \end{column}
  \end{columns}
\end{frame}

\begin{frame}{Backtraces}{CMM and disassembly of \funcname{definitely\_raise}}
  \begin{columns}[c]
    \begin{column}{0.5\textwidth}
      \minipage[c][0.66\textheight][s]{\columnwidth}
      \begin{itemize}
        \item<1-> An effect of compiling with debugging information (\funcarg{-g}): the CMM no longer uses \funcname{raise\_notrace} to raise the exception but \funcname{raise} instead
        \item<2-> Disassembly shows that CMM \funcname{raise} is actually a call to \funcname{caml\_raise\_exn}, a function in assembler provided by the runtime
      \end{itemize}
      \vfill
      \mintocaml[firstline=9,lastline=13,highlightlines=12]{../src/backtrace/backtrace.ml}
      \endminipage
    \end{column}
    \begin{column}{0.5\textwidth}
      \onslide*<1>{
        \mintcmm[highlightlines=5]{../src/backtrace/camlBacktrace\_\_definitely\_raise\_347.cmm}
      }
      \onslide*<2>{
        \mintobjdump[firstline=2,highlightlines=14]{../src/backtrace/camlBacktrace\_\_definitely\_raise\_347.objdump}
      }
    \end{column}
  \end{columns}
\end{frame}

\begin{frame}{Backtraces}{Introducing \funcname{caml\_raise\_exn}}
  \begin{columns}[c]
    \begin{column}{0.5\textwidth}
      \minipage[c][0.66\textheight][s]{\columnwidth}
      \onslide*<1>{
        \begin{itemize}
          \item Raising the exception is performed using \funcname{caml\_raise\_exn} which is
            \begin{itemize}
              \item An assembly function provided by the runtime
              \item Architecture specific
            \end{itemize}
          \item Let's see what it does differently from \funcname{raise\_notrace}\dots
          \item We are about to raise \typename{ExnC} with \funcname{caml\_raise\_exn} from \funcname{definitely\_raise}
        \end{itemize}
      }
      \onslide*<2>{
        \mintobjdump[firstline=2,highlightlines=14]{../src/backtrace/camlBacktrace\_\_definitely\_raise\_347.objdump}
      }
      \vfill
      \mintocaml[firstline=9,lastline=13,highlightlines=12]{../src/backtrace/backtrace.ml}
      \endminipage
    \end{column}
    \begin{column}{0.5\textwidth}
      \centering
      \providecommand\step{1}%
% Backtraces
%
\subsection{One advantage of raising with debugging information}
\frameSubsection{}{}

\begin{frame}{Backtraces}{Debugging information \& source code}
  \begin{columns}[c]
    \begin{column}{0.5\textwidth}
      \begin{itemize}
        \item Raising an exception is fun and games, but how do we know where the exception originated? $\rightarrow$ With the backtrace associated with the exception!
        \item In order to get a backtrace from the point where an exception was raised, it's necessary to compile with debugging information enabled (\funcarg{-g} flag for the compiler)
        \item Same example as in the `Nested handlers' section
      \end{itemize}
    \end{column}
    \begin{column}{0.5\textwidth}
      \listocaml[firstline=9,lastline=34]{../src/backtrace/backtrace.ml}{backtrace/backtrace.ml}
    \end{column}
  \end{columns}
\end{frame}

\begin{frame}{Backtraces}{CMM and disassembly of \funcname{definitely\_raise}}
  \begin{columns}[c]
    \begin{column}{0.5\textwidth}
      \minipage[c][0.66\textheight][s]{\columnwidth}
      \begin{itemize}
        \item<1-> An effect of compiling with debugging information (\funcarg{-g}): the CMM no longer uses \funcname{raise\_notrace} to raise the exception but \funcname{raise} instead
        \item<2-> Disassembly shows that CMM \funcname{raise} is actually a call to \funcname{caml\_raise\_exn}, a function in assembler provided by the runtime
      \end{itemize}
      \vfill
      \mintocaml[firstline=9,lastline=13,highlightlines=12]{../src/backtrace/backtrace.ml}
      \endminipage
    \end{column}
    \begin{column}{0.5\textwidth}
      \onslide*<1>{
        \mintcmm[highlightlines=5]{../src/backtrace/camlBacktrace\_\_definitely\_raise\_347.cmm}
      }
      \onslide*<2>{
        \mintobjdump[firstline=2,highlightlines=14]{../src/backtrace/camlBacktrace\_\_definitely\_raise\_347.objdump}
      }
    \end{column}
  \end{columns}
\end{frame}

\begin{frame}{Backtraces}{Introducing \funcname{caml\_raise\_exn}}
  \begin{columns}[c]
    \begin{column}{0.5\textwidth}
      \minipage[c][0.66\textheight][s]{\columnwidth}
      \onslide*<1>{
        \begin{itemize}
          \item Raising the exception is performed using \funcname{caml\_raise\_exn} which is
            \begin{itemize}
              \item An assembly function provided by the runtime
              \item Architecture specific
            \end{itemize}
          \item Let's see what it does differently from \funcname{raise\_notrace}\dots
          \item We are about to raise \typename{ExnC} with \funcname{caml\_raise\_exn} from \funcname{definitely\_raise}
        \end{itemize}
      }
      \onslide*<2>{
        \mintobjdump[firstline=2,highlightlines=14]{../src/backtrace/camlBacktrace\_\_definitely\_raise\_347.objdump}
      }
      \vfill
      \mintocaml[firstline=9,lastline=13,highlightlines=12]{../src/backtrace/backtrace.ml}
      \endminipage
    \end{column}
    \begin{column}{0.5\textwidth}
      \centering
      \providecommand\step{1}\input{diagrams/backtrace/backtrace.tex}
    \end{column}
  \end{columns}
\end{frame}

\begin{frame}{Backtraces}{But before that, we need more fields from \typename{caml\_domain\_state}}
  \begin{columns}
    \begin{column}{0.5\textwidth}
      \begin{itemize}
        \item Remember \typename{caml\_domain\_state} the per-domain runtime state?
        \item Some other members are of interest to us now\dots
        \item \localname{backtrace\_active} is used to know if the runtime should collect backtraces
        \item \localname{backtrace\_active} can be set by
          \begin{itemize}
            \item The \funcarg{b} flag in the \funcname{OCAMLRUNPARAM} environment variable
            \item \mintinline{ocaml}{Printexc.record_backtrace : bool -> unit} \footnotemark
          \end{itemize}
      \end{itemize}
    \end{column}
    \begin{column}{0.5\textwidth}
      \begin{listing}
        \mintc[highlightlines={9-11}]{src/ocaml/domain\_state\_backtrace.h}
        \caption{runtime/caml/domain\_state.h}
      \end{listing}
    \end{column}
  \end{columns}
  \footnotetext[1]{https://v2.ocaml.org/api/Printexc.html}
\end{frame}

\newcommand\listCamlRaiseExn[1][]{
  \listasm[firstline=702,lastline=725,#1]{../ocaml/runtime/amd64.S}{runtime/amd64.S:702}}

\begin{frame}{Backtraces}{Walk through \funcname{caml\_raise\_exn}}
  \begin{columns}[T]
    \begin{column}{0.5\textwidth}
      \begin{itemize}
        \item<1-> First checks whether exceptions backtrace should be recorded
        \item<1-> By checking the \funcarg{domain\_state->backtrace\_active} value
        \item<2-> If not, acts as \funcname{raise\_notrace} with \funcname{RESTORE\_EXN\_HANDLER\_OCAML}
          \begin{itemize}
            \item Set \regname{RSP} to \localname{domain\_state->exn\_handler} value
            \item Restore \localname{domain\_state->exn\_handler} from trap
            \item Jump with \funcname{ret} (same effect as using \funcname{pop} and \funcname{jmp})
          \end{itemize}
        \item<3-> Otherwise, switches to the C stack to be able to execute C code
        \item<4-> Calls \funcname{caml\_stash\_backtrace}, a C function provided by the runtime\ignorespaces
          \onslide<6->{, with
            \begin{itemize}
              \item \funcarg{exn}: exception value
              \item \funcarg{pc}: code address of the raising point
              \item \funcarg{sp}: stack address of the raising function
              \item \funcarg{trapsp}: stack address of the next handler
            \end{itemize}
          }
      \end{itemize}
      \bigskip
      \mintocaml[firstline=9,lastline=13,highlightlines=12]{../src/backtrace/backtrace.ml}
    \end{column}
    \begin{column}{0.5\textwidth}
      \raggedleft
      \onslide*<1,3-4>{
        \mintc[firstline=190,lastline=192]{../ocaml/runtime/amd64.S}
        \mintc[firstline=234,lastline=237]{../ocaml/runtime/amd64.S}
      }
      \onslide*<2>{
        \mintc[firstline=190,lastline=192,highlightlines={191-192}]{../ocaml/runtime/amd64.S}
        \mintc[firstline=234,lastline=237,highlightlines={235-237}]{../ocaml/runtime/amd64.S}
      }
      \onslide*<1>{\listCamlRaiseExn[highlightlines=705]{}}%
      \onslide*<2>{\listCamlRaiseExn[highlightlines={707-708}]{}}%
      \onslide*<3>{\listCamlRaiseExn[highlightlines={712-714}]{}}%
      \onslide*<4>{\listCamlRaiseExn[highlightlines={715-720}]{}}%
      \onslide*<5>{\providecommand\step{1}\input{diagrams/backtrace/backtrace.tex}}%
      \onslide*<6>{\providecommand\step{2}\input{diagrams/backtrace/backtrace.tex}}%
    \end{column}
  \end{columns}
\end{frame}

\newcommand\listStashBacktrace[1][]{
  \listc[firstline=91,lastline=120,#1]{../ocaml/runtime/backtrace\_nat.c}{runtime/backtrace\_nat.c:91}}

\begin{frame}{Backtraces}{Introducing \funcname{caml\_stash\_backtrace}}
  \begin{columns}[c]
    \begin{column}{0.5\textwidth}
      \minipage[c][0.66\textheight][s]{\columnwidth}
      \begin{itemize}
        \item<1-> A C function provided by the runtime (specific to native code)
        \item<2-> It scans the stack, between the stack pointer at the raising point up to the next trap, in order to collect the names of the detected function frames
          \onslide*<2>{
            \begin{itemize}
              \item And more information, like line number, column number...
            \end{itemize}
          }
        \item<3-> It uses the same technique and information as the Garbage Collector (GC) uses to scan the values live on the stack
          \onslide*<4>{
            \begin{itemize}
              \item \typename{frame\_descr} are at the core of stack unwinding
            \end{itemize}
          }
      \end{itemize}
      \vfill
      \mintocaml[firstline=9,lastline=13,highlightlines=12]{../src/backtrace/backtrace.ml}
      \endminipage
    \end{column}
    \begin{column}{0.5\textwidth}
      \onslide*<1>{\listStashBacktrace[highlightlines=91]{}}
      \onslide*<2>{\listStashBacktrace[highlightlines={91,117-118}]{}}
      \onslide*<3->{\listStashBacktrace[highlightlines={94,106,109-110}]{}}
    \end{column}
  \end{columns}
\end{frame}

\newcommand\listFrameDescr[1][]{
  \listocaml[firstline=111,lastline=124,#1]{../ocaml/asmcomp/emitaux.ml}{asmcomp/emitaux.ml:111}}
\newcommand\listDebugInfo[1][]{
  \listocaml[firstline=38,lastline=49,#1]{../ocaml/lambda/debuginfo.mli}{lambda/debuginfo.mli:38}}

\begin{frame}{Backtraces}{\typename{frame\_descr} and \typename{frame\_debuginfo} in a nutshell}
  \begin{columns}[c]
    \begin{column}{0.5\textwidth}
      \begin{itemize}
        \item<1-> For every return address in an OCaml program, there is a associated \typename{frame\_descr}
        \item<2-> A \typename{frame\_descr} specifies
        \begin{itemize}
          \item<2-> The size of the function stack frame
          \item<3-> Where live values are on the stack (so that the GC can mark them as live and prevent from being swept)
        \end{itemize}
        \item<4-> When compiled with debug information, \typename{frame\_descr} will be linked to a \typename{frame\_debuginfo}
        \begin{itemize}
          \item<5-> Contains information about the position in the source code (file name, line and column number)
        \end{itemize}
      \end{itemize}
    \end{column}
    \begin{column}{0.5\textwidth}
        \onslide*<1>{\listFrameDescr[highlightlines={118-122}]{}}
        \onslide*<2>{\listFrameDescr[highlightlines={120}]{}}
        \onslide*<3>{\listFrameDescr[highlightlines={121}]{}}
        \onslide*<4->{\listFrameDescr[highlightlines={122}]{}}
        \medskip
        \onslide*<1-3>{\listDebugInfo{}}
        \onslide*<4>{\listDebugInfo[highlightlines={38,49}]{}}
        \onslide*<5>{\listDebugInfo[highlightlines={39-46}]{}}
    \end{column}
  \end{columns}
\end{frame}

\begin{frame}{Backtraces}{Scanning the stack with \typename{frame\_descr}}
  \begin{columns}[c]
    \begin{column}{0.5\textwidth}
      \begin{itemize}
        \item Given the return address of the code that raises the exception by calling \funcname{caml\_raise\_exn} (\funcarg{pc} argument of \funcname{caml\_stash\_backtrace})
        \item And the position of the stack pointer (\funcarg{sp} argument of \funcname{caml\_stash\_backtrace})
        \item The frame size given by \typename{frame\_descr}.\funcarg{frame\_size}, allows to find the end of the previous frame
        \item And the return address of the caller of this function
        \bigskip
        \onslide<3->{
        \item Note that traps (here the reddish \funcname{ExnA}, \funcname{ExnB} and \funcname{ExnC} blocks) belong to the frame that installed them, and so the frame size changes when traps are removed
        }
      \end{itemize}
    \end{column}
    \begin{column}{0.5\textwidth}
      \raggedleft
      \onslide*<1>{\providecommand\step{2}\input{diagrams/backtrace/backtrace.tex}}%
      \onslide*<2>{\providecommand\step{1}\input{diagrams/backtrace/scanstack.tex}}%
      \onslide*<3>{\providecommand\step{2}\input{diagrams/backtrace/scanstack.tex}}%
      \onslide*<4>{\providecommand\step{3}\input{diagrams/backtrace/scanstack.tex}}%
      \onslide*<5>{\providecommand\step{4}\input{diagrams/backtrace/scanstack.tex}}%
      \onslide*<6>{\providecommand\step{5}\input{diagrams/backtrace/scanstack.tex}}%
    \end{column}
  \end{columns}
\end{frame}

% Duplicate of previous-previous slide
\begin{frame}{Backtraces}{Continuing on \funcname{caml\_stash\_backtrace}}
  \begin{columns}[c]
    \begin{column}{0.5\textwidth}
      \minipage[c][0.75\textheight][s]{\columnwidth}
      \begin{itemize}
          \color{gray}
        \item A C function provided by the runtime (specific to native code)
        \item It scans the stack, between the stack pointer at the raising point up to the next trap, in order to collect the names of the detected function frames
        \item It uses the same technique and information as the Garbage Collector (GC) uses to scan the values live on the stack
      \end{itemize}
      \begin{itemize}
        \item For each function frame detected, store a pointer to the \typename{frame\_descr} in the \localname{domain\_state->backtrace\_buffer} array, effectively constructing the backtrace
      \end{itemize}
      \onslide<2->{
        \begin{itemize}
            \smallskip
          \item Constructed backtrace:
            \funcname{definitely\_raise},
            \onslide<3->{\funcname{probably\_raise}}
        \end{itemize}
      }
      \vfill
      \mintocaml[firstline=9,lastline=13,highlightlines=12]{../src/backtrace/backtrace.ml}
      \endminipage
    \end{column}
    \begin{column}{0.5\textwidth}
      \raggedleft
      \onslide*<1>{\listStashBacktrace[highlightlines={114-115}]{}}
      \onslide*<2>{\providecommand\step{3}\input{diagrams/backtrace/backtrace.tex}}%
      \onslide*<3>{\providecommand\step{4}\input{diagrams/backtrace/backtrace.tex}}%
    \end{column}
  \end{columns}
\end{frame}

\begin{frame}{Backtraces}{Back to \funcname{caml\_raise\_exn}}
  \begin{columns}[c]
    \begin{column}{0.5\textwidth}
      \minipage[c][0.66\textheight][s]{\columnwidth}
      \begin{itemize}\color{gray}
        \item Checks wheteher exceptions backtrace should be recorded, by checking the \funcarg{domain\_state->backtrace\_active} value
        \item Switches to the C stack to be able to execute C code
        \item Calls \funcname{caml\_stash\_backtrace}, to collect the backtrace up to the next trap
      \end{itemize}
      \onslide*<2->{
        \begin{itemize}
          \item<2-> Executes the exception handler in \funcname{probably\_raise} with \funcname{RESTORE\_EXN\_HANDLER\_OCAML}
            \begin{itemize}
              \item Set \regname{RSP} to \localname{domain\_state->exn\_handler} value
              \item Restore \localname{domain\_state->exn\_handler} from trap
              \item Jump to the handler code with \funcname{ret}
            \end{itemize}
        \end{itemize}
      }
      \vfill
      \onslide*<1>{\mintocaml[firstline=9,lastline=13,highlightlines=12]{../src/backtrace/backtrace.ml}}
      \onslide*<2->{\mintocaml[firstline=15,lastline=21,highlightlines={19-21}]{../src/backtrace/backtrace.ml}}
      \endminipage
    \end{column}
    \begin{column}{0.5\textwidth}
      \centering
      \onslide*<1>{
        \mintc[firstline=190,lastline=192]{../ocaml/runtime/amd64.S}
        \mintc[firstline=234,lastline=237]{../ocaml/runtime/amd64.S}
      }
      \onslide*<2>{
        \mintc[firstline=190,lastline=192,highlightlines={191-192}]{../ocaml/runtime/amd64.S}
        \mintc[firstline=234,lastline=237,highlightlines={235-237}]{../ocaml/runtime/amd64.S}
      }
      \onslide*<1>{\listCamlRaiseExn[highlightlines=720]{}}
      \onslide*<2>{\listCamlRaiseExn[highlightlines=722]{}}
      \onslide*<3->{\providecommand\step{5}\input{diagrams/backtrace/backtrace.tex}}
    \end{column}
  \end{columns}
\end{frame}

\begin{frame}{Backtraces}{\funcname{caml\_probably\_raise}'s handler doesn't catch \typename{ExnC}}
  \begin{columns}[c]
    \begin{column}{0.45\textwidth}
      \minipage[c][0.75\textheight][s]{\columnwidth}
      \begin{itemize}
        \item<1-> \funcname{match \dots with}\footnotemark pattern matching can also match exceptions
        \item<1-> The CMM of \funcname{probably\_raise} shows that this \funcname{match \dots with} is implemented with a pattern matching and a \funcname{try \dots with}
        \item<1-> But this one only matches on \typename{ExnA} exception
        \item<2-> Since the pattern matching fails to catch the exception, \funcname{reraise} is called with our current \typename{ExnC} exception
        \item<3-> Disassembly shows that \funcname{reraise} is actually a call to \funcname{caml\_reraise\_exn}, which is also an assembly function provided by the runtime
      \end{itemize}
      \vfill
      \mintocaml[firstline=15,lastline=21,highlightlines={18-21}]{../src/backtrace/backtrace.ml}
      \endminipage
    \end{column}
    \begin{column}{0.55\textwidth}
      \centering
      \onslide*<1>{\mintcmm[highlightlines={10-18}]{../src/backtrace/camlBacktrace\_\_probably\_raise\_351.cmm}}
      \onslide*<2>{\listcmm[highlightlines=18]{../src/backtrace/camlBacktrace\_\_probably\_raise_351.cmm}{CMM of probably\_raise}}
      \onslide*<3>{\mintobjdump[firstline=5,lastline=35,highlightlines=31]{../src/backtrace/camlBacktrace\_\_probably\_raise_351.objdump}}
    \end{column}
  \end{columns}
  \only<1>{\footnotetext[1]{https://blog.janestreet.com/pattern-matching-and-exception-handling-unite/}}
\end{frame}

\newcommand\listCamlReraiseExn[1][]{
  \listasm[firstline=727,lastline=734,#1]{../ocaml/runtime/amd64.S}{runtime/amd64.S:727}}

\begin{frame}{Backtraces}{Introducing \funcname{caml\_reraise\_exn}}
  \begin{columns}[c]
    \begin{column}{0.5\textwidth}
      \minipage[c][0.66\textheight][s]{\columnwidth}
      \begin{itemize}
        \item<1-> The exception have to be reraised to the next handler
        \item<2-> First, checks if the backtrace should be collected with \funcarg{domain\_state->backtrace\_active}
        \only<3>{
        \begin{itemize}
            \item If not, behaves like a \funcname{raise\_notrace} with \funcname{RESTORE\_EXN\_HANDLER\_OCAML}
        \end{itemize}
        }
        \item<4-> Jumps into \funcname{caml\_raise\_exn}
        \item<5-> Calls \funcname{caml\_stash\_backtrace} again, to scan the next part of the stack: from the trap just pop-ed to the next trap
      \end{itemize}
      \vfill
      \mintocaml[firstline=15,lastline=21,highlightlines={18-21}]{../src/backtrace/backtrace.ml}
      \endminipage
    \end{column}
    \begin{column}{0.5\textwidth}
      \raggedleft
      \onslide*<1>{\listCamlReraiseExn{}}
      \onslide*<2>{\listCamlReraiseExn[highlightlines=729]{}}
      \onslide*<3>{
        \mintc[firstline=190,lastline=192,highlightlines={191-192}]{../ocaml/runtime/amd64.S}
        \mintc[firstline=234,lastline=237,highlightlines={235-237}]{../ocaml/runtime/amd64.S}
        \medskip
        \listCamlReraiseExn[highlightlines=731]{}
      }
      \onslide*<4-5>{
        % TODO refactor with \listCamlReraiseExn
        \mintasm[firstline=727,lastline=734,highlightlines=730]{../ocaml/runtime/amd64.S}
        \medskip
      }
      \onslide*<4>{\listCamlRaiseExn[highlightlines=711]{}}
      \onslide*<5>{\listCamlRaiseExn[highlightlines=720]{}}
    \end{column}
  \end{columns}
\end{frame}

\begin{frame}{Backtraces}{Backtrace collection continues}
  \begin{columns}[c]
    \begin{column}{0.55\textwidth}
      \minipage[c][0.75\textheight][s]{\columnwidth}
      \begin{itemize}
        \item<1-> \funcname{caml\_stash\_backtrace} scans the older part of the stack, up to the next trap
        \item<6-> Controls jumps to the nested exception handler in \funcname{maybe\_raise}, which matches only on \typename{ExnB}
        \item<7-> \funcname{caml\_reraise\_exn} is called again, backtrace collection resumes with \funcname{caml\_stash\_backtrace}
      \end{itemize}
      \medskip
      \begin{itemize}
        \item<2-> Constructed backtrace:
        \begin{itemize}
          \item <2-> \funcname{definitely\_raise}
          \item <2-> \funcname{probably\_raise}
          \item <3-> \funcname{probably\_raise} (re-raise)
          \item <4-> \funcname{maybe\_raise}
          \item <5-> \funcname{main}
          \item <8-> \funcname{main} (re-raise)
        \end{itemize}
      \end{itemize}
      \vfill
      \onslide*<1-5>{\mintocaml[firstline=15,lastline=21]{../src/backtrace/backtrace.ml}}
      \onslide*<6>{\mintocaml[firstline=28,lastline=34,highlightlines=31]{../src/backtrace/backtrace.ml}}
      \onslide*<7->{\mintocaml[firstline=28,lastline=34,highlightlines=33]{../src/backtrace/backtrace.ml}}
      \endminipage
    \end{column}
    \begin{column}{0.45\textwidth}
      \raggedleft
      \onslide*<1>{\providecommand\step{6}\input{diagrams/backtrace/backtrace.tex}}%
      \onslide*<2>{\providecommand\step{6}\input{diagrams/backtrace/backtrace.tex}}%
      \onslide*<3>{\providecommand\step{7}\input{diagrams/backtrace/backtrace.tex}}%
      \onslide*<4>{\providecommand\step{8}\input{diagrams/backtrace/backtrace.tex}}%
      \onslide*<5>{\providecommand\step{9}\input{diagrams/backtrace/backtrace.tex}}%
      \onslide*<6>{\providecommand\step{10}\input{diagrams/backtrace/backtrace.tex}}%
      \onslide*<7>{\providecommand\step{11}\input{diagrams/backtrace/backtrace.tex}}%
      \onslide*<8>{\providecommand\step{12}\input{diagrams/backtrace/backtrace.tex}}%
      \onslide*<9>{\providecommand\step{13}\input{diagrams/backtrace/backtrace.tex}}%
    \end{column}
  \end{columns}
\end{frame}

\begin{frame}{Backtraces}{Printing the backtrace}
  \begin{columns}[c]
    \begin{column}{0.45\textwidth}
      \minipage[c][0.66\textheight][s]{\columnwidth}
      \begin{itemize}
        \item<1-> The exception backtrace can now be printed to \funcarg{stdout} with \funcname{Printexc.print_backtrace}
        \item<2-> First the exception backtrace is retrieved into an OCaml array with \funcname{get_raw_backtrace }
        \item<3-> Which is implemented as the \funcname{caml_get_exception_raw_backtrace} C function in the runtime
        \item<4-> And finally printed with \funcname{print_exception_backtrace}
      \end{itemize}
      \vfill
      \mintocaml[firstline=28,lastline=34,highlightlines=33]{../src/backtrace/backtrace.ml}
      \endminipage
    \end{column}
    \begin{column}{0.55\textwidth}
      \onslide*<1,2>{
        \mintocaml[firstline=100,lastline=101]{../ocaml/stdlib/printexc.ml}
      }
      \only<1>{
        \listocaml[firstline=170,lastline=172]{../ocaml/stdlib/printexc.ml}{stdlib/printexc.ml:170}
      }
      \only<2>{
        \listocaml[firstline=170,lastline=172,highlightlines=172]{../ocaml/stdlib/printexc.ml}{stdlib/printexc.ml:170}
      }
      \only<3>{
        \mintc[firstline=154,lastline=156]{../ocaml/runtime/backtrace.c}
        \listc[firstline=171,lastline=192,highlightlines={184-187}]{../ocaml/runtime/backtrace.c}{runtime/backtrace.c:154}
      }
      \only<4>{
        \listocaml[firstline=155,lastline=165]{../ocaml/stdlib/printexc.ml}{stdlib/printexc.ml:55}
      }
    \end{column}
  \end{columns}
\end{frame}


\subsection{The multiple kinds of raise}
\frameSubsection{}{}

\begin{frame}{Backtraces}{When backtraces are not needed}
  \begin{columns}[c]
    \begin{column}{0.45\textwidth}
      \minipage[c][0.66\textheight][s]{\columnwidth}
      \begin{itemize}
        \item<1-> What if the backtrace for an exception is never needed?
        \item<2-> It's possible to raise the exception explicitly with \funcname{raise\_notrace} to make raising as cheap as possible
        \item<3-> An example of use in the \funcname{dynlink} library of the OCaml compiler
      \end{itemize}
      \vfill
      \onslide<3->{
      \listocaml[firstline=205,lastline=209,highlightlines=207]{../ocaml/otherlibs/dynlink/dynlink\_compilerlibs/misc.ml}{otherlibs/dynlink/dynlink\_compilerlibs/misc.ml:205}
      }
      \endminipage
    \end{column}
    \begin{column}{0.55\textwidth}
    \only<4>{
      \mintcmm[highlightlines=3]{../src/notrace/camlNotrace\_\_fun_347.cmm}
      \listcmm{../src/notrace/camlNotrace\_\_all\_somes\_327.cmm}{all\_somes}
    }
    \onslide*<5->{
      \listobjdump[highlightlines={6-9}]{../src/notrace/camlNotrace\_\_fun_347.objdump}{anonymous function from \funcname{all\_somes}}
    }
    \end{column}
  \end{columns}
\end{frame}

\begin{frame}{Backtraces}{Summary table of the multiple kinds of raise}
  \begin{itemize}
    \item When debugging information is enabled and if the backtrace of an exception isn't needed, it's possible to raise with \funcname{raise\_notrace} for performance
    \item When debugging information is \emph{not} enabled, the compiler optimises \funcname{raise} calls to inline assembly (same effect as using \funcname{raise\_notrace})
  \end{itemize}
  \bigskip
  \centering
  \begin{tabular}{| l | c | c | c | c |}
    \hline
    \parbox{10em}{Debug information \\ (\funcarg{-g} flag)}  & \multicolumn{2}{|c|}{Disabled}                                     & \multicolumn{2}{|c|}{Enabled} \\
    \hline
    Raise kind                                               & \funcname{raise} & \funcname{raise\_notrace}                       & \funcname{raise} & \funcname{raise\_notrace} \\
    \hline
    Compilation                                              & \parbox{8em}{inline assembly\\ (optimization)} & inline assembly   & \funcname{caml\_raise\_exn} & inline assembly \\
    \hline
  \end{tabular}
\end{frame}


%
% Takeaway #5
%
\subsection*{Takeaway \#5}
\frameSubsectionTakeaway{}
\againframe<5>{takeaway}

    \end{column}
  \end{columns}
\end{frame}

\begin{frame}{Backtraces}{But before that, we need more fields from \typename{caml\_domain\_state}}
  \begin{columns}
    \begin{column}{0.5\textwidth}
      \begin{itemize}
        \item Remember \typename{caml\_domain\_state} the per-domain runtime state?
        \item Some other members are of interest to us now\dots
        \item \localname{backtrace\_active} is used to know if the runtime should collect backtraces
        \item \localname{backtrace\_active} can be set by
          \begin{itemize}
            \item The \funcarg{b} flag in the \funcname{OCAMLRUNPARAM} environment variable
            \item \mintinline{ocaml}{Printexc.record_backtrace : bool -> unit} \footnotemark
          \end{itemize}
      \end{itemize}
    \end{column}
    \begin{column}{0.5\textwidth}
      \begin{listing}
        \mintc[highlightlines={9-11}]{src/ocaml/domain\_state\_backtrace.h}
        \caption{runtime/caml/domain\_state.h}
      \end{listing}
    \end{column}
  \end{columns}
  \footnotetext[1]{https://v2.ocaml.org/api/Printexc.html}
\end{frame}

\newcommand\listCamlRaiseExn[1][]{
  \listasm[firstline=702,lastline=725,#1]{../ocaml/runtime/amd64.S}{runtime/amd64.S:702}}

\begin{frame}{Backtraces}{Walk through \funcname{caml\_raise\_exn}}
  \begin{columns}[T]
    \begin{column}{0.5\textwidth}
      \begin{itemize}
        \item<1-> First checks whether exceptions backtrace should be recorded
        \item<1-> By checking the \funcarg{domain\_state->backtrace\_active} value
        \item<2-> If not, acts as \funcname{raise\_notrace} with \funcname{RESTORE\_EXN\_HANDLER\_OCAML}
          \begin{itemize}
            \item Set \regname{RSP} to \localname{domain\_state->exn\_handler} value
            \item Restore \localname{domain\_state->exn\_handler} from trap
            \item Jump with \funcname{ret} (same effect as using \funcname{pop} and \funcname{jmp})
          \end{itemize}
        \item<3-> Otherwise, switches to the C stack to be able to execute C code
        \item<4-> Calls \funcname{caml\_stash\_backtrace}, a C function provided by the runtime\ignorespaces
          \onslide<6->{, with
            \begin{itemize}
              \item \funcarg{exn}: exception value
              \item \funcarg{pc}: code address of the raising point
              \item \funcarg{sp}: stack address of the raising function
              \item \funcarg{trapsp}: stack address of the next handler
            \end{itemize}
          }
      \end{itemize}
      \bigskip
      \mintocaml[firstline=9,lastline=13,highlightlines=12]{../src/backtrace/backtrace.ml}
    \end{column}
    \begin{column}{0.5\textwidth}
      \raggedleft
      \onslide*<1,3-4>{
        \mintc[firstline=190,lastline=192]{../ocaml/runtime/amd64.S}
        \mintc[firstline=234,lastline=237]{../ocaml/runtime/amd64.S}
      }
      \onslide*<2>{
        \mintc[firstline=190,lastline=192,highlightlines={191-192}]{../ocaml/runtime/amd64.S}
        \mintc[firstline=234,lastline=237,highlightlines={235-237}]{../ocaml/runtime/amd64.S}
      }
      \onslide*<1>{\listCamlRaiseExn[highlightlines=705]{}}%
      \onslide*<2>{\listCamlRaiseExn[highlightlines={707-708}]{}}%
      \onslide*<3>{\listCamlRaiseExn[highlightlines={712-714}]{}}%
      \onslide*<4>{\listCamlRaiseExn[highlightlines={715-720}]{}}%
      \onslide*<5>{\providecommand\step{1}%
% Backtraces
%
\subsection{One advantage of raising with debugging information}
\frameSubsection{}{}

\begin{frame}{Backtraces}{Debugging information \& source code}
  \begin{columns}[c]
    \begin{column}{0.5\textwidth}
      \begin{itemize}
        \item Raising an exception is fun and games, but how do we know where the exception originated? $\rightarrow$ With the backtrace associated with the exception!
        \item In order to get a backtrace from the point where an exception was raised, it's necessary to compile with debugging information enabled (\funcarg{-g} flag for the compiler)
        \item Same example as in the `Nested handlers' section
      \end{itemize}
    \end{column}
    \begin{column}{0.5\textwidth}
      \listocaml[firstline=9,lastline=34]{../src/backtrace/backtrace.ml}{backtrace/backtrace.ml}
    \end{column}
  \end{columns}
\end{frame}

\begin{frame}{Backtraces}{CMM and disassembly of \funcname{definitely\_raise}}
  \begin{columns}[c]
    \begin{column}{0.5\textwidth}
      \minipage[c][0.66\textheight][s]{\columnwidth}
      \begin{itemize}
        \item<1-> An effect of compiling with debugging information (\funcarg{-g}): the CMM no longer uses \funcname{raise\_notrace} to raise the exception but \funcname{raise} instead
        \item<2-> Disassembly shows that CMM \funcname{raise} is actually a call to \funcname{caml\_raise\_exn}, a function in assembler provided by the runtime
      \end{itemize}
      \vfill
      \mintocaml[firstline=9,lastline=13,highlightlines=12]{../src/backtrace/backtrace.ml}
      \endminipage
    \end{column}
    \begin{column}{0.5\textwidth}
      \onslide*<1>{
        \mintcmm[highlightlines=5]{../src/backtrace/camlBacktrace\_\_definitely\_raise\_347.cmm}
      }
      \onslide*<2>{
        \mintobjdump[firstline=2,highlightlines=14]{../src/backtrace/camlBacktrace\_\_definitely\_raise\_347.objdump}
      }
    \end{column}
  \end{columns}
\end{frame}

\begin{frame}{Backtraces}{Introducing \funcname{caml\_raise\_exn}}
  \begin{columns}[c]
    \begin{column}{0.5\textwidth}
      \minipage[c][0.66\textheight][s]{\columnwidth}
      \onslide*<1>{
        \begin{itemize}
          \item Raising the exception is performed using \funcname{caml\_raise\_exn} which is
            \begin{itemize}
              \item An assembly function provided by the runtime
              \item Architecture specific
            \end{itemize}
          \item Let's see what it does differently from \funcname{raise\_notrace}\dots
          \item We are about to raise \typename{ExnC} with \funcname{caml\_raise\_exn} from \funcname{definitely\_raise}
        \end{itemize}
      }
      \onslide*<2>{
        \mintobjdump[firstline=2,highlightlines=14]{../src/backtrace/camlBacktrace\_\_definitely\_raise\_347.objdump}
      }
      \vfill
      \mintocaml[firstline=9,lastline=13,highlightlines=12]{../src/backtrace/backtrace.ml}
      \endminipage
    \end{column}
    \begin{column}{0.5\textwidth}
      \centering
      \providecommand\step{1}\input{diagrams/backtrace/backtrace.tex}
    \end{column}
  \end{columns}
\end{frame}

\begin{frame}{Backtraces}{But before that, we need more fields from \typename{caml\_domain\_state}}
  \begin{columns}
    \begin{column}{0.5\textwidth}
      \begin{itemize}
        \item Remember \typename{caml\_domain\_state} the per-domain runtime state?
        \item Some other members are of interest to us now\dots
        \item \localname{backtrace\_active} is used to know if the runtime should collect backtraces
        \item \localname{backtrace\_active} can be set by
          \begin{itemize}
            \item The \funcarg{b} flag in the \funcname{OCAMLRUNPARAM} environment variable
            \item \mintinline{ocaml}{Printexc.record_backtrace : bool -> unit} \footnotemark
          \end{itemize}
      \end{itemize}
    \end{column}
    \begin{column}{0.5\textwidth}
      \begin{listing}
        \mintc[highlightlines={9-11}]{src/ocaml/domain\_state\_backtrace.h}
        \caption{runtime/caml/domain\_state.h}
      \end{listing}
    \end{column}
  \end{columns}
  \footnotetext[1]{https://v2.ocaml.org/api/Printexc.html}
\end{frame}

\newcommand\listCamlRaiseExn[1][]{
  \listasm[firstline=702,lastline=725,#1]{../ocaml/runtime/amd64.S}{runtime/amd64.S:702}}

\begin{frame}{Backtraces}{Walk through \funcname{caml\_raise\_exn}}
  \begin{columns}[T]
    \begin{column}{0.5\textwidth}
      \begin{itemize}
        \item<1-> First checks whether exceptions backtrace should be recorded
        \item<1-> By checking the \funcarg{domain\_state->backtrace\_active} value
        \item<2-> If not, acts as \funcname{raise\_notrace} with \funcname{RESTORE\_EXN\_HANDLER\_OCAML}
          \begin{itemize}
            \item Set \regname{RSP} to \localname{domain\_state->exn\_handler} value
            \item Restore \localname{domain\_state->exn\_handler} from trap
            \item Jump with \funcname{ret} (same effect as using \funcname{pop} and \funcname{jmp})
          \end{itemize}
        \item<3-> Otherwise, switches to the C stack to be able to execute C code
        \item<4-> Calls \funcname{caml\_stash\_backtrace}, a C function provided by the runtime\ignorespaces
          \onslide<6->{, with
            \begin{itemize}
              \item \funcarg{exn}: exception value
              \item \funcarg{pc}: code address of the raising point
              \item \funcarg{sp}: stack address of the raising function
              \item \funcarg{trapsp}: stack address of the next handler
            \end{itemize}
          }
      \end{itemize}
      \bigskip
      \mintocaml[firstline=9,lastline=13,highlightlines=12]{../src/backtrace/backtrace.ml}
    \end{column}
    \begin{column}{0.5\textwidth}
      \raggedleft
      \onslide*<1,3-4>{
        \mintc[firstline=190,lastline=192]{../ocaml/runtime/amd64.S}
        \mintc[firstline=234,lastline=237]{../ocaml/runtime/amd64.S}
      }
      \onslide*<2>{
        \mintc[firstline=190,lastline=192,highlightlines={191-192}]{../ocaml/runtime/amd64.S}
        \mintc[firstline=234,lastline=237,highlightlines={235-237}]{../ocaml/runtime/amd64.S}
      }
      \onslide*<1>{\listCamlRaiseExn[highlightlines=705]{}}%
      \onslide*<2>{\listCamlRaiseExn[highlightlines={707-708}]{}}%
      \onslide*<3>{\listCamlRaiseExn[highlightlines={712-714}]{}}%
      \onslide*<4>{\listCamlRaiseExn[highlightlines={715-720}]{}}%
      \onslide*<5>{\providecommand\step{1}\input{diagrams/backtrace/backtrace.tex}}%
      \onslide*<6>{\providecommand\step{2}\input{diagrams/backtrace/backtrace.tex}}%
    \end{column}
  \end{columns}
\end{frame}

\newcommand\listStashBacktrace[1][]{
  \listc[firstline=91,lastline=120,#1]{../ocaml/runtime/backtrace\_nat.c}{runtime/backtrace\_nat.c:91}}

\begin{frame}{Backtraces}{Introducing \funcname{caml\_stash\_backtrace}}
  \begin{columns}[c]
    \begin{column}{0.5\textwidth}
      \minipage[c][0.66\textheight][s]{\columnwidth}
      \begin{itemize}
        \item<1-> A C function provided by the runtime (specific to native code)
        \item<2-> It scans the stack, between the stack pointer at the raising point up to the next trap, in order to collect the names of the detected function frames
          \onslide*<2>{
            \begin{itemize}
              \item And more information, like line number, column number...
            \end{itemize}
          }
        \item<3-> It uses the same technique and information as the Garbage Collector (GC) uses to scan the values live on the stack
          \onslide*<4>{
            \begin{itemize}
              \item \typename{frame\_descr} are at the core of stack unwinding
            \end{itemize}
          }
      \end{itemize}
      \vfill
      \mintocaml[firstline=9,lastline=13,highlightlines=12]{../src/backtrace/backtrace.ml}
      \endminipage
    \end{column}
    \begin{column}{0.5\textwidth}
      \onslide*<1>{\listStashBacktrace[highlightlines=91]{}}
      \onslide*<2>{\listStashBacktrace[highlightlines={91,117-118}]{}}
      \onslide*<3->{\listStashBacktrace[highlightlines={94,106,109-110}]{}}
    \end{column}
  \end{columns}
\end{frame}

\newcommand\listFrameDescr[1][]{
  \listocaml[firstline=111,lastline=124,#1]{../ocaml/asmcomp/emitaux.ml}{asmcomp/emitaux.ml:111}}
\newcommand\listDebugInfo[1][]{
  \listocaml[firstline=38,lastline=49,#1]{../ocaml/lambda/debuginfo.mli}{lambda/debuginfo.mli:38}}

\begin{frame}{Backtraces}{\typename{frame\_descr} and \typename{frame\_debuginfo} in a nutshell}
  \begin{columns}[c]
    \begin{column}{0.5\textwidth}
      \begin{itemize}
        \item<1-> For every return address in an OCaml program, there is a associated \typename{frame\_descr}
        \item<2-> A \typename{frame\_descr} specifies
        \begin{itemize}
          \item<2-> The size of the function stack frame
          \item<3-> Where live values are on the stack (so that the GC can mark them as live and prevent from being swept)
        \end{itemize}
        \item<4-> When compiled with debug information, \typename{frame\_descr} will be linked to a \typename{frame\_debuginfo}
        \begin{itemize}
          \item<5-> Contains information about the position in the source code (file name, line and column number)
        \end{itemize}
      \end{itemize}
    \end{column}
    \begin{column}{0.5\textwidth}
        \onslide*<1>{\listFrameDescr[highlightlines={118-122}]{}}
        \onslide*<2>{\listFrameDescr[highlightlines={120}]{}}
        \onslide*<3>{\listFrameDescr[highlightlines={121}]{}}
        \onslide*<4->{\listFrameDescr[highlightlines={122}]{}}
        \medskip
        \onslide*<1-3>{\listDebugInfo{}}
        \onslide*<4>{\listDebugInfo[highlightlines={38,49}]{}}
        \onslide*<5>{\listDebugInfo[highlightlines={39-46}]{}}
    \end{column}
  \end{columns}
\end{frame}

\begin{frame}{Backtraces}{Scanning the stack with \typename{frame\_descr}}
  \begin{columns}[c]
    \begin{column}{0.5\textwidth}
      \begin{itemize}
        \item Given the return address of the code that raises the exception by calling \funcname{caml\_raise\_exn} (\funcarg{pc} argument of \funcname{caml\_stash\_backtrace})
        \item And the position of the stack pointer (\funcarg{sp} argument of \funcname{caml\_stash\_backtrace})
        \item The frame size given by \typename{frame\_descr}.\funcarg{frame\_size}, allows to find the end of the previous frame
        \item And the return address of the caller of this function
        \bigskip
        \onslide<3->{
        \item Note that traps (here the reddish \funcname{ExnA}, \funcname{ExnB} and \funcname{ExnC} blocks) belong to the frame that installed them, and so the frame size changes when traps are removed
        }
      \end{itemize}
    \end{column}
    \begin{column}{0.5\textwidth}
      \raggedleft
      \onslide*<1>{\providecommand\step{2}\input{diagrams/backtrace/backtrace.tex}}%
      \onslide*<2>{\providecommand\step{1}\input{diagrams/backtrace/scanstack.tex}}%
      \onslide*<3>{\providecommand\step{2}\input{diagrams/backtrace/scanstack.tex}}%
      \onslide*<4>{\providecommand\step{3}\input{diagrams/backtrace/scanstack.tex}}%
      \onslide*<5>{\providecommand\step{4}\input{diagrams/backtrace/scanstack.tex}}%
      \onslide*<6>{\providecommand\step{5}\input{diagrams/backtrace/scanstack.tex}}%
    \end{column}
  \end{columns}
\end{frame}

% Duplicate of previous-previous slide
\begin{frame}{Backtraces}{Continuing on \funcname{caml\_stash\_backtrace}}
  \begin{columns}[c]
    \begin{column}{0.5\textwidth}
      \minipage[c][0.75\textheight][s]{\columnwidth}
      \begin{itemize}
          \color{gray}
        \item A C function provided by the runtime (specific to native code)
        \item It scans the stack, between the stack pointer at the raising point up to the next trap, in order to collect the names of the detected function frames
        \item It uses the same technique and information as the Garbage Collector (GC) uses to scan the values live on the stack
      \end{itemize}
      \begin{itemize}
        \item For each function frame detected, store a pointer to the \typename{frame\_descr} in the \localname{domain\_state->backtrace\_buffer} array, effectively constructing the backtrace
      \end{itemize}
      \onslide<2->{
        \begin{itemize}
            \smallskip
          \item Constructed backtrace:
            \funcname{definitely\_raise},
            \onslide<3->{\funcname{probably\_raise}}
        \end{itemize}
      }
      \vfill
      \mintocaml[firstline=9,lastline=13,highlightlines=12]{../src/backtrace/backtrace.ml}
      \endminipage
    \end{column}
    \begin{column}{0.5\textwidth}
      \raggedleft
      \onslide*<1>{\listStashBacktrace[highlightlines={114-115}]{}}
      \onslide*<2>{\providecommand\step{3}\input{diagrams/backtrace/backtrace.tex}}%
      \onslide*<3>{\providecommand\step{4}\input{diagrams/backtrace/backtrace.tex}}%
    \end{column}
  \end{columns}
\end{frame}

\begin{frame}{Backtraces}{Back to \funcname{caml\_raise\_exn}}
  \begin{columns}[c]
    \begin{column}{0.5\textwidth}
      \minipage[c][0.66\textheight][s]{\columnwidth}
      \begin{itemize}\color{gray}
        \item Checks wheteher exceptions backtrace should be recorded, by checking the \funcarg{domain\_state->backtrace\_active} value
        \item Switches to the C stack to be able to execute C code
        \item Calls \funcname{caml\_stash\_backtrace}, to collect the backtrace up to the next trap
      \end{itemize}
      \onslide*<2->{
        \begin{itemize}
          \item<2-> Executes the exception handler in \funcname{probably\_raise} with \funcname{RESTORE\_EXN\_HANDLER\_OCAML}
            \begin{itemize}
              \item Set \regname{RSP} to \localname{domain\_state->exn\_handler} value
              \item Restore \localname{domain\_state->exn\_handler} from trap
              \item Jump to the handler code with \funcname{ret}
            \end{itemize}
        \end{itemize}
      }
      \vfill
      \onslide*<1>{\mintocaml[firstline=9,lastline=13,highlightlines=12]{../src/backtrace/backtrace.ml}}
      \onslide*<2->{\mintocaml[firstline=15,lastline=21,highlightlines={19-21}]{../src/backtrace/backtrace.ml}}
      \endminipage
    \end{column}
    \begin{column}{0.5\textwidth}
      \centering
      \onslide*<1>{
        \mintc[firstline=190,lastline=192]{../ocaml/runtime/amd64.S}
        \mintc[firstline=234,lastline=237]{../ocaml/runtime/amd64.S}
      }
      \onslide*<2>{
        \mintc[firstline=190,lastline=192,highlightlines={191-192}]{../ocaml/runtime/amd64.S}
        \mintc[firstline=234,lastline=237,highlightlines={235-237}]{../ocaml/runtime/amd64.S}
      }
      \onslide*<1>{\listCamlRaiseExn[highlightlines=720]{}}
      \onslide*<2>{\listCamlRaiseExn[highlightlines=722]{}}
      \onslide*<3->{\providecommand\step{5}\input{diagrams/backtrace/backtrace.tex}}
    \end{column}
  \end{columns}
\end{frame}

\begin{frame}{Backtraces}{\funcname{caml\_probably\_raise}'s handler doesn't catch \typename{ExnC}}
  \begin{columns}[c]
    \begin{column}{0.45\textwidth}
      \minipage[c][0.75\textheight][s]{\columnwidth}
      \begin{itemize}
        \item<1-> \funcname{match \dots with}\footnotemark pattern matching can also match exceptions
        \item<1-> The CMM of \funcname{probably\_raise} shows that this \funcname{match \dots with} is implemented with a pattern matching and a \funcname{try \dots with}
        \item<1-> But this one only matches on \typename{ExnA} exception
        \item<2-> Since the pattern matching fails to catch the exception, \funcname{reraise} is called with our current \typename{ExnC} exception
        \item<3-> Disassembly shows that \funcname{reraise} is actually a call to \funcname{caml\_reraise\_exn}, which is also an assembly function provided by the runtime
      \end{itemize}
      \vfill
      \mintocaml[firstline=15,lastline=21,highlightlines={18-21}]{../src/backtrace/backtrace.ml}
      \endminipage
    \end{column}
    \begin{column}{0.55\textwidth}
      \centering
      \onslide*<1>{\mintcmm[highlightlines={10-18}]{../src/backtrace/camlBacktrace\_\_probably\_raise\_351.cmm}}
      \onslide*<2>{\listcmm[highlightlines=18]{../src/backtrace/camlBacktrace\_\_probably\_raise_351.cmm}{CMM of probably\_raise}}
      \onslide*<3>{\mintobjdump[firstline=5,lastline=35,highlightlines=31]{../src/backtrace/camlBacktrace\_\_probably\_raise_351.objdump}}
    \end{column}
  \end{columns}
  \only<1>{\footnotetext[1]{https://blog.janestreet.com/pattern-matching-and-exception-handling-unite/}}
\end{frame}

\newcommand\listCamlReraiseExn[1][]{
  \listasm[firstline=727,lastline=734,#1]{../ocaml/runtime/amd64.S}{runtime/amd64.S:727}}

\begin{frame}{Backtraces}{Introducing \funcname{caml\_reraise\_exn}}
  \begin{columns}[c]
    \begin{column}{0.5\textwidth}
      \minipage[c][0.66\textheight][s]{\columnwidth}
      \begin{itemize}
        \item<1-> The exception have to be reraised to the next handler
        \item<2-> First, checks if the backtrace should be collected with \funcarg{domain\_state->backtrace\_active}
        \only<3>{
        \begin{itemize}
            \item If not, behaves like a \funcname{raise\_notrace} with \funcname{RESTORE\_EXN\_HANDLER\_OCAML}
        \end{itemize}
        }
        \item<4-> Jumps into \funcname{caml\_raise\_exn}
        \item<5-> Calls \funcname{caml\_stash\_backtrace} again, to scan the next part of the stack: from the trap just pop-ed to the next trap
      \end{itemize}
      \vfill
      \mintocaml[firstline=15,lastline=21,highlightlines={18-21}]{../src/backtrace/backtrace.ml}
      \endminipage
    \end{column}
    \begin{column}{0.5\textwidth}
      \raggedleft
      \onslide*<1>{\listCamlReraiseExn{}}
      \onslide*<2>{\listCamlReraiseExn[highlightlines=729]{}}
      \onslide*<3>{
        \mintc[firstline=190,lastline=192,highlightlines={191-192}]{../ocaml/runtime/amd64.S}
        \mintc[firstline=234,lastline=237,highlightlines={235-237}]{../ocaml/runtime/amd64.S}
        \medskip
        \listCamlReraiseExn[highlightlines=731]{}
      }
      \onslide*<4-5>{
        % TODO refactor with \listCamlReraiseExn
        \mintasm[firstline=727,lastline=734,highlightlines=730]{../ocaml/runtime/amd64.S}
        \medskip
      }
      \onslide*<4>{\listCamlRaiseExn[highlightlines=711]{}}
      \onslide*<5>{\listCamlRaiseExn[highlightlines=720]{}}
    \end{column}
  \end{columns}
\end{frame}

\begin{frame}{Backtraces}{Backtrace collection continues}
  \begin{columns}[c]
    \begin{column}{0.55\textwidth}
      \minipage[c][0.75\textheight][s]{\columnwidth}
      \begin{itemize}
        \item<1-> \funcname{caml\_stash\_backtrace} scans the older part of the stack, up to the next trap
        \item<6-> Controls jumps to the nested exception handler in \funcname{maybe\_raise}, which matches only on \typename{ExnB}
        \item<7-> \funcname{caml\_reraise\_exn} is called again, backtrace collection resumes with \funcname{caml\_stash\_backtrace}
      \end{itemize}
      \medskip
      \begin{itemize}
        \item<2-> Constructed backtrace:
        \begin{itemize}
          \item <2-> \funcname{definitely\_raise}
          \item <2-> \funcname{probably\_raise}
          \item <3-> \funcname{probably\_raise} (re-raise)
          \item <4-> \funcname{maybe\_raise}
          \item <5-> \funcname{main}
          \item <8-> \funcname{main} (re-raise)
        \end{itemize}
      \end{itemize}
      \vfill
      \onslide*<1-5>{\mintocaml[firstline=15,lastline=21]{../src/backtrace/backtrace.ml}}
      \onslide*<6>{\mintocaml[firstline=28,lastline=34,highlightlines=31]{../src/backtrace/backtrace.ml}}
      \onslide*<7->{\mintocaml[firstline=28,lastline=34,highlightlines=33]{../src/backtrace/backtrace.ml}}
      \endminipage
    \end{column}
    \begin{column}{0.45\textwidth}
      \raggedleft
      \onslide*<1>{\providecommand\step{6}\input{diagrams/backtrace/backtrace.tex}}%
      \onslide*<2>{\providecommand\step{6}\input{diagrams/backtrace/backtrace.tex}}%
      \onslide*<3>{\providecommand\step{7}\input{diagrams/backtrace/backtrace.tex}}%
      \onslide*<4>{\providecommand\step{8}\input{diagrams/backtrace/backtrace.tex}}%
      \onslide*<5>{\providecommand\step{9}\input{diagrams/backtrace/backtrace.tex}}%
      \onslide*<6>{\providecommand\step{10}\input{diagrams/backtrace/backtrace.tex}}%
      \onslide*<7>{\providecommand\step{11}\input{diagrams/backtrace/backtrace.tex}}%
      \onslide*<8>{\providecommand\step{12}\input{diagrams/backtrace/backtrace.tex}}%
      \onslide*<9>{\providecommand\step{13}\input{diagrams/backtrace/backtrace.tex}}%
    \end{column}
  \end{columns}
\end{frame}

\begin{frame}{Backtraces}{Printing the backtrace}
  \begin{columns}[c]
    \begin{column}{0.45\textwidth}
      \minipage[c][0.66\textheight][s]{\columnwidth}
      \begin{itemize}
        \item<1-> The exception backtrace can now be printed to \funcarg{stdout} with \funcname{Printexc.print_backtrace}
        \item<2-> First the exception backtrace is retrieved into an OCaml array with \funcname{get_raw_backtrace }
        \item<3-> Which is implemented as the \funcname{caml_get_exception_raw_backtrace} C function in the runtime
        \item<4-> And finally printed with \funcname{print_exception_backtrace}
      \end{itemize}
      \vfill
      \mintocaml[firstline=28,lastline=34,highlightlines=33]{../src/backtrace/backtrace.ml}
      \endminipage
    \end{column}
    \begin{column}{0.55\textwidth}
      \onslide*<1,2>{
        \mintocaml[firstline=100,lastline=101]{../ocaml/stdlib/printexc.ml}
      }
      \only<1>{
        \listocaml[firstline=170,lastline=172]{../ocaml/stdlib/printexc.ml}{stdlib/printexc.ml:170}
      }
      \only<2>{
        \listocaml[firstline=170,lastline=172,highlightlines=172]{../ocaml/stdlib/printexc.ml}{stdlib/printexc.ml:170}
      }
      \only<3>{
        \mintc[firstline=154,lastline=156]{../ocaml/runtime/backtrace.c}
        \listc[firstline=171,lastline=192,highlightlines={184-187}]{../ocaml/runtime/backtrace.c}{runtime/backtrace.c:154}
      }
      \only<4>{
        \listocaml[firstline=155,lastline=165]{../ocaml/stdlib/printexc.ml}{stdlib/printexc.ml:55}
      }
    \end{column}
  \end{columns}
\end{frame}


\subsection{The multiple kinds of raise}
\frameSubsection{}{}

\begin{frame}{Backtraces}{When backtraces are not needed}
  \begin{columns}[c]
    \begin{column}{0.45\textwidth}
      \minipage[c][0.66\textheight][s]{\columnwidth}
      \begin{itemize}
        \item<1-> What if the backtrace for an exception is never needed?
        \item<2-> It's possible to raise the exception explicitly with \funcname{raise\_notrace} to make raising as cheap as possible
        \item<3-> An example of use in the \funcname{dynlink} library of the OCaml compiler
      \end{itemize}
      \vfill
      \onslide<3->{
      \listocaml[firstline=205,lastline=209,highlightlines=207]{../ocaml/otherlibs/dynlink/dynlink\_compilerlibs/misc.ml}{otherlibs/dynlink/dynlink\_compilerlibs/misc.ml:205}
      }
      \endminipage
    \end{column}
    \begin{column}{0.55\textwidth}
    \only<4>{
      \mintcmm[highlightlines=3]{../src/notrace/camlNotrace\_\_fun_347.cmm}
      \listcmm{../src/notrace/camlNotrace\_\_all\_somes\_327.cmm}{all\_somes}
    }
    \onslide*<5->{
      \listobjdump[highlightlines={6-9}]{../src/notrace/camlNotrace\_\_fun_347.objdump}{anonymous function from \funcname{all\_somes}}
    }
    \end{column}
  \end{columns}
\end{frame}

\begin{frame}{Backtraces}{Summary table of the multiple kinds of raise}
  \begin{itemize}
    \item When debugging information is enabled and if the backtrace of an exception isn't needed, it's possible to raise with \funcname{raise\_notrace} for performance
    \item When debugging information is \emph{not} enabled, the compiler optimises \funcname{raise} calls to inline assembly (same effect as using \funcname{raise\_notrace})
  \end{itemize}
  \bigskip
  \centering
  \begin{tabular}{| l | c | c | c | c |}
    \hline
    \parbox{10em}{Debug information \\ (\funcarg{-g} flag)}  & \multicolumn{2}{|c|}{Disabled}                                     & \multicolumn{2}{|c|}{Enabled} \\
    \hline
    Raise kind                                               & \funcname{raise} & \funcname{raise\_notrace}                       & \funcname{raise} & \funcname{raise\_notrace} \\
    \hline
    Compilation                                              & \parbox{8em}{inline assembly\\ (optimization)} & inline assembly   & \funcname{caml\_raise\_exn} & inline assembly \\
    \hline
  \end{tabular}
\end{frame}


%
% Takeaway #5
%
\subsection*{Takeaway \#5}
\frameSubsectionTakeaway{}
\againframe<5>{takeaway}
}%
      \onslide*<6>{\providecommand\step{2}%
% Backtraces
%
\subsection{One advantage of raising with debugging information}
\frameSubsection{}{}

\begin{frame}{Backtraces}{Debugging information \& source code}
  \begin{columns}[c]
    \begin{column}{0.5\textwidth}
      \begin{itemize}
        \item Raising an exception is fun and games, but how do we know where the exception originated? $\rightarrow$ With the backtrace associated with the exception!
        \item In order to get a backtrace from the point where an exception was raised, it's necessary to compile with debugging information enabled (\funcarg{-g} flag for the compiler)
        \item Same example as in the `Nested handlers' section
      \end{itemize}
    \end{column}
    \begin{column}{0.5\textwidth}
      \listocaml[firstline=9,lastline=34]{../src/backtrace/backtrace.ml}{backtrace/backtrace.ml}
    \end{column}
  \end{columns}
\end{frame}

\begin{frame}{Backtraces}{CMM and disassembly of \funcname{definitely\_raise}}
  \begin{columns}[c]
    \begin{column}{0.5\textwidth}
      \minipage[c][0.66\textheight][s]{\columnwidth}
      \begin{itemize}
        \item<1-> An effect of compiling with debugging information (\funcarg{-g}): the CMM no longer uses \funcname{raise\_notrace} to raise the exception but \funcname{raise} instead
        \item<2-> Disassembly shows that CMM \funcname{raise} is actually a call to \funcname{caml\_raise\_exn}, a function in assembler provided by the runtime
      \end{itemize}
      \vfill
      \mintocaml[firstline=9,lastline=13,highlightlines=12]{../src/backtrace/backtrace.ml}
      \endminipage
    \end{column}
    \begin{column}{0.5\textwidth}
      \onslide*<1>{
        \mintcmm[highlightlines=5]{../src/backtrace/camlBacktrace\_\_definitely\_raise\_347.cmm}
      }
      \onslide*<2>{
        \mintobjdump[firstline=2,highlightlines=14]{../src/backtrace/camlBacktrace\_\_definitely\_raise\_347.objdump}
      }
    \end{column}
  \end{columns}
\end{frame}

\begin{frame}{Backtraces}{Introducing \funcname{caml\_raise\_exn}}
  \begin{columns}[c]
    \begin{column}{0.5\textwidth}
      \minipage[c][0.66\textheight][s]{\columnwidth}
      \onslide*<1>{
        \begin{itemize}
          \item Raising the exception is performed using \funcname{caml\_raise\_exn} which is
            \begin{itemize}
              \item An assembly function provided by the runtime
              \item Architecture specific
            \end{itemize}
          \item Let's see what it does differently from \funcname{raise\_notrace}\dots
          \item We are about to raise \typename{ExnC} with \funcname{caml\_raise\_exn} from \funcname{definitely\_raise}
        \end{itemize}
      }
      \onslide*<2>{
        \mintobjdump[firstline=2,highlightlines=14]{../src/backtrace/camlBacktrace\_\_definitely\_raise\_347.objdump}
      }
      \vfill
      \mintocaml[firstline=9,lastline=13,highlightlines=12]{../src/backtrace/backtrace.ml}
      \endminipage
    \end{column}
    \begin{column}{0.5\textwidth}
      \centering
      \providecommand\step{1}\input{diagrams/backtrace/backtrace.tex}
    \end{column}
  \end{columns}
\end{frame}

\begin{frame}{Backtraces}{But before that, we need more fields from \typename{caml\_domain\_state}}
  \begin{columns}
    \begin{column}{0.5\textwidth}
      \begin{itemize}
        \item Remember \typename{caml\_domain\_state} the per-domain runtime state?
        \item Some other members are of interest to us now\dots
        \item \localname{backtrace\_active} is used to know if the runtime should collect backtraces
        \item \localname{backtrace\_active} can be set by
          \begin{itemize}
            \item The \funcarg{b} flag in the \funcname{OCAMLRUNPARAM} environment variable
            \item \mintinline{ocaml}{Printexc.record_backtrace : bool -> unit} \footnotemark
          \end{itemize}
      \end{itemize}
    \end{column}
    \begin{column}{0.5\textwidth}
      \begin{listing}
        \mintc[highlightlines={9-11}]{src/ocaml/domain\_state\_backtrace.h}
        \caption{runtime/caml/domain\_state.h}
      \end{listing}
    \end{column}
  \end{columns}
  \footnotetext[1]{https://v2.ocaml.org/api/Printexc.html}
\end{frame}

\newcommand\listCamlRaiseExn[1][]{
  \listasm[firstline=702,lastline=725,#1]{../ocaml/runtime/amd64.S}{runtime/amd64.S:702}}

\begin{frame}{Backtraces}{Walk through \funcname{caml\_raise\_exn}}
  \begin{columns}[T]
    \begin{column}{0.5\textwidth}
      \begin{itemize}
        \item<1-> First checks whether exceptions backtrace should be recorded
        \item<1-> By checking the \funcarg{domain\_state->backtrace\_active} value
        \item<2-> If not, acts as \funcname{raise\_notrace} with \funcname{RESTORE\_EXN\_HANDLER\_OCAML}
          \begin{itemize}
            \item Set \regname{RSP} to \localname{domain\_state->exn\_handler} value
            \item Restore \localname{domain\_state->exn\_handler} from trap
            \item Jump with \funcname{ret} (same effect as using \funcname{pop} and \funcname{jmp})
          \end{itemize}
        \item<3-> Otherwise, switches to the C stack to be able to execute C code
        \item<4-> Calls \funcname{caml\_stash\_backtrace}, a C function provided by the runtime\ignorespaces
          \onslide<6->{, with
            \begin{itemize}
              \item \funcarg{exn}: exception value
              \item \funcarg{pc}: code address of the raising point
              \item \funcarg{sp}: stack address of the raising function
              \item \funcarg{trapsp}: stack address of the next handler
            \end{itemize}
          }
      \end{itemize}
      \bigskip
      \mintocaml[firstline=9,lastline=13,highlightlines=12]{../src/backtrace/backtrace.ml}
    \end{column}
    \begin{column}{0.5\textwidth}
      \raggedleft
      \onslide*<1,3-4>{
        \mintc[firstline=190,lastline=192]{../ocaml/runtime/amd64.S}
        \mintc[firstline=234,lastline=237]{../ocaml/runtime/amd64.S}
      }
      \onslide*<2>{
        \mintc[firstline=190,lastline=192,highlightlines={191-192}]{../ocaml/runtime/amd64.S}
        \mintc[firstline=234,lastline=237,highlightlines={235-237}]{../ocaml/runtime/amd64.S}
      }
      \onslide*<1>{\listCamlRaiseExn[highlightlines=705]{}}%
      \onslide*<2>{\listCamlRaiseExn[highlightlines={707-708}]{}}%
      \onslide*<3>{\listCamlRaiseExn[highlightlines={712-714}]{}}%
      \onslide*<4>{\listCamlRaiseExn[highlightlines={715-720}]{}}%
      \onslide*<5>{\providecommand\step{1}\input{diagrams/backtrace/backtrace.tex}}%
      \onslide*<6>{\providecommand\step{2}\input{diagrams/backtrace/backtrace.tex}}%
    \end{column}
  \end{columns}
\end{frame}

\newcommand\listStashBacktrace[1][]{
  \listc[firstline=91,lastline=120,#1]{../ocaml/runtime/backtrace\_nat.c}{runtime/backtrace\_nat.c:91}}

\begin{frame}{Backtraces}{Introducing \funcname{caml\_stash\_backtrace}}
  \begin{columns}[c]
    \begin{column}{0.5\textwidth}
      \minipage[c][0.66\textheight][s]{\columnwidth}
      \begin{itemize}
        \item<1-> A C function provided by the runtime (specific to native code)
        \item<2-> It scans the stack, between the stack pointer at the raising point up to the next trap, in order to collect the names of the detected function frames
          \onslide*<2>{
            \begin{itemize}
              \item And more information, like line number, column number...
            \end{itemize}
          }
        \item<3-> It uses the same technique and information as the Garbage Collector (GC) uses to scan the values live on the stack
          \onslide*<4>{
            \begin{itemize}
              \item \typename{frame\_descr} are at the core of stack unwinding
            \end{itemize}
          }
      \end{itemize}
      \vfill
      \mintocaml[firstline=9,lastline=13,highlightlines=12]{../src/backtrace/backtrace.ml}
      \endminipage
    \end{column}
    \begin{column}{0.5\textwidth}
      \onslide*<1>{\listStashBacktrace[highlightlines=91]{}}
      \onslide*<2>{\listStashBacktrace[highlightlines={91,117-118}]{}}
      \onslide*<3->{\listStashBacktrace[highlightlines={94,106,109-110}]{}}
    \end{column}
  \end{columns}
\end{frame}

\newcommand\listFrameDescr[1][]{
  \listocaml[firstline=111,lastline=124,#1]{../ocaml/asmcomp/emitaux.ml}{asmcomp/emitaux.ml:111}}
\newcommand\listDebugInfo[1][]{
  \listocaml[firstline=38,lastline=49,#1]{../ocaml/lambda/debuginfo.mli}{lambda/debuginfo.mli:38}}

\begin{frame}{Backtraces}{\typename{frame\_descr} and \typename{frame\_debuginfo} in a nutshell}
  \begin{columns}[c]
    \begin{column}{0.5\textwidth}
      \begin{itemize}
        \item<1-> For every return address in an OCaml program, there is a associated \typename{frame\_descr}
        \item<2-> A \typename{frame\_descr} specifies
        \begin{itemize}
          \item<2-> The size of the function stack frame
          \item<3-> Where live values are on the stack (so that the GC can mark them as live and prevent from being swept)
        \end{itemize}
        \item<4-> When compiled with debug information, \typename{frame\_descr} will be linked to a \typename{frame\_debuginfo}
        \begin{itemize}
          \item<5-> Contains information about the position in the source code (file name, line and column number)
        \end{itemize}
      \end{itemize}
    \end{column}
    \begin{column}{0.5\textwidth}
        \onslide*<1>{\listFrameDescr[highlightlines={118-122}]{}}
        \onslide*<2>{\listFrameDescr[highlightlines={120}]{}}
        \onslide*<3>{\listFrameDescr[highlightlines={121}]{}}
        \onslide*<4->{\listFrameDescr[highlightlines={122}]{}}
        \medskip
        \onslide*<1-3>{\listDebugInfo{}}
        \onslide*<4>{\listDebugInfo[highlightlines={38,49}]{}}
        \onslide*<5>{\listDebugInfo[highlightlines={39-46}]{}}
    \end{column}
  \end{columns}
\end{frame}

\begin{frame}{Backtraces}{Scanning the stack with \typename{frame\_descr}}
  \begin{columns}[c]
    \begin{column}{0.5\textwidth}
      \begin{itemize}
        \item Given the return address of the code that raises the exception by calling \funcname{caml\_raise\_exn} (\funcarg{pc} argument of \funcname{caml\_stash\_backtrace})
        \item And the position of the stack pointer (\funcarg{sp} argument of \funcname{caml\_stash\_backtrace})
        \item The frame size given by \typename{frame\_descr}.\funcarg{frame\_size}, allows to find the end of the previous frame
        \item And the return address of the caller of this function
        \bigskip
        \onslide<3->{
        \item Note that traps (here the reddish \funcname{ExnA}, \funcname{ExnB} and \funcname{ExnC} blocks) belong to the frame that installed them, and so the frame size changes when traps are removed
        }
      \end{itemize}
    \end{column}
    \begin{column}{0.5\textwidth}
      \raggedleft
      \onslide*<1>{\providecommand\step{2}\input{diagrams/backtrace/backtrace.tex}}%
      \onslide*<2>{\providecommand\step{1}\input{diagrams/backtrace/scanstack.tex}}%
      \onslide*<3>{\providecommand\step{2}\input{diagrams/backtrace/scanstack.tex}}%
      \onslide*<4>{\providecommand\step{3}\input{diagrams/backtrace/scanstack.tex}}%
      \onslide*<5>{\providecommand\step{4}\input{diagrams/backtrace/scanstack.tex}}%
      \onslide*<6>{\providecommand\step{5}\input{diagrams/backtrace/scanstack.tex}}%
    \end{column}
  \end{columns}
\end{frame}

% Duplicate of previous-previous slide
\begin{frame}{Backtraces}{Continuing on \funcname{caml\_stash\_backtrace}}
  \begin{columns}[c]
    \begin{column}{0.5\textwidth}
      \minipage[c][0.75\textheight][s]{\columnwidth}
      \begin{itemize}
          \color{gray}
        \item A C function provided by the runtime (specific to native code)
        \item It scans the stack, between the stack pointer at the raising point up to the next trap, in order to collect the names of the detected function frames
        \item It uses the same technique and information as the Garbage Collector (GC) uses to scan the values live on the stack
      \end{itemize}
      \begin{itemize}
        \item For each function frame detected, store a pointer to the \typename{frame\_descr} in the \localname{domain\_state->backtrace\_buffer} array, effectively constructing the backtrace
      \end{itemize}
      \onslide<2->{
        \begin{itemize}
            \smallskip
          \item Constructed backtrace:
            \funcname{definitely\_raise},
            \onslide<3->{\funcname{probably\_raise}}
        \end{itemize}
      }
      \vfill
      \mintocaml[firstline=9,lastline=13,highlightlines=12]{../src/backtrace/backtrace.ml}
      \endminipage
    \end{column}
    \begin{column}{0.5\textwidth}
      \raggedleft
      \onslide*<1>{\listStashBacktrace[highlightlines={114-115}]{}}
      \onslide*<2>{\providecommand\step{3}\input{diagrams/backtrace/backtrace.tex}}%
      \onslide*<3>{\providecommand\step{4}\input{diagrams/backtrace/backtrace.tex}}%
    \end{column}
  \end{columns}
\end{frame}

\begin{frame}{Backtraces}{Back to \funcname{caml\_raise\_exn}}
  \begin{columns}[c]
    \begin{column}{0.5\textwidth}
      \minipage[c][0.66\textheight][s]{\columnwidth}
      \begin{itemize}\color{gray}
        \item Checks wheteher exceptions backtrace should be recorded, by checking the \funcarg{domain\_state->backtrace\_active} value
        \item Switches to the C stack to be able to execute C code
        \item Calls \funcname{caml\_stash\_backtrace}, to collect the backtrace up to the next trap
      \end{itemize}
      \onslide*<2->{
        \begin{itemize}
          \item<2-> Executes the exception handler in \funcname{probably\_raise} with \funcname{RESTORE\_EXN\_HANDLER\_OCAML}
            \begin{itemize}
              \item Set \regname{RSP} to \localname{domain\_state->exn\_handler} value
              \item Restore \localname{domain\_state->exn\_handler} from trap
              \item Jump to the handler code with \funcname{ret}
            \end{itemize}
        \end{itemize}
      }
      \vfill
      \onslide*<1>{\mintocaml[firstline=9,lastline=13,highlightlines=12]{../src/backtrace/backtrace.ml}}
      \onslide*<2->{\mintocaml[firstline=15,lastline=21,highlightlines={19-21}]{../src/backtrace/backtrace.ml}}
      \endminipage
    \end{column}
    \begin{column}{0.5\textwidth}
      \centering
      \onslide*<1>{
        \mintc[firstline=190,lastline=192]{../ocaml/runtime/amd64.S}
        \mintc[firstline=234,lastline=237]{../ocaml/runtime/amd64.S}
      }
      \onslide*<2>{
        \mintc[firstline=190,lastline=192,highlightlines={191-192}]{../ocaml/runtime/amd64.S}
        \mintc[firstline=234,lastline=237,highlightlines={235-237}]{../ocaml/runtime/amd64.S}
      }
      \onslide*<1>{\listCamlRaiseExn[highlightlines=720]{}}
      \onslide*<2>{\listCamlRaiseExn[highlightlines=722]{}}
      \onslide*<3->{\providecommand\step{5}\input{diagrams/backtrace/backtrace.tex}}
    \end{column}
  \end{columns}
\end{frame}

\begin{frame}{Backtraces}{\funcname{caml\_probably\_raise}'s handler doesn't catch \typename{ExnC}}
  \begin{columns}[c]
    \begin{column}{0.45\textwidth}
      \minipage[c][0.75\textheight][s]{\columnwidth}
      \begin{itemize}
        \item<1-> \funcname{match \dots with}\footnotemark pattern matching can also match exceptions
        \item<1-> The CMM of \funcname{probably\_raise} shows that this \funcname{match \dots with} is implemented with a pattern matching and a \funcname{try \dots with}
        \item<1-> But this one only matches on \typename{ExnA} exception
        \item<2-> Since the pattern matching fails to catch the exception, \funcname{reraise} is called with our current \typename{ExnC} exception
        \item<3-> Disassembly shows that \funcname{reraise} is actually a call to \funcname{caml\_reraise\_exn}, which is also an assembly function provided by the runtime
      \end{itemize}
      \vfill
      \mintocaml[firstline=15,lastline=21,highlightlines={18-21}]{../src/backtrace/backtrace.ml}
      \endminipage
    \end{column}
    \begin{column}{0.55\textwidth}
      \centering
      \onslide*<1>{\mintcmm[highlightlines={10-18}]{../src/backtrace/camlBacktrace\_\_probably\_raise\_351.cmm}}
      \onslide*<2>{\listcmm[highlightlines=18]{../src/backtrace/camlBacktrace\_\_probably\_raise_351.cmm}{CMM of probably\_raise}}
      \onslide*<3>{\mintobjdump[firstline=5,lastline=35,highlightlines=31]{../src/backtrace/camlBacktrace\_\_probably\_raise_351.objdump}}
    \end{column}
  \end{columns}
  \only<1>{\footnotetext[1]{https://blog.janestreet.com/pattern-matching-and-exception-handling-unite/}}
\end{frame}

\newcommand\listCamlReraiseExn[1][]{
  \listasm[firstline=727,lastline=734,#1]{../ocaml/runtime/amd64.S}{runtime/amd64.S:727}}

\begin{frame}{Backtraces}{Introducing \funcname{caml\_reraise\_exn}}
  \begin{columns}[c]
    \begin{column}{0.5\textwidth}
      \minipage[c][0.66\textheight][s]{\columnwidth}
      \begin{itemize}
        \item<1-> The exception have to be reraised to the next handler
        \item<2-> First, checks if the backtrace should be collected with \funcarg{domain\_state->backtrace\_active}
        \only<3>{
        \begin{itemize}
            \item If not, behaves like a \funcname{raise\_notrace} with \funcname{RESTORE\_EXN\_HANDLER\_OCAML}
        \end{itemize}
        }
        \item<4-> Jumps into \funcname{caml\_raise\_exn}
        \item<5-> Calls \funcname{caml\_stash\_backtrace} again, to scan the next part of the stack: from the trap just pop-ed to the next trap
      \end{itemize}
      \vfill
      \mintocaml[firstline=15,lastline=21,highlightlines={18-21}]{../src/backtrace/backtrace.ml}
      \endminipage
    \end{column}
    \begin{column}{0.5\textwidth}
      \raggedleft
      \onslide*<1>{\listCamlReraiseExn{}}
      \onslide*<2>{\listCamlReraiseExn[highlightlines=729]{}}
      \onslide*<3>{
        \mintc[firstline=190,lastline=192,highlightlines={191-192}]{../ocaml/runtime/amd64.S}
        \mintc[firstline=234,lastline=237,highlightlines={235-237}]{../ocaml/runtime/amd64.S}
        \medskip
        \listCamlReraiseExn[highlightlines=731]{}
      }
      \onslide*<4-5>{
        % TODO refactor with \listCamlReraiseExn
        \mintasm[firstline=727,lastline=734,highlightlines=730]{../ocaml/runtime/amd64.S}
        \medskip
      }
      \onslide*<4>{\listCamlRaiseExn[highlightlines=711]{}}
      \onslide*<5>{\listCamlRaiseExn[highlightlines=720]{}}
    \end{column}
  \end{columns}
\end{frame}

\begin{frame}{Backtraces}{Backtrace collection continues}
  \begin{columns}[c]
    \begin{column}{0.55\textwidth}
      \minipage[c][0.75\textheight][s]{\columnwidth}
      \begin{itemize}
        \item<1-> \funcname{caml\_stash\_backtrace} scans the older part of the stack, up to the next trap
        \item<6-> Controls jumps to the nested exception handler in \funcname{maybe\_raise}, which matches only on \typename{ExnB}
        \item<7-> \funcname{caml\_reraise\_exn} is called again, backtrace collection resumes with \funcname{caml\_stash\_backtrace}
      \end{itemize}
      \medskip
      \begin{itemize}
        \item<2-> Constructed backtrace:
        \begin{itemize}
          \item <2-> \funcname{definitely\_raise}
          \item <2-> \funcname{probably\_raise}
          \item <3-> \funcname{probably\_raise} (re-raise)
          \item <4-> \funcname{maybe\_raise}
          \item <5-> \funcname{main}
          \item <8-> \funcname{main} (re-raise)
        \end{itemize}
      \end{itemize}
      \vfill
      \onslide*<1-5>{\mintocaml[firstline=15,lastline=21]{../src/backtrace/backtrace.ml}}
      \onslide*<6>{\mintocaml[firstline=28,lastline=34,highlightlines=31]{../src/backtrace/backtrace.ml}}
      \onslide*<7->{\mintocaml[firstline=28,lastline=34,highlightlines=33]{../src/backtrace/backtrace.ml}}
      \endminipage
    \end{column}
    \begin{column}{0.45\textwidth}
      \raggedleft
      \onslide*<1>{\providecommand\step{6}\input{diagrams/backtrace/backtrace.tex}}%
      \onslide*<2>{\providecommand\step{6}\input{diagrams/backtrace/backtrace.tex}}%
      \onslide*<3>{\providecommand\step{7}\input{diagrams/backtrace/backtrace.tex}}%
      \onslide*<4>{\providecommand\step{8}\input{diagrams/backtrace/backtrace.tex}}%
      \onslide*<5>{\providecommand\step{9}\input{diagrams/backtrace/backtrace.tex}}%
      \onslide*<6>{\providecommand\step{10}\input{diagrams/backtrace/backtrace.tex}}%
      \onslide*<7>{\providecommand\step{11}\input{diagrams/backtrace/backtrace.tex}}%
      \onslide*<8>{\providecommand\step{12}\input{diagrams/backtrace/backtrace.tex}}%
      \onslide*<9>{\providecommand\step{13}\input{diagrams/backtrace/backtrace.tex}}%
    \end{column}
  \end{columns}
\end{frame}

\begin{frame}{Backtraces}{Printing the backtrace}
  \begin{columns}[c]
    \begin{column}{0.45\textwidth}
      \minipage[c][0.66\textheight][s]{\columnwidth}
      \begin{itemize}
        \item<1-> The exception backtrace can now be printed to \funcarg{stdout} with \funcname{Printexc.print_backtrace}
        \item<2-> First the exception backtrace is retrieved into an OCaml array with \funcname{get_raw_backtrace }
        \item<3-> Which is implemented as the \funcname{caml_get_exception_raw_backtrace} C function in the runtime
        \item<4-> And finally printed with \funcname{print_exception_backtrace}
      \end{itemize}
      \vfill
      \mintocaml[firstline=28,lastline=34,highlightlines=33]{../src/backtrace/backtrace.ml}
      \endminipage
    \end{column}
    \begin{column}{0.55\textwidth}
      \onslide*<1,2>{
        \mintocaml[firstline=100,lastline=101]{../ocaml/stdlib/printexc.ml}
      }
      \only<1>{
        \listocaml[firstline=170,lastline=172]{../ocaml/stdlib/printexc.ml}{stdlib/printexc.ml:170}
      }
      \only<2>{
        \listocaml[firstline=170,lastline=172,highlightlines=172]{../ocaml/stdlib/printexc.ml}{stdlib/printexc.ml:170}
      }
      \only<3>{
        \mintc[firstline=154,lastline=156]{../ocaml/runtime/backtrace.c}
        \listc[firstline=171,lastline=192,highlightlines={184-187}]{../ocaml/runtime/backtrace.c}{runtime/backtrace.c:154}
      }
      \only<4>{
        \listocaml[firstline=155,lastline=165]{../ocaml/stdlib/printexc.ml}{stdlib/printexc.ml:55}
      }
    \end{column}
  \end{columns}
\end{frame}


\subsection{The multiple kinds of raise}
\frameSubsection{}{}

\begin{frame}{Backtraces}{When backtraces are not needed}
  \begin{columns}[c]
    \begin{column}{0.45\textwidth}
      \minipage[c][0.66\textheight][s]{\columnwidth}
      \begin{itemize}
        \item<1-> What if the backtrace for an exception is never needed?
        \item<2-> It's possible to raise the exception explicitly with \funcname{raise\_notrace} to make raising as cheap as possible
        \item<3-> An example of use in the \funcname{dynlink} library of the OCaml compiler
      \end{itemize}
      \vfill
      \onslide<3->{
      \listocaml[firstline=205,lastline=209,highlightlines=207]{../ocaml/otherlibs/dynlink/dynlink\_compilerlibs/misc.ml}{otherlibs/dynlink/dynlink\_compilerlibs/misc.ml:205}
      }
      \endminipage
    \end{column}
    \begin{column}{0.55\textwidth}
    \only<4>{
      \mintcmm[highlightlines=3]{../src/notrace/camlNotrace\_\_fun_347.cmm}
      \listcmm{../src/notrace/camlNotrace\_\_all\_somes\_327.cmm}{all\_somes}
    }
    \onslide*<5->{
      \listobjdump[highlightlines={6-9}]{../src/notrace/camlNotrace\_\_fun_347.objdump}{anonymous function from \funcname{all\_somes}}
    }
    \end{column}
  \end{columns}
\end{frame}

\begin{frame}{Backtraces}{Summary table of the multiple kinds of raise}
  \begin{itemize}
    \item When debugging information is enabled and if the backtrace of an exception isn't needed, it's possible to raise with \funcname{raise\_notrace} for performance
    \item When debugging information is \emph{not} enabled, the compiler optimises \funcname{raise} calls to inline assembly (same effect as using \funcname{raise\_notrace})
  \end{itemize}
  \bigskip
  \centering
  \begin{tabular}{| l | c | c | c | c |}
    \hline
    \parbox{10em}{Debug information \\ (\funcarg{-g} flag)}  & \multicolumn{2}{|c|}{Disabled}                                     & \multicolumn{2}{|c|}{Enabled} \\
    \hline
    Raise kind                                               & \funcname{raise} & \funcname{raise\_notrace}                       & \funcname{raise} & \funcname{raise\_notrace} \\
    \hline
    Compilation                                              & \parbox{8em}{inline assembly\\ (optimization)} & inline assembly   & \funcname{caml\_raise\_exn} & inline assembly \\
    \hline
  \end{tabular}
\end{frame}


%
% Takeaway #5
%
\subsection*{Takeaway \#5}
\frameSubsectionTakeaway{}
\againframe<5>{takeaway}
}%
    \end{column}
  \end{columns}
\end{frame}

\newcommand\listStashBacktrace[1][]{
  \listc[firstline=91,lastline=120,#1]{../ocaml/runtime/backtrace\_nat.c}{runtime/backtrace\_nat.c:91}}

\begin{frame}{Backtraces}{Introducing \funcname{caml\_stash\_backtrace}}
  \begin{columns}[c]
    \begin{column}{0.5\textwidth}
      \minipage[c][0.66\textheight][s]{\columnwidth}
      \begin{itemize}
        \item<1-> A C function provided by the runtime (specific to native code)
        \item<2-> It scans the stack, between the stack pointer at the raising point up to the next trap, in order to collect the names of the detected function frames
          \onslide*<2>{
            \begin{itemize}
              \item And more information, like line number, column number...
            \end{itemize}
          }
        \item<3-> It uses the same technique and information as the Garbage Collector (GC) uses to scan the values live on the stack
          \onslide*<4>{
            \begin{itemize}
              \item \typename{frame\_descr} are at the core of stack unwinding
            \end{itemize}
          }
      \end{itemize}
      \vfill
      \mintocaml[firstline=9,lastline=13,highlightlines=12]{../src/backtrace/backtrace.ml}
      \endminipage
    \end{column}
    \begin{column}{0.5\textwidth}
      \onslide*<1>{\listStashBacktrace[highlightlines=91]{}}
      \onslide*<2>{\listStashBacktrace[highlightlines={91,117-118}]{}}
      \onslide*<3->{\listStashBacktrace[highlightlines={94,106,109-110}]{}}
    \end{column}
  \end{columns}
\end{frame}

\newcommand\listFrameDescr[1][]{
  \listocaml[firstline=111,lastline=124,#1]{../ocaml/asmcomp/emitaux.ml}{asmcomp/emitaux.ml:111}}
\newcommand\listDebugInfo[1][]{
  \listocaml[firstline=38,lastline=49,#1]{../ocaml/lambda/debuginfo.mli}{lambda/debuginfo.mli:38}}

\begin{frame}{Backtraces}{\typename{frame\_descr} and \typename{frame\_debuginfo} in a nutshell}
  \begin{columns}[c]
    \begin{column}{0.5\textwidth}
      \begin{itemize}
        \item<1-> For every return address in an OCaml program, there is a associated \typename{frame\_descr}
        \item<2-> A \typename{frame\_descr} specifies
        \begin{itemize}
          \item<2-> The size of the function stack frame
          \item<3-> Where live values are on the stack (so that the GC can mark them as live and prevent from being swept)
        \end{itemize}
        \item<4-> When compiled with debug information, \typename{frame\_descr} will be linked to a \typename{frame\_debuginfo}
        \begin{itemize}
          \item<5-> Contains information about the position in the source code (file name, line and column number)
        \end{itemize}
      \end{itemize}
    \end{column}
    \begin{column}{0.5\textwidth}
        \onslide*<1>{\listFrameDescr[highlightlines={118-122}]{}}
        \onslide*<2>{\listFrameDescr[highlightlines={120}]{}}
        \onslide*<3>{\listFrameDescr[highlightlines={121}]{}}
        \onslide*<4->{\listFrameDescr[highlightlines={122}]{}}
        \medskip
        \onslide*<1-3>{\listDebugInfo{}}
        \onslide*<4>{\listDebugInfo[highlightlines={38,49}]{}}
        \onslide*<5>{\listDebugInfo[highlightlines={39-46}]{}}
    \end{column}
  \end{columns}
\end{frame}

\begin{frame}{Backtraces}{Scanning the stack with \typename{frame\_descr}}
  \begin{columns}[c]
    \begin{column}{0.5\textwidth}
      \begin{itemize}
        \item Given the return address of the code that raises the exception by calling \funcname{caml\_raise\_exn} (\funcarg{pc} argument of \funcname{caml\_stash\_backtrace})
        \item And the position of the stack pointer (\funcarg{sp} argument of \funcname{caml\_stash\_backtrace})
        \item The frame size given by \typename{frame\_descr}.\funcarg{frame\_size}, allows to find the end of the previous frame
        \item And the return address of the caller of this function
        \bigskip
        \onslide<3->{
        \item Note that traps (here the reddish \funcname{ExnA}, \funcname{ExnB} and \funcname{ExnC} blocks) belong to the frame that installed them, and so the frame size changes when traps are removed
        }
      \end{itemize}
    \end{column}
    \begin{column}{0.5\textwidth}
      \raggedleft
      \onslide*<1>{\providecommand\step{2}%
% Backtraces
%
\subsection{One advantage of raising with debugging information}
\frameSubsection{}{}

\begin{frame}{Backtraces}{Debugging information \& source code}
  \begin{columns}[c]
    \begin{column}{0.5\textwidth}
      \begin{itemize}
        \item Raising an exception is fun and games, but how do we know where the exception originated? $\rightarrow$ With the backtrace associated with the exception!
        \item In order to get a backtrace from the point where an exception was raised, it's necessary to compile with debugging information enabled (\funcarg{-g} flag for the compiler)
        \item Same example as in the `Nested handlers' section
      \end{itemize}
    \end{column}
    \begin{column}{0.5\textwidth}
      \listocaml[firstline=9,lastline=34]{../src/backtrace/backtrace.ml}{backtrace/backtrace.ml}
    \end{column}
  \end{columns}
\end{frame}

\begin{frame}{Backtraces}{CMM and disassembly of \funcname{definitely\_raise}}
  \begin{columns}[c]
    \begin{column}{0.5\textwidth}
      \minipage[c][0.66\textheight][s]{\columnwidth}
      \begin{itemize}
        \item<1-> An effect of compiling with debugging information (\funcarg{-g}): the CMM no longer uses \funcname{raise\_notrace} to raise the exception but \funcname{raise} instead
        \item<2-> Disassembly shows that CMM \funcname{raise} is actually a call to \funcname{caml\_raise\_exn}, a function in assembler provided by the runtime
      \end{itemize}
      \vfill
      \mintocaml[firstline=9,lastline=13,highlightlines=12]{../src/backtrace/backtrace.ml}
      \endminipage
    \end{column}
    \begin{column}{0.5\textwidth}
      \onslide*<1>{
        \mintcmm[highlightlines=5]{../src/backtrace/camlBacktrace\_\_definitely\_raise\_347.cmm}
      }
      \onslide*<2>{
        \mintobjdump[firstline=2,highlightlines=14]{../src/backtrace/camlBacktrace\_\_definitely\_raise\_347.objdump}
      }
    \end{column}
  \end{columns}
\end{frame}

\begin{frame}{Backtraces}{Introducing \funcname{caml\_raise\_exn}}
  \begin{columns}[c]
    \begin{column}{0.5\textwidth}
      \minipage[c][0.66\textheight][s]{\columnwidth}
      \onslide*<1>{
        \begin{itemize}
          \item Raising the exception is performed using \funcname{caml\_raise\_exn} which is
            \begin{itemize}
              \item An assembly function provided by the runtime
              \item Architecture specific
            \end{itemize}
          \item Let's see what it does differently from \funcname{raise\_notrace}\dots
          \item We are about to raise \typename{ExnC} with \funcname{caml\_raise\_exn} from \funcname{definitely\_raise}
        \end{itemize}
      }
      \onslide*<2>{
        \mintobjdump[firstline=2,highlightlines=14]{../src/backtrace/camlBacktrace\_\_definitely\_raise\_347.objdump}
      }
      \vfill
      \mintocaml[firstline=9,lastline=13,highlightlines=12]{../src/backtrace/backtrace.ml}
      \endminipage
    \end{column}
    \begin{column}{0.5\textwidth}
      \centering
      \providecommand\step{1}\input{diagrams/backtrace/backtrace.tex}
    \end{column}
  \end{columns}
\end{frame}

\begin{frame}{Backtraces}{But before that, we need more fields from \typename{caml\_domain\_state}}
  \begin{columns}
    \begin{column}{0.5\textwidth}
      \begin{itemize}
        \item Remember \typename{caml\_domain\_state} the per-domain runtime state?
        \item Some other members are of interest to us now\dots
        \item \localname{backtrace\_active} is used to know if the runtime should collect backtraces
        \item \localname{backtrace\_active} can be set by
          \begin{itemize}
            \item The \funcarg{b} flag in the \funcname{OCAMLRUNPARAM} environment variable
            \item \mintinline{ocaml}{Printexc.record_backtrace : bool -> unit} \footnotemark
          \end{itemize}
      \end{itemize}
    \end{column}
    \begin{column}{0.5\textwidth}
      \begin{listing}
        \mintc[highlightlines={9-11}]{src/ocaml/domain\_state\_backtrace.h}
        \caption{runtime/caml/domain\_state.h}
      \end{listing}
    \end{column}
  \end{columns}
  \footnotetext[1]{https://v2.ocaml.org/api/Printexc.html}
\end{frame}

\newcommand\listCamlRaiseExn[1][]{
  \listasm[firstline=702,lastline=725,#1]{../ocaml/runtime/amd64.S}{runtime/amd64.S:702}}

\begin{frame}{Backtraces}{Walk through \funcname{caml\_raise\_exn}}
  \begin{columns}[T]
    \begin{column}{0.5\textwidth}
      \begin{itemize}
        \item<1-> First checks whether exceptions backtrace should be recorded
        \item<1-> By checking the \funcarg{domain\_state->backtrace\_active} value
        \item<2-> If not, acts as \funcname{raise\_notrace} with \funcname{RESTORE\_EXN\_HANDLER\_OCAML}
          \begin{itemize}
            \item Set \regname{RSP} to \localname{domain\_state->exn\_handler} value
            \item Restore \localname{domain\_state->exn\_handler} from trap
            \item Jump with \funcname{ret} (same effect as using \funcname{pop} and \funcname{jmp})
          \end{itemize}
        \item<3-> Otherwise, switches to the C stack to be able to execute C code
        \item<4-> Calls \funcname{caml\_stash\_backtrace}, a C function provided by the runtime\ignorespaces
          \onslide<6->{, with
            \begin{itemize}
              \item \funcarg{exn}: exception value
              \item \funcarg{pc}: code address of the raising point
              \item \funcarg{sp}: stack address of the raising function
              \item \funcarg{trapsp}: stack address of the next handler
            \end{itemize}
          }
      \end{itemize}
      \bigskip
      \mintocaml[firstline=9,lastline=13,highlightlines=12]{../src/backtrace/backtrace.ml}
    \end{column}
    \begin{column}{0.5\textwidth}
      \raggedleft
      \onslide*<1,3-4>{
        \mintc[firstline=190,lastline=192]{../ocaml/runtime/amd64.S}
        \mintc[firstline=234,lastline=237]{../ocaml/runtime/amd64.S}
      }
      \onslide*<2>{
        \mintc[firstline=190,lastline=192,highlightlines={191-192}]{../ocaml/runtime/amd64.S}
        \mintc[firstline=234,lastline=237,highlightlines={235-237}]{../ocaml/runtime/amd64.S}
      }
      \onslide*<1>{\listCamlRaiseExn[highlightlines=705]{}}%
      \onslide*<2>{\listCamlRaiseExn[highlightlines={707-708}]{}}%
      \onslide*<3>{\listCamlRaiseExn[highlightlines={712-714}]{}}%
      \onslide*<4>{\listCamlRaiseExn[highlightlines={715-720}]{}}%
      \onslide*<5>{\providecommand\step{1}\input{diagrams/backtrace/backtrace.tex}}%
      \onslide*<6>{\providecommand\step{2}\input{diagrams/backtrace/backtrace.tex}}%
    \end{column}
  \end{columns}
\end{frame}

\newcommand\listStashBacktrace[1][]{
  \listc[firstline=91,lastline=120,#1]{../ocaml/runtime/backtrace\_nat.c}{runtime/backtrace\_nat.c:91}}

\begin{frame}{Backtraces}{Introducing \funcname{caml\_stash\_backtrace}}
  \begin{columns}[c]
    \begin{column}{0.5\textwidth}
      \minipage[c][0.66\textheight][s]{\columnwidth}
      \begin{itemize}
        \item<1-> A C function provided by the runtime (specific to native code)
        \item<2-> It scans the stack, between the stack pointer at the raising point up to the next trap, in order to collect the names of the detected function frames
          \onslide*<2>{
            \begin{itemize}
              \item And more information, like line number, column number...
            \end{itemize}
          }
        \item<3-> It uses the same technique and information as the Garbage Collector (GC) uses to scan the values live on the stack
          \onslide*<4>{
            \begin{itemize}
              \item \typename{frame\_descr} are at the core of stack unwinding
            \end{itemize}
          }
      \end{itemize}
      \vfill
      \mintocaml[firstline=9,lastline=13,highlightlines=12]{../src/backtrace/backtrace.ml}
      \endminipage
    \end{column}
    \begin{column}{0.5\textwidth}
      \onslide*<1>{\listStashBacktrace[highlightlines=91]{}}
      \onslide*<2>{\listStashBacktrace[highlightlines={91,117-118}]{}}
      \onslide*<3->{\listStashBacktrace[highlightlines={94,106,109-110}]{}}
    \end{column}
  \end{columns}
\end{frame}

\newcommand\listFrameDescr[1][]{
  \listocaml[firstline=111,lastline=124,#1]{../ocaml/asmcomp/emitaux.ml}{asmcomp/emitaux.ml:111}}
\newcommand\listDebugInfo[1][]{
  \listocaml[firstline=38,lastline=49,#1]{../ocaml/lambda/debuginfo.mli}{lambda/debuginfo.mli:38}}

\begin{frame}{Backtraces}{\typename{frame\_descr} and \typename{frame\_debuginfo} in a nutshell}
  \begin{columns}[c]
    \begin{column}{0.5\textwidth}
      \begin{itemize}
        \item<1-> For every return address in an OCaml program, there is a associated \typename{frame\_descr}
        \item<2-> A \typename{frame\_descr} specifies
        \begin{itemize}
          \item<2-> The size of the function stack frame
          \item<3-> Where live values are on the stack (so that the GC can mark them as live and prevent from being swept)
        \end{itemize}
        \item<4-> When compiled with debug information, \typename{frame\_descr} will be linked to a \typename{frame\_debuginfo}
        \begin{itemize}
          \item<5-> Contains information about the position in the source code (file name, line and column number)
        \end{itemize}
      \end{itemize}
    \end{column}
    \begin{column}{0.5\textwidth}
        \onslide*<1>{\listFrameDescr[highlightlines={118-122}]{}}
        \onslide*<2>{\listFrameDescr[highlightlines={120}]{}}
        \onslide*<3>{\listFrameDescr[highlightlines={121}]{}}
        \onslide*<4->{\listFrameDescr[highlightlines={122}]{}}
        \medskip
        \onslide*<1-3>{\listDebugInfo{}}
        \onslide*<4>{\listDebugInfo[highlightlines={38,49}]{}}
        \onslide*<5>{\listDebugInfo[highlightlines={39-46}]{}}
    \end{column}
  \end{columns}
\end{frame}

\begin{frame}{Backtraces}{Scanning the stack with \typename{frame\_descr}}
  \begin{columns}[c]
    \begin{column}{0.5\textwidth}
      \begin{itemize}
        \item Given the return address of the code that raises the exception by calling \funcname{caml\_raise\_exn} (\funcarg{pc} argument of \funcname{caml\_stash\_backtrace})
        \item And the position of the stack pointer (\funcarg{sp} argument of \funcname{caml\_stash\_backtrace})
        \item The frame size given by \typename{frame\_descr}.\funcarg{frame\_size}, allows to find the end of the previous frame
        \item And the return address of the caller of this function
        \bigskip
        \onslide<3->{
        \item Note that traps (here the reddish \funcname{ExnA}, \funcname{ExnB} and \funcname{ExnC} blocks) belong to the frame that installed them, and so the frame size changes when traps are removed
        }
      \end{itemize}
    \end{column}
    \begin{column}{0.5\textwidth}
      \raggedleft
      \onslide*<1>{\providecommand\step{2}\input{diagrams/backtrace/backtrace.tex}}%
      \onslide*<2>{\providecommand\step{1}\input{diagrams/backtrace/scanstack.tex}}%
      \onslide*<3>{\providecommand\step{2}\input{diagrams/backtrace/scanstack.tex}}%
      \onslide*<4>{\providecommand\step{3}\input{diagrams/backtrace/scanstack.tex}}%
      \onslide*<5>{\providecommand\step{4}\input{diagrams/backtrace/scanstack.tex}}%
      \onslide*<6>{\providecommand\step{5}\input{diagrams/backtrace/scanstack.tex}}%
    \end{column}
  \end{columns}
\end{frame}

% Duplicate of previous-previous slide
\begin{frame}{Backtraces}{Continuing on \funcname{caml\_stash\_backtrace}}
  \begin{columns}[c]
    \begin{column}{0.5\textwidth}
      \minipage[c][0.75\textheight][s]{\columnwidth}
      \begin{itemize}
          \color{gray}
        \item A C function provided by the runtime (specific to native code)
        \item It scans the stack, between the stack pointer at the raising point up to the next trap, in order to collect the names of the detected function frames
        \item It uses the same technique and information as the Garbage Collector (GC) uses to scan the values live on the stack
      \end{itemize}
      \begin{itemize}
        \item For each function frame detected, store a pointer to the \typename{frame\_descr} in the \localname{domain\_state->backtrace\_buffer} array, effectively constructing the backtrace
      \end{itemize}
      \onslide<2->{
        \begin{itemize}
            \smallskip
          \item Constructed backtrace:
            \funcname{definitely\_raise},
            \onslide<3->{\funcname{probably\_raise}}
        \end{itemize}
      }
      \vfill
      \mintocaml[firstline=9,lastline=13,highlightlines=12]{../src/backtrace/backtrace.ml}
      \endminipage
    \end{column}
    \begin{column}{0.5\textwidth}
      \raggedleft
      \onslide*<1>{\listStashBacktrace[highlightlines={114-115}]{}}
      \onslide*<2>{\providecommand\step{3}\input{diagrams/backtrace/backtrace.tex}}%
      \onslide*<3>{\providecommand\step{4}\input{diagrams/backtrace/backtrace.tex}}%
    \end{column}
  \end{columns}
\end{frame}

\begin{frame}{Backtraces}{Back to \funcname{caml\_raise\_exn}}
  \begin{columns}[c]
    \begin{column}{0.5\textwidth}
      \minipage[c][0.66\textheight][s]{\columnwidth}
      \begin{itemize}\color{gray}
        \item Checks wheteher exceptions backtrace should be recorded, by checking the \funcarg{domain\_state->backtrace\_active} value
        \item Switches to the C stack to be able to execute C code
        \item Calls \funcname{caml\_stash\_backtrace}, to collect the backtrace up to the next trap
      \end{itemize}
      \onslide*<2->{
        \begin{itemize}
          \item<2-> Executes the exception handler in \funcname{probably\_raise} with \funcname{RESTORE\_EXN\_HANDLER\_OCAML}
            \begin{itemize}
              \item Set \regname{RSP} to \localname{domain\_state->exn\_handler} value
              \item Restore \localname{domain\_state->exn\_handler} from trap
              \item Jump to the handler code with \funcname{ret}
            \end{itemize}
        \end{itemize}
      }
      \vfill
      \onslide*<1>{\mintocaml[firstline=9,lastline=13,highlightlines=12]{../src/backtrace/backtrace.ml}}
      \onslide*<2->{\mintocaml[firstline=15,lastline=21,highlightlines={19-21}]{../src/backtrace/backtrace.ml}}
      \endminipage
    \end{column}
    \begin{column}{0.5\textwidth}
      \centering
      \onslide*<1>{
        \mintc[firstline=190,lastline=192]{../ocaml/runtime/amd64.S}
        \mintc[firstline=234,lastline=237]{../ocaml/runtime/amd64.S}
      }
      \onslide*<2>{
        \mintc[firstline=190,lastline=192,highlightlines={191-192}]{../ocaml/runtime/amd64.S}
        \mintc[firstline=234,lastline=237,highlightlines={235-237}]{../ocaml/runtime/amd64.S}
      }
      \onslide*<1>{\listCamlRaiseExn[highlightlines=720]{}}
      \onslide*<2>{\listCamlRaiseExn[highlightlines=722]{}}
      \onslide*<3->{\providecommand\step{5}\input{diagrams/backtrace/backtrace.tex}}
    \end{column}
  \end{columns}
\end{frame}

\begin{frame}{Backtraces}{\funcname{caml\_probably\_raise}'s handler doesn't catch \typename{ExnC}}
  \begin{columns}[c]
    \begin{column}{0.45\textwidth}
      \minipage[c][0.75\textheight][s]{\columnwidth}
      \begin{itemize}
        \item<1-> \funcname{match \dots with}\footnotemark pattern matching can also match exceptions
        \item<1-> The CMM of \funcname{probably\_raise} shows that this \funcname{match \dots with} is implemented with a pattern matching and a \funcname{try \dots with}
        \item<1-> But this one only matches on \typename{ExnA} exception
        \item<2-> Since the pattern matching fails to catch the exception, \funcname{reraise} is called with our current \typename{ExnC} exception
        \item<3-> Disassembly shows that \funcname{reraise} is actually a call to \funcname{caml\_reraise\_exn}, which is also an assembly function provided by the runtime
      \end{itemize}
      \vfill
      \mintocaml[firstline=15,lastline=21,highlightlines={18-21}]{../src/backtrace/backtrace.ml}
      \endminipage
    \end{column}
    \begin{column}{0.55\textwidth}
      \centering
      \onslide*<1>{\mintcmm[highlightlines={10-18}]{../src/backtrace/camlBacktrace\_\_probably\_raise\_351.cmm}}
      \onslide*<2>{\listcmm[highlightlines=18]{../src/backtrace/camlBacktrace\_\_probably\_raise_351.cmm}{CMM of probably\_raise}}
      \onslide*<3>{\mintobjdump[firstline=5,lastline=35,highlightlines=31]{../src/backtrace/camlBacktrace\_\_probably\_raise_351.objdump}}
    \end{column}
  \end{columns}
  \only<1>{\footnotetext[1]{https://blog.janestreet.com/pattern-matching-and-exception-handling-unite/}}
\end{frame}

\newcommand\listCamlReraiseExn[1][]{
  \listasm[firstline=727,lastline=734,#1]{../ocaml/runtime/amd64.S}{runtime/amd64.S:727}}

\begin{frame}{Backtraces}{Introducing \funcname{caml\_reraise\_exn}}
  \begin{columns}[c]
    \begin{column}{0.5\textwidth}
      \minipage[c][0.66\textheight][s]{\columnwidth}
      \begin{itemize}
        \item<1-> The exception have to be reraised to the next handler
        \item<2-> First, checks if the backtrace should be collected with \funcarg{domain\_state->backtrace\_active}
        \only<3>{
        \begin{itemize}
            \item If not, behaves like a \funcname{raise\_notrace} with \funcname{RESTORE\_EXN\_HANDLER\_OCAML}
        \end{itemize}
        }
        \item<4-> Jumps into \funcname{caml\_raise\_exn}
        \item<5-> Calls \funcname{caml\_stash\_backtrace} again, to scan the next part of the stack: from the trap just pop-ed to the next trap
      \end{itemize}
      \vfill
      \mintocaml[firstline=15,lastline=21,highlightlines={18-21}]{../src/backtrace/backtrace.ml}
      \endminipage
    \end{column}
    \begin{column}{0.5\textwidth}
      \raggedleft
      \onslide*<1>{\listCamlReraiseExn{}}
      \onslide*<2>{\listCamlReraiseExn[highlightlines=729]{}}
      \onslide*<3>{
        \mintc[firstline=190,lastline=192,highlightlines={191-192}]{../ocaml/runtime/amd64.S}
        \mintc[firstline=234,lastline=237,highlightlines={235-237}]{../ocaml/runtime/amd64.S}
        \medskip
        \listCamlReraiseExn[highlightlines=731]{}
      }
      \onslide*<4-5>{
        % TODO refactor with \listCamlReraiseExn
        \mintasm[firstline=727,lastline=734,highlightlines=730]{../ocaml/runtime/amd64.S}
        \medskip
      }
      \onslide*<4>{\listCamlRaiseExn[highlightlines=711]{}}
      \onslide*<5>{\listCamlRaiseExn[highlightlines=720]{}}
    \end{column}
  \end{columns}
\end{frame}

\begin{frame}{Backtraces}{Backtrace collection continues}
  \begin{columns}[c]
    \begin{column}{0.55\textwidth}
      \minipage[c][0.75\textheight][s]{\columnwidth}
      \begin{itemize}
        \item<1-> \funcname{caml\_stash\_backtrace} scans the older part of the stack, up to the next trap
        \item<6-> Controls jumps to the nested exception handler in \funcname{maybe\_raise}, which matches only on \typename{ExnB}
        \item<7-> \funcname{caml\_reraise\_exn} is called again, backtrace collection resumes with \funcname{caml\_stash\_backtrace}
      \end{itemize}
      \medskip
      \begin{itemize}
        \item<2-> Constructed backtrace:
        \begin{itemize}
          \item <2-> \funcname{definitely\_raise}
          \item <2-> \funcname{probably\_raise}
          \item <3-> \funcname{probably\_raise} (re-raise)
          \item <4-> \funcname{maybe\_raise}
          \item <5-> \funcname{main}
          \item <8-> \funcname{main} (re-raise)
        \end{itemize}
      \end{itemize}
      \vfill
      \onslide*<1-5>{\mintocaml[firstline=15,lastline=21]{../src/backtrace/backtrace.ml}}
      \onslide*<6>{\mintocaml[firstline=28,lastline=34,highlightlines=31]{../src/backtrace/backtrace.ml}}
      \onslide*<7->{\mintocaml[firstline=28,lastline=34,highlightlines=33]{../src/backtrace/backtrace.ml}}
      \endminipage
    \end{column}
    \begin{column}{0.45\textwidth}
      \raggedleft
      \onslide*<1>{\providecommand\step{6}\input{diagrams/backtrace/backtrace.tex}}%
      \onslide*<2>{\providecommand\step{6}\input{diagrams/backtrace/backtrace.tex}}%
      \onslide*<3>{\providecommand\step{7}\input{diagrams/backtrace/backtrace.tex}}%
      \onslide*<4>{\providecommand\step{8}\input{diagrams/backtrace/backtrace.tex}}%
      \onslide*<5>{\providecommand\step{9}\input{diagrams/backtrace/backtrace.tex}}%
      \onslide*<6>{\providecommand\step{10}\input{diagrams/backtrace/backtrace.tex}}%
      \onslide*<7>{\providecommand\step{11}\input{diagrams/backtrace/backtrace.tex}}%
      \onslide*<8>{\providecommand\step{12}\input{diagrams/backtrace/backtrace.tex}}%
      \onslide*<9>{\providecommand\step{13}\input{diagrams/backtrace/backtrace.tex}}%
    \end{column}
  \end{columns}
\end{frame}

\begin{frame}{Backtraces}{Printing the backtrace}
  \begin{columns}[c]
    \begin{column}{0.45\textwidth}
      \minipage[c][0.66\textheight][s]{\columnwidth}
      \begin{itemize}
        \item<1-> The exception backtrace can now be printed to \funcarg{stdout} with \funcname{Printexc.print_backtrace}
        \item<2-> First the exception backtrace is retrieved into an OCaml array with \funcname{get_raw_backtrace }
        \item<3-> Which is implemented as the \funcname{caml_get_exception_raw_backtrace} C function in the runtime
        \item<4-> And finally printed with \funcname{print_exception_backtrace}
      \end{itemize}
      \vfill
      \mintocaml[firstline=28,lastline=34,highlightlines=33]{../src/backtrace/backtrace.ml}
      \endminipage
    \end{column}
    \begin{column}{0.55\textwidth}
      \onslide*<1,2>{
        \mintocaml[firstline=100,lastline=101]{../ocaml/stdlib/printexc.ml}
      }
      \only<1>{
        \listocaml[firstline=170,lastline=172]{../ocaml/stdlib/printexc.ml}{stdlib/printexc.ml:170}
      }
      \only<2>{
        \listocaml[firstline=170,lastline=172,highlightlines=172]{../ocaml/stdlib/printexc.ml}{stdlib/printexc.ml:170}
      }
      \only<3>{
        \mintc[firstline=154,lastline=156]{../ocaml/runtime/backtrace.c}
        \listc[firstline=171,lastline=192,highlightlines={184-187}]{../ocaml/runtime/backtrace.c}{runtime/backtrace.c:154}
      }
      \only<4>{
        \listocaml[firstline=155,lastline=165]{../ocaml/stdlib/printexc.ml}{stdlib/printexc.ml:55}
      }
    \end{column}
  \end{columns}
\end{frame}


\subsection{The multiple kinds of raise}
\frameSubsection{}{}

\begin{frame}{Backtraces}{When backtraces are not needed}
  \begin{columns}[c]
    \begin{column}{0.45\textwidth}
      \minipage[c][0.66\textheight][s]{\columnwidth}
      \begin{itemize}
        \item<1-> What if the backtrace for an exception is never needed?
        \item<2-> It's possible to raise the exception explicitly with \funcname{raise\_notrace} to make raising as cheap as possible
        \item<3-> An example of use in the \funcname{dynlink} library of the OCaml compiler
      \end{itemize}
      \vfill
      \onslide<3->{
      \listocaml[firstline=205,lastline=209,highlightlines=207]{../ocaml/otherlibs/dynlink/dynlink\_compilerlibs/misc.ml}{otherlibs/dynlink/dynlink\_compilerlibs/misc.ml:205}
      }
      \endminipage
    \end{column}
    \begin{column}{0.55\textwidth}
    \only<4>{
      \mintcmm[highlightlines=3]{../src/notrace/camlNotrace\_\_fun_347.cmm}
      \listcmm{../src/notrace/camlNotrace\_\_all\_somes\_327.cmm}{all\_somes}
    }
    \onslide*<5->{
      \listobjdump[highlightlines={6-9}]{../src/notrace/camlNotrace\_\_fun_347.objdump}{anonymous function from \funcname{all\_somes}}
    }
    \end{column}
  \end{columns}
\end{frame}

\begin{frame}{Backtraces}{Summary table of the multiple kinds of raise}
  \begin{itemize}
    \item When debugging information is enabled and if the backtrace of an exception isn't needed, it's possible to raise with \funcname{raise\_notrace} for performance
    \item When debugging information is \emph{not} enabled, the compiler optimises \funcname{raise} calls to inline assembly (same effect as using \funcname{raise\_notrace})
  \end{itemize}
  \bigskip
  \centering
  \begin{tabular}{| l | c | c | c | c |}
    \hline
    \parbox{10em}{Debug information \\ (\funcarg{-g} flag)}  & \multicolumn{2}{|c|}{Disabled}                                     & \multicolumn{2}{|c|}{Enabled} \\
    \hline
    Raise kind                                               & \funcname{raise} & \funcname{raise\_notrace}                       & \funcname{raise} & \funcname{raise\_notrace} \\
    \hline
    Compilation                                              & \parbox{8em}{inline assembly\\ (optimization)} & inline assembly   & \funcname{caml\_raise\_exn} & inline assembly \\
    \hline
  \end{tabular}
\end{frame}


%
% Takeaway #5
%
\subsection*{Takeaway \#5}
\frameSubsectionTakeaway{}
\againframe<5>{takeaway}
}%
      \onslide*<2>{\providecommand\step{1}\input{diagrams/backtrace/scanstack.tex}}%
      \onslide*<3>{\providecommand\step{2}\input{diagrams/backtrace/scanstack.tex}}%
      \onslide*<4>{\providecommand\step{3}\input{diagrams/backtrace/scanstack.tex}}%
      \onslide*<5>{\providecommand\step{4}\input{diagrams/backtrace/scanstack.tex}}%
      \onslide*<6>{\providecommand\step{5}\input{diagrams/backtrace/scanstack.tex}}%
    \end{column}
  \end{columns}
\end{frame}

% Duplicate of previous-previous slide
\begin{frame}{Backtraces}{Continuing on \funcname{caml\_stash\_backtrace}}
  \begin{columns}[c]
    \begin{column}{0.5\textwidth}
      \minipage[c][0.75\textheight][s]{\columnwidth}
      \begin{itemize}
          \color{gray}
        \item A C function provided by the runtime (specific to native code)
        \item It scans the stack, between the stack pointer at the raising point up to the next trap, in order to collect the names of the detected function frames
        \item It uses the same technique and information as the Garbage Collector (GC) uses to scan the values live on the stack
      \end{itemize}
      \begin{itemize}
        \item For each function frame detected, store a pointer to the \typename{frame\_descr} in the \localname{domain\_state->backtrace\_buffer} array, effectively constructing the backtrace
      \end{itemize}
      \onslide<2->{
        \begin{itemize}
            \smallskip
          \item Constructed backtrace:
            \funcname{definitely\_raise},
            \onslide<3->{\funcname{probably\_raise}}
        \end{itemize}
      }
      \vfill
      \mintocaml[firstline=9,lastline=13,highlightlines=12]{../src/backtrace/backtrace.ml}
      \endminipage
    \end{column}
    \begin{column}{0.5\textwidth}
      \raggedleft
      \onslide*<1>{\listStashBacktrace[highlightlines={114-115}]{}}
      \onslide*<2>{\providecommand\step{3}%
% Backtraces
%
\subsection{One advantage of raising with debugging information}
\frameSubsection{}{}

\begin{frame}{Backtraces}{Debugging information \& source code}
  \begin{columns}[c]
    \begin{column}{0.5\textwidth}
      \begin{itemize}
        \item Raising an exception is fun and games, but how do we know where the exception originated? $\rightarrow$ With the backtrace associated with the exception!
        \item In order to get a backtrace from the point where an exception was raised, it's necessary to compile with debugging information enabled (\funcarg{-g} flag for the compiler)
        \item Same example as in the `Nested handlers' section
      \end{itemize}
    \end{column}
    \begin{column}{0.5\textwidth}
      \listocaml[firstline=9,lastline=34]{../src/backtrace/backtrace.ml}{backtrace/backtrace.ml}
    \end{column}
  \end{columns}
\end{frame}

\begin{frame}{Backtraces}{CMM and disassembly of \funcname{definitely\_raise}}
  \begin{columns}[c]
    \begin{column}{0.5\textwidth}
      \minipage[c][0.66\textheight][s]{\columnwidth}
      \begin{itemize}
        \item<1-> An effect of compiling with debugging information (\funcarg{-g}): the CMM no longer uses \funcname{raise\_notrace} to raise the exception but \funcname{raise} instead
        \item<2-> Disassembly shows that CMM \funcname{raise} is actually a call to \funcname{caml\_raise\_exn}, a function in assembler provided by the runtime
      \end{itemize}
      \vfill
      \mintocaml[firstline=9,lastline=13,highlightlines=12]{../src/backtrace/backtrace.ml}
      \endminipage
    \end{column}
    \begin{column}{0.5\textwidth}
      \onslide*<1>{
        \mintcmm[highlightlines=5]{../src/backtrace/camlBacktrace\_\_definitely\_raise\_347.cmm}
      }
      \onslide*<2>{
        \mintobjdump[firstline=2,highlightlines=14]{../src/backtrace/camlBacktrace\_\_definitely\_raise\_347.objdump}
      }
    \end{column}
  \end{columns}
\end{frame}

\begin{frame}{Backtraces}{Introducing \funcname{caml\_raise\_exn}}
  \begin{columns}[c]
    \begin{column}{0.5\textwidth}
      \minipage[c][0.66\textheight][s]{\columnwidth}
      \onslide*<1>{
        \begin{itemize}
          \item Raising the exception is performed using \funcname{caml\_raise\_exn} which is
            \begin{itemize}
              \item An assembly function provided by the runtime
              \item Architecture specific
            \end{itemize}
          \item Let's see what it does differently from \funcname{raise\_notrace}\dots
          \item We are about to raise \typename{ExnC} with \funcname{caml\_raise\_exn} from \funcname{definitely\_raise}
        \end{itemize}
      }
      \onslide*<2>{
        \mintobjdump[firstline=2,highlightlines=14]{../src/backtrace/camlBacktrace\_\_definitely\_raise\_347.objdump}
      }
      \vfill
      \mintocaml[firstline=9,lastline=13,highlightlines=12]{../src/backtrace/backtrace.ml}
      \endminipage
    \end{column}
    \begin{column}{0.5\textwidth}
      \centering
      \providecommand\step{1}\input{diagrams/backtrace/backtrace.tex}
    \end{column}
  \end{columns}
\end{frame}

\begin{frame}{Backtraces}{But before that, we need more fields from \typename{caml\_domain\_state}}
  \begin{columns}
    \begin{column}{0.5\textwidth}
      \begin{itemize}
        \item Remember \typename{caml\_domain\_state} the per-domain runtime state?
        \item Some other members are of interest to us now\dots
        \item \localname{backtrace\_active} is used to know if the runtime should collect backtraces
        \item \localname{backtrace\_active} can be set by
          \begin{itemize}
            \item The \funcarg{b} flag in the \funcname{OCAMLRUNPARAM} environment variable
            \item \mintinline{ocaml}{Printexc.record_backtrace : bool -> unit} \footnotemark
          \end{itemize}
      \end{itemize}
    \end{column}
    \begin{column}{0.5\textwidth}
      \begin{listing}
        \mintc[highlightlines={9-11}]{src/ocaml/domain\_state\_backtrace.h}
        \caption{runtime/caml/domain\_state.h}
      \end{listing}
    \end{column}
  \end{columns}
  \footnotetext[1]{https://v2.ocaml.org/api/Printexc.html}
\end{frame}

\newcommand\listCamlRaiseExn[1][]{
  \listasm[firstline=702,lastline=725,#1]{../ocaml/runtime/amd64.S}{runtime/amd64.S:702}}

\begin{frame}{Backtraces}{Walk through \funcname{caml\_raise\_exn}}
  \begin{columns}[T]
    \begin{column}{0.5\textwidth}
      \begin{itemize}
        \item<1-> First checks whether exceptions backtrace should be recorded
        \item<1-> By checking the \funcarg{domain\_state->backtrace\_active} value
        \item<2-> If not, acts as \funcname{raise\_notrace} with \funcname{RESTORE\_EXN\_HANDLER\_OCAML}
          \begin{itemize}
            \item Set \regname{RSP} to \localname{domain\_state->exn\_handler} value
            \item Restore \localname{domain\_state->exn\_handler} from trap
            \item Jump with \funcname{ret} (same effect as using \funcname{pop} and \funcname{jmp})
          \end{itemize}
        \item<3-> Otherwise, switches to the C stack to be able to execute C code
        \item<4-> Calls \funcname{caml\_stash\_backtrace}, a C function provided by the runtime\ignorespaces
          \onslide<6->{, with
            \begin{itemize}
              \item \funcarg{exn}: exception value
              \item \funcarg{pc}: code address of the raising point
              \item \funcarg{sp}: stack address of the raising function
              \item \funcarg{trapsp}: stack address of the next handler
            \end{itemize}
          }
      \end{itemize}
      \bigskip
      \mintocaml[firstline=9,lastline=13,highlightlines=12]{../src/backtrace/backtrace.ml}
    \end{column}
    \begin{column}{0.5\textwidth}
      \raggedleft
      \onslide*<1,3-4>{
        \mintc[firstline=190,lastline=192]{../ocaml/runtime/amd64.S}
        \mintc[firstline=234,lastline=237]{../ocaml/runtime/amd64.S}
      }
      \onslide*<2>{
        \mintc[firstline=190,lastline=192,highlightlines={191-192}]{../ocaml/runtime/amd64.S}
        \mintc[firstline=234,lastline=237,highlightlines={235-237}]{../ocaml/runtime/amd64.S}
      }
      \onslide*<1>{\listCamlRaiseExn[highlightlines=705]{}}%
      \onslide*<2>{\listCamlRaiseExn[highlightlines={707-708}]{}}%
      \onslide*<3>{\listCamlRaiseExn[highlightlines={712-714}]{}}%
      \onslide*<4>{\listCamlRaiseExn[highlightlines={715-720}]{}}%
      \onslide*<5>{\providecommand\step{1}\input{diagrams/backtrace/backtrace.tex}}%
      \onslide*<6>{\providecommand\step{2}\input{diagrams/backtrace/backtrace.tex}}%
    \end{column}
  \end{columns}
\end{frame}

\newcommand\listStashBacktrace[1][]{
  \listc[firstline=91,lastline=120,#1]{../ocaml/runtime/backtrace\_nat.c}{runtime/backtrace\_nat.c:91}}

\begin{frame}{Backtraces}{Introducing \funcname{caml\_stash\_backtrace}}
  \begin{columns}[c]
    \begin{column}{0.5\textwidth}
      \minipage[c][0.66\textheight][s]{\columnwidth}
      \begin{itemize}
        \item<1-> A C function provided by the runtime (specific to native code)
        \item<2-> It scans the stack, between the stack pointer at the raising point up to the next trap, in order to collect the names of the detected function frames
          \onslide*<2>{
            \begin{itemize}
              \item And more information, like line number, column number...
            \end{itemize}
          }
        \item<3-> It uses the same technique and information as the Garbage Collector (GC) uses to scan the values live on the stack
          \onslide*<4>{
            \begin{itemize}
              \item \typename{frame\_descr} are at the core of stack unwinding
            \end{itemize}
          }
      \end{itemize}
      \vfill
      \mintocaml[firstline=9,lastline=13,highlightlines=12]{../src/backtrace/backtrace.ml}
      \endminipage
    \end{column}
    \begin{column}{0.5\textwidth}
      \onslide*<1>{\listStashBacktrace[highlightlines=91]{}}
      \onslide*<2>{\listStashBacktrace[highlightlines={91,117-118}]{}}
      \onslide*<3->{\listStashBacktrace[highlightlines={94,106,109-110}]{}}
    \end{column}
  \end{columns}
\end{frame}

\newcommand\listFrameDescr[1][]{
  \listocaml[firstline=111,lastline=124,#1]{../ocaml/asmcomp/emitaux.ml}{asmcomp/emitaux.ml:111}}
\newcommand\listDebugInfo[1][]{
  \listocaml[firstline=38,lastline=49,#1]{../ocaml/lambda/debuginfo.mli}{lambda/debuginfo.mli:38}}

\begin{frame}{Backtraces}{\typename{frame\_descr} and \typename{frame\_debuginfo} in a nutshell}
  \begin{columns}[c]
    \begin{column}{0.5\textwidth}
      \begin{itemize}
        \item<1-> For every return address in an OCaml program, there is a associated \typename{frame\_descr}
        \item<2-> A \typename{frame\_descr} specifies
        \begin{itemize}
          \item<2-> The size of the function stack frame
          \item<3-> Where live values are on the stack (so that the GC can mark them as live and prevent from being swept)
        \end{itemize}
        \item<4-> When compiled with debug information, \typename{frame\_descr} will be linked to a \typename{frame\_debuginfo}
        \begin{itemize}
          \item<5-> Contains information about the position in the source code (file name, line and column number)
        \end{itemize}
      \end{itemize}
    \end{column}
    \begin{column}{0.5\textwidth}
        \onslide*<1>{\listFrameDescr[highlightlines={118-122}]{}}
        \onslide*<2>{\listFrameDescr[highlightlines={120}]{}}
        \onslide*<3>{\listFrameDescr[highlightlines={121}]{}}
        \onslide*<4->{\listFrameDescr[highlightlines={122}]{}}
        \medskip
        \onslide*<1-3>{\listDebugInfo{}}
        \onslide*<4>{\listDebugInfo[highlightlines={38,49}]{}}
        \onslide*<5>{\listDebugInfo[highlightlines={39-46}]{}}
    \end{column}
  \end{columns}
\end{frame}

\begin{frame}{Backtraces}{Scanning the stack with \typename{frame\_descr}}
  \begin{columns}[c]
    \begin{column}{0.5\textwidth}
      \begin{itemize}
        \item Given the return address of the code that raises the exception by calling \funcname{caml\_raise\_exn} (\funcarg{pc} argument of \funcname{caml\_stash\_backtrace})
        \item And the position of the stack pointer (\funcarg{sp} argument of \funcname{caml\_stash\_backtrace})
        \item The frame size given by \typename{frame\_descr}.\funcarg{frame\_size}, allows to find the end of the previous frame
        \item And the return address of the caller of this function
        \bigskip
        \onslide<3->{
        \item Note that traps (here the reddish \funcname{ExnA}, \funcname{ExnB} and \funcname{ExnC} blocks) belong to the frame that installed them, and so the frame size changes when traps are removed
        }
      \end{itemize}
    \end{column}
    \begin{column}{0.5\textwidth}
      \raggedleft
      \onslide*<1>{\providecommand\step{2}\input{diagrams/backtrace/backtrace.tex}}%
      \onslide*<2>{\providecommand\step{1}\input{diagrams/backtrace/scanstack.tex}}%
      \onslide*<3>{\providecommand\step{2}\input{diagrams/backtrace/scanstack.tex}}%
      \onslide*<4>{\providecommand\step{3}\input{diagrams/backtrace/scanstack.tex}}%
      \onslide*<5>{\providecommand\step{4}\input{diagrams/backtrace/scanstack.tex}}%
      \onslide*<6>{\providecommand\step{5}\input{diagrams/backtrace/scanstack.tex}}%
    \end{column}
  \end{columns}
\end{frame}

% Duplicate of previous-previous slide
\begin{frame}{Backtraces}{Continuing on \funcname{caml\_stash\_backtrace}}
  \begin{columns}[c]
    \begin{column}{0.5\textwidth}
      \minipage[c][0.75\textheight][s]{\columnwidth}
      \begin{itemize}
          \color{gray}
        \item A C function provided by the runtime (specific to native code)
        \item It scans the stack, between the stack pointer at the raising point up to the next trap, in order to collect the names of the detected function frames
        \item It uses the same technique and information as the Garbage Collector (GC) uses to scan the values live on the stack
      \end{itemize}
      \begin{itemize}
        \item For each function frame detected, store a pointer to the \typename{frame\_descr} in the \localname{domain\_state->backtrace\_buffer} array, effectively constructing the backtrace
      \end{itemize}
      \onslide<2->{
        \begin{itemize}
            \smallskip
          \item Constructed backtrace:
            \funcname{definitely\_raise},
            \onslide<3->{\funcname{probably\_raise}}
        \end{itemize}
      }
      \vfill
      \mintocaml[firstline=9,lastline=13,highlightlines=12]{../src/backtrace/backtrace.ml}
      \endminipage
    \end{column}
    \begin{column}{0.5\textwidth}
      \raggedleft
      \onslide*<1>{\listStashBacktrace[highlightlines={114-115}]{}}
      \onslide*<2>{\providecommand\step{3}\input{diagrams/backtrace/backtrace.tex}}%
      \onslide*<3>{\providecommand\step{4}\input{diagrams/backtrace/backtrace.tex}}%
    \end{column}
  \end{columns}
\end{frame}

\begin{frame}{Backtraces}{Back to \funcname{caml\_raise\_exn}}
  \begin{columns}[c]
    \begin{column}{0.5\textwidth}
      \minipage[c][0.66\textheight][s]{\columnwidth}
      \begin{itemize}\color{gray}
        \item Checks wheteher exceptions backtrace should be recorded, by checking the \funcarg{domain\_state->backtrace\_active} value
        \item Switches to the C stack to be able to execute C code
        \item Calls \funcname{caml\_stash\_backtrace}, to collect the backtrace up to the next trap
      \end{itemize}
      \onslide*<2->{
        \begin{itemize}
          \item<2-> Executes the exception handler in \funcname{probably\_raise} with \funcname{RESTORE\_EXN\_HANDLER\_OCAML}
            \begin{itemize}
              \item Set \regname{RSP} to \localname{domain\_state->exn\_handler} value
              \item Restore \localname{domain\_state->exn\_handler} from trap
              \item Jump to the handler code with \funcname{ret}
            \end{itemize}
        \end{itemize}
      }
      \vfill
      \onslide*<1>{\mintocaml[firstline=9,lastline=13,highlightlines=12]{../src/backtrace/backtrace.ml}}
      \onslide*<2->{\mintocaml[firstline=15,lastline=21,highlightlines={19-21}]{../src/backtrace/backtrace.ml}}
      \endminipage
    \end{column}
    \begin{column}{0.5\textwidth}
      \centering
      \onslide*<1>{
        \mintc[firstline=190,lastline=192]{../ocaml/runtime/amd64.S}
        \mintc[firstline=234,lastline=237]{../ocaml/runtime/amd64.S}
      }
      \onslide*<2>{
        \mintc[firstline=190,lastline=192,highlightlines={191-192}]{../ocaml/runtime/amd64.S}
        \mintc[firstline=234,lastline=237,highlightlines={235-237}]{../ocaml/runtime/amd64.S}
      }
      \onslide*<1>{\listCamlRaiseExn[highlightlines=720]{}}
      \onslide*<2>{\listCamlRaiseExn[highlightlines=722]{}}
      \onslide*<3->{\providecommand\step{5}\input{diagrams/backtrace/backtrace.tex}}
    \end{column}
  \end{columns}
\end{frame}

\begin{frame}{Backtraces}{\funcname{caml\_probably\_raise}'s handler doesn't catch \typename{ExnC}}
  \begin{columns}[c]
    \begin{column}{0.45\textwidth}
      \minipage[c][0.75\textheight][s]{\columnwidth}
      \begin{itemize}
        \item<1-> \funcname{match \dots with}\footnotemark pattern matching can also match exceptions
        \item<1-> The CMM of \funcname{probably\_raise} shows that this \funcname{match \dots with} is implemented with a pattern matching and a \funcname{try \dots with}
        \item<1-> But this one only matches on \typename{ExnA} exception
        \item<2-> Since the pattern matching fails to catch the exception, \funcname{reraise} is called with our current \typename{ExnC} exception
        \item<3-> Disassembly shows that \funcname{reraise} is actually a call to \funcname{caml\_reraise\_exn}, which is also an assembly function provided by the runtime
      \end{itemize}
      \vfill
      \mintocaml[firstline=15,lastline=21,highlightlines={18-21}]{../src/backtrace/backtrace.ml}
      \endminipage
    \end{column}
    \begin{column}{0.55\textwidth}
      \centering
      \onslide*<1>{\mintcmm[highlightlines={10-18}]{../src/backtrace/camlBacktrace\_\_probably\_raise\_351.cmm}}
      \onslide*<2>{\listcmm[highlightlines=18]{../src/backtrace/camlBacktrace\_\_probably\_raise_351.cmm}{CMM of probably\_raise}}
      \onslide*<3>{\mintobjdump[firstline=5,lastline=35,highlightlines=31]{../src/backtrace/camlBacktrace\_\_probably\_raise_351.objdump}}
    \end{column}
  \end{columns}
  \only<1>{\footnotetext[1]{https://blog.janestreet.com/pattern-matching-and-exception-handling-unite/}}
\end{frame}

\newcommand\listCamlReraiseExn[1][]{
  \listasm[firstline=727,lastline=734,#1]{../ocaml/runtime/amd64.S}{runtime/amd64.S:727}}

\begin{frame}{Backtraces}{Introducing \funcname{caml\_reraise\_exn}}
  \begin{columns}[c]
    \begin{column}{0.5\textwidth}
      \minipage[c][0.66\textheight][s]{\columnwidth}
      \begin{itemize}
        \item<1-> The exception have to be reraised to the next handler
        \item<2-> First, checks if the backtrace should be collected with \funcarg{domain\_state->backtrace\_active}
        \only<3>{
        \begin{itemize}
            \item If not, behaves like a \funcname{raise\_notrace} with \funcname{RESTORE\_EXN\_HANDLER\_OCAML}
        \end{itemize}
        }
        \item<4-> Jumps into \funcname{caml\_raise\_exn}
        \item<5-> Calls \funcname{caml\_stash\_backtrace} again, to scan the next part of the stack: from the trap just pop-ed to the next trap
      \end{itemize}
      \vfill
      \mintocaml[firstline=15,lastline=21,highlightlines={18-21}]{../src/backtrace/backtrace.ml}
      \endminipage
    \end{column}
    \begin{column}{0.5\textwidth}
      \raggedleft
      \onslide*<1>{\listCamlReraiseExn{}}
      \onslide*<2>{\listCamlReraiseExn[highlightlines=729]{}}
      \onslide*<3>{
        \mintc[firstline=190,lastline=192,highlightlines={191-192}]{../ocaml/runtime/amd64.S}
        \mintc[firstline=234,lastline=237,highlightlines={235-237}]{../ocaml/runtime/amd64.S}
        \medskip
        \listCamlReraiseExn[highlightlines=731]{}
      }
      \onslide*<4-5>{
        % TODO refactor with \listCamlReraiseExn
        \mintasm[firstline=727,lastline=734,highlightlines=730]{../ocaml/runtime/amd64.S}
        \medskip
      }
      \onslide*<4>{\listCamlRaiseExn[highlightlines=711]{}}
      \onslide*<5>{\listCamlRaiseExn[highlightlines=720]{}}
    \end{column}
  \end{columns}
\end{frame}

\begin{frame}{Backtraces}{Backtrace collection continues}
  \begin{columns}[c]
    \begin{column}{0.55\textwidth}
      \minipage[c][0.75\textheight][s]{\columnwidth}
      \begin{itemize}
        \item<1-> \funcname{caml\_stash\_backtrace} scans the older part of the stack, up to the next trap
        \item<6-> Controls jumps to the nested exception handler in \funcname{maybe\_raise}, which matches only on \typename{ExnB}
        \item<7-> \funcname{caml\_reraise\_exn} is called again, backtrace collection resumes with \funcname{caml\_stash\_backtrace}
      \end{itemize}
      \medskip
      \begin{itemize}
        \item<2-> Constructed backtrace:
        \begin{itemize}
          \item <2-> \funcname{definitely\_raise}
          \item <2-> \funcname{probably\_raise}
          \item <3-> \funcname{probably\_raise} (re-raise)
          \item <4-> \funcname{maybe\_raise}
          \item <5-> \funcname{main}
          \item <8-> \funcname{main} (re-raise)
        \end{itemize}
      \end{itemize}
      \vfill
      \onslide*<1-5>{\mintocaml[firstline=15,lastline=21]{../src/backtrace/backtrace.ml}}
      \onslide*<6>{\mintocaml[firstline=28,lastline=34,highlightlines=31]{../src/backtrace/backtrace.ml}}
      \onslide*<7->{\mintocaml[firstline=28,lastline=34,highlightlines=33]{../src/backtrace/backtrace.ml}}
      \endminipage
    \end{column}
    \begin{column}{0.45\textwidth}
      \raggedleft
      \onslide*<1>{\providecommand\step{6}\input{diagrams/backtrace/backtrace.tex}}%
      \onslide*<2>{\providecommand\step{6}\input{diagrams/backtrace/backtrace.tex}}%
      \onslide*<3>{\providecommand\step{7}\input{diagrams/backtrace/backtrace.tex}}%
      \onslide*<4>{\providecommand\step{8}\input{diagrams/backtrace/backtrace.tex}}%
      \onslide*<5>{\providecommand\step{9}\input{diagrams/backtrace/backtrace.tex}}%
      \onslide*<6>{\providecommand\step{10}\input{diagrams/backtrace/backtrace.tex}}%
      \onslide*<7>{\providecommand\step{11}\input{diagrams/backtrace/backtrace.tex}}%
      \onslide*<8>{\providecommand\step{12}\input{diagrams/backtrace/backtrace.tex}}%
      \onslide*<9>{\providecommand\step{13}\input{diagrams/backtrace/backtrace.tex}}%
    \end{column}
  \end{columns}
\end{frame}

\begin{frame}{Backtraces}{Printing the backtrace}
  \begin{columns}[c]
    \begin{column}{0.45\textwidth}
      \minipage[c][0.66\textheight][s]{\columnwidth}
      \begin{itemize}
        \item<1-> The exception backtrace can now be printed to \funcarg{stdout} with \funcname{Printexc.print_backtrace}
        \item<2-> First the exception backtrace is retrieved into an OCaml array with \funcname{get_raw_backtrace }
        \item<3-> Which is implemented as the \funcname{caml_get_exception_raw_backtrace} C function in the runtime
        \item<4-> And finally printed with \funcname{print_exception_backtrace}
      \end{itemize}
      \vfill
      \mintocaml[firstline=28,lastline=34,highlightlines=33]{../src/backtrace/backtrace.ml}
      \endminipage
    \end{column}
    \begin{column}{0.55\textwidth}
      \onslide*<1,2>{
        \mintocaml[firstline=100,lastline=101]{../ocaml/stdlib/printexc.ml}
      }
      \only<1>{
        \listocaml[firstline=170,lastline=172]{../ocaml/stdlib/printexc.ml}{stdlib/printexc.ml:170}
      }
      \only<2>{
        \listocaml[firstline=170,lastline=172,highlightlines=172]{../ocaml/stdlib/printexc.ml}{stdlib/printexc.ml:170}
      }
      \only<3>{
        \mintc[firstline=154,lastline=156]{../ocaml/runtime/backtrace.c}
        \listc[firstline=171,lastline=192,highlightlines={184-187}]{../ocaml/runtime/backtrace.c}{runtime/backtrace.c:154}
      }
      \only<4>{
        \listocaml[firstline=155,lastline=165]{../ocaml/stdlib/printexc.ml}{stdlib/printexc.ml:55}
      }
    \end{column}
  \end{columns}
\end{frame}


\subsection{The multiple kinds of raise}
\frameSubsection{}{}

\begin{frame}{Backtraces}{When backtraces are not needed}
  \begin{columns}[c]
    \begin{column}{0.45\textwidth}
      \minipage[c][0.66\textheight][s]{\columnwidth}
      \begin{itemize}
        \item<1-> What if the backtrace for an exception is never needed?
        \item<2-> It's possible to raise the exception explicitly with \funcname{raise\_notrace} to make raising as cheap as possible
        \item<3-> An example of use in the \funcname{dynlink} library of the OCaml compiler
      \end{itemize}
      \vfill
      \onslide<3->{
      \listocaml[firstline=205,lastline=209,highlightlines=207]{../ocaml/otherlibs/dynlink/dynlink\_compilerlibs/misc.ml}{otherlibs/dynlink/dynlink\_compilerlibs/misc.ml:205}
      }
      \endminipage
    \end{column}
    \begin{column}{0.55\textwidth}
    \only<4>{
      \mintcmm[highlightlines=3]{../src/notrace/camlNotrace\_\_fun_347.cmm}
      \listcmm{../src/notrace/camlNotrace\_\_all\_somes\_327.cmm}{all\_somes}
    }
    \onslide*<5->{
      \listobjdump[highlightlines={6-9}]{../src/notrace/camlNotrace\_\_fun_347.objdump}{anonymous function from \funcname{all\_somes}}
    }
    \end{column}
  \end{columns}
\end{frame}

\begin{frame}{Backtraces}{Summary table of the multiple kinds of raise}
  \begin{itemize}
    \item When debugging information is enabled and if the backtrace of an exception isn't needed, it's possible to raise with \funcname{raise\_notrace} for performance
    \item When debugging information is \emph{not} enabled, the compiler optimises \funcname{raise} calls to inline assembly (same effect as using \funcname{raise\_notrace})
  \end{itemize}
  \bigskip
  \centering
  \begin{tabular}{| l | c | c | c | c |}
    \hline
    \parbox{10em}{Debug information \\ (\funcarg{-g} flag)}  & \multicolumn{2}{|c|}{Disabled}                                     & \multicolumn{2}{|c|}{Enabled} \\
    \hline
    Raise kind                                               & \funcname{raise} & \funcname{raise\_notrace}                       & \funcname{raise} & \funcname{raise\_notrace} \\
    \hline
    Compilation                                              & \parbox{8em}{inline assembly\\ (optimization)} & inline assembly   & \funcname{caml\_raise\_exn} & inline assembly \\
    \hline
  \end{tabular}
\end{frame}


%
% Takeaway #5
%
\subsection*{Takeaway \#5}
\frameSubsectionTakeaway{}
\againframe<5>{takeaway}
}%
      \onslide*<3>{\providecommand\step{4}%
% Backtraces
%
\subsection{One advantage of raising with debugging information}
\frameSubsection{}{}

\begin{frame}{Backtraces}{Debugging information \& source code}
  \begin{columns}[c]
    \begin{column}{0.5\textwidth}
      \begin{itemize}
        \item Raising an exception is fun and games, but how do we know where the exception originated? $\rightarrow$ With the backtrace associated with the exception!
        \item In order to get a backtrace from the point where an exception was raised, it's necessary to compile with debugging information enabled (\funcarg{-g} flag for the compiler)
        \item Same example as in the `Nested handlers' section
      \end{itemize}
    \end{column}
    \begin{column}{0.5\textwidth}
      \listocaml[firstline=9,lastline=34]{../src/backtrace/backtrace.ml}{backtrace/backtrace.ml}
    \end{column}
  \end{columns}
\end{frame}

\begin{frame}{Backtraces}{CMM and disassembly of \funcname{definitely\_raise}}
  \begin{columns}[c]
    \begin{column}{0.5\textwidth}
      \minipage[c][0.66\textheight][s]{\columnwidth}
      \begin{itemize}
        \item<1-> An effect of compiling with debugging information (\funcarg{-g}): the CMM no longer uses \funcname{raise\_notrace} to raise the exception but \funcname{raise} instead
        \item<2-> Disassembly shows that CMM \funcname{raise} is actually a call to \funcname{caml\_raise\_exn}, a function in assembler provided by the runtime
      \end{itemize}
      \vfill
      \mintocaml[firstline=9,lastline=13,highlightlines=12]{../src/backtrace/backtrace.ml}
      \endminipage
    \end{column}
    \begin{column}{0.5\textwidth}
      \onslide*<1>{
        \mintcmm[highlightlines=5]{../src/backtrace/camlBacktrace\_\_definitely\_raise\_347.cmm}
      }
      \onslide*<2>{
        \mintobjdump[firstline=2,highlightlines=14]{../src/backtrace/camlBacktrace\_\_definitely\_raise\_347.objdump}
      }
    \end{column}
  \end{columns}
\end{frame}

\begin{frame}{Backtraces}{Introducing \funcname{caml\_raise\_exn}}
  \begin{columns}[c]
    \begin{column}{0.5\textwidth}
      \minipage[c][0.66\textheight][s]{\columnwidth}
      \onslide*<1>{
        \begin{itemize}
          \item Raising the exception is performed using \funcname{caml\_raise\_exn} which is
            \begin{itemize}
              \item An assembly function provided by the runtime
              \item Architecture specific
            \end{itemize}
          \item Let's see what it does differently from \funcname{raise\_notrace}\dots
          \item We are about to raise \typename{ExnC} with \funcname{caml\_raise\_exn} from \funcname{definitely\_raise}
        \end{itemize}
      }
      \onslide*<2>{
        \mintobjdump[firstline=2,highlightlines=14]{../src/backtrace/camlBacktrace\_\_definitely\_raise\_347.objdump}
      }
      \vfill
      \mintocaml[firstline=9,lastline=13,highlightlines=12]{../src/backtrace/backtrace.ml}
      \endminipage
    \end{column}
    \begin{column}{0.5\textwidth}
      \centering
      \providecommand\step{1}\input{diagrams/backtrace/backtrace.tex}
    \end{column}
  \end{columns}
\end{frame}

\begin{frame}{Backtraces}{But before that, we need more fields from \typename{caml\_domain\_state}}
  \begin{columns}
    \begin{column}{0.5\textwidth}
      \begin{itemize}
        \item Remember \typename{caml\_domain\_state} the per-domain runtime state?
        \item Some other members are of interest to us now\dots
        \item \localname{backtrace\_active} is used to know if the runtime should collect backtraces
        \item \localname{backtrace\_active} can be set by
          \begin{itemize}
            \item The \funcarg{b} flag in the \funcname{OCAMLRUNPARAM} environment variable
            \item \mintinline{ocaml}{Printexc.record_backtrace : bool -> unit} \footnotemark
          \end{itemize}
      \end{itemize}
    \end{column}
    \begin{column}{0.5\textwidth}
      \begin{listing}
        \mintc[highlightlines={9-11}]{src/ocaml/domain\_state\_backtrace.h}
        \caption{runtime/caml/domain\_state.h}
      \end{listing}
    \end{column}
  \end{columns}
  \footnotetext[1]{https://v2.ocaml.org/api/Printexc.html}
\end{frame}

\newcommand\listCamlRaiseExn[1][]{
  \listasm[firstline=702,lastline=725,#1]{../ocaml/runtime/amd64.S}{runtime/amd64.S:702}}

\begin{frame}{Backtraces}{Walk through \funcname{caml\_raise\_exn}}
  \begin{columns}[T]
    \begin{column}{0.5\textwidth}
      \begin{itemize}
        \item<1-> First checks whether exceptions backtrace should be recorded
        \item<1-> By checking the \funcarg{domain\_state->backtrace\_active} value
        \item<2-> If not, acts as \funcname{raise\_notrace} with \funcname{RESTORE\_EXN\_HANDLER\_OCAML}
          \begin{itemize}
            \item Set \regname{RSP} to \localname{domain\_state->exn\_handler} value
            \item Restore \localname{domain\_state->exn\_handler} from trap
            \item Jump with \funcname{ret} (same effect as using \funcname{pop} and \funcname{jmp})
          \end{itemize}
        \item<3-> Otherwise, switches to the C stack to be able to execute C code
        \item<4-> Calls \funcname{caml\_stash\_backtrace}, a C function provided by the runtime\ignorespaces
          \onslide<6->{, with
            \begin{itemize}
              \item \funcarg{exn}: exception value
              \item \funcarg{pc}: code address of the raising point
              \item \funcarg{sp}: stack address of the raising function
              \item \funcarg{trapsp}: stack address of the next handler
            \end{itemize}
          }
      \end{itemize}
      \bigskip
      \mintocaml[firstline=9,lastline=13,highlightlines=12]{../src/backtrace/backtrace.ml}
    \end{column}
    \begin{column}{0.5\textwidth}
      \raggedleft
      \onslide*<1,3-4>{
        \mintc[firstline=190,lastline=192]{../ocaml/runtime/amd64.S}
        \mintc[firstline=234,lastline=237]{../ocaml/runtime/amd64.S}
      }
      \onslide*<2>{
        \mintc[firstline=190,lastline=192,highlightlines={191-192}]{../ocaml/runtime/amd64.S}
        \mintc[firstline=234,lastline=237,highlightlines={235-237}]{../ocaml/runtime/amd64.S}
      }
      \onslide*<1>{\listCamlRaiseExn[highlightlines=705]{}}%
      \onslide*<2>{\listCamlRaiseExn[highlightlines={707-708}]{}}%
      \onslide*<3>{\listCamlRaiseExn[highlightlines={712-714}]{}}%
      \onslide*<4>{\listCamlRaiseExn[highlightlines={715-720}]{}}%
      \onslide*<5>{\providecommand\step{1}\input{diagrams/backtrace/backtrace.tex}}%
      \onslide*<6>{\providecommand\step{2}\input{diagrams/backtrace/backtrace.tex}}%
    \end{column}
  \end{columns}
\end{frame}

\newcommand\listStashBacktrace[1][]{
  \listc[firstline=91,lastline=120,#1]{../ocaml/runtime/backtrace\_nat.c}{runtime/backtrace\_nat.c:91}}

\begin{frame}{Backtraces}{Introducing \funcname{caml\_stash\_backtrace}}
  \begin{columns}[c]
    \begin{column}{0.5\textwidth}
      \minipage[c][0.66\textheight][s]{\columnwidth}
      \begin{itemize}
        \item<1-> A C function provided by the runtime (specific to native code)
        \item<2-> It scans the stack, between the stack pointer at the raising point up to the next trap, in order to collect the names of the detected function frames
          \onslide*<2>{
            \begin{itemize}
              \item And more information, like line number, column number...
            \end{itemize}
          }
        \item<3-> It uses the same technique and information as the Garbage Collector (GC) uses to scan the values live on the stack
          \onslide*<4>{
            \begin{itemize}
              \item \typename{frame\_descr} are at the core of stack unwinding
            \end{itemize}
          }
      \end{itemize}
      \vfill
      \mintocaml[firstline=9,lastline=13,highlightlines=12]{../src/backtrace/backtrace.ml}
      \endminipage
    \end{column}
    \begin{column}{0.5\textwidth}
      \onslide*<1>{\listStashBacktrace[highlightlines=91]{}}
      \onslide*<2>{\listStashBacktrace[highlightlines={91,117-118}]{}}
      \onslide*<3->{\listStashBacktrace[highlightlines={94,106,109-110}]{}}
    \end{column}
  \end{columns}
\end{frame}

\newcommand\listFrameDescr[1][]{
  \listocaml[firstline=111,lastline=124,#1]{../ocaml/asmcomp/emitaux.ml}{asmcomp/emitaux.ml:111}}
\newcommand\listDebugInfo[1][]{
  \listocaml[firstline=38,lastline=49,#1]{../ocaml/lambda/debuginfo.mli}{lambda/debuginfo.mli:38}}

\begin{frame}{Backtraces}{\typename{frame\_descr} and \typename{frame\_debuginfo} in a nutshell}
  \begin{columns}[c]
    \begin{column}{0.5\textwidth}
      \begin{itemize}
        \item<1-> For every return address in an OCaml program, there is a associated \typename{frame\_descr}
        \item<2-> A \typename{frame\_descr} specifies
        \begin{itemize}
          \item<2-> The size of the function stack frame
          \item<3-> Where live values are on the stack (so that the GC can mark them as live and prevent from being swept)
        \end{itemize}
        \item<4-> When compiled with debug information, \typename{frame\_descr} will be linked to a \typename{frame\_debuginfo}
        \begin{itemize}
          \item<5-> Contains information about the position in the source code (file name, line and column number)
        \end{itemize}
      \end{itemize}
    \end{column}
    \begin{column}{0.5\textwidth}
        \onslide*<1>{\listFrameDescr[highlightlines={118-122}]{}}
        \onslide*<2>{\listFrameDescr[highlightlines={120}]{}}
        \onslide*<3>{\listFrameDescr[highlightlines={121}]{}}
        \onslide*<4->{\listFrameDescr[highlightlines={122}]{}}
        \medskip
        \onslide*<1-3>{\listDebugInfo{}}
        \onslide*<4>{\listDebugInfo[highlightlines={38,49}]{}}
        \onslide*<5>{\listDebugInfo[highlightlines={39-46}]{}}
    \end{column}
  \end{columns}
\end{frame}

\begin{frame}{Backtraces}{Scanning the stack with \typename{frame\_descr}}
  \begin{columns}[c]
    \begin{column}{0.5\textwidth}
      \begin{itemize}
        \item Given the return address of the code that raises the exception by calling \funcname{caml\_raise\_exn} (\funcarg{pc} argument of \funcname{caml\_stash\_backtrace})
        \item And the position of the stack pointer (\funcarg{sp} argument of \funcname{caml\_stash\_backtrace})
        \item The frame size given by \typename{frame\_descr}.\funcarg{frame\_size}, allows to find the end of the previous frame
        \item And the return address of the caller of this function
        \bigskip
        \onslide<3->{
        \item Note that traps (here the reddish \funcname{ExnA}, \funcname{ExnB} and \funcname{ExnC} blocks) belong to the frame that installed them, and so the frame size changes when traps are removed
        }
      \end{itemize}
    \end{column}
    \begin{column}{0.5\textwidth}
      \raggedleft
      \onslide*<1>{\providecommand\step{2}\input{diagrams/backtrace/backtrace.tex}}%
      \onslide*<2>{\providecommand\step{1}\input{diagrams/backtrace/scanstack.tex}}%
      \onslide*<3>{\providecommand\step{2}\input{diagrams/backtrace/scanstack.tex}}%
      \onslide*<4>{\providecommand\step{3}\input{diagrams/backtrace/scanstack.tex}}%
      \onslide*<5>{\providecommand\step{4}\input{diagrams/backtrace/scanstack.tex}}%
      \onslide*<6>{\providecommand\step{5}\input{diagrams/backtrace/scanstack.tex}}%
    \end{column}
  \end{columns}
\end{frame}

% Duplicate of previous-previous slide
\begin{frame}{Backtraces}{Continuing on \funcname{caml\_stash\_backtrace}}
  \begin{columns}[c]
    \begin{column}{0.5\textwidth}
      \minipage[c][0.75\textheight][s]{\columnwidth}
      \begin{itemize}
          \color{gray}
        \item A C function provided by the runtime (specific to native code)
        \item It scans the stack, between the stack pointer at the raising point up to the next trap, in order to collect the names of the detected function frames
        \item It uses the same technique and information as the Garbage Collector (GC) uses to scan the values live on the stack
      \end{itemize}
      \begin{itemize}
        \item For each function frame detected, store a pointer to the \typename{frame\_descr} in the \localname{domain\_state->backtrace\_buffer} array, effectively constructing the backtrace
      \end{itemize}
      \onslide<2->{
        \begin{itemize}
            \smallskip
          \item Constructed backtrace:
            \funcname{definitely\_raise},
            \onslide<3->{\funcname{probably\_raise}}
        \end{itemize}
      }
      \vfill
      \mintocaml[firstline=9,lastline=13,highlightlines=12]{../src/backtrace/backtrace.ml}
      \endminipage
    \end{column}
    \begin{column}{0.5\textwidth}
      \raggedleft
      \onslide*<1>{\listStashBacktrace[highlightlines={114-115}]{}}
      \onslide*<2>{\providecommand\step{3}\input{diagrams/backtrace/backtrace.tex}}%
      \onslide*<3>{\providecommand\step{4}\input{diagrams/backtrace/backtrace.tex}}%
    \end{column}
  \end{columns}
\end{frame}

\begin{frame}{Backtraces}{Back to \funcname{caml\_raise\_exn}}
  \begin{columns}[c]
    \begin{column}{0.5\textwidth}
      \minipage[c][0.66\textheight][s]{\columnwidth}
      \begin{itemize}\color{gray}
        \item Checks wheteher exceptions backtrace should be recorded, by checking the \funcarg{domain\_state->backtrace\_active} value
        \item Switches to the C stack to be able to execute C code
        \item Calls \funcname{caml\_stash\_backtrace}, to collect the backtrace up to the next trap
      \end{itemize}
      \onslide*<2->{
        \begin{itemize}
          \item<2-> Executes the exception handler in \funcname{probably\_raise} with \funcname{RESTORE\_EXN\_HANDLER\_OCAML}
            \begin{itemize}
              \item Set \regname{RSP} to \localname{domain\_state->exn\_handler} value
              \item Restore \localname{domain\_state->exn\_handler} from trap
              \item Jump to the handler code with \funcname{ret}
            \end{itemize}
        \end{itemize}
      }
      \vfill
      \onslide*<1>{\mintocaml[firstline=9,lastline=13,highlightlines=12]{../src/backtrace/backtrace.ml}}
      \onslide*<2->{\mintocaml[firstline=15,lastline=21,highlightlines={19-21}]{../src/backtrace/backtrace.ml}}
      \endminipage
    \end{column}
    \begin{column}{0.5\textwidth}
      \centering
      \onslide*<1>{
        \mintc[firstline=190,lastline=192]{../ocaml/runtime/amd64.S}
        \mintc[firstline=234,lastline=237]{../ocaml/runtime/amd64.S}
      }
      \onslide*<2>{
        \mintc[firstline=190,lastline=192,highlightlines={191-192}]{../ocaml/runtime/amd64.S}
        \mintc[firstline=234,lastline=237,highlightlines={235-237}]{../ocaml/runtime/amd64.S}
      }
      \onslide*<1>{\listCamlRaiseExn[highlightlines=720]{}}
      \onslide*<2>{\listCamlRaiseExn[highlightlines=722]{}}
      \onslide*<3->{\providecommand\step{5}\input{diagrams/backtrace/backtrace.tex}}
    \end{column}
  \end{columns}
\end{frame}

\begin{frame}{Backtraces}{\funcname{caml\_probably\_raise}'s handler doesn't catch \typename{ExnC}}
  \begin{columns}[c]
    \begin{column}{0.45\textwidth}
      \minipage[c][0.75\textheight][s]{\columnwidth}
      \begin{itemize}
        \item<1-> \funcname{match \dots with}\footnotemark pattern matching can also match exceptions
        \item<1-> The CMM of \funcname{probably\_raise} shows that this \funcname{match \dots with} is implemented with a pattern matching and a \funcname{try \dots with}
        \item<1-> But this one only matches on \typename{ExnA} exception
        \item<2-> Since the pattern matching fails to catch the exception, \funcname{reraise} is called with our current \typename{ExnC} exception
        \item<3-> Disassembly shows that \funcname{reraise} is actually a call to \funcname{caml\_reraise\_exn}, which is also an assembly function provided by the runtime
      \end{itemize}
      \vfill
      \mintocaml[firstline=15,lastline=21,highlightlines={18-21}]{../src/backtrace/backtrace.ml}
      \endminipage
    \end{column}
    \begin{column}{0.55\textwidth}
      \centering
      \onslide*<1>{\mintcmm[highlightlines={10-18}]{../src/backtrace/camlBacktrace\_\_probably\_raise\_351.cmm}}
      \onslide*<2>{\listcmm[highlightlines=18]{../src/backtrace/camlBacktrace\_\_probably\_raise_351.cmm}{CMM of probably\_raise}}
      \onslide*<3>{\mintobjdump[firstline=5,lastline=35,highlightlines=31]{../src/backtrace/camlBacktrace\_\_probably\_raise_351.objdump}}
    \end{column}
  \end{columns}
  \only<1>{\footnotetext[1]{https://blog.janestreet.com/pattern-matching-and-exception-handling-unite/}}
\end{frame}

\newcommand\listCamlReraiseExn[1][]{
  \listasm[firstline=727,lastline=734,#1]{../ocaml/runtime/amd64.S}{runtime/amd64.S:727}}

\begin{frame}{Backtraces}{Introducing \funcname{caml\_reraise\_exn}}
  \begin{columns}[c]
    \begin{column}{0.5\textwidth}
      \minipage[c][0.66\textheight][s]{\columnwidth}
      \begin{itemize}
        \item<1-> The exception have to be reraised to the next handler
        \item<2-> First, checks if the backtrace should be collected with \funcarg{domain\_state->backtrace\_active}
        \only<3>{
        \begin{itemize}
            \item If not, behaves like a \funcname{raise\_notrace} with \funcname{RESTORE\_EXN\_HANDLER\_OCAML}
        \end{itemize}
        }
        \item<4-> Jumps into \funcname{caml\_raise\_exn}
        \item<5-> Calls \funcname{caml\_stash\_backtrace} again, to scan the next part of the stack: from the trap just pop-ed to the next trap
      \end{itemize}
      \vfill
      \mintocaml[firstline=15,lastline=21,highlightlines={18-21}]{../src/backtrace/backtrace.ml}
      \endminipage
    \end{column}
    \begin{column}{0.5\textwidth}
      \raggedleft
      \onslide*<1>{\listCamlReraiseExn{}}
      \onslide*<2>{\listCamlReraiseExn[highlightlines=729]{}}
      \onslide*<3>{
        \mintc[firstline=190,lastline=192,highlightlines={191-192}]{../ocaml/runtime/amd64.S}
        \mintc[firstline=234,lastline=237,highlightlines={235-237}]{../ocaml/runtime/amd64.S}
        \medskip
        \listCamlReraiseExn[highlightlines=731]{}
      }
      \onslide*<4-5>{
        % TODO refactor with \listCamlReraiseExn
        \mintasm[firstline=727,lastline=734,highlightlines=730]{../ocaml/runtime/amd64.S}
        \medskip
      }
      \onslide*<4>{\listCamlRaiseExn[highlightlines=711]{}}
      \onslide*<5>{\listCamlRaiseExn[highlightlines=720]{}}
    \end{column}
  \end{columns}
\end{frame}

\begin{frame}{Backtraces}{Backtrace collection continues}
  \begin{columns}[c]
    \begin{column}{0.55\textwidth}
      \minipage[c][0.75\textheight][s]{\columnwidth}
      \begin{itemize}
        \item<1-> \funcname{caml\_stash\_backtrace} scans the older part of the stack, up to the next trap
        \item<6-> Controls jumps to the nested exception handler in \funcname{maybe\_raise}, which matches only on \typename{ExnB}
        \item<7-> \funcname{caml\_reraise\_exn} is called again, backtrace collection resumes with \funcname{caml\_stash\_backtrace}
      \end{itemize}
      \medskip
      \begin{itemize}
        \item<2-> Constructed backtrace:
        \begin{itemize}
          \item <2-> \funcname{definitely\_raise}
          \item <2-> \funcname{probably\_raise}
          \item <3-> \funcname{probably\_raise} (re-raise)
          \item <4-> \funcname{maybe\_raise}
          \item <5-> \funcname{main}
          \item <8-> \funcname{main} (re-raise)
        \end{itemize}
      \end{itemize}
      \vfill
      \onslide*<1-5>{\mintocaml[firstline=15,lastline=21]{../src/backtrace/backtrace.ml}}
      \onslide*<6>{\mintocaml[firstline=28,lastline=34,highlightlines=31]{../src/backtrace/backtrace.ml}}
      \onslide*<7->{\mintocaml[firstline=28,lastline=34,highlightlines=33]{../src/backtrace/backtrace.ml}}
      \endminipage
    \end{column}
    \begin{column}{0.45\textwidth}
      \raggedleft
      \onslide*<1>{\providecommand\step{6}\input{diagrams/backtrace/backtrace.tex}}%
      \onslide*<2>{\providecommand\step{6}\input{diagrams/backtrace/backtrace.tex}}%
      \onslide*<3>{\providecommand\step{7}\input{diagrams/backtrace/backtrace.tex}}%
      \onslide*<4>{\providecommand\step{8}\input{diagrams/backtrace/backtrace.tex}}%
      \onslide*<5>{\providecommand\step{9}\input{diagrams/backtrace/backtrace.tex}}%
      \onslide*<6>{\providecommand\step{10}\input{diagrams/backtrace/backtrace.tex}}%
      \onslide*<7>{\providecommand\step{11}\input{diagrams/backtrace/backtrace.tex}}%
      \onslide*<8>{\providecommand\step{12}\input{diagrams/backtrace/backtrace.tex}}%
      \onslide*<9>{\providecommand\step{13}\input{diagrams/backtrace/backtrace.tex}}%
    \end{column}
  \end{columns}
\end{frame}

\begin{frame}{Backtraces}{Printing the backtrace}
  \begin{columns}[c]
    \begin{column}{0.45\textwidth}
      \minipage[c][0.66\textheight][s]{\columnwidth}
      \begin{itemize}
        \item<1-> The exception backtrace can now be printed to \funcarg{stdout} with \funcname{Printexc.print_backtrace}
        \item<2-> First the exception backtrace is retrieved into an OCaml array with \funcname{get_raw_backtrace }
        \item<3-> Which is implemented as the \funcname{caml_get_exception_raw_backtrace} C function in the runtime
        \item<4-> And finally printed with \funcname{print_exception_backtrace}
      \end{itemize}
      \vfill
      \mintocaml[firstline=28,lastline=34,highlightlines=33]{../src/backtrace/backtrace.ml}
      \endminipage
    \end{column}
    \begin{column}{0.55\textwidth}
      \onslide*<1,2>{
        \mintocaml[firstline=100,lastline=101]{../ocaml/stdlib/printexc.ml}
      }
      \only<1>{
        \listocaml[firstline=170,lastline=172]{../ocaml/stdlib/printexc.ml}{stdlib/printexc.ml:170}
      }
      \only<2>{
        \listocaml[firstline=170,lastline=172,highlightlines=172]{../ocaml/stdlib/printexc.ml}{stdlib/printexc.ml:170}
      }
      \only<3>{
        \mintc[firstline=154,lastline=156]{../ocaml/runtime/backtrace.c}
        \listc[firstline=171,lastline=192,highlightlines={184-187}]{../ocaml/runtime/backtrace.c}{runtime/backtrace.c:154}
      }
      \only<4>{
        \listocaml[firstline=155,lastline=165]{../ocaml/stdlib/printexc.ml}{stdlib/printexc.ml:55}
      }
    \end{column}
  \end{columns}
\end{frame}


\subsection{The multiple kinds of raise}
\frameSubsection{}{}

\begin{frame}{Backtraces}{When backtraces are not needed}
  \begin{columns}[c]
    \begin{column}{0.45\textwidth}
      \minipage[c][0.66\textheight][s]{\columnwidth}
      \begin{itemize}
        \item<1-> What if the backtrace for an exception is never needed?
        \item<2-> It's possible to raise the exception explicitly with \funcname{raise\_notrace} to make raising as cheap as possible
        \item<3-> An example of use in the \funcname{dynlink} library of the OCaml compiler
      \end{itemize}
      \vfill
      \onslide<3->{
      \listocaml[firstline=205,lastline=209,highlightlines=207]{../ocaml/otherlibs/dynlink/dynlink\_compilerlibs/misc.ml}{otherlibs/dynlink/dynlink\_compilerlibs/misc.ml:205}
      }
      \endminipage
    \end{column}
    \begin{column}{0.55\textwidth}
    \only<4>{
      \mintcmm[highlightlines=3]{../src/notrace/camlNotrace\_\_fun_347.cmm}
      \listcmm{../src/notrace/camlNotrace\_\_all\_somes\_327.cmm}{all\_somes}
    }
    \onslide*<5->{
      \listobjdump[highlightlines={6-9}]{../src/notrace/camlNotrace\_\_fun_347.objdump}{anonymous function from \funcname{all\_somes}}
    }
    \end{column}
  \end{columns}
\end{frame}

\begin{frame}{Backtraces}{Summary table of the multiple kinds of raise}
  \begin{itemize}
    \item When debugging information is enabled and if the backtrace of an exception isn't needed, it's possible to raise with \funcname{raise\_notrace} for performance
    \item When debugging information is \emph{not} enabled, the compiler optimises \funcname{raise} calls to inline assembly (same effect as using \funcname{raise\_notrace})
  \end{itemize}
  \bigskip
  \centering
  \begin{tabular}{| l | c | c | c | c |}
    \hline
    \parbox{10em}{Debug information \\ (\funcarg{-g} flag)}  & \multicolumn{2}{|c|}{Disabled}                                     & \multicolumn{2}{|c|}{Enabled} \\
    \hline
    Raise kind                                               & \funcname{raise} & \funcname{raise\_notrace}                       & \funcname{raise} & \funcname{raise\_notrace} \\
    \hline
    Compilation                                              & \parbox{8em}{inline assembly\\ (optimization)} & inline assembly   & \funcname{caml\_raise\_exn} & inline assembly \\
    \hline
  \end{tabular}
\end{frame}


%
% Takeaway #5
%
\subsection*{Takeaway \#5}
\frameSubsectionTakeaway{}
\againframe<5>{takeaway}
}%
    \end{column}
  \end{columns}
\end{frame}

\begin{frame}{Backtraces}{Back to \funcname{caml\_raise\_exn}}
  \begin{columns}[c]
    \begin{column}{0.5\textwidth}
      \minipage[c][0.66\textheight][s]{\columnwidth}
      \begin{itemize}\color{gray}
        \item Checks wheteher exceptions backtrace should be recorded, by checking the \funcarg{domain\_state->backtrace\_active} value
        \item Switches to the C stack to be able to execute C code
        \item Calls \funcname{caml\_stash\_backtrace}, to collect the backtrace up to the next trap
      \end{itemize}
      \onslide*<2->{
        \begin{itemize}
          \item<2-> Executes the exception handler in \funcname{probably\_raise} with \funcname{RESTORE\_EXN\_HANDLER\_OCAML}
            \begin{itemize}
              \item Set \regname{RSP} to \localname{domain\_state->exn\_handler} value
              \item Restore \localname{domain\_state->exn\_handler} from trap
              \item Jump to the handler code with \funcname{ret}
            \end{itemize}
        \end{itemize}
      }
      \vfill
      \onslide*<1>{\mintocaml[firstline=9,lastline=13,highlightlines=12]{../src/backtrace/backtrace.ml}}
      \onslide*<2->{\mintocaml[firstline=15,lastline=21,highlightlines={19-21}]{../src/backtrace/backtrace.ml}}
      \endminipage
    \end{column}
    \begin{column}{0.5\textwidth}
      \centering
      \onslide*<1>{
        \mintc[firstline=190,lastline=192]{../ocaml/runtime/amd64.S}
        \mintc[firstline=234,lastline=237]{../ocaml/runtime/amd64.S}
      }
      \onslide*<2>{
        \mintc[firstline=190,lastline=192,highlightlines={191-192}]{../ocaml/runtime/amd64.S}
        \mintc[firstline=234,lastline=237,highlightlines={235-237}]{../ocaml/runtime/amd64.S}
      }
      \onslide*<1>{\listCamlRaiseExn[highlightlines=720]{}}
      \onslide*<2>{\listCamlRaiseExn[highlightlines=722]{}}
      \onslide*<3->{\providecommand\step{5}%
% Backtraces
%
\subsection{One advantage of raising with debugging information}
\frameSubsection{}{}

\begin{frame}{Backtraces}{Debugging information \& source code}
  \begin{columns}[c]
    \begin{column}{0.5\textwidth}
      \begin{itemize}
        \item Raising an exception is fun and games, but how do we know where the exception originated? $\rightarrow$ With the backtrace associated with the exception!
        \item In order to get a backtrace from the point where an exception was raised, it's necessary to compile with debugging information enabled (\funcarg{-g} flag for the compiler)
        \item Same example as in the `Nested handlers' section
      \end{itemize}
    \end{column}
    \begin{column}{0.5\textwidth}
      \listocaml[firstline=9,lastline=34]{../src/backtrace/backtrace.ml}{backtrace/backtrace.ml}
    \end{column}
  \end{columns}
\end{frame}

\begin{frame}{Backtraces}{CMM and disassembly of \funcname{definitely\_raise}}
  \begin{columns}[c]
    \begin{column}{0.5\textwidth}
      \minipage[c][0.66\textheight][s]{\columnwidth}
      \begin{itemize}
        \item<1-> An effect of compiling with debugging information (\funcarg{-g}): the CMM no longer uses \funcname{raise\_notrace} to raise the exception but \funcname{raise} instead
        \item<2-> Disassembly shows that CMM \funcname{raise} is actually a call to \funcname{caml\_raise\_exn}, a function in assembler provided by the runtime
      \end{itemize}
      \vfill
      \mintocaml[firstline=9,lastline=13,highlightlines=12]{../src/backtrace/backtrace.ml}
      \endminipage
    \end{column}
    \begin{column}{0.5\textwidth}
      \onslide*<1>{
        \mintcmm[highlightlines=5]{../src/backtrace/camlBacktrace\_\_definitely\_raise\_347.cmm}
      }
      \onslide*<2>{
        \mintobjdump[firstline=2,highlightlines=14]{../src/backtrace/camlBacktrace\_\_definitely\_raise\_347.objdump}
      }
    \end{column}
  \end{columns}
\end{frame}

\begin{frame}{Backtraces}{Introducing \funcname{caml\_raise\_exn}}
  \begin{columns}[c]
    \begin{column}{0.5\textwidth}
      \minipage[c][0.66\textheight][s]{\columnwidth}
      \onslide*<1>{
        \begin{itemize}
          \item Raising the exception is performed using \funcname{caml\_raise\_exn} which is
            \begin{itemize}
              \item An assembly function provided by the runtime
              \item Architecture specific
            \end{itemize}
          \item Let's see what it does differently from \funcname{raise\_notrace}\dots
          \item We are about to raise \typename{ExnC} with \funcname{caml\_raise\_exn} from \funcname{definitely\_raise}
        \end{itemize}
      }
      \onslide*<2>{
        \mintobjdump[firstline=2,highlightlines=14]{../src/backtrace/camlBacktrace\_\_definitely\_raise\_347.objdump}
      }
      \vfill
      \mintocaml[firstline=9,lastline=13,highlightlines=12]{../src/backtrace/backtrace.ml}
      \endminipage
    \end{column}
    \begin{column}{0.5\textwidth}
      \centering
      \providecommand\step{1}\input{diagrams/backtrace/backtrace.tex}
    \end{column}
  \end{columns}
\end{frame}

\begin{frame}{Backtraces}{But before that, we need more fields from \typename{caml\_domain\_state}}
  \begin{columns}
    \begin{column}{0.5\textwidth}
      \begin{itemize}
        \item Remember \typename{caml\_domain\_state} the per-domain runtime state?
        \item Some other members are of interest to us now\dots
        \item \localname{backtrace\_active} is used to know if the runtime should collect backtraces
        \item \localname{backtrace\_active} can be set by
          \begin{itemize}
            \item The \funcarg{b} flag in the \funcname{OCAMLRUNPARAM} environment variable
            \item \mintinline{ocaml}{Printexc.record_backtrace : bool -> unit} \footnotemark
          \end{itemize}
      \end{itemize}
    \end{column}
    \begin{column}{0.5\textwidth}
      \begin{listing}
        \mintc[highlightlines={9-11}]{src/ocaml/domain\_state\_backtrace.h}
        \caption{runtime/caml/domain\_state.h}
      \end{listing}
    \end{column}
  \end{columns}
  \footnotetext[1]{https://v2.ocaml.org/api/Printexc.html}
\end{frame}

\newcommand\listCamlRaiseExn[1][]{
  \listasm[firstline=702,lastline=725,#1]{../ocaml/runtime/amd64.S}{runtime/amd64.S:702}}

\begin{frame}{Backtraces}{Walk through \funcname{caml\_raise\_exn}}
  \begin{columns}[T]
    \begin{column}{0.5\textwidth}
      \begin{itemize}
        \item<1-> First checks whether exceptions backtrace should be recorded
        \item<1-> By checking the \funcarg{domain\_state->backtrace\_active} value
        \item<2-> If not, acts as \funcname{raise\_notrace} with \funcname{RESTORE\_EXN\_HANDLER\_OCAML}
          \begin{itemize}
            \item Set \regname{RSP} to \localname{domain\_state->exn\_handler} value
            \item Restore \localname{domain\_state->exn\_handler} from trap
            \item Jump with \funcname{ret} (same effect as using \funcname{pop} and \funcname{jmp})
          \end{itemize}
        \item<3-> Otherwise, switches to the C stack to be able to execute C code
        \item<4-> Calls \funcname{caml\_stash\_backtrace}, a C function provided by the runtime\ignorespaces
          \onslide<6->{, with
            \begin{itemize}
              \item \funcarg{exn}: exception value
              \item \funcarg{pc}: code address of the raising point
              \item \funcarg{sp}: stack address of the raising function
              \item \funcarg{trapsp}: stack address of the next handler
            \end{itemize}
          }
      \end{itemize}
      \bigskip
      \mintocaml[firstline=9,lastline=13,highlightlines=12]{../src/backtrace/backtrace.ml}
    \end{column}
    \begin{column}{0.5\textwidth}
      \raggedleft
      \onslide*<1,3-4>{
        \mintc[firstline=190,lastline=192]{../ocaml/runtime/amd64.S}
        \mintc[firstline=234,lastline=237]{../ocaml/runtime/amd64.S}
      }
      \onslide*<2>{
        \mintc[firstline=190,lastline=192,highlightlines={191-192}]{../ocaml/runtime/amd64.S}
        \mintc[firstline=234,lastline=237,highlightlines={235-237}]{../ocaml/runtime/amd64.S}
      }
      \onslide*<1>{\listCamlRaiseExn[highlightlines=705]{}}%
      \onslide*<2>{\listCamlRaiseExn[highlightlines={707-708}]{}}%
      \onslide*<3>{\listCamlRaiseExn[highlightlines={712-714}]{}}%
      \onslide*<4>{\listCamlRaiseExn[highlightlines={715-720}]{}}%
      \onslide*<5>{\providecommand\step{1}\input{diagrams/backtrace/backtrace.tex}}%
      \onslide*<6>{\providecommand\step{2}\input{diagrams/backtrace/backtrace.tex}}%
    \end{column}
  \end{columns}
\end{frame}

\newcommand\listStashBacktrace[1][]{
  \listc[firstline=91,lastline=120,#1]{../ocaml/runtime/backtrace\_nat.c}{runtime/backtrace\_nat.c:91}}

\begin{frame}{Backtraces}{Introducing \funcname{caml\_stash\_backtrace}}
  \begin{columns}[c]
    \begin{column}{0.5\textwidth}
      \minipage[c][0.66\textheight][s]{\columnwidth}
      \begin{itemize}
        \item<1-> A C function provided by the runtime (specific to native code)
        \item<2-> It scans the stack, between the stack pointer at the raising point up to the next trap, in order to collect the names of the detected function frames
          \onslide*<2>{
            \begin{itemize}
              \item And more information, like line number, column number...
            \end{itemize}
          }
        \item<3-> It uses the same technique and information as the Garbage Collector (GC) uses to scan the values live on the stack
          \onslide*<4>{
            \begin{itemize}
              \item \typename{frame\_descr} are at the core of stack unwinding
            \end{itemize}
          }
      \end{itemize}
      \vfill
      \mintocaml[firstline=9,lastline=13,highlightlines=12]{../src/backtrace/backtrace.ml}
      \endminipage
    \end{column}
    \begin{column}{0.5\textwidth}
      \onslide*<1>{\listStashBacktrace[highlightlines=91]{}}
      \onslide*<2>{\listStashBacktrace[highlightlines={91,117-118}]{}}
      \onslide*<3->{\listStashBacktrace[highlightlines={94,106,109-110}]{}}
    \end{column}
  \end{columns}
\end{frame}

\newcommand\listFrameDescr[1][]{
  \listocaml[firstline=111,lastline=124,#1]{../ocaml/asmcomp/emitaux.ml}{asmcomp/emitaux.ml:111}}
\newcommand\listDebugInfo[1][]{
  \listocaml[firstline=38,lastline=49,#1]{../ocaml/lambda/debuginfo.mli}{lambda/debuginfo.mli:38}}

\begin{frame}{Backtraces}{\typename{frame\_descr} and \typename{frame\_debuginfo} in a nutshell}
  \begin{columns}[c]
    \begin{column}{0.5\textwidth}
      \begin{itemize}
        \item<1-> For every return address in an OCaml program, there is a associated \typename{frame\_descr}
        \item<2-> A \typename{frame\_descr} specifies
        \begin{itemize}
          \item<2-> The size of the function stack frame
          \item<3-> Where live values are on the stack (so that the GC can mark them as live and prevent from being swept)
        \end{itemize}
        \item<4-> When compiled with debug information, \typename{frame\_descr} will be linked to a \typename{frame\_debuginfo}
        \begin{itemize}
          \item<5-> Contains information about the position in the source code (file name, line and column number)
        \end{itemize}
      \end{itemize}
    \end{column}
    \begin{column}{0.5\textwidth}
        \onslide*<1>{\listFrameDescr[highlightlines={118-122}]{}}
        \onslide*<2>{\listFrameDescr[highlightlines={120}]{}}
        \onslide*<3>{\listFrameDescr[highlightlines={121}]{}}
        \onslide*<4->{\listFrameDescr[highlightlines={122}]{}}
        \medskip
        \onslide*<1-3>{\listDebugInfo{}}
        \onslide*<4>{\listDebugInfo[highlightlines={38,49}]{}}
        \onslide*<5>{\listDebugInfo[highlightlines={39-46}]{}}
    \end{column}
  \end{columns}
\end{frame}

\begin{frame}{Backtraces}{Scanning the stack with \typename{frame\_descr}}
  \begin{columns}[c]
    \begin{column}{0.5\textwidth}
      \begin{itemize}
        \item Given the return address of the code that raises the exception by calling \funcname{caml\_raise\_exn} (\funcarg{pc} argument of \funcname{caml\_stash\_backtrace})
        \item And the position of the stack pointer (\funcarg{sp} argument of \funcname{caml\_stash\_backtrace})
        \item The frame size given by \typename{frame\_descr}.\funcarg{frame\_size}, allows to find the end of the previous frame
        \item And the return address of the caller of this function
        \bigskip
        \onslide<3->{
        \item Note that traps (here the reddish \funcname{ExnA}, \funcname{ExnB} and \funcname{ExnC} blocks) belong to the frame that installed them, and so the frame size changes when traps are removed
        }
      \end{itemize}
    \end{column}
    \begin{column}{0.5\textwidth}
      \raggedleft
      \onslide*<1>{\providecommand\step{2}\input{diagrams/backtrace/backtrace.tex}}%
      \onslide*<2>{\providecommand\step{1}\input{diagrams/backtrace/scanstack.tex}}%
      \onslide*<3>{\providecommand\step{2}\input{diagrams/backtrace/scanstack.tex}}%
      \onslide*<4>{\providecommand\step{3}\input{diagrams/backtrace/scanstack.tex}}%
      \onslide*<5>{\providecommand\step{4}\input{diagrams/backtrace/scanstack.tex}}%
      \onslide*<6>{\providecommand\step{5}\input{diagrams/backtrace/scanstack.tex}}%
    \end{column}
  \end{columns}
\end{frame}

% Duplicate of previous-previous slide
\begin{frame}{Backtraces}{Continuing on \funcname{caml\_stash\_backtrace}}
  \begin{columns}[c]
    \begin{column}{0.5\textwidth}
      \minipage[c][0.75\textheight][s]{\columnwidth}
      \begin{itemize}
          \color{gray}
        \item A C function provided by the runtime (specific to native code)
        \item It scans the stack, between the stack pointer at the raising point up to the next trap, in order to collect the names of the detected function frames
        \item It uses the same technique and information as the Garbage Collector (GC) uses to scan the values live on the stack
      \end{itemize}
      \begin{itemize}
        \item For each function frame detected, store a pointer to the \typename{frame\_descr} in the \localname{domain\_state->backtrace\_buffer} array, effectively constructing the backtrace
      \end{itemize}
      \onslide<2->{
        \begin{itemize}
            \smallskip
          \item Constructed backtrace:
            \funcname{definitely\_raise},
            \onslide<3->{\funcname{probably\_raise}}
        \end{itemize}
      }
      \vfill
      \mintocaml[firstline=9,lastline=13,highlightlines=12]{../src/backtrace/backtrace.ml}
      \endminipage
    \end{column}
    \begin{column}{0.5\textwidth}
      \raggedleft
      \onslide*<1>{\listStashBacktrace[highlightlines={114-115}]{}}
      \onslide*<2>{\providecommand\step{3}\input{diagrams/backtrace/backtrace.tex}}%
      \onslide*<3>{\providecommand\step{4}\input{diagrams/backtrace/backtrace.tex}}%
    \end{column}
  \end{columns}
\end{frame}

\begin{frame}{Backtraces}{Back to \funcname{caml\_raise\_exn}}
  \begin{columns}[c]
    \begin{column}{0.5\textwidth}
      \minipage[c][0.66\textheight][s]{\columnwidth}
      \begin{itemize}\color{gray}
        \item Checks wheteher exceptions backtrace should be recorded, by checking the \funcarg{domain\_state->backtrace\_active} value
        \item Switches to the C stack to be able to execute C code
        \item Calls \funcname{caml\_stash\_backtrace}, to collect the backtrace up to the next trap
      \end{itemize}
      \onslide*<2->{
        \begin{itemize}
          \item<2-> Executes the exception handler in \funcname{probably\_raise} with \funcname{RESTORE\_EXN\_HANDLER\_OCAML}
            \begin{itemize}
              \item Set \regname{RSP} to \localname{domain\_state->exn\_handler} value
              \item Restore \localname{domain\_state->exn\_handler} from trap
              \item Jump to the handler code with \funcname{ret}
            \end{itemize}
        \end{itemize}
      }
      \vfill
      \onslide*<1>{\mintocaml[firstline=9,lastline=13,highlightlines=12]{../src/backtrace/backtrace.ml}}
      \onslide*<2->{\mintocaml[firstline=15,lastline=21,highlightlines={19-21}]{../src/backtrace/backtrace.ml}}
      \endminipage
    \end{column}
    \begin{column}{0.5\textwidth}
      \centering
      \onslide*<1>{
        \mintc[firstline=190,lastline=192]{../ocaml/runtime/amd64.S}
        \mintc[firstline=234,lastline=237]{../ocaml/runtime/amd64.S}
      }
      \onslide*<2>{
        \mintc[firstline=190,lastline=192,highlightlines={191-192}]{../ocaml/runtime/amd64.S}
        \mintc[firstline=234,lastline=237,highlightlines={235-237}]{../ocaml/runtime/amd64.S}
      }
      \onslide*<1>{\listCamlRaiseExn[highlightlines=720]{}}
      \onslide*<2>{\listCamlRaiseExn[highlightlines=722]{}}
      \onslide*<3->{\providecommand\step{5}\input{diagrams/backtrace/backtrace.tex}}
    \end{column}
  \end{columns}
\end{frame}

\begin{frame}{Backtraces}{\funcname{caml\_probably\_raise}'s handler doesn't catch \typename{ExnC}}
  \begin{columns}[c]
    \begin{column}{0.45\textwidth}
      \minipage[c][0.75\textheight][s]{\columnwidth}
      \begin{itemize}
        \item<1-> \funcname{match \dots with}\footnotemark pattern matching can also match exceptions
        \item<1-> The CMM of \funcname{probably\_raise} shows that this \funcname{match \dots with} is implemented with a pattern matching and a \funcname{try \dots with}
        \item<1-> But this one only matches on \typename{ExnA} exception
        \item<2-> Since the pattern matching fails to catch the exception, \funcname{reraise} is called with our current \typename{ExnC} exception
        \item<3-> Disassembly shows that \funcname{reraise} is actually a call to \funcname{caml\_reraise\_exn}, which is also an assembly function provided by the runtime
      \end{itemize}
      \vfill
      \mintocaml[firstline=15,lastline=21,highlightlines={18-21}]{../src/backtrace/backtrace.ml}
      \endminipage
    \end{column}
    \begin{column}{0.55\textwidth}
      \centering
      \onslide*<1>{\mintcmm[highlightlines={10-18}]{../src/backtrace/camlBacktrace\_\_probably\_raise\_351.cmm}}
      \onslide*<2>{\listcmm[highlightlines=18]{../src/backtrace/camlBacktrace\_\_probably\_raise_351.cmm}{CMM of probably\_raise}}
      \onslide*<3>{\mintobjdump[firstline=5,lastline=35,highlightlines=31]{../src/backtrace/camlBacktrace\_\_probably\_raise_351.objdump}}
    \end{column}
  \end{columns}
  \only<1>{\footnotetext[1]{https://blog.janestreet.com/pattern-matching-and-exception-handling-unite/}}
\end{frame}

\newcommand\listCamlReraiseExn[1][]{
  \listasm[firstline=727,lastline=734,#1]{../ocaml/runtime/amd64.S}{runtime/amd64.S:727}}

\begin{frame}{Backtraces}{Introducing \funcname{caml\_reraise\_exn}}
  \begin{columns}[c]
    \begin{column}{0.5\textwidth}
      \minipage[c][0.66\textheight][s]{\columnwidth}
      \begin{itemize}
        \item<1-> The exception have to be reraised to the next handler
        \item<2-> First, checks if the backtrace should be collected with \funcarg{domain\_state->backtrace\_active}
        \only<3>{
        \begin{itemize}
            \item If not, behaves like a \funcname{raise\_notrace} with \funcname{RESTORE\_EXN\_HANDLER\_OCAML}
        \end{itemize}
        }
        \item<4-> Jumps into \funcname{caml\_raise\_exn}
        \item<5-> Calls \funcname{caml\_stash\_backtrace} again, to scan the next part of the stack: from the trap just pop-ed to the next trap
      \end{itemize}
      \vfill
      \mintocaml[firstline=15,lastline=21,highlightlines={18-21}]{../src/backtrace/backtrace.ml}
      \endminipage
    \end{column}
    \begin{column}{0.5\textwidth}
      \raggedleft
      \onslide*<1>{\listCamlReraiseExn{}}
      \onslide*<2>{\listCamlReraiseExn[highlightlines=729]{}}
      \onslide*<3>{
        \mintc[firstline=190,lastline=192,highlightlines={191-192}]{../ocaml/runtime/amd64.S}
        \mintc[firstline=234,lastline=237,highlightlines={235-237}]{../ocaml/runtime/amd64.S}
        \medskip
        \listCamlReraiseExn[highlightlines=731]{}
      }
      \onslide*<4-5>{
        % TODO refactor with \listCamlReraiseExn
        \mintasm[firstline=727,lastline=734,highlightlines=730]{../ocaml/runtime/amd64.S}
        \medskip
      }
      \onslide*<4>{\listCamlRaiseExn[highlightlines=711]{}}
      \onslide*<5>{\listCamlRaiseExn[highlightlines=720]{}}
    \end{column}
  \end{columns}
\end{frame}

\begin{frame}{Backtraces}{Backtrace collection continues}
  \begin{columns}[c]
    \begin{column}{0.55\textwidth}
      \minipage[c][0.75\textheight][s]{\columnwidth}
      \begin{itemize}
        \item<1-> \funcname{caml\_stash\_backtrace} scans the older part of the stack, up to the next trap
        \item<6-> Controls jumps to the nested exception handler in \funcname{maybe\_raise}, which matches only on \typename{ExnB}
        \item<7-> \funcname{caml\_reraise\_exn} is called again, backtrace collection resumes with \funcname{caml\_stash\_backtrace}
      \end{itemize}
      \medskip
      \begin{itemize}
        \item<2-> Constructed backtrace:
        \begin{itemize}
          \item <2-> \funcname{definitely\_raise}
          \item <2-> \funcname{probably\_raise}
          \item <3-> \funcname{probably\_raise} (re-raise)
          \item <4-> \funcname{maybe\_raise}
          \item <5-> \funcname{main}
          \item <8-> \funcname{main} (re-raise)
        \end{itemize}
      \end{itemize}
      \vfill
      \onslide*<1-5>{\mintocaml[firstline=15,lastline=21]{../src/backtrace/backtrace.ml}}
      \onslide*<6>{\mintocaml[firstline=28,lastline=34,highlightlines=31]{../src/backtrace/backtrace.ml}}
      \onslide*<7->{\mintocaml[firstline=28,lastline=34,highlightlines=33]{../src/backtrace/backtrace.ml}}
      \endminipage
    \end{column}
    \begin{column}{0.45\textwidth}
      \raggedleft
      \onslide*<1>{\providecommand\step{6}\input{diagrams/backtrace/backtrace.tex}}%
      \onslide*<2>{\providecommand\step{6}\input{diagrams/backtrace/backtrace.tex}}%
      \onslide*<3>{\providecommand\step{7}\input{diagrams/backtrace/backtrace.tex}}%
      \onslide*<4>{\providecommand\step{8}\input{diagrams/backtrace/backtrace.tex}}%
      \onslide*<5>{\providecommand\step{9}\input{diagrams/backtrace/backtrace.tex}}%
      \onslide*<6>{\providecommand\step{10}\input{diagrams/backtrace/backtrace.tex}}%
      \onslide*<7>{\providecommand\step{11}\input{diagrams/backtrace/backtrace.tex}}%
      \onslide*<8>{\providecommand\step{12}\input{diagrams/backtrace/backtrace.tex}}%
      \onslide*<9>{\providecommand\step{13}\input{diagrams/backtrace/backtrace.tex}}%
    \end{column}
  \end{columns}
\end{frame}

\begin{frame}{Backtraces}{Printing the backtrace}
  \begin{columns}[c]
    \begin{column}{0.45\textwidth}
      \minipage[c][0.66\textheight][s]{\columnwidth}
      \begin{itemize}
        \item<1-> The exception backtrace can now be printed to \funcarg{stdout} with \funcname{Printexc.print_backtrace}
        \item<2-> First the exception backtrace is retrieved into an OCaml array with \funcname{get_raw_backtrace }
        \item<3-> Which is implemented as the \funcname{caml_get_exception_raw_backtrace} C function in the runtime
        \item<4-> And finally printed with \funcname{print_exception_backtrace}
      \end{itemize}
      \vfill
      \mintocaml[firstline=28,lastline=34,highlightlines=33]{../src/backtrace/backtrace.ml}
      \endminipage
    \end{column}
    \begin{column}{0.55\textwidth}
      \onslide*<1,2>{
        \mintocaml[firstline=100,lastline=101]{../ocaml/stdlib/printexc.ml}
      }
      \only<1>{
        \listocaml[firstline=170,lastline=172]{../ocaml/stdlib/printexc.ml}{stdlib/printexc.ml:170}
      }
      \only<2>{
        \listocaml[firstline=170,lastline=172,highlightlines=172]{../ocaml/stdlib/printexc.ml}{stdlib/printexc.ml:170}
      }
      \only<3>{
        \mintc[firstline=154,lastline=156]{../ocaml/runtime/backtrace.c}
        \listc[firstline=171,lastline=192,highlightlines={184-187}]{../ocaml/runtime/backtrace.c}{runtime/backtrace.c:154}
      }
      \only<4>{
        \listocaml[firstline=155,lastline=165]{../ocaml/stdlib/printexc.ml}{stdlib/printexc.ml:55}
      }
    \end{column}
  \end{columns}
\end{frame}


\subsection{The multiple kinds of raise}
\frameSubsection{}{}

\begin{frame}{Backtraces}{When backtraces are not needed}
  \begin{columns}[c]
    \begin{column}{0.45\textwidth}
      \minipage[c][0.66\textheight][s]{\columnwidth}
      \begin{itemize}
        \item<1-> What if the backtrace for an exception is never needed?
        \item<2-> It's possible to raise the exception explicitly with \funcname{raise\_notrace} to make raising as cheap as possible
        \item<3-> An example of use in the \funcname{dynlink} library of the OCaml compiler
      \end{itemize}
      \vfill
      \onslide<3->{
      \listocaml[firstline=205,lastline=209,highlightlines=207]{../ocaml/otherlibs/dynlink/dynlink\_compilerlibs/misc.ml}{otherlibs/dynlink/dynlink\_compilerlibs/misc.ml:205}
      }
      \endminipage
    \end{column}
    \begin{column}{0.55\textwidth}
    \only<4>{
      \mintcmm[highlightlines=3]{../src/notrace/camlNotrace\_\_fun_347.cmm}
      \listcmm{../src/notrace/camlNotrace\_\_all\_somes\_327.cmm}{all\_somes}
    }
    \onslide*<5->{
      \listobjdump[highlightlines={6-9}]{../src/notrace/camlNotrace\_\_fun_347.objdump}{anonymous function from \funcname{all\_somes}}
    }
    \end{column}
  \end{columns}
\end{frame}

\begin{frame}{Backtraces}{Summary table of the multiple kinds of raise}
  \begin{itemize}
    \item When debugging information is enabled and if the backtrace of an exception isn't needed, it's possible to raise with \funcname{raise\_notrace} for performance
    \item When debugging information is \emph{not} enabled, the compiler optimises \funcname{raise} calls to inline assembly (same effect as using \funcname{raise\_notrace})
  \end{itemize}
  \bigskip
  \centering
  \begin{tabular}{| l | c | c | c | c |}
    \hline
    \parbox{10em}{Debug information \\ (\funcarg{-g} flag)}  & \multicolumn{2}{|c|}{Disabled}                                     & \multicolumn{2}{|c|}{Enabled} \\
    \hline
    Raise kind                                               & \funcname{raise} & \funcname{raise\_notrace}                       & \funcname{raise} & \funcname{raise\_notrace} \\
    \hline
    Compilation                                              & \parbox{8em}{inline assembly\\ (optimization)} & inline assembly   & \funcname{caml\_raise\_exn} & inline assembly \\
    \hline
  \end{tabular}
\end{frame}


%
% Takeaway #5
%
\subsection*{Takeaway \#5}
\frameSubsectionTakeaway{}
\againframe<5>{takeaway}
}
    \end{column}
  \end{columns}
\end{frame}

\begin{frame}{Backtraces}{\funcname{caml\_probably\_raise}'s handler doesn't catch \typename{ExnC}}
  \begin{columns}[c]
    \begin{column}{0.45\textwidth}
      \minipage[c][0.75\textheight][s]{\columnwidth}
      \begin{itemize}
        \item<1-> \funcname{match \dots with}\footnotemark pattern matching can also match exceptions
        \item<1-> The CMM of \funcname{probably\_raise} shows that this \funcname{match \dots with} is implemented with a pattern matching and a \funcname{try \dots with}
        \item<1-> But this one only matches on \typename{ExnA} exception
        \item<2-> Since the pattern matching fails to catch the exception, \funcname{reraise} is called with our current \typename{ExnC} exception
        \item<3-> Disassembly shows that \funcname{reraise} is actually a call to \funcname{caml\_reraise\_exn}, which is also an assembly function provided by the runtime
      \end{itemize}
      \vfill
      \mintocaml[firstline=15,lastline=21,highlightlines={18-21}]{../src/backtrace/backtrace.ml}
      \endminipage
    \end{column}
    \begin{column}{0.55\textwidth}
      \centering
      \onslide*<1>{\mintcmm[highlightlines={10-18}]{../src/backtrace/camlBacktrace\_\_probably\_raise\_351.cmm}}
      \onslide*<2>{\listcmm[highlightlines=18]{../src/backtrace/camlBacktrace\_\_probably\_raise_351.cmm}{CMM of probably\_raise}}
      \onslide*<3>{\mintobjdump[firstline=5,lastline=35,highlightlines=31]{../src/backtrace/camlBacktrace\_\_probably\_raise_351.objdump}}
    \end{column}
  \end{columns}
  \only<1>{\footnotetext[1]{https://blog.janestreet.com/pattern-matching-and-exception-handling-unite/}}
\end{frame}

\newcommand\listCamlReraiseExn[1][]{
  \listasm[firstline=727,lastline=734,#1]{../ocaml/runtime/amd64.S}{runtime/amd64.S:727}}

\begin{frame}{Backtraces}{Introducing \funcname{caml\_reraise\_exn}}
  \begin{columns}[c]
    \begin{column}{0.5\textwidth}
      \minipage[c][0.66\textheight][s]{\columnwidth}
      \begin{itemize}
        \item<1-> The exception have to be reraised to the next handler
        \item<2-> First, checks if the backtrace should be collected with \funcarg{domain\_state->backtrace\_active}
        \only<3>{
        \begin{itemize}
            \item If not, behaves like a \funcname{raise\_notrace} with \funcname{RESTORE\_EXN\_HANDLER\_OCAML}
        \end{itemize}
        }
        \item<4-> Jumps into \funcname{caml\_raise\_exn}
        \item<5-> Calls \funcname{caml\_stash\_backtrace} again, to scan the next part of the stack: from the trap just pop-ed to the next trap
      \end{itemize}
      \vfill
      \mintocaml[firstline=15,lastline=21,highlightlines={18-21}]{../src/backtrace/backtrace.ml}
      \endminipage
    \end{column}
    \begin{column}{0.5\textwidth}
      \raggedleft
      \onslide*<1>{\listCamlReraiseExn{}}
      \onslide*<2>{\listCamlReraiseExn[highlightlines=729]{}}
      \onslide*<3>{
        \mintc[firstline=190,lastline=192,highlightlines={191-192}]{../ocaml/runtime/amd64.S}
        \mintc[firstline=234,lastline=237,highlightlines={235-237}]{../ocaml/runtime/amd64.S}
        \medskip
        \listCamlReraiseExn[highlightlines=731]{}
      }
      \onslide*<4-5>{
        % TODO refactor with \listCamlReraiseExn
        \mintasm[firstline=727,lastline=734,highlightlines=730]{../ocaml/runtime/amd64.S}
        \medskip
      }
      \onslide*<4>{\listCamlRaiseExn[highlightlines=711]{}}
      \onslide*<5>{\listCamlRaiseExn[highlightlines=720]{}}
    \end{column}
  \end{columns}
\end{frame}

\begin{frame}{Backtraces}{Backtrace collection continues}
  \begin{columns}[c]
    \begin{column}{0.55\textwidth}
      \minipage[c][0.75\textheight][s]{\columnwidth}
      \begin{itemize}
        \item<1-> \funcname{caml\_stash\_backtrace} scans the older part of the stack, up to the next trap
        \item<6-> Controls jumps to the nested exception handler in \funcname{maybe\_raise}, which matches only on \typename{ExnB}
        \item<7-> \funcname{caml\_reraise\_exn} is called again, backtrace collection resumes with \funcname{caml\_stash\_backtrace}
      \end{itemize}
      \medskip
      \begin{itemize}
        \item<2-> Constructed backtrace:
        \begin{itemize}
          \item <2-> \funcname{definitely\_raise}
          \item <2-> \funcname{probably\_raise}
          \item <3-> \funcname{probably\_raise} (re-raise)
          \item <4-> \funcname{maybe\_raise}
          \item <5-> \funcname{main}
          \item <8-> \funcname{main} (re-raise)
        \end{itemize}
      \end{itemize}
      \vfill
      \onslide*<1-5>{\mintocaml[firstline=15,lastline=21]{../src/backtrace/backtrace.ml}}
      \onslide*<6>{\mintocaml[firstline=28,lastline=34,highlightlines=31]{../src/backtrace/backtrace.ml}}
      \onslide*<7->{\mintocaml[firstline=28,lastline=34,highlightlines=33]{../src/backtrace/backtrace.ml}}
      \endminipage
    \end{column}
    \begin{column}{0.45\textwidth}
      \raggedleft
      \onslide*<1>{\providecommand\step{6}%
% Backtraces
%
\subsection{One advantage of raising with debugging information}
\frameSubsection{}{}

\begin{frame}{Backtraces}{Debugging information \& source code}
  \begin{columns}[c]
    \begin{column}{0.5\textwidth}
      \begin{itemize}
        \item Raising an exception is fun and games, but how do we know where the exception originated? $\rightarrow$ With the backtrace associated with the exception!
        \item In order to get a backtrace from the point where an exception was raised, it's necessary to compile with debugging information enabled (\funcarg{-g} flag for the compiler)
        \item Same example as in the `Nested handlers' section
      \end{itemize}
    \end{column}
    \begin{column}{0.5\textwidth}
      \listocaml[firstline=9,lastline=34]{../src/backtrace/backtrace.ml}{backtrace/backtrace.ml}
    \end{column}
  \end{columns}
\end{frame}

\begin{frame}{Backtraces}{CMM and disassembly of \funcname{definitely\_raise}}
  \begin{columns}[c]
    \begin{column}{0.5\textwidth}
      \minipage[c][0.66\textheight][s]{\columnwidth}
      \begin{itemize}
        \item<1-> An effect of compiling with debugging information (\funcarg{-g}): the CMM no longer uses \funcname{raise\_notrace} to raise the exception but \funcname{raise} instead
        \item<2-> Disassembly shows that CMM \funcname{raise} is actually a call to \funcname{caml\_raise\_exn}, a function in assembler provided by the runtime
      \end{itemize}
      \vfill
      \mintocaml[firstline=9,lastline=13,highlightlines=12]{../src/backtrace/backtrace.ml}
      \endminipage
    \end{column}
    \begin{column}{0.5\textwidth}
      \onslide*<1>{
        \mintcmm[highlightlines=5]{../src/backtrace/camlBacktrace\_\_definitely\_raise\_347.cmm}
      }
      \onslide*<2>{
        \mintobjdump[firstline=2,highlightlines=14]{../src/backtrace/camlBacktrace\_\_definitely\_raise\_347.objdump}
      }
    \end{column}
  \end{columns}
\end{frame}

\begin{frame}{Backtraces}{Introducing \funcname{caml\_raise\_exn}}
  \begin{columns}[c]
    \begin{column}{0.5\textwidth}
      \minipage[c][0.66\textheight][s]{\columnwidth}
      \onslide*<1>{
        \begin{itemize}
          \item Raising the exception is performed using \funcname{caml\_raise\_exn} which is
            \begin{itemize}
              \item An assembly function provided by the runtime
              \item Architecture specific
            \end{itemize}
          \item Let's see what it does differently from \funcname{raise\_notrace}\dots
          \item We are about to raise \typename{ExnC} with \funcname{caml\_raise\_exn} from \funcname{definitely\_raise}
        \end{itemize}
      }
      \onslide*<2>{
        \mintobjdump[firstline=2,highlightlines=14]{../src/backtrace/camlBacktrace\_\_definitely\_raise\_347.objdump}
      }
      \vfill
      \mintocaml[firstline=9,lastline=13,highlightlines=12]{../src/backtrace/backtrace.ml}
      \endminipage
    \end{column}
    \begin{column}{0.5\textwidth}
      \centering
      \providecommand\step{1}\input{diagrams/backtrace/backtrace.tex}
    \end{column}
  \end{columns}
\end{frame}

\begin{frame}{Backtraces}{But before that, we need more fields from \typename{caml\_domain\_state}}
  \begin{columns}
    \begin{column}{0.5\textwidth}
      \begin{itemize}
        \item Remember \typename{caml\_domain\_state} the per-domain runtime state?
        \item Some other members are of interest to us now\dots
        \item \localname{backtrace\_active} is used to know if the runtime should collect backtraces
        \item \localname{backtrace\_active} can be set by
          \begin{itemize}
            \item The \funcarg{b} flag in the \funcname{OCAMLRUNPARAM} environment variable
            \item \mintinline{ocaml}{Printexc.record_backtrace : bool -> unit} \footnotemark
          \end{itemize}
      \end{itemize}
    \end{column}
    \begin{column}{0.5\textwidth}
      \begin{listing}
        \mintc[highlightlines={9-11}]{src/ocaml/domain\_state\_backtrace.h}
        \caption{runtime/caml/domain\_state.h}
      \end{listing}
    \end{column}
  \end{columns}
  \footnotetext[1]{https://v2.ocaml.org/api/Printexc.html}
\end{frame}

\newcommand\listCamlRaiseExn[1][]{
  \listasm[firstline=702,lastline=725,#1]{../ocaml/runtime/amd64.S}{runtime/amd64.S:702}}

\begin{frame}{Backtraces}{Walk through \funcname{caml\_raise\_exn}}
  \begin{columns}[T]
    \begin{column}{0.5\textwidth}
      \begin{itemize}
        \item<1-> First checks whether exceptions backtrace should be recorded
        \item<1-> By checking the \funcarg{domain\_state->backtrace\_active} value
        \item<2-> If not, acts as \funcname{raise\_notrace} with \funcname{RESTORE\_EXN\_HANDLER\_OCAML}
          \begin{itemize}
            \item Set \regname{RSP} to \localname{domain\_state->exn\_handler} value
            \item Restore \localname{domain\_state->exn\_handler} from trap
            \item Jump with \funcname{ret} (same effect as using \funcname{pop} and \funcname{jmp})
          \end{itemize}
        \item<3-> Otherwise, switches to the C stack to be able to execute C code
        \item<4-> Calls \funcname{caml\_stash\_backtrace}, a C function provided by the runtime\ignorespaces
          \onslide<6->{, with
            \begin{itemize}
              \item \funcarg{exn}: exception value
              \item \funcarg{pc}: code address of the raising point
              \item \funcarg{sp}: stack address of the raising function
              \item \funcarg{trapsp}: stack address of the next handler
            \end{itemize}
          }
      \end{itemize}
      \bigskip
      \mintocaml[firstline=9,lastline=13,highlightlines=12]{../src/backtrace/backtrace.ml}
    \end{column}
    \begin{column}{0.5\textwidth}
      \raggedleft
      \onslide*<1,3-4>{
        \mintc[firstline=190,lastline=192]{../ocaml/runtime/amd64.S}
        \mintc[firstline=234,lastline=237]{../ocaml/runtime/amd64.S}
      }
      \onslide*<2>{
        \mintc[firstline=190,lastline=192,highlightlines={191-192}]{../ocaml/runtime/amd64.S}
        \mintc[firstline=234,lastline=237,highlightlines={235-237}]{../ocaml/runtime/amd64.S}
      }
      \onslide*<1>{\listCamlRaiseExn[highlightlines=705]{}}%
      \onslide*<2>{\listCamlRaiseExn[highlightlines={707-708}]{}}%
      \onslide*<3>{\listCamlRaiseExn[highlightlines={712-714}]{}}%
      \onslide*<4>{\listCamlRaiseExn[highlightlines={715-720}]{}}%
      \onslide*<5>{\providecommand\step{1}\input{diagrams/backtrace/backtrace.tex}}%
      \onslide*<6>{\providecommand\step{2}\input{diagrams/backtrace/backtrace.tex}}%
    \end{column}
  \end{columns}
\end{frame}

\newcommand\listStashBacktrace[1][]{
  \listc[firstline=91,lastline=120,#1]{../ocaml/runtime/backtrace\_nat.c}{runtime/backtrace\_nat.c:91}}

\begin{frame}{Backtraces}{Introducing \funcname{caml\_stash\_backtrace}}
  \begin{columns}[c]
    \begin{column}{0.5\textwidth}
      \minipage[c][0.66\textheight][s]{\columnwidth}
      \begin{itemize}
        \item<1-> A C function provided by the runtime (specific to native code)
        \item<2-> It scans the stack, between the stack pointer at the raising point up to the next trap, in order to collect the names of the detected function frames
          \onslide*<2>{
            \begin{itemize}
              \item And more information, like line number, column number...
            \end{itemize}
          }
        \item<3-> It uses the same technique and information as the Garbage Collector (GC) uses to scan the values live on the stack
          \onslide*<4>{
            \begin{itemize}
              \item \typename{frame\_descr} are at the core of stack unwinding
            \end{itemize}
          }
      \end{itemize}
      \vfill
      \mintocaml[firstline=9,lastline=13,highlightlines=12]{../src/backtrace/backtrace.ml}
      \endminipage
    \end{column}
    \begin{column}{0.5\textwidth}
      \onslide*<1>{\listStashBacktrace[highlightlines=91]{}}
      \onslide*<2>{\listStashBacktrace[highlightlines={91,117-118}]{}}
      \onslide*<3->{\listStashBacktrace[highlightlines={94,106,109-110}]{}}
    \end{column}
  \end{columns}
\end{frame}

\newcommand\listFrameDescr[1][]{
  \listocaml[firstline=111,lastline=124,#1]{../ocaml/asmcomp/emitaux.ml}{asmcomp/emitaux.ml:111}}
\newcommand\listDebugInfo[1][]{
  \listocaml[firstline=38,lastline=49,#1]{../ocaml/lambda/debuginfo.mli}{lambda/debuginfo.mli:38}}

\begin{frame}{Backtraces}{\typename{frame\_descr} and \typename{frame\_debuginfo} in a nutshell}
  \begin{columns}[c]
    \begin{column}{0.5\textwidth}
      \begin{itemize}
        \item<1-> For every return address in an OCaml program, there is a associated \typename{frame\_descr}
        \item<2-> A \typename{frame\_descr} specifies
        \begin{itemize}
          \item<2-> The size of the function stack frame
          \item<3-> Where live values are on the stack (so that the GC can mark them as live and prevent from being swept)
        \end{itemize}
        \item<4-> When compiled with debug information, \typename{frame\_descr} will be linked to a \typename{frame\_debuginfo}
        \begin{itemize}
          \item<5-> Contains information about the position in the source code (file name, line and column number)
        \end{itemize}
      \end{itemize}
    \end{column}
    \begin{column}{0.5\textwidth}
        \onslide*<1>{\listFrameDescr[highlightlines={118-122}]{}}
        \onslide*<2>{\listFrameDescr[highlightlines={120}]{}}
        \onslide*<3>{\listFrameDescr[highlightlines={121}]{}}
        \onslide*<4->{\listFrameDescr[highlightlines={122}]{}}
        \medskip
        \onslide*<1-3>{\listDebugInfo{}}
        \onslide*<4>{\listDebugInfo[highlightlines={38,49}]{}}
        \onslide*<5>{\listDebugInfo[highlightlines={39-46}]{}}
    \end{column}
  \end{columns}
\end{frame}

\begin{frame}{Backtraces}{Scanning the stack with \typename{frame\_descr}}
  \begin{columns}[c]
    \begin{column}{0.5\textwidth}
      \begin{itemize}
        \item Given the return address of the code that raises the exception by calling \funcname{caml\_raise\_exn} (\funcarg{pc} argument of \funcname{caml\_stash\_backtrace})
        \item And the position of the stack pointer (\funcarg{sp} argument of \funcname{caml\_stash\_backtrace})
        \item The frame size given by \typename{frame\_descr}.\funcarg{frame\_size}, allows to find the end of the previous frame
        \item And the return address of the caller of this function
        \bigskip
        \onslide<3->{
        \item Note that traps (here the reddish \funcname{ExnA}, \funcname{ExnB} and \funcname{ExnC} blocks) belong to the frame that installed them, and so the frame size changes when traps are removed
        }
      \end{itemize}
    \end{column}
    \begin{column}{0.5\textwidth}
      \raggedleft
      \onslide*<1>{\providecommand\step{2}\input{diagrams/backtrace/backtrace.tex}}%
      \onslide*<2>{\providecommand\step{1}\input{diagrams/backtrace/scanstack.tex}}%
      \onslide*<3>{\providecommand\step{2}\input{diagrams/backtrace/scanstack.tex}}%
      \onslide*<4>{\providecommand\step{3}\input{diagrams/backtrace/scanstack.tex}}%
      \onslide*<5>{\providecommand\step{4}\input{diagrams/backtrace/scanstack.tex}}%
      \onslide*<6>{\providecommand\step{5}\input{diagrams/backtrace/scanstack.tex}}%
    \end{column}
  \end{columns}
\end{frame}

% Duplicate of previous-previous slide
\begin{frame}{Backtraces}{Continuing on \funcname{caml\_stash\_backtrace}}
  \begin{columns}[c]
    \begin{column}{0.5\textwidth}
      \minipage[c][0.75\textheight][s]{\columnwidth}
      \begin{itemize}
          \color{gray}
        \item A C function provided by the runtime (specific to native code)
        \item It scans the stack, between the stack pointer at the raising point up to the next trap, in order to collect the names of the detected function frames
        \item It uses the same technique and information as the Garbage Collector (GC) uses to scan the values live on the stack
      \end{itemize}
      \begin{itemize}
        \item For each function frame detected, store a pointer to the \typename{frame\_descr} in the \localname{domain\_state->backtrace\_buffer} array, effectively constructing the backtrace
      \end{itemize}
      \onslide<2->{
        \begin{itemize}
            \smallskip
          \item Constructed backtrace:
            \funcname{definitely\_raise},
            \onslide<3->{\funcname{probably\_raise}}
        \end{itemize}
      }
      \vfill
      \mintocaml[firstline=9,lastline=13,highlightlines=12]{../src/backtrace/backtrace.ml}
      \endminipage
    \end{column}
    \begin{column}{0.5\textwidth}
      \raggedleft
      \onslide*<1>{\listStashBacktrace[highlightlines={114-115}]{}}
      \onslide*<2>{\providecommand\step{3}\input{diagrams/backtrace/backtrace.tex}}%
      \onslide*<3>{\providecommand\step{4}\input{diagrams/backtrace/backtrace.tex}}%
    \end{column}
  \end{columns}
\end{frame}

\begin{frame}{Backtraces}{Back to \funcname{caml\_raise\_exn}}
  \begin{columns}[c]
    \begin{column}{0.5\textwidth}
      \minipage[c][0.66\textheight][s]{\columnwidth}
      \begin{itemize}\color{gray}
        \item Checks wheteher exceptions backtrace should be recorded, by checking the \funcarg{domain\_state->backtrace\_active} value
        \item Switches to the C stack to be able to execute C code
        \item Calls \funcname{caml\_stash\_backtrace}, to collect the backtrace up to the next trap
      \end{itemize}
      \onslide*<2->{
        \begin{itemize}
          \item<2-> Executes the exception handler in \funcname{probably\_raise} with \funcname{RESTORE\_EXN\_HANDLER\_OCAML}
            \begin{itemize}
              \item Set \regname{RSP} to \localname{domain\_state->exn\_handler} value
              \item Restore \localname{domain\_state->exn\_handler} from trap
              \item Jump to the handler code with \funcname{ret}
            \end{itemize}
        \end{itemize}
      }
      \vfill
      \onslide*<1>{\mintocaml[firstline=9,lastline=13,highlightlines=12]{../src/backtrace/backtrace.ml}}
      \onslide*<2->{\mintocaml[firstline=15,lastline=21,highlightlines={19-21}]{../src/backtrace/backtrace.ml}}
      \endminipage
    \end{column}
    \begin{column}{0.5\textwidth}
      \centering
      \onslide*<1>{
        \mintc[firstline=190,lastline=192]{../ocaml/runtime/amd64.S}
        \mintc[firstline=234,lastline=237]{../ocaml/runtime/amd64.S}
      }
      \onslide*<2>{
        \mintc[firstline=190,lastline=192,highlightlines={191-192}]{../ocaml/runtime/amd64.S}
        \mintc[firstline=234,lastline=237,highlightlines={235-237}]{../ocaml/runtime/amd64.S}
      }
      \onslide*<1>{\listCamlRaiseExn[highlightlines=720]{}}
      \onslide*<2>{\listCamlRaiseExn[highlightlines=722]{}}
      \onslide*<3->{\providecommand\step{5}\input{diagrams/backtrace/backtrace.tex}}
    \end{column}
  \end{columns}
\end{frame}

\begin{frame}{Backtraces}{\funcname{caml\_probably\_raise}'s handler doesn't catch \typename{ExnC}}
  \begin{columns}[c]
    \begin{column}{0.45\textwidth}
      \minipage[c][0.75\textheight][s]{\columnwidth}
      \begin{itemize}
        \item<1-> \funcname{match \dots with}\footnotemark pattern matching can also match exceptions
        \item<1-> The CMM of \funcname{probably\_raise} shows that this \funcname{match \dots with} is implemented with a pattern matching and a \funcname{try \dots with}
        \item<1-> But this one only matches on \typename{ExnA} exception
        \item<2-> Since the pattern matching fails to catch the exception, \funcname{reraise} is called with our current \typename{ExnC} exception
        \item<3-> Disassembly shows that \funcname{reraise} is actually a call to \funcname{caml\_reraise\_exn}, which is also an assembly function provided by the runtime
      \end{itemize}
      \vfill
      \mintocaml[firstline=15,lastline=21,highlightlines={18-21}]{../src/backtrace/backtrace.ml}
      \endminipage
    \end{column}
    \begin{column}{0.55\textwidth}
      \centering
      \onslide*<1>{\mintcmm[highlightlines={10-18}]{../src/backtrace/camlBacktrace\_\_probably\_raise\_351.cmm}}
      \onslide*<2>{\listcmm[highlightlines=18]{../src/backtrace/camlBacktrace\_\_probably\_raise_351.cmm}{CMM of probably\_raise}}
      \onslide*<3>{\mintobjdump[firstline=5,lastline=35,highlightlines=31]{../src/backtrace/camlBacktrace\_\_probably\_raise_351.objdump}}
    \end{column}
  \end{columns}
  \only<1>{\footnotetext[1]{https://blog.janestreet.com/pattern-matching-and-exception-handling-unite/}}
\end{frame}

\newcommand\listCamlReraiseExn[1][]{
  \listasm[firstline=727,lastline=734,#1]{../ocaml/runtime/amd64.S}{runtime/amd64.S:727}}

\begin{frame}{Backtraces}{Introducing \funcname{caml\_reraise\_exn}}
  \begin{columns}[c]
    \begin{column}{0.5\textwidth}
      \minipage[c][0.66\textheight][s]{\columnwidth}
      \begin{itemize}
        \item<1-> The exception have to be reraised to the next handler
        \item<2-> First, checks if the backtrace should be collected with \funcarg{domain\_state->backtrace\_active}
        \only<3>{
        \begin{itemize}
            \item If not, behaves like a \funcname{raise\_notrace} with \funcname{RESTORE\_EXN\_HANDLER\_OCAML}
        \end{itemize}
        }
        \item<4-> Jumps into \funcname{caml\_raise\_exn}
        \item<5-> Calls \funcname{caml\_stash\_backtrace} again, to scan the next part of the stack: from the trap just pop-ed to the next trap
      \end{itemize}
      \vfill
      \mintocaml[firstline=15,lastline=21,highlightlines={18-21}]{../src/backtrace/backtrace.ml}
      \endminipage
    \end{column}
    \begin{column}{0.5\textwidth}
      \raggedleft
      \onslide*<1>{\listCamlReraiseExn{}}
      \onslide*<2>{\listCamlReraiseExn[highlightlines=729]{}}
      \onslide*<3>{
        \mintc[firstline=190,lastline=192,highlightlines={191-192}]{../ocaml/runtime/amd64.S}
        \mintc[firstline=234,lastline=237,highlightlines={235-237}]{../ocaml/runtime/amd64.S}
        \medskip
        \listCamlReraiseExn[highlightlines=731]{}
      }
      \onslide*<4-5>{
        % TODO refactor with \listCamlReraiseExn
        \mintasm[firstline=727,lastline=734,highlightlines=730]{../ocaml/runtime/amd64.S}
        \medskip
      }
      \onslide*<4>{\listCamlRaiseExn[highlightlines=711]{}}
      \onslide*<5>{\listCamlRaiseExn[highlightlines=720]{}}
    \end{column}
  \end{columns}
\end{frame}

\begin{frame}{Backtraces}{Backtrace collection continues}
  \begin{columns}[c]
    \begin{column}{0.55\textwidth}
      \minipage[c][0.75\textheight][s]{\columnwidth}
      \begin{itemize}
        \item<1-> \funcname{caml\_stash\_backtrace} scans the older part of the stack, up to the next trap
        \item<6-> Controls jumps to the nested exception handler in \funcname{maybe\_raise}, which matches only on \typename{ExnB}
        \item<7-> \funcname{caml\_reraise\_exn} is called again, backtrace collection resumes with \funcname{caml\_stash\_backtrace}
      \end{itemize}
      \medskip
      \begin{itemize}
        \item<2-> Constructed backtrace:
        \begin{itemize}
          \item <2-> \funcname{definitely\_raise}
          \item <2-> \funcname{probably\_raise}
          \item <3-> \funcname{probably\_raise} (re-raise)
          \item <4-> \funcname{maybe\_raise}
          \item <5-> \funcname{main}
          \item <8-> \funcname{main} (re-raise)
        \end{itemize}
      \end{itemize}
      \vfill
      \onslide*<1-5>{\mintocaml[firstline=15,lastline=21]{../src/backtrace/backtrace.ml}}
      \onslide*<6>{\mintocaml[firstline=28,lastline=34,highlightlines=31]{../src/backtrace/backtrace.ml}}
      \onslide*<7->{\mintocaml[firstline=28,lastline=34,highlightlines=33]{../src/backtrace/backtrace.ml}}
      \endminipage
    \end{column}
    \begin{column}{0.45\textwidth}
      \raggedleft
      \onslide*<1>{\providecommand\step{6}\input{diagrams/backtrace/backtrace.tex}}%
      \onslide*<2>{\providecommand\step{6}\input{diagrams/backtrace/backtrace.tex}}%
      \onslide*<3>{\providecommand\step{7}\input{diagrams/backtrace/backtrace.tex}}%
      \onslide*<4>{\providecommand\step{8}\input{diagrams/backtrace/backtrace.tex}}%
      \onslide*<5>{\providecommand\step{9}\input{diagrams/backtrace/backtrace.tex}}%
      \onslide*<6>{\providecommand\step{10}\input{diagrams/backtrace/backtrace.tex}}%
      \onslide*<7>{\providecommand\step{11}\input{diagrams/backtrace/backtrace.tex}}%
      \onslide*<8>{\providecommand\step{12}\input{diagrams/backtrace/backtrace.tex}}%
      \onslide*<9>{\providecommand\step{13}\input{diagrams/backtrace/backtrace.tex}}%
    \end{column}
  \end{columns}
\end{frame}

\begin{frame}{Backtraces}{Printing the backtrace}
  \begin{columns}[c]
    \begin{column}{0.45\textwidth}
      \minipage[c][0.66\textheight][s]{\columnwidth}
      \begin{itemize}
        \item<1-> The exception backtrace can now be printed to \funcarg{stdout} with \funcname{Printexc.print_backtrace}
        \item<2-> First the exception backtrace is retrieved into an OCaml array with \funcname{get_raw_backtrace }
        \item<3-> Which is implemented as the \funcname{caml_get_exception_raw_backtrace} C function in the runtime
        \item<4-> And finally printed with \funcname{print_exception_backtrace}
      \end{itemize}
      \vfill
      \mintocaml[firstline=28,lastline=34,highlightlines=33]{../src/backtrace/backtrace.ml}
      \endminipage
    \end{column}
    \begin{column}{0.55\textwidth}
      \onslide*<1,2>{
        \mintocaml[firstline=100,lastline=101]{../ocaml/stdlib/printexc.ml}
      }
      \only<1>{
        \listocaml[firstline=170,lastline=172]{../ocaml/stdlib/printexc.ml}{stdlib/printexc.ml:170}
      }
      \only<2>{
        \listocaml[firstline=170,lastline=172,highlightlines=172]{../ocaml/stdlib/printexc.ml}{stdlib/printexc.ml:170}
      }
      \only<3>{
        \mintc[firstline=154,lastline=156]{../ocaml/runtime/backtrace.c}
        \listc[firstline=171,lastline=192,highlightlines={184-187}]{../ocaml/runtime/backtrace.c}{runtime/backtrace.c:154}
      }
      \only<4>{
        \listocaml[firstline=155,lastline=165]{../ocaml/stdlib/printexc.ml}{stdlib/printexc.ml:55}
      }
    \end{column}
  \end{columns}
\end{frame}


\subsection{The multiple kinds of raise}
\frameSubsection{}{}

\begin{frame}{Backtraces}{When backtraces are not needed}
  \begin{columns}[c]
    \begin{column}{0.45\textwidth}
      \minipage[c][0.66\textheight][s]{\columnwidth}
      \begin{itemize}
        \item<1-> What if the backtrace for an exception is never needed?
        \item<2-> It's possible to raise the exception explicitly with \funcname{raise\_notrace} to make raising as cheap as possible
        \item<3-> An example of use in the \funcname{dynlink} library of the OCaml compiler
      \end{itemize}
      \vfill
      \onslide<3->{
      \listocaml[firstline=205,lastline=209,highlightlines=207]{../ocaml/otherlibs/dynlink/dynlink\_compilerlibs/misc.ml}{otherlibs/dynlink/dynlink\_compilerlibs/misc.ml:205}
      }
      \endminipage
    \end{column}
    \begin{column}{0.55\textwidth}
    \only<4>{
      \mintcmm[highlightlines=3]{../src/notrace/camlNotrace\_\_fun_347.cmm}
      \listcmm{../src/notrace/camlNotrace\_\_all\_somes\_327.cmm}{all\_somes}
    }
    \onslide*<5->{
      \listobjdump[highlightlines={6-9}]{../src/notrace/camlNotrace\_\_fun_347.objdump}{anonymous function from \funcname{all\_somes}}
    }
    \end{column}
  \end{columns}
\end{frame}

\begin{frame}{Backtraces}{Summary table of the multiple kinds of raise}
  \begin{itemize}
    \item When debugging information is enabled and if the backtrace of an exception isn't needed, it's possible to raise with \funcname{raise\_notrace} for performance
    \item When debugging information is \emph{not} enabled, the compiler optimises \funcname{raise} calls to inline assembly (same effect as using \funcname{raise\_notrace})
  \end{itemize}
  \bigskip
  \centering
  \begin{tabular}{| l | c | c | c | c |}
    \hline
    \parbox{10em}{Debug information \\ (\funcarg{-g} flag)}  & \multicolumn{2}{|c|}{Disabled}                                     & \multicolumn{2}{|c|}{Enabled} \\
    \hline
    Raise kind                                               & \funcname{raise} & \funcname{raise\_notrace}                       & \funcname{raise} & \funcname{raise\_notrace} \\
    \hline
    Compilation                                              & \parbox{8em}{inline assembly\\ (optimization)} & inline assembly   & \funcname{caml\_raise\_exn} & inline assembly \\
    \hline
  \end{tabular}
\end{frame}


%
% Takeaway #5
%
\subsection*{Takeaway \#5}
\frameSubsectionTakeaway{}
\againframe<5>{takeaway}
}%
      \onslide*<2>{\providecommand\step{6}%
% Backtraces
%
\subsection{One advantage of raising with debugging information}
\frameSubsection{}{}

\begin{frame}{Backtraces}{Debugging information \& source code}
  \begin{columns}[c]
    \begin{column}{0.5\textwidth}
      \begin{itemize}
        \item Raising an exception is fun and games, but how do we know where the exception originated? $\rightarrow$ With the backtrace associated with the exception!
        \item In order to get a backtrace from the point where an exception was raised, it's necessary to compile with debugging information enabled (\funcarg{-g} flag for the compiler)
        \item Same example as in the `Nested handlers' section
      \end{itemize}
    \end{column}
    \begin{column}{0.5\textwidth}
      \listocaml[firstline=9,lastline=34]{../src/backtrace/backtrace.ml}{backtrace/backtrace.ml}
    \end{column}
  \end{columns}
\end{frame}

\begin{frame}{Backtraces}{CMM and disassembly of \funcname{definitely\_raise}}
  \begin{columns}[c]
    \begin{column}{0.5\textwidth}
      \minipage[c][0.66\textheight][s]{\columnwidth}
      \begin{itemize}
        \item<1-> An effect of compiling with debugging information (\funcarg{-g}): the CMM no longer uses \funcname{raise\_notrace} to raise the exception but \funcname{raise} instead
        \item<2-> Disassembly shows that CMM \funcname{raise} is actually a call to \funcname{caml\_raise\_exn}, a function in assembler provided by the runtime
      \end{itemize}
      \vfill
      \mintocaml[firstline=9,lastline=13,highlightlines=12]{../src/backtrace/backtrace.ml}
      \endminipage
    \end{column}
    \begin{column}{0.5\textwidth}
      \onslide*<1>{
        \mintcmm[highlightlines=5]{../src/backtrace/camlBacktrace\_\_definitely\_raise\_347.cmm}
      }
      \onslide*<2>{
        \mintobjdump[firstline=2,highlightlines=14]{../src/backtrace/camlBacktrace\_\_definitely\_raise\_347.objdump}
      }
    \end{column}
  \end{columns}
\end{frame}

\begin{frame}{Backtraces}{Introducing \funcname{caml\_raise\_exn}}
  \begin{columns}[c]
    \begin{column}{0.5\textwidth}
      \minipage[c][0.66\textheight][s]{\columnwidth}
      \onslide*<1>{
        \begin{itemize}
          \item Raising the exception is performed using \funcname{caml\_raise\_exn} which is
            \begin{itemize}
              \item An assembly function provided by the runtime
              \item Architecture specific
            \end{itemize}
          \item Let's see what it does differently from \funcname{raise\_notrace}\dots
          \item We are about to raise \typename{ExnC} with \funcname{caml\_raise\_exn} from \funcname{definitely\_raise}
        \end{itemize}
      }
      \onslide*<2>{
        \mintobjdump[firstline=2,highlightlines=14]{../src/backtrace/camlBacktrace\_\_definitely\_raise\_347.objdump}
      }
      \vfill
      \mintocaml[firstline=9,lastline=13,highlightlines=12]{../src/backtrace/backtrace.ml}
      \endminipage
    \end{column}
    \begin{column}{0.5\textwidth}
      \centering
      \providecommand\step{1}\input{diagrams/backtrace/backtrace.tex}
    \end{column}
  \end{columns}
\end{frame}

\begin{frame}{Backtraces}{But before that, we need more fields from \typename{caml\_domain\_state}}
  \begin{columns}
    \begin{column}{0.5\textwidth}
      \begin{itemize}
        \item Remember \typename{caml\_domain\_state} the per-domain runtime state?
        \item Some other members are of interest to us now\dots
        \item \localname{backtrace\_active} is used to know if the runtime should collect backtraces
        \item \localname{backtrace\_active} can be set by
          \begin{itemize}
            \item The \funcarg{b} flag in the \funcname{OCAMLRUNPARAM} environment variable
            \item \mintinline{ocaml}{Printexc.record_backtrace : bool -> unit} \footnotemark
          \end{itemize}
      \end{itemize}
    \end{column}
    \begin{column}{0.5\textwidth}
      \begin{listing}
        \mintc[highlightlines={9-11}]{src/ocaml/domain\_state\_backtrace.h}
        \caption{runtime/caml/domain\_state.h}
      \end{listing}
    \end{column}
  \end{columns}
  \footnotetext[1]{https://v2.ocaml.org/api/Printexc.html}
\end{frame}

\newcommand\listCamlRaiseExn[1][]{
  \listasm[firstline=702,lastline=725,#1]{../ocaml/runtime/amd64.S}{runtime/amd64.S:702}}

\begin{frame}{Backtraces}{Walk through \funcname{caml\_raise\_exn}}
  \begin{columns}[T]
    \begin{column}{0.5\textwidth}
      \begin{itemize}
        \item<1-> First checks whether exceptions backtrace should be recorded
        \item<1-> By checking the \funcarg{domain\_state->backtrace\_active} value
        \item<2-> If not, acts as \funcname{raise\_notrace} with \funcname{RESTORE\_EXN\_HANDLER\_OCAML}
          \begin{itemize}
            \item Set \regname{RSP} to \localname{domain\_state->exn\_handler} value
            \item Restore \localname{domain\_state->exn\_handler} from trap
            \item Jump with \funcname{ret} (same effect as using \funcname{pop} and \funcname{jmp})
          \end{itemize}
        \item<3-> Otherwise, switches to the C stack to be able to execute C code
        \item<4-> Calls \funcname{caml\_stash\_backtrace}, a C function provided by the runtime\ignorespaces
          \onslide<6->{, with
            \begin{itemize}
              \item \funcarg{exn}: exception value
              \item \funcarg{pc}: code address of the raising point
              \item \funcarg{sp}: stack address of the raising function
              \item \funcarg{trapsp}: stack address of the next handler
            \end{itemize}
          }
      \end{itemize}
      \bigskip
      \mintocaml[firstline=9,lastline=13,highlightlines=12]{../src/backtrace/backtrace.ml}
    \end{column}
    \begin{column}{0.5\textwidth}
      \raggedleft
      \onslide*<1,3-4>{
        \mintc[firstline=190,lastline=192]{../ocaml/runtime/amd64.S}
        \mintc[firstline=234,lastline=237]{../ocaml/runtime/amd64.S}
      }
      \onslide*<2>{
        \mintc[firstline=190,lastline=192,highlightlines={191-192}]{../ocaml/runtime/amd64.S}
        \mintc[firstline=234,lastline=237,highlightlines={235-237}]{../ocaml/runtime/amd64.S}
      }
      \onslide*<1>{\listCamlRaiseExn[highlightlines=705]{}}%
      \onslide*<2>{\listCamlRaiseExn[highlightlines={707-708}]{}}%
      \onslide*<3>{\listCamlRaiseExn[highlightlines={712-714}]{}}%
      \onslide*<4>{\listCamlRaiseExn[highlightlines={715-720}]{}}%
      \onslide*<5>{\providecommand\step{1}\input{diagrams/backtrace/backtrace.tex}}%
      \onslide*<6>{\providecommand\step{2}\input{diagrams/backtrace/backtrace.tex}}%
    \end{column}
  \end{columns}
\end{frame}

\newcommand\listStashBacktrace[1][]{
  \listc[firstline=91,lastline=120,#1]{../ocaml/runtime/backtrace\_nat.c}{runtime/backtrace\_nat.c:91}}

\begin{frame}{Backtraces}{Introducing \funcname{caml\_stash\_backtrace}}
  \begin{columns}[c]
    \begin{column}{0.5\textwidth}
      \minipage[c][0.66\textheight][s]{\columnwidth}
      \begin{itemize}
        \item<1-> A C function provided by the runtime (specific to native code)
        \item<2-> It scans the stack, between the stack pointer at the raising point up to the next trap, in order to collect the names of the detected function frames
          \onslide*<2>{
            \begin{itemize}
              \item And more information, like line number, column number...
            \end{itemize}
          }
        \item<3-> It uses the same technique and information as the Garbage Collector (GC) uses to scan the values live on the stack
          \onslide*<4>{
            \begin{itemize}
              \item \typename{frame\_descr} are at the core of stack unwinding
            \end{itemize}
          }
      \end{itemize}
      \vfill
      \mintocaml[firstline=9,lastline=13,highlightlines=12]{../src/backtrace/backtrace.ml}
      \endminipage
    \end{column}
    \begin{column}{0.5\textwidth}
      \onslide*<1>{\listStashBacktrace[highlightlines=91]{}}
      \onslide*<2>{\listStashBacktrace[highlightlines={91,117-118}]{}}
      \onslide*<3->{\listStashBacktrace[highlightlines={94,106,109-110}]{}}
    \end{column}
  \end{columns}
\end{frame}

\newcommand\listFrameDescr[1][]{
  \listocaml[firstline=111,lastline=124,#1]{../ocaml/asmcomp/emitaux.ml}{asmcomp/emitaux.ml:111}}
\newcommand\listDebugInfo[1][]{
  \listocaml[firstline=38,lastline=49,#1]{../ocaml/lambda/debuginfo.mli}{lambda/debuginfo.mli:38}}

\begin{frame}{Backtraces}{\typename{frame\_descr} and \typename{frame\_debuginfo} in a nutshell}
  \begin{columns}[c]
    \begin{column}{0.5\textwidth}
      \begin{itemize}
        \item<1-> For every return address in an OCaml program, there is a associated \typename{frame\_descr}
        \item<2-> A \typename{frame\_descr} specifies
        \begin{itemize}
          \item<2-> The size of the function stack frame
          \item<3-> Where live values are on the stack (so that the GC can mark them as live and prevent from being swept)
        \end{itemize}
        \item<4-> When compiled with debug information, \typename{frame\_descr} will be linked to a \typename{frame\_debuginfo}
        \begin{itemize}
          \item<5-> Contains information about the position in the source code (file name, line and column number)
        \end{itemize}
      \end{itemize}
    \end{column}
    \begin{column}{0.5\textwidth}
        \onslide*<1>{\listFrameDescr[highlightlines={118-122}]{}}
        \onslide*<2>{\listFrameDescr[highlightlines={120}]{}}
        \onslide*<3>{\listFrameDescr[highlightlines={121}]{}}
        \onslide*<4->{\listFrameDescr[highlightlines={122}]{}}
        \medskip
        \onslide*<1-3>{\listDebugInfo{}}
        \onslide*<4>{\listDebugInfo[highlightlines={38,49}]{}}
        \onslide*<5>{\listDebugInfo[highlightlines={39-46}]{}}
    \end{column}
  \end{columns}
\end{frame}

\begin{frame}{Backtraces}{Scanning the stack with \typename{frame\_descr}}
  \begin{columns}[c]
    \begin{column}{0.5\textwidth}
      \begin{itemize}
        \item Given the return address of the code that raises the exception by calling \funcname{caml\_raise\_exn} (\funcarg{pc} argument of \funcname{caml\_stash\_backtrace})
        \item And the position of the stack pointer (\funcarg{sp} argument of \funcname{caml\_stash\_backtrace})
        \item The frame size given by \typename{frame\_descr}.\funcarg{frame\_size}, allows to find the end of the previous frame
        \item And the return address of the caller of this function
        \bigskip
        \onslide<3->{
        \item Note that traps (here the reddish \funcname{ExnA}, \funcname{ExnB} and \funcname{ExnC} blocks) belong to the frame that installed them, and so the frame size changes when traps are removed
        }
      \end{itemize}
    \end{column}
    \begin{column}{0.5\textwidth}
      \raggedleft
      \onslide*<1>{\providecommand\step{2}\input{diagrams/backtrace/backtrace.tex}}%
      \onslide*<2>{\providecommand\step{1}\input{diagrams/backtrace/scanstack.tex}}%
      \onslide*<3>{\providecommand\step{2}\input{diagrams/backtrace/scanstack.tex}}%
      \onslide*<4>{\providecommand\step{3}\input{diagrams/backtrace/scanstack.tex}}%
      \onslide*<5>{\providecommand\step{4}\input{diagrams/backtrace/scanstack.tex}}%
      \onslide*<6>{\providecommand\step{5}\input{diagrams/backtrace/scanstack.tex}}%
    \end{column}
  \end{columns}
\end{frame}

% Duplicate of previous-previous slide
\begin{frame}{Backtraces}{Continuing on \funcname{caml\_stash\_backtrace}}
  \begin{columns}[c]
    \begin{column}{0.5\textwidth}
      \minipage[c][0.75\textheight][s]{\columnwidth}
      \begin{itemize}
          \color{gray}
        \item A C function provided by the runtime (specific to native code)
        \item It scans the stack, between the stack pointer at the raising point up to the next trap, in order to collect the names of the detected function frames
        \item It uses the same technique and information as the Garbage Collector (GC) uses to scan the values live on the stack
      \end{itemize}
      \begin{itemize}
        \item For each function frame detected, store a pointer to the \typename{frame\_descr} in the \localname{domain\_state->backtrace\_buffer} array, effectively constructing the backtrace
      \end{itemize}
      \onslide<2->{
        \begin{itemize}
            \smallskip
          \item Constructed backtrace:
            \funcname{definitely\_raise},
            \onslide<3->{\funcname{probably\_raise}}
        \end{itemize}
      }
      \vfill
      \mintocaml[firstline=9,lastline=13,highlightlines=12]{../src/backtrace/backtrace.ml}
      \endminipage
    \end{column}
    \begin{column}{0.5\textwidth}
      \raggedleft
      \onslide*<1>{\listStashBacktrace[highlightlines={114-115}]{}}
      \onslide*<2>{\providecommand\step{3}\input{diagrams/backtrace/backtrace.tex}}%
      \onslide*<3>{\providecommand\step{4}\input{diagrams/backtrace/backtrace.tex}}%
    \end{column}
  \end{columns}
\end{frame}

\begin{frame}{Backtraces}{Back to \funcname{caml\_raise\_exn}}
  \begin{columns}[c]
    \begin{column}{0.5\textwidth}
      \minipage[c][0.66\textheight][s]{\columnwidth}
      \begin{itemize}\color{gray}
        \item Checks wheteher exceptions backtrace should be recorded, by checking the \funcarg{domain\_state->backtrace\_active} value
        \item Switches to the C stack to be able to execute C code
        \item Calls \funcname{caml\_stash\_backtrace}, to collect the backtrace up to the next trap
      \end{itemize}
      \onslide*<2->{
        \begin{itemize}
          \item<2-> Executes the exception handler in \funcname{probably\_raise} with \funcname{RESTORE\_EXN\_HANDLER\_OCAML}
            \begin{itemize}
              \item Set \regname{RSP} to \localname{domain\_state->exn\_handler} value
              \item Restore \localname{domain\_state->exn\_handler} from trap
              \item Jump to the handler code with \funcname{ret}
            \end{itemize}
        \end{itemize}
      }
      \vfill
      \onslide*<1>{\mintocaml[firstline=9,lastline=13,highlightlines=12]{../src/backtrace/backtrace.ml}}
      \onslide*<2->{\mintocaml[firstline=15,lastline=21,highlightlines={19-21}]{../src/backtrace/backtrace.ml}}
      \endminipage
    \end{column}
    \begin{column}{0.5\textwidth}
      \centering
      \onslide*<1>{
        \mintc[firstline=190,lastline=192]{../ocaml/runtime/amd64.S}
        \mintc[firstline=234,lastline=237]{../ocaml/runtime/amd64.S}
      }
      \onslide*<2>{
        \mintc[firstline=190,lastline=192,highlightlines={191-192}]{../ocaml/runtime/amd64.S}
        \mintc[firstline=234,lastline=237,highlightlines={235-237}]{../ocaml/runtime/amd64.S}
      }
      \onslide*<1>{\listCamlRaiseExn[highlightlines=720]{}}
      \onslide*<2>{\listCamlRaiseExn[highlightlines=722]{}}
      \onslide*<3->{\providecommand\step{5}\input{diagrams/backtrace/backtrace.tex}}
    \end{column}
  \end{columns}
\end{frame}

\begin{frame}{Backtraces}{\funcname{caml\_probably\_raise}'s handler doesn't catch \typename{ExnC}}
  \begin{columns}[c]
    \begin{column}{0.45\textwidth}
      \minipage[c][0.75\textheight][s]{\columnwidth}
      \begin{itemize}
        \item<1-> \funcname{match \dots with}\footnotemark pattern matching can also match exceptions
        \item<1-> The CMM of \funcname{probably\_raise} shows that this \funcname{match \dots with} is implemented with a pattern matching and a \funcname{try \dots with}
        \item<1-> But this one only matches on \typename{ExnA} exception
        \item<2-> Since the pattern matching fails to catch the exception, \funcname{reraise} is called with our current \typename{ExnC} exception
        \item<3-> Disassembly shows that \funcname{reraise} is actually a call to \funcname{caml\_reraise\_exn}, which is also an assembly function provided by the runtime
      \end{itemize}
      \vfill
      \mintocaml[firstline=15,lastline=21,highlightlines={18-21}]{../src/backtrace/backtrace.ml}
      \endminipage
    \end{column}
    \begin{column}{0.55\textwidth}
      \centering
      \onslide*<1>{\mintcmm[highlightlines={10-18}]{../src/backtrace/camlBacktrace\_\_probably\_raise\_351.cmm}}
      \onslide*<2>{\listcmm[highlightlines=18]{../src/backtrace/camlBacktrace\_\_probably\_raise_351.cmm}{CMM of probably\_raise}}
      \onslide*<3>{\mintobjdump[firstline=5,lastline=35,highlightlines=31]{../src/backtrace/camlBacktrace\_\_probably\_raise_351.objdump}}
    \end{column}
  \end{columns}
  \only<1>{\footnotetext[1]{https://blog.janestreet.com/pattern-matching-and-exception-handling-unite/}}
\end{frame}

\newcommand\listCamlReraiseExn[1][]{
  \listasm[firstline=727,lastline=734,#1]{../ocaml/runtime/amd64.S}{runtime/amd64.S:727}}

\begin{frame}{Backtraces}{Introducing \funcname{caml\_reraise\_exn}}
  \begin{columns}[c]
    \begin{column}{0.5\textwidth}
      \minipage[c][0.66\textheight][s]{\columnwidth}
      \begin{itemize}
        \item<1-> The exception have to be reraised to the next handler
        \item<2-> First, checks if the backtrace should be collected with \funcarg{domain\_state->backtrace\_active}
        \only<3>{
        \begin{itemize}
            \item If not, behaves like a \funcname{raise\_notrace} with \funcname{RESTORE\_EXN\_HANDLER\_OCAML}
        \end{itemize}
        }
        \item<4-> Jumps into \funcname{caml\_raise\_exn}
        \item<5-> Calls \funcname{caml\_stash\_backtrace} again, to scan the next part of the stack: from the trap just pop-ed to the next trap
      \end{itemize}
      \vfill
      \mintocaml[firstline=15,lastline=21,highlightlines={18-21}]{../src/backtrace/backtrace.ml}
      \endminipage
    \end{column}
    \begin{column}{0.5\textwidth}
      \raggedleft
      \onslide*<1>{\listCamlReraiseExn{}}
      \onslide*<2>{\listCamlReraiseExn[highlightlines=729]{}}
      \onslide*<3>{
        \mintc[firstline=190,lastline=192,highlightlines={191-192}]{../ocaml/runtime/amd64.S}
        \mintc[firstline=234,lastline=237,highlightlines={235-237}]{../ocaml/runtime/amd64.S}
        \medskip
        \listCamlReraiseExn[highlightlines=731]{}
      }
      \onslide*<4-5>{
        % TODO refactor with \listCamlReraiseExn
        \mintasm[firstline=727,lastline=734,highlightlines=730]{../ocaml/runtime/amd64.S}
        \medskip
      }
      \onslide*<4>{\listCamlRaiseExn[highlightlines=711]{}}
      \onslide*<5>{\listCamlRaiseExn[highlightlines=720]{}}
    \end{column}
  \end{columns}
\end{frame}

\begin{frame}{Backtraces}{Backtrace collection continues}
  \begin{columns}[c]
    \begin{column}{0.55\textwidth}
      \minipage[c][0.75\textheight][s]{\columnwidth}
      \begin{itemize}
        \item<1-> \funcname{caml\_stash\_backtrace} scans the older part of the stack, up to the next trap
        \item<6-> Controls jumps to the nested exception handler in \funcname{maybe\_raise}, which matches only on \typename{ExnB}
        \item<7-> \funcname{caml\_reraise\_exn} is called again, backtrace collection resumes with \funcname{caml\_stash\_backtrace}
      \end{itemize}
      \medskip
      \begin{itemize}
        \item<2-> Constructed backtrace:
        \begin{itemize}
          \item <2-> \funcname{definitely\_raise}
          \item <2-> \funcname{probably\_raise}
          \item <3-> \funcname{probably\_raise} (re-raise)
          \item <4-> \funcname{maybe\_raise}
          \item <5-> \funcname{main}
          \item <8-> \funcname{main} (re-raise)
        \end{itemize}
      \end{itemize}
      \vfill
      \onslide*<1-5>{\mintocaml[firstline=15,lastline=21]{../src/backtrace/backtrace.ml}}
      \onslide*<6>{\mintocaml[firstline=28,lastline=34,highlightlines=31]{../src/backtrace/backtrace.ml}}
      \onslide*<7->{\mintocaml[firstline=28,lastline=34,highlightlines=33]{../src/backtrace/backtrace.ml}}
      \endminipage
    \end{column}
    \begin{column}{0.45\textwidth}
      \raggedleft
      \onslide*<1>{\providecommand\step{6}\input{diagrams/backtrace/backtrace.tex}}%
      \onslide*<2>{\providecommand\step{6}\input{diagrams/backtrace/backtrace.tex}}%
      \onslide*<3>{\providecommand\step{7}\input{diagrams/backtrace/backtrace.tex}}%
      \onslide*<4>{\providecommand\step{8}\input{diagrams/backtrace/backtrace.tex}}%
      \onslide*<5>{\providecommand\step{9}\input{diagrams/backtrace/backtrace.tex}}%
      \onslide*<6>{\providecommand\step{10}\input{diagrams/backtrace/backtrace.tex}}%
      \onslide*<7>{\providecommand\step{11}\input{diagrams/backtrace/backtrace.tex}}%
      \onslide*<8>{\providecommand\step{12}\input{diagrams/backtrace/backtrace.tex}}%
      \onslide*<9>{\providecommand\step{13}\input{diagrams/backtrace/backtrace.tex}}%
    \end{column}
  \end{columns}
\end{frame}

\begin{frame}{Backtraces}{Printing the backtrace}
  \begin{columns}[c]
    \begin{column}{0.45\textwidth}
      \minipage[c][0.66\textheight][s]{\columnwidth}
      \begin{itemize}
        \item<1-> The exception backtrace can now be printed to \funcarg{stdout} with \funcname{Printexc.print_backtrace}
        \item<2-> First the exception backtrace is retrieved into an OCaml array with \funcname{get_raw_backtrace }
        \item<3-> Which is implemented as the \funcname{caml_get_exception_raw_backtrace} C function in the runtime
        \item<4-> And finally printed with \funcname{print_exception_backtrace}
      \end{itemize}
      \vfill
      \mintocaml[firstline=28,lastline=34,highlightlines=33]{../src/backtrace/backtrace.ml}
      \endminipage
    \end{column}
    \begin{column}{0.55\textwidth}
      \onslide*<1,2>{
        \mintocaml[firstline=100,lastline=101]{../ocaml/stdlib/printexc.ml}
      }
      \only<1>{
        \listocaml[firstline=170,lastline=172]{../ocaml/stdlib/printexc.ml}{stdlib/printexc.ml:170}
      }
      \only<2>{
        \listocaml[firstline=170,lastline=172,highlightlines=172]{../ocaml/stdlib/printexc.ml}{stdlib/printexc.ml:170}
      }
      \only<3>{
        \mintc[firstline=154,lastline=156]{../ocaml/runtime/backtrace.c}
        \listc[firstline=171,lastline=192,highlightlines={184-187}]{../ocaml/runtime/backtrace.c}{runtime/backtrace.c:154}
      }
      \only<4>{
        \listocaml[firstline=155,lastline=165]{../ocaml/stdlib/printexc.ml}{stdlib/printexc.ml:55}
      }
    \end{column}
  \end{columns}
\end{frame}


\subsection{The multiple kinds of raise}
\frameSubsection{}{}

\begin{frame}{Backtraces}{When backtraces are not needed}
  \begin{columns}[c]
    \begin{column}{0.45\textwidth}
      \minipage[c][0.66\textheight][s]{\columnwidth}
      \begin{itemize}
        \item<1-> What if the backtrace for an exception is never needed?
        \item<2-> It's possible to raise the exception explicitly with \funcname{raise\_notrace} to make raising as cheap as possible
        \item<3-> An example of use in the \funcname{dynlink} library of the OCaml compiler
      \end{itemize}
      \vfill
      \onslide<3->{
      \listocaml[firstline=205,lastline=209,highlightlines=207]{../ocaml/otherlibs/dynlink/dynlink\_compilerlibs/misc.ml}{otherlibs/dynlink/dynlink\_compilerlibs/misc.ml:205}
      }
      \endminipage
    \end{column}
    \begin{column}{0.55\textwidth}
    \only<4>{
      \mintcmm[highlightlines=3]{../src/notrace/camlNotrace\_\_fun_347.cmm}
      \listcmm{../src/notrace/camlNotrace\_\_all\_somes\_327.cmm}{all\_somes}
    }
    \onslide*<5->{
      \listobjdump[highlightlines={6-9}]{../src/notrace/camlNotrace\_\_fun_347.objdump}{anonymous function from \funcname{all\_somes}}
    }
    \end{column}
  \end{columns}
\end{frame}

\begin{frame}{Backtraces}{Summary table of the multiple kinds of raise}
  \begin{itemize}
    \item When debugging information is enabled and if the backtrace of an exception isn't needed, it's possible to raise with \funcname{raise\_notrace} for performance
    \item When debugging information is \emph{not} enabled, the compiler optimises \funcname{raise} calls to inline assembly (same effect as using \funcname{raise\_notrace})
  \end{itemize}
  \bigskip
  \centering
  \begin{tabular}{| l | c | c | c | c |}
    \hline
    \parbox{10em}{Debug information \\ (\funcarg{-g} flag)}  & \multicolumn{2}{|c|}{Disabled}                                     & \multicolumn{2}{|c|}{Enabled} \\
    \hline
    Raise kind                                               & \funcname{raise} & \funcname{raise\_notrace}                       & \funcname{raise} & \funcname{raise\_notrace} \\
    \hline
    Compilation                                              & \parbox{8em}{inline assembly\\ (optimization)} & inline assembly   & \funcname{caml\_raise\_exn} & inline assembly \\
    \hline
  \end{tabular}
\end{frame}


%
% Takeaway #5
%
\subsection*{Takeaway \#5}
\frameSubsectionTakeaway{}
\againframe<5>{takeaway}
}%
      \onslide*<3>{\providecommand\step{7}%
% Backtraces
%
\subsection{One advantage of raising with debugging information}
\frameSubsection{}{}

\begin{frame}{Backtraces}{Debugging information \& source code}
  \begin{columns}[c]
    \begin{column}{0.5\textwidth}
      \begin{itemize}
        \item Raising an exception is fun and games, but how do we know where the exception originated? $\rightarrow$ With the backtrace associated with the exception!
        \item In order to get a backtrace from the point where an exception was raised, it's necessary to compile with debugging information enabled (\funcarg{-g} flag for the compiler)
        \item Same example as in the `Nested handlers' section
      \end{itemize}
    \end{column}
    \begin{column}{0.5\textwidth}
      \listocaml[firstline=9,lastline=34]{../src/backtrace/backtrace.ml}{backtrace/backtrace.ml}
    \end{column}
  \end{columns}
\end{frame}

\begin{frame}{Backtraces}{CMM and disassembly of \funcname{definitely\_raise}}
  \begin{columns}[c]
    \begin{column}{0.5\textwidth}
      \minipage[c][0.66\textheight][s]{\columnwidth}
      \begin{itemize}
        \item<1-> An effect of compiling with debugging information (\funcarg{-g}): the CMM no longer uses \funcname{raise\_notrace} to raise the exception but \funcname{raise} instead
        \item<2-> Disassembly shows that CMM \funcname{raise} is actually a call to \funcname{caml\_raise\_exn}, a function in assembler provided by the runtime
      \end{itemize}
      \vfill
      \mintocaml[firstline=9,lastline=13,highlightlines=12]{../src/backtrace/backtrace.ml}
      \endminipage
    \end{column}
    \begin{column}{0.5\textwidth}
      \onslide*<1>{
        \mintcmm[highlightlines=5]{../src/backtrace/camlBacktrace\_\_definitely\_raise\_347.cmm}
      }
      \onslide*<2>{
        \mintobjdump[firstline=2,highlightlines=14]{../src/backtrace/camlBacktrace\_\_definitely\_raise\_347.objdump}
      }
    \end{column}
  \end{columns}
\end{frame}

\begin{frame}{Backtraces}{Introducing \funcname{caml\_raise\_exn}}
  \begin{columns}[c]
    \begin{column}{0.5\textwidth}
      \minipage[c][0.66\textheight][s]{\columnwidth}
      \onslide*<1>{
        \begin{itemize}
          \item Raising the exception is performed using \funcname{caml\_raise\_exn} which is
            \begin{itemize}
              \item An assembly function provided by the runtime
              \item Architecture specific
            \end{itemize}
          \item Let's see what it does differently from \funcname{raise\_notrace}\dots
          \item We are about to raise \typename{ExnC} with \funcname{caml\_raise\_exn} from \funcname{definitely\_raise}
        \end{itemize}
      }
      \onslide*<2>{
        \mintobjdump[firstline=2,highlightlines=14]{../src/backtrace/camlBacktrace\_\_definitely\_raise\_347.objdump}
      }
      \vfill
      \mintocaml[firstline=9,lastline=13,highlightlines=12]{../src/backtrace/backtrace.ml}
      \endminipage
    \end{column}
    \begin{column}{0.5\textwidth}
      \centering
      \providecommand\step{1}\input{diagrams/backtrace/backtrace.tex}
    \end{column}
  \end{columns}
\end{frame}

\begin{frame}{Backtraces}{But before that, we need more fields from \typename{caml\_domain\_state}}
  \begin{columns}
    \begin{column}{0.5\textwidth}
      \begin{itemize}
        \item Remember \typename{caml\_domain\_state} the per-domain runtime state?
        \item Some other members are of interest to us now\dots
        \item \localname{backtrace\_active} is used to know if the runtime should collect backtraces
        \item \localname{backtrace\_active} can be set by
          \begin{itemize}
            \item The \funcarg{b} flag in the \funcname{OCAMLRUNPARAM} environment variable
            \item \mintinline{ocaml}{Printexc.record_backtrace : bool -> unit} \footnotemark
          \end{itemize}
      \end{itemize}
    \end{column}
    \begin{column}{0.5\textwidth}
      \begin{listing}
        \mintc[highlightlines={9-11}]{src/ocaml/domain\_state\_backtrace.h}
        \caption{runtime/caml/domain\_state.h}
      \end{listing}
    \end{column}
  \end{columns}
  \footnotetext[1]{https://v2.ocaml.org/api/Printexc.html}
\end{frame}

\newcommand\listCamlRaiseExn[1][]{
  \listasm[firstline=702,lastline=725,#1]{../ocaml/runtime/amd64.S}{runtime/amd64.S:702}}

\begin{frame}{Backtraces}{Walk through \funcname{caml\_raise\_exn}}
  \begin{columns}[T]
    \begin{column}{0.5\textwidth}
      \begin{itemize}
        \item<1-> First checks whether exceptions backtrace should be recorded
        \item<1-> By checking the \funcarg{domain\_state->backtrace\_active} value
        \item<2-> If not, acts as \funcname{raise\_notrace} with \funcname{RESTORE\_EXN\_HANDLER\_OCAML}
          \begin{itemize}
            \item Set \regname{RSP} to \localname{domain\_state->exn\_handler} value
            \item Restore \localname{domain\_state->exn\_handler} from trap
            \item Jump with \funcname{ret} (same effect as using \funcname{pop} and \funcname{jmp})
          \end{itemize}
        \item<3-> Otherwise, switches to the C stack to be able to execute C code
        \item<4-> Calls \funcname{caml\_stash\_backtrace}, a C function provided by the runtime\ignorespaces
          \onslide<6->{, with
            \begin{itemize}
              \item \funcarg{exn}: exception value
              \item \funcarg{pc}: code address of the raising point
              \item \funcarg{sp}: stack address of the raising function
              \item \funcarg{trapsp}: stack address of the next handler
            \end{itemize}
          }
      \end{itemize}
      \bigskip
      \mintocaml[firstline=9,lastline=13,highlightlines=12]{../src/backtrace/backtrace.ml}
    \end{column}
    \begin{column}{0.5\textwidth}
      \raggedleft
      \onslide*<1,3-4>{
        \mintc[firstline=190,lastline=192]{../ocaml/runtime/amd64.S}
        \mintc[firstline=234,lastline=237]{../ocaml/runtime/amd64.S}
      }
      \onslide*<2>{
        \mintc[firstline=190,lastline=192,highlightlines={191-192}]{../ocaml/runtime/amd64.S}
        \mintc[firstline=234,lastline=237,highlightlines={235-237}]{../ocaml/runtime/amd64.S}
      }
      \onslide*<1>{\listCamlRaiseExn[highlightlines=705]{}}%
      \onslide*<2>{\listCamlRaiseExn[highlightlines={707-708}]{}}%
      \onslide*<3>{\listCamlRaiseExn[highlightlines={712-714}]{}}%
      \onslide*<4>{\listCamlRaiseExn[highlightlines={715-720}]{}}%
      \onslide*<5>{\providecommand\step{1}\input{diagrams/backtrace/backtrace.tex}}%
      \onslide*<6>{\providecommand\step{2}\input{diagrams/backtrace/backtrace.tex}}%
    \end{column}
  \end{columns}
\end{frame}

\newcommand\listStashBacktrace[1][]{
  \listc[firstline=91,lastline=120,#1]{../ocaml/runtime/backtrace\_nat.c}{runtime/backtrace\_nat.c:91}}

\begin{frame}{Backtraces}{Introducing \funcname{caml\_stash\_backtrace}}
  \begin{columns}[c]
    \begin{column}{0.5\textwidth}
      \minipage[c][0.66\textheight][s]{\columnwidth}
      \begin{itemize}
        \item<1-> A C function provided by the runtime (specific to native code)
        \item<2-> It scans the stack, between the stack pointer at the raising point up to the next trap, in order to collect the names of the detected function frames
          \onslide*<2>{
            \begin{itemize}
              \item And more information, like line number, column number...
            \end{itemize}
          }
        \item<3-> It uses the same technique and information as the Garbage Collector (GC) uses to scan the values live on the stack
          \onslide*<4>{
            \begin{itemize}
              \item \typename{frame\_descr} are at the core of stack unwinding
            \end{itemize}
          }
      \end{itemize}
      \vfill
      \mintocaml[firstline=9,lastline=13,highlightlines=12]{../src/backtrace/backtrace.ml}
      \endminipage
    \end{column}
    \begin{column}{0.5\textwidth}
      \onslide*<1>{\listStashBacktrace[highlightlines=91]{}}
      \onslide*<2>{\listStashBacktrace[highlightlines={91,117-118}]{}}
      \onslide*<3->{\listStashBacktrace[highlightlines={94,106,109-110}]{}}
    \end{column}
  \end{columns}
\end{frame}

\newcommand\listFrameDescr[1][]{
  \listocaml[firstline=111,lastline=124,#1]{../ocaml/asmcomp/emitaux.ml}{asmcomp/emitaux.ml:111}}
\newcommand\listDebugInfo[1][]{
  \listocaml[firstline=38,lastline=49,#1]{../ocaml/lambda/debuginfo.mli}{lambda/debuginfo.mli:38}}

\begin{frame}{Backtraces}{\typename{frame\_descr} and \typename{frame\_debuginfo} in a nutshell}
  \begin{columns}[c]
    \begin{column}{0.5\textwidth}
      \begin{itemize}
        \item<1-> For every return address in an OCaml program, there is a associated \typename{frame\_descr}
        \item<2-> A \typename{frame\_descr} specifies
        \begin{itemize}
          \item<2-> The size of the function stack frame
          \item<3-> Where live values are on the stack (so that the GC can mark them as live and prevent from being swept)
        \end{itemize}
        \item<4-> When compiled with debug information, \typename{frame\_descr} will be linked to a \typename{frame\_debuginfo}
        \begin{itemize}
          \item<5-> Contains information about the position in the source code (file name, line and column number)
        \end{itemize}
      \end{itemize}
    \end{column}
    \begin{column}{0.5\textwidth}
        \onslide*<1>{\listFrameDescr[highlightlines={118-122}]{}}
        \onslide*<2>{\listFrameDescr[highlightlines={120}]{}}
        \onslide*<3>{\listFrameDescr[highlightlines={121}]{}}
        \onslide*<4->{\listFrameDescr[highlightlines={122}]{}}
        \medskip
        \onslide*<1-3>{\listDebugInfo{}}
        \onslide*<4>{\listDebugInfo[highlightlines={38,49}]{}}
        \onslide*<5>{\listDebugInfo[highlightlines={39-46}]{}}
    \end{column}
  \end{columns}
\end{frame}

\begin{frame}{Backtraces}{Scanning the stack with \typename{frame\_descr}}
  \begin{columns}[c]
    \begin{column}{0.5\textwidth}
      \begin{itemize}
        \item Given the return address of the code that raises the exception by calling \funcname{caml\_raise\_exn} (\funcarg{pc} argument of \funcname{caml\_stash\_backtrace})
        \item And the position of the stack pointer (\funcarg{sp} argument of \funcname{caml\_stash\_backtrace})
        \item The frame size given by \typename{frame\_descr}.\funcarg{frame\_size}, allows to find the end of the previous frame
        \item And the return address of the caller of this function
        \bigskip
        \onslide<3->{
        \item Note that traps (here the reddish \funcname{ExnA}, \funcname{ExnB} and \funcname{ExnC} blocks) belong to the frame that installed them, and so the frame size changes when traps are removed
        }
      \end{itemize}
    \end{column}
    \begin{column}{0.5\textwidth}
      \raggedleft
      \onslide*<1>{\providecommand\step{2}\input{diagrams/backtrace/backtrace.tex}}%
      \onslide*<2>{\providecommand\step{1}\input{diagrams/backtrace/scanstack.tex}}%
      \onslide*<3>{\providecommand\step{2}\input{diagrams/backtrace/scanstack.tex}}%
      \onslide*<4>{\providecommand\step{3}\input{diagrams/backtrace/scanstack.tex}}%
      \onslide*<5>{\providecommand\step{4}\input{diagrams/backtrace/scanstack.tex}}%
      \onslide*<6>{\providecommand\step{5}\input{diagrams/backtrace/scanstack.tex}}%
    \end{column}
  \end{columns}
\end{frame}

% Duplicate of previous-previous slide
\begin{frame}{Backtraces}{Continuing on \funcname{caml\_stash\_backtrace}}
  \begin{columns}[c]
    \begin{column}{0.5\textwidth}
      \minipage[c][0.75\textheight][s]{\columnwidth}
      \begin{itemize}
          \color{gray}
        \item A C function provided by the runtime (specific to native code)
        \item It scans the stack, between the stack pointer at the raising point up to the next trap, in order to collect the names of the detected function frames
        \item It uses the same technique and information as the Garbage Collector (GC) uses to scan the values live on the stack
      \end{itemize}
      \begin{itemize}
        \item For each function frame detected, store a pointer to the \typename{frame\_descr} in the \localname{domain\_state->backtrace\_buffer} array, effectively constructing the backtrace
      \end{itemize}
      \onslide<2->{
        \begin{itemize}
            \smallskip
          \item Constructed backtrace:
            \funcname{definitely\_raise},
            \onslide<3->{\funcname{probably\_raise}}
        \end{itemize}
      }
      \vfill
      \mintocaml[firstline=9,lastline=13,highlightlines=12]{../src/backtrace/backtrace.ml}
      \endminipage
    \end{column}
    \begin{column}{0.5\textwidth}
      \raggedleft
      \onslide*<1>{\listStashBacktrace[highlightlines={114-115}]{}}
      \onslide*<2>{\providecommand\step{3}\input{diagrams/backtrace/backtrace.tex}}%
      \onslide*<3>{\providecommand\step{4}\input{diagrams/backtrace/backtrace.tex}}%
    \end{column}
  \end{columns}
\end{frame}

\begin{frame}{Backtraces}{Back to \funcname{caml\_raise\_exn}}
  \begin{columns}[c]
    \begin{column}{0.5\textwidth}
      \minipage[c][0.66\textheight][s]{\columnwidth}
      \begin{itemize}\color{gray}
        \item Checks wheteher exceptions backtrace should be recorded, by checking the \funcarg{domain\_state->backtrace\_active} value
        \item Switches to the C stack to be able to execute C code
        \item Calls \funcname{caml\_stash\_backtrace}, to collect the backtrace up to the next trap
      \end{itemize}
      \onslide*<2->{
        \begin{itemize}
          \item<2-> Executes the exception handler in \funcname{probably\_raise} with \funcname{RESTORE\_EXN\_HANDLER\_OCAML}
            \begin{itemize}
              \item Set \regname{RSP} to \localname{domain\_state->exn\_handler} value
              \item Restore \localname{domain\_state->exn\_handler} from trap
              \item Jump to the handler code with \funcname{ret}
            \end{itemize}
        \end{itemize}
      }
      \vfill
      \onslide*<1>{\mintocaml[firstline=9,lastline=13,highlightlines=12]{../src/backtrace/backtrace.ml}}
      \onslide*<2->{\mintocaml[firstline=15,lastline=21,highlightlines={19-21}]{../src/backtrace/backtrace.ml}}
      \endminipage
    \end{column}
    \begin{column}{0.5\textwidth}
      \centering
      \onslide*<1>{
        \mintc[firstline=190,lastline=192]{../ocaml/runtime/amd64.S}
        \mintc[firstline=234,lastline=237]{../ocaml/runtime/amd64.S}
      }
      \onslide*<2>{
        \mintc[firstline=190,lastline=192,highlightlines={191-192}]{../ocaml/runtime/amd64.S}
        \mintc[firstline=234,lastline=237,highlightlines={235-237}]{../ocaml/runtime/amd64.S}
      }
      \onslide*<1>{\listCamlRaiseExn[highlightlines=720]{}}
      \onslide*<2>{\listCamlRaiseExn[highlightlines=722]{}}
      \onslide*<3->{\providecommand\step{5}\input{diagrams/backtrace/backtrace.tex}}
    \end{column}
  \end{columns}
\end{frame}

\begin{frame}{Backtraces}{\funcname{caml\_probably\_raise}'s handler doesn't catch \typename{ExnC}}
  \begin{columns}[c]
    \begin{column}{0.45\textwidth}
      \minipage[c][0.75\textheight][s]{\columnwidth}
      \begin{itemize}
        \item<1-> \funcname{match \dots with}\footnotemark pattern matching can also match exceptions
        \item<1-> The CMM of \funcname{probably\_raise} shows that this \funcname{match \dots with} is implemented with a pattern matching and a \funcname{try \dots with}
        \item<1-> But this one only matches on \typename{ExnA} exception
        \item<2-> Since the pattern matching fails to catch the exception, \funcname{reraise} is called with our current \typename{ExnC} exception
        \item<3-> Disassembly shows that \funcname{reraise} is actually a call to \funcname{caml\_reraise\_exn}, which is also an assembly function provided by the runtime
      \end{itemize}
      \vfill
      \mintocaml[firstline=15,lastline=21,highlightlines={18-21}]{../src/backtrace/backtrace.ml}
      \endminipage
    \end{column}
    \begin{column}{0.55\textwidth}
      \centering
      \onslide*<1>{\mintcmm[highlightlines={10-18}]{../src/backtrace/camlBacktrace\_\_probably\_raise\_351.cmm}}
      \onslide*<2>{\listcmm[highlightlines=18]{../src/backtrace/camlBacktrace\_\_probably\_raise_351.cmm}{CMM of probably\_raise}}
      \onslide*<3>{\mintobjdump[firstline=5,lastline=35,highlightlines=31]{../src/backtrace/camlBacktrace\_\_probably\_raise_351.objdump}}
    \end{column}
  \end{columns}
  \only<1>{\footnotetext[1]{https://blog.janestreet.com/pattern-matching-and-exception-handling-unite/}}
\end{frame}

\newcommand\listCamlReraiseExn[1][]{
  \listasm[firstline=727,lastline=734,#1]{../ocaml/runtime/amd64.S}{runtime/amd64.S:727}}

\begin{frame}{Backtraces}{Introducing \funcname{caml\_reraise\_exn}}
  \begin{columns}[c]
    \begin{column}{0.5\textwidth}
      \minipage[c][0.66\textheight][s]{\columnwidth}
      \begin{itemize}
        \item<1-> The exception have to be reraised to the next handler
        \item<2-> First, checks if the backtrace should be collected with \funcarg{domain\_state->backtrace\_active}
        \only<3>{
        \begin{itemize}
            \item If not, behaves like a \funcname{raise\_notrace} with \funcname{RESTORE\_EXN\_HANDLER\_OCAML}
        \end{itemize}
        }
        \item<4-> Jumps into \funcname{caml\_raise\_exn}
        \item<5-> Calls \funcname{caml\_stash\_backtrace} again, to scan the next part of the stack: from the trap just pop-ed to the next trap
      \end{itemize}
      \vfill
      \mintocaml[firstline=15,lastline=21,highlightlines={18-21}]{../src/backtrace/backtrace.ml}
      \endminipage
    \end{column}
    \begin{column}{0.5\textwidth}
      \raggedleft
      \onslide*<1>{\listCamlReraiseExn{}}
      \onslide*<2>{\listCamlReraiseExn[highlightlines=729]{}}
      \onslide*<3>{
        \mintc[firstline=190,lastline=192,highlightlines={191-192}]{../ocaml/runtime/amd64.S}
        \mintc[firstline=234,lastline=237,highlightlines={235-237}]{../ocaml/runtime/amd64.S}
        \medskip
        \listCamlReraiseExn[highlightlines=731]{}
      }
      \onslide*<4-5>{
        % TODO refactor with \listCamlReraiseExn
        \mintasm[firstline=727,lastline=734,highlightlines=730]{../ocaml/runtime/amd64.S}
        \medskip
      }
      \onslide*<4>{\listCamlRaiseExn[highlightlines=711]{}}
      \onslide*<5>{\listCamlRaiseExn[highlightlines=720]{}}
    \end{column}
  \end{columns}
\end{frame}

\begin{frame}{Backtraces}{Backtrace collection continues}
  \begin{columns}[c]
    \begin{column}{0.55\textwidth}
      \minipage[c][0.75\textheight][s]{\columnwidth}
      \begin{itemize}
        \item<1-> \funcname{caml\_stash\_backtrace} scans the older part of the stack, up to the next trap
        \item<6-> Controls jumps to the nested exception handler in \funcname{maybe\_raise}, which matches only on \typename{ExnB}
        \item<7-> \funcname{caml\_reraise\_exn} is called again, backtrace collection resumes with \funcname{caml\_stash\_backtrace}
      \end{itemize}
      \medskip
      \begin{itemize}
        \item<2-> Constructed backtrace:
        \begin{itemize}
          \item <2-> \funcname{definitely\_raise}
          \item <2-> \funcname{probably\_raise}
          \item <3-> \funcname{probably\_raise} (re-raise)
          \item <4-> \funcname{maybe\_raise}
          \item <5-> \funcname{main}
          \item <8-> \funcname{main} (re-raise)
        \end{itemize}
      \end{itemize}
      \vfill
      \onslide*<1-5>{\mintocaml[firstline=15,lastline=21]{../src/backtrace/backtrace.ml}}
      \onslide*<6>{\mintocaml[firstline=28,lastline=34,highlightlines=31]{../src/backtrace/backtrace.ml}}
      \onslide*<7->{\mintocaml[firstline=28,lastline=34,highlightlines=33]{../src/backtrace/backtrace.ml}}
      \endminipage
    \end{column}
    \begin{column}{0.45\textwidth}
      \raggedleft
      \onslide*<1>{\providecommand\step{6}\input{diagrams/backtrace/backtrace.tex}}%
      \onslide*<2>{\providecommand\step{6}\input{diagrams/backtrace/backtrace.tex}}%
      \onslide*<3>{\providecommand\step{7}\input{diagrams/backtrace/backtrace.tex}}%
      \onslide*<4>{\providecommand\step{8}\input{diagrams/backtrace/backtrace.tex}}%
      \onslide*<5>{\providecommand\step{9}\input{diagrams/backtrace/backtrace.tex}}%
      \onslide*<6>{\providecommand\step{10}\input{diagrams/backtrace/backtrace.tex}}%
      \onslide*<7>{\providecommand\step{11}\input{diagrams/backtrace/backtrace.tex}}%
      \onslide*<8>{\providecommand\step{12}\input{diagrams/backtrace/backtrace.tex}}%
      \onslide*<9>{\providecommand\step{13}\input{diagrams/backtrace/backtrace.tex}}%
    \end{column}
  \end{columns}
\end{frame}

\begin{frame}{Backtraces}{Printing the backtrace}
  \begin{columns}[c]
    \begin{column}{0.45\textwidth}
      \minipage[c][0.66\textheight][s]{\columnwidth}
      \begin{itemize}
        \item<1-> The exception backtrace can now be printed to \funcarg{stdout} with \funcname{Printexc.print_backtrace}
        \item<2-> First the exception backtrace is retrieved into an OCaml array with \funcname{get_raw_backtrace }
        \item<3-> Which is implemented as the \funcname{caml_get_exception_raw_backtrace} C function in the runtime
        \item<4-> And finally printed with \funcname{print_exception_backtrace}
      \end{itemize}
      \vfill
      \mintocaml[firstline=28,lastline=34,highlightlines=33]{../src/backtrace/backtrace.ml}
      \endminipage
    \end{column}
    \begin{column}{0.55\textwidth}
      \onslide*<1,2>{
        \mintocaml[firstline=100,lastline=101]{../ocaml/stdlib/printexc.ml}
      }
      \only<1>{
        \listocaml[firstline=170,lastline=172]{../ocaml/stdlib/printexc.ml}{stdlib/printexc.ml:170}
      }
      \only<2>{
        \listocaml[firstline=170,lastline=172,highlightlines=172]{../ocaml/stdlib/printexc.ml}{stdlib/printexc.ml:170}
      }
      \only<3>{
        \mintc[firstline=154,lastline=156]{../ocaml/runtime/backtrace.c}
        \listc[firstline=171,lastline=192,highlightlines={184-187}]{../ocaml/runtime/backtrace.c}{runtime/backtrace.c:154}
      }
      \only<4>{
        \listocaml[firstline=155,lastline=165]{../ocaml/stdlib/printexc.ml}{stdlib/printexc.ml:55}
      }
    \end{column}
  \end{columns}
\end{frame}


\subsection{The multiple kinds of raise}
\frameSubsection{}{}

\begin{frame}{Backtraces}{When backtraces are not needed}
  \begin{columns}[c]
    \begin{column}{0.45\textwidth}
      \minipage[c][0.66\textheight][s]{\columnwidth}
      \begin{itemize}
        \item<1-> What if the backtrace for an exception is never needed?
        \item<2-> It's possible to raise the exception explicitly with \funcname{raise\_notrace} to make raising as cheap as possible
        \item<3-> An example of use in the \funcname{dynlink} library of the OCaml compiler
      \end{itemize}
      \vfill
      \onslide<3->{
      \listocaml[firstline=205,lastline=209,highlightlines=207]{../ocaml/otherlibs/dynlink/dynlink\_compilerlibs/misc.ml}{otherlibs/dynlink/dynlink\_compilerlibs/misc.ml:205}
      }
      \endminipage
    \end{column}
    \begin{column}{0.55\textwidth}
    \only<4>{
      \mintcmm[highlightlines=3]{../src/notrace/camlNotrace\_\_fun_347.cmm}
      \listcmm{../src/notrace/camlNotrace\_\_all\_somes\_327.cmm}{all\_somes}
    }
    \onslide*<5->{
      \listobjdump[highlightlines={6-9}]{../src/notrace/camlNotrace\_\_fun_347.objdump}{anonymous function from \funcname{all\_somes}}
    }
    \end{column}
  \end{columns}
\end{frame}

\begin{frame}{Backtraces}{Summary table of the multiple kinds of raise}
  \begin{itemize}
    \item When debugging information is enabled and if the backtrace of an exception isn't needed, it's possible to raise with \funcname{raise\_notrace} for performance
    \item When debugging information is \emph{not} enabled, the compiler optimises \funcname{raise} calls to inline assembly (same effect as using \funcname{raise\_notrace})
  \end{itemize}
  \bigskip
  \centering
  \begin{tabular}{| l | c | c | c | c |}
    \hline
    \parbox{10em}{Debug information \\ (\funcarg{-g} flag)}  & \multicolumn{2}{|c|}{Disabled}                                     & \multicolumn{2}{|c|}{Enabled} \\
    \hline
    Raise kind                                               & \funcname{raise} & \funcname{raise\_notrace}                       & \funcname{raise} & \funcname{raise\_notrace} \\
    \hline
    Compilation                                              & \parbox{8em}{inline assembly\\ (optimization)} & inline assembly   & \funcname{caml\_raise\_exn} & inline assembly \\
    \hline
  \end{tabular}
\end{frame}


%
% Takeaway #5
%
\subsection*{Takeaway \#5}
\frameSubsectionTakeaway{}
\againframe<5>{takeaway}
}%
      \onslide*<4>{\providecommand\step{8}%
% Backtraces
%
\subsection{One advantage of raising with debugging information}
\frameSubsection{}{}

\begin{frame}{Backtraces}{Debugging information \& source code}
  \begin{columns}[c]
    \begin{column}{0.5\textwidth}
      \begin{itemize}
        \item Raising an exception is fun and games, but how do we know where the exception originated? $\rightarrow$ With the backtrace associated with the exception!
        \item In order to get a backtrace from the point where an exception was raised, it's necessary to compile with debugging information enabled (\funcarg{-g} flag for the compiler)
        \item Same example as in the `Nested handlers' section
      \end{itemize}
    \end{column}
    \begin{column}{0.5\textwidth}
      \listocaml[firstline=9,lastline=34]{../src/backtrace/backtrace.ml}{backtrace/backtrace.ml}
    \end{column}
  \end{columns}
\end{frame}

\begin{frame}{Backtraces}{CMM and disassembly of \funcname{definitely\_raise}}
  \begin{columns}[c]
    \begin{column}{0.5\textwidth}
      \minipage[c][0.66\textheight][s]{\columnwidth}
      \begin{itemize}
        \item<1-> An effect of compiling with debugging information (\funcarg{-g}): the CMM no longer uses \funcname{raise\_notrace} to raise the exception but \funcname{raise} instead
        \item<2-> Disassembly shows that CMM \funcname{raise} is actually a call to \funcname{caml\_raise\_exn}, a function in assembler provided by the runtime
      \end{itemize}
      \vfill
      \mintocaml[firstline=9,lastline=13,highlightlines=12]{../src/backtrace/backtrace.ml}
      \endminipage
    \end{column}
    \begin{column}{0.5\textwidth}
      \onslide*<1>{
        \mintcmm[highlightlines=5]{../src/backtrace/camlBacktrace\_\_definitely\_raise\_347.cmm}
      }
      \onslide*<2>{
        \mintobjdump[firstline=2,highlightlines=14]{../src/backtrace/camlBacktrace\_\_definitely\_raise\_347.objdump}
      }
    \end{column}
  \end{columns}
\end{frame}

\begin{frame}{Backtraces}{Introducing \funcname{caml\_raise\_exn}}
  \begin{columns}[c]
    \begin{column}{0.5\textwidth}
      \minipage[c][0.66\textheight][s]{\columnwidth}
      \onslide*<1>{
        \begin{itemize}
          \item Raising the exception is performed using \funcname{caml\_raise\_exn} which is
            \begin{itemize}
              \item An assembly function provided by the runtime
              \item Architecture specific
            \end{itemize}
          \item Let's see what it does differently from \funcname{raise\_notrace}\dots
          \item We are about to raise \typename{ExnC} with \funcname{caml\_raise\_exn} from \funcname{definitely\_raise}
        \end{itemize}
      }
      \onslide*<2>{
        \mintobjdump[firstline=2,highlightlines=14]{../src/backtrace/camlBacktrace\_\_definitely\_raise\_347.objdump}
      }
      \vfill
      \mintocaml[firstline=9,lastline=13,highlightlines=12]{../src/backtrace/backtrace.ml}
      \endminipage
    \end{column}
    \begin{column}{0.5\textwidth}
      \centering
      \providecommand\step{1}\input{diagrams/backtrace/backtrace.tex}
    \end{column}
  \end{columns}
\end{frame}

\begin{frame}{Backtraces}{But before that, we need more fields from \typename{caml\_domain\_state}}
  \begin{columns}
    \begin{column}{0.5\textwidth}
      \begin{itemize}
        \item Remember \typename{caml\_domain\_state} the per-domain runtime state?
        \item Some other members are of interest to us now\dots
        \item \localname{backtrace\_active} is used to know if the runtime should collect backtraces
        \item \localname{backtrace\_active} can be set by
          \begin{itemize}
            \item The \funcarg{b} flag in the \funcname{OCAMLRUNPARAM} environment variable
            \item \mintinline{ocaml}{Printexc.record_backtrace : bool -> unit} \footnotemark
          \end{itemize}
      \end{itemize}
    \end{column}
    \begin{column}{0.5\textwidth}
      \begin{listing}
        \mintc[highlightlines={9-11}]{src/ocaml/domain\_state\_backtrace.h}
        \caption{runtime/caml/domain\_state.h}
      \end{listing}
    \end{column}
  \end{columns}
  \footnotetext[1]{https://v2.ocaml.org/api/Printexc.html}
\end{frame}

\newcommand\listCamlRaiseExn[1][]{
  \listasm[firstline=702,lastline=725,#1]{../ocaml/runtime/amd64.S}{runtime/amd64.S:702}}

\begin{frame}{Backtraces}{Walk through \funcname{caml\_raise\_exn}}
  \begin{columns}[T]
    \begin{column}{0.5\textwidth}
      \begin{itemize}
        \item<1-> First checks whether exceptions backtrace should be recorded
        \item<1-> By checking the \funcarg{domain\_state->backtrace\_active} value
        \item<2-> If not, acts as \funcname{raise\_notrace} with \funcname{RESTORE\_EXN\_HANDLER\_OCAML}
          \begin{itemize}
            \item Set \regname{RSP} to \localname{domain\_state->exn\_handler} value
            \item Restore \localname{domain\_state->exn\_handler} from trap
            \item Jump with \funcname{ret} (same effect as using \funcname{pop} and \funcname{jmp})
          \end{itemize}
        \item<3-> Otherwise, switches to the C stack to be able to execute C code
        \item<4-> Calls \funcname{caml\_stash\_backtrace}, a C function provided by the runtime\ignorespaces
          \onslide<6->{, with
            \begin{itemize}
              \item \funcarg{exn}: exception value
              \item \funcarg{pc}: code address of the raising point
              \item \funcarg{sp}: stack address of the raising function
              \item \funcarg{trapsp}: stack address of the next handler
            \end{itemize}
          }
      \end{itemize}
      \bigskip
      \mintocaml[firstline=9,lastline=13,highlightlines=12]{../src/backtrace/backtrace.ml}
    \end{column}
    \begin{column}{0.5\textwidth}
      \raggedleft
      \onslide*<1,3-4>{
        \mintc[firstline=190,lastline=192]{../ocaml/runtime/amd64.S}
        \mintc[firstline=234,lastline=237]{../ocaml/runtime/amd64.S}
      }
      \onslide*<2>{
        \mintc[firstline=190,lastline=192,highlightlines={191-192}]{../ocaml/runtime/amd64.S}
        \mintc[firstline=234,lastline=237,highlightlines={235-237}]{../ocaml/runtime/amd64.S}
      }
      \onslide*<1>{\listCamlRaiseExn[highlightlines=705]{}}%
      \onslide*<2>{\listCamlRaiseExn[highlightlines={707-708}]{}}%
      \onslide*<3>{\listCamlRaiseExn[highlightlines={712-714}]{}}%
      \onslide*<4>{\listCamlRaiseExn[highlightlines={715-720}]{}}%
      \onslide*<5>{\providecommand\step{1}\input{diagrams/backtrace/backtrace.tex}}%
      \onslide*<6>{\providecommand\step{2}\input{diagrams/backtrace/backtrace.tex}}%
    \end{column}
  \end{columns}
\end{frame}

\newcommand\listStashBacktrace[1][]{
  \listc[firstline=91,lastline=120,#1]{../ocaml/runtime/backtrace\_nat.c}{runtime/backtrace\_nat.c:91}}

\begin{frame}{Backtraces}{Introducing \funcname{caml\_stash\_backtrace}}
  \begin{columns}[c]
    \begin{column}{0.5\textwidth}
      \minipage[c][0.66\textheight][s]{\columnwidth}
      \begin{itemize}
        \item<1-> A C function provided by the runtime (specific to native code)
        \item<2-> It scans the stack, between the stack pointer at the raising point up to the next trap, in order to collect the names of the detected function frames
          \onslide*<2>{
            \begin{itemize}
              \item And more information, like line number, column number...
            \end{itemize}
          }
        \item<3-> It uses the same technique and information as the Garbage Collector (GC) uses to scan the values live on the stack
          \onslide*<4>{
            \begin{itemize}
              \item \typename{frame\_descr} are at the core of stack unwinding
            \end{itemize}
          }
      \end{itemize}
      \vfill
      \mintocaml[firstline=9,lastline=13,highlightlines=12]{../src/backtrace/backtrace.ml}
      \endminipage
    \end{column}
    \begin{column}{0.5\textwidth}
      \onslide*<1>{\listStashBacktrace[highlightlines=91]{}}
      \onslide*<2>{\listStashBacktrace[highlightlines={91,117-118}]{}}
      \onslide*<3->{\listStashBacktrace[highlightlines={94,106,109-110}]{}}
    \end{column}
  \end{columns}
\end{frame}

\newcommand\listFrameDescr[1][]{
  \listocaml[firstline=111,lastline=124,#1]{../ocaml/asmcomp/emitaux.ml}{asmcomp/emitaux.ml:111}}
\newcommand\listDebugInfo[1][]{
  \listocaml[firstline=38,lastline=49,#1]{../ocaml/lambda/debuginfo.mli}{lambda/debuginfo.mli:38}}

\begin{frame}{Backtraces}{\typename{frame\_descr} and \typename{frame\_debuginfo} in a nutshell}
  \begin{columns}[c]
    \begin{column}{0.5\textwidth}
      \begin{itemize}
        \item<1-> For every return address in an OCaml program, there is a associated \typename{frame\_descr}
        \item<2-> A \typename{frame\_descr} specifies
        \begin{itemize}
          \item<2-> The size of the function stack frame
          \item<3-> Where live values are on the stack (so that the GC can mark them as live and prevent from being swept)
        \end{itemize}
        \item<4-> When compiled with debug information, \typename{frame\_descr} will be linked to a \typename{frame\_debuginfo}
        \begin{itemize}
          \item<5-> Contains information about the position in the source code (file name, line and column number)
        \end{itemize}
      \end{itemize}
    \end{column}
    \begin{column}{0.5\textwidth}
        \onslide*<1>{\listFrameDescr[highlightlines={118-122}]{}}
        \onslide*<2>{\listFrameDescr[highlightlines={120}]{}}
        \onslide*<3>{\listFrameDescr[highlightlines={121}]{}}
        \onslide*<4->{\listFrameDescr[highlightlines={122}]{}}
        \medskip
        \onslide*<1-3>{\listDebugInfo{}}
        \onslide*<4>{\listDebugInfo[highlightlines={38,49}]{}}
        \onslide*<5>{\listDebugInfo[highlightlines={39-46}]{}}
    \end{column}
  \end{columns}
\end{frame}

\begin{frame}{Backtraces}{Scanning the stack with \typename{frame\_descr}}
  \begin{columns}[c]
    \begin{column}{0.5\textwidth}
      \begin{itemize}
        \item Given the return address of the code that raises the exception by calling \funcname{caml\_raise\_exn} (\funcarg{pc} argument of \funcname{caml\_stash\_backtrace})
        \item And the position of the stack pointer (\funcarg{sp} argument of \funcname{caml\_stash\_backtrace})
        \item The frame size given by \typename{frame\_descr}.\funcarg{frame\_size}, allows to find the end of the previous frame
        \item And the return address of the caller of this function
        \bigskip
        \onslide<3->{
        \item Note that traps (here the reddish \funcname{ExnA}, \funcname{ExnB} and \funcname{ExnC} blocks) belong to the frame that installed them, and so the frame size changes when traps are removed
        }
      \end{itemize}
    \end{column}
    \begin{column}{0.5\textwidth}
      \raggedleft
      \onslide*<1>{\providecommand\step{2}\input{diagrams/backtrace/backtrace.tex}}%
      \onslide*<2>{\providecommand\step{1}\input{diagrams/backtrace/scanstack.tex}}%
      \onslide*<3>{\providecommand\step{2}\input{diagrams/backtrace/scanstack.tex}}%
      \onslide*<4>{\providecommand\step{3}\input{diagrams/backtrace/scanstack.tex}}%
      \onslide*<5>{\providecommand\step{4}\input{diagrams/backtrace/scanstack.tex}}%
      \onslide*<6>{\providecommand\step{5}\input{diagrams/backtrace/scanstack.tex}}%
    \end{column}
  \end{columns}
\end{frame}

% Duplicate of previous-previous slide
\begin{frame}{Backtraces}{Continuing on \funcname{caml\_stash\_backtrace}}
  \begin{columns}[c]
    \begin{column}{0.5\textwidth}
      \minipage[c][0.75\textheight][s]{\columnwidth}
      \begin{itemize}
          \color{gray}
        \item A C function provided by the runtime (specific to native code)
        \item It scans the stack, between the stack pointer at the raising point up to the next trap, in order to collect the names of the detected function frames
        \item It uses the same technique and information as the Garbage Collector (GC) uses to scan the values live on the stack
      \end{itemize}
      \begin{itemize}
        \item For each function frame detected, store a pointer to the \typename{frame\_descr} in the \localname{domain\_state->backtrace\_buffer} array, effectively constructing the backtrace
      \end{itemize}
      \onslide<2->{
        \begin{itemize}
            \smallskip
          \item Constructed backtrace:
            \funcname{definitely\_raise},
            \onslide<3->{\funcname{probably\_raise}}
        \end{itemize}
      }
      \vfill
      \mintocaml[firstline=9,lastline=13,highlightlines=12]{../src/backtrace/backtrace.ml}
      \endminipage
    \end{column}
    \begin{column}{0.5\textwidth}
      \raggedleft
      \onslide*<1>{\listStashBacktrace[highlightlines={114-115}]{}}
      \onslide*<2>{\providecommand\step{3}\input{diagrams/backtrace/backtrace.tex}}%
      \onslide*<3>{\providecommand\step{4}\input{diagrams/backtrace/backtrace.tex}}%
    \end{column}
  \end{columns}
\end{frame}

\begin{frame}{Backtraces}{Back to \funcname{caml\_raise\_exn}}
  \begin{columns}[c]
    \begin{column}{0.5\textwidth}
      \minipage[c][0.66\textheight][s]{\columnwidth}
      \begin{itemize}\color{gray}
        \item Checks wheteher exceptions backtrace should be recorded, by checking the \funcarg{domain\_state->backtrace\_active} value
        \item Switches to the C stack to be able to execute C code
        \item Calls \funcname{caml\_stash\_backtrace}, to collect the backtrace up to the next trap
      \end{itemize}
      \onslide*<2->{
        \begin{itemize}
          \item<2-> Executes the exception handler in \funcname{probably\_raise} with \funcname{RESTORE\_EXN\_HANDLER\_OCAML}
            \begin{itemize}
              \item Set \regname{RSP} to \localname{domain\_state->exn\_handler} value
              \item Restore \localname{domain\_state->exn\_handler} from trap
              \item Jump to the handler code with \funcname{ret}
            \end{itemize}
        \end{itemize}
      }
      \vfill
      \onslide*<1>{\mintocaml[firstline=9,lastline=13,highlightlines=12]{../src/backtrace/backtrace.ml}}
      \onslide*<2->{\mintocaml[firstline=15,lastline=21,highlightlines={19-21}]{../src/backtrace/backtrace.ml}}
      \endminipage
    \end{column}
    \begin{column}{0.5\textwidth}
      \centering
      \onslide*<1>{
        \mintc[firstline=190,lastline=192]{../ocaml/runtime/amd64.S}
        \mintc[firstline=234,lastline=237]{../ocaml/runtime/amd64.S}
      }
      \onslide*<2>{
        \mintc[firstline=190,lastline=192,highlightlines={191-192}]{../ocaml/runtime/amd64.S}
        \mintc[firstline=234,lastline=237,highlightlines={235-237}]{../ocaml/runtime/amd64.S}
      }
      \onslide*<1>{\listCamlRaiseExn[highlightlines=720]{}}
      \onslide*<2>{\listCamlRaiseExn[highlightlines=722]{}}
      \onslide*<3->{\providecommand\step{5}\input{diagrams/backtrace/backtrace.tex}}
    \end{column}
  \end{columns}
\end{frame}

\begin{frame}{Backtraces}{\funcname{caml\_probably\_raise}'s handler doesn't catch \typename{ExnC}}
  \begin{columns}[c]
    \begin{column}{0.45\textwidth}
      \minipage[c][0.75\textheight][s]{\columnwidth}
      \begin{itemize}
        \item<1-> \funcname{match \dots with}\footnotemark pattern matching can also match exceptions
        \item<1-> The CMM of \funcname{probably\_raise} shows that this \funcname{match \dots with} is implemented with a pattern matching and a \funcname{try \dots with}
        \item<1-> But this one only matches on \typename{ExnA} exception
        \item<2-> Since the pattern matching fails to catch the exception, \funcname{reraise} is called with our current \typename{ExnC} exception
        \item<3-> Disassembly shows that \funcname{reraise} is actually a call to \funcname{caml\_reraise\_exn}, which is also an assembly function provided by the runtime
      \end{itemize}
      \vfill
      \mintocaml[firstline=15,lastline=21,highlightlines={18-21}]{../src/backtrace/backtrace.ml}
      \endminipage
    \end{column}
    \begin{column}{0.55\textwidth}
      \centering
      \onslide*<1>{\mintcmm[highlightlines={10-18}]{../src/backtrace/camlBacktrace\_\_probably\_raise\_351.cmm}}
      \onslide*<2>{\listcmm[highlightlines=18]{../src/backtrace/camlBacktrace\_\_probably\_raise_351.cmm}{CMM of probably\_raise}}
      \onslide*<3>{\mintobjdump[firstline=5,lastline=35,highlightlines=31]{../src/backtrace/camlBacktrace\_\_probably\_raise_351.objdump}}
    \end{column}
  \end{columns}
  \only<1>{\footnotetext[1]{https://blog.janestreet.com/pattern-matching-and-exception-handling-unite/}}
\end{frame}

\newcommand\listCamlReraiseExn[1][]{
  \listasm[firstline=727,lastline=734,#1]{../ocaml/runtime/amd64.S}{runtime/amd64.S:727}}

\begin{frame}{Backtraces}{Introducing \funcname{caml\_reraise\_exn}}
  \begin{columns}[c]
    \begin{column}{0.5\textwidth}
      \minipage[c][0.66\textheight][s]{\columnwidth}
      \begin{itemize}
        \item<1-> The exception have to be reraised to the next handler
        \item<2-> First, checks if the backtrace should be collected with \funcarg{domain\_state->backtrace\_active}
        \only<3>{
        \begin{itemize}
            \item If not, behaves like a \funcname{raise\_notrace} with \funcname{RESTORE\_EXN\_HANDLER\_OCAML}
        \end{itemize}
        }
        \item<4-> Jumps into \funcname{caml\_raise\_exn}
        \item<5-> Calls \funcname{caml\_stash\_backtrace} again, to scan the next part of the stack: from the trap just pop-ed to the next trap
      \end{itemize}
      \vfill
      \mintocaml[firstline=15,lastline=21,highlightlines={18-21}]{../src/backtrace/backtrace.ml}
      \endminipage
    \end{column}
    \begin{column}{0.5\textwidth}
      \raggedleft
      \onslide*<1>{\listCamlReraiseExn{}}
      \onslide*<2>{\listCamlReraiseExn[highlightlines=729]{}}
      \onslide*<3>{
        \mintc[firstline=190,lastline=192,highlightlines={191-192}]{../ocaml/runtime/amd64.S}
        \mintc[firstline=234,lastline=237,highlightlines={235-237}]{../ocaml/runtime/amd64.S}
        \medskip
        \listCamlReraiseExn[highlightlines=731]{}
      }
      \onslide*<4-5>{
        % TODO refactor with \listCamlReraiseExn
        \mintasm[firstline=727,lastline=734,highlightlines=730]{../ocaml/runtime/amd64.S}
        \medskip
      }
      \onslide*<4>{\listCamlRaiseExn[highlightlines=711]{}}
      \onslide*<5>{\listCamlRaiseExn[highlightlines=720]{}}
    \end{column}
  \end{columns}
\end{frame}

\begin{frame}{Backtraces}{Backtrace collection continues}
  \begin{columns}[c]
    \begin{column}{0.55\textwidth}
      \minipage[c][0.75\textheight][s]{\columnwidth}
      \begin{itemize}
        \item<1-> \funcname{caml\_stash\_backtrace} scans the older part of the stack, up to the next trap
        \item<6-> Controls jumps to the nested exception handler in \funcname{maybe\_raise}, which matches only on \typename{ExnB}
        \item<7-> \funcname{caml\_reraise\_exn} is called again, backtrace collection resumes with \funcname{caml\_stash\_backtrace}
      \end{itemize}
      \medskip
      \begin{itemize}
        \item<2-> Constructed backtrace:
        \begin{itemize}
          \item <2-> \funcname{definitely\_raise}
          \item <2-> \funcname{probably\_raise}
          \item <3-> \funcname{probably\_raise} (re-raise)
          \item <4-> \funcname{maybe\_raise}
          \item <5-> \funcname{main}
          \item <8-> \funcname{main} (re-raise)
        \end{itemize}
      \end{itemize}
      \vfill
      \onslide*<1-5>{\mintocaml[firstline=15,lastline=21]{../src/backtrace/backtrace.ml}}
      \onslide*<6>{\mintocaml[firstline=28,lastline=34,highlightlines=31]{../src/backtrace/backtrace.ml}}
      \onslide*<7->{\mintocaml[firstline=28,lastline=34,highlightlines=33]{../src/backtrace/backtrace.ml}}
      \endminipage
    \end{column}
    \begin{column}{0.45\textwidth}
      \raggedleft
      \onslide*<1>{\providecommand\step{6}\input{diagrams/backtrace/backtrace.tex}}%
      \onslide*<2>{\providecommand\step{6}\input{diagrams/backtrace/backtrace.tex}}%
      \onslide*<3>{\providecommand\step{7}\input{diagrams/backtrace/backtrace.tex}}%
      \onslide*<4>{\providecommand\step{8}\input{diagrams/backtrace/backtrace.tex}}%
      \onslide*<5>{\providecommand\step{9}\input{diagrams/backtrace/backtrace.tex}}%
      \onslide*<6>{\providecommand\step{10}\input{diagrams/backtrace/backtrace.tex}}%
      \onslide*<7>{\providecommand\step{11}\input{diagrams/backtrace/backtrace.tex}}%
      \onslide*<8>{\providecommand\step{12}\input{diagrams/backtrace/backtrace.tex}}%
      \onslide*<9>{\providecommand\step{13}\input{diagrams/backtrace/backtrace.tex}}%
    \end{column}
  \end{columns}
\end{frame}

\begin{frame}{Backtraces}{Printing the backtrace}
  \begin{columns}[c]
    \begin{column}{0.45\textwidth}
      \minipage[c][0.66\textheight][s]{\columnwidth}
      \begin{itemize}
        \item<1-> The exception backtrace can now be printed to \funcarg{stdout} with \funcname{Printexc.print_backtrace}
        \item<2-> First the exception backtrace is retrieved into an OCaml array with \funcname{get_raw_backtrace }
        \item<3-> Which is implemented as the \funcname{caml_get_exception_raw_backtrace} C function in the runtime
        \item<4-> And finally printed with \funcname{print_exception_backtrace}
      \end{itemize}
      \vfill
      \mintocaml[firstline=28,lastline=34,highlightlines=33]{../src/backtrace/backtrace.ml}
      \endminipage
    \end{column}
    \begin{column}{0.55\textwidth}
      \onslide*<1,2>{
        \mintocaml[firstline=100,lastline=101]{../ocaml/stdlib/printexc.ml}
      }
      \only<1>{
        \listocaml[firstline=170,lastline=172]{../ocaml/stdlib/printexc.ml}{stdlib/printexc.ml:170}
      }
      \only<2>{
        \listocaml[firstline=170,lastline=172,highlightlines=172]{../ocaml/stdlib/printexc.ml}{stdlib/printexc.ml:170}
      }
      \only<3>{
        \mintc[firstline=154,lastline=156]{../ocaml/runtime/backtrace.c}
        \listc[firstline=171,lastline=192,highlightlines={184-187}]{../ocaml/runtime/backtrace.c}{runtime/backtrace.c:154}
      }
      \only<4>{
        \listocaml[firstline=155,lastline=165]{../ocaml/stdlib/printexc.ml}{stdlib/printexc.ml:55}
      }
    \end{column}
  \end{columns}
\end{frame}


\subsection{The multiple kinds of raise}
\frameSubsection{}{}

\begin{frame}{Backtraces}{When backtraces are not needed}
  \begin{columns}[c]
    \begin{column}{0.45\textwidth}
      \minipage[c][0.66\textheight][s]{\columnwidth}
      \begin{itemize}
        \item<1-> What if the backtrace for an exception is never needed?
        \item<2-> It's possible to raise the exception explicitly with \funcname{raise\_notrace} to make raising as cheap as possible
        \item<3-> An example of use in the \funcname{dynlink} library of the OCaml compiler
      \end{itemize}
      \vfill
      \onslide<3->{
      \listocaml[firstline=205,lastline=209,highlightlines=207]{../ocaml/otherlibs/dynlink/dynlink\_compilerlibs/misc.ml}{otherlibs/dynlink/dynlink\_compilerlibs/misc.ml:205}
      }
      \endminipage
    \end{column}
    \begin{column}{0.55\textwidth}
    \only<4>{
      \mintcmm[highlightlines=3]{../src/notrace/camlNotrace\_\_fun_347.cmm}
      \listcmm{../src/notrace/camlNotrace\_\_all\_somes\_327.cmm}{all\_somes}
    }
    \onslide*<5->{
      \listobjdump[highlightlines={6-9}]{../src/notrace/camlNotrace\_\_fun_347.objdump}{anonymous function from \funcname{all\_somes}}
    }
    \end{column}
  \end{columns}
\end{frame}

\begin{frame}{Backtraces}{Summary table of the multiple kinds of raise}
  \begin{itemize}
    \item When debugging information is enabled and if the backtrace of an exception isn't needed, it's possible to raise with \funcname{raise\_notrace} for performance
    \item When debugging information is \emph{not} enabled, the compiler optimises \funcname{raise} calls to inline assembly (same effect as using \funcname{raise\_notrace})
  \end{itemize}
  \bigskip
  \centering
  \begin{tabular}{| l | c | c | c | c |}
    \hline
    \parbox{10em}{Debug information \\ (\funcarg{-g} flag)}  & \multicolumn{2}{|c|}{Disabled}                                     & \multicolumn{2}{|c|}{Enabled} \\
    \hline
    Raise kind                                               & \funcname{raise} & \funcname{raise\_notrace}                       & \funcname{raise} & \funcname{raise\_notrace} \\
    \hline
    Compilation                                              & \parbox{8em}{inline assembly\\ (optimization)} & inline assembly   & \funcname{caml\_raise\_exn} & inline assembly \\
    \hline
  \end{tabular}
\end{frame}


%
% Takeaway #5
%
\subsection*{Takeaway \#5}
\frameSubsectionTakeaway{}
\againframe<5>{takeaway}
}%
      \onslide*<5>{\providecommand\step{9}%
% Backtraces
%
\subsection{One advantage of raising with debugging information}
\frameSubsection{}{}

\begin{frame}{Backtraces}{Debugging information \& source code}
  \begin{columns}[c]
    \begin{column}{0.5\textwidth}
      \begin{itemize}
        \item Raising an exception is fun and games, but how do we know where the exception originated? $\rightarrow$ With the backtrace associated with the exception!
        \item In order to get a backtrace from the point where an exception was raised, it's necessary to compile with debugging information enabled (\funcarg{-g} flag for the compiler)
        \item Same example as in the `Nested handlers' section
      \end{itemize}
    \end{column}
    \begin{column}{0.5\textwidth}
      \listocaml[firstline=9,lastline=34]{../src/backtrace/backtrace.ml}{backtrace/backtrace.ml}
    \end{column}
  \end{columns}
\end{frame}

\begin{frame}{Backtraces}{CMM and disassembly of \funcname{definitely\_raise}}
  \begin{columns}[c]
    \begin{column}{0.5\textwidth}
      \minipage[c][0.66\textheight][s]{\columnwidth}
      \begin{itemize}
        \item<1-> An effect of compiling with debugging information (\funcarg{-g}): the CMM no longer uses \funcname{raise\_notrace} to raise the exception but \funcname{raise} instead
        \item<2-> Disassembly shows that CMM \funcname{raise} is actually a call to \funcname{caml\_raise\_exn}, a function in assembler provided by the runtime
      \end{itemize}
      \vfill
      \mintocaml[firstline=9,lastline=13,highlightlines=12]{../src/backtrace/backtrace.ml}
      \endminipage
    \end{column}
    \begin{column}{0.5\textwidth}
      \onslide*<1>{
        \mintcmm[highlightlines=5]{../src/backtrace/camlBacktrace\_\_definitely\_raise\_347.cmm}
      }
      \onslide*<2>{
        \mintobjdump[firstline=2,highlightlines=14]{../src/backtrace/camlBacktrace\_\_definitely\_raise\_347.objdump}
      }
    \end{column}
  \end{columns}
\end{frame}

\begin{frame}{Backtraces}{Introducing \funcname{caml\_raise\_exn}}
  \begin{columns}[c]
    \begin{column}{0.5\textwidth}
      \minipage[c][0.66\textheight][s]{\columnwidth}
      \onslide*<1>{
        \begin{itemize}
          \item Raising the exception is performed using \funcname{caml\_raise\_exn} which is
            \begin{itemize}
              \item An assembly function provided by the runtime
              \item Architecture specific
            \end{itemize}
          \item Let's see what it does differently from \funcname{raise\_notrace}\dots
          \item We are about to raise \typename{ExnC} with \funcname{caml\_raise\_exn} from \funcname{definitely\_raise}
        \end{itemize}
      }
      \onslide*<2>{
        \mintobjdump[firstline=2,highlightlines=14]{../src/backtrace/camlBacktrace\_\_definitely\_raise\_347.objdump}
      }
      \vfill
      \mintocaml[firstline=9,lastline=13,highlightlines=12]{../src/backtrace/backtrace.ml}
      \endminipage
    \end{column}
    \begin{column}{0.5\textwidth}
      \centering
      \providecommand\step{1}\input{diagrams/backtrace/backtrace.tex}
    \end{column}
  \end{columns}
\end{frame}

\begin{frame}{Backtraces}{But before that, we need more fields from \typename{caml\_domain\_state}}
  \begin{columns}
    \begin{column}{0.5\textwidth}
      \begin{itemize}
        \item Remember \typename{caml\_domain\_state} the per-domain runtime state?
        \item Some other members are of interest to us now\dots
        \item \localname{backtrace\_active} is used to know if the runtime should collect backtraces
        \item \localname{backtrace\_active} can be set by
          \begin{itemize}
            \item The \funcarg{b} flag in the \funcname{OCAMLRUNPARAM} environment variable
            \item \mintinline{ocaml}{Printexc.record_backtrace : bool -> unit} \footnotemark
          \end{itemize}
      \end{itemize}
    \end{column}
    \begin{column}{0.5\textwidth}
      \begin{listing}
        \mintc[highlightlines={9-11}]{src/ocaml/domain\_state\_backtrace.h}
        \caption{runtime/caml/domain\_state.h}
      \end{listing}
    \end{column}
  \end{columns}
  \footnotetext[1]{https://v2.ocaml.org/api/Printexc.html}
\end{frame}

\newcommand\listCamlRaiseExn[1][]{
  \listasm[firstline=702,lastline=725,#1]{../ocaml/runtime/amd64.S}{runtime/amd64.S:702}}

\begin{frame}{Backtraces}{Walk through \funcname{caml\_raise\_exn}}
  \begin{columns}[T]
    \begin{column}{0.5\textwidth}
      \begin{itemize}
        \item<1-> First checks whether exceptions backtrace should be recorded
        \item<1-> By checking the \funcarg{domain\_state->backtrace\_active} value
        \item<2-> If not, acts as \funcname{raise\_notrace} with \funcname{RESTORE\_EXN\_HANDLER\_OCAML}
          \begin{itemize}
            \item Set \regname{RSP} to \localname{domain\_state->exn\_handler} value
            \item Restore \localname{domain\_state->exn\_handler} from trap
            \item Jump with \funcname{ret} (same effect as using \funcname{pop} and \funcname{jmp})
          \end{itemize}
        \item<3-> Otherwise, switches to the C stack to be able to execute C code
        \item<4-> Calls \funcname{caml\_stash\_backtrace}, a C function provided by the runtime\ignorespaces
          \onslide<6->{, with
            \begin{itemize}
              \item \funcarg{exn}: exception value
              \item \funcarg{pc}: code address of the raising point
              \item \funcarg{sp}: stack address of the raising function
              \item \funcarg{trapsp}: stack address of the next handler
            \end{itemize}
          }
      \end{itemize}
      \bigskip
      \mintocaml[firstline=9,lastline=13,highlightlines=12]{../src/backtrace/backtrace.ml}
    \end{column}
    \begin{column}{0.5\textwidth}
      \raggedleft
      \onslide*<1,3-4>{
        \mintc[firstline=190,lastline=192]{../ocaml/runtime/amd64.S}
        \mintc[firstline=234,lastline=237]{../ocaml/runtime/amd64.S}
      }
      \onslide*<2>{
        \mintc[firstline=190,lastline=192,highlightlines={191-192}]{../ocaml/runtime/amd64.S}
        \mintc[firstline=234,lastline=237,highlightlines={235-237}]{../ocaml/runtime/amd64.S}
      }
      \onslide*<1>{\listCamlRaiseExn[highlightlines=705]{}}%
      \onslide*<2>{\listCamlRaiseExn[highlightlines={707-708}]{}}%
      \onslide*<3>{\listCamlRaiseExn[highlightlines={712-714}]{}}%
      \onslide*<4>{\listCamlRaiseExn[highlightlines={715-720}]{}}%
      \onslide*<5>{\providecommand\step{1}\input{diagrams/backtrace/backtrace.tex}}%
      \onslide*<6>{\providecommand\step{2}\input{diagrams/backtrace/backtrace.tex}}%
    \end{column}
  \end{columns}
\end{frame}

\newcommand\listStashBacktrace[1][]{
  \listc[firstline=91,lastline=120,#1]{../ocaml/runtime/backtrace\_nat.c}{runtime/backtrace\_nat.c:91}}

\begin{frame}{Backtraces}{Introducing \funcname{caml\_stash\_backtrace}}
  \begin{columns}[c]
    \begin{column}{0.5\textwidth}
      \minipage[c][0.66\textheight][s]{\columnwidth}
      \begin{itemize}
        \item<1-> A C function provided by the runtime (specific to native code)
        \item<2-> It scans the stack, between the stack pointer at the raising point up to the next trap, in order to collect the names of the detected function frames
          \onslide*<2>{
            \begin{itemize}
              \item And more information, like line number, column number...
            \end{itemize}
          }
        \item<3-> It uses the same technique and information as the Garbage Collector (GC) uses to scan the values live on the stack
          \onslide*<4>{
            \begin{itemize}
              \item \typename{frame\_descr} are at the core of stack unwinding
            \end{itemize}
          }
      \end{itemize}
      \vfill
      \mintocaml[firstline=9,lastline=13,highlightlines=12]{../src/backtrace/backtrace.ml}
      \endminipage
    \end{column}
    \begin{column}{0.5\textwidth}
      \onslide*<1>{\listStashBacktrace[highlightlines=91]{}}
      \onslide*<2>{\listStashBacktrace[highlightlines={91,117-118}]{}}
      \onslide*<3->{\listStashBacktrace[highlightlines={94,106,109-110}]{}}
    \end{column}
  \end{columns}
\end{frame}

\newcommand\listFrameDescr[1][]{
  \listocaml[firstline=111,lastline=124,#1]{../ocaml/asmcomp/emitaux.ml}{asmcomp/emitaux.ml:111}}
\newcommand\listDebugInfo[1][]{
  \listocaml[firstline=38,lastline=49,#1]{../ocaml/lambda/debuginfo.mli}{lambda/debuginfo.mli:38}}

\begin{frame}{Backtraces}{\typename{frame\_descr} and \typename{frame\_debuginfo} in a nutshell}
  \begin{columns}[c]
    \begin{column}{0.5\textwidth}
      \begin{itemize}
        \item<1-> For every return address in an OCaml program, there is a associated \typename{frame\_descr}
        \item<2-> A \typename{frame\_descr} specifies
        \begin{itemize}
          \item<2-> The size of the function stack frame
          \item<3-> Where live values are on the stack (so that the GC can mark them as live and prevent from being swept)
        \end{itemize}
        \item<4-> When compiled with debug information, \typename{frame\_descr} will be linked to a \typename{frame\_debuginfo}
        \begin{itemize}
          \item<5-> Contains information about the position in the source code (file name, line and column number)
        \end{itemize}
      \end{itemize}
    \end{column}
    \begin{column}{0.5\textwidth}
        \onslide*<1>{\listFrameDescr[highlightlines={118-122}]{}}
        \onslide*<2>{\listFrameDescr[highlightlines={120}]{}}
        \onslide*<3>{\listFrameDescr[highlightlines={121}]{}}
        \onslide*<4->{\listFrameDescr[highlightlines={122}]{}}
        \medskip
        \onslide*<1-3>{\listDebugInfo{}}
        \onslide*<4>{\listDebugInfo[highlightlines={38,49}]{}}
        \onslide*<5>{\listDebugInfo[highlightlines={39-46}]{}}
    \end{column}
  \end{columns}
\end{frame}

\begin{frame}{Backtraces}{Scanning the stack with \typename{frame\_descr}}
  \begin{columns}[c]
    \begin{column}{0.5\textwidth}
      \begin{itemize}
        \item Given the return address of the code that raises the exception by calling \funcname{caml\_raise\_exn} (\funcarg{pc} argument of \funcname{caml\_stash\_backtrace})
        \item And the position of the stack pointer (\funcarg{sp} argument of \funcname{caml\_stash\_backtrace})
        \item The frame size given by \typename{frame\_descr}.\funcarg{frame\_size}, allows to find the end of the previous frame
        \item And the return address of the caller of this function
        \bigskip
        \onslide<3->{
        \item Note that traps (here the reddish \funcname{ExnA}, \funcname{ExnB} and \funcname{ExnC} blocks) belong to the frame that installed them, and so the frame size changes when traps are removed
        }
      \end{itemize}
    \end{column}
    \begin{column}{0.5\textwidth}
      \raggedleft
      \onslide*<1>{\providecommand\step{2}\input{diagrams/backtrace/backtrace.tex}}%
      \onslide*<2>{\providecommand\step{1}\input{diagrams/backtrace/scanstack.tex}}%
      \onslide*<3>{\providecommand\step{2}\input{diagrams/backtrace/scanstack.tex}}%
      \onslide*<4>{\providecommand\step{3}\input{diagrams/backtrace/scanstack.tex}}%
      \onslide*<5>{\providecommand\step{4}\input{diagrams/backtrace/scanstack.tex}}%
      \onslide*<6>{\providecommand\step{5}\input{diagrams/backtrace/scanstack.tex}}%
    \end{column}
  \end{columns}
\end{frame}

% Duplicate of previous-previous slide
\begin{frame}{Backtraces}{Continuing on \funcname{caml\_stash\_backtrace}}
  \begin{columns}[c]
    \begin{column}{0.5\textwidth}
      \minipage[c][0.75\textheight][s]{\columnwidth}
      \begin{itemize}
          \color{gray}
        \item A C function provided by the runtime (specific to native code)
        \item It scans the stack, between the stack pointer at the raising point up to the next trap, in order to collect the names of the detected function frames
        \item It uses the same technique and information as the Garbage Collector (GC) uses to scan the values live on the stack
      \end{itemize}
      \begin{itemize}
        \item For each function frame detected, store a pointer to the \typename{frame\_descr} in the \localname{domain\_state->backtrace\_buffer} array, effectively constructing the backtrace
      \end{itemize}
      \onslide<2->{
        \begin{itemize}
            \smallskip
          \item Constructed backtrace:
            \funcname{definitely\_raise},
            \onslide<3->{\funcname{probably\_raise}}
        \end{itemize}
      }
      \vfill
      \mintocaml[firstline=9,lastline=13,highlightlines=12]{../src/backtrace/backtrace.ml}
      \endminipage
    \end{column}
    \begin{column}{0.5\textwidth}
      \raggedleft
      \onslide*<1>{\listStashBacktrace[highlightlines={114-115}]{}}
      \onslide*<2>{\providecommand\step{3}\input{diagrams/backtrace/backtrace.tex}}%
      \onslide*<3>{\providecommand\step{4}\input{diagrams/backtrace/backtrace.tex}}%
    \end{column}
  \end{columns}
\end{frame}

\begin{frame}{Backtraces}{Back to \funcname{caml\_raise\_exn}}
  \begin{columns}[c]
    \begin{column}{0.5\textwidth}
      \minipage[c][0.66\textheight][s]{\columnwidth}
      \begin{itemize}\color{gray}
        \item Checks wheteher exceptions backtrace should be recorded, by checking the \funcarg{domain\_state->backtrace\_active} value
        \item Switches to the C stack to be able to execute C code
        \item Calls \funcname{caml\_stash\_backtrace}, to collect the backtrace up to the next trap
      \end{itemize}
      \onslide*<2->{
        \begin{itemize}
          \item<2-> Executes the exception handler in \funcname{probably\_raise} with \funcname{RESTORE\_EXN\_HANDLER\_OCAML}
            \begin{itemize}
              \item Set \regname{RSP} to \localname{domain\_state->exn\_handler} value
              \item Restore \localname{domain\_state->exn\_handler} from trap
              \item Jump to the handler code with \funcname{ret}
            \end{itemize}
        \end{itemize}
      }
      \vfill
      \onslide*<1>{\mintocaml[firstline=9,lastline=13,highlightlines=12]{../src/backtrace/backtrace.ml}}
      \onslide*<2->{\mintocaml[firstline=15,lastline=21,highlightlines={19-21}]{../src/backtrace/backtrace.ml}}
      \endminipage
    \end{column}
    \begin{column}{0.5\textwidth}
      \centering
      \onslide*<1>{
        \mintc[firstline=190,lastline=192]{../ocaml/runtime/amd64.S}
        \mintc[firstline=234,lastline=237]{../ocaml/runtime/amd64.S}
      }
      \onslide*<2>{
        \mintc[firstline=190,lastline=192,highlightlines={191-192}]{../ocaml/runtime/amd64.S}
        \mintc[firstline=234,lastline=237,highlightlines={235-237}]{../ocaml/runtime/amd64.S}
      }
      \onslide*<1>{\listCamlRaiseExn[highlightlines=720]{}}
      \onslide*<2>{\listCamlRaiseExn[highlightlines=722]{}}
      \onslide*<3->{\providecommand\step{5}\input{diagrams/backtrace/backtrace.tex}}
    \end{column}
  \end{columns}
\end{frame}

\begin{frame}{Backtraces}{\funcname{caml\_probably\_raise}'s handler doesn't catch \typename{ExnC}}
  \begin{columns}[c]
    \begin{column}{0.45\textwidth}
      \minipage[c][0.75\textheight][s]{\columnwidth}
      \begin{itemize}
        \item<1-> \funcname{match \dots with}\footnotemark pattern matching can also match exceptions
        \item<1-> The CMM of \funcname{probably\_raise} shows that this \funcname{match \dots with} is implemented with a pattern matching and a \funcname{try \dots with}
        \item<1-> But this one only matches on \typename{ExnA} exception
        \item<2-> Since the pattern matching fails to catch the exception, \funcname{reraise} is called with our current \typename{ExnC} exception
        \item<3-> Disassembly shows that \funcname{reraise} is actually a call to \funcname{caml\_reraise\_exn}, which is also an assembly function provided by the runtime
      \end{itemize}
      \vfill
      \mintocaml[firstline=15,lastline=21,highlightlines={18-21}]{../src/backtrace/backtrace.ml}
      \endminipage
    \end{column}
    \begin{column}{0.55\textwidth}
      \centering
      \onslide*<1>{\mintcmm[highlightlines={10-18}]{../src/backtrace/camlBacktrace\_\_probably\_raise\_351.cmm}}
      \onslide*<2>{\listcmm[highlightlines=18]{../src/backtrace/camlBacktrace\_\_probably\_raise_351.cmm}{CMM of probably\_raise}}
      \onslide*<3>{\mintobjdump[firstline=5,lastline=35,highlightlines=31]{../src/backtrace/camlBacktrace\_\_probably\_raise_351.objdump}}
    \end{column}
  \end{columns}
  \only<1>{\footnotetext[1]{https://blog.janestreet.com/pattern-matching-and-exception-handling-unite/}}
\end{frame}

\newcommand\listCamlReraiseExn[1][]{
  \listasm[firstline=727,lastline=734,#1]{../ocaml/runtime/amd64.S}{runtime/amd64.S:727}}

\begin{frame}{Backtraces}{Introducing \funcname{caml\_reraise\_exn}}
  \begin{columns}[c]
    \begin{column}{0.5\textwidth}
      \minipage[c][0.66\textheight][s]{\columnwidth}
      \begin{itemize}
        \item<1-> The exception have to be reraised to the next handler
        \item<2-> First, checks if the backtrace should be collected with \funcarg{domain\_state->backtrace\_active}
        \only<3>{
        \begin{itemize}
            \item If not, behaves like a \funcname{raise\_notrace} with \funcname{RESTORE\_EXN\_HANDLER\_OCAML}
        \end{itemize}
        }
        \item<4-> Jumps into \funcname{caml\_raise\_exn}
        \item<5-> Calls \funcname{caml\_stash\_backtrace} again, to scan the next part of the stack: from the trap just pop-ed to the next trap
      \end{itemize}
      \vfill
      \mintocaml[firstline=15,lastline=21,highlightlines={18-21}]{../src/backtrace/backtrace.ml}
      \endminipage
    \end{column}
    \begin{column}{0.5\textwidth}
      \raggedleft
      \onslide*<1>{\listCamlReraiseExn{}}
      \onslide*<2>{\listCamlReraiseExn[highlightlines=729]{}}
      \onslide*<3>{
        \mintc[firstline=190,lastline=192,highlightlines={191-192}]{../ocaml/runtime/amd64.S}
        \mintc[firstline=234,lastline=237,highlightlines={235-237}]{../ocaml/runtime/amd64.S}
        \medskip
        \listCamlReraiseExn[highlightlines=731]{}
      }
      \onslide*<4-5>{
        % TODO refactor with \listCamlReraiseExn
        \mintasm[firstline=727,lastline=734,highlightlines=730]{../ocaml/runtime/amd64.S}
        \medskip
      }
      \onslide*<4>{\listCamlRaiseExn[highlightlines=711]{}}
      \onslide*<5>{\listCamlRaiseExn[highlightlines=720]{}}
    \end{column}
  \end{columns}
\end{frame}

\begin{frame}{Backtraces}{Backtrace collection continues}
  \begin{columns}[c]
    \begin{column}{0.55\textwidth}
      \minipage[c][0.75\textheight][s]{\columnwidth}
      \begin{itemize}
        \item<1-> \funcname{caml\_stash\_backtrace} scans the older part of the stack, up to the next trap
        \item<6-> Controls jumps to the nested exception handler in \funcname{maybe\_raise}, which matches only on \typename{ExnB}
        \item<7-> \funcname{caml\_reraise\_exn} is called again, backtrace collection resumes with \funcname{caml\_stash\_backtrace}
      \end{itemize}
      \medskip
      \begin{itemize}
        \item<2-> Constructed backtrace:
        \begin{itemize}
          \item <2-> \funcname{definitely\_raise}
          \item <2-> \funcname{probably\_raise}
          \item <3-> \funcname{probably\_raise} (re-raise)
          \item <4-> \funcname{maybe\_raise}
          \item <5-> \funcname{main}
          \item <8-> \funcname{main} (re-raise)
        \end{itemize}
      \end{itemize}
      \vfill
      \onslide*<1-5>{\mintocaml[firstline=15,lastline=21]{../src/backtrace/backtrace.ml}}
      \onslide*<6>{\mintocaml[firstline=28,lastline=34,highlightlines=31]{../src/backtrace/backtrace.ml}}
      \onslide*<7->{\mintocaml[firstline=28,lastline=34,highlightlines=33]{../src/backtrace/backtrace.ml}}
      \endminipage
    \end{column}
    \begin{column}{0.45\textwidth}
      \raggedleft
      \onslide*<1>{\providecommand\step{6}\input{diagrams/backtrace/backtrace.tex}}%
      \onslide*<2>{\providecommand\step{6}\input{diagrams/backtrace/backtrace.tex}}%
      \onslide*<3>{\providecommand\step{7}\input{diagrams/backtrace/backtrace.tex}}%
      \onslide*<4>{\providecommand\step{8}\input{diagrams/backtrace/backtrace.tex}}%
      \onslide*<5>{\providecommand\step{9}\input{diagrams/backtrace/backtrace.tex}}%
      \onslide*<6>{\providecommand\step{10}\input{diagrams/backtrace/backtrace.tex}}%
      \onslide*<7>{\providecommand\step{11}\input{diagrams/backtrace/backtrace.tex}}%
      \onslide*<8>{\providecommand\step{12}\input{diagrams/backtrace/backtrace.tex}}%
      \onslide*<9>{\providecommand\step{13}\input{diagrams/backtrace/backtrace.tex}}%
    \end{column}
  \end{columns}
\end{frame}

\begin{frame}{Backtraces}{Printing the backtrace}
  \begin{columns}[c]
    \begin{column}{0.45\textwidth}
      \minipage[c][0.66\textheight][s]{\columnwidth}
      \begin{itemize}
        \item<1-> The exception backtrace can now be printed to \funcarg{stdout} with \funcname{Printexc.print_backtrace}
        \item<2-> First the exception backtrace is retrieved into an OCaml array with \funcname{get_raw_backtrace }
        \item<3-> Which is implemented as the \funcname{caml_get_exception_raw_backtrace} C function in the runtime
        \item<4-> And finally printed with \funcname{print_exception_backtrace}
      \end{itemize}
      \vfill
      \mintocaml[firstline=28,lastline=34,highlightlines=33]{../src/backtrace/backtrace.ml}
      \endminipage
    \end{column}
    \begin{column}{0.55\textwidth}
      \onslide*<1,2>{
        \mintocaml[firstline=100,lastline=101]{../ocaml/stdlib/printexc.ml}
      }
      \only<1>{
        \listocaml[firstline=170,lastline=172]{../ocaml/stdlib/printexc.ml}{stdlib/printexc.ml:170}
      }
      \only<2>{
        \listocaml[firstline=170,lastline=172,highlightlines=172]{../ocaml/stdlib/printexc.ml}{stdlib/printexc.ml:170}
      }
      \only<3>{
        \mintc[firstline=154,lastline=156]{../ocaml/runtime/backtrace.c}
        \listc[firstline=171,lastline=192,highlightlines={184-187}]{../ocaml/runtime/backtrace.c}{runtime/backtrace.c:154}
      }
      \only<4>{
        \listocaml[firstline=155,lastline=165]{../ocaml/stdlib/printexc.ml}{stdlib/printexc.ml:55}
      }
    \end{column}
  \end{columns}
\end{frame}


\subsection{The multiple kinds of raise}
\frameSubsection{}{}

\begin{frame}{Backtraces}{When backtraces are not needed}
  \begin{columns}[c]
    \begin{column}{0.45\textwidth}
      \minipage[c][0.66\textheight][s]{\columnwidth}
      \begin{itemize}
        \item<1-> What if the backtrace for an exception is never needed?
        \item<2-> It's possible to raise the exception explicitly with \funcname{raise\_notrace} to make raising as cheap as possible
        \item<3-> An example of use in the \funcname{dynlink} library of the OCaml compiler
      \end{itemize}
      \vfill
      \onslide<3->{
      \listocaml[firstline=205,lastline=209,highlightlines=207]{../ocaml/otherlibs/dynlink/dynlink\_compilerlibs/misc.ml}{otherlibs/dynlink/dynlink\_compilerlibs/misc.ml:205}
      }
      \endminipage
    \end{column}
    \begin{column}{0.55\textwidth}
    \only<4>{
      \mintcmm[highlightlines=3]{../src/notrace/camlNotrace\_\_fun_347.cmm}
      \listcmm{../src/notrace/camlNotrace\_\_all\_somes\_327.cmm}{all\_somes}
    }
    \onslide*<5->{
      \listobjdump[highlightlines={6-9}]{../src/notrace/camlNotrace\_\_fun_347.objdump}{anonymous function from \funcname{all\_somes}}
    }
    \end{column}
  \end{columns}
\end{frame}

\begin{frame}{Backtraces}{Summary table of the multiple kinds of raise}
  \begin{itemize}
    \item When debugging information is enabled and if the backtrace of an exception isn't needed, it's possible to raise with \funcname{raise\_notrace} for performance
    \item When debugging information is \emph{not} enabled, the compiler optimises \funcname{raise} calls to inline assembly (same effect as using \funcname{raise\_notrace})
  \end{itemize}
  \bigskip
  \centering
  \begin{tabular}{| l | c | c | c | c |}
    \hline
    \parbox{10em}{Debug information \\ (\funcarg{-g} flag)}  & \multicolumn{2}{|c|}{Disabled}                                     & \multicolumn{2}{|c|}{Enabled} \\
    \hline
    Raise kind                                               & \funcname{raise} & \funcname{raise\_notrace}                       & \funcname{raise} & \funcname{raise\_notrace} \\
    \hline
    Compilation                                              & \parbox{8em}{inline assembly\\ (optimization)} & inline assembly   & \funcname{caml\_raise\_exn} & inline assembly \\
    \hline
  \end{tabular}
\end{frame}


%
% Takeaway #5
%
\subsection*{Takeaway \#5}
\frameSubsectionTakeaway{}
\againframe<5>{takeaway}
}%
      \onslide*<6>{\providecommand\step{10}%
% Backtraces
%
\subsection{One advantage of raising with debugging information}
\frameSubsection{}{}

\begin{frame}{Backtraces}{Debugging information \& source code}
  \begin{columns}[c]
    \begin{column}{0.5\textwidth}
      \begin{itemize}
        \item Raising an exception is fun and games, but how do we know where the exception originated? $\rightarrow$ With the backtrace associated with the exception!
        \item In order to get a backtrace from the point where an exception was raised, it's necessary to compile with debugging information enabled (\funcarg{-g} flag for the compiler)
        \item Same example as in the `Nested handlers' section
      \end{itemize}
    \end{column}
    \begin{column}{0.5\textwidth}
      \listocaml[firstline=9,lastline=34]{../src/backtrace/backtrace.ml}{backtrace/backtrace.ml}
    \end{column}
  \end{columns}
\end{frame}

\begin{frame}{Backtraces}{CMM and disassembly of \funcname{definitely\_raise}}
  \begin{columns}[c]
    \begin{column}{0.5\textwidth}
      \minipage[c][0.66\textheight][s]{\columnwidth}
      \begin{itemize}
        \item<1-> An effect of compiling with debugging information (\funcarg{-g}): the CMM no longer uses \funcname{raise\_notrace} to raise the exception but \funcname{raise} instead
        \item<2-> Disassembly shows that CMM \funcname{raise} is actually a call to \funcname{caml\_raise\_exn}, a function in assembler provided by the runtime
      \end{itemize}
      \vfill
      \mintocaml[firstline=9,lastline=13,highlightlines=12]{../src/backtrace/backtrace.ml}
      \endminipage
    \end{column}
    \begin{column}{0.5\textwidth}
      \onslide*<1>{
        \mintcmm[highlightlines=5]{../src/backtrace/camlBacktrace\_\_definitely\_raise\_347.cmm}
      }
      \onslide*<2>{
        \mintobjdump[firstline=2,highlightlines=14]{../src/backtrace/camlBacktrace\_\_definitely\_raise\_347.objdump}
      }
    \end{column}
  \end{columns}
\end{frame}

\begin{frame}{Backtraces}{Introducing \funcname{caml\_raise\_exn}}
  \begin{columns}[c]
    \begin{column}{0.5\textwidth}
      \minipage[c][0.66\textheight][s]{\columnwidth}
      \onslide*<1>{
        \begin{itemize}
          \item Raising the exception is performed using \funcname{caml\_raise\_exn} which is
            \begin{itemize}
              \item An assembly function provided by the runtime
              \item Architecture specific
            \end{itemize}
          \item Let's see what it does differently from \funcname{raise\_notrace}\dots
          \item We are about to raise \typename{ExnC} with \funcname{caml\_raise\_exn} from \funcname{definitely\_raise}
        \end{itemize}
      }
      \onslide*<2>{
        \mintobjdump[firstline=2,highlightlines=14]{../src/backtrace/camlBacktrace\_\_definitely\_raise\_347.objdump}
      }
      \vfill
      \mintocaml[firstline=9,lastline=13,highlightlines=12]{../src/backtrace/backtrace.ml}
      \endminipage
    \end{column}
    \begin{column}{0.5\textwidth}
      \centering
      \providecommand\step{1}\input{diagrams/backtrace/backtrace.tex}
    \end{column}
  \end{columns}
\end{frame}

\begin{frame}{Backtraces}{But before that, we need more fields from \typename{caml\_domain\_state}}
  \begin{columns}
    \begin{column}{0.5\textwidth}
      \begin{itemize}
        \item Remember \typename{caml\_domain\_state} the per-domain runtime state?
        \item Some other members are of interest to us now\dots
        \item \localname{backtrace\_active} is used to know if the runtime should collect backtraces
        \item \localname{backtrace\_active} can be set by
          \begin{itemize}
            \item The \funcarg{b} flag in the \funcname{OCAMLRUNPARAM} environment variable
            \item \mintinline{ocaml}{Printexc.record_backtrace : bool -> unit} \footnotemark
          \end{itemize}
      \end{itemize}
    \end{column}
    \begin{column}{0.5\textwidth}
      \begin{listing}
        \mintc[highlightlines={9-11}]{src/ocaml/domain\_state\_backtrace.h}
        \caption{runtime/caml/domain\_state.h}
      \end{listing}
    \end{column}
  \end{columns}
  \footnotetext[1]{https://v2.ocaml.org/api/Printexc.html}
\end{frame}

\newcommand\listCamlRaiseExn[1][]{
  \listasm[firstline=702,lastline=725,#1]{../ocaml/runtime/amd64.S}{runtime/amd64.S:702}}

\begin{frame}{Backtraces}{Walk through \funcname{caml\_raise\_exn}}
  \begin{columns}[T]
    \begin{column}{0.5\textwidth}
      \begin{itemize}
        \item<1-> First checks whether exceptions backtrace should be recorded
        \item<1-> By checking the \funcarg{domain\_state->backtrace\_active} value
        \item<2-> If not, acts as \funcname{raise\_notrace} with \funcname{RESTORE\_EXN\_HANDLER\_OCAML}
          \begin{itemize}
            \item Set \regname{RSP} to \localname{domain\_state->exn\_handler} value
            \item Restore \localname{domain\_state->exn\_handler} from trap
            \item Jump with \funcname{ret} (same effect as using \funcname{pop} and \funcname{jmp})
          \end{itemize}
        \item<3-> Otherwise, switches to the C stack to be able to execute C code
        \item<4-> Calls \funcname{caml\_stash\_backtrace}, a C function provided by the runtime\ignorespaces
          \onslide<6->{, with
            \begin{itemize}
              \item \funcarg{exn}: exception value
              \item \funcarg{pc}: code address of the raising point
              \item \funcarg{sp}: stack address of the raising function
              \item \funcarg{trapsp}: stack address of the next handler
            \end{itemize}
          }
      \end{itemize}
      \bigskip
      \mintocaml[firstline=9,lastline=13,highlightlines=12]{../src/backtrace/backtrace.ml}
    \end{column}
    \begin{column}{0.5\textwidth}
      \raggedleft
      \onslide*<1,3-4>{
        \mintc[firstline=190,lastline=192]{../ocaml/runtime/amd64.S}
        \mintc[firstline=234,lastline=237]{../ocaml/runtime/amd64.S}
      }
      \onslide*<2>{
        \mintc[firstline=190,lastline=192,highlightlines={191-192}]{../ocaml/runtime/amd64.S}
        \mintc[firstline=234,lastline=237,highlightlines={235-237}]{../ocaml/runtime/amd64.S}
      }
      \onslide*<1>{\listCamlRaiseExn[highlightlines=705]{}}%
      \onslide*<2>{\listCamlRaiseExn[highlightlines={707-708}]{}}%
      \onslide*<3>{\listCamlRaiseExn[highlightlines={712-714}]{}}%
      \onslide*<4>{\listCamlRaiseExn[highlightlines={715-720}]{}}%
      \onslide*<5>{\providecommand\step{1}\input{diagrams/backtrace/backtrace.tex}}%
      \onslide*<6>{\providecommand\step{2}\input{diagrams/backtrace/backtrace.tex}}%
    \end{column}
  \end{columns}
\end{frame}

\newcommand\listStashBacktrace[1][]{
  \listc[firstline=91,lastline=120,#1]{../ocaml/runtime/backtrace\_nat.c}{runtime/backtrace\_nat.c:91}}

\begin{frame}{Backtraces}{Introducing \funcname{caml\_stash\_backtrace}}
  \begin{columns}[c]
    \begin{column}{0.5\textwidth}
      \minipage[c][0.66\textheight][s]{\columnwidth}
      \begin{itemize}
        \item<1-> A C function provided by the runtime (specific to native code)
        \item<2-> It scans the stack, between the stack pointer at the raising point up to the next trap, in order to collect the names of the detected function frames
          \onslide*<2>{
            \begin{itemize}
              \item And more information, like line number, column number...
            \end{itemize}
          }
        \item<3-> It uses the same technique and information as the Garbage Collector (GC) uses to scan the values live on the stack
          \onslide*<4>{
            \begin{itemize}
              \item \typename{frame\_descr} are at the core of stack unwinding
            \end{itemize}
          }
      \end{itemize}
      \vfill
      \mintocaml[firstline=9,lastline=13,highlightlines=12]{../src/backtrace/backtrace.ml}
      \endminipage
    \end{column}
    \begin{column}{0.5\textwidth}
      \onslide*<1>{\listStashBacktrace[highlightlines=91]{}}
      \onslide*<2>{\listStashBacktrace[highlightlines={91,117-118}]{}}
      \onslide*<3->{\listStashBacktrace[highlightlines={94,106,109-110}]{}}
    \end{column}
  \end{columns}
\end{frame}

\newcommand\listFrameDescr[1][]{
  \listocaml[firstline=111,lastline=124,#1]{../ocaml/asmcomp/emitaux.ml}{asmcomp/emitaux.ml:111}}
\newcommand\listDebugInfo[1][]{
  \listocaml[firstline=38,lastline=49,#1]{../ocaml/lambda/debuginfo.mli}{lambda/debuginfo.mli:38}}

\begin{frame}{Backtraces}{\typename{frame\_descr} and \typename{frame\_debuginfo} in a nutshell}
  \begin{columns}[c]
    \begin{column}{0.5\textwidth}
      \begin{itemize}
        \item<1-> For every return address in an OCaml program, there is a associated \typename{frame\_descr}
        \item<2-> A \typename{frame\_descr} specifies
        \begin{itemize}
          \item<2-> The size of the function stack frame
          \item<3-> Where live values are on the stack (so that the GC can mark them as live and prevent from being swept)
        \end{itemize}
        \item<4-> When compiled with debug information, \typename{frame\_descr} will be linked to a \typename{frame\_debuginfo}
        \begin{itemize}
          \item<5-> Contains information about the position in the source code (file name, line and column number)
        \end{itemize}
      \end{itemize}
    \end{column}
    \begin{column}{0.5\textwidth}
        \onslide*<1>{\listFrameDescr[highlightlines={118-122}]{}}
        \onslide*<2>{\listFrameDescr[highlightlines={120}]{}}
        \onslide*<3>{\listFrameDescr[highlightlines={121}]{}}
        \onslide*<4->{\listFrameDescr[highlightlines={122}]{}}
        \medskip
        \onslide*<1-3>{\listDebugInfo{}}
        \onslide*<4>{\listDebugInfo[highlightlines={38,49}]{}}
        \onslide*<5>{\listDebugInfo[highlightlines={39-46}]{}}
    \end{column}
  \end{columns}
\end{frame}

\begin{frame}{Backtraces}{Scanning the stack with \typename{frame\_descr}}
  \begin{columns}[c]
    \begin{column}{0.5\textwidth}
      \begin{itemize}
        \item Given the return address of the code that raises the exception by calling \funcname{caml\_raise\_exn} (\funcarg{pc} argument of \funcname{caml\_stash\_backtrace})
        \item And the position of the stack pointer (\funcarg{sp} argument of \funcname{caml\_stash\_backtrace})
        \item The frame size given by \typename{frame\_descr}.\funcarg{frame\_size}, allows to find the end of the previous frame
        \item And the return address of the caller of this function
        \bigskip
        \onslide<3->{
        \item Note that traps (here the reddish \funcname{ExnA}, \funcname{ExnB} and \funcname{ExnC} blocks) belong to the frame that installed them, and so the frame size changes when traps are removed
        }
      \end{itemize}
    \end{column}
    \begin{column}{0.5\textwidth}
      \raggedleft
      \onslide*<1>{\providecommand\step{2}\input{diagrams/backtrace/backtrace.tex}}%
      \onslide*<2>{\providecommand\step{1}\input{diagrams/backtrace/scanstack.tex}}%
      \onslide*<3>{\providecommand\step{2}\input{diagrams/backtrace/scanstack.tex}}%
      \onslide*<4>{\providecommand\step{3}\input{diagrams/backtrace/scanstack.tex}}%
      \onslide*<5>{\providecommand\step{4}\input{diagrams/backtrace/scanstack.tex}}%
      \onslide*<6>{\providecommand\step{5}\input{diagrams/backtrace/scanstack.tex}}%
    \end{column}
  \end{columns}
\end{frame}

% Duplicate of previous-previous slide
\begin{frame}{Backtraces}{Continuing on \funcname{caml\_stash\_backtrace}}
  \begin{columns}[c]
    \begin{column}{0.5\textwidth}
      \minipage[c][0.75\textheight][s]{\columnwidth}
      \begin{itemize}
          \color{gray}
        \item A C function provided by the runtime (specific to native code)
        \item It scans the stack, between the stack pointer at the raising point up to the next trap, in order to collect the names of the detected function frames
        \item It uses the same technique and information as the Garbage Collector (GC) uses to scan the values live on the stack
      \end{itemize}
      \begin{itemize}
        \item For each function frame detected, store a pointer to the \typename{frame\_descr} in the \localname{domain\_state->backtrace\_buffer} array, effectively constructing the backtrace
      \end{itemize}
      \onslide<2->{
        \begin{itemize}
            \smallskip
          \item Constructed backtrace:
            \funcname{definitely\_raise},
            \onslide<3->{\funcname{probably\_raise}}
        \end{itemize}
      }
      \vfill
      \mintocaml[firstline=9,lastline=13,highlightlines=12]{../src/backtrace/backtrace.ml}
      \endminipage
    \end{column}
    \begin{column}{0.5\textwidth}
      \raggedleft
      \onslide*<1>{\listStashBacktrace[highlightlines={114-115}]{}}
      \onslide*<2>{\providecommand\step{3}\input{diagrams/backtrace/backtrace.tex}}%
      \onslide*<3>{\providecommand\step{4}\input{diagrams/backtrace/backtrace.tex}}%
    \end{column}
  \end{columns}
\end{frame}

\begin{frame}{Backtraces}{Back to \funcname{caml\_raise\_exn}}
  \begin{columns}[c]
    \begin{column}{0.5\textwidth}
      \minipage[c][0.66\textheight][s]{\columnwidth}
      \begin{itemize}\color{gray}
        \item Checks wheteher exceptions backtrace should be recorded, by checking the \funcarg{domain\_state->backtrace\_active} value
        \item Switches to the C stack to be able to execute C code
        \item Calls \funcname{caml\_stash\_backtrace}, to collect the backtrace up to the next trap
      \end{itemize}
      \onslide*<2->{
        \begin{itemize}
          \item<2-> Executes the exception handler in \funcname{probably\_raise} with \funcname{RESTORE\_EXN\_HANDLER\_OCAML}
            \begin{itemize}
              \item Set \regname{RSP} to \localname{domain\_state->exn\_handler} value
              \item Restore \localname{domain\_state->exn\_handler} from trap
              \item Jump to the handler code with \funcname{ret}
            \end{itemize}
        \end{itemize}
      }
      \vfill
      \onslide*<1>{\mintocaml[firstline=9,lastline=13,highlightlines=12]{../src/backtrace/backtrace.ml}}
      \onslide*<2->{\mintocaml[firstline=15,lastline=21,highlightlines={19-21}]{../src/backtrace/backtrace.ml}}
      \endminipage
    \end{column}
    \begin{column}{0.5\textwidth}
      \centering
      \onslide*<1>{
        \mintc[firstline=190,lastline=192]{../ocaml/runtime/amd64.S}
        \mintc[firstline=234,lastline=237]{../ocaml/runtime/amd64.S}
      }
      \onslide*<2>{
        \mintc[firstline=190,lastline=192,highlightlines={191-192}]{../ocaml/runtime/amd64.S}
        \mintc[firstline=234,lastline=237,highlightlines={235-237}]{../ocaml/runtime/amd64.S}
      }
      \onslide*<1>{\listCamlRaiseExn[highlightlines=720]{}}
      \onslide*<2>{\listCamlRaiseExn[highlightlines=722]{}}
      \onslide*<3->{\providecommand\step{5}\input{diagrams/backtrace/backtrace.tex}}
    \end{column}
  \end{columns}
\end{frame}

\begin{frame}{Backtraces}{\funcname{caml\_probably\_raise}'s handler doesn't catch \typename{ExnC}}
  \begin{columns}[c]
    \begin{column}{0.45\textwidth}
      \minipage[c][0.75\textheight][s]{\columnwidth}
      \begin{itemize}
        \item<1-> \funcname{match \dots with}\footnotemark pattern matching can also match exceptions
        \item<1-> The CMM of \funcname{probably\_raise} shows that this \funcname{match \dots with} is implemented with a pattern matching and a \funcname{try \dots with}
        \item<1-> But this one only matches on \typename{ExnA} exception
        \item<2-> Since the pattern matching fails to catch the exception, \funcname{reraise} is called with our current \typename{ExnC} exception
        \item<3-> Disassembly shows that \funcname{reraise} is actually a call to \funcname{caml\_reraise\_exn}, which is also an assembly function provided by the runtime
      \end{itemize}
      \vfill
      \mintocaml[firstline=15,lastline=21,highlightlines={18-21}]{../src/backtrace/backtrace.ml}
      \endminipage
    \end{column}
    \begin{column}{0.55\textwidth}
      \centering
      \onslide*<1>{\mintcmm[highlightlines={10-18}]{../src/backtrace/camlBacktrace\_\_probably\_raise\_351.cmm}}
      \onslide*<2>{\listcmm[highlightlines=18]{../src/backtrace/camlBacktrace\_\_probably\_raise_351.cmm}{CMM of probably\_raise}}
      \onslide*<3>{\mintobjdump[firstline=5,lastline=35,highlightlines=31]{../src/backtrace/camlBacktrace\_\_probably\_raise_351.objdump}}
    \end{column}
  \end{columns}
  \only<1>{\footnotetext[1]{https://blog.janestreet.com/pattern-matching-and-exception-handling-unite/}}
\end{frame}

\newcommand\listCamlReraiseExn[1][]{
  \listasm[firstline=727,lastline=734,#1]{../ocaml/runtime/amd64.S}{runtime/amd64.S:727}}

\begin{frame}{Backtraces}{Introducing \funcname{caml\_reraise\_exn}}
  \begin{columns}[c]
    \begin{column}{0.5\textwidth}
      \minipage[c][0.66\textheight][s]{\columnwidth}
      \begin{itemize}
        \item<1-> The exception have to be reraised to the next handler
        \item<2-> First, checks if the backtrace should be collected with \funcarg{domain\_state->backtrace\_active}
        \only<3>{
        \begin{itemize}
            \item If not, behaves like a \funcname{raise\_notrace} with \funcname{RESTORE\_EXN\_HANDLER\_OCAML}
        \end{itemize}
        }
        \item<4-> Jumps into \funcname{caml\_raise\_exn}
        \item<5-> Calls \funcname{caml\_stash\_backtrace} again, to scan the next part of the stack: from the trap just pop-ed to the next trap
      \end{itemize}
      \vfill
      \mintocaml[firstline=15,lastline=21,highlightlines={18-21}]{../src/backtrace/backtrace.ml}
      \endminipage
    \end{column}
    \begin{column}{0.5\textwidth}
      \raggedleft
      \onslide*<1>{\listCamlReraiseExn{}}
      \onslide*<2>{\listCamlReraiseExn[highlightlines=729]{}}
      \onslide*<3>{
        \mintc[firstline=190,lastline=192,highlightlines={191-192}]{../ocaml/runtime/amd64.S}
        \mintc[firstline=234,lastline=237,highlightlines={235-237}]{../ocaml/runtime/amd64.S}
        \medskip
        \listCamlReraiseExn[highlightlines=731]{}
      }
      \onslide*<4-5>{
        % TODO refactor with \listCamlReraiseExn
        \mintasm[firstline=727,lastline=734,highlightlines=730]{../ocaml/runtime/amd64.S}
        \medskip
      }
      \onslide*<4>{\listCamlRaiseExn[highlightlines=711]{}}
      \onslide*<5>{\listCamlRaiseExn[highlightlines=720]{}}
    \end{column}
  \end{columns}
\end{frame}

\begin{frame}{Backtraces}{Backtrace collection continues}
  \begin{columns}[c]
    \begin{column}{0.55\textwidth}
      \minipage[c][0.75\textheight][s]{\columnwidth}
      \begin{itemize}
        \item<1-> \funcname{caml\_stash\_backtrace} scans the older part of the stack, up to the next trap
        \item<6-> Controls jumps to the nested exception handler in \funcname{maybe\_raise}, which matches only on \typename{ExnB}
        \item<7-> \funcname{caml\_reraise\_exn} is called again, backtrace collection resumes with \funcname{caml\_stash\_backtrace}
      \end{itemize}
      \medskip
      \begin{itemize}
        \item<2-> Constructed backtrace:
        \begin{itemize}
          \item <2-> \funcname{definitely\_raise}
          \item <2-> \funcname{probably\_raise}
          \item <3-> \funcname{probably\_raise} (re-raise)
          \item <4-> \funcname{maybe\_raise}
          \item <5-> \funcname{main}
          \item <8-> \funcname{main} (re-raise)
        \end{itemize}
      \end{itemize}
      \vfill
      \onslide*<1-5>{\mintocaml[firstline=15,lastline=21]{../src/backtrace/backtrace.ml}}
      \onslide*<6>{\mintocaml[firstline=28,lastline=34,highlightlines=31]{../src/backtrace/backtrace.ml}}
      \onslide*<7->{\mintocaml[firstline=28,lastline=34,highlightlines=33]{../src/backtrace/backtrace.ml}}
      \endminipage
    \end{column}
    \begin{column}{0.45\textwidth}
      \raggedleft
      \onslide*<1>{\providecommand\step{6}\input{diagrams/backtrace/backtrace.tex}}%
      \onslide*<2>{\providecommand\step{6}\input{diagrams/backtrace/backtrace.tex}}%
      \onslide*<3>{\providecommand\step{7}\input{diagrams/backtrace/backtrace.tex}}%
      \onslide*<4>{\providecommand\step{8}\input{diagrams/backtrace/backtrace.tex}}%
      \onslide*<5>{\providecommand\step{9}\input{diagrams/backtrace/backtrace.tex}}%
      \onslide*<6>{\providecommand\step{10}\input{diagrams/backtrace/backtrace.tex}}%
      \onslide*<7>{\providecommand\step{11}\input{diagrams/backtrace/backtrace.tex}}%
      \onslide*<8>{\providecommand\step{12}\input{diagrams/backtrace/backtrace.tex}}%
      \onslide*<9>{\providecommand\step{13}\input{diagrams/backtrace/backtrace.tex}}%
    \end{column}
  \end{columns}
\end{frame}

\begin{frame}{Backtraces}{Printing the backtrace}
  \begin{columns}[c]
    \begin{column}{0.45\textwidth}
      \minipage[c][0.66\textheight][s]{\columnwidth}
      \begin{itemize}
        \item<1-> The exception backtrace can now be printed to \funcarg{stdout} with \funcname{Printexc.print_backtrace}
        \item<2-> First the exception backtrace is retrieved into an OCaml array with \funcname{get_raw_backtrace }
        \item<3-> Which is implemented as the \funcname{caml_get_exception_raw_backtrace} C function in the runtime
        \item<4-> And finally printed with \funcname{print_exception_backtrace}
      \end{itemize}
      \vfill
      \mintocaml[firstline=28,lastline=34,highlightlines=33]{../src/backtrace/backtrace.ml}
      \endminipage
    \end{column}
    \begin{column}{0.55\textwidth}
      \onslide*<1,2>{
        \mintocaml[firstline=100,lastline=101]{../ocaml/stdlib/printexc.ml}
      }
      \only<1>{
        \listocaml[firstline=170,lastline=172]{../ocaml/stdlib/printexc.ml}{stdlib/printexc.ml:170}
      }
      \only<2>{
        \listocaml[firstline=170,lastline=172,highlightlines=172]{../ocaml/stdlib/printexc.ml}{stdlib/printexc.ml:170}
      }
      \only<3>{
        \mintc[firstline=154,lastline=156]{../ocaml/runtime/backtrace.c}
        \listc[firstline=171,lastline=192,highlightlines={184-187}]{../ocaml/runtime/backtrace.c}{runtime/backtrace.c:154}
      }
      \only<4>{
        \listocaml[firstline=155,lastline=165]{../ocaml/stdlib/printexc.ml}{stdlib/printexc.ml:55}
      }
    \end{column}
  \end{columns}
\end{frame}


\subsection{The multiple kinds of raise}
\frameSubsection{}{}

\begin{frame}{Backtraces}{When backtraces are not needed}
  \begin{columns}[c]
    \begin{column}{0.45\textwidth}
      \minipage[c][0.66\textheight][s]{\columnwidth}
      \begin{itemize}
        \item<1-> What if the backtrace for an exception is never needed?
        \item<2-> It's possible to raise the exception explicitly with \funcname{raise\_notrace} to make raising as cheap as possible
        \item<3-> An example of use in the \funcname{dynlink} library of the OCaml compiler
      \end{itemize}
      \vfill
      \onslide<3->{
      \listocaml[firstline=205,lastline=209,highlightlines=207]{../ocaml/otherlibs/dynlink/dynlink\_compilerlibs/misc.ml}{otherlibs/dynlink/dynlink\_compilerlibs/misc.ml:205}
      }
      \endminipage
    \end{column}
    \begin{column}{0.55\textwidth}
    \only<4>{
      \mintcmm[highlightlines=3]{../src/notrace/camlNotrace\_\_fun_347.cmm}
      \listcmm{../src/notrace/camlNotrace\_\_all\_somes\_327.cmm}{all\_somes}
    }
    \onslide*<5->{
      \listobjdump[highlightlines={6-9}]{../src/notrace/camlNotrace\_\_fun_347.objdump}{anonymous function from \funcname{all\_somes}}
    }
    \end{column}
  \end{columns}
\end{frame}

\begin{frame}{Backtraces}{Summary table of the multiple kinds of raise}
  \begin{itemize}
    \item When debugging information is enabled and if the backtrace of an exception isn't needed, it's possible to raise with \funcname{raise\_notrace} for performance
    \item When debugging information is \emph{not} enabled, the compiler optimises \funcname{raise} calls to inline assembly (same effect as using \funcname{raise\_notrace})
  \end{itemize}
  \bigskip
  \centering
  \begin{tabular}{| l | c | c | c | c |}
    \hline
    \parbox{10em}{Debug information \\ (\funcarg{-g} flag)}  & \multicolumn{2}{|c|}{Disabled}                                     & \multicolumn{2}{|c|}{Enabled} \\
    \hline
    Raise kind                                               & \funcname{raise} & \funcname{raise\_notrace}                       & \funcname{raise} & \funcname{raise\_notrace} \\
    \hline
    Compilation                                              & \parbox{8em}{inline assembly\\ (optimization)} & inline assembly   & \funcname{caml\_raise\_exn} & inline assembly \\
    \hline
  \end{tabular}
\end{frame}


%
% Takeaway #5
%
\subsection*{Takeaway \#5}
\frameSubsectionTakeaway{}
\againframe<5>{takeaway}
}%
      \onslide*<7>{\providecommand\step{11}%
% Backtraces
%
\subsection{One advantage of raising with debugging information}
\frameSubsection{}{}

\begin{frame}{Backtraces}{Debugging information \& source code}
  \begin{columns}[c]
    \begin{column}{0.5\textwidth}
      \begin{itemize}
        \item Raising an exception is fun and games, but how do we know where the exception originated? $\rightarrow$ With the backtrace associated with the exception!
        \item In order to get a backtrace from the point where an exception was raised, it's necessary to compile with debugging information enabled (\funcarg{-g} flag for the compiler)
        \item Same example as in the `Nested handlers' section
      \end{itemize}
    \end{column}
    \begin{column}{0.5\textwidth}
      \listocaml[firstline=9,lastline=34]{../src/backtrace/backtrace.ml}{backtrace/backtrace.ml}
    \end{column}
  \end{columns}
\end{frame}

\begin{frame}{Backtraces}{CMM and disassembly of \funcname{definitely\_raise}}
  \begin{columns}[c]
    \begin{column}{0.5\textwidth}
      \minipage[c][0.66\textheight][s]{\columnwidth}
      \begin{itemize}
        \item<1-> An effect of compiling with debugging information (\funcarg{-g}): the CMM no longer uses \funcname{raise\_notrace} to raise the exception but \funcname{raise} instead
        \item<2-> Disassembly shows that CMM \funcname{raise} is actually a call to \funcname{caml\_raise\_exn}, a function in assembler provided by the runtime
      \end{itemize}
      \vfill
      \mintocaml[firstline=9,lastline=13,highlightlines=12]{../src/backtrace/backtrace.ml}
      \endminipage
    \end{column}
    \begin{column}{0.5\textwidth}
      \onslide*<1>{
        \mintcmm[highlightlines=5]{../src/backtrace/camlBacktrace\_\_definitely\_raise\_347.cmm}
      }
      \onslide*<2>{
        \mintobjdump[firstline=2,highlightlines=14]{../src/backtrace/camlBacktrace\_\_definitely\_raise\_347.objdump}
      }
    \end{column}
  \end{columns}
\end{frame}

\begin{frame}{Backtraces}{Introducing \funcname{caml\_raise\_exn}}
  \begin{columns}[c]
    \begin{column}{0.5\textwidth}
      \minipage[c][0.66\textheight][s]{\columnwidth}
      \onslide*<1>{
        \begin{itemize}
          \item Raising the exception is performed using \funcname{caml\_raise\_exn} which is
            \begin{itemize}
              \item An assembly function provided by the runtime
              \item Architecture specific
            \end{itemize}
          \item Let's see what it does differently from \funcname{raise\_notrace}\dots
          \item We are about to raise \typename{ExnC} with \funcname{caml\_raise\_exn} from \funcname{definitely\_raise}
        \end{itemize}
      }
      \onslide*<2>{
        \mintobjdump[firstline=2,highlightlines=14]{../src/backtrace/camlBacktrace\_\_definitely\_raise\_347.objdump}
      }
      \vfill
      \mintocaml[firstline=9,lastline=13,highlightlines=12]{../src/backtrace/backtrace.ml}
      \endminipage
    \end{column}
    \begin{column}{0.5\textwidth}
      \centering
      \providecommand\step{1}\input{diagrams/backtrace/backtrace.tex}
    \end{column}
  \end{columns}
\end{frame}

\begin{frame}{Backtraces}{But before that, we need more fields from \typename{caml\_domain\_state}}
  \begin{columns}
    \begin{column}{0.5\textwidth}
      \begin{itemize}
        \item Remember \typename{caml\_domain\_state} the per-domain runtime state?
        \item Some other members are of interest to us now\dots
        \item \localname{backtrace\_active} is used to know if the runtime should collect backtraces
        \item \localname{backtrace\_active} can be set by
          \begin{itemize}
            \item The \funcarg{b} flag in the \funcname{OCAMLRUNPARAM} environment variable
            \item \mintinline{ocaml}{Printexc.record_backtrace : bool -> unit} \footnotemark
          \end{itemize}
      \end{itemize}
    \end{column}
    \begin{column}{0.5\textwidth}
      \begin{listing}
        \mintc[highlightlines={9-11}]{src/ocaml/domain\_state\_backtrace.h}
        \caption{runtime/caml/domain\_state.h}
      \end{listing}
    \end{column}
  \end{columns}
  \footnotetext[1]{https://v2.ocaml.org/api/Printexc.html}
\end{frame}

\newcommand\listCamlRaiseExn[1][]{
  \listasm[firstline=702,lastline=725,#1]{../ocaml/runtime/amd64.S}{runtime/amd64.S:702}}

\begin{frame}{Backtraces}{Walk through \funcname{caml\_raise\_exn}}
  \begin{columns}[T]
    \begin{column}{0.5\textwidth}
      \begin{itemize}
        \item<1-> First checks whether exceptions backtrace should be recorded
        \item<1-> By checking the \funcarg{domain\_state->backtrace\_active} value
        \item<2-> If not, acts as \funcname{raise\_notrace} with \funcname{RESTORE\_EXN\_HANDLER\_OCAML}
          \begin{itemize}
            \item Set \regname{RSP} to \localname{domain\_state->exn\_handler} value
            \item Restore \localname{domain\_state->exn\_handler} from trap
            \item Jump with \funcname{ret} (same effect as using \funcname{pop} and \funcname{jmp})
          \end{itemize}
        \item<3-> Otherwise, switches to the C stack to be able to execute C code
        \item<4-> Calls \funcname{caml\_stash\_backtrace}, a C function provided by the runtime\ignorespaces
          \onslide<6->{, with
            \begin{itemize}
              \item \funcarg{exn}: exception value
              \item \funcarg{pc}: code address of the raising point
              \item \funcarg{sp}: stack address of the raising function
              \item \funcarg{trapsp}: stack address of the next handler
            \end{itemize}
          }
      \end{itemize}
      \bigskip
      \mintocaml[firstline=9,lastline=13,highlightlines=12]{../src/backtrace/backtrace.ml}
    \end{column}
    \begin{column}{0.5\textwidth}
      \raggedleft
      \onslide*<1,3-4>{
        \mintc[firstline=190,lastline=192]{../ocaml/runtime/amd64.S}
        \mintc[firstline=234,lastline=237]{../ocaml/runtime/amd64.S}
      }
      \onslide*<2>{
        \mintc[firstline=190,lastline=192,highlightlines={191-192}]{../ocaml/runtime/amd64.S}
        \mintc[firstline=234,lastline=237,highlightlines={235-237}]{../ocaml/runtime/amd64.S}
      }
      \onslide*<1>{\listCamlRaiseExn[highlightlines=705]{}}%
      \onslide*<2>{\listCamlRaiseExn[highlightlines={707-708}]{}}%
      \onslide*<3>{\listCamlRaiseExn[highlightlines={712-714}]{}}%
      \onslide*<4>{\listCamlRaiseExn[highlightlines={715-720}]{}}%
      \onslide*<5>{\providecommand\step{1}\input{diagrams/backtrace/backtrace.tex}}%
      \onslide*<6>{\providecommand\step{2}\input{diagrams/backtrace/backtrace.tex}}%
    \end{column}
  \end{columns}
\end{frame}

\newcommand\listStashBacktrace[1][]{
  \listc[firstline=91,lastline=120,#1]{../ocaml/runtime/backtrace\_nat.c}{runtime/backtrace\_nat.c:91}}

\begin{frame}{Backtraces}{Introducing \funcname{caml\_stash\_backtrace}}
  \begin{columns}[c]
    \begin{column}{0.5\textwidth}
      \minipage[c][0.66\textheight][s]{\columnwidth}
      \begin{itemize}
        \item<1-> A C function provided by the runtime (specific to native code)
        \item<2-> It scans the stack, between the stack pointer at the raising point up to the next trap, in order to collect the names of the detected function frames
          \onslide*<2>{
            \begin{itemize}
              \item And more information, like line number, column number...
            \end{itemize}
          }
        \item<3-> It uses the same technique and information as the Garbage Collector (GC) uses to scan the values live on the stack
          \onslide*<4>{
            \begin{itemize}
              \item \typename{frame\_descr} are at the core of stack unwinding
            \end{itemize}
          }
      \end{itemize}
      \vfill
      \mintocaml[firstline=9,lastline=13,highlightlines=12]{../src/backtrace/backtrace.ml}
      \endminipage
    \end{column}
    \begin{column}{0.5\textwidth}
      \onslide*<1>{\listStashBacktrace[highlightlines=91]{}}
      \onslide*<2>{\listStashBacktrace[highlightlines={91,117-118}]{}}
      \onslide*<3->{\listStashBacktrace[highlightlines={94,106,109-110}]{}}
    \end{column}
  \end{columns}
\end{frame}

\newcommand\listFrameDescr[1][]{
  \listocaml[firstline=111,lastline=124,#1]{../ocaml/asmcomp/emitaux.ml}{asmcomp/emitaux.ml:111}}
\newcommand\listDebugInfo[1][]{
  \listocaml[firstline=38,lastline=49,#1]{../ocaml/lambda/debuginfo.mli}{lambda/debuginfo.mli:38}}

\begin{frame}{Backtraces}{\typename{frame\_descr} and \typename{frame\_debuginfo} in a nutshell}
  \begin{columns}[c]
    \begin{column}{0.5\textwidth}
      \begin{itemize}
        \item<1-> For every return address in an OCaml program, there is a associated \typename{frame\_descr}
        \item<2-> A \typename{frame\_descr} specifies
        \begin{itemize}
          \item<2-> The size of the function stack frame
          \item<3-> Where live values are on the stack (so that the GC can mark them as live and prevent from being swept)
        \end{itemize}
        \item<4-> When compiled with debug information, \typename{frame\_descr} will be linked to a \typename{frame\_debuginfo}
        \begin{itemize}
          \item<5-> Contains information about the position in the source code (file name, line and column number)
        \end{itemize}
      \end{itemize}
    \end{column}
    \begin{column}{0.5\textwidth}
        \onslide*<1>{\listFrameDescr[highlightlines={118-122}]{}}
        \onslide*<2>{\listFrameDescr[highlightlines={120}]{}}
        \onslide*<3>{\listFrameDescr[highlightlines={121}]{}}
        \onslide*<4->{\listFrameDescr[highlightlines={122}]{}}
        \medskip
        \onslide*<1-3>{\listDebugInfo{}}
        \onslide*<4>{\listDebugInfo[highlightlines={38,49}]{}}
        \onslide*<5>{\listDebugInfo[highlightlines={39-46}]{}}
    \end{column}
  \end{columns}
\end{frame}

\begin{frame}{Backtraces}{Scanning the stack with \typename{frame\_descr}}
  \begin{columns}[c]
    \begin{column}{0.5\textwidth}
      \begin{itemize}
        \item Given the return address of the code that raises the exception by calling \funcname{caml\_raise\_exn} (\funcarg{pc} argument of \funcname{caml\_stash\_backtrace})
        \item And the position of the stack pointer (\funcarg{sp} argument of \funcname{caml\_stash\_backtrace})
        \item The frame size given by \typename{frame\_descr}.\funcarg{frame\_size}, allows to find the end of the previous frame
        \item And the return address of the caller of this function
        \bigskip
        \onslide<3->{
        \item Note that traps (here the reddish \funcname{ExnA}, \funcname{ExnB} and \funcname{ExnC} blocks) belong to the frame that installed them, and so the frame size changes when traps are removed
        }
      \end{itemize}
    \end{column}
    \begin{column}{0.5\textwidth}
      \raggedleft
      \onslide*<1>{\providecommand\step{2}\input{diagrams/backtrace/backtrace.tex}}%
      \onslide*<2>{\providecommand\step{1}\input{diagrams/backtrace/scanstack.tex}}%
      \onslide*<3>{\providecommand\step{2}\input{diagrams/backtrace/scanstack.tex}}%
      \onslide*<4>{\providecommand\step{3}\input{diagrams/backtrace/scanstack.tex}}%
      \onslide*<5>{\providecommand\step{4}\input{diagrams/backtrace/scanstack.tex}}%
      \onslide*<6>{\providecommand\step{5}\input{diagrams/backtrace/scanstack.tex}}%
    \end{column}
  \end{columns}
\end{frame}

% Duplicate of previous-previous slide
\begin{frame}{Backtraces}{Continuing on \funcname{caml\_stash\_backtrace}}
  \begin{columns}[c]
    \begin{column}{0.5\textwidth}
      \minipage[c][0.75\textheight][s]{\columnwidth}
      \begin{itemize}
          \color{gray}
        \item A C function provided by the runtime (specific to native code)
        \item It scans the stack, between the stack pointer at the raising point up to the next trap, in order to collect the names of the detected function frames
        \item It uses the same technique and information as the Garbage Collector (GC) uses to scan the values live on the stack
      \end{itemize}
      \begin{itemize}
        \item For each function frame detected, store a pointer to the \typename{frame\_descr} in the \localname{domain\_state->backtrace\_buffer} array, effectively constructing the backtrace
      \end{itemize}
      \onslide<2->{
        \begin{itemize}
            \smallskip
          \item Constructed backtrace:
            \funcname{definitely\_raise},
            \onslide<3->{\funcname{probably\_raise}}
        \end{itemize}
      }
      \vfill
      \mintocaml[firstline=9,lastline=13,highlightlines=12]{../src/backtrace/backtrace.ml}
      \endminipage
    \end{column}
    \begin{column}{0.5\textwidth}
      \raggedleft
      \onslide*<1>{\listStashBacktrace[highlightlines={114-115}]{}}
      \onslide*<2>{\providecommand\step{3}\input{diagrams/backtrace/backtrace.tex}}%
      \onslide*<3>{\providecommand\step{4}\input{diagrams/backtrace/backtrace.tex}}%
    \end{column}
  \end{columns}
\end{frame}

\begin{frame}{Backtraces}{Back to \funcname{caml\_raise\_exn}}
  \begin{columns}[c]
    \begin{column}{0.5\textwidth}
      \minipage[c][0.66\textheight][s]{\columnwidth}
      \begin{itemize}\color{gray}
        \item Checks wheteher exceptions backtrace should be recorded, by checking the \funcarg{domain\_state->backtrace\_active} value
        \item Switches to the C stack to be able to execute C code
        \item Calls \funcname{caml\_stash\_backtrace}, to collect the backtrace up to the next trap
      \end{itemize}
      \onslide*<2->{
        \begin{itemize}
          \item<2-> Executes the exception handler in \funcname{probably\_raise} with \funcname{RESTORE\_EXN\_HANDLER\_OCAML}
            \begin{itemize}
              \item Set \regname{RSP} to \localname{domain\_state->exn\_handler} value
              \item Restore \localname{domain\_state->exn\_handler} from trap
              \item Jump to the handler code with \funcname{ret}
            \end{itemize}
        \end{itemize}
      }
      \vfill
      \onslide*<1>{\mintocaml[firstline=9,lastline=13,highlightlines=12]{../src/backtrace/backtrace.ml}}
      \onslide*<2->{\mintocaml[firstline=15,lastline=21,highlightlines={19-21}]{../src/backtrace/backtrace.ml}}
      \endminipage
    \end{column}
    \begin{column}{0.5\textwidth}
      \centering
      \onslide*<1>{
        \mintc[firstline=190,lastline=192]{../ocaml/runtime/amd64.S}
        \mintc[firstline=234,lastline=237]{../ocaml/runtime/amd64.S}
      }
      \onslide*<2>{
        \mintc[firstline=190,lastline=192,highlightlines={191-192}]{../ocaml/runtime/amd64.S}
        \mintc[firstline=234,lastline=237,highlightlines={235-237}]{../ocaml/runtime/amd64.S}
      }
      \onslide*<1>{\listCamlRaiseExn[highlightlines=720]{}}
      \onslide*<2>{\listCamlRaiseExn[highlightlines=722]{}}
      \onslide*<3->{\providecommand\step{5}\input{diagrams/backtrace/backtrace.tex}}
    \end{column}
  \end{columns}
\end{frame}

\begin{frame}{Backtraces}{\funcname{caml\_probably\_raise}'s handler doesn't catch \typename{ExnC}}
  \begin{columns}[c]
    \begin{column}{0.45\textwidth}
      \minipage[c][0.75\textheight][s]{\columnwidth}
      \begin{itemize}
        \item<1-> \funcname{match \dots with}\footnotemark pattern matching can also match exceptions
        \item<1-> The CMM of \funcname{probably\_raise} shows that this \funcname{match \dots with} is implemented with a pattern matching and a \funcname{try \dots with}
        \item<1-> But this one only matches on \typename{ExnA} exception
        \item<2-> Since the pattern matching fails to catch the exception, \funcname{reraise} is called with our current \typename{ExnC} exception
        \item<3-> Disassembly shows that \funcname{reraise} is actually a call to \funcname{caml\_reraise\_exn}, which is also an assembly function provided by the runtime
      \end{itemize}
      \vfill
      \mintocaml[firstline=15,lastline=21,highlightlines={18-21}]{../src/backtrace/backtrace.ml}
      \endminipage
    \end{column}
    \begin{column}{0.55\textwidth}
      \centering
      \onslide*<1>{\mintcmm[highlightlines={10-18}]{../src/backtrace/camlBacktrace\_\_probably\_raise\_351.cmm}}
      \onslide*<2>{\listcmm[highlightlines=18]{../src/backtrace/camlBacktrace\_\_probably\_raise_351.cmm}{CMM of probably\_raise}}
      \onslide*<3>{\mintobjdump[firstline=5,lastline=35,highlightlines=31]{../src/backtrace/camlBacktrace\_\_probably\_raise_351.objdump}}
    \end{column}
  \end{columns}
  \only<1>{\footnotetext[1]{https://blog.janestreet.com/pattern-matching-and-exception-handling-unite/}}
\end{frame}

\newcommand\listCamlReraiseExn[1][]{
  \listasm[firstline=727,lastline=734,#1]{../ocaml/runtime/amd64.S}{runtime/amd64.S:727}}

\begin{frame}{Backtraces}{Introducing \funcname{caml\_reraise\_exn}}
  \begin{columns}[c]
    \begin{column}{0.5\textwidth}
      \minipage[c][0.66\textheight][s]{\columnwidth}
      \begin{itemize}
        \item<1-> The exception have to be reraised to the next handler
        \item<2-> First, checks if the backtrace should be collected with \funcarg{domain\_state->backtrace\_active}
        \only<3>{
        \begin{itemize}
            \item If not, behaves like a \funcname{raise\_notrace} with \funcname{RESTORE\_EXN\_HANDLER\_OCAML}
        \end{itemize}
        }
        \item<4-> Jumps into \funcname{caml\_raise\_exn}
        \item<5-> Calls \funcname{caml\_stash\_backtrace} again, to scan the next part of the stack: from the trap just pop-ed to the next trap
      \end{itemize}
      \vfill
      \mintocaml[firstline=15,lastline=21,highlightlines={18-21}]{../src/backtrace/backtrace.ml}
      \endminipage
    \end{column}
    \begin{column}{0.5\textwidth}
      \raggedleft
      \onslide*<1>{\listCamlReraiseExn{}}
      \onslide*<2>{\listCamlReraiseExn[highlightlines=729]{}}
      \onslide*<3>{
        \mintc[firstline=190,lastline=192,highlightlines={191-192}]{../ocaml/runtime/amd64.S}
        \mintc[firstline=234,lastline=237,highlightlines={235-237}]{../ocaml/runtime/amd64.S}
        \medskip
        \listCamlReraiseExn[highlightlines=731]{}
      }
      \onslide*<4-5>{
        % TODO refactor with \listCamlReraiseExn
        \mintasm[firstline=727,lastline=734,highlightlines=730]{../ocaml/runtime/amd64.S}
        \medskip
      }
      \onslide*<4>{\listCamlRaiseExn[highlightlines=711]{}}
      \onslide*<5>{\listCamlRaiseExn[highlightlines=720]{}}
    \end{column}
  \end{columns}
\end{frame}

\begin{frame}{Backtraces}{Backtrace collection continues}
  \begin{columns}[c]
    \begin{column}{0.55\textwidth}
      \minipage[c][0.75\textheight][s]{\columnwidth}
      \begin{itemize}
        \item<1-> \funcname{caml\_stash\_backtrace} scans the older part of the stack, up to the next trap
        \item<6-> Controls jumps to the nested exception handler in \funcname{maybe\_raise}, which matches only on \typename{ExnB}
        \item<7-> \funcname{caml\_reraise\_exn} is called again, backtrace collection resumes with \funcname{caml\_stash\_backtrace}
      \end{itemize}
      \medskip
      \begin{itemize}
        \item<2-> Constructed backtrace:
        \begin{itemize}
          \item <2-> \funcname{definitely\_raise}
          \item <2-> \funcname{probably\_raise}
          \item <3-> \funcname{probably\_raise} (re-raise)
          \item <4-> \funcname{maybe\_raise}
          \item <5-> \funcname{main}
          \item <8-> \funcname{main} (re-raise)
        \end{itemize}
      \end{itemize}
      \vfill
      \onslide*<1-5>{\mintocaml[firstline=15,lastline=21]{../src/backtrace/backtrace.ml}}
      \onslide*<6>{\mintocaml[firstline=28,lastline=34,highlightlines=31]{../src/backtrace/backtrace.ml}}
      \onslide*<7->{\mintocaml[firstline=28,lastline=34,highlightlines=33]{../src/backtrace/backtrace.ml}}
      \endminipage
    \end{column}
    \begin{column}{0.45\textwidth}
      \raggedleft
      \onslide*<1>{\providecommand\step{6}\input{diagrams/backtrace/backtrace.tex}}%
      \onslide*<2>{\providecommand\step{6}\input{diagrams/backtrace/backtrace.tex}}%
      \onslide*<3>{\providecommand\step{7}\input{diagrams/backtrace/backtrace.tex}}%
      \onslide*<4>{\providecommand\step{8}\input{diagrams/backtrace/backtrace.tex}}%
      \onslide*<5>{\providecommand\step{9}\input{diagrams/backtrace/backtrace.tex}}%
      \onslide*<6>{\providecommand\step{10}\input{diagrams/backtrace/backtrace.tex}}%
      \onslide*<7>{\providecommand\step{11}\input{diagrams/backtrace/backtrace.tex}}%
      \onslide*<8>{\providecommand\step{12}\input{diagrams/backtrace/backtrace.tex}}%
      \onslide*<9>{\providecommand\step{13}\input{diagrams/backtrace/backtrace.tex}}%
    \end{column}
  \end{columns}
\end{frame}

\begin{frame}{Backtraces}{Printing the backtrace}
  \begin{columns}[c]
    \begin{column}{0.45\textwidth}
      \minipage[c][0.66\textheight][s]{\columnwidth}
      \begin{itemize}
        \item<1-> The exception backtrace can now be printed to \funcarg{stdout} with \funcname{Printexc.print_backtrace}
        \item<2-> First the exception backtrace is retrieved into an OCaml array with \funcname{get_raw_backtrace }
        \item<3-> Which is implemented as the \funcname{caml_get_exception_raw_backtrace} C function in the runtime
        \item<4-> And finally printed with \funcname{print_exception_backtrace}
      \end{itemize}
      \vfill
      \mintocaml[firstline=28,lastline=34,highlightlines=33]{../src/backtrace/backtrace.ml}
      \endminipage
    \end{column}
    \begin{column}{0.55\textwidth}
      \onslide*<1,2>{
        \mintocaml[firstline=100,lastline=101]{../ocaml/stdlib/printexc.ml}
      }
      \only<1>{
        \listocaml[firstline=170,lastline=172]{../ocaml/stdlib/printexc.ml}{stdlib/printexc.ml:170}
      }
      \only<2>{
        \listocaml[firstline=170,lastline=172,highlightlines=172]{../ocaml/stdlib/printexc.ml}{stdlib/printexc.ml:170}
      }
      \only<3>{
        \mintc[firstline=154,lastline=156]{../ocaml/runtime/backtrace.c}
        \listc[firstline=171,lastline=192,highlightlines={184-187}]{../ocaml/runtime/backtrace.c}{runtime/backtrace.c:154}
      }
      \only<4>{
        \listocaml[firstline=155,lastline=165]{../ocaml/stdlib/printexc.ml}{stdlib/printexc.ml:55}
      }
    \end{column}
  \end{columns}
\end{frame}


\subsection{The multiple kinds of raise}
\frameSubsection{}{}

\begin{frame}{Backtraces}{When backtraces are not needed}
  \begin{columns}[c]
    \begin{column}{0.45\textwidth}
      \minipage[c][0.66\textheight][s]{\columnwidth}
      \begin{itemize}
        \item<1-> What if the backtrace for an exception is never needed?
        \item<2-> It's possible to raise the exception explicitly with \funcname{raise\_notrace} to make raising as cheap as possible
        \item<3-> An example of use in the \funcname{dynlink} library of the OCaml compiler
      \end{itemize}
      \vfill
      \onslide<3->{
      \listocaml[firstline=205,lastline=209,highlightlines=207]{../ocaml/otherlibs/dynlink/dynlink\_compilerlibs/misc.ml}{otherlibs/dynlink/dynlink\_compilerlibs/misc.ml:205}
      }
      \endminipage
    \end{column}
    \begin{column}{0.55\textwidth}
    \only<4>{
      \mintcmm[highlightlines=3]{../src/notrace/camlNotrace\_\_fun_347.cmm}
      \listcmm{../src/notrace/camlNotrace\_\_all\_somes\_327.cmm}{all\_somes}
    }
    \onslide*<5->{
      \listobjdump[highlightlines={6-9}]{../src/notrace/camlNotrace\_\_fun_347.objdump}{anonymous function from \funcname{all\_somes}}
    }
    \end{column}
  \end{columns}
\end{frame}

\begin{frame}{Backtraces}{Summary table of the multiple kinds of raise}
  \begin{itemize}
    \item When debugging information is enabled and if the backtrace of an exception isn't needed, it's possible to raise with \funcname{raise\_notrace} for performance
    \item When debugging information is \emph{not} enabled, the compiler optimises \funcname{raise} calls to inline assembly (same effect as using \funcname{raise\_notrace})
  \end{itemize}
  \bigskip
  \centering
  \begin{tabular}{| l | c | c | c | c |}
    \hline
    \parbox{10em}{Debug information \\ (\funcarg{-g} flag)}  & \multicolumn{2}{|c|}{Disabled}                                     & \multicolumn{2}{|c|}{Enabled} \\
    \hline
    Raise kind                                               & \funcname{raise} & \funcname{raise\_notrace}                       & \funcname{raise} & \funcname{raise\_notrace} \\
    \hline
    Compilation                                              & \parbox{8em}{inline assembly\\ (optimization)} & inline assembly   & \funcname{caml\_raise\_exn} & inline assembly \\
    \hline
  \end{tabular}
\end{frame}


%
% Takeaway #5
%
\subsection*{Takeaway \#5}
\frameSubsectionTakeaway{}
\againframe<5>{takeaway}
}%
      \onslide*<8>{\providecommand\step{12}%
% Backtraces
%
\subsection{One advantage of raising with debugging information}
\frameSubsection{}{}

\begin{frame}{Backtraces}{Debugging information \& source code}
  \begin{columns}[c]
    \begin{column}{0.5\textwidth}
      \begin{itemize}
        \item Raising an exception is fun and games, but how do we know where the exception originated? $\rightarrow$ With the backtrace associated with the exception!
        \item In order to get a backtrace from the point where an exception was raised, it's necessary to compile with debugging information enabled (\funcarg{-g} flag for the compiler)
        \item Same example as in the `Nested handlers' section
      \end{itemize}
    \end{column}
    \begin{column}{0.5\textwidth}
      \listocaml[firstline=9,lastline=34]{../src/backtrace/backtrace.ml}{backtrace/backtrace.ml}
    \end{column}
  \end{columns}
\end{frame}

\begin{frame}{Backtraces}{CMM and disassembly of \funcname{definitely\_raise}}
  \begin{columns}[c]
    \begin{column}{0.5\textwidth}
      \minipage[c][0.66\textheight][s]{\columnwidth}
      \begin{itemize}
        \item<1-> An effect of compiling with debugging information (\funcarg{-g}): the CMM no longer uses \funcname{raise\_notrace} to raise the exception but \funcname{raise} instead
        \item<2-> Disassembly shows that CMM \funcname{raise} is actually a call to \funcname{caml\_raise\_exn}, a function in assembler provided by the runtime
      \end{itemize}
      \vfill
      \mintocaml[firstline=9,lastline=13,highlightlines=12]{../src/backtrace/backtrace.ml}
      \endminipage
    \end{column}
    \begin{column}{0.5\textwidth}
      \onslide*<1>{
        \mintcmm[highlightlines=5]{../src/backtrace/camlBacktrace\_\_definitely\_raise\_347.cmm}
      }
      \onslide*<2>{
        \mintobjdump[firstline=2,highlightlines=14]{../src/backtrace/camlBacktrace\_\_definitely\_raise\_347.objdump}
      }
    \end{column}
  \end{columns}
\end{frame}

\begin{frame}{Backtraces}{Introducing \funcname{caml\_raise\_exn}}
  \begin{columns}[c]
    \begin{column}{0.5\textwidth}
      \minipage[c][0.66\textheight][s]{\columnwidth}
      \onslide*<1>{
        \begin{itemize}
          \item Raising the exception is performed using \funcname{caml\_raise\_exn} which is
            \begin{itemize}
              \item An assembly function provided by the runtime
              \item Architecture specific
            \end{itemize}
          \item Let's see what it does differently from \funcname{raise\_notrace}\dots
          \item We are about to raise \typename{ExnC} with \funcname{caml\_raise\_exn} from \funcname{definitely\_raise}
        \end{itemize}
      }
      \onslide*<2>{
        \mintobjdump[firstline=2,highlightlines=14]{../src/backtrace/camlBacktrace\_\_definitely\_raise\_347.objdump}
      }
      \vfill
      \mintocaml[firstline=9,lastline=13,highlightlines=12]{../src/backtrace/backtrace.ml}
      \endminipage
    \end{column}
    \begin{column}{0.5\textwidth}
      \centering
      \providecommand\step{1}\input{diagrams/backtrace/backtrace.tex}
    \end{column}
  \end{columns}
\end{frame}

\begin{frame}{Backtraces}{But before that, we need more fields from \typename{caml\_domain\_state}}
  \begin{columns}
    \begin{column}{0.5\textwidth}
      \begin{itemize}
        \item Remember \typename{caml\_domain\_state} the per-domain runtime state?
        \item Some other members are of interest to us now\dots
        \item \localname{backtrace\_active} is used to know if the runtime should collect backtraces
        \item \localname{backtrace\_active} can be set by
          \begin{itemize}
            \item The \funcarg{b} flag in the \funcname{OCAMLRUNPARAM} environment variable
            \item \mintinline{ocaml}{Printexc.record_backtrace : bool -> unit} \footnotemark
          \end{itemize}
      \end{itemize}
    \end{column}
    \begin{column}{0.5\textwidth}
      \begin{listing}
        \mintc[highlightlines={9-11}]{src/ocaml/domain\_state\_backtrace.h}
        \caption{runtime/caml/domain\_state.h}
      \end{listing}
    \end{column}
  \end{columns}
  \footnotetext[1]{https://v2.ocaml.org/api/Printexc.html}
\end{frame}

\newcommand\listCamlRaiseExn[1][]{
  \listasm[firstline=702,lastline=725,#1]{../ocaml/runtime/amd64.S}{runtime/amd64.S:702}}

\begin{frame}{Backtraces}{Walk through \funcname{caml\_raise\_exn}}
  \begin{columns}[T]
    \begin{column}{0.5\textwidth}
      \begin{itemize}
        \item<1-> First checks whether exceptions backtrace should be recorded
        \item<1-> By checking the \funcarg{domain\_state->backtrace\_active} value
        \item<2-> If not, acts as \funcname{raise\_notrace} with \funcname{RESTORE\_EXN\_HANDLER\_OCAML}
          \begin{itemize}
            \item Set \regname{RSP} to \localname{domain\_state->exn\_handler} value
            \item Restore \localname{domain\_state->exn\_handler} from trap
            \item Jump with \funcname{ret} (same effect as using \funcname{pop} and \funcname{jmp})
          \end{itemize}
        \item<3-> Otherwise, switches to the C stack to be able to execute C code
        \item<4-> Calls \funcname{caml\_stash\_backtrace}, a C function provided by the runtime\ignorespaces
          \onslide<6->{, with
            \begin{itemize}
              \item \funcarg{exn}: exception value
              \item \funcarg{pc}: code address of the raising point
              \item \funcarg{sp}: stack address of the raising function
              \item \funcarg{trapsp}: stack address of the next handler
            \end{itemize}
          }
      \end{itemize}
      \bigskip
      \mintocaml[firstline=9,lastline=13,highlightlines=12]{../src/backtrace/backtrace.ml}
    \end{column}
    \begin{column}{0.5\textwidth}
      \raggedleft
      \onslide*<1,3-4>{
        \mintc[firstline=190,lastline=192]{../ocaml/runtime/amd64.S}
        \mintc[firstline=234,lastline=237]{../ocaml/runtime/amd64.S}
      }
      \onslide*<2>{
        \mintc[firstline=190,lastline=192,highlightlines={191-192}]{../ocaml/runtime/amd64.S}
        \mintc[firstline=234,lastline=237,highlightlines={235-237}]{../ocaml/runtime/amd64.S}
      }
      \onslide*<1>{\listCamlRaiseExn[highlightlines=705]{}}%
      \onslide*<2>{\listCamlRaiseExn[highlightlines={707-708}]{}}%
      \onslide*<3>{\listCamlRaiseExn[highlightlines={712-714}]{}}%
      \onslide*<4>{\listCamlRaiseExn[highlightlines={715-720}]{}}%
      \onslide*<5>{\providecommand\step{1}\input{diagrams/backtrace/backtrace.tex}}%
      \onslide*<6>{\providecommand\step{2}\input{diagrams/backtrace/backtrace.tex}}%
    \end{column}
  \end{columns}
\end{frame}

\newcommand\listStashBacktrace[1][]{
  \listc[firstline=91,lastline=120,#1]{../ocaml/runtime/backtrace\_nat.c}{runtime/backtrace\_nat.c:91}}

\begin{frame}{Backtraces}{Introducing \funcname{caml\_stash\_backtrace}}
  \begin{columns}[c]
    \begin{column}{0.5\textwidth}
      \minipage[c][0.66\textheight][s]{\columnwidth}
      \begin{itemize}
        \item<1-> A C function provided by the runtime (specific to native code)
        \item<2-> It scans the stack, between the stack pointer at the raising point up to the next trap, in order to collect the names of the detected function frames
          \onslide*<2>{
            \begin{itemize}
              \item And more information, like line number, column number...
            \end{itemize}
          }
        \item<3-> It uses the same technique and information as the Garbage Collector (GC) uses to scan the values live on the stack
          \onslide*<4>{
            \begin{itemize}
              \item \typename{frame\_descr} are at the core of stack unwinding
            \end{itemize}
          }
      \end{itemize}
      \vfill
      \mintocaml[firstline=9,lastline=13,highlightlines=12]{../src/backtrace/backtrace.ml}
      \endminipage
    \end{column}
    \begin{column}{0.5\textwidth}
      \onslide*<1>{\listStashBacktrace[highlightlines=91]{}}
      \onslide*<2>{\listStashBacktrace[highlightlines={91,117-118}]{}}
      \onslide*<3->{\listStashBacktrace[highlightlines={94,106,109-110}]{}}
    \end{column}
  \end{columns}
\end{frame}

\newcommand\listFrameDescr[1][]{
  \listocaml[firstline=111,lastline=124,#1]{../ocaml/asmcomp/emitaux.ml}{asmcomp/emitaux.ml:111}}
\newcommand\listDebugInfo[1][]{
  \listocaml[firstline=38,lastline=49,#1]{../ocaml/lambda/debuginfo.mli}{lambda/debuginfo.mli:38}}

\begin{frame}{Backtraces}{\typename{frame\_descr} and \typename{frame\_debuginfo} in a nutshell}
  \begin{columns}[c]
    \begin{column}{0.5\textwidth}
      \begin{itemize}
        \item<1-> For every return address in an OCaml program, there is a associated \typename{frame\_descr}
        \item<2-> A \typename{frame\_descr} specifies
        \begin{itemize}
          \item<2-> The size of the function stack frame
          \item<3-> Where live values are on the stack (so that the GC can mark them as live and prevent from being swept)
        \end{itemize}
        \item<4-> When compiled with debug information, \typename{frame\_descr} will be linked to a \typename{frame\_debuginfo}
        \begin{itemize}
          \item<5-> Contains information about the position in the source code (file name, line and column number)
        \end{itemize}
      \end{itemize}
    \end{column}
    \begin{column}{0.5\textwidth}
        \onslide*<1>{\listFrameDescr[highlightlines={118-122}]{}}
        \onslide*<2>{\listFrameDescr[highlightlines={120}]{}}
        \onslide*<3>{\listFrameDescr[highlightlines={121}]{}}
        \onslide*<4->{\listFrameDescr[highlightlines={122}]{}}
        \medskip
        \onslide*<1-3>{\listDebugInfo{}}
        \onslide*<4>{\listDebugInfo[highlightlines={38,49}]{}}
        \onslide*<5>{\listDebugInfo[highlightlines={39-46}]{}}
    \end{column}
  \end{columns}
\end{frame}

\begin{frame}{Backtraces}{Scanning the stack with \typename{frame\_descr}}
  \begin{columns}[c]
    \begin{column}{0.5\textwidth}
      \begin{itemize}
        \item Given the return address of the code that raises the exception by calling \funcname{caml\_raise\_exn} (\funcarg{pc} argument of \funcname{caml\_stash\_backtrace})
        \item And the position of the stack pointer (\funcarg{sp} argument of \funcname{caml\_stash\_backtrace})
        \item The frame size given by \typename{frame\_descr}.\funcarg{frame\_size}, allows to find the end of the previous frame
        \item And the return address of the caller of this function
        \bigskip
        \onslide<3->{
        \item Note that traps (here the reddish \funcname{ExnA}, \funcname{ExnB} and \funcname{ExnC} blocks) belong to the frame that installed them, and so the frame size changes when traps are removed
        }
      \end{itemize}
    \end{column}
    \begin{column}{0.5\textwidth}
      \raggedleft
      \onslide*<1>{\providecommand\step{2}\input{diagrams/backtrace/backtrace.tex}}%
      \onslide*<2>{\providecommand\step{1}\input{diagrams/backtrace/scanstack.tex}}%
      \onslide*<3>{\providecommand\step{2}\input{diagrams/backtrace/scanstack.tex}}%
      \onslide*<4>{\providecommand\step{3}\input{diagrams/backtrace/scanstack.tex}}%
      \onslide*<5>{\providecommand\step{4}\input{diagrams/backtrace/scanstack.tex}}%
      \onslide*<6>{\providecommand\step{5}\input{diagrams/backtrace/scanstack.tex}}%
    \end{column}
  \end{columns}
\end{frame}

% Duplicate of previous-previous slide
\begin{frame}{Backtraces}{Continuing on \funcname{caml\_stash\_backtrace}}
  \begin{columns}[c]
    \begin{column}{0.5\textwidth}
      \minipage[c][0.75\textheight][s]{\columnwidth}
      \begin{itemize}
          \color{gray}
        \item A C function provided by the runtime (specific to native code)
        \item It scans the stack, between the stack pointer at the raising point up to the next trap, in order to collect the names of the detected function frames
        \item It uses the same technique and information as the Garbage Collector (GC) uses to scan the values live on the stack
      \end{itemize}
      \begin{itemize}
        \item For each function frame detected, store a pointer to the \typename{frame\_descr} in the \localname{domain\_state->backtrace\_buffer} array, effectively constructing the backtrace
      \end{itemize}
      \onslide<2->{
        \begin{itemize}
            \smallskip
          \item Constructed backtrace:
            \funcname{definitely\_raise},
            \onslide<3->{\funcname{probably\_raise}}
        \end{itemize}
      }
      \vfill
      \mintocaml[firstline=9,lastline=13,highlightlines=12]{../src/backtrace/backtrace.ml}
      \endminipage
    \end{column}
    \begin{column}{0.5\textwidth}
      \raggedleft
      \onslide*<1>{\listStashBacktrace[highlightlines={114-115}]{}}
      \onslide*<2>{\providecommand\step{3}\input{diagrams/backtrace/backtrace.tex}}%
      \onslide*<3>{\providecommand\step{4}\input{diagrams/backtrace/backtrace.tex}}%
    \end{column}
  \end{columns}
\end{frame}

\begin{frame}{Backtraces}{Back to \funcname{caml\_raise\_exn}}
  \begin{columns}[c]
    \begin{column}{0.5\textwidth}
      \minipage[c][0.66\textheight][s]{\columnwidth}
      \begin{itemize}\color{gray}
        \item Checks wheteher exceptions backtrace should be recorded, by checking the \funcarg{domain\_state->backtrace\_active} value
        \item Switches to the C stack to be able to execute C code
        \item Calls \funcname{caml\_stash\_backtrace}, to collect the backtrace up to the next trap
      \end{itemize}
      \onslide*<2->{
        \begin{itemize}
          \item<2-> Executes the exception handler in \funcname{probably\_raise} with \funcname{RESTORE\_EXN\_HANDLER\_OCAML}
            \begin{itemize}
              \item Set \regname{RSP} to \localname{domain\_state->exn\_handler} value
              \item Restore \localname{domain\_state->exn\_handler} from trap
              \item Jump to the handler code with \funcname{ret}
            \end{itemize}
        \end{itemize}
      }
      \vfill
      \onslide*<1>{\mintocaml[firstline=9,lastline=13,highlightlines=12]{../src/backtrace/backtrace.ml}}
      \onslide*<2->{\mintocaml[firstline=15,lastline=21,highlightlines={19-21}]{../src/backtrace/backtrace.ml}}
      \endminipage
    \end{column}
    \begin{column}{0.5\textwidth}
      \centering
      \onslide*<1>{
        \mintc[firstline=190,lastline=192]{../ocaml/runtime/amd64.S}
        \mintc[firstline=234,lastline=237]{../ocaml/runtime/amd64.S}
      }
      \onslide*<2>{
        \mintc[firstline=190,lastline=192,highlightlines={191-192}]{../ocaml/runtime/amd64.S}
        \mintc[firstline=234,lastline=237,highlightlines={235-237}]{../ocaml/runtime/amd64.S}
      }
      \onslide*<1>{\listCamlRaiseExn[highlightlines=720]{}}
      \onslide*<2>{\listCamlRaiseExn[highlightlines=722]{}}
      \onslide*<3->{\providecommand\step{5}\input{diagrams/backtrace/backtrace.tex}}
    \end{column}
  \end{columns}
\end{frame}

\begin{frame}{Backtraces}{\funcname{caml\_probably\_raise}'s handler doesn't catch \typename{ExnC}}
  \begin{columns}[c]
    \begin{column}{0.45\textwidth}
      \minipage[c][0.75\textheight][s]{\columnwidth}
      \begin{itemize}
        \item<1-> \funcname{match \dots with}\footnotemark pattern matching can also match exceptions
        \item<1-> The CMM of \funcname{probably\_raise} shows that this \funcname{match \dots with} is implemented with a pattern matching and a \funcname{try \dots with}
        \item<1-> But this one only matches on \typename{ExnA} exception
        \item<2-> Since the pattern matching fails to catch the exception, \funcname{reraise} is called with our current \typename{ExnC} exception
        \item<3-> Disassembly shows that \funcname{reraise} is actually a call to \funcname{caml\_reraise\_exn}, which is also an assembly function provided by the runtime
      \end{itemize}
      \vfill
      \mintocaml[firstline=15,lastline=21,highlightlines={18-21}]{../src/backtrace/backtrace.ml}
      \endminipage
    \end{column}
    \begin{column}{0.55\textwidth}
      \centering
      \onslide*<1>{\mintcmm[highlightlines={10-18}]{../src/backtrace/camlBacktrace\_\_probably\_raise\_351.cmm}}
      \onslide*<2>{\listcmm[highlightlines=18]{../src/backtrace/camlBacktrace\_\_probably\_raise_351.cmm}{CMM of probably\_raise}}
      \onslide*<3>{\mintobjdump[firstline=5,lastline=35,highlightlines=31]{../src/backtrace/camlBacktrace\_\_probably\_raise_351.objdump}}
    \end{column}
  \end{columns}
  \only<1>{\footnotetext[1]{https://blog.janestreet.com/pattern-matching-and-exception-handling-unite/}}
\end{frame}

\newcommand\listCamlReraiseExn[1][]{
  \listasm[firstline=727,lastline=734,#1]{../ocaml/runtime/amd64.S}{runtime/amd64.S:727}}

\begin{frame}{Backtraces}{Introducing \funcname{caml\_reraise\_exn}}
  \begin{columns}[c]
    \begin{column}{0.5\textwidth}
      \minipage[c][0.66\textheight][s]{\columnwidth}
      \begin{itemize}
        \item<1-> The exception have to be reraised to the next handler
        \item<2-> First, checks if the backtrace should be collected with \funcarg{domain\_state->backtrace\_active}
        \only<3>{
        \begin{itemize}
            \item If not, behaves like a \funcname{raise\_notrace} with \funcname{RESTORE\_EXN\_HANDLER\_OCAML}
        \end{itemize}
        }
        \item<4-> Jumps into \funcname{caml\_raise\_exn}
        \item<5-> Calls \funcname{caml\_stash\_backtrace} again, to scan the next part of the stack: from the trap just pop-ed to the next trap
      \end{itemize}
      \vfill
      \mintocaml[firstline=15,lastline=21,highlightlines={18-21}]{../src/backtrace/backtrace.ml}
      \endminipage
    \end{column}
    \begin{column}{0.5\textwidth}
      \raggedleft
      \onslide*<1>{\listCamlReraiseExn{}}
      \onslide*<2>{\listCamlReraiseExn[highlightlines=729]{}}
      \onslide*<3>{
        \mintc[firstline=190,lastline=192,highlightlines={191-192}]{../ocaml/runtime/amd64.S}
        \mintc[firstline=234,lastline=237,highlightlines={235-237}]{../ocaml/runtime/amd64.S}
        \medskip
        \listCamlReraiseExn[highlightlines=731]{}
      }
      \onslide*<4-5>{
        % TODO refactor with \listCamlReraiseExn
        \mintasm[firstline=727,lastline=734,highlightlines=730]{../ocaml/runtime/amd64.S}
        \medskip
      }
      \onslide*<4>{\listCamlRaiseExn[highlightlines=711]{}}
      \onslide*<5>{\listCamlRaiseExn[highlightlines=720]{}}
    \end{column}
  \end{columns}
\end{frame}

\begin{frame}{Backtraces}{Backtrace collection continues}
  \begin{columns}[c]
    \begin{column}{0.55\textwidth}
      \minipage[c][0.75\textheight][s]{\columnwidth}
      \begin{itemize}
        \item<1-> \funcname{caml\_stash\_backtrace} scans the older part of the stack, up to the next trap
        \item<6-> Controls jumps to the nested exception handler in \funcname{maybe\_raise}, which matches only on \typename{ExnB}
        \item<7-> \funcname{caml\_reraise\_exn} is called again, backtrace collection resumes with \funcname{caml\_stash\_backtrace}
      \end{itemize}
      \medskip
      \begin{itemize}
        \item<2-> Constructed backtrace:
        \begin{itemize}
          \item <2-> \funcname{definitely\_raise}
          \item <2-> \funcname{probably\_raise}
          \item <3-> \funcname{probably\_raise} (re-raise)
          \item <4-> \funcname{maybe\_raise}
          \item <5-> \funcname{main}
          \item <8-> \funcname{main} (re-raise)
        \end{itemize}
      \end{itemize}
      \vfill
      \onslide*<1-5>{\mintocaml[firstline=15,lastline=21]{../src/backtrace/backtrace.ml}}
      \onslide*<6>{\mintocaml[firstline=28,lastline=34,highlightlines=31]{../src/backtrace/backtrace.ml}}
      \onslide*<7->{\mintocaml[firstline=28,lastline=34,highlightlines=33]{../src/backtrace/backtrace.ml}}
      \endminipage
    \end{column}
    \begin{column}{0.45\textwidth}
      \raggedleft
      \onslide*<1>{\providecommand\step{6}\input{diagrams/backtrace/backtrace.tex}}%
      \onslide*<2>{\providecommand\step{6}\input{diagrams/backtrace/backtrace.tex}}%
      \onslide*<3>{\providecommand\step{7}\input{diagrams/backtrace/backtrace.tex}}%
      \onslide*<4>{\providecommand\step{8}\input{diagrams/backtrace/backtrace.tex}}%
      \onslide*<5>{\providecommand\step{9}\input{diagrams/backtrace/backtrace.tex}}%
      \onslide*<6>{\providecommand\step{10}\input{diagrams/backtrace/backtrace.tex}}%
      \onslide*<7>{\providecommand\step{11}\input{diagrams/backtrace/backtrace.tex}}%
      \onslide*<8>{\providecommand\step{12}\input{diagrams/backtrace/backtrace.tex}}%
      \onslide*<9>{\providecommand\step{13}\input{diagrams/backtrace/backtrace.tex}}%
    \end{column}
  \end{columns}
\end{frame}

\begin{frame}{Backtraces}{Printing the backtrace}
  \begin{columns}[c]
    \begin{column}{0.45\textwidth}
      \minipage[c][0.66\textheight][s]{\columnwidth}
      \begin{itemize}
        \item<1-> The exception backtrace can now be printed to \funcarg{stdout} with \funcname{Printexc.print_backtrace}
        \item<2-> First the exception backtrace is retrieved into an OCaml array with \funcname{get_raw_backtrace }
        \item<3-> Which is implemented as the \funcname{caml_get_exception_raw_backtrace} C function in the runtime
        \item<4-> And finally printed with \funcname{print_exception_backtrace}
      \end{itemize}
      \vfill
      \mintocaml[firstline=28,lastline=34,highlightlines=33]{../src/backtrace/backtrace.ml}
      \endminipage
    \end{column}
    \begin{column}{0.55\textwidth}
      \onslide*<1,2>{
        \mintocaml[firstline=100,lastline=101]{../ocaml/stdlib/printexc.ml}
      }
      \only<1>{
        \listocaml[firstline=170,lastline=172]{../ocaml/stdlib/printexc.ml}{stdlib/printexc.ml:170}
      }
      \only<2>{
        \listocaml[firstline=170,lastline=172,highlightlines=172]{../ocaml/stdlib/printexc.ml}{stdlib/printexc.ml:170}
      }
      \only<3>{
        \mintc[firstline=154,lastline=156]{../ocaml/runtime/backtrace.c}
        \listc[firstline=171,lastline=192,highlightlines={184-187}]{../ocaml/runtime/backtrace.c}{runtime/backtrace.c:154}
      }
      \only<4>{
        \listocaml[firstline=155,lastline=165]{../ocaml/stdlib/printexc.ml}{stdlib/printexc.ml:55}
      }
    \end{column}
  \end{columns}
\end{frame}


\subsection{The multiple kinds of raise}
\frameSubsection{}{}

\begin{frame}{Backtraces}{When backtraces are not needed}
  \begin{columns}[c]
    \begin{column}{0.45\textwidth}
      \minipage[c][0.66\textheight][s]{\columnwidth}
      \begin{itemize}
        \item<1-> What if the backtrace for an exception is never needed?
        \item<2-> It's possible to raise the exception explicitly with \funcname{raise\_notrace} to make raising as cheap as possible
        \item<3-> An example of use in the \funcname{dynlink} library of the OCaml compiler
      \end{itemize}
      \vfill
      \onslide<3->{
      \listocaml[firstline=205,lastline=209,highlightlines=207]{../ocaml/otherlibs/dynlink/dynlink\_compilerlibs/misc.ml}{otherlibs/dynlink/dynlink\_compilerlibs/misc.ml:205}
      }
      \endminipage
    \end{column}
    \begin{column}{0.55\textwidth}
    \only<4>{
      \mintcmm[highlightlines=3]{../src/notrace/camlNotrace\_\_fun_347.cmm}
      \listcmm{../src/notrace/camlNotrace\_\_all\_somes\_327.cmm}{all\_somes}
    }
    \onslide*<5->{
      \listobjdump[highlightlines={6-9}]{../src/notrace/camlNotrace\_\_fun_347.objdump}{anonymous function from \funcname{all\_somes}}
    }
    \end{column}
  \end{columns}
\end{frame}

\begin{frame}{Backtraces}{Summary table of the multiple kinds of raise}
  \begin{itemize}
    \item When debugging information is enabled and if the backtrace of an exception isn't needed, it's possible to raise with \funcname{raise\_notrace} for performance
    \item When debugging information is \emph{not} enabled, the compiler optimises \funcname{raise} calls to inline assembly (same effect as using \funcname{raise\_notrace})
  \end{itemize}
  \bigskip
  \centering
  \begin{tabular}{| l | c | c | c | c |}
    \hline
    \parbox{10em}{Debug information \\ (\funcarg{-g} flag)}  & \multicolumn{2}{|c|}{Disabled}                                     & \multicolumn{2}{|c|}{Enabled} \\
    \hline
    Raise kind                                               & \funcname{raise} & \funcname{raise\_notrace}                       & \funcname{raise} & \funcname{raise\_notrace} \\
    \hline
    Compilation                                              & \parbox{8em}{inline assembly\\ (optimization)} & inline assembly   & \funcname{caml\_raise\_exn} & inline assembly \\
    \hline
  \end{tabular}
\end{frame}


%
% Takeaway #5
%
\subsection*{Takeaway \#5}
\frameSubsectionTakeaway{}
\againframe<5>{takeaway}
}%
      \onslide*<9>{\providecommand\step{13}%
% Backtraces
%
\subsection{One advantage of raising with debugging information}
\frameSubsection{}{}

\begin{frame}{Backtraces}{Debugging information \& source code}
  \begin{columns}[c]
    \begin{column}{0.5\textwidth}
      \begin{itemize}
        \item Raising an exception is fun and games, but how do we know where the exception originated? $\rightarrow$ With the backtrace associated with the exception!
        \item In order to get a backtrace from the point where an exception was raised, it's necessary to compile with debugging information enabled (\funcarg{-g} flag for the compiler)
        \item Same example as in the `Nested handlers' section
      \end{itemize}
    \end{column}
    \begin{column}{0.5\textwidth}
      \listocaml[firstline=9,lastline=34]{../src/backtrace/backtrace.ml}{backtrace/backtrace.ml}
    \end{column}
  \end{columns}
\end{frame}

\begin{frame}{Backtraces}{CMM and disassembly of \funcname{definitely\_raise}}
  \begin{columns}[c]
    \begin{column}{0.5\textwidth}
      \minipage[c][0.66\textheight][s]{\columnwidth}
      \begin{itemize}
        \item<1-> An effect of compiling with debugging information (\funcarg{-g}): the CMM no longer uses \funcname{raise\_notrace} to raise the exception but \funcname{raise} instead
        \item<2-> Disassembly shows that CMM \funcname{raise} is actually a call to \funcname{caml\_raise\_exn}, a function in assembler provided by the runtime
      \end{itemize}
      \vfill
      \mintocaml[firstline=9,lastline=13,highlightlines=12]{../src/backtrace/backtrace.ml}
      \endminipage
    \end{column}
    \begin{column}{0.5\textwidth}
      \onslide*<1>{
        \mintcmm[highlightlines=5]{../src/backtrace/camlBacktrace\_\_definitely\_raise\_347.cmm}
      }
      \onslide*<2>{
        \mintobjdump[firstline=2,highlightlines=14]{../src/backtrace/camlBacktrace\_\_definitely\_raise\_347.objdump}
      }
    \end{column}
  \end{columns}
\end{frame}

\begin{frame}{Backtraces}{Introducing \funcname{caml\_raise\_exn}}
  \begin{columns}[c]
    \begin{column}{0.5\textwidth}
      \minipage[c][0.66\textheight][s]{\columnwidth}
      \onslide*<1>{
        \begin{itemize}
          \item Raising the exception is performed using \funcname{caml\_raise\_exn} which is
            \begin{itemize}
              \item An assembly function provided by the runtime
              \item Architecture specific
            \end{itemize}
          \item Let's see what it does differently from \funcname{raise\_notrace}\dots
          \item We are about to raise \typename{ExnC} with \funcname{caml\_raise\_exn} from \funcname{definitely\_raise}
        \end{itemize}
      }
      \onslide*<2>{
        \mintobjdump[firstline=2,highlightlines=14]{../src/backtrace/camlBacktrace\_\_definitely\_raise\_347.objdump}
      }
      \vfill
      \mintocaml[firstline=9,lastline=13,highlightlines=12]{../src/backtrace/backtrace.ml}
      \endminipage
    \end{column}
    \begin{column}{0.5\textwidth}
      \centering
      \providecommand\step{1}\input{diagrams/backtrace/backtrace.tex}
    \end{column}
  \end{columns}
\end{frame}

\begin{frame}{Backtraces}{But before that, we need more fields from \typename{caml\_domain\_state}}
  \begin{columns}
    \begin{column}{0.5\textwidth}
      \begin{itemize}
        \item Remember \typename{caml\_domain\_state} the per-domain runtime state?
        \item Some other members are of interest to us now\dots
        \item \localname{backtrace\_active} is used to know if the runtime should collect backtraces
        \item \localname{backtrace\_active} can be set by
          \begin{itemize}
            \item The \funcarg{b} flag in the \funcname{OCAMLRUNPARAM} environment variable
            \item \mintinline{ocaml}{Printexc.record_backtrace : bool -> unit} \footnotemark
          \end{itemize}
      \end{itemize}
    \end{column}
    \begin{column}{0.5\textwidth}
      \begin{listing}
        \mintc[highlightlines={9-11}]{src/ocaml/domain\_state\_backtrace.h}
        \caption{runtime/caml/domain\_state.h}
      \end{listing}
    \end{column}
  \end{columns}
  \footnotetext[1]{https://v2.ocaml.org/api/Printexc.html}
\end{frame}

\newcommand\listCamlRaiseExn[1][]{
  \listasm[firstline=702,lastline=725,#1]{../ocaml/runtime/amd64.S}{runtime/amd64.S:702}}

\begin{frame}{Backtraces}{Walk through \funcname{caml\_raise\_exn}}
  \begin{columns}[T]
    \begin{column}{0.5\textwidth}
      \begin{itemize}
        \item<1-> First checks whether exceptions backtrace should be recorded
        \item<1-> By checking the \funcarg{domain\_state->backtrace\_active} value
        \item<2-> If not, acts as \funcname{raise\_notrace} with \funcname{RESTORE\_EXN\_HANDLER\_OCAML}
          \begin{itemize}
            \item Set \regname{RSP} to \localname{domain\_state->exn\_handler} value
            \item Restore \localname{domain\_state->exn\_handler} from trap
            \item Jump with \funcname{ret} (same effect as using \funcname{pop} and \funcname{jmp})
          \end{itemize}
        \item<3-> Otherwise, switches to the C stack to be able to execute C code
        \item<4-> Calls \funcname{caml\_stash\_backtrace}, a C function provided by the runtime\ignorespaces
          \onslide<6->{, with
            \begin{itemize}
              \item \funcarg{exn}: exception value
              \item \funcarg{pc}: code address of the raising point
              \item \funcarg{sp}: stack address of the raising function
              \item \funcarg{trapsp}: stack address of the next handler
            \end{itemize}
          }
      \end{itemize}
      \bigskip
      \mintocaml[firstline=9,lastline=13,highlightlines=12]{../src/backtrace/backtrace.ml}
    \end{column}
    \begin{column}{0.5\textwidth}
      \raggedleft
      \onslide*<1,3-4>{
        \mintc[firstline=190,lastline=192]{../ocaml/runtime/amd64.S}
        \mintc[firstline=234,lastline=237]{../ocaml/runtime/amd64.S}
      }
      \onslide*<2>{
        \mintc[firstline=190,lastline=192,highlightlines={191-192}]{../ocaml/runtime/amd64.S}
        \mintc[firstline=234,lastline=237,highlightlines={235-237}]{../ocaml/runtime/amd64.S}
      }
      \onslide*<1>{\listCamlRaiseExn[highlightlines=705]{}}%
      \onslide*<2>{\listCamlRaiseExn[highlightlines={707-708}]{}}%
      \onslide*<3>{\listCamlRaiseExn[highlightlines={712-714}]{}}%
      \onslide*<4>{\listCamlRaiseExn[highlightlines={715-720}]{}}%
      \onslide*<5>{\providecommand\step{1}\input{diagrams/backtrace/backtrace.tex}}%
      \onslide*<6>{\providecommand\step{2}\input{diagrams/backtrace/backtrace.tex}}%
    \end{column}
  \end{columns}
\end{frame}

\newcommand\listStashBacktrace[1][]{
  \listc[firstline=91,lastline=120,#1]{../ocaml/runtime/backtrace\_nat.c}{runtime/backtrace\_nat.c:91}}

\begin{frame}{Backtraces}{Introducing \funcname{caml\_stash\_backtrace}}
  \begin{columns}[c]
    \begin{column}{0.5\textwidth}
      \minipage[c][0.66\textheight][s]{\columnwidth}
      \begin{itemize}
        \item<1-> A C function provided by the runtime (specific to native code)
        \item<2-> It scans the stack, between the stack pointer at the raising point up to the next trap, in order to collect the names of the detected function frames
          \onslide*<2>{
            \begin{itemize}
              \item And more information, like line number, column number...
            \end{itemize}
          }
        \item<3-> It uses the same technique and information as the Garbage Collector (GC) uses to scan the values live on the stack
          \onslide*<4>{
            \begin{itemize}
              \item \typename{frame\_descr} are at the core of stack unwinding
            \end{itemize}
          }
      \end{itemize}
      \vfill
      \mintocaml[firstline=9,lastline=13,highlightlines=12]{../src/backtrace/backtrace.ml}
      \endminipage
    \end{column}
    \begin{column}{0.5\textwidth}
      \onslide*<1>{\listStashBacktrace[highlightlines=91]{}}
      \onslide*<2>{\listStashBacktrace[highlightlines={91,117-118}]{}}
      \onslide*<3->{\listStashBacktrace[highlightlines={94,106,109-110}]{}}
    \end{column}
  \end{columns}
\end{frame}

\newcommand\listFrameDescr[1][]{
  \listocaml[firstline=111,lastline=124,#1]{../ocaml/asmcomp/emitaux.ml}{asmcomp/emitaux.ml:111}}
\newcommand\listDebugInfo[1][]{
  \listocaml[firstline=38,lastline=49,#1]{../ocaml/lambda/debuginfo.mli}{lambda/debuginfo.mli:38}}

\begin{frame}{Backtraces}{\typename{frame\_descr} and \typename{frame\_debuginfo} in a nutshell}
  \begin{columns}[c]
    \begin{column}{0.5\textwidth}
      \begin{itemize}
        \item<1-> For every return address in an OCaml program, there is a associated \typename{frame\_descr}
        \item<2-> A \typename{frame\_descr} specifies
        \begin{itemize}
          \item<2-> The size of the function stack frame
          \item<3-> Where live values are on the stack (so that the GC can mark them as live and prevent from being swept)
        \end{itemize}
        \item<4-> When compiled with debug information, \typename{frame\_descr} will be linked to a \typename{frame\_debuginfo}
        \begin{itemize}
          \item<5-> Contains information about the position in the source code (file name, line and column number)
        \end{itemize}
      \end{itemize}
    \end{column}
    \begin{column}{0.5\textwidth}
        \onslide*<1>{\listFrameDescr[highlightlines={118-122}]{}}
        \onslide*<2>{\listFrameDescr[highlightlines={120}]{}}
        \onslide*<3>{\listFrameDescr[highlightlines={121}]{}}
        \onslide*<4->{\listFrameDescr[highlightlines={122}]{}}
        \medskip
        \onslide*<1-3>{\listDebugInfo{}}
        \onslide*<4>{\listDebugInfo[highlightlines={38,49}]{}}
        \onslide*<5>{\listDebugInfo[highlightlines={39-46}]{}}
    \end{column}
  \end{columns}
\end{frame}

\begin{frame}{Backtraces}{Scanning the stack with \typename{frame\_descr}}
  \begin{columns}[c]
    \begin{column}{0.5\textwidth}
      \begin{itemize}
        \item Given the return address of the code that raises the exception by calling \funcname{caml\_raise\_exn} (\funcarg{pc} argument of \funcname{caml\_stash\_backtrace})
        \item And the position of the stack pointer (\funcarg{sp} argument of \funcname{caml\_stash\_backtrace})
        \item The frame size given by \typename{frame\_descr}.\funcarg{frame\_size}, allows to find the end of the previous frame
        \item And the return address of the caller of this function
        \bigskip
        \onslide<3->{
        \item Note that traps (here the reddish \funcname{ExnA}, \funcname{ExnB} and \funcname{ExnC} blocks) belong to the frame that installed them, and so the frame size changes when traps are removed
        }
      \end{itemize}
    \end{column}
    \begin{column}{0.5\textwidth}
      \raggedleft
      \onslide*<1>{\providecommand\step{2}\input{diagrams/backtrace/backtrace.tex}}%
      \onslide*<2>{\providecommand\step{1}\input{diagrams/backtrace/scanstack.tex}}%
      \onslide*<3>{\providecommand\step{2}\input{diagrams/backtrace/scanstack.tex}}%
      \onslide*<4>{\providecommand\step{3}\input{diagrams/backtrace/scanstack.tex}}%
      \onslide*<5>{\providecommand\step{4}\input{diagrams/backtrace/scanstack.tex}}%
      \onslide*<6>{\providecommand\step{5}\input{diagrams/backtrace/scanstack.tex}}%
    \end{column}
  \end{columns}
\end{frame}

% Duplicate of previous-previous slide
\begin{frame}{Backtraces}{Continuing on \funcname{caml\_stash\_backtrace}}
  \begin{columns}[c]
    \begin{column}{0.5\textwidth}
      \minipage[c][0.75\textheight][s]{\columnwidth}
      \begin{itemize}
          \color{gray}
        \item A C function provided by the runtime (specific to native code)
        \item It scans the stack, between the stack pointer at the raising point up to the next trap, in order to collect the names of the detected function frames
        \item It uses the same technique and information as the Garbage Collector (GC) uses to scan the values live on the stack
      \end{itemize}
      \begin{itemize}
        \item For each function frame detected, store a pointer to the \typename{frame\_descr} in the \localname{domain\_state->backtrace\_buffer} array, effectively constructing the backtrace
      \end{itemize}
      \onslide<2->{
        \begin{itemize}
            \smallskip
          \item Constructed backtrace:
            \funcname{definitely\_raise},
            \onslide<3->{\funcname{probably\_raise}}
        \end{itemize}
      }
      \vfill
      \mintocaml[firstline=9,lastline=13,highlightlines=12]{../src/backtrace/backtrace.ml}
      \endminipage
    \end{column}
    \begin{column}{0.5\textwidth}
      \raggedleft
      \onslide*<1>{\listStashBacktrace[highlightlines={114-115}]{}}
      \onslide*<2>{\providecommand\step{3}\input{diagrams/backtrace/backtrace.tex}}%
      \onslide*<3>{\providecommand\step{4}\input{diagrams/backtrace/backtrace.tex}}%
    \end{column}
  \end{columns}
\end{frame}

\begin{frame}{Backtraces}{Back to \funcname{caml\_raise\_exn}}
  \begin{columns}[c]
    \begin{column}{0.5\textwidth}
      \minipage[c][0.66\textheight][s]{\columnwidth}
      \begin{itemize}\color{gray}
        \item Checks wheteher exceptions backtrace should be recorded, by checking the \funcarg{domain\_state->backtrace\_active} value
        \item Switches to the C stack to be able to execute C code
        \item Calls \funcname{caml\_stash\_backtrace}, to collect the backtrace up to the next trap
      \end{itemize}
      \onslide*<2->{
        \begin{itemize}
          \item<2-> Executes the exception handler in \funcname{probably\_raise} with \funcname{RESTORE\_EXN\_HANDLER\_OCAML}
            \begin{itemize}
              \item Set \regname{RSP} to \localname{domain\_state->exn\_handler} value
              \item Restore \localname{domain\_state->exn\_handler} from trap
              \item Jump to the handler code with \funcname{ret}
            \end{itemize}
        \end{itemize}
      }
      \vfill
      \onslide*<1>{\mintocaml[firstline=9,lastline=13,highlightlines=12]{../src/backtrace/backtrace.ml}}
      \onslide*<2->{\mintocaml[firstline=15,lastline=21,highlightlines={19-21}]{../src/backtrace/backtrace.ml}}
      \endminipage
    \end{column}
    \begin{column}{0.5\textwidth}
      \centering
      \onslide*<1>{
        \mintc[firstline=190,lastline=192]{../ocaml/runtime/amd64.S}
        \mintc[firstline=234,lastline=237]{../ocaml/runtime/amd64.S}
      }
      \onslide*<2>{
        \mintc[firstline=190,lastline=192,highlightlines={191-192}]{../ocaml/runtime/amd64.S}
        \mintc[firstline=234,lastline=237,highlightlines={235-237}]{../ocaml/runtime/amd64.S}
      }
      \onslide*<1>{\listCamlRaiseExn[highlightlines=720]{}}
      \onslide*<2>{\listCamlRaiseExn[highlightlines=722]{}}
      \onslide*<3->{\providecommand\step{5}\input{diagrams/backtrace/backtrace.tex}}
    \end{column}
  \end{columns}
\end{frame}

\begin{frame}{Backtraces}{\funcname{caml\_probably\_raise}'s handler doesn't catch \typename{ExnC}}
  \begin{columns}[c]
    \begin{column}{0.45\textwidth}
      \minipage[c][0.75\textheight][s]{\columnwidth}
      \begin{itemize}
        \item<1-> \funcname{match \dots with}\footnotemark pattern matching can also match exceptions
        \item<1-> The CMM of \funcname{probably\_raise} shows that this \funcname{match \dots with} is implemented with a pattern matching and a \funcname{try \dots with}
        \item<1-> But this one only matches on \typename{ExnA} exception
        \item<2-> Since the pattern matching fails to catch the exception, \funcname{reraise} is called with our current \typename{ExnC} exception
        \item<3-> Disassembly shows that \funcname{reraise} is actually a call to \funcname{caml\_reraise\_exn}, which is also an assembly function provided by the runtime
      \end{itemize}
      \vfill
      \mintocaml[firstline=15,lastline=21,highlightlines={18-21}]{../src/backtrace/backtrace.ml}
      \endminipage
    \end{column}
    \begin{column}{0.55\textwidth}
      \centering
      \onslide*<1>{\mintcmm[highlightlines={10-18}]{../src/backtrace/camlBacktrace\_\_probably\_raise\_351.cmm}}
      \onslide*<2>{\listcmm[highlightlines=18]{../src/backtrace/camlBacktrace\_\_probably\_raise_351.cmm}{CMM of probably\_raise}}
      \onslide*<3>{\mintobjdump[firstline=5,lastline=35,highlightlines=31]{../src/backtrace/camlBacktrace\_\_probably\_raise_351.objdump}}
    \end{column}
  \end{columns}
  \only<1>{\footnotetext[1]{https://blog.janestreet.com/pattern-matching-and-exception-handling-unite/}}
\end{frame}

\newcommand\listCamlReraiseExn[1][]{
  \listasm[firstline=727,lastline=734,#1]{../ocaml/runtime/amd64.S}{runtime/amd64.S:727}}

\begin{frame}{Backtraces}{Introducing \funcname{caml\_reraise\_exn}}
  \begin{columns}[c]
    \begin{column}{0.5\textwidth}
      \minipage[c][0.66\textheight][s]{\columnwidth}
      \begin{itemize}
        \item<1-> The exception have to be reraised to the next handler
        \item<2-> First, checks if the backtrace should be collected with \funcarg{domain\_state->backtrace\_active}
        \only<3>{
        \begin{itemize}
            \item If not, behaves like a \funcname{raise\_notrace} with \funcname{RESTORE\_EXN\_HANDLER\_OCAML}
        \end{itemize}
        }
        \item<4-> Jumps into \funcname{caml\_raise\_exn}
        \item<5-> Calls \funcname{caml\_stash\_backtrace} again, to scan the next part of the stack: from the trap just pop-ed to the next trap
      \end{itemize}
      \vfill
      \mintocaml[firstline=15,lastline=21,highlightlines={18-21}]{../src/backtrace/backtrace.ml}
      \endminipage
    \end{column}
    \begin{column}{0.5\textwidth}
      \raggedleft
      \onslide*<1>{\listCamlReraiseExn{}}
      \onslide*<2>{\listCamlReraiseExn[highlightlines=729]{}}
      \onslide*<3>{
        \mintc[firstline=190,lastline=192,highlightlines={191-192}]{../ocaml/runtime/amd64.S}
        \mintc[firstline=234,lastline=237,highlightlines={235-237}]{../ocaml/runtime/amd64.S}
        \medskip
        \listCamlReraiseExn[highlightlines=731]{}
      }
      \onslide*<4-5>{
        % TODO refactor with \listCamlReraiseExn
        \mintasm[firstline=727,lastline=734,highlightlines=730]{../ocaml/runtime/amd64.S}
        \medskip
      }
      \onslide*<4>{\listCamlRaiseExn[highlightlines=711]{}}
      \onslide*<5>{\listCamlRaiseExn[highlightlines=720]{}}
    \end{column}
  \end{columns}
\end{frame}

\begin{frame}{Backtraces}{Backtrace collection continues}
  \begin{columns}[c]
    \begin{column}{0.55\textwidth}
      \minipage[c][0.75\textheight][s]{\columnwidth}
      \begin{itemize}
        \item<1-> \funcname{caml\_stash\_backtrace} scans the older part of the stack, up to the next trap
        \item<6-> Controls jumps to the nested exception handler in \funcname{maybe\_raise}, which matches only on \typename{ExnB}
        \item<7-> \funcname{caml\_reraise\_exn} is called again, backtrace collection resumes with \funcname{caml\_stash\_backtrace}
      \end{itemize}
      \medskip
      \begin{itemize}
        \item<2-> Constructed backtrace:
        \begin{itemize}
          \item <2-> \funcname{definitely\_raise}
          \item <2-> \funcname{probably\_raise}
          \item <3-> \funcname{probably\_raise} (re-raise)
          \item <4-> \funcname{maybe\_raise}
          \item <5-> \funcname{main}
          \item <8-> \funcname{main} (re-raise)
        \end{itemize}
      \end{itemize}
      \vfill
      \onslide*<1-5>{\mintocaml[firstline=15,lastline=21]{../src/backtrace/backtrace.ml}}
      \onslide*<6>{\mintocaml[firstline=28,lastline=34,highlightlines=31]{../src/backtrace/backtrace.ml}}
      \onslide*<7->{\mintocaml[firstline=28,lastline=34,highlightlines=33]{../src/backtrace/backtrace.ml}}
      \endminipage
    \end{column}
    \begin{column}{0.45\textwidth}
      \raggedleft
      \onslide*<1>{\providecommand\step{6}\input{diagrams/backtrace/backtrace.tex}}%
      \onslide*<2>{\providecommand\step{6}\input{diagrams/backtrace/backtrace.tex}}%
      \onslide*<3>{\providecommand\step{7}\input{diagrams/backtrace/backtrace.tex}}%
      \onslide*<4>{\providecommand\step{8}\input{diagrams/backtrace/backtrace.tex}}%
      \onslide*<5>{\providecommand\step{9}\input{diagrams/backtrace/backtrace.tex}}%
      \onslide*<6>{\providecommand\step{10}\input{diagrams/backtrace/backtrace.tex}}%
      \onslide*<7>{\providecommand\step{11}\input{diagrams/backtrace/backtrace.tex}}%
      \onslide*<8>{\providecommand\step{12}\input{diagrams/backtrace/backtrace.tex}}%
      \onslide*<9>{\providecommand\step{13}\input{diagrams/backtrace/backtrace.tex}}%
    \end{column}
  \end{columns}
\end{frame}

\begin{frame}{Backtraces}{Printing the backtrace}
  \begin{columns}[c]
    \begin{column}{0.45\textwidth}
      \minipage[c][0.66\textheight][s]{\columnwidth}
      \begin{itemize}
        \item<1-> The exception backtrace can now be printed to \funcarg{stdout} with \funcname{Printexc.print_backtrace}
        \item<2-> First the exception backtrace is retrieved into an OCaml array with \funcname{get_raw_backtrace }
        \item<3-> Which is implemented as the \funcname{caml_get_exception_raw_backtrace} C function in the runtime
        \item<4-> And finally printed with \funcname{print_exception_backtrace}
      \end{itemize}
      \vfill
      \mintocaml[firstline=28,lastline=34,highlightlines=33]{../src/backtrace/backtrace.ml}
      \endminipage
    \end{column}
    \begin{column}{0.55\textwidth}
      \onslide*<1,2>{
        \mintocaml[firstline=100,lastline=101]{../ocaml/stdlib/printexc.ml}
      }
      \only<1>{
        \listocaml[firstline=170,lastline=172]{../ocaml/stdlib/printexc.ml}{stdlib/printexc.ml:170}
      }
      \only<2>{
        \listocaml[firstline=170,lastline=172,highlightlines=172]{../ocaml/stdlib/printexc.ml}{stdlib/printexc.ml:170}
      }
      \only<3>{
        \mintc[firstline=154,lastline=156]{../ocaml/runtime/backtrace.c}
        \listc[firstline=171,lastline=192,highlightlines={184-187}]{../ocaml/runtime/backtrace.c}{runtime/backtrace.c:154}
      }
      \only<4>{
        \listocaml[firstline=155,lastline=165]{../ocaml/stdlib/printexc.ml}{stdlib/printexc.ml:55}
      }
    \end{column}
  \end{columns}
\end{frame}


\subsection{The multiple kinds of raise}
\frameSubsection{}{}

\begin{frame}{Backtraces}{When backtraces are not needed}
  \begin{columns}[c]
    \begin{column}{0.45\textwidth}
      \minipage[c][0.66\textheight][s]{\columnwidth}
      \begin{itemize}
        \item<1-> What if the backtrace for an exception is never needed?
        \item<2-> It's possible to raise the exception explicitly with \funcname{raise\_notrace} to make raising as cheap as possible
        \item<3-> An example of use in the \funcname{dynlink} library of the OCaml compiler
      \end{itemize}
      \vfill
      \onslide<3->{
      \listocaml[firstline=205,lastline=209,highlightlines=207]{../ocaml/otherlibs/dynlink/dynlink\_compilerlibs/misc.ml}{otherlibs/dynlink/dynlink\_compilerlibs/misc.ml:205}
      }
      \endminipage
    \end{column}
    \begin{column}{0.55\textwidth}
    \only<4>{
      \mintcmm[highlightlines=3]{../src/notrace/camlNotrace\_\_fun_347.cmm}
      \listcmm{../src/notrace/camlNotrace\_\_all\_somes\_327.cmm}{all\_somes}
    }
    \onslide*<5->{
      \listobjdump[highlightlines={6-9}]{../src/notrace/camlNotrace\_\_fun_347.objdump}{anonymous function from \funcname{all\_somes}}
    }
    \end{column}
  \end{columns}
\end{frame}

\begin{frame}{Backtraces}{Summary table of the multiple kinds of raise}
  \begin{itemize}
    \item When debugging information is enabled and if the backtrace of an exception isn't needed, it's possible to raise with \funcname{raise\_notrace} for performance
    \item When debugging information is \emph{not} enabled, the compiler optimises \funcname{raise} calls to inline assembly (same effect as using \funcname{raise\_notrace})
  \end{itemize}
  \bigskip
  \centering
  \begin{tabular}{| l | c | c | c | c |}
    \hline
    \parbox{10em}{Debug information \\ (\funcarg{-g} flag)}  & \multicolumn{2}{|c|}{Disabled}                                     & \multicolumn{2}{|c|}{Enabled} \\
    \hline
    Raise kind                                               & \funcname{raise} & \funcname{raise\_notrace}                       & \funcname{raise} & \funcname{raise\_notrace} \\
    \hline
    Compilation                                              & \parbox{8em}{inline assembly\\ (optimization)} & inline assembly   & \funcname{caml\_raise\_exn} & inline assembly \\
    \hline
  \end{tabular}
\end{frame}


%
% Takeaway #5
%
\subsection*{Takeaway \#5}
\frameSubsectionTakeaway{}
\againframe<5>{takeaway}
}%
    \end{column}
  \end{columns}
\end{frame}

\begin{frame}{Backtraces}{Printing the backtrace}
  \begin{columns}[c]
    \begin{column}{0.45\textwidth}
      \minipage[c][0.66\textheight][s]{\columnwidth}
      \begin{itemize}
        \item<1-> The exception backtrace can now be printed to \funcarg{stdout} with \funcname{Printexc.print_backtrace}
        \item<2-> First the exception backtrace is retrieved into an OCaml array with \funcname{get_raw_backtrace }
        \item<3-> Which is implemented as the \funcname{caml_get_exception_raw_backtrace} C function in the runtime
        \item<4-> And finally printed with \funcname{print_exception_backtrace}
      \end{itemize}
      \vfill
      \mintocaml[firstline=28,lastline=34,highlightlines=33]{../src/backtrace/backtrace.ml}
      \endminipage
    \end{column}
    \begin{column}{0.55\textwidth}
      \onslide*<1,2>{
        \mintocaml[firstline=100,lastline=101]{../ocaml/stdlib/printexc.ml}
      }
      \only<1>{
        \listocaml[firstline=170,lastline=172]{../ocaml/stdlib/printexc.ml}{stdlib/printexc.ml:170}
      }
      \only<2>{
        \listocaml[firstline=170,lastline=172,highlightlines=172]{../ocaml/stdlib/printexc.ml}{stdlib/printexc.ml:170}
      }
      \only<3>{
        \mintc[firstline=154,lastline=156]{../ocaml/runtime/backtrace.c}
        \listc[firstline=171,lastline=192,highlightlines={184-187}]{../ocaml/runtime/backtrace.c}{runtime/backtrace.c:154}
      }
      \only<4>{
        \listocaml[firstline=155,lastline=165]{../ocaml/stdlib/printexc.ml}{stdlib/printexc.ml:55}
      }
    \end{column}
  \end{columns}
\end{frame}


\subsection{The multiple kinds of raise}
\frameSubsection{}{}

\begin{frame}{Backtraces}{When backtraces are not needed}
  \begin{columns}[c]
    \begin{column}{0.45\textwidth}
      \minipage[c][0.66\textheight][s]{\columnwidth}
      \begin{itemize}
        \item<1-> What if the backtrace for an exception is never needed?
        \item<2-> It's possible to raise the exception explicitly with \funcname{raise\_notrace} to make raising as cheap as possible
        \item<3-> An example of use in the \funcname{dynlink} library of the OCaml compiler
      \end{itemize}
      \vfill
      \onslide<3->{
      \listocaml[firstline=205,lastline=209,highlightlines=207]{../ocaml/otherlibs/dynlink/dynlink\_compilerlibs/misc.ml}{otherlibs/dynlink/dynlink\_compilerlibs/misc.ml:205}
      }
      \endminipage
    \end{column}
    \begin{column}{0.55\textwidth}
    \only<4>{
      \mintcmm[highlightlines=3]{../src/notrace/camlNotrace\_\_fun_347.cmm}
      \listcmm{../src/notrace/camlNotrace\_\_all\_somes\_327.cmm}{all\_somes}
    }
    \onslide*<5->{
      \listobjdump[highlightlines={6-9}]{../src/notrace/camlNotrace\_\_fun_347.objdump}{anonymous function from \funcname{all\_somes}}
    }
    \end{column}
  \end{columns}
\end{frame}

\begin{frame}{Backtraces}{Summary table of the multiple kinds of raise}
  \begin{itemize}
    \item When debugging information is enabled and if the backtrace of an exception isn't needed, it's possible to raise with \funcname{raise\_notrace} for performance
    \item When debugging information is \emph{not} enabled, the compiler optimises \funcname{raise} calls to inline assembly (same effect as using \funcname{raise\_notrace})
  \end{itemize}
  \bigskip
  \centering
  \begin{tabular}{| l | c | c | c | c |}
    \hline
    \parbox{10em}{Debug information \\ (\funcarg{-g} flag)}  & \multicolumn{2}{|c|}{Disabled}                                     & \multicolumn{2}{|c|}{Enabled} \\
    \hline
    Raise kind                                               & \funcname{raise} & \funcname{raise\_notrace}                       & \funcname{raise} & \funcname{raise\_notrace} \\
    \hline
    Compilation                                              & \parbox{8em}{inline assembly\\ (optimization)} & inline assembly   & \funcname{caml\_raise\_exn} & inline assembly \\
    \hline
  \end{tabular}
\end{frame}


%
% Takeaway #5
%
\subsection*{Takeaway \#5}
\frameSubsectionTakeaway{}
\againframe<5>{takeaway}
}%
      \onslide*<3>{\providecommand\step{4}%
% Backtraces
%
\subsection{One advantage of raising with debugging information}
\frameSubsection{}{}

\begin{frame}{Backtraces}{Debugging information \& source code}
  \begin{columns}[c]
    \begin{column}{0.5\textwidth}
      \begin{itemize}
        \item Raising an exception is fun and games, but how do we know where the exception originated? $\rightarrow$ With the backtrace associated with the exception!
        \item In order to get a backtrace from the point where an exception was raised, it's necessary to compile with debugging information enabled (\funcarg{-g} flag for the compiler)
        \item Same example as in the `Nested handlers' section
      \end{itemize}
    \end{column}
    \begin{column}{0.5\textwidth}
      \listocaml[firstline=9,lastline=34]{../src/backtrace/backtrace.ml}{backtrace/backtrace.ml}
    \end{column}
  \end{columns}
\end{frame}

\begin{frame}{Backtraces}{CMM and disassembly of \funcname{definitely\_raise}}
  \begin{columns}[c]
    \begin{column}{0.5\textwidth}
      \minipage[c][0.66\textheight][s]{\columnwidth}
      \begin{itemize}
        \item<1-> An effect of compiling with debugging information (\funcarg{-g}): the CMM no longer uses \funcname{raise\_notrace} to raise the exception but \funcname{raise} instead
        \item<2-> Disassembly shows that CMM \funcname{raise} is actually a call to \funcname{caml\_raise\_exn}, a function in assembler provided by the runtime
      \end{itemize}
      \vfill
      \mintocaml[firstline=9,lastline=13,highlightlines=12]{../src/backtrace/backtrace.ml}
      \endminipage
    \end{column}
    \begin{column}{0.5\textwidth}
      \onslide*<1>{
        \mintcmm[highlightlines=5]{../src/backtrace/camlBacktrace\_\_definitely\_raise\_347.cmm}
      }
      \onslide*<2>{
        \mintobjdump[firstline=2,highlightlines=14]{../src/backtrace/camlBacktrace\_\_definitely\_raise\_347.objdump}
      }
    \end{column}
  \end{columns}
\end{frame}

\begin{frame}{Backtraces}{Introducing \funcname{caml\_raise\_exn}}
  \begin{columns}[c]
    \begin{column}{0.5\textwidth}
      \minipage[c][0.66\textheight][s]{\columnwidth}
      \onslide*<1>{
        \begin{itemize}
          \item Raising the exception is performed using \funcname{caml\_raise\_exn} which is
            \begin{itemize}
              \item An assembly function provided by the runtime
              \item Architecture specific
            \end{itemize}
          \item Let's see what it does differently from \funcname{raise\_notrace}\dots
          \item We are about to raise \typename{ExnC} with \funcname{caml\_raise\_exn} from \funcname{definitely\_raise}
        \end{itemize}
      }
      \onslide*<2>{
        \mintobjdump[firstline=2,highlightlines=14]{../src/backtrace/camlBacktrace\_\_definitely\_raise\_347.objdump}
      }
      \vfill
      \mintocaml[firstline=9,lastline=13,highlightlines=12]{../src/backtrace/backtrace.ml}
      \endminipage
    \end{column}
    \begin{column}{0.5\textwidth}
      \centering
      \providecommand\step{1}%
% Backtraces
%
\subsection{One advantage of raising with debugging information}
\frameSubsection{}{}

\begin{frame}{Backtraces}{Debugging information \& source code}
  \begin{columns}[c]
    \begin{column}{0.5\textwidth}
      \begin{itemize}
        \item Raising an exception is fun and games, but how do we know where the exception originated? $\rightarrow$ With the backtrace associated with the exception!
        \item In order to get a backtrace from the point where an exception was raised, it's necessary to compile with debugging information enabled (\funcarg{-g} flag for the compiler)
        \item Same example as in the `Nested handlers' section
      \end{itemize}
    \end{column}
    \begin{column}{0.5\textwidth}
      \listocaml[firstline=9,lastline=34]{../src/backtrace/backtrace.ml}{backtrace/backtrace.ml}
    \end{column}
  \end{columns}
\end{frame}

\begin{frame}{Backtraces}{CMM and disassembly of \funcname{definitely\_raise}}
  \begin{columns}[c]
    \begin{column}{0.5\textwidth}
      \minipage[c][0.66\textheight][s]{\columnwidth}
      \begin{itemize}
        \item<1-> An effect of compiling with debugging information (\funcarg{-g}): the CMM no longer uses \funcname{raise\_notrace} to raise the exception but \funcname{raise} instead
        \item<2-> Disassembly shows that CMM \funcname{raise} is actually a call to \funcname{caml\_raise\_exn}, a function in assembler provided by the runtime
      \end{itemize}
      \vfill
      \mintocaml[firstline=9,lastline=13,highlightlines=12]{../src/backtrace/backtrace.ml}
      \endminipage
    \end{column}
    \begin{column}{0.5\textwidth}
      \onslide*<1>{
        \mintcmm[highlightlines=5]{../src/backtrace/camlBacktrace\_\_definitely\_raise\_347.cmm}
      }
      \onslide*<2>{
        \mintobjdump[firstline=2,highlightlines=14]{../src/backtrace/camlBacktrace\_\_definitely\_raise\_347.objdump}
      }
    \end{column}
  \end{columns}
\end{frame}

\begin{frame}{Backtraces}{Introducing \funcname{caml\_raise\_exn}}
  \begin{columns}[c]
    \begin{column}{0.5\textwidth}
      \minipage[c][0.66\textheight][s]{\columnwidth}
      \onslide*<1>{
        \begin{itemize}
          \item Raising the exception is performed using \funcname{caml\_raise\_exn} which is
            \begin{itemize}
              \item An assembly function provided by the runtime
              \item Architecture specific
            \end{itemize}
          \item Let's see what it does differently from \funcname{raise\_notrace}\dots
          \item We are about to raise \typename{ExnC} with \funcname{caml\_raise\_exn} from \funcname{definitely\_raise}
        \end{itemize}
      }
      \onslide*<2>{
        \mintobjdump[firstline=2,highlightlines=14]{../src/backtrace/camlBacktrace\_\_definitely\_raise\_347.objdump}
      }
      \vfill
      \mintocaml[firstline=9,lastline=13,highlightlines=12]{../src/backtrace/backtrace.ml}
      \endminipage
    \end{column}
    \begin{column}{0.5\textwidth}
      \centering
      \providecommand\step{1}\input{diagrams/backtrace/backtrace.tex}
    \end{column}
  \end{columns}
\end{frame}

\begin{frame}{Backtraces}{But before that, we need more fields from \typename{caml\_domain\_state}}
  \begin{columns}
    \begin{column}{0.5\textwidth}
      \begin{itemize}
        \item Remember \typename{caml\_domain\_state} the per-domain runtime state?
        \item Some other members are of interest to us now\dots
        \item \localname{backtrace\_active} is used to know if the runtime should collect backtraces
        \item \localname{backtrace\_active} can be set by
          \begin{itemize}
            \item The \funcarg{b} flag in the \funcname{OCAMLRUNPARAM} environment variable
            \item \mintinline{ocaml}{Printexc.record_backtrace : bool -> unit} \footnotemark
          \end{itemize}
      \end{itemize}
    \end{column}
    \begin{column}{0.5\textwidth}
      \begin{listing}
        \mintc[highlightlines={9-11}]{src/ocaml/domain\_state\_backtrace.h}
        \caption{runtime/caml/domain\_state.h}
      \end{listing}
    \end{column}
  \end{columns}
  \footnotetext[1]{https://v2.ocaml.org/api/Printexc.html}
\end{frame}

\newcommand\listCamlRaiseExn[1][]{
  \listasm[firstline=702,lastline=725,#1]{../ocaml/runtime/amd64.S}{runtime/amd64.S:702}}

\begin{frame}{Backtraces}{Walk through \funcname{caml\_raise\_exn}}
  \begin{columns}[T]
    \begin{column}{0.5\textwidth}
      \begin{itemize}
        \item<1-> First checks whether exceptions backtrace should be recorded
        \item<1-> By checking the \funcarg{domain\_state->backtrace\_active} value
        \item<2-> If not, acts as \funcname{raise\_notrace} with \funcname{RESTORE\_EXN\_HANDLER\_OCAML}
          \begin{itemize}
            \item Set \regname{RSP} to \localname{domain\_state->exn\_handler} value
            \item Restore \localname{domain\_state->exn\_handler} from trap
            \item Jump with \funcname{ret} (same effect as using \funcname{pop} and \funcname{jmp})
          \end{itemize}
        \item<3-> Otherwise, switches to the C stack to be able to execute C code
        \item<4-> Calls \funcname{caml\_stash\_backtrace}, a C function provided by the runtime\ignorespaces
          \onslide<6->{, with
            \begin{itemize}
              \item \funcarg{exn}: exception value
              \item \funcarg{pc}: code address of the raising point
              \item \funcarg{sp}: stack address of the raising function
              \item \funcarg{trapsp}: stack address of the next handler
            \end{itemize}
          }
      \end{itemize}
      \bigskip
      \mintocaml[firstline=9,lastline=13,highlightlines=12]{../src/backtrace/backtrace.ml}
    \end{column}
    \begin{column}{0.5\textwidth}
      \raggedleft
      \onslide*<1,3-4>{
        \mintc[firstline=190,lastline=192]{../ocaml/runtime/amd64.S}
        \mintc[firstline=234,lastline=237]{../ocaml/runtime/amd64.S}
      }
      \onslide*<2>{
        \mintc[firstline=190,lastline=192,highlightlines={191-192}]{../ocaml/runtime/amd64.S}
        \mintc[firstline=234,lastline=237,highlightlines={235-237}]{../ocaml/runtime/amd64.S}
      }
      \onslide*<1>{\listCamlRaiseExn[highlightlines=705]{}}%
      \onslide*<2>{\listCamlRaiseExn[highlightlines={707-708}]{}}%
      \onslide*<3>{\listCamlRaiseExn[highlightlines={712-714}]{}}%
      \onslide*<4>{\listCamlRaiseExn[highlightlines={715-720}]{}}%
      \onslide*<5>{\providecommand\step{1}\input{diagrams/backtrace/backtrace.tex}}%
      \onslide*<6>{\providecommand\step{2}\input{diagrams/backtrace/backtrace.tex}}%
    \end{column}
  \end{columns}
\end{frame}

\newcommand\listStashBacktrace[1][]{
  \listc[firstline=91,lastline=120,#1]{../ocaml/runtime/backtrace\_nat.c}{runtime/backtrace\_nat.c:91}}

\begin{frame}{Backtraces}{Introducing \funcname{caml\_stash\_backtrace}}
  \begin{columns}[c]
    \begin{column}{0.5\textwidth}
      \minipage[c][0.66\textheight][s]{\columnwidth}
      \begin{itemize}
        \item<1-> A C function provided by the runtime (specific to native code)
        \item<2-> It scans the stack, between the stack pointer at the raising point up to the next trap, in order to collect the names of the detected function frames
          \onslide*<2>{
            \begin{itemize}
              \item And more information, like line number, column number...
            \end{itemize}
          }
        \item<3-> It uses the same technique and information as the Garbage Collector (GC) uses to scan the values live on the stack
          \onslide*<4>{
            \begin{itemize}
              \item \typename{frame\_descr} are at the core of stack unwinding
            \end{itemize}
          }
      \end{itemize}
      \vfill
      \mintocaml[firstline=9,lastline=13,highlightlines=12]{../src/backtrace/backtrace.ml}
      \endminipage
    \end{column}
    \begin{column}{0.5\textwidth}
      \onslide*<1>{\listStashBacktrace[highlightlines=91]{}}
      \onslide*<2>{\listStashBacktrace[highlightlines={91,117-118}]{}}
      \onslide*<3->{\listStashBacktrace[highlightlines={94,106,109-110}]{}}
    \end{column}
  \end{columns}
\end{frame}

\newcommand\listFrameDescr[1][]{
  \listocaml[firstline=111,lastline=124,#1]{../ocaml/asmcomp/emitaux.ml}{asmcomp/emitaux.ml:111}}
\newcommand\listDebugInfo[1][]{
  \listocaml[firstline=38,lastline=49,#1]{../ocaml/lambda/debuginfo.mli}{lambda/debuginfo.mli:38}}

\begin{frame}{Backtraces}{\typename{frame\_descr} and \typename{frame\_debuginfo} in a nutshell}
  \begin{columns}[c]
    \begin{column}{0.5\textwidth}
      \begin{itemize}
        \item<1-> For every return address in an OCaml program, there is a associated \typename{frame\_descr}
        \item<2-> A \typename{frame\_descr} specifies
        \begin{itemize}
          \item<2-> The size of the function stack frame
          \item<3-> Where live values are on the stack (so that the GC can mark them as live and prevent from being swept)
        \end{itemize}
        \item<4-> When compiled with debug information, \typename{frame\_descr} will be linked to a \typename{frame\_debuginfo}
        \begin{itemize}
          \item<5-> Contains information about the position in the source code (file name, line and column number)
        \end{itemize}
      \end{itemize}
    \end{column}
    \begin{column}{0.5\textwidth}
        \onslide*<1>{\listFrameDescr[highlightlines={118-122}]{}}
        \onslide*<2>{\listFrameDescr[highlightlines={120}]{}}
        \onslide*<3>{\listFrameDescr[highlightlines={121}]{}}
        \onslide*<4->{\listFrameDescr[highlightlines={122}]{}}
        \medskip
        \onslide*<1-3>{\listDebugInfo{}}
        \onslide*<4>{\listDebugInfo[highlightlines={38,49}]{}}
        \onslide*<5>{\listDebugInfo[highlightlines={39-46}]{}}
    \end{column}
  \end{columns}
\end{frame}

\begin{frame}{Backtraces}{Scanning the stack with \typename{frame\_descr}}
  \begin{columns}[c]
    \begin{column}{0.5\textwidth}
      \begin{itemize}
        \item Given the return address of the code that raises the exception by calling \funcname{caml\_raise\_exn} (\funcarg{pc} argument of \funcname{caml\_stash\_backtrace})
        \item And the position of the stack pointer (\funcarg{sp} argument of \funcname{caml\_stash\_backtrace})
        \item The frame size given by \typename{frame\_descr}.\funcarg{frame\_size}, allows to find the end of the previous frame
        \item And the return address of the caller of this function
        \bigskip
        \onslide<3->{
        \item Note that traps (here the reddish \funcname{ExnA}, \funcname{ExnB} and \funcname{ExnC} blocks) belong to the frame that installed them, and so the frame size changes when traps are removed
        }
      \end{itemize}
    \end{column}
    \begin{column}{0.5\textwidth}
      \raggedleft
      \onslide*<1>{\providecommand\step{2}\input{diagrams/backtrace/backtrace.tex}}%
      \onslide*<2>{\providecommand\step{1}\input{diagrams/backtrace/scanstack.tex}}%
      \onslide*<3>{\providecommand\step{2}\input{diagrams/backtrace/scanstack.tex}}%
      \onslide*<4>{\providecommand\step{3}\input{diagrams/backtrace/scanstack.tex}}%
      \onslide*<5>{\providecommand\step{4}\input{diagrams/backtrace/scanstack.tex}}%
      \onslide*<6>{\providecommand\step{5}\input{diagrams/backtrace/scanstack.tex}}%
    \end{column}
  \end{columns}
\end{frame}

% Duplicate of previous-previous slide
\begin{frame}{Backtraces}{Continuing on \funcname{caml\_stash\_backtrace}}
  \begin{columns}[c]
    \begin{column}{0.5\textwidth}
      \minipage[c][0.75\textheight][s]{\columnwidth}
      \begin{itemize}
          \color{gray}
        \item A C function provided by the runtime (specific to native code)
        \item It scans the stack, between the stack pointer at the raising point up to the next trap, in order to collect the names of the detected function frames
        \item It uses the same technique and information as the Garbage Collector (GC) uses to scan the values live on the stack
      \end{itemize}
      \begin{itemize}
        \item For each function frame detected, store a pointer to the \typename{frame\_descr} in the \localname{domain\_state->backtrace\_buffer} array, effectively constructing the backtrace
      \end{itemize}
      \onslide<2->{
        \begin{itemize}
            \smallskip
          \item Constructed backtrace:
            \funcname{definitely\_raise},
            \onslide<3->{\funcname{probably\_raise}}
        \end{itemize}
      }
      \vfill
      \mintocaml[firstline=9,lastline=13,highlightlines=12]{../src/backtrace/backtrace.ml}
      \endminipage
    \end{column}
    \begin{column}{0.5\textwidth}
      \raggedleft
      \onslide*<1>{\listStashBacktrace[highlightlines={114-115}]{}}
      \onslide*<2>{\providecommand\step{3}\input{diagrams/backtrace/backtrace.tex}}%
      \onslide*<3>{\providecommand\step{4}\input{diagrams/backtrace/backtrace.tex}}%
    \end{column}
  \end{columns}
\end{frame}

\begin{frame}{Backtraces}{Back to \funcname{caml\_raise\_exn}}
  \begin{columns}[c]
    \begin{column}{0.5\textwidth}
      \minipage[c][0.66\textheight][s]{\columnwidth}
      \begin{itemize}\color{gray}
        \item Checks wheteher exceptions backtrace should be recorded, by checking the \funcarg{domain\_state->backtrace\_active} value
        \item Switches to the C stack to be able to execute C code
        \item Calls \funcname{caml\_stash\_backtrace}, to collect the backtrace up to the next trap
      \end{itemize}
      \onslide*<2->{
        \begin{itemize}
          \item<2-> Executes the exception handler in \funcname{probably\_raise} with \funcname{RESTORE\_EXN\_HANDLER\_OCAML}
            \begin{itemize}
              \item Set \regname{RSP} to \localname{domain\_state->exn\_handler} value
              \item Restore \localname{domain\_state->exn\_handler} from trap
              \item Jump to the handler code with \funcname{ret}
            \end{itemize}
        \end{itemize}
      }
      \vfill
      \onslide*<1>{\mintocaml[firstline=9,lastline=13,highlightlines=12]{../src/backtrace/backtrace.ml}}
      \onslide*<2->{\mintocaml[firstline=15,lastline=21,highlightlines={19-21}]{../src/backtrace/backtrace.ml}}
      \endminipage
    \end{column}
    \begin{column}{0.5\textwidth}
      \centering
      \onslide*<1>{
        \mintc[firstline=190,lastline=192]{../ocaml/runtime/amd64.S}
        \mintc[firstline=234,lastline=237]{../ocaml/runtime/amd64.S}
      }
      \onslide*<2>{
        \mintc[firstline=190,lastline=192,highlightlines={191-192}]{../ocaml/runtime/amd64.S}
        \mintc[firstline=234,lastline=237,highlightlines={235-237}]{../ocaml/runtime/amd64.S}
      }
      \onslide*<1>{\listCamlRaiseExn[highlightlines=720]{}}
      \onslide*<2>{\listCamlRaiseExn[highlightlines=722]{}}
      \onslide*<3->{\providecommand\step{5}\input{diagrams/backtrace/backtrace.tex}}
    \end{column}
  \end{columns}
\end{frame}

\begin{frame}{Backtraces}{\funcname{caml\_probably\_raise}'s handler doesn't catch \typename{ExnC}}
  \begin{columns}[c]
    \begin{column}{0.45\textwidth}
      \minipage[c][0.75\textheight][s]{\columnwidth}
      \begin{itemize}
        \item<1-> \funcname{match \dots with}\footnotemark pattern matching can also match exceptions
        \item<1-> The CMM of \funcname{probably\_raise} shows that this \funcname{match \dots with} is implemented with a pattern matching and a \funcname{try \dots with}
        \item<1-> But this one only matches on \typename{ExnA} exception
        \item<2-> Since the pattern matching fails to catch the exception, \funcname{reraise} is called with our current \typename{ExnC} exception
        \item<3-> Disassembly shows that \funcname{reraise} is actually a call to \funcname{caml\_reraise\_exn}, which is also an assembly function provided by the runtime
      \end{itemize}
      \vfill
      \mintocaml[firstline=15,lastline=21,highlightlines={18-21}]{../src/backtrace/backtrace.ml}
      \endminipage
    \end{column}
    \begin{column}{0.55\textwidth}
      \centering
      \onslide*<1>{\mintcmm[highlightlines={10-18}]{../src/backtrace/camlBacktrace\_\_probably\_raise\_351.cmm}}
      \onslide*<2>{\listcmm[highlightlines=18]{../src/backtrace/camlBacktrace\_\_probably\_raise_351.cmm}{CMM of probably\_raise}}
      \onslide*<3>{\mintobjdump[firstline=5,lastline=35,highlightlines=31]{../src/backtrace/camlBacktrace\_\_probably\_raise_351.objdump}}
    \end{column}
  \end{columns}
  \only<1>{\footnotetext[1]{https://blog.janestreet.com/pattern-matching-and-exception-handling-unite/}}
\end{frame}

\newcommand\listCamlReraiseExn[1][]{
  \listasm[firstline=727,lastline=734,#1]{../ocaml/runtime/amd64.S}{runtime/amd64.S:727}}

\begin{frame}{Backtraces}{Introducing \funcname{caml\_reraise\_exn}}
  \begin{columns}[c]
    \begin{column}{0.5\textwidth}
      \minipage[c][0.66\textheight][s]{\columnwidth}
      \begin{itemize}
        \item<1-> The exception have to be reraised to the next handler
        \item<2-> First, checks if the backtrace should be collected with \funcarg{domain\_state->backtrace\_active}
        \only<3>{
        \begin{itemize}
            \item If not, behaves like a \funcname{raise\_notrace} with \funcname{RESTORE\_EXN\_HANDLER\_OCAML}
        \end{itemize}
        }
        \item<4-> Jumps into \funcname{caml\_raise\_exn}
        \item<5-> Calls \funcname{caml\_stash\_backtrace} again, to scan the next part of the stack: from the trap just pop-ed to the next trap
      \end{itemize}
      \vfill
      \mintocaml[firstline=15,lastline=21,highlightlines={18-21}]{../src/backtrace/backtrace.ml}
      \endminipage
    \end{column}
    \begin{column}{0.5\textwidth}
      \raggedleft
      \onslide*<1>{\listCamlReraiseExn{}}
      \onslide*<2>{\listCamlReraiseExn[highlightlines=729]{}}
      \onslide*<3>{
        \mintc[firstline=190,lastline=192,highlightlines={191-192}]{../ocaml/runtime/amd64.S}
        \mintc[firstline=234,lastline=237,highlightlines={235-237}]{../ocaml/runtime/amd64.S}
        \medskip
        \listCamlReraiseExn[highlightlines=731]{}
      }
      \onslide*<4-5>{
        % TODO refactor with \listCamlReraiseExn
        \mintasm[firstline=727,lastline=734,highlightlines=730]{../ocaml/runtime/amd64.S}
        \medskip
      }
      \onslide*<4>{\listCamlRaiseExn[highlightlines=711]{}}
      \onslide*<5>{\listCamlRaiseExn[highlightlines=720]{}}
    \end{column}
  \end{columns}
\end{frame}

\begin{frame}{Backtraces}{Backtrace collection continues}
  \begin{columns}[c]
    \begin{column}{0.55\textwidth}
      \minipage[c][0.75\textheight][s]{\columnwidth}
      \begin{itemize}
        \item<1-> \funcname{caml\_stash\_backtrace} scans the older part of the stack, up to the next trap
        \item<6-> Controls jumps to the nested exception handler in \funcname{maybe\_raise}, which matches only on \typename{ExnB}
        \item<7-> \funcname{caml\_reraise\_exn} is called again, backtrace collection resumes with \funcname{caml\_stash\_backtrace}
      \end{itemize}
      \medskip
      \begin{itemize}
        \item<2-> Constructed backtrace:
        \begin{itemize}
          \item <2-> \funcname{definitely\_raise}
          \item <2-> \funcname{probably\_raise}
          \item <3-> \funcname{probably\_raise} (re-raise)
          \item <4-> \funcname{maybe\_raise}
          \item <5-> \funcname{main}
          \item <8-> \funcname{main} (re-raise)
        \end{itemize}
      \end{itemize}
      \vfill
      \onslide*<1-5>{\mintocaml[firstline=15,lastline=21]{../src/backtrace/backtrace.ml}}
      \onslide*<6>{\mintocaml[firstline=28,lastline=34,highlightlines=31]{../src/backtrace/backtrace.ml}}
      \onslide*<7->{\mintocaml[firstline=28,lastline=34,highlightlines=33]{../src/backtrace/backtrace.ml}}
      \endminipage
    \end{column}
    \begin{column}{0.45\textwidth}
      \raggedleft
      \onslide*<1>{\providecommand\step{6}\input{diagrams/backtrace/backtrace.tex}}%
      \onslide*<2>{\providecommand\step{6}\input{diagrams/backtrace/backtrace.tex}}%
      \onslide*<3>{\providecommand\step{7}\input{diagrams/backtrace/backtrace.tex}}%
      \onslide*<4>{\providecommand\step{8}\input{diagrams/backtrace/backtrace.tex}}%
      \onslide*<5>{\providecommand\step{9}\input{diagrams/backtrace/backtrace.tex}}%
      \onslide*<6>{\providecommand\step{10}\input{diagrams/backtrace/backtrace.tex}}%
      \onslide*<7>{\providecommand\step{11}\input{diagrams/backtrace/backtrace.tex}}%
      \onslide*<8>{\providecommand\step{12}\input{diagrams/backtrace/backtrace.tex}}%
      \onslide*<9>{\providecommand\step{13}\input{diagrams/backtrace/backtrace.tex}}%
    \end{column}
  \end{columns}
\end{frame}

\begin{frame}{Backtraces}{Printing the backtrace}
  \begin{columns}[c]
    \begin{column}{0.45\textwidth}
      \minipage[c][0.66\textheight][s]{\columnwidth}
      \begin{itemize}
        \item<1-> The exception backtrace can now be printed to \funcarg{stdout} with \funcname{Printexc.print_backtrace}
        \item<2-> First the exception backtrace is retrieved into an OCaml array with \funcname{get_raw_backtrace }
        \item<3-> Which is implemented as the \funcname{caml_get_exception_raw_backtrace} C function in the runtime
        \item<4-> And finally printed with \funcname{print_exception_backtrace}
      \end{itemize}
      \vfill
      \mintocaml[firstline=28,lastline=34,highlightlines=33]{../src/backtrace/backtrace.ml}
      \endminipage
    \end{column}
    \begin{column}{0.55\textwidth}
      \onslide*<1,2>{
        \mintocaml[firstline=100,lastline=101]{../ocaml/stdlib/printexc.ml}
      }
      \only<1>{
        \listocaml[firstline=170,lastline=172]{../ocaml/stdlib/printexc.ml}{stdlib/printexc.ml:170}
      }
      \only<2>{
        \listocaml[firstline=170,lastline=172,highlightlines=172]{../ocaml/stdlib/printexc.ml}{stdlib/printexc.ml:170}
      }
      \only<3>{
        \mintc[firstline=154,lastline=156]{../ocaml/runtime/backtrace.c}
        \listc[firstline=171,lastline=192,highlightlines={184-187}]{../ocaml/runtime/backtrace.c}{runtime/backtrace.c:154}
      }
      \only<4>{
        \listocaml[firstline=155,lastline=165]{../ocaml/stdlib/printexc.ml}{stdlib/printexc.ml:55}
      }
    \end{column}
  \end{columns}
\end{frame}


\subsection{The multiple kinds of raise}
\frameSubsection{}{}

\begin{frame}{Backtraces}{When backtraces are not needed}
  \begin{columns}[c]
    \begin{column}{0.45\textwidth}
      \minipage[c][0.66\textheight][s]{\columnwidth}
      \begin{itemize}
        \item<1-> What if the backtrace for an exception is never needed?
        \item<2-> It's possible to raise the exception explicitly with \funcname{raise\_notrace} to make raising as cheap as possible
        \item<3-> An example of use in the \funcname{dynlink} library of the OCaml compiler
      \end{itemize}
      \vfill
      \onslide<3->{
      \listocaml[firstline=205,lastline=209,highlightlines=207]{../ocaml/otherlibs/dynlink/dynlink\_compilerlibs/misc.ml}{otherlibs/dynlink/dynlink\_compilerlibs/misc.ml:205}
      }
      \endminipage
    \end{column}
    \begin{column}{0.55\textwidth}
    \only<4>{
      \mintcmm[highlightlines=3]{../src/notrace/camlNotrace\_\_fun_347.cmm}
      \listcmm{../src/notrace/camlNotrace\_\_all\_somes\_327.cmm}{all\_somes}
    }
    \onslide*<5->{
      \listobjdump[highlightlines={6-9}]{../src/notrace/camlNotrace\_\_fun_347.objdump}{anonymous function from \funcname{all\_somes}}
    }
    \end{column}
  \end{columns}
\end{frame}

\begin{frame}{Backtraces}{Summary table of the multiple kinds of raise}
  \begin{itemize}
    \item When debugging information is enabled and if the backtrace of an exception isn't needed, it's possible to raise with \funcname{raise\_notrace} for performance
    \item When debugging information is \emph{not} enabled, the compiler optimises \funcname{raise} calls to inline assembly (same effect as using \funcname{raise\_notrace})
  \end{itemize}
  \bigskip
  \centering
  \begin{tabular}{| l | c | c | c | c |}
    \hline
    \parbox{10em}{Debug information \\ (\funcarg{-g} flag)}  & \multicolumn{2}{|c|}{Disabled}                                     & \multicolumn{2}{|c|}{Enabled} \\
    \hline
    Raise kind                                               & \funcname{raise} & \funcname{raise\_notrace}                       & \funcname{raise} & \funcname{raise\_notrace} \\
    \hline
    Compilation                                              & \parbox{8em}{inline assembly\\ (optimization)} & inline assembly   & \funcname{caml\_raise\_exn} & inline assembly \\
    \hline
  \end{tabular}
\end{frame}


%
% Takeaway #5
%
\subsection*{Takeaway \#5}
\frameSubsectionTakeaway{}
\againframe<5>{takeaway}

    \end{column}
  \end{columns}
\end{frame}

\begin{frame}{Backtraces}{But before that, we need more fields from \typename{caml\_domain\_state}}
  \begin{columns}
    \begin{column}{0.5\textwidth}
      \begin{itemize}
        \item Remember \typename{caml\_domain\_state} the per-domain runtime state?
        \item Some other members are of interest to us now\dots
        \item \localname{backtrace\_active} is used to know if the runtime should collect backtraces
        \item \localname{backtrace\_active} can be set by
          \begin{itemize}
            \item The \funcarg{b} flag in the \funcname{OCAMLRUNPARAM} environment variable
            \item \mintinline{ocaml}{Printexc.record_backtrace : bool -> unit} \footnotemark
          \end{itemize}
      \end{itemize}
    \end{column}
    \begin{column}{0.5\textwidth}
      \begin{listing}
        \mintc[highlightlines={9-11}]{src/ocaml/domain\_state\_backtrace.h}
        \caption{runtime/caml/domain\_state.h}
      \end{listing}
    \end{column}
  \end{columns}
  \footnotetext[1]{https://v2.ocaml.org/api/Printexc.html}
\end{frame}

\newcommand\listCamlRaiseExn[1][]{
  \listasm[firstline=702,lastline=725,#1]{../ocaml/runtime/amd64.S}{runtime/amd64.S:702}}

\begin{frame}{Backtraces}{Walk through \funcname{caml\_raise\_exn}}
  \begin{columns}[T]
    \begin{column}{0.5\textwidth}
      \begin{itemize}
        \item<1-> First checks whether exceptions backtrace should be recorded
        \item<1-> By checking the \funcarg{domain\_state->backtrace\_active} value
        \item<2-> If not, acts as \funcname{raise\_notrace} with \funcname{RESTORE\_EXN\_HANDLER\_OCAML}
          \begin{itemize}
            \item Set \regname{RSP} to \localname{domain\_state->exn\_handler} value
            \item Restore \localname{domain\_state->exn\_handler} from trap
            \item Jump with \funcname{ret} (same effect as using \funcname{pop} and \funcname{jmp})
          \end{itemize}
        \item<3-> Otherwise, switches to the C stack to be able to execute C code
        \item<4-> Calls \funcname{caml\_stash\_backtrace}, a C function provided by the runtime\ignorespaces
          \onslide<6->{, with
            \begin{itemize}
              \item \funcarg{exn}: exception value
              \item \funcarg{pc}: code address of the raising point
              \item \funcarg{sp}: stack address of the raising function
              \item \funcarg{trapsp}: stack address of the next handler
            \end{itemize}
          }
      \end{itemize}
      \bigskip
      \mintocaml[firstline=9,lastline=13,highlightlines=12]{../src/backtrace/backtrace.ml}
    \end{column}
    \begin{column}{0.5\textwidth}
      \raggedleft
      \onslide*<1,3-4>{
        \mintc[firstline=190,lastline=192]{../ocaml/runtime/amd64.S}
        \mintc[firstline=234,lastline=237]{../ocaml/runtime/amd64.S}
      }
      \onslide*<2>{
        \mintc[firstline=190,lastline=192,highlightlines={191-192}]{../ocaml/runtime/amd64.S}
        \mintc[firstline=234,lastline=237,highlightlines={235-237}]{../ocaml/runtime/amd64.S}
      }
      \onslide*<1>{\listCamlRaiseExn[highlightlines=705]{}}%
      \onslide*<2>{\listCamlRaiseExn[highlightlines={707-708}]{}}%
      \onslide*<3>{\listCamlRaiseExn[highlightlines={712-714}]{}}%
      \onslide*<4>{\listCamlRaiseExn[highlightlines={715-720}]{}}%
      \onslide*<5>{\providecommand\step{1}%
% Backtraces
%
\subsection{One advantage of raising with debugging information}
\frameSubsection{}{}

\begin{frame}{Backtraces}{Debugging information \& source code}
  \begin{columns}[c]
    \begin{column}{0.5\textwidth}
      \begin{itemize}
        \item Raising an exception is fun and games, but how do we know where the exception originated? $\rightarrow$ With the backtrace associated with the exception!
        \item In order to get a backtrace from the point where an exception was raised, it's necessary to compile with debugging information enabled (\funcarg{-g} flag for the compiler)
        \item Same example as in the `Nested handlers' section
      \end{itemize}
    \end{column}
    \begin{column}{0.5\textwidth}
      \listocaml[firstline=9,lastline=34]{../src/backtrace/backtrace.ml}{backtrace/backtrace.ml}
    \end{column}
  \end{columns}
\end{frame}

\begin{frame}{Backtraces}{CMM and disassembly of \funcname{definitely\_raise}}
  \begin{columns}[c]
    \begin{column}{0.5\textwidth}
      \minipage[c][0.66\textheight][s]{\columnwidth}
      \begin{itemize}
        \item<1-> An effect of compiling with debugging information (\funcarg{-g}): the CMM no longer uses \funcname{raise\_notrace} to raise the exception but \funcname{raise} instead
        \item<2-> Disassembly shows that CMM \funcname{raise} is actually a call to \funcname{caml\_raise\_exn}, a function in assembler provided by the runtime
      \end{itemize}
      \vfill
      \mintocaml[firstline=9,lastline=13,highlightlines=12]{../src/backtrace/backtrace.ml}
      \endminipage
    \end{column}
    \begin{column}{0.5\textwidth}
      \onslide*<1>{
        \mintcmm[highlightlines=5]{../src/backtrace/camlBacktrace\_\_definitely\_raise\_347.cmm}
      }
      \onslide*<2>{
        \mintobjdump[firstline=2,highlightlines=14]{../src/backtrace/camlBacktrace\_\_definitely\_raise\_347.objdump}
      }
    \end{column}
  \end{columns}
\end{frame}

\begin{frame}{Backtraces}{Introducing \funcname{caml\_raise\_exn}}
  \begin{columns}[c]
    \begin{column}{0.5\textwidth}
      \minipage[c][0.66\textheight][s]{\columnwidth}
      \onslide*<1>{
        \begin{itemize}
          \item Raising the exception is performed using \funcname{caml\_raise\_exn} which is
            \begin{itemize}
              \item An assembly function provided by the runtime
              \item Architecture specific
            \end{itemize}
          \item Let's see what it does differently from \funcname{raise\_notrace}\dots
          \item We are about to raise \typename{ExnC} with \funcname{caml\_raise\_exn} from \funcname{definitely\_raise}
        \end{itemize}
      }
      \onslide*<2>{
        \mintobjdump[firstline=2,highlightlines=14]{../src/backtrace/camlBacktrace\_\_definitely\_raise\_347.objdump}
      }
      \vfill
      \mintocaml[firstline=9,lastline=13,highlightlines=12]{../src/backtrace/backtrace.ml}
      \endminipage
    \end{column}
    \begin{column}{0.5\textwidth}
      \centering
      \providecommand\step{1}\input{diagrams/backtrace/backtrace.tex}
    \end{column}
  \end{columns}
\end{frame}

\begin{frame}{Backtraces}{But before that, we need more fields from \typename{caml\_domain\_state}}
  \begin{columns}
    \begin{column}{0.5\textwidth}
      \begin{itemize}
        \item Remember \typename{caml\_domain\_state} the per-domain runtime state?
        \item Some other members are of interest to us now\dots
        \item \localname{backtrace\_active} is used to know if the runtime should collect backtraces
        \item \localname{backtrace\_active} can be set by
          \begin{itemize}
            \item The \funcarg{b} flag in the \funcname{OCAMLRUNPARAM} environment variable
            \item \mintinline{ocaml}{Printexc.record_backtrace : bool -> unit} \footnotemark
          \end{itemize}
      \end{itemize}
    \end{column}
    \begin{column}{0.5\textwidth}
      \begin{listing}
        \mintc[highlightlines={9-11}]{src/ocaml/domain\_state\_backtrace.h}
        \caption{runtime/caml/domain\_state.h}
      \end{listing}
    \end{column}
  \end{columns}
  \footnotetext[1]{https://v2.ocaml.org/api/Printexc.html}
\end{frame}

\newcommand\listCamlRaiseExn[1][]{
  \listasm[firstline=702,lastline=725,#1]{../ocaml/runtime/amd64.S}{runtime/amd64.S:702}}

\begin{frame}{Backtraces}{Walk through \funcname{caml\_raise\_exn}}
  \begin{columns}[T]
    \begin{column}{0.5\textwidth}
      \begin{itemize}
        \item<1-> First checks whether exceptions backtrace should be recorded
        \item<1-> By checking the \funcarg{domain\_state->backtrace\_active} value
        \item<2-> If not, acts as \funcname{raise\_notrace} with \funcname{RESTORE\_EXN\_HANDLER\_OCAML}
          \begin{itemize}
            \item Set \regname{RSP} to \localname{domain\_state->exn\_handler} value
            \item Restore \localname{domain\_state->exn\_handler} from trap
            \item Jump with \funcname{ret} (same effect as using \funcname{pop} and \funcname{jmp})
          \end{itemize}
        \item<3-> Otherwise, switches to the C stack to be able to execute C code
        \item<4-> Calls \funcname{caml\_stash\_backtrace}, a C function provided by the runtime\ignorespaces
          \onslide<6->{, with
            \begin{itemize}
              \item \funcarg{exn}: exception value
              \item \funcarg{pc}: code address of the raising point
              \item \funcarg{sp}: stack address of the raising function
              \item \funcarg{trapsp}: stack address of the next handler
            \end{itemize}
          }
      \end{itemize}
      \bigskip
      \mintocaml[firstline=9,lastline=13,highlightlines=12]{../src/backtrace/backtrace.ml}
    \end{column}
    \begin{column}{0.5\textwidth}
      \raggedleft
      \onslide*<1,3-4>{
        \mintc[firstline=190,lastline=192]{../ocaml/runtime/amd64.S}
        \mintc[firstline=234,lastline=237]{../ocaml/runtime/amd64.S}
      }
      \onslide*<2>{
        \mintc[firstline=190,lastline=192,highlightlines={191-192}]{../ocaml/runtime/amd64.S}
        \mintc[firstline=234,lastline=237,highlightlines={235-237}]{../ocaml/runtime/amd64.S}
      }
      \onslide*<1>{\listCamlRaiseExn[highlightlines=705]{}}%
      \onslide*<2>{\listCamlRaiseExn[highlightlines={707-708}]{}}%
      \onslide*<3>{\listCamlRaiseExn[highlightlines={712-714}]{}}%
      \onslide*<4>{\listCamlRaiseExn[highlightlines={715-720}]{}}%
      \onslide*<5>{\providecommand\step{1}\input{diagrams/backtrace/backtrace.tex}}%
      \onslide*<6>{\providecommand\step{2}\input{diagrams/backtrace/backtrace.tex}}%
    \end{column}
  \end{columns}
\end{frame}

\newcommand\listStashBacktrace[1][]{
  \listc[firstline=91,lastline=120,#1]{../ocaml/runtime/backtrace\_nat.c}{runtime/backtrace\_nat.c:91}}

\begin{frame}{Backtraces}{Introducing \funcname{caml\_stash\_backtrace}}
  \begin{columns}[c]
    \begin{column}{0.5\textwidth}
      \minipage[c][0.66\textheight][s]{\columnwidth}
      \begin{itemize}
        \item<1-> A C function provided by the runtime (specific to native code)
        \item<2-> It scans the stack, between the stack pointer at the raising point up to the next trap, in order to collect the names of the detected function frames
          \onslide*<2>{
            \begin{itemize}
              \item And more information, like line number, column number...
            \end{itemize}
          }
        \item<3-> It uses the same technique and information as the Garbage Collector (GC) uses to scan the values live on the stack
          \onslide*<4>{
            \begin{itemize}
              \item \typename{frame\_descr} are at the core of stack unwinding
            \end{itemize}
          }
      \end{itemize}
      \vfill
      \mintocaml[firstline=9,lastline=13,highlightlines=12]{../src/backtrace/backtrace.ml}
      \endminipage
    \end{column}
    \begin{column}{0.5\textwidth}
      \onslide*<1>{\listStashBacktrace[highlightlines=91]{}}
      \onslide*<2>{\listStashBacktrace[highlightlines={91,117-118}]{}}
      \onslide*<3->{\listStashBacktrace[highlightlines={94,106,109-110}]{}}
    \end{column}
  \end{columns}
\end{frame}

\newcommand\listFrameDescr[1][]{
  \listocaml[firstline=111,lastline=124,#1]{../ocaml/asmcomp/emitaux.ml}{asmcomp/emitaux.ml:111}}
\newcommand\listDebugInfo[1][]{
  \listocaml[firstline=38,lastline=49,#1]{../ocaml/lambda/debuginfo.mli}{lambda/debuginfo.mli:38}}

\begin{frame}{Backtraces}{\typename{frame\_descr} and \typename{frame\_debuginfo} in a nutshell}
  \begin{columns}[c]
    \begin{column}{0.5\textwidth}
      \begin{itemize}
        \item<1-> For every return address in an OCaml program, there is a associated \typename{frame\_descr}
        \item<2-> A \typename{frame\_descr} specifies
        \begin{itemize}
          \item<2-> The size of the function stack frame
          \item<3-> Where live values are on the stack (so that the GC can mark them as live and prevent from being swept)
        \end{itemize}
        \item<4-> When compiled with debug information, \typename{frame\_descr} will be linked to a \typename{frame\_debuginfo}
        \begin{itemize}
          \item<5-> Contains information about the position in the source code (file name, line and column number)
        \end{itemize}
      \end{itemize}
    \end{column}
    \begin{column}{0.5\textwidth}
        \onslide*<1>{\listFrameDescr[highlightlines={118-122}]{}}
        \onslide*<2>{\listFrameDescr[highlightlines={120}]{}}
        \onslide*<3>{\listFrameDescr[highlightlines={121}]{}}
        \onslide*<4->{\listFrameDescr[highlightlines={122}]{}}
        \medskip
        \onslide*<1-3>{\listDebugInfo{}}
        \onslide*<4>{\listDebugInfo[highlightlines={38,49}]{}}
        \onslide*<5>{\listDebugInfo[highlightlines={39-46}]{}}
    \end{column}
  \end{columns}
\end{frame}

\begin{frame}{Backtraces}{Scanning the stack with \typename{frame\_descr}}
  \begin{columns}[c]
    \begin{column}{0.5\textwidth}
      \begin{itemize}
        \item Given the return address of the code that raises the exception by calling \funcname{caml\_raise\_exn} (\funcarg{pc} argument of \funcname{caml\_stash\_backtrace})
        \item And the position of the stack pointer (\funcarg{sp} argument of \funcname{caml\_stash\_backtrace})
        \item The frame size given by \typename{frame\_descr}.\funcarg{frame\_size}, allows to find the end of the previous frame
        \item And the return address of the caller of this function
        \bigskip
        \onslide<3->{
        \item Note that traps (here the reddish \funcname{ExnA}, \funcname{ExnB} and \funcname{ExnC} blocks) belong to the frame that installed them, and so the frame size changes when traps are removed
        }
      \end{itemize}
    \end{column}
    \begin{column}{0.5\textwidth}
      \raggedleft
      \onslide*<1>{\providecommand\step{2}\input{diagrams/backtrace/backtrace.tex}}%
      \onslide*<2>{\providecommand\step{1}\input{diagrams/backtrace/scanstack.tex}}%
      \onslide*<3>{\providecommand\step{2}\input{diagrams/backtrace/scanstack.tex}}%
      \onslide*<4>{\providecommand\step{3}\input{diagrams/backtrace/scanstack.tex}}%
      \onslide*<5>{\providecommand\step{4}\input{diagrams/backtrace/scanstack.tex}}%
      \onslide*<6>{\providecommand\step{5}\input{diagrams/backtrace/scanstack.tex}}%
    \end{column}
  \end{columns}
\end{frame}

% Duplicate of previous-previous slide
\begin{frame}{Backtraces}{Continuing on \funcname{caml\_stash\_backtrace}}
  \begin{columns}[c]
    \begin{column}{0.5\textwidth}
      \minipage[c][0.75\textheight][s]{\columnwidth}
      \begin{itemize}
          \color{gray}
        \item A C function provided by the runtime (specific to native code)
        \item It scans the stack, between the stack pointer at the raising point up to the next trap, in order to collect the names of the detected function frames
        \item It uses the same technique and information as the Garbage Collector (GC) uses to scan the values live on the stack
      \end{itemize}
      \begin{itemize}
        \item For each function frame detected, store a pointer to the \typename{frame\_descr} in the \localname{domain\_state->backtrace\_buffer} array, effectively constructing the backtrace
      \end{itemize}
      \onslide<2->{
        \begin{itemize}
            \smallskip
          \item Constructed backtrace:
            \funcname{definitely\_raise},
            \onslide<3->{\funcname{probably\_raise}}
        \end{itemize}
      }
      \vfill
      \mintocaml[firstline=9,lastline=13,highlightlines=12]{../src/backtrace/backtrace.ml}
      \endminipage
    \end{column}
    \begin{column}{0.5\textwidth}
      \raggedleft
      \onslide*<1>{\listStashBacktrace[highlightlines={114-115}]{}}
      \onslide*<2>{\providecommand\step{3}\input{diagrams/backtrace/backtrace.tex}}%
      \onslide*<3>{\providecommand\step{4}\input{diagrams/backtrace/backtrace.tex}}%
    \end{column}
  \end{columns}
\end{frame}

\begin{frame}{Backtraces}{Back to \funcname{caml\_raise\_exn}}
  \begin{columns}[c]
    \begin{column}{0.5\textwidth}
      \minipage[c][0.66\textheight][s]{\columnwidth}
      \begin{itemize}\color{gray}
        \item Checks wheteher exceptions backtrace should be recorded, by checking the \funcarg{domain\_state->backtrace\_active} value
        \item Switches to the C stack to be able to execute C code
        \item Calls \funcname{caml\_stash\_backtrace}, to collect the backtrace up to the next trap
      \end{itemize}
      \onslide*<2->{
        \begin{itemize}
          \item<2-> Executes the exception handler in \funcname{probably\_raise} with \funcname{RESTORE\_EXN\_HANDLER\_OCAML}
            \begin{itemize}
              \item Set \regname{RSP} to \localname{domain\_state->exn\_handler} value
              \item Restore \localname{domain\_state->exn\_handler} from trap
              \item Jump to the handler code with \funcname{ret}
            \end{itemize}
        \end{itemize}
      }
      \vfill
      \onslide*<1>{\mintocaml[firstline=9,lastline=13,highlightlines=12]{../src/backtrace/backtrace.ml}}
      \onslide*<2->{\mintocaml[firstline=15,lastline=21,highlightlines={19-21}]{../src/backtrace/backtrace.ml}}
      \endminipage
    \end{column}
    \begin{column}{0.5\textwidth}
      \centering
      \onslide*<1>{
        \mintc[firstline=190,lastline=192]{../ocaml/runtime/amd64.S}
        \mintc[firstline=234,lastline=237]{../ocaml/runtime/amd64.S}
      }
      \onslide*<2>{
        \mintc[firstline=190,lastline=192,highlightlines={191-192}]{../ocaml/runtime/amd64.S}
        \mintc[firstline=234,lastline=237,highlightlines={235-237}]{../ocaml/runtime/amd64.S}
      }
      \onslide*<1>{\listCamlRaiseExn[highlightlines=720]{}}
      \onslide*<2>{\listCamlRaiseExn[highlightlines=722]{}}
      \onslide*<3->{\providecommand\step{5}\input{diagrams/backtrace/backtrace.tex}}
    \end{column}
  \end{columns}
\end{frame}

\begin{frame}{Backtraces}{\funcname{caml\_probably\_raise}'s handler doesn't catch \typename{ExnC}}
  \begin{columns}[c]
    \begin{column}{0.45\textwidth}
      \minipage[c][0.75\textheight][s]{\columnwidth}
      \begin{itemize}
        \item<1-> \funcname{match \dots with}\footnotemark pattern matching can also match exceptions
        \item<1-> The CMM of \funcname{probably\_raise} shows that this \funcname{match \dots with} is implemented with a pattern matching and a \funcname{try \dots with}
        \item<1-> But this one only matches on \typename{ExnA} exception
        \item<2-> Since the pattern matching fails to catch the exception, \funcname{reraise} is called with our current \typename{ExnC} exception
        \item<3-> Disassembly shows that \funcname{reraise} is actually a call to \funcname{caml\_reraise\_exn}, which is also an assembly function provided by the runtime
      \end{itemize}
      \vfill
      \mintocaml[firstline=15,lastline=21,highlightlines={18-21}]{../src/backtrace/backtrace.ml}
      \endminipage
    \end{column}
    \begin{column}{0.55\textwidth}
      \centering
      \onslide*<1>{\mintcmm[highlightlines={10-18}]{../src/backtrace/camlBacktrace\_\_probably\_raise\_351.cmm}}
      \onslide*<2>{\listcmm[highlightlines=18]{../src/backtrace/camlBacktrace\_\_probably\_raise_351.cmm}{CMM of probably\_raise}}
      \onslide*<3>{\mintobjdump[firstline=5,lastline=35,highlightlines=31]{../src/backtrace/camlBacktrace\_\_probably\_raise_351.objdump}}
    \end{column}
  \end{columns}
  \only<1>{\footnotetext[1]{https://blog.janestreet.com/pattern-matching-and-exception-handling-unite/}}
\end{frame}

\newcommand\listCamlReraiseExn[1][]{
  \listasm[firstline=727,lastline=734,#1]{../ocaml/runtime/amd64.S}{runtime/amd64.S:727}}

\begin{frame}{Backtraces}{Introducing \funcname{caml\_reraise\_exn}}
  \begin{columns}[c]
    \begin{column}{0.5\textwidth}
      \minipage[c][0.66\textheight][s]{\columnwidth}
      \begin{itemize}
        \item<1-> The exception have to be reraised to the next handler
        \item<2-> First, checks if the backtrace should be collected with \funcarg{domain\_state->backtrace\_active}
        \only<3>{
        \begin{itemize}
            \item If not, behaves like a \funcname{raise\_notrace} with \funcname{RESTORE\_EXN\_HANDLER\_OCAML}
        \end{itemize}
        }
        \item<4-> Jumps into \funcname{caml\_raise\_exn}
        \item<5-> Calls \funcname{caml\_stash\_backtrace} again, to scan the next part of the stack: from the trap just pop-ed to the next trap
      \end{itemize}
      \vfill
      \mintocaml[firstline=15,lastline=21,highlightlines={18-21}]{../src/backtrace/backtrace.ml}
      \endminipage
    \end{column}
    \begin{column}{0.5\textwidth}
      \raggedleft
      \onslide*<1>{\listCamlReraiseExn{}}
      \onslide*<2>{\listCamlReraiseExn[highlightlines=729]{}}
      \onslide*<3>{
        \mintc[firstline=190,lastline=192,highlightlines={191-192}]{../ocaml/runtime/amd64.S}
        \mintc[firstline=234,lastline=237,highlightlines={235-237}]{../ocaml/runtime/amd64.S}
        \medskip
        \listCamlReraiseExn[highlightlines=731]{}
      }
      \onslide*<4-5>{
        % TODO refactor with \listCamlReraiseExn
        \mintasm[firstline=727,lastline=734,highlightlines=730]{../ocaml/runtime/amd64.S}
        \medskip
      }
      \onslide*<4>{\listCamlRaiseExn[highlightlines=711]{}}
      \onslide*<5>{\listCamlRaiseExn[highlightlines=720]{}}
    \end{column}
  \end{columns}
\end{frame}

\begin{frame}{Backtraces}{Backtrace collection continues}
  \begin{columns}[c]
    \begin{column}{0.55\textwidth}
      \minipage[c][0.75\textheight][s]{\columnwidth}
      \begin{itemize}
        \item<1-> \funcname{caml\_stash\_backtrace} scans the older part of the stack, up to the next trap
        \item<6-> Controls jumps to the nested exception handler in \funcname{maybe\_raise}, which matches only on \typename{ExnB}
        \item<7-> \funcname{caml\_reraise\_exn} is called again, backtrace collection resumes with \funcname{caml\_stash\_backtrace}
      \end{itemize}
      \medskip
      \begin{itemize}
        \item<2-> Constructed backtrace:
        \begin{itemize}
          \item <2-> \funcname{definitely\_raise}
          \item <2-> \funcname{probably\_raise}
          \item <3-> \funcname{probably\_raise} (re-raise)
          \item <4-> \funcname{maybe\_raise}
          \item <5-> \funcname{main}
          \item <8-> \funcname{main} (re-raise)
        \end{itemize}
      \end{itemize}
      \vfill
      \onslide*<1-5>{\mintocaml[firstline=15,lastline=21]{../src/backtrace/backtrace.ml}}
      \onslide*<6>{\mintocaml[firstline=28,lastline=34,highlightlines=31]{../src/backtrace/backtrace.ml}}
      \onslide*<7->{\mintocaml[firstline=28,lastline=34,highlightlines=33]{../src/backtrace/backtrace.ml}}
      \endminipage
    \end{column}
    \begin{column}{0.45\textwidth}
      \raggedleft
      \onslide*<1>{\providecommand\step{6}\input{diagrams/backtrace/backtrace.tex}}%
      \onslide*<2>{\providecommand\step{6}\input{diagrams/backtrace/backtrace.tex}}%
      \onslide*<3>{\providecommand\step{7}\input{diagrams/backtrace/backtrace.tex}}%
      \onslide*<4>{\providecommand\step{8}\input{diagrams/backtrace/backtrace.tex}}%
      \onslide*<5>{\providecommand\step{9}\input{diagrams/backtrace/backtrace.tex}}%
      \onslide*<6>{\providecommand\step{10}\input{diagrams/backtrace/backtrace.tex}}%
      \onslide*<7>{\providecommand\step{11}\input{diagrams/backtrace/backtrace.tex}}%
      \onslide*<8>{\providecommand\step{12}\input{diagrams/backtrace/backtrace.tex}}%
      \onslide*<9>{\providecommand\step{13}\input{diagrams/backtrace/backtrace.tex}}%
    \end{column}
  \end{columns}
\end{frame}

\begin{frame}{Backtraces}{Printing the backtrace}
  \begin{columns}[c]
    \begin{column}{0.45\textwidth}
      \minipage[c][0.66\textheight][s]{\columnwidth}
      \begin{itemize}
        \item<1-> The exception backtrace can now be printed to \funcarg{stdout} with \funcname{Printexc.print_backtrace}
        \item<2-> First the exception backtrace is retrieved into an OCaml array with \funcname{get_raw_backtrace }
        \item<3-> Which is implemented as the \funcname{caml_get_exception_raw_backtrace} C function in the runtime
        \item<4-> And finally printed with \funcname{print_exception_backtrace}
      \end{itemize}
      \vfill
      \mintocaml[firstline=28,lastline=34,highlightlines=33]{../src/backtrace/backtrace.ml}
      \endminipage
    \end{column}
    \begin{column}{0.55\textwidth}
      \onslide*<1,2>{
        \mintocaml[firstline=100,lastline=101]{../ocaml/stdlib/printexc.ml}
      }
      \only<1>{
        \listocaml[firstline=170,lastline=172]{../ocaml/stdlib/printexc.ml}{stdlib/printexc.ml:170}
      }
      \only<2>{
        \listocaml[firstline=170,lastline=172,highlightlines=172]{../ocaml/stdlib/printexc.ml}{stdlib/printexc.ml:170}
      }
      \only<3>{
        \mintc[firstline=154,lastline=156]{../ocaml/runtime/backtrace.c}
        \listc[firstline=171,lastline=192,highlightlines={184-187}]{../ocaml/runtime/backtrace.c}{runtime/backtrace.c:154}
      }
      \only<4>{
        \listocaml[firstline=155,lastline=165]{../ocaml/stdlib/printexc.ml}{stdlib/printexc.ml:55}
      }
    \end{column}
  \end{columns}
\end{frame}


\subsection{The multiple kinds of raise}
\frameSubsection{}{}

\begin{frame}{Backtraces}{When backtraces are not needed}
  \begin{columns}[c]
    \begin{column}{0.45\textwidth}
      \minipage[c][0.66\textheight][s]{\columnwidth}
      \begin{itemize}
        \item<1-> What if the backtrace for an exception is never needed?
        \item<2-> It's possible to raise the exception explicitly with \funcname{raise\_notrace} to make raising as cheap as possible
        \item<3-> An example of use in the \funcname{dynlink} library of the OCaml compiler
      \end{itemize}
      \vfill
      \onslide<3->{
      \listocaml[firstline=205,lastline=209,highlightlines=207]{../ocaml/otherlibs/dynlink/dynlink\_compilerlibs/misc.ml}{otherlibs/dynlink/dynlink\_compilerlibs/misc.ml:205}
      }
      \endminipage
    \end{column}
    \begin{column}{0.55\textwidth}
    \only<4>{
      \mintcmm[highlightlines=3]{../src/notrace/camlNotrace\_\_fun_347.cmm}
      \listcmm{../src/notrace/camlNotrace\_\_all\_somes\_327.cmm}{all\_somes}
    }
    \onslide*<5->{
      \listobjdump[highlightlines={6-9}]{../src/notrace/camlNotrace\_\_fun_347.objdump}{anonymous function from \funcname{all\_somes}}
    }
    \end{column}
  \end{columns}
\end{frame}

\begin{frame}{Backtraces}{Summary table of the multiple kinds of raise}
  \begin{itemize}
    \item When debugging information is enabled and if the backtrace of an exception isn't needed, it's possible to raise with \funcname{raise\_notrace} for performance
    \item When debugging information is \emph{not} enabled, the compiler optimises \funcname{raise} calls to inline assembly (same effect as using \funcname{raise\_notrace})
  \end{itemize}
  \bigskip
  \centering
  \begin{tabular}{| l | c | c | c | c |}
    \hline
    \parbox{10em}{Debug information \\ (\funcarg{-g} flag)}  & \multicolumn{2}{|c|}{Disabled}                                     & \multicolumn{2}{|c|}{Enabled} \\
    \hline
    Raise kind                                               & \funcname{raise} & \funcname{raise\_notrace}                       & \funcname{raise} & \funcname{raise\_notrace} \\
    \hline
    Compilation                                              & \parbox{8em}{inline assembly\\ (optimization)} & inline assembly   & \funcname{caml\_raise\_exn} & inline assembly \\
    \hline
  \end{tabular}
\end{frame}


%
% Takeaway #5
%
\subsection*{Takeaway \#5}
\frameSubsectionTakeaway{}
\againframe<5>{takeaway}
}%
      \onslide*<6>{\providecommand\step{2}%
% Backtraces
%
\subsection{One advantage of raising with debugging information}
\frameSubsection{}{}

\begin{frame}{Backtraces}{Debugging information \& source code}
  \begin{columns}[c]
    \begin{column}{0.5\textwidth}
      \begin{itemize}
        \item Raising an exception is fun and games, but how do we know where the exception originated? $\rightarrow$ With the backtrace associated with the exception!
        \item In order to get a backtrace from the point where an exception was raised, it's necessary to compile with debugging information enabled (\funcarg{-g} flag for the compiler)
        \item Same example as in the `Nested handlers' section
      \end{itemize}
    \end{column}
    \begin{column}{0.5\textwidth}
      \listocaml[firstline=9,lastline=34]{../src/backtrace/backtrace.ml}{backtrace/backtrace.ml}
    \end{column}
  \end{columns}
\end{frame}

\begin{frame}{Backtraces}{CMM and disassembly of \funcname{definitely\_raise}}
  \begin{columns}[c]
    \begin{column}{0.5\textwidth}
      \minipage[c][0.66\textheight][s]{\columnwidth}
      \begin{itemize}
        \item<1-> An effect of compiling with debugging information (\funcarg{-g}): the CMM no longer uses \funcname{raise\_notrace} to raise the exception but \funcname{raise} instead
        \item<2-> Disassembly shows that CMM \funcname{raise} is actually a call to \funcname{caml\_raise\_exn}, a function in assembler provided by the runtime
      \end{itemize}
      \vfill
      \mintocaml[firstline=9,lastline=13,highlightlines=12]{../src/backtrace/backtrace.ml}
      \endminipage
    \end{column}
    \begin{column}{0.5\textwidth}
      \onslide*<1>{
        \mintcmm[highlightlines=5]{../src/backtrace/camlBacktrace\_\_definitely\_raise\_347.cmm}
      }
      \onslide*<2>{
        \mintobjdump[firstline=2,highlightlines=14]{../src/backtrace/camlBacktrace\_\_definitely\_raise\_347.objdump}
      }
    \end{column}
  \end{columns}
\end{frame}

\begin{frame}{Backtraces}{Introducing \funcname{caml\_raise\_exn}}
  \begin{columns}[c]
    \begin{column}{0.5\textwidth}
      \minipage[c][0.66\textheight][s]{\columnwidth}
      \onslide*<1>{
        \begin{itemize}
          \item Raising the exception is performed using \funcname{caml\_raise\_exn} which is
            \begin{itemize}
              \item An assembly function provided by the runtime
              \item Architecture specific
            \end{itemize}
          \item Let's see what it does differently from \funcname{raise\_notrace}\dots
          \item We are about to raise \typename{ExnC} with \funcname{caml\_raise\_exn} from \funcname{definitely\_raise}
        \end{itemize}
      }
      \onslide*<2>{
        \mintobjdump[firstline=2,highlightlines=14]{../src/backtrace/camlBacktrace\_\_definitely\_raise\_347.objdump}
      }
      \vfill
      \mintocaml[firstline=9,lastline=13,highlightlines=12]{../src/backtrace/backtrace.ml}
      \endminipage
    \end{column}
    \begin{column}{0.5\textwidth}
      \centering
      \providecommand\step{1}\input{diagrams/backtrace/backtrace.tex}
    \end{column}
  \end{columns}
\end{frame}

\begin{frame}{Backtraces}{But before that, we need more fields from \typename{caml\_domain\_state}}
  \begin{columns}
    \begin{column}{0.5\textwidth}
      \begin{itemize}
        \item Remember \typename{caml\_domain\_state} the per-domain runtime state?
        \item Some other members are of interest to us now\dots
        \item \localname{backtrace\_active} is used to know if the runtime should collect backtraces
        \item \localname{backtrace\_active} can be set by
          \begin{itemize}
            \item The \funcarg{b} flag in the \funcname{OCAMLRUNPARAM} environment variable
            \item \mintinline{ocaml}{Printexc.record_backtrace : bool -> unit} \footnotemark
          \end{itemize}
      \end{itemize}
    \end{column}
    \begin{column}{0.5\textwidth}
      \begin{listing}
        \mintc[highlightlines={9-11}]{src/ocaml/domain\_state\_backtrace.h}
        \caption{runtime/caml/domain\_state.h}
      \end{listing}
    \end{column}
  \end{columns}
  \footnotetext[1]{https://v2.ocaml.org/api/Printexc.html}
\end{frame}

\newcommand\listCamlRaiseExn[1][]{
  \listasm[firstline=702,lastline=725,#1]{../ocaml/runtime/amd64.S}{runtime/amd64.S:702}}

\begin{frame}{Backtraces}{Walk through \funcname{caml\_raise\_exn}}
  \begin{columns}[T]
    \begin{column}{0.5\textwidth}
      \begin{itemize}
        \item<1-> First checks whether exceptions backtrace should be recorded
        \item<1-> By checking the \funcarg{domain\_state->backtrace\_active} value
        \item<2-> If not, acts as \funcname{raise\_notrace} with \funcname{RESTORE\_EXN\_HANDLER\_OCAML}
          \begin{itemize}
            \item Set \regname{RSP} to \localname{domain\_state->exn\_handler} value
            \item Restore \localname{domain\_state->exn\_handler} from trap
            \item Jump with \funcname{ret} (same effect as using \funcname{pop} and \funcname{jmp})
          \end{itemize}
        \item<3-> Otherwise, switches to the C stack to be able to execute C code
        \item<4-> Calls \funcname{caml\_stash\_backtrace}, a C function provided by the runtime\ignorespaces
          \onslide<6->{, with
            \begin{itemize}
              \item \funcarg{exn}: exception value
              \item \funcarg{pc}: code address of the raising point
              \item \funcarg{sp}: stack address of the raising function
              \item \funcarg{trapsp}: stack address of the next handler
            \end{itemize}
          }
      \end{itemize}
      \bigskip
      \mintocaml[firstline=9,lastline=13,highlightlines=12]{../src/backtrace/backtrace.ml}
    \end{column}
    \begin{column}{0.5\textwidth}
      \raggedleft
      \onslide*<1,3-4>{
        \mintc[firstline=190,lastline=192]{../ocaml/runtime/amd64.S}
        \mintc[firstline=234,lastline=237]{../ocaml/runtime/amd64.S}
      }
      \onslide*<2>{
        \mintc[firstline=190,lastline=192,highlightlines={191-192}]{../ocaml/runtime/amd64.S}
        \mintc[firstline=234,lastline=237,highlightlines={235-237}]{../ocaml/runtime/amd64.S}
      }
      \onslide*<1>{\listCamlRaiseExn[highlightlines=705]{}}%
      \onslide*<2>{\listCamlRaiseExn[highlightlines={707-708}]{}}%
      \onslide*<3>{\listCamlRaiseExn[highlightlines={712-714}]{}}%
      \onslide*<4>{\listCamlRaiseExn[highlightlines={715-720}]{}}%
      \onslide*<5>{\providecommand\step{1}\input{diagrams/backtrace/backtrace.tex}}%
      \onslide*<6>{\providecommand\step{2}\input{diagrams/backtrace/backtrace.tex}}%
    \end{column}
  \end{columns}
\end{frame}

\newcommand\listStashBacktrace[1][]{
  \listc[firstline=91,lastline=120,#1]{../ocaml/runtime/backtrace\_nat.c}{runtime/backtrace\_nat.c:91}}

\begin{frame}{Backtraces}{Introducing \funcname{caml\_stash\_backtrace}}
  \begin{columns}[c]
    \begin{column}{0.5\textwidth}
      \minipage[c][0.66\textheight][s]{\columnwidth}
      \begin{itemize}
        \item<1-> A C function provided by the runtime (specific to native code)
        \item<2-> It scans the stack, between the stack pointer at the raising point up to the next trap, in order to collect the names of the detected function frames
          \onslide*<2>{
            \begin{itemize}
              \item And more information, like line number, column number...
            \end{itemize}
          }
        \item<3-> It uses the same technique and information as the Garbage Collector (GC) uses to scan the values live on the stack
          \onslide*<4>{
            \begin{itemize}
              \item \typename{frame\_descr} are at the core of stack unwinding
            \end{itemize}
          }
      \end{itemize}
      \vfill
      \mintocaml[firstline=9,lastline=13,highlightlines=12]{../src/backtrace/backtrace.ml}
      \endminipage
    \end{column}
    \begin{column}{0.5\textwidth}
      \onslide*<1>{\listStashBacktrace[highlightlines=91]{}}
      \onslide*<2>{\listStashBacktrace[highlightlines={91,117-118}]{}}
      \onslide*<3->{\listStashBacktrace[highlightlines={94,106,109-110}]{}}
    \end{column}
  \end{columns}
\end{frame}

\newcommand\listFrameDescr[1][]{
  \listocaml[firstline=111,lastline=124,#1]{../ocaml/asmcomp/emitaux.ml}{asmcomp/emitaux.ml:111}}
\newcommand\listDebugInfo[1][]{
  \listocaml[firstline=38,lastline=49,#1]{../ocaml/lambda/debuginfo.mli}{lambda/debuginfo.mli:38}}

\begin{frame}{Backtraces}{\typename{frame\_descr} and \typename{frame\_debuginfo} in a nutshell}
  \begin{columns}[c]
    \begin{column}{0.5\textwidth}
      \begin{itemize}
        \item<1-> For every return address in an OCaml program, there is a associated \typename{frame\_descr}
        \item<2-> A \typename{frame\_descr} specifies
        \begin{itemize}
          \item<2-> The size of the function stack frame
          \item<3-> Where live values are on the stack (so that the GC can mark them as live and prevent from being swept)
        \end{itemize}
        \item<4-> When compiled with debug information, \typename{frame\_descr} will be linked to a \typename{frame\_debuginfo}
        \begin{itemize}
          \item<5-> Contains information about the position in the source code (file name, line and column number)
        \end{itemize}
      \end{itemize}
    \end{column}
    \begin{column}{0.5\textwidth}
        \onslide*<1>{\listFrameDescr[highlightlines={118-122}]{}}
        \onslide*<2>{\listFrameDescr[highlightlines={120}]{}}
        \onslide*<3>{\listFrameDescr[highlightlines={121}]{}}
        \onslide*<4->{\listFrameDescr[highlightlines={122}]{}}
        \medskip
        \onslide*<1-3>{\listDebugInfo{}}
        \onslide*<4>{\listDebugInfo[highlightlines={38,49}]{}}
        \onslide*<5>{\listDebugInfo[highlightlines={39-46}]{}}
    \end{column}
  \end{columns}
\end{frame}

\begin{frame}{Backtraces}{Scanning the stack with \typename{frame\_descr}}
  \begin{columns}[c]
    \begin{column}{0.5\textwidth}
      \begin{itemize}
        \item Given the return address of the code that raises the exception by calling \funcname{caml\_raise\_exn} (\funcarg{pc} argument of \funcname{caml\_stash\_backtrace})
        \item And the position of the stack pointer (\funcarg{sp} argument of \funcname{caml\_stash\_backtrace})
        \item The frame size given by \typename{frame\_descr}.\funcarg{frame\_size}, allows to find the end of the previous frame
        \item And the return address of the caller of this function
        \bigskip
        \onslide<3->{
        \item Note that traps (here the reddish \funcname{ExnA}, \funcname{ExnB} and \funcname{ExnC} blocks) belong to the frame that installed them, and so the frame size changes when traps are removed
        }
      \end{itemize}
    \end{column}
    \begin{column}{0.5\textwidth}
      \raggedleft
      \onslide*<1>{\providecommand\step{2}\input{diagrams/backtrace/backtrace.tex}}%
      \onslide*<2>{\providecommand\step{1}\input{diagrams/backtrace/scanstack.tex}}%
      \onslide*<3>{\providecommand\step{2}\input{diagrams/backtrace/scanstack.tex}}%
      \onslide*<4>{\providecommand\step{3}\input{diagrams/backtrace/scanstack.tex}}%
      \onslide*<5>{\providecommand\step{4}\input{diagrams/backtrace/scanstack.tex}}%
      \onslide*<6>{\providecommand\step{5}\input{diagrams/backtrace/scanstack.tex}}%
    \end{column}
  \end{columns}
\end{frame}

% Duplicate of previous-previous slide
\begin{frame}{Backtraces}{Continuing on \funcname{caml\_stash\_backtrace}}
  \begin{columns}[c]
    \begin{column}{0.5\textwidth}
      \minipage[c][0.75\textheight][s]{\columnwidth}
      \begin{itemize}
          \color{gray}
        \item A C function provided by the runtime (specific to native code)
        \item It scans the stack, between the stack pointer at the raising point up to the next trap, in order to collect the names of the detected function frames
        \item It uses the same technique and information as the Garbage Collector (GC) uses to scan the values live on the stack
      \end{itemize}
      \begin{itemize}
        \item For each function frame detected, store a pointer to the \typename{frame\_descr} in the \localname{domain\_state->backtrace\_buffer} array, effectively constructing the backtrace
      \end{itemize}
      \onslide<2->{
        \begin{itemize}
            \smallskip
          \item Constructed backtrace:
            \funcname{definitely\_raise},
            \onslide<3->{\funcname{probably\_raise}}
        \end{itemize}
      }
      \vfill
      \mintocaml[firstline=9,lastline=13,highlightlines=12]{../src/backtrace/backtrace.ml}
      \endminipage
    \end{column}
    \begin{column}{0.5\textwidth}
      \raggedleft
      \onslide*<1>{\listStashBacktrace[highlightlines={114-115}]{}}
      \onslide*<2>{\providecommand\step{3}\input{diagrams/backtrace/backtrace.tex}}%
      \onslide*<3>{\providecommand\step{4}\input{diagrams/backtrace/backtrace.tex}}%
    \end{column}
  \end{columns}
\end{frame}

\begin{frame}{Backtraces}{Back to \funcname{caml\_raise\_exn}}
  \begin{columns}[c]
    \begin{column}{0.5\textwidth}
      \minipage[c][0.66\textheight][s]{\columnwidth}
      \begin{itemize}\color{gray}
        \item Checks wheteher exceptions backtrace should be recorded, by checking the \funcarg{domain\_state->backtrace\_active} value
        \item Switches to the C stack to be able to execute C code
        \item Calls \funcname{caml\_stash\_backtrace}, to collect the backtrace up to the next trap
      \end{itemize}
      \onslide*<2->{
        \begin{itemize}
          \item<2-> Executes the exception handler in \funcname{probably\_raise} with \funcname{RESTORE\_EXN\_HANDLER\_OCAML}
            \begin{itemize}
              \item Set \regname{RSP} to \localname{domain\_state->exn\_handler} value
              \item Restore \localname{domain\_state->exn\_handler} from trap
              \item Jump to the handler code with \funcname{ret}
            \end{itemize}
        \end{itemize}
      }
      \vfill
      \onslide*<1>{\mintocaml[firstline=9,lastline=13,highlightlines=12]{../src/backtrace/backtrace.ml}}
      \onslide*<2->{\mintocaml[firstline=15,lastline=21,highlightlines={19-21}]{../src/backtrace/backtrace.ml}}
      \endminipage
    \end{column}
    \begin{column}{0.5\textwidth}
      \centering
      \onslide*<1>{
        \mintc[firstline=190,lastline=192]{../ocaml/runtime/amd64.S}
        \mintc[firstline=234,lastline=237]{../ocaml/runtime/amd64.S}
      }
      \onslide*<2>{
        \mintc[firstline=190,lastline=192,highlightlines={191-192}]{../ocaml/runtime/amd64.S}
        \mintc[firstline=234,lastline=237,highlightlines={235-237}]{../ocaml/runtime/amd64.S}
      }
      \onslide*<1>{\listCamlRaiseExn[highlightlines=720]{}}
      \onslide*<2>{\listCamlRaiseExn[highlightlines=722]{}}
      \onslide*<3->{\providecommand\step{5}\input{diagrams/backtrace/backtrace.tex}}
    \end{column}
  \end{columns}
\end{frame}

\begin{frame}{Backtraces}{\funcname{caml\_probably\_raise}'s handler doesn't catch \typename{ExnC}}
  \begin{columns}[c]
    \begin{column}{0.45\textwidth}
      \minipage[c][0.75\textheight][s]{\columnwidth}
      \begin{itemize}
        \item<1-> \funcname{match \dots with}\footnotemark pattern matching can also match exceptions
        \item<1-> The CMM of \funcname{probably\_raise} shows that this \funcname{match \dots with} is implemented with a pattern matching and a \funcname{try \dots with}
        \item<1-> But this one only matches on \typename{ExnA} exception
        \item<2-> Since the pattern matching fails to catch the exception, \funcname{reraise} is called with our current \typename{ExnC} exception
        \item<3-> Disassembly shows that \funcname{reraise} is actually a call to \funcname{caml\_reraise\_exn}, which is also an assembly function provided by the runtime
      \end{itemize}
      \vfill
      \mintocaml[firstline=15,lastline=21,highlightlines={18-21}]{../src/backtrace/backtrace.ml}
      \endminipage
    \end{column}
    \begin{column}{0.55\textwidth}
      \centering
      \onslide*<1>{\mintcmm[highlightlines={10-18}]{../src/backtrace/camlBacktrace\_\_probably\_raise\_351.cmm}}
      \onslide*<2>{\listcmm[highlightlines=18]{../src/backtrace/camlBacktrace\_\_probably\_raise_351.cmm}{CMM of probably\_raise}}
      \onslide*<3>{\mintobjdump[firstline=5,lastline=35,highlightlines=31]{../src/backtrace/camlBacktrace\_\_probably\_raise_351.objdump}}
    \end{column}
  \end{columns}
  \only<1>{\footnotetext[1]{https://blog.janestreet.com/pattern-matching-and-exception-handling-unite/}}
\end{frame}

\newcommand\listCamlReraiseExn[1][]{
  \listasm[firstline=727,lastline=734,#1]{../ocaml/runtime/amd64.S}{runtime/amd64.S:727}}

\begin{frame}{Backtraces}{Introducing \funcname{caml\_reraise\_exn}}
  \begin{columns}[c]
    \begin{column}{0.5\textwidth}
      \minipage[c][0.66\textheight][s]{\columnwidth}
      \begin{itemize}
        \item<1-> The exception have to be reraised to the next handler
        \item<2-> First, checks if the backtrace should be collected with \funcarg{domain\_state->backtrace\_active}
        \only<3>{
        \begin{itemize}
            \item If not, behaves like a \funcname{raise\_notrace} with \funcname{RESTORE\_EXN\_HANDLER\_OCAML}
        \end{itemize}
        }
        \item<4-> Jumps into \funcname{caml\_raise\_exn}
        \item<5-> Calls \funcname{caml\_stash\_backtrace} again, to scan the next part of the stack: from the trap just pop-ed to the next trap
      \end{itemize}
      \vfill
      \mintocaml[firstline=15,lastline=21,highlightlines={18-21}]{../src/backtrace/backtrace.ml}
      \endminipage
    \end{column}
    \begin{column}{0.5\textwidth}
      \raggedleft
      \onslide*<1>{\listCamlReraiseExn{}}
      \onslide*<2>{\listCamlReraiseExn[highlightlines=729]{}}
      \onslide*<3>{
        \mintc[firstline=190,lastline=192,highlightlines={191-192}]{../ocaml/runtime/amd64.S}
        \mintc[firstline=234,lastline=237,highlightlines={235-237}]{../ocaml/runtime/amd64.S}
        \medskip
        \listCamlReraiseExn[highlightlines=731]{}
      }
      \onslide*<4-5>{
        % TODO refactor with \listCamlReraiseExn
        \mintasm[firstline=727,lastline=734,highlightlines=730]{../ocaml/runtime/amd64.S}
        \medskip
      }
      \onslide*<4>{\listCamlRaiseExn[highlightlines=711]{}}
      \onslide*<5>{\listCamlRaiseExn[highlightlines=720]{}}
    \end{column}
  \end{columns}
\end{frame}

\begin{frame}{Backtraces}{Backtrace collection continues}
  \begin{columns}[c]
    \begin{column}{0.55\textwidth}
      \minipage[c][0.75\textheight][s]{\columnwidth}
      \begin{itemize}
        \item<1-> \funcname{caml\_stash\_backtrace} scans the older part of the stack, up to the next trap
        \item<6-> Controls jumps to the nested exception handler in \funcname{maybe\_raise}, which matches only on \typename{ExnB}
        \item<7-> \funcname{caml\_reraise\_exn} is called again, backtrace collection resumes with \funcname{caml\_stash\_backtrace}
      \end{itemize}
      \medskip
      \begin{itemize}
        \item<2-> Constructed backtrace:
        \begin{itemize}
          \item <2-> \funcname{definitely\_raise}
          \item <2-> \funcname{probably\_raise}
          \item <3-> \funcname{probably\_raise} (re-raise)
          \item <4-> \funcname{maybe\_raise}
          \item <5-> \funcname{main}
          \item <8-> \funcname{main} (re-raise)
        \end{itemize}
      \end{itemize}
      \vfill
      \onslide*<1-5>{\mintocaml[firstline=15,lastline=21]{../src/backtrace/backtrace.ml}}
      \onslide*<6>{\mintocaml[firstline=28,lastline=34,highlightlines=31]{../src/backtrace/backtrace.ml}}
      \onslide*<7->{\mintocaml[firstline=28,lastline=34,highlightlines=33]{../src/backtrace/backtrace.ml}}
      \endminipage
    \end{column}
    \begin{column}{0.45\textwidth}
      \raggedleft
      \onslide*<1>{\providecommand\step{6}\input{diagrams/backtrace/backtrace.tex}}%
      \onslide*<2>{\providecommand\step{6}\input{diagrams/backtrace/backtrace.tex}}%
      \onslide*<3>{\providecommand\step{7}\input{diagrams/backtrace/backtrace.tex}}%
      \onslide*<4>{\providecommand\step{8}\input{diagrams/backtrace/backtrace.tex}}%
      \onslide*<5>{\providecommand\step{9}\input{diagrams/backtrace/backtrace.tex}}%
      \onslide*<6>{\providecommand\step{10}\input{diagrams/backtrace/backtrace.tex}}%
      \onslide*<7>{\providecommand\step{11}\input{diagrams/backtrace/backtrace.tex}}%
      \onslide*<8>{\providecommand\step{12}\input{diagrams/backtrace/backtrace.tex}}%
      \onslide*<9>{\providecommand\step{13}\input{diagrams/backtrace/backtrace.tex}}%
    \end{column}
  \end{columns}
\end{frame}

\begin{frame}{Backtraces}{Printing the backtrace}
  \begin{columns}[c]
    \begin{column}{0.45\textwidth}
      \minipage[c][0.66\textheight][s]{\columnwidth}
      \begin{itemize}
        \item<1-> The exception backtrace can now be printed to \funcarg{stdout} with \funcname{Printexc.print_backtrace}
        \item<2-> First the exception backtrace is retrieved into an OCaml array with \funcname{get_raw_backtrace }
        \item<3-> Which is implemented as the \funcname{caml_get_exception_raw_backtrace} C function in the runtime
        \item<4-> And finally printed with \funcname{print_exception_backtrace}
      \end{itemize}
      \vfill
      \mintocaml[firstline=28,lastline=34,highlightlines=33]{../src/backtrace/backtrace.ml}
      \endminipage
    \end{column}
    \begin{column}{0.55\textwidth}
      \onslide*<1,2>{
        \mintocaml[firstline=100,lastline=101]{../ocaml/stdlib/printexc.ml}
      }
      \only<1>{
        \listocaml[firstline=170,lastline=172]{../ocaml/stdlib/printexc.ml}{stdlib/printexc.ml:170}
      }
      \only<2>{
        \listocaml[firstline=170,lastline=172,highlightlines=172]{../ocaml/stdlib/printexc.ml}{stdlib/printexc.ml:170}
      }
      \only<3>{
        \mintc[firstline=154,lastline=156]{../ocaml/runtime/backtrace.c}
        \listc[firstline=171,lastline=192,highlightlines={184-187}]{../ocaml/runtime/backtrace.c}{runtime/backtrace.c:154}
      }
      \only<4>{
        \listocaml[firstline=155,lastline=165]{../ocaml/stdlib/printexc.ml}{stdlib/printexc.ml:55}
      }
    \end{column}
  \end{columns}
\end{frame}


\subsection{The multiple kinds of raise}
\frameSubsection{}{}

\begin{frame}{Backtraces}{When backtraces are not needed}
  \begin{columns}[c]
    \begin{column}{0.45\textwidth}
      \minipage[c][0.66\textheight][s]{\columnwidth}
      \begin{itemize}
        \item<1-> What if the backtrace for an exception is never needed?
        \item<2-> It's possible to raise the exception explicitly with \funcname{raise\_notrace} to make raising as cheap as possible
        \item<3-> An example of use in the \funcname{dynlink} library of the OCaml compiler
      \end{itemize}
      \vfill
      \onslide<3->{
      \listocaml[firstline=205,lastline=209,highlightlines=207]{../ocaml/otherlibs/dynlink/dynlink\_compilerlibs/misc.ml}{otherlibs/dynlink/dynlink\_compilerlibs/misc.ml:205}
      }
      \endminipage
    \end{column}
    \begin{column}{0.55\textwidth}
    \only<4>{
      \mintcmm[highlightlines=3]{../src/notrace/camlNotrace\_\_fun_347.cmm}
      \listcmm{../src/notrace/camlNotrace\_\_all\_somes\_327.cmm}{all\_somes}
    }
    \onslide*<5->{
      \listobjdump[highlightlines={6-9}]{../src/notrace/camlNotrace\_\_fun_347.objdump}{anonymous function from \funcname{all\_somes}}
    }
    \end{column}
  \end{columns}
\end{frame}

\begin{frame}{Backtraces}{Summary table of the multiple kinds of raise}
  \begin{itemize}
    \item When debugging information is enabled and if the backtrace of an exception isn't needed, it's possible to raise with \funcname{raise\_notrace} for performance
    \item When debugging information is \emph{not} enabled, the compiler optimises \funcname{raise} calls to inline assembly (same effect as using \funcname{raise\_notrace})
  \end{itemize}
  \bigskip
  \centering
  \begin{tabular}{| l | c | c | c | c |}
    \hline
    \parbox{10em}{Debug information \\ (\funcarg{-g} flag)}  & \multicolumn{2}{|c|}{Disabled}                                     & \multicolumn{2}{|c|}{Enabled} \\
    \hline
    Raise kind                                               & \funcname{raise} & \funcname{raise\_notrace}                       & \funcname{raise} & \funcname{raise\_notrace} \\
    \hline
    Compilation                                              & \parbox{8em}{inline assembly\\ (optimization)} & inline assembly   & \funcname{caml\_raise\_exn} & inline assembly \\
    \hline
  \end{tabular}
\end{frame}


%
% Takeaway #5
%
\subsection*{Takeaway \#5}
\frameSubsectionTakeaway{}
\againframe<5>{takeaway}
}%
    \end{column}
  \end{columns}
\end{frame}

\newcommand\listStashBacktrace[1][]{
  \listc[firstline=91,lastline=120,#1]{../ocaml/runtime/backtrace\_nat.c}{runtime/backtrace\_nat.c:91}}

\begin{frame}{Backtraces}{Introducing \funcname{caml\_stash\_backtrace}}
  \begin{columns}[c]
    \begin{column}{0.5\textwidth}
      \minipage[c][0.66\textheight][s]{\columnwidth}
      \begin{itemize}
        \item<1-> A C function provided by the runtime (specific to native code)
        \item<2-> It scans the stack, between the stack pointer at the raising point up to the next trap, in order to collect the names of the detected function frames
          \onslide*<2>{
            \begin{itemize}
              \item And more information, like line number, column number...
            \end{itemize}
          }
        \item<3-> It uses the same technique and information as the Garbage Collector (GC) uses to scan the values live on the stack
          \onslide*<4>{
            \begin{itemize}
              \item \typename{frame\_descr} are at the core of stack unwinding
            \end{itemize}
          }
      \end{itemize}
      \vfill
      \mintocaml[firstline=9,lastline=13,highlightlines=12]{../src/backtrace/backtrace.ml}
      \endminipage
    \end{column}
    \begin{column}{0.5\textwidth}
      \onslide*<1>{\listStashBacktrace[highlightlines=91]{}}
      \onslide*<2>{\listStashBacktrace[highlightlines={91,117-118}]{}}
      \onslide*<3->{\listStashBacktrace[highlightlines={94,106,109-110}]{}}
    \end{column}
  \end{columns}
\end{frame}

\newcommand\listFrameDescr[1][]{
  \listocaml[firstline=111,lastline=124,#1]{../ocaml/asmcomp/emitaux.ml}{asmcomp/emitaux.ml:111}}
\newcommand\listDebugInfo[1][]{
  \listocaml[firstline=38,lastline=49,#1]{../ocaml/lambda/debuginfo.mli}{lambda/debuginfo.mli:38}}

\begin{frame}{Backtraces}{\typename{frame\_descr} and \typename{frame\_debuginfo} in a nutshell}
  \begin{columns}[c]
    \begin{column}{0.5\textwidth}
      \begin{itemize}
        \item<1-> For every return address in an OCaml program, there is a associated \typename{frame\_descr}
        \item<2-> A \typename{frame\_descr} specifies
        \begin{itemize}
          \item<2-> The size of the function stack frame
          \item<3-> Where live values are on the stack (so that the GC can mark them as live and prevent from being swept)
        \end{itemize}
        \item<4-> When compiled with debug information, \typename{frame\_descr} will be linked to a \typename{frame\_debuginfo}
        \begin{itemize}
          \item<5-> Contains information about the position in the source code (file name, line and column number)
        \end{itemize}
      \end{itemize}
    \end{column}
    \begin{column}{0.5\textwidth}
        \onslide*<1>{\listFrameDescr[highlightlines={118-122}]{}}
        \onslide*<2>{\listFrameDescr[highlightlines={120}]{}}
        \onslide*<3>{\listFrameDescr[highlightlines={121}]{}}
        \onslide*<4->{\listFrameDescr[highlightlines={122}]{}}
        \medskip
        \onslide*<1-3>{\listDebugInfo{}}
        \onslide*<4>{\listDebugInfo[highlightlines={38,49}]{}}
        \onslide*<5>{\listDebugInfo[highlightlines={39-46}]{}}
    \end{column}
  \end{columns}
\end{frame}

\begin{frame}{Backtraces}{Scanning the stack with \typename{frame\_descr}}
  \begin{columns}[c]
    \begin{column}{0.5\textwidth}
      \begin{itemize}
        \item Given the return address of the code that raises the exception by calling \funcname{caml\_raise\_exn} (\funcarg{pc} argument of \funcname{caml\_stash\_backtrace})
        \item And the position of the stack pointer (\funcarg{sp} argument of \funcname{caml\_stash\_backtrace})
        \item The frame size given by \typename{frame\_descr}.\funcarg{frame\_size}, allows to find the end of the previous frame
        \item And the return address of the caller of this function
        \bigskip
        \onslide<3->{
        \item Note that traps (here the reddish \funcname{ExnA}, \funcname{ExnB} and \funcname{ExnC} blocks) belong to the frame that installed them, and so the frame size changes when traps are removed
        }
      \end{itemize}
    \end{column}
    \begin{column}{0.5\textwidth}
      \raggedleft
      \onslide*<1>{\providecommand\step{2}%
% Backtraces
%
\subsection{One advantage of raising with debugging information}
\frameSubsection{}{}

\begin{frame}{Backtraces}{Debugging information \& source code}
  \begin{columns}[c]
    \begin{column}{0.5\textwidth}
      \begin{itemize}
        \item Raising an exception is fun and games, but how do we know where the exception originated? $\rightarrow$ With the backtrace associated with the exception!
        \item In order to get a backtrace from the point where an exception was raised, it's necessary to compile with debugging information enabled (\funcarg{-g} flag for the compiler)
        \item Same example as in the `Nested handlers' section
      \end{itemize}
    \end{column}
    \begin{column}{0.5\textwidth}
      \listocaml[firstline=9,lastline=34]{../src/backtrace/backtrace.ml}{backtrace/backtrace.ml}
    \end{column}
  \end{columns}
\end{frame}

\begin{frame}{Backtraces}{CMM and disassembly of \funcname{definitely\_raise}}
  \begin{columns}[c]
    \begin{column}{0.5\textwidth}
      \minipage[c][0.66\textheight][s]{\columnwidth}
      \begin{itemize}
        \item<1-> An effect of compiling with debugging information (\funcarg{-g}): the CMM no longer uses \funcname{raise\_notrace} to raise the exception but \funcname{raise} instead
        \item<2-> Disassembly shows that CMM \funcname{raise} is actually a call to \funcname{caml\_raise\_exn}, a function in assembler provided by the runtime
      \end{itemize}
      \vfill
      \mintocaml[firstline=9,lastline=13,highlightlines=12]{../src/backtrace/backtrace.ml}
      \endminipage
    \end{column}
    \begin{column}{0.5\textwidth}
      \onslide*<1>{
        \mintcmm[highlightlines=5]{../src/backtrace/camlBacktrace\_\_definitely\_raise\_347.cmm}
      }
      \onslide*<2>{
        \mintobjdump[firstline=2,highlightlines=14]{../src/backtrace/camlBacktrace\_\_definitely\_raise\_347.objdump}
      }
    \end{column}
  \end{columns}
\end{frame}

\begin{frame}{Backtraces}{Introducing \funcname{caml\_raise\_exn}}
  \begin{columns}[c]
    \begin{column}{0.5\textwidth}
      \minipage[c][0.66\textheight][s]{\columnwidth}
      \onslide*<1>{
        \begin{itemize}
          \item Raising the exception is performed using \funcname{caml\_raise\_exn} which is
            \begin{itemize}
              \item An assembly function provided by the runtime
              \item Architecture specific
            \end{itemize}
          \item Let's see what it does differently from \funcname{raise\_notrace}\dots
          \item We are about to raise \typename{ExnC} with \funcname{caml\_raise\_exn} from \funcname{definitely\_raise}
        \end{itemize}
      }
      \onslide*<2>{
        \mintobjdump[firstline=2,highlightlines=14]{../src/backtrace/camlBacktrace\_\_definitely\_raise\_347.objdump}
      }
      \vfill
      \mintocaml[firstline=9,lastline=13,highlightlines=12]{../src/backtrace/backtrace.ml}
      \endminipage
    \end{column}
    \begin{column}{0.5\textwidth}
      \centering
      \providecommand\step{1}\input{diagrams/backtrace/backtrace.tex}
    \end{column}
  \end{columns}
\end{frame}

\begin{frame}{Backtraces}{But before that, we need more fields from \typename{caml\_domain\_state}}
  \begin{columns}
    \begin{column}{0.5\textwidth}
      \begin{itemize}
        \item Remember \typename{caml\_domain\_state} the per-domain runtime state?
        \item Some other members are of interest to us now\dots
        \item \localname{backtrace\_active} is used to know if the runtime should collect backtraces
        \item \localname{backtrace\_active} can be set by
          \begin{itemize}
            \item The \funcarg{b} flag in the \funcname{OCAMLRUNPARAM} environment variable
            \item \mintinline{ocaml}{Printexc.record_backtrace : bool -> unit} \footnotemark
          \end{itemize}
      \end{itemize}
    \end{column}
    \begin{column}{0.5\textwidth}
      \begin{listing}
        \mintc[highlightlines={9-11}]{src/ocaml/domain\_state\_backtrace.h}
        \caption{runtime/caml/domain\_state.h}
      \end{listing}
    \end{column}
  \end{columns}
  \footnotetext[1]{https://v2.ocaml.org/api/Printexc.html}
\end{frame}

\newcommand\listCamlRaiseExn[1][]{
  \listasm[firstline=702,lastline=725,#1]{../ocaml/runtime/amd64.S}{runtime/amd64.S:702}}

\begin{frame}{Backtraces}{Walk through \funcname{caml\_raise\_exn}}
  \begin{columns}[T]
    \begin{column}{0.5\textwidth}
      \begin{itemize}
        \item<1-> First checks whether exceptions backtrace should be recorded
        \item<1-> By checking the \funcarg{domain\_state->backtrace\_active} value
        \item<2-> If not, acts as \funcname{raise\_notrace} with \funcname{RESTORE\_EXN\_HANDLER\_OCAML}
          \begin{itemize}
            \item Set \regname{RSP} to \localname{domain\_state->exn\_handler} value
            \item Restore \localname{domain\_state->exn\_handler} from trap
            \item Jump with \funcname{ret} (same effect as using \funcname{pop} and \funcname{jmp})
          \end{itemize}
        \item<3-> Otherwise, switches to the C stack to be able to execute C code
        \item<4-> Calls \funcname{caml\_stash\_backtrace}, a C function provided by the runtime\ignorespaces
          \onslide<6->{, with
            \begin{itemize}
              \item \funcarg{exn}: exception value
              \item \funcarg{pc}: code address of the raising point
              \item \funcarg{sp}: stack address of the raising function
              \item \funcarg{trapsp}: stack address of the next handler
            \end{itemize}
          }
      \end{itemize}
      \bigskip
      \mintocaml[firstline=9,lastline=13,highlightlines=12]{../src/backtrace/backtrace.ml}
    \end{column}
    \begin{column}{0.5\textwidth}
      \raggedleft
      \onslide*<1,3-4>{
        \mintc[firstline=190,lastline=192]{../ocaml/runtime/amd64.S}
        \mintc[firstline=234,lastline=237]{../ocaml/runtime/amd64.S}
      }
      \onslide*<2>{
        \mintc[firstline=190,lastline=192,highlightlines={191-192}]{../ocaml/runtime/amd64.S}
        \mintc[firstline=234,lastline=237,highlightlines={235-237}]{../ocaml/runtime/amd64.S}
      }
      \onslide*<1>{\listCamlRaiseExn[highlightlines=705]{}}%
      \onslide*<2>{\listCamlRaiseExn[highlightlines={707-708}]{}}%
      \onslide*<3>{\listCamlRaiseExn[highlightlines={712-714}]{}}%
      \onslide*<4>{\listCamlRaiseExn[highlightlines={715-720}]{}}%
      \onslide*<5>{\providecommand\step{1}\input{diagrams/backtrace/backtrace.tex}}%
      \onslide*<6>{\providecommand\step{2}\input{diagrams/backtrace/backtrace.tex}}%
    \end{column}
  \end{columns}
\end{frame}

\newcommand\listStashBacktrace[1][]{
  \listc[firstline=91,lastline=120,#1]{../ocaml/runtime/backtrace\_nat.c}{runtime/backtrace\_nat.c:91}}

\begin{frame}{Backtraces}{Introducing \funcname{caml\_stash\_backtrace}}
  \begin{columns}[c]
    \begin{column}{0.5\textwidth}
      \minipage[c][0.66\textheight][s]{\columnwidth}
      \begin{itemize}
        \item<1-> A C function provided by the runtime (specific to native code)
        \item<2-> It scans the stack, between the stack pointer at the raising point up to the next trap, in order to collect the names of the detected function frames
          \onslide*<2>{
            \begin{itemize}
              \item And more information, like line number, column number...
            \end{itemize}
          }
        \item<3-> It uses the same technique and information as the Garbage Collector (GC) uses to scan the values live on the stack
          \onslide*<4>{
            \begin{itemize}
              \item \typename{frame\_descr} are at the core of stack unwinding
            \end{itemize}
          }
      \end{itemize}
      \vfill
      \mintocaml[firstline=9,lastline=13,highlightlines=12]{../src/backtrace/backtrace.ml}
      \endminipage
    \end{column}
    \begin{column}{0.5\textwidth}
      \onslide*<1>{\listStashBacktrace[highlightlines=91]{}}
      \onslide*<2>{\listStashBacktrace[highlightlines={91,117-118}]{}}
      \onslide*<3->{\listStashBacktrace[highlightlines={94,106,109-110}]{}}
    \end{column}
  \end{columns}
\end{frame}

\newcommand\listFrameDescr[1][]{
  \listocaml[firstline=111,lastline=124,#1]{../ocaml/asmcomp/emitaux.ml}{asmcomp/emitaux.ml:111}}
\newcommand\listDebugInfo[1][]{
  \listocaml[firstline=38,lastline=49,#1]{../ocaml/lambda/debuginfo.mli}{lambda/debuginfo.mli:38}}

\begin{frame}{Backtraces}{\typename{frame\_descr} and \typename{frame\_debuginfo} in a nutshell}
  \begin{columns}[c]
    \begin{column}{0.5\textwidth}
      \begin{itemize}
        \item<1-> For every return address in an OCaml program, there is a associated \typename{frame\_descr}
        \item<2-> A \typename{frame\_descr} specifies
        \begin{itemize}
          \item<2-> The size of the function stack frame
          \item<3-> Where live values are on the stack (so that the GC can mark them as live and prevent from being swept)
        \end{itemize}
        \item<4-> When compiled with debug information, \typename{frame\_descr} will be linked to a \typename{frame\_debuginfo}
        \begin{itemize}
          \item<5-> Contains information about the position in the source code (file name, line and column number)
        \end{itemize}
      \end{itemize}
    \end{column}
    \begin{column}{0.5\textwidth}
        \onslide*<1>{\listFrameDescr[highlightlines={118-122}]{}}
        \onslide*<2>{\listFrameDescr[highlightlines={120}]{}}
        \onslide*<3>{\listFrameDescr[highlightlines={121}]{}}
        \onslide*<4->{\listFrameDescr[highlightlines={122}]{}}
        \medskip
        \onslide*<1-3>{\listDebugInfo{}}
        \onslide*<4>{\listDebugInfo[highlightlines={38,49}]{}}
        \onslide*<5>{\listDebugInfo[highlightlines={39-46}]{}}
    \end{column}
  \end{columns}
\end{frame}

\begin{frame}{Backtraces}{Scanning the stack with \typename{frame\_descr}}
  \begin{columns}[c]
    \begin{column}{0.5\textwidth}
      \begin{itemize}
        \item Given the return address of the code that raises the exception by calling \funcname{caml\_raise\_exn} (\funcarg{pc} argument of \funcname{caml\_stash\_backtrace})
        \item And the position of the stack pointer (\funcarg{sp} argument of \funcname{caml\_stash\_backtrace})
        \item The frame size given by \typename{frame\_descr}.\funcarg{frame\_size}, allows to find the end of the previous frame
        \item And the return address of the caller of this function
        \bigskip
        \onslide<3->{
        \item Note that traps (here the reddish \funcname{ExnA}, \funcname{ExnB} and \funcname{ExnC} blocks) belong to the frame that installed them, and so the frame size changes when traps are removed
        }
      \end{itemize}
    \end{column}
    \begin{column}{0.5\textwidth}
      \raggedleft
      \onslide*<1>{\providecommand\step{2}\input{diagrams/backtrace/backtrace.tex}}%
      \onslide*<2>{\providecommand\step{1}\input{diagrams/backtrace/scanstack.tex}}%
      \onslide*<3>{\providecommand\step{2}\input{diagrams/backtrace/scanstack.tex}}%
      \onslide*<4>{\providecommand\step{3}\input{diagrams/backtrace/scanstack.tex}}%
      \onslide*<5>{\providecommand\step{4}\input{diagrams/backtrace/scanstack.tex}}%
      \onslide*<6>{\providecommand\step{5}\input{diagrams/backtrace/scanstack.tex}}%
    \end{column}
  \end{columns}
\end{frame}

% Duplicate of previous-previous slide
\begin{frame}{Backtraces}{Continuing on \funcname{caml\_stash\_backtrace}}
  \begin{columns}[c]
    \begin{column}{0.5\textwidth}
      \minipage[c][0.75\textheight][s]{\columnwidth}
      \begin{itemize}
          \color{gray}
        \item A C function provided by the runtime (specific to native code)
        \item It scans the stack, between the stack pointer at the raising point up to the next trap, in order to collect the names of the detected function frames
        \item It uses the same technique and information as the Garbage Collector (GC) uses to scan the values live on the stack
      \end{itemize}
      \begin{itemize}
        \item For each function frame detected, store a pointer to the \typename{frame\_descr} in the \localname{domain\_state->backtrace\_buffer} array, effectively constructing the backtrace
      \end{itemize}
      \onslide<2->{
        \begin{itemize}
            \smallskip
          \item Constructed backtrace:
            \funcname{definitely\_raise},
            \onslide<3->{\funcname{probably\_raise}}
        \end{itemize}
      }
      \vfill
      \mintocaml[firstline=9,lastline=13,highlightlines=12]{../src/backtrace/backtrace.ml}
      \endminipage
    \end{column}
    \begin{column}{0.5\textwidth}
      \raggedleft
      \onslide*<1>{\listStashBacktrace[highlightlines={114-115}]{}}
      \onslide*<2>{\providecommand\step{3}\input{diagrams/backtrace/backtrace.tex}}%
      \onslide*<3>{\providecommand\step{4}\input{diagrams/backtrace/backtrace.tex}}%
    \end{column}
  \end{columns}
\end{frame}

\begin{frame}{Backtraces}{Back to \funcname{caml\_raise\_exn}}
  \begin{columns}[c]
    \begin{column}{0.5\textwidth}
      \minipage[c][0.66\textheight][s]{\columnwidth}
      \begin{itemize}\color{gray}
        \item Checks wheteher exceptions backtrace should be recorded, by checking the \funcarg{domain\_state->backtrace\_active} value
        \item Switches to the C stack to be able to execute C code
        \item Calls \funcname{caml\_stash\_backtrace}, to collect the backtrace up to the next trap
      \end{itemize}
      \onslide*<2->{
        \begin{itemize}
          \item<2-> Executes the exception handler in \funcname{probably\_raise} with \funcname{RESTORE\_EXN\_HANDLER\_OCAML}
            \begin{itemize}
              \item Set \regname{RSP} to \localname{domain\_state->exn\_handler} value
              \item Restore \localname{domain\_state->exn\_handler} from trap
              \item Jump to the handler code with \funcname{ret}
            \end{itemize}
        \end{itemize}
      }
      \vfill
      \onslide*<1>{\mintocaml[firstline=9,lastline=13,highlightlines=12]{../src/backtrace/backtrace.ml}}
      \onslide*<2->{\mintocaml[firstline=15,lastline=21,highlightlines={19-21}]{../src/backtrace/backtrace.ml}}
      \endminipage
    \end{column}
    \begin{column}{0.5\textwidth}
      \centering
      \onslide*<1>{
        \mintc[firstline=190,lastline=192]{../ocaml/runtime/amd64.S}
        \mintc[firstline=234,lastline=237]{../ocaml/runtime/amd64.S}
      }
      \onslide*<2>{
        \mintc[firstline=190,lastline=192,highlightlines={191-192}]{../ocaml/runtime/amd64.S}
        \mintc[firstline=234,lastline=237,highlightlines={235-237}]{../ocaml/runtime/amd64.S}
      }
      \onslide*<1>{\listCamlRaiseExn[highlightlines=720]{}}
      \onslide*<2>{\listCamlRaiseExn[highlightlines=722]{}}
      \onslide*<3->{\providecommand\step{5}\input{diagrams/backtrace/backtrace.tex}}
    \end{column}
  \end{columns}
\end{frame}

\begin{frame}{Backtraces}{\funcname{caml\_probably\_raise}'s handler doesn't catch \typename{ExnC}}
  \begin{columns}[c]
    \begin{column}{0.45\textwidth}
      \minipage[c][0.75\textheight][s]{\columnwidth}
      \begin{itemize}
        \item<1-> \funcname{match \dots with}\footnotemark pattern matching can also match exceptions
        \item<1-> The CMM of \funcname{probably\_raise} shows that this \funcname{match \dots with} is implemented with a pattern matching and a \funcname{try \dots with}
        \item<1-> But this one only matches on \typename{ExnA} exception
        \item<2-> Since the pattern matching fails to catch the exception, \funcname{reraise} is called with our current \typename{ExnC} exception
        \item<3-> Disassembly shows that \funcname{reraise} is actually a call to \funcname{caml\_reraise\_exn}, which is also an assembly function provided by the runtime
      \end{itemize}
      \vfill
      \mintocaml[firstline=15,lastline=21,highlightlines={18-21}]{../src/backtrace/backtrace.ml}
      \endminipage
    \end{column}
    \begin{column}{0.55\textwidth}
      \centering
      \onslide*<1>{\mintcmm[highlightlines={10-18}]{../src/backtrace/camlBacktrace\_\_probably\_raise\_351.cmm}}
      \onslide*<2>{\listcmm[highlightlines=18]{../src/backtrace/camlBacktrace\_\_probably\_raise_351.cmm}{CMM of probably\_raise}}
      \onslide*<3>{\mintobjdump[firstline=5,lastline=35,highlightlines=31]{../src/backtrace/camlBacktrace\_\_probably\_raise_351.objdump}}
    \end{column}
  \end{columns}
  \only<1>{\footnotetext[1]{https://blog.janestreet.com/pattern-matching-and-exception-handling-unite/}}
\end{frame}

\newcommand\listCamlReraiseExn[1][]{
  \listasm[firstline=727,lastline=734,#1]{../ocaml/runtime/amd64.S}{runtime/amd64.S:727}}

\begin{frame}{Backtraces}{Introducing \funcname{caml\_reraise\_exn}}
  \begin{columns}[c]
    \begin{column}{0.5\textwidth}
      \minipage[c][0.66\textheight][s]{\columnwidth}
      \begin{itemize}
        \item<1-> The exception have to be reraised to the next handler
        \item<2-> First, checks if the backtrace should be collected with \funcarg{domain\_state->backtrace\_active}
        \only<3>{
        \begin{itemize}
            \item If not, behaves like a \funcname{raise\_notrace} with \funcname{RESTORE\_EXN\_HANDLER\_OCAML}
        \end{itemize}
        }
        \item<4-> Jumps into \funcname{caml\_raise\_exn}
        \item<5-> Calls \funcname{caml\_stash\_backtrace} again, to scan the next part of the stack: from the trap just pop-ed to the next trap
      \end{itemize}
      \vfill
      \mintocaml[firstline=15,lastline=21,highlightlines={18-21}]{../src/backtrace/backtrace.ml}
      \endminipage
    \end{column}
    \begin{column}{0.5\textwidth}
      \raggedleft
      \onslide*<1>{\listCamlReraiseExn{}}
      \onslide*<2>{\listCamlReraiseExn[highlightlines=729]{}}
      \onslide*<3>{
        \mintc[firstline=190,lastline=192,highlightlines={191-192}]{../ocaml/runtime/amd64.S}
        \mintc[firstline=234,lastline=237,highlightlines={235-237}]{../ocaml/runtime/amd64.S}
        \medskip
        \listCamlReraiseExn[highlightlines=731]{}
      }
      \onslide*<4-5>{
        % TODO refactor with \listCamlReraiseExn
        \mintasm[firstline=727,lastline=734,highlightlines=730]{../ocaml/runtime/amd64.S}
        \medskip
      }
      \onslide*<4>{\listCamlRaiseExn[highlightlines=711]{}}
      \onslide*<5>{\listCamlRaiseExn[highlightlines=720]{}}
    \end{column}
  \end{columns}
\end{frame}

\begin{frame}{Backtraces}{Backtrace collection continues}
  \begin{columns}[c]
    \begin{column}{0.55\textwidth}
      \minipage[c][0.75\textheight][s]{\columnwidth}
      \begin{itemize}
        \item<1-> \funcname{caml\_stash\_backtrace} scans the older part of the stack, up to the next trap
        \item<6-> Controls jumps to the nested exception handler in \funcname{maybe\_raise}, which matches only on \typename{ExnB}
        \item<7-> \funcname{caml\_reraise\_exn} is called again, backtrace collection resumes with \funcname{caml\_stash\_backtrace}
      \end{itemize}
      \medskip
      \begin{itemize}
        \item<2-> Constructed backtrace:
        \begin{itemize}
          \item <2-> \funcname{definitely\_raise}
          \item <2-> \funcname{probably\_raise}
          \item <3-> \funcname{probably\_raise} (re-raise)
          \item <4-> \funcname{maybe\_raise}
          \item <5-> \funcname{main}
          \item <8-> \funcname{main} (re-raise)
        \end{itemize}
      \end{itemize}
      \vfill
      \onslide*<1-5>{\mintocaml[firstline=15,lastline=21]{../src/backtrace/backtrace.ml}}
      \onslide*<6>{\mintocaml[firstline=28,lastline=34,highlightlines=31]{../src/backtrace/backtrace.ml}}
      \onslide*<7->{\mintocaml[firstline=28,lastline=34,highlightlines=33]{../src/backtrace/backtrace.ml}}
      \endminipage
    \end{column}
    \begin{column}{0.45\textwidth}
      \raggedleft
      \onslide*<1>{\providecommand\step{6}\input{diagrams/backtrace/backtrace.tex}}%
      \onslide*<2>{\providecommand\step{6}\input{diagrams/backtrace/backtrace.tex}}%
      \onslide*<3>{\providecommand\step{7}\input{diagrams/backtrace/backtrace.tex}}%
      \onslide*<4>{\providecommand\step{8}\input{diagrams/backtrace/backtrace.tex}}%
      \onslide*<5>{\providecommand\step{9}\input{diagrams/backtrace/backtrace.tex}}%
      \onslide*<6>{\providecommand\step{10}\input{diagrams/backtrace/backtrace.tex}}%
      \onslide*<7>{\providecommand\step{11}\input{diagrams/backtrace/backtrace.tex}}%
      \onslide*<8>{\providecommand\step{12}\input{diagrams/backtrace/backtrace.tex}}%
      \onslide*<9>{\providecommand\step{13}\input{diagrams/backtrace/backtrace.tex}}%
    \end{column}
  \end{columns}
\end{frame}

\begin{frame}{Backtraces}{Printing the backtrace}
  \begin{columns}[c]
    \begin{column}{0.45\textwidth}
      \minipage[c][0.66\textheight][s]{\columnwidth}
      \begin{itemize}
        \item<1-> The exception backtrace can now be printed to \funcarg{stdout} with \funcname{Printexc.print_backtrace}
        \item<2-> First the exception backtrace is retrieved into an OCaml array with \funcname{get_raw_backtrace }
        \item<3-> Which is implemented as the \funcname{caml_get_exception_raw_backtrace} C function in the runtime
        \item<4-> And finally printed with \funcname{print_exception_backtrace}
      \end{itemize}
      \vfill
      \mintocaml[firstline=28,lastline=34,highlightlines=33]{../src/backtrace/backtrace.ml}
      \endminipage
    \end{column}
    \begin{column}{0.55\textwidth}
      \onslide*<1,2>{
        \mintocaml[firstline=100,lastline=101]{../ocaml/stdlib/printexc.ml}
      }
      \only<1>{
        \listocaml[firstline=170,lastline=172]{../ocaml/stdlib/printexc.ml}{stdlib/printexc.ml:170}
      }
      \only<2>{
        \listocaml[firstline=170,lastline=172,highlightlines=172]{../ocaml/stdlib/printexc.ml}{stdlib/printexc.ml:170}
      }
      \only<3>{
        \mintc[firstline=154,lastline=156]{../ocaml/runtime/backtrace.c}
        \listc[firstline=171,lastline=192,highlightlines={184-187}]{../ocaml/runtime/backtrace.c}{runtime/backtrace.c:154}
      }
      \only<4>{
        \listocaml[firstline=155,lastline=165]{../ocaml/stdlib/printexc.ml}{stdlib/printexc.ml:55}
      }
    \end{column}
  \end{columns}
\end{frame}


\subsection{The multiple kinds of raise}
\frameSubsection{}{}

\begin{frame}{Backtraces}{When backtraces are not needed}
  \begin{columns}[c]
    \begin{column}{0.45\textwidth}
      \minipage[c][0.66\textheight][s]{\columnwidth}
      \begin{itemize}
        \item<1-> What if the backtrace for an exception is never needed?
        \item<2-> It's possible to raise the exception explicitly with \funcname{raise\_notrace} to make raising as cheap as possible
        \item<3-> An example of use in the \funcname{dynlink} library of the OCaml compiler
      \end{itemize}
      \vfill
      \onslide<3->{
      \listocaml[firstline=205,lastline=209,highlightlines=207]{../ocaml/otherlibs/dynlink/dynlink\_compilerlibs/misc.ml}{otherlibs/dynlink/dynlink\_compilerlibs/misc.ml:205}
      }
      \endminipage
    \end{column}
    \begin{column}{0.55\textwidth}
    \only<4>{
      \mintcmm[highlightlines=3]{../src/notrace/camlNotrace\_\_fun_347.cmm}
      \listcmm{../src/notrace/camlNotrace\_\_all\_somes\_327.cmm}{all\_somes}
    }
    \onslide*<5->{
      \listobjdump[highlightlines={6-9}]{../src/notrace/camlNotrace\_\_fun_347.objdump}{anonymous function from \funcname{all\_somes}}
    }
    \end{column}
  \end{columns}
\end{frame}

\begin{frame}{Backtraces}{Summary table of the multiple kinds of raise}
  \begin{itemize}
    \item When debugging information is enabled and if the backtrace of an exception isn't needed, it's possible to raise with \funcname{raise\_notrace} for performance
    \item When debugging information is \emph{not} enabled, the compiler optimises \funcname{raise} calls to inline assembly (same effect as using \funcname{raise\_notrace})
  \end{itemize}
  \bigskip
  \centering
  \begin{tabular}{| l | c | c | c | c |}
    \hline
    \parbox{10em}{Debug information \\ (\funcarg{-g} flag)}  & \multicolumn{2}{|c|}{Disabled}                                     & \multicolumn{2}{|c|}{Enabled} \\
    \hline
    Raise kind                                               & \funcname{raise} & \funcname{raise\_notrace}                       & \funcname{raise} & \funcname{raise\_notrace} \\
    \hline
    Compilation                                              & \parbox{8em}{inline assembly\\ (optimization)} & inline assembly   & \funcname{caml\_raise\_exn} & inline assembly \\
    \hline
  \end{tabular}
\end{frame}


%
% Takeaway #5
%
\subsection*{Takeaway \#5}
\frameSubsectionTakeaway{}
\againframe<5>{takeaway}
}%
      \onslide*<2>{\providecommand\step{1}\input{diagrams/backtrace/scanstack.tex}}%
      \onslide*<3>{\providecommand\step{2}\input{diagrams/backtrace/scanstack.tex}}%
      \onslide*<4>{\providecommand\step{3}\input{diagrams/backtrace/scanstack.tex}}%
      \onslide*<5>{\providecommand\step{4}\input{diagrams/backtrace/scanstack.tex}}%
      \onslide*<6>{\providecommand\step{5}\input{diagrams/backtrace/scanstack.tex}}%
    \end{column}
  \end{columns}
\end{frame}

% Duplicate of previous-previous slide
\begin{frame}{Backtraces}{Continuing on \funcname{caml\_stash\_backtrace}}
  \begin{columns}[c]
    \begin{column}{0.5\textwidth}
      \minipage[c][0.75\textheight][s]{\columnwidth}
      \begin{itemize}
          \color{gray}
        \item A C function provided by the runtime (specific to native code)
        \item It scans the stack, between the stack pointer at the raising point up to the next trap, in order to collect the names of the detected function frames
        \item It uses the same technique and information as the Garbage Collector (GC) uses to scan the values live on the stack
      \end{itemize}
      \begin{itemize}
        \item For each function frame detected, store a pointer to the \typename{frame\_descr} in the \localname{domain\_state->backtrace\_buffer} array, effectively constructing the backtrace
      \end{itemize}
      \onslide<2->{
        \begin{itemize}
            \smallskip
          \item Constructed backtrace:
            \funcname{definitely\_raise},
            \onslide<3->{\funcname{probably\_raise}}
        \end{itemize}
      }
      \vfill
      \mintocaml[firstline=9,lastline=13,highlightlines=12]{../src/backtrace/backtrace.ml}
      \endminipage
    \end{column}
    \begin{column}{0.5\textwidth}
      \raggedleft
      \onslide*<1>{\listStashBacktrace[highlightlines={114-115}]{}}
      \onslide*<2>{\providecommand\step{3}%
% Backtraces
%
\subsection{One advantage of raising with debugging information}
\frameSubsection{}{}

\begin{frame}{Backtraces}{Debugging information \& source code}
  \begin{columns}[c]
    \begin{column}{0.5\textwidth}
      \begin{itemize}
        \item Raising an exception is fun and games, but how do we know where the exception originated? $\rightarrow$ With the backtrace associated with the exception!
        \item In order to get a backtrace from the point where an exception was raised, it's necessary to compile with debugging information enabled (\funcarg{-g} flag for the compiler)
        \item Same example as in the `Nested handlers' section
      \end{itemize}
    \end{column}
    \begin{column}{0.5\textwidth}
      \listocaml[firstline=9,lastline=34]{../src/backtrace/backtrace.ml}{backtrace/backtrace.ml}
    \end{column}
  \end{columns}
\end{frame}

\begin{frame}{Backtraces}{CMM and disassembly of \funcname{definitely\_raise}}
  \begin{columns}[c]
    \begin{column}{0.5\textwidth}
      \minipage[c][0.66\textheight][s]{\columnwidth}
      \begin{itemize}
        \item<1-> An effect of compiling with debugging information (\funcarg{-g}): the CMM no longer uses \funcname{raise\_notrace} to raise the exception but \funcname{raise} instead
        \item<2-> Disassembly shows that CMM \funcname{raise} is actually a call to \funcname{caml\_raise\_exn}, a function in assembler provided by the runtime
      \end{itemize}
      \vfill
      \mintocaml[firstline=9,lastline=13,highlightlines=12]{../src/backtrace/backtrace.ml}
      \endminipage
    \end{column}
    \begin{column}{0.5\textwidth}
      \onslide*<1>{
        \mintcmm[highlightlines=5]{../src/backtrace/camlBacktrace\_\_definitely\_raise\_347.cmm}
      }
      \onslide*<2>{
        \mintobjdump[firstline=2,highlightlines=14]{../src/backtrace/camlBacktrace\_\_definitely\_raise\_347.objdump}
      }
    \end{column}
  \end{columns}
\end{frame}

\begin{frame}{Backtraces}{Introducing \funcname{caml\_raise\_exn}}
  \begin{columns}[c]
    \begin{column}{0.5\textwidth}
      \minipage[c][0.66\textheight][s]{\columnwidth}
      \onslide*<1>{
        \begin{itemize}
          \item Raising the exception is performed using \funcname{caml\_raise\_exn} which is
            \begin{itemize}
              \item An assembly function provided by the runtime
              \item Architecture specific
            \end{itemize}
          \item Let's see what it does differently from \funcname{raise\_notrace}\dots
          \item We are about to raise \typename{ExnC} with \funcname{caml\_raise\_exn} from \funcname{definitely\_raise}
        \end{itemize}
      }
      \onslide*<2>{
        \mintobjdump[firstline=2,highlightlines=14]{../src/backtrace/camlBacktrace\_\_definitely\_raise\_347.objdump}
      }
      \vfill
      \mintocaml[firstline=9,lastline=13,highlightlines=12]{../src/backtrace/backtrace.ml}
      \endminipage
    \end{column}
    \begin{column}{0.5\textwidth}
      \centering
      \providecommand\step{1}\input{diagrams/backtrace/backtrace.tex}
    \end{column}
  \end{columns}
\end{frame}

\begin{frame}{Backtraces}{But before that, we need more fields from \typename{caml\_domain\_state}}
  \begin{columns}
    \begin{column}{0.5\textwidth}
      \begin{itemize}
        \item Remember \typename{caml\_domain\_state} the per-domain runtime state?
        \item Some other members are of interest to us now\dots
        \item \localname{backtrace\_active} is used to know if the runtime should collect backtraces
        \item \localname{backtrace\_active} can be set by
          \begin{itemize}
            \item The \funcarg{b} flag in the \funcname{OCAMLRUNPARAM} environment variable
            \item \mintinline{ocaml}{Printexc.record_backtrace : bool -> unit} \footnotemark
          \end{itemize}
      \end{itemize}
    \end{column}
    \begin{column}{0.5\textwidth}
      \begin{listing}
        \mintc[highlightlines={9-11}]{src/ocaml/domain\_state\_backtrace.h}
        \caption{runtime/caml/domain\_state.h}
      \end{listing}
    \end{column}
  \end{columns}
  \footnotetext[1]{https://v2.ocaml.org/api/Printexc.html}
\end{frame}

\newcommand\listCamlRaiseExn[1][]{
  \listasm[firstline=702,lastline=725,#1]{../ocaml/runtime/amd64.S}{runtime/amd64.S:702}}

\begin{frame}{Backtraces}{Walk through \funcname{caml\_raise\_exn}}
  \begin{columns}[T]
    \begin{column}{0.5\textwidth}
      \begin{itemize}
        \item<1-> First checks whether exceptions backtrace should be recorded
        \item<1-> By checking the \funcarg{domain\_state->backtrace\_active} value
        \item<2-> If not, acts as \funcname{raise\_notrace} with \funcname{RESTORE\_EXN\_HANDLER\_OCAML}
          \begin{itemize}
            \item Set \regname{RSP} to \localname{domain\_state->exn\_handler} value
            \item Restore \localname{domain\_state->exn\_handler} from trap
            \item Jump with \funcname{ret} (same effect as using \funcname{pop} and \funcname{jmp})
          \end{itemize}
        \item<3-> Otherwise, switches to the C stack to be able to execute C code
        \item<4-> Calls \funcname{caml\_stash\_backtrace}, a C function provided by the runtime\ignorespaces
          \onslide<6->{, with
            \begin{itemize}
              \item \funcarg{exn}: exception value
              \item \funcarg{pc}: code address of the raising point
              \item \funcarg{sp}: stack address of the raising function
              \item \funcarg{trapsp}: stack address of the next handler
            \end{itemize}
          }
      \end{itemize}
      \bigskip
      \mintocaml[firstline=9,lastline=13,highlightlines=12]{../src/backtrace/backtrace.ml}
    \end{column}
    \begin{column}{0.5\textwidth}
      \raggedleft
      \onslide*<1,3-4>{
        \mintc[firstline=190,lastline=192]{../ocaml/runtime/amd64.S}
        \mintc[firstline=234,lastline=237]{../ocaml/runtime/amd64.S}
      }
      \onslide*<2>{
        \mintc[firstline=190,lastline=192,highlightlines={191-192}]{../ocaml/runtime/amd64.S}
        \mintc[firstline=234,lastline=237,highlightlines={235-237}]{../ocaml/runtime/amd64.S}
      }
      \onslide*<1>{\listCamlRaiseExn[highlightlines=705]{}}%
      \onslide*<2>{\listCamlRaiseExn[highlightlines={707-708}]{}}%
      \onslide*<3>{\listCamlRaiseExn[highlightlines={712-714}]{}}%
      \onslide*<4>{\listCamlRaiseExn[highlightlines={715-720}]{}}%
      \onslide*<5>{\providecommand\step{1}\input{diagrams/backtrace/backtrace.tex}}%
      \onslide*<6>{\providecommand\step{2}\input{diagrams/backtrace/backtrace.tex}}%
    \end{column}
  \end{columns}
\end{frame}

\newcommand\listStashBacktrace[1][]{
  \listc[firstline=91,lastline=120,#1]{../ocaml/runtime/backtrace\_nat.c}{runtime/backtrace\_nat.c:91}}

\begin{frame}{Backtraces}{Introducing \funcname{caml\_stash\_backtrace}}
  \begin{columns}[c]
    \begin{column}{0.5\textwidth}
      \minipage[c][0.66\textheight][s]{\columnwidth}
      \begin{itemize}
        \item<1-> A C function provided by the runtime (specific to native code)
        \item<2-> It scans the stack, between the stack pointer at the raising point up to the next trap, in order to collect the names of the detected function frames
          \onslide*<2>{
            \begin{itemize}
              \item And more information, like line number, column number...
            \end{itemize}
          }
        \item<3-> It uses the same technique and information as the Garbage Collector (GC) uses to scan the values live on the stack
          \onslide*<4>{
            \begin{itemize}
              \item \typename{frame\_descr} are at the core of stack unwinding
            \end{itemize}
          }
      \end{itemize}
      \vfill
      \mintocaml[firstline=9,lastline=13,highlightlines=12]{../src/backtrace/backtrace.ml}
      \endminipage
    \end{column}
    \begin{column}{0.5\textwidth}
      \onslide*<1>{\listStashBacktrace[highlightlines=91]{}}
      \onslide*<2>{\listStashBacktrace[highlightlines={91,117-118}]{}}
      \onslide*<3->{\listStashBacktrace[highlightlines={94,106,109-110}]{}}
    \end{column}
  \end{columns}
\end{frame}

\newcommand\listFrameDescr[1][]{
  \listocaml[firstline=111,lastline=124,#1]{../ocaml/asmcomp/emitaux.ml}{asmcomp/emitaux.ml:111}}
\newcommand\listDebugInfo[1][]{
  \listocaml[firstline=38,lastline=49,#1]{../ocaml/lambda/debuginfo.mli}{lambda/debuginfo.mli:38}}

\begin{frame}{Backtraces}{\typename{frame\_descr} and \typename{frame\_debuginfo} in a nutshell}
  \begin{columns}[c]
    \begin{column}{0.5\textwidth}
      \begin{itemize}
        \item<1-> For every return address in an OCaml program, there is a associated \typename{frame\_descr}
        \item<2-> A \typename{frame\_descr} specifies
        \begin{itemize}
          \item<2-> The size of the function stack frame
          \item<3-> Where live values are on the stack (so that the GC can mark them as live and prevent from being swept)
        \end{itemize}
        \item<4-> When compiled with debug information, \typename{frame\_descr} will be linked to a \typename{frame\_debuginfo}
        \begin{itemize}
          \item<5-> Contains information about the position in the source code (file name, line and column number)
        \end{itemize}
      \end{itemize}
    \end{column}
    \begin{column}{0.5\textwidth}
        \onslide*<1>{\listFrameDescr[highlightlines={118-122}]{}}
        \onslide*<2>{\listFrameDescr[highlightlines={120}]{}}
        \onslide*<3>{\listFrameDescr[highlightlines={121}]{}}
        \onslide*<4->{\listFrameDescr[highlightlines={122}]{}}
        \medskip
        \onslide*<1-3>{\listDebugInfo{}}
        \onslide*<4>{\listDebugInfo[highlightlines={38,49}]{}}
        \onslide*<5>{\listDebugInfo[highlightlines={39-46}]{}}
    \end{column}
  \end{columns}
\end{frame}

\begin{frame}{Backtraces}{Scanning the stack with \typename{frame\_descr}}
  \begin{columns}[c]
    \begin{column}{0.5\textwidth}
      \begin{itemize}
        \item Given the return address of the code that raises the exception by calling \funcname{caml\_raise\_exn} (\funcarg{pc} argument of \funcname{caml\_stash\_backtrace})
        \item And the position of the stack pointer (\funcarg{sp} argument of \funcname{caml\_stash\_backtrace})
        \item The frame size given by \typename{frame\_descr}.\funcarg{frame\_size}, allows to find the end of the previous frame
        \item And the return address of the caller of this function
        \bigskip
        \onslide<3->{
        \item Note that traps (here the reddish \funcname{ExnA}, \funcname{ExnB} and \funcname{ExnC} blocks) belong to the frame that installed them, and so the frame size changes when traps are removed
        }
      \end{itemize}
    \end{column}
    \begin{column}{0.5\textwidth}
      \raggedleft
      \onslide*<1>{\providecommand\step{2}\input{diagrams/backtrace/backtrace.tex}}%
      \onslide*<2>{\providecommand\step{1}\input{diagrams/backtrace/scanstack.tex}}%
      \onslide*<3>{\providecommand\step{2}\input{diagrams/backtrace/scanstack.tex}}%
      \onslide*<4>{\providecommand\step{3}\input{diagrams/backtrace/scanstack.tex}}%
      \onslide*<5>{\providecommand\step{4}\input{diagrams/backtrace/scanstack.tex}}%
      \onslide*<6>{\providecommand\step{5}\input{diagrams/backtrace/scanstack.tex}}%
    \end{column}
  \end{columns}
\end{frame}

% Duplicate of previous-previous slide
\begin{frame}{Backtraces}{Continuing on \funcname{caml\_stash\_backtrace}}
  \begin{columns}[c]
    \begin{column}{0.5\textwidth}
      \minipage[c][0.75\textheight][s]{\columnwidth}
      \begin{itemize}
          \color{gray}
        \item A C function provided by the runtime (specific to native code)
        \item It scans the stack, between the stack pointer at the raising point up to the next trap, in order to collect the names of the detected function frames
        \item It uses the same technique and information as the Garbage Collector (GC) uses to scan the values live on the stack
      \end{itemize}
      \begin{itemize}
        \item For each function frame detected, store a pointer to the \typename{frame\_descr} in the \localname{domain\_state->backtrace\_buffer} array, effectively constructing the backtrace
      \end{itemize}
      \onslide<2->{
        \begin{itemize}
            \smallskip
          \item Constructed backtrace:
            \funcname{definitely\_raise},
            \onslide<3->{\funcname{probably\_raise}}
        \end{itemize}
      }
      \vfill
      \mintocaml[firstline=9,lastline=13,highlightlines=12]{../src/backtrace/backtrace.ml}
      \endminipage
    \end{column}
    \begin{column}{0.5\textwidth}
      \raggedleft
      \onslide*<1>{\listStashBacktrace[highlightlines={114-115}]{}}
      \onslide*<2>{\providecommand\step{3}\input{diagrams/backtrace/backtrace.tex}}%
      \onslide*<3>{\providecommand\step{4}\input{diagrams/backtrace/backtrace.tex}}%
    \end{column}
  \end{columns}
\end{frame}

\begin{frame}{Backtraces}{Back to \funcname{caml\_raise\_exn}}
  \begin{columns}[c]
    \begin{column}{0.5\textwidth}
      \minipage[c][0.66\textheight][s]{\columnwidth}
      \begin{itemize}\color{gray}
        \item Checks wheteher exceptions backtrace should be recorded, by checking the \funcarg{domain\_state->backtrace\_active} value
        \item Switches to the C stack to be able to execute C code
        \item Calls \funcname{caml\_stash\_backtrace}, to collect the backtrace up to the next trap
      \end{itemize}
      \onslide*<2->{
        \begin{itemize}
          \item<2-> Executes the exception handler in \funcname{probably\_raise} with \funcname{RESTORE\_EXN\_HANDLER\_OCAML}
            \begin{itemize}
              \item Set \regname{RSP} to \localname{domain\_state->exn\_handler} value
              \item Restore \localname{domain\_state->exn\_handler} from trap
              \item Jump to the handler code with \funcname{ret}
            \end{itemize}
        \end{itemize}
      }
      \vfill
      \onslide*<1>{\mintocaml[firstline=9,lastline=13,highlightlines=12]{../src/backtrace/backtrace.ml}}
      \onslide*<2->{\mintocaml[firstline=15,lastline=21,highlightlines={19-21}]{../src/backtrace/backtrace.ml}}
      \endminipage
    \end{column}
    \begin{column}{0.5\textwidth}
      \centering
      \onslide*<1>{
        \mintc[firstline=190,lastline=192]{../ocaml/runtime/amd64.S}
        \mintc[firstline=234,lastline=237]{../ocaml/runtime/amd64.S}
      }
      \onslide*<2>{
        \mintc[firstline=190,lastline=192,highlightlines={191-192}]{../ocaml/runtime/amd64.S}
        \mintc[firstline=234,lastline=237,highlightlines={235-237}]{../ocaml/runtime/amd64.S}
      }
      \onslide*<1>{\listCamlRaiseExn[highlightlines=720]{}}
      \onslide*<2>{\listCamlRaiseExn[highlightlines=722]{}}
      \onslide*<3->{\providecommand\step{5}\input{diagrams/backtrace/backtrace.tex}}
    \end{column}
  \end{columns}
\end{frame}

\begin{frame}{Backtraces}{\funcname{caml\_probably\_raise}'s handler doesn't catch \typename{ExnC}}
  \begin{columns}[c]
    \begin{column}{0.45\textwidth}
      \minipage[c][0.75\textheight][s]{\columnwidth}
      \begin{itemize}
        \item<1-> \funcname{match \dots with}\footnotemark pattern matching can also match exceptions
        \item<1-> The CMM of \funcname{probably\_raise} shows that this \funcname{match \dots with} is implemented with a pattern matching and a \funcname{try \dots with}
        \item<1-> But this one only matches on \typename{ExnA} exception
        \item<2-> Since the pattern matching fails to catch the exception, \funcname{reraise} is called with our current \typename{ExnC} exception
        \item<3-> Disassembly shows that \funcname{reraise} is actually a call to \funcname{caml\_reraise\_exn}, which is also an assembly function provided by the runtime
      \end{itemize}
      \vfill
      \mintocaml[firstline=15,lastline=21,highlightlines={18-21}]{../src/backtrace/backtrace.ml}
      \endminipage
    \end{column}
    \begin{column}{0.55\textwidth}
      \centering
      \onslide*<1>{\mintcmm[highlightlines={10-18}]{../src/backtrace/camlBacktrace\_\_probably\_raise\_351.cmm}}
      \onslide*<2>{\listcmm[highlightlines=18]{../src/backtrace/camlBacktrace\_\_probably\_raise_351.cmm}{CMM of probably\_raise}}
      \onslide*<3>{\mintobjdump[firstline=5,lastline=35,highlightlines=31]{../src/backtrace/camlBacktrace\_\_probably\_raise_351.objdump}}
    \end{column}
  \end{columns}
  \only<1>{\footnotetext[1]{https://blog.janestreet.com/pattern-matching-and-exception-handling-unite/}}
\end{frame}

\newcommand\listCamlReraiseExn[1][]{
  \listasm[firstline=727,lastline=734,#1]{../ocaml/runtime/amd64.S}{runtime/amd64.S:727}}

\begin{frame}{Backtraces}{Introducing \funcname{caml\_reraise\_exn}}
  \begin{columns}[c]
    \begin{column}{0.5\textwidth}
      \minipage[c][0.66\textheight][s]{\columnwidth}
      \begin{itemize}
        \item<1-> The exception have to be reraised to the next handler
        \item<2-> First, checks if the backtrace should be collected with \funcarg{domain\_state->backtrace\_active}
        \only<3>{
        \begin{itemize}
            \item If not, behaves like a \funcname{raise\_notrace} with \funcname{RESTORE\_EXN\_HANDLER\_OCAML}
        \end{itemize}
        }
        \item<4-> Jumps into \funcname{caml\_raise\_exn}
        \item<5-> Calls \funcname{caml\_stash\_backtrace} again, to scan the next part of the stack: from the trap just pop-ed to the next trap
      \end{itemize}
      \vfill
      \mintocaml[firstline=15,lastline=21,highlightlines={18-21}]{../src/backtrace/backtrace.ml}
      \endminipage
    \end{column}
    \begin{column}{0.5\textwidth}
      \raggedleft
      \onslide*<1>{\listCamlReraiseExn{}}
      \onslide*<2>{\listCamlReraiseExn[highlightlines=729]{}}
      \onslide*<3>{
        \mintc[firstline=190,lastline=192,highlightlines={191-192}]{../ocaml/runtime/amd64.S}
        \mintc[firstline=234,lastline=237,highlightlines={235-237}]{../ocaml/runtime/amd64.S}
        \medskip
        \listCamlReraiseExn[highlightlines=731]{}
      }
      \onslide*<4-5>{
        % TODO refactor with \listCamlReraiseExn
        \mintasm[firstline=727,lastline=734,highlightlines=730]{../ocaml/runtime/amd64.S}
        \medskip
      }
      \onslide*<4>{\listCamlRaiseExn[highlightlines=711]{}}
      \onslide*<5>{\listCamlRaiseExn[highlightlines=720]{}}
    \end{column}
  \end{columns}
\end{frame}

\begin{frame}{Backtraces}{Backtrace collection continues}
  \begin{columns}[c]
    \begin{column}{0.55\textwidth}
      \minipage[c][0.75\textheight][s]{\columnwidth}
      \begin{itemize}
        \item<1-> \funcname{caml\_stash\_backtrace} scans the older part of the stack, up to the next trap
        \item<6-> Controls jumps to the nested exception handler in \funcname{maybe\_raise}, which matches only on \typename{ExnB}
        \item<7-> \funcname{caml\_reraise\_exn} is called again, backtrace collection resumes with \funcname{caml\_stash\_backtrace}
      \end{itemize}
      \medskip
      \begin{itemize}
        \item<2-> Constructed backtrace:
        \begin{itemize}
          \item <2-> \funcname{definitely\_raise}
          \item <2-> \funcname{probably\_raise}
          \item <3-> \funcname{probably\_raise} (re-raise)
          \item <4-> \funcname{maybe\_raise}
          \item <5-> \funcname{main}
          \item <8-> \funcname{main} (re-raise)
        \end{itemize}
      \end{itemize}
      \vfill
      \onslide*<1-5>{\mintocaml[firstline=15,lastline=21]{../src/backtrace/backtrace.ml}}
      \onslide*<6>{\mintocaml[firstline=28,lastline=34,highlightlines=31]{../src/backtrace/backtrace.ml}}
      \onslide*<7->{\mintocaml[firstline=28,lastline=34,highlightlines=33]{../src/backtrace/backtrace.ml}}
      \endminipage
    \end{column}
    \begin{column}{0.45\textwidth}
      \raggedleft
      \onslide*<1>{\providecommand\step{6}\input{diagrams/backtrace/backtrace.tex}}%
      \onslide*<2>{\providecommand\step{6}\input{diagrams/backtrace/backtrace.tex}}%
      \onslide*<3>{\providecommand\step{7}\input{diagrams/backtrace/backtrace.tex}}%
      \onslide*<4>{\providecommand\step{8}\input{diagrams/backtrace/backtrace.tex}}%
      \onslide*<5>{\providecommand\step{9}\input{diagrams/backtrace/backtrace.tex}}%
      \onslide*<6>{\providecommand\step{10}\input{diagrams/backtrace/backtrace.tex}}%
      \onslide*<7>{\providecommand\step{11}\input{diagrams/backtrace/backtrace.tex}}%
      \onslide*<8>{\providecommand\step{12}\input{diagrams/backtrace/backtrace.tex}}%
      \onslide*<9>{\providecommand\step{13}\input{diagrams/backtrace/backtrace.tex}}%
    \end{column}
  \end{columns}
\end{frame}

\begin{frame}{Backtraces}{Printing the backtrace}
  \begin{columns}[c]
    \begin{column}{0.45\textwidth}
      \minipage[c][0.66\textheight][s]{\columnwidth}
      \begin{itemize}
        \item<1-> The exception backtrace can now be printed to \funcarg{stdout} with \funcname{Printexc.print_backtrace}
        \item<2-> First the exception backtrace is retrieved into an OCaml array with \funcname{get_raw_backtrace }
        \item<3-> Which is implemented as the \funcname{caml_get_exception_raw_backtrace} C function in the runtime
        \item<4-> And finally printed with \funcname{print_exception_backtrace}
      \end{itemize}
      \vfill
      \mintocaml[firstline=28,lastline=34,highlightlines=33]{../src/backtrace/backtrace.ml}
      \endminipage
    \end{column}
    \begin{column}{0.55\textwidth}
      \onslide*<1,2>{
        \mintocaml[firstline=100,lastline=101]{../ocaml/stdlib/printexc.ml}
      }
      \only<1>{
        \listocaml[firstline=170,lastline=172]{../ocaml/stdlib/printexc.ml}{stdlib/printexc.ml:170}
      }
      \only<2>{
        \listocaml[firstline=170,lastline=172,highlightlines=172]{../ocaml/stdlib/printexc.ml}{stdlib/printexc.ml:170}
      }
      \only<3>{
        \mintc[firstline=154,lastline=156]{../ocaml/runtime/backtrace.c}
        \listc[firstline=171,lastline=192,highlightlines={184-187}]{../ocaml/runtime/backtrace.c}{runtime/backtrace.c:154}
      }
      \only<4>{
        \listocaml[firstline=155,lastline=165]{../ocaml/stdlib/printexc.ml}{stdlib/printexc.ml:55}
      }
    \end{column}
  \end{columns}
\end{frame}


\subsection{The multiple kinds of raise}
\frameSubsection{}{}

\begin{frame}{Backtraces}{When backtraces are not needed}
  \begin{columns}[c]
    \begin{column}{0.45\textwidth}
      \minipage[c][0.66\textheight][s]{\columnwidth}
      \begin{itemize}
        \item<1-> What if the backtrace for an exception is never needed?
        \item<2-> It's possible to raise the exception explicitly with \funcname{raise\_notrace} to make raising as cheap as possible
        \item<3-> An example of use in the \funcname{dynlink} library of the OCaml compiler
      \end{itemize}
      \vfill
      \onslide<3->{
      \listocaml[firstline=205,lastline=209,highlightlines=207]{../ocaml/otherlibs/dynlink/dynlink\_compilerlibs/misc.ml}{otherlibs/dynlink/dynlink\_compilerlibs/misc.ml:205}
      }
      \endminipage
    \end{column}
    \begin{column}{0.55\textwidth}
    \only<4>{
      \mintcmm[highlightlines=3]{../src/notrace/camlNotrace\_\_fun_347.cmm}
      \listcmm{../src/notrace/camlNotrace\_\_all\_somes\_327.cmm}{all\_somes}
    }
    \onslide*<5->{
      \listobjdump[highlightlines={6-9}]{../src/notrace/camlNotrace\_\_fun_347.objdump}{anonymous function from \funcname{all\_somes}}
    }
    \end{column}
  \end{columns}
\end{frame}

\begin{frame}{Backtraces}{Summary table of the multiple kinds of raise}
  \begin{itemize}
    \item When debugging information is enabled and if the backtrace of an exception isn't needed, it's possible to raise with \funcname{raise\_notrace} for performance
    \item When debugging information is \emph{not} enabled, the compiler optimises \funcname{raise} calls to inline assembly (same effect as using \funcname{raise\_notrace})
  \end{itemize}
  \bigskip
  \centering
  \begin{tabular}{| l | c | c | c | c |}
    \hline
    \parbox{10em}{Debug information \\ (\funcarg{-g} flag)}  & \multicolumn{2}{|c|}{Disabled}                                     & \multicolumn{2}{|c|}{Enabled} \\
    \hline
    Raise kind                                               & \funcname{raise} & \funcname{raise\_notrace}                       & \funcname{raise} & \funcname{raise\_notrace} \\
    \hline
    Compilation                                              & \parbox{8em}{inline assembly\\ (optimization)} & inline assembly   & \funcname{caml\_raise\_exn} & inline assembly \\
    \hline
  \end{tabular}
\end{frame}


%
% Takeaway #5
%
\subsection*{Takeaway \#5}
\frameSubsectionTakeaway{}
\againframe<5>{takeaway}
}%
      \onslide*<3>{\providecommand\step{4}%
% Backtraces
%
\subsection{One advantage of raising with debugging information}
\frameSubsection{}{}

\begin{frame}{Backtraces}{Debugging information \& source code}
  \begin{columns}[c]
    \begin{column}{0.5\textwidth}
      \begin{itemize}
        \item Raising an exception is fun and games, but how do we know where the exception originated? $\rightarrow$ With the backtrace associated with the exception!
        \item In order to get a backtrace from the point where an exception was raised, it's necessary to compile with debugging information enabled (\funcarg{-g} flag for the compiler)
        \item Same example as in the `Nested handlers' section
      \end{itemize}
    \end{column}
    \begin{column}{0.5\textwidth}
      \listocaml[firstline=9,lastline=34]{../src/backtrace/backtrace.ml}{backtrace/backtrace.ml}
    \end{column}
  \end{columns}
\end{frame}

\begin{frame}{Backtraces}{CMM and disassembly of \funcname{definitely\_raise}}
  \begin{columns}[c]
    \begin{column}{0.5\textwidth}
      \minipage[c][0.66\textheight][s]{\columnwidth}
      \begin{itemize}
        \item<1-> An effect of compiling with debugging information (\funcarg{-g}): the CMM no longer uses \funcname{raise\_notrace} to raise the exception but \funcname{raise} instead
        \item<2-> Disassembly shows that CMM \funcname{raise} is actually a call to \funcname{caml\_raise\_exn}, a function in assembler provided by the runtime
      \end{itemize}
      \vfill
      \mintocaml[firstline=9,lastline=13,highlightlines=12]{../src/backtrace/backtrace.ml}
      \endminipage
    \end{column}
    \begin{column}{0.5\textwidth}
      \onslide*<1>{
        \mintcmm[highlightlines=5]{../src/backtrace/camlBacktrace\_\_definitely\_raise\_347.cmm}
      }
      \onslide*<2>{
        \mintobjdump[firstline=2,highlightlines=14]{../src/backtrace/camlBacktrace\_\_definitely\_raise\_347.objdump}
      }
    \end{column}
  \end{columns}
\end{frame}

\begin{frame}{Backtraces}{Introducing \funcname{caml\_raise\_exn}}
  \begin{columns}[c]
    \begin{column}{0.5\textwidth}
      \minipage[c][0.66\textheight][s]{\columnwidth}
      \onslide*<1>{
        \begin{itemize}
          \item Raising the exception is performed using \funcname{caml\_raise\_exn} which is
            \begin{itemize}
              \item An assembly function provided by the runtime
              \item Architecture specific
            \end{itemize}
          \item Let's see what it does differently from \funcname{raise\_notrace}\dots
          \item We are about to raise \typename{ExnC} with \funcname{caml\_raise\_exn} from \funcname{definitely\_raise}
        \end{itemize}
      }
      \onslide*<2>{
        \mintobjdump[firstline=2,highlightlines=14]{../src/backtrace/camlBacktrace\_\_definitely\_raise\_347.objdump}
      }
      \vfill
      \mintocaml[firstline=9,lastline=13,highlightlines=12]{../src/backtrace/backtrace.ml}
      \endminipage
    \end{column}
    \begin{column}{0.5\textwidth}
      \centering
      \providecommand\step{1}\input{diagrams/backtrace/backtrace.tex}
    \end{column}
  \end{columns}
\end{frame}

\begin{frame}{Backtraces}{But before that, we need more fields from \typename{caml\_domain\_state}}
  \begin{columns}
    \begin{column}{0.5\textwidth}
      \begin{itemize}
        \item Remember \typename{caml\_domain\_state} the per-domain runtime state?
        \item Some other members are of interest to us now\dots
        \item \localname{backtrace\_active} is used to know if the runtime should collect backtraces
        \item \localname{backtrace\_active} can be set by
          \begin{itemize}
            \item The \funcarg{b} flag in the \funcname{OCAMLRUNPARAM} environment variable
            \item \mintinline{ocaml}{Printexc.record_backtrace : bool -> unit} \footnotemark
          \end{itemize}
      \end{itemize}
    \end{column}
    \begin{column}{0.5\textwidth}
      \begin{listing}
        \mintc[highlightlines={9-11}]{src/ocaml/domain\_state\_backtrace.h}
        \caption{runtime/caml/domain\_state.h}
      \end{listing}
    \end{column}
  \end{columns}
  \footnotetext[1]{https://v2.ocaml.org/api/Printexc.html}
\end{frame}

\newcommand\listCamlRaiseExn[1][]{
  \listasm[firstline=702,lastline=725,#1]{../ocaml/runtime/amd64.S}{runtime/amd64.S:702}}

\begin{frame}{Backtraces}{Walk through \funcname{caml\_raise\_exn}}
  \begin{columns}[T]
    \begin{column}{0.5\textwidth}
      \begin{itemize}
        \item<1-> First checks whether exceptions backtrace should be recorded
        \item<1-> By checking the \funcarg{domain\_state->backtrace\_active} value
        \item<2-> If not, acts as \funcname{raise\_notrace} with \funcname{RESTORE\_EXN\_HANDLER\_OCAML}
          \begin{itemize}
            \item Set \regname{RSP} to \localname{domain\_state->exn\_handler} value
            \item Restore \localname{domain\_state->exn\_handler} from trap
            \item Jump with \funcname{ret} (same effect as using \funcname{pop} and \funcname{jmp})
          \end{itemize}
        \item<3-> Otherwise, switches to the C stack to be able to execute C code
        \item<4-> Calls \funcname{caml\_stash\_backtrace}, a C function provided by the runtime\ignorespaces
          \onslide<6->{, with
            \begin{itemize}
              \item \funcarg{exn}: exception value
              \item \funcarg{pc}: code address of the raising point
              \item \funcarg{sp}: stack address of the raising function
              \item \funcarg{trapsp}: stack address of the next handler
            \end{itemize}
          }
      \end{itemize}
      \bigskip
      \mintocaml[firstline=9,lastline=13,highlightlines=12]{../src/backtrace/backtrace.ml}
    \end{column}
    \begin{column}{0.5\textwidth}
      \raggedleft
      \onslide*<1,3-4>{
        \mintc[firstline=190,lastline=192]{../ocaml/runtime/amd64.S}
        \mintc[firstline=234,lastline=237]{../ocaml/runtime/amd64.S}
      }
      \onslide*<2>{
        \mintc[firstline=190,lastline=192,highlightlines={191-192}]{../ocaml/runtime/amd64.S}
        \mintc[firstline=234,lastline=237,highlightlines={235-237}]{../ocaml/runtime/amd64.S}
      }
      \onslide*<1>{\listCamlRaiseExn[highlightlines=705]{}}%
      \onslide*<2>{\listCamlRaiseExn[highlightlines={707-708}]{}}%
      \onslide*<3>{\listCamlRaiseExn[highlightlines={712-714}]{}}%
      \onslide*<4>{\listCamlRaiseExn[highlightlines={715-720}]{}}%
      \onslide*<5>{\providecommand\step{1}\input{diagrams/backtrace/backtrace.tex}}%
      \onslide*<6>{\providecommand\step{2}\input{diagrams/backtrace/backtrace.tex}}%
    \end{column}
  \end{columns}
\end{frame}

\newcommand\listStashBacktrace[1][]{
  \listc[firstline=91,lastline=120,#1]{../ocaml/runtime/backtrace\_nat.c}{runtime/backtrace\_nat.c:91}}

\begin{frame}{Backtraces}{Introducing \funcname{caml\_stash\_backtrace}}
  \begin{columns}[c]
    \begin{column}{0.5\textwidth}
      \minipage[c][0.66\textheight][s]{\columnwidth}
      \begin{itemize}
        \item<1-> A C function provided by the runtime (specific to native code)
        \item<2-> It scans the stack, between the stack pointer at the raising point up to the next trap, in order to collect the names of the detected function frames
          \onslide*<2>{
            \begin{itemize}
              \item And more information, like line number, column number...
            \end{itemize}
          }
        \item<3-> It uses the same technique and information as the Garbage Collector (GC) uses to scan the values live on the stack
          \onslide*<4>{
            \begin{itemize}
              \item \typename{frame\_descr} are at the core of stack unwinding
            \end{itemize}
          }
      \end{itemize}
      \vfill
      \mintocaml[firstline=9,lastline=13,highlightlines=12]{../src/backtrace/backtrace.ml}
      \endminipage
    \end{column}
    \begin{column}{0.5\textwidth}
      \onslide*<1>{\listStashBacktrace[highlightlines=91]{}}
      \onslide*<2>{\listStashBacktrace[highlightlines={91,117-118}]{}}
      \onslide*<3->{\listStashBacktrace[highlightlines={94,106,109-110}]{}}
    \end{column}
  \end{columns}
\end{frame}

\newcommand\listFrameDescr[1][]{
  \listocaml[firstline=111,lastline=124,#1]{../ocaml/asmcomp/emitaux.ml}{asmcomp/emitaux.ml:111}}
\newcommand\listDebugInfo[1][]{
  \listocaml[firstline=38,lastline=49,#1]{../ocaml/lambda/debuginfo.mli}{lambda/debuginfo.mli:38}}

\begin{frame}{Backtraces}{\typename{frame\_descr} and \typename{frame\_debuginfo} in a nutshell}
  \begin{columns}[c]
    \begin{column}{0.5\textwidth}
      \begin{itemize}
        \item<1-> For every return address in an OCaml program, there is a associated \typename{frame\_descr}
        \item<2-> A \typename{frame\_descr} specifies
        \begin{itemize}
          \item<2-> The size of the function stack frame
          \item<3-> Where live values are on the stack (so that the GC can mark them as live and prevent from being swept)
        \end{itemize}
        \item<4-> When compiled with debug information, \typename{frame\_descr} will be linked to a \typename{frame\_debuginfo}
        \begin{itemize}
          \item<5-> Contains information about the position in the source code (file name, line and column number)
        \end{itemize}
      \end{itemize}
    \end{column}
    \begin{column}{0.5\textwidth}
        \onslide*<1>{\listFrameDescr[highlightlines={118-122}]{}}
        \onslide*<2>{\listFrameDescr[highlightlines={120}]{}}
        \onslide*<3>{\listFrameDescr[highlightlines={121}]{}}
        \onslide*<4->{\listFrameDescr[highlightlines={122}]{}}
        \medskip
        \onslide*<1-3>{\listDebugInfo{}}
        \onslide*<4>{\listDebugInfo[highlightlines={38,49}]{}}
        \onslide*<5>{\listDebugInfo[highlightlines={39-46}]{}}
    \end{column}
  \end{columns}
\end{frame}

\begin{frame}{Backtraces}{Scanning the stack with \typename{frame\_descr}}
  \begin{columns}[c]
    \begin{column}{0.5\textwidth}
      \begin{itemize}
        \item Given the return address of the code that raises the exception by calling \funcname{caml\_raise\_exn} (\funcarg{pc} argument of \funcname{caml\_stash\_backtrace})
        \item And the position of the stack pointer (\funcarg{sp} argument of \funcname{caml\_stash\_backtrace})
        \item The frame size given by \typename{frame\_descr}.\funcarg{frame\_size}, allows to find the end of the previous frame
        \item And the return address of the caller of this function
        \bigskip
        \onslide<3->{
        \item Note that traps (here the reddish \funcname{ExnA}, \funcname{ExnB} and \funcname{ExnC} blocks) belong to the frame that installed them, and so the frame size changes when traps are removed
        }
      \end{itemize}
    \end{column}
    \begin{column}{0.5\textwidth}
      \raggedleft
      \onslide*<1>{\providecommand\step{2}\input{diagrams/backtrace/backtrace.tex}}%
      \onslide*<2>{\providecommand\step{1}\input{diagrams/backtrace/scanstack.tex}}%
      \onslide*<3>{\providecommand\step{2}\input{diagrams/backtrace/scanstack.tex}}%
      \onslide*<4>{\providecommand\step{3}\input{diagrams/backtrace/scanstack.tex}}%
      \onslide*<5>{\providecommand\step{4}\input{diagrams/backtrace/scanstack.tex}}%
      \onslide*<6>{\providecommand\step{5}\input{diagrams/backtrace/scanstack.tex}}%
    \end{column}
  \end{columns}
\end{frame}

% Duplicate of previous-previous slide
\begin{frame}{Backtraces}{Continuing on \funcname{caml\_stash\_backtrace}}
  \begin{columns}[c]
    \begin{column}{0.5\textwidth}
      \minipage[c][0.75\textheight][s]{\columnwidth}
      \begin{itemize}
          \color{gray}
        \item A C function provided by the runtime (specific to native code)
        \item It scans the stack, between the stack pointer at the raising point up to the next trap, in order to collect the names of the detected function frames
        \item It uses the same technique and information as the Garbage Collector (GC) uses to scan the values live on the stack
      \end{itemize}
      \begin{itemize}
        \item For each function frame detected, store a pointer to the \typename{frame\_descr} in the \localname{domain\_state->backtrace\_buffer} array, effectively constructing the backtrace
      \end{itemize}
      \onslide<2->{
        \begin{itemize}
            \smallskip
          \item Constructed backtrace:
            \funcname{definitely\_raise},
            \onslide<3->{\funcname{probably\_raise}}
        \end{itemize}
      }
      \vfill
      \mintocaml[firstline=9,lastline=13,highlightlines=12]{../src/backtrace/backtrace.ml}
      \endminipage
    \end{column}
    \begin{column}{0.5\textwidth}
      \raggedleft
      \onslide*<1>{\listStashBacktrace[highlightlines={114-115}]{}}
      \onslide*<2>{\providecommand\step{3}\input{diagrams/backtrace/backtrace.tex}}%
      \onslide*<3>{\providecommand\step{4}\input{diagrams/backtrace/backtrace.tex}}%
    \end{column}
  \end{columns}
\end{frame}

\begin{frame}{Backtraces}{Back to \funcname{caml\_raise\_exn}}
  \begin{columns}[c]
    \begin{column}{0.5\textwidth}
      \minipage[c][0.66\textheight][s]{\columnwidth}
      \begin{itemize}\color{gray}
        \item Checks wheteher exceptions backtrace should be recorded, by checking the \funcarg{domain\_state->backtrace\_active} value
        \item Switches to the C stack to be able to execute C code
        \item Calls \funcname{caml\_stash\_backtrace}, to collect the backtrace up to the next trap
      \end{itemize}
      \onslide*<2->{
        \begin{itemize}
          \item<2-> Executes the exception handler in \funcname{probably\_raise} with \funcname{RESTORE\_EXN\_HANDLER\_OCAML}
            \begin{itemize}
              \item Set \regname{RSP} to \localname{domain\_state->exn\_handler} value
              \item Restore \localname{domain\_state->exn\_handler} from trap
              \item Jump to the handler code with \funcname{ret}
            \end{itemize}
        \end{itemize}
      }
      \vfill
      \onslide*<1>{\mintocaml[firstline=9,lastline=13,highlightlines=12]{../src/backtrace/backtrace.ml}}
      \onslide*<2->{\mintocaml[firstline=15,lastline=21,highlightlines={19-21}]{../src/backtrace/backtrace.ml}}
      \endminipage
    \end{column}
    \begin{column}{0.5\textwidth}
      \centering
      \onslide*<1>{
        \mintc[firstline=190,lastline=192]{../ocaml/runtime/amd64.S}
        \mintc[firstline=234,lastline=237]{../ocaml/runtime/amd64.S}
      }
      \onslide*<2>{
        \mintc[firstline=190,lastline=192,highlightlines={191-192}]{../ocaml/runtime/amd64.S}
        \mintc[firstline=234,lastline=237,highlightlines={235-237}]{../ocaml/runtime/amd64.S}
      }
      \onslide*<1>{\listCamlRaiseExn[highlightlines=720]{}}
      \onslide*<2>{\listCamlRaiseExn[highlightlines=722]{}}
      \onslide*<3->{\providecommand\step{5}\input{diagrams/backtrace/backtrace.tex}}
    \end{column}
  \end{columns}
\end{frame}

\begin{frame}{Backtraces}{\funcname{caml\_probably\_raise}'s handler doesn't catch \typename{ExnC}}
  \begin{columns}[c]
    \begin{column}{0.45\textwidth}
      \minipage[c][0.75\textheight][s]{\columnwidth}
      \begin{itemize}
        \item<1-> \funcname{match \dots with}\footnotemark pattern matching can also match exceptions
        \item<1-> The CMM of \funcname{probably\_raise} shows that this \funcname{match \dots with} is implemented with a pattern matching and a \funcname{try \dots with}
        \item<1-> But this one only matches on \typename{ExnA} exception
        \item<2-> Since the pattern matching fails to catch the exception, \funcname{reraise} is called with our current \typename{ExnC} exception
        \item<3-> Disassembly shows that \funcname{reraise} is actually a call to \funcname{caml\_reraise\_exn}, which is also an assembly function provided by the runtime
      \end{itemize}
      \vfill
      \mintocaml[firstline=15,lastline=21,highlightlines={18-21}]{../src/backtrace/backtrace.ml}
      \endminipage
    \end{column}
    \begin{column}{0.55\textwidth}
      \centering
      \onslide*<1>{\mintcmm[highlightlines={10-18}]{../src/backtrace/camlBacktrace\_\_probably\_raise\_351.cmm}}
      \onslide*<2>{\listcmm[highlightlines=18]{../src/backtrace/camlBacktrace\_\_probably\_raise_351.cmm}{CMM of probably\_raise}}
      \onslide*<3>{\mintobjdump[firstline=5,lastline=35,highlightlines=31]{../src/backtrace/camlBacktrace\_\_probably\_raise_351.objdump}}
    \end{column}
  \end{columns}
  \only<1>{\footnotetext[1]{https://blog.janestreet.com/pattern-matching-and-exception-handling-unite/}}
\end{frame}

\newcommand\listCamlReraiseExn[1][]{
  \listasm[firstline=727,lastline=734,#1]{../ocaml/runtime/amd64.S}{runtime/amd64.S:727}}

\begin{frame}{Backtraces}{Introducing \funcname{caml\_reraise\_exn}}
  \begin{columns}[c]
    \begin{column}{0.5\textwidth}
      \minipage[c][0.66\textheight][s]{\columnwidth}
      \begin{itemize}
        \item<1-> The exception have to be reraised to the next handler
        \item<2-> First, checks if the backtrace should be collected with \funcarg{domain\_state->backtrace\_active}
        \only<3>{
        \begin{itemize}
            \item If not, behaves like a \funcname{raise\_notrace} with \funcname{RESTORE\_EXN\_HANDLER\_OCAML}
        \end{itemize}
        }
        \item<4-> Jumps into \funcname{caml\_raise\_exn}
        \item<5-> Calls \funcname{caml\_stash\_backtrace} again, to scan the next part of the stack: from the trap just pop-ed to the next trap
      \end{itemize}
      \vfill
      \mintocaml[firstline=15,lastline=21,highlightlines={18-21}]{../src/backtrace/backtrace.ml}
      \endminipage
    \end{column}
    \begin{column}{0.5\textwidth}
      \raggedleft
      \onslide*<1>{\listCamlReraiseExn{}}
      \onslide*<2>{\listCamlReraiseExn[highlightlines=729]{}}
      \onslide*<3>{
        \mintc[firstline=190,lastline=192,highlightlines={191-192}]{../ocaml/runtime/amd64.S}
        \mintc[firstline=234,lastline=237,highlightlines={235-237}]{../ocaml/runtime/amd64.S}
        \medskip
        \listCamlReraiseExn[highlightlines=731]{}
      }
      \onslide*<4-5>{
        % TODO refactor with \listCamlReraiseExn
        \mintasm[firstline=727,lastline=734,highlightlines=730]{../ocaml/runtime/amd64.S}
        \medskip
      }
      \onslide*<4>{\listCamlRaiseExn[highlightlines=711]{}}
      \onslide*<5>{\listCamlRaiseExn[highlightlines=720]{}}
    \end{column}
  \end{columns}
\end{frame}

\begin{frame}{Backtraces}{Backtrace collection continues}
  \begin{columns}[c]
    \begin{column}{0.55\textwidth}
      \minipage[c][0.75\textheight][s]{\columnwidth}
      \begin{itemize}
        \item<1-> \funcname{caml\_stash\_backtrace} scans the older part of the stack, up to the next trap
        \item<6-> Controls jumps to the nested exception handler in \funcname{maybe\_raise}, which matches only on \typename{ExnB}
        \item<7-> \funcname{caml\_reraise\_exn} is called again, backtrace collection resumes with \funcname{caml\_stash\_backtrace}
      \end{itemize}
      \medskip
      \begin{itemize}
        \item<2-> Constructed backtrace:
        \begin{itemize}
          \item <2-> \funcname{definitely\_raise}
          \item <2-> \funcname{probably\_raise}
          \item <3-> \funcname{probably\_raise} (re-raise)
          \item <4-> \funcname{maybe\_raise}
          \item <5-> \funcname{main}
          \item <8-> \funcname{main} (re-raise)
        \end{itemize}
      \end{itemize}
      \vfill
      \onslide*<1-5>{\mintocaml[firstline=15,lastline=21]{../src/backtrace/backtrace.ml}}
      \onslide*<6>{\mintocaml[firstline=28,lastline=34,highlightlines=31]{../src/backtrace/backtrace.ml}}
      \onslide*<7->{\mintocaml[firstline=28,lastline=34,highlightlines=33]{../src/backtrace/backtrace.ml}}
      \endminipage
    \end{column}
    \begin{column}{0.45\textwidth}
      \raggedleft
      \onslide*<1>{\providecommand\step{6}\input{diagrams/backtrace/backtrace.tex}}%
      \onslide*<2>{\providecommand\step{6}\input{diagrams/backtrace/backtrace.tex}}%
      \onslide*<3>{\providecommand\step{7}\input{diagrams/backtrace/backtrace.tex}}%
      \onslide*<4>{\providecommand\step{8}\input{diagrams/backtrace/backtrace.tex}}%
      \onslide*<5>{\providecommand\step{9}\input{diagrams/backtrace/backtrace.tex}}%
      \onslide*<6>{\providecommand\step{10}\input{diagrams/backtrace/backtrace.tex}}%
      \onslide*<7>{\providecommand\step{11}\input{diagrams/backtrace/backtrace.tex}}%
      \onslide*<8>{\providecommand\step{12}\input{diagrams/backtrace/backtrace.tex}}%
      \onslide*<9>{\providecommand\step{13}\input{diagrams/backtrace/backtrace.tex}}%
    \end{column}
  \end{columns}
\end{frame}

\begin{frame}{Backtraces}{Printing the backtrace}
  \begin{columns}[c]
    \begin{column}{0.45\textwidth}
      \minipage[c][0.66\textheight][s]{\columnwidth}
      \begin{itemize}
        \item<1-> The exception backtrace can now be printed to \funcarg{stdout} with \funcname{Printexc.print_backtrace}
        \item<2-> First the exception backtrace is retrieved into an OCaml array with \funcname{get_raw_backtrace }
        \item<3-> Which is implemented as the \funcname{caml_get_exception_raw_backtrace} C function in the runtime
        \item<4-> And finally printed with \funcname{print_exception_backtrace}
      \end{itemize}
      \vfill
      \mintocaml[firstline=28,lastline=34,highlightlines=33]{../src/backtrace/backtrace.ml}
      \endminipage
    \end{column}
    \begin{column}{0.55\textwidth}
      \onslide*<1,2>{
        \mintocaml[firstline=100,lastline=101]{../ocaml/stdlib/printexc.ml}
      }
      \only<1>{
        \listocaml[firstline=170,lastline=172]{../ocaml/stdlib/printexc.ml}{stdlib/printexc.ml:170}
      }
      \only<2>{
        \listocaml[firstline=170,lastline=172,highlightlines=172]{../ocaml/stdlib/printexc.ml}{stdlib/printexc.ml:170}
      }
      \only<3>{
        \mintc[firstline=154,lastline=156]{../ocaml/runtime/backtrace.c}
        \listc[firstline=171,lastline=192,highlightlines={184-187}]{../ocaml/runtime/backtrace.c}{runtime/backtrace.c:154}
      }
      \only<4>{
        \listocaml[firstline=155,lastline=165]{../ocaml/stdlib/printexc.ml}{stdlib/printexc.ml:55}
      }
    \end{column}
  \end{columns}
\end{frame}


\subsection{The multiple kinds of raise}
\frameSubsection{}{}

\begin{frame}{Backtraces}{When backtraces are not needed}
  \begin{columns}[c]
    \begin{column}{0.45\textwidth}
      \minipage[c][0.66\textheight][s]{\columnwidth}
      \begin{itemize}
        \item<1-> What if the backtrace for an exception is never needed?
        \item<2-> It's possible to raise the exception explicitly with \funcname{raise\_notrace} to make raising as cheap as possible
        \item<3-> An example of use in the \funcname{dynlink} library of the OCaml compiler
      \end{itemize}
      \vfill
      \onslide<3->{
      \listocaml[firstline=205,lastline=209,highlightlines=207]{../ocaml/otherlibs/dynlink/dynlink\_compilerlibs/misc.ml}{otherlibs/dynlink/dynlink\_compilerlibs/misc.ml:205}
      }
      \endminipage
    \end{column}
    \begin{column}{0.55\textwidth}
    \only<4>{
      \mintcmm[highlightlines=3]{../src/notrace/camlNotrace\_\_fun_347.cmm}
      \listcmm{../src/notrace/camlNotrace\_\_all\_somes\_327.cmm}{all\_somes}
    }
    \onslide*<5->{
      \listobjdump[highlightlines={6-9}]{../src/notrace/camlNotrace\_\_fun_347.objdump}{anonymous function from \funcname{all\_somes}}
    }
    \end{column}
  \end{columns}
\end{frame}

\begin{frame}{Backtraces}{Summary table of the multiple kinds of raise}
  \begin{itemize}
    \item When debugging information is enabled and if the backtrace of an exception isn't needed, it's possible to raise with \funcname{raise\_notrace} for performance
    \item When debugging information is \emph{not} enabled, the compiler optimises \funcname{raise} calls to inline assembly (same effect as using \funcname{raise\_notrace})
  \end{itemize}
  \bigskip
  \centering
  \begin{tabular}{| l | c | c | c | c |}
    \hline
    \parbox{10em}{Debug information \\ (\funcarg{-g} flag)}  & \multicolumn{2}{|c|}{Disabled}                                     & \multicolumn{2}{|c|}{Enabled} \\
    \hline
    Raise kind                                               & \funcname{raise} & \funcname{raise\_notrace}                       & \funcname{raise} & \funcname{raise\_notrace} \\
    \hline
    Compilation                                              & \parbox{8em}{inline assembly\\ (optimization)} & inline assembly   & \funcname{caml\_raise\_exn} & inline assembly \\
    \hline
  \end{tabular}
\end{frame}


%
% Takeaway #5
%
\subsection*{Takeaway \#5}
\frameSubsectionTakeaway{}
\againframe<5>{takeaway}
}%
    \end{column}
  \end{columns}
\end{frame}

\begin{frame}{Backtraces}{Back to \funcname{caml\_raise\_exn}}
  \begin{columns}[c]
    \begin{column}{0.5\textwidth}
      \minipage[c][0.66\textheight][s]{\columnwidth}
      \begin{itemize}\color{gray}
        \item Checks wheteher exceptions backtrace should be recorded, by checking the \funcarg{domain\_state->backtrace\_active} value
        \item Switches to the C stack to be able to execute C code
        \item Calls \funcname{caml\_stash\_backtrace}, to collect the backtrace up to the next trap
      \end{itemize}
      \onslide*<2->{
        \begin{itemize}
          \item<2-> Executes the exception handler in \funcname{probably\_raise} with \funcname{RESTORE\_EXN\_HANDLER\_OCAML}
            \begin{itemize}
              \item Set \regname{RSP} to \localname{domain\_state->exn\_handler} value
              \item Restore \localname{domain\_state->exn\_handler} from trap
              \item Jump to the handler code with \funcname{ret}
            \end{itemize}
        \end{itemize}
      }
      \vfill
      \onslide*<1>{\mintocaml[firstline=9,lastline=13,highlightlines=12]{../src/backtrace/backtrace.ml}}
      \onslide*<2->{\mintocaml[firstline=15,lastline=21,highlightlines={19-21}]{../src/backtrace/backtrace.ml}}
      \endminipage
    \end{column}
    \begin{column}{0.5\textwidth}
      \centering
      \onslide*<1>{
        \mintc[firstline=190,lastline=192]{../ocaml/runtime/amd64.S}
        \mintc[firstline=234,lastline=237]{../ocaml/runtime/amd64.S}
      }
      \onslide*<2>{
        \mintc[firstline=190,lastline=192,highlightlines={191-192}]{../ocaml/runtime/amd64.S}
        \mintc[firstline=234,lastline=237,highlightlines={235-237}]{../ocaml/runtime/amd64.S}
      }
      \onslide*<1>{\listCamlRaiseExn[highlightlines=720]{}}
      \onslide*<2>{\listCamlRaiseExn[highlightlines=722]{}}
      \onslide*<3->{\providecommand\step{5}%
% Backtraces
%
\subsection{One advantage of raising with debugging information}
\frameSubsection{}{}

\begin{frame}{Backtraces}{Debugging information \& source code}
  \begin{columns}[c]
    \begin{column}{0.5\textwidth}
      \begin{itemize}
        \item Raising an exception is fun and games, but how do we know where the exception originated? $\rightarrow$ With the backtrace associated with the exception!
        \item In order to get a backtrace from the point where an exception was raised, it's necessary to compile with debugging information enabled (\funcarg{-g} flag for the compiler)
        \item Same example as in the `Nested handlers' section
      \end{itemize}
    \end{column}
    \begin{column}{0.5\textwidth}
      \listocaml[firstline=9,lastline=34]{../src/backtrace/backtrace.ml}{backtrace/backtrace.ml}
    \end{column}
  \end{columns}
\end{frame}

\begin{frame}{Backtraces}{CMM and disassembly of \funcname{definitely\_raise}}
  \begin{columns}[c]
    \begin{column}{0.5\textwidth}
      \minipage[c][0.66\textheight][s]{\columnwidth}
      \begin{itemize}
        \item<1-> An effect of compiling with debugging information (\funcarg{-g}): the CMM no longer uses \funcname{raise\_notrace} to raise the exception but \funcname{raise} instead
        \item<2-> Disassembly shows that CMM \funcname{raise} is actually a call to \funcname{caml\_raise\_exn}, a function in assembler provided by the runtime
      \end{itemize}
      \vfill
      \mintocaml[firstline=9,lastline=13,highlightlines=12]{../src/backtrace/backtrace.ml}
      \endminipage
    \end{column}
    \begin{column}{0.5\textwidth}
      \onslide*<1>{
        \mintcmm[highlightlines=5]{../src/backtrace/camlBacktrace\_\_definitely\_raise\_347.cmm}
      }
      \onslide*<2>{
        \mintobjdump[firstline=2,highlightlines=14]{../src/backtrace/camlBacktrace\_\_definitely\_raise\_347.objdump}
      }
    \end{column}
  \end{columns}
\end{frame}

\begin{frame}{Backtraces}{Introducing \funcname{caml\_raise\_exn}}
  \begin{columns}[c]
    \begin{column}{0.5\textwidth}
      \minipage[c][0.66\textheight][s]{\columnwidth}
      \onslide*<1>{
        \begin{itemize}
          \item Raising the exception is performed using \funcname{caml\_raise\_exn} which is
            \begin{itemize}
              \item An assembly function provided by the runtime
              \item Architecture specific
            \end{itemize}
          \item Let's see what it does differently from \funcname{raise\_notrace}\dots
          \item We are about to raise \typename{ExnC} with \funcname{caml\_raise\_exn} from \funcname{definitely\_raise}
        \end{itemize}
      }
      \onslide*<2>{
        \mintobjdump[firstline=2,highlightlines=14]{../src/backtrace/camlBacktrace\_\_definitely\_raise\_347.objdump}
      }
      \vfill
      \mintocaml[firstline=9,lastline=13,highlightlines=12]{../src/backtrace/backtrace.ml}
      \endminipage
    \end{column}
    \begin{column}{0.5\textwidth}
      \centering
      \providecommand\step{1}\input{diagrams/backtrace/backtrace.tex}
    \end{column}
  \end{columns}
\end{frame}

\begin{frame}{Backtraces}{But before that, we need more fields from \typename{caml\_domain\_state}}
  \begin{columns}
    \begin{column}{0.5\textwidth}
      \begin{itemize}
        \item Remember \typename{caml\_domain\_state} the per-domain runtime state?
        \item Some other members are of interest to us now\dots
        \item \localname{backtrace\_active} is used to know if the runtime should collect backtraces
        \item \localname{backtrace\_active} can be set by
          \begin{itemize}
            \item The \funcarg{b} flag in the \funcname{OCAMLRUNPARAM} environment variable
            \item \mintinline{ocaml}{Printexc.record_backtrace : bool -> unit} \footnotemark
          \end{itemize}
      \end{itemize}
    \end{column}
    \begin{column}{0.5\textwidth}
      \begin{listing}
        \mintc[highlightlines={9-11}]{src/ocaml/domain\_state\_backtrace.h}
        \caption{runtime/caml/domain\_state.h}
      \end{listing}
    \end{column}
  \end{columns}
  \footnotetext[1]{https://v2.ocaml.org/api/Printexc.html}
\end{frame}

\newcommand\listCamlRaiseExn[1][]{
  \listasm[firstline=702,lastline=725,#1]{../ocaml/runtime/amd64.S}{runtime/amd64.S:702}}

\begin{frame}{Backtraces}{Walk through \funcname{caml\_raise\_exn}}
  \begin{columns}[T]
    \begin{column}{0.5\textwidth}
      \begin{itemize}
        \item<1-> First checks whether exceptions backtrace should be recorded
        \item<1-> By checking the \funcarg{domain\_state->backtrace\_active} value
        \item<2-> If not, acts as \funcname{raise\_notrace} with \funcname{RESTORE\_EXN\_HANDLER\_OCAML}
          \begin{itemize}
            \item Set \regname{RSP} to \localname{domain\_state->exn\_handler} value
            \item Restore \localname{domain\_state->exn\_handler} from trap
            \item Jump with \funcname{ret} (same effect as using \funcname{pop} and \funcname{jmp})
          \end{itemize}
        \item<3-> Otherwise, switches to the C stack to be able to execute C code
        \item<4-> Calls \funcname{caml\_stash\_backtrace}, a C function provided by the runtime\ignorespaces
          \onslide<6->{, with
            \begin{itemize}
              \item \funcarg{exn}: exception value
              \item \funcarg{pc}: code address of the raising point
              \item \funcarg{sp}: stack address of the raising function
              \item \funcarg{trapsp}: stack address of the next handler
            \end{itemize}
          }
      \end{itemize}
      \bigskip
      \mintocaml[firstline=9,lastline=13,highlightlines=12]{../src/backtrace/backtrace.ml}
    \end{column}
    \begin{column}{0.5\textwidth}
      \raggedleft
      \onslide*<1,3-4>{
        \mintc[firstline=190,lastline=192]{../ocaml/runtime/amd64.S}
        \mintc[firstline=234,lastline=237]{../ocaml/runtime/amd64.S}
      }
      \onslide*<2>{
        \mintc[firstline=190,lastline=192,highlightlines={191-192}]{../ocaml/runtime/amd64.S}
        \mintc[firstline=234,lastline=237,highlightlines={235-237}]{../ocaml/runtime/amd64.S}
      }
      \onslide*<1>{\listCamlRaiseExn[highlightlines=705]{}}%
      \onslide*<2>{\listCamlRaiseExn[highlightlines={707-708}]{}}%
      \onslide*<3>{\listCamlRaiseExn[highlightlines={712-714}]{}}%
      \onslide*<4>{\listCamlRaiseExn[highlightlines={715-720}]{}}%
      \onslide*<5>{\providecommand\step{1}\input{diagrams/backtrace/backtrace.tex}}%
      \onslide*<6>{\providecommand\step{2}\input{diagrams/backtrace/backtrace.tex}}%
    \end{column}
  \end{columns}
\end{frame}

\newcommand\listStashBacktrace[1][]{
  \listc[firstline=91,lastline=120,#1]{../ocaml/runtime/backtrace\_nat.c}{runtime/backtrace\_nat.c:91}}

\begin{frame}{Backtraces}{Introducing \funcname{caml\_stash\_backtrace}}
  \begin{columns}[c]
    \begin{column}{0.5\textwidth}
      \minipage[c][0.66\textheight][s]{\columnwidth}
      \begin{itemize}
        \item<1-> A C function provided by the runtime (specific to native code)
        \item<2-> It scans the stack, between the stack pointer at the raising point up to the next trap, in order to collect the names of the detected function frames
          \onslide*<2>{
            \begin{itemize}
              \item And more information, like line number, column number...
            \end{itemize}
          }
        \item<3-> It uses the same technique and information as the Garbage Collector (GC) uses to scan the values live on the stack
          \onslide*<4>{
            \begin{itemize}
              \item \typename{frame\_descr} are at the core of stack unwinding
            \end{itemize}
          }
      \end{itemize}
      \vfill
      \mintocaml[firstline=9,lastline=13,highlightlines=12]{../src/backtrace/backtrace.ml}
      \endminipage
    \end{column}
    \begin{column}{0.5\textwidth}
      \onslide*<1>{\listStashBacktrace[highlightlines=91]{}}
      \onslide*<2>{\listStashBacktrace[highlightlines={91,117-118}]{}}
      \onslide*<3->{\listStashBacktrace[highlightlines={94,106,109-110}]{}}
    \end{column}
  \end{columns}
\end{frame}

\newcommand\listFrameDescr[1][]{
  \listocaml[firstline=111,lastline=124,#1]{../ocaml/asmcomp/emitaux.ml}{asmcomp/emitaux.ml:111}}
\newcommand\listDebugInfo[1][]{
  \listocaml[firstline=38,lastline=49,#1]{../ocaml/lambda/debuginfo.mli}{lambda/debuginfo.mli:38}}

\begin{frame}{Backtraces}{\typename{frame\_descr} and \typename{frame\_debuginfo} in a nutshell}
  \begin{columns}[c]
    \begin{column}{0.5\textwidth}
      \begin{itemize}
        \item<1-> For every return address in an OCaml program, there is a associated \typename{frame\_descr}
        \item<2-> A \typename{frame\_descr} specifies
        \begin{itemize}
          \item<2-> The size of the function stack frame
          \item<3-> Where live values are on the stack (so that the GC can mark them as live and prevent from being swept)
        \end{itemize}
        \item<4-> When compiled with debug information, \typename{frame\_descr} will be linked to a \typename{frame\_debuginfo}
        \begin{itemize}
          \item<5-> Contains information about the position in the source code (file name, line and column number)
        \end{itemize}
      \end{itemize}
    \end{column}
    \begin{column}{0.5\textwidth}
        \onslide*<1>{\listFrameDescr[highlightlines={118-122}]{}}
        \onslide*<2>{\listFrameDescr[highlightlines={120}]{}}
        \onslide*<3>{\listFrameDescr[highlightlines={121}]{}}
        \onslide*<4->{\listFrameDescr[highlightlines={122}]{}}
        \medskip
        \onslide*<1-3>{\listDebugInfo{}}
        \onslide*<4>{\listDebugInfo[highlightlines={38,49}]{}}
        \onslide*<5>{\listDebugInfo[highlightlines={39-46}]{}}
    \end{column}
  \end{columns}
\end{frame}

\begin{frame}{Backtraces}{Scanning the stack with \typename{frame\_descr}}
  \begin{columns}[c]
    \begin{column}{0.5\textwidth}
      \begin{itemize}
        \item Given the return address of the code that raises the exception by calling \funcname{caml\_raise\_exn} (\funcarg{pc} argument of \funcname{caml\_stash\_backtrace})
        \item And the position of the stack pointer (\funcarg{sp} argument of \funcname{caml\_stash\_backtrace})
        \item The frame size given by \typename{frame\_descr}.\funcarg{frame\_size}, allows to find the end of the previous frame
        \item And the return address of the caller of this function
        \bigskip
        \onslide<3->{
        \item Note that traps (here the reddish \funcname{ExnA}, \funcname{ExnB} and \funcname{ExnC} blocks) belong to the frame that installed them, and so the frame size changes when traps are removed
        }
      \end{itemize}
    \end{column}
    \begin{column}{0.5\textwidth}
      \raggedleft
      \onslide*<1>{\providecommand\step{2}\input{diagrams/backtrace/backtrace.tex}}%
      \onslide*<2>{\providecommand\step{1}\input{diagrams/backtrace/scanstack.tex}}%
      \onslide*<3>{\providecommand\step{2}\input{diagrams/backtrace/scanstack.tex}}%
      \onslide*<4>{\providecommand\step{3}\input{diagrams/backtrace/scanstack.tex}}%
      \onslide*<5>{\providecommand\step{4}\input{diagrams/backtrace/scanstack.tex}}%
      \onslide*<6>{\providecommand\step{5}\input{diagrams/backtrace/scanstack.tex}}%
    \end{column}
  \end{columns}
\end{frame}

% Duplicate of previous-previous slide
\begin{frame}{Backtraces}{Continuing on \funcname{caml\_stash\_backtrace}}
  \begin{columns}[c]
    \begin{column}{0.5\textwidth}
      \minipage[c][0.75\textheight][s]{\columnwidth}
      \begin{itemize}
          \color{gray}
        \item A C function provided by the runtime (specific to native code)
        \item It scans the stack, between the stack pointer at the raising point up to the next trap, in order to collect the names of the detected function frames
        \item It uses the same technique and information as the Garbage Collector (GC) uses to scan the values live on the stack
      \end{itemize}
      \begin{itemize}
        \item For each function frame detected, store a pointer to the \typename{frame\_descr} in the \localname{domain\_state->backtrace\_buffer} array, effectively constructing the backtrace
      \end{itemize}
      \onslide<2->{
        \begin{itemize}
            \smallskip
          \item Constructed backtrace:
            \funcname{definitely\_raise},
            \onslide<3->{\funcname{probably\_raise}}
        \end{itemize}
      }
      \vfill
      \mintocaml[firstline=9,lastline=13,highlightlines=12]{../src/backtrace/backtrace.ml}
      \endminipage
    \end{column}
    \begin{column}{0.5\textwidth}
      \raggedleft
      \onslide*<1>{\listStashBacktrace[highlightlines={114-115}]{}}
      \onslide*<2>{\providecommand\step{3}\input{diagrams/backtrace/backtrace.tex}}%
      \onslide*<3>{\providecommand\step{4}\input{diagrams/backtrace/backtrace.tex}}%
    \end{column}
  \end{columns}
\end{frame}

\begin{frame}{Backtraces}{Back to \funcname{caml\_raise\_exn}}
  \begin{columns}[c]
    \begin{column}{0.5\textwidth}
      \minipage[c][0.66\textheight][s]{\columnwidth}
      \begin{itemize}\color{gray}
        \item Checks wheteher exceptions backtrace should be recorded, by checking the \funcarg{domain\_state->backtrace\_active} value
        \item Switches to the C stack to be able to execute C code
        \item Calls \funcname{caml\_stash\_backtrace}, to collect the backtrace up to the next trap
      \end{itemize}
      \onslide*<2->{
        \begin{itemize}
          \item<2-> Executes the exception handler in \funcname{probably\_raise} with \funcname{RESTORE\_EXN\_HANDLER\_OCAML}
            \begin{itemize}
              \item Set \regname{RSP} to \localname{domain\_state->exn\_handler} value
              \item Restore \localname{domain\_state->exn\_handler} from trap
              \item Jump to the handler code with \funcname{ret}
            \end{itemize}
        \end{itemize}
      }
      \vfill
      \onslide*<1>{\mintocaml[firstline=9,lastline=13,highlightlines=12]{../src/backtrace/backtrace.ml}}
      \onslide*<2->{\mintocaml[firstline=15,lastline=21,highlightlines={19-21}]{../src/backtrace/backtrace.ml}}
      \endminipage
    \end{column}
    \begin{column}{0.5\textwidth}
      \centering
      \onslide*<1>{
        \mintc[firstline=190,lastline=192]{../ocaml/runtime/amd64.S}
        \mintc[firstline=234,lastline=237]{../ocaml/runtime/amd64.S}
      }
      \onslide*<2>{
        \mintc[firstline=190,lastline=192,highlightlines={191-192}]{../ocaml/runtime/amd64.S}
        \mintc[firstline=234,lastline=237,highlightlines={235-237}]{../ocaml/runtime/amd64.S}
      }
      \onslide*<1>{\listCamlRaiseExn[highlightlines=720]{}}
      \onslide*<2>{\listCamlRaiseExn[highlightlines=722]{}}
      \onslide*<3->{\providecommand\step{5}\input{diagrams/backtrace/backtrace.tex}}
    \end{column}
  \end{columns}
\end{frame}

\begin{frame}{Backtraces}{\funcname{caml\_probably\_raise}'s handler doesn't catch \typename{ExnC}}
  \begin{columns}[c]
    \begin{column}{0.45\textwidth}
      \minipage[c][0.75\textheight][s]{\columnwidth}
      \begin{itemize}
        \item<1-> \funcname{match \dots with}\footnotemark pattern matching can also match exceptions
        \item<1-> The CMM of \funcname{probably\_raise} shows that this \funcname{match \dots with} is implemented with a pattern matching and a \funcname{try \dots with}
        \item<1-> But this one only matches on \typename{ExnA} exception
        \item<2-> Since the pattern matching fails to catch the exception, \funcname{reraise} is called with our current \typename{ExnC} exception
        \item<3-> Disassembly shows that \funcname{reraise} is actually a call to \funcname{caml\_reraise\_exn}, which is also an assembly function provided by the runtime
      \end{itemize}
      \vfill
      \mintocaml[firstline=15,lastline=21,highlightlines={18-21}]{../src/backtrace/backtrace.ml}
      \endminipage
    \end{column}
    \begin{column}{0.55\textwidth}
      \centering
      \onslide*<1>{\mintcmm[highlightlines={10-18}]{../src/backtrace/camlBacktrace\_\_probably\_raise\_351.cmm}}
      \onslide*<2>{\listcmm[highlightlines=18]{../src/backtrace/camlBacktrace\_\_probably\_raise_351.cmm}{CMM of probably\_raise}}
      \onslide*<3>{\mintobjdump[firstline=5,lastline=35,highlightlines=31]{../src/backtrace/camlBacktrace\_\_probably\_raise_351.objdump}}
    \end{column}
  \end{columns}
  \only<1>{\footnotetext[1]{https://blog.janestreet.com/pattern-matching-and-exception-handling-unite/}}
\end{frame}

\newcommand\listCamlReraiseExn[1][]{
  \listasm[firstline=727,lastline=734,#1]{../ocaml/runtime/amd64.S}{runtime/amd64.S:727}}

\begin{frame}{Backtraces}{Introducing \funcname{caml\_reraise\_exn}}
  \begin{columns}[c]
    \begin{column}{0.5\textwidth}
      \minipage[c][0.66\textheight][s]{\columnwidth}
      \begin{itemize}
        \item<1-> The exception have to be reraised to the next handler
        \item<2-> First, checks if the backtrace should be collected with \funcarg{domain\_state->backtrace\_active}
        \only<3>{
        \begin{itemize}
            \item If not, behaves like a \funcname{raise\_notrace} with \funcname{RESTORE\_EXN\_HANDLER\_OCAML}
        \end{itemize}
        }
        \item<4-> Jumps into \funcname{caml\_raise\_exn}
        \item<5-> Calls \funcname{caml\_stash\_backtrace} again, to scan the next part of the stack: from the trap just pop-ed to the next trap
      \end{itemize}
      \vfill
      \mintocaml[firstline=15,lastline=21,highlightlines={18-21}]{../src/backtrace/backtrace.ml}
      \endminipage
    \end{column}
    \begin{column}{0.5\textwidth}
      \raggedleft
      \onslide*<1>{\listCamlReraiseExn{}}
      \onslide*<2>{\listCamlReraiseExn[highlightlines=729]{}}
      \onslide*<3>{
        \mintc[firstline=190,lastline=192,highlightlines={191-192}]{../ocaml/runtime/amd64.S}
        \mintc[firstline=234,lastline=237,highlightlines={235-237}]{../ocaml/runtime/amd64.S}
        \medskip
        \listCamlReraiseExn[highlightlines=731]{}
      }
      \onslide*<4-5>{
        % TODO refactor with \listCamlReraiseExn
        \mintasm[firstline=727,lastline=734,highlightlines=730]{../ocaml/runtime/amd64.S}
        \medskip
      }
      \onslide*<4>{\listCamlRaiseExn[highlightlines=711]{}}
      \onslide*<5>{\listCamlRaiseExn[highlightlines=720]{}}
    \end{column}
  \end{columns}
\end{frame}

\begin{frame}{Backtraces}{Backtrace collection continues}
  \begin{columns}[c]
    \begin{column}{0.55\textwidth}
      \minipage[c][0.75\textheight][s]{\columnwidth}
      \begin{itemize}
        \item<1-> \funcname{caml\_stash\_backtrace} scans the older part of the stack, up to the next trap
        \item<6-> Controls jumps to the nested exception handler in \funcname{maybe\_raise}, which matches only on \typename{ExnB}
        \item<7-> \funcname{caml\_reraise\_exn} is called again, backtrace collection resumes with \funcname{caml\_stash\_backtrace}
      \end{itemize}
      \medskip
      \begin{itemize}
        \item<2-> Constructed backtrace:
        \begin{itemize}
          \item <2-> \funcname{definitely\_raise}
          \item <2-> \funcname{probably\_raise}
          \item <3-> \funcname{probably\_raise} (re-raise)
          \item <4-> \funcname{maybe\_raise}
          \item <5-> \funcname{main}
          \item <8-> \funcname{main} (re-raise)
        \end{itemize}
      \end{itemize}
      \vfill
      \onslide*<1-5>{\mintocaml[firstline=15,lastline=21]{../src/backtrace/backtrace.ml}}
      \onslide*<6>{\mintocaml[firstline=28,lastline=34,highlightlines=31]{../src/backtrace/backtrace.ml}}
      \onslide*<7->{\mintocaml[firstline=28,lastline=34,highlightlines=33]{../src/backtrace/backtrace.ml}}
      \endminipage
    \end{column}
    \begin{column}{0.45\textwidth}
      \raggedleft
      \onslide*<1>{\providecommand\step{6}\input{diagrams/backtrace/backtrace.tex}}%
      \onslide*<2>{\providecommand\step{6}\input{diagrams/backtrace/backtrace.tex}}%
      \onslide*<3>{\providecommand\step{7}\input{diagrams/backtrace/backtrace.tex}}%
      \onslide*<4>{\providecommand\step{8}\input{diagrams/backtrace/backtrace.tex}}%
      \onslide*<5>{\providecommand\step{9}\input{diagrams/backtrace/backtrace.tex}}%
      \onslide*<6>{\providecommand\step{10}\input{diagrams/backtrace/backtrace.tex}}%
      \onslide*<7>{\providecommand\step{11}\input{diagrams/backtrace/backtrace.tex}}%
      \onslide*<8>{\providecommand\step{12}\input{diagrams/backtrace/backtrace.tex}}%
      \onslide*<9>{\providecommand\step{13}\input{diagrams/backtrace/backtrace.tex}}%
    \end{column}
  \end{columns}
\end{frame}

\begin{frame}{Backtraces}{Printing the backtrace}
  \begin{columns}[c]
    \begin{column}{0.45\textwidth}
      \minipage[c][0.66\textheight][s]{\columnwidth}
      \begin{itemize}
        \item<1-> The exception backtrace can now be printed to \funcarg{stdout} with \funcname{Printexc.print_backtrace}
        \item<2-> First the exception backtrace is retrieved into an OCaml array with \funcname{get_raw_backtrace }
        \item<3-> Which is implemented as the \funcname{caml_get_exception_raw_backtrace} C function in the runtime
        \item<4-> And finally printed with \funcname{print_exception_backtrace}
      \end{itemize}
      \vfill
      \mintocaml[firstline=28,lastline=34,highlightlines=33]{../src/backtrace/backtrace.ml}
      \endminipage
    \end{column}
    \begin{column}{0.55\textwidth}
      \onslide*<1,2>{
        \mintocaml[firstline=100,lastline=101]{../ocaml/stdlib/printexc.ml}
      }
      \only<1>{
        \listocaml[firstline=170,lastline=172]{../ocaml/stdlib/printexc.ml}{stdlib/printexc.ml:170}
      }
      \only<2>{
        \listocaml[firstline=170,lastline=172,highlightlines=172]{../ocaml/stdlib/printexc.ml}{stdlib/printexc.ml:170}
      }
      \only<3>{
        \mintc[firstline=154,lastline=156]{../ocaml/runtime/backtrace.c}
        \listc[firstline=171,lastline=192,highlightlines={184-187}]{../ocaml/runtime/backtrace.c}{runtime/backtrace.c:154}
      }
      \only<4>{
        \listocaml[firstline=155,lastline=165]{../ocaml/stdlib/printexc.ml}{stdlib/printexc.ml:55}
      }
    \end{column}
  \end{columns}
\end{frame}


\subsection{The multiple kinds of raise}
\frameSubsection{}{}

\begin{frame}{Backtraces}{When backtraces are not needed}
  \begin{columns}[c]
    \begin{column}{0.45\textwidth}
      \minipage[c][0.66\textheight][s]{\columnwidth}
      \begin{itemize}
        \item<1-> What if the backtrace for an exception is never needed?
        \item<2-> It's possible to raise the exception explicitly with \funcname{raise\_notrace} to make raising as cheap as possible
        \item<3-> An example of use in the \funcname{dynlink} library of the OCaml compiler
      \end{itemize}
      \vfill
      \onslide<3->{
      \listocaml[firstline=205,lastline=209,highlightlines=207]{../ocaml/otherlibs/dynlink/dynlink\_compilerlibs/misc.ml}{otherlibs/dynlink/dynlink\_compilerlibs/misc.ml:205}
      }
      \endminipage
    \end{column}
    \begin{column}{0.55\textwidth}
    \only<4>{
      \mintcmm[highlightlines=3]{../src/notrace/camlNotrace\_\_fun_347.cmm}
      \listcmm{../src/notrace/camlNotrace\_\_all\_somes\_327.cmm}{all\_somes}
    }
    \onslide*<5->{
      \listobjdump[highlightlines={6-9}]{../src/notrace/camlNotrace\_\_fun_347.objdump}{anonymous function from \funcname{all\_somes}}
    }
    \end{column}
  \end{columns}
\end{frame}

\begin{frame}{Backtraces}{Summary table of the multiple kinds of raise}
  \begin{itemize}
    \item When debugging information is enabled and if the backtrace of an exception isn't needed, it's possible to raise with \funcname{raise\_notrace} for performance
    \item When debugging information is \emph{not} enabled, the compiler optimises \funcname{raise} calls to inline assembly (same effect as using \funcname{raise\_notrace})
  \end{itemize}
  \bigskip
  \centering
  \begin{tabular}{| l | c | c | c | c |}
    \hline
    \parbox{10em}{Debug information \\ (\funcarg{-g} flag)}  & \multicolumn{2}{|c|}{Disabled}                                     & \multicolumn{2}{|c|}{Enabled} \\
    \hline
    Raise kind                                               & \funcname{raise} & \funcname{raise\_notrace}                       & \funcname{raise} & \funcname{raise\_notrace} \\
    \hline
    Compilation                                              & \parbox{8em}{inline assembly\\ (optimization)} & inline assembly   & \funcname{caml\_raise\_exn} & inline assembly \\
    \hline
  \end{tabular}
\end{frame}


%
% Takeaway #5
%
\subsection*{Takeaway \#5}
\frameSubsectionTakeaway{}
\againframe<5>{takeaway}
}
    \end{column}
  \end{columns}
\end{frame}

\begin{frame}{Backtraces}{\funcname{caml\_probably\_raise}'s handler doesn't catch \typename{ExnC}}
  \begin{columns}[c]
    \begin{column}{0.45\textwidth}
      \minipage[c][0.75\textheight][s]{\columnwidth}
      \begin{itemize}
        \item<1-> \funcname{match \dots with}\footnotemark pattern matching can also match exceptions
        \item<1-> The CMM of \funcname{probably\_raise} shows that this \funcname{match \dots with} is implemented with a pattern matching and a \funcname{try \dots with}
        \item<1-> But this one only matches on \typename{ExnA} exception
        \item<2-> Since the pattern matching fails to catch the exception, \funcname{reraise} is called with our current \typename{ExnC} exception
        \item<3-> Disassembly shows that \funcname{reraise} is actually a call to \funcname{caml\_reraise\_exn}, which is also an assembly function provided by the runtime
      \end{itemize}
      \vfill
      \mintocaml[firstline=15,lastline=21,highlightlines={18-21}]{../src/backtrace/backtrace.ml}
      \endminipage
    \end{column}
    \begin{column}{0.55\textwidth}
      \centering
      \onslide*<1>{\mintcmm[highlightlines={10-18}]{../src/backtrace/camlBacktrace\_\_probably\_raise\_351.cmm}}
      \onslide*<2>{\listcmm[highlightlines=18]{../src/backtrace/camlBacktrace\_\_probably\_raise_351.cmm}{CMM of probably\_raise}}
      \onslide*<3>{\mintobjdump[firstline=5,lastline=35,highlightlines=31]{../src/backtrace/camlBacktrace\_\_probably\_raise_351.objdump}}
    \end{column}
  \end{columns}
  \only<1>{\footnotetext[1]{https://blog.janestreet.com/pattern-matching-and-exception-handling-unite/}}
\end{frame}

\newcommand\listCamlReraiseExn[1][]{
  \listasm[firstline=727,lastline=734,#1]{../ocaml/runtime/amd64.S}{runtime/amd64.S:727}}

\begin{frame}{Backtraces}{Introducing \funcname{caml\_reraise\_exn}}
  \begin{columns}[c]
    \begin{column}{0.5\textwidth}
      \minipage[c][0.66\textheight][s]{\columnwidth}
      \begin{itemize}
        \item<1-> The exception have to be reraised to the next handler
        \item<2-> First, checks if the backtrace should be collected with \funcarg{domain\_state->backtrace\_active}
        \only<3>{
        \begin{itemize}
            \item If not, behaves like a \funcname{raise\_notrace} with \funcname{RESTORE\_EXN\_HANDLER\_OCAML}
        \end{itemize}
        }
        \item<4-> Jumps into \funcname{caml\_raise\_exn}
        \item<5-> Calls \funcname{caml\_stash\_backtrace} again, to scan the next part of the stack: from the trap just pop-ed to the next trap
      \end{itemize}
      \vfill
      \mintocaml[firstline=15,lastline=21,highlightlines={18-21}]{../src/backtrace/backtrace.ml}
      \endminipage
    \end{column}
    \begin{column}{0.5\textwidth}
      \raggedleft
      \onslide*<1>{\listCamlReraiseExn{}}
      \onslide*<2>{\listCamlReraiseExn[highlightlines=729]{}}
      \onslide*<3>{
        \mintc[firstline=190,lastline=192,highlightlines={191-192}]{../ocaml/runtime/amd64.S}
        \mintc[firstline=234,lastline=237,highlightlines={235-237}]{../ocaml/runtime/amd64.S}
        \medskip
        \listCamlReraiseExn[highlightlines=731]{}
      }
      \onslide*<4-5>{
        % TODO refactor with \listCamlReraiseExn
        \mintasm[firstline=727,lastline=734,highlightlines=730]{../ocaml/runtime/amd64.S}
        \medskip
      }
      \onslide*<4>{\listCamlRaiseExn[highlightlines=711]{}}
      \onslide*<5>{\listCamlRaiseExn[highlightlines=720]{}}
    \end{column}
  \end{columns}
\end{frame}

\begin{frame}{Backtraces}{Backtrace collection continues}
  \begin{columns}[c]
    \begin{column}{0.55\textwidth}
      \minipage[c][0.75\textheight][s]{\columnwidth}
      \begin{itemize}
        \item<1-> \funcname{caml\_stash\_backtrace} scans the older part of the stack, up to the next trap
        \item<6-> Controls jumps to the nested exception handler in \funcname{maybe\_raise}, which matches only on \typename{ExnB}
        \item<7-> \funcname{caml\_reraise\_exn} is called again, backtrace collection resumes with \funcname{caml\_stash\_backtrace}
      \end{itemize}
      \medskip
      \begin{itemize}
        \item<2-> Constructed backtrace:
        \begin{itemize}
          \item <2-> \funcname{definitely\_raise}
          \item <2-> \funcname{probably\_raise}
          \item <3-> \funcname{probably\_raise} (re-raise)
          \item <4-> \funcname{maybe\_raise}
          \item <5-> \funcname{main}
          \item <8-> \funcname{main} (re-raise)
        \end{itemize}
      \end{itemize}
      \vfill
      \onslide*<1-5>{\mintocaml[firstline=15,lastline=21]{../src/backtrace/backtrace.ml}}
      \onslide*<6>{\mintocaml[firstline=28,lastline=34,highlightlines=31]{../src/backtrace/backtrace.ml}}
      \onslide*<7->{\mintocaml[firstline=28,lastline=34,highlightlines=33]{../src/backtrace/backtrace.ml}}
      \endminipage
    \end{column}
    \begin{column}{0.45\textwidth}
      \raggedleft
      \onslide*<1>{\providecommand\step{6}%
% Backtraces
%
\subsection{One advantage of raising with debugging information}
\frameSubsection{}{}

\begin{frame}{Backtraces}{Debugging information \& source code}
  \begin{columns}[c]
    \begin{column}{0.5\textwidth}
      \begin{itemize}
        \item Raising an exception is fun and games, but how do we know where the exception originated? $\rightarrow$ With the backtrace associated with the exception!
        \item In order to get a backtrace from the point where an exception was raised, it's necessary to compile with debugging information enabled (\funcarg{-g} flag for the compiler)
        \item Same example as in the `Nested handlers' section
      \end{itemize}
    \end{column}
    \begin{column}{0.5\textwidth}
      \listocaml[firstline=9,lastline=34]{../src/backtrace/backtrace.ml}{backtrace/backtrace.ml}
    \end{column}
  \end{columns}
\end{frame}

\begin{frame}{Backtraces}{CMM and disassembly of \funcname{definitely\_raise}}
  \begin{columns}[c]
    \begin{column}{0.5\textwidth}
      \minipage[c][0.66\textheight][s]{\columnwidth}
      \begin{itemize}
        \item<1-> An effect of compiling with debugging information (\funcarg{-g}): the CMM no longer uses \funcname{raise\_notrace} to raise the exception but \funcname{raise} instead
        \item<2-> Disassembly shows that CMM \funcname{raise} is actually a call to \funcname{caml\_raise\_exn}, a function in assembler provided by the runtime
      \end{itemize}
      \vfill
      \mintocaml[firstline=9,lastline=13,highlightlines=12]{../src/backtrace/backtrace.ml}
      \endminipage
    \end{column}
    \begin{column}{0.5\textwidth}
      \onslide*<1>{
        \mintcmm[highlightlines=5]{../src/backtrace/camlBacktrace\_\_definitely\_raise\_347.cmm}
      }
      \onslide*<2>{
        \mintobjdump[firstline=2,highlightlines=14]{../src/backtrace/camlBacktrace\_\_definitely\_raise\_347.objdump}
      }
    \end{column}
  \end{columns}
\end{frame}

\begin{frame}{Backtraces}{Introducing \funcname{caml\_raise\_exn}}
  \begin{columns}[c]
    \begin{column}{0.5\textwidth}
      \minipage[c][0.66\textheight][s]{\columnwidth}
      \onslide*<1>{
        \begin{itemize}
          \item Raising the exception is performed using \funcname{caml\_raise\_exn} which is
            \begin{itemize}
              \item An assembly function provided by the runtime
              \item Architecture specific
            \end{itemize}
          \item Let's see what it does differently from \funcname{raise\_notrace}\dots
          \item We are about to raise \typename{ExnC} with \funcname{caml\_raise\_exn} from \funcname{definitely\_raise}
        \end{itemize}
      }
      \onslide*<2>{
        \mintobjdump[firstline=2,highlightlines=14]{../src/backtrace/camlBacktrace\_\_definitely\_raise\_347.objdump}
      }
      \vfill
      \mintocaml[firstline=9,lastline=13,highlightlines=12]{../src/backtrace/backtrace.ml}
      \endminipage
    \end{column}
    \begin{column}{0.5\textwidth}
      \centering
      \providecommand\step{1}\input{diagrams/backtrace/backtrace.tex}
    \end{column}
  \end{columns}
\end{frame}

\begin{frame}{Backtraces}{But before that, we need more fields from \typename{caml\_domain\_state}}
  \begin{columns}
    \begin{column}{0.5\textwidth}
      \begin{itemize}
        \item Remember \typename{caml\_domain\_state} the per-domain runtime state?
        \item Some other members are of interest to us now\dots
        \item \localname{backtrace\_active} is used to know if the runtime should collect backtraces
        \item \localname{backtrace\_active} can be set by
          \begin{itemize}
            \item The \funcarg{b} flag in the \funcname{OCAMLRUNPARAM} environment variable
            \item \mintinline{ocaml}{Printexc.record_backtrace : bool -> unit} \footnotemark
          \end{itemize}
      \end{itemize}
    \end{column}
    \begin{column}{0.5\textwidth}
      \begin{listing}
        \mintc[highlightlines={9-11}]{src/ocaml/domain\_state\_backtrace.h}
        \caption{runtime/caml/domain\_state.h}
      \end{listing}
    \end{column}
  \end{columns}
  \footnotetext[1]{https://v2.ocaml.org/api/Printexc.html}
\end{frame}

\newcommand\listCamlRaiseExn[1][]{
  \listasm[firstline=702,lastline=725,#1]{../ocaml/runtime/amd64.S}{runtime/amd64.S:702}}

\begin{frame}{Backtraces}{Walk through \funcname{caml\_raise\_exn}}
  \begin{columns}[T]
    \begin{column}{0.5\textwidth}
      \begin{itemize}
        \item<1-> First checks whether exceptions backtrace should be recorded
        \item<1-> By checking the \funcarg{domain\_state->backtrace\_active} value
        \item<2-> If not, acts as \funcname{raise\_notrace} with \funcname{RESTORE\_EXN\_HANDLER\_OCAML}
          \begin{itemize}
            \item Set \regname{RSP} to \localname{domain\_state->exn\_handler} value
            \item Restore \localname{domain\_state->exn\_handler} from trap
            \item Jump with \funcname{ret} (same effect as using \funcname{pop} and \funcname{jmp})
          \end{itemize}
        \item<3-> Otherwise, switches to the C stack to be able to execute C code
        \item<4-> Calls \funcname{caml\_stash\_backtrace}, a C function provided by the runtime\ignorespaces
          \onslide<6->{, with
            \begin{itemize}
              \item \funcarg{exn}: exception value
              \item \funcarg{pc}: code address of the raising point
              \item \funcarg{sp}: stack address of the raising function
              \item \funcarg{trapsp}: stack address of the next handler
            \end{itemize}
          }
      \end{itemize}
      \bigskip
      \mintocaml[firstline=9,lastline=13,highlightlines=12]{../src/backtrace/backtrace.ml}
    \end{column}
    \begin{column}{0.5\textwidth}
      \raggedleft
      \onslide*<1,3-4>{
        \mintc[firstline=190,lastline=192]{../ocaml/runtime/amd64.S}
        \mintc[firstline=234,lastline=237]{../ocaml/runtime/amd64.S}
      }
      \onslide*<2>{
        \mintc[firstline=190,lastline=192,highlightlines={191-192}]{../ocaml/runtime/amd64.S}
        \mintc[firstline=234,lastline=237,highlightlines={235-237}]{../ocaml/runtime/amd64.S}
      }
      \onslide*<1>{\listCamlRaiseExn[highlightlines=705]{}}%
      \onslide*<2>{\listCamlRaiseExn[highlightlines={707-708}]{}}%
      \onslide*<3>{\listCamlRaiseExn[highlightlines={712-714}]{}}%
      \onslide*<4>{\listCamlRaiseExn[highlightlines={715-720}]{}}%
      \onslide*<5>{\providecommand\step{1}\input{diagrams/backtrace/backtrace.tex}}%
      \onslide*<6>{\providecommand\step{2}\input{diagrams/backtrace/backtrace.tex}}%
    \end{column}
  \end{columns}
\end{frame}

\newcommand\listStashBacktrace[1][]{
  \listc[firstline=91,lastline=120,#1]{../ocaml/runtime/backtrace\_nat.c}{runtime/backtrace\_nat.c:91}}

\begin{frame}{Backtraces}{Introducing \funcname{caml\_stash\_backtrace}}
  \begin{columns}[c]
    \begin{column}{0.5\textwidth}
      \minipage[c][0.66\textheight][s]{\columnwidth}
      \begin{itemize}
        \item<1-> A C function provided by the runtime (specific to native code)
        \item<2-> It scans the stack, between the stack pointer at the raising point up to the next trap, in order to collect the names of the detected function frames
          \onslide*<2>{
            \begin{itemize}
              \item And more information, like line number, column number...
            \end{itemize}
          }
        \item<3-> It uses the same technique and information as the Garbage Collector (GC) uses to scan the values live on the stack
          \onslide*<4>{
            \begin{itemize}
              \item \typename{frame\_descr} are at the core of stack unwinding
            \end{itemize}
          }
      \end{itemize}
      \vfill
      \mintocaml[firstline=9,lastline=13,highlightlines=12]{../src/backtrace/backtrace.ml}
      \endminipage
    \end{column}
    \begin{column}{0.5\textwidth}
      \onslide*<1>{\listStashBacktrace[highlightlines=91]{}}
      \onslide*<2>{\listStashBacktrace[highlightlines={91,117-118}]{}}
      \onslide*<3->{\listStashBacktrace[highlightlines={94,106,109-110}]{}}
    \end{column}
  \end{columns}
\end{frame}

\newcommand\listFrameDescr[1][]{
  \listocaml[firstline=111,lastline=124,#1]{../ocaml/asmcomp/emitaux.ml}{asmcomp/emitaux.ml:111}}
\newcommand\listDebugInfo[1][]{
  \listocaml[firstline=38,lastline=49,#1]{../ocaml/lambda/debuginfo.mli}{lambda/debuginfo.mli:38}}

\begin{frame}{Backtraces}{\typename{frame\_descr} and \typename{frame\_debuginfo} in a nutshell}
  \begin{columns}[c]
    \begin{column}{0.5\textwidth}
      \begin{itemize}
        \item<1-> For every return address in an OCaml program, there is a associated \typename{frame\_descr}
        \item<2-> A \typename{frame\_descr} specifies
        \begin{itemize}
          \item<2-> The size of the function stack frame
          \item<3-> Where live values are on the stack (so that the GC can mark them as live and prevent from being swept)
        \end{itemize}
        \item<4-> When compiled with debug information, \typename{frame\_descr} will be linked to a \typename{frame\_debuginfo}
        \begin{itemize}
          \item<5-> Contains information about the position in the source code (file name, line and column number)
        \end{itemize}
      \end{itemize}
    \end{column}
    \begin{column}{0.5\textwidth}
        \onslide*<1>{\listFrameDescr[highlightlines={118-122}]{}}
        \onslide*<2>{\listFrameDescr[highlightlines={120}]{}}
        \onslide*<3>{\listFrameDescr[highlightlines={121}]{}}
        \onslide*<4->{\listFrameDescr[highlightlines={122}]{}}
        \medskip
        \onslide*<1-3>{\listDebugInfo{}}
        \onslide*<4>{\listDebugInfo[highlightlines={38,49}]{}}
        \onslide*<5>{\listDebugInfo[highlightlines={39-46}]{}}
    \end{column}
  \end{columns}
\end{frame}

\begin{frame}{Backtraces}{Scanning the stack with \typename{frame\_descr}}
  \begin{columns}[c]
    \begin{column}{0.5\textwidth}
      \begin{itemize}
        \item Given the return address of the code that raises the exception by calling \funcname{caml\_raise\_exn} (\funcarg{pc} argument of \funcname{caml\_stash\_backtrace})
        \item And the position of the stack pointer (\funcarg{sp} argument of \funcname{caml\_stash\_backtrace})
        \item The frame size given by \typename{frame\_descr}.\funcarg{frame\_size}, allows to find the end of the previous frame
        \item And the return address of the caller of this function
        \bigskip
        \onslide<3->{
        \item Note that traps (here the reddish \funcname{ExnA}, \funcname{ExnB} and \funcname{ExnC} blocks) belong to the frame that installed them, and so the frame size changes when traps are removed
        }
      \end{itemize}
    \end{column}
    \begin{column}{0.5\textwidth}
      \raggedleft
      \onslide*<1>{\providecommand\step{2}\input{diagrams/backtrace/backtrace.tex}}%
      \onslide*<2>{\providecommand\step{1}\input{diagrams/backtrace/scanstack.tex}}%
      \onslide*<3>{\providecommand\step{2}\input{diagrams/backtrace/scanstack.tex}}%
      \onslide*<4>{\providecommand\step{3}\input{diagrams/backtrace/scanstack.tex}}%
      \onslide*<5>{\providecommand\step{4}\input{diagrams/backtrace/scanstack.tex}}%
      \onslide*<6>{\providecommand\step{5}\input{diagrams/backtrace/scanstack.tex}}%
    \end{column}
  \end{columns}
\end{frame}

% Duplicate of previous-previous slide
\begin{frame}{Backtraces}{Continuing on \funcname{caml\_stash\_backtrace}}
  \begin{columns}[c]
    \begin{column}{0.5\textwidth}
      \minipage[c][0.75\textheight][s]{\columnwidth}
      \begin{itemize}
          \color{gray}
        \item A C function provided by the runtime (specific to native code)
        \item It scans the stack, between the stack pointer at the raising point up to the next trap, in order to collect the names of the detected function frames
        \item It uses the same technique and information as the Garbage Collector (GC) uses to scan the values live on the stack
      \end{itemize}
      \begin{itemize}
        \item For each function frame detected, store a pointer to the \typename{frame\_descr} in the \localname{domain\_state->backtrace\_buffer} array, effectively constructing the backtrace
      \end{itemize}
      \onslide<2->{
        \begin{itemize}
            \smallskip
          \item Constructed backtrace:
            \funcname{definitely\_raise},
            \onslide<3->{\funcname{probably\_raise}}
        \end{itemize}
      }
      \vfill
      \mintocaml[firstline=9,lastline=13,highlightlines=12]{../src/backtrace/backtrace.ml}
      \endminipage
    \end{column}
    \begin{column}{0.5\textwidth}
      \raggedleft
      \onslide*<1>{\listStashBacktrace[highlightlines={114-115}]{}}
      \onslide*<2>{\providecommand\step{3}\input{diagrams/backtrace/backtrace.tex}}%
      \onslide*<3>{\providecommand\step{4}\input{diagrams/backtrace/backtrace.tex}}%
    \end{column}
  \end{columns}
\end{frame}

\begin{frame}{Backtraces}{Back to \funcname{caml\_raise\_exn}}
  \begin{columns}[c]
    \begin{column}{0.5\textwidth}
      \minipage[c][0.66\textheight][s]{\columnwidth}
      \begin{itemize}\color{gray}
        \item Checks wheteher exceptions backtrace should be recorded, by checking the \funcarg{domain\_state->backtrace\_active} value
        \item Switches to the C stack to be able to execute C code
        \item Calls \funcname{caml\_stash\_backtrace}, to collect the backtrace up to the next trap
      \end{itemize}
      \onslide*<2->{
        \begin{itemize}
          \item<2-> Executes the exception handler in \funcname{probably\_raise} with \funcname{RESTORE\_EXN\_HANDLER\_OCAML}
            \begin{itemize}
              \item Set \regname{RSP} to \localname{domain\_state->exn\_handler} value
              \item Restore \localname{domain\_state->exn\_handler} from trap
              \item Jump to the handler code with \funcname{ret}
            \end{itemize}
        \end{itemize}
      }
      \vfill
      \onslide*<1>{\mintocaml[firstline=9,lastline=13,highlightlines=12]{../src/backtrace/backtrace.ml}}
      \onslide*<2->{\mintocaml[firstline=15,lastline=21,highlightlines={19-21}]{../src/backtrace/backtrace.ml}}
      \endminipage
    \end{column}
    \begin{column}{0.5\textwidth}
      \centering
      \onslide*<1>{
        \mintc[firstline=190,lastline=192]{../ocaml/runtime/amd64.S}
        \mintc[firstline=234,lastline=237]{../ocaml/runtime/amd64.S}
      }
      \onslide*<2>{
        \mintc[firstline=190,lastline=192,highlightlines={191-192}]{../ocaml/runtime/amd64.S}
        \mintc[firstline=234,lastline=237,highlightlines={235-237}]{../ocaml/runtime/amd64.S}
      }
      \onslide*<1>{\listCamlRaiseExn[highlightlines=720]{}}
      \onslide*<2>{\listCamlRaiseExn[highlightlines=722]{}}
      \onslide*<3->{\providecommand\step{5}\input{diagrams/backtrace/backtrace.tex}}
    \end{column}
  \end{columns}
\end{frame}

\begin{frame}{Backtraces}{\funcname{caml\_probably\_raise}'s handler doesn't catch \typename{ExnC}}
  \begin{columns}[c]
    \begin{column}{0.45\textwidth}
      \minipage[c][0.75\textheight][s]{\columnwidth}
      \begin{itemize}
        \item<1-> \funcname{match \dots with}\footnotemark pattern matching can also match exceptions
        \item<1-> The CMM of \funcname{probably\_raise} shows that this \funcname{match \dots with} is implemented with a pattern matching and a \funcname{try \dots with}
        \item<1-> But this one only matches on \typename{ExnA} exception
        \item<2-> Since the pattern matching fails to catch the exception, \funcname{reraise} is called with our current \typename{ExnC} exception
        \item<3-> Disassembly shows that \funcname{reraise} is actually a call to \funcname{caml\_reraise\_exn}, which is also an assembly function provided by the runtime
      \end{itemize}
      \vfill
      \mintocaml[firstline=15,lastline=21,highlightlines={18-21}]{../src/backtrace/backtrace.ml}
      \endminipage
    \end{column}
    \begin{column}{0.55\textwidth}
      \centering
      \onslide*<1>{\mintcmm[highlightlines={10-18}]{../src/backtrace/camlBacktrace\_\_probably\_raise\_351.cmm}}
      \onslide*<2>{\listcmm[highlightlines=18]{../src/backtrace/camlBacktrace\_\_probably\_raise_351.cmm}{CMM of probably\_raise}}
      \onslide*<3>{\mintobjdump[firstline=5,lastline=35,highlightlines=31]{../src/backtrace/camlBacktrace\_\_probably\_raise_351.objdump}}
    \end{column}
  \end{columns}
  \only<1>{\footnotetext[1]{https://blog.janestreet.com/pattern-matching-and-exception-handling-unite/}}
\end{frame}

\newcommand\listCamlReraiseExn[1][]{
  \listasm[firstline=727,lastline=734,#1]{../ocaml/runtime/amd64.S}{runtime/amd64.S:727}}

\begin{frame}{Backtraces}{Introducing \funcname{caml\_reraise\_exn}}
  \begin{columns}[c]
    \begin{column}{0.5\textwidth}
      \minipage[c][0.66\textheight][s]{\columnwidth}
      \begin{itemize}
        \item<1-> The exception have to be reraised to the next handler
        \item<2-> First, checks if the backtrace should be collected with \funcarg{domain\_state->backtrace\_active}
        \only<3>{
        \begin{itemize}
            \item If not, behaves like a \funcname{raise\_notrace} with \funcname{RESTORE\_EXN\_HANDLER\_OCAML}
        \end{itemize}
        }
        \item<4-> Jumps into \funcname{caml\_raise\_exn}
        \item<5-> Calls \funcname{caml\_stash\_backtrace} again, to scan the next part of the stack: from the trap just pop-ed to the next trap
      \end{itemize}
      \vfill
      \mintocaml[firstline=15,lastline=21,highlightlines={18-21}]{../src/backtrace/backtrace.ml}
      \endminipage
    \end{column}
    \begin{column}{0.5\textwidth}
      \raggedleft
      \onslide*<1>{\listCamlReraiseExn{}}
      \onslide*<2>{\listCamlReraiseExn[highlightlines=729]{}}
      \onslide*<3>{
        \mintc[firstline=190,lastline=192,highlightlines={191-192}]{../ocaml/runtime/amd64.S}
        \mintc[firstline=234,lastline=237,highlightlines={235-237}]{../ocaml/runtime/amd64.S}
        \medskip
        \listCamlReraiseExn[highlightlines=731]{}
      }
      \onslide*<4-5>{
        % TODO refactor with \listCamlReraiseExn
        \mintasm[firstline=727,lastline=734,highlightlines=730]{../ocaml/runtime/amd64.S}
        \medskip
      }
      \onslide*<4>{\listCamlRaiseExn[highlightlines=711]{}}
      \onslide*<5>{\listCamlRaiseExn[highlightlines=720]{}}
    \end{column}
  \end{columns}
\end{frame}

\begin{frame}{Backtraces}{Backtrace collection continues}
  \begin{columns}[c]
    \begin{column}{0.55\textwidth}
      \minipage[c][0.75\textheight][s]{\columnwidth}
      \begin{itemize}
        \item<1-> \funcname{caml\_stash\_backtrace} scans the older part of the stack, up to the next trap
        \item<6-> Controls jumps to the nested exception handler in \funcname{maybe\_raise}, which matches only on \typename{ExnB}
        \item<7-> \funcname{caml\_reraise\_exn} is called again, backtrace collection resumes with \funcname{caml\_stash\_backtrace}
      \end{itemize}
      \medskip
      \begin{itemize}
        \item<2-> Constructed backtrace:
        \begin{itemize}
          \item <2-> \funcname{definitely\_raise}
          \item <2-> \funcname{probably\_raise}
          \item <3-> \funcname{probably\_raise} (re-raise)
          \item <4-> \funcname{maybe\_raise}
          \item <5-> \funcname{main}
          \item <8-> \funcname{main} (re-raise)
        \end{itemize}
      \end{itemize}
      \vfill
      \onslide*<1-5>{\mintocaml[firstline=15,lastline=21]{../src/backtrace/backtrace.ml}}
      \onslide*<6>{\mintocaml[firstline=28,lastline=34,highlightlines=31]{../src/backtrace/backtrace.ml}}
      \onslide*<7->{\mintocaml[firstline=28,lastline=34,highlightlines=33]{../src/backtrace/backtrace.ml}}
      \endminipage
    \end{column}
    \begin{column}{0.45\textwidth}
      \raggedleft
      \onslide*<1>{\providecommand\step{6}\input{diagrams/backtrace/backtrace.tex}}%
      \onslide*<2>{\providecommand\step{6}\input{diagrams/backtrace/backtrace.tex}}%
      \onslide*<3>{\providecommand\step{7}\input{diagrams/backtrace/backtrace.tex}}%
      \onslide*<4>{\providecommand\step{8}\input{diagrams/backtrace/backtrace.tex}}%
      \onslide*<5>{\providecommand\step{9}\input{diagrams/backtrace/backtrace.tex}}%
      \onslide*<6>{\providecommand\step{10}\input{diagrams/backtrace/backtrace.tex}}%
      \onslide*<7>{\providecommand\step{11}\input{diagrams/backtrace/backtrace.tex}}%
      \onslide*<8>{\providecommand\step{12}\input{diagrams/backtrace/backtrace.tex}}%
      \onslide*<9>{\providecommand\step{13}\input{diagrams/backtrace/backtrace.tex}}%
    \end{column}
  \end{columns}
\end{frame}

\begin{frame}{Backtraces}{Printing the backtrace}
  \begin{columns}[c]
    \begin{column}{0.45\textwidth}
      \minipage[c][0.66\textheight][s]{\columnwidth}
      \begin{itemize}
        \item<1-> The exception backtrace can now be printed to \funcarg{stdout} with \funcname{Printexc.print_backtrace}
        \item<2-> First the exception backtrace is retrieved into an OCaml array with \funcname{get_raw_backtrace }
        \item<3-> Which is implemented as the \funcname{caml_get_exception_raw_backtrace} C function in the runtime
        \item<4-> And finally printed with \funcname{print_exception_backtrace}
      \end{itemize}
      \vfill
      \mintocaml[firstline=28,lastline=34,highlightlines=33]{../src/backtrace/backtrace.ml}
      \endminipage
    \end{column}
    \begin{column}{0.55\textwidth}
      \onslide*<1,2>{
        \mintocaml[firstline=100,lastline=101]{../ocaml/stdlib/printexc.ml}
      }
      \only<1>{
        \listocaml[firstline=170,lastline=172]{../ocaml/stdlib/printexc.ml}{stdlib/printexc.ml:170}
      }
      \only<2>{
        \listocaml[firstline=170,lastline=172,highlightlines=172]{../ocaml/stdlib/printexc.ml}{stdlib/printexc.ml:170}
      }
      \only<3>{
        \mintc[firstline=154,lastline=156]{../ocaml/runtime/backtrace.c}
        \listc[firstline=171,lastline=192,highlightlines={184-187}]{../ocaml/runtime/backtrace.c}{runtime/backtrace.c:154}
      }
      \only<4>{
        \listocaml[firstline=155,lastline=165]{../ocaml/stdlib/printexc.ml}{stdlib/printexc.ml:55}
      }
    \end{column}
  \end{columns}
\end{frame}


\subsection{The multiple kinds of raise}
\frameSubsection{}{}

\begin{frame}{Backtraces}{When backtraces are not needed}
  \begin{columns}[c]
    \begin{column}{0.45\textwidth}
      \minipage[c][0.66\textheight][s]{\columnwidth}
      \begin{itemize}
        \item<1-> What if the backtrace for an exception is never needed?
        \item<2-> It's possible to raise the exception explicitly with \funcname{raise\_notrace} to make raising as cheap as possible
        \item<3-> An example of use in the \funcname{dynlink} library of the OCaml compiler
      \end{itemize}
      \vfill
      \onslide<3->{
      \listocaml[firstline=205,lastline=209,highlightlines=207]{../ocaml/otherlibs/dynlink/dynlink\_compilerlibs/misc.ml}{otherlibs/dynlink/dynlink\_compilerlibs/misc.ml:205}
      }
      \endminipage
    \end{column}
    \begin{column}{0.55\textwidth}
    \only<4>{
      \mintcmm[highlightlines=3]{../src/notrace/camlNotrace\_\_fun_347.cmm}
      \listcmm{../src/notrace/camlNotrace\_\_all\_somes\_327.cmm}{all\_somes}
    }
    \onslide*<5->{
      \listobjdump[highlightlines={6-9}]{../src/notrace/camlNotrace\_\_fun_347.objdump}{anonymous function from \funcname{all\_somes}}
    }
    \end{column}
  \end{columns}
\end{frame}

\begin{frame}{Backtraces}{Summary table of the multiple kinds of raise}
  \begin{itemize}
    \item When debugging information is enabled and if the backtrace of an exception isn't needed, it's possible to raise with \funcname{raise\_notrace} for performance
    \item When debugging information is \emph{not} enabled, the compiler optimises \funcname{raise} calls to inline assembly (same effect as using \funcname{raise\_notrace})
  \end{itemize}
  \bigskip
  \centering
  \begin{tabular}{| l | c | c | c | c |}
    \hline
    \parbox{10em}{Debug information \\ (\funcarg{-g} flag)}  & \multicolumn{2}{|c|}{Disabled}                                     & \multicolumn{2}{|c|}{Enabled} \\
    \hline
    Raise kind                                               & \funcname{raise} & \funcname{raise\_notrace}                       & \funcname{raise} & \funcname{raise\_notrace} \\
    \hline
    Compilation                                              & \parbox{8em}{inline assembly\\ (optimization)} & inline assembly   & \funcname{caml\_raise\_exn} & inline assembly \\
    \hline
  \end{tabular}
\end{frame}


%
% Takeaway #5
%
\subsection*{Takeaway \#5}
\frameSubsectionTakeaway{}
\againframe<5>{takeaway}
}%
      \onslide*<2>{\providecommand\step{6}%
% Backtraces
%
\subsection{One advantage of raising with debugging information}
\frameSubsection{}{}

\begin{frame}{Backtraces}{Debugging information \& source code}
  \begin{columns}[c]
    \begin{column}{0.5\textwidth}
      \begin{itemize}
        \item Raising an exception is fun and games, but how do we know where the exception originated? $\rightarrow$ With the backtrace associated with the exception!
        \item In order to get a backtrace from the point where an exception was raised, it's necessary to compile with debugging information enabled (\funcarg{-g} flag for the compiler)
        \item Same example as in the `Nested handlers' section
      \end{itemize}
    \end{column}
    \begin{column}{0.5\textwidth}
      \listocaml[firstline=9,lastline=34]{../src/backtrace/backtrace.ml}{backtrace/backtrace.ml}
    \end{column}
  \end{columns}
\end{frame}

\begin{frame}{Backtraces}{CMM and disassembly of \funcname{definitely\_raise}}
  \begin{columns}[c]
    \begin{column}{0.5\textwidth}
      \minipage[c][0.66\textheight][s]{\columnwidth}
      \begin{itemize}
        \item<1-> An effect of compiling with debugging information (\funcarg{-g}): the CMM no longer uses \funcname{raise\_notrace} to raise the exception but \funcname{raise} instead
        \item<2-> Disassembly shows that CMM \funcname{raise} is actually a call to \funcname{caml\_raise\_exn}, a function in assembler provided by the runtime
      \end{itemize}
      \vfill
      \mintocaml[firstline=9,lastline=13,highlightlines=12]{../src/backtrace/backtrace.ml}
      \endminipage
    \end{column}
    \begin{column}{0.5\textwidth}
      \onslide*<1>{
        \mintcmm[highlightlines=5]{../src/backtrace/camlBacktrace\_\_definitely\_raise\_347.cmm}
      }
      \onslide*<2>{
        \mintobjdump[firstline=2,highlightlines=14]{../src/backtrace/camlBacktrace\_\_definitely\_raise\_347.objdump}
      }
    \end{column}
  \end{columns}
\end{frame}

\begin{frame}{Backtraces}{Introducing \funcname{caml\_raise\_exn}}
  \begin{columns}[c]
    \begin{column}{0.5\textwidth}
      \minipage[c][0.66\textheight][s]{\columnwidth}
      \onslide*<1>{
        \begin{itemize}
          \item Raising the exception is performed using \funcname{caml\_raise\_exn} which is
            \begin{itemize}
              \item An assembly function provided by the runtime
              \item Architecture specific
            \end{itemize}
          \item Let's see what it does differently from \funcname{raise\_notrace}\dots
          \item We are about to raise \typename{ExnC} with \funcname{caml\_raise\_exn} from \funcname{definitely\_raise}
        \end{itemize}
      }
      \onslide*<2>{
        \mintobjdump[firstline=2,highlightlines=14]{../src/backtrace/camlBacktrace\_\_definitely\_raise\_347.objdump}
      }
      \vfill
      \mintocaml[firstline=9,lastline=13,highlightlines=12]{../src/backtrace/backtrace.ml}
      \endminipage
    \end{column}
    \begin{column}{0.5\textwidth}
      \centering
      \providecommand\step{1}\input{diagrams/backtrace/backtrace.tex}
    \end{column}
  \end{columns}
\end{frame}

\begin{frame}{Backtraces}{But before that, we need more fields from \typename{caml\_domain\_state}}
  \begin{columns}
    \begin{column}{0.5\textwidth}
      \begin{itemize}
        \item Remember \typename{caml\_domain\_state} the per-domain runtime state?
        \item Some other members are of interest to us now\dots
        \item \localname{backtrace\_active} is used to know if the runtime should collect backtraces
        \item \localname{backtrace\_active} can be set by
          \begin{itemize}
            \item The \funcarg{b} flag in the \funcname{OCAMLRUNPARAM} environment variable
            \item \mintinline{ocaml}{Printexc.record_backtrace : bool -> unit} \footnotemark
          \end{itemize}
      \end{itemize}
    \end{column}
    \begin{column}{0.5\textwidth}
      \begin{listing}
        \mintc[highlightlines={9-11}]{src/ocaml/domain\_state\_backtrace.h}
        \caption{runtime/caml/domain\_state.h}
      \end{listing}
    \end{column}
  \end{columns}
  \footnotetext[1]{https://v2.ocaml.org/api/Printexc.html}
\end{frame}

\newcommand\listCamlRaiseExn[1][]{
  \listasm[firstline=702,lastline=725,#1]{../ocaml/runtime/amd64.S}{runtime/amd64.S:702}}

\begin{frame}{Backtraces}{Walk through \funcname{caml\_raise\_exn}}
  \begin{columns}[T]
    \begin{column}{0.5\textwidth}
      \begin{itemize}
        \item<1-> First checks whether exceptions backtrace should be recorded
        \item<1-> By checking the \funcarg{domain\_state->backtrace\_active} value
        \item<2-> If not, acts as \funcname{raise\_notrace} with \funcname{RESTORE\_EXN\_HANDLER\_OCAML}
          \begin{itemize}
            \item Set \regname{RSP} to \localname{domain\_state->exn\_handler} value
            \item Restore \localname{domain\_state->exn\_handler} from trap
            \item Jump with \funcname{ret} (same effect as using \funcname{pop} and \funcname{jmp})
          \end{itemize}
        \item<3-> Otherwise, switches to the C stack to be able to execute C code
        \item<4-> Calls \funcname{caml\_stash\_backtrace}, a C function provided by the runtime\ignorespaces
          \onslide<6->{, with
            \begin{itemize}
              \item \funcarg{exn}: exception value
              \item \funcarg{pc}: code address of the raising point
              \item \funcarg{sp}: stack address of the raising function
              \item \funcarg{trapsp}: stack address of the next handler
            \end{itemize}
          }
      \end{itemize}
      \bigskip
      \mintocaml[firstline=9,lastline=13,highlightlines=12]{../src/backtrace/backtrace.ml}
    \end{column}
    \begin{column}{0.5\textwidth}
      \raggedleft
      \onslide*<1,3-4>{
        \mintc[firstline=190,lastline=192]{../ocaml/runtime/amd64.S}
        \mintc[firstline=234,lastline=237]{../ocaml/runtime/amd64.S}
      }
      \onslide*<2>{
        \mintc[firstline=190,lastline=192,highlightlines={191-192}]{../ocaml/runtime/amd64.S}
        \mintc[firstline=234,lastline=237,highlightlines={235-237}]{../ocaml/runtime/amd64.S}
      }
      \onslide*<1>{\listCamlRaiseExn[highlightlines=705]{}}%
      \onslide*<2>{\listCamlRaiseExn[highlightlines={707-708}]{}}%
      \onslide*<3>{\listCamlRaiseExn[highlightlines={712-714}]{}}%
      \onslide*<4>{\listCamlRaiseExn[highlightlines={715-720}]{}}%
      \onslide*<5>{\providecommand\step{1}\input{diagrams/backtrace/backtrace.tex}}%
      \onslide*<6>{\providecommand\step{2}\input{diagrams/backtrace/backtrace.tex}}%
    \end{column}
  \end{columns}
\end{frame}

\newcommand\listStashBacktrace[1][]{
  \listc[firstline=91,lastline=120,#1]{../ocaml/runtime/backtrace\_nat.c}{runtime/backtrace\_nat.c:91}}

\begin{frame}{Backtraces}{Introducing \funcname{caml\_stash\_backtrace}}
  \begin{columns}[c]
    \begin{column}{0.5\textwidth}
      \minipage[c][0.66\textheight][s]{\columnwidth}
      \begin{itemize}
        \item<1-> A C function provided by the runtime (specific to native code)
        \item<2-> It scans the stack, between the stack pointer at the raising point up to the next trap, in order to collect the names of the detected function frames
          \onslide*<2>{
            \begin{itemize}
              \item And more information, like line number, column number...
            \end{itemize}
          }
        \item<3-> It uses the same technique and information as the Garbage Collector (GC) uses to scan the values live on the stack
          \onslide*<4>{
            \begin{itemize}
              \item \typename{frame\_descr} are at the core of stack unwinding
            \end{itemize}
          }
      \end{itemize}
      \vfill
      \mintocaml[firstline=9,lastline=13,highlightlines=12]{../src/backtrace/backtrace.ml}
      \endminipage
    \end{column}
    \begin{column}{0.5\textwidth}
      \onslide*<1>{\listStashBacktrace[highlightlines=91]{}}
      \onslide*<2>{\listStashBacktrace[highlightlines={91,117-118}]{}}
      \onslide*<3->{\listStashBacktrace[highlightlines={94,106,109-110}]{}}
    \end{column}
  \end{columns}
\end{frame}

\newcommand\listFrameDescr[1][]{
  \listocaml[firstline=111,lastline=124,#1]{../ocaml/asmcomp/emitaux.ml}{asmcomp/emitaux.ml:111}}
\newcommand\listDebugInfo[1][]{
  \listocaml[firstline=38,lastline=49,#1]{../ocaml/lambda/debuginfo.mli}{lambda/debuginfo.mli:38}}

\begin{frame}{Backtraces}{\typename{frame\_descr} and \typename{frame\_debuginfo} in a nutshell}
  \begin{columns}[c]
    \begin{column}{0.5\textwidth}
      \begin{itemize}
        \item<1-> For every return address in an OCaml program, there is a associated \typename{frame\_descr}
        \item<2-> A \typename{frame\_descr} specifies
        \begin{itemize}
          \item<2-> The size of the function stack frame
          \item<3-> Where live values are on the stack (so that the GC can mark them as live and prevent from being swept)
        \end{itemize}
        \item<4-> When compiled with debug information, \typename{frame\_descr} will be linked to a \typename{frame\_debuginfo}
        \begin{itemize}
          \item<5-> Contains information about the position in the source code (file name, line and column number)
        \end{itemize}
      \end{itemize}
    \end{column}
    \begin{column}{0.5\textwidth}
        \onslide*<1>{\listFrameDescr[highlightlines={118-122}]{}}
        \onslide*<2>{\listFrameDescr[highlightlines={120}]{}}
        \onslide*<3>{\listFrameDescr[highlightlines={121}]{}}
        \onslide*<4->{\listFrameDescr[highlightlines={122}]{}}
        \medskip
        \onslide*<1-3>{\listDebugInfo{}}
        \onslide*<4>{\listDebugInfo[highlightlines={38,49}]{}}
        \onslide*<5>{\listDebugInfo[highlightlines={39-46}]{}}
    \end{column}
  \end{columns}
\end{frame}

\begin{frame}{Backtraces}{Scanning the stack with \typename{frame\_descr}}
  \begin{columns}[c]
    \begin{column}{0.5\textwidth}
      \begin{itemize}
        \item Given the return address of the code that raises the exception by calling \funcname{caml\_raise\_exn} (\funcarg{pc} argument of \funcname{caml\_stash\_backtrace})
        \item And the position of the stack pointer (\funcarg{sp} argument of \funcname{caml\_stash\_backtrace})
        \item The frame size given by \typename{frame\_descr}.\funcarg{frame\_size}, allows to find the end of the previous frame
        \item And the return address of the caller of this function
        \bigskip
        \onslide<3->{
        \item Note that traps (here the reddish \funcname{ExnA}, \funcname{ExnB} and \funcname{ExnC} blocks) belong to the frame that installed them, and so the frame size changes when traps are removed
        }
      \end{itemize}
    \end{column}
    \begin{column}{0.5\textwidth}
      \raggedleft
      \onslide*<1>{\providecommand\step{2}\input{diagrams/backtrace/backtrace.tex}}%
      \onslide*<2>{\providecommand\step{1}\input{diagrams/backtrace/scanstack.tex}}%
      \onslide*<3>{\providecommand\step{2}\input{diagrams/backtrace/scanstack.tex}}%
      \onslide*<4>{\providecommand\step{3}\input{diagrams/backtrace/scanstack.tex}}%
      \onslide*<5>{\providecommand\step{4}\input{diagrams/backtrace/scanstack.tex}}%
      \onslide*<6>{\providecommand\step{5}\input{diagrams/backtrace/scanstack.tex}}%
    \end{column}
  \end{columns}
\end{frame}

% Duplicate of previous-previous slide
\begin{frame}{Backtraces}{Continuing on \funcname{caml\_stash\_backtrace}}
  \begin{columns}[c]
    \begin{column}{0.5\textwidth}
      \minipage[c][0.75\textheight][s]{\columnwidth}
      \begin{itemize}
          \color{gray}
        \item A C function provided by the runtime (specific to native code)
        \item It scans the stack, between the stack pointer at the raising point up to the next trap, in order to collect the names of the detected function frames
        \item It uses the same technique and information as the Garbage Collector (GC) uses to scan the values live on the stack
      \end{itemize}
      \begin{itemize}
        \item For each function frame detected, store a pointer to the \typename{frame\_descr} in the \localname{domain\_state->backtrace\_buffer} array, effectively constructing the backtrace
      \end{itemize}
      \onslide<2->{
        \begin{itemize}
            \smallskip
          \item Constructed backtrace:
            \funcname{definitely\_raise},
            \onslide<3->{\funcname{probably\_raise}}
        \end{itemize}
      }
      \vfill
      \mintocaml[firstline=9,lastline=13,highlightlines=12]{../src/backtrace/backtrace.ml}
      \endminipage
    \end{column}
    \begin{column}{0.5\textwidth}
      \raggedleft
      \onslide*<1>{\listStashBacktrace[highlightlines={114-115}]{}}
      \onslide*<2>{\providecommand\step{3}\input{diagrams/backtrace/backtrace.tex}}%
      \onslide*<3>{\providecommand\step{4}\input{diagrams/backtrace/backtrace.tex}}%
    \end{column}
  \end{columns}
\end{frame}

\begin{frame}{Backtraces}{Back to \funcname{caml\_raise\_exn}}
  \begin{columns}[c]
    \begin{column}{0.5\textwidth}
      \minipage[c][0.66\textheight][s]{\columnwidth}
      \begin{itemize}\color{gray}
        \item Checks wheteher exceptions backtrace should be recorded, by checking the \funcarg{domain\_state->backtrace\_active} value
        \item Switches to the C stack to be able to execute C code
        \item Calls \funcname{caml\_stash\_backtrace}, to collect the backtrace up to the next trap
      \end{itemize}
      \onslide*<2->{
        \begin{itemize}
          \item<2-> Executes the exception handler in \funcname{probably\_raise} with \funcname{RESTORE\_EXN\_HANDLER\_OCAML}
            \begin{itemize}
              \item Set \regname{RSP} to \localname{domain\_state->exn\_handler} value
              \item Restore \localname{domain\_state->exn\_handler} from trap
              \item Jump to the handler code with \funcname{ret}
            \end{itemize}
        \end{itemize}
      }
      \vfill
      \onslide*<1>{\mintocaml[firstline=9,lastline=13,highlightlines=12]{../src/backtrace/backtrace.ml}}
      \onslide*<2->{\mintocaml[firstline=15,lastline=21,highlightlines={19-21}]{../src/backtrace/backtrace.ml}}
      \endminipage
    \end{column}
    \begin{column}{0.5\textwidth}
      \centering
      \onslide*<1>{
        \mintc[firstline=190,lastline=192]{../ocaml/runtime/amd64.S}
        \mintc[firstline=234,lastline=237]{../ocaml/runtime/amd64.S}
      }
      \onslide*<2>{
        \mintc[firstline=190,lastline=192,highlightlines={191-192}]{../ocaml/runtime/amd64.S}
        \mintc[firstline=234,lastline=237,highlightlines={235-237}]{../ocaml/runtime/amd64.S}
      }
      \onslide*<1>{\listCamlRaiseExn[highlightlines=720]{}}
      \onslide*<2>{\listCamlRaiseExn[highlightlines=722]{}}
      \onslide*<3->{\providecommand\step{5}\input{diagrams/backtrace/backtrace.tex}}
    \end{column}
  \end{columns}
\end{frame}

\begin{frame}{Backtraces}{\funcname{caml\_probably\_raise}'s handler doesn't catch \typename{ExnC}}
  \begin{columns}[c]
    \begin{column}{0.45\textwidth}
      \minipage[c][0.75\textheight][s]{\columnwidth}
      \begin{itemize}
        \item<1-> \funcname{match \dots with}\footnotemark pattern matching can also match exceptions
        \item<1-> The CMM of \funcname{probably\_raise} shows that this \funcname{match \dots with} is implemented with a pattern matching and a \funcname{try \dots with}
        \item<1-> But this one only matches on \typename{ExnA} exception
        \item<2-> Since the pattern matching fails to catch the exception, \funcname{reraise} is called with our current \typename{ExnC} exception
        \item<3-> Disassembly shows that \funcname{reraise} is actually a call to \funcname{caml\_reraise\_exn}, which is also an assembly function provided by the runtime
      \end{itemize}
      \vfill
      \mintocaml[firstline=15,lastline=21,highlightlines={18-21}]{../src/backtrace/backtrace.ml}
      \endminipage
    \end{column}
    \begin{column}{0.55\textwidth}
      \centering
      \onslide*<1>{\mintcmm[highlightlines={10-18}]{../src/backtrace/camlBacktrace\_\_probably\_raise\_351.cmm}}
      \onslide*<2>{\listcmm[highlightlines=18]{../src/backtrace/camlBacktrace\_\_probably\_raise_351.cmm}{CMM of probably\_raise}}
      \onslide*<3>{\mintobjdump[firstline=5,lastline=35,highlightlines=31]{../src/backtrace/camlBacktrace\_\_probably\_raise_351.objdump}}
    \end{column}
  \end{columns}
  \only<1>{\footnotetext[1]{https://blog.janestreet.com/pattern-matching-and-exception-handling-unite/}}
\end{frame}

\newcommand\listCamlReraiseExn[1][]{
  \listasm[firstline=727,lastline=734,#1]{../ocaml/runtime/amd64.S}{runtime/amd64.S:727}}

\begin{frame}{Backtraces}{Introducing \funcname{caml\_reraise\_exn}}
  \begin{columns}[c]
    \begin{column}{0.5\textwidth}
      \minipage[c][0.66\textheight][s]{\columnwidth}
      \begin{itemize}
        \item<1-> The exception have to be reraised to the next handler
        \item<2-> First, checks if the backtrace should be collected with \funcarg{domain\_state->backtrace\_active}
        \only<3>{
        \begin{itemize}
            \item If not, behaves like a \funcname{raise\_notrace} with \funcname{RESTORE\_EXN\_HANDLER\_OCAML}
        \end{itemize}
        }
        \item<4-> Jumps into \funcname{caml\_raise\_exn}
        \item<5-> Calls \funcname{caml\_stash\_backtrace} again, to scan the next part of the stack: from the trap just pop-ed to the next trap
      \end{itemize}
      \vfill
      \mintocaml[firstline=15,lastline=21,highlightlines={18-21}]{../src/backtrace/backtrace.ml}
      \endminipage
    \end{column}
    \begin{column}{0.5\textwidth}
      \raggedleft
      \onslide*<1>{\listCamlReraiseExn{}}
      \onslide*<2>{\listCamlReraiseExn[highlightlines=729]{}}
      \onslide*<3>{
        \mintc[firstline=190,lastline=192,highlightlines={191-192}]{../ocaml/runtime/amd64.S}
        \mintc[firstline=234,lastline=237,highlightlines={235-237}]{../ocaml/runtime/amd64.S}
        \medskip
        \listCamlReraiseExn[highlightlines=731]{}
      }
      \onslide*<4-5>{
        % TODO refactor with \listCamlReraiseExn
        \mintasm[firstline=727,lastline=734,highlightlines=730]{../ocaml/runtime/amd64.S}
        \medskip
      }
      \onslide*<4>{\listCamlRaiseExn[highlightlines=711]{}}
      \onslide*<5>{\listCamlRaiseExn[highlightlines=720]{}}
    \end{column}
  \end{columns}
\end{frame}

\begin{frame}{Backtraces}{Backtrace collection continues}
  \begin{columns}[c]
    \begin{column}{0.55\textwidth}
      \minipage[c][0.75\textheight][s]{\columnwidth}
      \begin{itemize}
        \item<1-> \funcname{caml\_stash\_backtrace} scans the older part of the stack, up to the next trap
        \item<6-> Controls jumps to the nested exception handler in \funcname{maybe\_raise}, which matches only on \typename{ExnB}
        \item<7-> \funcname{caml\_reraise\_exn} is called again, backtrace collection resumes with \funcname{caml\_stash\_backtrace}
      \end{itemize}
      \medskip
      \begin{itemize}
        \item<2-> Constructed backtrace:
        \begin{itemize}
          \item <2-> \funcname{definitely\_raise}
          \item <2-> \funcname{probably\_raise}
          \item <3-> \funcname{probably\_raise} (re-raise)
          \item <4-> \funcname{maybe\_raise}
          \item <5-> \funcname{main}
          \item <8-> \funcname{main} (re-raise)
        \end{itemize}
      \end{itemize}
      \vfill
      \onslide*<1-5>{\mintocaml[firstline=15,lastline=21]{../src/backtrace/backtrace.ml}}
      \onslide*<6>{\mintocaml[firstline=28,lastline=34,highlightlines=31]{../src/backtrace/backtrace.ml}}
      \onslide*<7->{\mintocaml[firstline=28,lastline=34,highlightlines=33]{../src/backtrace/backtrace.ml}}
      \endminipage
    \end{column}
    \begin{column}{0.45\textwidth}
      \raggedleft
      \onslide*<1>{\providecommand\step{6}\input{diagrams/backtrace/backtrace.tex}}%
      \onslide*<2>{\providecommand\step{6}\input{diagrams/backtrace/backtrace.tex}}%
      \onslide*<3>{\providecommand\step{7}\input{diagrams/backtrace/backtrace.tex}}%
      \onslide*<4>{\providecommand\step{8}\input{diagrams/backtrace/backtrace.tex}}%
      \onslide*<5>{\providecommand\step{9}\input{diagrams/backtrace/backtrace.tex}}%
      \onslide*<6>{\providecommand\step{10}\input{diagrams/backtrace/backtrace.tex}}%
      \onslide*<7>{\providecommand\step{11}\input{diagrams/backtrace/backtrace.tex}}%
      \onslide*<8>{\providecommand\step{12}\input{diagrams/backtrace/backtrace.tex}}%
      \onslide*<9>{\providecommand\step{13}\input{diagrams/backtrace/backtrace.tex}}%
    \end{column}
  \end{columns}
\end{frame}

\begin{frame}{Backtraces}{Printing the backtrace}
  \begin{columns}[c]
    \begin{column}{0.45\textwidth}
      \minipage[c][0.66\textheight][s]{\columnwidth}
      \begin{itemize}
        \item<1-> The exception backtrace can now be printed to \funcarg{stdout} with \funcname{Printexc.print_backtrace}
        \item<2-> First the exception backtrace is retrieved into an OCaml array with \funcname{get_raw_backtrace }
        \item<3-> Which is implemented as the \funcname{caml_get_exception_raw_backtrace} C function in the runtime
        \item<4-> And finally printed with \funcname{print_exception_backtrace}
      \end{itemize}
      \vfill
      \mintocaml[firstline=28,lastline=34,highlightlines=33]{../src/backtrace/backtrace.ml}
      \endminipage
    \end{column}
    \begin{column}{0.55\textwidth}
      \onslide*<1,2>{
        \mintocaml[firstline=100,lastline=101]{../ocaml/stdlib/printexc.ml}
      }
      \only<1>{
        \listocaml[firstline=170,lastline=172]{../ocaml/stdlib/printexc.ml}{stdlib/printexc.ml:170}
      }
      \only<2>{
        \listocaml[firstline=170,lastline=172,highlightlines=172]{../ocaml/stdlib/printexc.ml}{stdlib/printexc.ml:170}
      }
      \only<3>{
        \mintc[firstline=154,lastline=156]{../ocaml/runtime/backtrace.c}
        \listc[firstline=171,lastline=192,highlightlines={184-187}]{../ocaml/runtime/backtrace.c}{runtime/backtrace.c:154}
      }
      \only<4>{
        \listocaml[firstline=155,lastline=165]{../ocaml/stdlib/printexc.ml}{stdlib/printexc.ml:55}
      }
    \end{column}
  \end{columns}
\end{frame}


\subsection{The multiple kinds of raise}
\frameSubsection{}{}

\begin{frame}{Backtraces}{When backtraces are not needed}
  \begin{columns}[c]
    \begin{column}{0.45\textwidth}
      \minipage[c][0.66\textheight][s]{\columnwidth}
      \begin{itemize}
        \item<1-> What if the backtrace for an exception is never needed?
        \item<2-> It's possible to raise the exception explicitly with \funcname{raise\_notrace} to make raising as cheap as possible
        \item<3-> An example of use in the \funcname{dynlink} library of the OCaml compiler
      \end{itemize}
      \vfill
      \onslide<3->{
      \listocaml[firstline=205,lastline=209,highlightlines=207]{../ocaml/otherlibs/dynlink/dynlink\_compilerlibs/misc.ml}{otherlibs/dynlink/dynlink\_compilerlibs/misc.ml:205}
      }
      \endminipage
    \end{column}
    \begin{column}{0.55\textwidth}
    \only<4>{
      \mintcmm[highlightlines=3]{../src/notrace/camlNotrace\_\_fun_347.cmm}
      \listcmm{../src/notrace/camlNotrace\_\_all\_somes\_327.cmm}{all\_somes}
    }
    \onslide*<5->{
      \listobjdump[highlightlines={6-9}]{../src/notrace/camlNotrace\_\_fun_347.objdump}{anonymous function from \funcname{all\_somes}}
    }
    \end{column}
  \end{columns}
\end{frame}

\begin{frame}{Backtraces}{Summary table of the multiple kinds of raise}
  \begin{itemize}
    \item When debugging information is enabled and if the backtrace of an exception isn't needed, it's possible to raise with \funcname{raise\_notrace} for performance
    \item When debugging information is \emph{not} enabled, the compiler optimises \funcname{raise} calls to inline assembly (same effect as using \funcname{raise\_notrace})
  \end{itemize}
  \bigskip
  \centering
  \begin{tabular}{| l | c | c | c | c |}
    \hline
    \parbox{10em}{Debug information \\ (\funcarg{-g} flag)}  & \multicolumn{2}{|c|}{Disabled}                                     & \multicolumn{2}{|c|}{Enabled} \\
    \hline
    Raise kind                                               & \funcname{raise} & \funcname{raise\_notrace}                       & \funcname{raise} & \funcname{raise\_notrace} \\
    \hline
    Compilation                                              & \parbox{8em}{inline assembly\\ (optimization)} & inline assembly   & \funcname{caml\_raise\_exn} & inline assembly \\
    \hline
  \end{tabular}
\end{frame}


%
% Takeaway #5
%
\subsection*{Takeaway \#5}
\frameSubsectionTakeaway{}
\againframe<5>{takeaway}
}%
      \onslide*<3>{\providecommand\step{7}%
% Backtraces
%
\subsection{One advantage of raising with debugging information}
\frameSubsection{}{}

\begin{frame}{Backtraces}{Debugging information \& source code}
  \begin{columns}[c]
    \begin{column}{0.5\textwidth}
      \begin{itemize}
        \item Raising an exception is fun and games, but how do we know where the exception originated? $\rightarrow$ With the backtrace associated with the exception!
        \item In order to get a backtrace from the point where an exception was raised, it's necessary to compile with debugging information enabled (\funcarg{-g} flag for the compiler)
        \item Same example as in the `Nested handlers' section
      \end{itemize}
    \end{column}
    \begin{column}{0.5\textwidth}
      \listocaml[firstline=9,lastline=34]{../src/backtrace/backtrace.ml}{backtrace/backtrace.ml}
    \end{column}
  \end{columns}
\end{frame}

\begin{frame}{Backtraces}{CMM and disassembly of \funcname{definitely\_raise}}
  \begin{columns}[c]
    \begin{column}{0.5\textwidth}
      \minipage[c][0.66\textheight][s]{\columnwidth}
      \begin{itemize}
        \item<1-> An effect of compiling with debugging information (\funcarg{-g}): the CMM no longer uses \funcname{raise\_notrace} to raise the exception but \funcname{raise} instead
        \item<2-> Disassembly shows that CMM \funcname{raise} is actually a call to \funcname{caml\_raise\_exn}, a function in assembler provided by the runtime
      \end{itemize}
      \vfill
      \mintocaml[firstline=9,lastline=13,highlightlines=12]{../src/backtrace/backtrace.ml}
      \endminipage
    \end{column}
    \begin{column}{0.5\textwidth}
      \onslide*<1>{
        \mintcmm[highlightlines=5]{../src/backtrace/camlBacktrace\_\_definitely\_raise\_347.cmm}
      }
      \onslide*<2>{
        \mintobjdump[firstline=2,highlightlines=14]{../src/backtrace/camlBacktrace\_\_definitely\_raise\_347.objdump}
      }
    \end{column}
  \end{columns}
\end{frame}

\begin{frame}{Backtraces}{Introducing \funcname{caml\_raise\_exn}}
  \begin{columns}[c]
    \begin{column}{0.5\textwidth}
      \minipage[c][0.66\textheight][s]{\columnwidth}
      \onslide*<1>{
        \begin{itemize}
          \item Raising the exception is performed using \funcname{caml\_raise\_exn} which is
            \begin{itemize}
              \item An assembly function provided by the runtime
              \item Architecture specific
            \end{itemize}
          \item Let's see what it does differently from \funcname{raise\_notrace}\dots
          \item We are about to raise \typename{ExnC} with \funcname{caml\_raise\_exn} from \funcname{definitely\_raise}
        \end{itemize}
      }
      \onslide*<2>{
        \mintobjdump[firstline=2,highlightlines=14]{../src/backtrace/camlBacktrace\_\_definitely\_raise\_347.objdump}
      }
      \vfill
      \mintocaml[firstline=9,lastline=13,highlightlines=12]{../src/backtrace/backtrace.ml}
      \endminipage
    \end{column}
    \begin{column}{0.5\textwidth}
      \centering
      \providecommand\step{1}\input{diagrams/backtrace/backtrace.tex}
    \end{column}
  \end{columns}
\end{frame}

\begin{frame}{Backtraces}{But before that, we need more fields from \typename{caml\_domain\_state}}
  \begin{columns}
    \begin{column}{0.5\textwidth}
      \begin{itemize}
        \item Remember \typename{caml\_domain\_state} the per-domain runtime state?
        \item Some other members are of interest to us now\dots
        \item \localname{backtrace\_active} is used to know if the runtime should collect backtraces
        \item \localname{backtrace\_active} can be set by
          \begin{itemize}
            \item The \funcarg{b} flag in the \funcname{OCAMLRUNPARAM} environment variable
            \item \mintinline{ocaml}{Printexc.record_backtrace : bool -> unit} \footnotemark
          \end{itemize}
      \end{itemize}
    \end{column}
    \begin{column}{0.5\textwidth}
      \begin{listing}
        \mintc[highlightlines={9-11}]{src/ocaml/domain\_state\_backtrace.h}
        \caption{runtime/caml/domain\_state.h}
      \end{listing}
    \end{column}
  \end{columns}
  \footnotetext[1]{https://v2.ocaml.org/api/Printexc.html}
\end{frame}

\newcommand\listCamlRaiseExn[1][]{
  \listasm[firstline=702,lastline=725,#1]{../ocaml/runtime/amd64.S}{runtime/amd64.S:702}}

\begin{frame}{Backtraces}{Walk through \funcname{caml\_raise\_exn}}
  \begin{columns}[T]
    \begin{column}{0.5\textwidth}
      \begin{itemize}
        \item<1-> First checks whether exceptions backtrace should be recorded
        \item<1-> By checking the \funcarg{domain\_state->backtrace\_active} value
        \item<2-> If not, acts as \funcname{raise\_notrace} with \funcname{RESTORE\_EXN\_HANDLER\_OCAML}
          \begin{itemize}
            \item Set \regname{RSP} to \localname{domain\_state->exn\_handler} value
            \item Restore \localname{domain\_state->exn\_handler} from trap
            \item Jump with \funcname{ret} (same effect as using \funcname{pop} and \funcname{jmp})
          \end{itemize}
        \item<3-> Otherwise, switches to the C stack to be able to execute C code
        \item<4-> Calls \funcname{caml\_stash\_backtrace}, a C function provided by the runtime\ignorespaces
          \onslide<6->{, with
            \begin{itemize}
              \item \funcarg{exn}: exception value
              \item \funcarg{pc}: code address of the raising point
              \item \funcarg{sp}: stack address of the raising function
              \item \funcarg{trapsp}: stack address of the next handler
            \end{itemize}
          }
      \end{itemize}
      \bigskip
      \mintocaml[firstline=9,lastline=13,highlightlines=12]{../src/backtrace/backtrace.ml}
    \end{column}
    \begin{column}{0.5\textwidth}
      \raggedleft
      \onslide*<1,3-4>{
        \mintc[firstline=190,lastline=192]{../ocaml/runtime/amd64.S}
        \mintc[firstline=234,lastline=237]{../ocaml/runtime/amd64.S}
      }
      \onslide*<2>{
        \mintc[firstline=190,lastline=192,highlightlines={191-192}]{../ocaml/runtime/amd64.S}
        \mintc[firstline=234,lastline=237,highlightlines={235-237}]{../ocaml/runtime/amd64.S}
      }
      \onslide*<1>{\listCamlRaiseExn[highlightlines=705]{}}%
      \onslide*<2>{\listCamlRaiseExn[highlightlines={707-708}]{}}%
      \onslide*<3>{\listCamlRaiseExn[highlightlines={712-714}]{}}%
      \onslide*<4>{\listCamlRaiseExn[highlightlines={715-720}]{}}%
      \onslide*<5>{\providecommand\step{1}\input{diagrams/backtrace/backtrace.tex}}%
      \onslide*<6>{\providecommand\step{2}\input{diagrams/backtrace/backtrace.tex}}%
    \end{column}
  \end{columns}
\end{frame}

\newcommand\listStashBacktrace[1][]{
  \listc[firstline=91,lastline=120,#1]{../ocaml/runtime/backtrace\_nat.c}{runtime/backtrace\_nat.c:91}}

\begin{frame}{Backtraces}{Introducing \funcname{caml\_stash\_backtrace}}
  \begin{columns}[c]
    \begin{column}{0.5\textwidth}
      \minipage[c][0.66\textheight][s]{\columnwidth}
      \begin{itemize}
        \item<1-> A C function provided by the runtime (specific to native code)
        \item<2-> It scans the stack, between the stack pointer at the raising point up to the next trap, in order to collect the names of the detected function frames
          \onslide*<2>{
            \begin{itemize}
              \item And more information, like line number, column number...
            \end{itemize}
          }
        \item<3-> It uses the same technique and information as the Garbage Collector (GC) uses to scan the values live on the stack
          \onslide*<4>{
            \begin{itemize}
              \item \typename{frame\_descr} are at the core of stack unwinding
            \end{itemize}
          }
      \end{itemize}
      \vfill
      \mintocaml[firstline=9,lastline=13,highlightlines=12]{../src/backtrace/backtrace.ml}
      \endminipage
    \end{column}
    \begin{column}{0.5\textwidth}
      \onslide*<1>{\listStashBacktrace[highlightlines=91]{}}
      \onslide*<2>{\listStashBacktrace[highlightlines={91,117-118}]{}}
      \onslide*<3->{\listStashBacktrace[highlightlines={94,106,109-110}]{}}
    \end{column}
  \end{columns}
\end{frame}

\newcommand\listFrameDescr[1][]{
  \listocaml[firstline=111,lastline=124,#1]{../ocaml/asmcomp/emitaux.ml}{asmcomp/emitaux.ml:111}}
\newcommand\listDebugInfo[1][]{
  \listocaml[firstline=38,lastline=49,#1]{../ocaml/lambda/debuginfo.mli}{lambda/debuginfo.mli:38}}

\begin{frame}{Backtraces}{\typename{frame\_descr} and \typename{frame\_debuginfo} in a nutshell}
  \begin{columns}[c]
    \begin{column}{0.5\textwidth}
      \begin{itemize}
        \item<1-> For every return address in an OCaml program, there is a associated \typename{frame\_descr}
        \item<2-> A \typename{frame\_descr} specifies
        \begin{itemize}
          \item<2-> The size of the function stack frame
          \item<3-> Where live values are on the stack (so that the GC can mark them as live and prevent from being swept)
        \end{itemize}
        \item<4-> When compiled with debug information, \typename{frame\_descr} will be linked to a \typename{frame\_debuginfo}
        \begin{itemize}
          \item<5-> Contains information about the position in the source code (file name, line and column number)
        \end{itemize}
      \end{itemize}
    \end{column}
    \begin{column}{0.5\textwidth}
        \onslide*<1>{\listFrameDescr[highlightlines={118-122}]{}}
        \onslide*<2>{\listFrameDescr[highlightlines={120}]{}}
        \onslide*<3>{\listFrameDescr[highlightlines={121}]{}}
        \onslide*<4->{\listFrameDescr[highlightlines={122}]{}}
        \medskip
        \onslide*<1-3>{\listDebugInfo{}}
        \onslide*<4>{\listDebugInfo[highlightlines={38,49}]{}}
        \onslide*<5>{\listDebugInfo[highlightlines={39-46}]{}}
    \end{column}
  \end{columns}
\end{frame}

\begin{frame}{Backtraces}{Scanning the stack with \typename{frame\_descr}}
  \begin{columns}[c]
    \begin{column}{0.5\textwidth}
      \begin{itemize}
        \item Given the return address of the code that raises the exception by calling \funcname{caml\_raise\_exn} (\funcarg{pc} argument of \funcname{caml\_stash\_backtrace})
        \item And the position of the stack pointer (\funcarg{sp} argument of \funcname{caml\_stash\_backtrace})
        \item The frame size given by \typename{frame\_descr}.\funcarg{frame\_size}, allows to find the end of the previous frame
        \item And the return address of the caller of this function
        \bigskip
        \onslide<3->{
        \item Note that traps (here the reddish \funcname{ExnA}, \funcname{ExnB} and \funcname{ExnC} blocks) belong to the frame that installed them, and so the frame size changes when traps are removed
        }
      \end{itemize}
    \end{column}
    \begin{column}{0.5\textwidth}
      \raggedleft
      \onslide*<1>{\providecommand\step{2}\input{diagrams/backtrace/backtrace.tex}}%
      \onslide*<2>{\providecommand\step{1}\input{diagrams/backtrace/scanstack.tex}}%
      \onslide*<3>{\providecommand\step{2}\input{diagrams/backtrace/scanstack.tex}}%
      \onslide*<4>{\providecommand\step{3}\input{diagrams/backtrace/scanstack.tex}}%
      \onslide*<5>{\providecommand\step{4}\input{diagrams/backtrace/scanstack.tex}}%
      \onslide*<6>{\providecommand\step{5}\input{diagrams/backtrace/scanstack.tex}}%
    \end{column}
  \end{columns}
\end{frame}

% Duplicate of previous-previous slide
\begin{frame}{Backtraces}{Continuing on \funcname{caml\_stash\_backtrace}}
  \begin{columns}[c]
    \begin{column}{0.5\textwidth}
      \minipage[c][0.75\textheight][s]{\columnwidth}
      \begin{itemize}
          \color{gray}
        \item A C function provided by the runtime (specific to native code)
        \item It scans the stack, between the stack pointer at the raising point up to the next trap, in order to collect the names of the detected function frames
        \item It uses the same technique and information as the Garbage Collector (GC) uses to scan the values live on the stack
      \end{itemize}
      \begin{itemize}
        \item For each function frame detected, store a pointer to the \typename{frame\_descr} in the \localname{domain\_state->backtrace\_buffer} array, effectively constructing the backtrace
      \end{itemize}
      \onslide<2->{
        \begin{itemize}
            \smallskip
          \item Constructed backtrace:
            \funcname{definitely\_raise},
            \onslide<3->{\funcname{probably\_raise}}
        \end{itemize}
      }
      \vfill
      \mintocaml[firstline=9,lastline=13,highlightlines=12]{../src/backtrace/backtrace.ml}
      \endminipage
    \end{column}
    \begin{column}{0.5\textwidth}
      \raggedleft
      \onslide*<1>{\listStashBacktrace[highlightlines={114-115}]{}}
      \onslide*<2>{\providecommand\step{3}\input{diagrams/backtrace/backtrace.tex}}%
      \onslide*<3>{\providecommand\step{4}\input{diagrams/backtrace/backtrace.tex}}%
    \end{column}
  \end{columns}
\end{frame}

\begin{frame}{Backtraces}{Back to \funcname{caml\_raise\_exn}}
  \begin{columns}[c]
    \begin{column}{0.5\textwidth}
      \minipage[c][0.66\textheight][s]{\columnwidth}
      \begin{itemize}\color{gray}
        \item Checks wheteher exceptions backtrace should be recorded, by checking the \funcarg{domain\_state->backtrace\_active} value
        \item Switches to the C stack to be able to execute C code
        \item Calls \funcname{caml\_stash\_backtrace}, to collect the backtrace up to the next trap
      \end{itemize}
      \onslide*<2->{
        \begin{itemize}
          \item<2-> Executes the exception handler in \funcname{probably\_raise} with \funcname{RESTORE\_EXN\_HANDLER\_OCAML}
            \begin{itemize}
              \item Set \regname{RSP} to \localname{domain\_state->exn\_handler} value
              \item Restore \localname{domain\_state->exn\_handler} from trap
              \item Jump to the handler code with \funcname{ret}
            \end{itemize}
        \end{itemize}
      }
      \vfill
      \onslide*<1>{\mintocaml[firstline=9,lastline=13,highlightlines=12]{../src/backtrace/backtrace.ml}}
      \onslide*<2->{\mintocaml[firstline=15,lastline=21,highlightlines={19-21}]{../src/backtrace/backtrace.ml}}
      \endminipage
    \end{column}
    \begin{column}{0.5\textwidth}
      \centering
      \onslide*<1>{
        \mintc[firstline=190,lastline=192]{../ocaml/runtime/amd64.S}
        \mintc[firstline=234,lastline=237]{../ocaml/runtime/amd64.S}
      }
      \onslide*<2>{
        \mintc[firstline=190,lastline=192,highlightlines={191-192}]{../ocaml/runtime/amd64.S}
        \mintc[firstline=234,lastline=237,highlightlines={235-237}]{../ocaml/runtime/amd64.S}
      }
      \onslide*<1>{\listCamlRaiseExn[highlightlines=720]{}}
      \onslide*<2>{\listCamlRaiseExn[highlightlines=722]{}}
      \onslide*<3->{\providecommand\step{5}\input{diagrams/backtrace/backtrace.tex}}
    \end{column}
  \end{columns}
\end{frame}

\begin{frame}{Backtraces}{\funcname{caml\_probably\_raise}'s handler doesn't catch \typename{ExnC}}
  \begin{columns}[c]
    \begin{column}{0.45\textwidth}
      \minipage[c][0.75\textheight][s]{\columnwidth}
      \begin{itemize}
        \item<1-> \funcname{match \dots with}\footnotemark pattern matching can also match exceptions
        \item<1-> The CMM of \funcname{probably\_raise} shows that this \funcname{match \dots with} is implemented with a pattern matching and a \funcname{try \dots with}
        \item<1-> But this one only matches on \typename{ExnA} exception
        \item<2-> Since the pattern matching fails to catch the exception, \funcname{reraise} is called with our current \typename{ExnC} exception
        \item<3-> Disassembly shows that \funcname{reraise} is actually a call to \funcname{caml\_reraise\_exn}, which is also an assembly function provided by the runtime
      \end{itemize}
      \vfill
      \mintocaml[firstline=15,lastline=21,highlightlines={18-21}]{../src/backtrace/backtrace.ml}
      \endminipage
    \end{column}
    \begin{column}{0.55\textwidth}
      \centering
      \onslide*<1>{\mintcmm[highlightlines={10-18}]{../src/backtrace/camlBacktrace\_\_probably\_raise\_351.cmm}}
      \onslide*<2>{\listcmm[highlightlines=18]{../src/backtrace/camlBacktrace\_\_probably\_raise_351.cmm}{CMM of probably\_raise}}
      \onslide*<3>{\mintobjdump[firstline=5,lastline=35,highlightlines=31]{../src/backtrace/camlBacktrace\_\_probably\_raise_351.objdump}}
    \end{column}
  \end{columns}
  \only<1>{\footnotetext[1]{https://blog.janestreet.com/pattern-matching-and-exception-handling-unite/}}
\end{frame}

\newcommand\listCamlReraiseExn[1][]{
  \listasm[firstline=727,lastline=734,#1]{../ocaml/runtime/amd64.S}{runtime/amd64.S:727}}

\begin{frame}{Backtraces}{Introducing \funcname{caml\_reraise\_exn}}
  \begin{columns}[c]
    \begin{column}{0.5\textwidth}
      \minipage[c][0.66\textheight][s]{\columnwidth}
      \begin{itemize}
        \item<1-> The exception have to be reraised to the next handler
        \item<2-> First, checks if the backtrace should be collected with \funcarg{domain\_state->backtrace\_active}
        \only<3>{
        \begin{itemize}
            \item If not, behaves like a \funcname{raise\_notrace} with \funcname{RESTORE\_EXN\_HANDLER\_OCAML}
        \end{itemize}
        }
        \item<4-> Jumps into \funcname{caml\_raise\_exn}
        \item<5-> Calls \funcname{caml\_stash\_backtrace} again, to scan the next part of the stack: from the trap just pop-ed to the next trap
      \end{itemize}
      \vfill
      \mintocaml[firstline=15,lastline=21,highlightlines={18-21}]{../src/backtrace/backtrace.ml}
      \endminipage
    \end{column}
    \begin{column}{0.5\textwidth}
      \raggedleft
      \onslide*<1>{\listCamlReraiseExn{}}
      \onslide*<2>{\listCamlReraiseExn[highlightlines=729]{}}
      \onslide*<3>{
        \mintc[firstline=190,lastline=192,highlightlines={191-192}]{../ocaml/runtime/amd64.S}
        \mintc[firstline=234,lastline=237,highlightlines={235-237}]{../ocaml/runtime/amd64.S}
        \medskip
        \listCamlReraiseExn[highlightlines=731]{}
      }
      \onslide*<4-5>{
        % TODO refactor with \listCamlReraiseExn
        \mintasm[firstline=727,lastline=734,highlightlines=730]{../ocaml/runtime/amd64.S}
        \medskip
      }
      \onslide*<4>{\listCamlRaiseExn[highlightlines=711]{}}
      \onslide*<5>{\listCamlRaiseExn[highlightlines=720]{}}
    \end{column}
  \end{columns}
\end{frame}

\begin{frame}{Backtraces}{Backtrace collection continues}
  \begin{columns}[c]
    \begin{column}{0.55\textwidth}
      \minipage[c][0.75\textheight][s]{\columnwidth}
      \begin{itemize}
        \item<1-> \funcname{caml\_stash\_backtrace} scans the older part of the stack, up to the next trap
        \item<6-> Controls jumps to the nested exception handler in \funcname{maybe\_raise}, which matches only on \typename{ExnB}
        \item<7-> \funcname{caml\_reraise\_exn} is called again, backtrace collection resumes with \funcname{caml\_stash\_backtrace}
      \end{itemize}
      \medskip
      \begin{itemize}
        \item<2-> Constructed backtrace:
        \begin{itemize}
          \item <2-> \funcname{definitely\_raise}
          \item <2-> \funcname{probably\_raise}
          \item <3-> \funcname{probably\_raise} (re-raise)
          \item <4-> \funcname{maybe\_raise}
          \item <5-> \funcname{main}
          \item <8-> \funcname{main} (re-raise)
        \end{itemize}
      \end{itemize}
      \vfill
      \onslide*<1-5>{\mintocaml[firstline=15,lastline=21]{../src/backtrace/backtrace.ml}}
      \onslide*<6>{\mintocaml[firstline=28,lastline=34,highlightlines=31]{../src/backtrace/backtrace.ml}}
      \onslide*<7->{\mintocaml[firstline=28,lastline=34,highlightlines=33]{../src/backtrace/backtrace.ml}}
      \endminipage
    \end{column}
    \begin{column}{0.45\textwidth}
      \raggedleft
      \onslide*<1>{\providecommand\step{6}\input{diagrams/backtrace/backtrace.tex}}%
      \onslide*<2>{\providecommand\step{6}\input{diagrams/backtrace/backtrace.tex}}%
      \onslide*<3>{\providecommand\step{7}\input{diagrams/backtrace/backtrace.tex}}%
      \onslide*<4>{\providecommand\step{8}\input{diagrams/backtrace/backtrace.tex}}%
      \onslide*<5>{\providecommand\step{9}\input{diagrams/backtrace/backtrace.tex}}%
      \onslide*<6>{\providecommand\step{10}\input{diagrams/backtrace/backtrace.tex}}%
      \onslide*<7>{\providecommand\step{11}\input{diagrams/backtrace/backtrace.tex}}%
      \onslide*<8>{\providecommand\step{12}\input{diagrams/backtrace/backtrace.tex}}%
      \onslide*<9>{\providecommand\step{13}\input{diagrams/backtrace/backtrace.tex}}%
    \end{column}
  \end{columns}
\end{frame}

\begin{frame}{Backtraces}{Printing the backtrace}
  \begin{columns}[c]
    \begin{column}{0.45\textwidth}
      \minipage[c][0.66\textheight][s]{\columnwidth}
      \begin{itemize}
        \item<1-> The exception backtrace can now be printed to \funcarg{stdout} with \funcname{Printexc.print_backtrace}
        \item<2-> First the exception backtrace is retrieved into an OCaml array with \funcname{get_raw_backtrace }
        \item<3-> Which is implemented as the \funcname{caml_get_exception_raw_backtrace} C function in the runtime
        \item<4-> And finally printed with \funcname{print_exception_backtrace}
      \end{itemize}
      \vfill
      \mintocaml[firstline=28,lastline=34,highlightlines=33]{../src/backtrace/backtrace.ml}
      \endminipage
    \end{column}
    \begin{column}{0.55\textwidth}
      \onslide*<1,2>{
        \mintocaml[firstline=100,lastline=101]{../ocaml/stdlib/printexc.ml}
      }
      \only<1>{
        \listocaml[firstline=170,lastline=172]{../ocaml/stdlib/printexc.ml}{stdlib/printexc.ml:170}
      }
      \only<2>{
        \listocaml[firstline=170,lastline=172,highlightlines=172]{../ocaml/stdlib/printexc.ml}{stdlib/printexc.ml:170}
      }
      \only<3>{
        \mintc[firstline=154,lastline=156]{../ocaml/runtime/backtrace.c}
        \listc[firstline=171,lastline=192,highlightlines={184-187}]{../ocaml/runtime/backtrace.c}{runtime/backtrace.c:154}
      }
      \only<4>{
        \listocaml[firstline=155,lastline=165]{../ocaml/stdlib/printexc.ml}{stdlib/printexc.ml:55}
      }
    \end{column}
  \end{columns}
\end{frame}


\subsection{The multiple kinds of raise}
\frameSubsection{}{}

\begin{frame}{Backtraces}{When backtraces are not needed}
  \begin{columns}[c]
    \begin{column}{0.45\textwidth}
      \minipage[c][0.66\textheight][s]{\columnwidth}
      \begin{itemize}
        \item<1-> What if the backtrace for an exception is never needed?
        \item<2-> It's possible to raise the exception explicitly with \funcname{raise\_notrace} to make raising as cheap as possible
        \item<3-> An example of use in the \funcname{dynlink} library of the OCaml compiler
      \end{itemize}
      \vfill
      \onslide<3->{
      \listocaml[firstline=205,lastline=209,highlightlines=207]{../ocaml/otherlibs/dynlink/dynlink\_compilerlibs/misc.ml}{otherlibs/dynlink/dynlink\_compilerlibs/misc.ml:205}
      }
      \endminipage
    \end{column}
    \begin{column}{0.55\textwidth}
    \only<4>{
      \mintcmm[highlightlines=3]{../src/notrace/camlNotrace\_\_fun_347.cmm}
      \listcmm{../src/notrace/camlNotrace\_\_all\_somes\_327.cmm}{all\_somes}
    }
    \onslide*<5->{
      \listobjdump[highlightlines={6-9}]{../src/notrace/camlNotrace\_\_fun_347.objdump}{anonymous function from \funcname{all\_somes}}
    }
    \end{column}
  \end{columns}
\end{frame}

\begin{frame}{Backtraces}{Summary table of the multiple kinds of raise}
  \begin{itemize}
    \item When debugging information is enabled and if the backtrace of an exception isn't needed, it's possible to raise with \funcname{raise\_notrace} for performance
    \item When debugging information is \emph{not} enabled, the compiler optimises \funcname{raise} calls to inline assembly (same effect as using \funcname{raise\_notrace})
  \end{itemize}
  \bigskip
  \centering
  \begin{tabular}{| l | c | c | c | c |}
    \hline
    \parbox{10em}{Debug information \\ (\funcarg{-g} flag)}  & \multicolumn{2}{|c|}{Disabled}                                     & \multicolumn{2}{|c|}{Enabled} \\
    \hline
    Raise kind                                               & \funcname{raise} & \funcname{raise\_notrace}                       & \funcname{raise} & \funcname{raise\_notrace} \\
    \hline
    Compilation                                              & \parbox{8em}{inline assembly\\ (optimization)} & inline assembly   & \funcname{caml\_raise\_exn} & inline assembly \\
    \hline
  \end{tabular}
\end{frame}


%
% Takeaway #5
%
\subsection*{Takeaway \#5}
\frameSubsectionTakeaway{}
\againframe<5>{takeaway}
}%
      \onslide*<4>{\providecommand\step{8}%
% Backtraces
%
\subsection{One advantage of raising with debugging information}
\frameSubsection{}{}

\begin{frame}{Backtraces}{Debugging information \& source code}
  \begin{columns}[c]
    \begin{column}{0.5\textwidth}
      \begin{itemize}
        \item Raising an exception is fun and games, but how do we know where the exception originated? $\rightarrow$ With the backtrace associated with the exception!
        \item In order to get a backtrace from the point where an exception was raised, it's necessary to compile with debugging information enabled (\funcarg{-g} flag for the compiler)
        \item Same example as in the `Nested handlers' section
      \end{itemize}
    \end{column}
    \begin{column}{0.5\textwidth}
      \listocaml[firstline=9,lastline=34]{../src/backtrace/backtrace.ml}{backtrace/backtrace.ml}
    \end{column}
  \end{columns}
\end{frame}

\begin{frame}{Backtraces}{CMM and disassembly of \funcname{definitely\_raise}}
  \begin{columns}[c]
    \begin{column}{0.5\textwidth}
      \minipage[c][0.66\textheight][s]{\columnwidth}
      \begin{itemize}
        \item<1-> An effect of compiling with debugging information (\funcarg{-g}): the CMM no longer uses \funcname{raise\_notrace} to raise the exception but \funcname{raise} instead
        \item<2-> Disassembly shows that CMM \funcname{raise} is actually a call to \funcname{caml\_raise\_exn}, a function in assembler provided by the runtime
      \end{itemize}
      \vfill
      \mintocaml[firstline=9,lastline=13,highlightlines=12]{../src/backtrace/backtrace.ml}
      \endminipage
    \end{column}
    \begin{column}{0.5\textwidth}
      \onslide*<1>{
        \mintcmm[highlightlines=5]{../src/backtrace/camlBacktrace\_\_definitely\_raise\_347.cmm}
      }
      \onslide*<2>{
        \mintobjdump[firstline=2,highlightlines=14]{../src/backtrace/camlBacktrace\_\_definitely\_raise\_347.objdump}
      }
    \end{column}
  \end{columns}
\end{frame}

\begin{frame}{Backtraces}{Introducing \funcname{caml\_raise\_exn}}
  \begin{columns}[c]
    \begin{column}{0.5\textwidth}
      \minipage[c][0.66\textheight][s]{\columnwidth}
      \onslide*<1>{
        \begin{itemize}
          \item Raising the exception is performed using \funcname{caml\_raise\_exn} which is
            \begin{itemize}
              \item An assembly function provided by the runtime
              \item Architecture specific
            \end{itemize}
          \item Let's see what it does differently from \funcname{raise\_notrace}\dots
          \item We are about to raise \typename{ExnC} with \funcname{caml\_raise\_exn} from \funcname{definitely\_raise}
        \end{itemize}
      }
      \onslide*<2>{
        \mintobjdump[firstline=2,highlightlines=14]{../src/backtrace/camlBacktrace\_\_definitely\_raise\_347.objdump}
      }
      \vfill
      \mintocaml[firstline=9,lastline=13,highlightlines=12]{../src/backtrace/backtrace.ml}
      \endminipage
    \end{column}
    \begin{column}{0.5\textwidth}
      \centering
      \providecommand\step{1}\input{diagrams/backtrace/backtrace.tex}
    \end{column}
  \end{columns}
\end{frame}

\begin{frame}{Backtraces}{But before that, we need more fields from \typename{caml\_domain\_state}}
  \begin{columns}
    \begin{column}{0.5\textwidth}
      \begin{itemize}
        \item Remember \typename{caml\_domain\_state} the per-domain runtime state?
        \item Some other members are of interest to us now\dots
        \item \localname{backtrace\_active} is used to know if the runtime should collect backtraces
        \item \localname{backtrace\_active} can be set by
          \begin{itemize}
            \item The \funcarg{b} flag in the \funcname{OCAMLRUNPARAM} environment variable
            \item \mintinline{ocaml}{Printexc.record_backtrace : bool -> unit} \footnotemark
          \end{itemize}
      \end{itemize}
    \end{column}
    \begin{column}{0.5\textwidth}
      \begin{listing}
        \mintc[highlightlines={9-11}]{src/ocaml/domain\_state\_backtrace.h}
        \caption{runtime/caml/domain\_state.h}
      \end{listing}
    \end{column}
  \end{columns}
  \footnotetext[1]{https://v2.ocaml.org/api/Printexc.html}
\end{frame}

\newcommand\listCamlRaiseExn[1][]{
  \listasm[firstline=702,lastline=725,#1]{../ocaml/runtime/amd64.S}{runtime/amd64.S:702}}

\begin{frame}{Backtraces}{Walk through \funcname{caml\_raise\_exn}}
  \begin{columns}[T]
    \begin{column}{0.5\textwidth}
      \begin{itemize}
        \item<1-> First checks whether exceptions backtrace should be recorded
        \item<1-> By checking the \funcarg{domain\_state->backtrace\_active} value
        \item<2-> If not, acts as \funcname{raise\_notrace} with \funcname{RESTORE\_EXN\_HANDLER\_OCAML}
          \begin{itemize}
            \item Set \regname{RSP} to \localname{domain\_state->exn\_handler} value
            \item Restore \localname{domain\_state->exn\_handler} from trap
            \item Jump with \funcname{ret} (same effect as using \funcname{pop} and \funcname{jmp})
          \end{itemize}
        \item<3-> Otherwise, switches to the C stack to be able to execute C code
        \item<4-> Calls \funcname{caml\_stash\_backtrace}, a C function provided by the runtime\ignorespaces
          \onslide<6->{, with
            \begin{itemize}
              \item \funcarg{exn}: exception value
              \item \funcarg{pc}: code address of the raising point
              \item \funcarg{sp}: stack address of the raising function
              \item \funcarg{trapsp}: stack address of the next handler
            \end{itemize}
          }
      \end{itemize}
      \bigskip
      \mintocaml[firstline=9,lastline=13,highlightlines=12]{../src/backtrace/backtrace.ml}
    \end{column}
    \begin{column}{0.5\textwidth}
      \raggedleft
      \onslide*<1,3-4>{
        \mintc[firstline=190,lastline=192]{../ocaml/runtime/amd64.S}
        \mintc[firstline=234,lastline=237]{../ocaml/runtime/amd64.S}
      }
      \onslide*<2>{
        \mintc[firstline=190,lastline=192,highlightlines={191-192}]{../ocaml/runtime/amd64.S}
        \mintc[firstline=234,lastline=237,highlightlines={235-237}]{../ocaml/runtime/amd64.S}
      }
      \onslide*<1>{\listCamlRaiseExn[highlightlines=705]{}}%
      \onslide*<2>{\listCamlRaiseExn[highlightlines={707-708}]{}}%
      \onslide*<3>{\listCamlRaiseExn[highlightlines={712-714}]{}}%
      \onslide*<4>{\listCamlRaiseExn[highlightlines={715-720}]{}}%
      \onslide*<5>{\providecommand\step{1}\input{diagrams/backtrace/backtrace.tex}}%
      \onslide*<6>{\providecommand\step{2}\input{diagrams/backtrace/backtrace.tex}}%
    \end{column}
  \end{columns}
\end{frame}

\newcommand\listStashBacktrace[1][]{
  \listc[firstline=91,lastline=120,#1]{../ocaml/runtime/backtrace\_nat.c}{runtime/backtrace\_nat.c:91}}

\begin{frame}{Backtraces}{Introducing \funcname{caml\_stash\_backtrace}}
  \begin{columns}[c]
    \begin{column}{0.5\textwidth}
      \minipage[c][0.66\textheight][s]{\columnwidth}
      \begin{itemize}
        \item<1-> A C function provided by the runtime (specific to native code)
        \item<2-> It scans the stack, between the stack pointer at the raising point up to the next trap, in order to collect the names of the detected function frames
          \onslide*<2>{
            \begin{itemize}
              \item And more information, like line number, column number...
            \end{itemize}
          }
        \item<3-> It uses the same technique and information as the Garbage Collector (GC) uses to scan the values live on the stack
          \onslide*<4>{
            \begin{itemize}
              \item \typename{frame\_descr} are at the core of stack unwinding
            \end{itemize}
          }
      \end{itemize}
      \vfill
      \mintocaml[firstline=9,lastline=13,highlightlines=12]{../src/backtrace/backtrace.ml}
      \endminipage
    \end{column}
    \begin{column}{0.5\textwidth}
      \onslide*<1>{\listStashBacktrace[highlightlines=91]{}}
      \onslide*<2>{\listStashBacktrace[highlightlines={91,117-118}]{}}
      \onslide*<3->{\listStashBacktrace[highlightlines={94,106,109-110}]{}}
    \end{column}
  \end{columns}
\end{frame}

\newcommand\listFrameDescr[1][]{
  \listocaml[firstline=111,lastline=124,#1]{../ocaml/asmcomp/emitaux.ml}{asmcomp/emitaux.ml:111}}
\newcommand\listDebugInfo[1][]{
  \listocaml[firstline=38,lastline=49,#1]{../ocaml/lambda/debuginfo.mli}{lambda/debuginfo.mli:38}}

\begin{frame}{Backtraces}{\typename{frame\_descr} and \typename{frame\_debuginfo} in a nutshell}
  \begin{columns}[c]
    \begin{column}{0.5\textwidth}
      \begin{itemize}
        \item<1-> For every return address in an OCaml program, there is a associated \typename{frame\_descr}
        \item<2-> A \typename{frame\_descr} specifies
        \begin{itemize}
          \item<2-> The size of the function stack frame
          \item<3-> Where live values are on the stack (so that the GC can mark them as live and prevent from being swept)
        \end{itemize}
        \item<4-> When compiled with debug information, \typename{frame\_descr} will be linked to a \typename{frame\_debuginfo}
        \begin{itemize}
          \item<5-> Contains information about the position in the source code (file name, line and column number)
        \end{itemize}
      \end{itemize}
    \end{column}
    \begin{column}{0.5\textwidth}
        \onslide*<1>{\listFrameDescr[highlightlines={118-122}]{}}
        \onslide*<2>{\listFrameDescr[highlightlines={120}]{}}
        \onslide*<3>{\listFrameDescr[highlightlines={121}]{}}
        \onslide*<4->{\listFrameDescr[highlightlines={122}]{}}
        \medskip
        \onslide*<1-3>{\listDebugInfo{}}
        \onslide*<4>{\listDebugInfo[highlightlines={38,49}]{}}
        \onslide*<5>{\listDebugInfo[highlightlines={39-46}]{}}
    \end{column}
  \end{columns}
\end{frame}

\begin{frame}{Backtraces}{Scanning the stack with \typename{frame\_descr}}
  \begin{columns}[c]
    \begin{column}{0.5\textwidth}
      \begin{itemize}
        \item Given the return address of the code that raises the exception by calling \funcname{caml\_raise\_exn} (\funcarg{pc} argument of \funcname{caml\_stash\_backtrace})
        \item And the position of the stack pointer (\funcarg{sp} argument of \funcname{caml\_stash\_backtrace})
        \item The frame size given by \typename{frame\_descr}.\funcarg{frame\_size}, allows to find the end of the previous frame
        \item And the return address of the caller of this function
        \bigskip
        \onslide<3->{
        \item Note that traps (here the reddish \funcname{ExnA}, \funcname{ExnB} and \funcname{ExnC} blocks) belong to the frame that installed them, and so the frame size changes when traps are removed
        }
      \end{itemize}
    \end{column}
    \begin{column}{0.5\textwidth}
      \raggedleft
      \onslide*<1>{\providecommand\step{2}\input{diagrams/backtrace/backtrace.tex}}%
      \onslide*<2>{\providecommand\step{1}\input{diagrams/backtrace/scanstack.tex}}%
      \onslide*<3>{\providecommand\step{2}\input{diagrams/backtrace/scanstack.tex}}%
      \onslide*<4>{\providecommand\step{3}\input{diagrams/backtrace/scanstack.tex}}%
      \onslide*<5>{\providecommand\step{4}\input{diagrams/backtrace/scanstack.tex}}%
      \onslide*<6>{\providecommand\step{5}\input{diagrams/backtrace/scanstack.tex}}%
    \end{column}
  \end{columns}
\end{frame}

% Duplicate of previous-previous slide
\begin{frame}{Backtraces}{Continuing on \funcname{caml\_stash\_backtrace}}
  \begin{columns}[c]
    \begin{column}{0.5\textwidth}
      \minipage[c][0.75\textheight][s]{\columnwidth}
      \begin{itemize}
          \color{gray}
        \item A C function provided by the runtime (specific to native code)
        \item It scans the stack, between the stack pointer at the raising point up to the next trap, in order to collect the names of the detected function frames
        \item It uses the same technique and information as the Garbage Collector (GC) uses to scan the values live on the stack
      \end{itemize}
      \begin{itemize}
        \item For each function frame detected, store a pointer to the \typename{frame\_descr} in the \localname{domain\_state->backtrace\_buffer} array, effectively constructing the backtrace
      \end{itemize}
      \onslide<2->{
        \begin{itemize}
            \smallskip
          \item Constructed backtrace:
            \funcname{definitely\_raise},
            \onslide<3->{\funcname{probably\_raise}}
        \end{itemize}
      }
      \vfill
      \mintocaml[firstline=9,lastline=13,highlightlines=12]{../src/backtrace/backtrace.ml}
      \endminipage
    \end{column}
    \begin{column}{0.5\textwidth}
      \raggedleft
      \onslide*<1>{\listStashBacktrace[highlightlines={114-115}]{}}
      \onslide*<2>{\providecommand\step{3}\input{diagrams/backtrace/backtrace.tex}}%
      \onslide*<3>{\providecommand\step{4}\input{diagrams/backtrace/backtrace.tex}}%
    \end{column}
  \end{columns}
\end{frame}

\begin{frame}{Backtraces}{Back to \funcname{caml\_raise\_exn}}
  \begin{columns}[c]
    \begin{column}{0.5\textwidth}
      \minipage[c][0.66\textheight][s]{\columnwidth}
      \begin{itemize}\color{gray}
        \item Checks wheteher exceptions backtrace should be recorded, by checking the \funcarg{domain\_state->backtrace\_active} value
        \item Switches to the C stack to be able to execute C code
        \item Calls \funcname{caml\_stash\_backtrace}, to collect the backtrace up to the next trap
      \end{itemize}
      \onslide*<2->{
        \begin{itemize}
          \item<2-> Executes the exception handler in \funcname{probably\_raise} with \funcname{RESTORE\_EXN\_HANDLER\_OCAML}
            \begin{itemize}
              \item Set \regname{RSP} to \localname{domain\_state->exn\_handler} value
              \item Restore \localname{domain\_state->exn\_handler} from trap
              \item Jump to the handler code with \funcname{ret}
            \end{itemize}
        \end{itemize}
      }
      \vfill
      \onslide*<1>{\mintocaml[firstline=9,lastline=13,highlightlines=12]{../src/backtrace/backtrace.ml}}
      \onslide*<2->{\mintocaml[firstline=15,lastline=21,highlightlines={19-21}]{../src/backtrace/backtrace.ml}}
      \endminipage
    \end{column}
    \begin{column}{0.5\textwidth}
      \centering
      \onslide*<1>{
        \mintc[firstline=190,lastline=192]{../ocaml/runtime/amd64.S}
        \mintc[firstline=234,lastline=237]{../ocaml/runtime/amd64.S}
      }
      \onslide*<2>{
        \mintc[firstline=190,lastline=192,highlightlines={191-192}]{../ocaml/runtime/amd64.S}
        \mintc[firstline=234,lastline=237,highlightlines={235-237}]{../ocaml/runtime/amd64.S}
      }
      \onslide*<1>{\listCamlRaiseExn[highlightlines=720]{}}
      \onslide*<2>{\listCamlRaiseExn[highlightlines=722]{}}
      \onslide*<3->{\providecommand\step{5}\input{diagrams/backtrace/backtrace.tex}}
    \end{column}
  \end{columns}
\end{frame}

\begin{frame}{Backtraces}{\funcname{caml\_probably\_raise}'s handler doesn't catch \typename{ExnC}}
  \begin{columns}[c]
    \begin{column}{0.45\textwidth}
      \minipage[c][0.75\textheight][s]{\columnwidth}
      \begin{itemize}
        \item<1-> \funcname{match \dots with}\footnotemark pattern matching can also match exceptions
        \item<1-> The CMM of \funcname{probably\_raise} shows that this \funcname{match \dots with} is implemented with a pattern matching and a \funcname{try \dots with}
        \item<1-> But this one only matches on \typename{ExnA} exception
        \item<2-> Since the pattern matching fails to catch the exception, \funcname{reraise} is called with our current \typename{ExnC} exception
        \item<3-> Disassembly shows that \funcname{reraise} is actually a call to \funcname{caml\_reraise\_exn}, which is also an assembly function provided by the runtime
      \end{itemize}
      \vfill
      \mintocaml[firstline=15,lastline=21,highlightlines={18-21}]{../src/backtrace/backtrace.ml}
      \endminipage
    \end{column}
    \begin{column}{0.55\textwidth}
      \centering
      \onslide*<1>{\mintcmm[highlightlines={10-18}]{../src/backtrace/camlBacktrace\_\_probably\_raise\_351.cmm}}
      \onslide*<2>{\listcmm[highlightlines=18]{../src/backtrace/camlBacktrace\_\_probably\_raise_351.cmm}{CMM of probably\_raise}}
      \onslide*<3>{\mintobjdump[firstline=5,lastline=35,highlightlines=31]{../src/backtrace/camlBacktrace\_\_probably\_raise_351.objdump}}
    \end{column}
  \end{columns}
  \only<1>{\footnotetext[1]{https://blog.janestreet.com/pattern-matching-and-exception-handling-unite/}}
\end{frame}

\newcommand\listCamlReraiseExn[1][]{
  \listasm[firstline=727,lastline=734,#1]{../ocaml/runtime/amd64.S}{runtime/amd64.S:727}}

\begin{frame}{Backtraces}{Introducing \funcname{caml\_reraise\_exn}}
  \begin{columns}[c]
    \begin{column}{0.5\textwidth}
      \minipage[c][0.66\textheight][s]{\columnwidth}
      \begin{itemize}
        \item<1-> The exception have to be reraised to the next handler
        \item<2-> First, checks if the backtrace should be collected with \funcarg{domain\_state->backtrace\_active}
        \only<3>{
        \begin{itemize}
            \item If not, behaves like a \funcname{raise\_notrace} with \funcname{RESTORE\_EXN\_HANDLER\_OCAML}
        \end{itemize}
        }
        \item<4-> Jumps into \funcname{caml\_raise\_exn}
        \item<5-> Calls \funcname{caml\_stash\_backtrace} again, to scan the next part of the stack: from the trap just pop-ed to the next trap
      \end{itemize}
      \vfill
      \mintocaml[firstline=15,lastline=21,highlightlines={18-21}]{../src/backtrace/backtrace.ml}
      \endminipage
    \end{column}
    \begin{column}{0.5\textwidth}
      \raggedleft
      \onslide*<1>{\listCamlReraiseExn{}}
      \onslide*<2>{\listCamlReraiseExn[highlightlines=729]{}}
      \onslide*<3>{
        \mintc[firstline=190,lastline=192,highlightlines={191-192}]{../ocaml/runtime/amd64.S}
        \mintc[firstline=234,lastline=237,highlightlines={235-237}]{../ocaml/runtime/amd64.S}
        \medskip
        \listCamlReraiseExn[highlightlines=731]{}
      }
      \onslide*<4-5>{
        % TODO refactor with \listCamlReraiseExn
        \mintasm[firstline=727,lastline=734,highlightlines=730]{../ocaml/runtime/amd64.S}
        \medskip
      }
      \onslide*<4>{\listCamlRaiseExn[highlightlines=711]{}}
      \onslide*<5>{\listCamlRaiseExn[highlightlines=720]{}}
    \end{column}
  \end{columns}
\end{frame}

\begin{frame}{Backtraces}{Backtrace collection continues}
  \begin{columns}[c]
    \begin{column}{0.55\textwidth}
      \minipage[c][0.75\textheight][s]{\columnwidth}
      \begin{itemize}
        \item<1-> \funcname{caml\_stash\_backtrace} scans the older part of the stack, up to the next trap
        \item<6-> Controls jumps to the nested exception handler in \funcname{maybe\_raise}, which matches only on \typename{ExnB}
        \item<7-> \funcname{caml\_reraise\_exn} is called again, backtrace collection resumes with \funcname{caml\_stash\_backtrace}
      \end{itemize}
      \medskip
      \begin{itemize}
        \item<2-> Constructed backtrace:
        \begin{itemize}
          \item <2-> \funcname{definitely\_raise}
          \item <2-> \funcname{probably\_raise}
          \item <3-> \funcname{probably\_raise} (re-raise)
          \item <4-> \funcname{maybe\_raise}
          \item <5-> \funcname{main}
          \item <8-> \funcname{main} (re-raise)
        \end{itemize}
      \end{itemize}
      \vfill
      \onslide*<1-5>{\mintocaml[firstline=15,lastline=21]{../src/backtrace/backtrace.ml}}
      \onslide*<6>{\mintocaml[firstline=28,lastline=34,highlightlines=31]{../src/backtrace/backtrace.ml}}
      \onslide*<7->{\mintocaml[firstline=28,lastline=34,highlightlines=33]{../src/backtrace/backtrace.ml}}
      \endminipage
    \end{column}
    \begin{column}{0.45\textwidth}
      \raggedleft
      \onslide*<1>{\providecommand\step{6}\input{diagrams/backtrace/backtrace.tex}}%
      \onslide*<2>{\providecommand\step{6}\input{diagrams/backtrace/backtrace.tex}}%
      \onslide*<3>{\providecommand\step{7}\input{diagrams/backtrace/backtrace.tex}}%
      \onslide*<4>{\providecommand\step{8}\input{diagrams/backtrace/backtrace.tex}}%
      \onslide*<5>{\providecommand\step{9}\input{diagrams/backtrace/backtrace.tex}}%
      \onslide*<6>{\providecommand\step{10}\input{diagrams/backtrace/backtrace.tex}}%
      \onslide*<7>{\providecommand\step{11}\input{diagrams/backtrace/backtrace.tex}}%
      \onslide*<8>{\providecommand\step{12}\input{diagrams/backtrace/backtrace.tex}}%
      \onslide*<9>{\providecommand\step{13}\input{diagrams/backtrace/backtrace.tex}}%
    \end{column}
  \end{columns}
\end{frame}

\begin{frame}{Backtraces}{Printing the backtrace}
  \begin{columns}[c]
    \begin{column}{0.45\textwidth}
      \minipage[c][0.66\textheight][s]{\columnwidth}
      \begin{itemize}
        \item<1-> The exception backtrace can now be printed to \funcarg{stdout} with \funcname{Printexc.print_backtrace}
        \item<2-> First the exception backtrace is retrieved into an OCaml array with \funcname{get_raw_backtrace }
        \item<3-> Which is implemented as the \funcname{caml_get_exception_raw_backtrace} C function in the runtime
        \item<4-> And finally printed with \funcname{print_exception_backtrace}
      \end{itemize}
      \vfill
      \mintocaml[firstline=28,lastline=34,highlightlines=33]{../src/backtrace/backtrace.ml}
      \endminipage
    \end{column}
    \begin{column}{0.55\textwidth}
      \onslide*<1,2>{
        \mintocaml[firstline=100,lastline=101]{../ocaml/stdlib/printexc.ml}
      }
      \only<1>{
        \listocaml[firstline=170,lastline=172]{../ocaml/stdlib/printexc.ml}{stdlib/printexc.ml:170}
      }
      \only<2>{
        \listocaml[firstline=170,lastline=172,highlightlines=172]{../ocaml/stdlib/printexc.ml}{stdlib/printexc.ml:170}
      }
      \only<3>{
        \mintc[firstline=154,lastline=156]{../ocaml/runtime/backtrace.c}
        \listc[firstline=171,lastline=192,highlightlines={184-187}]{../ocaml/runtime/backtrace.c}{runtime/backtrace.c:154}
      }
      \only<4>{
        \listocaml[firstline=155,lastline=165]{../ocaml/stdlib/printexc.ml}{stdlib/printexc.ml:55}
      }
    \end{column}
  \end{columns}
\end{frame}


\subsection{The multiple kinds of raise}
\frameSubsection{}{}

\begin{frame}{Backtraces}{When backtraces are not needed}
  \begin{columns}[c]
    \begin{column}{0.45\textwidth}
      \minipage[c][0.66\textheight][s]{\columnwidth}
      \begin{itemize}
        \item<1-> What if the backtrace for an exception is never needed?
        \item<2-> It's possible to raise the exception explicitly with \funcname{raise\_notrace} to make raising as cheap as possible
        \item<3-> An example of use in the \funcname{dynlink} library of the OCaml compiler
      \end{itemize}
      \vfill
      \onslide<3->{
      \listocaml[firstline=205,lastline=209,highlightlines=207]{../ocaml/otherlibs/dynlink/dynlink\_compilerlibs/misc.ml}{otherlibs/dynlink/dynlink\_compilerlibs/misc.ml:205}
      }
      \endminipage
    \end{column}
    \begin{column}{0.55\textwidth}
    \only<4>{
      \mintcmm[highlightlines=3]{../src/notrace/camlNotrace\_\_fun_347.cmm}
      \listcmm{../src/notrace/camlNotrace\_\_all\_somes\_327.cmm}{all\_somes}
    }
    \onslide*<5->{
      \listobjdump[highlightlines={6-9}]{../src/notrace/camlNotrace\_\_fun_347.objdump}{anonymous function from \funcname{all\_somes}}
    }
    \end{column}
  \end{columns}
\end{frame}

\begin{frame}{Backtraces}{Summary table of the multiple kinds of raise}
  \begin{itemize}
    \item When debugging information is enabled and if the backtrace of an exception isn't needed, it's possible to raise with \funcname{raise\_notrace} for performance
    \item When debugging information is \emph{not} enabled, the compiler optimises \funcname{raise} calls to inline assembly (same effect as using \funcname{raise\_notrace})
  \end{itemize}
  \bigskip
  \centering
  \begin{tabular}{| l | c | c | c | c |}
    \hline
    \parbox{10em}{Debug information \\ (\funcarg{-g} flag)}  & \multicolumn{2}{|c|}{Disabled}                                     & \multicolumn{2}{|c|}{Enabled} \\
    \hline
    Raise kind                                               & \funcname{raise} & \funcname{raise\_notrace}                       & \funcname{raise} & \funcname{raise\_notrace} \\
    \hline
    Compilation                                              & \parbox{8em}{inline assembly\\ (optimization)} & inline assembly   & \funcname{caml\_raise\_exn} & inline assembly \\
    \hline
  \end{tabular}
\end{frame}


%
% Takeaway #5
%
\subsection*{Takeaway \#5}
\frameSubsectionTakeaway{}
\againframe<5>{takeaway}
}%
      \onslide*<5>{\providecommand\step{9}%
% Backtraces
%
\subsection{One advantage of raising with debugging information}
\frameSubsection{}{}

\begin{frame}{Backtraces}{Debugging information \& source code}
  \begin{columns}[c]
    \begin{column}{0.5\textwidth}
      \begin{itemize}
        \item Raising an exception is fun and games, but how do we know where the exception originated? $\rightarrow$ With the backtrace associated with the exception!
        \item In order to get a backtrace from the point where an exception was raised, it's necessary to compile with debugging information enabled (\funcarg{-g} flag for the compiler)
        \item Same example as in the `Nested handlers' section
      \end{itemize}
    \end{column}
    \begin{column}{0.5\textwidth}
      \listocaml[firstline=9,lastline=34]{../src/backtrace/backtrace.ml}{backtrace/backtrace.ml}
    \end{column}
  \end{columns}
\end{frame}

\begin{frame}{Backtraces}{CMM and disassembly of \funcname{definitely\_raise}}
  \begin{columns}[c]
    \begin{column}{0.5\textwidth}
      \minipage[c][0.66\textheight][s]{\columnwidth}
      \begin{itemize}
        \item<1-> An effect of compiling with debugging information (\funcarg{-g}): the CMM no longer uses \funcname{raise\_notrace} to raise the exception but \funcname{raise} instead
        \item<2-> Disassembly shows that CMM \funcname{raise} is actually a call to \funcname{caml\_raise\_exn}, a function in assembler provided by the runtime
      \end{itemize}
      \vfill
      \mintocaml[firstline=9,lastline=13,highlightlines=12]{../src/backtrace/backtrace.ml}
      \endminipage
    \end{column}
    \begin{column}{0.5\textwidth}
      \onslide*<1>{
        \mintcmm[highlightlines=5]{../src/backtrace/camlBacktrace\_\_definitely\_raise\_347.cmm}
      }
      \onslide*<2>{
        \mintobjdump[firstline=2,highlightlines=14]{../src/backtrace/camlBacktrace\_\_definitely\_raise\_347.objdump}
      }
    \end{column}
  \end{columns}
\end{frame}

\begin{frame}{Backtraces}{Introducing \funcname{caml\_raise\_exn}}
  \begin{columns}[c]
    \begin{column}{0.5\textwidth}
      \minipage[c][0.66\textheight][s]{\columnwidth}
      \onslide*<1>{
        \begin{itemize}
          \item Raising the exception is performed using \funcname{caml\_raise\_exn} which is
            \begin{itemize}
              \item An assembly function provided by the runtime
              \item Architecture specific
            \end{itemize}
          \item Let's see what it does differently from \funcname{raise\_notrace}\dots
          \item We are about to raise \typename{ExnC} with \funcname{caml\_raise\_exn} from \funcname{definitely\_raise}
        \end{itemize}
      }
      \onslide*<2>{
        \mintobjdump[firstline=2,highlightlines=14]{../src/backtrace/camlBacktrace\_\_definitely\_raise\_347.objdump}
      }
      \vfill
      \mintocaml[firstline=9,lastline=13,highlightlines=12]{../src/backtrace/backtrace.ml}
      \endminipage
    \end{column}
    \begin{column}{0.5\textwidth}
      \centering
      \providecommand\step{1}\input{diagrams/backtrace/backtrace.tex}
    \end{column}
  \end{columns}
\end{frame}

\begin{frame}{Backtraces}{But before that, we need more fields from \typename{caml\_domain\_state}}
  \begin{columns}
    \begin{column}{0.5\textwidth}
      \begin{itemize}
        \item Remember \typename{caml\_domain\_state} the per-domain runtime state?
        \item Some other members are of interest to us now\dots
        \item \localname{backtrace\_active} is used to know if the runtime should collect backtraces
        \item \localname{backtrace\_active} can be set by
          \begin{itemize}
            \item The \funcarg{b} flag in the \funcname{OCAMLRUNPARAM} environment variable
            \item \mintinline{ocaml}{Printexc.record_backtrace : bool -> unit} \footnotemark
          \end{itemize}
      \end{itemize}
    \end{column}
    \begin{column}{0.5\textwidth}
      \begin{listing}
        \mintc[highlightlines={9-11}]{src/ocaml/domain\_state\_backtrace.h}
        \caption{runtime/caml/domain\_state.h}
      \end{listing}
    \end{column}
  \end{columns}
  \footnotetext[1]{https://v2.ocaml.org/api/Printexc.html}
\end{frame}

\newcommand\listCamlRaiseExn[1][]{
  \listasm[firstline=702,lastline=725,#1]{../ocaml/runtime/amd64.S}{runtime/amd64.S:702}}

\begin{frame}{Backtraces}{Walk through \funcname{caml\_raise\_exn}}
  \begin{columns}[T]
    \begin{column}{0.5\textwidth}
      \begin{itemize}
        \item<1-> First checks whether exceptions backtrace should be recorded
        \item<1-> By checking the \funcarg{domain\_state->backtrace\_active} value
        \item<2-> If not, acts as \funcname{raise\_notrace} with \funcname{RESTORE\_EXN\_HANDLER\_OCAML}
          \begin{itemize}
            \item Set \regname{RSP} to \localname{domain\_state->exn\_handler} value
            \item Restore \localname{domain\_state->exn\_handler} from trap
            \item Jump with \funcname{ret} (same effect as using \funcname{pop} and \funcname{jmp})
          \end{itemize}
        \item<3-> Otherwise, switches to the C stack to be able to execute C code
        \item<4-> Calls \funcname{caml\_stash\_backtrace}, a C function provided by the runtime\ignorespaces
          \onslide<6->{, with
            \begin{itemize}
              \item \funcarg{exn}: exception value
              \item \funcarg{pc}: code address of the raising point
              \item \funcarg{sp}: stack address of the raising function
              \item \funcarg{trapsp}: stack address of the next handler
            \end{itemize}
          }
      \end{itemize}
      \bigskip
      \mintocaml[firstline=9,lastline=13,highlightlines=12]{../src/backtrace/backtrace.ml}
    \end{column}
    \begin{column}{0.5\textwidth}
      \raggedleft
      \onslide*<1,3-4>{
        \mintc[firstline=190,lastline=192]{../ocaml/runtime/amd64.S}
        \mintc[firstline=234,lastline=237]{../ocaml/runtime/amd64.S}
      }
      \onslide*<2>{
        \mintc[firstline=190,lastline=192,highlightlines={191-192}]{../ocaml/runtime/amd64.S}
        \mintc[firstline=234,lastline=237,highlightlines={235-237}]{../ocaml/runtime/amd64.S}
      }
      \onslide*<1>{\listCamlRaiseExn[highlightlines=705]{}}%
      \onslide*<2>{\listCamlRaiseExn[highlightlines={707-708}]{}}%
      \onslide*<3>{\listCamlRaiseExn[highlightlines={712-714}]{}}%
      \onslide*<4>{\listCamlRaiseExn[highlightlines={715-720}]{}}%
      \onslide*<5>{\providecommand\step{1}\input{diagrams/backtrace/backtrace.tex}}%
      \onslide*<6>{\providecommand\step{2}\input{diagrams/backtrace/backtrace.tex}}%
    \end{column}
  \end{columns}
\end{frame}

\newcommand\listStashBacktrace[1][]{
  \listc[firstline=91,lastline=120,#1]{../ocaml/runtime/backtrace\_nat.c}{runtime/backtrace\_nat.c:91}}

\begin{frame}{Backtraces}{Introducing \funcname{caml\_stash\_backtrace}}
  \begin{columns}[c]
    \begin{column}{0.5\textwidth}
      \minipage[c][0.66\textheight][s]{\columnwidth}
      \begin{itemize}
        \item<1-> A C function provided by the runtime (specific to native code)
        \item<2-> It scans the stack, between the stack pointer at the raising point up to the next trap, in order to collect the names of the detected function frames
          \onslide*<2>{
            \begin{itemize}
              \item And more information, like line number, column number...
            \end{itemize}
          }
        \item<3-> It uses the same technique and information as the Garbage Collector (GC) uses to scan the values live on the stack
          \onslide*<4>{
            \begin{itemize}
              \item \typename{frame\_descr} are at the core of stack unwinding
            \end{itemize}
          }
      \end{itemize}
      \vfill
      \mintocaml[firstline=9,lastline=13,highlightlines=12]{../src/backtrace/backtrace.ml}
      \endminipage
    \end{column}
    \begin{column}{0.5\textwidth}
      \onslide*<1>{\listStashBacktrace[highlightlines=91]{}}
      \onslide*<2>{\listStashBacktrace[highlightlines={91,117-118}]{}}
      \onslide*<3->{\listStashBacktrace[highlightlines={94,106,109-110}]{}}
    \end{column}
  \end{columns}
\end{frame}

\newcommand\listFrameDescr[1][]{
  \listocaml[firstline=111,lastline=124,#1]{../ocaml/asmcomp/emitaux.ml}{asmcomp/emitaux.ml:111}}
\newcommand\listDebugInfo[1][]{
  \listocaml[firstline=38,lastline=49,#1]{../ocaml/lambda/debuginfo.mli}{lambda/debuginfo.mli:38}}

\begin{frame}{Backtraces}{\typename{frame\_descr} and \typename{frame\_debuginfo} in a nutshell}
  \begin{columns}[c]
    \begin{column}{0.5\textwidth}
      \begin{itemize}
        \item<1-> For every return address in an OCaml program, there is a associated \typename{frame\_descr}
        \item<2-> A \typename{frame\_descr} specifies
        \begin{itemize}
          \item<2-> The size of the function stack frame
          \item<3-> Where live values are on the stack (so that the GC can mark them as live and prevent from being swept)
        \end{itemize}
        \item<4-> When compiled with debug information, \typename{frame\_descr} will be linked to a \typename{frame\_debuginfo}
        \begin{itemize}
          \item<5-> Contains information about the position in the source code (file name, line and column number)
        \end{itemize}
      \end{itemize}
    \end{column}
    \begin{column}{0.5\textwidth}
        \onslide*<1>{\listFrameDescr[highlightlines={118-122}]{}}
        \onslide*<2>{\listFrameDescr[highlightlines={120}]{}}
        \onslide*<3>{\listFrameDescr[highlightlines={121}]{}}
        \onslide*<4->{\listFrameDescr[highlightlines={122}]{}}
        \medskip
        \onslide*<1-3>{\listDebugInfo{}}
        \onslide*<4>{\listDebugInfo[highlightlines={38,49}]{}}
        \onslide*<5>{\listDebugInfo[highlightlines={39-46}]{}}
    \end{column}
  \end{columns}
\end{frame}

\begin{frame}{Backtraces}{Scanning the stack with \typename{frame\_descr}}
  \begin{columns}[c]
    \begin{column}{0.5\textwidth}
      \begin{itemize}
        \item Given the return address of the code that raises the exception by calling \funcname{caml\_raise\_exn} (\funcarg{pc} argument of \funcname{caml\_stash\_backtrace})
        \item And the position of the stack pointer (\funcarg{sp} argument of \funcname{caml\_stash\_backtrace})
        \item The frame size given by \typename{frame\_descr}.\funcarg{frame\_size}, allows to find the end of the previous frame
        \item And the return address of the caller of this function
        \bigskip
        \onslide<3->{
        \item Note that traps (here the reddish \funcname{ExnA}, \funcname{ExnB} and \funcname{ExnC} blocks) belong to the frame that installed them, and so the frame size changes when traps are removed
        }
      \end{itemize}
    \end{column}
    \begin{column}{0.5\textwidth}
      \raggedleft
      \onslide*<1>{\providecommand\step{2}\input{diagrams/backtrace/backtrace.tex}}%
      \onslide*<2>{\providecommand\step{1}\input{diagrams/backtrace/scanstack.tex}}%
      \onslide*<3>{\providecommand\step{2}\input{diagrams/backtrace/scanstack.tex}}%
      \onslide*<4>{\providecommand\step{3}\input{diagrams/backtrace/scanstack.tex}}%
      \onslide*<5>{\providecommand\step{4}\input{diagrams/backtrace/scanstack.tex}}%
      \onslide*<6>{\providecommand\step{5}\input{diagrams/backtrace/scanstack.tex}}%
    \end{column}
  \end{columns}
\end{frame}

% Duplicate of previous-previous slide
\begin{frame}{Backtraces}{Continuing on \funcname{caml\_stash\_backtrace}}
  \begin{columns}[c]
    \begin{column}{0.5\textwidth}
      \minipage[c][0.75\textheight][s]{\columnwidth}
      \begin{itemize}
          \color{gray}
        \item A C function provided by the runtime (specific to native code)
        \item It scans the stack, between the stack pointer at the raising point up to the next trap, in order to collect the names of the detected function frames
        \item It uses the same technique and information as the Garbage Collector (GC) uses to scan the values live on the stack
      \end{itemize}
      \begin{itemize}
        \item For each function frame detected, store a pointer to the \typename{frame\_descr} in the \localname{domain\_state->backtrace\_buffer} array, effectively constructing the backtrace
      \end{itemize}
      \onslide<2->{
        \begin{itemize}
            \smallskip
          \item Constructed backtrace:
            \funcname{definitely\_raise},
            \onslide<3->{\funcname{probably\_raise}}
        \end{itemize}
      }
      \vfill
      \mintocaml[firstline=9,lastline=13,highlightlines=12]{../src/backtrace/backtrace.ml}
      \endminipage
    \end{column}
    \begin{column}{0.5\textwidth}
      \raggedleft
      \onslide*<1>{\listStashBacktrace[highlightlines={114-115}]{}}
      \onslide*<2>{\providecommand\step{3}\input{diagrams/backtrace/backtrace.tex}}%
      \onslide*<3>{\providecommand\step{4}\input{diagrams/backtrace/backtrace.tex}}%
    \end{column}
  \end{columns}
\end{frame}

\begin{frame}{Backtraces}{Back to \funcname{caml\_raise\_exn}}
  \begin{columns}[c]
    \begin{column}{0.5\textwidth}
      \minipage[c][0.66\textheight][s]{\columnwidth}
      \begin{itemize}\color{gray}
        \item Checks wheteher exceptions backtrace should be recorded, by checking the \funcarg{domain\_state->backtrace\_active} value
        \item Switches to the C stack to be able to execute C code
        \item Calls \funcname{caml\_stash\_backtrace}, to collect the backtrace up to the next trap
      \end{itemize}
      \onslide*<2->{
        \begin{itemize}
          \item<2-> Executes the exception handler in \funcname{probably\_raise} with \funcname{RESTORE\_EXN\_HANDLER\_OCAML}
            \begin{itemize}
              \item Set \regname{RSP} to \localname{domain\_state->exn\_handler} value
              \item Restore \localname{domain\_state->exn\_handler} from trap
              \item Jump to the handler code with \funcname{ret}
            \end{itemize}
        \end{itemize}
      }
      \vfill
      \onslide*<1>{\mintocaml[firstline=9,lastline=13,highlightlines=12]{../src/backtrace/backtrace.ml}}
      \onslide*<2->{\mintocaml[firstline=15,lastline=21,highlightlines={19-21}]{../src/backtrace/backtrace.ml}}
      \endminipage
    \end{column}
    \begin{column}{0.5\textwidth}
      \centering
      \onslide*<1>{
        \mintc[firstline=190,lastline=192]{../ocaml/runtime/amd64.S}
        \mintc[firstline=234,lastline=237]{../ocaml/runtime/amd64.S}
      }
      \onslide*<2>{
        \mintc[firstline=190,lastline=192,highlightlines={191-192}]{../ocaml/runtime/amd64.S}
        \mintc[firstline=234,lastline=237,highlightlines={235-237}]{../ocaml/runtime/amd64.S}
      }
      \onslide*<1>{\listCamlRaiseExn[highlightlines=720]{}}
      \onslide*<2>{\listCamlRaiseExn[highlightlines=722]{}}
      \onslide*<3->{\providecommand\step{5}\input{diagrams/backtrace/backtrace.tex}}
    \end{column}
  \end{columns}
\end{frame}

\begin{frame}{Backtraces}{\funcname{caml\_probably\_raise}'s handler doesn't catch \typename{ExnC}}
  \begin{columns}[c]
    \begin{column}{0.45\textwidth}
      \minipage[c][0.75\textheight][s]{\columnwidth}
      \begin{itemize}
        \item<1-> \funcname{match \dots with}\footnotemark pattern matching can also match exceptions
        \item<1-> The CMM of \funcname{probably\_raise} shows that this \funcname{match \dots with} is implemented with a pattern matching and a \funcname{try \dots with}
        \item<1-> But this one only matches on \typename{ExnA} exception
        \item<2-> Since the pattern matching fails to catch the exception, \funcname{reraise} is called with our current \typename{ExnC} exception
        \item<3-> Disassembly shows that \funcname{reraise} is actually a call to \funcname{caml\_reraise\_exn}, which is also an assembly function provided by the runtime
      \end{itemize}
      \vfill
      \mintocaml[firstline=15,lastline=21,highlightlines={18-21}]{../src/backtrace/backtrace.ml}
      \endminipage
    \end{column}
    \begin{column}{0.55\textwidth}
      \centering
      \onslide*<1>{\mintcmm[highlightlines={10-18}]{../src/backtrace/camlBacktrace\_\_probably\_raise\_351.cmm}}
      \onslide*<2>{\listcmm[highlightlines=18]{../src/backtrace/camlBacktrace\_\_probably\_raise_351.cmm}{CMM of probably\_raise}}
      \onslide*<3>{\mintobjdump[firstline=5,lastline=35,highlightlines=31]{../src/backtrace/camlBacktrace\_\_probably\_raise_351.objdump}}
    \end{column}
  \end{columns}
  \only<1>{\footnotetext[1]{https://blog.janestreet.com/pattern-matching-and-exception-handling-unite/}}
\end{frame}

\newcommand\listCamlReraiseExn[1][]{
  \listasm[firstline=727,lastline=734,#1]{../ocaml/runtime/amd64.S}{runtime/amd64.S:727}}

\begin{frame}{Backtraces}{Introducing \funcname{caml\_reraise\_exn}}
  \begin{columns}[c]
    \begin{column}{0.5\textwidth}
      \minipage[c][0.66\textheight][s]{\columnwidth}
      \begin{itemize}
        \item<1-> The exception have to be reraised to the next handler
        \item<2-> First, checks if the backtrace should be collected with \funcarg{domain\_state->backtrace\_active}
        \only<3>{
        \begin{itemize}
            \item If not, behaves like a \funcname{raise\_notrace} with \funcname{RESTORE\_EXN\_HANDLER\_OCAML}
        \end{itemize}
        }
        \item<4-> Jumps into \funcname{caml\_raise\_exn}
        \item<5-> Calls \funcname{caml\_stash\_backtrace} again, to scan the next part of the stack: from the trap just pop-ed to the next trap
      \end{itemize}
      \vfill
      \mintocaml[firstline=15,lastline=21,highlightlines={18-21}]{../src/backtrace/backtrace.ml}
      \endminipage
    \end{column}
    \begin{column}{0.5\textwidth}
      \raggedleft
      \onslide*<1>{\listCamlReraiseExn{}}
      \onslide*<2>{\listCamlReraiseExn[highlightlines=729]{}}
      \onslide*<3>{
        \mintc[firstline=190,lastline=192,highlightlines={191-192}]{../ocaml/runtime/amd64.S}
        \mintc[firstline=234,lastline=237,highlightlines={235-237}]{../ocaml/runtime/amd64.S}
        \medskip
        \listCamlReraiseExn[highlightlines=731]{}
      }
      \onslide*<4-5>{
        % TODO refactor with \listCamlReraiseExn
        \mintasm[firstline=727,lastline=734,highlightlines=730]{../ocaml/runtime/amd64.S}
        \medskip
      }
      \onslide*<4>{\listCamlRaiseExn[highlightlines=711]{}}
      \onslide*<5>{\listCamlRaiseExn[highlightlines=720]{}}
    \end{column}
  \end{columns}
\end{frame}

\begin{frame}{Backtraces}{Backtrace collection continues}
  \begin{columns}[c]
    \begin{column}{0.55\textwidth}
      \minipage[c][0.75\textheight][s]{\columnwidth}
      \begin{itemize}
        \item<1-> \funcname{caml\_stash\_backtrace} scans the older part of the stack, up to the next trap
        \item<6-> Controls jumps to the nested exception handler in \funcname{maybe\_raise}, which matches only on \typename{ExnB}
        \item<7-> \funcname{caml\_reraise\_exn} is called again, backtrace collection resumes with \funcname{caml\_stash\_backtrace}
      \end{itemize}
      \medskip
      \begin{itemize}
        \item<2-> Constructed backtrace:
        \begin{itemize}
          \item <2-> \funcname{definitely\_raise}
          \item <2-> \funcname{probably\_raise}
          \item <3-> \funcname{probably\_raise} (re-raise)
          \item <4-> \funcname{maybe\_raise}
          \item <5-> \funcname{main}
          \item <8-> \funcname{main} (re-raise)
        \end{itemize}
      \end{itemize}
      \vfill
      \onslide*<1-5>{\mintocaml[firstline=15,lastline=21]{../src/backtrace/backtrace.ml}}
      \onslide*<6>{\mintocaml[firstline=28,lastline=34,highlightlines=31]{../src/backtrace/backtrace.ml}}
      \onslide*<7->{\mintocaml[firstline=28,lastline=34,highlightlines=33]{../src/backtrace/backtrace.ml}}
      \endminipage
    \end{column}
    \begin{column}{0.45\textwidth}
      \raggedleft
      \onslide*<1>{\providecommand\step{6}\input{diagrams/backtrace/backtrace.tex}}%
      \onslide*<2>{\providecommand\step{6}\input{diagrams/backtrace/backtrace.tex}}%
      \onslide*<3>{\providecommand\step{7}\input{diagrams/backtrace/backtrace.tex}}%
      \onslide*<4>{\providecommand\step{8}\input{diagrams/backtrace/backtrace.tex}}%
      \onslide*<5>{\providecommand\step{9}\input{diagrams/backtrace/backtrace.tex}}%
      \onslide*<6>{\providecommand\step{10}\input{diagrams/backtrace/backtrace.tex}}%
      \onslide*<7>{\providecommand\step{11}\input{diagrams/backtrace/backtrace.tex}}%
      \onslide*<8>{\providecommand\step{12}\input{diagrams/backtrace/backtrace.tex}}%
      \onslide*<9>{\providecommand\step{13}\input{diagrams/backtrace/backtrace.tex}}%
    \end{column}
  \end{columns}
\end{frame}

\begin{frame}{Backtraces}{Printing the backtrace}
  \begin{columns}[c]
    \begin{column}{0.45\textwidth}
      \minipage[c][0.66\textheight][s]{\columnwidth}
      \begin{itemize}
        \item<1-> The exception backtrace can now be printed to \funcarg{stdout} with \funcname{Printexc.print_backtrace}
        \item<2-> First the exception backtrace is retrieved into an OCaml array with \funcname{get_raw_backtrace }
        \item<3-> Which is implemented as the \funcname{caml_get_exception_raw_backtrace} C function in the runtime
        \item<4-> And finally printed with \funcname{print_exception_backtrace}
      \end{itemize}
      \vfill
      \mintocaml[firstline=28,lastline=34,highlightlines=33]{../src/backtrace/backtrace.ml}
      \endminipage
    \end{column}
    \begin{column}{0.55\textwidth}
      \onslide*<1,2>{
        \mintocaml[firstline=100,lastline=101]{../ocaml/stdlib/printexc.ml}
      }
      \only<1>{
        \listocaml[firstline=170,lastline=172]{../ocaml/stdlib/printexc.ml}{stdlib/printexc.ml:170}
      }
      \only<2>{
        \listocaml[firstline=170,lastline=172,highlightlines=172]{../ocaml/stdlib/printexc.ml}{stdlib/printexc.ml:170}
      }
      \only<3>{
        \mintc[firstline=154,lastline=156]{../ocaml/runtime/backtrace.c}
        \listc[firstline=171,lastline=192,highlightlines={184-187}]{../ocaml/runtime/backtrace.c}{runtime/backtrace.c:154}
      }
      \only<4>{
        \listocaml[firstline=155,lastline=165]{../ocaml/stdlib/printexc.ml}{stdlib/printexc.ml:55}
      }
    \end{column}
  \end{columns}
\end{frame}


\subsection{The multiple kinds of raise}
\frameSubsection{}{}

\begin{frame}{Backtraces}{When backtraces are not needed}
  \begin{columns}[c]
    \begin{column}{0.45\textwidth}
      \minipage[c][0.66\textheight][s]{\columnwidth}
      \begin{itemize}
        \item<1-> What if the backtrace for an exception is never needed?
        \item<2-> It's possible to raise the exception explicitly with \funcname{raise\_notrace} to make raising as cheap as possible
        \item<3-> An example of use in the \funcname{dynlink} library of the OCaml compiler
      \end{itemize}
      \vfill
      \onslide<3->{
      \listocaml[firstline=205,lastline=209,highlightlines=207]{../ocaml/otherlibs/dynlink/dynlink\_compilerlibs/misc.ml}{otherlibs/dynlink/dynlink\_compilerlibs/misc.ml:205}
      }
      \endminipage
    \end{column}
    \begin{column}{0.55\textwidth}
    \only<4>{
      \mintcmm[highlightlines=3]{../src/notrace/camlNotrace\_\_fun_347.cmm}
      \listcmm{../src/notrace/camlNotrace\_\_all\_somes\_327.cmm}{all\_somes}
    }
    \onslide*<5->{
      \listobjdump[highlightlines={6-9}]{../src/notrace/camlNotrace\_\_fun_347.objdump}{anonymous function from \funcname{all\_somes}}
    }
    \end{column}
  \end{columns}
\end{frame}

\begin{frame}{Backtraces}{Summary table of the multiple kinds of raise}
  \begin{itemize}
    \item When debugging information is enabled and if the backtrace of an exception isn't needed, it's possible to raise with \funcname{raise\_notrace} for performance
    \item When debugging information is \emph{not} enabled, the compiler optimises \funcname{raise} calls to inline assembly (same effect as using \funcname{raise\_notrace})
  \end{itemize}
  \bigskip
  \centering
  \begin{tabular}{| l | c | c | c | c |}
    \hline
    \parbox{10em}{Debug information \\ (\funcarg{-g} flag)}  & \multicolumn{2}{|c|}{Disabled}                                     & \multicolumn{2}{|c|}{Enabled} \\
    \hline
    Raise kind                                               & \funcname{raise} & \funcname{raise\_notrace}                       & \funcname{raise} & \funcname{raise\_notrace} \\
    \hline
    Compilation                                              & \parbox{8em}{inline assembly\\ (optimization)} & inline assembly   & \funcname{caml\_raise\_exn} & inline assembly \\
    \hline
  \end{tabular}
\end{frame}


%
% Takeaway #5
%
\subsection*{Takeaway \#5}
\frameSubsectionTakeaway{}
\againframe<5>{takeaway}
}%
      \onslide*<6>{\providecommand\step{10}%
% Backtraces
%
\subsection{One advantage of raising with debugging information}
\frameSubsection{}{}

\begin{frame}{Backtraces}{Debugging information \& source code}
  \begin{columns}[c]
    \begin{column}{0.5\textwidth}
      \begin{itemize}
        \item Raising an exception is fun and games, but how do we know where the exception originated? $\rightarrow$ With the backtrace associated with the exception!
        \item In order to get a backtrace from the point where an exception was raised, it's necessary to compile with debugging information enabled (\funcarg{-g} flag for the compiler)
        \item Same example as in the `Nested handlers' section
      \end{itemize}
    \end{column}
    \begin{column}{0.5\textwidth}
      \listocaml[firstline=9,lastline=34]{../src/backtrace/backtrace.ml}{backtrace/backtrace.ml}
    \end{column}
  \end{columns}
\end{frame}

\begin{frame}{Backtraces}{CMM and disassembly of \funcname{definitely\_raise}}
  \begin{columns}[c]
    \begin{column}{0.5\textwidth}
      \minipage[c][0.66\textheight][s]{\columnwidth}
      \begin{itemize}
        \item<1-> An effect of compiling with debugging information (\funcarg{-g}): the CMM no longer uses \funcname{raise\_notrace} to raise the exception but \funcname{raise} instead
        \item<2-> Disassembly shows that CMM \funcname{raise} is actually a call to \funcname{caml\_raise\_exn}, a function in assembler provided by the runtime
      \end{itemize}
      \vfill
      \mintocaml[firstline=9,lastline=13,highlightlines=12]{../src/backtrace/backtrace.ml}
      \endminipage
    \end{column}
    \begin{column}{0.5\textwidth}
      \onslide*<1>{
        \mintcmm[highlightlines=5]{../src/backtrace/camlBacktrace\_\_definitely\_raise\_347.cmm}
      }
      \onslide*<2>{
        \mintobjdump[firstline=2,highlightlines=14]{../src/backtrace/camlBacktrace\_\_definitely\_raise\_347.objdump}
      }
    \end{column}
  \end{columns}
\end{frame}

\begin{frame}{Backtraces}{Introducing \funcname{caml\_raise\_exn}}
  \begin{columns}[c]
    \begin{column}{0.5\textwidth}
      \minipage[c][0.66\textheight][s]{\columnwidth}
      \onslide*<1>{
        \begin{itemize}
          \item Raising the exception is performed using \funcname{caml\_raise\_exn} which is
            \begin{itemize}
              \item An assembly function provided by the runtime
              \item Architecture specific
            \end{itemize}
          \item Let's see what it does differently from \funcname{raise\_notrace}\dots
          \item We are about to raise \typename{ExnC} with \funcname{caml\_raise\_exn} from \funcname{definitely\_raise}
        \end{itemize}
      }
      \onslide*<2>{
        \mintobjdump[firstline=2,highlightlines=14]{../src/backtrace/camlBacktrace\_\_definitely\_raise\_347.objdump}
      }
      \vfill
      \mintocaml[firstline=9,lastline=13,highlightlines=12]{../src/backtrace/backtrace.ml}
      \endminipage
    \end{column}
    \begin{column}{0.5\textwidth}
      \centering
      \providecommand\step{1}\input{diagrams/backtrace/backtrace.tex}
    \end{column}
  \end{columns}
\end{frame}

\begin{frame}{Backtraces}{But before that, we need more fields from \typename{caml\_domain\_state}}
  \begin{columns}
    \begin{column}{0.5\textwidth}
      \begin{itemize}
        \item Remember \typename{caml\_domain\_state} the per-domain runtime state?
        \item Some other members are of interest to us now\dots
        \item \localname{backtrace\_active} is used to know if the runtime should collect backtraces
        \item \localname{backtrace\_active} can be set by
          \begin{itemize}
            \item The \funcarg{b} flag in the \funcname{OCAMLRUNPARAM} environment variable
            \item \mintinline{ocaml}{Printexc.record_backtrace : bool -> unit} \footnotemark
          \end{itemize}
      \end{itemize}
    \end{column}
    \begin{column}{0.5\textwidth}
      \begin{listing}
        \mintc[highlightlines={9-11}]{src/ocaml/domain\_state\_backtrace.h}
        \caption{runtime/caml/domain\_state.h}
      \end{listing}
    \end{column}
  \end{columns}
  \footnotetext[1]{https://v2.ocaml.org/api/Printexc.html}
\end{frame}

\newcommand\listCamlRaiseExn[1][]{
  \listasm[firstline=702,lastline=725,#1]{../ocaml/runtime/amd64.S}{runtime/amd64.S:702}}

\begin{frame}{Backtraces}{Walk through \funcname{caml\_raise\_exn}}
  \begin{columns}[T]
    \begin{column}{0.5\textwidth}
      \begin{itemize}
        \item<1-> First checks whether exceptions backtrace should be recorded
        \item<1-> By checking the \funcarg{domain\_state->backtrace\_active} value
        \item<2-> If not, acts as \funcname{raise\_notrace} with \funcname{RESTORE\_EXN\_HANDLER\_OCAML}
          \begin{itemize}
            \item Set \regname{RSP} to \localname{domain\_state->exn\_handler} value
            \item Restore \localname{domain\_state->exn\_handler} from trap
            \item Jump with \funcname{ret} (same effect as using \funcname{pop} and \funcname{jmp})
          \end{itemize}
        \item<3-> Otherwise, switches to the C stack to be able to execute C code
        \item<4-> Calls \funcname{caml\_stash\_backtrace}, a C function provided by the runtime\ignorespaces
          \onslide<6->{, with
            \begin{itemize}
              \item \funcarg{exn}: exception value
              \item \funcarg{pc}: code address of the raising point
              \item \funcarg{sp}: stack address of the raising function
              \item \funcarg{trapsp}: stack address of the next handler
            \end{itemize}
          }
      \end{itemize}
      \bigskip
      \mintocaml[firstline=9,lastline=13,highlightlines=12]{../src/backtrace/backtrace.ml}
    \end{column}
    \begin{column}{0.5\textwidth}
      \raggedleft
      \onslide*<1,3-4>{
        \mintc[firstline=190,lastline=192]{../ocaml/runtime/amd64.S}
        \mintc[firstline=234,lastline=237]{../ocaml/runtime/amd64.S}
      }
      \onslide*<2>{
        \mintc[firstline=190,lastline=192,highlightlines={191-192}]{../ocaml/runtime/amd64.S}
        \mintc[firstline=234,lastline=237,highlightlines={235-237}]{../ocaml/runtime/amd64.S}
      }
      \onslide*<1>{\listCamlRaiseExn[highlightlines=705]{}}%
      \onslide*<2>{\listCamlRaiseExn[highlightlines={707-708}]{}}%
      \onslide*<3>{\listCamlRaiseExn[highlightlines={712-714}]{}}%
      \onslide*<4>{\listCamlRaiseExn[highlightlines={715-720}]{}}%
      \onslide*<5>{\providecommand\step{1}\input{diagrams/backtrace/backtrace.tex}}%
      \onslide*<6>{\providecommand\step{2}\input{diagrams/backtrace/backtrace.tex}}%
    \end{column}
  \end{columns}
\end{frame}

\newcommand\listStashBacktrace[1][]{
  \listc[firstline=91,lastline=120,#1]{../ocaml/runtime/backtrace\_nat.c}{runtime/backtrace\_nat.c:91}}

\begin{frame}{Backtraces}{Introducing \funcname{caml\_stash\_backtrace}}
  \begin{columns}[c]
    \begin{column}{0.5\textwidth}
      \minipage[c][0.66\textheight][s]{\columnwidth}
      \begin{itemize}
        \item<1-> A C function provided by the runtime (specific to native code)
        \item<2-> It scans the stack, between the stack pointer at the raising point up to the next trap, in order to collect the names of the detected function frames
          \onslide*<2>{
            \begin{itemize}
              \item And more information, like line number, column number...
            \end{itemize}
          }
        \item<3-> It uses the same technique and information as the Garbage Collector (GC) uses to scan the values live on the stack
          \onslide*<4>{
            \begin{itemize}
              \item \typename{frame\_descr} are at the core of stack unwinding
            \end{itemize}
          }
      \end{itemize}
      \vfill
      \mintocaml[firstline=9,lastline=13,highlightlines=12]{../src/backtrace/backtrace.ml}
      \endminipage
    \end{column}
    \begin{column}{0.5\textwidth}
      \onslide*<1>{\listStashBacktrace[highlightlines=91]{}}
      \onslide*<2>{\listStashBacktrace[highlightlines={91,117-118}]{}}
      \onslide*<3->{\listStashBacktrace[highlightlines={94,106,109-110}]{}}
    \end{column}
  \end{columns}
\end{frame}

\newcommand\listFrameDescr[1][]{
  \listocaml[firstline=111,lastline=124,#1]{../ocaml/asmcomp/emitaux.ml}{asmcomp/emitaux.ml:111}}
\newcommand\listDebugInfo[1][]{
  \listocaml[firstline=38,lastline=49,#1]{../ocaml/lambda/debuginfo.mli}{lambda/debuginfo.mli:38}}

\begin{frame}{Backtraces}{\typename{frame\_descr} and \typename{frame\_debuginfo} in a nutshell}
  \begin{columns}[c]
    \begin{column}{0.5\textwidth}
      \begin{itemize}
        \item<1-> For every return address in an OCaml program, there is a associated \typename{frame\_descr}
        \item<2-> A \typename{frame\_descr} specifies
        \begin{itemize}
          \item<2-> The size of the function stack frame
          \item<3-> Where live values are on the stack (so that the GC can mark them as live and prevent from being swept)
        \end{itemize}
        \item<4-> When compiled with debug information, \typename{frame\_descr} will be linked to a \typename{frame\_debuginfo}
        \begin{itemize}
          \item<5-> Contains information about the position in the source code (file name, line and column number)
        \end{itemize}
      \end{itemize}
    \end{column}
    \begin{column}{0.5\textwidth}
        \onslide*<1>{\listFrameDescr[highlightlines={118-122}]{}}
        \onslide*<2>{\listFrameDescr[highlightlines={120}]{}}
        \onslide*<3>{\listFrameDescr[highlightlines={121}]{}}
        \onslide*<4->{\listFrameDescr[highlightlines={122}]{}}
        \medskip
        \onslide*<1-3>{\listDebugInfo{}}
        \onslide*<4>{\listDebugInfo[highlightlines={38,49}]{}}
        \onslide*<5>{\listDebugInfo[highlightlines={39-46}]{}}
    \end{column}
  \end{columns}
\end{frame}

\begin{frame}{Backtraces}{Scanning the stack with \typename{frame\_descr}}
  \begin{columns}[c]
    \begin{column}{0.5\textwidth}
      \begin{itemize}
        \item Given the return address of the code that raises the exception by calling \funcname{caml\_raise\_exn} (\funcarg{pc} argument of \funcname{caml\_stash\_backtrace})
        \item And the position of the stack pointer (\funcarg{sp} argument of \funcname{caml\_stash\_backtrace})
        \item The frame size given by \typename{frame\_descr}.\funcarg{frame\_size}, allows to find the end of the previous frame
        \item And the return address of the caller of this function
        \bigskip
        \onslide<3->{
        \item Note that traps (here the reddish \funcname{ExnA}, \funcname{ExnB} and \funcname{ExnC} blocks) belong to the frame that installed them, and so the frame size changes when traps are removed
        }
      \end{itemize}
    \end{column}
    \begin{column}{0.5\textwidth}
      \raggedleft
      \onslide*<1>{\providecommand\step{2}\input{diagrams/backtrace/backtrace.tex}}%
      \onslide*<2>{\providecommand\step{1}\input{diagrams/backtrace/scanstack.tex}}%
      \onslide*<3>{\providecommand\step{2}\input{diagrams/backtrace/scanstack.tex}}%
      \onslide*<4>{\providecommand\step{3}\input{diagrams/backtrace/scanstack.tex}}%
      \onslide*<5>{\providecommand\step{4}\input{diagrams/backtrace/scanstack.tex}}%
      \onslide*<6>{\providecommand\step{5}\input{diagrams/backtrace/scanstack.tex}}%
    \end{column}
  \end{columns}
\end{frame}

% Duplicate of previous-previous slide
\begin{frame}{Backtraces}{Continuing on \funcname{caml\_stash\_backtrace}}
  \begin{columns}[c]
    \begin{column}{0.5\textwidth}
      \minipage[c][0.75\textheight][s]{\columnwidth}
      \begin{itemize}
          \color{gray}
        \item A C function provided by the runtime (specific to native code)
        \item It scans the stack, between the stack pointer at the raising point up to the next trap, in order to collect the names of the detected function frames
        \item It uses the same technique and information as the Garbage Collector (GC) uses to scan the values live on the stack
      \end{itemize}
      \begin{itemize}
        \item For each function frame detected, store a pointer to the \typename{frame\_descr} in the \localname{domain\_state->backtrace\_buffer} array, effectively constructing the backtrace
      \end{itemize}
      \onslide<2->{
        \begin{itemize}
            \smallskip
          \item Constructed backtrace:
            \funcname{definitely\_raise},
            \onslide<3->{\funcname{probably\_raise}}
        \end{itemize}
      }
      \vfill
      \mintocaml[firstline=9,lastline=13,highlightlines=12]{../src/backtrace/backtrace.ml}
      \endminipage
    \end{column}
    \begin{column}{0.5\textwidth}
      \raggedleft
      \onslide*<1>{\listStashBacktrace[highlightlines={114-115}]{}}
      \onslide*<2>{\providecommand\step{3}\input{diagrams/backtrace/backtrace.tex}}%
      \onslide*<3>{\providecommand\step{4}\input{diagrams/backtrace/backtrace.tex}}%
    \end{column}
  \end{columns}
\end{frame}

\begin{frame}{Backtraces}{Back to \funcname{caml\_raise\_exn}}
  \begin{columns}[c]
    \begin{column}{0.5\textwidth}
      \minipage[c][0.66\textheight][s]{\columnwidth}
      \begin{itemize}\color{gray}
        \item Checks wheteher exceptions backtrace should be recorded, by checking the \funcarg{domain\_state->backtrace\_active} value
        \item Switches to the C stack to be able to execute C code
        \item Calls \funcname{caml\_stash\_backtrace}, to collect the backtrace up to the next trap
      \end{itemize}
      \onslide*<2->{
        \begin{itemize}
          \item<2-> Executes the exception handler in \funcname{probably\_raise} with \funcname{RESTORE\_EXN\_HANDLER\_OCAML}
            \begin{itemize}
              \item Set \regname{RSP} to \localname{domain\_state->exn\_handler} value
              \item Restore \localname{domain\_state->exn\_handler} from trap
              \item Jump to the handler code with \funcname{ret}
            \end{itemize}
        \end{itemize}
      }
      \vfill
      \onslide*<1>{\mintocaml[firstline=9,lastline=13,highlightlines=12]{../src/backtrace/backtrace.ml}}
      \onslide*<2->{\mintocaml[firstline=15,lastline=21,highlightlines={19-21}]{../src/backtrace/backtrace.ml}}
      \endminipage
    \end{column}
    \begin{column}{0.5\textwidth}
      \centering
      \onslide*<1>{
        \mintc[firstline=190,lastline=192]{../ocaml/runtime/amd64.S}
        \mintc[firstline=234,lastline=237]{../ocaml/runtime/amd64.S}
      }
      \onslide*<2>{
        \mintc[firstline=190,lastline=192,highlightlines={191-192}]{../ocaml/runtime/amd64.S}
        \mintc[firstline=234,lastline=237,highlightlines={235-237}]{../ocaml/runtime/amd64.S}
      }
      \onslide*<1>{\listCamlRaiseExn[highlightlines=720]{}}
      \onslide*<2>{\listCamlRaiseExn[highlightlines=722]{}}
      \onslide*<3->{\providecommand\step{5}\input{diagrams/backtrace/backtrace.tex}}
    \end{column}
  \end{columns}
\end{frame}

\begin{frame}{Backtraces}{\funcname{caml\_probably\_raise}'s handler doesn't catch \typename{ExnC}}
  \begin{columns}[c]
    \begin{column}{0.45\textwidth}
      \minipage[c][0.75\textheight][s]{\columnwidth}
      \begin{itemize}
        \item<1-> \funcname{match \dots with}\footnotemark pattern matching can also match exceptions
        \item<1-> The CMM of \funcname{probably\_raise} shows that this \funcname{match \dots with} is implemented with a pattern matching and a \funcname{try \dots with}
        \item<1-> But this one only matches on \typename{ExnA} exception
        \item<2-> Since the pattern matching fails to catch the exception, \funcname{reraise} is called with our current \typename{ExnC} exception
        \item<3-> Disassembly shows that \funcname{reraise} is actually a call to \funcname{caml\_reraise\_exn}, which is also an assembly function provided by the runtime
      \end{itemize}
      \vfill
      \mintocaml[firstline=15,lastline=21,highlightlines={18-21}]{../src/backtrace/backtrace.ml}
      \endminipage
    \end{column}
    \begin{column}{0.55\textwidth}
      \centering
      \onslide*<1>{\mintcmm[highlightlines={10-18}]{../src/backtrace/camlBacktrace\_\_probably\_raise\_351.cmm}}
      \onslide*<2>{\listcmm[highlightlines=18]{../src/backtrace/camlBacktrace\_\_probably\_raise_351.cmm}{CMM of probably\_raise}}
      \onslide*<3>{\mintobjdump[firstline=5,lastline=35,highlightlines=31]{../src/backtrace/camlBacktrace\_\_probably\_raise_351.objdump}}
    \end{column}
  \end{columns}
  \only<1>{\footnotetext[1]{https://blog.janestreet.com/pattern-matching-and-exception-handling-unite/}}
\end{frame}

\newcommand\listCamlReraiseExn[1][]{
  \listasm[firstline=727,lastline=734,#1]{../ocaml/runtime/amd64.S}{runtime/amd64.S:727}}

\begin{frame}{Backtraces}{Introducing \funcname{caml\_reraise\_exn}}
  \begin{columns}[c]
    \begin{column}{0.5\textwidth}
      \minipage[c][0.66\textheight][s]{\columnwidth}
      \begin{itemize}
        \item<1-> The exception have to be reraised to the next handler
        \item<2-> First, checks if the backtrace should be collected with \funcarg{domain\_state->backtrace\_active}
        \only<3>{
        \begin{itemize}
            \item If not, behaves like a \funcname{raise\_notrace} with \funcname{RESTORE\_EXN\_HANDLER\_OCAML}
        \end{itemize}
        }
        \item<4-> Jumps into \funcname{caml\_raise\_exn}
        \item<5-> Calls \funcname{caml\_stash\_backtrace} again, to scan the next part of the stack: from the trap just pop-ed to the next trap
      \end{itemize}
      \vfill
      \mintocaml[firstline=15,lastline=21,highlightlines={18-21}]{../src/backtrace/backtrace.ml}
      \endminipage
    \end{column}
    \begin{column}{0.5\textwidth}
      \raggedleft
      \onslide*<1>{\listCamlReraiseExn{}}
      \onslide*<2>{\listCamlReraiseExn[highlightlines=729]{}}
      \onslide*<3>{
        \mintc[firstline=190,lastline=192,highlightlines={191-192}]{../ocaml/runtime/amd64.S}
        \mintc[firstline=234,lastline=237,highlightlines={235-237}]{../ocaml/runtime/amd64.S}
        \medskip
        \listCamlReraiseExn[highlightlines=731]{}
      }
      \onslide*<4-5>{
        % TODO refactor with \listCamlReraiseExn
        \mintasm[firstline=727,lastline=734,highlightlines=730]{../ocaml/runtime/amd64.S}
        \medskip
      }
      \onslide*<4>{\listCamlRaiseExn[highlightlines=711]{}}
      \onslide*<5>{\listCamlRaiseExn[highlightlines=720]{}}
    \end{column}
  \end{columns}
\end{frame}

\begin{frame}{Backtraces}{Backtrace collection continues}
  \begin{columns}[c]
    \begin{column}{0.55\textwidth}
      \minipage[c][0.75\textheight][s]{\columnwidth}
      \begin{itemize}
        \item<1-> \funcname{caml\_stash\_backtrace} scans the older part of the stack, up to the next trap
        \item<6-> Controls jumps to the nested exception handler in \funcname{maybe\_raise}, which matches only on \typename{ExnB}
        \item<7-> \funcname{caml\_reraise\_exn} is called again, backtrace collection resumes with \funcname{caml\_stash\_backtrace}
      \end{itemize}
      \medskip
      \begin{itemize}
        \item<2-> Constructed backtrace:
        \begin{itemize}
          \item <2-> \funcname{definitely\_raise}
          \item <2-> \funcname{probably\_raise}
          \item <3-> \funcname{probably\_raise} (re-raise)
          \item <4-> \funcname{maybe\_raise}
          \item <5-> \funcname{main}
          \item <8-> \funcname{main} (re-raise)
        \end{itemize}
      \end{itemize}
      \vfill
      \onslide*<1-5>{\mintocaml[firstline=15,lastline=21]{../src/backtrace/backtrace.ml}}
      \onslide*<6>{\mintocaml[firstline=28,lastline=34,highlightlines=31]{../src/backtrace/backtrace.ml}}
      \onslide*<7->{\mintocaml[firstline=28,lastline=34,highlightlines=33]{../src/backtrace/backtrace.ml}}
      \endminipage
    \end{column}
    \begin{column}{0.45\textwidth}
      \raggedleft
      \onslide*<1>{\providecommand\step{6}\input{diagrams/backtrace/backtrace.tex}}%
      \onslide*<2>{\providecommand\step{6}\input{diagrams/backtrace/backtrace.tex}}%
      \onslide*<3>{\providecommand\step{7}\input{diagrams/backtrace/backtrace.tex}}%
      \onslide*<4>{\providecommand\step{8}\input{diagrams/backtrace/backtrace.tex}}%
      \onslide*<5>{\providecommand\step{9}\input{diagrams/backtrace/backtrace.tex}}%
      \onslide*<6>{\providecommand\step{10}\input{diagrams/backtrace/backtrace.tex}}%
      \onslide*<7>{\providecommand\step{11}\input{diagrams/backtrace/backtrace.tex}}%
      \onslide*<8>{\providecommand\step{12}\input{diagrams/backtrace/backtrace.tex}}%
      \onslide*<9>{\providecommand\step{13}\input{diagrams/backtrace/backtrace.tex}}%
    \end{column}
  \end{columns}
\end{frame}

\begin{frame}{Backtraces}{Printing the backtrace}
  \begin{columns}[c]
    \begin{column}{0.45\textwidth}
      \minipage[c][0.66\textheight][s]{\columnwidth}
      \begin{itemize}
        \item<1-> The exception backtrace can now be printed to \funcarg{stdout} with \funcname{Printexc.print_backtrace}
        \item<2-> First the exception backtrace is retrieved into an OCaml array with \funcname{get_raw_backtrace }
        \item<3-> Which is implemented as the \funcname{caml_get_exception_raw_backtrace} C function in the runtime
        \item<4-> And finally printed with \funcname{print_exception_backtrace}
      \end{itemize}
      \vfill
      \mintocaml[firstline=28,lastline=34,highlightlines=33]{../src/backtrace/backtrace.ml}
      \endminipage
    \end{column}
    \begin{column}{0.55\textwidth}
      \onslide*<1,2>{
        \mintocaml[firstline=100,lastline=101]{../ocaml/stdlib/printexc.ml}
      }
      \only<1>{
        \listocaml[firstline=170,lastline=172]{../ocaml/stdlib/printexc.ml}{stdlib/printexc.ml:170}
      }
      \only<2>{
        \listocaml[firstline=170,lastline=172,highlightlines=172]{../ocaml/stdlib/printexc.ml}{stdlib/printexc.ml:170}
      }
      \only<3>{
        \mintc[firstline=154,lastline=156]{../ocaml/runtime/backtrace.c}
        \listc[firstline=171,lastline=192,highlightlines={184-187}]{../ocaml/runtime/backtrace.c}{runtime/backtrace.c:154}
      }
      \only<4>{
        \listocaml[firstline=155,lastline=165]{../ocaml/stdlib/printexc.ml}{stdlib/printexc.ml:55}
      }
    \end{column}
  \end{columns}
\end{frame}


\subsection{The multiple kinds of raise}
\frameSubsection{}{}

\begin{frame}{Backtraces}{When backtraces are not needed}
  \begin{columns}[c]
    \begin{column}{0.45\textwidth}
      \minipage[c][0.66\textheight][s]{\columnwidth}
      \begin{itemize}
        \item<1-> What if the backtrace for an exception is never needed?
        \item<2-> It's possible to raise the exception explicitly with \funcname{raise\_notrace} to make raising as cheap as possible
        \item<3-> An example of use in the \funcname{dynlink} library of the OCaml compiler
      \end{itemize}
      \vfill
      \onslide<3->{
      \listocaml[firstline=205,lastline=209,highlightlines=207]{../ocaml/otherlibs/dynlink/dynlink\_compilerlibs/misc.ml}{otherlibs/dynlink/dynlink\_compilerlibs/misc.ml:205}
      }
      \endminipage
    \end{column}
    \begin{column}{0.55\textwidth}
    \only<4>{
      \mintcmm[highlightlines=3]{../src/notrace/camlNotrace\_\_fun_347.cmm}
      \listcmm{../src/notrace/camlNotrace\_\_all\_somes\_327.cmm}{all\_somes}
    }
    \onslide*<5->{
      \listobjdump[highlightlines={6-9}]{../src/notrace/camlNotrace\_\_fun_347.objdump}{anonymous function from \funcname{all\_somes}}
    }
    \end{column}
  \end{columns}
\end{frame}

\begin{frame}{Backtraces}{Summary table of the multiple kinds of raise}
  \begin{itemize}
    \item When debugging information is enabled and if the backtrace of an exception isn't needed, it's possible to raise with \funcname{raise\_notrace} for performance
    \item When debugging information is \emph{not} enabled, the compiler optimises \funcname{raise} calls to inline assembly (same effect as using \funcname{raise\_notrace})
  \end{itemize}
  \bigskip
  \centering
  \begin{tabular}{| l | c | c | c | c |}
    \hline
    \parbox{10em}{Debug information \\ (\funcarg{-g} flag)}  & \multicolumn{2}{|c|}{Disabled}                                     & \multicolumn{2}{|c|}{Enabled} \\
    \hline
    Raise kind                                               & \funcname{raise} & \funcname{raise\_notrace}                       & \funcname{raise} & \funcname{raise\_notrace} \\
    \hline
    Compilation                                              & \parbox{8em}{inline assembly\\ (optimization)} & inline assembly   & \funcname{caml\_raise\_exn} & inline assembly \\
    \hline
  \end{tabular}
\end{frame}


%
% Takeaway #5
%
\subsection*{Takeaway \#5}
\frameSubsectionTakeaway{}
\againframe<5>{takeaway}
}%
      \onslide*<7>{\providecommand\step{11}%
% Backtraces
%
\subsection{One advantage of raising with debugging information}
\frameSubsection{}{}

\begin{frame}{Backtraces}{Debugging information \& source code}
  \begin{columns}[c]
    \begin{column}{0.5\textwidth}
      \begin{itemize}
        \item Raising an exception is fun and games, but how do we know where the exception originated? $\rightarrow$ With the backtrace associated with the exception!
        \item In order to get a backtrace from the point where an exception was raised, it's necessary to compile with debugging information enabled (\funcarg{-g} flag for the compiler)
        \item Same example as in the `Nested handlers' section
      \end{itemize}
    \end{column}
    \begin{column}{0.5\textwidth}
      \listocaml[firstline=9,lastline=34]{../src/backtrace/backtrace.ml}{backtrace/backtrace.ml}
    \end{column}
  \end{columns}
\end{frame}

\begin{frame}{Backtraces}{CMM and disassembly of \funcname{definitely\_raise}}
  \begin{columns}[c]
    \begin{column}{0.5\textwidth}
      \minipage[c][0.66\textheight][s]{\columnwidth}
      \begin{itemize}
        \item<1-> An effect of compiling with debugging information (\funcarg{-g}): the CMM no longer uses \funcname{raise\_notrace} to raise the exception but \funcname{raise} instead
        \item<2-> Disassembly shows that CMM \funcname{raise} is actually a call to \funcname{caml\_raise\_exn}, a function in assembler provided by the runtime
      \end{itemize}
      \vfill
      \mintocaml[firstline=9,lastline=13,highlightlines=12]{../src/backtrace/backtrace.ml}
      \endminipage
    \end{column}
    \begin{column}{0.5\textwidth}
      \onslide*<1>{
        \mintcmm[highlightlines=5]{../src/backtrace/camlBacktrace\_\_definitely\_raise\_347.cmm}
      }
      \onslide*<2>{
        \mintobjdump[firstline=2,highlightlines=14]{../src/backtrace/camlBacktrace\_\_definitely\_raise\_347.objdump}
      }
    \end{column}
  \end{columns}
\end{frame}

\begin{frame}{Backtraces}{Introducing \funcname{caml\_raise\_exn}}
  \begin{columns}[c]
    \begin{column}{0.5\textwidth}
      \minipage[c][0.66\textheight][s]{\columnwidth}
      \onslide*<1>{
        \begin{itemize}
          \item Raising the exception is performed using \funcname{caml\_raise\_exn} which is
            \begin{itemize}
              \item An assembly function provided by the runtime
              \item Architecture specific
            \end{itemize}
          \item Let's see what it does differently from \funcname{raise\_notrace}\dots
          \item We are about to raise \typename{ExnC} with \funcname{caml\_raise\_exn} from \funcname{definitely\_raise}
        \end{itemize}
      }
      \onslide*<2>{
        \mintobjdump[firstline=2,highlightlines=14]{../src/backtrace/camlBacktrace\_\_definitely\_raise\_347.objdump}
      }
      \vfill
      \mintocaml[firstline=9,lastline=13,highlightlines=12]{../src/backtrace/backtrace.ml}
      \endminipage
    \end{column}
    \begin{column}{0.5\textwidth}
      \centering
      \providecommand\step{1}\input{diagrams/backtrace/backtrace.tex}
    \end{column}
  \end{columns}
\end{frame}

\begin{frame}{Backtraces}{But before that, we need more fields from \typename{caml\_domain\_state}}
  \begin{columns}
    \begin{column}{0.5\textwidth}
      \begin{itemize}
        \item Remember \typename{caml\_domain\_state} the per-domain runtime state?
        \item Some other members are of interest to us now\dots
        \item \localname{backtrace\_active} is used to know if the runtime should collect backtraces
        \item \localname{backtrace\_active} can be set by
          \begin{itemize}
            \item The \funcarg{b} flag in the \funcname{OCAMLRUNPARAM} environment variable
            \item \mintinline{ocaml}{Printexc.record_backtrace : bool -> unit} \footnotemark
          \end{itemize}
      \end{itemize}
    \end{column}
    \begin{column}{0.5\textwidth}
      \begin{listing}
        \mintc[highlightlines={9-11}]{src/ocaml/domain\_state\_backtrace.h}
        \caption{runtime/caml/domain\_state.h}
      \end{listing}
    \end{column}
  \end{columns}
  \footnotetext[1]{https://v2.ocaml.org/api/Printexc.html}
\end{frame}

\newcommand\listCamlRaiseExn[1][]{
  \listasm[firstline=702,lastline=725,#1]{../ocaml/runtime/amd64.S}{runtime/amd64.S:702}}

\begin{frame}{Backtraces}{Walk through \funcname{caml\_raise\_exn}}
  \begin{columns}[T]
    \begin{column}{0.5\textwidth}
      \begin{itemize}
        \item<1-> First checks whether exceptions backtrace should be recorded
        \item<1-> By checking the \funcarg{domain\_state->backtrace\_active} value
        \item<2-> If not, acts as \funcname{raise\_notrace} with \funcname{RESTORE\_EXN\_HANDLER\_OCAML}
          \begin{itemize}
            \item Set \regname{RSP} to \localname{domain\_state->exn\_handler} value
            \item Restore \localname{domain\_state->exn\_handler} from trap
            \item Jump with \funcname{ret} (same effect as using \funcname{pop} and \funcname{jmp})
          \end{itemize}
        \item<3-> Otherwise, switches to the C stack to be able to execute C code
        \item<4-> Calls \funcname{caml\_stash\_backtrace}, a C function provided by the runtime\ignorespaces
          \onslide<6->{, with
            \begin{itemize}
              \item \funcarg{exn}: exception value
              \item \funcarg{pc}: code address of the raising point
              \item \funcarg{sp}: stack address of the raising function
              \item \funcarg{trapsp}: stack address of the next handler
            \end{itemize}
          }
      \end{itemize}
      \bigskip
      \mintocaml[firstline=9,lastline=13,highlightlines=12]{../src/backtrace/backtrace.ml}
    \end{column}
    \begin{column}{0.5\textwidth}
      \raggedleft
      \onslide*<1,3-4>{
        \mintc[firstline=190,lastline=192]{../ocaml/runtime/amd64.S}
        \mintc[firstline=234,lastline=237]{../ocaml/runtime/amd64.S}
      }
      \onslide*<2>{
        \mintc[firstline=190,lastline=192,highlightlines={191-192}]{../ocaml/runtime/amd64.S}
        \mintc[firstline=234,lastline=237,highlightlines={235-237}]{../ocaml/runtime/amd64.S}
      }
      \onslide*<1>{\listCamlRaiseExn[highlightlines=705]{}}%
      \onslide*<2>{\listCamlRaiseExn[highlightlines={707-708}]{}}%
      \onslide*<3>{\listCamlRaiseExn[highlightlines={712-714}]{}}%
      \onslide*<4>{\listCamlRaiseExn[highlightlines={715-720}]{}}%
      \onslide*<5>{\providecommand\step{1}\input{diagrams/backtrace/backtrace.tex}}%
      \onslide*<6>{\providecommand\step{2}\input{diagrams/backtrace/backtrace.tex}}%
    \end{column}
  \end{columns}
\end{frame}

\newcommand\listStashBacktrace[1][]{
  \listc[firstline=91,lastline=120,#1]{../ocaml/runtime/backtrace\_nat.c}{runtime/backtrace\_nat.c:91}}

\begin{frame}{Backtraces}{Introducing \funcname{caml\_stash\_backtrace}}
  \begin{columns}[c]
    \begin{column}{0.5\textwidth}
      \minipage[c][0.66\textheight][s]{\columnwidth}
      \begin{itemize}
        \item<1-> A C function provided by the runtime (specific to native code)
        \item<2-> It scans the stack, between the stack pointer at the raising point up to the next trap, in order to collect the names of the detected function frames
          \onslide*<2>{
            \begin{itemize}
              \item And more information, like line number, column number...
            \end{itemize}
          }
        \item<3-> It uses the same technique and information as the Garbage Collector (GC) uses to scan the values live on the stack
          \onslide*<4>{
            \begin{itemize}
              \item \typename{frame\_descr} are at the core of stack unwinding
            \end{itemize}
          }
      \end{itemize}
      \vfill
      \mintocaml[firstline=9,lastline=13,highlightlines=12]{../src/backtrace/backtrace.ml}
      \endminipage
    \end{column}
    \begin{column}{0.5\textwidth}
      \onslide*<1>{\listStashBacktrace[highlightlines=91]{}}
      \onslide*<2>{\listStashBacktrace[highlightlines={91,117-118}]{}}
      \onslide*<3->{\listStashBacktrace[highlightlines={94,106,109-110}]{}}
    \end{column}
  \end{columns}
\end{frame}

\newcommand\listFrameDescr[1][]{
  \listocaml[firstline=111,lastline=124,#1]{../ocaml/asmcomp/emitaux.ml}{asmcomp/emitaux.ml:111}}
\newcommand\listDebugInfo[1][]{
  \listocaml[firstline=38,lastline=49,#1]{../ocaml/lambda/debuginfo.mli}{lambda/debuginfo.mli:38}}

\begin{frame}{Backtraces}{\typename{frame\_descr} and \typename{frame\_debuginfo} in a nutshell}
  \begin{columns}[c]
    \begin{column}{0.5\textwidth}
      \begin{itemize}
        \item<1-> For every return address in an OCaml program, there is a associated \typename{frame\_descr}
        \item<2-> A \typename{frame\_descr} specifies
        \begin{itemize}
          \item<2-> The size of the function stack frame
          \item<3-> Where live values are on the stack (so that the GC can mark them as live and prevent from being swept)
        \end{itemize}
        \item<4-> When compiled with debug information, \typename{frame\_descr} will be linked to a \typename{frame\_debuginfo}
        \begin{itemize}
          \item<5-> Contains information about the position in the source code (file name, line and column number)
        \end{itemize}
      \end{itemize}
    \end{column}
    \begin{column}{0.5\textwidth}
        \onslide*<1>{\listFrameDescr[highlightlines={118-122}]{}}
        \onslide*<2>{\listFrameDescr[highlightlines={120}]{}}
        \onslide*<3>{\listFrameDescr[highlightlines={121}]{}}
        \onslide*<4->{\listFrameDescr[highlightlines={122}]{}}
        \medskip
        \onslide*<1-3>{\listDebugInfo{}}
        \onslide*<4>{\listDebugInfo[highlightlines={38,49}]{}}
        \onslide*<5>{\listDebugInfo[highlightlines={39-46}]{}}
    \end{column}
  \end{columns}
\end{frame}

\begin{frame}{Backtraces}{Scanning the stack with \typename{frame\_descr}}
  \begin{columns}[c]
    \begin{column}{0.5\textwidth}
      \begin{itemize}
        \item Given the return address of the code that raises the exception by calling \funcname{caml\_raise\_exn} (\funcarg{pc} argument of \funcname{caml\_stash\_backtrace})
        \item And the position of the stack pointer (\funcarg{sp} argument of \funcname{caml\_stash\_backtrace})
        \item The frame size given by \typename{frame\_descr}.\funcarg{frame\_size}, allows to find the end of the previous frame
        \item And the return address of the caller of this function
        \bigskip
        \onslide<3->{
        \item Note that traps (here the reddish \funcname{ExnA}, \funcname{ExnB} and \funcname{ExnC} blocks) belong to the frame that installed them, and so the frame size changes when traps are removed
        }
      \end{itemize}
    \end{column}
    \begin{column}{0.5\textwidth}
      \raggedleft
      \onslide*<1>{\providecommand\step{2}\input{diagrams/backtrace/backtrace.tex}}%
      \onslide*<2>{\providecommand\step{1}\input{diagrams/backtrace/scanstack.tex}}%
      \onslide*<3>{\providecommand\step{2}\input{diagrams/backtrace/scanstack.tex}}%
      \onslide*<4>{\providecommand\step{3}\input{diagrams/backtrace/scanstack.tex}}%
      \onslide*<5>{\providecommand\step{4}\input{diagrams/backtrace/scanstack.tex}}%
      \onslide*<6>{\providecommand\step{5}\input{diagrams/backtrace/scanstack.tex}}%
    \end{column}
  \end{columns}
\end{frame}

% Duplicate of previous-previous slide
\begin{frame}{Backtraces}{Continuing on \funcname{caml\_stash\_backtrace}}
  \begin{columns}[c]
    \begin{column}{0.5\textwidth}
      \minipage[c][0.75\textheight][s]{\columnwidth}
      \begin{itemize}
          \color{gray}
        \item A C function provided by the runtime (specific to native code)
        \item It scans the stack, between the stack pointer at the raising point up to the next trap, in order to collect the names of the detected function frames
        \item It uses the same technique and information as the Garbage Collector (GC) uses to scan the values live on the stack
      \end{itemize}
      \begin{itemize}
        \item For each function frame detected, store a pointer to the \typename{frame\_descr} in the \localname{domain\_state->backtrace\_buffer} array, effectively constructing the backtrace
      \end{itemize}
      \onslide<2->{
        \begin{itemize}
            \smallskip
          \item Constructed backtrace:
            \funcname{definitely\_raise},
            \onslide<3->{\funcname{probably\_raise}}
        \end{itemize}
      }
      \vfill
      \mintocaml[firstline=9,lastline=13,highlightlines=12]{../src/backtrace/backtrace.ml}
      \endminipage
    \end{column}
    \begin{column}{0.5\textwidth}
      \raggedleft
      \onslide*<1>{\listStashBacktrace[highlightlines={114-115}]{}}
      \onslide*<2>{\providecommand\step{3}\input{diagrams/backtrace/backtrace.tex}}%
      \onslide*<3>{\providecommand\step{4}\input{diagrams/backtrace/backtrace.tex}}%
    \end{column}
  \end{columns}
\end{frame}

\begin{frame}{Backtraces}{Back to \funcname{caml\_raise\_exn}}
  \begin{columns}[c]
    \begin{column}{0.5\textwidth}
      \minipage[c][0.66\textheight][s]{\columnwidth}
      \begin{itemize}\color{gray}
        \item Checks wheteher exceptions backtrace should be recorded, by checking the \funcarg{domain\_state->backtrace\_active} value
        \item Switches to the C stack to be able to execute C code
        \item Calls \funcname{caml\_stash\_backtrace}, to collect the backtrace up to the next trap
      \end{itemize}
      \onslide*<2->{
        \begin{itemize}
          \item<2-> Executes the exception handler in \funcname{probably\_raise} with \funcname{RESTORE\_EXN\_HANDLER\_OCAML}
            \begin{itemize}
              \item Set \regname{RSP} to \localname{domain\_state->exn\_handler} value
              \item Restore \localname{domain\_state->exn\_handler} from trap
              \item Jump to the handler code with \funcname{ret}
            \end{itemize}
        \end{itemize}
      }
      \vfill
      \onslide*<1>{\mintocaml[firstline=9,lastline=13,highlightlines=12]{../src/backtrace/backtrace.ml}}
      \onslide*<2->{\mintocaml[firstline=15,lastline=21,highlightlines={19-21}]{../src/backtrace/backtrace.ml}}
      \endminipage
    \end{column}
    \begin{column}{0.5\textwidth}
      \centering
      \onslide*<1>{
        \mintc[firstline=190,lastline=192]{../ocaml/runtime/amd64.S}
        \mintc[firstline=234,lastline=237]{../ocaml/runtime/amd64.S}
      }
      \onslide*<2>{
        \mintc[firstline=190,lastline=192,highlightlines={191-192}]{../ocaml/runtime/amd64.S}
        \mintc[firstline=234,lastline=237,highlightlines={235-237}]{../ocaml/runtime/amd64.S}
      }
      \onslide*<1>{\listCamlRaiseExn[highlightlines=720]{}}
      \onslide*<2>{\listCamlRaiseExn[highlightlines=722]{}}
      \onslide*<3->{\providecommand\step{5}\input{diagrams/backtrace/backtrace.tex}}
    \end{column}
  \end{columns}
\end{frame}

\begin{frame}{Backtraces}{\funcname{caml\_probably\_raise}'s handler doesn't catch \typename{ExnC}}
  \begin{columns}[c]
    \begin{column}{0.45\textwidth}
      \minipage[c][0.75\textheight][s]{\columnwidth}
      \begin{itemize}
        \item<1-> \funcname{match \dots with}\footnotemark pattern matching can also match exceptions
        \item<1-> The CMM of \funcname{probably\_raise} shows that this \funcname{match \dots with} is implemented with a pattern matching and a \funcname{try \dots with}
        \item<1-> But this one only matches on \typename{ExnA} exception
        \item<2-> Since the pattern matching fails to catch the exception, \funcname{reraise} is called with our current \typename{ExnC} exception
        \item<3-> Disassembly shows that \funcname{reraise} is actually a call to \funcname{caml\_reraise\_exn}, which is also an assembly function provided by the runtime
      \end{itemize}
      \vfill
      \mintocaml[firstline=15,lastline=21,highlightlines={18-21}]{../src/backtrace/backtrace.ml}
      \endminipage
    \end{column}
    \begin{column}{0.55\textwidth}
      \centering
      \onslide*<1>{\mintcmm[highlightlines={10-18}]{../src/backtrace/camlBacktrace\_\_probably\_raise\_351.cmm}}
      \onslide*<2>{\listcmm[highlightlines=18]{../src/backtrace/camlBacktrace\_\_probably\_raise_351.cmm}{CMM of probably\_raise}}
      \onslide*<3>{\mintobjdump[firstline=5,lastline=35,highlightlines=31]{../src/backtrace/camlBacktrace\_\_probably\_raise_351.objdump}}
    \end{column}
  \end{columns}
  \only<1>{\footnotetext[1]{https://blog.janestreet.com/pattern-matching-and-exception-handling-unite/}}
\end{frame}

\newcommand\listCamlReraiseExn[1][]{
  \listasm[firstline=727,lastline=734,#1]{../ocaml/runtime/amd64.S}{runtime/amd64.S:727}}

\begin{frame}{Backtraces}{Introducing \funcname{caml\_reraise\_exn}}
  \begin{columns}[c]
    \begin{column}{0.5\textwidth}
      \minipage[c][0.66\textheight][s]{\columnwidth}
      \begin{itemize}
        \item<1-> The exception have to be reraised to the next handler
        \item<2-> First, checks if the backtrace should be collected with \funcarg{domain\_state->backtrace\_active}
        \only<3>{
        \begin{itemize}
            \item If not, behaves like a \funcname{raise\_notrace} with \funcname{RESTORE\_EXN\_HANDLER\_OCAML}
        \end{itemize}
        }
        \item<4-> Jumps into \funcname{caml\_raise\_exn}
        \item<5-> Calls \funcname{caml\_stash\_backtrace} again, to scan the next part of the stack: from the trap just pop-ed to the next trap
      \end{itemize}
      \vfill
      \mintocaml[firstline=15,lastline=21,highlightlines={18-21}]{../src/backtrace/backtrace.ml}
      \endminipage
    \end{column}
    \begin{column}{0.5\textwidth}
      \raggedleft
      \onslide*<1>{\listCamlReraiseExn{}}
      \onslide*<2>{\listCamlReraiseExn[highlightlines=729]{}}
      \onslide*<3>{
        \mintc[firstline=190,lastline=192,highlightlines={191-192}]{../ocaml/runtime/amd64.S}
        \mintc[firstline=234,lastline=237,highlightlines={235-237}]{../ocaml/runtime/amd64.S}
        \medskip
        \listCamlReraiseExn[highlightlines=731]{}
      }
      \onslide*<4-5>{
        % TODO refactor with \listCamlReraiseExn
        \mintasm[firstline=727,lastline=734,highlightlines=730]{../ocaml/runtime/amd64.S}
        \medskip
      }
      \onslide*<4>{\listCamlRaiseExn[highlightlines=711]{}}
      \onslide*<5>{\listCamlRaiseExn[highlightlines=720]{}}
    \end{column}
  \end{columns}
\end{frame}

\begin{frame}{Backtraces}{Backtrace collection continues}
  \begin{columns}[c]
    \begin{column}{0.55\textwidth}
      \minipage[c][0.75\textheight][s]{\columnwidth}
      \begin{itemize}
        \item<1-> \funcname{caml\_stash\_backtrace} scans the older part of the stack, up to the next trap
        \item<6-> Controls jumps to the nested exception handler in \funcname{maybe\_raise}, which matches only on \typename{ExnB}
        \item<7-> \funcname{caml\_reraise\_exn} is called again, backtrace collection resumes with \funcname{caml\_stash\_backtrace}
      \end{itemize}
      \medskip
      \begin{itemize}
        \item<2-> Constructed backtrace:
        \begin{itemize}
          \item <2-> \funcname{definitely\_raise}
          \item <2-> \funcname{probably\_raise}
          \item <3-> \funcname{probably\_raise} (re-raise)
          \item <4-> \funcname{maybe\_raise}
          \item <5-> \funcname{main}
          \item <8-> \funcname{main} (re-raise)
        \end{itemize}
      \end{itemize}
      \vfill
      \onslide*<1-5>{\mintocaml[firstline=15,lastline=21]{../src/backtrace/backtrace.ml}}
      \onslide*<6>{\mintocaml[firstline=28,lastline=34,highlightlines=31]{../src/backtrace/backtrace.ml}}
      \onslide*<7->{\mintocaml[firstline=28,lastline=34,highlightlines=33]{../src/backtrace/backtrace.ml}}
      \endminipage
    \end{column}
    \begin{column}{0.45\textwidth}
      \raggedleft
      \onslide*<1>{\providecommand\step{6}\input{diagrams/backtrace/backtrace.tex}}%
      \onslide*<2>{\providecommand\step{6}\input{diagrams/backtrace/backtrace.tex}}%
      \onslide*<3>{\providecommand\step{7}\input{diagrams/backtrace/backtrace.tex}}%
      \onslide*<4>{\providecommand\step{8}\input{diagrams/backtrace/backtrace.tex}}%
      \onslide*<5>{\providecommand\step{9}\input{diagrams/backtrace/backtrace.tex}}%
      \onslide*<6>{\providecommand\step{10}\input{diagrams/backtrace/backtrace.tex}}%
      \onslide*<7>{\providecommand\step{11}\input{diagrams/backtrace/backtrace.tex}}%
      \onslide*<8>{\providecommand\step{12}\input{diagrams/backtrace/backtrace.tex}}%
      \onslide*<9>{\providecommand\step{13}\input{diagrams/backtrace/backtrace.tex}}%
    \end{column}
  \end{columns}
\end{frame}

\begin{frame}{Backtraces}{Printing the backtrace}
  \begin{columns}[c]
    \begin{column}{0.45\textwidth}
      \minipage[c][0.66\textheight][s]{\columnwidth}
      \begin{itemize}
        \item<1-> The exception backtrace can now be printed to \funcarg{stdout} with \funcname{Printexc.print_backtrace}
        \item<2-> First the exception backtrace is retrieved into an OCaml array with \funcname{get_raw_backtrace }
        \item<3-> Which is implemented as the \funcname{caml_get_exception_raw_backtrace} C function in the runtime
        \item<4-> And finally printed with \funcname{print_exception_backtrace}
      \end{itemize}
      \vfill
      \mintocaml[firstline=28,lastline=34,highlightlines=33]{../src/backtrace/backtrace.ml}
      \endminipage
    \end{column}
    \begin{column}{0.55\textwidth}
      \onslide*<1,2>{
        \mintocaml[firstline=100,lastline=101]{../ocaml/stdlib/printexc.ml}
      }
      \only<1>{
        \listocaml[firstline=170,lastline=172]{../ocaml/stdlib/printexc.ml}{stdlib/printexc.ml:170}
      }
      \only<2>{
        \listocaml[firstline=170,lastline=172,highlightlines=172]{../ocaml/stdlib/printexc.ml}{stdlib/printexc.ml:170}
      }
      \only<3>{
        \mintc[firstline=154,lastline=156]{../ocaml/runtime/backtrace.c}
        \listc[firstline=171,lastline=192,highlightlines={184-187}]{../ocaml/runtime/backtrace.c}{runtime/backtrace.c:154}
      }
      \only<4>{
        \listocaml[firstline=155,lastline=165]{../ocaml/stdlib/printexc.ml}{stdlib/printexc.ml:55}
      }
    \end{column}
  \end{columns}
\end{frame}


\subsection{The multiple kinds of raise}
\frameSubsection{}{}

\begin{frame}{Backtraces}{When backtraces are not needed}
  \begin{columns}[c]
    \begin{column}{0.45\textwidth}
      \minipage[c][0.66\textheight][s]{\columnwidth}
      \begin{itemize}
        \item<1-> What if the backtrace for an exception is never needed?
        \item<2-> It's possible to raise the exception explicitly with \funcname{raise\_notrace} to make raising as cheap as possible
        \item<3-> An example of use in the \funcname{dynlink} library of the OCaml compiler
      \end{itemize}
      \vfill
      \onslide<3->{
      \listocaml[firstline=205,lastline=209,highlightlines=207]{../ocaml/otherlibs/dynlink/dynlink\_compilerlibs/misc.ml}{otherlibs/dynlink/dynlink\_compilerlibs/misc.ml:205}
      }
      \endminipage
    \end{column}
    \begin{column}{0.55\textwidth}
    \only<4>{
      \mintcmm[highlightlines=3]{../src/notrace/camlNotrace\_\_fun_347.cmm}
      \listcmm{../src/notrace/camlNotrace\_\_all\_somes\_327.cmm}{all\_somes}
    }
    \onslide*<5->{
      \listobjdump[highlightlines={6-9}]{../src/notrace/camlNotrace\_\_fun_347.objdump}{anonymous function from \funcname{all\_somes}}
    }
    \end{column}
  \end{columns}
\end{frame}

\begin{frame}{Backtraces}{Summary table of the multiple kinds of raise}
  \begin{itemize}
    \item When debugging information is enabled and if the backtrace of an exception isn't needed, it's possible to raise with \funcname{raise\_notrace} for performance
    \item When debugging information is \emph{not} enabled, the compiler optimises \funcname{raise} calls to inline assembly (same effect as using \funcname{raise\_notrace})
  \end{itemize}
  \bigskip
  \centering
  \begin{tabular}{| l | c | c | c | c |}
    \hline
    \parbox{10em}{Debug information \\ (\funcarg{-g} flag)}  & \multicolumn{2}{|c|}{Disabled}                                     & \multicolumn{2}{|c|}{Enabled} \\
    \hline
    Raise kind                                               & \funcname{raise} & \funcname{raise\_notrace}                       & \funcname{raise} & \funcname{raise\_notrace} \\
    \hline
    Compilation                                              & \parbox{8em}{inline assembly\\ (optimization)} & inline assembly   & \funcname{caml\_raise\_exn} & inline assembly \\
    \hline
  \end{tabular}
\end{frame}


%
% Takeaway #5
%
\subsection*{Takeaway \#5}
\frameSubsectionTakeaway{}
\againframe<5>{takeaway}
}%
      \onslide*<8>{\providecommand\step{12}%
% Backtraces
%
\subsection{One advantage of raising with debugging information}
\frameSubsection{}{}

\begin{frame}{Backtraces}{Debugging information \& source code}
  \begin{columns}[c]
    \begin{column}{0.5\textwidth}
      \begin{itemize}
        \item Raising an exception is fun and games, but how do we know where the exception originated? $\rightarrow$ With the backtrace associated with the exception!
        \item In order to get a backtrace from the point where an exception was raised, it's necessary to compile with debugging information enabled (\funcarg{-g} flag for the compiler)
        \item Same example as in the `Nested handlers' section
      \end{itemize}
    \end{column}
    \begin{column}{0.5\textwidth}
      \listocaml[firstline=9,lastline=34]{../src/backtrace/backtrace.ml}{backtrace/backtrace.ml}
    \end{column}
  \end{columns}
\end{frame}

\begin{frame}{Backtraces}{CMM and disassembly of \funcname{definitely\_raise}}
  \begin{columns}[c]
    \begin{column}{0.5\textwidth}
      \minipage[c][0.66\textheight][s]{\columnwidth}
      \begin{itemize}
        \item<1-> An effect of compiling with debugging information (\funcarg{-g}): the CMM no longer uses \funcname{raise\_notrace} to raise the exception but \funcname{raise} instead
        \item<2-> Disassembly shows that CMM \funcname{raise} is actually a call to \funcname{caml\_raise\_exn}, a function in assembler provided by the runtime
      \end{itemize}
      \vfill
      \mintocaml[firstline=9,lastline=13,highlightlines=12]{../src/backtrace/backtrace.ml}
      \endminipage
    \end{column}
    \begin{column}{0.5\textwidth}
      \onslide*<1>{
        \mintcmm[highlightlines=5]{../src/backtrace/camlBacktrace\_\_definitely\_raise\_347.cmm}
      }
      \onslide*<2>{
        \mintobjdump[firstline=2,highlightlines=14]{../src/backtrace/camlBacktrace\_\_definitely\_raise\_347.objdump}
      }
    \end{column}
  \end{columns}
\end{frame}

\begin{frame}{Backtraces}{Introducing \funcname{caml\_raise\_exn}}
  \begin{columns}[c]
    \begin{column}{0.5\textwidth}
      \minipage[c][0.66\textheight][s]{\columnwidth}
      \onslide*<1>{
        \begin{itemize}
          \item Raising the exception is performed using \funcname{caml\_raise\_exn} which is
            \begin{itemize}
              \item An assembly function provided by the runtime
              \item Architecture specific
            \end{itemize}
          \item Let's see what it does differently from \funcname{raise\_notrace}\dots
          \item We are about to raise \typename{ExnC} with \funcname{caml\_raise\_exn} from \funcname{definitely\_raise}
        \end{itemize}
      }
      \onslide*<2>{
        \mintobjdump[firstline=2,highlightlines=14]{../src/backtrace/camlBacktrace\_\_definitely\_raise\_347.objdump}
      }
      \vfill
      \mintocaml[firstline=9,lastline=13,highlightlines=12]{../src/backtrace/backtrace.ml}
      \endminipage
    \end{column}
    \begin{column}{0.5\textwidth}
      \centering
      \providecommand\step{1}\input{diagrams/backtrace/backtrace.tex}
    \end{column}
  \end{columns}
\end{frame}

\begin{frame}{Backtraces}{But before that, we need more fields from \typename{caml\_domain\_state}}
  \begin{columns}
    \begin{column}{0.5\textwidth}
      \begin{itemize}
        \item Remember \typename{caml\_domain\_state} the per-domain runtime state?
        \item Some other members are of interest to us now\dots
        \item \localname{backtrace\_active} is used to know if the runtime should collect backtraces
        \item \localname{backtrace\_active} can be set by
          \begin{itemize}
            \item The \funcarg{b} flag in the \funcname{OCAMLRUNPARAM} environment variable
            \item \mintinline{ocaml}{Printexc.record_backtrace : bool -> unit} \footnotemark
          \end{itemize}
      \end{itemize}
    \end{column}
    \begin{column}{0.5\textwidth}
      \begin{listing}
        \mintc[highlightlines={9-11}]{src/ocaml/domain\_state\_backtrace.h}
        \caption{runtime/caml/domain\_state.h}
      \end{listing}
    \end{column}
  \end{columns}
  \footnotetext[1]{https://v2.ocaml.org/api/Printexc.html}
\end{frame}

\newcommand\listCamlRaiseExn[1][]{
  \listasm[firstline=702,lastline=725,#1]{../ocaml/runtime/amd64.S}{runtime/amd64.S:702}}

\begin{frame}{Backtraces}{Walk through \funcname{caml\_raise\_exn}}
  \begin{columns}[T]
    \begin{column}{0.5\textwidth}
      \begin{itemize}
        \item<1-> First checks whether exceptions backtrace should be recorded
        \item<1-> By checking the \funcarg{domain\_state->backtrace\_active} value
        \item<2-> If not, acts as \funcname{raise\_notrace} with \funcname{RESTORE\_EXN\_HANDLER\_OCAML}
          \begin{itemize}
            \item Set \regname{RSP} to \localname{domain\_state->exn\_handler} value
            \item Restore \localname{domain\_state->exn\_handler} from trap
            \item Jump with \funcname{ret} (same effect as using \funcname{pop} and \funcname{jmp})
          \end{itemize}
        \item<3-> Otherwise, switches to the C stack to be able to execute C code
        \item<4-> Calls \funcname{caml\_stash\_backtrace}, a C function provided by the runtime\ignorespaces
          \onslide<6->{, with
            \begin{itemize}
              \item \funcarg{exn}: exception value
              \item \funcarg{pc}: code address of the raising point
              \item \funcarg{sp}: stack address of the raising function
              \item \funcarg{trapsp}: stack address of the next handler
            \end{itemize}
          }
      \end{itemize}
      \bigskip
      \mintocaml[firstline=9,lastline=13,highlightlines=12]{../src/backtrace/backtrace.ml}
    \end{column}
    \begin{column}{0.5\textwidth}
      \raggedleft
      \onslide*<1,3-4>{
        \mintc[firstline=190,lastline=192]{../ocaml/runtime/amd64.S}
        \mintc[firstline=234,lastline=237]{../ocaml/runtime/amd64.S}
      }
      \onslide*<2>{
        \mintc[firstline=190,lastline=192,highlightlines={191-192}]{../ocaml/runtime/amd64.S}
        \mintc[firstline=234,lastline=237,highlightlines={235-237}]{../ocaml/runtime/amd64.S}
      }
      \onslide*<1>{\listCamlRaiseExn[highlightlines=705]{}}%
      \onslide*<2>{\listCamlRaiseExn[highlightlines={707-708}]{}}%
      \onslide*<3>{\listCamlRaiseExn[highlightlines={712-714}]{}}%
      \onslide*<4>{\listCamlRaiseExn[highlightlines={715-720}]{}}%
      \onslide*<5>{\providecommand\step{1}\input{diagrams/backtrace/backtrace.tex}}%
      \onslide*<6>{\providecommand\step{2}\input{diagrams/backtrace/backtrace.tex}}%
    \end{column}
  \end{columns}
\end{frame}

\newcommand\listStashBacktrace[1][]{
  \listc[firstline=91,lastline=120,#1]{../ocaml/runtime/backtrace\_nat.c}{runtime/backtrace\_nat.c:91}}

\begin{frame}{Backtraces}{Introducing \funcname{caml\_stash\_backtrace}}
  \begin{columns}[c]
    \begin{column}{0.5\textwidth}
      \minipage[c][0.66\textheight][s]{\columnwidth}
      \begin{itemize}
        \item<1-> A C function provided by the runtime (specific to native code)
        \item<2-> It scans the stack, between the stack pointer at the raising point up to the next trap, in order to collect the names of the detected function frames
          \onslide*<2>{
            \begin{itemize}
              \item And more information, like line number, column number...
            \end{itemize}
          }
        \item<3-> It uses the same technique and information as the Garbage Collector (GC) uses to scan the values live on the stack
          \onslide*<4>{
            \begin{itemize}
              \item \typename{frame\_descr} are at the core of stack unwinding
            \end{itemize}
          }
      \end{itemize}
      \vfill
      \mintocaml[firstline=9,lastline=13,highlightlines=12]{../src/backtrace/backtrace.ml}
      \endminipage
    \end{column}
    \begin{column}{0.5\textwidth}
      \onslide*<1>{\listStashBacktrace[highlightlines=91]{}}
      \onslide*<2>{\listStashBacktrace[highlightlines={91,117-118}]{}}
      \onslide*<3->{\listStashBacktrace[highlightlines={94,106,109-110}]{}}
    \end{column}
  \end{columns}
\end{frame}

\newcommand\listFrameDescr[1][]{
  \listocaml[firstline=111,lastline=124,#1]{../ocaml/asmcomp/emitaux.ml}{asmcomp/emitaux.ml:111}}
\newcommand\listDebugInfo[1][]{
  \listocaml[firstline=38,lastline=49,#1]{../ocaml/lambda/debuginfo.mli}{lambda/debuginfo.mli:38}}

\begin{frame}{Backtraces}{\typename{frame\_descr} and \typename{frame\_debuginfo} in a nutshell}
  \begin{columns}[c]
    \begin{column}{0.5\textwidth}
      \begin{itemize}
        \item<1-> For every return address in an OCaml program, there is a associated \typename{frame\_descr}
        \item<2-> A \typename{frame\_descr} specifies
        \begin{itemize}
          \item<2-> The size of the function stack frame
          \item<3-> Where live values are on the stack (so that the GC can mark them as live and prevent from being swept)
        \end{itemize}
        \item<4-> When compiled with debug information, \typename{frame\_descr} will be linked to a \typename{frame\_debuginfo}
        \begin{itemize}
          \item<5-> Contains information about the position in the source code (file name, line and column number)
        \end{itemize}
      \end{itemize}
    \end{column}
    \begin{column}{0.5\textwidth}
        \onslide*<1>{\listFrameDescr[highlightlines={118-122}]{}}
        \onslide*<2>{\listFrameDescr[highlightlines={120}]{}}
        \onslide*<3>{\listFrameDescr[highlightlines={121}]{}}
        \onslide*<4->{\listFrameDescr[highlightlines={122}]{}}
        \medskip
        \onslide*<1-3>{\listDebugInfo{}}
        \onslide*<4>{\listDebugInfo[highlightlines={38,49}]{}}
        \onslide*<5>{\listDebugInfo[highlightlines={39-46}]{}}
    \end{column}
  \end{columns}
\end{frame}

\begin{frame}{Backtraces}{Scanning the stack with \typename{frame\_descr}}
  \begin{columns}[c]
    \begin{column}{0.5\textwidth}
      \begin{itemize}
        \item Given the return address of the code that raises the exception by calling \funcname{caml\_raise\_exn} (\funcarg{pc} argument of \funcname{caml\_stash\_backtrace})
        \item And the position of the stack pointer (\funcarg{sp} argument of \funcname{caml\_stash\_backtrace})
        \item The frame size given by \typename{frame\_descr}.\funcarg{frame\_size}, allows to find the end of the previous frame
        \item And the return address of the caller of this function
        \bigskip
        \onslide<3->{
        \item Note that traps (here the reddish \funcname{ExnA}, \funcname{ExnB} and \funcname{ExnC} blocks) belong to the frame that installed them, and so the frame size changes when traps are removed
        }
      \end{itemize}
    \end{column}
    \begin{column}{0.5\textwidth}
      \raggedleft
      \onslide*<1>{\providecommand\step{2}\input{diagrams/backtrace/backtrace.tex}}%
      \onslide*<2>{\providecommand\step{1}\input{diagrams/backtrace/scanstack.tex}}%
      \onslide*<3>{\providecommand\step{2}\input{diagrams/backtrace/scanstack.tex}}%
      \onslide*<4>{\providecommand\step{3}\input{diagrams/backtrace/scanstack.tex}}%
      \onslide*<5>{\providecommand\step{4}\input{diagrams/backtrace/scanstack.tex}}%
      \onslide*<6>{\providecommand\step{5}\input{diagrams/backtrace/scanstack.tex}}%
    \end{column}
  \end{columns}
\end{frame}

% Duplicate of previous-previous slide
\begin{frame}{Backtraces}{Continuing on \funcname{caml\_stash\_backtrace}}
  \begin{columns}[c]
    \begin{column}{0.5\textwidth}
      \minipage[c][0.75\textheight][s]{\columnwidth}
      \begin{itemize}
          \color{gray}
        \item A C function provided by the runtime (specific to native code)
        \item It scans the stack, between the stack pointer at the raising point up to the next trap, in order to collect the names of the detected function frames
        \item It uses the same technique and information as the Garbage Collector (GC) uses to scan the values live on the stack
      \end{itemize}
      \begin{itemize}
        \item For each function frame detected, store a pointer to the \typename{frame\_descr} in the \localname{domain\_state->backtrace\_buffer} array, effectively constructing the backtrace
      \end{itemize}
      \onslide<2->{
        \begin{itemize}
            \smallskip
          \item Constructed backtrace:
            \funcname{definitely\_raise},
            \onslide<3->{\funcname{probably\_raise}}
        \end{itemize}
      }
      \vfill
      \mintocaml[firstline=9,lastline=13,highlightlines=12]{../src/backtrace/backtrace.ml}
      \endminipage
    \end{column}
    \begin{column}{0.5\textwidth}
      \raggedleft
      \onslide*<1>{\listStashBacktrace[highlightlines={114-115}]{}}
      \onslide*<2>{\providecommand\step{3}\input{diagrams/backtrace/backtrace.tex}}%
      \onslide*<3>{\providecommand\step{4}\input{diagrams/backtrace/backtrace.tex}}%
    \end{column}
  \end{columns}
\end{frame}

\begin{frame}{Backtraces}{Back to \funcname{caml\_raise\_exn}}
  \begin{columns}[c]
    \begin{column}{0.5\textwidth}
      \minipage[c][0.66\textheight][s]{\columnwidth}
      \begin{itemize}\color{gray}
        \item Checks wheteher exceptions backtrace should be recorded, by checking the \funcarg{domain\_state->backtrace\_active} value
        \item Switches to the C stack to be able to execute C code
        \item Calls \funcname{caml\_stash\_backtrace}, to collect the backtrace up to the next trap
      \end{itemize}
      \onslide*<2->{
        \begin{itemize}
          \item<2-> Executes the exception handler in \funcname{probably\_raise} with \funcname{RESTORE\_EXN\_HANDLER\_OCAML}
            \begin{itemize}
              \item Set \regname{RSP} to \localname{domain\_state->exn\_handler} value
              \item Restore \localname{domain\_state->exn\_handler} from trap
              \item Jump to the handler code with \funcname{ret}
            \end{itemize}
        \end{itemize}
      }
      \vfill
      \onslide*<1>{\mintocaml[firstline=9,lastline=13,highlightlines=12]{../src/backtrace/backtrace.ml}}
      \onslide*<2->{\mintocaml[firstline=15,lastline=21,highlightlines={19-21}]{../src/backtrace/backtrace.ml}}
      \endminipage
    \end{column}
    \begin{column}{0.5\textwidth}
      \centering
      \onslide*<1>{
        \mintc[firstline=190,lastline=192]{../ocaml/runtime/amd64.S}
        \mintc[firstline=234,lastline=237]{../ocaml/runtime/amd64.S}
      }
      \onslide*<2>{
        \mintc[firstline=190,lastline=192,highlightlines={191-192}]{../ocaml/runtime/amd64.S}
        \mintc[firstline=234,lastline=237,highlightlines={235-237}]{../ocaml/runtime/amd64.S}
      }
      \onslide*<1>{\listCamlRaiseExn[highlightlines=720]{}}
      \onslide*<2>{\listCamlRaiseExn[highlightlines=722]{}}
      \onslide*<3->{\providecommand\step{5}\input{diagrams/backtrace/backtrace.tex}}
    \end{column}
  \end{columns}
\end{frame}

\begin{frame}{Backtraces}{\funcname{caml\_probably\_raise}'s handler doesn't catch \typename{ExnC}}
  \begin{columns}[c]
    \begin{column}{0.45\textwidth}
      \minipage[c][0.75\textheight][s]{\columnwidth}
      \begin{itemize}
        \item<1-> \funcname{match \dots with}\footnotemark pattern matching can also match exceptions
        \item<1-> The CMM of \funcname{probably\_raise} shows that this \funcname{match \dots with} is implemented with a pattern matching and a \funcname{try \dots with}
        \item<1-> But this one only matches on \typename{ExnA} exception
        \item<2-> Since the pattern matching fails to catch the exception, \funcname{reraise} is called with our current \typename{ExnC} exception
        \item<3-> Disassembly shows that \funcname{reraise} is actually a call to \funcname{caml\_reraise\_exn}, which is also an assembly function provided by the runtime
      \end{itemize}
      \vfill
      \mintocaml[firstline=15,lastline=21,highlightlines={18-21}]{../src/backtrace/backtrace.ml}
      \endminipage
    \end{column}
    \begin{column}{0.55\textwidth}
      \centering
      \onslide*<1>{\mintcmm[highlightlines={10-18}]{../src/backtrace/camlBacktrace\_\_probably\_raise\_351.cmm}}
      \onslide*<2>{\listcmm[highlightlines=18]{../src/backtrace/camlBacktrace\_\_probably\_raise_351.cmm}{CMM of probably\_raise}}
      \onslide*<3>{\mintobjdump[firstline=5,lastline=35,highlightlines=31]{../src/backtrace/camlBacktrace\_\_probably\_raise_351.objdump}}
    \end{column}
  \end{columns}
  \only<1>{\footnotetext[1]{https://blog.janestreet.com/pattern-matching-and-exception-handling-unite/}}
\end{frame}

\newcommand\listCamlReraiseExn[1][]{
  \listasm[firstline=727,lastline=734,#1]{../ocaml/runtime/amd64.S}{runtime/amd64.S:727}}

\begin{frame}{Backtraces}{Introducing \funcname{caml\_reraise\_exn}}
  \begin{columns}[c]
    \begin{column}{0.5\textwidth}
      \minipage[c][0.66\textheight][s]{\columnwidth}
      \begin{itemize}
        \item<1-> The exception have to be reraised to the next handler
        \item<2-> First, checks if the backtrace should be collected with \funcarg{domain\_state->backtrace\_active}
        \only<3>{
        \begin{itemize}
            \item If not, behaves like a \funcname{raise\_notrace} with \funcname{RESTORE\_EXN\_HANDLER\_OCAML}
        \end{itemize}
        }
        \item<4-> Jumps into \funcname{caml\_raise\_exn}
        \item<5-> Calls \funcname{caml\_stash\_backtrace} again, to scan the next part of the stack: from the trap just pop-ed to the next trap
      \end{itemize}
      \vfill
      \mintocaml[firstline=15,lastline=21,highlightlines={18-21}]{../src/backtrace/backtrace.ml}
      \endminipage
    \end{column}
    \begin{column}{0.5\textwidth}
      \raggedleft
      \onslide*<1>{\listCamlReraiseExn{}}
      \onslide*<2>{\listCamlReraiseExn[highlightlines=729]{}}
      \onslide*<3>{
        \mintc[firstline=190,lastline=192,highlightlines={191-192}]{../ocaml/runtime/amd64.S}
        \mintc[firstline=234,lastline=237,highlightlines={235-237}]{../ocaml/runtime/amd64.S}
        \medskip
        \listCamlReraiseExn[highlightlines=731]{}
      }
      \onslide*<4-5>{
        % TODO refactor with \listCamlReraiseExn
        \mintasm[firstline=727,lastline=734,highlightlines=730]{../ocaml/runtime/amd64.S}
        \medskip
      }
      \onslide*<4>{\listCamlRaiseExn[highlightlines=711]{}}
      \onslide*<5>{\listCamlRaiseExn[highlightlines=720]{}}
    \end{column}
  \end{columns}
\end{frame}

\begin{frame}{Backtraces}{Backtrace collection continues}
  \begin{columns}[c]
    \begin{column}{0.55\textwidth}
      \minipage[c][0.75\textheight][s]{\columnwidth}
      \begin{itemize}
        \item<1-> \funcname{caml\_stash\_backtrace} scans the older part of the stack, up to the next trap
        \item<6-> Controls jumps to the nested exception handler in \funcname{maybe\_raise}, which matches only on \typename{ExnB}
        \item<7-> \funcname{caml\_reraise\_exn} is called again, backtrace collection resumes with \funcname{caml\_stash\_backtrace}
      \end{itemize}
      \medskip
      \begin{itemize}
        \item<2-> Constructed backtrace:
        \begin{itemize}
          \item <2-> \funcname{definitely\_raise}
          \item <2-> \funcname{probably\_raise}
          \item <3-> \funcname{probably\_raise} (re-raise)
          \item <4-> \funcname{maybe\_raise}
          \item <5-> \funcname{main}
          \item <8-> \funcname{main} (re-raise)
        \end{itemize}
      \end{itemize}
      \vfill
      \onslide*<1-5>{\mintocaml[firstline=15,lastline=21]{../src/backtrace/backtrace.ml}}
      \onslide*<6>{\mintocaml[firstline=28,lastline=34,highlightlines=31]{../src/backtrace/backtrace.ml}}
      \onslide*<7->{\mintocaml[firstline=28,lastline=34,highlightlines=33]{../src/backtrace/backtrace.ml}}
      \endminipage
    \end{column}
    \begin{column}{0.45\textwidth}
      \raggedleft
      \onslide*<1>{\providecommand\step{6}\input{diagrams/backtrace/backtrace.tex}}%
      \onslide*<2>{\providecommand\step{6}\input{diagrams/backtrace/backtrace.tex}}%
      \onslide*<3>{\providecommand\step{7}\input{diagrams/backtrace/backtrace.tex}}%
      \onslide*<4>{\providecommand\step{8}\input{diagrams/backtrace/backtrace.tex}}%
      \onslide*<5>{\providecommand\step{9}\input{diagrams/backtrace/backtrace.tex}}%
      \onslide*<6>{\providecommand\step{10}\input{diagrams/backtrace/backtrace.tex}}%
      \onslide*<7>{\providecommand\step{11}\input{diagrams/backtrace/backtrace.tex}}%
      \onslide*<8>{\providecommand\step{12}\input{diagrams/backtrace/backtrace.tex}}%
      \onslide*<9>{\providecommand\step{13}\input{diagrams/backtrace/backtrace.tex}}%
    \end{column}
  \end{columns}
\end{frame}

\begin{frame}{Backtraces}{Printing the backtrace}
  \begin{columns}[c]
    \begin{column}{0.45\textwidth}
      \minipage[c][0.66\textheight][s]{\columnwidth}
      \begin{itemize}
        \item<1-> The exception backtrace can now be printed to \funcarg{stdout} with \funcname{Printexc.print_backtrace}
        \item<2-> First the exception backtrace is retrieved into an OCaml array with \funcname{get_raw_backtrace }
        \item<3-> Which is implemented as the \funcname{caml_get_exception_raw_backtrace} C function in the runtime
        \item<4-> And finally printed with \funcname{print_exception_backtrace}
      \end{itemize}
      \vfill
      \mintocaml[firstline=28,lastline=34,highlightlines=33]{../src/backtrace/backtrace.ml}
      \endminipage
    \end{column}
    \begin{column}{0.55\textwidth}
      \onslide*<1,2>{
        \mintocaml[firstline=100,lastline=101]{../ocaml/stdlib/printexc.ml}
      }
      \only<1>{
        \listocaml[firstline=170,lastline=172]{../ocaml/stdlib/printexc.ml}{stdlib/printexc.ml:170}
      }
      \only<2>{
        \listocaml[firstline=170,lastline=172,highlightlines=172]{../ocaml/stdlib/printexc.ml}{stdlib/printexc.ml:170}
      }
      \only<3>{
        \mintc[firstline=154,lastline=156]{../ocaml/runtime/backtrace.c}
        \listc[firstline=171,lastline=192,highlightlines={184-187}]{../ocaml/runtime/backtrace.c}{runtime/backtrace.c:154}
      }
      \only<4>{
        \listocaml[firstline=155,lastline=165]{../ocaml/stdlib/printexc.ml}{stdlib/printexc.ml:55}
      }
    \end{column}
  \end{columns}
\end{frame}


\subsection{The multiple kinds of raise}
\frameSubsection{}{}

\begin{frame}{Backtraces}{When backtraces are not needed}
  \begin{columns}[c]
    \begin{column}{0.45\textwidth}
      \minipage[c][0.66\textheight][s]{\columnwidth}
      \begin{itemize}
        \item<1-> What if the backtrace for an exception is never needed?
        \item<2-> It's possible to raise the exception explicitly with \funcname{raise\_notrace} to make raising as cheap as possible
        \item<3-> An example of use in the \funcname{dynlink} library of the OCaml compiler
      \end{itemize}
      \vfill
      \onslide<3->{
      \listocaml[firstline=205,lastline=209,highlightlines=207]{../ocaml/otherlibs/dynlink/dynlink\_compilerlibs/misc.ml}{otherlibs/dynlink/dynlink\_compilerlibs/misc.ml:205}
      }
      \endminipage
    \end{column}
    \begin{column}{0.55\textwidth}
    \only<4>{
      \mintcmm[highlightlines=3]{../src/notrace/camlNotrace\_\_fun_347.cmm}
      \listcmm{../src/notrace/camlNotrace\_\_all\_somes\_327.cmm}{all\_somes}
    }
    \onslide*<5->{
      \listobjdump[highlightlines={6-9}]{../src/notrace/camlNotrace\_\_fun_347.objdump}{anonymous function from \funcname{all\_somes}}
    }
    \end{column}
  \end{columns}
\end{frame}

\begin{frame}{Backtraces}{Summary table of the multiple kinds of raise}
  \begin{itemize}
    \item When debugging information is enabled and if the backtrace of an exception isn't needed, it's possible to raise with \funcname{raise\_notrace} for performance
    \item When debugging information is \emph{not} enabled, the compiler optimises \funcname{raise} calls to inline assembly (same effect as using \funcname{raise\_notrace})
  \end{itemize}
  \bigskip
  \centering
  \begin{tabular}{| l | c | c | c | c |}
    \hline
    \parbox{10em}{Debug information \\ (\funcarg{-g} flag)}  & \multicolumn{2}{|c|}{Disabled}                                     & \multicolumn{2}{|c|}{Enabled} \\
    \hline
    Raise kind                                               & \funcname{raise} & \funcname{raise\_notrace}                       & \funcname{raise} & \funcname{raise\_notrace} \\
    \hline
    Compilation                                              & \parbox{8em}{inline assembly\\ (optimization)} & inline assembly   & \funcname{caml\_raise\_exn} & inline assembly \\
    \hline
  \end{tabular}
\end{frame}


%
% Takeaway #5
%
\subsection*{Takeaway \#5}
\frameSubsectionTakeaway{}
\againframe<5>{takeaway}
}%
      \onslide*<9>{\providecommand\step{13}%
% Backtraces
%
\subsection{One advantage of raising with debugging information}
\frameSubsection{}{}

\begin{frame}{Backtraces}{Debugging information \& source code}
  \begin{columns}[c]
    \begin{column}{0.5\textwidth}
      \begin{itemize}
        \item Raising an exception is fun and games, but how do we know where the exception originated? $\rightarrow$ With the backtrace associated with the exception!
        \item In order to get a backtrace from the point where an exception was raised, it's necessary to compile with debugging information enabled (\funcarg{-g} flag for the compiler)
        \item Same example as in the `Nested handlers' section
      \end{itemize}
    \end{column}
    \begin{column}{0.5\textwidth}
      \listocaml[firstline=9,lastline=34]{../src/backtrace/backtrace.ml}{backtrace/backtrace.ml}
    \end{column}
  \end{columns}
\end{frame}

\begin{frame}{Backtraces}{CMM and disassembly of \funcname{definitely\_raise}}
  \begin{columns}[c]
    \begin{column}{0.5\textwidth}
      \minipage[c][0.66\textheight][s]{\columnwidth}
      \begin{itemize}
        \item<1-> An effect of compiling with debugging information (\funcarg{-g}): the CMM no longer uses \funcname{raise\_notrace} to raise the exception but \funcname{raise} instead
        \item<2-> Disassembly shows that CMM \funcname{raise} is actually a call to \funcname{caml\_raise\_exn}, a function in assembler provided by the runtime
      \end{itemize}
      \vfill
      \mintocaml[firstline=9,lastline=13,highlightlines=12]{../src/backtrace/backtrace.ml}
      \endminipage
    \end{column}
    \begin{column}{0.5\textwidth}
      \onslide*<1>{
        \mintcmm[highlightlines=5]{../src/backtrace/camlBacktrace\_\_definitely\_raise\_347.cmm}
      }
      \onslide*<2>{
        \mintobjdump[firstline=2,highlightlines=14]{../src/backtrace/camlBacktrace\_\_definitely\_raise\_347.objdump}
      }
    \end{column}
  \end{columns}
\end{frame}

\begin{frame}{Backtraces}{Introducing \funcname{caml\_raise\_exn}}
  \begin{columns}[c]
    \begin{column}{0.5\textwidth}
      \minipage[c][0.66\textheight][s]{\columnwidth}
      \onslide*<1>{
        \begin{itemize}
          \item Raising the exception is performed using \funcname{caml\_raise\_exn} which is
            \begin{itemize}
              \item An assembly function provided by the runtime
              \item Architecture specific
            \end{itemize}
          \item Let's see what it does differently from \funcname{raise\_notrace}\dots
          \item We are about to raise \typename{ExnC} with \funcname{caml\_raise\_exn} from \funcname{definitely\_raise}
        \end{itemize}
      }
      \onslide*<2>{
        \mintobjdump[firstline=2,highlightlines=14]{../src/backtrace/camlBacktrace\_\_definitely\_raise\_347.objdump}
      }
      \vfill
      \mintocaml[firstline=9,lastline=13,highlightlines=12]{../src/backtrace/backtrace.ml}
      \endminipage
    \end{column}
    \begin{column}{0.5\textwidth}
      \centering
      \providecommand\step{1}\input{diagrams/backtrace/backtrace.tex}
    \end{column}
  \end{columns}
\end{frame}

\begin{frame}{Backtraces}{But before that, we need more fields from \typename{caml\_domain\_state}}
  \begin{columns}
    \begin{column}{0.5\textwidth}
      \begin{itemize}
        \item Remember \typename{caml\_domain\_state} the per-domain runtime state?
        \item Some other members are of interest to us now\dots
        \item \localname{backtrace\_active} is used to know if the runtime should collect backtraces
        \item \localname{backtrace\_active} can be set by
          \begin{itemize}
            \item The \funcarg{b} flag in the \funcname{OCAMLRUNPARAM} environment variable
            \item \mintinline{ocaml}{Printexc.record_backtrace : bool -> unit} \footnotemark
          \end{itemize}
      \end{itemize}
    \end{column}
    \begin{column}{0.5\textwidth}
      \begin{listing}
        \mintc[highlightlines={9-11}]{src/ocaml/domain\_state\_backtrace.h}
        \caption{runtime/caml/domain\_state.h}
      \end{listing}
    \end{column}
  \end{columns}
  \footnotetext[1]{https://v2.ocaml.org/api/Printexc.html}
\end{frame}

\newcommand\listCamlRaiseExn[1][]{
  \listasm[firstline=702,lastline=725,#1]{../ocaml/runtime/amd64.S}{runtime/amd64.S:702}}

\begin{frame}{Backtraces}{Walk through \funcname{caml\_raise\_exn}}
  \begin{columns}[T]
    \begin{column}{0.5\textwidth}
      \begin{itemize}
        \item<1-> First checks whether exceptions backtrace should be recorded
        \item<1-> By checking the \funcarg{domain\_state->backtrace\_active} value
        \item<2-> If not, acts as \funcname{raise\_notrace} with \funcname{RESTORE\_EXN\_HANDLER\_OCAML}
          \begin{itemize}
            \item Set \regname{RSP} to \localname{domain\_state->exn\_handler} value
            \item Restore \localname{domain\_state->exn\_handler} from trap
            \item Jump with \funcname{ret} (same effect as using \funcname{pop} and \funcname{jmp})
          \end{itemize}
        \item<3-> Otherwise, switches to the C stack to be able to execute C code
        \item<4-> Calls \funcname{caml\_stash\_backtrace}, a C function provided by the runtime\ignorespaces
          \onslide<6->{, with
            \begin{itemize}
              \item \funcarg{exn}: exception value
              \item \funcarg{pc}: code address of the raising point
              \item \funcarg{sp}: stack address of the raising function
              \item \funcarg{trapsp}: stack address of the next handler
            \end{itemize}
          }
      \end{itemize}
      \bigskip
      \mintocaml[firstline=9,lastline=13,highlightlines=12]{../src/backtrace/backtrace.ml}
    \end{column}
    \begin{column}{0.5\textwidth}
      \raggedleft
      \onslide*<1,3-4>{
        \mintc[firstline=190,lastline=192]{../ocaml/runtime/amd64.S}
        \mintc[firstline=234,lastline=237]{../ocaml/runtime/amd64.S}
      }
      \onslide*<2>{
        \mintc[firstline=190,lastline=192,highlightlines={191-192}]{../ocaml/runtime/amd64.S}
        \mintc[firstline=234,lastline=237,highlightlines={235-237}]{../ocaml/runtime/amd64.S}
      }
      \onslide*<1>{\listCamlRaiseExn[highlightlines=705]{}}%
      \onslide*<2>{\listCamlRaiseExn[highlightlines={707-708}]{}}%
      \onslide*<3>{\listCamlRaiseExn[highlightlines={712-714}]{}}%
      \onslide*<4>{\listCamlRaiseExn[highlightlines={715-720}]{}}%
      \onslide*<5>{\providecommand\step{1}\input{diagrams/backtrace/backtrace.tex}}%
      \onslide*<6>{\providecommand\step{2}\input{diagrams/backtrace/backtrace.tex}}%
    \end{column}
  \end{columns}
\end{frame}

\newcommand\listStashBacktrace[1][]{
  \listc[firstline=91,lastline=120,#1]{../ocaml/runtime/backtrace\_nat.c}{runtime/backtrace\_nat.c:91}}

\begin{frame}{Backtraces}{Introducing \funcname{caml\_stash\_backtrace}}
  \begin{columns}[c]
    \begin{column}{0.5\textwidth}
      \minipage[c][0.66\textheight][s]{\columnwidth}
      \begin{itemize}
        \item<1-> A C function provided by the runtime (specific to native code)
        \item<2-> It scans the stack, between the stack pointer at the raising point up to the next trap, in order to collect the names of the detected function frames
          \onslide*<2>{
            \begin{itemize}
              \item And more information, like line number, column number...
            \end{itemize}
          }
        \item<3-> It uses the same technique and information as the Garbage Collector (GC) uses to scan the values live on the stack
          \onslide*<4>{
            \begin{itemize}
              \item \typename{frame\_descr} are at the core of stack unwinding
            \end{itemize}
          }
      \end{itemize}
      \vfill
      \mintocaml[firstline=9,lastline=13,highlightlines=12]{../src/backtrace/backtrace.ml}
      \endminipage
    \end{column}
    \begin{column}{0.5\textwidth}
      \onslide*<1>{\listStashBacktrace[highlightlines=91]{}}
      \onslide*<2>{\listStashBacktrace[highlightlines={91,117-118}]{}}
      \onslide*<3->{\listStashBacktrace[highlightlines={94,106,109-110}]{}}
    \end{column}
  \end{columns}
\end{frame}

\newcommand\listFrameDescr[1][]{
  \listocaml[firstline=111,lastline=124,#1]{../ocaml/asmcomp/emitaux.ml}{asmcomp/emitaux.ml:111}}
\newcommand\listDebugInfo[1][]{
  \listocaml[firstline=38,lastline=49,#1]{../ocaml/lambda/debuginfo.mli}{lambda/debuginfo.mli:38}}

\begin{frame}{Backtraces}{\typename{frame\_descr} and \typename{frame\_debuginfo} in a nutshell}
  \begin{columns}[c]
    \begin{column}{0.5\textwidth}
      \begin{itemize}
        \item<1-> For every return address in an OCaml program, there is a associated \typename{frame\_descr}
        \item<2-> A \typename{frame\_descr} specifies
        \begin{itemize}
          \item<2-> The size of the function stack frame
          \item<3-> Where live values are on the stack (so that the GC can mark them as live and prevent from being swept)
        \end{itemize}
        \item<4-> When compiled with debug information, \typename{frame\_descr} will be linked to a \typename{frame\_debuginfo}
        \begin{itemize}
          \item<5-> Contains information about the position in the source code (file name, line and column number)
        \end{itemize}
      \end{itemize}
    \end{column}
    \begin{column}{0.5\textwidth}
        \onslide*<1>{\listFrameDescr[highlightlines={118-122}]{}}
        \onslide*<2>{\listFrameDescr[highlightlines={120}]{}}
        \onslide*<3>{\listFrameDescr[highlightlines={121}]{}}
        \onslide*<4->{\listFrameDescr[highlightlines={122}]{}}
        \medskip
        \onslide*<1-3>{\listDebugInfo{}}
        \onslide*<4>{\listDebugInfo[highlightlines={38,49}]{}}
        \onslide*<5>{\listDebugInfo[highlightlines={39-46}]{}}
    \end{column}
  \end{columns}
\end{frame}

\begin{frame}{Backtraces}{Scanning the stack with \typename{frame\_descr}}
  \begin{columns}[c]
    \begin{column}{0.5\textwidth}
      \begin{itemize}
        \item Given the return address of the code that raises the exception by calling \funcname{caml\_raise\_exn} (\funcarg{pc} argument of \funcname{caml\_stash\_backtrace})
        \item And the position of the stack pointer (\funcarg{sp} argument of \funcname{caml\_stash\_backtrace})
        \item The frame size given by \typename{frame\_descr}.\funcarg{frame\_size}, allows to find the end of the previous frame
        \item And the return address of the caller of this function
        \bigskip
        \onslide<3->{
        \item Note that traps (here the reddish \funcname{ExnA}, \funcname{ExnB} and \funcname{ExnC} blocks) belong to the frame that installed them, and so the frame size changes when traps are removed
        }
      \end{itemize}
    \end{column}
    \begin{column}{0.5\textwidth}
      \raggedleft
      \onslide*<1>{\providecommand\step{2}\input{diagrams/backtrace/backtrace.tex}}%
      \onslide*<2>{\providecommand\step{1}\input{diagrams/backtrace/scanstack.tex}}%
      \onslide*<3>{\providecommand\step{2}\input{diagrams/backtrace/scanstack.tex}}%
      \onslide*<4>{\providecommand\step{3}\input{diagrams/backtrace/scanstack.tex}}%
      \onslide*<5>{\providecommand\step{4}\input{diagrams/backtrace/scanstack.tex}}%
      \onslide*<6>{\providecommand\step{5}\input{diagrams/backtrace/scanstack.tex}}%
    \end{column}
  \end{columns}
\end{frame}

% Duplicate of previous-previous slide
\begin{frame}{Backtraces}{Continuing on \funcname{caml\_stash\_backtrace}}
  \begin{columns}[c]
    \begin{column}{0.5\textwidth}
      \minipage[c][0.75\textheight][s]{\columnwidth}
      \begin{itemize}
          \color{gray}
        \item A C function provided by the runtime (specific to native code)
        \item It scans the stack, between the stack pointer at the raising point up to the next trap, in order to collect the names of the detected function frames
        \item It uses the same technique and information as the Garbage Collector (GC) uses to scan the values live on the stack
      \end{itemize}
      \begin{itemize}
        \item For each function frame detected, store a pointer to the \typename{frame\_descr} in the \localname{domain\_state->backtrace\_buffer} array, effectively constructing the backtrace
      \end{itemize}
      \onslide<2->{
        \begin{itemize}
            \smallskip
          \item Constructed backtrace:
            \funcname{definitely\_raise},
            \onslide<3->{\funcname{probably\_raise}}
        \end{itemize}
      }
      \vfill
      \mintocaml[firstline=9,lastline=13,highlightlines=12]{../src/backtrace/backtrace.ml}
      \endminipage
    \end{column}
    \begin{column}{0.5\textwidth}
      \raggedleft
      \onslide*<1>{\listStashBacktrace[highlightlines={114-115}]{}}
      \onslide*<2>{\providecommand\step{3}\input{diagrams/backtrace/backtrace.tex}}%
      \onslide*<3>{\providecommand\step{4}\input{diagrams/backtrace/backtrace.tex}}%
    \end{column}
  \end{columns}
\end{frame}

\begin{frame}{Backtraces}{Back to \funcname{caml\_raise\_exn}}
  \begin{columns}[c]
    \begin{column}{0.5\textwidth}
      \minipage[c][0.66\textheight][s]{\columnwidth}
      \begin{itemize}\color{gray}
        \item Checks wheteher exceptions backtrace should be recorded, by checking the \funcarg{domain\_state->backtrace\_active} value
        \item Switches to the C stack to be able to execute C code
        \item Calls \funcname{caml\_stash\_backtrace}, to collect the backtrace up to the next trap
      \end{itemize}
      \onslide*<2->{
        \begin{itemize}
          \item<2-> Executes the exception handler in \funcname{probably\_raise} with \funcname{RESTORE\_EXN\_HANDLER\_OCAML}
            \begin{itemize}
              \item Set \regname{RSP} to \localname{domain\_state->exn\_handler} value
              \item Restore \localname{domain\_state->exn\_handler} from trap
              \item Jump to the handler code with \funcname{ret}
            \end{itemize}
        \end{itemize}
      }
      \vfill
      \onslide*<1>{\mintocaml[firstline=9,lastline=13,highlightlines=12]{../src/backtrace/backtrace.ml}}
      \onslide*<2->{\mintocaml[firstline=15,lastline=21,highlightlines={19-21}]{../src/backtrace/backtrace.ml}}
      \endminipage
    \end{column}
    \begin{column}{0.5\textwidth}
      \centering
      \onslide*<1>{
        \mintc[firstline=190,lastline=192]{../ocaml/runtime/amd64.S}
        \mintc[firstline=234,lastline=237]{../ocaml/runtime/amd64.S}
      }
      \onslide*<2>{
        \mintc[firstline=190,lastline=192,highlightlines={191-192}]{../ocaml/runtime/amd64.S}
        \mintc[firstline=234,lastline=237,highlightlines={235-237}]{../ocaml/runtime/amd64.S}
      }
      \onslide*<1>{\listCamlRaiseExn[highlightlines=720]{}}
      \onslide*<2>{\listCamlRaiseExn[highlightlines=722]{}}
      \onslide*<3->{\providecommand\step{5}\input{diagrams/backtrace/backtrace.tex}}
    \end{column}
  \end{columns}
\end{frame}

\begin{frame}{Backtraces}{\funcname{caml\_probably\_raise}'s handler doesn't catch \typename{ExnC}}
  \begin{columns}[c]
    \begin{column}{0.45\textwidth}
      \minipage[c][0.75\textheight][s]{\columnwidth}
      \begin{itemize}
        \item<1-> \funcname{match \dots with}\footnotemark pattern matching can also match exceptions
        \item<1-> The CMM of \funcname{probably\_raise} shows that this \funcname{match \dots with} is implemented with a pattern matching and a \funcname{try \dots with}
        \item<1-> But this one only matches on \typename{ExnA} exception
        \item<2-> Since the pattern matching fails to catch the exception, \funcname{reraise} is called with our current \typename{ExnC} exception
        \item<3-> Disassembly shows that \funcname{reraise} is actually a call to \funcname{caml\_reraise\_exn}, which is also an assembly function provided by the runtime
      \end{itemize}
      \vfill
      \mintocaml[firstline=15,lastline=21,highlightlines={18-21}]{../src/backtrace/backtrace.ml}
      \endminipage
    \end{column}
    \begin{column}{0.55\textwidth}
      \centering
      \onslide*<1>{\mintcmm[highlightlines={10-18}]{../src/backtrace/camlBacktrace\_\_probably\_raise\_351.cmm}}
      \onslide*<2>{\listcmm[highlightlines=18]{../src/backtrace/camlBacktrace\_\_probably\_raise_351.cmm}{CMM of probably\_raise}}
      \onslide*<3>{\mintobjdump[firstline=5,lastline=35,highlightlines=31]{../src/backtrace/camlBacktrace\_\_probably\_raise_351.objdump}}
    \end{column}
  \end{columns}
  \only<1>{\footnotetext[1]{https://blog.janestreet.com/pattern-matching-and-exception-handling-unite/}}
\end{frame}

\newcommand\listCamlReraiseExn[1][]{
  \listasm[firstline=727,lastline=734,#1]{../ocaml/runtime/amd64.S}{runtime/amd64.S:727}}

\begin{frame}{Backtraces}{Introducing \funcname{caml\_reraise\_exn}}
  \begin{columns}[c]
    \begin{column}{0.5\textwidth}
      \minipage[c][0.66\textheight][s]{\columnwidth}
      \begin{itemize}
        \item<1-> The exception have to be reraised to the next handler
        \item<2-> First, checks if the backtrace should be collected with \funcarg{domain\_state->backtrace\_active}
        \only<3>{
        \begin{itemize}
            \item If not, behaves like a \funcname{raise\_notrace} with \funcname{RESTORE\_EXN\_HANDLER\_OCAML}
        \end{itemize}
        }
        \item<4-> Jumps into \funcname{caml\_raise\_exn}
        \item<5-> Calls \funcname{caml\_stash\_backtrace} again, to scan the next part of the stack: from the trap just pop-ed to the next trap
      \end{itemize}
      \vfill
      \mintocaml[firstline=15,lastline=21,highlightlines={18-21}]{../src/backtrace/backtrace.ml}
      \endminipage
    \end{column}
    \begin{column}{0.5\textwidth}
      \raggedleft
      \onslide*<1>{\listCamlReraiseExn{}}
      \onslide*<2>{\listCamlReraiseExn[highlightlines=729]{}}
      \onslide*<3>{
        \mintc[firstline=190,lastline=192,highlightlines={191-192}]{../ocaml/runtime/amd64.S}
        \mintc[firstline=234,lastline=237,highlightlines={235-237}]{../ocaml/runtime/amd64.S}
        \medskip
        \listCamlReraiseExn[highlightlines=731]{}
      }
      \onslide*<4-5>{
        % TODO refactor with \listCamlReraiseExn
        \mintasm[firstline=727,lastline=734,highlightlines=730]{../ocaml/runtime/amd64.S}
        \medskip
      }
      \onslide*<4>{\listCamlRaiseExn[highlightlines=711]{}}
      \onslide*<5>{\listCamlRaiseExn[highlightlines=720]{}}
    \end{column}
  \end{columns}
\end{frame}

\begin{frame}{Backtraces}{Backtrace collection continues}
  \begin{columns}[c]
    \begin{column}{0.55\textwidth}
      \minipage[c][0.75\textheight][s]{\columnwidth}
      \begin{itemize}
        \item<1-> \funcname{caml\_stash\_backtrace} scans the older part of the stack, up to the next trap
        \item<6-> Controls jumps to the nested exception handler in \funcname{maybe\_raise}, which matches only on \typename{ExnB}
        \item<7-> \funcname{caml\_reraise\_exn} is called again, backtrace collection resumes with \funcname{caml\_stash\_backtrace}
      \end{itemize}
      \medskip
      \begin{itemize}
        \item<2-> Constructed backtrace:
        \begin{itemize}
          \item <2-> \funcname{definitely\_raise}
          \item <2-> \funcname{probably\_raise}
          \item <3-> \funcname{probably\_raise} (re-raise)
          \item <4-> \funcname{maybe\_raise}
          \item <5-> \funcname{main}
          \item <8-> \funcname{main} (re-raise)
        \end{itemize}
      \end{itemize}
      \vfill
      \onslide*<1-5>{\mintocaml[firstline=15,lastline=21]{../src/backtrace/backtrace.ml}}
      \onslide*<6>{\mintocaml[firstline=28,lastline=34,highlightlines=31]{../src/backtrace/backtrace.ml}}
      \onslide*<7->{\mintocaml[firstline=28,lastline=34,highlightlines=33]{../src/backtrace/backtrace.ml}}
      \endminipage
    \end{column}
    \begin{column}{0.45\textwidth}
      \raggedleft
      \onslide*<1>{\providecommand\step{6}\input{diagrams/backtrace/backtrace.tex}}%
      \onslide*<2>{\providecommand\step{6}\input{diagrams/backtrace/backtrace.tex}}%
      \onslide*<3>{\providecommand\step{7}\input{diagrams/backtrace/backtrace.tex}}%
      \onslide*<4>{\providecommand\step{8}\input{diagrams/backtrace/backtrace.tex}}%
      \onslide*<5>{\providecommand\step{9}\input{diagrams/backtrace/backtrace.tex}}%
      \onslide*<6>{\providecommand\step{10}\input{diagrams/backtrace/backtrace.tex}}%
      \onslide*<7>{\providecommand\step{11}\input{diagrams/backtrace/backtrace.tex}}%
      \onslide*<8>{\providecommand\step{12}\input{diagrams/backtrace/backtrace.tex}}%
      \onslide*<9>{\providecommand\step{13}\input{diagrams/backtrace/backtrace.tex}}%
    \end{column}
  \end{columns}
\end{frame}

\begin{frame}{Backtraces}{Printing the backtrace}
  \begin{columns}[c]
    \begin{column}{0.45\textwidth}
      \minipage[c][0.66\textheight][s]{\columnwidth}
      \begin{itemize}
        \item<1-> The exception backtrace can now be printed to \funcarg{stdout} with \funcname{Printexc.print_backtrace}
        \item<2-> First the exception backtrace is retrieved into an OCaml array with \funcname{get_raw_backtrace }
        \item<3-> Which is implemented as the \funcname{caml_get_exception_raw_backtrace} C function in the runtime
        \item<4-> And finally printed with \funcname{print_exception_backtrace}
      \end{itemize}
      \vfill
      \mintocaml[firstline=28,lastline=34,highlightlines=33]{../src/backtrace/backtrace.ml}
      \endminipage
    \end{column}
    \begin{column}{0.55\textwidth}
      \onslide*<1,2>{
        \mintocaml[firstline=100,lastline=101]{../ocaml/stdlib/printexc.ml}
      }
      \only<1>{
        \listocaml[firstline=170,lastline=172]{../ocaml/stdlib/printexc.ml}{stdlib/printexc.ml:170}
      }
      \only<2>{
        \listocaml[firstline=170,lastline=172,highlightlines=172]{../ocaml/stdlib/printexc.ml}{stdlib/printexc.ml:170}
      }
      \only<3>{
        \mintc[firstline=154,lastline=156]{../ocaml/runtime/backtrace.c}
        \listc[firstline=171,lastline=192,highlightlines={184-187}]{../ocaml/runtime/backtrace.c}{runtime/backtrace.c:154}
      }
      \only<4>{
        \listocaml[firstline=155,lastline=165]{../ocaml/stdlib/printexc.ml}{stdlib/printexc.ml:55}
      }
    \end{column}
  \end{columns}
\end{frame}


\subsection{The multiple kinds of raise}
\frameSubsection{}{}

\begin{frame}{Backtraces}{When backtraces are not needed}
  \begin{columns}[c]
    \begin{column}{0.45\textwidth}
      \minipage[c][0.66\textheight][s]{\columnwidth}
      \begin{itemize}
        \item<1-> What if the backtrace for an exception is never needed?
        \item<2-> It's possible to raise the exception explicitly with \funcname{raise\_notrace} to make raising as cheap as possible
        \item<3-> An example of use in the \funcname{dynlink} library of the OCaml compiler
      \end{itemize}
      \vfill
      \onslide<3->{
      \listocaml[firstline=205,lastline=209,highlightlines=207]{../ocaml/otherlibs/dynlink/dynlink\_compilerlibs/misc.ml}{otherlibs/dynlink/dynlink\_compilerlibs/misc.ml:205}
      }
      \endminipage
    \end{column}
    \begin{column}{0.55\textwidth}
    \only<4>{
      \mintcmm[highlightlines=3]{../src/notrace/camlNotrace\_\_fun_347.cmm}
      \listcmm{../src/notrace/camlNotrace\_\_all\_somes\_327.cmm}{all\_somes}
    }
    \onslide*<5->{
      \listobjdump[highlightlines={6-9}]{../src/notrace/camlNotrace\_\_fun_347.objdump}{anonymous function from \funcname{all\_somes}}
    }
    \end{column}
  \end{columns}
\end{frame}

\begin{frame}{Backtraces}{Summary table of the multiple kinds of raise}
  \begin{itemize}
    \item When debugging information is enabled and if the backtrace of an exception isn't needed, it's possible to raise with \funcname{raise\_notrace} for performance
    \item When debugging information is \emph{not} enabled, the compiler optimises \funcname{raise} calls to inline assembly (same effect as using \funcname{raise\_notrace})
  \end{itemize}
  \bigskip
  \centering
  \begin{tabular}{| l | c | c | c | c |}
    \hline
    \parbox{10em}{Debug information \\ (\funcarg{-g} flag)}  & \multicolumn{2}{|c|}{Disabled}                                     & \multicolumn{2}{|c|}{Enabled} \\
    \hline
    Raise kind                                               & \funcname{raise} & \funcname{raise\_notrace}                       & \funcname{raise} & \funcname{raise\_notrace} \\
    \hline
    Compilation                                              & \parbox{8em}{inline assembly\\ (optimization)} & inline assembly   & \funcname{caml\_raise\_exn} & inline assembly \\
    \hline
  \end{tabular}
\end{frame}


%
% Takeaway #5
%
\subsection*{Takeaway \#5}
\frameSubsectionTakeaway{}
\againframe<5>{takeaway}
}%
    \end{column}
  \end{columns}
\end{frame}

\begin{frame}{Backtraces}{Printing the backtrace}
  \begin{columns}[c]
    \begin{column}{0.45\textwidth}
      \minipage[c][0.66\textheight][s]{\columnwidth}
      \begin{itemize}
        \item<1-> The exception backtrace can now be printed to \funcarg{stdout} with \funcname{Printexc.print_backtrace}
        \item<2-> First the exception backtrace is retrieved into an OCaml array with \funcname{get_raw_backtrace }
        \item<3-> Which is implemented as the \funcname{caml_get_exception_raw_backtrace} C function in the runtime
        \item<4-> And finally printed with \funcname{print_exception_backtrace}
      \end{itemize}
      \vfill
      \mintocaml[firstline=28,lastline=34,highlightlines=33]{../src/backtrace/backtrace.ml}
      \endminipage
    \end{column}
    \begin{column}{0.55\textwidth}
      \onslide*<1,2>{
        \mintocaml[firstline=100,lastline=101]{../ocaml/stdlib/printexc.ml}
      }
      \only<1>{
        \listocaml[firstline=170,lastline=172]{../ocaml/stdlib/printexc.ml}{stdlib/printexc.ml:170}
      }
      \only<2>{
        \listocaml[firstline=170,lastline=172,highlightlines=172]{../ocaml/stdlib/printexc.ml}{stdlib/printexc.ml:170}
      }
      \only<3>{
        \mintc[firstline=154,lastline=156]{../ocaml/runtime/backtrace.c}
        \listc[firstline=171,lastline=192,highlightlines={184-187}]{../ocaml/runtime/backtrace.c}{runtime/backtrace.c:154}
      }
      \only<4>{
        \listocaml[firstline=155,lastline=165]{../ocaml/stdlib/printexc.ml}{stdlib/printexc.ml:55}
      }
    \end{column}
  \end{columns}
\end{frame}


\subsection{The multiple kinds of raise}
\frameSubsection{}{}

\begin{frame}{Backtraces}{When backtraces are not needed}
  \begin{columns}[c]
    \begin{column}{0.45\textwidth}
      \minipage[c][0.66\textheight][s]{\columnwidth}
      \begin{itemize}
        \item<1-> What if the backtrace for an exception is never needed?
        \item<2-> It's possible to raise the exception explicitly with \funcname{raise\_notrace} to make raising as cheap as possible
        \item<3-> An example of use in the \funcname{dynlink} library of the OCaml compiler
      \end{itemize}
      \vfill
      \onslide<3->{
      \listocaml[firstline=205,lastline=209,highlightlines=207]{../ocaml/otherlibs/dynlink/dynlink\_compilerlibs/misc.ml}{otherlibs/dynlink/dynlink\_compilerlibs/misc.ml:205}
      }
      \endminipage
    \end{column}
    \begin{column}{0.55\textwidth}
    \only<4>{
      \mintcmm[highlightlines=3]{../src/notrace/camlNotrace\_\_fun_347.cmm}
      \listcmm{../src/notrace/camlNotrace\_\_all\_somes\_327.cmm}{all\_somes}
    }
    \onslide*<5->{
      \listobjdump[highlightlines={6-9}]{../src/notrace/camlNotrace\_\_fun_347.objdump}{anonymous function from \funcname{all\_somes}}
    }
    \end{column}
  \end{columns}
\end{frame}

\begin{frame}{Backtraces}{Summary table of the multiple kinds of raise}
  \begin{itemize}
    \item When debugging information is enabled and if the backtrace of an exception isn't needed, it's possible to raise with \funcname{raise\_notrace} for performance
    \item When debugging information is \emph{not} enabled, the compiler optimises \funcname{raise} calls to inline assembly (same effect as using \funcname{raise\_notrace})
  \end{itemize}
  \bigskip
  \centering
  \begin{tabular}{| l | c | c | c | c |}
    \hline
    \parbox{10em}{Debug information \\ (\funcarg{-g} flag)}  & \multicolumn{2}{|c|}{Disabled}                                     & \multicolumn{2}{|c|}{Enabled} \\
    \hline
    Raise kind                                               & \funcname{raise} & \funcname{raise\_notrace}                       & \funcname{raise} & \funcname{raise\_notrace} \\
    \hline
    Compilation                                              & \parbox{8em}{inline assembly\\ (optimization)} & inline assembly   & \funcname{caml\_raise\_exn} & inline assembly \\
    \hline
  \end{tabular}
\end{frame}


%
% Takeaway #5
%
\subsection*{Takeaway \#5}
\frameSubsectionTakeaway{}
\againframe<5>{takeaway}
}%
    \end{column}
  \end{columns}
\end{frame}

\begin{frame}{Backtraces}{Back to \funcname{caml\_raise\_exn}}
  \begin{columns}[c]
    \begin{column}{0.5\textwidth}
      \minipage[c][0.66\textheight][s]{\columnwidth}
      \begin{itemize}\color{gray}
        \item Checks wheteher exceptions backtrace should be recorded, by checking the \funcarg{domain\_state->backtrace\_active} value
        \item Switches to the C stack to be able to execute C code
        \item Calls \funcname{caml\_stash\_backtrace}, to collect the backtrace up to the next trap
      \end{itemize}
      \onslide*<2->{
        \begin{itemize}
          \item<2-> Executes the exception handler in \funcname{probably\_raise} with \funcname{RESTORE\_EXN\_HANDLER\_OCAML}
            \begin{itemize}
              \item Set \regname{RSP} to \localname{domain\_state->exn\_handler} value
              \item Restore \localname{domain\_state->exn\_handler} from trap
              \item Jump to the handler code with \funcname{ret}
            \end{itemize}
        \end{itemize}
      }
      \vfill
      \onslide*<1>{\mintocaml[firstline=9,lastline=13,highlightlines=12]{../src/backtrace/backtrace.ml}}
      \onslide*<2->{\mintocaml[firstline=15,lastline=21,highlightlines={19-21}]{../src/backtrace/backtrace.ml}}
      \endminipage
    \end{column}
    \begin{column}{0.5\textwidth}
      \centering
      \onslide*<1>{
        \mintc[firstline=190,lastline=192]{../ocaml/runtime/amd64.S}
        \mintc[firstline=234,lastline=237]{../ocaml/runtime/amd64.S}
      }
      \onslide*<2>{
        \mintc[firstline=190,lastline=192,highlightlines={191-192}]{../ocaml/runtime/amd64.S}
        \mintc[firstline=234,lastline=237,highlightlines={235-237}]{../ocaml/runtime/amd64.S}
      }
      \onslide*<1>{\listCamlRaiseExn[highlightlines=720]{}}
      \onslide*<2>{\listCamlRaiseExn[highlightlines=722]{}}
      \onslide*<3->{\providecommand\step{5}%
% Backtraces
%
\subsection{One advantage of raising with debugging information}
\frameSubsection{}{}

\begin{frame}{Backtraces}{Debugging information \& source code}
  \begin{columns}[c]
    \begin{column}{0.5\textwidth}
      \begin{itemize}
        \item Raising an exception is fun and games, but how do we know where the exception originated? $\rightarrow$ With the backtrace associated with the exception!
        \item In order to get a backtrace from the point where an exception was raised, it's necessary to compile with debugging information enabled (\funcarg{-g} flag for the compiler)
        \item Same example as in the `Nested handlers' section
      \end{itemize}
    \end{column}
    \begin{column}{0.5\textwidth}
      \listocaml[firstline=9,lastline=34]{../src/backtrace/backtrace.ml}{backtrace/backtrace.ml}
    \end{column}
  \end{columns}
\end{frame}

\begin{frame}{Backtraces}{CMM and disassembly of \funcname{definitely\_raise}}
  \begin{columns}[c]
    \begin{column}{0.5\textwidth}
      \minipage[c][0.66\textheight][s]{\columnwidth}
      \begin{itemize}
        \item<1-> An effect of compiling with debugging information (\funcarg{-g}): the CMM no longer uses \funcname{raise\_notrace} to raise the exception but \funcname{raise} instead
        \item<2-> Disassembly shows that CMM \funcname{raise} is actually a call to \funcname{caml\_raise\_exn}, a function in assembler provided by the runtime
      \end{itemize}
      \vfill
      \mintocaml[firstline=9,lastline=13,highlightlines=12]{../src/backtrace/backtrace.ml}
      \endminipage
    \end{column}
    \begin{column}{0.5\textwidth}
      \onslide*<1>{
        \mintcmm[highlightlines=5]{../src/backtrace/camlBacktrace\_\_definitely\_raise\_347.cmm}
      }
      \onslide*<2>{
        \mintobjdump[firstline=2,highlightlines=14]{../src/backtrace/camlBacktrace\_\_definitely\_raise\_347.objdump}
      }
    \end{column}
  \end{columns}
\end{frame}

\begin{frame}{Backtraces}{Introducing \funcname{caml\_raise\_exn}}
  \begin{columns}[c]
    \begin{column}{0.5\textwidth}
      \minipage[c][0.66\textheight][s]{\columnwidth}
      \onslide*<1>{
        \begin{itemize}
          \item Raising the exception is performed using \funcname{caml\_raise\_exn} which is
            \begin{itemize}
              \item An assembly function provided by the runtime
              \item Architecture specific
            \end{itemize}
          \item Let's see what it does differently from \funcname{raise\_notrace}\dots
          \item We are about to raise \typename{ExnC} with \funcname{caml\_raise\_exn} from \funcname{definitely\_raise}
        \end{itemize}
      }
      \onslide*<2>{
        \mintobjdump[firstline=2,highlightlines=14]{../src/backtrace/camlBacktrace\_\_definitely\_raise\_347.objdump}
      }
      \vfill
      \mintocaml[firstline=9,lastline=13,highlightlines=12]{../src/backtrace/backtrace.ml}
      \endminipage
    \end{column}
    \begin{column}{0.5\textwidth}
      \centering
      \providecommand\step{1}%
% Backtraces
%
\subsection{One advantage of raising with debugging information}
\frameSubsection{}{}

\begin{frame}{Backtraces}{Debugging information \& source code}
  \begin{columns}[c]
    \begin{column}{0.5\textwidth}
      \begin{itemize}
        \item Raising an exception is fun and games, but how do we know where the exception originated? $\rightarrow$ With the backtrace associated with the exception!
        \item In order to get a backtrace from the point where an exception was raised, it's necessary to compile with debugging information enabled (\funcarg{-g} flag for the compiler)
        \item Same example as in the `Nested handlers' section
      \end{itemize}
    \end{column}
    \begin{column}{0.5\textwidth}
      \listocaml[firstline=9,lastline=34]{../src/backtrace/backtrace.ml}{backtrace/backtrace.ml}
    \end{column}
  \end{columns}
\end{frame}

\begin{frame}{Backtraces}{CMM and disassembly of \funcname{definitely\_raise}}
  \begin{columns}[c]
    \begin{column}{0.5\textwidth}
      \minipage[c][0.66\textheight][s]{\columnwidth}
      \begin{itemize}
        \item<1-> An effect of compiling with debugging information (\funcarg{-g}): the CMM no longer uses \funcname{raise\_notrace} to raise the exception but \funcname{raise} instead
        \item<2-> Disassembly shows that CMM \funcname{raise} is actually a call to \funcname{caml\_raise\_exn}, a function in assembler provided by the runtime
      \end{itemize}
      \vfill
      \mintocaml[firstline=9,lastline=13,highlightlines=12]{../src/backtrace/backtrace.ml}
      \endminipage
    \end{column}
    \begin{column}{0.5\textwidth}
      \onslide*<1>{
        \mintcmm[highlightlines=5]{../src/backtrace/camlBacktrace\_\_definitely\_raise\_347.cmm}
      }
      \onslide*<2>{
        \mintobjdump[firstline=2,highlightlines=14]{../src/backtrace/camlBacktrace\_\_definitely\_raise\_347.objdump}
      }
    \end{column}
  \end{columns}
\end{frame}

\begin{frame}{Backtraces}{Introducing \funcname{caml\_raise\_exn}}
  \begin{columns}[c]
    \begin{column}{0.5\textwidth}
      \minipage[c][0.66\textheight][s]{\columnwidth}
      \onslide*<1>{
        \begin{itemize}
          \item Raising the exception is performed using \funcname{caml\_raise\_exn} which is
            \begin{itemize}
              \item An assembly function provided by the runtime
              \item Architecture specific
            \end{itemize}
          \item Let's see what it does differently from \funcname{raise\_notrace}\dots
          \item We are about to raise \typename{ExnC} with \funcname{caml\_raise\_exn} from \funcname{definitely\_raise}
        \end{itemize}
      }
      \onslide*<2>{
        \mintobjdump[firstline=2,highlightlines=14]{../src/backtrace/camlBacktrace\_\_definitely\_raise\_347.objdump}
      }
      \vfill
      \mintocaml[firstline=9,lastline=13,highlightlines=12]{../src/backtrace/backtrace.ml}
      \endminipage
    \end{column}
    \begin{column}{0.5\textwidth}
      \centering
      \providecommand\step{1}\input{diagrams/backtrace/backtrace.tex}
    \end{column}
  \end{columns}
\end{frame}

\begin{frame}{Backtraces}{But before that, we need more fields from \typename{caml\_domain\_state}}
  \begin{columns}
    \begin{column}{0.5\textwidth}
      \begin{itemize}
        \item Remember \typename{caml\_domain\_state} the per-domain runtime state?
        \item Some other members are of interest to us now\dots
        \item \localname{backtrace\_active} is used to know if the runtime should collect backtraces
        \item \localname{backtrace\_active} can be set by
          \begin{itemize}
            \item The \funcarg{b} flag in the \funcname{OCAMLRUNPARAM} environment variable
            \item \mintinline{ocaml}{Printexc.record_backtrace : bool -> unit} \footnotemark
          \end{itemize}
      \end{itemize}
    \end{column}
    \begin{column}{0.5\textwidth}
      \begin{listing}
        \mintc[highlightlines={9-11}]{src/ocaml/domain\_state\_backtrace.h}
        \caption{runtime/caml/domain\_state.h}
      \end{listing}
    \end{column}
  \end{columns}
  \footnotetext[1]{https://v2.ocaml.org/api/Printexc.html}
\end{frame}

\newcommand\listCamlRaiseExn[1][]{
  \listasm[firstline=702,lastline=725,#1]{../ocaml/runtime/amd64.S}{runtime/amd64.S:702}}

\begin{frame}{Backtraces}{Walk through \funcname{caml\_raise\_exn}}
  \begin{columns}[T]
    \begin{column}{0.5\textwidth}
      \begin{itemize}
        \item<1-> First checks whether exceptions backtrace should be recorded
        \item<1-> By checking the \funcarg{domain\_state->backtrace\_active} value
        \item<2-> If not, acts as \funcname{raise\_notrace} with \funcname{RESTORE\_EXN\_HANDLER\_OCAML}
          \begin{itemize}
            \item Set \regname{RSP} to \localname{domain\_state->exn\_handler} value
            \item Restore \localname{domain\_state->exn\_handler} from trap
            \item Jump with \funcname{ret} (same effect as using \funcname{pop} and \funcname{jmp})
          \end{itemize}
        \item<3-> Otherwise, switches to the C stack to be able to execute C code
        \item<4-> Calls \funcname{caml\_stash\_backtrace}, a C function provided by the runtime\ignorespaces
          \onslide<6->{, with
            \begin{itemize}
              \item \funcarg{exn}: exception value
              \item \funcarg{pc}: code address of the raising point
              \item \funcarg{sp}: stack address of the raising function
              \item \funcarg{trapsp}: stack address of the next handler
            \end{itemize}
          }
      \end{itemize}
      \bigskip
      \mintocaml[firstline=9,lastline=13,highlightlines=12]{../src/backtrace/backtrace.ml}
    \end{column}
    \begin{column}{0.5\textwidth}
      \raggedleft
      \onslide*<1,3-4>{
        \mintc[firstline=190,lastline=192]{../ocaml/runtime/amd64.S}
        \mintc[firstline=234,lastline=237]{../ocaml/runtime/amd64.S}
      }
      \onslide*<2>{
        \mintc[firstline=190,lastline=192,highlightlines={191-192}]{../ocaml/runtime/amd64.S}
        \mintc[firstline=234,lastline=237,highlightlines={235-237}]{../ocaml/runtime/amd64.S}
      }
      \onslide*<1>{\listCamlRaiseExn[highlightlines=705]{}}%
      \onslide*<2>{\listCamlRaiseExn[highlightlines={707-708}]{}}%
      \onslide*<3>{\listCamlRaiseExn[highlightlines={712-714}]{}}%
      \onslide*<4>{\listCamlRaiseExn[highlightlines={715-720}]{}}%
      \onslide*<5>{\providecommand\step{1}\input{diagrams/backtrace/backtrace.tex}}%
      \onslide*<6>{\providecommand\step{2}\input{diagrams/backtrace/backtrace.tex}}%
    \end{column}
  \end{columns}
\end{frame}

\newcommand\listStashBacktrace[1][]{
  \listc[firstline=91,lastline=120,#1]{../ocaml/runtime/backtrace\_nat.c}{runtime/backtrace\_nat.c:91}}

\begin{frame}{Backtraces}{Introducing \funcname{caml\_stash\_backtrace}}
  \begin{columns}[c]
    \begin{column}{0.5\textwidth}
      \minipage[c][0.66\textheight][s]{\columnwidth}
      \begin{itemize}
        \item<1-> A C function provided by the runtime (specific to native code)
        \item<2-> It scans the stack, between the stack pointer at the raising point up to the next trap, in order to collect the names of the detected function frames
          \onslide*<2>{
            \begin{itemize}
              \item And more information, like line number, column number...
            \end{itemize}
          }
        \item<3-> It uses the same technique and information as the Garbage Collector (GC) uses to scan the values live on the stack
          \onslide*<4>{
            \begin{itemize}
              \item \typename{frame\_descr} are at the core of stack unwinding
            \end{itemize}
          }
      \end{itemize}
      \vfill
      \mintocaml[firstline=9,lastline=13,highlightlines=12]{../src/backtrace/backtrace.ml}
      \endminipage
    \end{column}
    \begin{column}{0.5\textwidth}
      \onslide*<1>{\listStashBacktrace[highlightlines=91]{}}
      \onslide*<2>{\listStashBacktrace[highlightlines={91,117-118}]{}}
      \onslide*<3->{\listStashBacktrace[highlightlines={94,106,109-110}]{}}
    \end{column}
  \end{columns}
\end{frame}

\newcommand\listFrameDescr[1][]{
  \listocaml[firstline=111,lastline=124,#1]{../ocaml/asmcomp/emitaux.ml}{asmcomp/emitaux.ml:111}}
\newcommand\listDebugInfo[1][]{
  \listocaml[firstline=38,lastline=49,#1]{../ocaml/lambda/debuginfo.mli}{lambda/debuginfo.mli:38}}

\begin{frame}{Backtraces}{\typename{frame\_descr} and \typename{frame\_debuginfo} in a nutshell}
  \begin{columns}[c]
    \begin{column}{0.5\textwidth}
      \begin{itemize}
        \item<1-> For every return address in an OCaml program, there is a associated \typename{frame\_descr}
        \item<2-> A \typename{frame\_descr} specifies
        \begin{itemize}
          \item<2-> The size of the function stack frame
          \item<3-> Where live values are on the stack (so that the GC can mark them as live and prevent from being swept)
        \end{itemize}
        \item<4-> When compiled with debug information, \typename{frame\_descr} will be linked to a \typename{frame\_debuginfo}
        \begin{itemize}
          \item<5-> Contains information about the position in the source code (file name, line and column number)
        \end{itemize}
      \end{itemize}
    \end{column}
    \begin{column}{0.5\textwidth}
        \onslide*<1>{\listFrameDescr[highlightlines={118-122}]{}}
        \onslide*<2>{\listFrameDescr[highlightlines={120}]{}}
        \onslide*<3>{\listFrameDescr[highlightlines={121}]{}}
        \onslide*<4->{\listFrameDescr[highlightlines={122}]{}}
        \medskip
        \onslide*<1-3>{\listDebugInfo{}}
        \onslide*<4>{\listDebugInfo[highlightlines={38,49}]{}}
        \onslide*<5>{\listDebugInfo[highlightlines={39-46}]{}}
    \end{column}
  \end{columns}
\end{frame}

\begin{frame}{Backtraces}{Scanning the stack with \typename{frame\_descr}}
  \begin{columns}[c]
    \begin{column}{0.5\textwidth}
      \begin{itemize}
        \item Given the return address of the code that raises the exception by calling \funcname{caml\_raise\_exn} (\funcarg{pc} argument of \funcname{caml\_stash\_backtrace})
        \item And the position of the stack pointer (\funcarg{sp} argument of \funcname{caml\_stash\_backtrace})
        \item The frame size given by \typename{frame\_descr}.\funcarg{frame\_size}, allows to find the end of the previous frame
        \item And the return address of the caller of this function
        \bigskip
        \onslide<3->{
        \item Note that traps (here the reddish \funcname{ExnA}, \funcname{ExnB} and \funcname{ExnC} blocks) belong to the frame that installed them, and so the frame size changes when traps are removed
        }
      \end{itemize}
    \end{column}
    \begin{column}{0.5\textwidth}
      \raggedleft
      \onslide*<1>{\providecommand\step{2}\input{diagrams/backtrace/backtrace.tex}}%
      \onslide*<2>{\providecommand\step{1}\input{diagrams/backtrace/scanstack.tex}}%
      \onslide*<3>{\providecommand\step{2}\input{diagrams/backtrace/scanstack.tex}}%
      \onslide*<4>{\providecommand\step{3}\input{diagrams/backtrace/scanstack.tex}}%
      \onslide*<5>{\providecommand\step{4}\input{diagrams/backtrace/scanstack.tex}}%
      \onslide*<6>{\providecommand\step{5}\input{diagrams/backtrace/scanstack.tex}}%
    \end{column}
  \end{columns}
\end{frame}

% Duplicate of previous-previous slide
\begin{frame}{Backtraces}{Continuing on \funcname{caml\_stash\_backtrace}}
  \begin{columns}[c]
    \begin{column}{0.5\textwidth}
      \minipage[c][0.75\textheight][s]{\columnwidth}
      \begin{itemize}
          \color{gray}
        \item A C function provided by the runtime (specific to native code)
        \item It scans the stack, between the stack pointer at the raising point up to the next trap, in order to collect the names of the detected function frames
        \item It uses the same technique and information as the Garbage Collector (GC) uses to scan the values live on the stack
      \end{itemize}
      \begin{itemize}
        \item For each function frame detected, store a pointer to the \typename{frame\_descr} in the \localname{domain\_state->backtrace\_buffer} array, effectively constructing the backtrace
      \end{itemize}
      \onslide<2->{
        \begin{itemize}
            \smallskip
          \item Constructed backtrace:
            \funcname{definitely\_raise},
            \onslide<3->{\funcname{probably\_raise}}
        \end{itemize}
      }
      \vfill
      \mintocaml[firstline=9,lastline=13,highlightlines=12]{../src/backtrace/backtrace.ml}
      \endminipage
    \end{column}
    \begin{column}{0.5\textwidth}
      \raggedleft
      \onslide*<1>{\listStashBacktrace[highlightlines={114-115}]{}}
      \onslide*<2>{\providecommand\step{3}\input{diagrams/backtrace/backtrace.tex}}%
      \onslide*<3>{\providecommand\step{4}\input{diagrams/backtrace/backtrace.tex}}%
    \end{column}
  \end{columns}
\end{frame}

\begin{frame}{Backtraces}{Back to \funcname{caml\_raise\_exn}}
  \begin{columns}[c]
    \begin{column}{0.5\textwidth}
      \minipage[c][0.66\textheight][s]{\columnwidth}
      \begin{itemize}\color{gray}
        \item Checks wheteher exceptions backtrace should be recorded, by checking the \funcarg{domain\_state->backtrace\_active} value
        \item Switches to the C stack to be able to execute C code
        \item Calls \funcname{caml\_stash\_backtrace}, to collect the backtrace up to the next trap
      \end{itemize}
      \onslide*<2->{
        \begin{itemize}
          \item<2-> Executes the exception handler in \funcname{probably\_raise} with \funcname{RESTORE\_EXN\_HANDLER\_OCAML}
            \begin{itemize}
              \item Set \regname{RSP} to \localname{domain\_state->exn\_handler} value
              \item Restore \localname{domain\_state->exn\_handler} from trap
              \item Jump to the handler code with \funcname{ret}
            \end{itemize}
        \end{itemize}
      }
      \vfill
      \onslide*<1>{\mintocaml[firstline=9,lastline=13,highlightlines=12]{../src/backtrace/backtrace.ml}}
      \onslide*<2->{\mintocaml[firstline=15,lastline=21,highlightlines={19-21}]{../src/backtrace/backtrace.ml}}
      \endminipage
    \end{column}
    \begin{column}{0.5\textwidth}
      \centering
      \onslide*<1>{
        \mintc[firstline=190,lastline=192]{../ocaml/runtime/amd64.S}
        \mintc[firstline=234,lastline=237]{../ocaml/runtime/amd64.S}
      }
      \onslide*<2>{
        \mintc[firstline=190,lastline=192,highlightlines={191-192}]{../ocaml/runtime/amd64.S}
        \mintc[firstline=234,lastline=237,highlightlines={235-237}]{../ocaml/runtime/amd64.S}
      }
      \onslide*<1>{\listCamlRaiseExn[highlightlines=720]{}}
      \onslide*<2>{\listCamlRaiseExn[highlightlines=722]{}}
      \onslide*<3->{\providecommand\step{5}\input{diagrams/backtrace/backtrace.tex}}
    \end{column}
  \end{columns}
\end{frame}

\begin{frame}{Backtraces}{\funcname{caml\_probably\_raise}'s handler doesn't catch \typename{ExnC}}
  \begin{columns}[c]
    \begin{column}{0.45\textwidth}
      \minipage[c][0.75\textheight][s]{\columnwidth}
      \begin{itemize}
        \item<1-> \funcname{match \dots with}\footnotemark pattern matching can also match exceptions
        \item<1-> The CMM of \funcname{probably\_raise} shows that this \funcname{match \dots with} is implemented with a pattern matching and a \funcname{try \dots with}
        \item<1-> But this one only matches on \typename{ExnA} exception
        \item<2-> Since the pattern matching fails to catch the exception, \funcname{reraise} is called with our current \typename{ExnC} exception
        \item<3-> Disassembly shows that \funcname{reraise} is actually a call to \funcname{caml\_reraise\_exn}, which is also an assembly function provided by the runtime
      \end{itemize}
      \vfill
      \mintocaml[firstline=15,lastline=21,highlightlines={18-21}]{../src/backtrace/backtrace.ml}
      \endminipage
    \end{column}
    \begin{column}{0.55\textwidth}
      \centering
      \onslide*<1>{\mintcmm[highlightlines={10-18}]{../src/backtrace/camlBacktrace\_\_probably\_raise\_351.cmm}}
      \onslide*<2>{\listcmm[highlightlines=18]{../src/backtrace/camlBacktrace\_\_probably\_raise_351.cmm}{CMM of probably\_raise}}
      \onslide*<3>{\mintobjdump[firstline=5,lastline=35,highlightlines=31]{../src/backtrace/camlBacktrace\_\_probably\_raise_351.objdump}}
    \end{column}
  \end{columns}
  \only<1>{\footnotetext[1]{https://blog.janestreet.com/pattern-matching-and-exception-handling-unite/}}
\end{frame}

\newcommand\listCamlReraiseExn[1][]{
  \listasm[firstline=727,lastline=734,#1]{../ocaml/runtime/amd64.S}{runtime/amd64.S:727}}

\begin{frame}{Backtraces}{Introducing \funcname{caml\_reraise\_exn}}
  \begin{columns}[c]
    \begin{column}{0.5\textwidth}
      \minipage[c][0.66\textheight][s]{\columnwidth}
      \begin{itemize}
        \item<1-> The exception have to be reraised to the next handler
        \item<2-> First, checks if the backtrace should be collected with \funcarg{domain\_state->backtrace\_active}
        \only<3>{
        \begin{itemize}
            \item If not, behaves like a \funcname{raise\_notrace} with \funcname{RESTORE\_EXN\_HANDLER\_OCAML}
        \end{itemize}
        }
        \item<4-> Jumps into \funcname{caml\_raise\_exn}
        \item<5-> Calls \funcname{caml\_stash\_backtrace} again, to scan the next part of the stack: from the trap just pop-ed to the next trap
      \end{itemize}
      \vfill
      \mintocaml[firstline=15,lastline=21,highlightlines={18-21}]{../src/backtrace/backtrace.ml}
      \endminipage
    \end{column}
    \begin{column}{0.5\textwidth}
      \raggedleft
      \onslide*<1>{\listCamlReraiseExn{}}
      \onslide*<2>{\listCamlReraiseExn[highlightlines=729]{}}
      \onslide*<3>{
        \mintc[firstline=190,lastline=192,highlightlines={191-192}]{../ocaml/runtime/amd64.S}
        \mintc[firstline=234,lastline=237,highlightlines={235-237}]{../ocaml/runtime/amd64.S}
        \medskip
        \listCamlReraiseExn[highlightlines=731]{}
      }
      \onslide*<4-5>{
        % TODO refactor with \listCamlReraiseExn
        \mintasm[firstline=727,lastline=734,highlightlines=730]{../ocaml/runtime/amd64.S}
        \medskip
      }
      \onslide*<4>{\listCamlRaiseExn[highlightlines=711]{}}
      \onslide*<5>{\listCamlRaiseExn[highlightlines=720]{}}
    \end{column}
  \end{columns}
\end{frame}

\begin{frame}{Backtraces}{Backtrace collection continues}
  \begin{columns}[c]
    \begin{column}{0.55\textwidth}
      \minipage[c][0.75\textheight][s]{\columnwidth}
      \begin{itemize}
        \item<1-> \funcname{caml\_stash\_backtrace} scans the older part of the stack, up to the next trap
        \item<6-> Controls jumps to the nested exception handler in \funcname{maybe\_raise}, which matches only on \typename{ExnB}
        \item<7-> \funcname{caml\_reraise\_exn} is called again, backtrace collection resumes with \funcname{caml\_stash\_backtrace}
      \end{itemize}
      \medskip
      \begin{itemize}
        \item<2-> Constructed backtrace:
        \begin{itemize}
          \item <2-> \funcname{definitely\_raise}
          \item <2-> \funcname{probably\_raise}
          \item <3-> \funcname{probably\_raise} (re-raise)
          \item <4-> \funcname{maybe\_raise}
          \item <5-> \funcname{main}
          \item <8-> \funcname{main} (re-raise)
        \end{itemize}
      \end{itemize}
      \vfill
      \onslide*<1-5>{\mintocaml[firstline=15,lastline=21]{../src/backtrace/backtrace.ml}}
      \onslide*<6>{\mintocaml[firstline=28,lastline=34,highlightlines=31]{../src/backtrace/backtrace.ml}}
      \onslide*<7->{\mintocaml[firstline=28,lastline=34,highlightlines=33]{../src/backtrace/backtrace.ml}}
      \endminipage
    \end{column}
    \begin{column}{0.45\textwidth}
      \raggedleft
      \onslide*<1>{\providecommand\step{6}\input{diagrams/backtrace/backtrace.tex}}%
      \onslide*<2>{\providecommand\step{6}\input{diagrams/backtrace/backtrace.tex}}%
      \onslide*<3>{\providecommand\step{7}\input{diagrams/backtrace/backtrace.tex}}%
      \onslide*<4>{\providecommand\step{8}\input{diagrams/backtrace/backtrace.tex}}%
      \onslide*<5>{\providecommand\step{9}\input{diagrams/backtrace/backtrace.tex}}%
      \onslide*<6>{\providecommand\step{10}\input{diagrams/backtrace/backtrace.tex}}%
      \onslide*<7>{\providecommand\step{11}\input{diagrams/backtrace/backtrace.tex}}%
      \onslide*<8>{\providecommand\step{12}\input{diagrams/backtrace/backtrace.tex}}%
      \onslide*<9>{\providecommand\step{13}\input{diagrams/backtrace/backtrace.tex}}%
    \end{column}
  \end{columns}
\end{frame}

\begin{frame}{Backtraces}{Printing the backtrace}
  \begin{columns}[c]
    \begin{column}{0.45\textwidth}
      \minipage[c][0.66\textheight][s]{\columnwidth}
      \begin{itemize}
        \item<1-> The exception backtrace can now be printed to \funcarg{stdout} with \funcname{Printexc.print_backtrace}
        \item<2-> First the exception backtrace is retrieved into an OCaml array with \funcname{get_raw_backtrace }
        \item<3-> Which is implemented as the \funcname{caml_get_exception_raw_backtrace} C function in the runtime
        \item<4-> And finally printed with \funcname{print_exception_backtrace}
      \end{itemize}
      \vfill
      \mintocaml[firstline=28,lastline=34,highlightlines=33]{../src/backtrace/backtrace.ml}
      \endminipage
    \end{column}
    \begin{column}{0.55\textwidth}
      \onslide*<1,2>{
        \mintocaml[firstline=100,lastline=101]{../ocaml/stdlib/printexc.ml}
      }
      \only<1>{
        \listocaml[firstline=170,lastline=172]{../ocaml/stdlib/printexc.ml}{stdlib/printexc.ml:170}
      }
      \only<2>{
        \listocaml[firstline=170,lastline=172,highlightlines=172]{../ocaml/stdlib/printexc.ml}{stdlib/printexc.ml:170}
      }
      \only<3>{
        \mintc[firstline=154,lastline=156]{../ocaml/runtime/backtrace.c}
        \listc[firstline=171,lastline=192,highlightlines={184-187}]{../ocaml/runtime/backtrace.c}{runtime/backtrace.c:154}
      }
      \only<4>{
        \listocaml[firstline=155,lastline=165]{../ocaml/stdlib/printexc.ml}{stdlib/printexc.ml:55}
      }
    \end{column}
  \end{columns}
\end{frame}


\subsection{The multiple kinds of raise}
\frameSubsection{}{}

\begin{frame}{Backtraces}{When backtraces are not needed}
  \begin{columns}[c]
    \begin{column}{0.45\textwidth}
      \minipage[c][0.66\textheight][s]{\columnwidth}
      \begin{itemize}
        \item<1-> What if the backtrace for an exception is never needed?
        \item<2-> It's possible to raise the exception explicitly with \funcname{raise\_notrace} to make raising as cheap as possible
        \item<3-> An example of use in the \funcname{dynlink} library of the OCaml compiler
      \end{itemize}
      \vfill
      \onslide<3->{
      \listocaml[firstline=205,lastline=209,highlightlines=207]{../ocaml/otherlibs/dynlink/dynlink\_compilerlibs/misc.ml}{otherlibs/dynlink/dynlink\_compilerlibs/misc.ml:205}
      }
      \endminipage
    \end{column}
    \begin{column}{0.55\textwidth}
    \only<4>{
      \mintcmm[highlightlines=3]{../src/notrace/camlNotrace\_\_fun_347.cmm}
      \listcmm{../src/notrace/camlNotrace\_\_all\_somes\_327.cmm}{all\_somes}
    }
    \onslide*<5->{
      \listobjdump[highlightlines={6-9}]{../src/notrace/camlNotrace\_\_fun_347.objdump}{anonymous function from \funcname{all\_somes}}
    }
    \end{column}
  \end{columns}
\end{frame}

\begin{frame}{Backtraces}{Summary table of the multiple kinds of raise}
  \begin{itemize}
    \item When debugging information is enabled and if the backtrace of an exception isn't needed, it's possible to raise with \funcname{raise\_notrace} for performance
    \item When debugging information is \emph{not} enabled, the compiler optimises \funcname{raise} calls to inline assembly (same effect as using \funcname{raise\_notrace})
  \end{itemize}
  \bigskip
  \centering
  \begin{tabular}{| l | c | c | c | c |}
    \hline
    \parbox{10em}{Debug information \\ (\funcarg{-g} flag)}  & \multicolumn{2}{|c|}{Disabled}                                     & \multicolumn{2}{|c|}{Enabled} \\
    \hline
    Raise kind                                               & \funcname{raise} & \funcname{raise\_notrace}                       & \funcname{raise} & \funcname{raise\_notrace} \\
    \hline
    Compilation                                              & \parbox{8em}{inline assembly\\ (optimization)} & inline assembly   & \funcname{caml\_raise\_exn} & inline assembly \\
    \hline
  \end{tabular}
\end{frame}


%
% Takeaway #5
%
\subsection*{Takeaway \#5}
\frameSubsectionTakeaway{}
\againframe<5>{takeaway}

    \end{column}
  \end{columns}
\end{frame}

\begin{frame}{Backtraces}{But before that, we need more fields from \typename{caml\_domain\_state}}
  \begin{columns}
    \begin{column}{0.5\textwidth}
      \begin{itemize}
        \item Remember \typename{caml\_domain\_state} the per-domain runtime state?
        \item Some other members are of interest to us now\dots
        \item \localname{backtrace\_active} is used to know if the runtime should collect backtraces
        \item \localname{backtrace\_active} can be set by
          \begin{itemize}
            \item The \funcarg{b} flag in the \funcname{OCAMLRUNPARAM} environment variable
            \item \mintinline{ocaml}{Printexc.record_backtrace : bool -> unit} \footnotemark
          \end{itemize}
      \end{itemize}
    \end{column}
    \begin{column}{0.5\textwidth}
      \begin{listing}
        \mintc[highlightlines={9-11}]{src/ocaml/domain\_state\_backtrace.h}
        \caption{runtime/caml/domain\_state.h}
      \end{listing}
    \end{column}
  \end{columns}
  \footnotetext[1]{https://v2.ocaml.org/api/Printexc.html}
\end{frame}

\newcommand\listCamlRaiseExn[1][]{
  \listasm[firstline=702,lastline=725,#1]{../ocaml/runtime/amd64.S}{runtime/amd64.S:702}}

\begin{frame}{Backtraces}{Walk through \funcname{caml\_raise\_exn}}
  \begin{columns}[T]
    \begin{column}{0.5\textwidth}
      \begin{itemize}
        \item<1-> First checks whether exceptions backtrace should be recorded
        \item<1-> By checking the \funcarg{domain\_state->backtrace\_active} value
        \item<2-> If not, acts as \funcname{raise\_notrace} with \funcname{RESTORE\_EXN\_HANDLER\_OCAML}
          \begin{itemize}
            \item Set \regname{RSP} to \localname{domain\_state->exn\_handler} value
            \item Restore \localname{domain\_state->exn\_handler} from trap
            \item Jump with \funcname{ret} (same effect as using \funcname{pop} and \funcname{jmp})
          \end{itemize}
        \item<3-> Otherwise, switches to the C stack to be able to execute C code
        \item<4-> Calls \funcname{caml\_stash\_backtrace}, a C function provided by the runtime\ignorespaces
          \onslide<6->{, with
            \begin{itemize}
              \item \funcarg{exn}: exception value
              \item \funcarg{pc}: code address of the raising point
              \item \funcarg{sp}: stack address of the raising function
              \item \funcarg{trapsp}: stack address of the next handler
            \end{itemize}
          }
      \end{itemize}
      \bigskip
      \mintocaml[firstline=9,lastline=13,highlightlines=12]{../src/backtrace/backtrace.ml}
    \end{column}
    \begin{column}{0.5\textwidth}
      \raggedleft
      \onslide*<1,3-4>{
        \mintc[firstline=190,lastline=192]{../ocaml/runtime/amd64.S}
        \mintc[firstline=234,lastline=237]{../ocaml/runtime/amd64.S}
      }
      \onslide*<2>{
        \mintc[firstline=190,lastline=192,highlightlines={191-192}]{../ocaml/runtime/amd64.S}
        \mintc[firstline=234,lastline=237,highlightlines={235-237}]{../ocaml/runtime/amd64.S}
      }
      \onslide*<1>{\listCamlRaiseExn[highlightlines=705]{}}%
      \onslide*<2>{\listCamlRaiseExn[highlightlines={707-708}]{}}%
      \onslide*<3>{\listCamlRaiseExn[highlightlines={712-714}]{}}%
      \onslide*<4>{\listCamlRaiseExn[highlightlines={715-720}]{}}%
      \onslide*<5>{\providecommand\step{1}%
% Backtraces
%
\subsection{One advantage of raising with debugging information}
\frameSubsection{}{}

\begin{frame}{Backtraces}{Debugging information \& source code}
  \begin{columns}[c]
    \begin{column}{0.5\textwidth}
      \begin{itemize}
        \item Raising an exception is fun and games, but how do we know where the exception originated? $\rightarrow$ With the backtrace associated with the exception!
        \item In order to get a backtrace from the point where an exception was raised, it's necessary to compile with debugging information enabled (\funcarg{-g} flag for the compiler)
        \item Same example as in the `Nested handlers' section
      \end{itemize}
    \end{column}
    \begin{column}{0.5\textwidth}
      \listocaml[firstline=9,lastline=34]{../src/backtrace/backtrace.ml}{backtrace/backtrace.ml}
    \end{column}
  \end{columns}
\end{frame}

\begin{frame}{Backtraces}{CMM and disassembly of \funcname{definitely\_raise}}
  \begin{columns}[c]
    \begin{column}{0.5\textwidth}
      \minipage[c][0.66\textheight][s]{\columnwidth}
      \begin{itemize}
        \item<1-> An effect of compiling with debugging information (\funcarg{-g}): the CMM no longer uses \funcname{raise\_notrace} to raise the exception but \funcname{raise} instead
        \item<2-> Disassembly shows that CMM \funcname{raise} is actually a call to \funcname{caml\_raise\_exn}, a function in assembler provided by the runtime
      \end{itemize}
      \vfill
      \mintocaml[firstline=9,lastline=13,highlightlines=12]{../src/backtrace/backtrace.ml}
      \endminipage
    \end{column}
    \begin{column}{0.5\textwidth}
      \onslide*<1>{
        \mintcmm[highlightlines=5]{../src/backtrace/camlBacktrace\_\_definitely\_raise\_347.cmm}
      }
      \onslide*<2>{
        \mintobjdump[firstline=2,highlightlines=14]{../src/backtrace/camlBacktrace\_\_definitely\_raise\_347.objdump}
      }
    \end{column}
  \end{columns}
\end{frame}

\begin{frame}{Backtraces}{Introducing \funcname{caml\_raise\_exn}}
  \begin{columns}[c]
    \begin{column}{0.5\textwidth}
      \minipage[c][0.66\textheight][s]{\columnwidth}
      \onslide*<1>{
        \begin{itemize}
          \item Raising the exception is performed using \funcname{caml\_raise\_exn} which is
            \begin{itemize}
              \item An assembly function provided by the runtime
              \item Architecture specific
            \end{itemize}
          \item Let's see what it does differently from \funcname{raise\_notrace}\dots
          \item We are about to raise \typename{ExnC} with \funcname{caml\_raise\_exn} from \funcname{definitely\_raise}
        \end{itemize}
      }
      \onslide*<2>{
        \mintobjdump[firstline=2,highlightlines=14]{../src/backtrace/camlBacktrace\_\_definitely\_raise\_347.objdump}
      }
      \vfill
      \mintocaml[firstline=9,lastline=13,highlightlines=12]{../src/backtrace/backtrace.ml}
      \endminipage
    \end{column}
    \begin{column}{0.5\textwidth}
      \centering
      \providecommand\step{1}\input{diagrams/backtrace/backtrace.tex}
    \end{column}
  \end{columns}
\end{frame}

\begin{frame}{Backtraces}{But before that, we need more fields from \typename{caml\_domain\_state}}
  \begin{columns}
    \begin{column}{0.5\textwidth}
      \begin{itemize}
        \item Remember \typename{caml\_domain\_state} the per-domain runtime state?
        \item Some other members are of interest to us now\dots
        \item \localname{backtrace\_active} is used to know if the runtime should collect backtraces
        \item \localname{backtrace\_active} can be set by
          \begin{itemize}
            \item The \funcarg{b} flag in the \funcname{OCAMLRUNPARAM} environment variable
            \item \mintinline{ocaml}{Printexc.record_backtrace : bool -> unit} \footnotemark
          \end{itemize}
      \end{itemize}
    \end{column}
    \begin{column}{0.5\textwidth}
      \begin{listing}
        \mintc[highlightlines={9-11}]{src/ocaml/domain\_state\_backtrace.h}
        \caption{runtime/caml/domain\_state.h}
      \end{listing}
    \end{column}
  \end{columns}
  \footnotetext[1]{https://v2.ocaml.org/api/Printexc.html}
\end{frame}

\newcommand\listCamlRaiseExn[1][]{
  \listasm[firstline=702,lastline=725,#1]{../ocaml/runtime/amd64.S}{runtime/amd64.S:702}}

\begin{frame}{Backtraces}{Walk through \funcname{caml\_raise\_exn}}
  \begin{columns}[T]
    \begin{column}{0.5\textwidth}
      \begin{itemize}
        \item<1-> First checks whether exceptions backtrace should be recorded
        \item<1-> By checking the \funcarg{domain\_state->backtrace\_active} value
        \item<2-> If not, acts as \funcname{raise\_notrace} with \funcname{RESTORE\_EXN\_HANDLER\_OCAML}
          \begin{itemize}
            \item Set \regname{RSP} to \localname{domain\_state->exn\_handler} value
            \item Restore \localname{domain\_state->exn\_handler} from trap
            \item Jump with \funcname{ret} (same effect as using \funcname{pop} and \funcname{jmp})
          \end{itemize}
        \item<3-> Otherwise, switches to the C stack to be able to execute C code
        \item<4-> Calls \funcname{caml\_stash\_backtrace}, a C function provided by the runtime\ignorespaces
          \onslide<6->{, with
            \begin{itemize}
              \item \funcarg{exn}: exception value
              \item \funcarg{pc}: code address of the raising point
              \item \funcarg{sp}: stack address of the raising function
              \item \funcarg{trapsp}: stack address of the next handler
            \end{itemize}
          }
      \end{itemize}
      \bigskip
      \mintocaml[firstline=9,lastline=13,highlightlines=12]{../src/backtrace/backtrace.ml}
    \end{column}
    \begin{column}{0.5\textwidth}
      \raggedleft
      \onslide*<1,3-4>{
        \mintc[firstline=190,lastline=192]{../ocaml/runtime/amd64.S}
        \mintc[firstline=234,lastline=237]{../ocaml/runtime/amd64.S}
      }
      \onslide*<2>{
        \mintc[firstline=190,lastline=192,highlightlines={191-192}]{../ocaml/runtime/amd64.S}
        \mintc[firstline=234,lastline=237,highlightlines={235-237}]{../ocaml/runtime/amd64.S}
      }
      \onslide*<1>{\listCamlRaiseExn[highlightlines=705]{}}%
      \onslide*<2>{\listCamlRaiseExn[highlightlines={707-708}]{}}%
      \onslide*<3>{\listCamlRaiseExn[highlightlines={712-714}]{}}%
      \onslide*<4>{\listCamlRaiseExn[highlightlines={715-720}]{}}%
      \onslide*<5>{\providecommand\step{1}\input{diagrams/backtrace/backtrace.tex}}%
      \onslide*<6>{\providecommand\step{2}\input{diagrams/backtrace/backtrace.tex}}%
    \end{column}
  \end{columns}
\end{frame}

\newcommand\listStashBacktrace[1][]{
  \listc[firstline=91,lastline=120,#1]{../ocaml/runtime/backtrace\_nat.c}{runtime/backtrace\_nat.c:91}}

\begin{frame}{Backtraces}{Introducing \funcname{caml\_stash\_backtrace}}
  \begin{columns}[c]
    \begin{column}{0.5\textwidth}
      \minipage[c][0.66\textheight][s]{\columnwidth}
      \begin{itemize}
        \item<1-> A C function provided by the runtime (specific to native code)
        \item<2-> It scans the stack, between the stack pointer at the raising point up to the next trap, in order to collect the names of the detected function frames
          \onslide*<2>{
            \begin{itemize}
              \item And more information, like line number, column number...
            \end{itemize}
          }
        \item<3-> It uses the same technique and information as the Garbage Collector (GC) uses to scan the values live on the stack
          \onslide*<4>{
            \begin{itemize}
              \item \typename{frame\_descr} are at the core of stack unwinding
            \end{itemize}
          }
      \end{itemize}
      \vfill
      \mintocaml[firstline=9,lastline=13,highlightlines=12]{../src/backtrace/backtrace.ml}
      \endminipage
    \end{column}
    \begin{column}{0.5\textwidth}
      \onslide*<1>{\listStashBacktrace[highlightlines=91]{}}
      \onslide*<2>{\listStashBacktrace[highlightlines={91,117-118}]{}}
      \onslide*<3->{\listStashBacktrace[highlightlines={94,106,109-110}]{}}
    \end{column}
  \end{columns}
\end{frame}

\newcommand\listFrameDescr[1][]{
  \listocaml[firstline=111,lastline=124,#1]{../ocaml/asmcomp/emitaux.ml}{asmcomp/emitaux.ml:111}}
\newcommand\listDebugInfo[1][]{
  \listocaml[firstline=38,lastline=49,#1]{../ocaml/lambda/debuginfo.mli}{lambda/debuginfo.mli:38}}

\begin{frame}{Backtraces}{\typename{frame\_descr} and \typename{frame\_debuginfo} in a nutshell}
  \begin{columns}[c]
    \begin{column}{0.5\textwidth}
      \begin{itemize}
        \item<1-> For every return address in an OCaml program, there is a associated \typename{frame\_descr}
        \item<2-> A \typename{frame\_descr} specifies
        \begin{itemize}
          \item<2-> The size of the function stack frame
          \item<3-> Where live values are on the stack (so that the GC can mark them as live and prevent from being swept)
        \end{itemize}
        \item<4-> When compiled with debug information, \typename{frame\_descr} will be linked to a \typename{frame\_debuginfo}
        \begin{itemize}
          \item<5-> Contains information about the position in the source code (file name, line and column number)
        \end{itemize}
      \end{itemize}
    \end{column}
    \begin{column}{0.5\textwidth}
        \onslide*<1>{\listFrameDescr[highlightlines={118-122}]{}}
        \onslide*<2>{\listFrameDescr[highlightlines={120}]{}}
        \onslide*<3>{\listFrameDescr[highlightlines={121}]{}}
        \onslide*<4->{\listFrameDescr[highlightlines={122}]{}}
        \medskip
        \onslide*<1-3>{\listDebugInfo{}}
        \onslide*<4>{\listDebugInfo[highlightlines={38,49}]{}}
        \onslide*<5>{\listDebugInfo[highlightlines={39-46}]{}}
    \end{column}
  \end{columns}
\end{frame}

\begin{frame}{Backtraces}{Scanning the stack with \typename{frame\_descr}}
  \begin{columns}[c]
    \begin{column}{0.5\textwidth}
      \begin{itemize}
        \item Given the return address of the code that raises the exception by calling \funcname{caml\_raise\_exn} (\funcarg{pc} argument of \funcname{caml\_stash\_backtrace})
        \item And the position of the stack pointer (\funcarg{sp} argument of \funcname{caml\_stash\_backtrace})
        \item The frame size given by \typename{frame\_descr}.\funcarg{frame\_size}, allows to find the end of the previous frame
        \item And the return address of the caller of this function
        \bigskip
        \onslide<3->{
        \item Note that traps (here the reddish \funcname{ExnA}, \funcname{ExnB} and \funcname{ExnC} blocks) belong to the frame that installed them, and so the frame size changes when traps are removed
        }
      \end{itemize}
    \end{column}
    \begin{column}{0.5\textwidth}
      \raggedleft
      \onslide*<1>{\providecommand\step{2}\input{diagrams/backtrace/backtrace.tex}}%
      \onslide*<2>{\providecommand\step{1}\input{diagrams/backtrace/scanstack.tex}}%
      \onslide*<3>{\providecommand\step{2}\input{diagrams/backtrace/scanstack.tex}}%
      \onslide*<4>{\providecommand\step{3}\input{diagrams/backtrace/scanstack.tex}}%
      \onslide*<5>{\providecommand\step{4}\input{diagrams/backtrace/scanstack.tex}}%
      \onslide*<6>{\providecommand\step{5}\input{diagrams/backtrace/scanstack.tex}}%
    \end{column}
  \end{columns}
\end{frame}

% Duplicate of previous-previous slide
\begin{frame}{Backtraces}{Continuing on \funcname{caml\_stash\_backtrace}}
  \begin{columns}[c]
    \begin{column}{0.5\textwidth}
      \minipage[c][0.75\textheight][s]{\columnwidth}
      \begin{itemize}
          \color{gray}
        \item A C function provided by the runtime (specific to native code)
        \item It scans the stack, between the stack pointer at the raising point up to the next trap, in order to collect the names of the detected function frames
        \item It uses the same technique and information as the Garbage Collector (GC) uses to scan the values live on the stack
      \end{itemize}
      \begin{itemize}
        \item For each function frame detected, store a pointer to the \typename{frame\_descr} in the \localname{domain\_state->backtrace\_buffer} array, effectively constructing the backtrace
      \end{itemize}
      \onslide<2->{
        \begin{itemize}
            \smallskip
          \item Constructed backtrace:
            \funcname{definitely\_raise},
            \onslide<3->{\funcname{probably\_raise}}
        \end{itemize}
      }
      \vfill
      \mintocaml[firstline=9,lastline=13,highlightlines=12]{../src/backtrace/backtrace.ml}
      \endminipage
    \end{column}
    \begin{column}{0.5\textwidth}
      \raggedleft
      \onslide*<1>{\listStashBacktrace[highlightlines={114-115}]{}}
      \onslide*<2>{\providecommand\step{3}\input{diagrams/backtrace/backtrace.tex}}%
      \onslide*<3>{\providecommand\step{4}\input{diagrams/backtrace/backtrace.tex}}%
    \end{column}
  \end{columns}
\end{frame}

\begin{frame}{Backtraces}{Back to \funcname{caml\_raise\_exn}}
  \begin{columns}[c]
    \begin{column}{0.5\textwidth}
      \minipage[c][0.66\textheight][s]{\columnwidth}
      \begin{itemize}\color{gray}
        \item Checks wheteher exceptions backtrace should be recorded, by checking the \funcarg{domain\_state->backtrace\_active} value
        \item Switches to the C stack to be able to execute C code
        \item Calls \funcname{caml\_stash\_backtrace}, to collect the backtrace up to the next trap
      \end{itemize}
      \onslide*<2->{
        \begin{itemize}
          \item<2-> Executes the exception handler in \funcname{probably\_raise} with \funcname{RESTORE\_EXN\_HANDLER\_OCAML}
            \begin{itemize}
              \item Set \regname{RSP} to \localname{domain\_state->exn\_handler} value
              \item Restore \localname{domain\_state->exn\_handler} from trap
              \item Jump to the handler code with \funcname{ret}
            \end{itemize}
        \end{itemize}
      }
      \vfill
      \onslide*<1>{\mintocaml[firstline=9,lastline=13,highlightlines=12]{../src/backtrace/backtrace.ml}}
      \onslide*<2->{\mintocaml[firstline=15,lastline=21,highlightlines={19-21}]{../src/backtrace/backtrace.ml}}
      \endminipage
    \end{column}
    \begin{column}{0.5\textwidth}
      \centering
      \onslide*<1>{
        \mintc[firstline=190,lastline=192]{../ocaml/runtime/amd64.S}
        \mintc[firstline=234,lastline=237]{../ocaml/runtime/amd64.S}
      }
      \onslide*<2>{
        \mintc[firstline=190,lastline=192,highlightlines={191-192}]{../ocaml/runtime/amd64.S}
        \mintc[firstline=234,lastline=237,highlightlines={235-237}]{../ocaml/runtime/amd64.S}
      }
      \onslide*<1>{\listCamlRaiseExn[highlightlines=720]{}}
      \onslide*<2>{\listCamlRaiseExn[highlightlines=722]{}}
      \onslide*<3->{\providecommand\step{5}\input{diagrams/backtrace/backtrace.tex}}
    \end{column}
  \end{columns}
\end{frame}

\begin{frame}{Backtraces}{\funcname{caml\_probably\_raise}'s handler doesn't catch \typename{ExnC}}
  \begin{columns}[c]
    \begin{column}{0.45\textwidth}
      \minipage[c][0.75\textheight][s]{\columnwidth}
      \begin{itemize}
        \item<1-> \funcname{match \dots with}\footnotemark pattern matching can also match exceptions
        \item<1-> The CMM of \funcname{probably\_raise} shows that this \funcname{match \dots with} is implemented with a pattern matching and a \funcname{try \dots with}
        \item<1-> But this one only matches on \typename{ExnA} exception
        \item<2-> Since the pattern matching fails to catch the exception, \funcname{reraise} is called with our current \typename{ExnC} exception
        \item<3-> Disassembly shows that \funcname{reraise} is actually a call to \funcname{caml\_reraise\_exn}, which is also an assembly function provided by the runtime
      \end{itemize}
      \vfill
      \mintocaml[firstline=15,lastline=21,highlightlines={18-21}]{../src/backtrace/backtrace.ml}
      \endminipage
    \end{column}
    \begin{column}{0.55\textwidth}
      \centering
      \onslide*<1>{\mintcmm[highlightlines={10-18}]{../src/backtrace/camlBacktrace\_\_probably\_raise\_351.cmm}}
      \onslide*<2>{\listcmm[highlightlines=18]{../src/backtrace/camlBacktrace\_\_probably\_raise_351.cmm}{CMM of probably\_raise}}
      \onslide*<3>{\mintobjdump[firstline=5,lastline=35,highlightlines=31]{../src/backtrace/camlBacktrace\_\_probably\_raise_351.objdump}}
    \end{column}
  \end{columns}
  \only<1>{\footnotetext[1]{https://blog.janestreet.com/pattern-matching-and-exception-handling-unite/}}
\end{frame}

\newcommand\listCamlReraiseExn[1][]{
  \listasm[firstline=727,lastline=734,#1]{../ocaml/runtime/amd64.S}{runtime/amd64.S:727}}

\begin{frame}{Backtraces}{Introducing \funcname{caml\_reraise\_exn}}
  \begin{columns}[c]
    \begin{column}{0.5\textwidth}
      \minipage[c][0.66\textheight][s]{\columnwidth}
      \begin{itemize}
        \item<1-> The exception have to be reraised to the next handler
        \item<2-> First, checks if the backtrace should be collected with \funcarg{domain\_state->backtrace\_active}
        \only<3>{
        \begin{itemize}
            \item If not, behaves like a \funcname{raise\_notrace} with \funcname{RESTORE\_EXN\_HANDLER\_OCAML}
        \end{itemize}
        }
        \item<4-> Jumps into \funcname{caml\_raise\_exn}
        \item<5-> Calls \funcname{caml\_stash\_backtrace} again, to scan the next part of the stack: from the trap just pop-ed to the next trap
      \end{itemize}
      \vfill
      \mintocaml[firstline=15,lastline=21,highlightlines={18-21}]{../src/backtrace/backtrace.ml}
      \endminipage
    \end{column}
    \begin{column}{0.5\textwidth}
      \raggedleft
      \onslide*<1>{\listCamlReraiseExn{}}
      \onslide*<2>{\listCamlReraiseExn[highlightlines=729]{}}
      \onslide*<3>{
        \mintc[firstline=190,lastline=192,highlightlines={191-192}]{../ocaml/runtime/amd64.S}
        \mintc[firstline=234,lastline=237,highlightlines={235-237}]{../ocaml/runtime/amd64.S}
        \medskip
        \listCamlReraiseExn[highlightlines=731]{}
      }
      \onslide*<4-5>{
        % TODO refactor with \listCamlReraiseExn
        \mintasm[firstline=727,lastline=734,highlightlines=730]{../ocaml/runtime/amd64.S}
        \medskip
      }
      \onslide*<4>{\listCamlRaiseExn[highlightlines=711]{}}
      \onslide*<5>{\listCamlRaiseExn[highlightlines=720]{}}
    \end{column}
  \end{columns}
\end{frame}

\begin{frame}{Backtraces}{Backtrace collection continues}
  \begin{columns}[c]
    \begin{column}{0.55\textwidth}
      \minipage[c][0.75\textheight][s]{\columnwidth}
      \begin{itemize}
        \item<1-> \funcname{caml\_stash\_backtrace} scans the older part of the stack, up to the next trap
        \item<6-> Controls jumps to the nested exception handler in \funcname{maybe\_raise}, which matches only on \typename{ExnB}
        \item<7-> \funcname{caml\_reraise\_exn} is called again, backtrace collection resumes with \funcname{caml\_stash\_backtrace}
      \end{itemize}
      \medskip
      \begin{itemize}
        \item<2-> Constructed backtrace:
        \begin{itemize}
          \item <2-> \funcname{definitely\_raise}
          \item <2-> \funcname{probably\_raise}
          \item <3-> \funcname{probably\_raise} (re-raise)
          \item <4-> \funcname{maybe\_raise}
          \item <5-> \funcname{main}
          \item <8-> \funcname{main} (re-raise)
        \end{itemize}
      \end{itemize}
      \vfill
      \onslide*<1-5>{\mintocaml[firstline=15,lastline=21]{../src/backtrace/backtrace.ml}}
      \onslide*<6>{\mintocaml[firstline=28,lastline=34,highlightlines=31]{../src/backtrace/backtrace.ml}}
      \onslide*<7->{\mintocaml[firstline=28,lastline=34,highlightlines=33]{../src/backtrace/backtrace.ml}}
      \endminipage
    \end{column}
    \begin{column}{0.45\textwidth}
      \raggedleft
      \onslide*<1>{\providecommand\step{6}\input{diagrams/backtrace/backtrace.tex}}%
      \onslide*<2>{\providecommand\step{6}\input{diagrams/backtrace/backtrace.tex}}%
      \onslide*<3>{\providecommand\step{7}\input{diagrams/backtrace/backtrace.tex}}%
      \onslide*<4>{\providecommand\step{8}\input{diagrams/backtrace/backtrace.tex}}%
      \onslide*<5>{\providecommand\step{9}\input{diagrams/backtrace/backtrace.tex}}%
      \onslide*<6>{\providecommand\step{10}\input{diagrams/backtrace/backtrace.tex}}%
      \onslide*<7>{\providecommand\step{11}\input{diagrams/backtrace/backtrace.tex}}%
      \onslide*<8>{\providecommand\step{12}\input{diagrams/backtrace/backtrace.tex}}%
      \onslide*<9>{\providecommand\step{13}\input{diagrams/backtrace/backtrace.tex}}%
    \end{column}
  \end{columns}
\end{frame}

\begin{frame}{Backtraces}{Printing the backtrace}
  \begin{columns}[c]
    \begin{column}{0.45\textwidth}
      \minipage[c][0.66\textheight][s]{\columnwidth}
      \begin{itemize}
        \item<1-> The exception backtrace can now be printed to \funcarg{stdout} with \funcname{Printexc.print_backtrace}
        \item<2-> First the exception backtrace is retrieved into an OCaml array with \funcname{get_raw_backtrace }
        \item<3-> Which is implemented as the \funcname{caml_get_exception_raw_backtrace} C function in the runtime
        \item<4-> And finally printed with \funcname{print_exception_backtrace}
      \end{itemize}
      \vfill
      \mintocaml[firstline=28,lastline=34,highlightlines=33]{../src/backtrace/backtrace.ml}
      \endminipage
    \end{column}
    \begin{column}{0.55\textwidth}
      \onslide*<1,2>{
        \mintocaml[firstline=100,lastline=101]{../ocaml/stdlib/printexc.ml}
      }
      \only<1>{
        \listocaml[firstline=170,lastline=172]{../ocaml/stdlib/printexc.ml}{stdlib/printexc.ml:170}
      }
      \only<2>{
        \listocaml[firstline=170,lastline=172,highlightlines=172]{../ocaml/stdlib/printexc.ml}{stdlib/printexc.ml:170}
      }
      \only<3>{
        \mintc[firstline=154,lastline=156]{../ocaml/runtime/backtrace.c}
        \listc[firstline=171,lastline=192,highlightlines={184-187}]{../ocaml/runtime/backtrace.c}{runtime/backtrace.c:154}
      }
      \only<4>{
        \listocaml[firstline=155,lastline=165]{../ocaml/stdlib/printexc.ml}{stdlib/printexc.ml:55}
      }
    \end{column}
  \end{columns}
\end{frame}


\subsection{The multiple kinds of raise}
\frameSubsection{}{}

\begin{frame}{Backtraces}{When backtraces are not needed}
  \begin{columns}[c]
    \begin{column}{0.45\textwidth}
      \minipage[c][0.66\textheight][s]{\columnwidth}
      \begin{itemize}
        \item<1-> What if the backtrace for an exception is never needed?
        \item<2-> It's possible to raise the exception explicitly with \funcname{raise\_notrace} to make raising as cheap as possible
        \item<3-> An example of use in the \funcname{dynlink} library of the OCaml compiler
      \end{itemize}
      \vfill
      \onslide<3->{
      \listocaml[firstline=205,lastline=209,highlightlines=207]{../ocaml/otherlibs/dynlink/dynlink\_compilerlibs/misc.ml}{otherlibs/dynlink/dynlink\_compilerlibs/misc.ml:205}
      }
      \endminipage
    \end{column}
    \begin{column}{0.55\textwidth}
    \only<4>{
      \mintcmm[highlightlines=3]{../src/notrace/camlNotrace\_\_fun_347.cmm}
      \listcmm{../src/notrace/camlNotrace\_\_all\_somes\_327.cmm}{all\_somes}
    }
    \onslide*<5->{
      \listobjdump[highlightlines={6-9}]{../src/notrace/camlNotrace\_\_fun_347.objdump}{anonymous function from \funcname{all\_somes}}
    }
    \end{column}
  \end{columns}
\end{frame}

\begin{frame}{Backtraces}{Summary table of the multiple kinds of raise}
  \begin{itemize}
    \item When debugging information is enabled and if the backtrace of an exception isn't needed, it's possible to raise with \funcname{raise\_notrace} for performance
    \item When debugging information is \emph{not} enabled, the compiler optimises \funcname{raise} calls to inline assembly (same effect as using \funcname{raise\_notrace})
  \end{itemize}
  \bigskip
  \centering
  \begin{tabular}{| l | c | c | c | c |}
    \hline
    \parbox{10em}{Debug information \\ (\funcarg{-g} flag)}  & \multicolumn{2}{|c|}{Disabled}                                     & \multicolumn{2}{|c|}{Enabled} \\
    \hline
    Raise kind                                               & \funcname{raise} & \funcname{raise\_notrace}                       & \funcname{raise} & \funcname{raise\_notrace} \\
    \hline
    Compilation                                              & \parbox{8em}{inline assembly\\ (optimization)} & inline assembly   & \funcname{caml\_raise\_exn} & inline assembly \\
    \hline
  \end{tabular}
\end{frame}


%
% Takeaway #5
%
\subsection*{Takeaway \#5}
\frameSubsectionTakeaway{}
\againframe<5>{takeaway}
}%
      \onslide*<6>{\providecommand\step{2}%
% Backtraces
%
\subsection{One advantage of raising with debugging information}
\frameSubsection{}{}

\begin{frame}{Backtraces}{Debugging information \& source code}
  \begin{columns}[c]
    \begin{column}{0.5\textwidth}
      \begin{itemize}
        \item Raising an exception is fun and games, but how do we know where the exception originated? $\rightarrow$ With the backtrace associated with the exception!
        \item In order to get a backtrace from the point where an exception was raised, it's necessary to compile with debugging information enabled (\funcarg{-g} flag for the compiler)
        \item Same example as in the `Nested handlers' section
      \end{itemize}
    \end{column}
    \begin{column}{0.5\textwidth}
      \listocaml[firstline=9,lastline=34]{../src/backtrace/backtrace.ml}{backtrace/backtrace.ml}
    \end{column}
  \end{columns}
\end{frame}

\begin{frame}{Backtraces}{CMM and disassembly of \funcname{definitely\_raise}}
  \begin{columns}[c]
    \begin{column}{0.5\textwidth}
      \minipage[c][0.66\textheight][s]{\columnwidth}
      \begin{itemize}
        \item<1-> An effect of compiling with debugging information (\funcarg{-g}): the CMM no longer uses \funcname{raise\_notrace} to raise the exception but \funcname{raise} instead
        \item<2-> Disassembly shows that CMM \funcname{raise} is actually a call to \funcname{caml\_raise\_exn}, a function in assembler provided by the runtime
      \end{itemize}
      \vfill
      \mintocaml[firstline=9,lastline=13,highlightlines=12]{../src/backtrace/backtrace.ml}
      \endminipage
    \end{column}
    \begin{column}{0.5\textwidth}
      \onslide*<1>{
        \mintcmm[highlightlines=5]{../src/backtrace/camlBacktrace\_\_definitely\_raise\_347.cmm}
      }
      \onslide*<2>{
        \mintobjdump[firstline=2,highlightlines=14]{../src/backtrace/camlBacktrace\_\_definitely\_raise\_347.objdump}
      }
    \end{column}
  \end{columns}
\end{frame}

\begin{frame}{Backtraces}{Introducing \funcname{caml\_raise\_exn}}
  \begin{columns}[c]
    \begin{column}{0.5\textwidth}
      \minipage[c][0.66\textheight][s]{\columnwidth}
      \onslide*<1>{
        \begin{itemize}
          \item Raising the exception is performed using \funcname{caml\_raise\_exn} which is
            \begin{itemize}
              \item An assembly function provided by the runtime
              \item Architecture specific
            \end{itemize}
          \item Let's see what it does differently from \funcname{raise\_notrace}\dots
          \item We are about to raise \typename{ExnC} with \funcname{caml\_raise\_exn} from \funcname{definitely\_raise}
        \end{itemize}
      }
      \onslide*<2>{
        \mintobjdump[firstline=2,highlightlines=14]{../src/backtrace/camlBacktrace\_\_definitely\_raise\_347.objdump}
      }
      \vfill
      \mintocaml[firstline=9,lastline=13,highlightlines=12]{../src/backtrace/backtrace.ml}
      \endminipage
    \end{column}
    \begin{column}{0.5\textwidth}
      \centering
      \providecommand\step{1}\input{diagrams/backtrace/backtrace.tex}
    \end{column}
  \end{columns}
\end{frame}

\begin{frame}{Backtraces}{But before that, we need more fields from \typename{caml\_domain\_state}}
  \begin{columns}
    \begin{column}{0.5\textwidth}
      \begin{itemize}
        \item Remember \typename{caml\_domain\_state} the per-domain runtime state?
        \item Some other members are of interest to us now\dots
        \item \localname{backtrace\_active} is used to know if the runtime should collect backtraces
        \item \localname{backtrace\_active} can be set by
          \begin{itemize}
            \item The \funcarg{b} flag in the \funcname{OCAMLRUNPARAM} environment variable
            \item \mintinline{ocaml}{Printexc.record_backtrace : bool -> unit} \footnotemark
          \end{itemize}
      \end{itemize}
    \end{column}
    \begin{column}{0.5\textwidth}
      \begin{listing}
        \mintc[highlightlines={9-11}]{src/ocaml/domain\_state\_backtrace.h}
        \caption{runtime/caml/domain\_state.h}
      \end{listing}
    \end{column}
  \end{columns}
  \footnotetext[1]{https://v2.ocaml.org/api/Printexc.html}
\end{frame}

\newcommand\listCamlRaiseExn[1][]{
  \listasm[firstline=702,lastline=725,#1]{../ocaml/runtime/amd64.S}{runtime/amd64.S:702}}

\begin{frame}{Backtraces}{Walk through \funcname{caml\_raise\_exn}}
  \begin{columns}[T]
    \begin{column}{0.5\textwidth}
      \begin{itemize}
        \item<1-> First checks whether exceptions backtrace should be recorded
        \item<1-> By checking the \funcarg{domain\_state->backtrace\_active} value
        \item<2-> If not, acts as \funcname{raise\_notrace} with \funcname{RESTORE\_EXN\_HANDLER\_OCAML}
          \begin{itemize}
            \item Set \regname{RSP} to \localname{domain\_state->exn\_handler} value
            \item Restore \localname{domain\_state->exn\_handler} from trap
            \item Jump with \funcname{ret} (same effect as using \funcname{pop} and \funcname{jmp})
          \end{itemize}
        \item<3-> Otherwise, switches to the C stack to be able to execute C code
        \item<4-> Calls \funcname{caml\_stash\_backtrace}, a C function provided by the runtime\ignorespaces
          \onslide<6->{, with
            \begin{itemize}
              \item \funcarg{exn}: exception value
              \item \funcarg{pc}: code address of the raising point
              \item \funcarg{sp}: stack address of the raising function
              \item \funcarg{trapsp}: stack address of the next handler
            \end{itemize}
          }
      \end{itemize}
      \bigskip
      \mintocaml[firstline=9,lastline=13,highlightlines=12]{../src/backtrace/backtrace.ml}
    \end{column}
    \begin{column}{0.5\textwidth}
      \raggedleft
      \onslide*<1,3-4>{
        \mintc[firstline=190,lastline=192]{../ocaml/runtime/amd64.S}
        \mintc[firstline=234,lastline=237]{../ocaml/runtime/amd64.S}
      }
      \onslide*<2>{
        \mintc[firstline=190,lastline=192,highlightlines={191-192}]{../ocaml/runtime/amd64.S}
        \mintc[firstline=234,lastline=237,highlightlines={235-237}]{../ocaml/runtime/amd64.S}
      }
      \onslide*<1>{\listCamlRaiseExn[highlightlines=705]{}}%
      \onslide*<2>{\listCamlRaiseExn[highlightlines={707-708}]{}}%
      \onslide*<3>{\listCamlRaiseExn[highlightlines={712-714}]{}}%
      \onslide*<4>{\listCamlRaiseExn[highlightlines={715-720}]{}}%
      \onslide*<5>{\providecommand\step{1}\input{diagrams/backtrace/backtrace.tex}}%
      \onslide*<6>{\providecommand\step{2}\input{diagrams/backtrace/backtrace.tex}}%
    \end{column}
  \end{columns}
\end{frame}

\newcommand\listStashBacktrace[1][]{
  \listc[firstline=91,lastline=120,#1]{../ocaml/runtime/backtrace\_nat.c}{runtime/backtrace\_nat.c:91}}

\begin{frame}{Backtraces}{Introducing \funcname{caml\_stash\_backtrace}}
  \begin{columns}[c]
    \begin{column}{0.5\textwidth}
      \minipage[c][0.66\textheight][s]{\columnwidth}
      \begin{itemize}
        \item<1-> A C function provided by the runtime (specific to native code)
        \item<2-> It scans the stack, between the stack pointer at the raising point up to the next trap, in order to collect the names of the detected function frames
          \onslide*<2>{
            \begin{itemize}
              \item And more information, like line number, column number...
            \end{itemize}
          }
        \item<3-> It uses the same technique and information as the Garbage Collector (GC) uses to scan the values live on the stack
          \onslide*<4>{
            \begin{itemize}
              \item \typename{frame\_descr} are at the core of stack unwinding
            \end{itemize}
          }
      \end{itemize}
      \vfill
      \mintocaml[firstline=9,lastline=13,highlightlines=12]{../src/backtrace/backtrace.ml}
      \endminipage
    \end{column}
    \begin{column}{0.5\textwidth}
      \onslide*<1>{\listStashBacktrace[highlightlines=91]{}}
      \onslide*<2>{\listStashBacktrace[highlightlines={91,117-118}]{}}
      \onslide*<3->{\listStashBacktrace[highlightlines={94,106,109-110}]{}}
    \end{column}
  \end{columns}
\end{frame}

\newcommand\listFrameDescr[1][]{
  \listocaml[firstline=111,lastline=124,#1]{../ocaml/asmcomp/emitaux.ml}{asmcomp/emitaux.ml:111}}
\newcommand\listDebugInfo[1][]{
  \listocaml[firstline=38,lastline=49,#1]{../ocaml/lambda/debuginfo.mli}{lambda/debuginfo.mli:38}}

\begin{frame}{Backtraces}{\typename{frame\_descr} and \typename{frame\_debuginfo} in a nutshell}
  \begin{columns}[c]
    \begin{column}{0.5\textwidth}
      \begin{itemize}
        \item<1-> For every return address in an OCaml program, there is a associated \typename{frame\_descr}
        \item<2-> A \typename{frame\_descr} specifies
        \begin{itemize}
          \item<2-> The size of the function stack frame
          \item<3-> Where live values are on the stack (so that the GC can mark them as live and prevent from being swept)
        \end{itemize}
        \item<4-> When compiled with debug information, \typename{frame\_descr} will be linked to a \typename{frame\_debuginfo}
        \begin{itemize}
          \item<5-> Contains information about the position in the source code (file name, line and column number)
        \end{itemize}
      \end{itemize}
    \end{column}
    \begin{column}{0.5\textwidth}
        \onslide*<1>{\listFrameDescr[highlightlines={118-122}]{}}
        \onslide*<2>{\listFrameDescr[highlightlines={120}]{}}
        \onslide*<3>{\listFrameDescr[highlightlines={121}]{}}
        \onslide*<4->{\listFrameDescr[highlightlines={122}]{}}
        \medskip
        \onslide*<1-3>{\listDebugInfo{}}
        \onslide*<4>{\listDebugInfo[highlightlines={38,49}]{}}
        \onslide*<5>{\listDebugInfo[highlightlines={39-46}]{}}
    \end{column}
  \end{columns}
\end{frame}

\begin{frame}{Backtraces}{Scanning the stack with \typename{frame\_descr}}
  \begin{columns}[c]
    \begin{column}{0.5\textwidth}
      \begin{itemize}
        \item Given the return address of the code that raises the exception by calling \funcname{caml\_raise\_exn} (\funcarg{pc} argument of \funcname{caml\_stash\_backtrace})
        \item And the position of the stack pointer (\funcarg{sp} argument of \funcname{caml\_stash\_backtrace})
        \item The frame size given by \typename{frame\_descr}.\funcarg{frame\_size}, allows to find the end of the previous frame
        \item And the return address of the caller of this function
        \bigskip
        \onslide<3->{
        \item Note that traps (here the reddish \funcname{ExnA}, \funcname{ExnB} and \funcname{ExnC} blocks) belong to the frame that installed them, and so the frame size changes when traps are removed
        }
      \end{itemize}
    \end{column}
    \begin{column}{0.5\textwidth}
      \raggedleft
      \onslide*<1>{\providecommand\step{2}\input{diagrams/backtrace/backtrace.tex}}%
      \onslide*<2>{\providecommand\step{1}\input{diagrams/backtrace/scanstack.tex}}%
      \onslide*<3>{\providecommand\step{2}\input{diagrams/backtrace/scanstack.tex}}%
      \onslide*<4>{\providecommand\step{3}\input{diagrams/backtrace/scanstack.tex}}%
      \onslide*<5>{\providecommand\step{4}\input{diagrams/backtrace/scanstack.tex}}%
      \onslide*<6>{\providecommand\step{5}\input{diagrams/backtrace/scanstack.tex}}%
    \end{column}
  \end{columns}
\end{frame}

% Duplicate of previous-previous slide
\begin{frame}{Backtraces}{Continuing on \funcname{caml\_stash\_backtrace}}
  \begin{columns}[c]
    \begin{column}{0.5\textwidth}
      \minipage[c][0.75\textheight][s]{\columnwidth}
      \begin{itemize}
          \color{gray}
        \item A C function provided by the runtime (specific to native code)
        \item It scans the stack, between the stack pointer at the raising point up to the next trap, in order to collect the names of the detected function frames
        \item It uses the same technique and information as the Garbage Collector (GC) uses to scan the values live on the stack
      \end{itemize}
      \begin{itemize}
        \item For each function frame detected, store a pointer to the \typename{frame\_descr} in the \localname{domain\_state->backtrace\_buffer} array, effectively constructing the backtrace
      \end{itemize}
      \onslide<2->{
        \begin{itemize}
            \smallskip
          \item Constructed backtrace:
            \funcname{definitely\_raise},
            \onslide<3->{\funcname{probably\_raise}}
        \end{itemize}
      }
      \vfill
      \mintocaml[firstline=9,lastline=13,highlightlines=12]{../src/backtrace/backtrace.ml}
      \endminipage
    \end{column}
    \begin{column}{0.5\textwidth}
      \raggedleft
      \onslide*<1>{\listStashBacktrace[highlightlines={114-115}]{}}
      \onslide*<2>{\providecommand\step{3}\input{diagrams/backtrace/backtrace.tex}}%
      \onslide*<3>{\providecommand\step{4}\input{diagrams/backtrace/backtrace.tex}}%
    \end{column}
  \end{columns}
\end{frame}

\begin{frame}{Backtraces}{Back to \funcname{caml\_raise\_exn}}
  \begin{columns}[c]
    \begin{column}{0.5\textwidth}
      \minipage[c][0.66\textheight][s]{\columnwidth}
      \begin{itemize}\color{gray}
        \item Checks wheteher exceptions backtrace should be recorded, by checking the \funcarg{domain\_state->backtrace\_active} value
        \item Switches to the C stack to be able to execute C code
        \item Calls \funcname{caml\_stash\_backtrace}, to collect the backtrace up to the next trap
      \end{itemize}
      \onslide*<2->{
        \begin{itemize}
          \item<2-> Executes the exception handler in \funcname{probably\_raise} with \funcname{RESTORE\_EXN\_HANDLER\_OCAML}
            \begin{itemize}
              \item Set \regname{RSP} to \localname{domain\_state->exn\_handler} value
              \item Restore \localname{domain\_state->exn\_handler} from trap
              \item Jump to the handler code with \funcname{ret}
            \end{itemize}
        \end{itemize}
      }
      \vfill
      \onslide*<1>{\mintocaml[firstline=9,lastline=13,highlightlines=12]{../src/backtrace/backtrace.ml}}
      \onslide*<2->{\mintocaml[firstline=15,lastline=21,highlightlines={19-21}]{../src/backtrace/backtrace.ml}}
      \endminipage
    \end{column}
    \begin{column}{0.5\textwidth}
      \centering
      \onslide*<1>{
        \mintc[firstline=190,lastline=192]{../ocaml/runtime/amd64.S}
        \mintc[firstline=234,lastline=237]{../ocaml/runtime/amd64.S}
      }
      \onslide*<2>{
        \mintc[firstline=190,lastline=192,highlightlines={191-192}]{../ocaml/runtime/amd64.S}
        \mintc[firstline=234,lastline=237,highlightlines={235-237}]{../ocaml/runtime/amd64.S}
      }
      \onslide*<1>{\listCamlRaiseExn[highlightlines=720]{}}
      \onslide*<2>{\listCamlRaiseExn[highlightlines=722]{}}
      \onslide*<3->{\providecommand\step{5}\input{diagrams/backtrace/backtrace.tex}}
    \end{column}
  \end{columns}
\end{frame}

\begin{frame}{Backtraces}{\funcname{caml\_probably\_raise}'s handler doesn't catch \typename{ExnC}}
  \begin{columns}[c]
    \begin{column}{0.45\textwidth}
      \minipage[c][0.75\textheight][s]{\columnwidth}
      \begin{itemize}
        \item<1-> \funcname{match \dots with}\footnotemark pattern matching can also match exceptions
        \item<1-> The CMM of \funcname{probably\_raise} shows that this \funcname{match \dots with} is implemented with a pattern matching and a \funcname{try \dots with}
        \item<1-> But this one only matches on \typename{ExnA} exception
        \item<2-> Since the pattern matching fails to catch the exception, \funcname{reraise} is called with our current \typename{ExnC} exception
        \item<3-> Disassembly shows that \funcname{reraise} is actually a call to \funcname{caml\_reraise\_exn}, which is also an assembly function provided by the runtime
      \end{itemize}
      \vfill
      \mintocaml[firstline=15,lastline=21,highlightlines={18-21}]{../src/backtrace/backtrace.ml}
      \endminipage
    \end{column}
    \begin{column}{0.55\textwidth}
      \centering
      \onslide*<1>{\mintcmm[highlightlines={10-18}]{../src/backtrace/camlBacktrace\_\_probably\_raise\_351.cmm}}
      \onslide*<2>{\listcmm[highlightlines=18]{../src/backtrace/camlBacktrace\_\_probably\_raise_351.cmm}{CMM of probably\_raise}}
      \onslide*<3>{\mintobjdump[firstline=5,lastline=35,highlightlines=31]{../src/backtrace/camlBacktrace\_\_probably\_raise_351.objdump}}
    \end{column}
  \end{columns}
  \only<1>{\footnotetext[1]{https://blog.janestreet.com/pattern-matching-and-exception-handling-unite/}}
\end{frame}

\newcommand\listCamlReraiseExn[1][]{
  \listasm[firstline=727,lastline=734,#1]{../ocaml/runtime/amd64.S}{runtime/amd64.S:727}}

\begin{frame}{Backtraces}{Introducing \funcname{caml\_reraise\_exn}}
  \begin{columns}[c]
    \begin{column}{0.5\textwidth}
      \minipage[c][0.66\textheight][s]{\columnwidth}
      \begin{itemize}
        \item<1-> The exception have to be reraised to the next handler
        \item<2-> First, checks if the backtrace should be collected with \funcarg{domain\_state->backtrace\_active}
        \only<3>{
        \begin{itemize}
            \item If not, behaves like a \funcname{raise\_notrace} with \funcname{RESTORE\_EXN\_HANDLER\_OCAML}
        \end{itemize}
        }
        \item<4-> Jumps into \funcname{caml\_raise\_exn}
        \item<5-> Calls \funcname{caml\_stash\_backtrace} again, to scan the next part of the stack: from the trap just pop-ed to the next trap
      \end{itemize}
      \vfill
      \mintocaml[firstline=15,lastline=21,highlightlines={18-21}]{../src/backtrace/backtrace.ml}
      \endminipage
    \end{column}
    \begin{column}{0.5\textwidth}
      \raggedleft
      \onslide*<1>{\listCamlReraiseExn{}}
      \onslide*<2>{\listCamlReraiseExn[highlightlines=729]{}}
      \onslide*<3>{
        \mintc[firstline=190,lastline=192,highlightlines={191-192}]{../ocaml/runtime/amd64.S}
        \mintc[firstline=234,lastline=237,highlightlines={235-237}]{../ocaml/runtime/amd64.S}
        \medskip
        \listCamlReraiseExn[highlightlines=731]{}
      }
      \onslide*<4-5>{
        % TODO refactor with \listCamlReraiseExn
        \mintasm[firstline=727,lastline=734,highlightlines=730]{../ocaml/runtime/amd64.S}
        \medskip
      }
      \onslide*<4>{\listCamlRaiseExn[highlightlines=711]{}}
      \onslide*<5>{\listCamlRaiseExn[highlightlines=720]{}}
    \end{column}
  \end{columns}
\end{frame}

\begin{frame}{Backtraces}{Backtrace collection continues}
  \begin{columns}[c]
    \begin{column}{0.55\textwidth}
      \minipage[c][0.75\textheight][s]{\columnwidth}
      \begin{itemize}
        \item<1-> \funcname{caml\_stash\_backtrace} scans the older part of the stack, up to the next trap
        \item<6-> Controls jumps to the nested exception handler in \funcname{maybe\_raise}, which matches only on \typename{ExnB}
        \item<7-> \funcname{caml\_reraise\_exn} is called again, backtrace collection resumes with \funcname{caml\_stash\_backtrace}
      \end{itemize}
      \medskip
      \begin{itemize}
        \item<2-> Constructed backtrace:
        \begin{itemize}
          \item <2-> \funcname{definitely\_raise}
          \item <2-> \funcname{probably\_raise}
          \item <3-> \funcname{probably\_raise} (re-raise)
          \item <4-> \funcname{maybe\_raise}
          \item <5-> \funcname{main}
          \item <8-> \funcname{main} (re-raise)
        \end{itemize}
      \end{itemize}
      \vfill
      \onslide*<1-5>{\mintocaml[firstline=15,lastline=21]{../src/backtrace/backtrace.ml}}
      \onslide*<6>{\mintocaml[firstline=28,lastline=34,highlightlines=31]{../src/backtrace/backtrace.ml}}
      \onslide*<7->{\mintocaml[firstline=28,lastline=34,highlightlines=33]{../src/backtrace/backtrace.ml}}
      \endminipage
    \end{column}
    \begin{column}{0.45\textwidth}
      \raggedleft
      \onslide*<1>{\providecommand\step{6}\input{diagrams/backtrace/backtrace.tex}}%
      \onslide*<2>{\providecommand\step{6}\input{diagrams/backtrace/backtrace.tex}}%
      \onslide*<3>{\providecommand\step{7}\input{diagrams/backtrace/backtrace.tex}}%
      \onslide*<4>{\providecommand\step{8}\input{diagrams/backtrace/backtrace.tex}}%
      \onslide*<5>{\providecommand\step{9}\input{diagrams/backtrace/backtrace.tex}}%
      \onslide*<6>{\providecommand\step{10}\input{diagrams/backtrace/backtrace.tex}}%
      \onslide*<7>{\providecommand\step{11}\input{diagrams/backtrace/backtrace.tex}}%
      \onslide*<8>{\providecommand\step{12}\input{diagrams/backtrace/backtrace.tex}}%
      \onslide*<9>{\providecommand\step{13}\input{diagrams/backtrace/backtrace.tex}}%
    \end{column}
  \end{columns}
\end{frame}

\begin{frame}{Backtraces}{Printing the backtrace}
  \begin{columns}[c]
    \begin{column}{0.45\textwidth}
      \minipage[c][0.66\textheight][s]{\columnwidth}
      \begin{itemize}
        \item<1-> The exception backtrace can now be printed to \funcarg{stdout} with \funcname{Printexc.print_backtrace}
        \item<2-> First the exception backtrace is retrieved into an OCaml array with \funcname{get_raw_backtrace }
        \item<3-> Which is implemented as the \funcname{caml_get_exception_raw_backtrace} C function in the runtime
        \item<4-> And finally printed with \funcname{print_exception_backtrace}
      \end{itemize}
      \vfill
      \mintocaml[firstline=28,lastline=34,highlightlines=33]{../src/backtrace/backtrace.ml}
      \endminipage
    \end{column}
    \begin{column}{0.55\textwidth}
      \onslide*<1,2>{
        \mintocaml[firstline=100,lastline=101]{../ocaml/stdlib/printexc.ml}
      }
      \only<1>{
        \listocaml[firstline=170,lastline=172]{../ocaml/stdlib/printexc.ml}{stdlib/printexc.ml:170}
      }
      \only<2>{
        \listocaml[firstline=170,lastline=172,highlightlines=172]{../ocaml/stdlib/printexc.ml}{stdlib/printexc.ml:170}
      }
      \only<3>{
        \mintc[firstline=154,lastline=156]{../ocaml/runtime/backtrace.c}
        \listc[firstline=171,lastline=192,highlightlines={184-187}]{../ocaml/runtime/backtrace.c}{runtime/backtrace.c:154}
      }
      \only<4>{
        \listocaml[firstline=155,lastline=165]{../ocaml/stdlib/printexc.ml}{stdlib/printexc.ml:55}
      }
    \end{column}
  \end{columns}
\end{frame}


\subsection{The multiple kinds of raise}
\frameSubsection{}{}

\begin{frame}{Backtraces}{When backtraces are not needed}
  \begin{columns}[c]
    \begin{column}{0.45\textwidth}
      \minipage[c][0.66\textheight][s]{\columnwidth}
      \begin{itemize}
        \item<1-> What if the backtrace for an exception is never needed?
        \item<2-> It's possible to raise the exception explicitly with \funcname{raise\_notrace} to make raising as cheap as possible
        \item<3-> An example of use in the \funcname{dynlink} library of the OCaml compiler
      \end{itemize}
      \vfill
      \onslide<3->{
      \listocaml[firstline=205,lastline=209,highlightlines=207]{../ocaml/otherlibs/dynlink/dynlink\_compilerlibs/misc.ml}{otherlibs/dynlink/dynlink\_compilerlibs/misc.ml:205}
      }
      \endminipage
    \end{column}
    \begin{column}{0.55\textwidth}
    \only<4>{
      \mintcmm[highlightlines=3]{../src/notrace/camlNotrace\_\_fun_347.cmm}
      \listcmm{../src/notrace/camlNotrace\_\_all\_somes\_327.cmm}{all\_somes}
    }
    \onslide*<5->{
      \listobjdump[highlightlines={6-9}]{../src/notrace/camlNotrace\_\_fun_347.objdump}{anonymous function from \funcname{all\_somes}}
    }
    \end{column}
  \end{columns}
\end{frame}

\begin{frame}{Backtraces}{Summary table of the multiple kinds of raise}
  \begin{itemize}
    \item When debugging information is enabled and if the backtrace of an exception isn't needed, it's possible to raise with \funcname{raise\_notrace} for performance
    \item When debugging information is \emph{not} enabled, the compiler optimises \funcname{raise} calls to inline assembly (same effect as using \funcname{raise\_notrace})
  \end{itemize}
  \bigskip
  \centering
  \begin{tabular}{| l | c | c | c | c |}
    \hline
    \parbox{10em}{Debug information \\ (\funcarg{-g} flag)}  & \multicolumn{2}{|c|}{Disabled}                                     & \multicolumn{2}{|c|}{Enabled} \\
    \hline
    Raise kind                                               & \funcname{raise} & \funcname{raise\_notrace}                       & \funcname{raise} & \funcname{raise\_notrace} \\
    \hline
    Compilation                                              & \parbox{8em}{inline assembly\\ (optimization)} & inline assembly   & \funcname{caml\_raise\_exn} & inline assembly \\
    \hline
  \end{tabular}
\end{frame}


%
% Takeaway #5
%
\subsection*{Takeaway \#5}
\frameSubsectionTakeaway{}
\againframe<5>{takeaway}
}%
    \end{column}
  \end{columns}
\end{frame}

\newcommand\listStashBacktrace[1][]{
  \listc[firstline=91,lastline=120,#1]{../ocaml/runtime/backtrace\_nat.c}{runtime/backtrace\_nat.c:91}}

\begin{frame}{Backtraces}{Introducing \funcname{caml\_stash\_backtrace}}
  \begin{columns}[c]
    \begin{column}{0.5\textwidth}
      \minipage[c][0.66\textheight][s]{\columnwidth}
      \begin{itemize}
        \item<1-> A C function provided by the runtime (specific to native code)
        \item<2-> It scans the stack, between the stack pointer at the raising point up to the next trap, in order to collect the names of the detected function frames
          \onslide*<2>{
            \begin{itemize}
              \item And more information, like line number, column number...
            \end{itemize}
          }
        \item<3-> It uses the same technique and information as the Garbage Collector (GC) uses to scan the values live on the stack
          \onslide*<4>{
            \begin{itemize}
              \item \typename{frame\_descr} are at the core of stack unwinding
            \end{itemize}
          }
      \end{itemize}
      \vfill
      \mintocaml[firstline=9,lastline=13,highlightlines=12]{../src/backtrace/backtrace.ml}
      \endminipage
    \end{column}
    \begin{column}{0.5\textwidth}
      \onslide*<1>{\listStashBacktrace[highlightlines=91]{}}
      \onslide*<2>{\listStashBacktrace[highlightlines={91,117-118}]{}}
      \onslide*<3->{\listStashBacktrace[highlightlines={94,106,109-110}]{}}
    \end{column}
  \end{columns}
\end{frame}

\newcommand\listFrameDescr[1][]{
  \listocaml[firstline=111,lastline=124,#1]{../ocaml/asmcomp/emitaux.ml}{asmcomp/emitaux.ml:111}}
\newcommand\listDebugInfo[1][]{
  \listocaml[firstline=38,lastline=49,#1]{../ocaml/lambda/debuginfo.mli}{lambda/debuginfo.mli:38}}

\begin{frame}{Backtraces}{\typename{frame\_descr} and \typename{frame\_debuginfo} in a nutshell}
  \begin{columns}[c]
    \begin{column}{0.5\textwidth}
      \begin{itemize}
        \item<1-> For every return address in an OCaml program, there is a associated \typename{frame\_descr}
        \item<2-> A \typename{frame\_descr} specifies
        \begin{itemize}
          \item<2-> The size of the function stack frame
          \item<3-> Where live values are on the stack (so that the GC can mark them as live and prevent from being swept)
        \end{itemize}
        \item<4-> When compiled with debug information, \typename{frame\_descr} will be linked to a \typename{frame\_debuginfo}
        \begin{itemize}
          \item<5-> Contains information about the position in the source code (file name, line and column number)
        \end{itemize}
      \end{itemize}
    \end{column}
    \begin{column}{0.5\textwidth}
        \onslide*<1>{\listFrameDescr[highlightlines={118-122}]{}}
        \onslide*<2>{\listFrameDescr[highlightlines={120}]{}}
        \onslide*<3>{\listFrameDescr[highlightlines={121}]{}}
        \onslide*<4->{\listFrameDescr[highlightlines={122}]{}}
        \medskip
        \onslide*<1-3>{\listDebugInfo{}}
        \onslide*<4>{\listDebugInfo[highlightlines={38,49}]{}}
        \onslide*<5>{\listDebugInfo[highlightlines={39-46}]{}}
    \end{column}
  \end{columns}
\end{frame}

\begin{frame}{Backtraces}{Scanning the stack with \typename{frame\_descr}}
  \begin{columns}[c]
    \begin{column}{0.5\textwidth}
      \begin{itemize}
        \item Given the return address of the code that raises the exception by calling \funcname{caml\_raise\_exn} (\funcarg{pc} argument of \funcname{caml\_stash\_backtrace})
        \item And the position of the stack pointer (\funcarg{sp} argument of \funcname{caml\_stash\_backtrace})
        \item The frame size given by \typename{frame\_descr}.\funcarg{frame\_size}, allows to find the end of the previous frame
        \item And the return address of the caller of this function
        \bigskip
        \onslide<3->{
        \item Note that traps (here the reddish \funcname{ExnA}, \funcname{ExnB} and \funcname{ExnC} blocks) belong to the frame that installed them, and so the frame size changes when traps are removed
        }
      \end{itemize}
    \end{column}
    \begin{column}{0.5\textwidth}
      \raggedleft
      \onslide*<1>{\providecommand\step{2}%
% Backtraces
%
\subsection{One advantage of raising with debugging information}
\frameSubsection{}{}

\begin{frame}{Backtraces}{Debugging information \& source code}
  \begin{columns}[c]
    \begin{column}{0.5\textwidth}
      \begin{itemize}
        \item Raising an exception is fun and games, but how do we know where the exception originated? $\rightarrow$ With the backtrace associated with the exception!
        \item In order to get a backtrace from the point where an exception was raised, it's necessary to compile with debugging information enabled (\funcarg{-g} flag for the compiler)
        \item Same example as in the `Nested handlers' section
      \end{itemize}
    \end{column}
    \begin{column}{0.5\textwidth}
      \listocaml[firstline=9,lastline=34]{../src/backtrace/backtrace.ml}{backtrace/backtrace.ml}
    \end{column}
  \end{columns}
\end{frame}

\begin{frame}{Backtraces}{CMM and disassembly of \funcname{definitely\_raise}}
  \begin{columns}[c]
    \begin{column}{0.5\textwidth}
      \minipage[c][0.66\textheight][s]{\columnwidth}
      \begin{itemize}
        \item<1-> An effect of compiling with debugging information (\funcarg{-g}): the CMM no longer uses \funcname{raise\_notrace} to raise the exception but \funcname{raise} instead
        \item<2-> Disassembly shows that CMM \funcname{raise} is actually a call to \funcname{caml\_raise\_exn}, a function in assembler provided by the runtime
      \end{itemize}
      \vfill
      \mintocaml[firstline=9,lastline=13,highlightlines=12]{../src/backtrace/backtrace.ml}
      \endminipage
    \end{column}
    \begin{column}{0.5\textwidth}
      \onslide*<1>{
        \mintcmm[highlightlines=5]{../src/backtrace/camlBacktrace\_\_definitely\_raise\_347.cmm}
      }
      \onslide*<2>{
        \mintobjdump[firstline=2,highlightlines=14]{../src/backtrace/camlBacktrace\_\_definitely\_raise\_347.objdump}
      }
    \end{column}
  \end{columns}
\end{frame}

\begin{frame}{Backtraces}{Introducing \funcname{caml\_raise\_exn}}
  \begin{columns}[c]
    \begin{column}{0.5\textwidth}
      \minipage[c][0.66\textheight][s]{\columnwidth}
      \onslide*<1>{
        \begin{itemize}
          \item Raising the exception is performed using \funcname{caml\_raise\_exn} which is
            \begin{itemize}
              \item An assembly function provided by the runtime
              \item Architecture specific
            \end{itemize}
          \item Let's see what it does differently from \funcname{raise\_notrace}\dots
          \item We are about to raise \typename{ExnC} with \funcname{caml\_raise\_exn} from \funcname{definitely\_raise}
        \end{itemize}
      }
      \onslide*<2>{
        \mintobjdump[firstline=2,highlightlines=14]{../src/backtrace/camlBacktrace\_\_definitely\_raise\_347.objdump}
      }
      \vfill
      \mintocaml[firstline=9,lastline=13,highlightlines=12]{../src/backtrace/backtrace.ml}
      \endminipage
    \end{column}
    \begin{column}{0.5\textwidth}
      \centering
      \providecommand\step{1}\input{diagrams/backtrace/backtrace.tex}
    \end{column}
  \end{columns}
\end{frame}

\begin{frame}{Backtraces}{But before that, we need more fields from \typename{caml\_domain\_state}}
  \begin{columns}
    \begin{column}{0.5\textwidth}
      \begin{itemize}
        \item Remember \typename{caml\_domain\_state} the per-domain runtime state?
        \item Some other members are of interest to us now\dots
        \item \localname{backtrace\_active} is used to know if the runtime should collect backtraces
        \item \localname{backtrace\_active} can be set by
          \begin{itemize}
            \item The \funcarg{b} flag in the \funcname{OCAMLRUNPARAM} environment variable
            \item \mintinline{ocaml}{Printexc.record_backtrace : bool -> unit} \footnotemark
          \end{itemize}
      \end{itemize}
    \end{column}
    \begin{column}{0.5\textwidth}
      \begin{listing}
        \mintc[highlightlines={9-11}]{src/ocaml/domain\_state\_backtrace.h}
        \caption{runtime/caml/domain\_state.h}
      \end{listing}
    \end{column}
  \end{columns}
  \footnotetext[1]{https://v2.ocaml.org/api/Printexc.html}
\end{frame}

\newcommand\listCamlRaiseExn[1][]{
  \listasm[firstline=702,lastline=725,#1]{../ocaml/runtime/amd64.S}{runtime/amd64.S:702}}

\begin{frame}{Backtraces}{Walk through \funcname{caml\_raise\_exn}}
  \begin{columns}[T]
    \begin{column}{0.5\textwidth}
      \begin{itemize}
        \item<1-> First checks whether exceptions backtrace should be recorded
        \item<1-> By checking the \funcarg{domain\_state->backtrace\_active} value
        \item<2-> If not, acts as \funcname{raise\_notrace} with \funcname{RESTORE\_EXN\_HANDLER\_OCAML}
          \begin{itemize}
            \item Set \regname{RSP} to \localname{domain\_state->exn\_handler} value
            \item Restore \localname{domain\_state->exn\_handler} from trap
            \item Jump with \funcname{ret} (same effect as using \funcname{pop} and \funcname{jmp})
          \end{itemize}
        \item<3-> Otherwise, switches to the C stack to be able to execute C code
        \item<4-> Calls \funcname{caml\_stash\_backtrace}, a C function provided by the runtime\ignorespaces
          \onslide<6->{, with
            \begin{itemize}
              \item \funcarg{exn}: exception value
              \item \funcarg{pc}: code address of the raising point
              \item \funcarg{sp}: stack address of the raising function
              \item \funcarg{trapsp}: stack address of the next handler
            \end{itemize}
          }
      \end{itemize}
      \bigskip
      \mintocaml[firstline=9,lastline=13,highlightlines=12]{../src/backtrace/backtrace.ml}
    \end{column}
    \begin{column}{0.5\textwidth}
      \raggedleft
      \onslide*<1,3-4>{
        \mintc[firstline=190,lastline=192]{../ocaml/runtime/amd64.S}
        \mintc[firstline=234,lastline=237]{../ocaml/runtime/amd64.S}
      }
      \onslide*<2>{
        \mintc[firstline=190,lastline=192,highlightlines={191-192}]{../ocaml/runtime/amd64.S}
        \mintc[firstline=234,lastline=237,highlightlines={235-237}]{../ocaml/runtime/amd64.S}
      }
      \onslide*<1>{\listCamlRaiseExn[highlightlines=705]{}}%
      \onslide*<2>{\listCamlRaiseExn[highlightlines={707-708}]{}}%
      \onslide*<3>{\listCamlRaiseExn[highlightlines={712-714}]{}}%
      \onslide*<4>{\listCamlRaiseExn[highlightlines={715-720}]{}}%
      \onslide*<5>{\providecommand\step{1}\input{diagrams/backtrace/backtrace.tex}}%
      \onslide*<6>{\providecommand\step{2}\input{diagrams/backtrace/backtrace.tex}}%
    \end{column}
  \end{columns}
\end{frame}

\newcommand\listStashBacktrace[1][]{
  \listc[firstline=91,lastline=120,#1]{../ocaml/runtime/backtrace\_nat.c}{runtime/backtrace\_nat.c:91}}

\begin{frame}{Backtraces}{Introducing \funcname{caml\_stash\_backtrace}}
  \begin{columns}[c]
    \begin{column}{0.5\textwidth}
      \minipage[c][0.66\textheight][s]{\columnwidth}
      \begin{itemize}
        \item<1-> A C function provided by the runtime (specific to native code)
        \item<2-> It scans the stack, between the stack pointer at the raising point up to the next trap, in order to collect the names of the detected function frames
          \onslide*<2>{
            \begin{itemize}
              \item And more information, like line number, column number...
            \end{itemize}
          }
        \item<3-> It uses the same technique and information as the Garbage Collector (GC) uses to scan the values live on the stack
          \onslide*<4>{
            \begin{itemize}
              \item \typename{frame\_descr} are at the core of stack unwinding
            \end{itemize}
          }
      \end{itemize}
      \vfill
      \mintocaml[firstline=9,lastline=13,highlightlines=12]{../src/backtrace/backtrace.ml}
      \endminipage
    \end{column}
    \begin{column}{0.5\textwidth}
      \onslide*<1>{\listStashBacktrace[highlightlines=91]{}}
      \onslide*<2>{\listStashBacktrace[highlightlines={91,117-118}]{}}
      \onslide*<3->{\listStashBacktrace[highlightlines={94,106,109-110}]{}}
    \end{column}
  \end{columns}
\end{frame}

\newcommand\listFrameDescr[1][]{
  \listocaml[firstline=111,lastline=124,#1]{../ocaml/asmcomp/emitaux.ml}{asmcomp/emitaux.ml:111}}
\newcommand\listDebugInfo[1][]{
  \listocaml[firstline=38,lastline=49,#1]{../ocaml/lambda/debuginfo.mli}{lambda/debuginfo.mli:38}}

\begin{frame}{Backtraces}{\typename{frame\_descr} and \typename{frame\_debuginfo} in a nutshell}
  \begin{columns}[c]
    \begin{column}{0.5\textwidth}
      \begin{itemize}
        \item<1-> For every return address in an OCaml program, there is a associated \typename{frame\_descr}
        \item<2-> A \typename{frame\_descr} specifies
        \begin{itemize}
          \item<2-> The size of the function stack frame
          \item<3-> Where live values are on the stack (so that the GC can mark them as live and prevent from being swept)
        \end{itemize}
        \item<4-> When compiled with debug information, \typename{frame\_descr} will be linked to a \typename{frame\_debuginfo}
        \begin{itemize}
          \item<5-> Contains information about the position in the source code (file name, line and column number)
        \end{itemize}
      \end{itemize}
    \end{column}
    \begin{column}{0.5\textwidth}
        \onslide*<1>{\listFrameDescr[highlightlines={118-122}]{}}
        \onslide*<2>{\listFrameDescr[highlightlines={120}]{}}
        \onslide*<3>{\listFrameDescr[highlightlines={121}]{}}
        \onslide*<4->{\listFrameDescr[highlightlines={122}]{}}
        \medskip
        \onslide*<1-3>{\listDebugInfo{}}
        \onslide*<4>{\listDebugInfo[highlightlines={38,49}]{}}
        \onslide*<5>{\listDebugInfo[highlightlines={39-46}]{}}
    \end{column}
  \end{columns}
\end{frame}

\begin{frame}{Backtraces}{Scanning the stack with \typename{frame\_descr}}
  \begin{columns}[c]
    \begin{column}{0.5\textwidth}
      \begin{itemize}
        \item Given the return address of the code that raises the exception by calling \funcname{caml\_raise\_exn} (\funcarg{pc} argument of \funcname{caml\_stash\_backtrace})
        \item And the position of the stack pointer (\funcarg{sp} argument of \funcname{caml\_stash\_backtrace})
        \item The frame size given by \typename{frame\_descr}.\funcarg{frame\_size}, allows to find the end of the previous frame
        \item And the return address of the caller of this function
        \bigskip
        \onslide<3->{
        \item Note that traps (here the reddish \funcname{ExnA}, \funcname{ExnB} and \funcname{ExnC} blocks) belong to the frame that installed them, and so the frame size changes when traps are removed
        }
      \end{itemize}
    \end{column}
    \begin{column}{0.5\textwidth}
      \raggedleft
      \onslide*<1>{\providecommand\step{2}\input{diagrams/backtrace/backtrace.tex}}%
      \onslide*<2>{\providecommand\step{1}\input{diagrams/backtrace/scanstack.tex}}%
      \onslide*<3>{\providecommand\step{2}\input{diagrams/backtrace/scanstack.tex}}%
      \onslide*<4>{\providecommand\step{3}\input{diagrams/backtrace/scanstack.tex}}%
      \onslide*<5>{\providecommand\step{4}\input{diagrams/backtrace/scanstack.tex}}%
      \onslide*<6>{\providecommand\step{5}\input{diagrams/backtrace/scanstack.tex}}%
    \end{column}
  \end{columns}
\end{frame}

% Duplicate of previous-previous slide
\begin{frame}{Backtraces}{Continuing on \funcname{caml\_stash\_backtrace}}
  \begin{columns}[c]
    \begin{column}{0.5\textwidth}
      \minipage[c][0.75\textheight][s]{\columnwidth}
      \begin{itemize}
          \color{gray}
        \item A C function provided by the runtime (specific to native code)
        \item It scans the stack, between the stack pointer at the raising point up to the next trap, in order to collect the names of the detected function frames
        \item It uses the same technique and information as the Garbage Collector (GC) uses to scan the values live on the stack
      \end{itemize}
      \begin{itemize}
        \item For each function frame detected, store a pointer to the \typename{frame\_descr} in the \localname{domain\_state->backtrace\_buffer} array, effectively constructing the backtrace
      \end{itemize}
      \onslide<2->{
        \begin{itemize}
            \smallskip
          \item Constructed backtrace:
            \funcname{definitely\_raise},
            \onslide<3->{\funcname{probably\_raise}}
        \end{itemize}
      }
      \vfill
      \mintocaml[firstline=9,lastline=13,highlightlines=12]{../src/backtrace/backtrace.ml}
      \endminipage
    \end{column}
    \begin{column}{0.5\textwidth}
      \raggedleft
      \onslide*<1>{\listStashBacktrace[highlightlines={114-115}]{}}
      \onslide*<2>{\providecommand\step{3}\input{diagrams/backtrace/backtrace.tex}}%
      \onslide*<3>{\providecommand\step{4}\input{diagrams/backtrace/backtrace.tex}}%
    \end{column}
  \end{columns}
\end{frame}

\begin{frame}{Backtraces}{Back to \funcname{caml\_raise\_exn}}
  \begin{columns}[c]
    \begin{column}{0.5\textwidth}
      \minipage[c][0.66\textheight][s]{\columnwidth}
      \begin{itemize}\color{gray}
        \item Checks wheteher exceptions backtrace should be recorded, by checking the \funcarg{domain\_state->backtrace\_active} value
        \item Switches to the C stack to be able to execute C code
        \item Calls \funcname{caml\_stash\_backtrace}, to collect the backtrace up to the next trap
      \end{itemize}
      \onslide*<2->{
        \begin{itemize}
          \item<2-> Executes the exception handler in \funcname{probably\_raise} with \funcname{RESTORE\_EXN\_HANDLER\_OCAML}
            \begin{itemize}
              \item Set \regname{RSP} to \localname{domain\_state->exn\_handler} value
              \item Restore \localname{domain\_state->exn\_handler} from trap
              \item Jump to the handler code with \funcname{ret}
            \end{itemize}
        \end{itemize}
      }
      \vfill
      \onslide*<1>{\mintocaml[firstline=9,lastline=13,highlightlines=12]{../src/backtrace/backtrace.ml}}
      \onslide*<2->{\mintocaml[firstline=15,lastline=21,highlightlines={19-21}]{../src/backtrace/backtrace.ml}}
      \endminipage
    \end{column}
    \begin{column}{0.5\textwidth}
      \centering
      \onslide*<1>{
        \mintc[firstline=190,lastline=192]{../ocaml/runtime/amd64.S}
        \mintc[firstline=234,lastline=237]{../ocaml/runtime/amd64.S}
      }
      \onslide*<2>{
        \mintc[firstline=190,lastline=192,highlightlines={191-192}]{../ocaml/runtime/amd64.S}
        \mintc[firstline=234,lastline=237,highlightlines={235-237}]{../ocaml/runtime/amd64.S}
      }
      \onslide*<1>{\listCamlRaiseExn[highlightlines=720]{}}
      \onslide*<2>{\listCamlRaiseExn[highlightlines=722]{}}
      \onslide*<3->{\providecommand\step{5}\input{diagrams/backtrace/backtrace.tex}}
    \end{column}
  \end{columns}
\end{frame}

\begin{frame}{Backtraces}{\funcname{caml\_probably\_raise}'s handler doesn't catch \typename{ExnC}}
  \begin{columns}[c]
    \begin{column}{0.45\textwidth}
      \minipage[c][0.75\textheight][s]{\columnwidth}
      \begin{itemize}
        \item<1-> \funcname{match \dots with}\footnotemark pattern matching can also match exceptions
        \item<1-> The CMM of \funcname{probably\_raise} shows that this \funcname{match \dots with} is implemented with a pattern matching and a \funcname{try \dots with}
        \item<1-> But this one only matches on \typename{ExnA} exception
        \item<2-> Since the pattern matching fails to catch the exception, \funcname{reraise} is called with our current \typename{ExnC} exception
        \item<3-> Disassembly shows that \funcname{reraise} is actually a call to \funcname{caml\_reraise\_exn}, which is also an assembly function provided by the runtime
      \end{itemize}
      \vfill
      \mintocaml[firstline=15,lastline=21,highlightlines={18-21}]{../src/backtrace/backtrace.ml}
      \endminipage
    \end{column}
    \begin{column}{0.55\textwidth}
      \centering
      \onslide*<1>{\mintcmm[highlightlines={10-18}]{../src/backtrace/camlBacktrace\_\_probably\_raise\_351.cmm}}
      \onslide*<2>{\listcmm[highlightlines=18]{../src/backtrace/camlBacktrace\_\_probably\_raise_351.cmm}{CMM of probably\_raise}}
      \onslide*<3>{\mintobjdump[firstline=5,lastline=35,highlightlines=31]{../src/backtrace/camlBacktrace\_\_probably\_raise_351.objdump}}
    \end{column}
  \end{columns}
  \only<1>{\footnotetext[1]{https://blog.janestreet.com/pattern-matching-and-exception-handling-unite/}}
\end{frame}

\newcommand\listCamlReraiseExn[1][]{
  \listasm[firstline=727,lastline=734,#1]{../ocaml/runtime/amd64.S}{runtime/amd64.S:727}}

\begin{frame}{Backtraces}{Introducing \funcname{caml\_reraise\_exn}}
  \begin{columns}[c]
    \begin{column}{0.5\textwidth}
      \minipage[c][0.66\textheight][s]{\columnwidth}
      \begin{itemize}
        \item<1-> The exception have to be reraised to the next handler
        \item<2-> First, checks if the backtrace should be collected with \funcarg{domain\_state->backtrace\_active}
        \only<3>{
        \begin{itemize}
            \item If not, behaves like a \funcname{raise\_notrace} with \funcname{RESTORE\_EXN\_HANDLER\_OCAML}
        \end{itemize}
        }
        \item<4-> Jumps into \funcname{caml\_raise\_exn}
        \item<5-> Calls \funcname{caml\_stash\_backtrace} again, to scan the next part of the stack: from the trap just pop-ed to the next trap
      \end{itemize}
      \vfill
      \mintocaml[firstline=15,lastline=21,highlightlines={18-21}]{../src/backtrace/backtrace.ml}
      \endminipage
    \end{column}
    \begin{column}{0.5\textwidth}
      \raggedleft
      \onslide*<1>{\listCamlReraiseExn{}}
      \onslide*<2>{\listCamlReraiseExn[highlightlines=729]{}}
      \onslide*<3>{
        \mintc[firstline=190,lastline=192,highlightlines={191-192}]{../ocaml/runtime/amd64.S}
        \mintc[firstline=234,lastline=237,highlightlines={235-237}]{../ocaml/runtime/amd64.S}
        \medskip
        \listCamlReraiseExn[highlightlines=731]{}
      }
      \onslide*<4-5>{
        % TODO refactor with \listCamlReraiseExn
        \mintasm[firstline=727,lastline=734,highlightlines=730]{../ocaml/runtime/amd64.S}
        \medskip
      }
      \onslide*<4>{\listCamlRaiseExn[highlightlines=711]{}}
      \onslide*<5>{\listCamlRaiseExn[highlightlines=720]{}}
    \end{column}
  \end{columns}
\end{frame}

\begin{frame}{Backtraces}{Backtrace collection continues}
  \begin{columns}[c]
    \begin{column}{0.55\textwidth}
      \minipage[c][0.75\textheight][s]{\columnwidth}
      \begin{itemize}
        \item<1-> \funcname{caml\_stash\_backtrace} scans the older part of the stack, up to the next trap
        \item<6-> Controls jumps to the nested exception handler in \funcname{maybe\_raise}, which matches only on \typename{ExnB}
        \item<7-> \funcname{caml\_reraise\_exn} is called again, backtrace collection resumes with \funcname{caml\_stash\_backtrace}
      \end{itemize}
      \medskip
      \begin{itemize}
        \item<2-> Constructed backtrace:
        \begin{itemize}
          \item <2-> \funcname{definitely\_raise}
          \item <2-> \funcname{probably\_raise}
          \item <3-> \funcname{probably\_raise} (re-raise)
          \item <4-> \funcname{maybe\_raise}
          \item <5-> \funcname{main}
          \item <8-> \funcname{main} (re-raise)
        \end{itemize}
      \end{itemize}
      \vfill
      \onslide*<1-5>{\mintocaml[firstline=15,lastline=21]{../src/backtrace/backtrace.ml}}
      \onslide*<6>{\mintocaml[firstline=28,lastline=34,highlightlines=31]{../src/backtrace/backtrace.ml}}
      \onslide*<7->{\mintocaml[firstline=28,lastline=34,highlightlines=33]{../src/backtrace/backtrace.ml}}
      \endminipage
    \end{column}
    \begin{column}{0.45\textwidth}
      \raggedleft
      \onslide*<1>{\providecommand\step{6}\input{diagrams/backtrace/backtrace.tex}}%
      \onslide*<2>{\providecommand\step{6}\input{diagrams/backtrace/backtrace.tex}}%
      \onslide*<3>{\providecommand\step{7}\input{diagrams/backtrace/backtrace.tex}}%
      \onslide*<4>{\providecommand\step{8}\input{diagrams/backtrace/backtrace.tex}}%
      \onslide*<5>{\providecommand\step{9}\input{diagrams/backtrace/backtrace.tex}}%
      \onslide*<6>{\providecommand\step{10}\input{diagrams/backtrace/backtrace.tex}}%
      \onslide*<7>{\providecommand\step{11}\input{diagrams/backtrace/backtrace.tex}}%
      \onslide*<8>{\providecommand\step{12}\input{diagrams/backtrace/backtrace.tex}}%
      \onslide*<9>{\providecommand\step{13}\input{diagrams/backtrace/backtrace.tex}}%
    \end{column}
  \end{columns}
\end{frame}

\begin{frame}{Backtraces}{Printing the backtrace}
  \begin{columns}[c]
    \begin{column}{0.45\textwidth}
      \minipage[c][0.66\textheight][s]{\columnwidth}
      \begin{itemize}
        \item<1-> The exception backtrace can now be printed to \funcarg{stdout} with \funcname{Printexc.print_backtrace}
        \item<2-> First the exception backtrace is retrieved into an OCaml array with \funcname{get_raw_backtrace }
        \item<3-> Which is implemented as the \funcname{caml_get_exception_raw_backtrace} C function in the runtime
        \item<4-> And finally printed with \funcname{print_exception_backtrace}
      \end{itemize}
      \vfill
      \mintocaml[firstline=28,lastline=34,highlightlines=33]{../src/backtrace/backtrace.ml}
      \endminipage
    \end{column}
    \begin{column}{0.55\textwidth}
      \onslide*<1,2>{
        \mintocaml[firstline=100,lastline=101]{../ocaml/stdlib/printexc.ml}
      }
      \only<1>{
        \listocaml[firstline=170,lastline=172]{../ocaml/stdlib/printexc.ml}{stdlib/printexc.ml:170}
      }
      \only<2>{
        \listocaml[firstline=170,lastline=172,highlightlines=172]{../ocaml/stdlib/printexc.ml}{stdlib/printexc.ml:170}
      }
      \only<3>{
        \mintc[firstline=154,lastline=156]{../ocaml/runtime/backtrace.c}
        \listc[firstline=171,lastline=192,highlightlines={184-187}]{../ocaml/runtime/backtrace.c}{runtime/backtrace.c:154}
      }
      \only<4>{
        \listocaml[firstline=155,lastline=165]{../ocaml/stdlib/printexc.ml}{stdlib/printexc.ml:55}
      }
    \end{column}
  \end{columns}
\end{frame}


\subsection{The multiple kinds of raise}
\frameSubsection{}{}

\begin{frame}{Backtraces}{When backtraces are not needed}
  \begin{columns}[c]
    \begin{column}{0.45\textwidth}
      \minipage[c][0.66\textheight][s]{\columnwidth}
      \begin{itemize}
        \item<1-> What if the backtrace for an exception is never needed?
        \item<2-> It's possible to raise the exception explicitly with \funcname{raise\_notrace} to make raising as cheap as possible
        \item<3-> An example of use in the \funcname{dynlink} library of the OCaml compiler
      \end{itemize}
      \vfill
      \onslide<3->{
      \listocaml[firstline=205,lastline=209,highlightlines=207]{../ocaml/otherlibs/dynlink/dynlink\_compilerlibs/misc.ml}{otherlibs/dynlink/dynlink\_compilerlibs/misc.ml:205}
      }
      \endminipage
    \end{column}
    \begin{column}{0.55\textwidth}
    \only<4>{
      \mintcmm[highlightlines=3]{../src/notrace/camlNotrace\_\_fun_347.cmm}
      \listcmm{../src/notrace/camlNotrace\_\_all\_somes\_327.cmm}{all\_somes}
    }
    \onslide*<5->{
      \listobjdump[highlightlines={6-9}]{../src/notrace/camlNotrace\_\_fun_347.objdump}{anonymous function from \funcname{all\_somes}}
    }
    \end{column}
  \end{columns}
\end{frame}

\begin{frame}{Backtraces}{Summary table of the multiple kinds of raise}
  \begin{itemize}
    \item When debugging information is enabled and if the backtrace of an exception isn't needed, it's possible to raise with \funcname{raise\_notrace} for performance
    \item When debugging information is \emph{not} enabled, the compiler optimises \funcname{raise} calls to inline assembly (same effect as using \funcname{raise\_notrace})
  \end{itemize}
  \bigskip
  \centering
  \begin{tabular}{| l | c | c | c | c |}
    \hline
    \parbox{10em}{Debug information \\ (\funcarg{-g} flag)}  & \multicolumn{2}{|c|}{Disabled}                                     & \multicolumn{2}{|c|}{Enabled} \\
    \hline
    Raise kind                                               & \funcname{raise} & \funcname{raise\_notrace}                       & \funcname{raise} & \funcname{raise\_notrace} \\
    \hline
    Compilation                                              & \parbox{8em}{inline assembly\\ (optimization)} & inline assembly   & \funcname{caml\_raise\_exn} & inline assembly \\
    \hline
  \end{tabular}
\end{frame}


%
% Takeaway #5
%
\subsection*{Takeaway \#5}
\frameSubsectionTakeaway{}
\againframe<5>{takeaway}
}%
      \onslide*<2>{\providecommand\step{1}\input{diagrams/backtrace/scanstack.tex}}%
      \onslide*<3>{\providecommand\step{2}\input{diagrams/backtrace/scanstack.tex}}%
      \onslide*<4>{\providecommand\step{3}\input{diagrams/backtrace/scanstack.tex}}%
      \onslide*<5>{\providecommand\step{4}\input{diagrams/backtrace/scanstack.tex}}%
      \onslide*<6>{\providecommand\step{5}\input{diagrams/backtrace/scanstack.tex}}%
    \end{column}
  \end{columns}
\end{frame}

% Duplicate of previous-previous slide
\begin{frame}{Backtraces}{Continuing on \funcname{caml\_stash\_backtrace}}
  \begin{columns}[c]
    \begin{column}{0.5\textwidth}
      \minipage[c][0.75\textheight][s]{\columnwidth}
      \begin{itemize}
          \color{gray}
        \item A C function provided by the runtime (specific to native code)
        \item It scans the stack, between the stack pointer at the raising point up to the next trap, in order to collect the names of the detected function frames
        \item It uses the same technique and information as the Garbage Collector (GC) uses to scan the values live on the stack
      \end{itemize}
      \begin{itemize}
        \item For each function frame detected, store a pointer to the \typename{frame\_descr} in the \localname{domain\_state->backtrace\_buffer} array, effectively constructing the backtrace
      \end{itemize}
      \onslide<2->{
        \begin{itemize}
            \smallskip
          \item Constructed backtrace:
            \funcname{definitely\_raise},
            \onslide<3->{\funcname{probably\_raise}}
        \end{itemize}
      }
      \vfill
      \mintocaml[firstline=9,lastline=13,highlightlines=12]{../src/backtrace/backtrace.ml}
      \endminipage
    \end{column}
    \begin{column}{0.5\textwidth}
      \raggedleft
      \onslide*<1>{\listStashBacktrace[highlightlines={114-115}]{}}
      \onslide*<2>{\providecommand\step{3}%
% Backtraces
%
\subsection{One advantage of raising with debugging information}
\frameSubsection{}{}

\begin{frame}{Backtraces}{Debugging information \& source code}
  \begin{columns}[c]
    \begin{column}{0.5\textwidth}
      \begin{itemize}
        \item Raising an exception is fun and games, but how do we know where the exception originated? $\rightarrow$ With the backtrace associated with the exception!
        \item In order to get a backtrace from the point where an exception was raised, it's necessary to compile with debugging information enabled (\funcarg{-g} flag for the compiler)
        \item Same example as in the `Nested handlers' section
      \end{itemize}
    \end{column}
    \begin{column}{0.5\textwidth}
      \listocaml[firstline=9,lastline=34]{../src/backtrace/backtrace.ml}{backtrace/backtrace.ml}
    \end{column}
  \end{columns}
\end{frame}

\begin{frame}{Backtraces}{CMM and disassembly of \funcname{definitely\_raise}}
  \begin{columns}[c]
    \begin{column}{0.5\textwidth}
      \minipage[c][0.66\textheight][s]{\columnwidth}
      \begin{itemize}
        \item<1-> An effect of compiling with debugging information (\funcarg{-g}): the CMM no longer uses \funcname{raise\_notrace} to raise the exception but \funcname{raise} instead
        \item<2-> Disassembly shows that CMM \funcname{raise} is actually a call to \funcname{caml\_raise\_exn}, a function in assembler provided by the runtime
      \end{itemize}
      \vfill
      \mintocaml[firstline=9,lastline=13,highlightlines=12]{../src/backtrace/backtrace.ml}
      \endminipage
    \end{column}
    \begin{column}{0.5\textwidth}
      \onslide*<1>{
        \mintcmm[highlightlines=5]{../src/backtrace/camlBacktrace\_\_definitely\_raise\_347.cmm}
      }
      \onslide*<2>{
        \mintobjdump[firstline=2,highlightlines=14]{../src/backtrace/camlBacktrace\_\_definitely\_raise\_347.objdump}
      }
    \end{column}
  \end{columns}
\end{frame}

\begin{frame}{Backtraces}{Introducing \funcname{caml\_raise\_exn}}
  \begin{columns}[c]
    \begin{column}{0.5\textwidth}
      \minipage[c][0.66\textheight][s]{\columnwidth}
      \onslide*<1>{
        \begin{itemize}
          \item Raising the exception is performed using \funcname{caml\_raise\_exn} which is
            \begin{itemize}
              \item An assembly function provided by the runtime
              \item Architecture specific
            \end{itemize}
          \item Let's see what it does differently from \funcname{raise\_notrace}\dots
          \item We are about to raise \typename{ExnC} with \funcname{caml\_raise\_exn} from \funcname{definitely\_raise}
        \end{itemize}
      }
      \onslide*<2>{
        \mintobjdump[firstline=2,highlightlines=14]{../src/backtrace/camlBacktrace\_\_definitely\_raise\_347.objdump}
      }
      \vfill
      \mintocaml[firstline=9,lastline=13,highlightlines=12]{../src/backtrace/backtrace.ml}
      \endminipage
    \end{column}
    \begin{column}{0.5\textwidth}
      \centering
      \providecommand\step{1}\input{diagrams/backtrace/backtrace.tex}
    \end{column}
  \end{columns}
\end{frame}

\begin{frame}{Backtraces}{But before that, we need more fields from \typename{caml\_domain\_state}}
  \begin{columns}
    \begin{column}{0.5\textwidth}
      \begin{itemize}
        \item Remember \typename{caml\_domain\_state} the per-domain runtime state?
        \item Some other members are of interest to us now\dots
        \item \localname{backtrace\_active} is used to know if the runtime should collect backtraces
        \item \localname{backtrace\_active} can be set by
          \begin{itemize}
            \item The \funcarg{b} flag in the \funcname{OCAMLRUNPARAM} environment variable
            \item \mintinline{ocaml}{Printexc.record_backtrace : bool -> unit} \footnotemark
          \end{itemize}
      \end{itemize}
    \end{column}
    \begin{column}{0.5\textwidth}
      \begin{listing}
        \mintc[highlightlines={9-11}]{src/ocaml/domain\_state\_backtrace.h}
        \caption{runtime/caml/domain\_state.h}
      \end{listing}
    \end{column}
  \end{columns}
  \footnotetext[1]{https://v2.ocaml.org/api/Printexc.html}
\end{frame}

\newcommand\listCamlRaiseExn[1][]{
  \listasm[firstline=702,lastline=725,#1]{../ocaml/runtime/amd64.S}{runtime/amd64.S:702}}

\begin{frame}{Backtraces}{Walk through \funcname{caml\_raise\_exn}}
  \begin{columns}[T]
    \begin{column}{0.5\textwidth}
      \begin{itemize}
        \item<1-> First checks whether exceptions backtrace should be recorded
        \item<1-> By checking the \funcarg{domain\_state->backtrace\_active} value
        \item<2-> If not, acts as \funcname{raise\_notrace} with \funcname{RESTORE\_EXN\_HANDLER\_OCAML}
          \begin{itemize}
            \item Set \regname{RSP} to \localname{domain\_state->exn\_handler} value
            \item Restore \localname{domain\_state->exn\_handler} from trap
            \item Jump with \funcname{ret} (same effect as using \funcname{pop} and \funcname{jmp})
          \end{itemize}
        \item<3-> Otherwise, switches to the C stack to be able to execute C code
        \item<4-> Calls \funcname{caml\_stash\_backtrace}, a C function provided by the runtime\ignorespaces
          \onslide<6->{, with
            \begin{itemize}
              \item \funcarg{exn}: exception value
              \item \funcarg{pc}: code address of the raising point
              \item \funcarg{sp}: stack address of the raising function
              \item \funcarg{trapsp}: stack address of the next handler
            \end{itemize}
          }
      \end{itemize}
      \bigskip
      \mintocaml[firstline=9,lastline=13,highlightlines=12]{../src/backtrace/backtrace.ml}
    \end{column}
    \begin{column}{0.5\textwidth}
      \raggedleft
      \onslide*<1,3-4>{
        \mintc[firstline=190,lastline=192]{../ocaml/runtime/amd64.S}
        \mintc[firstline=234,lastline=237]{../ocaml/runtime/amd64.S}
      }
      \onslide*<2>{
        \mintc[firstline=190,lastline=192,highlightlines={191-192}]{../ocaml/runtime/amd64.S}
        \mintc[firstline=234,lastline=237,highlightlines={235-237}]{../ocaml/runtime/amd64.S}
      }
      \onslide*<1>{\listCamlRaiseExn[highlightlines=705]{}}%
      \onslide*<2>{\listCamlRaiseExn[highlightlines={707-708}]{}}%
      \onslide*<3>{\listCamlRaiseExn[highlightlines={712-714}]{}}%
      \onslide*<4>{\listCamlRaiseExn[highlightlines={715-720}]{}}%
      \onslide*<5>{\providecommand\step{1}\input{diagrams/backtrace/backtrace.tex}}%
      \onslide*<6>{\providecommand\step{2}\input{diagrams/backtrace/backtrace.tex}}%
    \end{column}
  \end{columns}
\end{frame}

\newcommand\listStashBacktrace[1][]{
  \listc[firstline=91,lastline=120,#1]{../ocaml/runtime/backtrace\_nat.c}{runtime/backtrace\_nat.c:91}}

\begin{frame}{Backtraces}{Introducing \funcname{caml\_stash\_backtrace}}
  \begin{columns}[c]
    \begin{column}{0.5\textwidth}
      \minipage[c][0.66\textheight][s]{\columnwidth}
      \begin{itemize}
        \item<1-> A C function provided by the runtime (specific to native code)
        \item<2-> It scans the stack, between the stack pointer at the raising point up to the next trap, in order to collect the names of the detected function frames
          \onslide*<2>{
            \begin{itemize}
              \item And more information, like line number, column number...
            \end{itemize}
          }
        \item<3-> It uses the same technique and information as the Garbage Collector (GC) uses to scan the values live on the stack
          \onslide*<4>{
            \begin{itemize}
              \item \typename{frame\_descr} are at the core of stack unwinding
            \end{itemize}
          }
      \end{itemize}
      \vfill
      \mintocaml[firstline=9,lastline=13,highlightlines=12]{../src/backtrace/backtrace.ml}
      \endminipage
    \end{column}
    \begin{column}{0.5\textwidth}
      \onslide*<1>{\listStashBacktrace[highlightlines=91]{}}
      \onslide*<2>{\listStashBacktrace[highlightlines={91,117-118}]{}}
      \onslide*<3->{\listStashBacktrace[highlightlines={94,106,109-110}]{}}
    \end{column}
  \end{columns}
\end{frame}

\newcommand\listFrameDescr[1][]{
  \listocaml[firstline=111,lastline=124,#1]{../ocaml/asmcomp/emitaux.ml}{asmcomp/emitaux.ml:111}}
\newcommand\listDebugInfo[1][]{
  \listocaml[firstline=38,lastline=49,#1]{../ocaml/lambda/debuginfo.mli}{lambda/debuginfo.mli:38}}

\begin{frame}{Backtraces}{\typename{frame\_descr} and \typename{frame\_debuginfo} in a nutshell}
  \begin{columns}[c]
    \begin{column}{0.5\textwidth}
      \begin{itemize}
        \item<1-> For every return address in an OCaml program, there is a associated \typename{frame\_descr}
        \item<2-> A \typename{frame\_descr} specifies
        \begin{itemize}
          \item<2-> The size of the function stack frame
          \item<3-> Where live values are on the stack (so that the GC can mark them as live and prevent from being swept)
        \end{itemize}
        \item<4-> When compiled with debug information, \typename{frame\_descr} will be linked to a \typename{frame\_debuginfo}
        \begin{itemize}
          \item<5-> Contains information about the position in the source code (file name, line and column number)
        \end{itemize}
      \end{itemize}
    \end{column}
    \begin{column}{0.5\textwidth}
        \onslide*<1>{\listFrameDescr[highlightlines={118-122}]{}}
        \onslide*<2>{\listFrameDescr[highlightlines={120}]{}}
        \onslide*<3>{\listFrameDescr[highlightlines={121}]{}}
        \onslide*<4->{\listFrameDescr[highlightlines={122}]{}}
        \medskip
        \onslide*<1-3>{\listDebugInfo{}}
        \onslide*<4>{\listDebugInfo[highlightlines={38,49}]{}}
        \onslide*<5>{\listDebugInfo[highlightlines={39-46}]{}}
    \end{column}
  \end{columns}
\end{frame}

\begin{frame}{Backtraces}{Scanning the stack with \typename{frame\_descr}}
  \begin{columns}[c]
    \begin{column}{0.5\textwidth}
      \begin{itemize}
        \item Given the return address of the code that raises the exception by calling \funcname{caml\_raise\_exn} (\funcarg{pc} argument of \funcname{caml\_stash\_backtrace})
        \item And the position of the stack pointer (\funcarg{sp} argument of \funcname{caml\_stash\_backtrace})
        \item The frame size given by \typename{frame\_descr}.\funcarg{frame\_size}, allows to find the end of the previous frame
        \item And the return address of the caller of this function
        \bigskip
        \onslide<3->{
        \item Note that traps (here the reddish \funcname{ExnA}, \funcname{ExnB} and \funcname{ExnC} blocks) belong to the frame that installed them, and so the frame size changes when traps are removed
        }
      \end{itemize}
    \end{column}
    \begin{column}{0.5\textwidth}
      \raggedleft
      \onslide*<1>{\providecommand\step{2}\input{diagrams/backtrace/backtrace.tex}}%
      \onslide*<2>{\providecommand\step{1}\input{diagrams/backtrace/scanstack.tex}}%
      \onslide*<3>{\providecommand\step{2}\input{diagrams/backtrace/scanstack.tex}}%
      \onslide*<4>{\providecommand\step{3}\input{diagrams/backtrace/scanstack.tex}}%
      \onslide*<5>{\providecommand\step{4}\input{diagrams/backtrace/scanstack.tex}}%
      \onslide*<6>{\providecommand\step{5}\input{diagrams/backtrace/scanstack.tex}}%
    \end{column}
  \end{columns}
\end{frame}

% Duplicate of previous-previous slide
\begin{frame}{Backtraces}{Continuing on \funcname{caml\_stash\_backtrace}}
  \begin{columns}[c]
    \begin{column}{0.5\textwidth}
      \minipage[c][0.75\textheight][s]{\columnwidth}
      \begin{itemize}
          \color{gray}
        \item A C function provided by the runtime (specific to native code)
        \item It scans the stack, between the stack pointer at the raising point up to the next trap, in order to collect the names of the detected function frames
        \item It uses the same technique and information as the Garbage Collector (GC) uses to scan the values live on the stack
      \end{itemize}
      \begin{itemize}
        \item For each function frame detected, store a pointer to the \typename{frame\_descr} in the \localname{domain\_state->backtrace\_buffer} array, effectively constructing the backtrace
      \end{itemize}
      \onslide<2->{
        \begin{itemize}
            \smallskip
          \item Constructed backtrace:
            \funcname{definitely\_raise},
            \onslide<3->{\funcname{probably\_raise}}
        \end{itemize}
      }
      \vfill
      \mintocaml[firstline=9,lastline=13,highlightlines=12]{../src/backtrace/backtrace.ml}
      \endminipage
    \end{column}
    \begin{column}{0.5\textwidth}
      \raggedleft
      \onslide*<1>{\listStashBacktrace[highlightlines={114-115}]{}}
      \onslide*<2>{\providecommand\step{3}\input{diagrams/backtrace/backtrace.tex}}%
      \onslide*<3>{\providecommand\step{4}\input{diagrams/backtrace/backtrace.tex}}%
    \end{column}
  \end{columns}
\end{frame}

\begin{frame}{Backtraces}{Back to \funcname{caml\_raise\_exn}}
  \begin{columns}[c]
    \begin{column}{0.5\textwidth}
      \minipage[c][0.66\textheight][s]{\columnwidth}
      \begin{itemize}\color{gray}
        \item Checks wheteher exceptions backtrace should be recorded, by checking the \funcarg{domain\_state->backtrace\_active} value
        \item Switches to the C stack to be able to execute C code
        \item Calls \funcname{caml\_stash\_backtrace}, to collect the backtrace up to the next trap
      \end{itemize}
      \onslide*<2->{
        \begin{itemize}
          \item<2-> Executes the exception handler in \funcname{probably\_raise} with \funcname{RESTORE\_EXN\_HANDLER\_OCAML}
            \begin{itemize}
              \item Set \regname{RSP} to \localname{domain\_state->exn\_handler} value
              \item Restore \localname{domain\_state->exn\_handler} from trap
              \item Jump to the handler code with \funcname{ret}
            \end{itemize}
        \end{itemize}
      }
      \vfill
      \onslide*<1>{\mintocaml[firstline=9,lastline=13,highlightlines=12]{../src/backtrace/backtrace.ml}}
      \onslide*<2->{\mintocaml[firstline=15,lastline=21,highlightlines={19-21}]{../src/backtrace/backtrace.ml}}
      \endminipage
    \end{column}
    \begin{column}{0.5\textwidth}
      \centering
      \onslide*<1>{
        \mintc[firstline=190,lastline=192]{../ocaml/runtime/amd64.S}
        \mintc[firstline=234,lastline=237]{../ocaml/runtime/amd64.S}
      }
      \onslide*<2>{
        \mintc[firstline=190,lastline=192,highlightlines={191-192}]{../ocaml/runtime/amd64.S}
        \mintc[firstline=234,lastline=237,highlightlines={235-237}]{../ocaml/runtime/amd64.S}
      }
      \onslide*<1>{\listCamlRaiseExn[highlightlines=720]{}}
      \onslide*<2>{\listCamlRaiseExn[highlightlines=722]{}}
      \onslide*<3->{\providecommand\step{5}\input{diagrams/backtrace/backtrace.tex}}
    \end{column}
  \end{columns}
\end{frame}

\begin{frame}{Backtraces}{\funcname{caml\_probably\_raise}'s handler doesn't catch \typename{ExnC}}
  \begin{columns}[c]
    \begin{column}{0.45\textwidth}
      \minipage[c][0.75\textheight][s]{\columnwidth}
      \begin{itemize}
        \item<1-> \funcname{match \dots with}\footnotemark pattern matching can also match exceptions
        \item<1-> The CMM of \funcname{probably\_raise} shows that this \funcname{match \dots with} is implemented with a pattern matching and a \funcname{try \dots with}
        \item<1-> But this one only matches on \typename{ExnA} exception
        \item<2-> Since the pattern matching fails to catch the exception, \funcname{reraise} is called with our current \typename{ExnC} exception
        \item<3-> Disassembly shows that \funcname{reraise} is actually a call to \funcname{caml\_reraise\_exn}, which is also an assembly function provided by the runtime
      \end{itemize}
      \vfill
      \mintocaml[firstline=15,lastline=21,highlightlines={18-21}]{../src/backtrace/backtrace.ml}
      \endminipage
    \end{column}
    \begin{column}{0.55\textwidth}
      \centering
      \onslide*<1>{\mintcmm[highlightlines={10-18}]{../src/backtrace/camlBacktrace\_\_probably\_raise\_351.cmm}}
      \onslide*<2>{\listcmm[highlightlines=18]{../src/backtrace/camlBacktrace\_\_probably\_raise_351.cmm}{CMM of probably\_raise}}
      \onslide*<3>{\mintobjdump[firstline=5,lastline=35,highlightlines=31]{../src/backtrace/camlBacktrace\_\_probably\_raise_351.objdump}}
    \end{column}
  \end{columns}
  \only<1>{\footnotetext[1]{https://blog.janestreet.com/pattern-matching-and-exception-handling-unite/}}
\end{frame}

\newcommand\listCamlReraiseExn[1][]{
  \listasm[firstline=727,lastline=734,#1]{../ocaml/runtime/amd64.S}{runtime/amd64.S:727}}

\begin{frame}{Backtraces}{Introducing \funcname{caml\_reraise\_exn}}
  \begin{columns}[c]
    \begin{column}{0.5\textwidth}
      \minipage[c][0.66\textheight][s]{\columnwidth}
      \begin{itemize}
        \item<1-> The exception have to be reraised to the next handler
        \item<2-> First, checks if the backtrace should be collected with \funcarg{domain\_state->backtrace\_active}
        \only<3>{
        \begin{itemize}
            \item If not, behaves like a \funcname{raise\_notrace} with \funcname{RESTORE\_EXN\_HANDLER\_OCAML}
        \end{itemize}
        }
        \item<4-> Jumps into \funcname{caml\_raise\_exn}
        \item<5-> Calls \funcname{caml\_stash\_backtrace} again, to scan the next part of the stack: from the trap just pop-ed to the next trap
      \end{itemize}
      \vfill
      \mintocaml[firstline=15,lastline=21,highlightlines={18-21}]{../src/backtrace/backtrace.ml}
      \endminipage
    \end{column}
    \begin{column}{0.5\textwidth}
      \raggedleft
      \onslide*<1>{\listCamlReraiseExn{}}
      \onslide*<2>{\listCamlReraiseExn[highlightlines=729]{}}
      \onslide*<3>{
        \mintc[firstline=190,lastline=192,highlightlines={191-192}]{../ocaml/runtime/amd64.S}
        \mintc[firstline=234,lastline=237,highlightlines={235-237}]{../ocaml/runtime/amd64.S}
        \medskip
        \listCamlReraiseExn[highlightlines=731]{}
      }
      \onslide*<4-5>{
        % TODO refactor with \listCamlReraiseExn
        \mintasm[firstline=727,lastline=734,highlightlines=730]{../ocaml/runtime/amd64.S}
        \medskip
      }
      \onslide*<4>{\listCamlRaiseExn[highlightlines=711]{}}
      \onslide*<5>{\listCamlRaiseExn[highlightlines=720]{}}
    \end{column}
  \end{columns}
\end{frame}

\begin{frame}{Backtraces}{Backtrace collection continues}
  \begin{columns}[c]
    \begin{column}{0.55\textwidth}
      \minipage[c][0.75\textheight][s]{\columnwidth}
      \begin{itemize}
        \item<1-> \funcname{caml\_stash\_backtrace} scans the older part of the stack, up to the next trap
        \item<6-> Controls jumps to the nested exception handler in \funcname{maybe\_raise}, which matches only on \typename{ExnB}
        \item<7-> \funcname{caml\_reraise\_exn} is called again, backtrace collection resumes with \funcname{caml\_stash\_backtrace}
      \end{itemize}
      \medskip
      \begin{itemize}
        \item<2-> Constructed backtrace:
        \begin{itemize}
          \item <2-> \funcname{definitely\_raise}
          \item <2-> \funcname{probably\_raise}
          \item <3-> \funcname{probably\_raise} (re-raise)
          \item <4-> \funcname{maybe\_raise}
          \item <5-> \funcname{main}
          \item <8-> \funcname{main} (re-raise)
        \end{itemize}
      \end{itemize}
      \vfill
      \onslide*<1-5>{\mintocaml[firstline=15,lastline=21]{../src/backtrace/backtrace.ml}}
      \onslide*<6>{\mintocaml[firstline=28,lastline=34,highlightlines=31]{../src/backtrace/backtrace.ml}}
      \onslide*<7->{\mintocaml[firstline=28,lastline=34,highlightlines=33]{../src/backtrace/backtrace.ml}}
      \endminipage
    \end{column}
    \begin{column}{0.45\textwidth}
      \raggedleft
      \onslide*<1>{\providecommand\step{6}\input{diagrams/backtrace/backtrace.tex}}%
      \onslide*<2>{\providecommand\step{6}\input{diagrams/backtrace/backtrace.tex}}%
      \onslide*<3>{\providecommand\step{7}\input{diagrams/backtrace/backtrace.tex}}%
      \onslide*<4>{\providecommand\step{8}\input{diagrams/backtrace/backtrace.tex}}%
      \onslide*<5>{\providecommand\step{9}\input{diagrams/backtrace/backtrace.tex}}%
      \onslide*<6>{\providecommand\step{10}\input{diagrams/backtrace/backtrace.tex}}%
      \onslide*<7>{\providecommand\step{11}\input{diagrams/backtrace/backtrace.tex}}%
      \onslide*<8>{\providecommand\step{12}\input{diagrams/backtrace/backtrace.tex}}%
      \onslide*<9>{\providecommand\step{13}\input{diagrams/backtrace/backtrace.tex}}%
    \end{column}
  \end{columns}
\end{frame}

\begin{frame}{Backtraces}{Printing the backtrace}
  \begin{columns}[c]
    \begin{column}{0.45\textwidth}
      \minipage[c][0.66\textheight][s]{\columnwidth}
      \begin{itemize}
        \item<1-> The exception backtrace can now be printed to \funcarg{stdout} with \funcname{Printexc.print_backtrace}
        \item<2-> First the exception backtrace is retrieved into an OCaml array with \funcname{get_raw_backtrace }
        \item<3-> Which is implemented as the \funcname{caml_get_exception_raw_backtrace} C function in the runtime
        \item<4-> And finally printed with \funcname{print_exception_backtrace}
      \end{itemize}
      \vfill
      \mintocaml[firstline=28,lastline=34,highlightlines=33]{../src/backtrace/backtrace.ml}
      \endminipage
    \end{column}
    \begin{column}{0.55\textwidth}
      \onslide*<1,2>{
        \mintocaml[firstline=100,lastline=101]{../ocaml/stdlib/printexc.ml}
      }
      \only<1>{
        \listocaml[firstline=170,lastline=172]{../ocaml/stdlib/printexc.ml}{stdlib/printexc.ml:170}
      }
      \only<2>{
        \listocaml[firstline=170,lastline=172,highlightlines=172]{../ocaml/stdlib/printexc.ml}{stdlib/printexc.ml:170}
      }
      \only<3>{
        \mintc[firstline=154,lastline=156]{../ocaml/runtime/backtrace.c}
        \listc[firstline=171,lastline=192,highlightlines={184-187}]{../ocaml/runtime/backtrace.c}{runtime/backtrace.c:154}
      }
      \only<4>{
        \listocaml[firstline=155,lastline=165]{../ocaml/stdlib/printexc.ml}{stdlib/printexc.ml:55}
      }
    \end{column}
  \end{columns}
\end{frame}


\subsection{The multiple kinds of raise}
\frameSubsection{}{}

\begin{frame}{Backtraces}{When backtraces are not needed}
  \begin{columns}[c]
    \begin{column}{0.45\textwidth}
      \minipage[c][0.66\textheight][s]{\columnwidth}
      \begin{itemize}
        \item<1-> What if the backtrace for an exception is never needed?
        \item<2-> It's possible to raise the exception explicitly with \funcname{raise\_notrace} to make raising as cheap as possible
        \item<3-> An example of use in the \funcname{dynlink} library of the OCaml compiler
      \end{itemize}
      \vfill
      \onslide<3->{
      \listocaml[firstline=205,lastline=209,highlightlines=207]{../ocaml/otherlibs/dynlink/dynlink\_compilerlibs/misc.ml}{otherlibs/dynlink/dynlink\_compilerlibs/misc.ml:205}
      }
      \endminipage
    \end{column}
    \begin{column}{0.55\textwidth}
    \only<4>{
      \mintcmm[highlightlines=3]{../src/notrace/camlNotrace\_\_fun_347.cmm}
      \listcmm{../src/notrace/camlNotrace\_\_all\_somes\_327.cmm}{all\_somes}
    }
    \onslide*<5->{
      \listobjdump[highlightlines={6-9}]{../src/notrace/camlNotrace\_\_fun_347.objdump}{anonymous function from \funcname{all\_somes}}
    }
    \end{column}
  \end{columns}
\end{frame}

\begin{frame}{Backtraces}{Summary table of the multiple kinds of raise}
  \begin{itemize}
    \item When debugging information is enabled and if the backtrace of an exception isn't needed, it's possible to raise with \funcname{raise\_notrace} for performance
    \item When debugging information is \emph{not} enabled, the compiler optimises \funcname{raise} calls to inline assembly (same effect as using \funcname{raise\_notrace})
  \end{itemize}
  \bigskip
  \centering
  \begin{tabular}{| l | c | c | c | c |}
    \hline
    \parbox{10em}{Debug information \\ (\funcarg{-g} flag)}  & \multicolumn{2}{|c|}{Disabled}                                     & \multicolumn{2}{|c|}{Enabled} \\
    \hline
    Raise kind                                               & \funcname{raise} & \funcname{raise\_notrace}                       & \funcname{raise} & \funcname{raise\_notrace} \\
    \hline
    Compilation                                              & \parbox{8em}{inline assembly\\ (optimization)} & inline assembly   & \funcname{caml\_raise\_exn} & inline assembly \\
    \hline
  \end{tabular}
\end{frame}


%
% Takeaway #5
%
\subsection*{Takeaway \#5}
\frameSubsectionTakeaway{}
\againframe<5>{takeaway}
}%
      \onslide*<3>{\providecommand\step{4}%
% Backtraces
%
\subsection{One advantage of raising with debugging information}
\frameSubsection{}{}

\begin{frame}{Backtraces}{Debugging information \& source code}
  \begin{columns}[c]
    \begin{column}{0.5\textwidth}
      \begin{itemize}
        \item Raising an exception is fun and games, but how do we know where the exception originated? $\rightarrow$ With the backtrace associated with the exception!
        \item In order to get a backtrace from the point where an exception was raised, it's necessary to compile with debugging information enabled (\funcarg{-g} flag for the compiler)
        \item Same example as in the `Nested handlers' section
      \end{itemize}
    \end{column}
    \begin{column}{0.5\textwidth}
      \listocaml[firstline=9,lastline=34]{../src/backtrace/backtrace.ml}{backtrace/backtrace.ml}
    \end{column}
  \end{columns}
\end{frame}

\begin{frame}{Backtraces}{CMM and disassembly of \funcname{definitely\_raise}}
  \begin{columns}[c]
    \begin{column}{0.5\textwidth}
      \minipage[c][0.66\textheight][s]{\columnwidth}
      \begin{itemize}
        \item<1-> An effect of compiling with debugging information (\funcarg{-g}): the CMM no longer uses \funcname{raise\_notrace} to raise the exception but \funcname{raise} instead
        \item<2-> Disassembly shows that CMM \funcname{raise} is actually a call to \funcname{caml\_raise\_exn}, a function in assembler provided by the runtime
      \end{itemize}
      \vfill
      \mintocaml[firstline=9,lastline=13,highlightlines=12]{../src/backtrace/backtrace.ml}
      \endminipage
    \end{column}
    \begin{column}{0.5\textwidth}
      \onslide*<1>{
        \mintcmm[highlightlines=5]{../src/backtrace/camlBacktrace\_\_definitely\_raise\_347.cmm}
      }
      \onslide*<2>{
        \mintobjdump[firstline=2,highlightlines=14]{../src/backtrace/camlBacktrace\_\_definitely\_raise\_347.objdump}
      }
    \end{column}
  \end{columns}
\end{frame}

\begin{frame}{Backtraces}{Introducing \funcname{caml\_raise\_exn}}
  \begin{columns}[c]
    \begin{column}{0.5\textwidth}
      \minipage[c][0.66\textheight][s]{\columnwidth}
      \onslide*<1>{
        \begin{itemize}
          \item Raising the exception is performed using \funcname{caml\_raise\_exn} which is
            \begin{itemize}
              \item An assembly function provided by the runtime
              \item Architecture specific
            \end{itemize}
          \item Let's see what it does differently from \funcname{raise\_notrace}\dots
          \item We are about to raise \typename{ExnC} with \funcname{caml\_raise\_exn} from \funcname{definitely\_raise}
        \end{itemize}
      }
      \onslide*<2>{
        \mintobjdump[firstline=2,highlightlines=14]{../src/backtrace/camlBacktrace\_\_definitely\_raise\_347.objdump}
      }
      \vfill
      \mintocaml[firstline=9,lastline=13,highlightlines=12]{../src/backtrace/backtrace.ml}
      \endminipage
    \end{column}
    \begin{column}{0.5\textwidth}
      \centering
      \providecommand\step{1}\input{diagrams/backtrace/backtrace.tex}
    \end{column}
  \end{columns}
\end{frame}

\begin{frame}{Backtraces}{But before that, we need more fields from \typename{caml\_domain\_state}}
  \begin{columns}
    \begin{column}{0.5\textwidth}
      \begin{itemize}
        \item Remember \typename{caml\_domain\_state} the per-domain runtime state?
        \item Some other members are of interest to us now\dots
        \item \localname{backtrace\_active} is used to know if the runtime should collect backtraces
        \item \localname{backtrace\_active} can be set by
          \begin{itemize}
            \item The \funcarg{b} flag in the \funcname{OCAMLRUNPARAM} environment variable
            \item \mintinline{ocaml}{Printexc.record_backtrace : bool -> unit} \footnotemark
          \end{itemize}
      \end{itemize}
    \end{column}
    \begin{column}{0.5\textwidth}
      \begin{listing}
        \mintc[highlightlines={9-11}]{src/ocaml/domain\_state\_backtrace.h}
        \caption{runtime/caml/domain\_state.h}
      \end{listing}
    \end{column}
  \end{columns}
  \footnotetext[1]{https://v2.ocaml.org/api/Printexc.html}
\end{frame}

\newcommand\listCamlRaiseExn[1][]{
  \listasm[firstline=702,lastline=725,#1]{../ocaml/runtime/amd64.S}{runtime/amd64.S:702}}

\begin{frame}{Backtraces}{Walk through \funcname{caml\_raise\_exn}}
  \begin{columns}[T]
    \begin{column}{0.5\textwidth}
      \begin{itemize}
        \item<1-> First checks whether exceptions backtrace should be recorded
        \item<1-> By checking the \funcarg{domain\_state->backtrace\_active} value
        \item<2-> If not, acts as \funcname{raise\_notrace} with \funcname{RESTORE\_EXN\_HANDLER\_OCAML}
          \begin{itemize}
            \item Set \regname{RSP} to \localname{domain\_state->exn\_handler} value
            \item Restore \localname{domain\_state->exn\_handler} from trap
            \item Jump with \funcname{ret} (same effect as using \funcname{pop} and \funcname{jmp})
          \end{itemize}
        \item<3-> Otherwise, switches to the C stack to be able to execute C code
        \item<4-> Calls \funcname{caml\_stash\_backtrace}, a C function provided by the runtime\ignorespaces
          \onslide<6->{, with
            \begin{itemize}
              \item \funcarg{exn}: exception value
              \item \funcarg{pc}: code address of the raising point
              \item \funcarg{sp}: stack address of the raising function
              \item \funcarg{trapsp}: stack address of the next handler
            \end{itemize}
          }
      \end{itemize}
      \bigskip
      \mintocaml[firstline=9,lastline=13,highlightlines=12]{../src/backtrace/backtrace.ml}
    \end{column}
    \begin{column}{0.5\textwidth}
      \raggedleft
      \onslide*<1,3-4>{
        \mintc[firstline=190,lastline=192]{../ocaml/runtime/amd64.S}
        \mintc[firstline=234,lastline=237]{../ocaml/runtime/amd64.S}
      }
      \onslide*<2>{
        \mintc[firstline=190,lastline=192,highlightlines={191-192}]{../ocaml/runtime/amd64.S}
        \mintc[firstline=234,lastline=237,highlightlines={235-237}]{../ocaml/runtime/amd64.S}
      }
      \onslide*<1>{\listCamlRaiseExn[highlightlines=705]{}}%
      \onslide*<2>{\listCamlRaiseExn[highlightlines={707-708}]{}}%
      \onslide*<3>{\listCamlRaiseExn[highlightlines={712-714}]{}}%
      \onslide*<4>{\listCamlRaiseExn[highlightlines={715-720}]{}}%
      \onslide*<5>{\providecommand\step{1}\input{diagrams/backtrace/backtrace.tex}}%
      \onslide*<6>{\providecommand\step{2}\input{diagrams/backtrace/backtrace.tex}}%
    \end{column}
  \end{columns}
\end{frame}

\newcommand\listStashBacktrace[1][]{
  \listc[firstline=91,lastline=120,#1]{../ocaml/runtime/backtrace\_nat.c}{runtime/backtrace\_nat.c:91}}

\begin{frame}{Backtraces}{Introducing \funcname{caml\_stash\_backtrace}}
  \begin{columns}[c]
    \begin{column}{0.5\textwidth}
      \minipage[c][0.66\textheight][s]{\columnwidth}
      \begin{itemize}
        \item<1-> A C function provided by the runtime (specific to native code)
        \item<2-> It scans the stack, between the stack pointer at the raising point up to the next trap, in order to collect the names of the detected function frames
          \onslide*<2>{
            \begin{itemize}
              \item And more information, like line number, column number...
            \end{itemize}
          }
        \item<3-> It uses the same technique and information as the Garbage Collector (GC) uses to scan the values live on the stack
          \onslide*<4>{
            \begin{itemize}
              \item \typename{frame\_descr} are at the core of stack unwinding
            \end{itemize}
          }
      \end{itemize}
      \vfill
      \mintocaml[firstline=9,lastline=13,highlightlines=12]{../src/backtrace/backtrace.ml}
      \endminipage
    \end{column}
    \begin{column}{0.5\textwidth}
      \onslide*<1>{\listStashBacktrace[highlightlines=91]{}}
      \onslide*<2>{\listStashBacktrace[highlightlines={91,117-118}]{}}
      \onslide*<3->{\listStashBacktrace[highlightlines={94,106,109-110}]{}}
    \end{column}
  \end{columns}
\end{frame}

\newcommand\listFrameDescr[1][]{
  \listocaml[firstline=111,lastline=124,#1]{../ocaml/asmcomp/emitaux.ml}{asmcomp/emitaux.ml:111}}
\newcommand\listDebugInfo[1][]{
  \listocaml[firstline=38,lastline=49,#1]{../ocaml/lambda/debuginfo.mli}{lambda/debuginfo.mli:38}}

\begin{frame}{Backtraces}{\typename{frame\_descr} and \typename{frame\_debuginfo} in a nutshell}
  \begin{columns}[c]
    \begin{column}{0.5\textwidth}
      \begin{itemize}
        \item<1-> For every return address in an OCaml program, there is a associated \typename{frame\_descr}
        \item<2-> A \typename{frame\_descr} specifies
        \begin{itemize}
          \item<2-> The size of the function stack frame
          \item<3-> Where live values are on the stack (so that the GC can mark them as live and prevent from being swept)
        \end{itemize}
        \item<4-> When compiled with debug information, \typename{frame\_descr} will be linked to a \typename{frame\_debuginfo}
        \begin{itemize}
          \item<5-> Contains information about the position in the source code (file name, line and column number)
        \end{itemize}
      \end{itemize}
    \end{column}
    \begin{column}{0.5\textwidth}
        \onslide*<1>{\listFrameDescr[highlightlines={118-122}]{}}
        \onslide*<2>{\listFrameDescr[highlightlines={120}]{}}
        \onslide*<3>{\listFrameDescr[highlightlines={121}]{}}
        \onslide*<4->{\listFrameDescr[highlightlines={122}]{}}
        \medskip
        \onslide*<1-3>{\listDebugInfo{}}
        \onslide*<4>{\listDebugInfo[highlightlines={38,49}]{}}
        \onslide*<5>{\listDebugInfo[highlightlines={39-46}]{}}
    \end{column}
  \end{columns}
\end{frame}

\begin{frame}{Backtraces}{Scanning the stack with \typename{frame\_descr}}
  \begin{columns}[c]
    \begin{column}{0.5\textwidth}
      \begin{itemize}
        \item Given the return address of the code that raises the exception by calling \funcname{caml\_raise\_exn} (\funcarg{pc} argument of \funcname{caml\_stash\_backtrace})
        \item And the position of the stack pointer (\funcarg{sp} argument of \funcname{caml\_stash\_backtrace})
        \item The frame size given by \typename{frame\_descr}.\funcarg{frame\_size}, allows to find the end of the previous frame
        \item And the return address of the caller of this function
        \bigskip
        \onslide<3->{
        \item Note that traps (here the reddish \funcname{ExnA}, \funcname{ExnB} and \funcname{ExnC} blocks) belong to the frame that installed them, and so the frame size changes when traps are removed
        }
      \end{itemize}
    \end{column}
    \begin{column}{0.5\textwidth}
      \raggedleft
      \onslide*<1>{\providecommand\step{2}\input{diagrams/backtrace/backtrace.tex}}%
      \onslide*<2>{\providecommand\step{1}\input{diagrams/backtrace/scanstack.tex}}%
      \onslide*<3>{\providecommand\step{2}\input{diagrams/backtrace/scanstack.tex}}%
      \onslide*<4>{\providecommand\step{3}\input{diagrams/backtrace/scanstack.tex}}%
      \onslide*<5>{\providecommand\step{4}\input{diagrams/backtrace/scanstack.tex}}%
      \onslide*<6>{\providecommand\step{5}\input{diagrams/backtrace/scanstack.tex}}%
    \end{column}
  \end{columns}
\end{frame}

% Duplicate of previous-previous slide
\begin{frame}{Backtraces}{Continuing on \funcname{caml\_stash\_backtrace}}
  \begin{columns}[c]
    \begin{column}{0.5\textwidth}
      \minipage[c][0.75\textheight][s]{\columnwidth}
      \begin{itemize}
          \color{gray}
        \item A C function provided by the runtime (specific to native code)
        \item It scans the stack, between the stack pointer at the raising point up to the next trap, in order to collect the names of the detected function frames
        \item It uses the same technique and information as the Garbage Collector (GC) uses to scan the values live on the stack
      \end{itemize}
      \begin{itemize}
        \item For each function frame detected, store a pointer to the \typename{frame\_descr} in the \localname{domain\_state->backtrace\_buffer} array, effectively constructing the backtrace
      \end{itemize}
      \onslide<2->{
        \begin{itemize}
            \smallskip
          \item Constructed backtrace:
            \funcname{definitely\_raise},
            \onslide<3->{\funcname{probably\_raise}}
        \end{itemize}
      }
      \vfill
      \mintocaml[firstline=9,lastline=13,highlightlines=12]{../src/backtrace/backtrace.ml}
      \endminipage
    \end{column}
    \begin{column}{0.5\textwidth}
      \raggedleft
      \onslide*<1>{\listStashBacktrace[highlightlines={114-115}]{}}
      \onslide*<2>{\providecommand\step{3}\input{diagrams/backtrace/backtrace.tex}}%
      \onslide*<3>{\providecommand\step{4}\input{diagrams/backtrace/backtrace.tex}}%
    \end{column}
  \end{columns}
\end{frame}

\begin{frame}{Backtraces}{Back to \funcname{caml\_raise\_exn}}
  \begin{columns}[c]
    \begin{column}{0.5\textwidth}
      \minipage[c][0.66\textheight][s]{\columnwidth}
      \begin{itemize}\color{gray}
        \item Checks wheteher exceptions backtrace should be recorded, by checking the \funcarg{domain\_state->backtrace\_active} value
        \item Switches to the C stack to be able to execute C code
        \item Calls \funcname{caml\_stash\_backtrace}, to collect the backtrace up to the next trap
      \end{itemize}
      \onslide*<2->{
        \begin{itemize}
          \item<2-> Executes the exception handler in \funcname{probably\_raise} with \funcname{RESTORE\_EXN\_HANDLER\_OCAML}
            \begin{itemize}
              \item Set \regname{RSP} to \localname{domain\_state->exn\_handler} value
              \item Restore \localname{domain\_state->exn\_handler} from trap
              \item Jump to the handler code with \funcname{ret}
            \end{itemize}
        \end{itemize}
      }
      \vfill
      \onslide*<1>{\mintocaml[firstline=9,lastline=13,highlightlines=12]{../src/backtrace/backtrace.ml}}
      \onslide*<2->{\mintocaml[firstline=15,lastline=21,highlightlines={19-21}]{../src/backtrace/backtrace.ml}}
      \endminipage
    \end{column}
    \begin{column}{0.5\textwidth}
      \centering
      \onslide*<1>{
        \mintc[firstline=190,lastline=192]{../ocaml/runtime/amd64.S}
        \mintc[firstline=234,lastline=237]{../ocaml/runtime/amd64.S}
      }
      \onslide*<2>{
        \mintc[firstline=190,lastline=192,highlightlines={191-192}]{../ocaml/runtime/amd64.S}
        \mintc[firstline=234,lastline=237,highlightlines={235-237}]{../ocaml/runtime/amd64.S}
      }
      \onslide*<1>{\listCamlRaiseExn[highlightlines=720]{}}
      \onslide*<2>{\listCamlRaiseExn[highlightlines=722]{}}
      \onslide*<3->{\providecommand\step{5}\input{diagrams/backtrace/backtrace.tex}}
    \end{column}
  \end{columns}
\end{frame}

\begin{frame}{Backtraces}{\funcname{caml\_probably\_raise}'s handler doesn't catch \typename{ExnC}}
  \begin{columns}[c]
    \begin{column}{0.45\textwidth}
      \minipage[c][0.75\textheight][s]{\columnwidth}
      \begin{itemize}
        \item<1-> \funcname{match \dots with}\footnotemark pattern matching can also match exceptions
        \item<1-> The CMM of \funcname{probably\_raise} shows that this \funcname{match \dots with} is implemented with a pattern matching and a \funcname{try \dots with}
        \item<1-> But this one only matches on \typename{ExnA} exception
        \item<2-> Since the pattern matching fails to catch the exception, \funcname{reraise} is called with our current \typename{ExnC} exception
        \item<3-> Disassembly shows that \funcname{reraise} is actually a call to \funcname{caml\_reraise\_exn}, which is also an assembly function provided by the runtime
      \end{itemize}
      \vfill
      \mintocaml[firstline=15,lastline=21,highlightlines={18-21}]{../src/backtrace/backtrace.ml}
      \endminipage
    \end{column}
    \begin{column}{0.55\textwidth}
      \centering
      \onslide*<1>{\mintcmm[highlightlines={10-18}]{../src/backtrace/camlBacktrace\_\_probably\_raise\_351.cmm}}
      \onslide*<2>{\listcmm[highlightlines=18]{../src/backtrace/camlBacktrace\_\_probably\_raise_351.cmm}{CMM of probably\_raise}}
      \onslide*<3>{\mintobjdump[firstline=5,lastline=35,highlightlines=31]{../src/backtrace/camlBacktrace\_\_probably\_raise_351.objdump}}
    \end{column}
  \end{columns}
  \only<1>{\footnotetext[1]{https://blog.janestreet.com/pattern-matching-and-exception-handling-unite/}}
\end{frame}

\newcommand\listCamlReraiseExn[1][]{
  \listasm[firstline=727,lastline=734,#1]{../ocaml/runtime/amd64.S}{runtime/amd64.S:727}}

\begin{frame}{Backtraces}{Introducing \funcname{caml\_reraise\_exn}}
  \begin{columns}[c]
    \begin{column}{0.5\textwidth}
      \minipage[c][0.66\textheight][s]{\columnwidth}
      \begin{itemize}
        \item<1-> The exception have to be reraised to the next handler
        \item<2-> First, checks if the backtrace should be collected with \funcarg{domain\_state->backtrace\_active}
        \only<3>{
        \begin{itemize}
            \item If not, behaves like a \funcname{raise\_notrace} with \funcname{RESTORE\_EXN\_HANDLER\_OCAML}
        \end{itemize}
        }
        \item<4-> Jumps into \funcname{caml\_raise\_exn}
        \item<5-> Calls \funcname{caml\_stash\_backtrace} again, to scan the next part of the stack: from the trap just pop-ed to the next trap
      \end{itemize}
      \vfill
      \mintocaml[firstline=15,lastline=21,highlightlines={18-21}]{../src/backtrace/backtrace.ml}
      \endminipage
    \end{column}
    \begin{column}{0.5\textwidth}
      \raggedleft
      \onslide*<1>{\listCamlReraiseExn{}}
      \onslide*<2>{\listCamlReraiseExn[highlightlines=729]{}}
      \onslide*<3>{
        \mintc[firstline=190,lastline=192,highlightlines={191-192}]{../ocaml/runtime/amd64.S}
        \mintc[firstline=234,lastline=237,highlightlines={235-237}]{../ocaml/runtime/amd64.S}
        \medskip
        \listCamlReraiseExn[highlightlines=731]{}
      }
      \onslide*<4-5>{
        % TODO refactor with \listCamlReraiseExn
        \mintasm[firstline=727,lastline=734,highlightlines=730]{../ocaml/runtime/amd64.S}
        \medskip
      }
      \onslide*<4>{\listCamlRaiseExn[highlightlines=711]{}}
      \onslide*<5>{\listCamlRaiseExn[highlightlines=720]{}}
    \end{column}
  \end{columns}
\end{frame}

\begin{frame}{Backtraces}{Backtrace collection continues}
  \begin{columns}[c]
    \begin{column}{0.55\textwidth}
      \minipage[c][0.75\textheight][s]{\columnwidth}
      \begin{itemize}
        \item<1-> \funcname{caml\_stash\_backtrace} scans the older part of the stack, up to the next trap
        \item<6-> Controls jumps to the nested exception handler in \funcname{maybe\_raise}, which matches only on \typename{ExnB}
        \item<7-> \funcname{caml\_reraise\_exn} is called again, backtrace collection resumes with \funcname{caml\_stash\_backtrace}
      \end{itemize}
      \medskip
      \begin{itemize}
        \item<2-> Constructed backtrace:
        \begin{itemize}
          \item <2-> \funcname{definitely\_raise}
          \item <2-> \funcname{probably\_raise}
          \item <3-> \funcname{probably\_raise} (re-raise)
          \item <4-> \funcname{maybe\_raise}
          \item <5-> \funcname{main}
          \item <8-> \funcname{main} (re-raise)
        \end{itemize}
      \end{itemize}
      \vfill
      \onslide*<1-5>{\mintocaml[firstline=15,lastline=21]{../src/backtrace/backtrace.ml}}
      \onslide*<6>{\mintocaml[firstline=28,lastline=34,highlightlines=31]{../src/backtrace/backtrace.ml}}
      \onslide*<7->{\mintocaml[firstline=28,lastline=34,highlightlines=33]{../src/backtrace/backtrace.ml}}
      \endminipage
    \end{column}
    \begin{column}{0.45\textwidth}
      \raggedleft
      \onslide*<1>{\providecommand\step{6}\input{diagrams/backtrace/backtrace.tex}}%
      \onslide*<2>{\providecommand\step{6}\input{diagrams/backtrace/backtrace.tex}}%
      \onslide*<3>{\providecommand\step{7}\input{diagrams/backtrace/backtrace.tex}}%
      \onslide*<4>{\providecommand\step{8}\input{diagrams/backtrace/backtrace.tex}}%
      \onslide*<5>{\providecommand\step{9}\input{diagrams/backtrace/backtrace.tex}}%
      \onslide*<6>{\providecommand\step{10}\input{diagrams/backtrace/backtrace.tex}}%
      \onslide*<7>{\providecommand\step{11}\input{diagrams/backtrace/backtrace.tex}}%
      \onslide*<8>{\providecommand\step{12}\input{diagrams/backtrace/backtrace.tex}}%
      \onslide*<9>{\providecommand\step{13}\input{diagrams/backtrace/backtrace.tex}}%
    \end{column}
  \end{columns}
\end{frame}

\begin{frame}{Backtraces}{Printing the backtrace}
  \begin{columns}[c]
    \begin{column}{0.45\textwidth}
      \minipage[c][0.66\textheight][s]{\columnwidth}
      \begin{itemize}
        \item<1-> The exception backtrace can now be printed to \funcarg{stdout} with \funcname{Printexc.print_backtrace}
        \item<2-> First the exception backtrace is retrieved into an OCaml array with \funcname{get_raw_backtrace }
        \item<3-> Which is implemented as the \funcname{caml_get_exception_raw_backtrace} C function in the runtime
        \item<4-> And finally printed with \funcname{print_exception_backtrace}
      \end{itemize}
      \vfill
      \mintocaml[firstline=28,lastline=34,highlightlines=33]{../src/backtrace/backtrace.ml}
      \endminipage
    \end{column}
    \begin{column}{0.55\textwidth}
      \onslide*<1,2>{
        \mintocaml[firstline=100,lastline=101]{../ocaml/stdlib/printexc.ml}
      }
      \only<1>{
        \listocaml[firstline=170,lastline=172]{../ocaml/stdlib/printexc.ml}{stdlib/printexc.ml:170}
      }
      \only<2>{
        \listocaml[firstline=170,lastline=172,highlightlines=172]{../ocaml/stdlib/printexc.ml}{stdlib/printexc.ml:170}
      }
      \only<3>{
        \mintc[firstline=154,lastline=156]{../ocaml/runtime/backtrace.c}
        \listc[firstline=171,lastline=192,highlightlines={184-187}]{../ocaml/runtime/backtrace.c}{runtime/backtrace.c:154}
      }
      \only<4>{
        \listocaml[firstline=155,lastline=165]{../ocaml/stdlib/printexc.ml}{stdlib/printexc.ml:55}
      }
    \end{column}
  \end{columns}
\end{frame}


\subsection{The multiple kinds of raise}
\frameSubsection{}{}

\begin{frame}{Backtraces}{When backtraces are not needed}
  \begin{columns}[c]
    \begin{column}{0.45\textwidth}
      \minipage[c][0.66\textheight][s]{\columnwidth}
      \begin{itemize}
        \item<1-> What if the backtrace for an exception is never needed?
        \item<2-> It's possible to raise the exception explicitly with \funcname{raise\_notrace} to make raising as cheap as possible
        \item<3-> An example of use in the \funcname{dynlink} library of the OCaml compiler
      \end{itemize}
      \vfill
      \onslide<3->{
      \listocaml[firstline=205,lastline=209,highlightlines=207]{../ocaml/otherlibs/dynlink/dynlink\_compilerlibs/misc.ml}{otherlibs/dynlink/dynlink\_compilerlibs/misc.ml:205}
      }
      \endminipage
    \end{column}
    \begin{column}{0.55\textwidth}
    \only<4>{
      \mintcmm[highlightlines=3]{../src/notrace/camlNotrace\_\_fun_347.cmm}
      \listcmm{../src/notrace/camlNotrace\_\_all\_somes\_327.cmm}{all\_somes}
    }
    \onslide*<5->{
      \listobjdump[highlightlines={6-9}]{../src/notrace/camlNotrace\_\_fun_347.objdump}{anonymous function from \funcname{all\_somes}}
    }
    \end{column}
  \end{columns}
\end{frame}

\begin{frame}{Backtraces}{Summary table of the multiple kinds of raise}
  \begin{itemize}
    \item When debugging information is enabled and if the backtrace of an exception isn't needed, it's possible to raise with \funcname{raise\_notrace} for performance
    \item When debugging information is \emph{not} enabled, the compiler optimises \funcname{raise} calls to inline assembly (same effect as using \funcname{raise\_notrace})
  \end{itemize}
  \bigskip
  \centering
  \begin{tabular}{| l | c | c | c | c |}
    \hline
    \parbox{10em}{Debug information \\ (\funcarg{-g} flag)}  & \multicolumn{2}{|c|}{Disabled}                                     & \multicolumn{2}{|c|}{Enabled} \\
    \hline
    Raise kind                                               & \funcname{raise} & \funcname{raise\_notrace}                       & \funcname{raise} & \funcname{raise\_notrace} \\
    \hline
    Compilation                                              & \parbox{8em}{inline assembly\\ (optimization)} & inline assembly   & \funcname{caml\_raise\_exn} & inline assembly \\
    \hline
  \end{tabular}
\end{frame}


%
% Takeaway #5
%
\subsection*{Takeaway \#5}
\frameSubsectionTakeaway{}
\againframe<5>{takeaway}
}%
    \end{column}
  \end{columns}
\end{frame}

\begin{frame}{Backtraces}{Back to \funcname{caml\_raise\_exn}}
  \begin{columns}[c]
    \begin{column}{0.5\textwidth}
      \minipage[c][0.66\textheight][s]{\columnwidth}
      \begin{itemize}\color{gray}
        \item Checks wheteher exceptions backtrace should be recorded, by checking the \funcarg{domain\_state->backtrace\_active} value
        \item Switches to the C stack to be able to execute C code
        \item Calls \funcname{caml\_stash\_backtrace}, to collect the backtrace up to the next trap
      \end{itemize}
      \onslide*<2->{
        \begin{itemize}
          \item<2-> Executes the exception handler in \funcname{probably\_raise} with \funcname{RESTORE\_EXN\_HANDLER\_OCAML}
            \begin{itemize}
              \item Set \regname{RSP} to \localname{domain\_state->exn\_handler} value
              \item Restore \localname{domain\_state->exn\_handler} from trap
              \item Jump to the handler code with \funcname{ret}
            \end{itemize}
        \end{itemize}
      }
      \vfill
      \onslide*<1>{\mintocaml[firstline=9,lastline=13,highlightlines=12]{../src/backtrace/backtrace.ml}}
      \onslide*<2->{\mintocaml[firstline=15,lastline=21,highlightlines={19-21}]{../src/backtrace/backtrace.ml}}
      \endminipage
    \end{column}
    \begin{column}{0.5\textwidth}
      \centering
      \onslide*<1>{
        \mintc[firstline=190,lastline=192]{../ocaml/runtime/amd64.S}
        \mintc[firstline=234,lastline=237]{../ocaml/runtime/amd64.S}
      }
      \onslide*<2>{
        \mintc[firstline=190,lastline=192,highlightlines={191-192}]{../ocaml/runtime/amd64.S}
        \mintc[firstline=234,lastline=237,highlightlines={235-237}]{../ocaml/runtime/amd64.S}
      }
      \onslide*<1>{\listCamlRaiseExn[highlightlines=720]{}}
      \onslide*<2>{\listCamlRaiseExn[highlightlines=722]{}}
      \onslide*<3->{\providecommand\step{5}%
% Backtraces
%
\subsection{One advantage of raising with debugging information}
\frameSubsection{}{}

\begin{frame}{Backtraces}{Debugging information \& source code}
  \begin{columns}[c]
    \begin{column}{0.5\textwidth}
      \begin{itemize}
        \item Raising an exception is fun and games, but how do we know where the exception originated? $\rightarrow$ With the backtrace associated with the exception!
        \item In order to get a backtrace from the point where an exception was raised, it's necessary to compile with debugging information enabled (\funcarg{-g} flag for the compiler)
        \item Same example as in the `Nested handlers' section
      \end{itemize}
    \end{column}
    \begin{column}{0.5\textwidth}
      \listocaml[firstline=9,lastline=34]{../src/backtrace/backtrace.ml}{backtrace/backtrace.ml}
    \end{column}
  \end{columns}
\end{frame}

\begin{frame}{Backtraces}{CMM and disassembly of \funcname{definitely\_raise}}
  \begin{columns}[c]
    \begin{column}{0.5\textwidth}
      \minipage[c][0.66\textheight][s]{\columnwidth}
      \begin{itemize}
        \item<1-> An effect of compiling with debugging information (\funcarg{-g}): the CMM no longer uses \funcname{raise\_notrace} to raise the exception but \funcname{raise} instead
        \item<2-> Disassembly shows that CMM \funcname{raise} is actually a call to \funcname{caml\_raise\_exn}, a function in assembler provided by the runtime
      \end{itemize}
      \vfill
      \mintocaml[firstline=9,lastline=13,highlightlines=12]{../src/backtrace/backtrace.ml}
      \endminipage
    \end{column}
    \begin{column}{0.5\textwidth}
      \onslide*<1>{
        \mintcmm[highlightlines=5]{../src/backtrace/camlBacktrace\_\_definitely\_raise\_347.cmm}
      }
      \onslide*<2>{
        \mintobjdump[firstline=2,highlightlines=14]{../src/backtrace/camlBacktrace\_\_definitely\_raise\_347.objdump}
      }
    \end{column}
  \end{columns}
\end{frame}

\begin{frame}{Backtraces}{Introducing \funcname{caml\_raise\_exn}}
  \begin{columns}[c]
    \begin{column}{0.5\textwidth}
      \minipage[c][0.66\textheight][s]{\columnwidth}
      \onslide*<1>{
        \begin{itemize}
          \item Raising the exception is performed using \funcname{caml\_raise\_exn} which is
            \begin{itemize}
              \item An assembly function provided by the runtime
              \item Architecture specific
            \end{itemize}
          \item Let's see what it does differently from \funcname{raise\_notrace}\dots
          \item We are about to raise \typename{ExnC} with \funcname{caml\_raise\_exn} from \funcname{definitely\_raise}
        \end{itemize}
      }
      \onslide*<2>{
        \mintobjdump[firstline=2,highlightlines=14]{../src/backtrace/camlBacktrace\_\_definitely\_raise\_347.objdump}
      }
      \vfill
      \mintocaml[firstline=9,lastline=13,highlightlines=12]{../src/backtrace/backtrace.ml}
      \endminipage
    \end{column}
    \begin{column}{0.5\textwidth}
      \centering
      \providecommand\step{1}\input{diagrams/backtrace/backtrace.tex}
    \end{column}
  \end{columns}
\end{frame}

\begin{frame}{Backtraces}{But before that, we need more fields from \typename{caml\_domain\_state}}
  \begin{columns}
    \begin{column}{0.5\textwidth}
      \begin{itemize}
        \item Remember \typename{caml\_domain\_state} the per-domain runtime state?
        \item Some other members are of interest to us now\dots
        \item \localname{backtrace\_active} is used to know if the runtime should collect backtraces
        \item \localname{backtrace\_active} can be set by
          \begin{itemize}
            \item The \funcarg{b} flag in the \funcname{OCAMLRUNPARAM} environment variable
            \item \mintinline{ocaml}{Printexc.record_backtrace : bool -> unit} \footnotemark
          \end{itemize}
      \end{itemize}
    \end{column}
    \begin{column}{0.5\textwidth}
      \begin{listing}
        \mintc[highlightlines={9-11}]{src/ocaml/domain\_state\_backtrace.h}
        \caption{runtime/caml/domain\_state.h}
      \end{listing}
    \end{column}
  \end{columns}
  \footnotetext[1]{https://v2.ocaml.org/api/Printexc.html}
\end{frame}

\newcommand\listCamlRaiseExn[1][]{
  \listasm[firstline=702,lastline=725,#1]{../ocaml/runtime/amd64.S}{runtime/amd64.S:702}}

\begin{frame}{Backtraces}{Walk through \funcname{caml\_raise\_exn}}
  \begin{columns}[T]
    \begin{column}{0.5\textwidth}
      \begin{itemize}
        \item<1-> First checks whether exceptions backtrace should be recorded
        \item<1-> By checking the \funcarg{domain\_state->backtrace\_active} value
        \item<2-> If not, acts as \funcname{raise\_notrace} with \funcname{RESTORE\_EXN\_HANDLER\_OCAML}
          \begin{itemize}
            \item Set \regname{RSP} to \localname{domain\_state->exn\_handler} value
            \item Restore \localname{domain\_state->exn\_handler} from trap
            \item Jump with \funcname{ret} (same effect as using \funcname{pop} and \funcname{jmp})
          \end{itemize}
        \item<3-> Otherwise, switches to the C stack to be able to execute C code
        \item<4-> Calls \funcname{caml\_stash\_backtrace}, a C function provided by the runtime\ignorespaces
          \onslide<6->{, with
            \begin{itemize}
              \item \funcarg{exn}: exception value
              \item \funcarg{pc}: code address of the raising point
              \item \funcarg{sp}: stack address of the raising function
              \item \funcarg{trapsp}: stack address of the next handler
            \end{itemize}
          }
      \end{itemize}
      \bigskip
      \mintocaml[firstline=9,lastline=13,highlightlines=12]{../src/backtrace/backtrace.ml}
    \end{column}
    \begin{column}{0.5\textwidth}
      \raggedleft
      \onslide*<1,3-4>{
        \mintc[firstline=190,lastline=192]{../ocaml/runtime/amd64.S}
        \mintc[firstline=234,lastline=237]{../ocaml/runtime/amd64.S}
      }
      \onslide*<2>{
        \mintc[firstline=190,lastline=192,highlightlines={191-192}]{../ocaml/runtime/amd64.S}
        \mintc[firstline=234,lastline=237,highlightlines={235-237}]{../ocaml/runtime/amd64.S}
      }
      \onslide*<1>{\listCamlRaiseExn[highlightlines=705]{}}%
      \onslide*<2>{\listCamlRaiseExn[highlightlines={707-708}]{}}%
      \onslide*<3>{\listCamlRaiseExn[highlightlines={712-714}]{}}%
      \onslide*<4>{\listCamlRaiseExn[highlightlines={715-720}]{}}%
      \onslide*<5>{\providecommand\step{1}\input{diagrams/backtrace/backtrace.tex}}%
      \onslide*<6>{\providecommand\step{2}\input{diagrams/backtrace/backtrace.tex}}%
    \end{column}
  \end{columns}
\end{frame}

\newcommand\listStashBacktrace[1][]{
  \listc[firstline=91,lastline=120,#1]{../ocaml/runtime/backtrace\_nat.c}{runtime/backtrace\_nat.c:91}}

\begin{frame}{Backtraces}{Introducing \funcname{caml\_stash\_backtrace}}
  \begin{columns}[c]
    \begin{column}{0.5\textwidth}
      \minipage[c][0.66\textheight][s]{\columnwidth}
      \begin{itemize}
        \item<1-> A C function provided by the runtime (specific to native code)
        \item<2-> It scans the stack, between the stack pointer at the raising point up to the next trap, in order to collect the names of the detected function frames
          \onslide*<2>{
            \begin{itemize}
              \item And more information, like line number, column number...
            \end{itemize}
          }
        \item<3-> It uses the same technique and information as the Garbage Collector (GC) uses to scan the values live on the stack
          \onslide*<4>{
            \begin{itemize}
              \item \typename{frame\_descr} are at the core of stack unwinding
            \end{itemize}
          }
      \end{itemize}
      \vfill
      \mintocaml[firstline=9,lastline=13,highlightlines=12]{../src/backtrace/backtrace.ml}
      \endminipage
    \end{column}
    \begin{column}{0.5\textwidth}
      \onslide*<1>{\listStashBacktrace[highlightlines=91]{}}
      \onslide*<2>{\listStashBacktrace[highlightlines={91,117-118}]{}}
      \onslide*<3->{\listStashBacktrace[highlightlines={94,106,109-110}]{}}
    \end{column}
  \end{columns}
\end{frame}

\newcommand\listFrameDescr[1][]{
  \listocaml[firstline=111,lastline=124,#1]{../ocaml/asmcomp/emitaux.ml}{asmcomp/emitaux.ml:111}}
\newcommand\listDebugInfo[1][]{
  \listocaml[firstline=38,lastline=49,#1]{../ocaml/lambda/debuginfo.mli}{lambda/debuginfo.mli:38}}

\begin{frame}{Backtraces}{\typename{frame\_descr} and \typename{frame\_debuginfo} in a nutshell}
  \begin{columns}[c]
    \begin{column}{0.5\textwidth}
      \begin{itemize}
        \item<1-> For every return address in an OCaml program, there is a associated \typename{frame\_descr}
        \item<2-> A \typename{frame\_descr} specifies
        \begin{itemize}
          \item<2-> The size of the function stack frame
          \item<3-> Where live values are on the stack (so that the GC can mark them as live and prevent from being swept)
        \end{itemize}
        \item<4-> When compiled with debug information, \typename{frame\_descr} will be linked to a \typename{frame\_debuginfo}
        \begin{itemize}
          \item<5-> Contains information about the position in the source code (file name, line and column number)
        \end{itemize}
      \end{itemize}
    \end{column}
    \begin{column}{0.5\textwidth}
        \onslide*<1>{\listFrameDescr[highlightlines={118-122}]{}}
        \onslide*<2>{\listFrameDescr[highlightlines={120}]{}}
        \onslide*<3>{\listFrameDescr[highlightlines={121}]{}}
        \onslide*<4->{\listFrameDescr[highlightlines={122}]{}}
        \medskip
        \onslide*<1-3>{\listDebugInfo{}}
        \onslide*<4>{\listDebugInfo[highlightlines={38,49}]{}}
        \onslide*<5>{\listDebugInfo[highlightlines={39-46}]{}}
    \end{column}
  \end{columns}
\end{frame}

\begin{frame}{Backtraces}{Scanning the stack with \typename{frame\_descr}}
  \begin{columns}[c]
    \begin{column}{0.5\textwidth}
      \begin{itemize}
        \item Given the return address of the code that raises the exception by calling \funcname{caml\_raise\_exn} (\funcarg{pc} argument of \funcname{caml\_stash\_backtrace})
        \item And the position of the stack pointer (\funcarg{sp} argument of \funcname{caml\_stash\_backtrace})
        \item The frame size given by \typename{frame\_descr}.\funcarg{frame\_size}, allows to find the end of the previous frame
        \item And the return address of the caller of this function
        \bigskip
        \onslide<3->{
        \item Note that traps (here the reddish \funcname{ExnA}, \funcname{ExnB} and \funcname{ExnC} blocks) belong to the frame that installed them, and so the frame size changes when traps are removed
        }
      \end{itemize}
    \end{column}
    \begin{column}{0.5\textwidth}
      \raggedleft
      \onslide*<1>{\providecommand\step{2}\input{diagrams/backtrace/backtrace.tex}}%
      \onslide*<2>{\providecommand\step{1}\input{diagrams/backtrace/scanstack.tex}}%
      \onslide*<3>{\providecommand\step{2}\input{diagrams/backtrace/scanstack.tex}}%
      \onslide*<4>{\providecommand\step{3}\input{diagrams/backtrace/scanstack.tex}}%
      \onslide*<5>{\providecommand\step{4}\input{diagrams/backtrace/scanstack.tex}}%
      \onslide*<6>{\providecommand\step{5}\input{diagrams/backtrace/scanstack.tex}}%
    \end{column}
  \end{columns}
\end{frame}

% Duplicate of previous-previous slide
\begin{frame}{Backtraces}{Continuing on \funcname{caml\_stash\_backtrace}}
  \begin{columns}[c]
    \begin{column}{0.5\textwidth}
      \minipage[c][0.75\textheight][s]{\columnwidth}
      \begin{itemize}
          \color{gray}
        \item A C function provided by the runtime (specific to native code)
        \item It scans the stack, between the stack pointer at the raising point up to the next trap, in order to collect the names of the detected function frames
        \item It uses the same technique and information as the Garbage Collector (GC) uses to scan the values live on the stack
      \end{itemize}
      \begin{itemize}
        \item For each function frame detected, store a pointer to the \typename{frame\_descr} in the \localname{domain\_state->backtrace\_buffer} array, effectively constructing the backtrace
      \end{itemize}
      \onslide<2->{
        \begin{itemize}
            \smallskip
          \item Constructed backtrace:
            \funcname{definitely\_raise},
            \onslide<3->{\funcname{probably\_raise}}
        \end{itemize}
      }
      \vfill
      \mintocaml[firstline=9,lastline=13,highlightlines=12]{../src/backtrace/backtrace.ml}
      \endminipage
    \end{column}
    \begin{column}{0.5\textwidth}
      \raggedleft
      \onslide*<1>{\listStashBacktrace[highlightlines={114-115}]{}}
      \onslide*<2>{\providecommand\step{3}\input{diagrams/backtrace/backtrace.tex}}%
      \onslide*<3>{\providecommand\step{4}\input{diagrams/backtrace/backtrace.tex}}%
    \end{column}
  \end{columns}
\end{frame}

\begin{frame}{Backtraces}{Back to \funcname{caml\_raise\_exn}}
  \begin{columns}[c]
    \begin{column}{0.5\textwidth}
      \minipage[c][0.66\textheight][s]{\columnwidth}
      \begin{itemize}\color{gray}
        \item Checks wheteher exceptions backtrace should be recorded, by checking the \funcarg{domain\_state->backtrace\_active} value
        \item Switches to the C stack to be able to execute C code
        \item Calls \funcname{caml\_stash\_backtrace}, to collect the backtrace up to the next trap
      \end{itemize}
      \onslide*<2->{
        \begin{itemize}
          \item<2-> Executes the exception handler in \funcname{probably\_raise} with \funcname{RESTORE\_EXN\_HANDLER\_OCAML}
            \begin{itemize}
              \item Set \regname{RSP} to \localname{domain\_state->exn\_handler} value
              \item Restore \localname{domain\_state->exn\_handler} from trap
              \item Jump to the handler code with \funcname{ret}
            \end{itemize}
        \end{itemize}
      }
      \vfill
      \onslide*<1>{\mintocaml[firstline=9,lastline=13,highlightlines=12]{../src/backtrace/backtrace.ml}}
      \onslide*<2->{\mintocaml[firstline=15,lastline=21,highlightlines={19-21}]{../src/backtrace/backtrace.ml}}
      \endminipage
    \end{column}
    \begin{column}{0.5\textwidth}
      \centering
      \onslide*<1>{
        \mintc[firstline=190,lastline=192]{../ocaml/runtime/amd64.S}
        \mintc[firstline=234,lastline=237]{../ocaml/runtime/amd64.S}
      }
      \onslide*<2>{
        \mintc[firstline=190,lastline=192,highlightlines={191-192}]{../ocaml/runtime/amd64.S}
        \mintc[firstline=234,lastline=237,highlightlines={235-237}]{../ocaml/runtime/amd64.S}
      }
      \onslide*<1>{\listCamlRaiseExn[highlightlines=720]{}}
      \onslide*<2>{\listCamlRaiseExn[highlightlines=722]{}}
      \onslide*<3->{\providecommand\step{5}\input{diagrams/backtrace/backtrace.tex}}
    \end{column}
  \end{columns}
\end{frame}

\begin{frame}{Backtraces}{\funcname{caml\_probably\_raise}'s handler doesn't catch \typename{ExnC}}
  \begin{columns}[c]
    \begin{column}{0.45\textwidth}
      \minipage[c][0.75\textheight][s]{\columnwidth}
      \begin{itemize}
        \item<1-> \funcname{match \dots with}\footnotemark pattern matching can also match exceptions
        \item<1-> The CMM of \funcname{probably\_raise} shows that this \funcname{match \dots with} is implemented with a pattern matching and a \funcname{try \dots with}
        \item<1-> But this one only matches on \typename{ExnA} exception
        \item<2-> Since the pattern matching fails to catch the exception, \funcname{reraise} is called with our current \typename{ExnC} exception
        \item<3-> Disassembly shows that \funcname{reraise} is actually a call to \funcname{caml\_reraise\_exn}, which is also an assembly function provided by the runtime
      \end{itemize}
      \vfill
      \mintocaml[firstline=15,lastline=21,highlightlines={18-21}]{../src/backtrace/backtrace.ml}
      \endminipage
    \end{column}
    \begin{column}{0.55\textwidth}
      \centering
      \onslide*<1>{\mintcmm[highlightlines={10-18}]{../src/backtrace/camlBacktrace\_\_probably\_raise\_351.cmm}}
      \onslide*<2>{\listcmm[highlightlines=18]{../src/backtrace/camlBacktrace\_\_probably\_raise_351.cmm}{CMM of probably\_raise}}
      \onslide*<3>{\mintobjdump[firstline=5,lastline=35,highlightlines=31]{../src/backtrace/camlBacktrace\_\_probably\_raise_351.objdump}}
    \end{column}
  \end{columns}
  \only<1>{\footnotetext[1]{https://blog.janestreet.com/pattern-matching-and-exception-handling-unite/}}
\end{frame}

\newcommand\listCamlReraiseExn[1][]{
  \listasm[firstline=727,lastline=734,#1]{../ocaml/runtime/amd64.S}{runtime/amd64.S:727}}

\begin{frame}{Backtraces}{Introducing \funcname{caml\_reraise\_exn}}
  \begin{columns}[c]
    \begin{column}{0.5\textwidth}
      \minipage[c][0.66\textheight][s]{\columnwidth}
      \begin{itemize}
        \item<1-> The exception have to be reraised to the next handler
        \item<2-> First, checks if the backtrace should be collected with \funcarg{domain\_state->backtrace\_active}
        \only<3>{
        \begin{itemize}
            \item If not, behaves like a \funcname{raise\_notrace} with \funcname{RESTORE\_EXN\_HANDLER\_OCAML}
        \end{itemize}
        }
        \item<4-> Jumps into \funcname{caml\_raise\_exn}
        \item<5-> Calls \funcname{caml\_stash\_backtrace} again, to scan the next part of the stack: from the trap just pop-ed to the next trap
      \end{itemize}
      \vfill
      \mintocaml[firstline=15,lastline=21,highlightlines={18-21}]{../src/backtrace/backtrace.ml}
      \endminipage
    \end{column}
    \begin{column}{0.5\textwidth}
      \raggedleft
      \onslide*<1>{\listCamlReraiseExn{}}
      \onslide*<2>{\listCamlReraiseExn[highlightlines=729]{}}
      \onslide*<3>{
        \mintc[firstline=190,lastline=192,highlightlines={191-192}]{../ocaml/runtime/amd64.S}
        \mintc[firstline=234,lastline=237,highlightlines={235-237}]{../ocaml/runtime/amd64.S}
        \medskip
        \listCamlReraiseExn[highlightlines=731]{}
      }
      \onslide*<4-5>{
        % TODO refactor with \listCamlReraiseExn
        \mintasm[firstline=727,lastline=734,highlightlines=730]{../ocaml/runtime/amd64.S}
        \medskip
      }
      \onslide*<4>{\listCamlRaiseExn[highlightlines=711]{}}
      \onslide*<5>{\listCamlRaiseExn[highlightlines=720]{}}
    \end{column}
  \end{columns}
\end{frame}

\begin{frame}{Backtraces}{Backtrace collection continues}
  \begin{columns}[c]
    \begin{column}{0.55\textwidth}
      \minipage[c][0.75\textheight][s]{\columnwidth}
      \begin{itemize}
        \item<1-> \funcname{caml\_stash\_backtrace} scans the older part of the stack, up to the next trap
        \item<6-> Controls jumps to the nested exception handler in \funcname{maybe\_raise}, which matches only on \typename{ExnB}
        \item<7-> \funcname{caml\_reraise\_exn} is called again, backtrace collection resumes with \funcname{caml\_stash\_backtrace}
      \end{itemize}
      \medskip
      \begin{itemize}
        \item<2-> Constructed backtrace:
        \begin{itemize}
          \item <2-> \funcname{definitely\_raise}
          \item <2-> \funcname{probably\_raise}
          \item <3-> \funcname{probably\_raise} (re-raise)
          \item <4-> \funcname{maybe\_raise}
          \item <5-> \funcname{main}
          \item <8-> \funcname{main} (re-raise)
        \end{itemize}
      \end{itemize}
      \vfill
      \onslide*<1-5>{\mintocaml[firstline=15,lastline=21]{../src/backtrace/backtrace.ml}}
      \onslide*<6>{\mintocaml[firstline=28,lastline=34,highlightlines=31]{../src/backtrace/backtrace.ml}}
      \onslide*<7->{\mintocaml[firstline=28,lastline=34,highlightlines=33]{../src/backtrace/backtrace.ml}}
      \endminipage
    \end{column}
    \begin{column}{0.45\textwidth}
      \raggedleft
      \onslide*<1>{\providecommand\step{6}\input{diagrams/backtrace/backtrace.tex}}%
      \onslide*<2>{\providecommand\step{6}\input{diagrams/backtrace/backtrace.tex}}%
      \onslide*<3>{\providecommand\step{7}\input{diagrams/backtrace/backtrace.tex}}%
      \onslide*<4>{\providecommand\step{8}\input{diagrams/backtrace/backtrace.tex}}%
      \onslide*<5>{\providecommand\step{9}\input{diagrams/backtrace/backtrace.tex}}%
      \onslide*<6>{\providecommand\step{10}\input{diagrams/backtrace/backtrace.tex}}%
      \onslide*<7>{\providecommand\step{11}\input{diagrams/backtrace/backtrace.tex}}%
      \onslide*<8>{\providecommand\step{12}\input{diagrams/backtrace/backtrace.tex}}%
      \onslide*<9>{\providecommand\step{13}\input{diagrams/backtrace/backtrace.tex}}%
    \end{column}
  \end{columns}
\end{frame}

\begin{frame}{Backtraces}{Printing the backtrace}
  \begin{columns}[c]
    \begin{column}{0.45\textwidth}
      \minipage[c][0.66\textheight][s]{\columnwidth}
      \begin{itemize}
        \item<1-> The exception backtrace can now be printed to \funcarg{stdout} with \funcname{Printexc.print_backtrace}
        \item<2-> First the exception backtrace is retrieved into an OCaml array with \funcname{get_raw_backtrace }
        \item<3-> Which is implemented as the \funcname{caml_get_exception_raw_backtrace} C function in the runtime
        \item<4-> And finally printed with \funcname{print_exception_backtrace}
      \end{itemize}
      \vfill
      \mintocaml[firstline=28,lastline=34,highlightlines=33]{../src/backtrace/backtrace.ml}
      \endminipage
    \end{column}
    \begin{column}{0.55\textwidth}
      \onslide*<1,2>{
        \mintocaml[firstline=100,lastline=101]{../ocaml/stdlib/printexc.ml}
      }
      \only<1>{
        \listocaml[firstline=170,lastline=172]{../ocaml/stdlib/printexc.ml}{stdlib/printexc.ml:170}
      }
      \only<2>{
        \listocaml[firstline=170,lastline=172,highlightlines=172]{../ocaml/stdlib/printexc.ml}{stdlib/printexc.ml:170}
      }
      \only<3>{
        \mintc[firstline=154,lastline=156]{../ocaml/runtime/backtrace.c}
        \listc[firstline=171,lastline=192,highlightlines={184-187}]{../ocaml/runtime/backtrace.c}{runtime/backtrace.c:154}
      }
      \only<4>{
        \listocaml[firstline=155,lastline=165]{../ocaml/stdlib/printexc.ml}{stdlib/printexc.ml:55}
      }
    \end{column}
  \end{columns}
\end{frame}


\subsection{The multiple kinds of raise}
\frameSubsection{}{}

\begin{frame}{Backtraces}{When backtraces are not needed}
  \begin{columns}[c]
    \begin{column}{0.45\textwidth}
      \minipage[c][0.66\textheight][s]{\columnwidth}
      \begin{itemize}
        \item<1-> What if the backtrace for an exception is never needed?
        \item<2-> It's possible to raise the exception explicitly with \funcname{raise\_notrace} to make raising as cheap as possible
        \item<3-> An example of use in the \funcname{dynlink} library of the OCaml compiler
      \end{itemize}
      \vfill
      \onslide<3->{
      \listocaml[firstline=205,lastline=209,highlightlines=207]{../ocaml/otherlibs/dynlink/dynlink\_compilerlibs/misc.ml}{otherlibs/dynlink/dynlink\_compilerlibs/misc.ml:205}
      }
      \endminipage
    \end{column}
    \begin{column}{0.55\textwidth}
    \only<4>{
      \mintcmm[highlightlines=3]{../src/notrace/camlNotrace\_\_fun_347.cmm}
      \listcmm{../src/notrace/camlNotrace\_\_all\_somes\_327.cmm}{all\_somes}
    }
    \onslide*<5->{
      \listobjdump[highlightlines={6-9}]{../src/notrace/camlNotrace\_\_fun_347.objdump}{anonymous function from \funcname{all\_somes}}
    }
    \end{column}
  \end{columns}
\end{frame}

\begin{frame}{Backtraces}{Summary table of the multiple kinds of raise}
  \begin{itemize}
    \item When debugging information is enabled and if the backtrace of an exception isn't needed, it's possible to raise with \funcname{raise\_notrace} for performance
    \item When debugging information is \emph{not} enabled, the compiler optimises \funcname{raise} calls to inline assembly (same effect as using \funcname{raise\_notrace})
  \end{itemize}
  \bigskip
  \centering
  \begin{tabular}{| l | c | c | c | c |}
    \hline
    \parbox{10em}{Debug information \\ (\funcarg{-g} flag)}  & \multicolumn{2}{|c|}{Disabled}                                     & \multicolumn{2}{|c|}{Enabled} \\
    \hline
    Raise kind                                               & \funcname{raise} & \funcname{raise\_notrace}                       & \funcname{raise} & \funcname{raise\_notrace} \\
    \hline
    Compilation                                              & \parbox{8em}{inline assembly\\ (optimization)} & inline assembly   & \funcname{caml\_raise\_exn} & inline assembly \\
    \hline
  \end{tabular}
\end{frame}


%
% Takeaway #5
%
\subsection*{Takeaway \#5}
\frameSubsectionTakeaway{}
\againframe<5>{takeaway}
}
    \end{column}
  \end{columns}
\end{frame}

\begin{frame}{Backtraces}{\funcname{caml\_probably\_raise}'s handler doesn't catch \typename{ExnC}}
  \begin{columns}[c]
    \begin{column}{0.45\textwidth}
      \minipage[c][0.75\textheight][s]{\columnwidth}
      \begin{itemize}
        \item<1-> \funcname{match \dots with}\footnotemark pattern matching can also match exceptions
        \item<1-> The CMM of \funcname{probably\_raise} shows that this \funcname{match \dots with} is implemented with a pattern matching and a \funcname{try \dots with}
        \item<1-> But this one only matches on \typename{ExnA} exception
        \item<2-> Since the pattern matching fails to catch the exception, \funcname{reraise} is called with our current \typename{ExnC} exception
        \item<3-> Disassembly shows that \funcname{reraise} is actually a call to \funcname{caml\_reraise\_exn}, which is also an assembly function provided by the runtime
      \end{itemize}
      \vfill
      \mintocaml[firstline=15,lastline=21,highlightlines={18-21}]{../src/backtrace/backtrace.ml}
      \endminipage
    \end{column}
    \begin{column}{0.55\textwidth}
      \centering
      \onslide*<1>{\mintcmm[highlightlines={10-18}]{../src/backtrace/camlBacktrace\_\_probably\_raise\_351.cmm}}
      \onslide*<2>{\listcmm[highlightlines=18]{../src/backtrace/camlBacktrace\_\_probably\_raise_351.cmm}{CMM of probably\_raise}}
      \onslide*<3>{\mintobjdump[firstline=5,lastline=35,highlightlines=31]{../src/backtrace/camlBacktrace\_\_probably\_raise_351.objdump}}
    \end{column}
  \end{columns}
  \only<1>{\footnotetext[1]{https://blog.janestreet.com/pattern-matching-and-exception-handling-unite/}}
\end{frame}

\newcommand\listCamlReraiseExn[1][]{
  \listasm[firstline=727,lastline=734,#1]{../ocaml/runtime/amd64.S}{runtime/amd64.S:727}}

\begin{frame}{Backtraces}{Introducing \funcname{caml\_reraise\_exn}}
  \begin{columns}[c]
    \begin{column}{0.5\textwidth}
      \minipage[c][0.66\textheight][s]{\columnwidth}
      \begin{itemize}
        \item<1-> The exception have to be reraised to the next handler
        \item<2-> First, checks if the backtrace should be collected with \funcarg{domain\_state->backtrace\_active}
        \only<3>{
        \begin{itemize}
            \item If not, behaves like a \funcname{raise\_notrace} with \funcname{RESTORE\_EXN\_HANDLER\_OCAML}
        \end{itemize}
        }
        \item<4-> Jumps into \funcname{caml\_raise\_exn}
        \item<5-> Calls \funcname{caml\_stash\_backtrace} again, to scan the next part of the stack: from the trap just pop-ed to the next trap
      \end{itemize}
      \vfill
      \mintocaml[firstline=15,lastline=21,highlightlines={18-21}]{../src/backtrace/backtrace.ml}
      \endminipage
    \end{column}
    \begin{column}{0.5\textwidth}
      \raggedleft
      \onslide*<1>{\listCamlReraiseExn{}}
      \onslide*<2>{\listCamlReraiseExn[highlightlines=729]{}}
      \onslide*<3>{
        \mintc[firstline=190,lastline=192,highlightlines={191-192}]{../ocaml/runtime/amd64.S}
        \mintc[firstline=234,lastline=237,highlightlines={235-237}]{../ocaml/runtime/amd64.S}
        \medskip
        \listCamlReraiseExn[highlightlines=731]{}
      }
      \onslide*<4-5>{
        % TODO refactor with \listCamlReraiseExn
        \mintasm[firstline=727,lastline=734,highlightlines=730]{../ocaml/runtime/amd64.S}
        \medskip
      }
      \onslide*<4>{\listCamlRaiseExn[highlightlines=711]{}}
      \onslide*<5>{\listCamlRaiseExn[highlightlines=720]{}}
    \end{column}
  \end{columns}
\end{frame}

\begin{frame}{Backtraces}{Backtrace collection continues}
  \begin{columns}[c]
    \begin{column}{0.55\textwidth}
      \minipage[c][0.75\textheight][s]{\columnwidth}
      \begin{itemize}
        \item<1-> \funcname{caml\_stash\_backtrace} scans the older part of the stack, up to the next trap
        \item<6-> Controls jumps to the nested exception handler in \funcname{maybe\_raise}, which matches only on \typename{ExnB}
        \item<7-> \funcname{caml\_reraise\_exn} is called again, backtrace collection resumes with \funcname{caml\_stash\_backtrace}
      \end{itemize}
      \medskip
      \begin{itemize}
        \item<2-> Constructed backtrace:
        \begin{itemize}
          \item <2-> \funcname{definitely\_raise}
          \item <2-> \funcname{probably\_raise}
          \item <3-> \funcname{probably\_raise} (re-raise)
          \item <4-> \funcname{maybe\_raise}
          \item <5-> \funcname{main}
          \item <8-> \funcname{main} (re-raise)
        \end{itemize}
      \end{itemize}
      \vfill
      \onslide*<1-5>{\mintocaml[firstline=15,lastline=21]{../src/backtrace/backtrace.ml}}
      \onslide*<6>{\mintocaml[firstline=28,lastline=34,highlightlines=31]{../src/backtrace/backtrace.ml}}
      \onslide*<7->{\mintocaml[firstline=28,lastline=34,highlightlines=33]{../src/backtrace/backtrace.ml}}
      \endminipage
    \end{column}
    \begin{column}{0.45\textwidth}
      \raggedleft
      \onslide*<1>{\providecommand\step{6}%
% Backtraces
%
\subsection{One advantage of raising with debugging information}
\frameSubsection{}{}

\begin{frame}{Backtraces}{Debugging information \& source code}
  \begin{columns}[c]
    \begin{column}{0.5\textwidth}
      \begin{itemize}
        \item Raising an exception is fun and games, but how do we know where the exception originated? $\rightarrow$ With the backtrace associated with the exception!
        \item In order to get a backtrace from the point where an exception was raised, it's necessary to compile with debugging information enabled (\funcarg{-g} flag for the compiler)
        \item Same example as in the `Nested handlers' section
      \end{itemize}
    \end{column}
    \begin{column}{0.5\textwidth}
      \listocaml[firstline=9,lastline=34]{../src/backtrace/backtrace.ml}{backtrace/backtrace.ml}
    \end{column}
  \end{columns}
\end{frame}

\begin{frame}{Backtraces}{CMM and disassembly of \funcname{definitely\_raise}}
  \begin{columns}[c]
    \begin{column}{0.5\textwidth}
      \minipage[c][0.66\textheight][s]{\columnwidth}
      \begin{itemize}
        \item<1-> An effect of compiling with debugging information (\funcarg{-g}): the CMM no longer uses \funcname{raise\_notrace} to raise the exception but \funcname{raise} instead
        \item<2-> Disassembly shows that CMM \funcname{raise} is actually a call to \funcname{caml\_raise\_exn}, a function in assembler provided by the runtime
      \end{itemize}
      \vfill
      \mintocaml[firstline=9,lastline=13,highlightlines=12]{../src/backtrace/backtrace.ml}
      \endminipage
    \end{column}
    \begin{column}{0.5\textwidth}
      \onslide*<1>{
        \mintcmm[highlightlines=5]{../src/backtrace/camlBacktrace\_\_definitely\_raise\_347.cmm}
      }
      \onslide*<2>{
        \mintobjdump[firstline=2,highlightlines=14]{../src/backtrace/camlBacktrace\_\_definitely\_raise\_347.objdump}
      }
    \end{column}
  \end{columns}
\end{frame}

\begin{frame}{Backtraces}{Introducing \funcname{caml\_raise\_exn}}
  \begin{columns}[c]
    \begin{column}{0.5\textwidth}
      \minipage[c][0.66\textheight][s]{\columnwidth}
      \onslide*<1>{
        \begin{itemize}
          \item Raising the exception is performed using \funcname{caml\_raise\_exn} which is
            \begin{itemize}
              \item An assembly function provided by the runtime
              \item Architecture specific
            \end{itemize}
          \item Let's see what it does differently from \funcname{raise\_notrace}\dots
          \item We are about to raise \typename{ExnC} with \funcname{caml\_raise\_exn} from \funcname{definitely\_raise}
        \end{itemize}
      }
      \onslide*<2>{
        \mintobjdump[firstline=2,highlightlines=14]{../src/backtrace/camlBacktrace\_\_definitely\_raise\_347.objdump}
      }
      \vfill
      \mintocaml[firstline=9,lastline=13,highlightlines=12]{../src/backtrace/backtrace.ml}
      \endminipage
    \end{column}
    \begin{column}{0.5\textwidth}
      \centering
      \providecommand\step{1}\input{diagrams/backtrace/backtrace.tex}
    \end{column}
  \end{columns}
\end{frame}

\begin{frame}{Backtraces}{But before that, we need more fields from \typename{caml\_domain\_state}}
  \begin{columns}
    \begin{column}{0.5\textwidth}
      \begin{itemize}
        \item Remember \typename{caml\_domain\_state} the per-domain runtime state?
        \item Some other members are of interest to us now\dots
        \item \localname{backtrace\_active} is used to know if the runtime should collect backtraces
        \item \localname{backtrace\_active} can be set by
          \begin{itemize}
            \item The \funcarg{b} flag in the \funcname{OCAMLRUNPARAM} environment variable
            \item \mintinline{ocaml}{Printexc.record_backtrace : bool -> unit} \footnotemark
          \end{itemize}
      \end{itemize}
    \end{column}
    \begin{column}{0.5\textwidth}
      \begin{listing}
        \mintc[highlightlines={9-11}]{src/ocaml/domain\_state\_backtrace.h}
        \caption{runtime/caml/domain\_state.h}
      \end{listing}
    \end{column}
  \end{columns}
  \footnotetext[1]{https://v2.ocaml.org/api/Printexc.html}
\end{frame}

\newcommand\listCamlRaiseExn[1][]{
  \listasm[firstline=702,lastline=725,#1]{../ocaml/runtime/amd64.S}{runtime/amd64.S:702}}

\begin{frame}{Backtraces}{Walk through \funcname{caml\_raise\_exn}}
  \begin{columns}[T]
    \begin{column}{0.5\textwidth}
      \begin{itemize}
        \item<1-> First checks whether exceptions backtrace should be recorded
        \item<1-> By checking the \funcarg{domain\_state->backtrace\_active} value
        \item<2-> If not, acts as \funcname{raise\_notrace} with \funcname{RESTORE\_EXN\_HANDLER\_OCAML}
          \begin{itemize}
            \item Set \regname{RSP} to \localname{domain\_state->exn\_handler} value
            \item Restore \localname{domain\_state->exn\_handler} from trap
            \item Jump with \funcname{ret} (same effect as using \funcname{pop} and \funcname{jmp})
          \end{itemize}
        \item<3-> Otherwise, switches to the C stack to be able to execute C code
        \item<4-> Calls \funcname{caml\_stash\_backtrace}, a C function provided by the runtime\ignorespaces
          \onslide<6->{, with
            \begin{itemize}
              \item \funcarg{exn}: exception value
              \item \funcarg{pc}: code address of the raising point
              \item \funcarg{sp}: stack address of the raising function
              \item \funcarg{trapsp}: stack address of the next handler
            \end{itemize}
          }
      \end{itemize}
      \bigskip
      \mintocaml[firstline=9,lastline=13,highlightlines=12]{../src/backtrace/backtrace.ml}
    \end{column}
    \begin{column}{0.5\textwidth}
      \raggedleft
      \onslide*<1,3-4>{
        \mintc[firstline=190,lastline=192]{../ocaml/runtime/amd64.S}
        \mintc[firstline=234,lastline=237]{../ocaml/runtime/amd64.S}
      }
      \onslide*<2>{
        \mintc[firstline=190,lastline=192,highlightlines={191-192}]{../ocaml/runtime/amd64.S}
        \mintc[firstline=234,lastline=237,highlightlines={235-237}]{../ocaml/runtime/amd64.S}
      }
      \onslide*<1>{\listCamlRaiseExn[highlightlines=705]{}}%
      \onslide*<2>{\listCamlRaiseExn[highlightlines={707-708}]{}}%
      \onslide*<3>{\listCamlRaiseExn[highlightlines={712-714}]{}}%
      \onslide*<4>{\listCamlRaiseExn[highlightlines={715-720}]{}}%
      \onslide*<5>{\providecommand\step{1}\input{diagrams/backtrace/backtrace.tex}}%
      \onslide*<6>{\providecommand\step{2}\input{diagrams/backtrace/backtrace.tex}}%
    \end{column}
  \end{columns}
\end{frame}

\newcommand\listStashBacktrace[1][]{
  \listc[firstline=91,lastline=120,#1]{../ocaml/runtime/backtrace\_nat.c}{runtime/backtrace\_nat.c:91}}

\begin{frame}{Backtraces}{Introducing \funcname{caml\_stash\_backtrace}}
  \begin{columns}[c]
    \begin{column}{0.5\textwidth}
      \minipage[c][0.66\textheight][s]{\columnwidth}
      \begin{itemize}
        \item<1-> A C function provided by the runtime (specific to native code)
        \item<2-> It scans the stack, between the stack pointer at the raising point up to the next trap, in order to collect the names of the detected function frames
          \onslide*<2>{
            \begin{itemize}
              \item And more information, like line number, column number...
            \end{itemize}
          }
        \item<3-> It uses the same technique and information as the Garbage Collector (GC) uses to scan the values live on the stack
          \onslide*<4>{
            \begin{itemize}
              \item \typename{frame\_descr} are at the core of stack unwinding
            \end{itemize}
          }
      \end{itemize}
      \vfill
      \mintocaml[firstline=9,lastline=13,highlightlines=12]{../src/backtrace/backtrace.ml}
      \endminipage
    \end{column}
    \begin{column}{0.5\textwidth}
      \onslide*<1>{\listStashBacktrace[highlightlines=91]{}}
      \onslide*<2>{\listStashBacktrace[highlightlines={91,117-118}]{}}
      \onslide*<3->{\listStashBacktrace[highlightlines={94,106,109-110}]{}}
    \end{column}
  \end{columns}
\end{frame}

\newcommand\listFrameDescr[1][]{
  \listocaml[firstline=111,lastline=124,#1]{../ocaml/asmcomp/emitaux.ml}{asmcomp/emitaux.ml:111}}
\newcommand\listDebugInfo[1][]{
  \listocaml[firstline=38,lastline=49,#1]{../ocaml/lambda/debuginfo.mli}{lambda/debuginfo.mli:38}}

\begin{frame}{Backtraces}{\typename{frame\_descr} and \typename{frame\_debuginfo} in a nutshell}
  \begin{columns}[c]
    \begin{column}{0.5\textwidth}
      \begin{itemize}
        \item<1-> For every return address in an OCaml program, there is a associated \typename{frame\_descr}
        \item<2-> A \typename{frame\_descr} specifies
        \begin{itemize}
          \item<2-> The size of the function stack frame
          \item<3-> Where live values are on the stack (so that the GC can mark them as live and prevent from being swept)
        \end{itemize}
        \item<4-> When compiled with debug information, \typename{frame\_descr} will be linked to a \typename{frame\_debuginfo}
        \begin{itemize}
          \item<5-> Contains information about the position in the source code (file name, line and column number)
        \end{itemize}
      \end{itemize}
    \end{column}
    \begin{column}{0.5\textwidth}
        \onslide*<1>{\listFrameDescr[highlightlines={118-122}]{}}
        \onslide*<2>{\listFrameDescr[highlightlines={120}]{}}
        \onslide*<3>{\listFrameDescr[highlightlines={121}]{}}
        \onslide*<4->{\listFrameDescr[highlightlines={122}]{}}
        \medskip
        \onslide*<1-3>{\listDebugInfo{}}
        \onslide*<4>{\listDebugInfo[highlightlines={38,49}]{}}
        \onslide*<5>{\listDebugInfo[highlightlines={39-46}]{}}
    \end{column}
  \end{columns}
\end{frame}

\begin{frame}{Backtraces}{Scanning the stack with \typename{frame\_descr}}
  \begin{columns}[c]
    \begin{column}{0.5\textwidth}
      \begin{itemize}
        \item Given the return address of the code that raises the exception by calling \funcname{caml\_raise\_exn} (\funcarg{pc} argument of \funcname{caml\_stash\_backtrace})
        \item And the position of the stack pointer (\funcarg{sp} argument of \funcname{caml\_stash\_backtrace})
        \item The frame size given by \typename{frame\_descr}.\funcarg{frame\_size}, allows to find the end of the previous frame
        \item And the return address of the caller of this function
        \bigskip
        \onslide<3->{
        \item Note that traps (here the reddish \funcname{ExnA}, \funcname{ExnB} and \funcname{ExnC} blocks) belong to the frame that installed them, and so the frame size changes when traps are removed
        }
      \end{itemize}
    \end{column}
    \begin{column}{0.5\textwidth}
      \raggedleft
      \onslide*<1>{\providecommand\step{2}\input{diagrams/backtrace/backtrace.tex}}%
      \onslide*<2>{\providecommand\step{1}\input{diagrams/backtrace/scanstack.tex}}%
      \onslide*<3>{\providecommand\step{2}\input{diagrams/backtrace/scanstack.tex}}%
      \onslide*<4>{\providecommand\step{3}\input{diagrams/backtrace/scanstack.tex}}%
      \onslide*<5>{\providecommand\step{4}\input{diagrams/backtrace/scanstack.tex}}%
      \onslide*<6>{\providecommand\step{5}\input{diagrams/backtrace/scanstack.tex}}%
    \end{column}
  \end{columns}
\end{frame}

% Duplicate of previous-previous slide
\begin{frame}{Backtraces}{Continuing on \funcname{caml\_stash\_backtrace}}
  \begin{columns}[c]
    \begin{column}{0.5\textwidth}
      \minipage[c][0.75\textheight][s]{\columnwidth}
      \begin{itemize}
          \color{gray}
        \item A C function provided by the runtime (specific to native code)
        \item It scans the stack, between the stack pointer at the raising point up to the next trap, in order to collect the names of the detected function frames
        \item It uses the same technique and information as the Garbage Collector (GC) uses to scan the values live on the stack
      \end{itemize}
      \begin{itemize}
        \item For each function frame detected, store a pointer to the \typename{frame\_descr} in the \localname{domain\_state->backtrace\_buffer} array, effectively constructing the backtrace
      \end{itemize}
      \onslide<2->{
        \begin{itemize}
            \smallskip
          \item Constructed backtrace:
            \funcname{definitely\_raise},
            \onslide<3->{\funcname{probably\_raise}}
        \end{itemize}
      }
      \vfill
      \mintocaml[firstline=9,lastline=13,highlightlines=12]{../src/backtrace/backtrace.ml}
      \endminipage
    \end{column}
    \begin{column}{0.5\textwidth}
      \raggedleft
      \onslide*<1>{\listStashBacktrace[highlightlines={114-115}]{}}
      \onslide*<2>{\providecommand\step{3}\input{diagrams/backtrace/backtrace.tex}}%
      \onslide*<3>{\providecommand\step{4}\input{diagrams/backtrace/backtrace.tex}}%
    \end{column}
  \end{columns}
\end{frame}

\begin{frame}{Backtraces}{Back to \funcname{caml\_raise\_exn}}
  \begin{columns}[c]
    \begin{column}{0.5\textwidth}
      \minipage[c][0.66\textheight][s]{\columnwidth}
      \begin{itemize}\color{gray}
        \item Checks wheteher exceptions backtrace should be recorded, by checking the \funcarg{domain\_state->backtrace\_active} value
        \item Switches to the C stack to be able to execute C code
        \item Calls \funcname{caml\_stash\_backtrace}, to collect the backtrace up to the next trap
      \end{itemize}
      \onslide*<2->{
        \begin{itemize}
          \item<2-> Executes the exception handler in \funcname{probably\_raise} with \funcname{RESTORE\_EXN\_HANDLER\_OCAML}
            \begin{itemize}
              \item Set \regname{RSP} to \localname{domain\_state->exn\_handler} value
              \item Restore \localname{domain\_state->exn\_handler} from trap
              \item Jump to the handler code with \funcname{ret}
            \end{itemize}
        \end{itemize}
      }
      \vfill
      \onslide*<1>{\mintocaml[firstline=9,lastline=13,highlightlines=12]{../src/backtrace/backtrace.ml}}
      \onslide*<2->{\mintocaml[firstline=15,lastline=21,highlightlines={19-21}]{../src/backtrace/backtrace.ml}}
      \endminipage
    \end{column}
    \begin{column}{0.5\textwidth}
      \centering
      \onslide*<1>{
        \mintc[firstline=190,lastline=192]{../ocaml/runtime/amd64.S}
        \mintc[firstline=234,lastline=237]{../ocaml/runtime/amd64.S}
      }
      \onslide*<2>{
        \mintc[firstline=190,lastline=192,highlightlines={191-192}]{../ocaml/runtime/amd64.S}
        \mintc[firstline=234,lastline=237,highlightlines={235-237}]{../ocaml/runtime/amd64.S}
      }
      \onslide*<1>{\listCamlRaiseExn[highlightlines=720]{}}
      \onslide*<2>{\listCamlRaiseExn[highlightlines=722]{}}
      \onslide*<3->{\providecommand\step{5}\input{diagrams/backtrace/backtrace.tex}}
    \end{column}
  \end{columns}
\end{frame}

\begin{frame}{Backtraces}{\funcname{caml\_probably\_raise}'s handler doesn't catch \typename{ExnC}}
  \begin{columns}[c]
    \begin{column}{0.45\textwidth}
      \minipage[c][0.75\textheight][s]{\columnwidth}
      \begin{itemize}
        \item<1-> \funcname{match \dots with}\footnotemark pattern matching can also match exceptions
        \item<1-> The CMM of \funcname{probably\_raise} shows that this \funcname{match \dots with} is implemented with a pattern matching and a \funcname{try \dots with}
        \item<1-> But this one only matches on \typename{ExnA} exception
        \item<2-> Since the pattern matching fails to catch the exception, \funcname{reraise} is called with our current \typename{ExnC} exception
        \item<3-> Disassembly shows that \funcname{reraise} is actually a call to \funcname{caml\_reraise\_exn}, which is also an assembly function provided by the runtime
      \end{itemize}
      \vfill
      \mintocaml[firstline=15,lastline=21,highlightlines={18-21}]{../src/backtrace/backtrace.ml}
      \endminipage
    \end{column}
    \begin{column}{0.55\textwidth}
      \centering
      \onslide*<1>{\mintcmm[highlightlines={10-18}]{../src/backtrace/camlBacktrace\_\_probably\_raise\_351.cmm}}
      \onslide*<2>{\listcmm[highlightlines=18]{../src/backtrace/camlBacktrace\_\_probably\_raise_351.cmm}{CMM of probably\_raise}}
      \onslide*<3>{\mintobjdump[firstline=5,lastline=35,highlightlines=31]{../src/backtrace/camlBacktrace\_\_probably\_raise_351.objdump}}
    \end{column}
  \end{columns}
  \only<1>{\footnotetext[1]{https://blog.janestreet.com/pattern-matching-and-exception-handling-unite/}}
\end{frame}

\newcommand\listCamlReraiseExn[1][]{
  \listasm[firstline=727,lastline=734,#1]{../ocaml/runtime/amd64.S}{runtime/amd64.S:727}}

\begin{frame}{Backtraces}{Introducing \funcname{caml\_reraise\_exn}}
  \begin{columns}[c]
    \begin{column}{0.5\textwidth}
      \minipage[c][0.66\textheight][s]{\columnwidth}
      \begin{itemize}
        \item<1-> The exception have to be reraised to the next handler
        \item<2-> First, checks if the backtrace should be collected with \funcarg{domain\_state->backtrace\_active}
        \only<3>{
        \begin{itemize}
            \item If not, behaves like a \funcname{raise\_notrace} with \funcname{RESTORE\_EXN\_HANDLER\_OCAML}
        \end{itemize}
        }
        \item<4-> Jumps into \funcname{caml\_raise\_exn}
        \item<5-> Calls \funcname{caml\_stash\_backtrace} again, to scan the next part of the stack: from the trap just pop-ed to the next trap
      \end{itemize}
      \vfill
      \mintocaml[firstline=15,lastline=21,highlightlines={18-21}]{../src/backtrace/backtrace.ml}
      \endminipage
    \end{column}
    \begin{column}{0.5\textwidth}
      \raggedleft
      \onslide*<1>{\listCamlReraiseExn{}}
      \onslide*<2>{\listCamlReraiseExn[highlightlines=729]{}}
      \onslide*<3>{
        \mintc[firstline=190,lastline=192,highlightlines={191-192}]{../ocaml/runtime/amd64.S}
        \mintc[firstline=234,lastline=237,highlightlines={235-237}]{../ocaml/runtime/amd64.S}
        \medskip
        \listCamlReraiseExn[highlightlines=731]{}
      }
      \onslide*<4-5>{
        % TODO refactor with \listCamlReraiseExn
        \mintasm[firstline=727,lastline=734,highlightlines=730]{../ocaml/runtime/amd64.S}
        \medskip
      }
      \onslide*<4>{\listCamlRaiseExn[highlightlines=711]{}}
      \onslide*<5>{\listCamlRaiseExn[highlightlines=720]{}}
    \end{column}
  \end{columns}
\end{frame}

\begin{frame}{Backtraces}{Backtrace collection continues}
  \begin{columns}[c]
    \begin{column}{0.55\textwidth}
      \minipage[c][0.75\textheight][s]{\columnwidth}
      \begin{itemize}
        \item<1-> \funcname{caml\_stash\_backtrace} scans the older part of the stack, up to the next trap
        \item<6-> Controls jumps to the nested exception handler in \funcname{maybe\_raise}, which matches only on \typename{ExnB}
        \item<7-> \funcname{caml\_reraise\_exn} is called again, backtrace collection resumes with \funcname{caml\_stash\_backtrace}
      \end{itemize}
      \medskip
      \begin{itemize}
        \item<2-> Constructed backtrace:
        \begin{itemize}
          \item <2-> \funcname{definitely\_raise}
          \item <2-> \funcname{probably\_raise}
          \item <3-> \funcname{probably\_raise} (re-raise)
          \item <4-> \funcname{maybe\_raise}
          \item <5-> \funcname{main}
          \item <8-> \funcname{main} (re-raise)
        \end{itemize}
      \end{itemize}
      \vfill
      \onslide*<1-5>{\mintocaml[firstline=15,lastline=21]{../src/backtrace/backtrace.ml}}
      \onslide*<6>{\mintocaml[firstline=28,lastline=34,highlightlines=31]{../src/backtrace/backtrace.ml}}
      \onslide*<7->{\mintocaml[firstline=28,lastline=34,highlightlines=33]{../src/backtrace/backtrace.ml}}
      \endminipage
    \end{column}
    \begin{column}{0.45\textwidth}
      \raggedleft
      \onslide*<1>{\providecommand\step{6}\input{diagrams/backtrace/backtrace.tex}}%
      \onslide*<2>{\providecommand\step{6}\input{diagrams/backtrace/backtrace.tex}}%
      \onslide*<3>{\providecommand\step{7}\input{diagrams/backtrace/backtrace.tex}}%
      \onslide*<4>{\providecommand\step{8}\input{diagrams/backtrace/backtrace.tex}}%
      \onslide*<5>{\providecommand\step{9}\input{diagrams/backtrace/backtrace.tex}}%
      \onslide*<6>{\providecommand\step{10}\input{diagrams/backtrace/backtrace.tex}}%
      \onslide*<7>{\providecommand\step{11}\input{diagrams/backtrace/backtrace.tex}}%
      \onslide*<8>{\providecommand\step{12}\input{diagrams/backtrace/backtrace.tex}}%
      \onslide*<9>{\providecommand\step{13}\input{diagrams/backtrace/backtrace.tex}}%
    \end{column}
  \end{columns}
\end{frame}

\begin{frame}{Backtraces}{Printing the backtrace}
  \begin{columns}[c]
    \begin{column}{0.45\textwidth}
      \minipage[c][0.66\textheight][s]{\columnwidth}
      \begin{itemize}
        \item<1-> The exception backtrace can now be printed to \funcarg{stdout} with \funcname{Printexc.print_backtrace}
        \item<2-> First the exception backtrace is retrieved into an OCaml array with \funcname{get_raw_backtrace }
        \item<3-> Which is implemented as the \funcname{caml_get_exception_raw_backtrace} C function in the runtime
        \item<4-> And finally printed with \funcname{print_exception_backtrace}
      \end{itemize}
      \vfill
      \mintocaml[firstline=28,lastline=34,highlightlines=33]{../src/backtrace/backtrace.ml}
      \endminipage
    \end{column}
    \begin{column}{0.55\textwidth}
      \onslide*<1,2>{
        \mintocaml[firstline=100,lastline=101]{../ocaml/stdlib/printexc.ml}
      }
      \only<1>{
        \listocaml[firstline=170,lastline=172]{../ocaml/stdlib/printexc.ml}{stdlib/printexc.ml:170}
      }
      \only<2>{
        \listocaml[firstline=170,lastline=172,highlightlines=172]{../ocaml/stdlib/printexc.ml}{stdlib/printexc.ml:170}
      }
      \only<3>{
        \mintc[firstline=154,lastline=156]{../ocaml/runtime/backtrace.c}
        \listc[firstline=171,lastline=192,highlightlines={184-187}]{../ocaml/runtime/backtrace.c}{runtime/backtrace.c:154}
      }
      \only<4>{
        \listocaml[firstline=155,lastline=165]{../ocaml/stdlib/printexc.ml}{stdlib/printexc.ml:55}
      }
    \end{column}
  \end{columns}
\end{frame}


\subsection{The multiple kinds of raise}
\frameSubsection{}{}

\begin{frame}{Backtraces}{When backtraces are not needed}
  \begin{columns}[c]
    \begin{column}{0.45\textwidth}
      \minipage[c][0.66\textheight][s]{\columnwidth}
      \begin{itemize}
        \item<1-> What if the backtrace for an exception is never needed?
        \item<2-> It's possible to raise the exception explicitly with \funcname{raise\_notrace} to make raising as cheap as possible
        \item<3-> An example of use in the \funcname{dynlink} library of the OCaml compiler
      \end{itemize}
      \vfill
      \onslide<3->{
      \listocaml[firstline=205,lastline=209,highlightlines=207]{../ocaml/otherlibs/dynlink/dynlink\_compilerlibs/misc.ml}{otherlibs/dynlink/dynlink\_compilerlibs/misc.ml:205}
      }
      \endminipage
    \end{column}
    \begin{column}{0.55\textwidth}
    \only<4>{
      \mintcmm[highlightlines=3]{../src/notrace/camlNotrace\_\_fun_347.cmm}
      \listcmm{../src/notrace/camlNotrace\_\_all\_somes\_327.cmm}{all\_somes}
    }
    \onslide*<5->{
      \listobjdump[highlightlines={6-9}]{../src/notrace/camlNotrace\_\_fun_347.objdump}{anonymous function from \funcname{all\_somes}}
    }
    \end{column}
  \end{columns}
\end{frame}

\begin{frame}{Backtraces}{Summary table of the multiple kinds of raise}
  \begin{itemize}
    \item When debugging information is enabled and if the backtrace of an exception isn't needed, it's possible to raise with \funcname{raise\_notrace} for performance
    \item When debugging information is \emph{not} enabled, the compiler optimises \funcname{raise} calls to inline assembly (same effect as using \funcname{raise\_notrace})
  \end{itemize}
  \bigskip
  \centering
  \begin{tabular}{| l | c | c | c | c |}
    \hline
    \parbox{10em}{Debug information \\ (\funcarg{-g} flag)}  & \multicolumn{2}{|c|}{Disabled}                                     & \multicolumn{2}{|c|}{Enabled} \\
    \hline
    Raise kind                                               & \funcname{raise} & \funcname{raise\_notrace}                       & \funcname{raise} & \funcname{raise\_notrace} \\
    \hline
    Compilation                                              & \parbox{8em}{inline assembly\\ (optimization)} & inline assembly   & \funcname{caml\_raise\_exn} & inline assembly \\
    \hline
  \end{tabular}
\end{frame}


%
% Takeaway #5
%
\subsection*{Takeaway \#5}
\frameSubsectionTakeaway{}
\againframe<5>{takeaway}
}%
      \onslide*<2>{\providecommand\step{6}%
% Backtraces
%
\subsection{One advantage of raising with debugging information}
\frameSubsection{}{}

\begin{frame}{Backtraces}{Debugging information \& source code}
  \begin{columns}[c]
    \begin{column}{0.5\textwidth}
      \begin{itemize}
        \item Raising an exception is fun and games, but how do we know where the exception originated? $\rightarrow$ With the backtrace associated with the exception!
        \item In order to get a backtrace from the point where an exception was raised, it's necessary to compile with debugging information enabled (\funcarg{-g} flag for the compiler)
        \item Same example as in the `Nested handlers' section
      \end{itemize}
    \end{column}
    \begin{column}{0.5\textwidth}
      \listocaml[firstline=9,lastline=34]{../src/backtrace/backtrace.ml}{backtrace/backtrace.ml}
    \end{column}
  \end{columns}
\end{frame}

\begin{frame}{Backtraces}{CMM and disassembly of \funcname{definitely\_raise}}
  \begin{columns}[c]
    \begin{column}{0.5\textwidth}
      \minipage[c][0.66\textheight][s]{\columnwidth}
      \begin{itemize}
        \item<1-> An effect of compiling with debugging information (\funcarg{-g}): the CMM no longer uses \funcname{raise\_notrace} to raise the exception but \funcname{raise} instead
        \item<2-> Disassembly shows that CMM \funcname{raise} is actually a call to \funcname{caml\_raise\_exn}, a function in assembler provided by the runtime
      \end{itemize}
      \vfill
      \mintocaml[firstline=9,lastline=13,highlightlines=12]{../src/backtrace/backtrace.ml}
      \endminipage
    \end{column}
    \begin{column}{0.5\textwidth}
      \onslide*<1>{
        \mintcmm[highlightlines=5]{../src/backtrace/camlBacktrace\_\_definitely\_raise\_347.cmm}
      }
      \onslide*<2>{
        \mintobjdump[firstline=2,highlightlines=14]{../src/backtrace/camlBacktrace\_\_definitely\_raise\_347.objdump}
      }
    \end{column}
  \end{columns}
\end{frame}

\begin{frame}{Backtraces}{Introducing \funcname{caml\_raise\_exn}}
  \begin{columns}[c]
    \begin{column}{0.5\textwidth}
      \minipage[c][0.66\textheight][s]{\columnwidth}
      \onslide*<1>{
        \begin{itemize}
          \item Raising the exception is performed using \funcname{caml\_raise\_exn} which is
            \begin{itemize}
              \item An assembly function provided by the runtime
              \item Architecture specific
            \end{itemize}
          \item Let's see what it does differently from \funcname{raise\_notrace}\dots
          \item We are about to raise \typename{ExnC} with \funcname{caml\_raise\_exn} from \funcname{definitely\_raise}
        \end{itemize}
      }
      \onslide*<2>{
        \mintobjdump[firstline=2,highlightlines=14]{../src/backtrace/camlBacktrace\_\_definitely\_raise\_347.objdump}
      }
      \vfill
      \mintocaml[firstline=9,lastline=13,highlightlines=12]{../src/backtrace/backtrace.ml}
      \endminipage
    \end{column}
    \begin{column}{0.5\textwidth}
      \centering
      \providecommand\step{1}\input{diagrams/backtrace/backtrace.tex}
    \end{column}
  \end{columns}
\end{frame}

\begin{frame}{Backtraces}{But before that, we need more fields from \typename{caml\_domain\_state}}
  \begin{columns}
    \begin{column}{0.5\textwidth}
      \begin{itemize}
        \item Remember \typename{caml\_domain\_state} the per-domain runtime state?
        \item Some other members are of interest to us now\dots
        \item \localname{backtrace\_active} is used to know if the runtime should collect backtraces
        \item \localname{backtrace\_active} can be set by
          \begin{itemize}
            \item The \funcarg{b} flag in the \funcname{OCAMLRUNPARAM} environment variable
            \item \mintinline{ocaml}{Printexc.record_backtrace : bool -> unit} \footnotemark
          \end{itemize}
      \end{itemize}
    \end{column}
    \begin{column}{0.5\textwidth}
      \begin{listing}
        \mintc[highlightlines={9-11}]{src/ocaml/domain\_state\_backtrace.h}
        \caption{runtime/caml/domain\_state.h}
      \end{listing}
    \end{column}
  \end{columns}
  \footnotetext[1]{https://v2.ocaml.org/api/Printexc.html}
\end{frame}

\newcommand\listCamlRaiseExn[1][]{
  \listasm[firstline=702,lastline=725,#1]{../ocaml/runtime/amd64.S}{runtime/amd64.S:702}}

\begin{frame}{Backtraces}{Walk through \funcname{caml\_raise\_exn}}
  \begin{columns}[T]
    \begin{column}{0.5\textwidth}
      \begin{itemize}
        \item<1-> First checks whether exceptions backtrace should be recorded
        \item<1-> By checking the \funcarg{domain\_state->backtrace\_active} value
        \item<2-> If not, acts as \funcname{raise\_notrace} with \funcname{RESTORE\_EXN\_HANDLER\_OCAML}
          \begin{itemize}
            \item Set \regname{RSP} to \localname{domain\_state->exn\_handler} value
            \item Restore \localname{domain\_state->exn\_handler} from trap
            \item Jump with \funcname{ret} (same effect as using \funcname{pop} and \funcname{jmp})
          \end{itemize}
        \item<3-> Otherwise, switches to the C stack to be able to execute C code
        \item<4-> Calls \funcname{caml\_stash\_backtrace}, a C function provided by the runtime\ignorespaces
          \onslide<6->{, with
            \begin{itemize}
              \item \funcarg{exn}: exception value
              \item \funcarg{pc}: code address of the raising point
              \item \funcarg{sp}: stack address of the raising function
              \item \funcarg{trapsp}: stack address of the next handler
            \end{itemize}
          }
      \end{itemize}
      \bigskip
      \mintocaml[firstline=9,lastline=13,highlightlines=12]{../src/backtrace/backtrace.ml}
    \end{column}
    \begin{column}{0.5\textwidth}
      \raggedleft
      \onslide*<1,3-4>{
        \mintc[firstline=190,lastline=192]{../ocaml/runtime/amd64.S}
        \mintc[firstline=234,lastline=237]{../ocaml/runtime/amd64.S}
      }
      \onslide*<2>{
        \mintc[firstline=190,lastline=192,highlightlines={191-192}]{../ocaml/runtime/amd64.S}
        \mintc[firstline=234,lastline=237,highlightlines={235-237}]{../ocaml/runtime/amd64.S}
      }
      \onslide*<1>{\listCamlRaiseExn[highlightlines=705]{}}%
      \onslide*<2>{\listCamlRaiseExn[highlightlines={707-708}]{}}%
      \onslide*<3>{\listCamlRaiseExn[highlightlines={712-714}]{}}%
      \onslide*<4>{\listCamlRaiseExn[highlightlines={715-720}]{}}%
      \onslide*<5>{\providecommand\step{1}\input{diagrams/backtrace/backtrace.tex}}%
      \onslide*<6>{\providecommand\step{2}\input{diagrams/backtrace/backtrace.tex}}%
    \end{column}
  \end{columns}
\end{frame}

\newcommand\listStashBacktrace[1][]{
  \listc[firstline=91,lastline=120,#1]{../ocaml/runtime/backtrace\_nat.c}{runtime/backtrace\_nat.c:91}}

\begin{frame}{Backtraces}{Introducing \funcname{caml\_stash\_backtrace}}
  \begin{columns}[c]
    \begin{column}{0.5\textwidth}
      \minipage[c][0.66\textheight][s]{\columnwidth}
      \begin{itemize}
        \item<1-> A C function provided by the runtime (specific to native code)
        \item<2-> It scans the stack, between the stack pointer at the raising point up to the next trap, in order to collect the names of the detected function frames
          \onslide*<2>{
            \begin{itemize}
              \item And more information, like line number, column number...
            \end{itemize}
          }
        \item<3-> It uses the same technique and information as the Garbage Collector (GC) uses to scan the values live on the stack
          \onslide*<4>{
            \begin{itemize}
              \item \typename{frame\_descr} are at the core of stack unwinding
            \end{itemize}
          }
      \end{itemize}
      \vfill
      \mintocaml[firstline=9,lastline=13,highlightlines=12]{../src/backtrace/backtrace.ml}
      \endminipage
    \end{column}
    \begin{column}{0.5\textwidth}
      \onslide*<1>{\listStashBacktrace[highlightlines=91]{}}
      \onslide*<2>{\listStashBacktrace[highlightlines={91,117-118}]{}}
      \onslide*<3->{\listStashBacktrace[highlightlines={94,106,109-110}]{}}
    \end{column}
  \end{columns}
\end{frame}

\newcommand\listFrameDescr[1][]{
  \listocaml[firstline=111,lastline=124,#1]{../ocaml/asmcomp/emitaux.ml}{asmcomp/emitaux.ml:111}}
\newcommand\listDebugInfo[1][]{
  \listocaml[firstline=38,lastline=49,#1]{../ocaml/lambda/debuginfo.mli}{lambda/debuginfo.mli:38}}

\begin{frame}{Backtraces}{\typename{frame\_descr} and \typename{frame\_debuginfo} in a nutshell}
  \begin{columns}[c]
    \begin{column}{0.5\textwidth}
      \begin{itemize}
        \item<1-> For every return address in an OCaml program, there is a associated \typename{frame\_descr}
        \item<2-> A \typename{frame\_descr} specifies
        \begin{itemize}
          \item<2-> The size of the function stack frame
          \item<3-> Where live values are on the stack (so that the GC can mark them as live and prevent from being swept)
        \end{itemize}
        \item<4-> When compiled with debug information, \typename{frame\_descr} will be linked to a \typename{frame\_debuginfo}
        \begin{itemize}
          \item<5-> Contains information about the position in the source code (file name, line and column number)
        \end{itemize}
      \end{itemize}
    \end{column}
    \begin{column}{0.5\textwidth}
        \onslide*<1>{\listFrameDescr[highlightlines={118-122}]{}}
        \onslide*<2>{\listFrameDescr[highlightlines={120}]{}}
        \onslide*<3>{\listFrameDescr[highlightlines={121}]{}}
        \onslide*<4->{\listFrameDescr[highlightlines={122}]{}}
        \medskip
        \onslide*<1-3>{\listDebugInfo{}}
        \onslide*<4>{\listDebugInfo[highlightlines={38,49}]{}}
        \onslide*<5>{\listDebugInfo[highlightlines={39-46}]{}}
    \end{column}
  \end{columns}
\end{frame}

\begin{frame}{Backtraces}{Scanning the stack with \typename{frame\_descr}}
  \begin{columns}[c]
    \begin{column}{0.5\textwidth}
      \begin{itemize}
        \item Given the return address of the code that raises the exception by calling \funcname{caml\_raise\_exn} (\funcarg{pc} argument of \funcname{caml\_stash\_backtrace})
        \item And the position of the stack pointer (\funcarg{sp} argument of \funcname{caml\_stash\_backtrace})
        \item The frame size given by \typename{frame\_descr}.\funcarg{frame\_size}, allows to find the end of the previous frame
        \item And the return address of the caller of this function
        \bigskip
        \onslide<3->{
        \item Note that traps (here the reddish \funcname{ExnA}, \funcname{ExnB} and \funcname{ExnC} blocks) belong to the frame that installed them, and so the frame size changes when traps are removed
        }
      \end{itemize}
    \end{column}
    \begin{column}{0.5\textwidth}
      \raggedleft
      \onslide*<1>{\providecommand\step{2}\input{diagrams/backtrace/backtrace.tex}}%
      \onslide*<2>{\providecommand\step{1}\input{diagrams/backtrace/scanstack.tex}}%
      \onslide*<3>{\providecommand\step{2}\input{diagrams/backtrace/scanstack.tex}}%
      \onslide*<4>{\providecommand\step{3}\input{diagrams/backtrace/scanstack.tex}}%
      \onslide*<5>{\providecommand\step{4}\input{diagrams/backtrace/scanstack.tex}}%
      \onslide*<6>{\providecommand\step{5}\input{diagrams/backtrace/scanstack.tex}}%
    \end{column}
  \end{columns}
\end{frame}

% Duplicate of previous-previous slide
\begin{frame}{Backtraces}{Continuing on \funcname{caml\_stash\_backtrace}}
  \begin{columns}[c]
    \begin{column}{0.5\textwidth}
      \minipage[c][0.75\textheight][s]{\columnwidth}
      \begin{itemize}
          \color{gray}
        \item A C function provided by the runtime (specific to native code)
        \item It scans the stack, between the stack pointer at the raising point up to the next trap, in order to collect the names of the detected function frames
        \item It uses the same technique and information as the Garbage Collector (GC) uses to scan the values live on the stack
      \end{itemize}
      \begin{itemize}
        \item For each function frame detected, store a pointer to the \typename{frame\_descr} in the \localname{domain\_state->backtrace\_buffer} array, effectively constructing the backtrace
      \end{itemize}
      \onslide<2->{
        \begin{itemize}
            \smallskip
          \item Constructed backtrace:
            \funcname{definitely\_raise},
            \onslide<3->{\funcname{probably\_raise}}
        \end{itemize}
      }
      \vfill
      \mintocaml[firstline=9,lastline=13,highlightlines=12]{../src/backtrace/backtrace.ml}
      \endminipage
    \end{column}
    \begin{column}{0.5\textwidth}
      \raggedleft
      \onslide*<1>{\listStashBacktrace[highlightlines={114-115}]{}}
      \onslide*<2>{\providecommand\step{3}\input{diagrams/backtrace/backtrace.tex}}%
      \onslide*<3>{\providecommand\step{4}\input{diagrams/backtrace/backtrace.tex}}%
    \end{column}
  \end{columns}
\end{frame}

\begin{frame}{Backtraces}{Back to \funcname{caml\_raise\_exn}}
  \begin{columns}[c]
    \begin{column}{0.5\textwidth}
      \minipage[c][0.66\textheight][s]{\columnwidth}
      \begin{itemize}\color{gray}
        \item Checks wheteher exceptions backtrace should be recorded, by checking the \funcarg{domain\_state->backtrace\_active} value
        \item Switches to the C stack to be able to execute C code
        \item Calls \funcname{caml\_stash\_backtrace}, to collect the backtrace up to the next trap
      \end{itemize}
      \onslide*<2->{
        \begin{itemize}
          \item<2-> Executes the exception handler in \funcname{probably\_raise} with \funcname{RESTORE\_EXN\_HANDLER\_OCAML}
            \begin{itemize}
              \item Set \regname{RSP} to \localname{domain\_state->exn\_handler} value
              \item Restore \localname{domain\_state->exn\_handler} from trap
              \item Jump to the handler code with \funcname{ret}
            \end{itemize}
        \end{itemize}
      }
      \vfill
      \onslide*<1>{\mintocaml[firstline=9,lastline=13,highlightlines=12]{../src/backtrace/backtrace.ml}}
      \onslide*<2->{\mintocaml[firstline=15,lastline=21,highlightlines={19-21}]{../src/backtrace/backtrace.ml}}
      \endminipage
    \end{column}
    \begin{column}{0.5\textwidth}
      \centering
      \onslide*<1>{
        \mintc[firstline=190,lastline=192]{../ocaml/runtime/amd64.S}
        \mintc[firstline=234,lastline=237]{../ocaml/runtime/amd64.S}
      }
      \onslide*<2>{
        \mintc[firstline=190,lastline=192,highlightlines={191-192}]{../ocaml/runtime/amd64.S}
        \mintc[firstline=234,lastline=237,highlightlines={235-237}]{../ocaml/runtime/amd64.S}
      }
      \onslide*<1>{\listCamlRaiseExn[highlightlines=720]{}}
      \onslide*<2>{\listCamlRaiseExn[highlightlines=722]{}}
      \onslide*<3->{\providecommand\step{5}\input{diagrams/backtrace/backtrace.tex}}
    \end{column}
  \end{columns}
\end{frame}

\begin{frame}{Backtraces}{\funcname{caml\_probably\_raise}'s handler doesn't catch \typename{ExnC}}
  \begin{columns}[c]
    \begin{column}{0.45\textwidth}
      \minipage[c][0.75\textheight][s]{\columnwidth}
      \begin{itemize}
        \item<1-> \funcname{match \dots with}\footnotemark pattern matching can also match exceptions
        \item<1-> The CMM of \funcname{probably\_raise} shows that this \funcname{match \dots with} is implemented with a pattern matching and a \funcname{try \dots with}
        \item<1-> But this one only matches on \typename{ExnA} exception
        \item<2-> Since the pattern matching fails to catch the exception, \funcname{reraise} is called with our current \typename{ExnC} exception
        \item<3-> Disassembly shows that \funcname{reraise} is actually a call to \funcname{caml\_reraise\_exn}, which is also an assembly function provided by the runtime
      \end{itemize}
      \vfill
      \mintocaml[firstline=15,lastline=21,highlightlines={18-21}]{../src/backtrace/backtrace.ml}
      \endminipage
    \end{column}
    \begin{column}{0.55\textwidth}
      \centering
      \onslide*<1>{\mintcmm[highlightlines={10-18}]{../src/backtrace/camlBacktrace\_\_probably\_raise\_351.cmm}}
      \onslide*<2>{\listcmm[highlightlines=18]{../src/backtrace/camlBacktrace\_\_probably\_raise_351.cmm}{CMM of probably\_raise}}
      \onslide*<3>{\mintobjdump[firstline=5,lastline=35,highlightlines=31]{../src/backtrace/camlBacktrace\_\_probably\_raise_351.objdump}}
    \end{column}
  \end{columns}
  \only<1>{\footnotetext[1]{https://blog.janestreet.com/pattern-matching-and-exception-handling-unite/}}
\end{frame}

\newcommand\listCamlReraiseExn[1][]{
  \listasm[firstline=727,lastline=734,#1]{../ocaml/runtime/amd64.S}{runtime/amd64.S:727}}

\begin{frame}{Backtraces}{Introducing \funcname{caml\_reraise\_exn}}
  \begin{columns}[c]
    \begin{column}{0.5\textwidth}
      \minipage[c][0.66\textheight][s]{\columnwidth}
      \begin{itemize}
        \item<1-> The exception have to be reraised to the next handler
        \item<2-> First, checks if the backtrace should be collected with \funcarg{domain\_state->backtrace\_active}
        \only<3>{
        \begin{itemize}
            \item If not, behaves like a \funcname{raise\_notrace} with \funcname{RESTORE\_EXN\_HANDLER\_OCAML}
        \end{itemize}
        }
        \item<4-> Jumps into \funcname{caml\_raise\_exn}
        \item<5-> Calls \funcname{caml\_stash\_backtrace} again, to scan the next part of the stack: from the trap just pop-ed to the next trap
      \end{itemize}
      \vfill
      \mintocaml[firstline=15,lastline=21,highlightlines={18-21}]{../src/backtrace/backtrace.ml}
      \endminipage
    \end{column}
    \begin{column}{0.5\textwidth}
      \raggedleft
      \onslide*<1>{\listCamlReraiseExn{}}
      \onslide*<2>{\listCamlReraiseExn[highlightlines=729]{}}
      \onslide*<3>{
        \mintc[firstline=190,lastline=192,highlightlines={191-192}]{../ocaml/runtime/amd64.S}
        \mintc[firstline=234,lastline=237,highlightlines={235-237}]{../ocaml/runtime/amd64.S}
        \medskip
        \listCamlReraiseExn[highlightlines=731]{}
      }
      \onslide*<4-5>{
        % TODO refactor with \listCamlReraiseExn
        \mintasm[firstline=727,lastline=734,highlightlines=730]{../ocaml/runtime/amd64.S}
        \medskip
      }
      \onslide*<4>{\listCamlRaiseExn[highlightlines=711]{}}
      \onslide*<5>{\listCamlRaiseExn[highlightlines=720]{}}
    \end{column}
  \end{columns}
\end{frame}

\begin{frame}{Backtraces}{Backtrace collection continues}
  \begin{columns}[c]
    \begin{column}{0.55\textwidth}
      \minipage[c][0.75\textheight][s]{\columnwidth}
      \begin{itemize}
        \item<1-> \funcname{caml\_stash\_backtrace} scans the older part of the stack, up to the next trap
        \item<6-> Controls jumps to the nested exception handler in \funcname{maybe\_raise}, which matches only on \typename{ExnB}
        \item<7-> \funcname{caml\_reraise\_exn} is called again, backtrace collection resumes with \funcname{caml\_stash\_backtrace}
      \end{itemize}
      \medskip
      \begin{itemize}
        \item<2-> Constructed backtrace:
        \begin{itemize}
          \item <2-> \funcname{definitely\_raise}
          \item <2-> \funcname{probably\_raise}
          \item <3-> \funcname{probably\_raise} (re-raise)
          \item <4-> \funcname{maybe\_raise}
          \item <5-> \funcname{main}
          \item <8-> \funcname{main} (re-raise)
        \end{itemize}
      \end{itemize}
      \vfill
      \onslide*<1-5>{\mintocaml[firstline=15,lastline=21]{../src/backtrace/backtrace.ml}}
      \onslide*<6>{\mintocaml[firstline=28,lastline=34,highlightlines=31]{../src/backtrace/backtrace.ml}}
      \onslide*<7->{\mintocaml[firstline=28,lastline=34,highlightlines=33]{../src/backtrace/backtrace.ml}}
      \endminipage
    \end{column}
    \begin{column}{0.45\textwidth}
      \raggedleft
      \onslide*<1>{\providecommand\step{6}\input{diagrams/backtrace/backtrace.tex}}%
      \onslide*<2>{\providecommand\step{6}\input{diagrams/backtrace/backtrace.tex}}%
      \onslide*<3>{\providecommand\step{7}\input{diagrams/backtrace/backtrace.tex}}%
      \onslide*<4>{\providecommand\step{8}\input{diagrams/backtrace/backtrace.tex}}%
      \onslide*<5>{\providecommand\step{9}\input{diagrams/backtrace/backtrace.tex}}%
      \onslide*<6>{\providecommand\step{10}\input{diagrams/backtrace/backtrace.tex}}%
      \onslide*<7>{\providecommand\step{11}\input{diagrams/backtrace/backtrace.tex}}%
      \onslide*<8>{\providecommand\step{12}\input{diagrams/backtrace/backtrace.tex}}%
      \onslide*<9>{\providecommand\step{13}\input{diagrams/backtrace/backtrace.tex}}%
    \end{column}
  \end{columns}
\end{frame}

\begin{frame}{Backtraces}{Printing the backtrace}
  \begin{columns}[c]
    \begin{column}{0.45\textwidth}
      \minipage[c][0.66\textheight][s]{\columnwidth}
      \begin{itemize}
        \item<1-> The exception backtrace can now be printed to \funcarg{stdout} with \funcname{Printexc.print_backtrace}
        \item<2-> First the exception backtrace is retrieved into an OCaml array with \funcname{get_raw_backtrace }
        \item<3-> Which is implemented as the \funcname{caml_get_exception_raw_backtrace} C function in the runtime
        \item<4-> And finally printed with \funcname{print_exception_backtrace}
      \end{itemize}
      \vfill
      \mintocaml[firstline=28,lastline=34,highlightlines=33]{../src/backtrace/backtrace.ml}
      \endminipage
    \end{column}
    \begin{column}{0.55\textwidth}
      \onslide*<1,2>{
        \mintocaml[firstline=100,lastline=101]{../ocaml/stdlib/printexc.ml}
      }
      \only<1>{
        \listocaml[firstline=170,lastline=172]{../ocaml/stdlib/printexc.ml}{stdlib/printexc.ml:170}
      }
      \only<2>{
        \listocaml[firstline=170,lastline=172,highlightlines=172]{../ocaml/stdlib/printexc.ml}{stdlib/printexc.ml:170}
      }
      \only<3>{
        \mintc[firstline=154,lastline=156]{../ocaml/runtime/backtrace.c}
        \listc[firstline=171,lastline=192,highlightlines={184-187}]{../ocaml/runtime/backtrace.c}{runtime/backtrace.c:154}
      }
      \only<4>{
        \listocaml[firstline=155,lastline=165]{../ocaml/stdlib/printexc.ml}{stdlib/printexc.ml:55}
      }
    \end{column}
  \end{columns}
\end{frame}


\subsection{The multiple kinds of raise}
\frameSubsection{}{}

\begin{frame}{Backtraces}{When backtraces are not needed}
  \begin{columns}[c]
    \begin{column}{0.45\textwidth}
      \minipage[c][0.66\textheight][s]{\columnwidth}
      \begin{itemize}
        \item<1-> What if the backtrace for an exception is never needed?
        \item<2-> It's possible to raise the exception explicitly with \funcname{raise\_notrace} to make raising as cheap as possible
        \item<3-> An example of use in the \funcname{dynlink} library of the OCaml compiler
      \end{itemize}
      \vfill
      \onslide<3->{
      \listocaml[firstline=205,lastline=209,highlightlines=207]{../ocaml/otherlibs/dynlink/dynlink\_compilerlibs/misc.ml}{otherlibs/dynlink/dynlink\_compilerlibs/misc.ml:205}
      }
      \endminipage
    \end{column}
    \begin{column}{0.55\textwidth}
    \only<4>{
      \mintcmm[highlightlines=3]{../src/notrace/camlNotrace\_\_fun_347.cmm}
      \listcmm{../src/notrace/camlNotrace\_\_all\_somes\_327.cmm}{all\_somes}
    }
    \onslide*<5->{
      \listobjdump[highlightlines={6-9}]{../src/notrace/camlNotrace\_\_fun_347.objdump}{anonymous function from \funcname{all\_somes}}
    }
    \end{column}
  \end{columns}
\end{frame}

\begin{frame}{Backtraces}{Summary table of the multiple kinds of raise}
  \begin{itemize}
    \item When debugging information is enabled and if the backtrace of an exception isn't needed, it's possible to raise with \funcname{raise\_notrace} for performance
    \item When debugging information is \emph{not} enabled, the compiler optimises \funcname{raise} calls to inline assembly (same effect as using \funcname{raise\_notrace})
  \end{itemize}
  \bigskip
  \centering
  \begin{tabular}{| l | c | c | c | c |}
    \hline
    \parbox{10em}{Debug information \\ (\funcarg{-g} flag)}  & \multicolumn{2}{|c|}{Disabled}                                     & \multicolumn{2}{|c|}{Enabled} \\
    \hline
    Raise kind                                               & \funcname{raise} & \funcname{raise\_notrace}                       & \funcname{raise} & \funcname{raise\_notrace} \\
    \hline
    Compilation                                              & \parbox{8em}{inline assembly\\ (optimization)} & inline assembly   & \funcname{caml\_raise\_exn} & inline assembly \\
    \hline
  \end{tabular}
\end{frame}


%
% Takeaway #5
%
\subsection*{Takeaway \#5}
\frameSubsectionTakeaway{}
\againframe<5>{takeaway}
}%
      \onslide*<3>{\providecommand\step{7}%
% Backtraces
%
\subsection{One advantage of raising with debugging information}
\frameSubsection{}{}

\begin{frame}{Backtraces}{Debugging information \& source code}
  \begin{columns}[c]
    \begin{column}{0.5\textwidth}
      \begin{itemize}
        \item Raising an exception is fun and games, but how do we know where the exception originated? $\rightarrow$ With the backtrace associated with the exception!
        \item In order to get a backtrace from the point where an exception was raised, it's necessary to compile with debugging information enabled (\funcarg{-g} flag for the compiler)
        \item Same example as in the `Nested handlers' section
      \end{itemize}
    \end{column}
    \begin{column}{0.5\textwidth}
      \listocaml[firstline=9,lastline=34]{../src/backtrace/backtrace.ml}{backtrace/backtrace.ml}
    \end{column}
  \end{columns}
\end{frame}

\begin{frame}{Backtraces}{CMM and disassembly of \funcname{definitely\_raise}}
  \begin{columns}[c]
    \begin{column}{0.5\textwidth}
      \minipage[c][0.66\textheight][s]{\columnwidth}
      \begin{itemize}
        \item<1-> An effect of compiling with debugging information (\funcarg{-g}): the CMM no longer uses \funcname{raise\_notrace} to raise the exception but \funcname{raise} instead
        \item<2-> Disassembly shows that CMM \funcname{raise} is actually a call to \funcname{caml\_raise\_exn}, a function in assembler provided by the runtime
      \end{itemize}
      \vfill
      \mintocaml[firstline=9,lastline=13,highlightlines=12]{../src/backtrace/backtrace.ml}
      \endminipage
    \end{column}
    \begin{column}{0.5\textwidth}
      \onslide*<1>{
        \mintcmm[highlightlines=5]{../src/backtrace/camlBacktrace\_\_definitely\_raise\_347.cmm}
      }
      \onslide*<2>{
        \mintobjdump[firstline=2,highlightlines=14]{../src/backtrace/camlBacktrace\_\_definitely\_raise\_347.objdump}
      }
    \end{column}
  \end{columns}
\end{frame}

\begin{frame}{Backtraces}{Introducing \funcname{caml\_raise\_exn}}
  \begin{columns}[c]
    \begin{column}{0.5\textwidth}
      \minipage[c][0.66\textheight][s]{\columnwidth}
      \onslide*<1>{
        \begin{itemize}
          \item Raising the exception is performed using \funcname{caml\_raise\_exn} which is
            \begin{itemize}
              \item An assembly function provided by the runtime
              \item Architecture specific
            \end{itemize}
          \item Let's see what it does differently from \funcname{raise\_notrace}\dots
          \item We are about to raise \typename{ExnC} with \funcname{caml\_raise\_exn} from \funcname{definitely\_raise}
        \end{itemize}
      }
      \onslide*<2>{
        \mintobjdump[firstline=2,highlightlines=14]{../src/backtrace/camlBacktrace\_\_definitely\_raise\_347.objdump}
      }
      \vfill
      \mintocaml[firstline=9,lastline=13,highlightlines=12]{../src/backtrace/backtrace.ml}
      \endminipage
    \end{column}
    \begin{column}{0.5\textwidth}
      \centering
      \providecommand\step{1}\input{diagrams/backtrace/backtrace.tex}
    \end{column}
  \end{columns}
\end{frame}

\begin{frame}{Backtraces}{But before that, we need more fields from \typename{caml\_domain\_state}}
  \begin{columns}
    \begin{column}{0.5\textwidth}
      \begin{itemize}
        \item Remember \typename{caml\_domain\_state} the per-domain runtime state?
        \item Some other members are of interest to us now\dots
        \item \localname{backtrace\_active} is used to know if the runtime should collect backtraces
        \item \localname{backtrace\_active} can be set by
          \begin{itemize}
            \item The \funcarg{b} flag in the \funcname{OCAMLRUNPARAM} environment variable
            \item \mintinline{ocaml}{Printexc.record_backtrace : bool -> unit} \footnotemark
          \end{itemize}
      \end{itemize}
    \end{column}
    \begin{column}{0.5\textwidth}
      \begin{listing}
        \mintc[highlightlines={9-11}]{src/ocaml/domain\_state\_backtrace.h}
        \caption{runtime/caml/domain\_state.h}
      \end{listing}
    \end{column}
  \end{columns}
  \footnotetext[1]{https://v2.ocaml.org/api/Printexc.html}
\end{frame}

\newcommand\listCamlRaiseExn[1][]{
  \listasm[firstline=702,lastline=725,#1]{../ocaml/runtime/amd64.S}{runtime/amd64.S:702}}

\begin{frame}{Backtraces}{Walk through \funcname{caml\_raise\_exn}}
  \begin{columns}[T]
    \begin{column}{0.5\textwidth}
      \begin{itemize}
        \item<1-> First checks whether exceptions backtrace should be recorded
        \item<1-> By checking the \funcarg{domain\_state->backtrace\_active} value
        \item<2-> If not, acts as \funcname{raise\_notrace} with \funcname{RESTORE\_EXN\_HANDLER\_OCAML}
          \begin{itemize}
            \item Set \regname{RSP} to \localname{domain\_state->exn\_handler} value
            \item Restore \localname{domain\_state->exn\_handler} from trap
            \item Jump with \funcname{ret} (same effect as using \funcname{pop} and \funcname{jmp})
          \end{itemize}
        \item<3-> Otherwise, switches to the C stack to be able to execute C code
        \item<4-> Calls \funcname{caml\_stash\_backtrace}, a C function provided by the runtime\ignorespaces
          \onslide<6->{, with
            \begin{itemize}
              \item \funcarg{exn}: exception value
              \item \funcarg{pc}: code address of the raising point
              \item \funcarg{sp}: stack address of the raising function
              \item \funcarg{trapsp}: stack address of the next handler
            \end{itemize}
          }
      \end{itemize}
      \bigskip
      \mintocaml[firstline=9,lastline=13,highlightlines=12]{../src/backtrace/backtrace.ml}
    \end{column}
    \begin{column}{0.5\textwidth}
      \raggedleft
      \onslide*<1,3-4>{
        \mintc[firstline=190,lastline=192]{../ocaml/runtime/amd64.S}
        \mintc[firstline=234,lastline=237]{../ocaml/runtime/amd64.S}
      }
      \onslide*<2>{
        \mintc[firstline=190,lastline=192,highlightlines={191-192}]{../ocaml/runtime/amd64.S}
        \mintc[firstline=234,lastline=237,highlightlines={235-237}]{../ocaml/runtime/amd64.S}
      }
      \onslide*<1>{\listCamlRaiseExn[highlightlines=705]{}}%
      \onslide*<2>{\listCamlRaiseExn[highlightlines={707-708}]{}}%
      \onslide*<3>{\listCamlRaiseExn[highlightlines={712-714}]{}}%
      \onslide*<4>{\listCamlRaiseExn[highlightlines={715-720}]{}}%
      \onslide*<5>{\providecommand\step{1}\input{diagrams/backtrace/backtrace.tex}}%
      \onslide*<6>{\providecommand\step{2}\input{diagrams/backtrace/backtrace.tex}}%
    \end{column}
  \end{columns}
\end{frame}

\newcommand\listStashBacktrace[1][]{
  \listc[firstline=91,lastline=120,#1]{../ocaml/runtime/backtrace\_nat.c}{runtime/backtrace\_nat.c:91}}

\begin{frame}{Backtraces}{Introducing \funcname{caml\_stash\_backtrace}}
  \begin{columns}[c]
    \begin{column}{0.5\textwidth}
      \minipage[c][0.66\textheight][s]{\columnwidth}
      \begin{itemize}
        \item<1-> A C function provided by the runtime (specific to native code)
        \item<2-> It scans the stack, between the stack pointer at the raising point up to the next trap, in order to collect the names of the detected function frames
          \onslide*<2>{
            \begin{itemize}
              \item And more information, like line number, column number...
            \end{itemize}
          }
        \item<3-> It uses the same technique and information as the Garbage Collector (GC) uses to scan the values live on the stack
          \onslide*<4>{
            \begin{itemize}
              \item \typename{frame\_descr} are at the core of stack unwinding
            \end{itemize}
          }
      \end{itemize}
      \vfill
      \mintocaml[firstline=9,lastline=13,highlightlines=12]{../src/backtrace/backtrace.ml}
      \endminipage
    \end{column}
    \begin{column}{0.5\textwidth}
      \onslide*<1>{\listStashBacktrace[highlightlines=91]{}}
      \onslide*<2>{\listStashBacktrace[highlightlines={91,117-118}]{}}
      \onslide*<3->{\listStashBacktrace[highlightlines={94,106,109-110}]{}}
    \end{column}
  \end{columns}
\end{frame}

\newcommand\listFrameDescr[1][]{
  \listocaml[firstline=111,lastline=124,#1]{../ocaml/asmcomp/emitaux.ml}{asmcomp/emitaux.ml:111}}
\newcommand\listDebugInfo[1][]{
  \listocaml[firstline=38,lastline=49,#1]{../ocaml/lambda/debuginfo.mli}{lambda/debuginfo.mli:38}}

\begin{frame}{Backtraces}{\typename{frame\_descr} and \typename{frame\_debuginfo} in a nutshell}
  \begin{columns}[c]
    \begin{column}{0.5\textwidth}
      \begin{itemize}
        \item<1-> For every return address in an OCaml program, there is a associated \typename{frame\_descr}
        \item<2-> A \typename{frame\_descr} specifies
        \begin{itemize}
          \item<2-> The size of the function stack frame
          \item<3-> Where live values are on the stack (so that the GC can mark them as live and prevent from being swept)
        \end{itemize}
        \item<4-> When compiled with debug information, \typename{frame\_descr} will be linked to a \typename{frame\_debuginfo}
        \begin{itemize}
          \item<5-> Contains information about the position in the source code (file name, line and column number)
        \end{itemize}
      \end{itemize}
    \end{column}
    \begin{column}{0.5\textwidth}
        \onslide*<1>{\listFrameDescr[highlightlines={118-122}]{}}
        \onslide*<2>{\listFrameDescr[highlightlines={120}]{}}
        \onslide*<3>{\listFrameDescr[highlightlines={121}]{}}
        \onslide*<4->{\listFrameDescr[highlightlines={122}]{}}
        \medskip
        \onslide*<1-3>{\listDebugInfo{}}
        \onslide*<4>{\listDebugInfo[highlightlines={38,49}]{}}
        \onslide*<5>{\listDebugInfo[highlightlines={39-46}]{}}
    \end{column}
  \end{columns}
\end{frame}

\begin{frame}{Backtraces}{Scanning the stack with \typename{frame\_descr}}
  \begin{columns}[c]
    \begin{column}{0.5\textwidth}
      \begin{itemize}
        \item Given the return address of the code that raises the exception by calling \funcname{caml\_raise\_exn} (\funcarg{pc} argument of \funcname{caml\_stash\_backtrace})
        \item And the position of the stack pointer (\funcarg{sp} argument of \funcname{caml\_stash\_backtrace})
        \item The frame size given by \typename{frame\_descr}.\funcarg{frame\_size}, allows to find the end of the previous frame
        \item And the return address of the caller of this function
        \bigskip
        \onslide<3->{
        \item Note that traps (here the reddish \funcname{ExnA}, \funcname{ExnB} and \funcname{ExnC} blocks) belong to the frame that installed them, and so the frame size changes when traps are removed
        }
      \end{itemize}
    \end{column}
    \begin{column}{0.5\textwidth}
      \raggedleft
      \onslide*<1>{\providecommand\step{2}\input{diagrams/backtrace/backtrace.tex}}%
      \onslide*<2>{\providecommand\step{1}\input{diagrams/backtrace/scanstack.tex}}%
      \onslide*<3>{\providecommand\step{2}\input{diagrams/backtrace/scanstack.tex}}%
      \onslide*<4>{\providecommand\step{3}\input{diagrams/backtrace/scanstack.tex}}%
      \onslide*<5>{\providecommand\step{4}\input{diagrams/backtrace/scanstack.tex}}%
      \onslide*<6>{\providecommand\step{5}\input{diagrams/backtrace/scanstack.tex}}%
    \end{column}
  \end{columns}
\end{frame}

% Duplicate of previous-previous slide
\begin{frame}{Backtraces}{Continuing on \funcname{caml\_stash\_backtrace}}
  \begin{columns}[c]
    \begin{column}{0.5\textwidth}
      \minipage[c][0.75\textheight][s]{\columnwidth}
      \begin{itemize}
          \color{gray}
        \item A C function provided by the runtime (specific to native code)
        \item It scans the stack, between the stack pointer at the raising point up to the next trap, in order to collect the names of the detected function frames
        \item It uses the same technique and information as the Garbage Collector (GC) uses to scan the values live on the stack
      \end{itemize}
      \begin{itemize}
        \item For each function frame detected, store a pointer to the \typename{frame\_descr} in the \localname{domain\_state->backtrace\_buffer} array, effectively constructing the backtrace
      \end{itemize}
      \onslide<2->{
        \begin{itemize}
            \smallskip
          \item Constructed backtrace:
            \funcname{definitely\_raise},
            \onslide<3->{\funcname{probably\_raise}}
        \end{itemize}
      }
      \vfill
      \mintocaml[firstline=9,lastline=13,highlightlines=12]{../src/backtrace/backtrace.ml}
      \endminipage
    \end{column}
    \begin{column}{0.5\textwidth}
      \raggedleft
      \onslide*<1>{\listStashBacktrace[highlightlines={114-115}]{}}
      \onslide*<2>{\providecommand\step{3}\input{diagrams/backtrace/backtrace.tex}}%
      \onslide*<3>{\providecommand\step{4}\input{diagrams/backtrace/backtrace.tex}}%
    \end{column}
  \end{columns}
\end{frame}

\begin{frame}{Backtraces}{Back to \funcname{caml\_raise\_exn}}
  \begin{columns}[c]
    \begin{column}{0.5\textwidth}
      \minipage[c][0.66\textheight][s]{\columnwidth}
      \begin{itemize}\color{gray}
        \item Checks wheteher exceptions backtrace should be recorded, by checking the \funcarg{domain\_state->backtrace\_active} value
        \item Switches to the C stack to be able to execute C code
        \item Calls \funcname{caml\_stash\_backtrace}, to collect the backtrace up to the next trap
      \end{itemize}
      \onslide*<2->{
        \begin{itemize}
          \item<2-> Executes the exception handler in \funcname{probably\_raise} with \funcname{RESTORE\_EXN\_HANDLER\_OCAML}
            \begin{itemize}
              \item Set \regname{RSP} to \localname{domain\_state->exn\_handler} value
              \item Restore \localname{domain\_state->exn\_handler} from trap
              \item Jump to the handler code with \funcname{ret}
            \end{itemize}
        \end{itemize}
      }
      \vfill
      \onslide*<1>{\mintocaml[firstline=9,lastline=13,highlightlines=12]{../src/backtrace/backtrace.ml}}
      \onslide*<2->{\mintocaml[firstline=15,lastline=21,highlightlines={19-21}]{../src/backtrace/backtrace.ml}}
      \endminipage
    \end{column}
    \begin{column}{0.5\textwidth}
      \centering
      \onslide*<1>{
        \mintc[firstline=190,lastline=192]{../ocaml/runtime/amd64.S}
        \mintc[firstline=234,lastline=237]{../ocaml/runtime/amd64.S}
      }
      \onslide*<2>{
        \mintc[firstline=190,lastline=192,highlightlines={191-192}]{../ocaml/runtime/amd64.S}
        \mintc[firstline=234,lastline=237,highlightlines={235-237}]{../ocaml/runtime/amd64.S}
      }
      \onslide*<1>{\listCamlRaiseExn[highlightlines=720]{}}
      \onslide*<2>{\listCamlRaiseExn[highlightlines=722]{}}
      \onslide*<3->{\providecommand\step{5}\input{diagrams/backtrace/backtrace.tex}}
    \end{column}
  \end{columns}
\end{frame}

\begin{frame}{Backtraces}{\funcname{caml\_probably\_raise}'s handler doesn't catch \typename{ExnC}}
  \begin{columns}[c]
    \begin{column}{0.45\textwidth}
      \minipage[c][0.75\textheight][s]{\columnwidth}
      \begin{itemize}
        \item<1-> \funcname{match \dots with}\footnotemark pattern matching can also match exceptions
        \item<1-> The CMM of \funcname{probably\_raise} shows that this \funcname{match \dots with} is implemented with a pattern matching and a \funcname{try \dots with}
        \item<1-> But this one only matches on \typename{ExnA} exception
        \item<2-> Since the pattern matching fails to catch the exception, \funcname{reraise} is called with our current \typename{ExnC} exception
        \item<3-> Disassembly shows that \funcname{reraise} is actually a call to \funcname{caml\_reraise\_exn}, which is also an assembly function provided by the runtime
      \end{itemize}
      \vfill
      \mintocaml[firstline=15,lastline=21,highlightlines={18-21}]{../src/backtrace/backtrace.ml}
      \endminipage
    \end{column}
    \begin{column}{0.55\textwidth}
      \centering
      \onslide*<1>{\mintcmm[highlightlines={10-18}]{../src/backtrace/camlBacktrace\_\_probably\_raise\_351.cmm}}
      \onslide*<2>{\listcmm[highlightlines=18]{../src/backtrace/camlBacktrace\_\_probably\_raise_351.cmm}{CMM of probably\_raise}}
      \onslide*<3>{\mintobjdump[firstline=5,lastline=35,highlightlines=31]{../src/backtrace/camlBacktrace\_\_probably\_raise_351.objdump}}
    \end{column}
  \end{columns}
  \only<1>{\footnotetext[1]{https://blog.janestreet.com/pattern-matching-and-exception-handling-unite/}}
\end{frame}

\newcommand\listCamlReraiseExn[1][]{
  \listasm[firstline=727,lastline=734,#1]{../ocaml/runtime/amd64.S}{runtime/amd64.S:727}}

\begin{frame}{Backtraces}{Introducing \funcname{caml\_reraise\_exn}}
  \begin{columns}[c]
    \begin{column}{0.5\textwidth}
      \minipage[c][0.66\textheight][s]{\columnwidth}
      \begin{itemize}
        \item<1-> The exception have to be reraised to the next handler
        \item<2-> First, checks if the backtrace should be collected with \funcarg{domain\_state->backtrace\_active}
        \only<3>{
        \begin{itemize}
            \item If not, behaves like a \funcname{raise\_notrace} with \funcname{RESTORE\_EXN\_HANDLER\_OCAML}
        \end{itemize}
        }
        \item<4-> Jumps into \funcname{caml\_raise\_exn}
        \item<5-> Calls \funcname{caml\_stash\_backtrace} again, to scan the next part of the stack: from the trap just pop-ed to the next trap
      \end{itemize}
      \vfill
      \mintocaml[firstline=15,lastline=21,highlightlines={18-21}]{../src/backtrace/backtrace.ml}
      \endminipage
    \end{column}
    \begin{column}{0.5\textwidth}
      \raggedleft
      \onslide*<1>{\listCamlReraiseExn{}}
      \onslide*<2>{\listCamlReraiseExn[highlightlines=729]{}}
      \onslide*<3>{
        \mintc[firstline=190,lastline=192,highlightlines={191-192}]{../ocaml/runtime/amd64.S}
        \mintc[firstline=234,lastline=237,highlightlines={235-237}]{../ocaml/runtime/amd64.S}
        \medskip
        \listCamlReraiseExn[highlightlines=731]{}
      }
      \onslide*<4-5>{
        % TODO refactor with \listCamlReraiseExn
        \mintasm[firstline=727,lastline=734,highlightlines=730]{../ocaml/runtime/amd64.S}
        \medskip
      }
      \onslide*<4>{\listCamlRaiseExn[highlightlines=711]{}}
      \onslide*<5>{\listCamlRaiseExn[highlightlines=720]{}}
    \end{column}
  \end{columns}
\end{frame}

\begin{frame}{Backtraces}{Backtrace collection continues}
  \begin{columns}[c]
    \begin{column}{0.55\textwidth}
      \minipage[c][0.75\textheight][s]{\columnwidth}
      \begin{itemize}
        \item<1-> \funcname{caml\_stash\_backtrace} scans the older part of the stack, up to the next trap
        \item<6-> Controls jumps to the nested exception handler in \funcname{maybe\_raise}, which matches only on \typename{ExnB}
        \item<7-> \funcname{caml\_reraise\_exn} is called again, backtrace collection resumes with \funcname{caml\_stash\_backtrace}
      \end{itemize}
      \medskip
      \begin{itemize}
        \item<2-> Constructed backtrace:
        \begin{itemize}
          \item <2-> \funcname{definitely\_raise}
          \item <2-> \funcname{probably\_raise}
          \item <3-> \funcname{probably\_raise} (re-raise)
          \item <4-> \funcname{maybe\_raise}
          \item <5-> \funcname{main}
          \item <8-> \funcname{main} (re-raise)
        \end{itemize}
      \end{itemize}
      \vfill
      \onslide*<1-5>{\mintocaml[firstline=15,lastline=21]{../src/backtrace/backtrace.ml}}
      \onslide*<6>{\mintocaml[firstline=28,lastline=34,highlightlines=31]{../src/backtrace/backtrace.ml}}
      \onslide*<7->{\mintocaml[firstline=28,lastline=34,highlightlines=33]{../src/backtrace/backtrace.ml}}
      \endminipage
    \end{column}
    \begin{column}{0.45\textwidth}
      \raggedleft
      \onslide*<1>{\providecommand\step{6}\input{diagrams/backtrace/backtrace.tex}}%
      \onslide*<2>{\providecommand\step{6}\input{diagrams/backtrace/backtrace.tex}}%
      \onslide*<3>{\providecommand\step{7}\input{diagrams/backtrace/backtrace.tex}}%
      \onslide*<4>{\providecommand\step{8}\input{diagrams/backtrace/backtrace.tex}}%
      \onslide*<5>{\providecommand\step{9}\input{diagrams/backtrace/backtrace.tex}}%
      \onslide*<6>{\providecommand\step{10}\input{diagrams/backtrace/backtrace.tex}}%
      \onslide*<7>{\providecommand\step{11}\input{diagrams/backtrace/backtrace.tex}}%
      \onslide*<8>{\providecommand\step{12}\input{diagrams/backtrace/backtrace.tex}}%
      \onslide*<9>{\providecommand\step{13}\input{diagrams/backtrace/backtrace.tex}}%
    \end{column}
  \end{columns}
\end{frame}

\begin{frame}{Backtraces}{Printing the backtrace}
  \begin{columns}[c]
    \begin{column}{0.45\textwidth}
      \minipage[c][0.66\textheight][s]{\columnwidth}
      \begin{itemize}
        \item<1-> The exception backtrace can now be printed to \funcarg{stdout} with \funcname{Printexc.print_backtrace}
        \item<2-> First the exception backtrace is retrieved into an OCaml array with \funcname{get_raw_backtrace }
        \item<3-> Which is implemented as the \funcname{caml_get_exception_raw_backtrace} C function in the runtime
        \item<4-> And finally printed with \funcname{print_exception_backtrace}
      \end{itemize}
      \vfill
      \mintocaml[firstline=28,lastline=34,highlightlines=33]{../src/backtrace/backtrace.ml}
      \endminipage
    \end{column}
    \begin{column}{0.55\textwidth}
      \onslide*<1,2>{
        \mintocaml[firstline=100,lastline=101]{../ocaml/stdlib/printexc.ml}
      }
      \only<1>{
        \listocaml[firstline=170,lastline=172]{../ocaml/stdlib/printexc.ml}{stdlib/printexc.ml:170}
      }
      \only<2>{
        \listocaml[firstline=170,lastline=172,highlightlines=172]{../ocaml/stdlib/printexc.ml}{stdlib/printexc.ml:170}
      }
      \only<3>{
        \mintc[firstline=154,lastline=156]{../ocaml/runtime/backtrace.c}
        \listc[firstline=171,lastline=192,highlightlines={184-187}]{../ocaml/runtime/backtrace.c}{runtime/backtrace.c:154}
      }
      \only<4>{
        \listocaml[firstline=155,lastline=165]{../ocaml/stdlib/printexc.ml}{stdlib/printexc.ml:55}
      }
    \end{column}
  \end{columns}
\end{frame}


\subsection{The multiple kinds of raise}
\frameSubsection{}{}

\begin{frame}{Backtraces}{When backtraces are not needed}
  \begin{columns}[c]
    \begin{column}{0.45\textwidth}
      \minipage[c][0.66\textheight][s]{\columnwidth}
      \begin{itemize}
        \item<1-> What if the backtrace for an exception is never needed?
        \item<2-> It's possible to raise the exception explicitly with \funcname{raise\_notrace} to make raising as cheap as possible
        \item<3-> An example of use in the \funcname{dynlink} library of the OCaml compiler
      \end{itemize}
      \vfill
      \onslide<3->{
      \listocaml[firstline=205,lastline=209,highlightlines=207]{../ocaml/otherlibs/dynlink/dynlink\_compilerlibs/misc.ml}{otherlibs/dynlink/dynlink\_compilerlibs/misc.ml:205}
      }
      \endminipage
    \end{column}
    \begin{column}{0.55\textwidth}
    \only<4>{
      \mintcmm[highlightlines=3]{../src/notrace/camlNotrace\_\_fun_347.cmm}
      \listcmm{../src/notrace/camlNotrace\_\_all\_somes\_327.cmm}{all\_somes}
    }
    \onslide*<5->{
      \listobjdump[highlightlines={6-9}]{../src/notrace/camlNotrace\_\_fun_347.objdump}{anonymous function from \funcname{all\_somes}}
    }
    \end{column}
  \end{columns}
\end{frame}

\begin{frame}{Backtraces}{Summary table of the multiple kinds of raise}
  \begin{itemize}
    \item When debugging information is enabled and if the backtrace of an exception isn't needed, it's possible to raise with \funcname{raise\_notrace} for performance
    \item When debugging information is \emph{not} enabled, the compiler optimises \funcname{raise} calls to inline assembly (same effect as using \funcname{raise\_notrace})
  \end{itemize}
  \bigskip
  \centering
  \begin{tabular}{| l | c | c | c | c |}
    \hline
    \parbox{10em}{Debug information \\ (\funcarg{-g} flag)}  & \multicolumn{2}{|c|}{Disabled}                                     & \multicolumn{2}{|c|}{Enabled} \\
    \hline
    Raise kind                                               & \funcname{raise} & \funcname{raise\_notrace}                       & \funcname{raise} & \funcname{raise\_notrace} \\
    \hline
    Compilation                                              & \parbox{8em}{inline assembly\\ (optimization)} & inline assembly   & \funcname{caml\_raise\_exn} & inline assembly \\
    \hline
  \end{tabular}
\end{frame}


%
% Takeaway #5
%
\subsection*{Takeaway \#5}
\frameSubsectionTakeaway{}
\againframe<5>{takeaway}
}%
      \onslide*<4>{\providecommand\step{8}%
% Backtraces
%
\subsection{One advantage of raising with debugging information}
\frameSubsection{}{}

\begin{frame}{Backtraces}{Debugging information \& source code}
  \begin{columns}[c]
    \begin{column}{0.5\textwidth}
      \begin{itemize}
        \item Raising an exception is fun and games, but how do we know where the exception originated? $\rightarrow$ With the backtrace associated with the exception!
        \item In order to get a backtrace from the point where an exception was raised, it's necessary to compile with debugging information enabled (\funcarg{-g} flag for the compiler)
        \item Same example as in the `Nested handlers' section
      \end{itemize}
    \end{column}
    \begin{column}{0.5\textwidth}
      \listocaml[firstline=9,lastline=34]{../src/backtrace/backtrace.ml}{backtrace/backtrace.ml}
    \end{column}
  \end{columns}
\end{frame}

\begin{frame}{Backtraces}{CMM and disassembly of \funcname{definitely\_raise}}
  \begin{columns}[c]
    \begin{column}{0.5\textwidth}
      \minipage[c][0.66\textheight][s]{\columnwidth}
      \begin{itemize}
        \item<1-> An effect of compiling with debugging information (\funcarg{-g}): the CMM no longer uses \funcname{raise\_notrace} to raise the exception but \funcname{raise} instead
        \item<2-> Disassembly shows that CMM \funcname{raise} is actually a call to \funcname{caml\_raise\_exn}, a function in assembler provided by the runtime
      \end{itemize}
      \vfill
      \mintocaml[firstline=9,lastline=13,highlightlines=12]{../src/backtrace/backtrace.ml}
      \endminipage
    \end{column}
    \begin{column}{0.5\textwidth}
      \onslide*<1>{
        \mintcmm[highlightlines=5]{../src/backtrace/camlBacktrace\_\_definitely\_raise\_347.cmm}
      }
      \onslide*<2>{
        \mintobjdump[firstline=2,highlightlines=14]{../src/backtrace/camlBacktrace\_\_definitely\_raise\_347.objdump}
      }
    \end{column}
  \end{columns}
\end{frame}

\begin{frame}{Backtraces}{Introducing \funcname{caml\_raise\_exn}}
  \begin{columns}[c]
    \begin{column}{0.5\textwidth}
      \minipage[c][0.66\textheight][s]{\columnwidth}
      \onslide*<1>{
        \begin{itemize}
          \item Raising the exception is performed using \funcname{caml\_raise\_exn} which is
            \begin{itemize}
              \item An assembly function provided by the runtime
              \item Architecture specific
            \end{itemize}
          \item Let's see what it does differently from \funcname{raise\_notrace}\dots
          \item We are about to raise \typename{ExnC} with \funcname{caml\_raise\_exn} from \funcname{definitely\_raise}
        \end{itemize}
      }
      \onslide*<2>{
        \mintobjdump[firstline=2,highlightlines=14]{../src/backtrace/camlBacktrace\_\_definitely\_raise\_347.objdump}
      }
      \vfill
      \mintocaml[firstline=9,lastline=13,highlightlines=12]{../src/backtrace/backtrace.ml}
      \endminipage
    \end{column}
    \begin{column}{0.5\textwidth}
      \centering
      \providecommand\step{1}\input{diagrams/backtrace/backtrace.tex}
    \end{column}
  \end{columns}
\end{frame}

\begin{frame}{Backtraces}{But before that, we need more fields from \typename{caml\_domain\_state}}
  \begin{columns}
    \begin{column}{0.5\textwidth}
      \begin{itemize}
        \item Remember \typename{caml\_domain\_state} the per-domain runtime state?
        \item Some other members are of interest to us now\dots
        \item \localname{backtrace\_active} is used to know if the runtime should collect backtraces
        \item \localname{backtrace\_active} can be set by
          \begin{itemize}
            \item The \funcarg{b} flag in the \funcname{OCAMLRUNPARAM} environment variable
            \item \mintinline{ocaml}{Printexc.record_backtrace : bool -> unit} \footnotemark
          \end{itemize}
      \end{itemize}
    \end{column}
    \begin{column}{0.5\textwidth}
      \begin{listing}
        \mintc[highlightlines={9-11}]{src/ocaml/domain\_state\_backtrace.h}
        \caption{runtime/caml/domain\_state.h}
      \end{listing}
    \end{column}
  \end{columns}
  \footnotetext[1]{https://v2.ocaml.org/api/Printexc.html}
\end{frame}

\newcommand\listCamlRaiseExn[1][]{
  \listasm[firstline=702,lastline=725,#1]{../ocaml/runtime/amd64.S}{runtime/amd64.S:702}}

\begin{frame}{Backtraces}{Walk through \funcname{caml\_raise\_exn}}
  \begin{columns}[T]
    \begin{column}{0.5\textwidth}
      \begin{itemize}
        \item<1-> First checks whether exceptions backtrace should be recorded
        \item<1-> By checking the \funcarg{domain\_state->backtrace\_active} value
        \item<2-> If not, acts as \funcname{raise\_notrace} with \funcname{RESTORE\_EXN\_HANDLER\_OCAML}
          \begin{itemize}
            \item Set \regname{RSP} to \localname{domain\_state->exn\_handler} value
            \item Restore \localname{domain\_state->exn\_handler} from trap
            \item Jump with \funcname{ret} (same effect as using \funcname{pop} and \funcname{jmp})
          \end{itemize}
        \item<3-> Otherwise, switches to the C stack to be able to execute C code
        \item<4-> Calls \funcname{caml\_stash\_backtrace}, a C function provided by the runtime\ignorespaces
          \onslide<6->{, with
            \begin{itemize}
              \item \funcarg{exn}: exception value
              \item \funcarg{pc}: code address of the raising point
              \item \funcarg{sp}: stack address of the raising function
              \item \funcarg{trapsp}: stack address of the next handler
            \end{itemize}
          }
      \end{itemize}
      \bigskip
      \mintocaml[firstline=9,lastline=13,highlightlines=12]{../src/backtrace/backtrace.ml}
    \end{column}
    \begin{column}{0.5\textwidth}
      \raggedleft
      \onslide*<1,3-4>{
        \mintc[firstline=190,lastline=192]{../ocaml/runtime/amd64.S}
        \mintc[firstline=234,lastline=237]{../ocaml/runtime/amd64.S}
      }
      \onslide*<2>{
        \mintc[firstline=190,lastline=192,highlightlines={191-192}]{../ocaml/runtime/amd64.S}
        \mintc[firstline=234,lastline=237,highlightlines={235-237}]{../ocaml/runtime/amd64.S}
      }
      \onslide*<1>{\listCamlRaiseExn[highlightlines=705]{}}%
      \onslide*<2>{\listCamlRaiseExn[highlightlines={707-708}]{}}%
      \onslide*<3>{\listCamlRaiseExn[highlightlines={712-714}]{}}%
      \onslide*<4>{\listCamlRaiseExn[highlightlines={715-720}]{}}%
      \onslide*<5>{\providecommand\step{1}\input{diagrams/backtrace/backtrace.tex}}%
      \onslide*<6>{\providecommand\step{2}\input{diagrams/backtrace/backtrace.tex}}%
    \end{column}
  \end{columns}
\end{frame}

\newcommand\listStashBacktrace[1][]{
  \listc[firstline=91,lastline=120,#1]{../ocaml/runtime/backtrace\_nat.c}{runtime/backtrace\_nat.c:91}}

\begin{frame}{Backtraces}{Introducing \funcname{caml\_stash\_backtrace}}
  \begin{columns}[c]
    \begin{column}{0.5\textwidth}
      \minipage[c][0.66\textheight][s]{\columnwidth}
      \begin{itemize}
        \item<1-> A C function provided by the runtime (specific to native code)
        \item<2-> It scans the stack, between the stack pointer at the raising point up to the next trap, in order to collect the names of the detected function frames
          \onslide*<2>{
            \begin{itemize}
              \item And more information, like line number, column number...
            \end{itemize}
          }
        \item<3-> It uses the same technique and information as the Garbage Collector (GC) uses to scan the values live on the stack
          \onslide*<4>{
            \begin{itemize}
              \item \typename{frame\_descr} are at the core of stack unwinding
            \end{itemize}
          }
      \end{itemize}
      \vfill
      \mintocaml[firstline=9,lastline=13,highlightlines=12]{../src/backtrace/backtrace.ml}
      \endminipage
    \end{column}
    \begin{column}{0.5\textwidth}
      \onslide*<1>{\listStashBacktrace[highlightlines=91]{}}
      \onslide*<2>{\listStashBacktrace[highlightlines={91,117-118}]{}}
      \onslide*<3->{\listStashBacktrace[highlightlines={94,106,109-110}]{}}
    \end{column}
  \end{columns}
\end{frame}

\newcommand\listFrameDescr[1][]{
  \listocaml[firstline=111,lastline=124,#1]{../ocaml/asmcomp/emitaux.ml}{asmcomp/emitaux.ml:111}}
\newcommand\listDebugInfo[1][]{
  \listocaml[firstline=38,lastline=49,#1]{../ocaml/lambda/debuginfo.mli}{lambda/debuginfo.mli:38}}

\begin{frame}{Backtraces}{\typename{frame\_descr} and \typename{frame\_debuginfo} in a nutshell}
  \begin{columns}[c]
    \begin{column}{0.5\textwidth}
      \begin{itemize}
        \item<1-> For every return address in an OCaml program, there is a associated \typename{frame\_descr}
        \item<2-> A \typename{frame\_descr} specifies
        \begin{itemize}
          \item<2-> The size of the function stack frame
          \item<3-> Where live values are on the stack (so that the GC can mark them as live and prevent from being swept)
        \end{itemize}
        \item<4-> When compiled with debug information, \typename{frame\_descr} will be linked to a \typename{frame\_debuginfo}
        \begin{itemize}
          \item<5-> Contains information about the position in the source code (file name, line and column number)
        \end{itemize}
      \end{itemize}
    \end{column}
    \begin{column}{0.5\textwidth}
        \onslide*<1>{\listFrameDescr[highlightlines={118-122}]{}}
        \onslide*<2>{\listFrameDescr[highlightlines={120}]{}}
        \onslide*<3>{\listFrameDescr[highlightlines={121}]{}}
        \onslide*<4->{\listFrameDescr[highlightlines={122}]{}}
        \medskip
        \onslide*<1-3>{\listDebugInfo{}}
        \onslide*<4>{\listDebugInfo[highlightlines={38,49}]{}}
        \onslide*<5>{\listDebugInfo[highlightlines={39-46}]{}}
    \end{column}
  \end{columns}
\end{frame}

\begin{frame}{Backtraces}{Scanning the stack with \typename{frame\_descr}}
  \begin{columns}[c]
    \begin{column}{0.5\textwidth}
      \begin{itemize}
        \item Given the return address of the code that raises the exception by calling \funcname{caml\_raise\_exn} (\funcarg{pc} argument of \funcname{caml\_stash\_backtrace})
        \item And the position of the stack pointer (\funcarg{sp} argument of \funcname{caml\_stash\_backtrace})
        \item The frame size given by \typename{frame\_descr}.\funcarg{frame\_size}, allows to find the end of the previous frame
        \item And the return address of the caller of this function
        \bigskip
        \onslide<3->{
        \item Note that traps (here the reddish \funcname{ExnA}, \funcname{ExnB} and \funcname{ExnC} blocks) belong to the frame that installed them, and so the frame size changes when traps are removed
        }
      \end{itemize}
    \end{column}
    \begin{column}{0.5\textwidth}
      \raggedleft
      \onslide*<1>{\providecommand\step{2}\input{diagrams/backtrace/backtrace.tex}}%
      \onslide*<2>{\providecommand\step{1}\input{diagrams/backtrace/scanstack.tex}}%
      \onslide*<3>{\providecommand\step{2}\input{diagrams/backtrace/scanstack.tex}}%
      \onslide*<4>{\providecommand\step{3}\input{diagrams/backtrace/scanstack.tex}}%
      \onslide*<5>{\providecommand\step{4}\input{diagrams/backtrace/scanstack.tex}}%
      \onslide*<6>{\providecommand\step{5}\input{diagrams/backtrace/scanstack.tex}}%
    \end{column}
  \end{columns}
\end{frame}

% Duplicate of previous-previous slide
\begin{frame}{Backtraces}{Continuing on \funcname{caml\_stash\_backtrace}}
  \begin{columns}[c]
    \begin{column}{0.5\textwidth}
      \minipage[c][0.75\textheight][s]{\columnwidth}
      \begin{itemize}
          \color{gray}
        \item A C function provided by the runtime (specific to native code)
        \item It scans the stack, between the stack pointer at the raising point up to the next trap, in order to collect the names of the detected function frames
        \item It uses the same technique and information as the Garbage Collector (GC) uses to scan the values live on the stack
      \end{itemize}
      \begin{itemize}
        \item For each function frame detected, store a pointer to the \typename{frame\_descr} in the \localname{domain\_state->backtrace\_buffer} array, effectively constructing the backtrace
      \end{itemize}
      \onslide<2->{
        \begin{itemize}
            \smallskip
          \item Constructed backtrace:
            \funcname{definitely\_raise},
            \onslide<3->{\funcname{probably\_raise}}
        \end{itemize}
      }
      \vfill
      \mintocaml[firstline=9,lastline=13,highlightlines=12]{../src/backtrace/backtrace.ml}
      \endminipage
    \end{column}
    \begin{column}{0.5\textwidth}
      \raggedleft
      \onslide*<1>{\listStashBacktrace[highlightlines={114-115}]{}}
      \onslide*<2>{\providecommand\step{3}\input{diagrams/backtrace/backtrace.tex}}%
      \onslide*<3>{\providecommand\step{4}\input{diagrams/backtrace/backtrace.tex}}%
    \end{column}
  \end{columns}
\end{frame}

\begin{frame}{Backtraces}{Back to \funcname{caml\_raise\_exn}}
  \begin{columns}[c]
    \begin{column}{0.5\textwidth}
      \minipage[c][0.66\textheight][s]{\columnwidth}
      \begin{itemize}\color{gray}
        \item Checks wheteher exceptions backtrace should be recorded, by checking the \funcarg{domain\_state->backtrace\_active} value
        \item Switches to the C stack to be able to execute C code
        \item Calls \funcname{caml\_stash\_backtrace}, to collect the backtrace up to the next trap
      \end{itemize}
      \onslide*<2->{
        \begin{itemize}
          \item<2-> Executes the exception handler in \funcname{probably\_raise} with \funcname{RESTORE\_EXN\_HANDLER\_OCAML}
            \begin{itemize}
              \item Set \regname{RSP} to \localname{domain\_state->exn\_handler} value
              \item Restore \localname{domain\_state->exn\_handler} from trap
              \item Jump to the handler code with \funcname{ret}
            \end{itemize}
        \end{itemize}
      }
      \vfill
      \onslide*<1>{\mintocaml[firstline=9,lastline=13,highlightlines=12]{../src/backtrace/backtrace.ml}}
      \onslide*<2->{\mintocaml[firstline=15,lastline=21,highlightlines={19-21}]{../src/backtrace/backtrace.ml}}
      \endminipage
    \end{column}
    \begin{column}{0.5\textwidth}
      \centering
      \onslide*<1>{
        \mintc[firstline=190,lastline=192]{../ocaml/runtime/amd64.S}
        \mintc[firstline=234,lastline=237]{../ocaml/runtime/amd64.S}
      }
      \onslide*<2>{
        \mintc[firstline=190,lastline=192,highlightlines={191-192}]{../ocaml/runtime/amd64.S}
        \mintc[firstline=234,lastline=237,highlightlines={235-237}]{../ocaml/runtime/amd64.S}
      }
      \onslide*<1>{\listCamlRaiseExn[highlightlines=720]{}}
      \onslide*<2>{\listCamlRaiseExn[highlightlines=722]{}}
      \onslide*<3->{\providecommand\step{5}\input{diagrams/backtrace/backtrace.tex}}
    \end{column}
  \end{columns}
\end{frame}

\begin{frame}{Backtraces}{\funcname{caml\_probably\_raise}'s handler doesn't catch \typename{ExnC}}
  \begin{columns}[c]
    \begin{column}{0.45\textwidth}
      \minipage[c][0.75\textheight][s]{\columnwidth}
      \begin{itemize}
        \item<1-> \funcname{match \dots with}\footnotemark pattern matching can also match exceptions
        \item<1-> The CMM of \funcname{probably\_raise} shows that this \funcname{match \dots with} is implemented with a pattern matching and a \funcname{try \dots with}
        \item<1-> But this one only matches on \typename{ExnA} exception
        \item<2-> Since the pattern matching fails to catch the exception, \funcname{reraise} is called with our current \typename{ExnC} exception
        \item<3-> Disassembly shows that \funcname{reraise} is actually a call to \funcname{caml\_reraise\_exn}, which is also an assembly function provided by the runtime
      \end{itemize}
      \vfill
      \mintocaml[firstline=15,lastline=21,highlightlines={18-21}]{../src/backtrace/backtrace.ml}
      \endminipage
    \end{column}
    \begin{column}{0.55\textwidth}
      \centering
      \onslide*<1>{\mintcmm[highlightlines={10-18}]{../src/backtrace/camlBacktrace\_\_probably\_raise\_351.cmm}}
      \onslide*<2>{\listcmm[highlightlines=18]{../src/backtrace/camlBacktrace\_\_probably\_raise_351.cmm}{CMM of probably\_raise}}
      \onslide*<3>{\mintobjdump[firstline=5,lastline=35,highlightlines=31]{../src/backtrace/camlBacktrace\_\_probably\_raise_351.objdump}}
    \end{column}
  \end{columns}
  \only<1>{\footnotetext[1]{https://blog.janestreet.com/pattern-matching-and-exception-handling-unite/}}
\end{frame}

\newcommand\listCamlReraiseExn[1][]{
  \listasm[firstline=727,lastline=734,#1]{../ocaml/runtime/amd64.S}{runtime/amd64.S:727}}

\begin{frame}{Backtraces}{Introducing \funcname{caml\_reraise\_exn}}
  \begin{columns}[c]
    \begin{column}{0.5\textwidth}
      \minipage[c][0.66\textheight][s]{\columnwidth}
      \begin{itemize}
        \item<1-> The exception have to be reraised to the next handler
        \item<2-> First, checks if the backtrace should be collected with \funcarg{domain\_state->backtrace\_active}
        \only<3>{
        \begin{itemize}
            \item If not, behaves like a \funcname{raise\_notrace} with \funcname{RESTORE\_EXN\_HANDLER\_OCAML}
        \end{itemize}
        }
        \item<4-> Jumps into \funcname{caml\_raise\_exn}
        \item<5-> Calls \funcname{caml\_stash\_backtrace} again, to scan the next part of the stack: from the trap just pop-ed to the next trap
      \end{itemize}
      \vfill
      \mintocaml[firstline=15,lastline=21,highlightlines={18-21}]{../src/backtrace/backtrace.ml}
      \endminipage
    \end{column}
    \begin{column}{0.5\textwidth}
      \raggedleft
      \onslide*<1>{\listCamlReraiseExn{}}
      \onslide*<2>{\listCamlReraiseExn[highlightlines=729]{}}
      \onslide*<3>{
        \mintc[firstline=190,lastline=192,highlightlines={191-192}]{../ocaml/runtime/amd64.S}
        \mintc[firstline=234,lastline=237,highlightlines={235-237}]{../ocaml/runtime/amd64.S}
        \medskip
        \listCamlReraiseExn[highlightlines=731]{}
      }
      \onslide*<4-5>{
        % TODO refactor with \listCamlReraiseExn
        \mintasm[firstline=727,lastline=734,highlightlines=730]{../ocaml/runtime/amd64.S}
        \medskip
      }
      \onslide*<4>{\listCamlRaiseExn[highlightlines=711]{}}
      \onslide*<5>{\listCamlRaiseExn[highlightlines=720]{}}
    \end{column}
  \end{columns}
\end{frame}

\begin{frame}{Backtraces}{Backtrace collection continues}
  \begin{columns}[c]
    \begin{column}{0.55\textwidth}
      \minipage[c][0.75\textheight][s]{\columnwidth}
      \begin{itemize}
        \item<1-> \funcname{caml\_stash\_backtrace} scans the older part of the stack, up to the next trap
        \item<6-> Controls jumps to the nested exception handler in \funcname{maybe\_raise}, which matches only on \typename{ExnB}
        \item<7-> \funcname{caml\_reraise\_exn} is called again, backtrace collection resumes with \funcname{caml\_stash\_backtrace}
      \end{itemize}
      \medskip
      \begin{itemize}
        \item<2-> Constructed backtrace:
        \begin{itemize}
          \item <2-> \funcname{definitely\_raise}
          \item <2-> \funcname{probably\_raise}
          \item <3-> \funcname{probably\_raise} (re-raise)
          \item <4-> \funcname{maybe\_raise}
          \item <5-> \funcname{main}
          \item <8-> \funcname{main} (re-raise)
        \end{itemize}
      \end{itemize}
      \vfill
      \onslide*<1-5>{\mintocaml[firstline=15,lastline=21]{../src/backtrace/backtrace.ml}}
      \onslide*<6>{\mintocaml[firstline=28,lastline=34,highlightlines=31]{../src/backtrace/backtrace.ml}}
      \onslide*<7->{\mintocaml[firstline=28,lastline=34,highlightlines=33]{../src/backtrace/backtrace.ml}}
      \endminipage
    \end{column}
    \begin{column}{0.45\textwidth}
      \raggedleft
      \onslide*<1>{\providecommand\step{6}\input{diagrams/backtrace/backtrace.tex}}%
      \onslide*<2>{\providecommand\step{6}\input{diagrams/backtrace/backtrace.tex}}%
      \onslide*<3>{\providecommand\step{7}\input{diagrams/backtrace/backtrace.tex}}%
      \onslide*<4>{\providecommand\step{8}\input{diagrams/backtrace/backtrace.tex}}%
      \onslide*<5>{\providecommand\step{9}\input{diagrams/backtrace/backtrace.tex}}%
      \onslide*<6>{\providecommand\step{10}\input{diagrams/backtrace/backtrace.tex}}%
      \onslide*<7>{\providecommand\step{11}\input{diagrams/backtrace/backtrace.tex}}%
      \onslide*<8>{\providecommand\step{12}\input{diagrams/backtrace/backtrace.tex}}%
      \onslide*<9>{\providecommand\step{13}\input{diagrams/backtrace/backtrace.tex}}%
    \end{column}
  \end{columns}
\end{frame}

\begin{frame}{Backtraces}{Printing the backtrace}
  \begin{columns}[c]
    \begin{column}{0.45\textwidth}
      \minipage[c][0.66\textheight][s]{\columnwidth}
      \begin{itemize}
        \item<1-> The exception backtrace can now be printed to \funcarg{stdout} with \funcname{Printexc.print_backtrace}
        \item<2-> First the exception backtrace is retrieved into an OCaml array with \funcname{get_raw_backtrace }
        \item<3-> Which is implemented as the \funcname{caml_get_exception_raw_backtrace} C function in the runtime
        \item<4-> And finally printed with \funcname{print_exception_backtrace}
      \end{itemize}
      \vfill
      \mintocaml[firstline=28,lastline=34,highlightlines=33]{../src/backtrace/backtrace.ml}
      \endminipage
    \end{column}
    \begin{column}{0.55\textwidth}
      \onslide*<1,2>{
        \mintocaml[firstline=100,lastline=101]{../ocaml/stdlib/printexc.ml}
      }
      \only<1>{
        \listocaml[firstline=170,lastline=172]{../ocaml/stdlib/printexc.ml}{stdlib/printexc.ml:170}
      }
      \only<2>{
        \listocaml[firstline=170,lastline=172,highlightlines=172]{../ocaml/stdlib/printexc.ml}{stdlib/printexc.ml:170}
      }
      \only<3>{
        \mintc[firstline=154,lastline=156]{../ocaml/runtime/backtrace.c}
        \listc[firstline=171,lastline=192,highlightlines={184-187}]{../ocaml/runtime/backtrace.c}{runtime/backtrace.c:154}
      }
      \only<4>{
        \listocaml[firstline=155,lastline=165]{../ocaml/stdlib/printexc.ml}{stdlib/printexc.ml:55}
      }
    \end{column}
  \end{columns}
\end{frame}


\subsection{The multiple kinds of raise}
\frameSubsection{}{}

\begin{frame}{Backtraces}{When backtraces are not needed}
  \begin{columns}[c]
    \begin{column}{0.45\textwidth}
      \minipage[c][0.66\textheight][s]{\columnwidth}
      \begin{itemize}
        \item<1-> What if the backtrace for an exception is never needed?
        \item<2-> It's possible to raise the exception explicitly with \funcname{raise\_notrace} to make raising as cheap as possible
        \item<3-> An example of use in the \funcname{dynlink} library of the OCaml compiler
      \end{itemize}
      \vfill
      \onslide<3->{
      \listocaml[firstline=205,lastline=209,highlightlines=207]{../ocaml/otherlibs/dynlink/dynlink\_compilerlibs/misc.ml}{otherlibs/dynlink/dynlink\_compilerlibs/misc.ml:205}
      }
      \endminipage
    \end{column}
    \begin{column}{0.55\textwidth}
    \only<4>{
      \mintcmm[highlightlines=3]{../src/notrace/camlNotrace\_\_fun_347.cmm}
      \listcmm{../src/notrace/camlNotrace\_\_all\_somes\_327.cmm}{all\_somes}
    }
    \onslide*<5->{
      \listobjdump[highlightlines={6-9}]{../src/notrace/camlNotrace\_\_fun_347.objdump}{anonymous function from \funcname{all\_somes}}
    }
    \end{column}
  \end{columns}
\end{frame}

\begin{frame}{Backtraces}{Summary table of the multiple kinds of raise}
  \begin{itemize}
    \item When debugging information is enabled and if the backtrace of an exception isn't needed, it's possible to raise with \funcname{raise\_notrace} for performance
    \item When debugging information is \emph{not} enabled, the compiler optimises \funcname{raise} calls to inline assembly (same effect as using \funcname{raise\_notrace})
  \end{itemize}
  \bigskip
  \centering
  \begin{tabular}{| l | c | c | c | c |}
    \hline
    \parbox{10em}{Debug information \\ (\funcarg{-g} flag)}  & \multicolumn{2}{|c|}{Disabled}                                     & \multicolumn{2}{|c|}{Enabled} \\
    \hline
    Raise kind                                               & \funcname{raise} & \funcname{raise\_notrace}                       & \funcname{raise} & \funcname{raise\_notrace} \\
    \hline
    Compilation                                              & \parbox{8em}{inline assembly\\ (optimization)} & inline assembly   & \funcname{caml\_raise\_exn} & inline assembly \\
    \hline
  \end{tabular}
\end{frame}


%
% Takeaway #5
%
\subsection*{Takeaway \#5}
\frameSubsectionTakeaway{}
\againframe<5>{takeaway}
}%
      \onslide*<5>{\providecommand\step{9}%
% Backtraces
%
\subsection{One advantage of raising with debugging information}
\frameSubsection{}{}

\begin{frame}{Backtraces}{Debugging information \& source code}
  \begin{columns}[c]
    \begin{column}{0.5\textwidth}
      \begin{itemize}
        \item Raising an exception is fun and games, but how do we know where the exception originated? $\rightarrow$ With the backtrace associated with the exception!
        \item In order to get a backtrace from the point where an exception was raised, it's necessary to compile with debugging information enabled (\funcarg{-g} flag for the compiler)
        \item Same example as in the `Nested handlers' section
      \end{itemize}
    \end{column}
    \begin{column}{0.5\textwidth}
      \listocaml[firstline=9,lastline=34]{../src/backtrace/backtrace.ml}{backtrace/backtrace.ml}
    \end{column}
  \end{columns}
\end{frame}

\begin{frame}{Backtraces}{CMM and disassembly of \funcname{definitely\_raise}}
  \begin{columns}[c]
    \begin{column}{0.5\textwidth}
      \minipage[c][0.66\textheight][s]{\columnwidth}
      \begin{itemize}
        \item<1-> An effect of compiling with debugging information (\funcarg{-g}): the CMM no longer uses \funcname{raise\_notrace} to raise the exception but \funcname{raise} instead
        \item<2-> Disassembly shows that CMM \funcname{raise} is actually a call to \funcname{caml\_raise\_exn}, a function in assembler provided by the runtime
      \end{itemize}
      \vfill
      \mintocaml[firstline=9,lastline=13,highlightlines=12]{../src/backtrace/backtrace.ml}
      \endminipage
    \end{column}
    \begin{column}{0.5\textwidth}
      \onslide*<1>{
        \mintcmm[highlightlines=5]{../src/backtrace/camlBacktrace\_\_definitely\_raise\_347.cmm}
      }
      \onslide*<2>{
        \mintobjdump[firstline=2,highlightlines=14]{../src/backtrace/camlBacktrace\_\_definitely\_raise\_347.objdump}
      }
    \end{column}
  \end{columns}
\end{frame}

\begin{frame}{Backtraces}{Introducing \funcname{caml\_raise\_exn}}
  \begin{columns}[c]
    \begin{column}{0.5\textwidth}
      \minipage[c][0.66\textheight][s]{\columnwidth}
      \onslide*<1>{
        \begin{itemize}
          \item Raising the exception is performed using \funcname{caml\_raise\_exn} which is
            \begin{itemize}
              \item An assembly function provided by the runtime
              \item Architecture specific
            \end{itemize}
          \item Let's see what it does differently from \funcname{raise\_notrace}\dots
          \item We are about to raise \typename{ExnC} with \funcname{caml\_raise\_exn} from \funcname{definitely\_raise}
        \end{itemize}
      }
      \onslide*<2>{
        \mintobjdump[firstline=2,highlightlines=14]{../src/backtrace/camlBacktrace\_\_definitely\_raise\_347.objdump}
      }
      \vfill
      \mintocaml[firstline=9,lastline=13,highlightlines=12]{../src/backtrace/backtrace.ml}
      \endminipage
    \end{column}
    \begin{column}{0.5\textwidth}
      \centering
      \providecommand\step{1}\input{diagrams/backtrace/backtrace.tex}
    \end{column}
  \end{columns}
\end{frame}

\begin{frame}{Backtraces}{But before that, we need more fields from \typename{caml\_domain\_state}}
  \begin{columns}
    \begin{column}{0.5\textwidth}
      \begin{itemize}
        \item Remember \typename{caml\_domain\_state} the per-domain runtime state?
        \item Some other members are of interest to us now\dots
        \item \localname{backtrace\_active} is used to know if the runtime should collect backtraces
        \item \localname{backtrace\_active} can be set by
          \begin{itemize}
            \item The \funcarg{b} flag in the \funcname{OCAMLRUNPARAM} environment variable
            \item \mintinline{ocaml}{Printexc.record_backtrace : bool -> unit} \footnotemark
          \end{itemize}
      \end{itemize}
    \end{column}
    \begin{column}{0.5\textwidth}
      \begin{listing}
        \mintc[highlightlines={9-11}]{src/ocaml/domain\_state\_backtrace.h}
        \caption{runtime/caml/domain\_state.h}
      \end{listing}
    \end{column}
  \end{columns}
  \footnotetext[1]{https://v2.ocaml.org/api/Printexc.html}
\end{frame}

\newcommand\listCamlRaiseExn[1][]{
  \listasm[firstline=702,lastline=725,#1]{../ocaml/runtime/amd64.S}{runtime/amd64.S:702}}

\begin{frame}{Backtraces}{Walk through \funcname{caml\_raise\_exn}}
  \begin{columns}[T]
    \begin{column}{0.5\textwidth}
      \begin{itemize}
        \item<1-> First checks whether exceptions backtrace should be recorded
        \item<1-> By checking the \funcarg{domain\_state->backtrace\_active} value
        \item<2-> If not, acts as \funcname{raise\_notrace} with \funcname{RESTORE\_EXN\_HANDLER\_OCAML}
          \begin{itemize}
            \item Set \regname{RSP} to \localname{domain\_state->exn\_handler} value
            \item Restore \localname{domain\_state->exn\_handler} from trap
            \item Jump with \funcname{ret} (same effect as using \funcname{pop} and \funcname{jmp})
          \end{itemize}
        \item<3-> Otherwise, switches to the C stack to be able to execute C code
        \item<4-> Calls \funcname{caml\_stash\_backtrace}, a C function provided by the runtime\ignorespaces
          \onslide<6->{, with
            \begin{itemize}
              \item \funcarg{exn}: exception value
              \item \funcarg{pc}: code address of the raising point
              \item \funcarg{sp}: stack address of the raising function
              \item \funcarg{trapsp}: stack address of the next handler
            \end{itemize}
          }
      \end{itemize}
      \bigskip
      \mintocaml[firstline=9,lastline=13,highlightlines=12]{../src/backtrace/backtrace.ml}
    \end{column}
    \begin{column}{0.5\textwidth}
      \raggedleft
      \onslide*<1,3-4>{
        \mintc[firstline=190,lastline=192]{../ocaml/runtime/amd64.S}
        \mintc[firstline=234,lastline=237]{../ocaml/runtime/amd64.S}
      }
      \onslide*<2>{
        \mintc[firstline=190,lastline=192,highlightlines={191-192}]{../ocaml/runtime/amd64.S}
        \mintc[firstline=234,lastline=237,highlightlines={235-237}]{../ocaml/runtime/amd64.S}
      }
      \onslide*<1>{\listCamlRaiseExn[highlightlines=705]{}}%
      \onslide*<2>{\listCamlRaiseExn[highlightlines={707-708}]{}}%
      \onslide*<3>{\listCamlRaiseExn[highlightlines={712-714}]{}}%
      \onslide*<4>{\listCamlRaiseExn[highlightlines={715-720}]{}}%
      \onslide*<5>{\providecommand\step{1}\input{diagrams/backtrace/backtrace.tex}}%
      \onslide*<6>{\providecommand\step{2}\input{diagrams/backtrace/backtrace.tex}}%
    \end{column}
  \end{columns}
\end{frame}

\newcommand\listStashBacktrace[1][]{
  \listc[firstline=91,lastline=120,#1]{../ocaml/runtime/backtrace\_nat.c}{runtime/backtrace\_nat.c:91}}

\begin{frame}{Backtraces}{Introducing \funcname{caml\_stash\_backtrace}}
  \begin{columns}[c]
    \begin{column}{0.5\textwidth}
      \minipage[c][0.66\textheight][s]{\columnwidth}
      \begin{itemize}
        \item<1-> A C function provided by the runtime (specific to native code)
        \item<2-> It scans the stack, between the stack pointer at the raising point up to the next trap, in order to collect the names of the detected function frames
          \onslide*<2>{
            \begin{itemize}
              \item And more information, like line number, column number...
            \end{itemize}
          }
        \item<3-> It uses the same technique and information as the Garbage Collector (GC) uses to scan the values live on the stack
          \onslide*<4>{
            \begin{itemize}
              \item \typename{frame\_descr} are at the core of stack unwinding
            \end{itemize}
          }
      \end{itemize}
      \vfill
      \mintocaml[firstline=9,lastline=13,highlightlines=12]{../src/backtrace/backtrace.ml}
      \endminipage
    \end{column}
    \begin{column}{0.5\textwidth}
      \onslide*<1>{\listStashBacktrace[highlightlines=91]{}}
      \onslide*<2>{\listStashBacktrace[highlightlines={91,117-118}]{}}
      \onslide*<3->{\listStashBacktrace[highlightlines={94,106,109-110}]{}}
    \end{column}
  \end{columns}
\end{frame}

\newcommand\listFrameDescr[1][]{
  \listocaml[firstline=111,lastline=124,#1]{../ocaml/asmcomp/emitaux.ml}{asmcomp/emitaux.ml:111}}
\newcommand\listDebugInfo[1][]{
  \listocaml[firstline=38,lastline=49,#1]{../ocaml/lambda/debuginfo.mli}{lambda/debuginfo.mli:38}}

\begin{frame}{Backtraces}{\typename{frame\_descr} and \typename{frame\_debuginfo} in a nutshell}
  \begin{columns}[c]
    \begin{column}{0.5\textwidth}
      \begin{itemize}
        \item<1-> For every return address in an OCaml program, there is a associated \typename{frame\_descr}
        \item<2-> A \typename{frame\_descr} specifies
        \begin{itemize}
          \item<2-> The size of the function stack frame
          \item<3-> Where live values are on the stack (so that the GC can mark them as live and prevent from being swept)
        \end{itemize}
        \item<4-> When compiled with debug information, \typename{frame\_descr} will be linked to a \typename{frame\_debuginfo}
        \begin{itemize}
          \item<5-> Contains information about the position in the source code (file name, line and column number)
        \end{itemize}
      \end{itemize}
    \end{column}
    \begin{column}{0.5\textwidth}
        \onslide*<1>{\listFrameDescr[highlightlines={118-122}]{}}
        \onslide*<2>{\listFrameDescr[highlightlines={120}]{}}
        \onslide*<3>{\listFrameDescr[highlightlines={121}]{}}
        \onslide*<4->{\listFrameDescr[highlightlines={122}]{}}
        \medskip
        \onslide*<1-3>{\listDebugInfo{}}
        \onslide*<4>{\listDebugInfo[highlightlines={38,49}]{}}
        \onslide*<5>{\listDebugInfo[highlightlines={39-46}]{}}
    \end{column}
  \end{columns}
\end{frame}

\begin{frame}{Backtraces}{Scanning the stack with \typename{frame\_descr}}
  \begin{columns}[c]
    \begin{column}{0.5\textwidth}
      \begin{itemize}
        \item Given the return address of the code that raises the exception by calling \funcname{caml\_raise\_exn} (\funcarg{pc} argument of \funcname{caml\_stash\_backtrace})
        \item And the position of the stack pointer (\funcarg{sp} argument of \funcname{caml\_stash\_backtrace})
        \item The frame size given by \typename{frame\_descr}.\funcarg{frame\_size}, allows to find the end of the previous frame
        \item And the return address of the caller of this function
        \bigskip
        \onslide<3->{
        \item Note that traps (here the reddish \funcname{ExnA}, \funcname{ExnB} and \funcname{ExnC} blocks) belong to the frame that installed them, and so the frame size changes when traps are removed
        }
      \end{itemize}
    \end{column}
    \begin{column}{0.5\textwidth}
      \raggedleft
      \onslide*<1>{\providecommand\step{2}\input{diagrams/backtrace/backtrace.tex}}%
      \onslide*<2>{\providecommand\step{1}\input{diagrams/backtrace/scanstack.tex}}%
      \onslide*<3>{\providecommand\step{2}\input{diagrams/backtrace/scanstack.tex}}%
      \onslide*<4>{\providecommand\step{3}\input{diagrams/backtrace/scanstack.tex}}%
      \onslide*<5>{\providecommand\step{4}\input{diagrams/backtrace/scanstack.tex}}%
      \onslide*<6>{\providecommand\step{5}\input{diagrams/backtrace/scanstack.tex}}%
    \end{column}
  \end{columns}
\end{frame}

% Duplicate of previous-previous slide
\begin{frame}{Backtraces}{Continuing on \funcname{caml\_stash\_backtrace}}
  \begin{columns}[c]
    \begin{column}{0.5\textwidth}
      \minipage[c][0.75\textheight][s]{\columnwidth}
      \begin{itemize}
          \color{gray}
        \item A C function provided by the runtime (specific to native code)
        \item It scans the stack, between the stack pointer at the raising point up to the next trap, in order to collect the names of the detected function frames
        \item It uses the same technique and information as the Garbage Collector (GC) uses to scan the values live on the stack
      \end{itemize}
      \begin{itemize}
        \item For each function frame detected, store a pointer to the \typename{frame\_descr} in the \localname{domain\_state->backtrace\_buffer} array, effectively constructing the backtrace
      \end{itemize}
      \onslide<2->{
        \begin{itemize}
            \smallskip
          \item Constructed backtrace:
            \funcname{definitely\_raise},
            \onslide<3->{\funcname{probably\_raise}}
        \end{itemize}
      }
      \vfill
      \mintocaml[firstline=9,lastline=13,highlightlines=12]{../src/backtrace/backtrace.ml}
      \endminipage
    \end{column}
    \begin{column}{0.5\textwidth}
      \raggedleft
      \onslide*<1>{\listStashBacktrace[highlightlines={114-115}]{}}
      \onslide*<2>{\providecommand\step{3}\input{diagrams/backtrace/backtrace.tex}}%
      \onslide*<3>{\providecommand\step{4}\input{diagrams/backtrace/backtrace.tex}}%
    \end{column}
  \end{columns}
\end{frame}

\begin{frame}{Backtraces}{Back to \funcname{caml\_raise\_exn}}
  \begin{columns}[c]
    \begin{column}{0.5\textwidth}
      \minipage[c][0.66\textheight][s]{\columnwidth}
      \begin{itemize}\color{gray}
        \item Checks wheteher exceptions backtrace should be recorded, by checking the \funcarg{domain\_state->backtrace\_active} value
        \item Switches to the C stack to be able to execute C code
        \item Calls \funcname{caml\_stash\_backtrace}, to collect the backtrace up to the next trap
      \end{itemize}
      \onslide*<2->{
        \begin{itemize}
          \item<2-> Executes the exception handler in \funcname{probably\_raise} with \funcname{RESTORE\_EXN\_HANDLER\_OCAML}
            \begin{itemize}
              \item Set \regname{RSP} to \localname{domain\_state->exn\_handler} value
              \item Restore \localname{domain\_state->exn\_handler} from trap
              \item Jump to the handler code with \funcname{ret}
            \end{itemize}
        \end{itemize}
      }
      \vfill
      \onslide*<1>{\mintocaml[firstline=9,lastline=13,highlightlines=12]{../src/backtrace/backtrace.ml}}
      \onslide*<2->{\mintocaml[firstline=15,lastline=21,highlightlines={19-21}]{../src/backtrace/backtrace.ml}}
      \endminipage
    \end{column}
    \begin{column}{0.5\textwidth}
      \centering
      \onslide*<1>{
        \mintc[firstline=190,lastline=192]{../ocaml/runtime/amd64.S}
        \mintc[firstline=234,lastline=237]{../ocaml/runtime/amd64.S}
      }
      \onslide*<2>{
        \mintc[firstline=190,lastline=192,highlightlines={191-192}]{../ocaml/runtime/amd64.S}
        \mintc[firstline=234,lastline=237,highlightlines={235-237}]{../ocaml/runtime/amd64.S}
      }
      \onslide*<1>{\listCamlRaiseExn[highlightlines=720]{}}
      \onslide*<2>{\listCamlRaiseExn[highlightlines=722]{}}
      \onslide*<3->{\providecommand\step{5}\input{diagrams/backtrace/backtrace.tex}}
    \end{column}
  \end{columns}
\end{frame}

\begin{frame}{Backtraces}{\funcname{caml\_probably\_raise}'s handler doesn't catch \typename{ExnC}}
  \begin{columns}[c]
    \begin{column}{0.45\textwidth}
      \minipage[c][0.75\textheight][s]{\columnwidth}
      \begin{itemize}
        \item<1-> \funcname{match \dots with}\footnotemark pattern matching can also match exceptions
        \item<1-> The CMM of \funcname{probably\_raise} shows that this \funcname{match \dots with} is implemented with a pattern matching and a \funcname{try \dots with}
        \item<1-> But this one only matches on \typename{ExnA} exception
        \item<2-> Since the pattern matching fails to catch the exception, \funcname{reraise} is called with our current \typename{ExnC} exception
        \item<3-> Disassembly shows that \funcname{reraise} is actually a call to \funcname{caml\_reraise\_exn}, which is also an assembly function provided by the runtime
      \end{itemize}
      \vfill
      \mintocaml[firstline=15,lastline=21,highlightlines={18-21}]{../src/backtrace/backtrace.ml}
      \endminipage
    \end{column}
    \begin{column}{0.55\textwidth}
      \centering
      \onslide*<1>{\mintcmm[highlightlines={10-18}]{../src/backtrace/camlBacktrace\_\_probably\_raise\_351.cmm}}
      \onslide*<2>{\listcmm[highlightlines=18]{../src/backtrace/camlBacktrace\_\_probably\_raise_351.cmm}{CMM of probably\_raise}}
      \onslide*<3>{\mintobjdump[firstline=5,lastline=35,highlightlines=31]{../src/backtrace/camlBacktrace\_\_probably\_raise_351.objdump}}
    \end{column}
  \end{columns}
  \only<1>{\footnotetext[1]{https://blog.janestreet.com/pattern-matching-and-exception-handling-unite/}}
\end{frame}

\newcommand\listCamlReraiseExn[1][]{
  \listasm[firstline=727,lastline=734,#1]{../ocaml/runtime/amd64.S}{runtime/amd64.S:727}}

\begin{frame}{Backtraces}{Introducing \funcname{caml\_reraise\_exn}}
  \begin{columns}[c]
    \begin{column}{0.5\textwidth}
      \minipage[c][0.66\textheight][s]{\columnwidth}
      \begin{itemize}
        \item<1-> The exception have to be reraised to the next handler
        \item<2-> First, checks if the backtrace should be collected with \funcarg{domain\_state->backtrace\_active}
        \only<3>{
        \begin{itemize}
            \item If not, behaves like a \funcname{raise\_notrace} with \funcname{RESTORE\_EXN\_HANDLER\_OCAML}
        \end{itemize}
        }
        \item<4-> Jumps into \funcname{caml\_raise\_exn}
        \item<5-> Calls \funcname{caml\_stash\_backtrace} again, to scan the next part of the stack: from the trap just pop-ed to the next trap
      \end{itemize}
      \vfill
      \mintocaml[firstline=15,lastline=21,highlightlines={18-21}]{../src/backtrace/backtrace.ml}
      \endminipage
    \end{column}
    \begin{column}{0.5\textwidth}
      \raggedleft
      \onslide*<1>{\listCamlReraiseExn{}}
      \onslide*<2>{\listCamlReraiseExn[highlightlines=729]{}}
      \onslide*<3>{
        \mintc[firstline=190,lastline=192,highlightlines={191-192}]{../ocaml/runtime/amd64.S}
        \mintc[firstline=234,lastline=237,highlightlines={235-237}]{../ocaml/runtime/amd64.S}
        \medskip
        \listCamlReraiseExn[highlightlines=731]{}
      }
      \onslide*<4-5>{
        % TODO refactor with \listCamlReraiseExn
        \mintasm[firstline=727,lastline=734,highlightlines=730]{../ocaml/runtime/amd64.S}
        \medskip
      }
      \onslide*<4>{\listCamlRaiseExn[highlightlines=711]{}}
      \onslide*<5>{\listCamlRaiseExn[highlightlines=720]{}}
    \end{column}
  \end{columns}
\end{frame}

\begin{frame}{Backtraces}{Backtrace collection continues}
  \begin{columns}[c]
    \begin{column}{0.55\textwidth}
      \minipage[c][0.75\textheight][s]{\columnwidth}
      \begin{itemize}
        \item<1-> \funcname{caml\_stash\_backtrace} scans the older part of the stack, up to the next trap
        \item<6-> Controls jumps to the nested exception handler in \funcname{maybe\_raise}, which matches only on \typename{ExnB}
        \item<7-> \funcname{caml\_reraise\_exn} is called again, backtrace collection resumes with \funcname{caml\_stash\_backtrace}
      \end{itemize}
      \medskip
      \begin{itemize}
        \item<2-> Constructed backtrace:
        \begin{itemize}
          \item <2-> \funcname{definitely\_raise}
          \item <2-> \funcname{probably\_raise}
          \item <3-> \funcname{probably\_raise} (re-raise)
          \item <4-> \funcname{maybe\_raise}
          \item <5-> \funcname{main}
          \item <8-> \funcname{main} (re-raise)
        \end{itemize}
      \end{itemize}
      \vfill
      \onslide*<1-5>{\mintocaml[firstline=15,lastline=21]{../src/backtrace/backtrace.ml}}
      \onslide*<6>{\mintocaml[firstline=28,lastline=34,highlightlines=31]{../src/backtrace/backtrace.ml}}
      \onslide*<7->{\mintocaml[firstline=28,lastline=34,highlightlines=33]{../src/backtrace/backtrace.ml}}
      \endminipage
    \end{column}
    \begin{column}{0.45\textwidth}
      \raggedleft
      \onslide*<1>{\providecommand\step{6}\input{diagrams/backtrace/backtrace.tex}}%
      \onslide*<2>{\providecommand\step{6}\input{diagrams/backtrace/backtrace.tex}}%
      \onslide*<3>{\providecommand\step{7}\input{diagrams/backtrace/backtrace.tex}}%
      \onslide*<4>{\providecommand\step{8}\input{diagrams/backtrace/backtrace.tex}}%
      \onslide*<5>{\providecommand\step{9}\input{diagrams/backtrace/backtrace.tex}}%
      \onslide*<6>{\providecommand\step{10}\input{diagrams/backtrace/backtrace.tex}}%
      \onslide*<7>{\providecommand\step{11}\input{diagrams/backtrace/backtrace.tex}}%
      \onslide*<8>{\providecommand\step{12}\input{diagrams/backtrace/backtrace.tex}}%
      \onslide*<9>{\providecommand\step{13}\input{diagrams/backtrace/backtrace.tex}}%
    \end{column}
  \end{columns}
\end{frame}

\begin{frame}{Backtraces}{Printing the backtrace}
  \begin{columns}[c]
    \begin{column}{0.45\textwidth}
      \minipage[c][0.66\textheight][s]{\columnwidth}
      \begin{itemize}
        \item<1-> The exception backtrace can now be printed to \funcarg{stdout} with \funcname{Printexc.print_backtrace}
        \item<2-> First the exception backtrace is retrieved into an OCaml array with \funcname{get_raw_backtrace }
        \item<3-> Which is implemented as the \funcname{caml_get_exception_raw_backtrace} C function in the runtime
        \item<4-> And finally printed with \funcname{print_exception_backtrace}
      \end{itemize}
      \vfill
      \mintocaml[firstline=28,lastline=34,highlightlines=33]{../src/backtrace/backtrace.ml}
      \endminipage
    \end{column}
    \begin{column}{0.55\textwidth}
      \onslide*<1,2>{
        \mintocaml[firstline=100,lastline=101]{../ocaml/stdlib/printexc.ml}
      }
      \only<1>{
        \listocaml[firstline=170,lastline=172]{../ocaml/stdlib/printexc.ml}{stdlib/printexc.ml:170}
      }
      \only<2>{
        \listocaml[firstline=170,lastline=172,highlightlines=172]{../ocaml/stdlib/printexc.ml}{stdlib/printexc.ml:170}
      }
      \only<3>{
        \mintc[firstline=154,lastline=156]{../ocaml/runtime/backtrace.c}
        \listc[firstline=171,lastline=192,highlightlines={184-187}]{../ocaml/runtime/backtrace.c}{runtime/backtrace.c:154}
      }
      \only<4>{
        \listocaml[firstline=155,lastline=165]{../ocaml/stdlib/printexc.ml}{stdlib/printexc.ml:55}
      }
    \end{column}
  \end{columns}
\end{frame}


\subsection{The multiple kinds of raise}
\frameSubsection{}{}

\begin{frame}{Backtraces}{When backtraces are not needed}
  \begin{columns}[c]
    \begin{column}{0.45\textwidth}
      \minipage[c][0.66\textheight][s]{\columnwidth}
      \begin{itemize}
        \item<1-> What if the backtrace for an exception is never needed?
        \item<2-> It's possible to raise the exception explicitly with \funcname{raise\_notrace} to make raising as cheap as possible
        \item<3-> An example of use in the \funcname{dynlink} library of the OCaml compiler
      \end{itemize}
      \vfill
      \onslide<3->{
      \listocaml[firstline=205,lastline=209,highlightlines=207]{../ocaml/otherlibs/dynlink/dynlink\_compilerlibs/misc.ml}{otherlibs/dynlink/dynlink\_compilerlibs/misc.ml:205}
      }
      \endminipage
    \end{column}
    \begin{column}{0.55\textwidth}
    \only<4>{
      \mintcmm[highlightlines=3]{../src/notrace/camlNotrace\_\_fun_347.cmm}
      \listcmm{../src/notrace/camlNotrace\_\_all\_somes\_327.cmm}{all\_somes}
    }
    \onslide*<5->{
      \listobjdump[highlightlines={6-9}]{../src/notrace/camlNotrace\_\_fun_347.objdump}{anonymous function from \funcname{all\_somes}}
    }
    \end{column}
  \end{columns}
\end{frame}

\begin{frame}{Backtraces}{Summary table of the multiple kinds of raise}
  \begin{itemize}
    \item When debugging information is enabled and if the backtrace of an exception isn't needed, it's possible to raise with \funcname{raise\_notrace} for performance
    \item When debugging information is \emph{not} enabled, the compiler optimises \funcname{raise} calls to inline assembly (same effect as using \funcname{raise\_notrace})
  \end{itemize}
  \bigskip
  \centering
  \begin{tabular}{| l | c | c | c | c |}
    \hline
    \parbox{10em}{Debug information \\ (\funcarg{-g} flag)}  & \multicolumn{2}{|c|}{Disabled}                                     & \multicolumn{2}{|c|}{Enabled} \\
    \hline
    Raise kind                                               & \funcname{raise} & \funcname{raise\_notrace}                       & \funcname{raise} & \funcname{raise\_notrace} \\
    \hline
    Compilation                                              & \parbox{8em}{inline assembly\\ (optimization)} & inline assembly   & \funcname{caml\_raise\_exn} & inline assembly \\
    \hline
  \end{tabular}
\end{frame}


%
% Takeaway #5
%
\subsection*{Takeaway \#5}
\frameSubsectionTakeaway{}
\againframe<5>{takeaway}
}%
      \onslide*<6>{\providecommand\step{10}%
% Backtraces
%
\subsection{One advantage of raising with debugging information}
\frameSubsection{}{}

\begin{frame}{Backtraces}{Debugging information \& source code}
  \begin{columns}[c]
    \begin{column}{0.5\textwidth}
      \begin{itemize}
        \item Raising an exception is fun and games, but how do we know where the exception originated? $\rightarrow$ With the backtrace associated with the exception!
        \item In order to get a backtrace from the point where an exception was raised, it's necessary to compile with debugging information enabled (\funcarg{-g} flag for the compiler)
        \item Same example as in the `Nested handlers' section
      \end{itemize}
    \end{column}
    \begin{column}{0.5\textwidth}
      \listocaml[firstline=9,lastline=34]{../src/backtrace/backtrace.ml}{backtrace/backtrace.ml}
    \end{column}
  \end{columns}
\end{frame}

\begin{frame}{Backtraces}{CMM and disassembly of \funcname{definitely\_raise}}
  \begin{columns}[c]
    \begin{column}{0.5\textwidth}
      \minipage[c][0.66\textheight][s]{\columnwidth}
      \begin{itemize}
        \item<1-> An effect of compiling with debugging information (\funcarg{-g}): the CMM no longer uses \funcname{raise\_notrace} to raise the exception but \funcname{raise} instead
        \item<2-> Disassembly shows that CMM \funcname{raise} is actually a call to \funcname{caml\_raise\_exn}, a function in assembler provided by the runtime
      \end{itemize}
      \vfill
      \mintocaml[firstline=9,lastline=13,highlightlines=12]{../src/backtrace/backtrace.ml}
      \endminipage
    \end{column}
    \begin{column}{0.5\textwidth}
      \onslide*<1>{
        \mintcmm[highlightlines=5]{../src/backtrace/camlBacktrace\_\_definitely\_raise\_347.cmm}
      }
      \onslide*<2>{
        \mintobjdump[firstline=2,highlightlines=14]{../src/backtrace/camlBacktrace\_\_definitely\_raise\_347.objdump}
      }
    \end{column}
  \end{columns}
\end{frame}

\begin{frame}{Backtraces}{Introducing \funcname{caml\_raise\_exn}}
  \begin{columns}[c]
    \begin{column}{0.5\textwidth}
      \minipage[c][0.66\textheight][s]{\columnwidth}
      \onslide*<1>{
        \begin{itemize}
          \item Raising the exception is performed using \funcname{caml\_raise\_exn} which is
            \begin{itemize}
              \item An assembly function provided by the runtime
              \item Architecture specific
            \end{itemize}
          \item Let's see what it does differently from \funcname{raise\_notrace}\dots
          \item We are about to raise \typename{ExnC} with \funcname{caml\_raise\_exn} from \funcname{definitely\_raise}
        \end{itemize}
      }
      \onslide*<2>{
        \mintobjdump[firstline=2,highlightlines=14]{../src/backtrace/camlBacktrace\_\_definitely\_raise\_347.objdump}
      }
      \vfill
      \mintocaml[firstline=9,lastline=13,highlightlines=12]{../src/backtrace/backtrace.ml}
      \endminipage
    \end{column}
    \begin{column}{0.5\textwidth}
      \centering
      \providecommand\step{1}\input{diagrams/backtrace/backtrace.tex}
    \end{column}
  \end{columns}
\end{frame}

\begin{frame}{Backtraces}{But before that, we need more fields from \typename{caml\_domain\_state}}
  \begin{columns}
    \begin{column}{0.5\textwidth}
      \begin{itemize}
        \item Remember \typename{caml\_domain\_state} the per-domain runtime state?
        \item Some other members are of interest to us now\dots
        \item \localname{backtrace\_active} is used to know if the runtime should collect backtraces
        \item \localname{backtrace\_active} can be set by
          \begin{itemize}
            \item The \funcarg{b} flag in the \funcname{OCAMLRUNPARAM} environment variable
            \item \mintinline{ocaml}{Printexc.record_backtrace : bool -> unit} \footnotemark
          \end{itemize}
      \end{itemize}
    \end{column}
    \begin{column}{0.5\textwidth}
      \begin{listing}
        \mintc[highlightlines={9-11}]{src/ocaml/domain\_state\_backtrace.h}
        \caption{runtime/caml/domain\_state.h}
      \end{listing}
    \end{column}
  \end{columns}
  \footnotetext[1]{https://v2.ocaml.org/api/Printexc.html}
\end{frame}

\newcommand\listCamlRaiseExn[1][]{
  \listasm[firstline=702,lastline=725,#1]{../ocaml/runtime/amd64.S}{runtime/amd64.S:702}}

\begin{frame}{Backtraces}{Walk through \funcname{caml\_raise\_exn}}
  \begin{columns}[T]
    \begin{column}{0.5\textwidth}
      \begin{itemize}
        \item<1-> First checks whether exceptions backtrace should be recorded
        \item<1-> By checking the \funcarg{domain\_state->backtrace\_active} value
        \item<2-> If not, acts as \funcname{raise\_notrace} with \funcname{RESTORE\_EXN\_HANDLER\_OCAML}
          \begin{itemize}
            \item Set \regname{RSP} to \localname{domain\_state->exn\_handler} value
            \item Restore \localname{domain\_state->exn\_handler} from trap
            \item Jump with \funcname{ret} (same effect as using \funcname{pop} and \funcname{jmp})
          \end{itemize}
        \item<3-> Otherwise, switches to the C stack to be able to execute C code
        \item<4-> Calls \funcname{caml\_stash\_backtrace}, a C function provided by the runtime\ignorespaces
          \onslide<6->{, with
            \begin{itemize}
              \item \funcarg{exn}: exception value
              \item \funcarg{pc}: code address of the raising point
              \item \funcarg{sp}: stack address of the raising function
              \item \funcarg{trapsp}: stack address of the next handler
            \end{itemize}
          }
      \end{itemize}
      \bigskip
      \mintocaml[firstline=9,lastline=13,highlightlines=12]{../src/backtrace/backtrace.ml}
    \end{column}
    \begin{column}{0.5\textwidth}
      \raggedleft
      \onslide*<1,3-4>{
        \mintc[firstline=190,lastline=192]{../ocaml/runtime/amd64.S}
        \mintc[firstline=234,lastline=237]{../ocaml/runtime/amd64.S}
      }
      \onslide*<2>{
        \mintc[firstline=190,lastline=192,highlightlines={191-192}]{../ocaml/runtime/amd64.S}
        \mintc[firstline=234,lastline=237,highlightlines={235-237}]{../ocaml/runtime/amd64.S}
      }
      \onslide*<1>{\listCamlRaiseExn[highlightlines=705]{}}%
      \onslide*<2>{\listCamlRaiseExn[highlightlines={707-708}]{}}%
      \onslide*<3>{\listCamlRaiseExn[highlightlines={712-714}]{}}%
      \onslide*<4>{\listCamlRaiseExn[highlightlines={715-720}]{}}%
      \onslide*<5>{\providecommand\step{1}\input{diagrams/backtrace/backtrace.tex}}%
      \onslide*<6>{\providecommand\step{2}\input{diagrams/backtrace/backtrace.tex}}%
    \end{column}
  \end{columns}
\end{frame}

\newcommand\listStashBacktrace[1][]{
  \listc[firstline=91,lastline=120,#1]{../ocaml/runtime/backtrace\_nat.c}{runtime/backtrace\_nat.c:91}}

\begin{frame}{Backtraces}{Introducing \funcname{caml\_stash\_backtrace}}
  \begin{columns}[c]
    \begin{column}{0.5\textwidth}
      \minipage[c][0.66\textheight][s]{\columnwidth}
      \begin{itemize}
        \item<1-> A C function provided by the runtime (specific to native code)
        \item<2-> It scans the stack, between the stack pointer at the raising point up to the next trap, in order to collect the names of the detected function frames
          \onslide*<2>{
            \begin{itemize}
              \item And more information, like line number, column number...
            \end{itemize}
          }
        \item<3-> It uses the same technique and information as the Garbage Collector (GC) uses to scan the values live on the stack
          \onslide*<4>{
            \begin{itemize}
              \item \typename{frame\_descr} are at the core of stack unwinding
            \end{itemize}
          }
      \end{itemize}
      \vfill
      \mintocaml[firstline=9,lastline=13,highlightlines=12]{../src/backtrace/backtrace.ml}
      \endminipage
    \end{column}
    \begin{column}{0.5\textwidth}
      \onslide*<1>{\listStashBacktrace[highlightlines=91]{}}
      \onslide*<2>{\listStashBacktrace[highlightlines={91,117-118}]{}}
      \onslide*<3->{\listStashBacktrace[highlightlines={94,106,109-110}]{}}
    \end{column}
  \end{columns}
\end{frame}

\newcommand\listFrameDescr[1][]{
  \listocaml[firstline=111,lastline=124,#1]{../ocaml/asmcomp/emitaux.ml}{asmcomp/emitaux.ml:111}}
\newcommand\listDebugInfo[1][]{
  \listocaml[firstline=38,lastline=49,#1]{../ocaml/lambda/debuginfo.mli}{lambda/debuginfo.mli:38}}

\begin{frame}{Backtraces}{\typename{frame\_descr} and \typename{frame\_debuginfo} in a nutshell}
  \begin{columns}[c]
    \begin{column}{0.5\textwidth}
      \begin{itemize}
        \item<1-> For every return address in an OCaml program, there is a associated \typename{frame\_descr}
        \item<2-> A \typename{frame\_descr} specifies
        \begin{itemize}
          \item<2-> The size of the function stack frame
          \item<3-> Where live values are on the stack (so that the GC can mark them as live and prevent from being swept)
        \end{itemize}
        \item<4-> When compiled with debug information, \typename{frame\_descr} will be linked to a \typename{frame\_debuginfo}
        \begin{itemize}
          \item<5-> Contains information about the position in the source code (file name, line and column number)
        \end{itemize}
      \end{itemize}
    \end{column}
    \begin{column}{0.5\textwidth}
        \onslide*<1>{\listFrameDescr[highlightlines={118-122}]{}}
        \onslide*<2>{\listFrameDescr[highlightlines={120}]{}}
        \onslide*<3>{\listFrameDescr[highlightlines={121}]{}}
        \onslide*<4->{\listFrameDescr[highlightlines={122}]{}}
        \medskip
        \onslide*<1-3>{\listDebugInfo{}}
        \onslide*<4>{\listDebugInfo[highlightlines={38,49}]{}}
        \onslide*<5>{\listDebugInfo[highlightlines={39-46}]{}}
    \end{column}
  \end{columns}
\end{frame}

\begin{frame}{Backtraces}{Scanning the stack with \typename{frame\_descr}}
  \begin{columns}[c]
    \begin{column}{0.5\textwidth}
      \begin{itemize}
        \item Given the return address of the code that raises the exception by calling \funcname{caml\_raise\_exn} (\funcarg{pc} argument of \funcname{caml\_stash\_backtrace})
        \item And the position of the stack pointer (\funcarg{sp} argument of \funcname{caml\_stash\_backtrace})
        \item The frame size given by \typename{frame\_descr}.\funcarg{frame\_size}, allows to find the end of the previous frame
        \item And the return address of the caller of this function
        \bigskip
        \onslide<3->{
        \item Note that traps (here the reddish \funcname{ExnA}, \funcname{ExnB} and \funcname{ExnC} blocks) belong to the frame that installed them, and so the frame size changes when traps are removed
        }
      \end{itemize}
    \end{column}
    \begin{column}{0.5\textwidth}
      \raggedleft
      \onslide*<1>{\providecommand\step{2}\input{diagrams/backtrace/backtrace.tex}}%
      \onslide*<2>{\providecommand\step{1}\input{diagrams/backtrace/scanstack.tex}}%
      \onslide*<3>{\providecommand\step{2}\input{diagrams/backtrace/scanstack.tex}}%
      \onslide*<4>{\providecommand\step{3}\input{diagrams/backtrace/scanstack.tex}}%
      \onslide*<5>{\providecommand\step{4}\input{diagrams/backtrace/scanstack.tex}}%
      \onslide*<6>{\providecommand\step{5}\input{diagrams/backtrace/scanstack.tex}}%
    \end{column}
  \end{columns}
\end{frame}

% Duplicate of previous-previous slide
\begin{frame}{Backtraces}{Continuing on \funcname{caml\_stash\_backtrace}}
  \begin{columns}[c]
    \begin{column}{0.5\textwidth}
      \minipage[c][0.75\textheight][s]{\columnwidth}
      \begin{itemize}
          \color{gray}
        \item A C function provided by the runtime (specific to native code)
        \item It scans the stack, between the stack pointer at the raising point up to the next trap, in order to collect the names of the detected function frames
        \item It uses the same technique and information as the Garbage Collector (GC) uses to scan the values live on the stack
      \end{itemize}
      \begin{itemize}
        \item For each function frame detected, store a pointer to the \typename{frame\_descr} in the \localname{domain\_state->backtrace\_buffer} array, effectively constructing the backtrace
      \end{itemize}
      \onslide<2->{
        \begin{itemize}
            \smallskip
          \item Constructed backtrace:
            \funcname{definitely\_raise},
            \onslide<3->{\funcname{probably\_raise}}
        \end{itemize}
      }
      \vfill
      \mintocaml[firstline=9,lastline=13,highlightlines=12]{../src/backtrace/backtrace.ml}
      \endminipage
    \end{column}
    \begin{column}{0.5\textwidth}
      \raggedleft
      \onslide*<1>{\listStashBacktrace[highlightlines={114-115}]{}}
      \onslide*<2>{\providecommand\step{3}\input{diagrams/backtrace/backtrace.tex}}%
      \onslide*<3>{\providecommand\step{4}\input{diagrams/backtrace/backtrace.tex}}%
    \end{column}
  \end{columns}
\end{frame}

\begin{frame}{Backtraces}{Back to \funcname{caml\_raise\_exn}}
  \begin{columns}[c]
    \begin{column}{0.5\textwidth}
      \minipage[c][0.66\textheight][s]{\columnwidth}
      \begin{itemize}\color{gray}
        \item Checks wheteher exceptions backtrace should be recorded, by checking the \funcarg{domain\_state->backtrace\_active} value
        \item Switches to the C stack to be able to execute C code
        \item Calls \funcname{caml\_stash\_backtrace}, to collect the backtrace up to the next trap
      \end{itemize}
      \onslide*<2->{
        \begin{itemize}
          \item<2-> Executes the exception handler in \funcname{probably\_raise} with \funcname{RESTORE\_EXN\_HANDLER\_OCAML}
            \begin{itemize}
              \item Set \regname{RSP} to \localname{domain\_state->exn\_handler} value
              \item Restore \localname{domain\_state->exn\_handler} from trap
              \item Jump to the handler code with \funcname{ret}
            \end{itemize}
        \end{itemize}
      }
      \vfill
      \onslide*<1>{\mintocaml[firstline=9,lastline=13,highlightlines=12]{../src/backtrace/backtrace.ml}}
      \onslide*<2->{\mintocaml[firstline=15,lastline=21,highlightlines={19-21}]{../src/backtrace/backtrace.ml}}
      \endminipage
    \end{column}
    \begin{column}{0.5\textwidth}
      \centering
      \onslide*<1>{
        \mintc[firstline=190,lastline=192]{../ocaml/runtime/amd64.S}
        \mintc[firstline=234,lastline=237]{../ocaml/runtime/amd64.S}
      }
      \onslide*<2>{
        \mintc[firstline=190,lastline=192,highlightlines={191-192}]{../ocaml/runtime/amd64.S}
        \mintc[firstline=234,lastline=237,highlightlines={235-237}]{../ocaml/runtime/amd64.S}
      }
      \onslide*<1>{\listCamlRaiseExn[highlightlines=720]{}}
      \onslide*<2>{\listCamlRaiseExn[highlightlines=722]{}}
      \onslide*<3->{\providecommand\step{5}\input{diagrams/backtrace/backtrace.tex}}
    \end{column}
  \end{columns}
\end{frame}

\begin{frame}{Backtraces}{\funcname{caml\_probably\_raise}'s handler doesn't catch \typename{ExnC}}
  \begin{columns}[c]
    \begin{column}{0.45\textwidth}
      \minipage[c][0.75\textheight][s]{\columnwidth}
      \begin{itemize}
        \item<1-> \funcname{match \dots with}\footnotemark pattern matching can also match exceptions
        \item<1-> The CMM of \funcname{probably\_raise} shows that this \funcname{match \dots with} is implemented with a pattern matching and a \funcname{try \dots with}
        \item<1-> But this one only matches on \typename{ExnA} exception
        \item<2-> Since the pattern matching fails to catch the exception, \funcname{reraise} is called with our current \typename{ExnC} exception
        \item<3-> Disassembly shows that \funcname{reraise} is actually a call to \funcname{caml\_reraise\_exn}, which is also an assembly function provided by the runtime
      \end{itemize}
      \vfill
      \mintocaml[firstline=15,lastline=21,highlightlines={18-21}]{../src/backtrace/backtrace.ml}
      \endminipage
    \end{column}
    \begin{column}{0.55\textwidth}
      \centering
      \onslide*<1>{\mintcmm[highlightlines={10-18}]{../src/backtrace/camlBacktrace\_\_probably\_raise\_351.cmm}}
      \onslide*<2>{\listcmm[highlightlines=18]{../src/backtrace/camlBacktrace\_\_probably\_raise_351.cmm}{CMM of probably\_raise}}
      \onslide*<3>{\mintobjdump[firstline=5,lastline=35,highlightlines=31]{../src/backtrace/camlBacktrace\_\_probably\_raise_351.objdump}}
    \end{column}
  \end{columns}
  \only<1>{\footnotetext[1]{https://blog.janestreet.com/pattern-matching-and-exception-handling-unite/}}
\end{frame}

\newcommand\listCamlReraiseExn[1][]{
  \listasm[firstline=727,lastline=734,#1]{../ocaml/runtime/amd64.S}{runtime/amd64.S:727}}

\begin{frame}{Backtraces}{Introducing \funcname{caml\_reraise\_exn}}
  \begin{columns}[c]
    \begin{column}{0.5\textwidth}
      \minipage[c][0.66\textheight][s]{\columnwidth}
      \begin{itemize}
        \item<1-> The exception have to be reraised to the next handler
        \item<2-> First, checks if the backtrace should be collected with \funcarg{domain\_state->backtrace\_active}
        \only<3>{
        \begin{itemize}
            \item If not, behaves like a \funcname{raise\_notrace} with \funcname{RESTORE\_EXN\_HANDLER\_OCAML}
        \end{itemize}
        }
        \item<4-> Jumps into \funcname{caml\_raise\_exn}
        \item<5-> Calls \funcname{caml\_stash\_backtrace} again, to scan the next part of the stack: from the trap just pop-ed to the next trap
      \end{itemize}
      \vfill
      \mintocaml[firstline=15,lastline=21,highlightlines={18-21}]{../src/backtrace/backtrace.ml}
      \endminipage
    \end{column}
    \begin{column}{0.5\textwidth}
      \raggedleft
      \onslide*<1>{\listCamlReraiseExn{}}
      \onslide*<2>{\listCamlReraiseExn[highlightlines=729]{}}
      \onslide*<3>{
        \mintc[firstline=190,lastline=192,highlightlines={191-192}]{../ocaml/runtime/amd64.S}
        \mintc[firstline=234,lastline=237,highlightlines={235-237}]{../ocaml/runtime/amd64.S}
        \medskip
        \listCamlReraiseExn[highlightlines=731]{}
      }
      \onslide*<4-5>{
        % TODO refactor with \listCamlReraiseExn
        \mintasm[firstline=727,lastline=734,highlightlines=730]{../ocaml/runtime/amd64.S}
        \medskip
      }
      \onslide*<4>{\listCamlRaiseExn[highlightlines=711]{}}
      \onslide*<5>{\listCamlRaiseExn[highlightlines=720]{}}
    \end{column}
  \end{columns}
\end{frame}

\begin{frame}{Backtraces}{Backtrace collection continues}
  \begin{columns}[c]
    \begin{column}{0.55\textwidth}
      \minipage[c][0.75\textheight][s]{\columnwidth}
      \begin{itemize}
        \item<1-> \funcname{caml\_stash\_backtrace} scans the older part of the stack, up to the next trap
        \item<6-> Controls jumps to the nested exception handler in \funcname{maybe\_raise}, which matches only on \typename{ExnB}
        \item<7-> \funcname{caml\_reraise\_exn} is called again, backtrace collection resumes with \funcname{caml\_stash\_backtrace}
      \end{itemize}
      \medskip
      \begin{itemize}
        \item<2-> Constructed backtrace:
        \begin{itemize}
          \item <2-> \funcname{definitely\_raise}
          \item <2-> \funcname{probably\_raise}
          \item <3-> \funcname{probably\_raise} (re-raise)
          \item <4-> \funcname{maybe\_raise}
          \item <5-> \funcname{main}
          \item <8-> \funcname{main} (re-raise)
        \end{itemize}
      \end{itemize}
      \vfill
      \onslide*<1-5>{\mintocaml[firstline=15,lastline=21]{../src/backtrace/backtrace.ml}}
      \onslide*<6>{\mintocaml[firstline=28,lastline=34,highlightlines=31]{../src/backtrace/backtrace.ml}}
      \onslide*<7->{\mintocaml[firstline=28,lastline=34,highlightlines=33]{../src/backtrace/backtrace.ml}}
      \endminipage
    \end{column}
    \begin{column}{0.45\textwidth}
      \raggedleft
      \onslide*<1>{\providecommand\step{6}\input{diagrams/backtrace/backtrace.tex}}%
      \onslide*<2>{\providecommand\step{6}\input{diagrams/backtrace/backtrace.tex}}%
      \onslide*<3>{\providecommand\step{7}\input{diagrams/backtrace/backtrace.tex}}%
      \onslide*<4>{\providecommand\step{8}\input{diagrams/backtrace/backtrace.tex}}%
      \onslide*<5>{\providecommand\step{9}\input{diagrams/backtrace/backtrace.tex}}%
      \onslide*<6>{\providecommand\step{10}\input{diagrams/backtrace/backtrace.tex}}%
      \onslide*<7>{\providecommand\step{11}\input{diagrams/backtrace/backtrace.tex}}%
      \onslide*<8>{\providecommand\step{12}\input{diagrams/backtrace/backtrace.tex}}%
      \onslide*<9>{\providecommand\step{13}\input{diagrams/backtrace/backtrace.tex}}%
    \end{column}
  \end{columns}
\end{frame}

\begin{frame}{Backtraces}{Printing the backtrace}
  \begin{columns}[c]
    \begin{column}{0.45\textwidth}
      \minipage[c][0.66\textheight][s]{\columnwidth}
      \begin{itemize}
        \item<1-> The exception backtrace can now be printed to \funcarg{stdout} with \funcname{Printexc.print_backtrace}
        \item<2-> First the exception backtrace is retrieved into an OCaml array with \funcname{get_raw_backtrace }
        \item<3-> Which is implemented as the \funcname{caml_get_exception_raw_backtrace} C function in the runtime
        \item<4-> And finally printed with \funcname{print_exception_backtrace}
      \end{itemize}
      \vfill
      \mintocaml[firstline=28,lastline=34,highlightlines=33]{../src/backtrace/backtrace.ml}
      \endminipage
    \end{column}
    \begin{column}{0.55\textwidth}
      \onslide*<1,2>{
        \mintocaml[firstline=100,lastline=101]{../ocaml/stdlib/printexc.ml}
      }
      \only<1>{
        \listocaml[firstline=170,lastline=172]{../ocaml/stdlib/printexc.ml}{stdlib/printexc.ml:170}
      }
      \only<2>{
        \listocaml[firstline=170,lastline=172,highlightlines=172]{../ocaml/stdlib/printexc.ml}{stdlib/printexc.ml:170}
      }
      \only<3>{
        \mintc[firstline=154,lastline=156]{../ocaml/runtime/backtrace.c}
        \listc[firstline=171,lastline=192,highlightlines={184-187}]{../ocaml/runtime/backtrace.c}{runtime/backtrace.c:154}
      }
      \only<4>{
        \listocaml[firstline=155,lastline=165]{../ocaml/stdlib/printexc.ml}{stdlib/printexc.ml:55}
      }
    \end{column}
  \end{columns}
\end{frame}


\subsection{The multiple kinds of raise}
\frameSubsection{}{}

\begin{frame}{Backtraces}{When backtraces are not needed}
  \begin{columns}[c]
    \begin{column}{0.45\textwidth}
      \minipage[c][0.66\textheight][s]{\columnwidth}
      \begin{itemize}
        \item<1-> What if the backtrace for an exception is never needed?
        \item<2-> It's possible to raise the exception explicitly with \funcname{raise\_notrace} to make raising as cheap as possible
        \item<3-> An example of use in the \funcname{dynlink} library of the OCaml compiler
      \end{itemize}
      \vfill
      \onslide<3->{
      \listocaml[firstline=205,lastline=209,highlightlines=207]{../ocaml/otherlibs/dynlink/dynlink\_compilerlibs/misc.ml}{otherlibs/dynlink/dynlink\_compilerlibs/misc.ml:205}
      }
      \endminipage
    \end{column}
    \begin{column}{0.55\textwidth}
    \only<4>{
      \mintcmm[highlightlines=3]{../src/notrace/camlNotrace\_\_fun_347.cmm}
      \listcmm{../src/notrace/camlNotrace\_\_all\_somes\_327.cmm}{all\_somes}
    }
    \onslide*<5->{
      \listobjdump[highlightlines={6-9}]{../src/notrace/camlNotrace\_\_fun_347.objdump}{anonymous function from \funcname{all\_somes}}
    }
    \end{column}
  \end{columns}
\end{frame}

\begin{frame}{Backtraces}{Summary table of the multiple kinds of raise}
  \begin{itemize}
    \item When debugging information is enabled and if the backtrace of an exception isn't needed, it's possible to raise with \funcname{raise\_notrace} for performance
    \item When debugging information is \emph{not} enabled, the compiler optimises \funcname{raise} calls to inline assembly (same effect as using \funcname{raise\_notrace})
  \end{itemize}
  \bigskip
  \centering
  \begin{tabular}{| l | c | c | c | c |}
    \hline
    \parbox{10em}{Debug information \\ (\funcarg{-g} flag)}  & \multicolumn{2}{|c|}{Disabled}                                     & \multicolumn{2}{|c|}{Enabled} \\
    \hline
    Raise kind                                               & \funcname{raise} & \funcname{raise\_notrace}                       & \funcname{raise} & \funcname{raise\_notrace} \\
    \hline
    Compilation                                              & \parbox{8em}{inline assembly\\ (optimization)} & inline assembly   & \funcname{caml\_raise\_exn} & inline assembly \\
    \hline
  \end{tabular}
\end{frame}


%
% Takeaway #5
%
\subsection*{Takeaway \#5}
\frameSubsectionTakeaway{}
\againframe<5>{takeaway}
}%
      \onslide*<7>{\providecommand\step{11}%
% Backtraces
%
\subsection{One advantage of raising with debugging information}
\frameSubsection{}{}

\begin{frame}{Backtraces}{Debugging information \& source code}
  \begin{columns}[c]
    \begin{column}{0.5\textwidth}
      \begin{itemize}
        \item Raising an exception is fun and games, but how do we know where the exception originated? $\rightarrow$ With the backtrace associated with the exception!
        \item In order to get a backtrace from the point where an exception was raised, it's necessary to compile with debugging information enabled (\funcarg{-g} flag for the compiler)
        \item Same example as in the `Nested handlers' section
      \end{itemize}
    \end{column}
    \begin{column}{0.5\textwidth}
      \listocaml[firstline=9,lastline=34]{../src/backtrace/backtrace.ml}{backtrace/backtrace.ml}
    \end{column}
  \end{columns}
\end{frame}

\begin{frame}{Backtraces}{CMM and disassembly of \funcname{definitely\_raise}}
  \begin{columns}[c]
    \begin{column}{0.5\textwidth}
      \minipage[c][0.66\textheight][s]{\columnwidth}
      \begin{itemize}
        \item<1-> An effect of compiling with debugging information (\funcarg{-g}): the CMM no longer uses \funcname{raise\_notrace} to raise the exception but \funcname{raise} instead
        \item<2-> Disassembly shows that CMM \funcname{raise} is actually a call to \funcname{caml\_raise\_exn}, a function in assembler provided by the runtime
      \end{itemize}
      \vfill
      \mintocaml[firstline=9,lastline=13,highlightlines=12]{../src/backtrace/backtrace.ml}
      \endminipage
    \end{column}
    \begin{column}{0.5\textwidth}
      \onslide*<1>{
        \mintcmm[highlightlines=5]{../src/backtrace/camlBacktrace\_\_definitely\_raise\_347.cmm}
      }
      \onslide*<2>{
        \mintobjdump[firstline=2,highlightlines=14]{../src/backtrace/camlBacktrace\_\_definitely\_raise\_347.objdump}
      }
    \end{column}
  \end{columns}
\end{frame}

\begin{frame}{Backtraces}{Introducing \funcname{caml\_raise\_exn}}
  \begin{columns}[c]
    \begin{column}{0.5\textwidth}
      \minipage[c][0.66\textheight][s]{\columnwidth}
      \onslide*<1>{
        \begin{itemize}
          \item Raising the exception is performed using \funcname{caml\_raise\_exn} which is
            \begin{itemize}
              \item An assembly function provided by the runtime
              \item Architecture specific
            \end{itemize}
          \item Let's see what it does differently from \funcname{raise\_notrace}\dots
          \item We are about to raise \typename{ExnC} with \funcname{caml\_raise\_exn} from \funcname{definitely\_raise}
        \end{itemize}
      }
      \onslide*<2>{
        \mintobjdump[firstline=2,highlightlines=14]{../src/backtrace/camlBacktrace\_\_definitely\_raise\_347.objdump}
      }
      \vfill
      \mintocaml[firstline=9,lastline=13,highlightlines=12]{../src/backtrace/backtrace.ml}
      \endminipage
    \end{column}
    \begin{column}{0.5\textwidth}
      \centering
      \providecommand\step{1}\input{diagrams/backtrace/backtrace.tex}
    \end{column}
  \end{columns}
\end{frame}

\begin{frame}{Backtraces}{But before that, we need more fields from \typename{caml\_domain\_state}}
  \begin{columns}
    \begin{column}{0.5\textwidth}
      \begin{itemize}
        \item Remember \typename{caml\_domain\_state} the per-domain runtime state?
        \item Some other members are of interest to us now\dots
        \item \localname{backtrace\_active} is used to know if the runtime should collect backtraces
        \item \localname{backtrace\_active} can be set by
          \begin{itemize}
            \item The \funcarg{b} flag in the \funcname{OCAMLRUNPARAM} environment variable
            \item \mintinline{ocaml}{Printexc.record_backtrace : bool -> unit} \footnotemark
          \end{itemize}
      \end{itemize}
    \end{column}
    \begin{column}{0.5\textwidth}
      \begin{listing}
        \mintc[highlightlines={9-11}]{src/ocaml/domain\_state\_backtrace.h}
        \caption{runtime/caml/domain\_state.h}
      \end{listing}
    \end{column}
  \end{columns}
  \footnotetext[1]{https://v2.ocaml.org/api/Printexc.html}
\end{frame}

\newcommand\listCamlRaiseExn[1][]{
  \listasm[firstline=702,lastline=725,#1]{../ocaml/runtime/amd64.S}{runtime/amd64.S:702}}

\begin{frame}{Backtraces}{Walk through \funcname{caml\_raise\_exn}}
  \begin{columns}[T]
    \begin{column}{0.5\textwidth}
      \begin{itemize}
        \item<1-> First checks whether exceptions backtrace should be recorded
        \item<1-> By checking the \funcarg{domain\_state->backtrace\_active} value
        \item<2-> If not, acts as \funcname{raise\_notrace} with \funcname{RESTORE\_EXN\_HANDLER\_OCAML}
          \begin{itemize}
            \item Set \regname{RSP} to \localname{domain\_state->exn\_handler} value
            \item Restore \localname{domain\_state->exn\_handler} from trap
            \item Jump with \funcname{ret} (same effect as using \funcname{pop} and \funcname{jmp})
          \end{itemize}
        \item<3-> Otherwise, switches to the C stack to be able to execute C code
        \item<4-> Calls \funcname{caml\_stash\_backtrace}, a C function provided by the runtime\ignorespaces
          \onslide<6->{, with
            \begin{itemize}
              \item \funcarg{exn}: exception value
              \item \funcarg{pc}: code address of the raising point
              \item \funcarg{sp}: stack address of the raising function
              \item \funcarg{trapsp}: stack address of the next handler
            \end{itemize}
          }
      \end{itemize}
      \bigskip
      \mintocaml[firstline=9,lastline=13,highlightlines=12]{../src/backtrace/backtrace.ml}
    \end{column}
    \begin{column}{0.5\textwidth}
      \raggedleft
      \onslide*<1,3-4>{
        \mintc[firstline=190,lastline=192]{../ocaml/runtime/amd64.S}
        \mintc[firstline=234,lastline=237]{../ocaml/runtime/amd64.S}
      }
      \onslide*<2>{
        \mintc[firstline=190,lastline=192,highlightlines={191-192}]{../ocaml/runtime/amd64.S}
        \mintc[firstline=234,lastline=237,highlightlines={235-237}]{../ocaml/runtime/amd64.S}
      }
      \onslide*<1>{\listCamlRaiseExn[highlightlines=705]{}}%
      \onslide*<2>{\listCamlRaiseExn[highlightlines={707-708}]{}}%
      \onslide*<3>{\listCamlRaiseExn[highlightlines={712-714}]{}}%
      \onslide*<4>{\listCamlRaiseExn[highlightlines={715-720}]{}}%
      \onslide*<5>{\providecommand\step{1}\input{diagrams/backtrace/backtrace.tex}}%
      \onslide*<6>{\providecommand\step{2}\input{diagrams/backtrace/backtrace.tex}}%
    \end{column}
  \end{columns}
\end{frame}

\newcommand\listStashBacktrace[1][]{
  \listc[firstline=91,lastline=120,#1]{../ocaml/runtime/backtrace\_nat.c}{runtime/backtrace\_nat.c:91}}

\begin{frame}{Backtraces}{Introducing \funcname{caml\_stash\_backtrace}}
  \begin{columns}[c]
    \begin{column}{0.5\textwidth}
      \minipage[c][0.66\textheight][s]{\columnwidth}
      \begin{itemize}
        \item<1-> A C function provided by the runtime (specific to native code)
        \item<2-> It scans the stack, between the stack pointer at the raising point up to the next trap, in order to collect the names of the detected function frames
          \onslide*<2>{
            \begin{itemize}
              \item And more information, like line number, column number...
            \end{itemize}
          }
        \item<3-> It uses the same technique and information as the Garbage Collector (GC) uses to scan the values live on the stack
          \onslide*<4>{
            \begin{itemize}
              \item \typename{frame\_descr} are at the core of stack unwinding
            \end{itemize}
          }
      \end{itemize}
      \vfill
      \mintocaml[firstline=9,lastline=13,highlightlines=12]{../src/backtrace/backtrace.ml}
      \endminipage
    \end{column}
    \begin{column}{0.5\textwidth}
      \onslide*<1>{\listStashBacktrace[highlightlines=91]{}}
      \onslide*<2>{\listStashBacktrace[highlightlines={91,117-118}]{}}
      \onslide*<3->{\listStashBacktrace[highlightlines={94,106,109-110}]{}}
    \end{column}
  \end{columns}
\end{frame}

\newcommand\listFrameDescr[1][]{
  \listocaml[firstline=111,lastline=124,#1]{../ocaml/asmcomp/emitaux.ml}{asmcomp/emitaux.ml:111}}
\newcommand\listDebugInfo[1][]{
  \listocaml[firstline=38,lastline=49,#1]{../ocaml/lambda/debuginfo.mli}{lambda/debuginfo.mli:38}}

\begin{frame}{Backtraces}{\typename{frame\_descr} and \typename{frame\_debuginfo} in a nutshell}
  \begin{columns}[c]
    \begin{column}{0.5\textwidth}
      \begin{itemize}
        \item<1-> For every return address in an OCaml program, there is a associated \typename{frame\_descr}
        \item<2-> A \typename{frame\_descr} specifies
        \begin{itemize}
          \item<2-> The size of the function stack frame
          \item<3-> Where live values are on the stack (so that the GC can mark them as live and prevent from being swept)
        \end{itemize}
        \item<4-> When compiled with debug information, \typename{frame\_descr} will be linked to a \typename{frame\_debuginfo}
        \begin{itemize}
          \item<5-> Contains information about the position in the source code (file name, line and column number)
        \end{itemize}
      \end{itemize}
    \end{column}
    \begin{column}{0.5\textwidth}
        \onslide*<1>{\listFrameDescr[highlightlines={118-122}]{}}
        \onslide*<2>{\listFrameDescr[highlightlines={120}]{}}
        \onslide*<3>{\listFrameDescr[highlightlines={121}]{}}
        \onslide*<4->{\listFrameDescr[highlightlines={122}]{}}
        \medskip
        \onslide*<1-3>{\listDebugInfo{}}
        \onslide*<4>{\listDebugInfo[highlightlines={38,49}]{}}
        \onslide*<5>{\listDebugInfo[highlightlines={39-46}]{}}
    \end{column}
  \end{columns}
\end{frame}

\begin{frame}{Backtraces}{Scanning the stack with \typename{frame\_descr}}
  \begin{columns}[c]
    \begin{column}{0.5\textwidth}
      \begin{itemize}
        \item Given the return address of the code that raises the exception by calling \funcname{caml\_raise\_exn} (\funcarg{pc} argument of \funcname{caml\_stash\_backtrace})
        \item And the position of the stack pointer (\funcarg{sp} argument of \funcname{caml\_stash\_backtrace})
        \item The frame size given by \typename{frame\_descr}.\funcarg{frame\_size}, allows to find the end of the previous frame
        \item And the return address of the caller of this function
        \bigskip
        \onslide<3->{
        \item Note that traps (here the reddish \funcname{ExnA}, \funcname{ExnB} and \funcname{ExnC} blocks) belong to the frame that installed them, and so the frame size changes when traps are removed
        }
      \end{itemize}
    \end{column}
    \begin{column}{0.5\textwidth}
      \raggedleft
      \onslide*<1>{\providecommand\step{2}\input{diagrams/backtrace/backtrace.tex}}%
      \onslide*<2>{\providecommand\step{1}\input{diagrams/backtrace/scanstack.tex}}%
      \onslide*<3>{\providecommand\step{2}\input{diagrams/backtrace/scanstack.tex}}%
      \onslide*<4>{\providecommand\step{3}\input{diagrams/backtrace/scanstack.tex}}%
      \onslide*<5>{\providecommand\step{4}\input{diagrams/backtrace/scanstack.tex}}%
      \onslide*<6>{\providecommand\step{5}\input{diagrams/backtrace/scanstack.tex}}%
    \end{column}
  \end{columns}
\end{frame}

% Duplicate of previous-previous slide
\begin{frame}{Backtraces}{Continuing on \funcname{caml\_stash\_backtrace}}
  \begin{columns}[c]
    \begin{column}{0.5\textwidth}
      \minipage[c][0.75\textheight][s]{\columnwidth}
      \begin{itemize}
          \color{gray}
        \item A C function provided by the runtime (specific to native code)
        \item It scans the stack, between the stack pointer at the raising point up to the next trap, in order to collect the names of the detected function frames
        \item It uses the same technique and information as the Garbage Collector (GC) uses to scan the values live on the stack
      \end{itemize}
      \begin{itemize}
        \item For each function frame detected, store a pointer to the \typename{frame\_descr} in the \localname{domain\_state->backtrace\_buffer} array, effectively constructing the backtrace
      \end{itemize}
      \onslide<2->{
        \begin{itemize}
            \smallskip
          \item Constructed backtrace:
            \funcname{definitely\_raise},
            \onslide<3->{\funcname{probably\_raise}}
        \end{itemize}
      }
      \vfill
      \mintocaml[firstline=9,lastline=13,highlightlines=12]{../src/backtrace/backtrace.ml}
      \endminipage
    \end{column}
    \begin{column}{0.5\textwidth}
      \raggedleft
      \onslide*<1>{\listStashBacktrace[highlightlines={114-115}]{}}
      \onslide*<2>{\providecommand\step{3}\input{diagrams/backtrace/backtrace.tex}}%
      \onslide*<3>{\providecommand\step{4}\input{diagrams/backtrace/backtrace.tex}}%
    \end{column}
  \end{columns}
\end{frame}

\begin{frame}{Backtraces}{Back to \funcname{caml\_raise\_exn}}
  \begin{columns}[c]
    \begin{column}{0.5\textwidth}
      \minipage[c][0.66\textheight][s]{\columnwidth}
      \begin{itemize}\color{gray}
        \item Checks wheteher exceptions backtrace should be recorded, by checking the \funcarg{domain\_state->backtrace\_active} value
        \item Switches to the C stack to be able to execute C code
        \item Calls \funcname{caml\_stash\_backtrace}, to collect the backtrace up to the next trap
      \end{itemize}
      \onslide*<2->{
        \begin{itemize}
          \item<2-> Executes the exception handler in \funcname{probably\_raise} with \funcname{RESTORE\_EXN\_HANDLER\_OCAML}
            \begin{itemize}
              \item Set \regname{RSP} to \localname{domain\_state->exn\_handler} value
              \item Restore \localname{domain\_state->exn\_handler} from trap
              \item Jump to the handler code with \funcname{ret}
            \end{itemize}
        \end{itemize}
      }
      \vfill
      \onslide*<1>{\mintocaml[firstline=9,lastline=13,highlightlines=12]{../src/backtrace/backtrace.ml}}
      \onslide*<2->{\mintocaml[firstline=15,lastline=21,highlightlines={19-21}]{../src/backtrace/backtrace.ml}}
      \endminipage
    \end{column}
    \begin{column}{0.5\textwidth}
      \centering
      \onslide*<1>{
        \mintc[firstline=190,lastline=192]{../ocaml/runtime/amd64.S}
        \mintc[firstline=234,lastline=237]{../ocaml/runtime/amd64.S}
      }
      \onslide*<2>{
        \mintc[firstline=190,lastline=192,highlightlines={191-192}]{../ocaml/runtime/amd64.S}
        \mintc[firstline=234,lastline=237,highlightlines={235-237}]{../ocaml/runtime/amd64.S}
      }
      \onslide*<1>{\listCamlRaiseExn[highlightlines=720]{}}
      \onslide*<2>{\listCamlRaiseExn[highlightlines=722]{}}
      \onslide*<3->{\providecommand\step{5}\input{diagrams/backtrace/backtrace.tex}}
    \end{column}
  \end{columns}
\end{frame}

\begin{frame}{Backtraces}{\funcname{caml\_probably\_raise}'s handler doesn't catch \typename{ExnC}}
  \begin{columns}[c]
    \begin{column}{0.45\textwidth}
      \minipage[c][0.75\textheight][s]{\columnwidth}
      \begin{itemize}
        \item<1-> \funcname{match \dots with}\footnotemark pattern matching can also match exceptions
        \item<1-> The CMM of \funcname{probably\_raise} shows that this \funcname{match \dots with} is implemented with a pattern matching and a \funcname{try \dots with}
        \item<1-> But this one only matches on \typename{ExnA} exception
        \item<2-> Since the pattern matching fails to catch the exception, \funcname{reraise} is called with our current \typename{ExnC} exception
        \item<3-> Disassembly shows that \funcname{reraise} is actually a call to \funcname{caml\_reraise\_exn}, which is also an assembly function provided by the runtime
      \end{itemize}
      \vfill
      \mintocaml[firstline=15,lastline=21,highlightlines={18-21}]{../src/backtrace/backtrace.ml}
      \endminipage
    \end{column}
    \begin{column}{0.55\textwidth}
      \centering
      \onslide*<1>{\mintcmm[highlightlines={10-18}]{../src/backtrace/camlBacktrace\_\_probably\_raise\_351.cmm}}
      \onslide*<2>{\listcmm[highlightlines=18]{../src/backtrace/camlBacktrace\_\_probably\_raise_351.cmm}{CMM of probably\_raise}}
      \onslide*<3>{\mintobjdump[firstline=5,lastline=35,highlightlines=31]{../src/backtrace/camlBacktrace\_\_probably\_raise_351.objdump}}
    \end{column}
  \end{columns}
  \only<1>{\footnotetext[1]{https://blog.janestreet.com/pattern-matching-and-exception-handling-unite/}}
\end{frame}

\newcommand\listCamlReraiseExn[1][]{
  \listasm[firstline=727,lastline=734,#1]{../ocaml/runtime/amd64.S}{runtime/amd64.S:727}}

\begin{frame}{Backtraces}{Introducing \funcname{caml\_reraise\_exn}}
  \begin{columns}[c]
    \begin{column}{0.5\textwidth}
      \minipage[c][0.66\textheight][s]{\columnwidth}
      \begin{itemize}
        \item<1-> The exception have to be reraised to the next handler
        \item<2-> First, checks if the backtrace should be collected with \funcarg{domain\_state->backtrace\_active}
        \only<3>{
        \begin{itemize}
            \item If not, behaves like a \funcname{raise\_notrace} with \funcname{RESTORE\_EXN\_HANDLER\_OCAML}
        \end{itemize}
        }
        \item<4-> Jumps into \funcname{caml\_raise\_exn}
        \item<5-> Calls \funcname{caml\_stash\_backtrace} again, to scan the next part of the stack: from the trap just pop-ed to the next trap
      \end{itemize}
      \vfill
      \mintocaml[firstline=15,lastline=21,highlightlines={18-21}]{../src/backtrace/backtrace.ml}
      \endminipage
    \end{column}
    \begin{column}{0.5\textwidth}
      \raggedleft
      \onslide*<1>{\listCamlReraiseExn{}}
      \onslide*<2>{\listCamlReraiseExn[highlightlines=729]{}}
      \onslide*<3>{
        \mintc[firstline=190,lastline=192,highlightlines={191-192}]{../ocaml/runtime/amd64.S}
        \mintc[firstline=234,lastline=237,highlightlines={235-237}]{../ocaml/runtime/amd64.S}
        \medskip
        \listCamlReraiseExn[highlightlines=731]{}
      }
      \onslide*<4-5>{
        % TODO refactor with \listCamlReraiseExn
        \mintasm[firstline=727,lastline=734,highlightlines=730]{../ocaml/runtime/amd64.S}
        \medskip
      }
      \onslide*<4>{\listCamlRaiseExn[highlightlines=711]{}}
      \onslide*<5>{\listCamlRaiseExn[highlightlines=720]{}}
    \end{column}
  \end{columns}
\end{frame}

\begin{frame}{Backtraces}{Backtrace collection continues}
  \begin{columns}[c]
    \begin{column}{0.55\textwidth}
      \minipage[c][0.75\textheight][s]{\columnwidth}
      \begin{itemize}
        \item<1-> \funcname{caml\_stash\_backtrace} scans the older part of the stack, up to the next trap
        \item<6-> Controls jumps to the nested exception handler in \funcname{maybe\_raise}, which matches only on \typename{ExnB}
        \item<7-> \funcname{caml\_reraise\_exn} is called again, backtrace collection resumes with \funcname{caml\_stash\_backtrace}
      \end{itemize}
      \medskip
      \begin{itemize}
        \item<2-> Constructed backtrace:
        \begin{itemize}
          \item <2-> \funcname{definitely\_raise}
          \item <2-> \funcname{probably\_raise}
          \item <3-> \funcname{probably\_raise} (re-raise)
          \item <4-> \funcname{maybe\_raise}
          \item <5-> \funcname{main}
          \item <8-> \funcname{main} (re-raise)
        \end{itemize}
      \end{itemize}
      \vfill
      \onslide*<1-5>{\mintocaml[firstline=15,lastline=21]{../src/backtrace/backtrace.ml}}
      \onslide*<6>{\mintocaml[firstline=28,lastline=34,highlightlines=31]{../src/backtrace/backtrace.ml}}
      \onslide*<7->{\mintocaml[firstline=28,lastline=34,highlightlines=33]{../src/backtrace/backtrace.ml}}
      \endminipage
    \end{column}
    \begin{column}{0.45\textwidth}
      \raggedleft
      \onslide*<1>{\providecommand\step{6}\input{diagrams/backtrace/backtrace.tex}}%
      \onslide*<2>{\providecommand\step{6}\input{diagrams/backtrace/backtrace.tex}}%
      \onslide*<3>{\providecommand\step{7}\input{diagrams/backtrace/backtrace.tex}}%
      \onslide*<4>{\providecommand\step{8}\input{diagrams/backtrace/backtrace.tex}}%
      \onslide*<5>{\providecommand\step{9}\input{diagrams/backtrace/backtrace.tex}}%
      \onslide*<6>{\providecommand\step{10}\input{diagrams/backtrace/backtrace.tex}}%
      \onslide*<7>{\providecommand\step{11}\input{diagrams/backtrace/backtrace.tex}}%
      \onslide*<8>{\providecommand\step{12}\input{diagrams/backtrace/backtrace.tex}}%
      \onslide*<9>{\providecommand\step{13}\input{diagrams/backtrace/backtrace.tex}}%
    \end{column}
  \end{columns}
\end{frame}

\begin{frame}{Backtraces}{Printing the backtrace}
  \begin{columns}[c]
    \begin{column}{0.45\textwidth}
      \minipage[c][0.66\textheight][s]{\columnwidth}
      \begin{itemize}
        \item<1-> The exception backtrace can now be printed to \funcarg{stdout} with \funcname{Printexc.print_backtrace}
        \item<2-> First the exception backtrace is retrieved into an OCaml array with \funcname{get_raw_backtrace }
        \item<3-> Which is implemented as the \funcname{caml_get_exception_raw_backtrace} C function in the runtime
        \item<4-> And finally printed with \funcname{print_exception_backtrace}
      \end{itemize}
      \vfill
      \mintocaml[firstline=28,lastline=34,highlightlines=33]{../src/backtrace/backtrace.ml}
      \endminipage
    \end{column}
    \begin{column}{0.55\textwidth}
      \onslide*<1,2>{
        \mintocaml[firstline=100,lastline=101]{../ocaml/stdlib/printexc.ml}
      }
      \only<1>{
        \listocaml[firstline=170,lastline=172]{../ocaml/stdlib/printexc.ml}{stdlib/printexc.ml:170}
      }
      \only<2>{
        \listocaml[firstline=170,lastline=172,highlightlines=172]{../ocaml/stdlib/printexc.ml}{stdlib/printexc.ml:170}
      }
      \only<3>{
        \mintc[firstline=154,lastline=156]{../ocaml/runtime/backtrace.c}
        \listc[firstline=171,lastline=192,highlightlines={184-187}]{../ocaml/runtime/backtrace.c}{runtime/backtrace.c:154}
      }
      \only<4>{
        \listocaml[firstline=155,lastline=165]{../ocaml/stdlib/printexc.ml}{stdlib/printexc.ml:55}
      }
    \end{column}
  \end{columns}
\end{frame}


\subsection{The multiple kinds of raise}
\frameSubsection{}{}

\begin{frame}{Backtraces}{When backtraces are not needed}
  \begin{columns}[c]
    \begin{column}{0.45\textwidth}
      \minipage[c][0.66\textheight][s]{\columnwidth}
      \begin{itemize}
        \item<1-> What if the backtrace for an exception is never needed?
        \item<2-> It's possible to raise the exception explicitly with \funcname{raise\_notrace} to make raising as cheap as possible
        \item<3-> An example of use in the \funcname{dynlink} library of the OCaml compiler
      \end{itemize}
      \vfill
      \onslide<3->{
      \listocaml[firstline=205,lastline=209,highlightlines=207]{../ocaml/otherlibs/dynlink/dynlink\_compilerlibs/misc.ml}{otherlibs/dynlink/dynlink\_compilerlibs/misc.ml:205}
      }
      \endminipage
    \end{column}
    \begin{column}{0.55\textwidth}
    \only<4>{
      \mintcmm[highlightlines=3]{../src/notrace/camlNotrace\_\_fun_347.cmm}
      \listcmm{../src/notrace/camlNotrace\_\_all\_somes\_327.cmm}{all\_somes}
    }
    \onslide*<5->{
      \listobjdump[highlightlines={6-9}]{../src/notrace/camlNotrace\_\_fun_347.objdump}{anonymous function from \funcname{all\_somes}}
    }
    \end{column}
  \end{columns}
\end{frame}

\begin{frame}{Backtraces}{Summary table of the multiple kinds of raise}
  \begin{itemize}
    \item When debugging information is enabled and if the backtrace of an exception isn't needed, it's possible to raise with \funcname{raise\_notrace} for performance
    \item When debugging information is \emph{not} enabled, the compiler optimises \funcname{raise} calls to inline assembly (same effect as using \funcname{raise\_notrace})
  \end{itemize}
  \bigskip
  \centering
  \begin{tabular}{| l | c | c | c | c |}
    \hline
    \parbox{10em}{Debug information \\ (\funcarg{-g} flag)}  & \multicolumn{2}{|c|}{Disabled}                                     & \multicolumn{2}{|c|}{Enabled} \\
    \hline
    Raise kind                                               & \funcname{raise} & \funcname{raise\_notrace}                       & \funcname{raise} & \funcname{raise\_notrace} \\
    \hline
    Compilation                                              & \parbox{8em}{inline assembly\\ (optimization)} & inline assembly   & \funcname{caml\_raise\_exn} & inline assembly \\
    \hline
  \end{tabular}
\end{frame}


%
% Takeaway #5
%
\subsection*{Takeaway \#5}
\frameSubsectionTakeaway{}
\againframe<5>{takeaway}
}%
      \onslide*<8>{\providecommand\step{12}%
% Backtraces
%
\subsection{One advantage of raising with debugging information}
\frameSubsection{}{}

\begin{frame}{Backtraces}{Debugging information \& source code}
  \begin{columns}[c]
    \begin{column}{0.5\textwidth}
      \begin{itemize}
        \item Raising an exception is fun and games, but how do we know where the exception originated? $\rightarrow$ With the backtrace associated with the exception!
        \item In order to get a backtrace from the point where an exception was raised, it's necessary to compile with debugging information enabled (\funcarg{-g} flag for the compiler)
        \item Same example as in the `Nested handlers' section
      \end{itemize}
    \end{column}
    \begin{column}{0.5\textwidth}
      \listocaml[firstline=9,lastline=34]{../src/backtrace/backtrace.ml}{backtrace/backtrace.ml}
    \end{column}
  \end{columns}
\end{frame}

\begin{frame}{Backtraces}{CMM and disassembly of \funcname{definitely\_raise}}
  \begin{columns}[c]
    \begin{column}{0.5\textwidth}
      \minipage[c][0.66\textheight][s]{\columnwidth}
      \begin{itemize}
        \item<1-> An effect of compiling with debugging information (\funcarg{-g}): the CMM no longer uses \funcname{raise\_notrace} to raise the exception but \funcname{raise} instead
        \item<2-> Disassembly shows that CMM \funcname{raise} is actually a call to \funcname{caml\_raise\_exn}, a function in assembler provided by the runtime
      \end{itemize}
      \vfill
      \mintocaml[firstline=9,lastline=13,highlightlines=12]{../src/backtrace/backtrace.ml}
      \endminipage
    \end{column}
    \begin{column}{0.5\textwidth}
      \onslide*<1>{
        \mintcmm[highlightlines=5]{../src/backtrace/camlBacktrace\_\_definitely\_raise\_347.cmm}
      }
      \onslide*<2>{
        \mintobjdump[firstline=2,highlightlines=14]{../src/backtrace/camlBacktrace\_\_definitely\_raise\_347.objdump}
      }
    \end{column}
  \end{columns}
\end{frame}

\begin{frame}{Backtraces}{Introducing \funcname{caml\_raise\_exn}}
  \begin{columns}[c]
    \begin{column}{0.5\textwidth}
      \minipage[c][0.66\textheight][s]{\columnwidth}
      \onslide*<1>{
        \begin{itemize}
          \item Raising the exception is performed using \funcname{caml\_raise\_exn} which is
            \begin{itemize}
              \item An assembly function provided by the runtime
              \item Architecture specific
            \end{itemize}
          \item Let's see what it does differently from \funcname{raise\_notrace}\dots
          \item We are about to raise \typename{ExnC} with \funcname{caml\_raise\_exn} from \funcname{definitely\_raise}
        \end{itemize}
      }
      \onslide*<2>{
        \mintobjdump[firstline=2,highlightlines=14]{../src/backtrace/camlBacktrace\_\_definitely\_raise\_347.objdump}
      }
      \vfill
      \mintocaml[firstline=9,lastline=13,highlightlines=12]{../src/backtrace/backtrace.ml}
      \endminipage
    \end{column}
    \begin{column}{0.5\textwidth}
      \centering
      \providecommand\step{1}\input{diagrams/backtrace/backtrace.tex}
    \end{column}
  \end{columns}
\end{frame}

\begin{frame}{Backtraces}{But before that, we need more fields from \typename{caml\_domain\_state}}
  \begin{columns}
    \begin{column}{0.5\textwidth}
      \begin{itemize}
        \item Remember \typename{caml\_domain\_state} the per-domain runtime state?
        \item Some other members are of interest to us now\dots
        \item \localname{backtrace\_active} is used to know if the runtime should collect backtraces
        \item \localname{backtrace\_active} can be set by
          \begin{itemize}
            \item The \funcarg{b} flag in the \funcname{OCAMLRUNPARAM} environment variable
            \item \mintinline{ocaml}{Printexc.record_backtrace : bool -> unit} \footnotemark
          \end{itemize}
      \end{itemize}
    \end{column}
    \begin{column}{0.5\textwidth}
      \begin{listing}
        \mintc[highlightlines={9-11}]{src/ocaml/domain\_state\_backtrace.h}
        \caption{runtime/caml/domain\_state.h}
      \end{listing}
    \end{column}
  \end{columns}
  \footnotetext[1]{https://v2.ocaml.org/api/Printexc.html}
\end{frame}

\newcommand\listCamlRaiseExn[1][]{
  \listasm[firstline=702,lastline=725,#1]{../ocaml/runtime/amd64.S}{runtime/amd64.S:702}}

\begin{frame}{Backtraces}{Walk through \funcname{caml\_raise\_exn}}
  \begin{columns}[T]
    \begin{column}{0.5\textwidth}
      \begin{itemize}
        \item<1-> First checks whether exceptions backtrace should be recorded
        \item<1-> By checking the \funcarg{domain\_state->backtrace\_active} value
        \item<2-> If not, acts as \funcname{raise\_notrace} with \funcname{RESTORE\_EXN\_HANDLER\_OCAML}
          \begin{itemize}
            \item Set \regname{RSP} to \localname{domain\_state->exn\_handler} value
            \item Restore \localname{domain\_state->exn\_handler} from trap
            \item Jump with \funcname{ret} (same effect as using \funcname{pop} and \funcname{jmp})
          \end{itemize}
        \item<3-> Otherwise, switches to the C stack to be able to execute C code
        \item<4-> Calls \funcname{caml\_stash\_backtrace}, a C function provided by the runtime\ignorespaces
          \onslide<6->{, with
            \begin{itemize}
              \item \funcarg{exn}: exception value
              \item \funcarg{pc}: code address of the raising point
              \item \funcarg{sp}: stack address of the raising function
              \item \funcarg{trapsp}: stack address of the next handler
            \end{itemize}
          }
      \end{itemize}
      \bigskip
      \mintocaml[firstline=9,lastline=13,highlightlines=12]{../src/backtrace/backtrace.ml}
    \end{column}
    \begin{column}{0.5\textwidth}
      \raggedleft
      \onslide*<1,3-4>{
        \mintc[firstline=190,lastline=192]{../ocaml/runtime/amd64.S}
        \mintc[firstline=234,lastline=237]{../ocaml/runtime/amd64.S}
      }
      \onslide*<2>{
        \mintc[firstline=190,lastline=192,highlightlines={191-192}]{../ocaml/runtime/amd64.S}
        \mintc[firstline=234,lastline=237,highlightlines={235-237}]{../ocaml/runtime/amd64.S}
      }
      \onslide*<1>{\listCamlRaiseExn[highlightlines=705]{}}%
      \onslide*<2>{\listCamlRaiseExn[highlightlines={707-708}]{}}%
      \onslide*<3>{\listCamlRaiseExn[highlightlines={712-714}]{}}%
      \onslide*<4>{\listCamlRaiseExn[highlightlines={715-720}]{}}%
      \onslide*<5>{\providecommand\step{1}\input{diagrams/backtrace/backtrace.tex}}%
      \onslide*<6>{\providecommand\step{2}\input{diagrams/backtrace/backtrace.tex}}%
    \end{column}
  \end{columns}
\end{frame}

\newcommand\listStashBacktrace[1][]{
  \listc[firstline=91,lastline=120,#1]{../ocaml/runtime/backtrace\_nat.c}{runtime/backtrace\_nat.c:91}}

\begin{frame}{Backtraces}{Introducing \funcname{caml\_stash\_backtrace}}
  \begin{columns}[c]
    \begin{column}{0.5\textwidth}
      \minipage[c][0.66\textheight][s]{\columnwidth}
      \begin{itemize}
        \item<1-> A C function provided by the runtime (specific to native code)
        \item<2-> It scans the stack, between the stack pointer at the raising point up to the next trap, in order to collect the names of the detected function frames
          \onslide*<2>{
            \begin{itemize}
              \item And more information, like line number, column number...
            \end{itemize}
          }
        \item<3-> It uses the same technique and information as the Garbage Collector (GC) uses to scan the values live on the stack
          \onslide*<4>{
            \begin{itemize}
              \item \typename{frame\_descr} are at the core of stack unwinding
            \end{itemize}
          }
      \end{itemize}
      \vfill
      \mintocaml[firstline=9,lastline=13,highlightlines=12]{../src/backtrace/backtrace.ml}
      \endminipage
    \end{column}
    \begin{column}{0.5\textwidth}
      \onslide*<1>{\listStashBacktrace[highlightlines=91]{}}
      \onslide*<2>{\listStashBacktrace[highlightlines={91,117-118}]{}}
      \onslide*<3->{\listStashBacktrace[highlightlines={94,106,109-110}]{}}
    \end{column}
  \end{columns}
\end{frame}

\newcommand\listFrameDescr[1][]{
  \listocaml[firstline=111,lastline=124,#1]{../ocaml/asmcomp/emitaux.ml}{asmcomp/emitaux.ml:111}}
\newcommand\listDebugInfo[1][]{
  \listocaml[firstline=38,lastline=49,#1]{../ocaml/lambda/debuginfo.mli}{lambda/debuginfo.mli:38}}

\begin{frame}{Backtraces}{\typename{frame\_descr} and \typename{frame\_debuginfo} in a nutshell}
  \begin{columns}[c]
    \begin{column}{0.5\textwidth}
      \begin{itemize}
        \item<1-> For every return address in an OCaml program, there is a associated \typename{frame\_descr}
        \item<2-> A \typename{frame\_descr} specifies
        \begin{itemize}
          \item<2-> The size of the function stack frame
          \item<3-> Where live values are on the stack (so that the GC can mark them as live and prevent from being swept)
        \end{itemize}
        \item<4-> When compiled with debug information, \typename{frame\_descr} will be linked to a \typename{frame\_debuginfo}
        \begin{itemize}
          \item<5-> Contains information about the position in the source code (file name, line and column number)
        \end{itemize}
      \end{itemize}
    \end{column}
    \begin{column}{0.5\textwidth}
        \onslide*<1>{\listFrameDescr[highlightlines={118-122}]{}}
        \onslide*<2>{\listFrameDescr[highlightlines={120}]{}}
        \onslide*<3>{\listFrameDescr[highlightlines={121}]{}}
        \onslide*<4->{\listFrameDescr[highlightlines={122}]{}}
        \medskip
        \onslide*<1-3>{\listDebugInfo{}}
        \onslide*<4>{\listDebugInfo[highlightlines={38,49}]{}}
        \onslide*<5>{\listDebugInfo[highlightlines={39-46}]{}}
    \end{column}
  \end{columns}
\end{frame}

\begin{frame}{Backtraces}{Scanning the stack with \typename{frame\_descr}}
  \begin{columns}[c]
    \begin{column}{0.5\textwidth}
      \begin{itemize}
        \item Given the return address of the code that raises the exception by calling \funcname{caml\_raise\_exn} (\funcarg{pc} argument of \funcname{caml\_stash\_backtrace})
        \item And the position of the stack pointer (\funcarg{sp} argument of \funcname{caml\_stash\_backtrace})
        \item The frame size given by \typename{frame\_descr}.\funcarg{frame\_size}, allows to find the end of the previous frame
        \item And the return address of the caller of this function
        \bigskip
        \onslide<3->{
        \item Note that traps (here the reddish \funcname{ExnA}, \funcname{ExnB} and \funcname{ExnC} blocks) belong to the frame that installed them, and so the frame size changes when traps are removed
        }
      \end{itemize}
    \end{column}
    \begin{column}{0.5\textwidth}
      \raggedleft
      \onslide*<1>{\providecommand\step{2}\input{diagrams/backtrace/backtrace.tex}}%
      \onslide*<2>{\providecommand\step{1}\input{diagrams/backtrace/scanstack.tex}}%
      \onslide*<3>{\providecommand\step{2}\input{diagrams/backtrace/scanstack.tex}}%
      \onslide*<4>{\providecommand\step{3}\input{diagrams/backtrace/scanstack.tex}}%
      \onslide*<5>{\providecommand\step{4}\input{diagrams/backtrace/scanstack.tex}}%
      \onslide*<6>{\providecommand\step{5}\input{diagrams/backtrace/scanstack.tex}}%
    \end{column}
  \end{columns}
\end{frame}

% Duplicate of previous-previous slide
\begin{frame}{Backtraces}{Continuing on \funcname{caml\_stash\_backtrace}}
  \begin{columns}[c]
    \begin{column}{0.5\textwidth}
      \minipage[c][0.75\textheight][s]{\columnwidth}
      \begin{itemize}
          \color{gray}
        \item A C function provided by the runtime (specific to native code)
        \item It scans the stack, between the stack pointer at the raising point up to the next trap, in order to collect the names of the detected function frames
        \item It uses the same technique and information as the Garbage Collector (GC) uses to scan the values live on the stack
      \end{itemize}
      \begin{itemize}
        \item For each function frame detected, store a pointer to the \typename{frame\_descr} in the \localname{domain\_state->backtrace\_buffer} array, effectively constructing the backtrace
      \end{itemize}
      \onslide<2->{
        \begin{itemize}
            \smallskip
          \item Constructed backtrace:
            \funcname{definitely\_raise},
            \onslide<3->{\funcname{probably\_raise}}
        \end{itemize}
      }
      \vfill
      \mintocaml[firstline=9,lastline=13,highlightlines=12]{../src/backtrace/backtrace.ml}
      \endminipage
    \end{column}
    \begin{column}{0.5\textwidth}
      \raggedleft
      \onslide*<1>{\listStashBacktrace[highlightlines={114-115}]{}}
      \onslide*<2>{\providecommand\step{3}\input{diagrams/backtrace/backtrace.tex}}%
      \onslide*<3>{\providecommand\step{4}\input{diagrams/backtrace/backtrace.tex}}%
    \end{column}
  \end{columns}
\end{frame}

\begin{frame}{Backtraces}{Back to \funcname{caml\_raise\_exn}}
  \begin{columns}[c]
    \begin{column}{0.5\textwidth}
      \minipage[c][0.66\textheight][s]{\columnwidth}
      \begin{itemize}\color{gray}
        \item Checks wheteher exceptions backtrace should be recorded, by checking the \funcarg{domain\_state->backtrace\_active} value
        \item Switches to the C stack to be able to execute C code
        \item Calls \funcname{caml\_stash\_backtrace}, to collect the backtrace up to the next trap
      \end{itemize}
      \onslide*<2->{
        \begin{itemize}
          \item<2-> Executes the exception handler in \funcname{probably\_raise} with \funcname{RESTORE\_EXN\_HANDLER\_OCAML}
            \begin{itemize}
              \item Set \regname{RSP} to \localname{domain\_state->exn\_handler} value
              \item Restore \localname{domain\_state->exn\_handler} from trap
              \item Jump to the handler code with \funcname{ret}
            \end{itemize}
        \end{itemize}
      }
      \vfill
      \onslide*<1>{\mintocaml[firstline=9,lastline=13,highlightlines=12]{../src/backtrace/backtrace.ml}}
      \onslide*<2->{\mintocaml[firstline=15,lastline=21,highlightlines={19-21}]{../src/backtrace/backtrace.ml}}
      \endminipage
    \end{column}
    \begin{column}{0.5\textwidth}
      \centering
      \onslide*<1>{
        \mintc[firstline=190,lastline=192]{../ocaml/runtime/amd64.S}
        \mintc[firstline=234,lastline=237]{../ocaml/runtime/amd64.S}
      }
      \onslide*<2>{
        \mintc[firstline=190,lastline=192,highlightlines={191-192}]{../ocaml/runtime/amd64.S}
        \mintc[firstline=234,lastline=237,highlightlines={235-237}]{../ocaml/runtime/amd64.S}
      }
      \onslide*<1>{\listCamlRaiseExn[highlightlines=720]{}}
      \onslide*<2>{\listCamlRaiseExn[highlightlines=722]{}}
      \onslide*<3->{\providecommand\step{5}\input{diagrams/backtrace/backtrace.tex}}
    \end{column}
  \end{columns}
\end{frame}

\begin{frame}{Backtraces}{\funcname{caml\_probably\_raise}'s handler doesn't catch \typename{ExnC}}
  \begin{columns}[c]
    \begin{column}{0.45\textwidth}
      \minipage[c][0.75\textheight][s]{\columnwidth}
      \begin{itemize}
        \item<1-> \funcname{match \dots with}\footnotemark pattern matching can also match exceptions
        \item<1-> The CMM of \funcname{probably\_raise} shows that this \funcname{match \dots with} is implemented with a pattern matching and a \funcname{try \dots with}
        \item<1-> But this one only matches on \typename{ExnA} exception
        \item<2-> Since the pattern matching fails to catch the exception, \funcname{reraise} is called with our current \typename{ExnC} exception
        \item<3-> Disassembly shows that \funcname{reraise} is actually a call to \funcname{caml\_reraise\_exn}, which is also an assembly function provided by the runtime
      \end{itemize}
      \vfill
      \mintocaml[firstline=15,lastline=21,highlightlines={18-21}]{../src/backtrace/backtrace.ml}
      \endminipage
    \end{column}
    \begin{column}{0.55\textwidth}
      \centering
      \onslide*<1>{\mintcmm[highlightlines={10-18}]{../src/backtrace/camlBacktrace\_\_probably\_raise\_351.cmm}}
      \onslide*<2>{\listcmm[highlightlines=18]{../src/backtrace/camlBacktrace\_\_probably\_raise_351.cmm}{CMM of probably\_raise}}
      \onslide*<3>{\mintobjdump[firstline=5,lastline=35,highlightlines=31]{../src/backtrace/camlBacktrace\_\_probably\_raise_351.objdump}}
    \end{column}
  \end{columns}
  \only<1>{\footnotetext[1]{https://blog.janestreet.com/pattern-matching-and-exception-handling-unite/}}
\end{frame}

\newcommand\listCamlReraiseExn[1][]{
  \listasm[firstline=727,lastline=734,#1]{../ocaml/runtime/amd64.S}{runtime/amd64.S:727}}

\begin{frame}{Backtraces}{Introducing \funcname{caml\_reraise\_exn}}
  \begin{columns}[c]
    \begin{column}{0.5\textwidth}
      \minipage[c][0.66\textheight][s]{\columnwidth}
      \begin{itemize}
        \item<1-> The exception have to be reraised to the next handler
        \item<2-> First, checks if the backtrace should be collected with \funcarg{domain\_state->backtrace\_active}
        \only<3>{
        \begin{itemize}
            \item If not, behaves like a \funcname{raise\_notrace} with \funcname{RESTORE\_EXN\_HANDLER\_OCAML}
        \end{itemize}
        }
        \item<4-> Jumps into \funcname{caml\_raise\_exn}
        \item<5-> Calls \funcname{caml\_stash\_backtrace} again, to scan the next part of the stack: from the trap just pop-ed to the next trap
      \end{itemize}
      \vfill
      \mintocaml[firstline=15,lastline=21,highlightlines={18-21}]{../src/backtrace/backtrace.ml}
      \endminipage
    \end{column}
    \begin{column}{0.5\textwidth}
      \raggedleft
      \onslide*<1>{\listCamlReraiseExn{}}
      \onslide*<2>{\listCamlReraiseExn[highlightlines=729]{}}
      \onslide*<3>{
        \mintc[firstline=190,lastline=192,highlightlines={191-192}]{../ocaml/runtime/amd64.S}
        \mintc[firstline=234,lastline=237,highlightlines={235-237}]{../ocaml/runtime/amd64.S}
        \medskip
        \listCamlReraiseExn[highlightlines=731]{}
      }
      \onslide*<4-5>{
        % TODO refactor with \listCamlReraiseExn
        \mintasm[firstline=727,lastline=734,highlightlines=730]{../ocaml/runtime/amd64.S}
        \medskip
      }
      \onslide*<4>{\listCamlRaiseExn[highlightlines=711]{}}
      \onslide*<5>{\listCamlRaiseExn[highlightlines=720]{}}
    \end{column}
  \end{columns}
\end{frame}

\begin{frame}{Backtraces}{Backtrace collection continues}
  \begin{columns}[c]
    \begin{column}{0.55\textwidth}
      \minipage[c][0.75\textheight][s]{\columnwidth}
      \begin{itemize}
        \item<1-> \funcname{caml\_stash\_backtrace} scans the older part of the stack, up to the next trap
        \item<6-> Controls jumps to the nested exception handler in \funcname{maybe\_raise}, which matches only on \typename{ExnB}
        \item<7-> \funcname{caml\_reraise\_exn} is called again, backtrace collection resumes with \funcname{caml\_stash\_backtrace}
      \end{itemize}
      \medskip
      \begin{itemize}
        \item<2-> Constructed backtrace:
        \begin{itemize}
          \item <2-> \funcname{definitely\_raise}
          \item <2-> \funcname{probably\_raise}
          \item <3-> \funcname{probably\_raise} (re-raise)
          \item <4-> \funcname{maybe\_raise}
          \item <5-> \funcname{main}
          \item <8-> \funcname{main} (re-raise)
        \end{itemize}
      \end{itemize}
      \vfill
      \onslide*<1-5>{\mintocaml[firstline=15,lastline=21]{../src/backtrace/backtrace.ml}}
      \onslide*<6>{\mintocaml[firstline=28,lastline=34,highlightlines=31]{../src/backtrace/backtrace.ml}}
      \onslide*<7->{\mintocaml[firstline=28,lastline=34,highlightlines=33]{../src/backtrace/backtrace.ml}}
      \endminipage
    \end{column}
    \begin{column}{0.45\textwidth}
      \raggedleft
      \onslide*<1>{\providecommand\step{6}\input{diagrams/backtrace/backtrace.tex}}%
      \onslide*<2>{\providecommand\step{6}\input{diagrams/backtrace/backtrace.tex}}%
      \onslide*<3>{\providecommand\step{7}\input{diagrams/backtrace/backtrace.tex}}%
      \onslide*<4>{\providecommand\step{8}\input{diagrams/backtrace/backtrace.tex}}%
      \onslide*<5>{\providecommand\step{9}\input{diagrams/backtrace/backtrace.tex}}%
      \onslide*<6>{\providecommand\step{10}\input{diagrams/backtrace/backtrace.tex}}%
      \onslide*<7>{\providecommand\step{11}\input{diagrams/backtrace/backtrace.tex}}%
      \onslide*<8>{\providecommand\step{12}\input{diagrams/backtrace/backtrace.tex}}%
      \onslide*<9>{\providecommand\step{13}\input{diagrams/backtrace/backtrace.tex}}%
    \end{column}
  \end{columns}
\end{frame}

\begin{frame}{Backtraces}{Printing the backtrace}
  \begin{columns}[c]
    \begin{column}{0.45\textwidth}
      \minipage[c][0.66\textheight][s]{\columnwidth}
      \begin{itemize}
        \item<1-> The exception backtrace can now be printed to \funcarg{stdout} with \funcname{Printexc.print_backtrace}
        \item<2-> First the exception backtrace is retrieved into an OCaml array with \funcname{get_raw_backtrace }
        \item<3-> Which is implemented as the \funcname{caml_get_exception_raw_backtrace} C function in the runtime
        \item<4-> And finally printed with \funcname{print_exception_backtrace}
      \end{itemize}
      \vfill
      \mintocaml[firstline=28,lastline=34,highlightlines=33]{../src/backtrace/backtrace.ml}
      \endminipage
    \end{column}
    \begin{column}{0.55\textwidth}
      \onslide*<1,2>{
        \mintocaml[firstline=100,lastline=101]{../ocaml/stdlib/printexc.ml}
      }
      \only<1>{
        \listocaml[firstline=170,lastline=172]{../ocaml/stdlib/printexc.ml}{stdlib/printexc.ml:170}
      }
      \only<2>{
        \listocaml[firstline=170,lastline=172,highlightlines=172]{../ocaml/stdlib/printexc.ml}{stdlib/printexc.ml:170}
      }
      \only<3>{
        \mintc[firstline=154,lastline=156]{../ocaml/runtime/backtrace.c}
        \listc[firstline=171,lastline=192,highlightlines={184-187}]{../ocaml/runtime/backtrace.c}{runtime/backtrace.c:154}
      }
      \only<4>{
        \listocaml[firstline=155,lastline=165]{../ocaml/stdlib/printexc.ml}{stdlib/printexc.ml:55}
      }
    \end{column}
  \end{columns}
\end{frame}


\subsection{The multiple kinds of raise}
\frameSubsection{}{}

\begin{frame}{Backtraces}{When backtraces are not needed}
  \begin{columns}[c]
    \begin{column}{0.45\textwidth}
      \minipage[c][0.66\textheight][s]{\columnwidth}
      \begin{itemize}
        \item<1-> What if the backtrace for an exception is never needed?
        \item<2-> It's possible to raise the exception explicitly with \funcname{raise\_notrace} to make raising as cheap as possible
        \item<3-> An example of use in the \funcname{dynlink} library of the OCaml compiler
      \end{itemize}
      \vfill
      \onslide<3->{
      \listocaml[firstline=205,lastline=209,highlightlines=207]{../ocaml/otherlibs/dynlink/dynlink\_compilerlibs/misc.ml}{otherlibs/dynlink/dynlink\_compilerlibs/misc.ml:205}
      }
      \endminipage
    \end{column}
    \begin{column}{0.55\textwidth}
    \only<4>{
      \mintcmm[highlightlines=3]{../src/notrace/camlNotrace\_\_fun_347.cmm}
      \listcmm{../src/notrace/camlNotrace\_\_all\_somes\_327.cmm}{all\_somes}
    }
    \onslide*<5->{
      \listobjdump[highlightlines={6-9}]{../src/notrace/camlNotrace\_\_fun_347.objdump}{anonymous function from \funcname{all\_somes}}
    }
    \end{column}
  \end{columns}
\end{frame}

\begin{frame}{Backtraces}{Summary table of the multiple kinds of raise}
  \begin{itemize}
    \item When debugging information is enabled and if the backtrace of an exception isn't needed, it's possible to raise with \funcname{raise\_notrace} for performance
    \item When debugging information is \emph{not} enabled, the compiler optimises \funcname{raise} calls to inline assembly (same effect as using \funcname{raise\_notrace})
  \end{itemize}
  \bigskip
  \centering
  \begin{tabular}{| l | c | c | c | c |}
    \hline
    \parbox{10em}{Debug information \\ (\funcarg{-g} flag)}  & \multicolumn{2}{|c|}{Disabled}                                     & \multicolumn{2}{|c|}{Enabled} \\
    \hline
    Raise kind                                               & \funcname{raise} & \funcname{raise\_notrace}                       & \funcname{raise} & \funcname{raise\_notrace} \\
    \hline
    Compilation                                              & \parbox{8em}{inline assembly\\ (optimization)} & inline assembly   & \funcname{caml\_raise\_exn} & inline assembly \\
    \hline
  \end{tabular}
\end{frame}


%
% Takeaway #5
%
\subsection*{Takeaway \#5}
\frameSubsectionTakeaway{}
\againframe<5>{takeaway}
}%
      \onslide*<9>{\providecommand\step{13}%
% Backtraces
%
\subsection{One advantage of raising with debugging information}
\frameSubsection{}{}

\begin{frame}{Backtraces}{Debugging information \& source code}
  \begin{columns}[c]
    \begin{column}{0.5\textwidth}
      \begin{itemize}
        \item Raising an exception is fun and games, but how do we know where the exception originated? $\rightarrow$ With the backtrace associated with the exception!
        \item In order to get a backtrace from the point where an exception was raised, it's necessary to compile with debugging information enabled (\funcarg{-g} flag for the compiler)
        \item Same example as in the `Nested handlers' section
      \end{itemize}
    \end{column}
    \begin{column}{0.5\textwidth}
      \listocaml[firstline=9,lastline=34]{../src/backtrace/backtrace.ml}{backtrace/backtrace.ml}
    \end{column}
  \end{columns}
\end{frame}

\begin{frame}{Backtraces}{CMM and disassembly of \funcname{definitely\_raise}}
  \begin{columns}[c]
    \begin{column}{0.5\textwidth}
      \minipage[c][0.66\textheight][s]{\columnwidth}
      \begin{itemize}
        \item<1-> An effect of compiling with debugging information (\funcarg{-g}): the CMM no longer uses \funcname{raise\_notrace} to raise the exception but \funcname{raise} instead
        \item<2-> Disassembly shows that CMM \funcname{raise} is actually a call to \funcname{caml\_raise\_exn}, a function in assembler provided by the runtime
      \end{itemize}
      \vfill
      \mintocaml[firstline=9,lastline=13,highlightlines=12]{../src/backtrace/backtrace.ml}
      \endminipage
    \end{column}
    \begin{column}{0.5\textwidth}
      \onslide*<1>{
        \mintcmm[highlightlines=5]{../src/backtrace/camlBacktrace\_\_definitely\_raise\_347.cmm}
      }
      \onslide*<2>{
        \mintobjdump[firstline=2,highlightlines=14]{../src/backtrace/camlBacktrace\_\_definitely\_raise\_347.objdump}
      }
    \end{column}
  \end{columns}
\end{frame}

\begin{frame}{Backtraces}{Introducing \funcname{caml\_raise\_exn}}
  \begin{columns}[c]
    \begin{column}{0.5\textwidth}
      \minipage[c][0.66\textheight][s]{\columnwidth}
      \onslide*<1>{
        \begin{itemize}
          \item Raising the exception is performed using \funcname{caml\_raise\_exn} which is
            \begin{itemize}
              \item An assembly function provided by the runtime
              \item Architecture specific
            \end{itemize}
          \item Let's see what it does differently from \funcname{raise\_notrace}\dots
          \item We are about to raise \typename{ExnC} with \funcname{caml\_raise\_exn} from \funcname{definitely\_raise}
        \end{itemize}
      }
      \onslide*<2>{
        \mintobjdump[firstline=2,highlightlines=14]{../src/backtrace/camlBacktrace\_\_definitely\_raise\_347.objdump}
      }
      \vfill
      \mintocaml[firstline=9,lastline=13,highlightlines=12]{../src/backtrace/backtrace.ml}
      \endminipage
    \end{column}
    \begin{column}{0.5\textwidth}
      \centering
      \providecommand\step{1}\input{diagrams/backtrace/backtrace.tex}
    \end{column}
  \end{columns}
\end{frame}

\begin{frame}{Backtraces}{But before that, we need more fields from \typename{caml\_domain\_state}}
  \begin{columns}
    \begin{column}{0.5\textwidth}
      \begin{itemize}
        \item Remember \typename{caml\_domain\_state} the per-domain runtime state?
        \item Some other members are of interest to us now\dots
        \item \localname{backtrace\_active} is used to know if the runtime should collect backtraces
        \item \localname{backtrace\_active} can be set by
          \begin{itemize}
            \item The \funcarg{b} flag in the \funcname{OCAMLRUNPARAM} environment variable
            \item \mintinline{ocaml}{Printexc.record_backtrace : bool -> unit} \footnotemark
          \end{itemize}
      \end{itemize}
    \end{column}
    \begin{column}{0.5\textwidth}
      \begin{listing}
        \mintc[highlightlines={9-11}]{src/ocaml/domain\_state\_backtrace.h}
        \caption{runtime/caml/domain\_state.h}
      \end{listing}
    \end{column}
  \end{columns}
  \footnotetext[1]{https://v2.ocaml.org/api/Printexc.html}
\end{frame}

\newcommand\listCamlRaiseExn[1][]{
  \listasm[firstline=702,lastline=725,#1]{../ocaml/runtime/amd64.S}{runtime/amd64.S:702}}

\begin{frame}{Backtraces}{Walk through \funcname{caml\_raise\_exn}}
  \begin{columns}[T]
    \begin{column}{0.5\textwidth}
      \begin{itemize}
        \item<1-> First checks whether exceptions backtrace should be recorded
        \item<1-> By checking the \funcarg{domain\_state->backtrace\_active} value
        \item<2-> If not, acts as \funcname{raise\_notrace} with \funcname{RESTORE\_EXN\_HANDLER\_OCAML}
          \begin{itemize}
            \item Set \regname{RSP} to \localname{domain\_state->exn\_handler} value
            \item Restore \localname{domain\_state->exn\_handler} from trap
            \item Jump with \funcname{ret} (same effect as using \funcname{pop} and \funcname{jmp})
          \end{itemize}
        \item<3-> Otherwise, switches to the C stack to be able to execute C code
        \item<4-> Calls \funcname{caml\_stash\_backtrace}, a C function provided by the runtime\ignorespaces
          \onslide<6->{, with
            \begin{itemize}
              \item \funcarg{exn}: exception value
              \item \funcarg{pc}: code address of the raising point
              \item \funcarg{sp}: stack address of the raising function
              \item \funcarg{trapsp}: stack address of the next handler
            \end{itemize}
          }
      \end{itemize}
      \bigskip
      \mintocaml[firstline=9,lastline=13,highlightlines=12]{../src/backtrace/backtrace.ml}
    \end{column}
    \begin{column}{0.5\textwidth}
      \raggedleft
      \onslide*<1,3-4>{
        \mintc[firstline=190,lastline=192]{../ocaml/runtime/amd64.S}
        \mintc[firstline=234,lastline=237]{../ocaml/runtime/amd64.S}
      }
      \onslide*<2>{
        \mintc[firstline=190,lastline=192,highlightlines={191-192}]{../ocaml/runtime/amd64.S}
        \mintc[firstline=234,lastline=237,highlightlines={235-237}]{../ocaml/runtime/amd64.S}
      }
      \onslide*<1>{\listCamlRaiseExn[highlightlines=705]{}}%
      \onslide*<2>{\listCamlRaiseExn[highlightlines={707-708}]{}}%
      \onslide*<3>{\listCamlRaiseExn[highlightlines={712-714}]{}}%
      \onslide*<4>{\listCamlRaiseExn[highlightlines={715-720}]{}}%
      \onslide*<5>{\providecommand\step{1}\input{diagrams/backtrace/backtrace.tex}}%
      \onslide*<6>{\providecommand\step{2}\input{diagrams/backtrace/backtrace.tex}}%
    \end{column}
  \end{columns}
\end{frame}

\newcommand\listStashBacktrace[1][]{
  \listc[firstline=91,lastline=120,#1]{../ocaml/runtime/backtrace\_nat.c}{runtime/backtrace\_nat.c:91}}

\begin{frame}{Backtraces}{Introducing \funcname{caml\_stash\_backtrace}}
  \begin{columns}[c]
    \begin{column}{0.5\textwidth}
      \minipage[c][0.66\textheight][s]{\columnwidth}
      \begin{itemize}
        \item<1-> A C function provided by the runtime (specific to native code)
        \item<2-> It scans the stack, between the stack pointer at the raising point up to the next trap, in order to collect the names of the detected function frames
          \onslide*<2>{
            \begin{itemize}
              \item And more information, like line number, column number...
            \end{itemize}
          }
        \item<3-> It uses the same technique and information as the Garbage Collector (GC) uses to scan the values live on the stack
          \onslide*<4>{
            \begin{itemize}
              \item \typename{frame\_descr} are at the core of stack unwinding
            \end{itemize}
          }
      \end{itemize}
      \vfill
      \mintocaml[firstline=9,lastline=13,highlightlines=12]{../src/backtrace/backtrace.ml}
      \endminipage
    \end{column}
    \begin{column}{0.5\textwidth}
      \onslide*<1>{\listStashBacktrace[highlightlines=91]{}}
      \onslide*<2>{\listStashBacktrace[highlightlines={91,117-118}]{}}
      \onslide*<3->{\listStashBacktrace[highlightlines={94,106,109-110}]{}}
    \end{column}
  \end{columns}
\end{frame}

\newcommand\listFrameDescr[1][]{
  \listocaml[firstline=111,lastline=124,#1]{../ocaml/asmcomp/emitaux.ml}{asmcomp/emitaux.ml:111}}
\newcommand\listDebugInfo[1][]{
  \listocaml[firstline=38,lastline=49,#1]{../ocaml/lambda/debuginfo.mli}{lambda/debuginfo.mli:38}}

\begin{frame}{Backtraces}{\typename{frame\_descr} and \typename{frame\_debuginfo} in a nutshell}
  \begin{columns}[c]
    \begin{column}{0.5\textwidth}
      \begin{itemize}
        \item<1-> For every return address in an OCaml program, there is a associated \typename{frame\_descr}
        \item<2-> A \typename{frame\_descr} specifies
        \begin{itemize}
          \item<2-> The size of the function stack frame
          \item<3-> Where live values are on the stack (so that the GC can mark them as live and prevent from being swept)
        \end{itemize}
        \item<4-> When compiled with debug information, \typename{frame\_descr} will be linked to a \typename{frame\_debuginfo}
        \begin{itemize}
          \item<5-> Contains information about the position in the source code (file name, line and column number)
        \end{itemize}
      \end{itemize}
    \end{column}
    \begin{column}{0.5\textwidth}
        \onslide*<1>{\listFrameDescr[highlightlines={118-122}]{}}
        \onslide*<2>{\listFrameDescr[highlightlines={120}]{}}
        \onslide*<3>{\listFrameDescr[highlightlines={121}]{}}
        \onslide*<4->{\listFrameDescr[highlightlines={122}]{}}
        \medskip
        \onslide*<1-3>{\listDebugInfo{}}
        \onslide*<4>{\listDebugInfo[highlightlines={38,49}]{}}
        \onslide*<5>{\listDebugInfo[highlightlines={39-46}]{}}
    \end{column}
  \end{columns}
\end{frame}

\begin{frame}{Backtraces}{Scanning the stack with \typename{frame\_descr}}
  \begin{columns}[c]
    \begin{column}{0.5\textwidth}
      \begin{itemize}
        \item Given the return address of the code that raises the exception by calling \funcname{caml\_raise\_exn} (\funcarg{pc} argument of \funcname{caml\_stash\_backtrace})
        \item And the position of the stack pointer (\funcarg{sp} argument of \funcname{caml\_stash\_backtrace})
        \item The frame size given by \typename{frame\_descr}.\funcarg{frame\_size}, allows to find the end of the previous frame
        \item And the return address of the caller of this function
        \bigskip
        \onslide<3->{
        \item Note that traps (here the reddish \funcname{ExnA}, \funcname{ExnB} and \funcname{ExnC} blocks) belong to the frame that installed them, and so the frame size changes when traps are removed
        }
      \end{itemize}
    \end{column}
    \begin{column}{0.5\textwidth}
      \raggedleft
      \onslide*<1>{\providecommand\step{2}\input{diagrams/backtrace/backtrace.tex}}%
      \onslide*<2>{\providecommand\step{1}\input{diagrams/backtrace/scanstack.tex}}%
      \onslide*<3>{\providecommand\step{2}\input{diagrams/backtrace/scanstack.tex}}%
      \onslide*<4>{\providecommand\step{3}\input{diagrams/backtrace/scanstack.tex}}%
      \onslide*<5>{\providecommand\step{4}\input{diagrams/backtrace/scanstack.tex}}%
      \onslide*<6>{\providecommand\step{5}\input{diagrams/backtrace/scanstack.tex}}%
    \end{column}
  \end{columns}
\end{frame}

% Duplicate of previous-previous slide
\begin{frame}{Backtraces}{Continuing on \funcname{caml\_stash\_backtrace}}
  \begin{columns}[c]
    \begin{column}{0.5\textwidth}
      \minipage[c][0.75\textheight][s]{\columnwidth}
      \begin{itemize}
          \color{gray}
        \item A C function provided by the runtime (specific to native code)
        \item It scans the stack, between the stack pointer at the raising point up to the next trap, in order to collect the names of the detected function frames
        \item It uses the same technique and information as the Garbage Collector (GC) uses to scan the values live on the stack
      \end{itemize}
      \begin{itemize}
        \item For each function frame detected, store a pointer to the \typename{frame\_descr} in the \localname{domain\_state->backtrace\_buffer} array, effectively constructing the backtrace
      \end{itemize}
      \onslide<2->{
        \begin{itemize}
            \smallskip
          \item Constructed backtrace:
            \funcname{definitely\_raise},
            \onslide<3->{\funcname{probably\_raise}}
        \end{itemize}
      }
      \vfill
      \mintocaml[firstline=9,lastline=13,highlightlines=12]{../src/backtrace/backtrace.ml}
      \endminipage
    \end{column}
    \begin{column}{0.5\textwidth}
      \raggedleft
      \onslide*<1>{\listStashBacktrace[highlightlines={114-115}]{}}
      \onslide*<2>{\providecommand\step{3}\input{diagrams/backtrace/backtrace.tex}}%
      \onslide*<3>{\providecommand\step{4}\input{diagrams/backtrace/backtrace.tex}}%
    \end{column}
  \end{columns}
\end{frame}

\begin{frame}{Backtraces}{Back to \funcname{caml\_raise\_exn}}
  \begin{columns}[c]
    \begin{column}{0.5\textwidth}
      \minipage[c][0.66\textheight][s]{\columnwidth}
      \begin{itemize}\color{gray}
        \item Checks wheteher exceptions backtrace should be recorded, by checking the \funcarg{domain\_state->backtrace\_active} value
        \item Switches to the C stack to be able to execute C code
        \item Calls \funcname{caml\_stash\_backtrace}, to collect the backtrace up to the next trap
      \end{itemize}
      \onslide*<2->{
        \begin{itemize}
          \item<2-> Executes the exception handler in \funcname{probably\_raise} with \funcname{RESTORE\_EXN\_HANDLER\_OCAML}
            \begin{itemize}
              \item Set \regname{RSP} to \localname{domain\_state->exn\_handler} value
              \item Restore \localname{domain\_state->exn\_handler} from trap
              \item Jump to the handler code with \funcname{ret}
            \end{itemize}
        \end{itemize}
      }
      \vfill
      \onslide*<1>{\mintocaml[firstline=9,lastline=13,highlightlines=12]{../src/backtrace/backtrace.ml}}
      \onslide*<2->{\mintocaml[firstline=15,lastline=21,highlightlines={19-21}]{../src/backtrace/backtrace.ml}}
      \endminipage
    \end{column}
    \begin{column}{0.5\textwidth}
      \centering
      \onslide*<1>{
        \mintc[firstline=190,lastline=192]{../ocaml/runtime/amd64.S}
        \mintc[firstline=234,lastline=237]{../ocaml/runtime/amd64.S}
      }
      \onslide*<2>{
        \mintc[firstline=190,lastline=192,highlightlines={191-192}]{../ocaml/runtime/amd64.S}
        \mintc[firstline=234,lastline=237,highlightlines={235-237}]{../ocaml/runtime/amd64.S}
      }
      \onslide*<1>{\listCamlRaiseExn[highlightlines=720]{}}
      \onslide*<2>{\listCamlRaiseExn[highlightlines=722]{}}
      \onslide*<3->{\providecommand\step{5}\input{diagrams/backtrace/backtrace.tex}}
    \end{column}
  \end{columns}
\end{frame}

\begin{frame}{Backtraces}{\funcname{caml\_probably\_raise}'s handler doesn't catch \typename{ExnC}}
  \begin{columns}[c]
    \begin{column}{0.45\textwidth}
      \minipage[c][0.75\textheight][s]{\columnwidth}
      \begin{itemize}
        \item<1-> \funcname{match \dots with}\footnotemark pattern matching can also match exceptions
        \item<1-> The CMM of \funcname{probably\_raise} shows that this \funcname{match \dots with} is implemented with a pattern matching and a \funcname{try \dots with}
        \item<1-> But this one only matches on \typename{ExnA} exception
        \item<2-> Since the pattern matching fails to catch the exception, \funcname{reraise} is called with our current \typename{ExnC} exception
        \item<3-> Disassembly shows that \funcname{reraise} is actually a call to \funcname{caml\_reraise\_exn}, which is also an assembly function provided by the runtime
      \end{itemize}
      \vfill
      \mintocaml[firstline=15,lastline=21,highlightlines={18-21}]{../src/backtrace/backtrace.ml}
      \endminipage
    \end{column}
    \begin{column}{0.55\textwidth}
      \centering
      \onslide*<1>{\mintcmm[highlightlines={10-18}]{../src/backtrace/camlBacktrace\_\_probably\_raise\_351.cmm}}
      \onslide*<2>{\listcmm[highlightlines=18]{../src/backtrace/camlBacktrace\_\_probably\_raise_351.cmm}{CMM of probably\_raise}}
      \onslide*<3>{\mintobjdump[firstline=5,lastline=35,highlightlines=31]{../src/backtrace/camlBacktrace\_\_probably\_raise_351.objdump}}
    \end{column}
  \end{columns}
  \only<1>{\footnotetext[1]{https://blog.janestreet.com/pattern-matching-and-exception-handling-unite/}}
\end{frame}

\newcommand\listCamlReraiseExn[1][]{
  \listasm[firstline=727,lastline=734,#1]{../ocaml/runtime/amd64.S}{runtime/amd64.S:727}}

\begin{frame}{Backtraces}{Introducing \funcname{caml\_reraise\_exn}}
  \begin{columns}[c]
    \begin{column}{0.5\textwidth}
      \minipage[c][0.66\textheight][s]{\columnwidth}
      \begin{itemize}
        \item<1-> The exception have to be reraised to the next handler
        \item<2-> First, checks if the backtrace should be collected with \funcarg{domain\_state->backtrace\_active}
        \only<3>{
        \begin{itemize}
            \item If not, behaves like a \funcname{raise\_notrace} with \funcname{RESTORE\_EXN\_HANDLER\_OCAML}
        \end{itemize}
        }
        \item<4-> Jumps into \funcname{caml\_raise\_exn}
        \item<5-> Calls \funcname{caml\_stash\_backtrace} again, to scan the next part of the stack: from the trap just pop-ed to the next trap
      \end{itemize}
      \vfill
      \mintocaml[firstline=15,lastline=21,highlightlines={18-21}]{../src/backtrace/backtrace.ml}
      \endminipage
    \end{column}
    \begin{column}{0.5\textwidth}
      \raggedleft
      \onslide*<1>{\listCamlReraiseExn{}}
      \onslide*<2>{\listCamlReraiseExn[highlightlines=729]{}}
      \onslide*<3>{
        \mintc[firstline=190,lastline=192,highlightlines={191-192}]{../ocaml/runtime/amd64.S}
        \mintc[firstline=234,lastline=237,highlightlines={235-237}]{../ocaml/runtime/amd64.S}
        \medskip
        \listCamlReraiseExn[highlightlines=731]{}
      }
      \onslide*<4-5>{
        % TODO refactor with \listCamlReraiseExn
        \mintasm[firstline=727,lastline=734,highlightlines=730]{../ocaml/runtime/amd64.S}
        \medskip
      }
      \onslide*<4>{\listCamlRaiseExn[highlightlines=711]{}}
      \onslide*<5>{\listCamlRaiseExn[highlightlines=720]{}}
    \end{column}
  \end{columns}
\end{frame}

\begin{frame}{Backtraces}{Backtrace collection continues}
  \begin{columns}[c]
    \begin{column}{0.55\textwidth}
      \minipage[c][0.75\textheight][s]{\columnwidth}
      \begin{itemize}
        \item<1-> \funcname{caml\_stash\_backtrace} scans the older part of the stack, up to the next trap
        \item<6-> Controls jumps to the nested exception handler in \funcname{maybe\_raise}, which matches only on \typename{ExnB}
        \item<7-> \funcname{caml\_reraise\_exn} is called again, backtrace collection resumes with \funcname{caml\_stash\_backtrace}
      \end{itemize}
      \medskip
      \begin{itemize}
        \item<2-> Constructed backtrace:
        \begin{itemize}
          \item <2-> \funcname{definitely\_raise}
          \item <2-> \funcname{probably\_raise}
          \item <3-> \funcname{probably\_raise} (re-raise)
          \item <4-> \funcname{maybe\_raise}
          \item <5-> \funcname{main}
          \item <8-> \funcname{main} (re-raise)
        \end{itemize}
      \end{itemize}
      \vfill
      \onslide*<1-5>{\mintocaml[firstline=15,lastline=21]{../src/backtrace/backtrace.ml}}
      \onslide*<6>{\mintocaml[firstline=28,lastline=34,highlightlines=31]{../src/backtrace/backtrace.ml}}
      \onslide*<7->{\mintocaml[firstline=28,lastline=34,highlightlines=33]{../src/backtrace/backtrace.ml}}
      \endminipage
    \end{column}
    \begin{column}{0.45\textwidth}
      \raggedleft
      \onslide*<1>{\providecommand\step{6}\input{diagrams/backtrace/backtrace.tex}}%
      \onslide*<2>{\providecommand\step{6}\input{diagrams/backtrace/backtrace.tex}}%
      \onslide*<3>{\providecommand\step{7}\input{diagrams/backtrace/backtrace.tex}}%
      \onslide*<4>{\providecommand\step{8}\input{diagrams/backtrace/backtrace.tex}}%
      \onslide*<5>{\providecommand\step{9}\input{diagrams/backtrace/backtrace.tex}}%
      \onslide*<6>{\providecommand\step{10}\input{diagrams/backtrace/backtrace.tex}}%
      \onslide*<7>{\providecommand\step{11}\input{diagrams/backtrace/backtrace.tex}}%
      \onslide*<8>{\providecommand\step{12}\input{diagrams/backtrace/backtrace.tex}}%
      \onslide*<9>{\providecommand\step{13}\input{diagrams/backtrace/backtrace.tex}}%
    \end{column}
  \end{columns}
\end{frame}

\begin{frame}{Backtraces}{Printing the backtrace}
  \begin{columns}[c]
    \begin{column}{0.45\textwidth}
      \minipage[c][0.66\textheight][s]{\columnwidth}
      \begin{itemize}
        \item<1-> The exception backtrace can now be printed to \funcarg{stdout} with \funcname{Printexc.print_backtrace}
        \item<2-> First the exception backtrace is retrieved into an OCaml array with \funcname{get_raw_backtrace }
        \item<3-> Which is implemented as the \funcname{caml_get_exception_raw_backtrace} C function in the runtime
        \item<4-> And finally printed with \funcname{print_exception_backtrace}
      \end{itemize}
      \vfill
      \mintocaml[firstline=28,lastline=34,highlightlines=33]{../src/backtrace/backtrace.ml}
      \endminipage
    \end{column}
    \begin{column}{0.55\textwidth}
      \onslide*<1,2>{
        \mintocaml[firstline=100,lastline=101]{../ocaml/stdlib/printexc.ml}
      }
      \only<1>{
        \listocaml[firstline=170,lastline=172]{../ocaml/stdlib/printexc.ml}{stdlib/printexc.ml:170}
      }
      \only<2>{
        \listocaml[firstline=170,lastline=172,highlightlines=172]{../ocaml/stdlib/printexc.ml}{stdlib/printexc.ml:170}
      }
      \only<3>{
        \mintc[firstline=154,lastline=156]{../ocaml/runtime/backtrace.c}
        \listc[firstline=171,lastline=192,highlightlines={184-187}]{../ocaml/runtime/backtrace.c}{runtime/backtrace.c:154}
      }
      \only<4>{
        \listocaml[firstline=155,lastline=165]{../ocaml/stdlib/printexc.ml}{stdlib/printexc.ml:55}
      }
    \end{column}
  \end{columns}
\end{frame}


\subsection{The multiple kinds of raise}
\frameSubsection{}{}

\begin{frame}{Backtraces}{When backtraces are not needed}
  \begin{columns}[c]
    \begin{column}{0.45\textwidth}
      \minipage[c][0.66\textheight][s]{\columnwidth}
      \begin{itemize}
        \item<1-> What if the backtrace for an exception is never needed?
        \item<2-> It's possible to raise the exception explicitly with \funcname{raise\_notrace} to make raising as cheap as possible
        \item<3-> An example of use in the \funcname{dynlink} library of the OCaml compiler
      \end{itemize}
      \vfill
      \onslide<3->{
      \listocaml[firstline=205,lastline=209,highlightlines=207]{../ocaml/otherlibs/dynlink/dynlink\_compilerlibs/misc.ml}{otherlibs/dynlink/dynlink\_compilerlibs/misc.ml:205}
      }
      \endminipage
    \end{column}
    \begin{column}{0.55\textwidth}
    \only<4>{
      \mintcmm[highlightlines=3]{../src/notrace/camlNotrace\_\_fun_347.cmm}
      \listcmm{../src/notrace/camlNotrace\_\_all\_somes\_327.cmm}{all\_somes}
    }
    \onslide*<5->{
      \listobjdump[highlightlines={6-9}]{../src/notrace/camlNotrace\_\_fun_347.objdump}{anonymous function from \funcname{all\_somes}}
    }
    \end{column}
  \end{columns}
\end{frame}

\begin{frame}{Backtraces}{Summary table of the multiple kinds of raise}
  \begin{itemize}
    \item When debugging information is enabled and if the backtrace of an exception isn't needed, it's possible to raise with \funcname{raise\_notrace} for performance
    \item When debugging information is \emph{not} enabled, the compiler optimises \funcname{raise} calls to inline assembly (same effect as using \funcname{raise\_notrace})
  \end{itemize}
  \bigskip
  \centering
  \begin{tabular}{| l | c | c | c | c |}
    \hline
    \parbox{10em}{Debug information \\ (\funcarg{-g} flag)}  & \multicolumn{2}{|c|}{Disabled}                                     & \multicolumn{2}{|c|}{Enabled} \\
    \hline
    Raise kind                                               & \funcname{raise} & \funcname{raise\_notrace}                       & \funcname{raise} & \funcname{raise\_notrace} \\
    \hline
    Compilation                                              & \parbox{8em}{inline assembly\\ (optimization)} & inline assembly   & \funcname{caml\_raise\_exn} & inline assembly \\
    \hline
  \end{tabular}
\end{frame}


%
% Takeaway #5
%
\subsection*{Takeaway \#5}
\frameSubsectionTakeaway{}
\againframe<5>{takeaway}
}%
    \end{column}
  \end{columns}
\end{frame}

\begin{frame}{Backtraces}{Printing the backtrace}
  \begin{columns}[c]
    \begin{column}{0.45\textwidth}
      \minipage[c][0.66\textheight][s]{\columnwidth}
      \begin{itemize}
        \item<1-> The exception backtrace can now be printed to \funcarg{stdout} with \funcname{Printexc.print_backtrace}
        \item<2-> First the exception backtrace is retrieved into an OCaml array with \funcname{get_raw_backtrace }
        \item<3-> Which is implemented as the \funcname{caml_get_exception_raw_backtrace} C function in the runtime
        \item<4-> And finally printed with \funcname{print_exception_backtrace}
      \end{itemize}
      \vfill
      \mintocaml[firstline=28,lastline=34,highlightlines=33]{../src/backtrace/backtrace.ml}
      \endminipage
    \end{column}
    \begin{column}{0.55\textwidth}
      \onslide*<1,2>{
        \mintocaml[firstline=100,lastline=101]{../ocaml/stdlib/printexc.ml}
      }
      \only<1>{
        \listocaml[firstline=170,lastline=172]{../ocaml/stdlib/printexc.ml}{stdlib/printexc.ml:170}
      }
      \only<2>{
        \listocaml[firstline=170,lastline=172,highlightlines=172]{../ocaml/stdlib/printexc.ml}{stdlib/printexc.ml:170}
      }
      \only<3>{
        \mintc[firstline=154,lastline=156]{../ocaml/runtime/backtrace.c}
        \listc[firstline=171,lastline=192,highlightlines={184-187}]{../ocaml/runtime/backtrace.c}{runtime/backtrace.c:154}
      }
      \only<4>{
        \listocaml[firstline=155,lastline=165]{../ocaml/stdlib/printexc.ml}{stdlib/printexc.ml:55}
      }
    \end{column}
  \end{columns}
\end{frame}


\subsection{The multiple kinds of raise}
\frameSubsection{}{}

\begin{frame}{Backtraces}{When backtraces are not needed}
  \begin{columns}[c]
    \begin{column}{0.45\textwidth}
      \minipage[c][0.66\textheight][s]{\columnwidth}
      \begin{itemize}
        \item<1-> What if the backtrace for an exception is never needed?
        \item<2-> It's possible to raise the exception explicitly with \funcname{raise\_notrace} to make raising as cheap as possible
        \item<3-> An example of use in the \funcname{dynlink} library of the OCaml compiler
      \end{itemize}
      \vfill
      \onslide<3->{
      \listocaml[firstline=205,lastline=209,highlightlines=207]{../ocaml/otherlibs/dynlink/dynlink\_compilerlibs/misc.ml}{otherlibs/dynlink/dynlink\_compilerlibs/misc.ml:205}
      }
      \endminipage
    \end{column}
    \begin{column}{0.55\textwidth}
    \only<4>{
      \mintcmm[highlightlines=3]{../src/notrace/camlNotrace\_\_fun_347.cmm}
      \listcmm{../src/notrace/camlNotrace\_\_all\_somes\_327.cmm}{all\_somes}
    }
    \onslide*<5->{
      \listobjdump[highlightlines={6-9}]{../src/notrace/camlNotrace\_\_fun_347.objdump}{anonymous function from \funcname{all\_somes}}
    }
    \end{column}
  \end{columns}
\end{frame}

\begin{frame}{Backtraces}{Summary table of the multiple kinds of raise}
  \begin{itemize}
    \item When debugging information is enabled and if the backtrace of an exception isn't needed, it's possible to raise with \funcname{raise\_notrace} for performance
    \item When debugging information is \emph{not} enabled, the compiler optimises \funcname{raise} calls to inline assembly (same effect as using \funcname{raise\_notrace})
  \end{itemize}
  \bigskip
  \centering
  \begin{tabular}{| l | c | c | c | c |}
    \hline
    \parbox{10em}{Debug information \\ (\funcarg{-g} flag)}  & \multicolumn{2}{|c|}{Disabled}                                     & \multicolumn{2}{|c|}{Enabled} \\
    \hline
    Raise kind                                               & \funcname{raise} & \funcname{raise\_notrace}                       & \funcname{raise} & \funcname{raise\_notrace} \\
    \hline
    Compilation                                              & \parbox{8em}{inline assembly\\ (optimization)} & inline assembly   & \funcname{caml\_raise\_exn} & inline assembly \\
    \hline
  \end{tabular}
\end{frame}


%
% Takeaway #5
%
\subsection*{Takeaway \#5}
\frameSubsectionTakeaway{}
\againframe<5>{takeaway}
}
    \end{column}
  \end{columns}
\end{frame}

\begin{frame}{Backtraces}{\funcname{caml\_probably\_raise}'s handler doesn't catch \typename{ExnC}}
  \begin{columns}[c]
    \begin{column}{0.45\textwidth}
      \minipage[c][0.75\textheight][s]{\columnwidth}
      \begin{itemize}
        \item<1-> \funcname{match \dots with}\footnotemark pattern matching can also match exceptions
        \item<1-> The CMM of \funcname{probably\_raise} shows that this \funcname{match \dots with} is implemented with a pattern matching and a \funcname{try \dots with}
        \item<1-> But this one only matches on \typename{ExnA} exception
        \item<2-> Since the pattern matching fails to catch the exception, \funcname{reraise} is called with our current \typename{ExnC} exception
        \item<3-> Disassembly shows that \funcname{reraise} is actually a call to \funcname{caml\_reraise\_exn}, which is also an assembly function provided by the runtime
      \end{itemize}
      \vfill
      \mintocaml[firstline=15,lastline=21,highlightlines={18-21}]{../src/backtrace/backtrace.ml}
      \endminipage
    \end{column}
    \begin{column}{0.55\textwidth}
      \centering
      \onslide*<1>{\mintcmm[highlightlines={10-18}]{../src/backtrace/camlBacktrace\_\_probably\_raise\_351.cmm}}
      \onslide*<2>{\listcmm[highlightlines=18]{../src/backtrace/camlBacktrace\_\_probably\_raise_351.cmm}{CMM of probably\_raise}}
      \onslide*<3>{\mintobjdump[firstline=5,lastline=35,highlightlines=31]{../src/backtrace/camlBacktrace\_\_probably\_raise_351.objdump}}
    \end{column}
  \end{columns}
  \only<1>{\footnotetext[1]{https://blog.janestreet.com/pattern-matching-and-exception-handling-unite/}}
\end{frame}

\newcommand\listCamlReraiseExn[1][]{
  \listasm[firstline=727,lastline=734,#1]{../ocaml/runtime/amd64.S}{runtime/amd64.S:727}}

\begin{frame}{Backtraces}{Introducing \funcname{caml\_reraise\_exn}}
  \begin{columns}[c]
    \begin{column}{0.5\textwidth}
      \minipage[c][0.66\textheight][s]{\columnwidth}
      \begin{itemize}
        \item<1-> The exception have to be reraised to the next handler
        \item<2-> First, checks if the backtrace should be collected with \funcarg{domain\_state->backtrace\_active}
        \only<3>{
        \begin{itemize}
            \item If not, behaves like a \funcname{raise\_notrace} with \funcname{RESTORE\_EXN\_HANDLER\_OCAML}
        \end{itemize}
        }
        \item<4-> Jumps into \funcname{caml\_raise\_exn}
        \item<5-> Calls \funcname{caml\_stash\_backtrace} again, to scan the next part of the stack: from the trap just pop-ed to the next trap
      \end{itemize}
      \vfill
      \mintocaml[firstline=15,lastline=21,highlightlines={18-21}]{../src/backtrace/backtrace.ml}
      \endminipage
    \end{column}
    \begin{column}{0.5\textwidth}
      \raggedleft
      \onslide*<1>{\listCamlReraiseExn{}}
      \onslide*<2>{\listCamlReraiseExn[highlightlines=729]{}}
      \onslide*<3>{
        \mintc[firstline=190,lastline=192,highlightlines={191-192}]{../ocaml/runtime/amd64.S}
        \mintc[firstline=234,lastline=237,highlightlines={235-237}]{../ocaml/runtime/amd64.S}
        \medskip
        \listCamlReraiseExn[highlightlines=731]{}
      }
      \onslide*<4-5>{
        % TODO refactor with \listCamlReraiseExn
        \mintasm[firstline=727,lastline=734,highlightlines=730]{../ocaml/runtime/amd64.S}
        \medskip
      }
      \onslide*<4>{\listCamlRaiseExn[highlightlines=711]{}}
      \onslide*<5>{\listCamlRaiseExn[highlightlines=720]{}}
    \end{column}
  \end{columns}
\end{frame}

\begin{frame}{Backtraces}{Backtrace collection continues}
  \begin{columns}[c]
    \begin{column}{0.55\textwidth}
      \minipage[c][0.75\textheight][s]{\columnwidth}
      \begin{itemize}
        \item<1-> \funcname{caml\_stash\_backtrace} scans the older part of the stack, up to the next trap
        \item<6-> Controls jumps to the nested exception handler in \funcname{maybe\_raise}, which matches only on \typename{ExnB}
        \item<7-> \funcname{caml\_reraise\_exn} is called again, backtrace collection resumes with \funcname{caml\_stash\_backtrace}
      \end{itemize}
      \medskip
      \begin{itemize}
        \item<2-> Constructed backtrace:
        \begin{itemize}
          \item <2-> \funcname{definitely\_raise}
          \item <2-> \funcname{probably\_raise}
          \item <3-> \funcname{probably\_raise} (re-raise)
          \item <4-> \funcname{maybe\_raise}
          \item <5-> \funcname{main}
          \item <8-> \funcname{main} (re-raise)
        \end{itemize}
      \end{itemize}
      \vfill
      \onslide*<1-5>{\mintocaml[firstline=15,lastline=21]{../src/backtrace/backtrace.ml}}
      \onslide*<6>{\mintocaml[firstline=28,lastline=34,highlightlines=31]{../src/backtrace/backtrace.ml}}
      \onslide*<7->{\mintocaml[firstline=28,lastline=34,highlightlines=33]{../src/backtrace/backtrace.ml}}
      \endminipage
    \end{column}
    \begin{column}{0.45\textwidth}
      \raggedleft
      \onslide*<1>{\providecommand\step{6}%
% Backtraces
%
\subsection{One advantage of raising with debugging information}
\frameSubsection{}{}

\begin{frame}{Backtraces}{Debugging information \& source code}
  \begin{columns}[c]
    \begin{column}{0.5\textwidth}
      \begin{itemize}
        \item Raising an exception is fun and games, but how do we know where the exception originated? $\rightarrow$ With the backtrace associated with the exception!
        \item In order to get a backtrace from the point where an exception was raised, it's necessary to compile with debugging information enabled (\funcarg{-g} flag for the compiler)
        \item Same example as in the `Nested handlers' section
      \end{itemize}
    \end{column}
    \begin{column}{0.5\textwidth}
      \listocaml[firstline=9,lastline=34]{../src/backtrace/backtrace.ml}{backtrace/backtrace.ml}
    \end{column}
  \end{columns}
\end{frame}

\begin{frame}{Backtraces}{CMM and disassembly of \funcname{definitely\_raise}}
  \begin{columns}[c]
    \begin{column}{0.5\textwidth}
      \minipage[c][0.66\textheight][s]{\columnwidth}
      \begin{itemize}
        \item<1-> An effect of compiling with debugging information (\funcarg{-g}): the CMM no longer uses \funcname{raise\_notrace} to raise the exception but \funcname{raise} instead
        \item<2-> Disassembly shows that CMM \funcname{raise} is actually a call to \funcname{caml\_raise\_exn}, a function in assembler provided by the runtime
      \end{itemize}
      \vfill
      \mintocaml[firstline=9,lastline=13,highlightlines=12]{../src/backtrace/backtrace.ml}
      \endminipage
    \end{column}
    \begin{column}{0.5\textwidth}
      \onslide*<1>{
        \mintcmm[highlightlines=5]{../src/backtrace/camlBacktrace\_\_definitely\_raise\_347.cmm}
      }
      \onslide*<2>{
        \mintobjdump[firstline=2,highlightlines=14]{../src/backtrace/camlBacktrace\_\_definitely\_raise\_347.objdump}
      }
    \end{column}
  \end{columns}
\end{frame}

\begin{frame}{Backtraces}{Introducing \funcname{caml\_raise\_exn}}
  \begin{columns}[c]
    \begin{column}{0.5\textwidth}
      \minipage[c][0.66\textheight][s]{\columnwidth}
      \onslide*<1>{
        \begin{itemize}
          \item Raising the exception is performed using \funcname{caml\_raise\_exn} which is
            \begin{itemize}
              \item An assembly function provided by the runtime
              \item Architecture specific
            \end{itemize}
          \item Let's see what it does differently from \funcname{raise\_notrace}\dots
          \item We are about to raise \typename{ExnC} with \funcname{caml\_raise\_exn} from \funcname{definitely\_raise}
        \end{itemize}
      }
      \onslide*<2>{
        \mintobjdump[firstline=2,highlightlines=14]{../src/backtrace/camlBacktrace\_\_definitely\_raise\_347.objdump}
      }
      \vfill
      \mintocaml[firstline=9,lastline=13,highlightlines=12]{../src/backtrace/backtrace.ml}
      \endminipage
    \end{column}
    \begin{column}{0.5\textwidth}
      \centering
      \providecommand\step{1}%
% Backtraces
%
\subsection{One advantage of raising with debugging information}
\frameSubsection{}{}

\begin{frame}{Backtraces}{Debugging information \& source code}
  \begin{columns}[c]
    \begin{column}{0.5\textwidth}
      \begin{itemize}
        \item Raising an exception is fun and games, but how do we know where the exception originated? $\rightarrow$ With the backtrace associated with the exception!
        \item In order to get a backtrace from the point where an exception was raised, it's necessary to compile with debugging information enabled (\funcarg{-g} flag for the compiler)
        \item Same example as in the `Nested handlers' section
      \end{itemize}
    \end{column}
    \begin{column}{0.5\textwidth}
      \listocaml[firstline=9,lastline=34]{../src/backtrace/backtrace.ml}{backtrace/backtrace.ml}
    \end{column}
  \end{columns}
\end{frame}

\begin{frame}{Backtraces}{CMM and disassembly of \funcname{definitely\_raise}}
  \begin{columns}[c]
    \begin{column}{0.5\textwidth}
      \minipage[c][0.66\textheight][s]{\columnwidth}
      \begin{itemize}
        \item<1-> An effect of compiling with debugging information (\funcarg{-g}): the CMM no longer uses \funcname{raise\_notrace} to raise the exception but \funcname{raise} instead
        \item<2-> Disassembly shows that CMM \funcname{raise} is actually a call to \funcname{caml\_raise\_exn}, a function in assembler provided by the runtime
      \end{itemize}
      \vfill
      \mintocaml[firstline=9,lastline=13,highlightlines=12]{../src/backtrace/backtrace.ml}
      \endminipage
    \end{column}
    \begin{column}{0.5\textwidth}
      \onslide*<1>{
        \mintcmm[highlightlines=5]{../src/backtrace/camlBacktrace\_\_definitely\_raise\_347.cmm}
      }
      \onslide*<2>{
        \mintobjdump[firstline=2,highlightlines=14]{../src/backtrace/camlBacktrace\_\_definitely\_raise\_347.objdump}
      }
    \end{column}
  \end{columns}
\end{frame}

\begin{frame}{Backtraces}{Introducing \funcname{caml\_raise\_exn}}
  \begin{columns}[c]
    \begin{column}{0.5\textwidth}
      \minipage[c][0.66\textheight][s]{\columnwidth}
      \onslide*<1>{
        \begin{itemize}
          \item Raising the exception is performed using \funcname{caml\_raise\_exn} which is
            \begin{itemize}
              \item An assembly function provided by the runtime
              \item Architecture specific
            \end{itemize}
          \item Let's see what it does differently from \funcname{raise\_notrace}\dots
          \item We are about to raise \typename{ExnC} with \funcname{caml\_raise\_exn} from \funcname{definitely\_raise}
        \end{itemize}
      }
      \onslide*<2>{
        \mintobjdump[firstline=2,highlightlines=14]{../src/backtrace/camlBacktrace\_\_definitely\_raise\_347.objdump}
      }
      \vfill
      \mintocaml[firstline=9,lastline=13,highlightlines=12]{../src/backtrace/backtrace.ml}
      \endminipage
    \end{column}
    \begin{column}{0.5\textwidth}
      \centering
      \providecommand\step{1}\input{diagrams/backtrace/backtrace.tex}
    \end{column}
  \end{columns}
\end{frame}

\begin{frame}{Backtraces}{But before that, we need more fields from \typename{caml\_domain\_state}}
  \begin{columns}
    \begin{column}{0.5\textwidth}
      \begin{itemize}
        \item Remember \typename{caml\_domain\_state} the per-domain runtime state?
        \item Some other members are of interest to us now\dots
        \item \localname{backtrace\_active} is used to know if the runtime should collect backtraces
        \item \localname{backtrace\_active} can be set by
          \begin{itemize}
            \item The \funcarg{b} flag in the \funcname{OCAMLRUNPARAM} environment variable
            \item \mintinline{ocaml}{Printexc.record_backtrace : bool -> unit} \footnotemark
          \end{itemize}
      \end{itemize}
    \end{column}
    \begin{column}{0.5\textwidth}
      \begin{listing}
        \mintc[highlightlines={9-11}]{src/ocaml/domain\_state\_backtrace.h}
        \caption{runtime/caml/domain\_state.h}
      \end{listing}
    \end{column}
  \end{columns}
  \footnotetext[1]{https://v2.ocaml.org/api/Printexc.html}
\end{frame}

\newcommand\listCamlRaiseExn[1][]{
  \listasm[firstline=702,lastline=725,#1]{../ocaml/runtime/amd64.S}{runtime/amd64.S:702}}

\begin{frame}{Backtraces}{Walk through \funcname{caml\_raise\_exn}}
  \begin{columns}[T]
    \begin{column}{0.5\textwidth}
      \begin{itemize}
        \item<1-> First checks whether exceptions backtrace should be recorded
        \item<1-> By checking the \funcarg{domain\_state->backtrace\_active} value
        \item<2-> If not, acts as \funcname{raise\_notrace} with \funcname{RESTORE\_EXN\_HANDLER\_OCAML}
          \begin{itemize}
            \item Set \regname{RSP} to \localname{domain\_state->exn\_handler} value
            \item Restore \localname{domain\_state->exn\_handler} from trap
            \item Jump with \funcname{ret} (same effect as using \funcname{pop} and \funcname{jmp})
          \end{itemize}
        \item<3-> Otherwise, switches to the C stack to be able to execute C code
        \item<4-> Calls \funcname{caml\_stash\_backtrace}, a C function provided by the runtime\ignorespaces
          \onslide<6->{, with
            \begin{itemize}
              \item \funcarg{exn}: exception value
              \item \funcarg{pc}: code address of the raising point
              \item \funcarg{sp}: stack address of the raising function
              \item \funcarg{trapsp}: stack address of the next handler
            \end{itemize}
          }
      \end{itemize}
      \bigskip
      \mintocaml[firstline=9,lastline=13,highlightlines=12]{../src/backtrace/backtrace.ml}
    \end{column}
    \begin{column}{0.5\textwidth}
      \raggedleft
      \onslide*<1,3-4>{
        \mintc[firstline=190,lastline=192]{../ocaml/runtime/amd64.S}
        \mintc[firstline=234,lastline=237]{../ocaml/runtime/amd64.S}
      }
      \onslide*<2>{
        \mintc[firstline=190,lastline=192,highlightlines={191-192}]{../ocaml/runtime/amd64.S}
        \mintc[firstline=234,lastline=237,highlightlines={235-237}]{../ocaml/runtime/amd64.S}
      }
      \onslide*<1>{\listCamlRaiseExn[highlightlines=705]{}}%
      \onslide*<2>{\listCamlRaiseExn[highlightlines={707-708}]{}}%
      \onslide*<3>{\listCamlRaiseExn[highlightlines={712-714}]{}}%
      \onslide*<4>{\listCamlRaiseExn[highlightlines={715-720}]{}}%
      \onslide*<5>{\providecommand\step{1}\input{diagrams/backtrace/backtrace.tex}}%
      \onslide*<6>{\providecommand\step{2}\input{diagrams/backtrace/backtrace.tex}}%
    \end{column}
  \end{columns}
\end{frame}

\newcommand\listStashBacktrace[1][]{
  \listc[firstline=91,lastline=120,#1]{../ocaml/runtime/backtrace\_nat.c}{runtime/backtrace\_nat.c:91}}

\begin{frame}{Backtraces}{Introducing \funcname{caml\_stash\_backtrace}}
  \begin{columns}[c]
    \begin{column}{0.5\textwidth}
      \minipage[c][0.66\textheight][s]{\columnwidth}
      \begin{itemize}
        \item<1-> A C function provided by the runtime (specific to native code)
        \item<2-> It scans the stack, between the stack pointer at the raising point up to the next trap, in order to collect the names of the detected function frames
          \onslide*<2>{
            \begin{itemize}
              \item And more information, like line number, column number...
            \end{itemize}
          }
        \item<3-> It uses the same technique and information as the Garbage Collector (GC) uses to scan the values live on the stack
          \onslide*<4>{
            \begin{itemize}
              \item \typename{frame\_descr} are at the core of stack unwinding
            \end{itemize}
          }
      \end{itemize}
      \vfill
      \mintocaml[firstline=9,lastline=13,highlightlines=12]{../src/backtrace/backtrace.ml}
      \endminipage
    \end{column}
    \begin{column}{0.5\textwidth}
      \onslide*<1>{\listStashBacktrace[highlightlines=91]{}}
      \onslide*<2>{\listStashBacktrace[highlightlines={91,117-118}]{}}
      \onslide*<3->{\listStashBacktrace[highlightlines={94,106,109-110}]{}}
    \end{column}
  \end{columns}
\end{frame}

\newcommand\listFrameDescr[1][]{
  \listocaml[firstline=111,lastline=124,#1]{../ocaml/asmcomp/emitaux.ml}{asmcomp/emitaux.ml:111}}
\newcommand\listDebugInfo[1][]{
  \listocaml[firstline=38,lastline=49,#1]{../ocaml/lambda/debuginfo.mli}{lambda/debuginfo.mli:38}}

\begin{frame}{Backtraces}{\typename{frame\_descr} and \typename{frame\_debuginfo} in a nutshell}
  \begin{columns}[c]
    \begin{column}{0.5\textwidth}
      \begin{itemize}
        \item<1-> For every return address in an OCaml program, there is a associated \typename{frame\_descr}
        \item<2-> A \typename{frame\_descr} specifies
        \begin{itemize}
          \item<2-> The size of the function stack frame
          \item<3-> Where live values are on the stack (so that the GC can mark them as live and prevent from being swept)
        \end{itemize}
        \item<4-> When compiled with debug information, \typename{frame\_descr} will be linked to a \typename{frame\_debuginfo}
        \begin{itemize}
          \item<5-> Contains information about the position in the source code (file name, line and column number)
        \end{itemize}
      \end{itemize}
    \end{column}
    \begin{column}{0.5\textwidth}
        \onslide*<1>{\listFrameDescr[highlightlines={118-122}]{}}
        \onslide*<2>{\listFrameDescr[highlightlines={120}]{}}
        \onslide*<3>{\listFrameDescr[highlightlines={121}]{}}
        \onslide*<4->{\listFrameDescr[highlightlines={122}]{}}
        \medskip
        \onslide*<1-3>{\listDebugInfo{}}
        \onslide*<4>{\listDebugInfo[highlightlines={38,49}]{}}
        \onslide*<5>{\listDebugInfo[highlightlines={39-46}]{}}
    \end{column}
  \end{columns}
\end{frame}

\begin{frame}{Backtraces}{Scanning the stack with \typename{frame\_descr}}
  \begin{columns}[c]
    \begin{column}{0.5\textwidth}
      \begin{itemize}
        \item Given the return address of the code that raises the exception by calling \funcname{caml\_raise\_exn} (\funcarg{pc} argument of \funcname{caml\_stash\_backtrace})
        \item And the position of the stack pointer (\funcarg{sp} argument of \funcname{caml\_stash\_backtrace})
        \item The frame size given by \typename{frame\_descr}.\funcarg{frame\_size}, allows to find the end of the previous frame
        \item And the return address of the caller of this function
        \bigskip
        \onslide<3->{
        \item Note that traps (here the reddish \funcname{ExnA}, \funcname{ExnB} and \funcname{ExnC} blocks) belong to the frame that installed them, and so the frame size changes when traps are removed
        }
      \end{itemize}
    \end{column}
    \begin{column}{0.5\textwidth}
      \raggedleft
      \onslide*<1>{\providecommand\step{2}\input{diagrams/backtrace/backtrace.tex}}%
      \onslide*<2>{\providecommand\step{1}\input{diagrams/backtrace/scanstack.tex}}%
      \onslide*<3>{\providecommand\step{2}\input{diagrams/backtrace/scanstack.tex}}%
      \onslide*<4>{\providecommand\step{3}\input{diagrams/backtrace/scanstack.tex}}%
      \onslide*<5>{\providecommand\step{4}\input{diagrams/backtrace/scanstack.tex}}%
      \onslide*<6>{\providecommand\step{5}\input{diagrams/backtrace/scanstack.tex}}%
    \end{column}
  \end{columns}
\end{frame}

% Duplicate of previous-previous slide
\begin{frame}{Backtraces}{Continuing on \funcname{caml\_stash\_backtrace}}
  \begin{columns}[c]
    \begin{column}{0.5\textwidth}
      \minipage[c][0.75\textheight][s]{\columnwidth}
      \begin{itemize}
          \color{gray}
        \item A C function provided by the runtime (specific to native code)
        \item It scans the stack, between the stack pointer at the raising point up to the next trap, in order to collect the names of the detected function frames
        \item It uses the same technique and information as the Garbage Collector (GC) uses to scan the values live on the stack
      \end{itemize}
      \begin{itemize}
        \item For each function frame detected, store a pointer to the \typename{frame\_descr} in the \localname{domain\_state->backtrace\_buffer} array, effectively constructing the backtrace
      \end{itemize}
      \onslide<2->{
        \begin{itemize}
            \smallskip
          \item Constructed backtrace:
            \funcname{definitely\_raise},
            \onslide<3->{\funcname{probably\_raise}}
        \end{itemize}
      }
      \vfill
      \mintocaml[firstline=9,lastline=13,highlightlines=12]{../src/backtrace/backtrace.ml}
      \endminipage
    \end{column}
    \begin{column}{0.5\textwidth}
      \raggedleft
      \onslide*<1>{\listStashBacktrace[highlightlines={114-115}]{}}
      \onslide*<2>{\providecommand\step{3}\input{diagrams/backtrace/backtrace.tex}}%
      \onslide*<3>{\providecommand\step{4}\input{diagrams/backtrace/backtrace.tex}}%
    \end{column}
  \end{columns}
\end{frame}

\begin{frame}{Backtraces}{Back to \funcname{caml\_raise\_exn}}
  \begin{columns}[c]
    \begin{column}{0.5\textwidth}
      \minipage[c][0.66\textheight][s]{\columnwidth}
      \begin{itemize}\color{gray}
        \item Checks wheteher exceptions backtrace should be recorded, by checking the \funcarg{domain\_state->backtrace\_active} value
        \item Switches to the C stack to be able to execute C code
        \item Calls \funcname{caml\_stash\_backtrace}, to collect the backtrace up to the next trap
      \end{itemize}
      \onslide*<2->{
        \begin{itemize}
          \item<2-> Executes the exception handler in \funcname{probably\_raise} with \funcname{RESTORE\_EXN\_HANDLER\_OCAML}
            \begin{itemize}
              \item Set \regname{RSP} to \localname{domain\_state->exn\_handler} value
              \item Restore \localname{domain\_state->exn\_handler} from trap
              \item Jump to the handler code with \funcname{ret}
            \end{itemize}
        \end{itemize}
      }
      \vfill
      \onslide*<1>{\mintocaml[firstline=9,lastline=13,highlightlines=12]{../src/backtrace/backtrace.ml}}
      \onslide*<2->{\mintocaml[firstline=15,lastline=21,highlightlines={19-21}]{../src/backtrace/backtrace.ml}}
      \endminipage
    \end{column}
    \begin{column}{0.5\textwidth}
      \centering
      \onslide*<1>{
        \mintc[firstline=190,lastline=192]{../ocaml/runtime/amd64.S}
        \mintc[firstline=234,lastline=237]{../ocaml/runtime/amd64.S}
      }
      \onslide*<2>{
        \mintc[firstline=190,lastline=192,highlightlines={191-192}]{../ocaml/runtime/amd64.S}
        \mintc[firstline=234,lastline=237,highlightlines={235-237}]{../ocaml/runtime/amd64.S}
      }
      \onslide*<1>{\listCamlRaiseExn[highlightlines=720]{}}
      \onslide*<2>{\listCamlRaiseExn[highlightlines=722]{}}
      \onslide*<3->{\providecommand\step{5}\input{diagrams/backtrace/backtrace.tex}}
    \end{column}
  \end{columns}
\end{frame}

\begin{frame}{Backtraces}{\funcname{caml\_probably\_raise}'s handler doesn't catch \typename{ExnC}}
  \begin{columns}[c]
    \begin{column}{0.45\textwidth}
      \minipage[c][0.75\textheight][s]{\columnwidth}
      \begin{itemize}
        \item<1-> \funcname{match \dots with}\footnotemark pattern matching can also match exceptions
        \item<1-> The CMM of \funcname{probably\_raise} shows that this \funcname{match \dots with} is implemented with a pattern matching and a \funcname{try \dots with}
        \item<1-> But this one only matches on \typename{ExnA} exception
        \item<2-> Since the pattern matching fails to catch the exception, \funcname{reraise} is called with our current \typename{ExnC} exception
        \item<3-> Disassembly shows that \funcname{reraise} is actually a call to \funcname{caml\_reraise\_exn}, which is also an assembly function provided by the runtime
      \end{itemize}
      \vfill
      \mintocaml[firstline=15,lastline=21,highlightlines={18-21}]{../src/backtrace/backtrace.ml}
      \endminipage
    \end{column}
    \begin{column}{0.55\textwidth}
      \centering
      \onslide*<1>{\mintcmm[highlightlines={10-18}]{../src/backtrace/camlBacktrace\_\_probably\_raise\_351.cmm}}
      \onslide*<2>{\listcmm[highlightlines=18]{../src/backtrace/camlBacktrace\_\_probably\_raise_351.cmm}{CMM of probably\_raise}}
      \onslide*<3>{\mintobjdump[firstline=5,lastline=35,highlightlines=31]{../src/backtrace/camlBacktrace\_\_probably\_raise_351.objdump}}
    \end{column}
  \end{columns}
  \only<1>{\footnotetext[1]{https://blog.janestreet.com/pattern-matching-and-exception-handling-unite/}}
\end{frame}

\newcommand\listCamlReraiseExn[1][]{
  \listasm[firstline=727,lastline=734,#1]{../ocaml/runtime/amd64.S}{runtime/amd64.S:727}}

\begin{frame}{Backtraces}{Introducing \funcname{caml\_reraise\_exn}}
  \begin{columns}[c]
    \begin{column}{0.5\textwidth}
      \minipage[c][0.66\textheight][s]{\columnwidth}
      \begin{itemize}
        \item<1-> The exception have to be reraised to the next handler
        \item<2-> First, checks if the backtrace should be collected with \funcarg{domain\_state->backtrace\_active}
        \only<3>{
        \begin{itemize}
            \item If not, behaves like a \funcname{raise\_notrace} with \funcname{RESTORE\_EXN\_HANDLER\_OCAML}
        \end{itemize}
        }
        \item<4-> Jumps into \funcname{caml\_raise\_exn}
        \item<5-> Calls \funcname{caml\_stash\_backtrace} again, to scan the next part of the stack: from the trap just pop-ed to the next trap
      \end{itemize}
      \vfill
      \mintocaml[firstline=15,lastline=21,highlightlines={18-21}]{../src/backtrace/backtrace.ml}
      \endminipage
    \end{column}
    \begin{column}{0.5\textwidth}
      \raggedleft
      \onslide*<1>{\listCamlReraiseExn{}}
      \onslide*<2>{\listCamlReraiseExn[highlightlines=729]{}}
      \onslide*<3>{
        \mintc[firstline=190,lastline=192,highlightlines={191-192}]{../ocaml/runtime/amd64.S}
        \mintc[firstline=234,lastline=237,highlightlines={235-237}]{../ocaml/runtime/amd64.S}
        \medskip
        \listCamlReraiseExn[highlightlines=731]{}
      }
      \onslide*<4-5>{
        % TODO refactor with \listCamlReraiseExn
        \mintasm[firstline=727,lastline=734,highlightlines=730]{../ocaml/runtime/amd64.S}
        \medskip
      }
      \onslide*<4>{\listCamlRaiseExn[highlightlines=711]{}}
      \onslide*<5>{\listCamlRaiseExn[highlightlines=720]{}}
    \end{column}
  \end{columns}
\end{frame}

\begin{frame}{Backtraces}{Backtrace collection continues}
  \begin{columns}[c]
    \begin{column}{0.55\textwidth}
      \minipage[c][0.75\textheight][s]{\columnwidth}
      \begin{itemize}
        \item<1-> \funcname{caml\_stash\_backtrace} scans the older part of the stack, up to the next trap
        \item<6-> Controls jumps to the nested exception handler in \funcname{maybe\_raise}, which matches only on \typename{ExnB}
        \item<7-> \funcname{caml\_reraise\_exn} is called again, backtrace collection resumes with \funcname{caml\_stash\_backtrace}
      \end{itemize}
      \medskip
      \begin{itemize}
        \item<2-> Constructed backtrace:
        \begin{itemize}
          \item <2-> \funcname{definitely\_raise}
          \item <2-> \funcname{probably\_raise}
          \item <3-> \funcname{probably\_raise} (re-raise)
          \item <4-> \funcname{maybe\_raise}
          \item <5-> \funcname{main}
          \item <8-> \funcname{main} (re-raise)
        \end{itemize}
      \end{itemize}
      \vfill
      \onslide*<1-5>{\mintocaml[firstline=15,lastline=21]{../src/backtrace/backtrace.ml}}
      \onslide*<6>{\mintocaml[firstline=28,lastline=34,highlightlines=31]{../src/backtrace/backtrace.ml}}
      \onslide*<7->{\mintocaml[firstline=28,lastline=34,highlightlines=33]{../src/backtrace/backtrace.ml}}
      \endminipage
    \end{column}
    \begin{column}{0.45\textwidth}
      \raggedleft
      \onslide*<1>{\providecommand\step{6}\input{diagrams/backtrace/backtrace.tex}}%
      \onslide*<2>{\providecommand\step{6}\input{diagrams/backtrace/backtrace.tex}}%
      \onslide*<3>{\providecommand\step{7}\input{diagrams/backtrace/backtrace.tex}}%
      \onslide*<4>{\providecommand\step{8}\input{diagrams/backtrace/backtrace.tex}}%
      \onslide*<5>{\providecommand\step{9}\input{diagrams/backtrace/backtrace.tex}}%
      \onslide*<6>{\providecommand\step{10}\input{diagrams/backtrace/backtrace.tex}}%
      \onslide*<7>{\providecommand\step{11}\input{diagrams/backtrace/backtrace.tex}}%
      \onslide*<8>{\providecommand\step{12}\input{diagrams/backtrace/backtrace.tex}}%
      \onslide*<9>{\providecommand\step{13}\input{diagrams/backtrace/backtrace.tex}}%
    \end{column}
  \end{columns}
\end{frame}

\begin{frame}{Backtraces}{Printing the backtrace}
  \begin{columns}[c]
    \begin{column}{0.45\textwidth}
      \minipage[c][0.66\textheight][s]{\columnwidth}
      \begin{itemize}
        \item<1-> The exception backtrace can now be printed to \funcarg{stdout} with \funcname{Printexc.print_backtrace}
        \item<2-> First the exception backtrace is retrieved into an OCaml array with \funcname{get_raw_backtrace }
        \item<3-> Which is implemented as the \funcname{caml_get_exception_raw_backtrace} C function in the runtime
        \item<4-> And finally printed with \funcname{print_exception_backtrace}
      \end{itemize}
      \vfill
      \mintocaml[firstline=28,lastline=34,highlightlines=33]{../src/backtrace/backtrace.ml}
      \endminipage
    \end{column}
    \begin{column}{0.55\textwidth}
      \onslide*<1,2>{
        \mintocaml[firstline=100,lastline=101]{../ocaml/stdlib/printexc.ml}
      }
      \only<1>{
        \listocaml[firstline=170,lastline=172]{../ocaml/stdlib/printexc.ml}{stdlib/printexc.ml:170}
      }
      \only<2>{
        \listocaml[firstline=170,lastline=172,highlightlines=172]{../ocaml/stdlib/printexc.ml}{stdlib/printexc.ml:170}
      }
      \only<3>{
        \mintc[firstline=154,lastline=156]{../ocaml/runtime/backtrace.c}
        \listc[firstline=171,lastline=192,highlightlines={184-187}]{../ocaml/runtime/backtrace.c}{runtime/backtrace.c:154}
      }
      \only<4>{
        \listocaml[firstline=155,lastline=165]{../ocaml/stdlib/printexc.ml}{stdlib/printexc.ml:55}
      }
    \end{column}
  \end{columns}
\end{frame}


\subsection{The multiple kinds of raise}
\frameSubsection{}{}

\begin{frame}{Backtraces}{When backtraces are not needed}
  \begin{columns}[c]
    \begin{column}{0.45\textwidth}
      \minipage[c][0.66\textheight][s]{\columnwidth}
      \begin{itemize}
        \item<1-> What if the backtrace for an exception is never needed?
        \item<2-> It's possible to raise the exception explicitly with \funcname{raise\_notrace} to make raising as cheap as possible
        \item<3-> An example of use in the \funcname{dynlink} library of the OCaml compiler
      \end{itemize}
      \vfill
      \onslide<3->{
      \listocaml[firstline=205,lastline=209,highlightlines=207]{../ocaml/otherlibs/dynlink/dynlink\_compilerlibs/misc.ml}{otherlibs/dynlink/dynlink\_compilerlibs/misc.ml:205}
      }
      \endminipage
    \end{column}
    \begin{column}{0.55\textwidth}
    \only<4>{
      \mintcmm[highlightlines=3]{../src/notrace/camlNotrace\_\_fun_347.cmm}
      \listcmm{../src/notrace/camlNotrace\_\_all\_somes\_327.cmm}{all\_somes}
    }
    \onslide*<5->{
      \listobjdump[highlightlines={6-9}]{../src/notrace/camlNotrace\_\_fun_347.objdump}{anonymous function from \funcname{all\_somes}}
    }
    \end{column}
  \end{columns}
\end{frame}

\begin{frame}{Backtraces}{Summary table of the multiple kinds of raise}
  \begin{itemize}
    \item When debugging information is enabled and if the backtrace of an exception isn't needed, it's possible to raise with \funcname{raise\_notrace} for performance
    \item When debugging information is \emph{not} enabled, the compiler optimises \funcname{raise} calls to inline assembly (same effect as using \funcname{raise\_notrace})
  \end{itemize}
  \bigskip
  \centering
  \begin{tabular}{| l | c | c | c | c |}
    \hline
    \parbox{10em}{Debug information \\ (\funcarg{-g} flag)}  & \multicolumn{2}{|c|}{Disabled}                                     & \multicolumn{2}{|c|}{Enabled} \\
    \hline
    Raise kind                                               & \funcname{raise} & \funcname{raise\_notrace}                       & \funcname{raise} & \funcname{raise\_notrace} \\
    \hline
    Compilation                                              & \parbox{8em}{inline assembly\\ (optimization)} & inline assembly   & \funcname{caml\_raise\_exn} & inline assembly \\
    \hline
  \end{tabular}
\end{frame}


%
% Takeaway #5
%
\subsection*{Takeaway \#5}
\frameSubsectionTakeaway{}
\againframe<5>{takeaway}

    \end{column}
  \end{columns}
\end{frame}

\begin{frame}{Backtraces}{But before that, we need more fields from \typename{caml\_domain\_state}}
  \begin{columns}
    \begin{column}{0.5\textwidth}
      \begin{itemize}
        \item Remember \typename{caml\_domain\_state} the per-domain runtime state?
        \item Some other members are of interest to us now\dots
        \item \localname{backtrace\_active} is used to know if the runtime should collect backtraces
        \item \localname{backtrace\_active} can be set by
          \begin{itemize}
            \item The \funcarg{b} flag in the \funcname{OCAMLRUNPARAM} environment variable
            \item \mintinline{ocaml}{Printexc.record_backtrace : bool -> unit} \footnotemark
          \end{itemize}
      \end{itemize}
    \end{column}
    \begin{column}{0.5\textwidth}
      \begin{listing}
        \mintc[highlightlines={9-11}]{src/ocaml/domain\_state\_backtrace.h}
        \caption{runtime/caml/domain\_state.h}
      \end{listing}
    \end{column}
  \end{columns}
  \footnotetext[1]{https://v2.ocaml.org/api/Printexc.html}
\end{frame}

\newcommand\listCamlRaiseExn[1][]{
  \listasm[firstline=702,lastline=725,#1]{../ocaml/runtime/amd64.S}{runtime/amd64.S:702}}

\begin{frame}{Backtraces}{Walk through \funcname{caml\_raise\_exn}}
  \begin{columns}[T]
    \begin{column}{0.5\textwidth}
      \begin{itemize}
        \item<1-> First checks whether exceptions backtrace should be recorded
        \item<1-> By checking the \funcarg{domain\_state->backtrace\_active} value
        \item<2-> If not, acts as \funcname{raise\_notrace} with \funcname{RESTORE\_EXN\_HANDLER\_OCAML}
          \begin{itemize}
            \item Set \regname{RSP} to \localname{domain\_state->exn\_handler} value
            \item Restore \localname{domain\_state->exn\_handler} from trap
            \item Jump with \funcname{ret} (same effect as using \funcname{pop} and \funcname{jmp})
          \end{itemize}
        \item<3-> Otherwise, switches to the C stack to be able to execute C code
        \item<4-> Calls \funcname{caml\_stash\_backtrace}, a C function provided by the runtime\ignorespaces
          \onslide<6->{, with
            \begin{itemize}
              \item \funcarg{exn}: exception value
              \item \funcarg{pc}: code address of the raising point
              \item \funcarg{sp}: stack address of the raising function
              \item \funcarg{trapsp}: stack address of the next handler
            \end{itemize}
          }
      \end{itemize}
      \bigskip
      \mintocaml[firstline=9,lastline=13,highlightlines=12]{../src/backtrace/backtrace.ml}
    \end{column}
    \begin{column}{0.5\textwidth}
      \raggedleft
      \onslide*<1,3-4>{
        \mintc[firstline=190,lastline=192]{../ocaml/runtime/amd64.S}
        \mintc[firstline=234,lastline=237]{../ocaml/runtime/amd64.S}
      }
      \onslide*<2>{
        \mintc[firstline=190,lastline=192,highlightlines={191-192}]{../ocaml/runtime/amd64.S}
        \mintc[firstline=234,lastline=237,highlightlines={235-237}]{../ocaml/runtime/amd64.S}
      }
      \onslide*<1>{\listCamlRaiseExn[highlightlines=705]{}}%
      \onslide*<2>{\listCamlRaiseExn[highlightlines={707-708}]{}}%
      \onslide*<3>{\listCamlRaiseExn[highlightlines={712-714}]{}}%
      \onslide*<4>{\listCamlRaiseExn[highlightlines={715-720}]{}}%
      \onslide*<5>{\providecommand\step{1}%
% Backtraces
%
\subsection{One advantage of raising with debugging information}
\frameSubsection{}{}

\begin{frame}{Backtraces}{Debugging information \& source code}
  \begin{columns}[c]
    \begin{column}{0.5\textwidth}
      \begin{itemize}
        \item Raising an exception is fun and games, but how do we know where the exception originated? $\rightarrow$ With the backtrace associated with the exception!
        \item In order to get a backtrace from the point where an exception was raised, it's necessary to compile with debugging information enabled (\funcarg{-g} flag for the compiler)
        \item Same example as in the `Nested handlers' section
      \end{itemize}
    \end{column}
    \begin{column}{0.5\textwidth}
      \listocaml[firstline=9,lastline=34]{../src/backtrace/backtrace.ml}{backtrace/backtrace.ml}
    \end{column}
  \end{columns}
\end{frame}

\begin{frame}{Backtraces}{CMM and disassembly of \funcname{definitely\_raise}}
  \begin{columns}[c]
    \begin{column}{0.5\textwidth}
      \minipage[c][0.66\textheight][s]{\columnwidth}
      \begin{itemize}
        \item<1-> An effect of compiling with debugging information (\funcarg{-g}): the CMM no longer uses \funcname{raise\_notrace} to raise the exception but \funcname{raise} instead
        \item<2-> Disassembly shows that CMM \funcname{raise} is actually a call to \funcname{caml\_raise\_exn}, a function in assembler provided by the runtime
      \end{itemize}
      \vfill
      \mintocaml[firstline=9,lastline=13,highlightlines=12]{../src/backtrace/backtrace.ml}
      \endminipage
    \end{column}
    \begin{column}{0.5\textwidth}
      \onslide*<1>{
        \mintcmm[highlightlines=5]{../src/backtrace/camlBacktrace\_\_definitely\_raise\_347.cmm}
      }
      \onslide*<2>{
        \mintobjdump[firstline=2,highlightlines=14]{../src/backtrace/camlBacktrace\_\_definitely\_raise\_347.objdump}
      }
    \end{column}
  \end{columns}
\end{frame}

\begin{frame}{Backtraces}{Introducing \funcname{caml\_raise\_exn}}
  \begin{columns}[c]
    \begin{column}{0.5\textwidth}
      \minipage[c][0.66\textheight][s]{\columnwidth}
      \onslide*<1>{
        \begin{itemize}
          \item Raising the exception is performed using \funcname{caml\_raise\_exn} which is
            \begin{itemize}
              \item An assembly function provided by the runtime
              \item Architecture specific
            \end{itemize}
          \item Let's see what it does differently from \funcname{raise\_notrace}\dots
          \item We are about to raise \typename{ExnC} with \funcname{caml\_raise\_exn} from \funcname{definitely\_raise}
        \end{itemize}
      }
      \onslide*<2>{
        \mintobjdump[firstline=2,highlightlines=14]{../src/backtrace/camlBacktrace\_\_definitely\_raise\_347.objdump}
      }
      \vfill
      \mintocaml[firstline=9,lastline=13,highlightlines=12]{../src/backtrace/backtrace.ml}
      \endminipage
    \end{column}
    \begin{column}{0.5\textwidth}
      \centering
      \providecommand\step{1}\input{diagrams/backtrace/backtrace.tex}
    \end{column}
  \end{columns}
\end{frame}

\begin{frame}{Backtraces}{But before that, we need more fields from \typename{caml\_domain\_state}}
  \begin{columns}
    \begin{column}{0.5\textwidth}
      \begin{itemize}
        \item Remember \typename{caml\_domain\_state} the per-domain runtime state?
        \item Some other members are of interest to us now\dots
        \item \localname{backtrace\_active} is used to know if the runtime should collect backtraces
        \item \localname{backtrace\_active} can be set by
          \begin{itemize}
            \item The \funcarg{b} flag in the \funcname{OCAMLRUNPARAM} environment variable
            \item \mintinline{ocaml}{Printexc.record_backtrace : bool -> unit} \footnotemark
          \end{itemize}
      \end{itemize}
    \end{column}
    \begin{column}{0.5\textwidth}
      \begin{listing}
        \mintc[highlightlines={9-11}]{src/ocaml/domain\_state\_backtrace.h}
        \caption{runtime/caml/domain\_state.h}
      \end{listing}
    \end{column}
  \end{columns}
  \footnotetext[1]{https://v2.ocaml.org/api/Printexc.html}
\end{frame}

\newcommand\listCamlRaiseExn[1][]{
  \listasm[firstline=702,lastline=725,#1]{../ocaml/runtime/amd64.S}{runtime/amd64.S:702}}

\begin{frame}{Backtraces}{Walk through \funcname{caml\_raise\_exn}}
  \begin{columns}[T]
    \begin{column}{0.5\textwidth}
      \begin{itemize}
        \item<1-> First checks whether exceptions backtrace should be recorded
        \item<1-> By checking the \funcarg{domain\_state->backtrace\_active} value
        \item<2-> If not, acts as \funcname{raise\_notrace} with \funcname{RESTORE\_EXN\_HANDLER\_OCAML}
          \begin{itemize}
            \item Set \regname{RSP} to \localname{domain\_state->exn\_handler} value
            \item Restore \localname{domain\_state->exn\_handler} from trap
            \item Jump with \funcname{ret} (same effect as using \funcname{pop} and \funcname{jmp})
          \end{itemize}
        \item<3-> Otherwise, switches to the C stack to be able to execute C code
        \item<4-> Calls \funcname{caml\_stash\_backtrace}, a C function provided by the runtime\ignorespaces
          \onslide<6->{, with
            \begin{itemize}
              \item \funcarg{exn}: exception value
              \item \funcarg{pc}: code address of the raising point
              \item \funcarg{sp}: stack address of the raising function
              \item \funcarg{trapsp}: stack address of the next handler
            \end{itemize}
          }
      \end{itemize}
      \bigskip
      \mintocaml[firstline=9,lastline=13,highlightlines=12]{../src/backtrace/backtrace.ml}
    \end{column}
    \begin{column}{0.5\textwidth}
      \raggedleft
      \onslide*<1,3-4>{
        \mintc[firstline=190,lastline=192]{../ocaml/runtime/amd64.S}
        \mintc[firstline=234,lastline=237]{../ocaml/runtime/amd64.S}
      }
      \onslide*<2>{
        \mintc[firstline=190,lastline=192,highlightlines={191-192}]{../ocaml/runtime/amd64.S}
        \mintc[firstline=234,lastline=237,highlightlines={235-237}]{../ocaml/runtime/amd64.S}
      }
      \onslide*<1>{\listCamlRaiseExn[highlightlines=705]{}}%
      \onslide*<2>{\listCamlRaiseExn[highlightlines={707-708}]{}}%
      \onslide*<3>{\listCamlRaiseExn[highlightlines={712-714}]{}}%
      \onslide*<4>{\listCamlRaiseExn[highlightlines={715-720}]{}}%
      \onslide*<5>{\providecommand\step{1}\input{diagrams/backtrace/backtrace.tex}}%
      \onslide*<6>{\providecommand\step{2}\input{diagrams/backtrace/backtrace.tex}}%
    \end{column}
  \end{columns}
\end{frame}

\newcommand\listStashBacktrace[1][]{
  \listc[firstline=91,lastline=120,#1]{../ocaml/runtime/backtrace\_nat.c}{runtime/backtrace\_nat.c:91}}

\begin{frame}{Backtraces}{Introducing \funcname{caml\_stash\_backtrace}}
  \begin{columns}[c]
    \begin{column}{0.5\textwidth}
      \minipage[c][0.66\textheight][s]{\columnwidth}
      \begin{itemize}
        \item<1-> A C function provided by the runtime (specific to native code)
        \item<2-> It scans the stack, between the stack pointer at the raising point up to the next trap, in order to collect the names of the detected function frames
          \onslide*<2>{
            \begin{itemize}
              \item And more information, like line number, column number...
            \end{itemize}
          }
        \item<3-> It uses the same technique and information as the Garbage Collector (GC) uses to scan the values live on the stack
          \onslide*<4>{
            \begin{itemize}
              \item \typename{frame\_descr} are at the core of stack unwinding
            \end{itemize}
          }
      \end{itemize}
      \vfill
      \mintocaml[firstline=9,lastline=13,highlightlines=12]{../src/backtrace/backtrace.ml}
      \endminipage
    \end{column}
    \begin{column}{0.5\textwidth}
      \onslide*<1>{\listStashBacktrace[highlightlines=91]{}}
      \onslide*<2>{\listStashBacktrace[highlightlines={91,117-118}]{}}
      \onslide*<3->{\listStashBacktrace[highlightlines={94,106,109-110}]{}}
    \end{column}
  \end{columns}
\end{frame}

\newcommand\listFrameDescr[1][]{
  \listocaml[firstline=111,lastline=124,#1]{../ocaml/asmcomp/emitaux.ml}{asmcomp/emitaux.ml:111}}
\newcommand\listDebugInfo[1][]{
  \listocaml[firstline=38,lastline=49,#1]{../ocaml/lambda/debuginfo.mli}{lambda/debuginfo.mli:38}}

\begin{frame}{Backtraces}{\typename{frame\_descr} and \typename{frame\_debuginfo} in a nutshell}
  \begin{columns}[c]
    \begin{column}{0.5\textwidth}
      \begin{itemize}
        \item<1-> For every return address in an OCaml program, there is a associated \typename{frame\_descr}
        \item<2-> A \typename{frame\_descr} specifies
        \begin{itemize}
          \item<2-> The size of the function stack frame
          \item<3-> Where live values are on the stack (so that the GC can mark them as live and prevent from being swept)
        \end{itemize}
        \item<4-> When compiled with debug information, \typename{frame\_descr} will be linked to a \typename{frame\_debuginfo}
        \begin{itemize}
          \item<5-> Contains information about the position in the source code (file name, line and column number)
        \end{itemize}
      \end{itemize}
    \end{column}
    \begin{column}{0.5\textwidth}
        \onslide*<1>{\listFrameDescr[highlightlines={118-122}]{}}
        \onslide*<2>{\listFrameDescr[highlightlines={120}]{}}
        \onslide*<3>{\listFrameDescr[highlightlines={121}]{}}
        \onslide*<4->{\listFrameDescr[highlightlines={122}]{}}
        \medskip
        \onslide*<1-3>{\listDebugInfo{}}
        \onslide*<4>{\listDebugInfo[highlightlines={38,49}]{}}
        \onslide*<5>{\listDebugInfo[highlightlines={39-46}]{}}
    \end{column}
  \end{columns}
\end{frame}

\begin{frame}{Backtraces}{Scanning the stack with \typename{frame\_descr}}
  \begin{columns}[c]
    \begin{column}{0.5\textwidth}
      \begin{itemize}
        \item Given the return address of the code that raises the exception by calling \funcname{caml\_raise\_exn} (\funcarg{pc} argument of \funcname{caml\_stash\_backtrace})
        \item And the position of the stack pointer (\funcarg{sp} argument of \funcname{caml\_stash\_backtrace})
        \item The frame size given by \typename{frame\_descr}.\funcarg{frame\_size}, allows to find the end of the previous frame
        \item And the return address of the caller of this function
        \bigskip
        \onslide<3->{
        \item Note that traps (here the reddish \funcname{ExnA}, \funcname{ExnB} and \funcname{ExnC} blocks) belong to the frame that installed them, and so the frame size changes when traps are removed
        }
      \end{itemize}
    \end{column}
    \begin{column}{0.5\textwidth}
      \raggedleft
      \onslide*<1>{\providecommand\step{2}\input{diagrams/backtrace/backtrace.tex}}%
      \onslide*<2>{\providecommand\step{1}\input{diagrams/backtrace/scanstack.tex}}%
      \onslide*<3>{\providecommand\step{2}\input{diagrams/backtrace/scanstack.tex}}%
      \onslide*<4>{\providecommand\step{3}\input{diagrams/backtrace/scanstack.tex}}%
      \onslide*<5>{\providecommand\step{4}\input{diagrams/backtrace/scanstack.tex}}%
      \onslide*<6>{\providecommand\step{5}\input{diagrams/backtrace/scanstack.tex}}%
    \end{column}
  \end{columns}
\end{frame}

% Duplicate of previous-previous slide
\begin{frame}{Backtraces}{Continuing on \funcname{caml\_stash\_backtrace}}
  \begin{columns}[c]
    \begin{column}{0.5\textwidth}
      \minipage[c][0.75\textheight][s]{\columnwidth}
      \begin{itemize}
          \color{gray}
        \item A C function provided by the runtime (specific to native code)
        \item It scans the stack, between the stack pointer at the raising point up to the next trap, in order to collect the names of the detected function frames
        \item It uses the same technique and information as the Garbage Collector (GC) uses to scan the values live on the stack
      \end{itemize}
      \begin{itemize}
        \item For each function frame detected, store a pointer to the \typename{frame\_descr} in the \localname{domain\_state->backtrace\_buffer} array, effectively constructing the backtrace
      \end{itemize}
      \onslide<2->{
        \begin{itemize}
            \smallskip
          \item Constructed backtrace:
            \funcname{definitely\_raise},
            \onslide<3->{\funcname{probably\_raise}}
        \end{itemize}
      }
      \vfill
      \mintocaml[firstline=9,lastline=13,highlightlines=12]{../src/backtrace/backtrace.ml}
      \endminipage
    \end{column}
    \begin{column}{0.5\textwidth}
      \raggedleft
      \onslide*<1>{\listStashBacktrace[highlightlines={114-115}]{}}
      \onslide*<2>{\providecommand\step{3}\input{diagrams/backtrace/backtrace.tex}}%
      \onslide*<3>{\providecommand\step{4}\input{diagrams/backtrace/backtrace.tex}}%
    \end{column}
  \end{columns}
\end{frame}

\begin{frame}{Backtraces}{Back to \funcname{caml\_raise\_exn}}
  \begin{columns}[c]
    \begin{column}{0.5\textwidth}
      \minipage[c][0.66\textheight][s]{\columnwidth}
      \begin{itemize}\color{gray}
        \item Checks wheteher exceptions backtrace should be recorded, by checking the \funcarg{domain\_state->backtrace\_active} value
        \item Switches to the C stack to be able to execute C code
        \item Calls \funcname{caml\_stash\_backtrace}, to collect the backtrace up to the next trap
      \end{itemize}
      \onslide*<2->{
        \begin{itemize}
          \item<2-> Executes the exception handler in \funcname{probably\_raise} with \funcname{RESTORE\_EXN\_HANDLER\_OCAML}
            \begin{itemize}
              \item Set \regname{RSP} to \localname{domain\_state->exn\_handler} value
              \item Restore \localname{domain\_state->exn\_handler} from trap
              \item Jump to the handler code with \funcname{ret}
            \end{itemize}
        \end{itemize}
      }
      \vfill
      \onslide*<1>{\mintocaml[firstline=9,lastline=13,highlightlines=12]{../src/backtrace/backtrace.ml}}
      \onslide*<2->{\mintocaml[firstline=15,lastline=21,highlightlines={19-21}]{../src/backtrace/backtrace.ml}}
      \endminipage
    \end{column}
    \begin{column}{0.5\textwidth}
      \centering
      \onslide*<1>{
        \mintc[firstline=190,lastline=192]{../ocaml/runtime/amd64.S}
        \mintc[firstline=234,lastline=237]{../ocaml/runtime/amd64.S}
      }
      \onslide*<2>{
        \mintc[firstline=190,lastline=192,highlightlines={191-192}]{../ocaml/runtime/amd64.S}
        \mintc[firstline=234,lastline=237,highlightlines={235-237}]{../ocaml/runtime/amd64.S}
      }
      \onslide*<1>{\listCamlRaiseExn[highlightlines=720]{}}
      \onslide*<2>{\listCamlRaiseExn[highlightlines=722]{}}
      \onslide*<3->{\providecommand\step{5}\input{diagrams/backtrace/backtrace.tex}}
    \end{column}
  \end{columns}
\end{frame}

\begin{frame}{Backtraces}{\funcname{caml\_probably\_raise}'s handler doesn't catch \typename{ExnC}}
  \begin{columns}[c]
    \begin{column}{0.45\textwidth}
      \minipage[c][0.75\textheight][s]{\columnwidth}
      \begin{itemize}
        \item<1-> \funcname{match \dots with}\footnotemark pattern matching can also match exceptions
        \item<1-> The CMM of \funcname{probably\_raise} shows that this \funcname{match \dots with} is implemented with a pattern matching and a \funcname{try \dots with}
        \item<1-> But this one only matches on \typename{ExnA} exception
        \item<2-> Since the pattern matching fails to catch the exception, \funcname{reraise} is called with our current \typename{ExnC} exception
        \item<3-> Disassembly shows that \funcname{reraise} is actually a call to \funcname{caml\_reraise\_exn}, which is also an assembly function provided by the runtime
      \end{itemize}
      \vfill
      \mintocaml[firstline=15,lastline=21,highlightlines={18-21}]{../src/backtrace/backtrace.ml}
      \endminipage
    \end{column}
    \begin{column}{0.55\textwidth}
      \centering
      \onslide*<1>{\mintcmm[highlightlines={10-18}]{../src/backtrace/camlBacktrace\_\_probably\_raise\_351.cmm}}
      \onslide*<2>{\listcmm[highlightlines=18]{../src/backtrace/camlBacktrace\_\_probably\_raise_351.cmm}{CMM of probably\_raise}}
      \onslide*<3>{\mintobjdump[firstline=5,lastline=35,highlightlines=31]{../src/backtrace/camlBacktrace\_\_probably\_raise_351.objdump}}
    \end{column}
  \end{columns}
  \only<1>{\footnotetext[1]{https://blog.janestreet.com/pattern-matching-and-exception-handling-unite/}}
\end{frame}

\newcommand\listCamlReraiseExn[1][]{
  \listasm[firstline=727,lastline=734,#1]{../ocaml/runtime/amd64.S}{runtime/amd64.S:727}}

\begin{frame}{Backtraces}{Introducing \funcname{caml\_reraise\_exn}}
  \begin{columns}[c]
    \begin{column}{0.5\textwidth}
      \minipage[c][0.66\textheight][s]{\columnwidth}
      \begin{itemize}
        \item<1-> The exception have to be reraised to the next handler
        \item<2-> First, checks if the backtrace should be collected with \funcarg{domain\_state->backtrace\_active}
        \only<3>{
        \begin{itemize}
            \item If not, behaves like a \funcname{raise\_notrace} with \funcname{RESTORE\_EXN\_HANDLER\_OCAML}
        \end{itemize}
        }
        \item<4-> Jumps into \funcname{caml\_raise\_exn}
        \item<5-> Calls \funcname{caml\_stash\_backtrace} again, to scan the next part of the stack: from the trap just pop-ed to the next trap
      \end{itemize}
      \vfill
      \mintocaml[firstline=15,lastline=21,highlightlines={18-21}]{../src/backtrace/backtrace.ml}
      \endminipage
    \end{column}
    \begin{column}{0.5\textwidth}
      \raggedleft
      \onslide*<1>{\listCamlReraiseExn{}}
      \onslide*<2>{\listCamlReraiseExn[highlightlines=729]{}}
      \onslide*<3>{
        \mintc[firstline=190,lastline=192,highlightlines={191-192}]{../ocaml/runtime/amd64.S}
        \mintc[firstline=234,lastline=237,highlightlines={235-237}]{../ocaml/runtime/amd64.S}
        \medskip
        \listCamlReraiseExn[highlightlines=731]{}
      }
      \onslide*<4-5>{
        % TODO refactor with \listCamlReraiseExn
        \mintasm[firstline=727,lastline=734,highlightlines=730]{../ocaml/runtime/amd64.S}
        \medskip
      }
      \onslide*<4>{\listCamlRaiseExn[highlightlines=711]{}}
      \onslide*<5>{\listCamlRaiseExn[highlightlines=720]{}}
    \end{column}
  \end{columns}
\end{frame}

\begin{frame}{Backtraces}{Backtrace collection continues}
  \begin{columns}[c]
    \begin{column}{0.55\textwidth}
      \minipage[c][0.75\textheight][s]{\columnwidth}
      \begin{itemize}
        \item<1-> \funcname{caml\_stash\_backtrace} scans the older part of the stack, up to the next trap
        \item<6-> Controls jumps to the nested exception handler in \funcname{maybe\_raise}, which matches only on \typename{ExnB}
        \item<7-> \funcname{caml\_reraise\_exn} is called again, backtrace collection resumes with \funcname{caml\_stash\_backtrace}
      \end{itemize}
      \medskip
      \begin{itemize}
        \item<2-> Constructed backtrace:
        \begin{itemize}
          \item <2-> \funcname{definitely\_raise}
          \item <2-> \funcname{probably\_raise}
          \item <3-> \funcname{probably\_raise} (re-raise)
          \item <4-> \funcname{maybe\_raise}
          \item <5-> \funcname{main}
          \item <8-> \funcname{main} (re-raise)
        \end{itemize}
      \end{itemize}
      \vfill
      \onslide*<1-5>{\mintocaml[firstline=15,lastline=21]{../src/backtrace/backtrace.ml}}
      \onslide*<6>{\mintocaml[firstline=28,lastline=34,highlightlines=31]{../src/backtrace/backtrace.ml}}
      \onslide*<7->{\mintocaml[firstline=28,lastline=34,highlightlines=33]{../src/backtrace/backtrace.ml}}
      \endminipage
    \end{column}
    \begin{column}{0.45\textwidth}
      \raggedleft
      \onslide*<1>{\providecommand\step{6}\input{diagrams/backtrace/backtrace.tex}}%
      \onslide*<2>{\providecommand\step{6}\input{diagrams/backtrace/backtrace.tex}}%
      \onslide*<3>{\providecommand\step{7}\input{diagrams/backtrace/backtrace.tex}}%
      \onslide*<4>{\providecommand\step{8}\input{diagrams/backtrace/backtrace.tex}}%
      \onslide*<5>{\providecommand\step{9}\input{diagrams/backtrace/backtrace.tex}}%
      \onslide*<6>{\providecommand\step{10}\input{diagrams/backtrace/backtrace.tex}}%
      \onslide*<7>{\providecommand\step{11}\input{diagrams/backtrace/backtrace.tex}}%
      \onslide*<8>{\providecommand\step{12}\input{diagrams/backtrace/backtrace.tex}}%
      \onslide*<9>{\providecommand\step{13}\input{diagrams/backtrace/backtrace.tex}}%
    \end{column}
  \end{columns}
\end{frame}

\begin{frame}{Backtraces}{Printing the backtrace}
  \begin{columns}[c]
    \begin{column}{0.45\textwidth}
      \minipage[c][0.66\textheight][s]{\columnwidth}
      \begin{itemize}
        \item<1-> The exception backtrace can now be printed to \funcarg{stdout} with \funcname{Printexc.print_backtrace}
        \item<2-> First the exception backtrace is retrieved into an OCaml array with \funcname{get_raw_backtrace }
        \item<3-> Which is implemented as the \funcname{caml_get_exception_raw_backtrace} C function in the runtime
        \item<4-> And finally printed with \funcname{print_exception_backtrace}
      \end{itemize}
      \vfill
      \mintocaml[firstline=28,lastline=34,highlightlines=33]{../src/backtrace/backtrace.ml}
      \endminipage
    \end{column}
    \begin{column}{0.55\textwidth}
      \onslide*<1,2>{
        \mintocaml[firstline=100,lastline=101]{../ocaml/stdlib/printexc.ml}
      }
      \only<1>{
        \listocaml[firstline=170,lastline=172]{../ocaml/stdlib/printexc.ml}{stdlib/printexc.ml:170}
      }
      \only<2>{
        \listocaml[firstline=170,lastline=172,highlightlines=172]{../ocaml/stdlib/printexc.ml}{stdlib/printexc.ml:170}
      }
      \only<3>{
        \mintc[firstline=154,lastline=156]{../ocaml/runtime/backtrace.c}
        \listc[firstline=171,lastline=192,highlightlines={184-187}]{../ocaml/runtime/backtrace.c}{runtime/backtrace.c:154}
      }
      \only<4>{
        \listocaml[firstline=155,lastline=165]{../ocaml/stdlib/printexc.ml}{stdlib/printexc.ml:55}
      }
    \end{column}
  \end{columns}
\end{frame}


\subsection{The multiple kinds of raise}
\frameSubsection{}{}

\begin{frame}{Backtraces}{When backtraces are not needed}
  \begin{columns}[c]
    \begin{column}{0.45\textwidth}
      \minipage[c][0.66\textheight][s]{\columnwidth}
      \begin{itemize}
        \item<1-> What if the backtrace for an exception is never needed?
        \item<2-> It's possible to raise the exception explicitly with \funcname{raise\_notrace} to make raising as cheap as possible
        \item<3-> An example of use in the \funcname{dynlink} library of the OCaml compiler
      \end{itemize}
      \vfill
      \onslide<3->{
      \listocaml[firstline=205,lastline=209,highlightlines=207]{../ocaml/otherlibs/dynlink/dynlink\_compilerlibs/misc.ml}{otherlibs/dynlink/dynlink\_compilerlibs/misc.ml:205}
      }
      \endminipage
    \end{column}
    \begin{column}{0.55\textwidth}
    \only<4>{
      \mintcmm[highlightlines=3]{../src/notrace/camlNotrace\_\_fun_347.cmm}
      \listcmm{../src/notrace/camlNotrace\_\_all\_somes\_327.cmm}{all\_somes}
    }
    \onslide*<5->{
      \listobjdump[highlightlines={6-9}]{../src/notrace/camlNotrace\_\_fun_347.objdump}{anonymous function from \funcname{all\_somes}}
    }
    \end{column}
  \end{columns}
\end{frame}

\begin{frame}{Backtraces}{Summary table of the multiple kinds of raise}
  \begin{itemize}
    \item When debugging information is enabled and if the backtrace of an exception isn't needed, it's possible to raise with \funcname{raise\_notrace} for performance
    \item When debugging information is \emph{not} enabled, the compiler optimises \funcname{raise} calls to inline assembly (same effect as using \funcname{raise\_notrace})
  \end{itemize}
  \bigskip
  \centering
  \begin{tabular}{| l | c | c | c | c |}
    \hline
    \parbox{10em}{Debug information \\ (\funcarg{-g} flag)}  & \multicolumn{2}{|c|}{Disabled}                                     & \multicolumn{2}{|c|}{Enabled} \\
    \hline
    Raise kind                                               & \funcname{raise} & \funcname{raise\_notrace}                       & \funcname{raise} & \funcname{raise\_notrace} \\
    \hline
    Compilation                                              & \parbox{8em}{inline assembly\\ (optimization)} & inline assembly   & \funcname{caml\_raise\_exn} & inline assembly \\
    \hline
  \end{tabular}
\end{frame}


%
% Takeaway #5
%
\subsection*{Takeaway \#5}
\frameSubsectionTakeaway{}
\againframe<5>{takeaway}
}%
      \onslide*<6>{\providecommand\step{2}%
% Backtraces
%
\subsection{One advantage of raising with debugging information}
\frameSubsection{}{}

\begin{frame}{Backtraces}{Debugging information \& source code}
  \begin{columns}[c]
    \begin{column}{0.5\textwidth}
      \begin{itemize}
        \item Raising an exception is fun and games, but how do we know where the exception originated? $\rightarrow$ With the backtrace associated with the exception!
        \item In order to get a backtrace from the point where an exception was raised, it's necessary to compile with debugging information enabled (\funcarg{-g} flag for the compiler)
        \item Same example as in the `Nested handlers' section
      \end{itemize}
    \end{column}
    \begin{column}{0.5\textwidth}
      \listocaml[firstline=9,lastline=34]{../src/backtrace/backtrace.ml}{backtrace/backtrace.ml}
    \end{column}
  \end{columns}
\end{frame}

\begin{frame}{Backtraces}{CMM and disassembly of \funcname{definitely\_raise}}
  \begin{columns}[c]
    \begin{column}{0.5\textwidth}
      \minipage[c][0.66\textheight][s]{\columnwidth}
      \begin{itemize}
        \item<1-> An effect of compiling with debugging information (\funcarg{-g}): the CMM no longer uses \funcname{raise\_notrace} to raise the exception but \funcname{raise} instead
        \item<2-> Disassembly shows that CMM \funcname{raise} is actually a call to \funcname{caml\_raise\_exn}, a function in assembler provided by the runtime
      \end{itemize}
      \vfill
      \mintocaml[firstline=9,lastline=13,highlightlines=12]{../src/backtrace/backtrace.ml}
      \endminipage
    \end{column}
    \begin{column}{0.5\textwidth}
      \onslide*<1>{
        \mintcmm[highlightlines=5]{../src/backtrace/camlBacktrace\_\_definitely\_raise\_347.cmm}
      }
      \onslide*<2>{
        \mintobjdump[firstline=2,highlightlines=14]{../src/backtrace/camlBacktrace\_\_definitely\_raise\_347.objdump}
      }
    \end{column}
  \end{columns}
\end{frame}

\begin{frame}{Backtraces}{Introducing \funcname{caml\_raise\_exn}}
  \begin{columns}[c]
    \begin{column}{0.5\textwidth}
      \minipage[c][0.66\textheight][s]{\columnwidth}
      \onslide*<1>{
        \begin{itemize}
          \item Raising the exception is performed using \funcname{caml\_raise\_exn} which is
            \begin{itemize}
              \item An assembly function provided by the runtime
              \item Architecture specific
            \end{itemize}
          \item Let's see what it does differently from \funcname{raise\_notrace}\dots
          \item We are about to raise \typename{ExnC} with \funcname{caml\_raise\_exn} from \funcname{definitely\_raise}
        \end{itemize}
      }
      \onslide*<2>{
        \mintobjdump[firstline=2,highlightlines=14]{../src/backtrace/camlBacktrace\_\_definitely\_raise\_347.objdump}
      }
      \vfill
      \mintocaml[firstline=9,lastline=13,highlightlines=12]{../src/backtrace/backtrace.ml}
      \endminipage
    \end{column}
    \begin{column}{0.5\textwidth}
      \centering
      \providecommand\step{1}\input{diagrams/backtrace/backtrace.tex}
    \end{column}
  \end{columns}
\end{frame}

\begin{frame}{Backtraces}{But before that, we need more fields from \typename{caml\_domain\_state}}
  \begin{columns}
    \begin{column}{0.5\textwidth}
      \begin{itemize}
        \item Remember \typename{caml\_domain\_state} the per-domain runtime state?
        \item Some other members are of interest to us now\dots
        \item \localname{backtrace\_active} is used to know if the runtime should collect backtraces
        \item \localname{backtrace\_active} can be set by
          \begin{itemize}
            \item The \funcarg{b} flag in the \funcname{OCAMLRUNPARAM} environment variable
            \item \mintinline{ocaml}{Printexc.record_backtrace : bool -> unit} \footnotemark
          \end{itemize}
      \end{itemize}
    \end{column}
    \begin{column}{0.5\textwidth}
      \begin{listing}
        \mintc[highlightlines={9-11}]{src/ocaml/domain\_state\_backtrace.h}
        \caption{runtime/caml/domain\_state.h}
      \end{listing}
    \end{column}
  \end{columns}
  \footnotetext[1]{https://v2.ocaml.org/api/Printexc.html}
\end{frame}

\newcommand\listCamlRaiseExn[1][]{
  \listasm[firstline=702,lastline=725,#1]{../ocaml/runtime/amd64.S}{runtime/amd64.S:702}}

\begin{frame}{Backtraces}{Walk through \funcname{caml\_raise\_exn}}
  \begin{columns}[T]
    \begin{column}{0.5\textwidth}
      \begin{itemize}
        \item<1-> First checks whether exceptions backtrace should be recorded
        \item<1-> By checking the \funcarg{domain\_state->backtrace\_active} value
        \item<2-> If not, acts as \funcname{raise\_notrace} with \funcname{RESTORE\_EXN\_HANDLER\_OCAML}
          \begin{itemize}
            \item Set \regname{RSP} to \localname{domain\_state->exn\_handler} value
            \item Restore \localname{domain\_state->exn\_handler} from trap
            \item Jump with \funcname{ret} (same effect as using \funcname{pop} and \funcname{jmp})
          \end{itemize}
        \item<3-> Otherwise, switches to the C stack to be able to execute C code
        \item<4-> Calls \funcname{caml\_stash\_backtrace}, a C function provided by the runtime\ignorespaces
          \onslide<6->{, with
            \begin{itemize}
              \item \funcarg{exn}: exception value
              \item \funcarg{pc}: code address of the raising point
              \item \funcarg{sp}: stack address of the raising function
              \item \funcarg{trapsp}: stack address of the next handler
            \end{itemize}
          }
      \end{itemize}
      \bigskip
      \mintocaml[firstline=9,lastline=13,highlightlines=12]{../src/backtrace/backtrace.ml}
    \end{column}
    \begin{column}{0.5\textwidth}
      \raggedleft
      \onslide*<1,3-4>{
        \mintc[firstline=190,lastline=192]{../ocaml/runtime/amd64.S}
        \mintc[firstline=234,lastline=237]{../ocaml/runtime/amd64.S}
      }
      \onslide*<2>{
        \mintc[firstline=190,lastline=192,highlightlines={191-192}]{../ocaml/runtime/amd64.S}
        \mintc[firstline=234,lastline=237,highlightlines={235-237}]{../ocaml/runtime/amd64.S}
      }
      \onslide*<1>{\listCamlRaiseExn[highlightlines=705]{}}%
      \onslide*<2>{\listCamlRaiseExn[highlightlines={707-708}]{}}%
      \onslide*<3>{\listCamlRaiseExn[highlightlines={712-714}]{}}%
      \onslide*<4>{\listCamlRaiseExn[highlightlines={715-720}]{}}%
      \onslide*<5>{\providecommand\step{1}\input{diagrams/backtrace/backtrace.tex}}%
      \onslide*<6>{\providecommand\step{2}\input{diagrams/backtrace/backtrace.tex}}%
    \end{column}
  \end{columns}
\end{frame}

\newcommand\listStashBacktrace[1][]{
  \listc[firstline=91,lastline=120,#1]{../ocaml/runtime/backtrace\_nat.c}{runtime/backtrace\_nat.c:91}}

\begin{frame}{Backtraces}{Introducing \funcname{caml\_stash\_backtrace}}
  \begin{columns}[c]
    \begin{column}{0.5\textwidth}
      \minipage[c][0.66\textheight][s]{\columnwidth}
      \begin{itemize}
        \item<1-> A C function provided by the runtime (specific to native code)
        \item<2-> It scans the stack, between the stack pointer at the raising point up to the next trap, in order to collect the names of the detected function frames
          \onslide*<2>{
            \begin{itemize}
              \item And more information, like line number, column number...
            \end{itemize}
          }
        \item<3-> It uses the same technique and information as the Garbage Collector (GC) uses to scan the values live on the stack
          \onslide*<4>{
            \begin{itemize}
              \item \typename{frame\_descr} are at the core of stack unwinding
            \end{itemize}
          }
      \end{itemize}
      \vfill
      \mintocaml[firstline=9,lastline=13,highlightlines=12]{../src/backtrace/backtrace.ml}
      \endminipage
    \end{column}
    \begin{column}{0.5\textwidth}
      \onslide*<1>{\listStashBacktrace[highlightlines=91]{}}
      \onslide*<2>{\listStashBacktrace[highlightlines={91,117-118}]{}}
      \onslide*<3->{\listStashBacktrace[highlightlines={94,106,109-110}]{}}
    \end{column}
  \end{columns}
\end{frame}

\newcommand\listFrameDescr[1][]{
  \listocaml[firstline=111,lastline=124,#1]{../ocaml/asmcomp/emitaux.ml}{asmcomp/emitaux.ml:111}}
\newcommand\listDebugInfo[1][]{
  \listocaml[firstline=38,lastline=49,#1]{../ocaml/lambda/debuginfo.mli}{lambda/debuginfo.mli:38}}

\begin{frame}{Backtraces}{\typename{frame\_descr} and \typename{frame\_debuginfo} in a nutshell}
  \begin{columns}[c]
    \begin{column}{0.5\textwidth}
      \begin{itemize}
        \item<1-> For every return address in an OCaml program, there is a associated \typename{frame\_descr}
        \item<2-> A \typename{frame\_descr} specifies
        \begin{itemize}
          \item<2-> The size of the function stack frame
          \item<3-> Where live values are on the stack (so that the GC can mark them as live and prevent from being swept)
        \end{itemize}
        \item<4-> When compiled with debug information, \typename{frame\_descr} will be linked to a \typename{frame\_debuginfo}
        \begin{itemize}
          \item<5-> Contains information about the position in the source code (file name, line and column number)
        \end{itemize}
      \end{itemize}
    \end{column}
    \begin{column}{0.5\textwidth}
        \onslide*<1>{\listFrameDescr[highlightlines={118-122}]{}}
        \onslide*<2>{\listFrameDescr[highlightlines={120}]{}}
        \onslide*<3>{\listFrameDescr[highlightlines={121}]{}}
        \onslide*<4->{\listFrameDescr[highlightlines={122}]{}}
        \medskip
        \onslide*<1-3>{\listDebugInfo{}}
        \onslide*<4>{\listDebugInfo[highlightlines={38,49}]{}}
        \onslide*<5>{\listDebugInfo[highlightlines={39-46}]{}}
    \end{column}
  \end{columns}
\end{frame}

\begin{frame}{Backtraces}{Scanning the stack with \typename{frame\_descr}}
  \begin{columns}[c]
    \begin{column}{0.5\textwidth}
      \begin{itemize}
        \item Given the return address of the code that raises the exception by calling \funcname{caml\_raise\_exn} (\funcarg{pc} argument of \funcname{caml\_stash\_backtrace})
        \item And the position of the stack pointer (\funcarg{sp} argument of \funcname{caml\_stash\_backtrace})
        \item The frame size given by \typename{frame\_descr}.\funcarg{frame\_size}, allows to find the end of the previous frame
        \item And the return address of the caller of this function
        \bigskip
        \onslide<3->{
        \item Note that traps (here the reddish \funcname{ExnA}, \funcname{ExnB} and \funcname{ExnC} blocks) belong to the frame that installed them, and so the frame size changes when traps are removed
        }
      \end{itemize}
    \end{column}
    \begin{column}{0.5\textwidth}
      \raggedleft
      \onslide*<1>{\providecommand\step{2}\input{diagrams/backtrace/backtrace.tex}}%
      \onslide*<2>{\providecommand\step{1}\input{diagrams/backtrace/scanstack.tex}}%
      \onslide*<3>{\providecommand\step{2}\input{diagrams/backtrace/scanstack.tex}}%
      \onslide*<4>{\providecommand\step{3}\input{diagrams/backtrace/scanstack.tex}}%
      \onslide*<5>{\providecommand\step{4}\input{diagrams/backtrace/scanstack.tex}}%
      \onslide*<6>{\providecommand\step{5}\input{diagrams/backtrace/scanstack.tex}}%
    \end{column}
  \end{columns}
\end{frame}

% Duplicate of previous-previous slide
\begin{frame}{Backtraces}{Continuing on \funcname{caml\_stash\_backtrace}}
  \begin{columns}[c]
    \begin{column}{0.5\textwidth}
      \minipage[c][0.75\textheight][s]{\columnwidth}
      \begin{itemize}
          \color{gray}
        \item A C function provided by the runtime (specific to native code)
        \item It scans the stack, between the stack pointer at the raising point up to the next trap, in order to collect the names of the detected function frames
        \item It uses the same technique and information as the Garbage Collector (GC) uses to scan the values live on the stack
      \end{itemize}
      \begin{itemize}
        \item For each function frame detected, store a pointer to the \typename{frame\_descr} in the \localname{domain\_state->backtrace\_buffer} array, effectively constructing the backtrace
      \end{itemize}
      \onslide<2->{
        \begin{itemize}
            \smallskip
          \item Constructed backtrace:
            \funcname{definitely\_raise},
            \onslide<3->{\funcname{probably\_raise}}
        \end{itemize}
      }
      \vfill
      \mintocaml[firstline=9,lastline=13,highlightlines=12]{../src/backtrace/backtrace.ml}
      \endminipage
    \end{column}
    \begin{column}{0.5\textwidth}
      \raggedleft
      \onslide*<1>{\listStashBacktrace[highlightlines={114-115}]{}}
      \onslide*<2>{\providecommand\step{3}\input{diagrams/backtrace/backtrace.tex}}%
      \onslide*<3>{\providecommand\step{4}\input{diagrams/backtrace/backtrace.tex}}%
    \end{column}
  \end{columns}
\end{frame}

\begin{frame}{Backtraces}{Back to \funcname{caml\_raise\_exn}}
  \begin{columns}[c]
    \begin{column}{0.5\textwidth}
      \minipage[c][0.66\textheight][s]{\columnwidth}
      \begin{itemize}\color{gray}
        \item Checks wheteher exceptions backtrace should be recorded, by checking the \funcarg{domain\_state->backtrace\_active} value
        \item Switches to the C stack to be able to execute C code
        \item Calls \funcname{caml\_stash\_backtrace}, to collect the backtrace up to the next trap
      \end{itemize}
      \onslide*<2->{
        \begin{itemize}
          \item<2-> Executes the exception handler in \funcname{probably\_raise} with \funcname{RESTORE\_EXN\_HANDLER\_OCAML}
            \begin{itemize}
              \item Set \regname{RSP} to \localname{domain\_state->exn\_handler} value
              \item Restore \localname{domain\_state->exn\_handler} from trap
              \item Jump to the handler code with \funcname{ret}
            \end{itemize}
        \end{itemize}
      }
      \vfill
      \onslide*<1>{\mintocaml[firstline=9,lastline=13,highlightlines=12]{../src/backtrace/backtrace.ml}}
      \onslide*<2->{\mintocaml[firstline=15,lastline=21,highlightlines={19-21}]{../src/backtrace/backtrace.ml}}
      \endminipage
    \end{column}
    \begin{column}{0.5\textwidth}
      \centering
      \onslide*<1>{
        \mintc[firstline=190,lastline=192]{../ocaml/runtime/amd64.S}
        \mintc[firstline=234,lastline=237]{../ocaml/runtime/amd64.S}
      }
      \onslide*<2>{
        \mintc[firstline=190,lastline=192,highlightlines={191-192}]{../ocaml/runtime/amd64.S}
        \mintc[firstline=234,lastline=237,highlightlines={235-237}]{../ocaml/runtime/amd64.S}
      }
      \onslide*<1>{\listCamlRaiseExn[highlightlines=720]{}}
      \onslide*<2>{\listCamlRaiseExn[highlightlines=722]{}}
      \onslide*<3->{\providecommand\step{5}\input{diagrams/backtrace/backtrace.tex}}
    \end{column}
  \end{columns}
\end{frame}

\begin{frame}{Backtraces}{\funcname{caml\_probably\_raise}'s handler doesn't catch \typename{ExnC}}
  \begin{columns}[c]
    \begin{column}{0.45\textwidth}
      \minipage[c][0.75\textheight][s]{\columnwidth}
      \begin{itemize}
        \item<1-> \funcname{match \dots with}\footnotemark pattern matching can also match exceptions
        \item<1-> The CMM of \funcname{probably\_raise} shows that this \funcname{match \dots with} is implemented with a pattern matching and a \funcname{try \dots with}
        \item<1-> But this one only matches on \typename{ExnA} exception
        \item<2-> Since the pattern matching fails to catch the exception, \funcname{reraise} is called with our current \typename{ExnC} exception
        \item<3-> Disassembly shows that \funcname{reraise} is actually a call to \funcname{caml\_reraise\_exn}, which is also an assembly function provided by the runtime
      \end{itemize}
      \vfill
      \mintocaml[firstline=15,lastline=21,highlightlines={18-21}]{../src/backtrace/backtrace.ml}
      \endminipage
    \end{column}
    \begin{column}{0.55\textwidth}
      \centering
      \onslide*<1>{\mintcmm[highlightlines={10-18}]{../src/backtrace/camlBacktrace\_\_probably\_raise\_351.cmm}}
      \onslide*<2>{\listcmm[highlightlines=18]{../src/backtrace/camlBacktrace\_\_probably\_raise_351.cmm}{CMM of probably\_raise}}
      \onslide*<3>{\mintobjdump[firstline=5,lastline=35,highlightlines=31]{../src/backtrace/camlBacktrace\_\_probably\_raise_351.objdump}}
    \end{column}
  \end{columns}
  \only<1>{\footnotetext[1]{https://blog.janestreet.com/pattern-matching-and-exception-handling-unite/}}
\end{frame}

\newcommand\listCamlReraiseExn[1][]{
  \listasm[firstline=727,lastline=734,#1]{../ocaml/runtime/amd64.S}{runtime/amd64.S:727}}

\begin{frame}{Backtraces}{Introducing \funcname{caml\_reraise\_exn}}
  \begin{columns}[c]
    \begin{column}{0.5\textwidth}
      \minipage[c][0.66\textheight][s]{\columnwidth}
      \begin{itemize}
        \item<1-> The exception have to be reraised to the next handler
        \item<2-> First, checks if the backtrace should be collected with \funcarg{domain\_state->backtrace\_active}
        \only<3>{
        \begin{itemize}
            \item If not, behaves like a \funcname{raise\_notrace} with \funcname{RESTORE\_EXN\_HANDLER\_OCAML}
        \end{itemize}
        }
        \item<4-> Jumps into \funcname{caml\_raise\_exn}
        \item<5-> Calls \funcname{caml\_stash\_backtrace} again, to scan the next part of the stack: from the trap just pop-ed to the next trap
      \end{itemize}
      \vfill
      \mintocaml[firstline=15,lastline=21,highlightlines={18-21}]{../src/backtrace/backtrace.ml}
      \endminipage
    \end{column}
    \begin{column}{0.5\textwidth}
      \raggedleft
      \onslide*<1>{\listCamlReraiseExn{}}
      \onslide*<2>{\listCamlReraiseExn[highlightlines=729]{}}
      \onslide*<3>{
        \mintc[firstline=190,lastline=192,highlightlines={191-192}]{../ocaml/runtime/amd64.S}
        \mintc[firstline=234,lastline=237,highlightlines={235-237}]{../ocaml/runtime/amd64.S}
        \medskip
        \listCamlReraiseExn[highlightlines=731]{}
      }
      \onslide*<4-5>{
        % TODO refactor with \listCamlReraiseExn
        \mintasm[firstline=727,lastline=734,highlightlines=730]{../ocaml/runtime/amd64.S}
        \medskip
      }
      \onslide*<4>{\listCamlRaiseExn[highlightlines=711]{}}
      \onslide*<5>{\listCamlRaiseExn[highlightlines=720]{}}
    \end{column}
  \end{columns}
\end{frame}

\begin{frame}{Backtraces}{Backtrace collection continues}
  \begin{columns}[c]
    \begin{column}{0.55\textwidth}
      \minipage[c][0.75\textheight][s]{\columnwidth}
      \begin{itemize}
        \item<1-> \funcname{caml\_stash\_backtrace} scans the older part of the stack, up to the next trap
        \item<6-> Controls jumps to the nested exception handler in \funcname{maybe\_raise}, which matches only on \typename{ExnB}
        \item<7-> \funcname{caml\_reraise\_exn} is called again, backtrace collection resumes with \funcname{caml\_stash\_backtrace}
      \end{itemize}
      \medskip
      \begin{itemize}
        \item<2-> Constructed backtrace:
        \begin{itemize}
          \item <2-> \funcname{definitely\_raise}
          \item <2-> \funcname{probably\_raise}
          \item <3-> \funcname{probably\_raise} (re-raise)
          \item <4-> \funcname{maybe\_raise}
          \item <5-> \funcname{main}
          \item <8-> \funcname{main} (re-raise)
        \end{itemize}
      \end{itemize}
      \vfill
      \onslide*<1-5>{\mintocaml[firstline=15,lastline=21]{../src/backtrace/backtrace.ml}}
      \onslide*<6>{\mintocaml[firstline=28,lastline=34,highlightlines=31]{../src/backtrace/backtrace.ml}}
      \onslide*<7->{\mintocaml[firstline=28,lastline=34,highlightlines=33]{../src/backtrace/backtrace.ml}}
      \endminipage
    \end{column}
    \begin{column}{0.45\textwidth}
      \raggedleft
      \onslide*<1>{\providecommand\step{6}\input{diagrams/backtrace/backtrace.tex}}%
      \onslide*<2>{\providecommand\step{6}\input{diagrams/backtrace/backtrace.tex}}%
      \onslide*<3>{\providecommand\step{7}\input{diagrams/backtrace/backtrace.tex}}%
      \onslide*<4>{\providecommand\step{8}\input{diagrams/backtrace/backtrace.tex}}%
      \onslide*<5>{\providecommand\step{9}\input{diagrams/backtrace/backtrace.tex}}%
      \onslide*<6>{\providecommand\step{10}\input{diagrams/backtrace/backtrace.tex}}%
      \onslide*<7>{\providecommand\step{11}\input{diagrams/backtrace/backtrace.tex}}%
      \onslide*<8>{\providecommand\step{12}\input{diagrams/backtrace/backtrace.tex}}%
      \onslide*<9>{\providecommand\step{13}\input{diagrams/backtrace/backtrace.tex}}%
    \end{column}
  \end{columns}
\end{frame}

\begin{frame}{Backtraces}{Printing the backtrace}
  \begin{columns}[c]
    \begin{column}{0.45\textwidth}
      \minipage[c][0.66\textheight][s]{\columnwidth}
      \begin{itemize}
        \item<1-> The exception backtrace can now be printed to \funcarg{stdout} with \funcname{Printexc.print_backtrace}
        \item<2-> First the exception backtrace is retrieved into an OCaml array with \funcname{get_raw_backtrace }
        \item<3-> Which is implemented as the \funcname{caml_get_exception_raw_backtrace} C function in the runtime
        \item<4-> And finally printed with \funcname{print_exception_backtrace}
      \end{itemize}
      \vfill
      \mintocaml[firstline=28,lastline=34,highlightlines=33]{../src/backtrace/backtrace.ml}
      \endminipage
    \end{column}
    \begin{column}{0.55\textwidth}
      \onslide*<1,2>{
        \mintocaml[firstline=100,lastline=101]{../ocaml/stdlib/printexc.ml}
      }
      \only<1>{
        \listocaml[firstline=170,lastline=172]{../ocaml/stdlib/printexc.ml}{stdlib/printexc.ml:170}
      }
      \only<2>{
        \listocaml[firstline=170,lastline=172,highlightlines=172]{../ocaml/stdlib/printexc.ml}{stdlib/printexc.ml:170}
      }
      \only<3>{
        \mintc[firstline=154,lastline=156]{../ocaml/runtime/backtrace.c}
        \listc[firstline=171,lastline=192,highlightlines={184-187}]{../ocaml/runtime/backtrace.c}{runtime/backtrace.c:154}
      }
      \only<4>{
        \listocaml[firstline=155,lastline=165]{../ocaml/stdlib/printexc.ml}{stdlib/printexc.ml:55}
      }
    \end{column}
  \end{columns}
\end{frame}


\subsection{The multiple kinds of raise}
\frameSubsection{}{}

\begin{frame}{Backtraces}{When backtraces are not needed}
  \begin{columns}[c]
    \begin{column}{0.45\textwidth}
      \minipage[c][0.66\textheight][s]{\columnwidth}
      \begin{itemize}
        \item<1-> What if the backtrace for an exception is never needed?
        \item<2-> It's possible to raise the exception explicitly with \funcname{raise\_notrace} to make raising as cheap as possible
        \item<3-> An example of use in the \funcname{dynlink} library of the OCaml compiler
      \end{itemize}
      \vfill
      \onslide<3->{
      \listocaml[firstline=205,lastline=209,highlightlines=207]{../ocaml/otherlibs/dynlink/dynlink\_compilerlibs/misc.ml}{otherlibs/dynlink/dynlink\_compilerlibs/misc.ml:205}
      }
      \endminipage
    \end{column}
    \begin{column}{0.55\textwidth}
    \only<4>{
      \mintcmm[highlightlines=3]{../src/notrace/camlNotrace\_\_fun_347.cmm}
      \listcmm{../src/notrace/camlNotrace\_\_all\_somes\_327.cmm}{all\_somes}
    }
    \onslide*<5->{
      \listobjdump[highlightlines={6-9}]{../src/notrace/camlNotrace\_\_fun_347.objdump}{anonymous function from \funcname{all\_somes}}
    }
    \end{column}
  \end{columns}
\end{frame}

\begin{frame}{Backtraces}{Summary table of the multiple kinds of raise}
  \begin{itemize}
    \item When debugging information is enabled and if the backtrace of an exception isn't needed, it's possible to raise with \funcname{raise\_notrace} for performance
    \item When debugging information is \emph{not} enabled, the compiler optimises \funcname{raise} calls to inline assembly (same effect as using \funcname{raise\_notrace})
  \end{itemize}
  \bigskip
  \centering
  \begin{tabular}{| l | c | c | c | c |}
    \hline
    \parbox{10em}{Debug information \\ (\funcarg{-g} flag)}  & \multicolumn{2}{|c|}{Disabled}                                     & \multicolumn{2}{|c|}{Enabled} \\
    \hline
    Raise kind                                               & \funcname{raise} & \funcname{raise\_notrace}                       & \funcname{raise} & \funcname{raise\_notrace} \\
    \hline
    Compilation                                              & \parbox{8em}{inline assembly\\ (optimization)} & inline assembly   & \funcname{caml\_raise\_exn} & inline assembly \\
    \hline
  \end{tabular}
\end{frame}


%
% Takeaway #5
%
\subsection*{Takeaway \#5}
\frameSubsectionTakeaway{}
\againframe<5>{takeaway}
}%
    \end{column}
  \end{columns}
\end{frame}

\newcommand\listStashBacktrace[1][]{
  \listc[firstline=91,lastline=120,#1]{../ocaml/runtime/backtrace\_nat.c}{runtime/backtrace\_nat.c:91}}

\begin{frame}{Backtraces}{Introducing \funcname{caml\_stash\_backtrace}}
  \begin{columns}[c]
    \begin{column}{0.5\textwidth}
      \minipage[c][0.66\textheight][s]{\columnwidth}
      \begin{itemize}
        \item<1-> A C function provided by the runtime (specific to native code)
        \item<2-> It scans the stack, between the stack pointer at the raising point up to the next trap, in order to collect the names of the detected function frames
          \onslide*<2>{
            \begin{itemize}
              \item And more information, like line number, column number...
            \end{itemize}
          }
        \item<3-> It uses the same technique and information as the Garbage Collector (GC) uses to scan the values live on the stack
          \onslide*<4>{
            \begin{itemize}
              \item \typename{frame\_descr} are at the core of stack unwinding
            \end{itemize}
          }
      \end{itemize}
      \vfill
      \mintocaml[firstline=9,lastline=13,highlightlines=12]{../src/backtrace/backtrace.ml}
      \endminipage
    \end{column}
    \begin{column}{0.5\textwidth}
      \onslide*<1>{\listStashBacktrace[highlightlines=91]{}}
      \onslide*<2>{\listStashBacktrace[highlightlines={91,117-118}]{}}
      \onslide*<3->{\listStashBacktrace[highlightlines={94,106,109-110}]{}}
    \end{column}
  \end{columns}
\end{frame}

\newcommand\listFrameDescr[1][]{
  \listocaml[firstline=111,lastline=124,#1]{../ocaml/asmcomp/emitaux.ml}{asmcomp/emitaux.ml:111}}
\newcommand\listDebugInfo[1][]{
  \listocaml[firstline=38,lastline=49,#1]{../ocaml/lambda/debuginfo.mli}{lambda/debuginfo.mli:38}}

\begin{frame}{Backtraces}{\typename{frame\_descr} and \typename{frame\_debuginfo} in a nutshell}
  \begin{columns}[c]
    \begin{column}{0.5\textwidth}
      \begin{itemize}
        \item<1-> For every return address in an OCaml program, there is a associated \typename{frame\_descr}
        \item<2-> A \typename{frame\_descr} specifies
        \begin{itemize}
          \item<2-> The size of the function stack frame
          \item<3-> Where live values are on the stack (so that the GC can mark them as live and prevent from being swept)
        \end{itemize}
        \item<4-> When compiled with debug information, \typename{frame\_descr} will be linked to a \typename{frame\_debuginfo}
        \begin{itemize}
          \item<5-> Contains information about the position in the source code (file name, line and column number)
        \end{itemize}
      \end{itemize}
    \end{column}
    \begin{column}{0.5\textwidth}
        \onslide*<1>{\listFrameDescr[highlightlines={118-122}]{}}
        \onslide*<2>{\listFrameDescr[highlightlines={120}]{}}
        \onslide*<3>{\listFrameDescr[highlightlines={121}]{}}
        \onslide*<4->{\listFrameDescr[highlightlines={122}]{}}
        \medskip
        \onslide*<1-3>{\listDebugInfo{}}
        \onslide*<4>{\listDebugInfo[highlightlines={38,49}]{}}
        \onslide*<5>{\listDebugInfo[highlightlines={39-46}]{}}
    \end{column}
  \end{columns}
\end{frame}

\begin{frame}{Backtraces}{Scanning the stack with \typename{frame\_descr}}
  \begin{columns}[c]
    \begin{column}{0.5\textwidth}
      \begin{itemize}
        \item Given the return address of the code that raises the exception by calling \funcname{caml\_raise\_exn} (\funcarg{pc} argument of \funcname{caml\_stash\_backtrace})
        \item And the position of the stack pointer (\funcarg{sp} argument of \funcname{caml\_stash\_backtrace})
        \item The frame size given by \typename{frame\_descr}.\funcarg{frame\_size}, allows to find the end of the previous frame
        \item And the return address of the caller of this function
        \bigskip
        \onslide<3->{
        \item Note that traps (here the reddish \funcname{ExnA}, \funcname{ExnB} and \funcname{ExnC} blocks) belong to the frame that installed them, and so the frame size changes when traps are removed
        }
      \end{itemize}
    \end{column}
    \begin{column}{0.5\textwidth}
      \raggedleft
      \onslide*<1>{\providecommand\step{2}%
% Backtraces
%
\subsection{One advantage of raising with debugging information}
\frameSubsection{}{}

\begin{frame}{Backtraces}{Debugging information \& source code}
  \begin{columns}[c]
    \begin{column}{0.5\textwidth}
      \begin{itemize}
        \item Raising an exception is fun and games, but how do we know where the exception originated? $\rightarrow$ With the backtrace associated with the exception!
        \item In order to get a backtrace from the point where an exception was raised, it's necessary to compile with debugging information enabled (\funcarg{-g} flag for the compiler)
        \item Same example as in the `Nested handlers' section
      \end{itemize}
    \end{column}
    \begin{column}{0.5\textwidth}
      \listocaml[firstline=9,lastline=34]{../src/backtrace/backtrace.ml}{backtrace/backtrace.ml}
    \end{column}
  \end{columns}
\end{frame}

\begin{frame}{Backtraces}{CMM and disassembly of \funcname{definitely\_raise}}
  \begin{columns}[c]
    \begin{column}{0.5\textwidth}
      \minipage[c][0.66\textheight][s]{\columnwidth}
      \begin{itemize}
        \item<1-> An effect of compiling with debugging information (\funcarg{-g}): the CMM no longer uses \funcname{raise\_notrace} to raise the exception but \funcname{raise} instead
        \item<2-> Disassembly shows that CMM \funcname{raise} is actually a call to \funcname{caml\_raise\_exn}, a function in assembler provided by the runtime
      \end{itemize}
      \vfill
      \mintocaml[firstline=9,lastline=13,highlightlines=12]{../src/backtrace/backtrace.ml}
      \endminipage
    \end{column}
    \begin{column}{0.5\textwidth}
      \onslide*<1>{
        \mintcmm[highlightlines=5]{../src/backtrace/camlBacktrace\_\_definitely\_raise\_347.cmm}
      }
      \onslide*<2>{
        \mintobjdump[firstline=2,highlightlines=14]{../src/backtrace/camlBacktrace\_\_definitely\_raise\_347.objdump}
      }
    \end{column}
  \end{columns}
\end{frame}

\begin{frame}{Backtraces}{Introducing \funcname{caml\_raise\_exn}}
  \begin{columns}[c]
    \begin{column}{0.5\textwidth}
      \minipage[c][0.66\textheight][s]{\columnwidth}
      \onslide*<1>{
        \begin{itemize}
          \item Raising the exception is performed using \funcname{caml\_raise\_exn} which is
            \begin{itemize}
              \item An assembly function provided by the runtime
              \item Architecture specific
            \end{itemize}
          \item Let's see what it does differently from \funcname{raise\_notrace}\dots
          \item We are about to raise \typename{ExnC} with \funcname{caml\_raise\_exn} from \funcname{definitely\_raise}
        \end{itemize}
      }
      \onslide*<2>{
        \mintobjdump[firstline=2,highlightlines=14]{../src/backtrace/camlBacktrace\_\_definitely\_raise\_347.objdump}
      }
      \vfill
      \mintocaml[firstline=9,lastline=13,highlightlines=12]{../src/backtrace/backtrace.ml}
      \endminipage
    \end{column}
    \begin{column}{0.5\textwidth}
      \centering
      \providecommand\step{1}\input{diagrams/backtrace/backtrace.tex}
    \end{column}
  \end{columns}
\end{frame}

\begin{frame}{Backtraces}{But before that, we need more fields from \typename{caml\_domain\_state}}
  \begin{columns}
    \begin{column}{0.5\textwidth}
      \begin{itemize}
        \item Remember \typename{caml\_domain\_state} the per-domain runtime state?
        \item Some other members are of interest to us now\dots
        \item \localname{backtrace\_active} is used to know if the runtime should collect backtraces
        \item \localname{backtrace\_active} can be set by
          \begin{itemize}
            \item The \funcarg{b} flag in the \funcname{OCAMLRUNPARAM} environment variable
            \item \mintinline{ocaml}{Printexc.record_backtrace : bool -> unit} \footnotemark
          \end{itemize}
      \end{itemize}
    \end{column}
    \begin{column}{0.5\textwidth}
      \begin{listing}
        \mintc[highlightlines={9-11}]{src/ocaml/domain\_state\_backtrace.h}
        \caption{runtime/caml/domain\_state.h}
      \end{listing}
    \end{column}
  \end{columns}
  \footnotetext[1]{https://v2.ocaml.org/api/Printexc.html}
\end{frame}

\newcommand\listCamlRaiseExn[1][]{
  \listasm[firstline=702,lastline=725,#1]{../ocaml/runtime/amd64.S}{runtime/amd64.S:702}}

\begin{frame}{Backtraces}{Walk through \funcname{caml\_raise\_exn}}
  \begin{columns}[T]
    \begin{column}{0.5\textwidth}
      \begin{itemize}
        \item<1-> First checks whether exceptions backtrace should be recorded
        \item<1-> By checking the \funcarg{domain\_state->backtrace\_active} value
        \item<2-> If not, acts as \funcname{raise\_notrace} with \funcname{RESTORE\_EXN\_HANDLER\_OCAML}
          \begin{itemize}
            \item Set \regname{RSP} to \localname{domain\_state->exn\_handler} value
            \item Restore \localname{domain\_state->exn\_handler} from trap
            \item Jump with \funcname{ret} (same effect as using \funcname{pop} and \funcname{jmp})
          \end{itemize}
        \item<3-> Otherwise, switches to the C stack to be able to execute C code
        \item<4-> Calls \funcname{caml\_stash\_backtrace}, a C function provided by the runtime\ignorespaces
          \onslide<6->{, with
            \begin{itemize}
              \item \funcarg{exn}: exception value
              \item \funcarg{pc}: code address of the raising point
              \item \funcarg{sp}: stack address of the raising function
              \item \funcarg{trapsp}: stack address of the next handler
            \end{itemize}
          }
      \end{itemize}
      \bigskip
      \mintocaml[firstline=9,lastline=13,highlightlines=12]{../src/backtrace/backtrace.ml}
    \end{column}
    \begin{column}{0.5\textwidth}
      \raggedleft
      \onslide*<1,3-4>{
        \mintc[firstline=190,lastline=192]{../ocaml/runtime/amd64.S}
        \mintc[firstline=234,lastline=237]{../ocaml/runtime/amd64.S}
      }
      \onslide*<2>{
        \mintc[firstline=190,lastline=192,highlightlines={191-192}]{../ocaml/runtime/amd64.S}
        \mintc[firstline=234,lastline=237,highlightlines={235-237}]{../ocaml/runtime/amd64.S}
      }
      \onslide*<1>{\listCamlRaiseExn[highlightlines=705]{}}%
      \onslide*<2>{\listCamlRaiseExn[highlightlines={707-708}]{}}%
      \onslide*<3>{\listCamlRaiseExn[highlightlines={712-714}]{}}%
      \onslide*<4>{\listCamlRaiseExn[highlightlines={715-720}]{}}%
      \onslide*<5>{\providecommand\step{1}\input{diagrams/backtrace/backtrace.tex}}%
      \onslide*<6>{\providecommand\step{2}\input{diagrams/backtrace/backtrace.tex}}%
    \end{column}
  \end{columns}
\end{frame}

\newcommand\listStashBacktrace[1][]{
  \listc[firstline=91,lastline=120,#1]{../ocaml/runtime/backtrace\_nat.c}{runtime/backtrace\_nat.c:91}}

\begin{frame}{Backtraces}{Introducing \funcname{caml\_stash\_backtrace}}
  \begin{columns}[c]
    \begin{column}{0.5\textwidth}
      \minipage[c][0.66\textheight][s]{\columnwidth}
      \begin{itemize}
        \item<1-> A C function provided by the runtime (specific to native code)
        \item<2-> It scans the stack, between the stack pointer at the raising point up to the next trap, in order to collect the names of the detected function frames
          \onslide*<2>{
            \begin{itemize}
              \item And more information, like line number, column number...
            \end{itemize}
          }
        \item<3-> It uses the same technique and information as the Garbage Collector (GC) uses to scan the values live on the stack
          \onslide*<4>{
            \begin{itemize}
              \item \typename{frame\_descr} are at the core of stack unwinding
            \end{itemize}
          }
      \end{itemize}
      \vfill
      \mintocaml[firstline=9,lastline=13,highlightlines=12]{../src/backtrace/backtrace.ml}
      \endminipage
    \end{column}
    \begin{column}{0.5\textwidth}
      \onslide*<1>{\listStashBacktrace[highlightlines=91]{}}
      \onslide*<2>{\listStashBacktrace[highlightlines={91,117-118}]{}}
      \onslide*<3->{\listStashBacktrace[highlightlines={94,106,109-110}]{}}
    \end{column}
  \end{columns}
\end{frame}

\newcommand\listFrameDescr[1][]{
  \listocaml[firstline=111,lastline=124,#1]{../ocaml/asmcomp/emitaux.ml}{asmcomp/emitaux.ml:111}}
\newcommand\listDebugInfo[1][]{
  \listocaml[firstline=38,lastline=49,#1]{../ocaml/lambda/debuginfo.mli}{lambda/debuginfo.mli:38}}

\begin{frame}{Backtraces}{\typename{frame\_descr} and \typename{frame\_debuginfo} in a nutshell}
  \begin{columns}[c]
    \begin{column}{0.5\textwidth}
      \begin{itemize}
        \item<1-> For every return address in an OCaml program, there is a associated \typename{frame\_descr}
        \item<2-> A \typename{frame\_descr} specifies
        \begin{itemize}
          \item<2-> The size of the function stack frame
          \item<3-> Where live values are on the stack (so that the GC can mark them as live and prevent from being swept)
        \end{itemize}
        \item<4-> When compiled with debug information, \typename{frame\_descr} will be linked to a \typename{frame\_debuginfo}
        \begin{itemize}
          \item<5-> Contains information about the position in the source code (file name, line and column number)
        \end{itemize}
      \end{itemize}
    \end{column}
    \begin{column}{0.5\textwidth}
        \onslide*<1>{\listFrameDescr[highlightlines={118-122}]{}}
        \onslide*<2>{\listFrameDescr[highlightlines={120}]{}}
        \onslide*<3>{\listFrameDescr[highlightlines={121}]{}}
        \onslide*<4->{\listFrameDescr[highlightlines={122}]{}}
        \medskip
        \onslide*<1-3>{\listDebugInfo{}}
        \onslide*<4>{\listDebugInfo[highlightlines={38,49}]{}}
        \onslide*<5>{\listDebugInfo[highlightlines={39-46}]{}}
    \end{column}
  \end{columns}
\end{frame}

\begin{frame}{Backtraces}{Scanning the stack with \typename{frame\_descr}}
  \begin{columns}[c]
    \begin{column}{0.5\textwidth}
      \begin{itemize}
        \item Given the return address of the code that raises the exception by calling \funcname{caml\_raise\_exn} (\funcarg{pc} argument of \funcname{caml\_stash\_backtrace})
        \item And the position of the stack pointer (\funcarg{sp} argument of \funcname{caml\_stash\_backtrace})
        \item The frame size given by \typename{frame\_descr}.\funcarg{frame\_size}, allows to find the end of the previous frame
        \item And the return address of the caller of this function
        \bigskip
        \onslide<3->{
        \item Note that traps (here the reddish \funcname{ExnA}, \funcname{ExnB} and \funcname{ExnC} blocks) belong to the frame that installed them, and so the frame size changes when traps are removed
        }
      \end{itemize}
    \end{column}
    \begin{column}{0.5\textwidth}
      \raggedleft
      \onslide*<1>{\providecommand\step{2}\input{diagrams/backtrace/backtrace.tex}}%
      \onslide*<2>{\providecommand\step{1}\input{diagrams/backtrace/scanstack.tex}}%
      \onslide*<3>{\providecommand\step{2}\input{diagrams/backtrace/scanstack.tex}}%
      \onslide*<4>{\providecommand\step{3}\input{diagrams/backtrace/scanstack.tex}}%
      \onslide*<5>{\providecommand\step{4}\input{diagrams/backtrace/scanstack.tex}}%
      \onslide*<6>{\providecommand\step{5}\input{diagrams/backtrace/scanstack.tex}}%
    \end{column}
  \end{columns}
\end{frame}

% Duplicate of previous-previous slide
\begin{frame}{Backtraces}{Continuing on \funcname{caml\_stash\_backtrace}}
  \begin{columns}[c]
    \begin{column}{0.5\textwidth}
      \minipage[c][0.75\textheight][s]{\columnwidth}
      \begin{itemize}
          \color{gray}
        \item A C function provided by the runtime (specific to native code)
        \item It scans the stack, between the stack pointer at the raising point up to the next trap, in order to collect the names of the detected function frames
        \item It uses the same technique and information as the Garbage Collector (GC) uses to scan the values live on the stack
      \end{itemize}
      \begin{itemize}
        \item For each function frame detected, store a pointer to the \typename{frame\_descr} in the \localname{domain\_state->backtrace\_buffer} array, effectively constructing the backtrace
      \end{itemize}
      \onslide<2->{
        \begin{itemize}
            \smallskip
          \item Constructed backtrace:
            \funcname{definitely\_raise},
            \onslide<3->{\funcname{probably\_raise}}
        \end{itemize}
      }
      \vfill
      \mintocaml[firstline=9,lastline=13,highlightlines=12]{../src/backtrace/backtrace.ml}
      \endminipage
    \end{column}
    \begin{column}{0.5\textwidth}
      \raggedleft
      \onslide*<1>{\listStashBacktrace[highlightlines={114-115}]{}}
      \onslide*<2>{\providecommand\step{3}\input{diagrams/backtrace/backtrace.tex}}%
      \onslide*<3>{\providecommand\step{4}\input{diagrams/backtrace/backtrace.tex}}%
    \end{column}
  \end{columns}
\end{frame}

\begin{frame}{Backtraces}{Back to \funcname{caml\_raise\_exn}}
  \begin{columns}[c]
    \begin{column}{0.5\textwidth}
      \minipage[c][0.66\textheight][s]{\columnwidth}
      \begin{itemize}\color{gray}
        \item Checks wheteher exceptions backtrace should be recorded, by checking the \funcarg{domain\_state->backtrace\_active} value
        \item Switches to the C stack to be able to execute C code
        \item Calls \funcname{caml\_stash\_backtrace}, to collect the backtrace up to the next trap
      \end{itemize}
      \onslide*<2->{
        \begin{itemize}
          \item<2-> Executes the exception handler in \funcname{probably\_raise} with \funcname{RESTORE\_EXN\_HANDLER\_OCAML}
            \begin{itemize}
              \item Set \regname{RSP} to \localname{domain\_state->exn\_handler} value
              \item Restore \localname{domain\_state->exn\_handler} from trap
              \item Jump to the handler code with \funcname{ret}
            \end{itemize}
        \end{itemize}
      }
      \vfill
      \onslide*<1>{\mintocaml[firstline=9,lastline=13,highlightlines=12]{../src/backtrace/backtrace.ml}}
      \onslide*<2->{\mintocaml[firstline=15,lastline=21,highlightlines={19-21}]{../src/backtrace/backtrace.ml}}
      \endminipage
    \end{column}
    \begin{column}{0.5\textwidth}
      \centering
      \onslide*<1>{
        \mintc[firstline=190,lastline=192]{../ocaml/runtime/amd64.S}
        \mintc[firstline=234,lastline=237]{../ocaml/runtime/amd64.S}
      }
      \onslide*<2>{
        \mintc[firstline=190,lastline=192,highlightlines={191-192}]{../ocaml/runtime/amd64.S}
        \mintc[firstline=234,lastline=237,highlightlines={235-237}]{../ocaml/runtime/amd64.S}
      }
      \onslide*<1>{\listCamlRaiseExn[highlightlines=720]{}}
      \onslide*<2>{\listCamlRaiseExn[highlightlines=722]{}}
      \onslide*<3->{\providecommand\step{5}\input{diagrams/backtrace/backtrace.tex}}
    \end{column}
  \end{columns}
\end{frame}

\begin{frame}{Backtraces}{\funcname{caml\_probably\_raise}'s handler doesn't catch \typename{ExnC}}
  \begin{columns}[c]
    \begin{column}{0.45\textwidth}
      \minipage[c][0.75\textheight][s]{\columnwidth}
      \begin{itemize}
        \item<1-> \funcname{match \dots with}\footnotemark pattern matching can also match exceptions
        \item<1-> The CMM of \funcname{probably\_raise} shows that this \funcname{match \dots with} is implemented with a pattern matching and a \funcname{try \dots with}
        \item<1-> But this one only matches on \typename{ExnA} exception
        \item<2-> Since the pattern matching fails to catch the exception, \funcname{reraise} is called with our current \typename{ExnC} exception
        \item<3-> Disassembly shows that \funcname{reraise} is actually a call to \funcname{caml\_reraise\_exn}, which is also an assembly function provided by the runtime
      \end{itemize}
      \vfill
      \mintocaml[firstline=15,lastline=21,highlightlines={18-21}]{../src/backtrace/backtrace.ml}
      \endminipage
    \end{column}
    \begin{column}{0.55\textwidth}
      \centering
      \onslide*<1>{\mintcmm[highlightlines={10-18}]{../src/backtrace/camlBacktrace\_\_probably\_raise\_351.cmm}}
      \onslide*<2>{\listcmm[highlightlines=18]{../src/backtrace/camlBacktrace\_\_probably\_raise_351.cmm}{CMM of probably\_raise}}
      \onslide*<3>{\mintobjdump[firstline=5,lastline=35,highlightlines=31]{../src/backtrace/camlBacktrace\_\_probably\_raise_351.objdump}}
    \end{column}
  \end{columns}
  \only<1>{\footnotetext[1]{https://blog.janestreet.com/pattern-matching-and-exception-handling-unite/}}
\end{frame}

\newcommand\listCamlReraiseExn[1][]{
  \listasm[firstline=727,lastline=734,#1]{../ocaml/runtime/amd64.S}{runtime/amd64.S:727}}

\begin{frame}{Backtraces}{Introducing \funcname{caml\_reraise\_exn}}
  \begin{columns}[c]
    \begin{column}{0.5\textwidth}
      \minipage[c][0.66\textheight][s]{\columnwidth}
      \begin{itemize}
        \item<1-> The exception have to be reraised to the next handler
        \item<2-> First, checks if the backtrace should be collected with \funcarg{domain\_state->backtrace\_active}
        \only<3>{
        \begin{itemize}
            \item If not, behaves like a \funcname{raise\_notrace} with \funcname{RESTORE\_EXN\_HANDLER\_OCAML}
        \end{itemize}
        }
        \item<4-> Jumps into \funcname{caml\_raise\_exn}
        \item<5-> Calls \funcname{caml\_stash\_backtrace} again, to scan the next part of the stack: from the trap just pop-ed to the next trap
      \end{itemize}
      \vfill
      \mintocaml[firstline=15,lastline=21,highlightlines={18-21}]{../src/backtrace/backtrace.ml}
      \endminipage
    \end{column}
    \begin{column}{0.5\textwidth}
      \raggedleft
      \onslide*<1>{\listCamlReraiseExn{}}
      \onslide*<2>{\listCamlReraiseExn[highlightlines=729]{}}
      \onslide*<3>{
        \mintc[firstline=190,lastline=192,highlightlines={191-192}]{../ocaml/runtime/amd64.S}
        \mintc[firstline=234,lastline=237,highlightlines={235-237}]{../ocaml/runtime/amd64.S}
        \medskip
        \listCamlReraiseExn[highlightlines=731]{}
      }
      \onslide*<4-5>{
        % TODO refactor with \listCamlReraiseExn
        \mintasm[firstline=727,lastline=734,highlightlines=730]{../ocaml/runtime/amd64.S}
        \medskip
      }
      \onslide*<4>{\listCamlRaiseExn[highlightlines=711]{}}
      \onslide*<5>{\listCamlRaiseExn[highlightlines=720]{}}
    \end{column}
  \end{columns}
\end{frame}

\begin{frame}{Backtraces}{Backtrace collection continues}
  \begin{columns}[c]
    \begin{column}{0.55\textwidth}
      \minipage[c][0.75\textheight][s]{\columnwidth}
      \begin{itemize}
        \item<1-> \funcname{caml\_stash\_backtrace} scans the older part of the stack, up to the next trap
        \item<6-> Controls jumps to the nested exception handler in \funcname{maybe\_raise}, which matches only on \typename{ExnB}
        \item<7-> \funcname{caml\_reraise\_exn} is called again, backtrace collection resumes with \funcname{caml\_stash\_backtrace}
      \end{itemize}
      \medskip
      \begin{itemize}
        \item<2-> Constructed backtrace:
        \begin{itemize}
          \item <2-> \funcname{definitely\_raise}
          \item <2-> \funcname{probably\_raise}
          \item <3-> \funcname{probably\_raise} (re-raise)
          \item <4-> \funcname{maybe\_raise}
          \item <5-> \funcname{main}
          \item <8-> \funcname{main} (re-raise)
        \end{itemize}
      \end{itemize}
      \vfill
      \onslide*<1-5>{\mintocaml[firstline=15,lastline=21]{../src/backtrace/backtrace.ml}}
      \onslide*<6>{\mintocaml[firstline=28,lastline=34,highlightlines=31]{../src/backtrace/backtrace.ml}}
      \onslide*<7->{\mintocaml[firstline=28,lastline=34,highlightlines=33]{../src/backtrace/backtrace.ml}}
      \endminipage
    \end{column}
    \begin{column}{0.45\textwidth}
      \raggedleft
      \onslide*<1>{\providecommand\step{6}\input{diagrams/backtrace/backtrace.tex}}%
      \onslide*<2>{\providecommand\step{6}\input{diagrams/backtrace/backtrace.tex}}%
      \onslide*<3>{\providecommand\step{7}\input{diagrams/backtrace/backtrace.tex}}%
      \onslide*<4>{\providecommand\step{8}\input{diagrams/backtrace/backtrace.tex}}%
      \onslide*<5>{\providecommand\step{9}\input{diagrams/backtrace/backtrace.tex}}%
      \onslide*<6>{\providecommand\step{10}\input{diagrams/backtrace/backtrace.tex}}%
      \onslide*<7>{\providecommand\step{11}\input{diagrams/backtrace/backtrace.tex}}%
      \onslide*<8>{\providecommand\step{12}\input{diagrams/backtrace/backtrace.tex}}%
      \onslide*<9>{\providecommand\step{13}\input{diagrams/backtrace/backtrace.tex}}%
    \end{column}
  \end{columns}
\end{frame}

\begin{frame}{Backtraces}{Printing the backtrace}
  \begin{columns}[c]
    \begin{column}{0.45\textwidth}
      \minipage[c][0.66\textheight][s]{\columnwidth}
      \begin{itemize}
        \item<1-> The exception backtrace can now be printed to \funcarg{stdout} with \funcname{Printexc.print_backtrace}
        \item<2-> First the exception backtrace is retrieved into an OCaml array with \funcname{get_raw_backtrace }
        \item<3-> Which is implemented as the \funcname{caml_get_exception_raw_backtrace} C function in the runtime
        \item<4-> And finally printed with \funcname{print_exception_backtrace}
      \end{itemize}
      \vfill
      \mintocaml[firstline=28,lastline=34,highlightlines=33]{../src/backtrace/backtrace.ml}
      \endminipage
    \end{column}
    \begin{column}{0.55\textwidth}
      \onslide*<1,2>{
        \mintocaml[firstline=100,lastline=101]{../ocaml/stdlib/printexc.ml}
      }
      \only<1>{
        \listocaml[firstline=170,lastline=172]{../ocaml/stdlib/printexc.ml}{stdlib/printexc.ml:170}
      }
      \only<2>{
        \listocaml[firstline=170,lastline=172,highlightlines=172]{../ocaml/stdlib/printexc.ml}{stdlib/printexc.ml:170}
      }
      \only<3>{
        \mintc[firstline=154,lastline=156]{../ocaml/runtime/backtrace.c}
        \listc[firstline=171,lastline=192,highlightlines={184-187}]{../ocaml/runtime/backtrace.c}{runtime/backtrace.c:154}
      }
      \only<4>{
        \listocaml[firstline=155,lastline=165]{../ocaml/stdlib/printexc.ml}{stdlib/printexc.ml:55}
      }
    \end{column}
  \end{columns}
\end{frame}


\subsection{The multiple kinds of raise}
\frameSubsection{}{}

\begin{frame}{Backtraces}{When backtraces are not needed}
  \begin{columns}[c]
    \begin{column}{0.45\textwidth}
      \minipage[c][0.66\textheight][s]{\columnwidth}
      \begin{itemize}
        \item<1-> What if the backtrace for an exception is never needed?
        \item<2-> It's possible to raise the exception explicitly with \funcname{raise\_notrace} to make raising as cheap as possible
        \item<3-> An example of use in the \funcname{dynlink} library of the OCaml compiler
      \end{itemize}
      \vfill
      \onslide<3->{
      \listocaml[firstline=205,lastline=209,highlightlines=207]{../ocaml/otherlibs/dynlink/dynlink\_compilerlibs/misc.ml}{otherlibs/dynlink/dynlink\_compilerlibs/misc.ml:205}
      }
      \endminipage
    \end{column}
    \begin{column}{0.55\textwidth}
    \only<4>{
      \mintcmm[highlightlines=3]{../src/notrace/camlNotrace\_\_fun_347.cmm}
      \listcmm{../src/notrace/camlNotrace\_\_all\_somes\_327.cmm}{all\_somes}
    }
    \onslide*<5->{
      \listobjdump[highlightlines={6-9}]{../src/notrace/camlNotrace\_\_fun_347.objdump}{anonymous function from \funcname{all\_somes}}
    }
    \end{column}
  \end{columns}
\end{frame}

\begin{frame}{Backtraces}{Summary table of the multiple kinds of raise}
  \begin{itemize}
    \item When debugging information is enabled and if the backtrace of an exception isn't needed, it's possible to raise with \funcname{raise\_notrace} for performance
    \item When debugging information is \emph{not} enabled, the compiler optimises \funcname{raise} calls to inline assembly (same effect as using \funcname{raise\_notrace})
  \end{itemize}
  \bigskip
  \centering
  \begin{tabular}{| l | c | c | c | c |}
    \hline
    \parbox{10em}{Debug information \\ (\funcarg{-g} flag)}  & \multicolumn{2}{|c|}{Disabled}                                     & \multicolumn{2}{|c|}{Enabled} \\
    \hline
    Raise kind                                               & \funcname{raise} & \funcname{raise\_notrace}                       & \funcname{raise} & \funcname{raise\_notrace} \\
    \hline
    Compilation                                              & \parbox{8em}{inline assembly\\ (optimization)} & inline assembly   & \funcname{caml\_raise\_exn} & inline assembly \\
    \hline
  \end{tabular}
\end{frame}


%
% Takeaway #5
%
\subsection*{Takeaway \#5}
\frameSubsectionTakeaway{}
\againframe<5>{takeaway}
}%
      \onslide*<2>{\providecommand\step{1}\input{diagrams/backtrace/scanstack.tex}}%
      \onslide*<3>{\providecommand\step{2}\input{diagrams/backtrace/scanstack.tex}}%
      \onslide*<4>{\providecommand\step{3}\input{diagrams/backtrace/scanstack.tex}}%
      \onslide*<5>{\providecommand\step{4}\input{diagrams/backtrace/scanstack.tex}}%
      \onslide*<6>{\providecommand\step{5}\input{diagrams/backtrace/scanstack.tex}}%
    \end{column}
  \end{columns}
\end{frame}

% Duplicate of previous-previous slide
\begin{frame}{Backtraces}{Continuing on \funcname{caml\_stash\_backtrace}}
  \begin{columns}[c]
    \begin{column}{0.5\textwidth}
      \minipage[c][0.75\textheight][s]{\columnwidth}
      \begin{itemize}
          \color{gray}
        \item A C function provided by the runtime (specific to native code)
        \item It scans the stack, between the stack pointer at the raising point up to the next trap, in order to collect the names of the detected function frames
        \item It uses the same technique and information as the Garbage Collector (GC) uses to scan the values live on the stack
      \end{itemize}
      \begin{itemize}
        \item For each function frame detected, store a pointer to the \typename{frame\_descr} in the \localname{domain\_state->backtrace\_buffer} array, effectively constructing the backtrace
      \end{itemize}
      \onslide<2->{
        \begin{itemize}
            \smallskip
          \item Constructed backtrace:
            \funcname{definitely\_raise},
            \onslide<3->{\funcname{probably\_raise}}
        \end{itemize}
      }
      \vfill
      \mintocaml[firstline=9,lastline=13,highlightlines=12]{../src/backtrace/backtrace.ml}
      \endminipage
    \end{column}
    \begin{column}{0.5\textwidth}
      \raggedleft
      \onslide*<1>{\listStashBacktrace[highlightlines={114-115}]{}}
      \onslide*<2>{\providecommand\step{3}%
% Backtraces
%
\subsection{One advantage of raising with debugging information}
\frameSubsection{}{}

\begin{frame}{Backtraces}{Debugging information \& source code}
  \begin{columns}[c]
    \begin{column}{0.5\textwidth}
      \begin{itemize}
        \item Raising an exception is fun and games, but how do we know where the exception originated? $\rightarrow$ With the backtrace associated with the exception!
        \item In order to get a backtrace from the point where an exception was raised, it's necessary to compile with debugging information enabled (\funcarg{-g} flag for the compiler)
        \item Same example as in the `Nested handlers' section
      \end{itemize}
    \end{column}
    \begin{column}{0.5\textwidth}
      \listocaml[firstline=9,lastline=34]{../src/backtrace/backtrace.ml}{backtrace/backtrace.ml}
    \end{column}
  \end{columns}
\end{frame}

\begin{frame}{Backtraces}{CMM and disassembly of \funcname{definitely\_raise}}
  \begin{columns}[c]
    \begin{column}{0.5\textwidth}
      \minipage[c][0.66\textheight][s]{\columnwidth}
      \begin{itemize}
        \item<1-> An effect of compiling with debugging information (\funcarg{-g}): the CMM no longer uses \funcname{raise\_notrace} to raise the exception but \funcname{raise} instead
        \item<2-> Disassembly shows that CMM \funcname{raise} is actually a call to \funcname{caml\_raise\_exn}, a function in assembler provided by the runtime
      \end{itemize}
      \vfill
      \mintocaml[firstline=9,lastline=13,highlightlines=12]{../src/backtrace/backtrace.ml}
      \endminipage
    \end{column}
    \begin{column}{0.5\textwidth}
      \onslide*<1>{
        \mintcmm[highlightlines=5]{../src/backtrace/camlBacktrace\_\_definitely\_raise\_347.cmm}
      }
      \onslide*<2>{
        \mintobjdump[firstline=2,highlightlines=14]{../src/backtrace/camlBacktrace\_\_definitely\_raise\_347.objdump}
      }
    \end{column}
  \end{columns}
\end{frame}

\begin{frame}{Backtraces}{Introducing \funcname{caml\_raise\_exn}}
  \begin{columns}[c]
    \begin{column}{0.5\textwidth}
      \minipage[c][0.66\textheight][s]{\columnwidth}
      \onslide*<1>{
        \begin{itemize}
          \item Raising the exception is performed using \funcname{caml\_raise\_exn} which is
            \begin{itemize}
              \item An assembly function provided by the runtime
              \item Architecture specific
            \end{itemize}
          \item Let's see what it does differently from \funcname{raise\_notrace}\dots
          \item We are about to raise \typename{ExnC} with \funcname{caml\_raise\_exn} from \funcname{definitely\_raise}
        \end{itemize}
      }
      \onslide*<2>{
        \mintobjdump[firstline=2,highlightlines=14]{../src/backtrace/camlBacktrace\_\_definitely\_raise\_347.objdump}
      }
      \vfill
      \mintocaml[firstline=9,lastline=13,highlightlines=12]{../src/backtrace/backtrace.ml}
      \endminipage
    \end{column}
    \begin{column}{0.5\textwidth}
      \centering
      \providecommand\step{1}\input{diagrams/backtrace/backtrace.tex}
    \end{column}
  \end{columns}
\end{frame}

\begin{frame}{Backtraces}{But before that, we need more fields from \typename{caml\_domain\_state}}
  \begin{columns}
    \begin{column}{0.5\textwidth}
      \begin{itemize}
        \item Remember \typename{caml\_domain\_state} the per-domain runtime state?
        \item Some other members are of interest to us now\dots
        \item \localname{backtrace\_active} is used to know if the runtime should collect backtraces
        \item \localname{backtrace\_active} can be set by
          \begin{itemize}
            \item The \funcarg{b} flag in the \funcname{OCAMLRUNPARAM} environment variable
            \item \mintinline{ocaml}{Printexc.record_backtrace : bool -> unit} \footnotemark
          \end{itemize}
      \end{itemize}
    \end{column}
    \begin{column}{0.5\textwidth}
      \begin{listing}
        \mintc[highlightlines={9-11}]{src/ocaml/domain\_state\_backtrace.h}
        \caption{runtime/caml/domain\_state.h}
      \end{listing}
    \end{column}
  \end{columns}
  \footnotetext[1]{https://v2.ocaml.org/api/Printexc.html}
\end{frame}

\newcommand\listCamlRaiseExn[1][]{
  \listasm[firstline=702,lastline=725,#1]{../ocaml/runtime/amd64.S}{runtime/amd64.S:702}}

\begin{frame}{Backtraces}{Walk through \funcname{caml\_raise\_exn}}
  \begin{columns}[T]
    \begin{column}{0.5\textwidth}
      \begin{itemize}
        \item<1-> First checks whether exceptions backtrace should be recorded
        \item<1-> By checking the \funcarg{domain\_state->backtrace\_active} value
        \item<2-> If not, acts as \funcname{raise\_notrace} with \funcname{RESTORE\_EXN\_HANDLER\_OCAML}
          \begin{itemize}
            \item Set \regname{RSP} to \localname{domain\_state->exn\_handler} value
            \item Restore \localname{domain\_state->exn\_handler} from trap
            \item Jump with \funcname{ret} (same effect as using \funcname{pop} and \funcname{jmp})
          \end{itemize}
        \item<3-> Otherwise, switches to the C stack to be able to execute C code
        \item<4-> Calls \funcname{caml\_stash\_backtrace}, a C function provided by the runtime\ignorespaces
          \onslide<6->{, with
            \begin{itemize}
              \item \funcarg{exn}: exception value
              \item \funcarg{pc}: code address of the raising point
              \item \funcarg{sp}: stack address of the raising function
              \item \funcarg{trapsp}: stack address of the next handler
            \end{itemize}
          }
      \end{itemize}
      \bigskip
      \mintocaml[firstline=9,lastline=13,highlightlines=12]{../src/backtrace/backtrace.ml}
    \end{column}
    \begin{column}{0.5\textwidth}
      \raggedleft
      \onslide*<1,3-4>{
        \mintc[firstline=190,lastline=192]{../ocaml/runtime/amd64.S}
        \mintc[firstline=234,lastline=237]{../ocaml/runtime/amd64.S}
      }
      \onslide*<2>{
        \mintc[firstline=190,lastline=192,highlightlines={191-192}]{../ocaml/runtime/amd64.S}
        \mintc[firstline=234,lastline=237,highlightlines={235-237}]{../ocaml/runtime/amd64.S}
      }
      \onslide*<1>{\listCamlRaiseExn[highlightlines=705]{}}%
      \onslide*<2>{\listCamlRaiseExn[highlightlines={707-708}]{}}%
      \onslide*<3>{\listCamlRaiseExn[highlightlines={712-714}]{}}%
      \onslide*<4>{\listCamlRaiseExn[highlightlines={715-720}]{}}%
      \onslide*<5>{\providecommand\step{1}\input{diagrams/backtrace/backtrace.tex}}%
      \onslide*<6>{\providecommand\step{2}\input{diagrams/backtrace/backtrace.tex}}%
    \end{column}
  \end{columns}
\end{frame}

\newcommand\listStashBacktrace[1][]{
  \listc[firstline=91,lastline=120,#1]{../ocaml/runtime/backtrace\_nat.c}{runtime/backtrace\_nat.c:91}}

\begin{frame}{Backtraces}{Introducing \funcname{caml\_stash\_backtrace}}
  \begin{columns}[c]
    \begin{column}{0.5\textwidth}
      \minipage[c][0.66\textheight][s]{\columnwidth}
      \begin{itemize}
        \item<1-> A C function provided by the runtime (specific to native code)
        \item<2-> It scans the stack, between the stack pointer at the raising point up to the next trap, in order to collect the names of the detected function frames
          \onslide*<2>{
            \begin{itemize}
              \item And more information, like line number, column number...
            \end{itemize}
          }
        \item<3-> It uses the same technique and information as the Garbage Collector (GC) uses to scan the values live on the stack
          \onslide*<4>{
            \begin{itemize}
              \item \typename{frame\_descr} are at the core of stack unwinding
            \end{itemize}
          }
      \end{itemize}
      \vfill
      \mintocaml[firstline=9,lastline=13,highlightlines=12]{../src/backtrace/backtrace.ml}
      \endminipage
    \end{column}
    \begin{column}{0.5\textwidth}
      \onslide*<1>{\listStashBacktrace[highlightlines=91]{}}
      \onslide*<2>{\listStashBacktrace[highlightlines={91,117-118}]{}}
      \onslide*<3->{\listStashBacktrace[highlightlines={94,106,109-110}]{}}
    \end{column}
  \end{columns}
\end{frame}

\newcommand\listFrameDescr[1][]{
  \listocaml[firstline=111,lastline=124,#1]{../ocaml/asmcomp/emitaux.ml}{asmcomp/emitaux.ml:111}}
\newcommand\listDebugInfo[1][]{
  \listocaml[firstline=38,lastline=49,#1]{../ocaml/lambda/debuginfo.mli}{lambda/debuginfo.mli:38}}

\begin{frame}{Backtraces}{\typename{frame\_descr} and \typename{frame\_debuginfo} in a nutshell}
  \begin{columns}[c]
    \begin{column}{0.5\textwidth}
      \begin{itemize}
        \item<1-> For every return address in an OCaml program, there is a associated \typename{frame\_descr}
        \item<2-> A \typename{frame\_descr} specifies
        \begin{itemize}
          \item<2-> The size of the function stack frame
          \item<3-> Where live values are on the stack (so that the GC can mark them as live and prevent from being swept)
        \end{itemize}
        \item<4-> When compiled with debug information, \typename{frame\_descr} will be linked to a \typename{frame\_debuginfo}
        \begin{itemize}
          \item<5-> Contains information about the position in the source code (file name, line and column number)
        \end{itemize}
      \end{itemize}
    \end{column}
    \begin{column}{0.5\textwidth}
        \onslide*<1>{\listFrameDescr[highlightlines={118-122}]{}}
        \onslide*<2>{\listFrameDescr[highlightlines={120}]{}}
        \onslide*<3>{\listFrameDescr[highlightlines={121}]{}}
        \onslide*<4->{\listFrameDescr[highlightlines={122}]{}}
        \medskip
        \onslide*<1-3>{\listDebugInfo{}}
        \onslide*<4>{\listDebugInfo[highlightlines={38,49}]{}}
        \onslide*<5>{\listDebugInfo[highlightlines={39-46}]{}}
    \end{column}
  \end{columns}
\end{frame}

\begin{frame}{Backtraces}{Scanning the stack with \typename{frame\_descr}}
  \begin{columns}[c]
    \begin{column}{0.5\textwidth}
      \begin{itemize}
        \item Given the return address of the code that raises the exception by calling \funcname{caml\_raise\_exn} (\funcarg{pc} argument of \funcname{caml\_stash\_backtrace})
        \item And the position of the stack pointer (\funcarg{sp} argument of \funcname{caml\_stash\_backtrace})
        \item The frame size given by \typename{frame\_descr}.\funcarg{frame\_size}, allows to find the end of the previous frame
        \item And the return address of the caller of this function
        \bigskip
        \onslide<3->{
        \item Note that traps (here the reddish \funcname{ExnA}, \funcname{ExnB} and \funcname{ExnC} blocks) belong to the frame that installed them, and so the frame size changes when traps are removed
        }
      \end{itemize}
    \end{column}
    \begin{column}{0.5\textwidth}
      \raggedleft
      \onslide*<1>{\providecommand\step{2}\input{diagrams/backtrace/backtrace.tex}}%
      \onslide*<2>{\providecommand\step{1}\input{diagrams/backtrace/scanstack.tex}}%
      \onslide*<3>{\providecommand\step{2}\input{diagrams/backtrace/scanstack.tex}}%
      \onslide*<4>{\providecommand\step{3}\input{diagrams/backtrace/scanstack.tex}}%
      \onslide*<5>{\providecommand\step{4}\input{diagrams/backtrace/scanstack.tex}}%
      \onslide*<6>{\providecommand\step{5}\input{diagrams/backtrace/scanstack.tex}}%
    \end{column}
  \end{columns}
\end{frame}

% Duplicate of previous-previous slide
\begin{frame}{Backtraces}{Continuing on \funcname{caml\_stash\_backtrace}}
  \begin{columns}[c]
    \begin{column}{0.5\textwidth}
      \minipage[c][0.75\textheight][s]{\columnwidth}
      \begin{itemize}
          \color{gray}
        \item A C function provided by the runtime (specific to native code)
        \item It scans the stack, between the stack pointer at the raising point up to the next trap, in order to collect the names of the detected function frames
        \item It uses the same technique and information as the Garbage Collector (GC) uses to scan the values live on the stack
      \end{itemize}
      \begin{itemize}
        \item For each function frame detected, store a pointer to the \typename{frame\_descr} in the \localname{domain\_state->backtrace\_buffer} array, effectively constructing the backtrace
      \end{itemize}
      \onslide<2->{
        \begin{itemize}
            \smallskip
          \item Constructed backtrace:
            \funcname{definitely\_raise},
            \onslide<3->{\funcname{probably\_raise}}
        \end{itemize}
      }
      \vfill
      \mintocaml[firstline=9,lastline=13,highlightlines=12]{../src/backtrace/backtrace.ml}
      \endminipage
    \end{column}
    \begin{column}{0.5\textwidth}
      \raggedleft
      \onslide*<1>{\listStashBacktrace[highlightlines={114-115}]{}}
      \onslide*<2>{\providecommand\step{3}\input{diagrams/backtrace/backtrace.tex}}%
      \onslide*<3>{\providecommand\step{4}\input{diagrams/backtrace/backtrace.tex}}%
    \end{column}
  \end{columns}
\end{frame}

\begin{frame}{Backtraces}{Back to \funcname{caml\_raise\_exn}}
  \begin{columns}[c]
    \begin{column}{0.5\textwidth}
      \minipage[c][0.66\textheight][s]{\columnwidth}
      \begin{itemize}\color{gray}
        \item Checks wheteher exceptions backtrace should be recorded, by checking the \funcarg{domain\_state->backtrace\_active} value
        \item Switches to the C stack to be able to execute C code
        \item Calls \funcname{caml\_stash\_backtrace}, to collect the backtrace up to the next trap
      \end{itemize}
      \onslide*<2->{
        \begin{itemize}
          \item<2-> Executes the exception handler in \funcname{probably\_raise} with \funcname{RESTORE\_EXN\_HANDLER\_OCAML}
            \begin{itemize}
              \item Set \regname{RSP} to \localname{domain\_state->exn\_handler} value
              \item Restore \localname{domain\_state->exn\_handler} from trap
              \item Jump to the handler code with \funcname{ret}
            \end{itemize}
        \end{itemize}
      }
      \vfill
      \onslide*<1>{\mintocaml[firstline=9,lastline=13,highlightlines=12]{../src/backtrace/backtrace.ml}}
      \onslide*<2->{\mintocaml[firstline=15,lastline=21,highlightlines={19-21}]{../src/backtrace/backtrace.ml}}
      \endminipage
    \end{column}
    \begin{column}{0.5\textwidth}
      \centering
      \onslide*<1>{
        \mintc[firstline=190,lastline=192]{../ocaml/runtime/amd64.S}
        \mintc[firstline=234,lastline=237]{../ocaml/runtime/amd64.S}
      }
      \onslide*<2>{
        \mintc[firstline=190,lastline=192,highlightlines={191-192}]{../ocaml/runtime/amd64.S}
        \mintc[firstline=234,lastline=237,highlightlines={235-237}]{../ocaml/runtime/amd64.S}
      }
      \onslide*<1>{\listCamlRaiseExn[highlightlines=720]{}}
      \onslide*<2>{\listCamlRaiseExn[highlightlines=722]{}}
      \onslide*<3->{\providecommand\step{5}\input{diagrams/backtrace/backtrace.tex}}
    \end{column}
  \end{columns}
\end{frame}

\begin{frame}{Backtraces}{\funcname{caml\_probably\_raise}'s handler doesn't catch \typename{ExnC}}
  \begin{columns}[c]
    \begin{column}{0.45\textwidth}
      \minipage[c][0.75\textheight][s]{\columnwidth}
      \begin{itemize}
        \item<1-> \funcname{match \dots with}\footnotemark pattern matching can also match exceptions
        \item<1-> The CMM of \funcname{probably\_raise} shows that this \funcname{match \dots with} is implemented with a pattern matching and a \funcname{try \dots with}
        \item<1-> But this one only matches on \typename{ExnA} exception
        \item<2-> Since the pattern matching fails to catch the exception, \funcname{reraise} is called with our current \typename{ExnC} exception
        \item<3-> Disassembly shows that \funcname{reraise} is actually a call to \funcname{caml\_reraise\_exn}, which is also an assembly function provided by the runtime
      \end{itemize}
      \vfill
      \mintocaml[firstline=15,lastline=21,highlightlines={18-21}]{../src/backtrace/backtrace.ml}
      \endminipage
    \end{column}
    \begin{column}{0.55\textwidth}
      \centering
      \onslide*<1>{\mintcmm[highlightlines={10-18}]{../src/backtrace/camlBacktrace\_\_probably\_raise\_351.cmm}}
      \onslide*<2>{\listcmm[highlightlines=18]{../src/backtrace/camlBacktrace\_\_probably\_raise_351.cmm}{CMM of probably\_raise}}
      \onslide*<3>{\mintobjdump[firstline=5,lastline=35,highlightlines=31]{../src/backtrace/camlBacktrace\_\_probably\_raise_351.objdump}}
    \end{column}
  \end{columns}
  \only<1>{\footnotetext[1]{https://blog.janestreet.com/pattern-matching-and-exception-handling-unite/}}
\end{frame}

\newcommand\listCamlReraiseExn[1][]{
  \listasm[firstline=727,lastline=734,#1]{../ocaml/runtime/amd64.S}{runtime/amd64.S:727}}

\begin{frame}{Backtraces}{Introducing \funcname{caml\_reraise\_exn}}
  \begin{columns}[c]
    \begin{column}{0.5\textwidth}
      \minipage[c][0.66\textheight][s]{\columnwidth}
      \begin{itemize}
        \item<1-> The exception have to be reraised to the next handler
        \item<2-> First, checks if the backtrace should be collected with \funcarg{domain\_state->backtrace\_active}
        \only<3>{
        \begin{itemize}
            \item If not, behaves like a \funcname{raise\_notrace} with \funcname{RESTORE\_EXN\_HANDLER\_OCAML}
        \end{itemize}
        }
        \item<4-> Jumps into \funcname{caml\_raise\_exn}
        \item<5-> Calls \funcname{caml\_stash\_backtrace} again, to scan the next part of the stack: from the trap just pop-ed to the next trap
      \end{itemize}
      \vfill
      \mintocaml[firstline=15,lastline=21,highlightlines={18-21}]{../src/backtrace/backtrace.ml}
      \endminipage
    \end{column}
    \begin{column}{0.5\textwidth}
      \raggedleft
      \onslide*<1>{\listCamlReraiseExn{}}
      \onslide*<2>{\listCamlReraiseExn[highlightlines=729]{}}
      \onslide*<3>{
        \mintc[firstline=190,lastline=192,highlightlines={191-192}]{../ocaml/runtime/amd64.S}
        \mintc[firstline=234,lastline=237,highlightlines={235-237}]{../ocaml/runtime/amd64.S}
        \medskip
        \listCamlReraiseExn[highlightlines=731]{}
      }
      \onslide*<4-5>{
        % TODO refactor with \listCamlReraiseExn
        \mintasm[firstline=727,lastline=734,highlightlines=730]{../ocaml/runtime/amd64.S}
        \medskip
      }
      \onslide*<4>{\listCamlRaiseExn[highlightlines=711]{}}
      \onslide*<5>{\listCamlRaiseExn[highlightlines=720]{}}
    \end{column}
  \end{columns}
\end{frame}

\begin{frame}{Backtraces}{Backtrace collection continues}
  \begin{columns}[c]
    \begin{column}{0.55\textwidth}
      \minipage[c][0.75\textheight][s]{\columnwidth}
      \begin{itemize}
        \item<1-> \funcname{caml\_stash\_backtrace} scans the older part of the stack, up to the next trap
        \item<6-> Controls jumps to the nested exception handler in \funcname{maybe\_raise}, which matches only on \typename{ExnB}
        \item<7-> \funcname{caml\_reraise\_exn} is called again, backtrace collection resumes with \funcname{caml\_stash\_backtrace}
      \end{itemize}
      \medskip
      \begin{itemize}
        \item<2-> Constructed backtrace:
        \begin{itemize}
          \item <2-> \funcname{definitely\_raise}
          \item <2-> \funcname{probably\_raise}
          \item <3-> \funcname{probably\_raise} (re-raise)
          \item <4-> \funcname{maybe\_raise}
          \item <5-> \funcname{main}
          \item <8-> \funcname{main} (re-raise)
        \end{itemize}
      \end{itemize}
      \vfill
      \onslide*<1-5>{\mintocaml[firstline=15,lastline=21]{../src/backtrace/backtrace.ml}}
      \onslide*<6>{\mintocaml[firstline=28,lastline=34,highlightlines=31]{../src/backtrace/backtrace.ml}}
      \onslide*<7->{\mintocaml[firstline=28,lastline=34,highlightlines=33]{../src/backtrace/backtrace.ml}}
      \endminipage
    \end{column}
    \begin{column}{0.45\textwidth}
      \raggedleft
      \onslide*<1>{\providecommand\step{6}\input{diagrams/backtrace/backtrace.tex}}%
      \onslide*<2>{\providecommand\step{6}\input{diagrams/backtrace/backtrace.tex}}%
      \onslide*<3>{\providecommand\step{7}\input{diagrams/backtrace/backtrace.tex}}%
      \onslide*<4>{\providecommand\step{8}\input{diagrams/backtrace/backtrace.tex}}%
      \onslide*<5>{\providecommand\step{9}\input{diagrams/backtrace/backtrace.tex}}%
      \onslide*<6>{\providecommand\step{10}\input{diagrams/backtrace/backtrace.tex}}%
      \onslide*<7>{\providecommand\step{11}\input{diagrams/backtrace/backtrace.tex}}%
      \onslide*<8>{\providecommand\step{12}\input{diagrams/backtrace/backtrace.tex}}%
      \onslide*<9>{\providecommand\step{13}\input{diagrams/backtrace/backtrace.tex}}%
    \end{column}
  \end{columns}
\end{frame}

\begin{frame}{Backtraces}{Printing the backtrace}
  \begin{columns}[c]
    \begin{column}{0.45\textwidth}
      \minipage[c][0.66\textheight][s]{\columnwidth}
      \begin{itemize}
        \item<1-> The exception backtrace can now be printed to \funcarg{stdout} with \funcname{Printexc.print_backtrace}
        \item<2-> First the exception backtrace is retrieved into an OCaml array with \funcname{get_raw_backtrace }
        \item<3-> Which is implemented as the \funcname{caml_get_exception_raw_backtrace} C function in the runtime
        \item<4-> And finally printed with \funcname{print_exception_backtrace}
      \end{itemize}
      \vfill
      \mintocaml[firstline=28,lastline=34,highlightlines=33]{../src/backtrace/backtrace.ml}
      \endminipage
    \end{column}
    \begin{column}{0.55\textwidth}
      \onslide*<1,2>{
        \mintocaml[firstline=100,lastline=101]{../ocaml/stdlib/printexc.ml}
      }
      \only<1>{
        \listocaml[firstline=170,lastline=172]{../ocaml/stdlib/printexc.ml}{stdlib/printexc.ml:170}
      }
      \only<2>{
        \listocaml[firstline=170,lastline=172,highlightlines=172]{../ocaml/stdlib/printexc.ml}{stdlib/printexc.ml:170}
      }
      \only<3>{
        \mintc[firstline=154,lastline=156]{../ocaml/runtime/backtrace.c}
        \listc[firstline=171,lastline=192,highlightlines={184-187}]{../ocaml/runtime/backtrace.c}{runtime/backtrace.c:154}
      }
      \only<4>{
        \listocaml[firstline=155,lastline=165]{../ocaml/stdlib/printexc.ml}{stdlib/printexc.ml:55}
      }
    \end{column}
  \end{columns}
\end{frame}


\subsection{The multiple kinds of raise}
\frameSubsection{}{}

\begin{frame}{Backtraces}{When backtraces are not needed}
  \begin{columns}[c]
    \begin{column}{0.45\textwidth}
      \minipage[c][0.66\textheight][s]{\columnwidth}
      \begin{itemize}
        \item<1-> What if the backtrace for an exception is never needed?
        \item<2-> It's possible to raise the exception explicitly with \funcname{raise\_notrace} to make raising as cheap as possible
        \item<3-> An example of use in the \funcname{dynlink} library of the OCaml compiler
      \end{itemize}
      \vfill
      \onslide<3->{
      \listocaml[firstline=205,lastline=209,highlightlines=207]{../ocaml/otherlibs/dynlink/dynlink\_compilerlibs/misc.ml}{otherlibs/dynlink/dynlink\_compilerlibs/misc.ml:205}
      }
      \endminipage
    \end{column}
    \begin{column}{0.55\textwidth}
    \only<4>{
      \mintcmm[highlightlines=3]{../src/notrace/camlNotrace\_\_fun_347.cmm}
      \listcmm{../src/notrace/camlNotrace\_\_all\_somes\_327.cmm}{all\_somes}
    }
    \onslide*<5->{
      \listobjdump[highlightlines={6-9}]{../src/notrace/camlNotrace\_\_fun_347.objdump}{anonymous function from \funcname{all\_somes}}
    }
    \end{column}
  \end{columns}
\end{frame}

\begin{frame}{Backtraces}{Summary table of the multiple kinds of raise}
  \begin{itemize}
    \item When debugging information is enabled and if the backtrace of an exception isn't needed, it's possible to raise with \funcname{raise\_notrace} for performance
    \item When debugging information is \emph{not} enabled, the compiler optimises \funcname{raise} calls to inline assembly (same effect as using \funcname{raise\_notrace})
  \end{itemize}
  \bigskip
  \centering
  \begin{tabular}{| l | c | c | c | c |}
    \hline
    \parbox{10em}{Debug information \\ (\funcarg{-g} flag)}  & \multicolumn{2}{|c|}{Disabled}                                     & \multicolumn{2}{|c|}{Enabled} \\
    \hline
    Raise kind                                               & \funcname{raise} & \funcname{raise\_notrace}                       & \funcname{raise} & \funcname{raise\_notrace} \\
    \hline
    Compilation                                              & \parbox{8em}{inline assembly\\ (optimization)} & inline assembly   & \funcname{caml\_raise\_exn} & inline assembly \\
    \hline
  \end{tabular}
\end{frame}


%
% Takeaway #5
%
\subsection*{Takeaway \#5}
\frameSubsectionTakeaway{}
\againframe<5>{takeaway}
}%
      \onslide*<3>{\providecommand\step{4}%
% Backtraces
%
\subsection{One advantage of raising with debugging information}
\frameSubsection{}{}

\begin{frame}{Backtraces}{Debugging information \& source code}
  \begin{columns}[c]
    \begin{column}{0.5\textwidth}
      \begin{itemize}
        \item Raising an exception is fun and games, but how do we know where the exception originated? $\rightarrow$ With the backtrace associated with the exception!
        \item In order to get a backtrace from the point where an exception was raised, it's necessary to compile with debugging information enabled (\funcarg{-g} flag for the compiler)
        \item Same example as in the `Nested handlers' section
      \end{itemize}
    \end{column}
    \begin{column}{0.5\textwidth}
      \listocaml[firstline=9,lastline=34]{../src/backtrace/backtrace.ml}{backtrace/backtrace.ml}
    \end{column}
  \end{columns}
\end{frame}

\begin{frame}{Backtraces}{CMM and disassembly of \funcname{definitely\_raise}}
  \begin{columns}[c]
    \begin{column}{0.5\textwidth}
      \minipage[c][0.66\textheight][s]{\columnwidth}
      \begin{itemize}
        \item<1-> An effect of compiling with debugging information (\funcarg{-g}): the CMM no longer uses \funcname{raise\_notrace} to raise the exception but \funcname{raise} instead
        \item<2-> Disassembly shows that CMM \funcname{raise} is actually a call to \funcname{caml\_raise\_exn}, a function in assembler provided by the runtime
      \end{itemize}
      \vfill
      \mintocaml[firstline=9,lastline=13,highlightlines=12]{../src/backtrace/backtrace.ml}
      \endminipage
    \end{column}
    \begin{column}{0.5\textwidth}
      \onslide*<1>{
        \mintcmm[highlightlines=5]{../src/backtrace/camlBacktrace\_\_definitely\_raise\_347.cmm}
      }
      \onslide*<2>{
        \mintobjdump[firstline=2,highlightlines=14]{../src/backtrace/camlBacktrace\_\_definitely\_raise\_347.objdump}
      }
    \end{column}
  \end{columns}
\end{frame}

\begin{frame}{Backtraces}{Introducing \funcname{caml\_raise\_exn}}
  \begin{columns}[c]
    \begin{column}{0.5\textwidth}
      \minipage[c][0.66\textheight][s]{\columnwidth}
      \onslide*<1>{
        \begin{itemize}
          \item Raising the exception is performed using \funcname{caml\_raise\_exn} which is
            \begin{itemize}
              \item An assembly function provided by the runtime
              \item Architecture specific
            \end{itemize}
          \item Let's see what it does differently from \funcname{raise\_notrace}\dots
          \item We are about to raise \typename{ExnC} with \funcname{caml\_raise\_exn} from \funcname{definitely\_raise}
        \end{itemize}
      }
      \onslide*<2>{
        \mintobjdump[firstline=2,highlightlines=14]{../src/backtrace/camlBacktrace\_\_definitely\_raise\_347.objdump}
      }
      \vfill
      \mintocaml[firstline=9,lastline=13,highlightlines=12]{../src/backtrace/backtrace.ml}
      \endminipage
    \end{column}
    \begin{column}{0.5\textwidth}
      \centering
      \providecommand\step{1}\input{diagrams/backtrace/backtrace.tex}
    \end{column}
  \end{columns}
\end{frame}

\begin{frame}{Backtraces}{But before that, we need more fields from \typename{caml\_domain\_state}}
  \begin{columns}
    \begin{column}{0.5\textwidth}
      \begin{itemize}
        \item Remember \typename{caml\_domain\_state} the per-domain runtime state?
        \item Some other members are of interest to us now\dots
        \item \localname{backtrace\_active} is used to know if the runtime should collect backtraces
        \item \localname{backtrace\_active} can be set by
          \begin{itemize}
            \item The \funcarg{b} flag in the \funcname{OCAMLRUNPARAM} environment variable
            \item \mintinline{ocaml}{Printexc.record_backtrace : bool -> unit} \footnotemark
          \end{itemize}
      \end{itemize}
    \end{column}
    \begin{column}{0.5\textwidth}
      \begin{listing}
        \mintc[highlightlines={9-11}]{src/ocaml/domain\_state\_backtrace.h}
        \caption{runtime/caml/domain\_state.h}
      \end{listing}
    \end{column}
  \end{columns}
  \footnotetext[1]{https://v2.ocaml.org/api/Printexc.html}
\end{frame}

\newcommand\listCamlRaiseExn[1][]{
  \listasm[firstline=702,lastline=725,#1]{../ocaml/runtime/amd64.S}{runtime/amd64.S:702}}

\begin{frame}{Backtraces}{Walk through \funcname{caml\_raise\_exn}}
  \begin{columns}[T]
    \begin{column}{0.5\textwidth}
      \begin{itemize}
        \item<1-> First checks whether exceptions backtrace should be recorded
        \item<1-> By checking the \funcarg{domain\_state->backtrace\_active} value
        \item<2-> If not, acts as \funcname{raise\_notrace} with \funcname{RESTORE\_EXN\_HANDLER\_OCAML}
          \begin{itemize}
            \item Set \regname{RSP} to \localname{domain\_state->exn\_handler} value
            \item Restore \localname{domain\_state->exn\_handler} from trap
            \item Jump with \funcname{ret} (same effect as using \funcname{pop} and \funcname{jmp})
          \end{itemize}
        \item<3-> Otherwise, switches to the C stack to be able to execute C code
        \item<4-> Calls \funcname{caml\_stash\_backtrace}, a C function provided by the runtime\ignorespaces
          \onslide<6->{, with
            \begin{itemize}
              \item \funcarg{exn}: exception value
              \item \funcarg{pc}: code address of the raising point
              \item \funcarg{sp}: stack address of the raising function
              \item \funcarg{trapsp}: stack address of the next handler
            \end{itemize}
          }
      \end{itemize}
      \bigskip
      \mintocaml[firstline=9,lastline=13,highlightlines=12]{../src/backtrace/backtrace.ml}
    \end{column}
    \begin{column}{0.5\textwidth}
      \raggedleft
      \onslide*<1,3-4>{
        \mintc[firstline=190,lastline=192]{../ocaml/runtime/amd64.S}
        \mintc[firstline=234,lastline=237]{../ocaml/runtime/amd64.S}
      }
      \onslide*<2>{
        \mintc[firstline=190,lastline=192,highlightlines={191-192}]{../ocaml/runtime/amd64.S}
        \mintc[firstline=234,lastline=237,highlightlines={235-237}]{../ocaml/runtime/amd64.S}
      }
      \onslide*<1>{\listCamlRaiseExn[highlightlines=705]{}}%
      \onslide*<2>{\listCamlRaiseExn[highlightlines={707-708}]{}}%
      \onslide*<3>{\listCamlRaiseExn[highlightlines={712-714}]{}}%
      \onslide*<4>{\listCamlRaiseExn[highlightlines={715-720}]{}}%
      \onslide*<5>{\providecommand\step{1}\input{diagrams/backtrace/backtrace.tex}}%
      \onslide*<6>{\providecommand\step{2}\input{diagrams/backtrace/backtrace.tex}}%
    \end{column}
  \end{columns}
\end{frame}

\newcommand\listStashBacktrace[1][]{
  \listc[firstline=91,lastline=120,#1]{../ocaml/runtime/backtrace\_nat.c}{runtime/backtrace\_nat.c:91}}

\begin{frame}{Backtraces}{Introducing \funcname{caml\_stash\_backtrace}}
  \begin{columns}[c]
    \begin{column}{0.5\textwidth}
      \minipage[c][0.66\textheight][s]{\columnwidth}
      \begin{itemize}
        \item<1-> A C function provided by the runtime (specific to native code)
        \item<2-> It scans the stack, between the stack pointer at the raising point up to the next trap, in order to collect the names of the detected function frames
          \onslide*<2>{
            \begin{itemize}
              \item And more information, like line number, column number...
            \end{itemize}
          }
        \item<3-> It uses the same technique and information as the Garbage Collector (GC) uses to scan the values live on the stack
          \onslide*<4>{
            \begin{itemize}
              \item \typename{frame\_descr} are at the core of stack unwinding
            \end{itemize}
          }
      \end{itemize}
      \vfill
      \mintocaml[firstline=9,lastline=13,highlightlines=12]{../src/backtrace/backtrace.ml}
      \endminipage
    \end{column}
    \begin{column}{0.5\textwidth}
      \onslide*<1>{\listStashBacktrace[highlightlines=91]{}}
      \onslide*<2>{\listStashBacktrace[highlightlines={91,117-118}]{}}
      \onslide*<3->{\listStashBacktrace[highlightlines={94,106,109-110}]{}}
    \end{column}
  \end{columns}
\end{frame}

\newcommand\listFrameDescr[1][]{
  \listocaml[firstline=111,lastline=124,#1]{../ocaml/asmcomp/emitaux.ml}{asmcomp/emitaux.ml:111}}
\newcommand\listDebugInfo[1][]{
  \listocaml[firstline=38,lastline=49,#1]{../ocaml/lambda/debuginfo.mli}{lambda/debuginfo.mli:38}}

\begin{frame}{Backtraces}{\typename{frame\_descr} and \typename{frame\_debuginfo} in a nutshell}
  \begin{columns}[c]
    \begin{column}{0.5\textwidth}
      \begin{itemize}
        \item<1-> For every return address in an OCaml program, there is a associated \typename{frame\_descr}
        \item<2-> A \typename{frame\_descr} specifies
        \begin{itemize}
          \item<2-> The size of the function stack frame
          \item<3-> Where live values are on the stack (so that the GC can mark them as live and prevent from being swept)
        \end{itemize}
        \item<4-> When compiled with debug information, \typename{frame\_descr} will be linked to a \typename{frame\_debuginfo}
        \begin{itemize}
          \item<5-> Contains information about the position in the source code (file name, line and column number)
        \end{itemize}
      \end{itemize}
    \end{column}
    \begin{column}{0.5\textwidth}
        \onslide*<1>{\listFrameDescr[highlightlines={118-122}]{}}
        \onslide*<2>{\listFrameDescr[highlightlines={120}]{}}
        \onslide*<3>{\listFrameDescr[highlightlines={121}]{}}
        \onslide*<4->{\listFrameDescr[highlightlines={122}]{}}
        \medskip
        \onslide*<1-3>{\listDebugInfo{}}
        \onslide*<4>{\listDebugInfo[highlightlines={38,49}]{}}
        \onslide*<5>{\listDebugInfo[highlightlines={39-46}]{}}
    \end{column}
  \end{columns}
\end{frame}

\begin{frame}{Backtraces}{Scanning the stack with \typename{frame\_descr}}
  \begin{columns}[c]
    \begin{column}{0.5\textwidth}
      \begin{itemize}
        \item Given the return address of the code that raises the exception by calling \funcname{caml\_raise\_exn} (\funcarg{pc} argument of \funcname{caml\_stash\_backtrace})
        \item And the position of the stack pointer (\funcarg{sp} argument of \funcname{caml\_stash\_backtrace})
        \item The frame size given by \typename{frame\_descr}.\funcarg{frame\_size}, allows to find the end of the previous frame
        \item And the return address of the caller of this function
        \bigskip
        \onslide<3->{
        \item Note that traps (here the reddish \funcname{ExnA}, \funcname{ExnB} and \funcname{ExnC} blocks) belong to the frame that installed them, and so the frame size changes when traps are removed
        }
      \end{itemize}
    \end{column}
    \begin{column}{0.5\textwidth}
      \raggedleft
      \onslide*<1>{\providecommand\step{2}\input{diagrams/backtrace/backtrace.tex}}%
      \onslide*<2>{\providecommand\step{1}\input{diagrams/backtrace/scanstack.tex}}%
      \onslide*<3>{\providecommand\step{2}\input{diagrams/backtrace/scanstack.tex}}%
      \onslide*<4>{\providecommand\step{3}\input{diagrams/backtrace/scanstack.tex}}%
      \onslide*<5>{\providecommand\step{4}\input{diagrams/backtrace/scanstack.tex}}%
      \onslide*<6>{\providecommand\step{5}\input{diagrams/backtrace/scanstack.tex}}%
    \end{column}
  \end{columns}
\end{frame}

% Duplicate of previous-previous slide
\begin{frame}{Backtraces}{Continuing on \funcname{caml\_stash\_backtrace}}
  \begin{columns}[c]
    \begin{column}{0.5\textwidth}
      \minipage[c][0.75\textheight][s]{\columnwidth}
      \begin{itemize}
          \color{gray}
        \item A C function provided by the runtime (specific to native code)
        \item It scans the stack, between the stack pointer at the raising point up to the next trap, in order to collect the names of the detected function frames
        \item It uses the same technique and information as the Garbage Collector (GC) uses to scan the values live on the stack
      \end{itemize}
      \begin{itemize}
        \item For each function frame detected, store a pointer to the \typename{frame\_descr} in the \localname{domain\_state->backtrace\_buffer} array, effectively constructing the backtrace
      \end{itemize}
      \onslide<2->{
        \begin{itemize}
            \smallskip
          \item Constructed backtrace:
            \funcname{definitely\_raise},
            \onslide<3->{\funcname{probably\_raise}}
        \end{itemize}
      }
      \vfill
      \mintocaml[firstline=9,lastline=13,highlightlines=12]{../src/backtrace/backtrace.ml}
      \endminipage
    \end{column}
    \begin{column}{0.5\textwidth}
      \raggedleft
      \onslide*<1>{\listStashBacktrace[highlightlines={114-115}]{}}
      \onslide*<2>{\providecommand\step{3}\input{diagrams/backtrace/backtrace.tex}}%
      \onslide*<3>{\providecommand\step{4}\input{diagrams/backtrace/backtrace.tex}}%
    \end{column}
  \end{columns}
\end{frame}

\begin{frame}{Backtraces}{Back to \funcname{caml\_raise\_exn}}
  \begin{columns}[c]
    \begin{column}{0.5\textwidth}
      \minipage[c][0.66\textheight][s]{\columnwidth}
      \begin{itemize}\color{gray}
        \item Checks wheteher exceptions backtrace should be recorded, by checking the \funcarg{domain\_state->backtrace\_active} value
        \item Switches to the C stack to be able to execute C code
        \item Calls \funcname{caml\_stash\_backtrace}, to collect the backtrace up to the next trap
      \end{itemize}
      \onslide*<2->{
        \begin{itemize}
          \item<2-> Executes the exception handler in \funcname{probably\_raise} with \funcname{RESTORE\_EXN\_HANDLER\_OCAML}
            \begin{itemize}
              \item Set \regname{RSP} to \localname{domain\_state->exn\_handler} value
              \item Restore \localname{domain\_state->exn\_handler} from trap
              \item Jump to the handler code with \funcname{ret}
            \end{itemize}
        \end{itemize}
      }
      \vfill
      \onslide*<1>{\mintocaml[firstline=9,lastline=13,highlightlines=12]{../src/backtrace/backtrace.ml}}
      \onslide*<2->{\mintocaml[firstline=15,lastline=21,highlightlines={19-21}]{../src/backtrace/backtrace.ml}}
      \endminipage
    \end{column}
    \begin{column}{0.5\textwidth}
      \centering
      \onslide*<1>{
        \mintc[firstline=190,lastline=192]{../ocaml/runtime/amd64.S}
        \mintc[firstline=234,lastline=237]{../ocaml/runtime/amd64.S}
      }
      \onslide*<2>{
        \mintc[firstline=190,lastline=192,highlightlines={191-192}]{../ocaml/runtime/amd64.S}
        \mintc[firstline=234,lastline=237,highlightlines={235-237}]{../ocaml/runtime/amd64.S}
      }
      \onslide*<1>{\listCamlRaiseExn[highlightlines=720]{}}
      \onslide*<2>{\listCamlRaiseExn[highlightlines=722]{}}
      \onslide*<3->{\providecommand\step{5}\input{diagrams/backtrace/backtrace.tex}}
    \end{column}
  \end{columns}
\end{frame}

\begin{frame}{Backtraces}{\funcname{caml\_probably\_raise}'s handler doesn't catch \typename{ExnC}}
  \begin{columns}[c]
    \begin{column}{0.45\textwidth}
      \minipage[c][0.75\textheight][s]{\columnwidth}
      \begin{itemize}
        \item<1-> \funcname{match \dots with}\footnotemark pattern matching can also match exceptions
        \item<1-> The CMM of \funcname{probably\_raise} shows that this \funcname{match \dots with} is implemented with a pattern matching and a \funcname{try \dots with}
        \item<1-> But this one only matches on \typename{ExnA} exception
        \item<2-> Since the pattern matching fails to catch the exception, \funcname{reraise} is called with our current \typename{ExnC} exception
        \item<3-> Disassembly shows that \funcname{reraise} is actually a call to \funcname{caml\_reraise\_exn}, which is also an assembly function provided by the runtime
      \end{itemize}
      \vfill
      \mintocaml[firstline=15,lastline=21,highlightlines={18-21}]{../src/backtrace/backtrace.ml}
      \endminipage
    \end{column}
    \begin{column}{0.55\textwidth}
      \centering
      \onslide*<1>{\mintcmm[highlightlines={10-18}]{../src/backtrace/camlBacktrace\_\_probably\_raise\_351.cmm}}
      \onslide*<2>{\listcmm[highlightlines=18]{../src/backtrace/camlBacktrace\_\_probably\_raise_351.cmm}{CMM of probably\_raise}}
      \onslide*<3>{\mintobjdump[firstline=5,lastline=35,highlightlines=31]{../src/backtrace/camlBacktrace\_\_probably\_raise_351.objdump}}
    \end{column}
  \end{columns}
  \only<1>{\footnotetext[1]{https://blog.janestreet.com/pattern-matching-and-exception-handling-unite/}}
\end{frame}

\newcommand\listCamlReraiseExn[1][]{
  \listasm[firstline=727,lastline=734,#1]{../ocaml/runtime/amd64.S}{runtime/amd64.S:727}}

\begin{frame}{Backtraces}{Introducing \funcname{caml\_reraise\_exn}}
  \begin{columns}[c]
    \begin{column}{0.5\textwidth}
      \minipage[c][0.66\textheight][s]{\columnwidth}
      \begin{itemize}
        \item<1-> The exception have to be reraised to the next handler
        \item<2-> First, checks if the backtrace should be collected with \funcarg{domain\_state->backtrace\_active}
        \only<3>{
        \begin{itemize}
            \item If not, behaves like a \funcname{raise\_notrace} with \funcname{RESTORE\_EXN\_HANDLER\_OCAML}
        \end{itemize}
        }
        \item<4-> Jumps into \funcname{caml\_raise\_exn}
        \item<5-> Calls \funcname{caml\_stash\_backtrace} again, to scan the next part of the stack: from the trap just pop-ed to the next trap
      \end{itemize}
      \vfill
      \mintocaml[firstline=15,lastline=21,highlightlines={18-21}]{../src/backtrace/backtrace.ml}
      \endminipage
    \end{column}
    \begin{column}{0.5\textwidth}
      \raggedleft
      \onslide*<1>{\listCamlReraiseExn{}}
      \onslide*<2>{\listCamlReraiseExn[highlightlines=729]{}}
      \onslide*<3>{
        \mintc[firstline=190,lastline=192,highlightlines={191-192}]{../ocaml/runtime/amd64.S}
        \mintc[firstline=234,lastline=237,highlightlines={235-237}]{../ocaml/runtime/amd64.S}
        \medskip
        \listCamlReraiseExn[highlightlines=731]{}
      }
      \onslide*<4-5>{
        % TODO refactor with \listCamlReraiseExn
        \mintasm[firstline=727,lastline=734,highlightlines=730]{../ocaml/runtime/amd64.S}
        \medskip
      }
      \onslide*<4>{\listCamlRaiseExn[highlightlines=711]{}}
      \onslide*<5>{\listCamlRaiseExn[highlightlines=720]{}}
    \end{column}
  \end{columns}
\end{frame}

\begin{frame}{Backtraces}{Backtrace collection continues}
  \begin{columns}[c]
    \begin{column}{0.55\textwidth}
      \minipage[c][0.75\textheight][s]{\columnwidth}
      \begin{itemize}
        \item<1-> \funcname{caml\_stash\_backtrace} scans the older part of the stack, up to the next trap
        \item<6-> Controls jumps to the nested exception handler in \funcname{maybe\_raise}, which matches only on \typename{ExnB}
        \item<7-> \funcname{caml\_reraise\_exn} is called again, backtrace collection resumes with \funcname{caml\_stash\_backtrace}
      \end{itemize}
      \medskip
      \begin{itemize}
        \item<2-> Constructed backtrace:
        \begin{itemize}
          \item <2-> \funcname{definitely\_raise}
          \item <2-> \funcname{probably\_raise}
          \item <3-> \funcname{probably\_raise} (re-raise)
          \item <4-> \funcname{maybe\_raise}
          \item <5-> \funcname{main}
          \item <8-> \funcname{main} (re-raise)
        \end{itemize}
      \end{itemize}
      \vfill
      \onslide*<1-5>{\mintocaml[firstline=15,lastline=21]{../src/backtrace/backtrace.ml}}
      \onslide*<6>{\mintocaml[firstline=28,lastline=34,highlightlines=31]{../src/backtrace/backtrace.ml}}
      \onslide*<7->{\mintocaml[firstline=28,lastline=34,highlightlines=33]{../src/backtrace/backtrace.ml}}
      \endminipage
    \end{column}
    \begin{column}{0.45\textwidth}
      \raggedleft
      \onslide*<1>{\providecommand\step{6}\input{diagrams/backtrace/backtrace.tex}}%
      \onslide*<2>{\providecommand\step{6}\input{diagrams/backtrace/backtrace.tex}}%
      \onslide*<3>{\providecommand\step{7}\input{diagrams/backtrace/backtrace.tex}}%
      \onslide*<4>{\providecommand\step{8}\input{diagrams/backtrace/backtrace.tex}}%
      \onslide*<5>{\providecommand\step{9}\input{diagrams/backtrace/backtrace.tex}}%
      \onslide*<6>{\providecommand\step{10}\input{diagrams/backtrace/backtrace.tex}}%
      \onslide*<7>{\providecommand\step{11}\input{diagrams/backtrace/backtrace.tex}}%
      \onslide*<8>{\providecommand\step{12}\input{diagrams/backtrace/backtrace.tex}}%
      \onslide*<9>{\providecommand\step{13}\input{diagrams/backtrace/backtrace.tex}}%
    \end{column}
  \end{columns}
\end{frame}

\begin{frame}{Backtraces}{Printing the backtrace}
  \begin{columns}[c]
    \begin{column}{0.45\textwidth}
      \minipage[c][0.66\textheight][s]{\columnwidth}
      \begin{itemize}
        \item<1-> The exception backtrace can now be printed to \funcarg{stdout} with \funcname{Printexc.print_backtrace}
        \item<2-> First the exception backtrace is retrieved into an OCaml array with \funcname{get_raw_backtrace }
        \item<3-> Which is implemented as the \funcname{caml_get_exception_raw_backtrace} C function in the runtime
        \item<4-> And finally printed with \funcname{print_exception_backtrace}
      \end{itemize}
      \vfill
      \mintocaml[firstline=28,lastline=34,highlightlines=33]{../src/backtrace/backtrace.ml}
      \endminipage
    \end{column}
    \begin{column}{0.55\textwidth}
      \onslide*<1,2>{
        \mintocaml[firstline=100,lastline=101]{../ocaml/stdlib/printexc.ml}
      }
      \only<1>{
        \listocaml[firstline=170,lastline=172]{../ocaml/stdlib/printexc.ml}{stdlib/printexc.ml:170}
      }
      \only<2>{
        \listocaml[firstline=170,lastline=172,highlightlines=172]{../ocaml/stdlib/printexc.ml}{stdlib/printexc.ml:170}
      }
      \only<3>{
        \mintc[firstline=154,lastline=156]{../ocaml/runtime/backtrace.c}
        \listc[firstline=171,lastline=192,highlightlines={184-187}]{../ocaml/runtime/backtrace.c}{runtime/backtrace.c:154}
      }
      \only<4>{
        \listocaml[firstline=155,lastline=165]{../ocaml/stdlib/printexc.ml}{stdlib/printexc.ml:55}
      }
    \end{column}
  \end{columns}
\end{frame}


\subsection{The multiple kinds of raise}
\frameSubsection{}{}

\begin{frame}{Backtraces}{When backtraces are not needed}
  \begin{columns}[c]
    \begin{column}{0.45\textwidth}
      \minipage[c][0.66\textheight][s]{\columnwidth}
      \begin{itemize}
        \item<1-> What if the backtrace for an exception is never needed?
        \item<2-> It's possible to raise the exception explicitly with \funcname{raise\_notrace} to make raising as cheap as possible
        \item<3-> An example of use in the \funcname{dynlink} library of the OCaml compiler
      \end{itemize}
      \vfill
      \onslide<3->{
      \listocaml[firstline=205,lastline=209,highlightlines=207]{../ocaml/otherlibs/dynlink/dynlink\_compilerlibs/misc.ml}{otherlibs/dynlink/dynlink\_compilerlibs/misc.ml:205}
      }
      \endminipage
    \end{column}
    \begin{column}{0.55\textwidth}
    \only<4>{
      \mintcmm[highlightlines=3]{../src/notrace/camlNotrace\_\_fun_347.cmm}
      \listcmm{../src/notrace/camlNotrace\_\_all\_somes\_327.cmm}{all\_somes}
    }
    \onslide*<5->{
      \listobjdump[highlightlines={6-9}]{../src/notrace/camlNotrace\_\_fun_347.objdump}{anonymous function from \funcname{all\_somes}}
    }
    \end{column}
  \end{columns}
\end{frame}

\begin{frame}{Backtraces}{Summary table of the multiple kinds of raise}
  \begin{itemize}
    \item When debugging information is enabled and if the backtrace of an exception isn't needed, it's possible to raise with \funcname{raise\_notrace} for performance
    \item When debugging information is \emph{not} enabled, the compiler optimises \funcname{raise} calls to inline assembly (same effect as using \funcname{raise\_notrace})
  \end{itemize}
  \bigskip
  \centering
  \begin{tabular}{| l | c | c | c | c |}
    \hline
    \parbox{10em}{Debug information \\ (\funcarg{-g} flag)}  & \multicolumn{2}{|c|}{Disabled}                                     & \multicolumn{2}{|c|}{Enabled} \\
    \hline
    Raise kind                                               & \funcname{raise} & \funcname{raise\_notrace}                       & \funcname{raise} & \funcname{raise\_notrace} \\
    \hline
    Compilation                                              & \parbox{8em}{inline assembly\\ (optimization)} & inline assembly   & \funcname{caml\_raise\_exn} & inline assembly \\
    \hline
  \end{tabular}
\end{frame}


%
% Takeaway #5
%
\subsection*{Takeaway \#5}
\frameSubsectionTakeaway{}
\againframe<5>{takeaway}
}%
    \end{column}
  \end{columns}
\end{frame}

\begin{frame}{Backtraces}{Back to \funcname{caml\_raise\_exn}}
  \begin{columns}[c]
    \begin{column}{0.5\textwidth}
      \minipage[c][0.66\textheight][s]{\columnwidth}
      \begin{itemize}\color{gray}
        \item Checks wheteher exceptions backtrace should be recorded, by checking the \funcarg{domain\_state->backtrace\_active} value
        \item Switches to the C stack to be able to execute C code
        \item Calls \funcname{caml\_stash\_backtrace}, to collect the backtrace up to the next trap
      \end{itemize}
      \onslide*<2->{
        \begin{itemize}
          \item<2-> Executes the exception handler in \funcname{probably\_raise} with \funcname{RESTORE\_EXN\_HANDLER\_OCAML}
            \begin{itemize}
              \item Set \regname{RSP} to \localname{domain\_state->exn\_handler} value
              \item Restore \localname{domain\_state->exn\_handler} from trap
              \item Jump to the handler code with \funcname{ret}
            \end{itemize}
        \end{itemize}
      }
      \vfill
      \onslide*<1>{\mintocaml[firstline=9,lastline=13,highlightlines=12]{../src/backtrace/backtrace.ml}}
      \onslide*<2->{\mintocaml[firstline=15,lastline=21,highlightlines={19-21}]{../src/backtrace/backtrace.ml}}
      \endminipage
    \end{column}
    \begin{column}{0.5\textwidth}
      \centering
      \onslide*<1>{
        \mintc[firstline=190,lastline=192]{../ocaml/runtime/amd64.S}
        \mintc[firstline=234,lastline=237]{../ocaml/runtime/amd64.S}
      }
      \onslide*<2>{
        \mintc[firstline=190,lastline=192,highlightlines={191-192}]{../ocaml/runtime/amd64.S}
        \mintc[firstline=234,lastline=237,highlightlines={235-237}]{../ocaml/runtime/amd64.S}
      }
      \onslide*<1>{\listCamlRaiseExn[highlightlines=720]{}}
      \onslide*<2>{\listCamlRaiseExn[highlightlines=722]{}}
      \onslide*<3->{\providecommand\step{5}%
% Backtraces
%
\subsection{One advantage of raising with debugging information}
\frameSubsection{}{}

\begin{frame}{Backtraces}{Debugging information \& source code}
  \begin{columns}[c]
    \begin{column}{0.5\textwidth}
      \begin{itemize}
        \item Raising an exception is fun and games, but how do we know where the exception originated? $\rightarrow$ With the backtrace associated with the exception!
        \item In order to get a backtrace from the point where an exception was raised, it's necessary to compile with debugging information enabled (\funcarg{-g} flag for the compiler)
        \item Same example as in the `Nested handlers' section
      \end{itemize}
    \end{column}
    \begin{column}{0.5\textwidth}
      \listocaml[firstline=9,lastline=34]{../src/backtrace/backtrace.ml}{backtrace/backtrace.ml}
    \end{column}
  \end{columns}
\end{frame}

\begin{frame}{Backtraces}{CMM and disassembly of \funcname{definitely\_raise}}
  \begin{columns}[c]
    \begin{column}{0.5\textwidth}
      \minipage[c][0.66\textheight][s]{\columnwidth}
      \begin{itemize}
        \item<1-> An effect of compiling with debugging information (\funcarg{-g}): the CMM no longer uses \funcname{raise\_notrace} to raise the exception but \funcname{raise} instead
        \item<2-> Disassembly shows that CMM \funcname{raise} is actually a call to \funcname{caml\_raise\_exn}, a function in assembler provided by the runtime
      \end{itemize}
      \vfill
      \mintocaml[firstline=9,lastline=13,highlightlines=12]{../src/backtrace/backtrace.ml}
      \endminipage
    \end{column}
    \begin{column}{0.5\textwidth}
      \onslide*<1>{
        \mintcmm[highlightlines=5]{../src/backtrace/camlBacktrace\_\_definitely\_raise\_347.cmm}
      }
      \onslide*<2>{
        \mintobjdump[firstline=2,highlightlines=14]{../src/backtrace/camlBacktrace\_\_definitely\_raise\_347.objdump}
      }
    \end{column}
  \end{columns}
\end{frame}

\begin{frame}{Backtraces}{Introducing \funcname{caml\_raise\_exn}}
  \begin{columns}[c]
    \begin{column}{0.5\textwidth}
      \minipage[c][0.66\textheight][s]{\columnwidth}
      \onslide*<1>{
        \begin{itemize}
          \item Raising the exception is performed using \funcname{caml\_raise\_exn} which is
            \begin{itemize}
              \item An assembly function provided by the runtime
              \item Architecture specific
            \end{itemize}
          \item Let's see what it does differently from \funcname{raise\_notrace}\dots
          \item We are about to raise \typename{ExnC} with \funcname{caml\_raise\_exn} from \funcname{definitely\_raise}
        \end{itemize}
      }
      \onslide*<2>{
        \mintobjdump[firstline=2,highlightlines=14]{../src/backtrace/camlBacktrace\_\_definitely\_raise\_347.objdump}
      }
      \vfill
      \mintocaml[firstline=9,lastline=13,highlightlines=12]{../src/backtrace/backtrace.ml}
      \endminipage
    \end{column}
    \begin{column}{0.5\textwidth}
      \centering
      \providecommand\step{1}\input{diagrams/backtrace/backtrace.tex}
    \end{column}
  \end{columns}
\end{frame}

\begin{frame}{Backtraces}{But before that, we need more fields from \typename{caml\_domain\_state}}
  \begin{columns}
    \begin{column}{0.5\textwidth}
      \begin{itemize}
        \item Remember \typename{caml\_domain\_state} the per-domain runtime state?
        \item Some other members are of interest to us now\dots
        \item \localname{backtrace\_active} is used to know if the runtime should collect backtraces
        \item \localname{backtrace\_active} can be set by
          \begin{itemize}
            \item The \funcarg{b} flag in the \funcname{OCAMLRUNPARAM} environment variable
            \item \mintinline{ocaml}{Printexc.record_backtrace : bool -> unit} \footnotemark
          \end{itemize}
      \end{itemize}
    \end{column}
    \begin{column}{0.5\textwidth}
      \begin{listing}
        \mintc[highlightlines={9-11}]{src/ocaml/domain\_state\_backtrace.h}
        \caption{runtime/caml/domain\_state.h}
      \end{listing}
    \end{column}
  \end{columns}
  \footnotetext[1]{https://v2.ocaml.org/api/Printexc.html}
\end{frame}

\newcommand\listCamlRaiseExn[1][]{
  \listasm[firstline=702,lastline=725,#1]{../ocaml/runtime/amd64.S}{runtime/amd64.S:702}}

\begin{frame}{Backtraces}{Walk through \funcname{caml\_raise\_exn}}
  \begin{columns}[T]
    \begin{column}{0.5\textwidth}
      \begin{itemize}
        \item<1-> First checks whether exceptions backtrace should be recorded
        \item<1-> By checking the \funcarg{domain\_state->backtrace\_active} value
        \item<2-> If not, acts as \funcname{raise\_notrace} with \funcname{RESTORE\_EXN\_HANDLER\_OCAML}
          \begin{itemize}
            \item Set \regname{RSP} to \localname{domain\_state->exn\_handler} value
            \item Restore \localname{domain\_state->exn\_handler} from trap
            \item Jump with \funcname{ret} (same effect as using \funcname{pop} and \funcname{jmp})
          \end{itemize}
        \item<3-> Otherwise, switches to the C stack to be able to execute C code
        \item<4-> Calls \funcname{caml\_stash\_backtrace}, a C function provided by the runtime\ignorespaces
          \onslide<6->{, with
            \begin{itemize}
              \item \funcarg{exn}: exception value
              \item \funcarg{pc}: code address of the raising point
              \item \funcarg{sp}: stack address of the raising function
              \item \funcarg{trapsp}: stack address of the next handler
            \end{itemize}
          }
      \end{itemize}
      \bigskip
      \mintocaml[firstline=9,lastline=13,highlightlines=12]{../src/backtrace/backtrace.ml}
    \end{column}
    \begin{column}{0.5\textwidth}
      \raggedleft
      \onslide*<1,3-4>{
        \mintc[firstline=190,lastline=192]{../ocaml/runtime/amd64.S}
        \mintc[firstline=234,lastline=237]{../ocaml/runtime/amd64.S}
      }
      \onslide*<2>{
        \mintc[firstline=190,lastline=192,highlightlines={191-192}]{../ocaml/runtime/amd64.S}
        \mintc[firstline=234,lastline=237,highlightlines={235-237}]{../ocaml/runtime/amd64.S}
      }
      \onslide*<1>{\listCamlRaiseExn[highlightlines=705]{}}%
      \onslide*<2>{\listCamlRaiseExn[highlightlines={707-708}]{}}%
      \onslide*<3>{\listCamlRaiseExn[highlightlines={712-714}]{}}%
      \onslide*<4>{\listCamlRaiseExn[highlightlines={715-720}]{}}%
      \onslide*<5>{\providecommand\step{1}\input{diagrams/backtrace/backtrace.tex}}%
      \onslide*<6>{\providecommand\step{2}\input{diagrams/backtrace/backtrace.tex}}%
    \end{column}
  \end{columns}
\end{frame}

\newcommand\listStashBacktrace[1][]{
  \listc[firstline=91,lastline=120,#1]{../ocaml/runtime/backtrace\_nat.c}{runtime/backtrace\_nat.c:91}}

\begin{frame}{Backtraces}{Introducing \funcname{caml\_stash\_backtrace}}
  \begin{columns}[c]
    \begin{column}{0.5\textwidth}
      \minipage[c][0.66\textheight][s]{\columnwidth}
      \begin{itemize}
        \item<1-> A C function provided by the runtime (specific to native code)
        \item<2-> It scans the stack, between the stack pointer at the raising point up to the next trap, in order to collect the names of the detected function frames
          \onslide*<2>{
            \begin{itemize}
              \item And more information, like line number, column number...
            \end{itemize}
          }
        \item<3-> It uses the same technique and information as the Garbage Collector (GC) uses to scan the values live on the stack
          \onslide*<4>{
            \begin{itemize}
              \item \typename{frame\_descr} are at the core of stack unwinding
            \end{itemize}
          }
      \end{itemize}
      \vfill
      \mintocaml[firstline=9,lastline=13,highlightlines=12]{../src/backtrace/backtrace.ml}
      \endminipage
    \end{column}
    \begin{column}{0.5\textwidth}
      \onslide*<1>{\listStashBacktrace[highlightlines=91]{}}
      \onslide*<2>{\listStashBacktrace[highlightlines={91,117-118}]{}}
      \onslide*<3->{\listStashBacktrace[highlightlines={94,106,109-110}]{}}
    \end{column}
  \end{columns}
\end{frame}

\newcommand\listFrameDescr[1][]{
  \listocaml[firstline=111,lastline=124,#1]{../ocaml/asmcomp/emitaux.ml}{asmcomp/emitaux.ml:111}}
\newcommand\listDebugInfo[1][]{
  \listocaml[firstline=38,lastline=49,#1]{../ocaml/lambda/debuginfo.mli}{lambda/debuginfo.mli:38}}

\begin{frame}{Backtraces}{\typename{frame\_descr} and \typename{frame\_debuginfo} in a nutshell}
  \begin{columns}[c]
    \begin{column}{0.5\textwidth}
      \begin{itemize}
        \item<1-> For every return address in an OCaml program, there is a associated \typename{frame\_descr}
        \item<2-> A \typename{frame\_descr} specifies
        \begin{itemize}
          \item<2-> The size of the function stack frame
          \item<3-> Where live values are on the stack (so that the GC can mark them as live and prevent from being swept)
        \end{itemize}
        \item<4-> When compiled with debug information, \typename{frame\_descr} will be linked to a \typename{frame\_debuginfo}
        \begin{itemize}
          \item<5-> Contains information about the position in the source code (file name, line and column number)
        \end{itemize}
      \end{itemize}
    \end{column}
    \begin{column}{0.5\textwidth}
        \onslide*<1>{\listFrameDescr[highlightlines={118-122}]{}}
        \onslide*<2>{\listFrameDescr[highlightlines={120}]{}}
        \onslide*<3>{\listFrameDescr[highlightlines={121}]{}}
        \onslide*<4->{\listFrameDescr[highlightlines={122}]{}}
        \medskip
        \onslide*<1-3>{\listDebugInfo{}}
        \onslide*<4>{\listDebugInfo[highlightlines={38,49}]{}}
        \onslide*<5>{\listDebugInfo[highlightlines={39-46}]{}}
    \end{column}
  \end{columns}
\end{frame}

\begin{frame}{Backtraces}{Scanning the stack with \typename{frame\_descr}}
  \begin{columns}[c]
    \begin{column}{0.5\textwidth}
      \begin{itemize}
        \item Given the return address of the code that raises the exception by calling \funcname{caml\_raise\_exn} (\funcarg{pc} argument of \funcname{caml\_stash\_backtrace})
        \item And the position of the stack pointer (\funcarg{sp} argument of \funcname{caml\_stash\_backtrace})
        \item The frame size given by \typename{frame\_descr}.\funcarg{frame\_size}, allows to find the end of the previous frame
        \item And the return address of the caller of this function
        \bigskip
        \onslide<3->{
        \item Note that traps (here the reddish \funcname{ExnA}, \funcname{ExnB} and \funcname{ExnC} blocks) belong to the frame that installed them, and so the frame size changes when traps are removed
        }
      \end{itemize}
    \end{column}
    \begin{column}{0.5\textwidth}
      \raggedleft
      \onslide*<1>{\providecommand\step{2}\input{diagrams/backtrace/backtrace.tex}}%
      \onslide*<2>{\providecommand\step{1}\input{diagrams/backtrace/scanstack.tex}}%
      \onslide*<3>{\providecommand\step{2}\input{diagrams/backtrace/scanstack.tex}}%
      \onslide*<4>{\providecommand\step{3}\input{diagrams/backtrace/scanstack.tex}}%
      \onslide*<5>{\providecommand\step{4}\input{diagrams/backtrace/scanstack.tex}}%
      \onslide*<6>{\providecommand\step{5}\input{diagrams/backtrace/scanstack.tex}}%
    \end{column}
  \end{columns}
\end{frame}

% Duplicate of previous-previous slide
\begin{frame}{Backtraces}{Continuing on \funcname{caml\_stash\_backtrace}}
  \begin{columns}[c]
    \begin{column}{0.5\textwidth}
      \minipage[c][0.75\textheight][s]{\columnwidth}
      \begin{itemize}
          \color{gray}
        \item A C function provided by the runtime (specific to native code)
        \item It scans the stack, between the stack pointer at the raising point up to the next trap, in order to collect the names of the detected function frames
        \item It uses the same technique and information as the Garbage Collector (GC) uses to scan the values live on the stack
      \end{itemize}
      \begin{itemize}
        \item For each function frame detected, store a pointer to the \typename{frame\_descr} in the \localname{domain\_state->backtrace\_buffer} array, effectively constructing the backtrace
      \end{itemize}
      \onslide<2->{
        \begin{itemize}
            \smallskip
          \item Constructed backtrace:
            \funcname{definitely\_raise},
            \onslide<3->{\funcname{probably\_raise}}
        \end{itemize}
      }
      \vfill
      \mintocaml[firstline=9,lastline=13,highlightlines=12]{../src/backtrace/backtrace.ml}
      \endminipage
    \end{column}
    \begin{column}{0.5\textwidth}
      \raggedleft
      \onslide*<1>{\listStashBacktrace[highlightlines={114-115}]{}}
      \onslide*<2>{\providecommand\step{3}\input{diagrams/backtrace/backtrace.tex}}%
      \onslide*<3>{\providecommand\step{4}\input{diagrams/backtrace/backtrace.tex}}%
    \end{column}
  \end{columns}
\end{frame}

\begin{frame}{Backtraces}{Back to \funcname{caml\_raise\_exn}}
  \begin{columns}[c]
    \begin{column}{0.5\textwidth}
      \minipage[c][0.66\textheight][s]{\columnwidth}
      \begin{itemize}\color{gray}
        \item Checks wheteher exceptions backtrace should be recorded, by checking the \funcarg{domain\_state->backtrace\_active} value
        \item Switches to the C stack to be able to execute C code
        \item Calls \funcname{caml\_stash\_backtrace}, to collect the backtrace up to the next trap
      \end{itemize}
      \onslide*<2->{
        \begin{itemize}
          \item<2-> Executes the exception handler in \funcname{probably\_raise} with \funcname{RESTORE\_EXN\_HANDLER\_OCAML}
            \begin{itemize}
              \item Set \regname{RSP} to \localname{domain\_state->exn\_handler} value
              \item Restore \localname{domain\_state->exn\_handler} from trap
              \item Jump to the handler code with \funcname{ret}
            \end{itemize}
        \end{itemize}
      }
      \vfill
      \onslide*<1>{\mintocaml[firstline=9,lastline=13,highlightlines=12]{../src/backtrace/backtrace.ml}}
      \onslide*<2->{\mintocaml[firstline=15,lastline=21,highlightlines={19-21}]{../src/backtrace/backtrace.ml}}
      \endminipage
    \end{column}
    \begin{column}{0.5\textwidth}
      \centering
      \onslide*<1>{
        \mintc[firstline=190,lastline=192]{../ocaml/runtime/amd64.S}
        \mintc[firstline=234,lastline=237]{../ocaml/runtime/amd64.S}
      }
      \onslide*<2>{
        \mintc[firstline=190,lastline=192,highlightlines={191-192}]{../ocaml/runtime/amd64.S}
        \mintc[firstline=234,lastline=237,highlightlines={235-237}]{../ocaml/runtime/amd64.S}
      }
      \onslide*<1>{\listCamlRaiseExn[highlightlines=720]{}}
      \onslide*<2>{\listCamlRaiseExn[highlightlines=722]{}}
      \onslide*<3->{\providecommand\step{5}\input{diagrams/backtrace/backtrace.tex}}
    \end{column}
  \end{columns}
\end{frame}

\begin{frame}{Backtraces}{\funcname{caml\_probably\_raise}'s handler doesn't catch \typename{ExnC}}
  \begin{columns}[c]
    \begin{column}{0.45\textwidth}
      \minipage[c][0.75\textheight][s]{\columnwidth}
      \begin{itemize}
        \item<1-> \funcname{match \dots with}\footnotemark pattern matching can also match exceptions
        \item<1-> The CMM of \funcname{probably\_raise} shows that this \funcname{match \dots with} is implemented with a pattern matching and a \funcname{try \dots with}
        \item<1-> But this one only matches on \typename{ExnA} exception
        \item<2-> Since the pattern matching fails to catch the exception, \funcname{reraise} is called with our current \typename{ExnC} exception
        \item<3-> Disassembly shows that \funcname{reraise} is actually a call to \funcname{caml\_reraise\_exn}, which is also an assembly function provided by the runtime
      \end{itemize}
      \vfill
      \mintocaml[firstline=15,lastline=21,highlightlines={18-21}]{../src/backtrace/backtrace.ml}
      \endminipage
    \end{column}
    \begin{column}{0.55\textwidth}
      \centering
      \onslide*<1>{\mintcmm[highlightlines={10-18}]{../src/backtrace/camlBacktrace\_\_probably\_raise\_351.cmm}}
      \onslide*<2>{\listcmm[highlightlines=18]{../src/backtrace/camlBacktrace\_\_probably\_raise_351.cmm}{CMM of probably\_raise}}
      \onslide*<3>{\mintobjdump[firstline=5,lastline=35,highlightlines=31]{../src/backtrace/camlBacktrace\_\_probably\_raise_351.objdump}}
    \end{column}
  \end{columns}
  \only<1>{\footnotetext[1]{https://blog.janestreet.com/pattern-matching-and-exception-handling-unite/}}
\end{frame}

\newcommand\listCamlReraiseExn[1][]{
  \listasm[firstline=727,lastline=734,#1]{../ocaml/runtime/amd64.S}{runtime/amd64.S:727}}

\begin{frame}{Backtraces}{Introducing \funcname{caml\_reraise\_exn}}
  \begin{columns}[c]
    \begin{column}{0.5\textwidth}
      \minipage[c][0.66\textheight][s]{\columnwidth}
      \begin{itemize}
        \item<1-> The exception have to be reraised to the next handler
        \item<2-> First, checks if the backtrace should be collected with \funcarg{domain\_state->backtrace\_active}
        \only<3>{
        \begin{itemize}
            \item If not, behaves like a \funcname{raise\_notrace} with \funcname{RESTORE\_EXN\_HANDLER\_OCAML}
        \end{itemize}
        }
        \item<4-> Jumps into \funcname{caml\_raise\_exn}
        \item<5-> Calls \funcname{caml\_stash\_backtrace} again, to scan the next part of the stack: from the trap just pop-ed to the next trap
      \end{itemize}
      \vfill
      \mintocaml[firstline=15,lastline=21,highlightlines={18-21}]{../src/backtrace/backtrace.ml}
      \endminipage
    \end{column}
    \begin{column}{0.5\textwidth}
      \raggedleft
      \onslide*<1>{\listCamlReraiseExn{}}
      \onslide*<2>{\listCamlReraiseExn[highlightlines=729]{}}
      \onslide*<3>{
        \mintc[firstline=190,lastline=192,highlightlines={191-192}]{../ocaml/runtime/amd64.S}
        \mintc[firstline=234,lastline=237,highlightlines={235-237}]{../ocaml/runtime/amd64.S}
        \medskip
        \listCamlReraiseExn[highlightlines=731]{}
      }
      \onslide*<4-5>{
        % TODO refactor with \listCamlReraiseExn
        \mintasm[firstline=727,lastline=734,highlightlines=730]{../ocaml/runtime/amd64.S}
        \medskip
      }
      \onslide*<4>{\listCamlRaiseExn[highlightlines=711]{}}
      \onslide*<5>{\listCamlRaiseExn[highlightlines=720]{}}
    \end{column}
  \end{columns}
\end{frame}

\begin{frame}{Backtraces}{Backtrace collection continues}
  \begin{columns}[c]
    \begin{column}{0.55\textwidth}
      \minipage[c][0.75\textheight][s]{\columnwidth}
      \begin{itemize}
        \item<1-> \funcname{caml\_stash\_backtrace} scans the older part of the stack, up to the next trap
        \item<6-> Controls jumps to the nested exception handler in \funcname{maybe\_raise}, which matches only on \typename{ExnB}
        \item<7-> \funcname{caml\_reraise\_exn} is called again, backtrace collection resumes with \funcname{caml\_stash\_backtrace}
      \end{itemize}
      \medskip
      \begin{itemize}
        \item<2-> Constructed backtrace:
        \begin{itemize}
          \item <2-> \funcname{definitely\_raise}
          \item <2-> \funcname{probably\_raise}
          \item <3-> \funcname{probably\_raise} (re-raise)
          \item <4-> \funcname{maybe\_raise}
          \item <5-> \funcname{main}
          \item <8-> \funcname{main} (re-raise)
        \end{itemize}
      \end{itemize}
      \vfill
      \onslide*<1-5>{\mintocaml[firstline=15,lastline=21]{../src/backtrace/backtrace.ml}}
      \onslide*<6>{\mintocaml[firstline=28,lastline=34,highlightlines=31]{../src/backtrace/backtrace.ml}}
      \onslide*<7->{\mintocaml[firstline=28,lastline=34,highlightlines=33]{../src/backtrace/backtrace.ml}}
      \endminipage
    \end{column}
    \begin{column}{0.45\textwidth}
      \raggedleft
      \onslide*<1>{\providecommand\step{6}\input{diagrams/backtrace/backtrace.tex}}%
      \onslide*<2>{\providecommand\step{6}\input{diagrams/backtrace/backtrace.tex}}%
      \onslide*<3>{\providecommand\step{7}\input{diagrams/backtrace/backtrace.tex}}%
      \onslide*<4>{\providecommand\step{8}\input{diagrams/backtrace/backtrace.tex}}%
      \onslide*<5>{\providecommand\step{9}\input{diagrams/backtrace/backtrace.tex}}%
      \onslide*<6>{\providecommand\step{10}\input{diagrams/backtrace/backtrace.tex}}%
      \onslide*<7>{\providecommand\step{11}\input{diagrams/backtrace/backtrace.tex}}%
      \onslide*<8>{\providecommand\step{12}\input{diagrams/backtrace/backtrace.tex}}%
      \onslide*<9>{\providecommand\step{13}\input{diagrams/backtrace/backtrace.tex}}%
    \end{column}
  \end{columns}
\end{frame}

\begin{frame}{Backtraces}{Printing the backtrace}
  \begin{columns}[c]
    \begin{column}{0.45\textwidth}
      \minipage[c][0.66\textheight][s]{\columnwidth}
      \begin{itemize}
        \item<1-> The exception backtrace can now be printed to \funcarg{stdout} with \funcname{Printexc.print_backtrace}
        \item<2-> First the exception backtrace is retrieved into an OCaml array with \funcname{get_raw_backtrace }
        \item<3-> Which is implemented as the \funcname{caml_get_exception_raw_backtrace} C function in the runtime
        \item<4-> And finally printed with \funcname{print_exception_backtrace}
      \end{itemize}
      \vfill
      \mintocaml[firstline=28,lastline=34,highlightlines=33]{../src/backtrace/backtrace.ml}
      \endminipage
    \end{column}
    \begin{column}{0.55\textwidth}
      \onslide*<1,2>{
        \mintocaml[firstline=100,lastline=101]{../ocaml/stdlib/printexc.ml}
      }
      \only<1>{
        \listocaml[firstline=170,lastline=172]{../ocaml/stdlib/printexc.ml}{stdlib/printexc.ml:170}
      }
      \only<2>{
        \listocaml[firstline=170,lastline=172,highlightlines=172]{../ocaml/stdlib/printexc.ml}{stdlib/printexc.ml:170}
      }
      \only<3>{
        \mintc[firstline=154,lastline=156]{../ocaml/runtime/backtrace.c}
        \listc[firstline=171,lastline=192,highlightlines={184-187}]{../ocaml/runtime/backtrace.c}{runtime/backtrace.c:154}
      }
      \only<4>{
        \listocaml[firstline=155,lastline=165]{../ocaml/stdlib/printexc.ml}{stdlib/printexc.ml:55}
      }
    \end{column}
  \end{columns}
\end{frame}


\subsection{The multiple kinds of raise}
\frameSubsection{}{}

\begin{frame}{Backtraces}{When backtraces are not needed}
  \begin{columns}[c]
    \begin{column}{0.45\textwidth}
      \minipage[c][0.66\textheight][s]{\columnwidth}
      \begin{itemize}
        \item<1-> What if the backtrace for an exception is never needed?
        \item<2-> It's possible to raise the exception explicitly with \funcname{raise\_notrace} to make raising as cheap as possible
        \item<3-> An example of use in the \funcname{dynlink} library of the OCaml compiler
      \end{itemize}
      \vfill
      \onslide<3->{
      \listocaml[firstline=205,lastline=209,highlightlines=207]{../ocaml/otherlibs/dynlink/dynlink\_compilerlibs/misc.ml}{otherlibs/dynlink/dynlink\_compilerlibs/misc.ml:205}
      }
      \endminipage
    \end{column}
    \begin{column}{0.55\textwidth}
    \only<4>{
      \mintcmm[highlightlines=3]{../src/notrace/camlNotrace\_\_fun_347.cmm}
      \listcmm{../src/notrace/camlNotrace\_\_all\_somes\_327.cmm}{all\_somes}
    }
    \onslide*<5->{
      \listobjdump[highlightlines={6-9}]{../src/notrace/camlNotrace\_\_fun_347.objdump}{anonymous function from \funcname{all\_somes}}
    }
    \end{column}
  \end{columns}
\end{frame}

\begin{frame}{Backtraces}{Summary table of the multiple kinds of raise}
  \begin{itemize}
    \item When debugging information is enabled and if the backtrace of an exception isn't needed, it's possible to raise with \funcname{raise\_notrace} for performance
    \item When debugging information is \emph{not} enabled, the compiler optimises \funcname{raise} calls to inline assembly (same effect as using \funcname{raise\_notrace})
  \end{itemize}
  \bigskip
  \centering
  \begin{tabular}{| l | c | c | c | c |}
    \hline
    \parbox{10em}{Debug information \\ (\funcarg{-g} flag)}  & \multicolumn{2}{|c|}{Disabled}                                     & \multicolumn{2}{|c|}{Enabled} \\
    \hline
    Raise kind                                               & \funcname{raise} & \funcname{raise\_notrace}                       & \funcname{raise} & \funcname{raise\_notrace} \\
    \hline
    Compilation                                              & \parbox{8em}{inline assembly\\ (optimization)} & inline assembly   & \funcname{caml\_raise\_exn} & inline assembly \\
    \hline
  \end{tabular}
\end{frame}


%
% Takeaway #5
%
\subsection*{Takeaway \#5}
\frameSubsectionTakeaway{}
\againframe<5>{takeaway}
}
    \end{column}
  \end{columns}
\end{frame}

\begin{frame}{Backtraces}{\funcname{caml\_probably\_raise}'s handler doesn't catch \typename{ExnC}}
  \begin{columns}[c]
    \begin{column}{0.45\textwidth}
      \minipage[c][0.75\textheight][s]{\columnwidth}
      \begin{itemize}
        \item<1-> \funcname{match \dots with}\footnotemark pattern matching can also match exceptions
        \item<1-> The CMM of \funcname{probably\_raise} shows that this \funcname{match \dots with} is implemented with a pattern matching and a \funcname{try \dots with}
        \item<1-> But this one only matches on \typename{ExnA} exception
        \item<2-> Since the pattern matching fails to catch the exception, \funcname{reraise} is called with our current \typename{ExnC} exception
        \item<3-> Disassembly shows that \funcname{reraise} is actually a call to \funcname{caml\_reraise\_exn}, which is also an assembly function provided by the runtime
      \end{itemize}
      \vfill
      \mintocaml[firstline=15,lastline=21,highlightlines={18-21}]{../src/backtrace/backtrace.ml}
      \endminipage
    \end{column}
    \begin{column}{0.55\textwidth}
      \centering
      \onslide*<1>{\mintcmm[highlightlines={10-18}]{../src/backtrace/camlBacktrace\_\_probably\_raise\_351.cmm}}
      \onslide*<2>{\listcmm[highlightlines=18]{../src/backtrace/camlBacktrace\_\_probably\_raise_351.cmm}{CMM of probably\_raise}}
      \onslide*<3>{\mintobjdump[firstline=5,lastline=35,highlightlines=31]{../src/backtrace/camlBacktrace\_\_probably\_raise_351.objdump}}
    \end{column}
  \end{columns}
  \only<1>{\footnotetext[1]{https://blog.janestreet.com/pattern-matching-and-exception-handling-unite/}}
\end{frame}

\newcommand\listCamlReraiseExn[1][]{
  \listasm[firstline=727,lastline=734,#1]{../ocaml/runtime/amd64.S}{runtime/amd64.S:727}}

\begin{frame}{Backtraces}{Introducing \funcname{caml\_reraise\_exn}}
  \begin{columns}[c]
    \begin{column}{0.5\textwidth}
      \minipage[c][0.66\textheight][s]{\columnwidth}
      \begin{itemize}
        \item<1-> The exception have to be reraised to the next handler
        \item<2-> First, checks if the backtrace should be collected with \funcarg{domain\_state->backtrace\_active}
        \only<3>{
        \begin{itemize}
            \item If not, behaves like a \funcname{raise\_notrace} with \funcname{RESTORE\_EXN\_HANDLER\_OCAML}
        \end{itemize}
        }
        \item<4-> Jumps into \funcname{caml\_raise\_exn}
        \item<5-> Calls \funcname{caml\_stash\_backtrace} again, to scan the next part of the stack: from the trap just pop-ed to the next trap
      \end{itemize}
      \vfill
      \mintocaml[firstline=15,lastline=21,highlightlines={18-21}]{../src/backtrace/backtrace.ml}
      \endminipage
    \end{column}
    \begin{column}{0.5\textwidth}
      \raggedleft
      \onslide*<1>{\listCamlReraiseExn{}}
      \onslide*<2>{\listCamlReraiseExn[highlightlines=729]{}}
      \onslide*<3>{
        \mintc[firstline=190,lastline=192,highlightlines={191-192}]{../ocaml/runtime/amd64.S}
        \mintc[firstline=234,lastline=237,highlightlines={235-237}]{../ocaml/runtime/amd64.S}
        \medskip
        \listCamlReraiseExn[highlightlines=731]{}
      }
      \onslide*<4-5>{
        % TODO refactor with \listCamlReraiseExn
        \mintasm[firstline=727,lastline=734,highlightlines=730]{../ocaml/runtime/amd64.S}
        \medskip
      }
      \onslide*<4>{\listCamlRaiseExn[highlightlines=711]{}}
      \onslide*<5>{\listCamlRaiseExn[highlightlines=720]{}}
    \end{column}
  \end{columns}
\end{frame}

\begin{frame}{Backtraces}{Backtrace collection continues}
  \begin{columns}[c]
    \begin{column}{0.55\textwidth}
      \minipage[c][0.75\textheight][s]{\columnwidth}
      \begin{itemize}
        \item<1-> \funcname{caml\_stash\_backtrace} scans the older part of the stack, up to the next trap
        \item<6-> Controls jumps to the nested exception handler in \funcname{maybe\_raise}, which matches only on \typename{ExnB}
        \item<7-> \funcname{caml\_reraise\_exn} is called again, backtrace collection resumes with \funcname{caml\_stash\_backtrace}
      \end{itemize}
      \medskip
      \begin{itemize}
        \item<2-> Constructed backtrace:
        \begin{itemize}
          \item <2-> \funcname{definitely\_raise}
          \item <2-> \funcname{probably\_raise}
          \item <3-> \funcname{probably\_raise} (re-raise)
          \item <4-> \funcname{maybe\_raise}
          \item <5-> \funcname{main}
          \item <8-> \funcname{main} (re-raise)
        \end{itemize}
      \end{itemize}
      \vfill
      \onslide*<1-5>{\mintocaml[firstline=15,lastline=21]{../src/backtrace/backtrace.ml}}
      \onslide*<6>{\mintocaml[firstline=28,lastline=34,highlightlines=31]{../src/backtrace/backtrace.ml}}
      \onslide*<7->{\mintocaml[firstline=28,lastline=34,highlightlines=33]{../src/backtrace/backtrace.ml}}
      \endminipage
    \end{column}
    \begin{column}{0.45\textwidth}
      \raggedleft
      \onslide*<1>{\providecommand\step{6}%
% Backtraces
%
\subsection{One advantage of raising with debugging information}
\frameSubsection{}{}

\begin{frame}{Backtraces}{Debugging information \& source code}
  \begin{columns}[c]
    \begin{column}{0.5\textwidth}
      \begin{itemize}
        \item Raising an exception is fun and games, but how do we know where the exception originated? $\rightarrow$ With the backtrace associated with the exception!
        \item In order to get a backtrace from the point where an exception was raised, it's necessary to compile with debugging information enabled (\funcarg{-g} flag for the compiler)
        \item Same example as in the `Nested handlers' section
      \end{itemize}
    \end{column}
    \begin{column}{0.5\textwidth}
      \listocaml[firstline=9,lastline=34]{../src/backtrace/backtrace.ml}{backtrace/backtrace.ml}
    \end{column}
  \end{columns}
\end{frame}

\begin{frame}{Backtraces}{CMM and disassembly of \funcname{definitely\_raise}}
  \begin{columns}[c]
    \begin{column}{0.5\textwidth}
      \minipage[c][0.66\textheight][s]{\columnwidth}
      \begin{itemize}
        \item<1-> An effect of compiling with debugging information (\funcarg{-g}): the CMM no longer uses \funcname{raise\_notrace} to raise the exception but \funcname{raise} instead
        \item<2-> Disassembly shows that CMM \funcname{raise} is actually a call to \funcname{caml\_raise\_exn}, a function in assembler provided by the runtime
      \end{itemize}
      \vfill
      \mintocaml[firstline=9,lastline=13,highlightlines=12]{../src/backtrace/backtrace.ml}
      \endminipage
    \end{column}
    \begin{column}{0.5\textwidth}
      \onslide*<1>{
        \mintcmm[highlightlines=5]{../src/backtrace/camlBacktrace\_\_definitely\_raise\_347.cmm}
      }
      \onslide*<2>{
        \mintobjdump[firstline=2,highlightlines=14]{../src/backtrace/camlBacktrace\_\_definitely\_raise\_347.objdump}
      }
    \end{column}
  \end{columns}
\end{frame}

\begin{frame}{Backtraces}{Introducing \funcname{caml\_raise\_exn}}
  \begin{columns}[c]
    \begin{column}{0.5\textwidth}
      \minipage[c][0.66\textheight][s]{\columnwidth}
      \onslide*<1>{
        \begin{itemize}
          \item Raising the exception is performed using \funcname{caml\_raise\_exn} which is
            \begin{itemize}
              \item An assembly function provided by the runtime
              \item Architecture specific
            \end{itemize}
          \item Let's see what it does differently from \funcname{raise\_notrace}\dots
          \item We are about to raise \typename{ExnC} with \funcname{caml\_raise\_exn} from \funcname{definitely\_raise}
        \end{itemize}
      }
      \onslide*<2>{
        \mintobjdump[firstline=2,highlightlines=14]{../src/backtrace/camlBacktrace\_\_definitely\_raise\_347.objdump}
      }
      \vfill
      \mintocaml[firstline=9,lastline=13,highlightlines=12]{../src/backtrace/backtrace.ml}
      \endminipage
    \end{column}
    \begin{column}{0.5\textwidth}
      \centering
      \providecommand\step{1}\input{diagrams/backtrace/backtrace.tex}
    \end{column}
  \end{columns}
\end{frame}

\begin{frame}{Backtraces}{But before that, we need more fields from \typename{caml\_domain\_state}}
  \begin{columns}
    \begin{column}{0.5\textwidth}
      \begin{itemize}
        \item Remember \typename{caml\_domain\_state} the per-domain runtime state?
        \item Some other members are of interest to us now\dots
        \item \localname{backtrace\_active} is used to know if the runtime should collect backtraces
        \item \localname{backtrace\_active} can be set by
          \begin{itemize}
            \item The \funcarg{b} flag in the \funcname{OCAMLRUNPARAM} environment variable
            \item \mintinline{ocaml}{Printexc.record_backtrace : bool -> unit} \footnotemark
          \end{itemize}
      \end{itemize}
    \end{column}
    \begin{column}{0.5\textwidth}
      \begin{listing}
        \mintc[highlightlines={9-11}]{src/ocaml/domain\_state\_backtrace.h}
        \caption{runtime/caml/domain\_state.h}
      \end{listing}
    \end{column}
  \end{columns}
  \footnotetext[1]{https://v2.ocaml.org/api/Printexc.html}
\end{frame}

\newcommand\listCamlRaiseExn[1][]{
  \listasm[firstline=702,lastline=725,#1]{../ocaml/runtime/amd64.S}{runtime/amd64.S:702}}

\begin{frame}{Backtraces}{Walk through \funcname{caml\_raise\_exn}}
  \begin{columns}[T]
    \begin{column}{0.5\textwidth}
      \begin{itemize}
        \item<1-> First checks whether exceptions backtrace should be recorded
        \item<1-> By checking the \funcarg{domain\_state->backtrace\_active} value
        \item<2-> If not, acts as \funcname{raise\_notrace} with \funcname{RESTORE\_EXN\_HANDLER\_OCAML}
          \begin{itemize}
            \item Set \regname{RSP} to \localname{domain\_state->exn\_handler} value
            \item Restore \localname{domain\_state->exn\_handler} from trap
            \item Jump with \funcname{ret} (same effect as using \funcname{pop} and \funcname{jmp})
          \end{itemize}
        \item<3-> Otherwise, switches to the C stack to be able to execute C code
        \item<4-> Calls \funcname{caml\_stash\_backtrace}, a C function provided by the runtime\ignorespaces
          \onslide<6->{, with
            \begin{itemize}
              \item \funcarg{exn}: exception value
              \item \funcarg{pc}: code address of the raising point
              \item \funcarg{sp}: stack address of the raising function
              \item \funcarg{trapsp}: stack address of the next handler
            \end{itemize}
          }
      \end{itemize}
      \bigskip
      \mintocaml[firstline=9,lastline=13,highlightlines=12]{../src/backtrace/backtrace.ml}
    \end{column}
    \begin{column}{0.5\textwidth}
      \raggedleft
      \onslide*<1,3-4>{
        \mintc[firstline=190,lastline=192]{../ocaml/runtime/amd64.S}
        \mintc[firstline=234,lastline=237]{../ocaml/runtime/amd64.S}
      }
      \onslide*<2>{
        \mintc[firstline=190,lastline=192,highlightlines={191-192}]{../ocaml/runtime/amd64.S}
        \mintc[firstline=234,lastline=237,highlightlines={235-237}]{../ocaml/runtime/amd64.S}
      }
      \onslide*<1>{\listCamlRaiseExn[highlightlines=705]{}}%
      \onslide*<2>{\listCamlRaiseExn[highlightlines={707-708}]{}}%
      \onslide*<3>{\listCamlRaiseExn[highlightlines={712-714}]{}}%
      \onslide*<4>{\listCamlRaiseExn[highlightlines={715-720}]{}}%
      \onslide*<5>{\providecommand\step{1}\input{diagrams/backtrace/backtrace.tex}}%
      \onslide*<6>{\providecommand\step{2}\input{diagrams/backtrace/backtrace.tex}}%
    \end{column}
  \end{columns}
\end{frame}

\newcommand\listStashBacktrace[1][]{
  \listc[firstline=91,lastline=120,#1]{../ocaml/runtime/backtrace\_nat.c}{runtime/backtrace\_nat.c:91}}

\begin{frame}{Backtraces}{Introducing \funcname{caml\_stash\_backtrace}}
  \begin{columns}[c]
    \begin{column}{0.5\textwidth}
      \minipage[c][0.66\textheight][s]{\columnwidth}
      \begin{itemize}
        \item<1-> A C function provided by the runtime (specific to native code)
        \item<2-> It scans the stack, between the stack pointer at the raising point up to the next trap, in order to collect the names of the detected function frames
          \onslide*<2>{
            \begin{itemize}
              \item And more information, like line number, column number...
            \end{itemize}
          }
        \item<3-> It uses the same technique and information as the Garbage Collector (GC) uses to scan the values live on the stack
          \onslide*<4>{
            \begin{itemize}
              \item \typename{frame\_descr} are at the core of stack unwinding
            \end{itemize}
          }
      \end{itemize}
      \vfill
      \mintocaml[firstline=9,lastline=13,highlightlines=12]{../src/backtrace/backtrace.ml}
      \endminipage
    \end{column}
    \begin{column}{0.5\textwidth}
      \onslide*<1>{\listStashBacktrace[highlightlines=91]{}}
      \onslide*<2>{\listStashBacktrace[highlightlines={91,117-118}]{}}
      \onslide*<3->{\listStashBacktrace[highlightlines={94,106,109-110}]{}}
    \end{column}
  \end{columns}
\end{frame}

\newcommand\listFrameDescr[1][]{
  \listocaml[firstline=111,lastline=124,#1]{../ocaml/asmcomp/emitaux.ml}{asmcomp/emitaux.ml:111}}
\newcommand\listDebugInfo[1][]{
  \listocaml[firstline=38,lastline=49,#1]{../ocaml/lambda/debuginfo.mli}{lambda/debuginfo.mli:38}}

\begin{frame}{Backtraces}{\typename{frame\_descr} and \typename{frame\_debuginfo} in a nutshell}
  \begin{columns}[c]
    \begin{column}{0.5\textwidth}
      \begin{itemize}
        \item<1-> For every return address in an OCaml program, there is a associated \typename{frame\_descr}
        \item<2-> A \typename{frame\_descr} specifies
        \begin{itemize}
          \item<2-> The size of the function stack frame
          \item<3-> Where live values are on the stack (so that the GC can mark them as live and prevent from being swept)
        \end{itemize}
        \item<4-> When compiled with debug information, \typename{frame\_descr} will be linked to a \typename{frame\_debuginfo}
        \begin{itemize}
          \item<5-> Contains information about the position in the source code (file name, line and column number)
        \end{itemize}
      \end{itemize}
    \end{column}
    \begin{column}{0.5\textwidth}
        \onslide*<1>{\listFrameDescr[highlightlines={118-122}]{}}
        \onslide*<2>{\listFrameDescr[highlightlines={120}]{}}
        \onslide*<3>{\listFrameDescr[highlightlines={121}]{}}
        \onslide*<4->{\listFrameDescr[highlightlines={122}]{}}
        \medskip
        \onslide*<1-3>{\listDebugInfo{}}
        \onslide*<4>{\listDebugInfo[highlightlines={38,49}]{}}
        \onslide*<5>{\listDebugInfo[highlightlines={39-46}]{}}
    \end{column}
  \end{columns}
\end{frame}

\begin{frame}{Backtraces}{Scanning the stack with \typename{frame\_descr}}
  \begin{columns}[c]
    \begin{column}{0.5\textwidth}
      \begin{itemize}
        \item Given the return address of the code that raises the exception by calling \funcname{caml\_raise\_exn} (\funcarg{pc} argument of \funcname{caml\_stash\_backtrace})
        \item And the position of the stack pointer (\funcarg{sp} argument of \funcname{caml\_stash\_backtrace})
        \item The frame size given by \typename{frame\_descr}.\funcarg{frame\_size}, allows to find the end of the previous frame
        \item And the return address of the caller of this function
        \bigskip
        \onslide<3->{
        \item Note that traps (here the reddish \funcname{ExnA}, \funcname{ExnB} and \funcname{ExnC} blocks) belong to the frame that installed them, and so the frame size changes when traps are removed
        }
      \end{itemize}
    \end{column}
    \begin{column}{0.5\textwidth}
      \raggedleft
      \onslide*<1>{\providecommand\step{2}\input{diagrams/backtrace/backtrace.tex}}%
      \onslide*<2>{\providecommand\step{1}\input{diagrams/backtrace/scanstack.tex}}%
      \onslide*<3>{\providecommand\step{2}\input{diagrams/backtrace/scanstack.tex}}%
      \onslide*<4>{\providecommand\step{3}\input{diagrams/backtrace/scanstack.tex}}%
      \onslide*<5>{\providecommand\step{4}\input{diagrams/backtrace/scanstack.tex}}%
      \onslide*<6>{\providecommand\step{5}\input{diagrams/backtrace/scanstack.tex}}%
    \end{column}
  \end{columns}
\end{frame}

% Duplicate of previous-previous slide
\begin{frame}{Backtraces}{Continuing on \funcname{caml\_stash\_backtrace}}
  \begin{columns}[c]
    \begin{column}{0.5\textwidth}
      \minipage[c][0.75\textheight][s]{\columnwidth}
      \begin{itemize}
          \color{gray}
        \item A C function provided by the runtime (specific to native code)
        \item It scans the stack, between the stack pointer at the raising point up to the next trap, in order to collect the names of the detected function frames
        \item It uses the same technique and information as the Garbage Collector (GC) uses to scan the values live on the stack
      \end{itemize}
      \begin{itemize}
        \item For each function frame detected, store a pointer to the \typename{frame\_descr} in the \localname{domain\_state->backtrace\_buffer} array, effectively constructing the backtrace
      \end{itemize}
      \onslide<2->{
        \begin{itemize}
            \smallskip
          \item Constructed backtrace:
            \funcname{definitely\_raise},
            \onslide<3->{\funcname{probably\_raise}}
        \end{itemize}
      }
      \vfill
      \mintocaml[firstline=9,lastline=13,highlightlines=12]{../src/backtrace/backtrace.ml}
      \endminipage
    \end{column}
    \begin{column}{0.5\textwidth}
      \raggedleft
      \onslide*<1>{\listStashBacktrace[highlightlines={114-115}]{}}
      \onslide*<2>{\providecommand\step{3}\input{diagrams/backtrace/backtrace.tex}}%
      \onslide*<3>{\providecommand\step{4}\input{diagrams/backtrace/backtrace.tex}}%
    \end{column}
  \end{columns}
\end{frame}

\begin{frame}{Backtraces}{Back to \funcname{caml\_raise\_exn}}
  \begin{columns}[c]
    \begin{column}{0.5\textwidth}
      \minipage[c][0.66\textheight][s]{\columnwidth}
      \begin{itemize}\color{gray}
        \item Checks wheteher exceptions backtrace should be recorded, by checking the \funcarg{domain\_state->backtrace\_active} value
        \item Switches to the C stack to be able to execute C code
        \item Calls \funcname{caml\_stash\_backtrace}, to collect the backtrace up to the next trap
      \end{itemize}
      \onslide*<2->{
        \begin{itemize}
          \item<2-> Executes the exception handler in \funcname{probably\_raise} with \funcname{RESTORE\_EXN\_HANDLER\_OCAML}
            \begin{itemize}
              \item Set \regname{RSP} to \localname{domain\_state->exn\_handler} value
              \item Restore \localname{domain\_state->exn\_handler} from trap
              \item Jump to the handler code with \funcname{ret}
            \end{itemize}
        \end{itemize}
      }
      \vfill
      \onslide*<1>{\mintocaml[firstline=9,lastline=13,highlightlines=12]{../src/backtrace/backtrace.ml}}
      \onslide*<2->{\mintocaml[firstline=15,lastline=21,highlightlines={19-21}]{../src/backtrace/backtrace.ml}}
      \endminipage
    \end{column}
    \begin{column}{0.5\textwidth}
      \centering
      \onslide*<1>{
        \mintc[firstline=190,lastline=192]{../ocaml/runtime/amd64.S}
        \mintc[firstline=234,lastline=237]{../ocaml/runtime/amd64.S}
      }
      \onslide*<2>{
        \mintc[firstline=190,lastline=192,highlightlines={191-192}]{../ocaml/runtime/amd64.S}
        \mintc[firstline=234,lastline=237,highlightlines={235-237}]{../ocaml/runtime/amd64.S}
      }
      \onslide*<1>{\listCamlRaiseExn[highlightlines=720]{}}
      \onslide*<2>{\listCamlRaiseExn[highlightlines=722]{}}
      \onslide*<3->{\providecommand\step{5}\input{diagrams/backtrace/backtrace.tex}}
    \end{column}
  \end{columns}
\end{frame}

\begin{frame}{Backtraces}{\funcname{caml\_probably\_raise}'s handler doesn't catch \typename{ExnC}}
  \begin{columns}[c]
    \begin{column}{0.45\textwidth}
      \minipage[c][0.75\textheight][s]{\columnwidth}
      \begin{itemize}
        \item<1-> \funcname{match \dots with}\footnotemark pattern matching can also match exceptions
        \item<1-> The CMM of \funcname{probably\_raise} shows that this \funcname{match \dots with} is implemented with a pattern matching and a \funcname{try \dots with}
        \item<1-> But this one only matches on \typename{ExnA} exception
        \item<2-> Since the pattern matching fails to catch the exception, \funcname{reraise} is called with our current \typename{ExnC} exception
        \item<3-> Disassembly shows that \funcname{reraise} is actually a call to \funcname{caml\_reraise\_exn}, which is also an assembly function provided by the runtime
      \end{itemize}
      \vfill
      \mintocaml[firstline=15,lastline=21,highlightlines={18-21}]{../src/backtrace/backtrace.ml}
      \endminipage
    \end{column}
    \begin{column}{0.55\textwidth}
      \centering
      \onslide*<1>{\mintcmm[highlightlines={10-18}]{../src/backtrace/camlBacktrace\_\_probably\_raise\_351.cmm}}
      \onslide*<2>{\listcmm[highlightlines=18]{../src/backtrace/camlBacktrace\_\_probably\_raise_351.cmm}{CMM of probably\_raise}}
      \onslide*<3>{\mintobjdump[firstline=5,lastline=35,highlightlines=31]{../src/backtrace/camlBacktrace\_\_probably\_raise_351.objdump}}
    \end{column}
  \end{columns}
  \only<1>{\footnotetext[1]{https://blog.janestreet.com/pattern-matching-and-exception-handling-unite/}}
\end{frame}

\newcommand\listCamlReraiseExn[1][]{
  \listasm[firstline=727,lastline=734,#1]{../ocaml/runtime/amd64.S}{runtime/amd64.S:727}}

\begin{frame}{Backtraces}{Introducing \funcname{caml\_reraise\_exn}}
  \begin{columns}[c]
    \begin{column}{0.5\textwidth}
      \minipage[c][0.66\textheight][s]{\columnwidth}
      \begin{itemize}
        \item<1-> The exception have to be reraised to the next handler
        \item<2-> First, checks if the backtrace should be collected with \funcarg{domain\_state->backtrace\_active}
        \only<3>{
        \begin{itemize}
            \item If not, behaves like a \funcname{raise\_notrace} with \funcname{RESTORE\_EXN\_HANDLER\_OCAML}
        \end{itemize}
        }
        \item<4-> Jumps into \funcname{caml\_raise\_exn}
        \item<5-> Calls \funcname{caml\_stash\_backtrace} again, to scan the next part of the stack: from the trap just pop-ed to the next trap
      \end{itemize}
      \vfill
      \mintocaml[firstline=15,lastline=21,highlightlines={18-21}]{../src/backtrace/backtrace.ml}
      \endminipage
    \end{column}
    \begin{column}{0.5\textwidth}
      \raggedleft
      \onslide*<1>{\listCamlReraiseExn{}}
      \onslide*<2>{\listCamlReraiseExn[highlightlines=729]{}}
      \onslide*<3>{
        \mintc[firstline=190,lastline=192,highlightlines={191-192}]{../ocaml/runtime/amd64.S}
        \mintc[firstline=234,lastline=237,highlightlines={235-237}]{../ocaml/runtime/amd64.S}
        \medskip
        \listCamlReraiseExn[highlightlines=731]{}
      }
      \onslide*<4-5>{
        % TODO refactor with \listCamlReraiseExn
        \mintasm[firstline=727,lastline=734,highlightlines=730]{../ocaml/runtime/amd64.S}
        \medskip
      }
      \onslide*<4>{\listCamlRaiseExn[highlightlines=711]{}}
      \onslide*<5>{\listCamlRaiseExn[highlightlines=720]{}}
    \end{column}
  \end{columns}
\end{frame}

\begin{frame}{Backtraces}{Backtrace collection continues}
  \begin{columns}[c]
    \begin{column}{0.55\textwidth}
      \minipage[c][0.75\textheight][s]{\columnwidth}
      \begin{itemize}
        \item<1-> \funcname{caml\_stash\_backtrace} scans the older part of the stack, up to the next trap
        \item<6-> Controls jumps to the nested exception handler in \funcname{maybe\_raise}, which matches only on \typename{ExnB}
        \item<7-> \funcname{caml\_reraise\_exn} is called again, backtrace collection resumes with \funcname{caml\_stash\_backtrace}
      \end{itemize}
      \medskip
      \begin{itemize}
        \item<2-> Constructed backtrace:
        \begin{itemize}
          \item <2-> \funcname{definitely\_raise}
          \item <2-> \funcname{probably\_raise}
          \item <3-> \funcname{probably\_raise} (re-raise)
          \item <4-> \funcname{maybe\_raise}
          \item <5-> \funcname{main}
          \item <8-> \funcname{main} (re-raise)
        \end{itemize}
      \end{itemize}
      \vfill
      \onslide*<1-5>{\mintocaml[firstline=15,lastline=21]{../src/backtrace/backtrace.ml}}
      \onslide*<6>{\mintocaml[firstline=28,lastline=34,highlightlines=31]{../src/backtrace/backtrace.ml}}
      \onslide*<7->{\mintocaml[firstline=28,lastline=34,highlightlines=33]{../src/backtrace/backtrace.ml}}
      \endminipage
    \end{column}
    \begin{column}{0.45\textwidth}
      \raggedleft
      \onslide*<1>{\providecommand\step{6}\input{diagrams/backtrace/backtrace.tex}}%
      \onslide*<2>{\providecommand\step{6}\input{diagrams/backtrace/backtrace.tex}}%
      \onslide*<3>{\providecommand\step{7}\input{diagrams/backtrace/backtrace.tex}}%
      \onslide*<4>{\providecommand\step{8}\input{diagrams/backtrace/backtrace.tex}}%
      \onslide*<5>{\providecommand\step{9}\input{diagrams/backtrace/backtrace.tex}}%
      \onslide*<6>{\providecommand\step{10}\input{diagrams/backtrace/backtrace.tex}}%
      \onslide*<7>{\providecommand\step{11}\input{diagrams/backtrace/backtrace.tex}}%
      \onslide*<8>{\providecommand\step{12}\input{diagrams/backtrace/backtrace.tex}}%
      \onslide*<9>{\providecommand\step{13}\input{diagrams/backtrace/backtrace.tex}}%
    \end{column}
  \end{columns}
\end{frame}

\begin{frame}{Backtraces}{Printing the backtrace}
  \begin{columns}[c]
    \begin{column}{0.45\textwidth}
      \minipage[c][0.66\textheight][s]{\columnwidth}
      \begin{itemize}
        \item<1-> The exception backtrace can now be printed to \funcarg{stdout} with \funcname{Printexc.print_backtrace}
        \item<2-> First the exception backtrace is retrieved into an OCaml array with \funcname{get_raw_backtrace }
        \item<3-> Which is implemented as the \funcname{caml_get_exception_raw_backtrace} C function in the runtime
        \item<4-> And finally printed with \funcname{print_exception_backtrace}
      \end{itemize}
      \vfill
      \mintocaml[firstline=28,lastline=34,highlightlines=33]{../src/backtrace/backtrace.ml}
      \endminipage
    \end{column}
    \begin{column}{0.55\textwidth}
      \onslide*<1,2>{
        \mintocaml[firstline=100,lastline=101]{../ocaml/stdlib/printexc.ml}
      }
      \only<1>{
        \listocaml[firstline=170,lastline=172]{../ocaml/stdlib/printexc.ml}{stdlib/printexc.ml:170}
      }
      \only<2>{
        \listocaml[firstline=170,lastline=172,highlightlines=172]{../ocaml/stdlib/printexc.ml}{stdlib/printexc.ml:170}
      }
      \only<3>{
        \mintc[firstline=154,lastline=156]{../ocaml/runtime/backtrace.c}
        \listc[firstline=171,lastline=192,highlightlines={184-187}]{../ocaml/runtime/backtrace.c}{runtime/backtrace.c:154}
      }
      \only<4>{
        \listocaml[firstline=155,lastline=165]{../ocaml/stdlib/printexc.ml}{stdlib/printexc.ml:55}
      }
    \end{column}
  \end{columns}
\end{frame}


\subsection{The multiple kinds of raise}
\frameSubsection{}{}

\begin{frame}{Backtraces}{When backtraces are not needed}
  \begin{columns}[c]
    \begin{column}{0.45\textwidth}
      \minipage[c][0.66\textheight][s]{\columnwidth}
      \begin{itemize}
        \item<1-> What if the backtrace for an exception is never needed?
        \item<2-> It's possible to raise the exception explicitly with \funcname{raise\_notrace} to make raising as cheap as possible
        \item<3-> An example of use in the \funcname{dynlink} library of the OCaml compiler
      \end{itemize}
      \vfill
      \onslide<3->{
      \listocaml[firstline=205,lastline=209,highlightlines=207]{../ocaml/otherlibs/dynlink/dynlink\_compilerlibs/misc.ml}{otherlibs/dynlink/dynlink\_compilerlibs/misc.ml:205}
      }
      \endminipage
    \end{column}
    \begin{column}{0.55\textwidth}
    \only<4>{
      \mintcmm[highlightlines=3]{../src/notrace/camlNotrace\_\_fun_347.cmm}
      \listcmm{../src/notrace/camlNotrace\_\_all\_somes\_327.cmm}{all\_somes}
    }
    \onslide*<5->{
      \listobjdump[highlightlines={6-9}]{../src/notrace/camlNotrace\_\_fun_347.objdump}{anonymous function from \funcname{all\_somes}}
    }
    \end{column}
  \end{columns}
\end{frame}

\begin{frame}{Backtraces}{Summary table of the multiple kinds of raise}
  \begin{itemize}
    \item When debugging information is enabled and if the backtrace of an exception isn't needed, it's possible to raise with \funcname{raise\_notrace} for performance
    \item When debugging information is \emph{not} enabled, the compiler optimises \funcname{raise} calls to inline assembly (same effect as using \funcname{raise\_notrace})
  \end{itemize}
  \bigskip
  \centering
  \begin{tabular}{| l | c | c | c | c |}
    \hline
    \parbox{10em}{Debug information \\ (\funcarg{-g} flag)}  & \multicolumn{2}{|c|}{Disabled}                                     & \multicolumn{2}{|c|}{Enabled} \\
    \hline
    Raise kind                                               & \funcname{raise} & \funcname{raise\_notrace}                       & \funcname{raise} & \funcname{raise\_notrace} \\
    \hline
    Compilation                                              & \parbox{8em}{inline assembly\\ (optimization)} & inline assembly   & \funcname{caml\_raise\_exn} & inline assembly \\
    \hline
  \end{tabular}
\end{frame}


%
% Takeaway #5
%
\subsection*{Takeaway \#5}
\frameSubsectionTakeaway{}
\againframe<5>{takeaway}
}%
      \onslide*<2>{\providecommand\step{6}%
% Backtraces
%
\subsection{One advantage of raising with debugging information}
\frameSubsection{}{}

\begin{frame}{Backtraces}{Debugging information \& source code}
  \begin{columns}[c]
    \begin{column}{0.5\textwidth}
      \begin{itemize}
        \item Raising an exception is fun and games, but how do we know where the exception originated? $\rightarrow$ With the backtrace associated with the exception!
        \item In order to get a backtrace from the point where an exception was raised, it's necessary to compile with debugging information enabled (\funcarg{-g} flag for the compiler)
        \item Same example as in the `Nested handlers' section
      \end{itemize}
    \end{column}
    \begin{column}{0.5\textwidth}
      \listocaml[firstline=9,lastline=34]{../src/backtrace/backtrace.ml}{backtrace/backtrace.ml}
    \end{column}
  \end{columns}
\end{frame}

\begin{frame}{Backtraces}{CMM and disassembly of \funcname{definitely\_raise}}
  \begin{columns}[c]
    \begin{column}{0.5\textwidth}
      \minipage[c][0.66\textheight][s]{\columnwidth}
      \begin{itemize}
        \item<1-> An effect of compiling with debugging information (\funcarg{-g}): the CMM no longer uses \funcname{raise\_notrace} to raise the exception but \funcname{raise} instead
        \item<2-> Disassembly shows that CMM \funcname{raise} is actually a call to \funcname{caml\_raise\_exn}, a function in assembler provided by the runtime
      \end{itemize}
      \vfill
      \mintocaml[firstline=9,lastline=13,highlightlines=12]{../src/backtrace/backtrace.ml}
      \endminipage
    \end{column}
    \begin{column}{0.5\textwidth}
      \onslide*<1>{
        \mintcmm[highlightlines=5]{../src/backtrace/camlBacktrace\_\_definitely\_raise\_347.cmm}
      }
      \onslide*<2>{
        \mintobjdump[firstline=2,highlightlines=14]{../src/backtrace/camlBacktrace\_\_definitely\_raise\_347.objdump}
      }
    \end{column}
  \end{columns}
\end{frame}

\begin{frame}{Backtraces}{Introducing \funcname{caml\_raise\_exn}}
  \begin{columns}[c]
    \begin{column}{0.5\textwidth}
      \minipage[c][0.66\textheight][s]{\columnwidth}
      \onslide*<1>{
        \begin{itemize}
          \item Raising the exception is performed using \funcname{caml\_raise\_exn} which is
            \begin{itemize}
              \item An assembly function provided by the runtime
              \item Architecture specific
            \end{itemize}
          \item Let's see what it does differently from \funcname{raise\_notrace}\dots
          \item We are about to raise \typename{ExnC} with \funcname{caml\_raise\_exn} from \funcname{definitely\_raise}
        \end{itemize}
      }
      \onslide*<2>{
        \mintobjdump[firstline=2,highlightlines=14]{../src/backtrace/camlBacktrace\_\_definitely\_raise\_347.objdump}
      }
      \vfill
      \mintocaml[firstline=9,lastline=13,highlightlines=12]{../src/backtrace/backtrace.ml}
      \endminipage
    \end{column}
    \begin{column}{0.5\textwidth}
      \centering
      \providecommand\step{1}\input{diagrams/backtrace/backtrace.tex}
    \end{column}
  \end{columns}
\end{frame}

\begin{frame}{Backtraces}{But before that, we need more fields from \typename{caml\_domain\_state}}
  \begin{columns}
    \begin{column}{0.5\textwidth}
      \begin{itemize}
        \item Remember \typename{caml\_domain\_state} the per-domain runtime state?
        \item Some other members are of interest to us now\dots
        \item \localname{backtrace\_active} is used to know if the runtime should collect backtraces
        \item \localname{backtrace\_active} can be set by
          \begin{itemize}
            \item The \funcarg{b} flag in the \funcname{OCAMLRUNPARAM} environment variable
            \item \mintinline{ocaml}{Printexc.record_backtrace : bool -> unit} \footnotemark
          \end{itemize}
      \end{itemize}
    \end{column}
    \begin{column}{0.5\textwidth}
      \begin{listing}
        \mintc[highlightlines={9-11}]{src/ocaml/domain\_state\_backtrace.h}
        \caption{runtime/caml/domain\_state.h}
      \end{listing}
    \end{column}
  \end{columns}
  \footnotetext[1]{https://v2.ocaml.org/api/Printexc.html}
\end{frame}

\newcommand\listCamlRaiseExn[1][]{
  \listasm[firstline=702,lastline=725,#1]{../ocaml/runtime/amd64.S}{runtime/amd64.S:702}}

\begin{frame}{Backtraces}{Walk through \funcname{caml\_raise\_exn}}
  \begin{columns}[T]
    \begin{column}{0.5\textwidth}
      \begin{itemize}
        \item<1-> First checks whether exceptions backtrace should be recorded
        \item<1-> By checking the \funcarg{domain\_state->backtrace\_active} value
        \item<2-> If not, acts as \funcname{raise\_notrace} with \funcname{RESTORE\_EXN\_HANDLER\_OCAML}
          \begin{itemize}
            \item Set \regname{RSP} to \localname{domain\_state->exn\_handler} value
            \item Restore \localname{domain\_state->exn\_handler} from trap
            \item Jump with \funcname{ret} (same effect as using \funcname{pop} and \funcname{jmp})
          \end{itemize}
        \item<3-> Otherwise, switches to the C stack to be able to execute C code
        \item<4-> Calls \funcname{caml\_stash\_backtrace}, a C function provided by the runtime\ignorespaces
          \onslide<6->{, with
            \begin{itemize}
              \item \funcarg{exn}: exception value
              \item \funcarg{pc}: code address of the raising point
              \item \funcarg{sp}: stack address of the raising function
              \item \funcarg{trapsp}: stack address of the next handler
            \end{itemize}
          }
      \end{itemize}
      \bigskip
      \mintocaml[firstline=9,lastline=13,highlightlines=12]{../src/backtrace/backtrace.ml}
    \end{column}
    \begin{column}{0.5\textwidth}
      \raggedleft
      \onslide*<1,3-4>{
        \mintc[firstline=190,lastline=192]{../ocaml/runtime/amd64.S}
        \mintc[firstline=234,lastline=237]{../ocaml/runtime/amd64.S}
      }
      \onslide*<2>{
        \mintc[firstline=190,lastline=192,highlightlines={191-192}]{../ocaml/runtime/amd64.S}
        \mintc[firstline=234,lastline=237,highlightlines={235-237}]{../ocaml/runtime/amd64.S}
      }
      \onslide*<1>{\listCamlRaiseExn[highlightlines=705]{}}%
      \onslide*<2>{\listCamlRaiseExn[highlightlines={707-708}]{}}%
      \onslide*<3>{\listCamlRaiseExn[highlightlines={712-714}]{}}%
      \onslide*<4>{\listCamlRaiseExn[highlightlines={715-720}]{}}%
      \onslide*<5>{\providecommand\step{1}\input{diagrams/backtrace/backtrace.tex}}%
      \onslide*<6>{\providecommand\step{2}\input{diagrams/backtrace/backtrace.tex}}%
    \end{column}
  \end{columns}
\end{frame}

\newcommand\listStashBacktrace[1][]{
  \listc[firstline=91,lastline=120,#1]{../ocaml/runtime/backtrace\_nat.c}{runtime/backtrace\_nat.c:91}}

\begin{frame}{Backtraces}{Introducing \funcname{caml\_stash\_backtrace}}
  \begin{columns}[c]
    \begin{column}{0.5\textwidth}
      \minipage[c][0.66\textheight][s]{\columnwidth}
      \begin{itemize}
        \item<1-> A C function provided by the runtime (specific to native code)
        \item<2-> It scans the stack, between the stack pointer at the raising point up to the next trap, in order to collect the names of the detected function frames
          \onslide*<2>{
            \begin{itemize}
              \item And more information, like line number, column number...
            \end{itemize}
          }
        \item<3-> It uses the same technique and information as the Garbage Collector (GC) uses to scan the values live on the stack
          \onslide*<4>{
            \begin{itemize}
              \item \typename{frame\_descr} are at the core of stack unwinding
            \end{itemize}
          }
      \end{itemize}
      \vfill
      \mintocaml[firstline=9,lastline=13,highlightlines=12]{../src/backtrace/backtrace.ml}
      \endminipage
    \end{column}
    \begin{column}{0.5\textwidth}
      \onslide*<1>{\listStashBacktrace[highlightlines=91]{}}
      \onslide*<2>{\listStashBacktrace[highlightlines={91,117-118}]{}}
      \onslide*<3->{\listStashBacktrace[highlightlines={94,106,109-110}]{}}
    \end{column}
  \end{columns}
\end{frame}

\newcommand\listFrameDescr[1][]{
  \listocaml[firstline=111,lastline=124,#1]{../ocaml/asmcomp/emitaux.ml}{asmcomp/emitaux.ml:111}}
\newcommand\listDebugInfo[1][]{
  \listocaml[firstline=38,lastline=49,#1]{../ocaml/lambda/debuginfo.mli}{lambda/debuginfo.mli:38}}

\begin{frame}{Backtraces}{\typename{frame\_descr} and \typename{frame\_debuginfo} in a nutshell}
  \begin{columns}[c]
    \begin{column}{0.5\textwidth}
      \begin{itemize}
        \item<1-> For every return address in an OCaml program, there is a associated \typename{frame\_descr}
        \item<2-> A \typename{frame\_descr} specifies
        \begin{itemize}
          \item<2-> The size of the function stack frame
          \item<3-> Where live values are on the stack (so that the GC can mark them as live and prevent from being swept)
        \end{itemize}
        \item<4-> When compiled with debug information, \typename{frame\_descr} will be linked to a \typename{frame\_debuginfo}
        \begin{itemize}
          \item<5-> Contains information about the position in the source code (file name, line and column number)
        \end{itemize}
      \end{itemize}
    \end{column}
    \begin{column}{0.5\textwidth}
        \onslide*<1>{\listFrameDescr[highlightlines={118-122}]{}}
        \onslide*<2>{\listFrameDescr[highlightlines={120}]{}}
        \onslide*<3>{\listFrameDescr[highlightlines={121}]{}}
        \onslide*<4->{\listFrameDescr[highlightlines={122}]{}}
        \medskip
        \onslide*<1-3>{\listDebugInfo{}}
        \onslide*<4>{\listDebugInfo[highlightlines={38,49}]{}}
        \onslide*<5>{\listDebugInfo[highlightlines={39-46}]{}}
    \end{column}
  \end{columns}
\end{frame}

\begin{frame}{Backtraces}{Scanning the stack with \typename{frame\_descr}}
  \begin{columns}[c]
    \begin{column}{0.5\textwidth}
      \begin{itemize}
        \item Given the return address of the code that raises the exception by calling \funcname{caml\_raise\_exn} (\funcarg{pc} argument of \funcname{caml\_stash\_backtrace})
        \item And the position of the stack pointer (\funcarg{sp} argument of \funcname{caml\_stash\_backtrace})
        \item The frame size given by \typename{frame\_descr}.\funcarg{frame\_size}, allows to find the end of the previous frame
        \item And the return address of the caller of this function
        \bigskip
        \onslide<3->{
        \item Note that traps (here the reddish \funcname{ExnA}, \funcname{ExnB} and \funcname{ExnC} blocks) belong to the frame that installed them, and so the frame size changes when traps are removed
        }
      \end{itemize}
    \end{column}
    \begin{column}{0.5\textwidth}
      \raggedleft
      \onslide*<1>{\providecommand\step{2}\input{diagrams/backtrace/backtrace.tex}}%
      \onslide*<2>{\providecommand\step{1}\input{diagrams/backtrace/scanstack.tex}}%
      \onslide*<3>{\providecommand\step{2}\input{diagrams/backtrace/scanstack.tex}}%
      \onslide*<4>{\providecommand\step{3}\input{diagrams/backtrace/scanstack.tex}}%
      \onslide*<5>{\providecommand\step{4}\input{diagrams/backtrace/scanstack.tex}}%
      \onslide*<6>{\providecommand\step{5}\input{diagrams/backtrace/scanstack.tex}}%
    \end{column}
  \end{columns}
\end{frame}

% Duplicate of previous-previous slide
\begin{frame}{Backtraces}{Continuing on \funcname{caml\_stash\_backtrace}}
  \begin{columns}[c]
    \begin{column}{0.5\textwidth}
      \minipage[c][0.75\textheight][s]{\columnwidth}
      \begin{itemize}
          \color{gray}
        \item A C function provided by the runtime (specific to native code)
        \item It scans the stack, between the stack pointer at the raising point up to the next trap, in order to collect the names of the detected function frames
        \item It uses the same technique and information as the Garbage Collector (GC) uses to scan the values live on the stack
      \end{itemize}
      \begin{itemize}
        \item For each function frame detected, store a pointer to the \typename{frame\_descr} in the \localname{domain\_state->backtrace\_buffer} array, effectively constructing the backtrace
      \end{itemize}
      \onslide<2->{
        \begin{itemize}
            \smallskip
          \item Constructed backtrace:
            \funcname{definitely\_raise},
            \onslide<3->{\funcname{probably\_raise}}
        \end{itemize}
      }
      \vfill
      \mintocaml[firstline=9,lastline=13,highlightlines=12]{../src/backtrace/backtrace.ml}
      \endminipage
    \end{column}
    \begin{column}{0.5\textwidth}
      \raggedleft
      \onslide*<1>{\listStashBacktrace[highlightlines={114-115}]{}}
      \onslide*<2>{\providecommand\step{3}\input{diagrams/backtrace/backtrace.tex}}%
      \onslide*<3>{\providecommand\step{4}\input{diagrams/backtrace/backtrace.tex}}%
    \end{column}
  \end{columns}
\end{frame}

\begin{frame}{Backtraces}{Back to \funcname{caml\_raise\_exn}}
  \begin{columns}[c]
    \begin{column}{0.5\textwidth}
      \minipage[c][0.66\textheight][s]{\columnwidth}
      \begin{itemize}\color{gray}
        \item Checks wheteher exceptions backtrace should be recorded, by checking the \funcarg{domain\_state->backtrace\_active} value
        \item Switches to the C stack to be able to execute C code
        \item Calls \funcname{caml\_stash\_backtrace}, to collect the backtrace up to the next trap
      \end{itemize}
      \onslide*<2->{
        \begin{itemize}
          \item<2-> Executes the exception handler in \funcname{probably\_raise} with \funcname{RESTORE\_EXN\_HANDLER\_OCAML}
            \begin{itemize}
              \item Set \regname{RSP} to \localname{domain\_state->exn\_handler} value
              \item Restore \localname{domain\_state->exn\_handler} from trap
              \item Jump to the handler code with \funcname{ret}
            \end{itemize}
        \end{itemize}
      }
      \vfill
      \onslide*<1>{\mintocaml[firstline=9,lastline=13,highlightlines=12]{../src/backtrace/backtrace.ml}}
      \onslide*<2->{\mintocaml[firstline=15,lastline=21,highlightlines={19-21}]{../src/backtrace/backtrace.ml}}
      \endminipage
    \end{column}
    \begin{column}{0.5\textwidth}
      \centering
      \onslide*<1>{
        \mintc[firstline=190,lastline=192]{../ocaml/runtime/amd64.S}
        \mintc[firstline=234,lastline=237]{../ocaml/runtime/amd64.S}
      }
      \onslide*<2>{
        \mintc[firstline=190,lastline=192,highlightlines={191-192}]{../ocaml/runtime/amd64.S}
        \mintc[firstline=234,lastline=237,highlightlines={235-237}]{../ocaml/runtime/amd64.S}
      }
      \onslide*<1>{\listCamlRaiseExn[highlightlines=720]{}}
      \onslide*<2>{\listCamlRaiseExn[highlightlines=722]{}}
      \onslide*<3->{\providecommand\step{5}\input{diagrams/backtrace/backtrace.tex}}
    \end{column}
  \end{columns}
\end{frame}

\begin{frame}{Backtraces}{\funcname{caml\_probably\_raise}'s handler doesn't catch \typename{ExnC}}
  \begin{columns}[c]
    \begin{column}{0.45\textwidth}
      \minipage[c][0.75\textheight][s]{\columnwidth}
      \begin{itemize}
        \item<1-> \funcname{match \dots with}\footnotemark pattern matching can also match exceptions
        \item<1-> The CMM of \funcname{probably\_raise} shows that this \funcname{match \dots with} is implemented with a pattern matching and a \funcname{try \dots with}
        \item<1-> But this one only matches on \typename{ExnA} exception
        \item<2-> Since the pattern matching fails to catch the exception, \funcname{reraise} is called with our current \typename{ExnC} exception
        \item<3-> Disassembly shows that \funcname{reraise} is actually a call to \funcname{caml\_reraise\_exn}, which is also an assembly function provided by the runtime
      \end{itemize}
      \vfill
      \mintocaml[firstline=15,lastline=21,highlightlines={18-21}]{../src/backtrace/backtrace.ml}
      \endminipage
    \end{column}
    \begin{column}{0.55\textwidth}
      \centering
      \onslide*<1>{\mintcmm[highlightlines={10-18}]{../src/backtrace/camlBacktrace\_\_probably\_raise\_351.cmm}}
      \onslide*<2>{\listcmm[highlightlines=18]{../src/backtrace/camlBacktrace\_\_probably\_raise_351.cmm}{CMM of probably\_raise}}
      \onslide*<3>{\mintobjdump[firstline=5,lastline=35,highlightlines=31]{../src/backtrace/camlBacktrace\_\_probably\_raise_351.objdump}}
    \end{column}
  \end{columns}
  \only<1>{\footnotetext[1]{https://blog.janestreet.com/pattern-matching-and-exception-handling-unite/}}
\end{frame}

\newcommand\listCamlReraiseExn[1][]{
  \listasm[firstline=727,lastline=734,#1]{../ocaml/runtime/amd64.S}{runtime/amd64.S:727}}

\begin{frame}{Backtraces}{Introducing \funcname{caml\_reraise\_exn}}
  \begin{columns}[c]
    \begin{column}{0.5\textwidth}
      \minipage[c][0.66\textheight][s]{\columnwidth}
      \begin{itemize}
        \item<1-> The exception have to be reraised to the next handler
        \item<2-> First, checks if the backtrace should be collected with \funcarg{domain\_state->backtrace\_active}
        \only<3>{
        \begin{itemize}
            \item If not, behaves like a \funcname{raise\_notrace} with \funcname{RESTORE\_EXN\_HANDLER\_OCAML}
        \end{itemize}
        }
        \item<4-> Jumps into \funcname{caml\_raise\_exn}
        \item<5-> Calls \funcname{caml\_stash\_backtrace} again, to scan the next part of the stack: from the trap just pop-ed to the next trap
      \end{itemize}
      \vfill
      \mintocaml[firstline=15,lastline=21,highlightlines={18-21}]{../src/backtrace/backtrace.ml}
      \endminipage
    \end{column}
    \begin{column}{0.5\textwidth}
      \raggedleft
      \onslide*<1>{\listCamlReraiseExn{}}
      \onslide*<2>{\listCamlReraiseExn[highlightlines=729]{}}
      \onslide*<3>{
        \mintc[firstline=190,lastline=192,highlightlines={191-192}]{../ocaml/runtime/amd64.S}
        \mintc[firstline=234,lastline=237,highlightlines={235-237}]{../ocaml/runtime/amd64.S}
        \medskip
        \listCamlReraiseExn[highlightlines=731]{}
      }
      \onslide*<4-5>{
        % TODO refactor with \listCamlReraiseExn
        \mintasm[firstline=727,lastline=734,highlightlines=730]{../ocaml/runtime/amd64.S}
        \medskip
      }
      \onslide*<4>{\listCamlRaiseExn[highlightlines=711]{}}
      \onslide*<5>{\listCamlRaiseExn[highlightlines=720]{}}
    \end{column}
  \end{columns}
\end{frame}

\begin{frame}{Backtraces}{Backtrace collection continues}
  \begin{columns}[c]
    \begin{column}{0.55\textwidth}
      \minipage[c][0.75\textheight][s]{\columnwidth}
      \begin{itemize}
        \item<1-> \funcname{caml\_stash\_backtrace} scans the older part of the stack, up to the next trap
        \item<6-> Controls jumps to the nested exception handler in \funcname{maybe\_raise}, which matches only on \typename{ExnB}
        \item<7-> \funcname{caml\_reraise\_exn} is called again, backtrace collection resumes with \funcname{caml\_stash\_backtrace}
      \end{itemize}
      \medskip
      \begin{itemize}
        \item<2-> Constructed backtrace:
        \begin{itemize}
          \item <2-> \funcname{definitely\_raise}
          \item <2-> \funcname{probably\_raise}
          \item <3-> \funcname{probably\_raise} (re-raise)
          \item <4-> \funcname{maybe\_raise}
          \item <5-> \funcname{main}
          \item <8-> \funcname{main} (re-raise)
        \end{itemize}
      \end{itemize}
      \vfill
      \onslide*<1-5>{\mintocaml[firstline=15,lastline=21]{../src/backtrace/backtrace.ml}}
      \onslide*<6>{\mintocaml[firstline=28,lastline=34,highlightlines=31]{../src/backtrace/backtrace.ml}}
      \onslide*<7->{\mintocaml[firstline=28,lastline=34,highlightlines=33]{../src/backtrace/backtrace.ml}}
      \endminipage
    \end{column}
    \begin{column}{0.45\textwidth}
      \raggedleft
      \onslide*<1>{\providecommand\step{6}\input{diagrams/backtrace/backtrace.tex}}%
      \onslide*<2>{\providecommand\step{6}\input{diagrams/backtrace/backtrace.tex}}%
      \onslide*<3>{\providecommand\step{7}\input{diagrams/backtrace/backtrace.tex}}%
      \onslide*<4>{\providecommand\step{8}\input{diagrams/backtrace/backtrace.tex}}%
      \onslide*<5>{\providecommand\step{9}\input{diagrams/backtrace/backtrace.tex}}%
      \onslide*<6>{\providecommand\step{10}\input{diagrams/backtrace/backtrace.tex}}%
      \onslide*<7>{\providecommand\step{11}\input{diagrams/backtrace/backtrace.tex}}%
      \onslide*<8>{\providecommand\step{12}\input{diagrams/backtrace/backtrace.tex}}%
      \onslide*<9>{\providecommand\step{13}\input{diagrams/backtrace/backtrace.tex}}%
    \end{column}
  \end{columns}
\end{frame}

\begin{frame}{Backtraces}{Printing the backtrace}
  \begin{columns}[c]
    \begin{column}{0.45\textwidth}
      \minipage[c][0.66\textheight][s]{\columnwidth}
      \begin{itemize}
        \item<1-> The exception backtrace can now be printed to \funcarg{stdout} with \funcname{Printexc.print_backtrace}
        \item<2-> First the exception backtrace is retrieved into an OCaml array with \funcname{get_raw_backtrace }
        \item<3-> Which is implemented as the \funcname{caml_get_exception_raw_backtrace} C function in the runtime
        \item<4-> And finally printed with \funcname{print_exception_backtrace}
      \end{itemize}
      \vfill
      \mintocaml[firstline=28,lastline=34,highlightlines=33]{../src/backtrace/backtrace.ml}
      \endminipage
    \end{column}
    \begin{column}{0.55\textwidth}
      \onslide*<1,2>{
        \mintocaml[firstline=100,lastline=101]{../ocaml/stdlib/printexc.ml}
      }
      \only<1>{
        \listocaml[firstline=170,lastline=172]{../ocaml/stdlib/printexc.ml}{stdlib/printexc.ml:170}
      }
      \only<2>{
        \listocaml[firstline=170,lastline=172,highlightlines=172]{../ocaml/stdlib/printexc.ml}{stdlib/printexc.ml:170}
      }
      \only<3>{
        \mintc[firstline=154,lastline=156]{../ocaml/runtime/backtrace.c}
        \listc[firstline=171,lastline=192,highlightlines={184-187}]{../ocaml/runtime/backtrace.c}{runtime/backtrace.c:154}
      }
      \only<4>{
        \listocaml[firstline=155,lastline=165]{../ocaml/stdlib/printexc.ml}{stdlib/printexc.ml:55}
      }
    \end{column}
  \end{columns}
\end{frame}


\subsection{The multiple kinds of raise}
\frameSubsection{}{}

\begin{frame}{Backtraces}{When backtraces are not needed}
  \begin{columns}[c]
    \begin{column}{0.45\textwidth}
      \minipage[c][0.66\textheight][s]{\columnwidth}
      \begin{itemize}
        \item<1-> What if the backtrace for an exception is never needed?
        \item<2-> It's possible to raise the exception explicitly with \funcname{raise\_notrace} to make raising as cheap as possible
        \item<3-> An example of use in the \funcname{dynlink} library of the OCaml compiler
      \end{itemize}
      \vfill
      \onslide<3->{
      \listocaml[firstline=205,lastline=209,highlightlines=207]{../ocaml/otherlibs/dynlink/dynlink\_compilerlibs/misc.ml}{otherlibs/dynlink/dynlink\_compilerlibs/misc.ml:205}
      }
      \endminipage
    \end{column}
    \begin{column}{0.55\textwidth}
    \only<4>{
      \mintcmm[highlightlines=3]{../src/notrace/camlNotrace\_\_fun_347.cmm}
      \listcmm{../src/notrace/camlNotrace\_\_all\_somes\_327.cmm}{all\_somes}
    }
    \onslide*<5->{
      \listobjdump[highlightlines={6-9}]{../src/notrace/camlNotrace\_\_fun_347.objdump}{anonymous function from \funcname{all\_somes}}
    }
    \end{column}
  \end{columns}
\end{frame}

\begin{frame}{Backtraces}{Summary table of the multiple kinds of raise}
  \begin{itemize}
    \item When debugging information is enabled and if the backtrace of an exception isn't needed, it's possible to raise with \funcname{raise\_notrace} for performance
    \item When debugging information is \emph{not} enabled, the compiler optimises \funcname{raise} calls to inline assembly (same effect as using \funcname{raise\_notrace})
  \end{itemize}
  \bigskip
  \centering
  \begin{tabular}{| l | c | c | c | c |}
    \hline
    \parbox{10em}{Debug information \\ (\funcarg{-g} flag)}  & \multicolumn{2}{|c|}{Disabled}                                     & \multicolumn{2}{|c|}{Enabled} \\
    \hline
    Raise kind                                               & \funcname{raise} & \funcname{raise\_notrace}                       & \funcname{raise} & \funcname{raise\_notrace} \\
    \hline
    Compilation                                              & \parbox{8em}{inline assembly\\ (optimization)} & inline assembly   & \funcname{caml\_raise\_exn} & inline assembly \\
    \hline
  \end{tabular}
\end{frame}


%
% Takeaway #5
%
\subsection*{Takeaway \#5}
\frameSubsectionTakeaway{}
\againframe<5>{takeaway}
}%
      \onslide*<3>{\providecommand\step{7}%
% Backtraces
%
\subsection{One advantage of raising with debugging information}
\frameSubsection{}{}

\begin{frame}{Backtraces}{Debugging information \& source code}
  \begin{columns}[c]
    \begin{column}{0.5\textwidth}
      \begin{itemize}
        \item Raising an exception is fun and games, but how do we know where the exception originated? $\rightarrow$ With the backtrace associated with the exception!
        \item In order to get a backtrace from the point where an exception was raised, it's necessary to compile with debugging information enabled (\funcarg{-g} flag for the compiler)
        \item Same example as in the `Nested handlers' section
      \end{itemize}
    \end{column}
    \begin{column}{0.5\textwidth}
      \listocaml[firstline=9,lastline=34]{../src/backtrace/backtrace.ml}{backtrace/backtrace.ml}
    \end{column}
  \end{columns}
\end{frame}

\begin{frame}{Backtraces}{CMM and disassembly of \funcname{definitely\_raise}}
  \begin{columns}[c]
    \begin{column}{0.5\textwidth}
      \minipage[c][0.66\textheight][s]{\columnwidth}
      \begin{itemize}
        \item<1-> An effect of compiling with debugging information (\funcarg{-g}): the CMM no longer uses \funcname{raise\_notrace} to raise the exception but \funcname{raise} instead
        \item<2-> Disassembly shows that CMM \funcname{raise} is actually a call to \funcname{caml\_raise\_exn}, a function in assembler provided by the runtime
      \end{itemize}
      \vfill
      \mintocaml[firstline=9,lastline=13,highlightlines=12]{../src/backtrace/backtrace.ml}
      \endminipage
    \end{column}
    \begin{column}{0.5\textwidth}
      \onslide*<1>{
        \mintcmm[highlightlines=5]{../src/backtrace/camlBacktrace\_\_definitely\_raise\_347.cmm}
      }
      \onslide*<2>{
        \mintobjdump[firstline=2,highlightlines=14]{../src/backtrace/camlBacktrace\_\_definitely\_raise\_347.objdump}
      }
    \end{column}
  \end{columns}
\end{frame}

\begin{frame}{Backtraces}{Introducing \funcname{caml\_raise\_exn}}
  \begin{columns}[c]
    \begin{column}{0.5\textwidth}
      \minipage[c][0.66\textheight][s]{\columnwidth}
      \onslide*<1>{
        \begin{itemize}
          \item Raising the exception is performed using \funcname{caml\_raise\_exn} which is
            \begin{itemize}
              \item An assembly function provided by the runtime
              \item Architecture specific
            \end{itemize}
          \item Let's see what it does differently from \funcname{raise\_notrace}\dots
          \item We are about to raise \typename{ExnC} with \funcname{caml\_raise\_exn} from \funcname{definitely\_raise}
        \end{itemize}
      }
      \onslide*<2>{
        \mintobjdump[firstline=2,highlightlines=14]{../src/backtrace/camlBacktrace\_\_definitely\_raise\_347.objdump}
      }
      \vfill
      \mintocaml[firstline=9,lastline=13,highlightlines=12]{../src/backtrace/backtrace.ml}
      \endminipage
    \end{column}
    \begin{column}{0.5\textwidth}
      \centering
      \providecommand\step{1}\input{diagrams/backtrace/backtrace.tex}
    \end{column}
  \end{columns}
\end{frame}

\begin{frame}{Backtraces}{But before that, we need more fields from \typename{caml\_domain\_state}}
  \begin{columns}
    \begin{column}{0.5\textwidth}
      \begin{itemize}
        \item Remember \typename{caml\_domain\_state} the per-domain runtime state?
        \item Some other members are of interest to us now\dots
        \item \localname{backtrace\_active} is used to know if the runtime should collect backtraces
        \item \localname{backtrace\_active} can be set by
          \begin{itemize}
            \item The \funcarg{b} flag in the \funcname{OCAMLRUNPARAM} environment variable
            \item \mintinline{ocaml}{Printexc.record_backtrace : bool -> unit} \footnotemark
          \end{itemize}
      \end{itemize}
    \end{column}
    \begin{column}{0.5\textwidth}
      \begin{listing}
        \mintc[highlightlines={9-11}]{src/ocaml/domain\_state\_backtrace.h}
        \caption{runtime/caml/domain\_state.h}
      \end{listing}
    \end{column}
  \end{columns}
  \footnotetext[1]{https://v2.ocaml.org/api/Printexc.html}
\end{frame}

\newcommand\listCamlRaiseExn[1][]{
  \listasm[firstline=702,lastline=725,#1]{../ocaml/runtime/amd64.S}{runtime/amd64.S:702}}

\begin{frame}{Backtraces}{Walk through \funcname{caml\_raise\_exn}}
  \begin{columns}[T]
    \begin{column}{0.5\textwidth}
      \begin{itemize}
        \item<1-> First checks whether exceptions backtrace should be recorded
        \item<1-> By checking the \funcarg{domain\_state->backtrace\_active} value
        \item<2-> If not, acts as \funcname{raise\_notrace} with \funcname{RESTORE\_EXN\_HANDLER\_OCAML}
          \begin{itemize}
            \item Set \regname{RSP} to \localname{domain\_state->exn\_handler} value
            \item Restore \localname{domain\_state->exn\_handler} from trap
            \item Jump with \funcname{ret} (same effect as using \funcname{pop} and \funcname{jmp})
          \end{itemize}
        \item<3-> Otherwise, switches to the C stack to be able to execute C code
        \item<4-> Calls \funcname{caml\_stash\_backtrace}, a C function provided by the runtime\ignorespaces
          \onslide<6->{, with
            \begin{itemize}
              \item \funcarg{exn}: exception value
              \item \funcarg{pc}: code address of the raising point
              \item \funcarg{sp}: stack address of the raising function
              \item \funcarg{trapsp}: stack address of the next handler
            \end{itemize}
          }
      \end{itemize}
      \bigskip
      \mintocaml[firstline=9,lastline=13,highlightlines=12]{../src/backtrace/backtrace.ml}
    \end{column}
    \begin{column}{0.5\textwidth}
      \raggedleft
      \onslide*<1,3-4>{
        \mintc[firstline=190,lastline=192]{../ocaml/runtime/amd64.S}
        \mintc[firstline=234,lastline=237]{../ocaml/runtime/amd64.S}
      }
      \onslide*<2>{
        \mintc[firstline=190,lastline=192,highlightlines={191-192}]{../ocaml/runtime/amd64.S}
        \mintc[firstline=234,lastline=237,highlightlines={235-237}]{../ocaml/runtime/amd64.S}
      }
      \onslide*<1>{\listCamlRaiseExn[highlightlines=705]{}}%
      \onslide*<2>{\listCamlRaiseExn[highlightlines={707-708}]{}}%
      \onslide*<3>{\listCamlRaiseExn[highlightlines={712-714}]{}}%
      \onslide*<4>{\listCamlRaiseExn[highlightlines={715-720}]{}}%
      \onslide*<5>{\providecommand\step{1}\input{diagrams/backtrace/backtrace.tex}}%
      \onslide*<6>{\providecommand\step{2}\input{diagrams/backtrace/backtrace.tex}}%
    \end{column}
  \end{columns}
\end{frame}

\newcommand\listStashBacktrace[1][]{
  \listc[firstline=91,lastline=120,#1]{../ocaml/runtime/backtrace\_nat.c}{runtime/backtrace\_nat.c:91}}

\begin{frame}{Backtraces}{Introducing \funcname{caml\_stash\_backtrace}}
  \begin{columns}[c]
    \begin{column}{0.5\textwidth}
      \minipage[c][0.66\textheight][s]{\columnwidth}
      \begin{itemize}
        \item<1-> A C function provided by the runtime (specific to native code)
        \item<2-> It scans the stack, between the stack pointer at the raising point up to the next trap, in order to collect the names of the detected function frames
          \onslide*<2>{
            \begin{itemize}
              \item And more information, like line number, column number...
            \end{itemize}
          }
        \item<3-> It uses the same technique and information as the Garbage Collector (GC) uses to scan the values live on the stack
          \onslide*<4>{
            \begin{itemize}
              \item \typename{frame\_descr} are at the core of stack unwinding
            \end{itemize}
          }
      \end{itemize}
      \vfill
      \mintocaml[firstline=9,lastline=13,highlightlines=12]{../src/backtrace/backtrace.ml}
      \endminipage
    \end{column}
    \begin{column}{0.5\textwidth}
      \onslide*<1>{\listStashBacktrace[highlightlines=91]{}}
      \onslide*<2>{\listStashBacktrace[highlightlines={91,117-118}]{}}
      \onslide*<3->{\listStashBacktrace[highlightlines={94,106,109-110}]{}}
    \end{column}
  \end{columns}
\end{frame}

\newcommand\listFrameDescr[1][]{
  \listocaml[firstline=111,lastline=124,#1]{../ocaml/asmcomp/emitaux.ml}{asmcomp/emitaux.ml:111}}
\newcommand\listDebugInfo[1][]{
  \listocaml[firstline=38,lastline=49,#1]{../ocaml/lambda/debuginfo.mli}{lambda/debuginfo.mli:38}}

\begin{frame}{Backtraces}{\typename{frame\_descr} and \typename{frame\_debuginfo} in a nutshell}
  \begin{columns}[c]
    \begin{column}{0.5\textwidth}
      \begin{itemize}
        \item<1-> For every return address in an OCaml program, there is a associated \typename{frame\_descr}
        \item<2-> A \typename{frame\_descr} specifies
        \begin{itemize}
          \item<2-> The size of the function stack frame
          \item<3-> Where live values are on the stack (so that the GC can mark them as live and prevent from being swept)
        \end{itemize}
        \item<4-> When compiled with debug information, \typename{frame\_descr} will be linked to a \typename{frame\_debuginfo}
        \begin{itemize}
          \item<5-> Contains information about the position in the source code (file name, line and column number)
        \end{itemize}
      \end{itemize}
    \end{column}
    \begin{column}{0.5\textwidth}
        \onslide*<1>{\listFrameDescr[highlightlines={118-122}]{}}
        \onslide*<2>{\listFrameDescr[highlightlines={120}]{}}
        \onslide*<3>{\listFrameDescr[highlightlines={121}]{}}
        \onslide*<4->{\listFrameDescr[highlightlines={122}]{}}
        \medskip
        \onslide*<1-3>{\listDebugInfo{}}
        \onslide*<4>{\listDebugInfo[highlightlines={38,49}]{}}
        \onslide*<5>{\listDebugInfo[highlightlines={39-46}]{}}
    \end{column}
  \end{columns}
\end{frame}

\begin{frame}{Backtraces}{Scanning the stack with \typename{frame\_descr}}
  \begin{columns}[c]
    \begin{column}{0.5\textwidth}
      \begin{itemize}
        \item Given the return address of the code that raises the exception by calling \funcname{caml\_raise\_exn} (\funcarg{pc} argument of \funcname{caml\_stash\_backtrace})
        \item And the position of the stack pointer (\funcarg{sp} argument of \funcname{caml\_stash\_backtrace})
        \item The frame size given by \typename{frame\_descr}.\funcarg{frame\_size}, allows to find the end of the previous frame
        \item And the return address of the caller of this function
        \bigskip
        \onslide<3->{
        \item Note that traps (here the reddish \funcname{ExnA}, \funcname{ExnB} and \funcname{ExnC} blocks) belong to the frame that installed them, and so the frame size changes when traps are removed
        }
      \end{itemize}
    \end{column}
    \begin{column}{0.5\textwidth}
      \raggedleft
      \onslide*<1>{\providecommand\step{2}\input{diagrams/backtrace/backtrace.tex}}%
      \onslide*<2>{\providecommand\step{1}\input{diagrams/backtrace/scanstack.tex}}%
      \onslide*<3>{\providecommand\step{2}\input{diagrams/backtrace/scanstack.tex}}%
      \onslide*<4>{\providecommand\step{3}\input{diagrams/backtrace/scanstack.tex}}%
      \onslide*<5>{\providecommand\step{4}\input{diagrams/backtrace/scanstack.tex}}%
      \onslide*<6>{\providecommand\step{5}\input{diagrams/backtrace/scanstack.tex}}%
    \end{column}
  \end{columns}
\end{frame}

% Duplicate of previous-previous slide
\begin{frame}{Backtraces}{Continuing on \funcname{caml\_stash\_backtrace}}
  \begin{columns}[c]
    \begin{column}{0.5\textwidth}
      \minipage[c][0.75\textheight][s]{\columnwidth}
      \begin{itemize}
          \color{gray}
        \item A C function provided by the runtime (specific to native code)
        \item It scans the stack, between the stack pointer at the raising point up to the next trap, in order to collect the names of the detected function frames
        \item It uses the same technique and information as the Garbage Collector (GC) uses to scan the values live on the stack
      \end{itemize}
      \begin{itemize}
        \item For each function frame detected, store a pointer to the \typename{frame\_descr} in the \localname{domain\_state->backtrace\_buffer} array, effectively constructing the backtrace
      \end{itemize}
      \onslide<2->{
        \begin{itemize}
            \smallskip
          \item Constructed backtrace:
            \funcname{definitely\_raise},
            \onslide<3->{\funcname{probably\_raise}}
        \end{itemize}
      }
      \vfill
      \mintocaml[firstline=9,lastline=13,highlightlines=12]{../src/backtrace/backtrace.ml}
      \endminipage
    \end{column}
    \begin{column}{0.5\textwidth}
      \raggedleft
      \onslide*<1>{\listStashBacktrace[highlightlines={114-115}]{}}
      \onslide*<2>{\providecommand\step{3}\input{diagrams/backtrace/backtrace.tex}}%
      \onslide*<3>{\providecommand\step{4}\input{diagrams/backtrace/backtrace.tex}}%
    \end{column}
  \end{columns}
\end{frame}

\begin{frame}{Backtraces}{Back to \funcname{caml\_raise\_exn}}
  \begin{columns}[c]
    \begin{column}{0.5\textwidth}
      \minipage[c][0.66\textheight][s]{\columnwidth}
      \begin{itemize}\color{gray}
        \item Checks wheteher exceptions backtrace should be recorded, by checking the \funcarg{domain\_state->backtrace\_active} value
        \item Switches to the C stack to be able to execute C code
        \item Calls \funcname{caml\_stash\_backtrace}, to collect the backtrace up to the next trap
      \end{itemize}
      \onslide*<2->{
        \begin{itemize}
          \item<2-> Executes the exception handler in \funcname{probably\_raise} with \funcname{RESTORE\_EXN\_HANDLER\_OCAML}
            \begin{itemize}
              \item Set \regname{RSP} to \localname{domain\_state->exn\_handler} value
              \item Restore \localname{domain\_state->exn\_handler} from trap
              \item Jump to the handler code with \funcname{ret}
            \end{itemize}
        \end{itemize}
      }
      \vfill
      \onslide*<1>{\mintocaml[firstline=9,lastline=13,highlightlines=12]{../src/backtrace/backtrace.ml}}
      \onslide*<2->{\mintocaml[firstline=15,lastline=21,highlightlines={19-21}]{../src/backtrace/backtrace.ml}}
      \endminipage
    \end{column}
    \begin{column}{0.5\textwidth}
      \centering
      \onslide*<1>{
        \mintc[firstline=190,lastline=192]{../ocaml/runtime/amd64.S}
        \mintc[firstline=234,lastline=237]{../ocaml/runtime/amd64.S}
      }
      \onslide*<2>{
        \mintc[firstline=190,lastline=192,highlightlines={191-192}]{../ocaml/runtime/amd64.S}
        \mintc[firstline=234,lastline=237,highlightlines={235-237}]{../ocaml/runtime/amd64.S}
      }
      \onslide*<1>{\listCamlRaiseExn[highlightlines=720]{}}
      \onslide*<2>{\listCamlRaiseExn[highlightlines=722]{}}
      \onslide*<3->{\providecommand\step{5}\input{diagrams/backtrace/backtrace.tex}}
    \end{column}
  \end{columns}
\end{frame}

\begin{frame}{Backtraces}{\funcname{caml\_probably\_raise}'s handler doesn't catch \typename{ExnC}}
  \begin{columns}[c]
    \begin{column}{0.45\textwidth}
      \minipage[c][0.75\textheight][s]{\columnwidth}
      \begin{itemize}
        \item<1-> \funcname{match \dots with}\footnotemark pattern matching can also match exceptions
        \item<1-> The CMM of \funcname{probably\_raise} shows that this \funcname{match \dots with} is implemented with a pattern matching and a \funcname{try \dots with}
        \item<1-> But this one only matches on \typename{ExnA} exception
        \item<2-> Since the pattern matching fails to catch the exception, \funcname{reraise} is called with our current \typename{ExnC} exception
        \item<3-> Disassembly shows that \funcname{reraise} is actually a call to \funcname{caml\_reraise\_exn}, which is also an assembly function provided by the runtime
      \end{itemize}
      \vfill
      \mintocaml[firstline=15,lastline=21,highlightlines={18-21}]{../src/backtrace/backtrace.ml}
      \endminipage
    \end{column}
    \begin{column}{0.55\textwidth}
      \centering
      \onslide*<1>{\mintcmm[highlightlines={10-18}]{../src/backtrace/camlBacktrace\_\_probably\_raise\_351.cmm}}
      \onslide*<2>{\listcmm[highlightlines=18]{../src/backtrace/camlBacktrace\_\_probably\_raise_351.cmm}{CMM of probably\_raise}}
      \onslide*<3>{\mintobjdump[firstline=5,lastline=35,highlightlines=31]{../src/backtrace/camlBacktrace\_\_probably\_raise_351.objdump}}
    \end{column}
  \end{columns}
  \only<1>{\footnotetext[1]{https://blog.janestreet.com/pattern-matching-and-exception-handling-unite/}}
\end{frame}

\newcommand\listCamlReraiseExn[1][]{
  \listasm[firstline=727,lastline=734,#1]{../ocaml/runtime/amd64.S}{runtime/amd64.S:727}}

\begin{frame}{Backtraces}{Introducing \funcname{caml\_reraise\_exn}}
  \begin{columns}[c]
    \begin{column}{0.5\textwidth}
      \minipage[c][0.66\textheight][s]{\columnwidth}
      \begin{itemize}
        \item<1-> The exception have to be reraised to the next handler
        \item<2-> First, checks if the backtrace should be collected with \funcarg{domain\_state->backtrace\_active}
        \only<3>{
        \begin{itemize}
            \item If not, behaves like a \funcname{raise\_notrace} with \funcname{RESTORE\_EXN\_HANDLER\_OCAML}
        \end{itemize}
        }
        \item<4-> Jumps into \funcname{caml\_raise\_exn}
        \item<5-> Calls \funcname{caml\_stash\_backtrace} again, to scan the next part of the stack: from the trap just pop-ed to the next trap
      \end{itemize}
      \vfill
      \mintocaml[firstline=15,lastline=21,highlightlines={18-21}]{../src/backtrace/backtrace.ml}
      \endminipage
    \end{column}
    \begin{column}{0.5\textwidth}
      \raggedleft
      \onslide*<1>{\listCamlReraiseExn{}}
      \onslide*<2>{\listCamlReraiseExn[highlightlines=729]{}}
      \onslide*<3>{
        \mintc[firstline=190,lastline=192,highlightlines={191-192}]{../ocaml/runtime/amd64.S}
        \mintc[firstline=234,lastline=237,highlightlines={235-237}]{../ocaml/runtime/amd64.S}
        \medskip
        \listCamlReraiseExn[highlightlines=731]{}
      }
      \onslide*<4-5>{
        % TODO refactor with \listCamlReraiseExn
        \mintasm[firstline=727,lastline=734,highlightlines=730]{../ocaml/runtime/amd64.S}
        \medskip
      }
      \onslide*<4>{\listCamlRaiseExn[highlightlines=711]{}}
      \onslide*<5>{\listCamlRaiseExn[highlightlines=720]{}}
    \end{column}
  \end{columns}
\end{frame}

\begin{frame}{Backtraces}{Backtrace collection continues}
  \begin{columns}[c]
    \begin{column}{0.55\textwidth}
      \minipage[c][0.75\textheight][s]{\columnwidth}
      \begin{itemize}
        \item<1-> \funcname{caml\_stash\_backtrace} scans the older part of the stack, up to the next trap
        \item<6-> Controls jumps to the nested exception handler in \funcname{maybe\_raise}, which matches only on \typename{ExnB}
        \item<7-> \funcname{caml\_reraise\_exn} is called again, backtrace collection resumes with \funcname{caml\_stash\_backtrace}
      \end{itemize}
      \medskip
      \begin{itemize}
        \item<2-> Constructed backtrace:
        \begin{itemize}
          \item <2-> \funcname{definitely\_raise}
          \item <2-> \funcname{probably\_raise}
          \item <3-> \funcname{probably\_raise} (re-raise)
          \item <4-> \funcname{maybe\_raise}
          \item <5-> \funcname{main}
          \item <8-> \funcname{main} (re-raise)
        \end{itemize}
      \end{itemize}
      \vfill
      \onslide*<1-5>{\mintocaml[firstline=15,lastline=21]{../src/backtrace/backtrace.ml}}
      \onslide*<6>{\mintocaml[firstline=28,lastline=34,highlightlines=31]{../src/backtrace/backtrace.ml}}
      \onslide*<7->{\mintocaml[firstline=28,lastline=34,highlightlines=33]{../src/backtrace/backtrace.ml}}
      \endminipage
    \end{column}
    \begin{column}{0.45\textwidth}
      \raggedleft
      \onslide*<1>{\providecommand\step{6}\input{diagrams/backtrace/backtrace.tex}}%
      \onslide*<2>{\providecommand\step{6}\input{diagrams/backtrace/backtrace.tex}}%
      \onslide*<3>{\providecommand\step{7}\input{diagrams/backtrace/backtrace.tex}}%
      \onslide*<4>{\providecommand\step{8}\input{diagrams/backtrace/backtrace.tex}}%
      \onslide*<5>{\providecommand\step{9}\input{diagrams/backtrace/backtrace.tex}}%
      \onslide*<6>{\providecommand\step{10}\input{diagrams/backtrace/backtrace.tex}}%
      \onslide*<7>{\providecommand\step{11}\input{diagrams/backtrace/backtrace.tex}}%
      \onslide*<8>{\providecommand\step{12}\input{diagrams/backtrace/backtrace.tex}}%
      \onslide*<9>{\providecommand\step{13}\input{diagrams/backtrace/backtrace.tex}}%
    \end{column}
  \end{columns}
\end{frame}

\begin{frame}{Backtraces}{Printing the backtrace}
  \begin{columns}[c]
    \begin{column}{0.45\textwidth}
      \minipage[c][0.66\textheight][s]{\columnwidth}
      \begin{itemize}
        \item<1-> The exception backtrace can now be printed to \funcarg{stdout} with \funcname{Printexc.print_backtrace}
        \item<2-> First the exception backtrace is retrieved into an OCaml array with \funcname{get_raw_backtrace }
        \item<3-> Which is implemented as the \funcname{caml_get_exception_raw_backtrace} C function in the runtime
        \item<4-> And finally printed with \funcname{print_exception_backtrace}
      \end{itemize}
      \vfill
      \mintocaml[firstline=28,lastline=34,highlightlines=33]{../src/backtrace/backtrace.ml}
      \endminipage
    \end{column}
    \begin{column}{0.55\textwidth}
      \onslide*<1,2>{
        \mintocaml[firstline=100,lastline=101]{../ocaml/stdlib/printexc.ml}
      }
      \only<1>{
        \listocaml[firstline=170,lastline=172]{../ocaml/stdlib/printexc.ml}{stdlib/printexc.ml:170}
      }
      \only<2>{
        \listocaml[firstline=170,lastline=172,highlightlines=172]{../ocaml/stdlib/printexc.ml}{stdlib/printexc.ml:170}
      }
      \only<3>{
        \mintc[firstline=154,lastline=156]{../ocaml/runtime/backtrace.c}
        \listc[firstline=171,lastline=192,highlightlines={184-187}]{../ocaml/runtime/backtrace.c}{runtime/backtrace.c:154}
      }
      \only<4>{
        \listocaml[firstline=155,lastline=165]{../ocaml/stdlib/printexc.ml}{stdlib/printexc.ml:55}
      }
    \end{column}
  \end{columns}
\end{frame}


\subsection{The multiple kinds of raise}
\frameSubsection{}{}

\begin{frame}{Backtraces}{When backtraces are not needed}
  \begin{columns}[c]
    \begin{column}{0.45\textwidth}
      \minipage[c][0.66\textheight][s]{\columnwidth}
      \begin{itemize}
        \item<1-> What if the backtrace for an exception is never needed?
        \item<2-> It's possible to raise the exception explicitly with \funcname{raise\_notrace} to make raising as cheap as possible
        \item<3-> An example of use in the \funcname{dynlink} library of the OCaml compiler
      \end{itemize}
      \vfill
      \onslide<3->{
      \listocaml[firstline=205,lastline=209,highlightlines=207]{../ocaml/otherlibs/dynlink/dynlink\_compilerlibs/misc.ml}{otherlibs/dynlink/dynlink\_compilerlibs/misc.ml:205}
      }
      \endminipage
    \end{column}
    \begin{column}{0.55\textwidth}
    \only<4>{
      \mintcmm[highlightlines=3]{../src/notrace/camlNotrace\_\_fun_347.cmm}
      \listcmm{../src/notrace/camlNotrace\_\_all\_somes\_327.cmm}{all\_somes}
    }
    \onslide*<5->{
      \listobjdump[highlightlines={6-9}]{../src/notrace/camlNotrace\_\_fun_347.objdump}{anonymous function from \funcname{all\_somes}}
    }
    \end{column}
  \end{columns}
\end{frame}

\begin{frame}{Backtraces}{Summary table of the multiple kinds of raise}
  \begin{itemize}
    \item When debugging information is enabled and if the backtrace of an exception isn't needed, it's possible to raise with \funcname{raise\_notrace} for performance
    \item When debugging information is \emph{not} enabled, the compiler optimises \funcname{raise} calls to inline assembly (same effect as using \funcname{raise\_notrace})
  \end{itemize}
  \bigskip
  \centering
  \begin{tabular}{| l | c | c | c | c |}
    \hline
    \parbox{10em}{Debug information \\ (\funcarg{-g} flag)}  & \multicolumn{2}{|c|}{Disabled}                                     & \multicolumn{2}{|c|}{Enabled} \\
    \hline
    Raise kind                                               & \funcname{raise} & \funcname{raise\_notrace}                       & \funcname{raise} & \funcname{raise\_notrace} \\
    \hline
    Compilation                                              & \parbox{8em}{inline assembly\\ (optimization)} & inline assembly   & \funcname{caml\_raise\_exn} & inline assembly \\
    \hline
  \end{tabular}
\end{frame}


%
% Takeaway #5
%
\subsection*{Takeaway \#5}
\frameSubsectionTakeaway{}
\againframe<5>{takeaway}
}%
      \onslide*<4>{\providecommand\step{8}%
% Backtraces
%
\subsection{One advantage of raising with debugging information}
\frameSubsection{}{}

\begin{frame}{Backtraces}{Debugging information \& source code}
  \begin{columns}[c]
    \begin{column}{0.5\textwidth}
      \begin{itemize}
        \item Raising an exception is fun and games, but how do we know where the exception originated? $\rightarrow$ With the backtrace associated with the exception!
        \item In order to get a backtrace from the point where an exception was raised, it's necessary to compile with debugging information enabled (\funcarg{-g} flag for the compiler)
        \item Same example as in the `Nested handlers' section
      \end{itemize}
    \end{column}
    \begin{column}{0.5\textwidth}
      \listocaml[firstline=9,lastline=34]{../src/backtrace/backtrace.ml}{backtrace/backtrace.ml}
    \end{column}
  \end{columns}
\end{frame}

\begin{frame}{Backtraces}{CMM and disassembly of \funcname{definitely\_raise}}
  \begin{columns}[c]
    \begin{column}{0.5\textwidth}
      \minipage[c][0.66\textheight][s]{\columnwidth}
      \begin{itemize}
        \item<1-> An effect of compiling with debugging information (\funcarg{-g}): the CMM no longer uses \funcname{raise\_notrace} to raise the exception but \funcname{raise} instead
        \item<2-> Disassembly shows that CMM \funcname{raise} is actually a call to \funcname{caml\_raise\_exn}, a function in assembler provided by the runtime
      \end{itemize}
      \vfill
      \mintocaml[firstline=9,lastline=13,highlightlines=12]{../src/backtrace/backtrace.ml}
      \endminipage
    \end{column}
    \begin{column}{0.5\textwidth}
      \onslide*<1>{
        \mintcmm[highlightlines=5]{../src/backtrace/camlBacktrace\_\_definitely\_raise\_347.cmm}
      }
      \onslide*<2>{
        \mintobjdump[firstline=2,highlightlines=14]{../src/backtrace/camlBacktrace\_\_definitely\_raise\_347.objdump}
      }
    \end{column}
  \end{columns}
\end{frame}

\begin{frame}{Backtraces}{Introducing \funcname{caml\_raise\_exn}}
  \begin{columns}[c]
    \begin{column}{0.5\textwidth}
      \minipage[c][0.66\textheight][s]{\columnwidth}
      \onslide*<1>{
        \begin{itemize}
          \item Raising the exception is performed using \funcname{caml\_raise\_exn} which is
            \begin{itemize}
              \item An assembly function provided by the runtime
              \item Architecture specific
            \end{itemize}
          \item Let's see what it does differently from \funcname{raise\_notrace}\dots
          \item We are about to raise \typename{ExnC} with \funcname{caml\_raise\_exn} from \funcname{definitely\_raise}
        \end{itemize}
      }
      \onslide*<2>{
        \mintobjdump[firstline=2,highlightlines=14]{../src/backtrace/camlBacktrace\_\_definitely\_raise\_347.objdump}
      }
      \vfill
      \mintocaml[firstline=9,lastline=13,highlightlines=12]{../src/backtrace/backtrace.ml}
      \endminipage
    \end{column}
    \begin{column}{0.5\textwidth}
      \centering
      \providecommand\step{1}\input{diagrams/backtrace/backtrace.tex}
    \end{column}
  \end{columns}
\end{frame}

\begin{frame}{Backtraces}{But before that, we need more fields from \typename{caml\_domain\_state}}
  \begin{columns}
    \begin{column}{0.5\textwidth}
      \begin{itemize}
        \item Remember \typename{caml\_domain\_state} the per-domain runtime state?
        \item Some other members are of interest to us now\dots
        \item \localname{backtrace\_active} is used to know if the runtime should collect backtraces
        \item \localname{backtrace\_active} can be set by
          \begin{itemize}
            \item The \funcarg{b} flag in the \funcname{OCAMLRUNPARAM} environment variable
            \item \mintinline{ocaml}{Printexc.record_backtrace : bool -> unit} \footnotemark
          \end{itemize}
      \end{itemize}
    \end{column}
    \begin{column}{0.5\textwidth}
      \begin{listing}
        \mintc[highlightlines={9-11}]{src/ocaml/domain\_state\_backtrace.h}
        \caption{runtime/caml/domain\_state.h}
      \end{listing}
    \end{column}
  \end{columns}
  \footnotetext[1]{https://v2.ocaml.org/api/Printexc.html}
\end{frame}

\newcommand\listCamlRaiseExn[1][]{
  \listasm[firstline=702,lastline=725,#1]{../ocaml/runtime/amd64.S}{runtime/amd64.S:702}}

\begin{frame}{Backtraces}{Walk through \funcname{caml\_raise\_exn}}
  \begin{columns}[T]
    \begin{column}{0.5\textwidth}
      \begin{itemize}
        \item<1-> First checks whether exceptions backtrace should be recorded
        \item<1-> By checking the \funcarg{domain\_state->backtrace\_active} value
        \item<2-> If not, acts as \funcname{raise\_notrace} with \funcname{RESTORE\_EXN\_HANDLER\_OCAML}
          \begin{itemize}
            \item Set \regname{RSP} to \localname{domain\_state->exn\_handler} value
            \item Restore \localname{domain\_state->exn\_handler} from trap
            \item Jump with \funcname{ret} (same effect as using \funcname{pop} and \funcname{jmp})
          \end{itemize}
        \item<3-> Otherwise, switches to the C stack to be able to execute C code
        \item<4-> Calls \funcname{caml\_stash\_backtrace}, a C function provided by the runtime\ignorespaces
          \onslide<6->{, with
            \begin{itemize}
              \item \funcarg{exn}: exception value
              \item \funcarg{pc}: code address of the raising point
              \item \funcarg{sp}: stack address of the raising function
              \item \funcarg{trapsp}: stack address of the next handler
            \end{itemize}
          }
      \end{itemize}
      \bigskip
      \mintocaml[firstline=9,lastline=13,highlightlines=12]{../src/backtrace/backtrace.ml}
    \end{column}
    \begin{column}{0.5\textwidth}
      \raggedleft
      \onslide*<1,3-4>{
        \mintc[firstline=190,lastline=192]{../ocaml/runtime/amd64.S}
        \mintc[firstline=234,lastline=237]{../ocaml/runtime/amd64.S}
      }
      \onslide*<2>{
        \mintc[firstline=190,lastline=192,highlightlines={191-192}]{../ocaml/runtime/amd64.S}
        \mintc[firstline=234,lastline=237,highlightlines={235-237}]{../ocaml/runtime/amd64.S}
      }
      \onslide*<1>{\listCamlRaiseExn[highlightlines=705]{}}%
      \onslide*<2>{\listCamlRaiseExn[highlightlines={707-708}]{}}%
      \onslide*<3>{\listCamlRaiseExn[highlightlines={712-714}]{}}%
      \onslide*<4>{\listCamlRaiseExn[highlightlines={715-720}]{}}%
      \onslide*<5>{\providecommand\step{1}\input{diagrams/backtrace/backtrace.tex}}%
      \onslide*<6>{\providecommand\step{2}\input{diagrams/backtrace/backtrace.tex}}%
    \end{column}
  \end{columns}
\end{frame}

\newcommand\listStashBacktrace[1][]{
  \listc[firstline=91,lastline=120,#1]{../ocaml/runtime/backtrace\_nat.c}{runtime/backtrace\_nat.c:91}}

\begin{frame}{Backtraces}{Introducing \funcname{caml\_stash\_backtrace}}
  \begin{columns}[c]
    \begin{column}{0.5\textwidth}
      \minipage[c][0.66\textheight][s]{\columnwidth}
      \begin{itemize}
        \item<1-> A C function provided by the runtime (specific to native code)
        \item<2-> It scans the stack, between the stack pointer at the raising point up to the next trap, in order to collect the names of the detected function frames
          \onslide*<2>{
            \begin{itemize}
              \item And more information, like line number, column number...
            \end{itemize}
          }
        \item<3-> It uses the same technique and information as the Garbage Collector (GC) uses to scan the values live on the stack
          \onslide*<4>{
            \begin{itemize}
              \item \typename{frame\_descr} are at the core of stack unwinding
            \end{itemize}
          }
      \end{itemize}
      \vfill
      \mintocaml[firstline=9,lastline=13,highlightlines=12]{../src/backtrace/backtrace.ml}
      \endminipage
    \end{column}
    \begin{column}{0.5\textwidth}
      \onslide*<1>{\listStashBacktrace[highlightlines=91]{}}
      \onslide*<2>{\listStashBacktrace[highlightlines={91,117-118}]{}}
      \onslide*<3->{\listStashBacktrace[highlightlines={94,106,109-110}]{}}
    \end{column}
  \end{columns}
\end{frame}

\newcommand\listFrameDescr[1][]{
  \listocaml[firstline=111,lastline=124,#1]{../ocaml/asmcomp/emitaux.ml}{asmcomp/emitaux.ml:111}}
\newcommand\listDebugInfo[1][]{
  \listocaml[firstline=38,lastline=49,#1]{../ocaml/lambda/debuginfo.mli}{lambda/debuginfo.mli:38}}

\begin{frame}{Backtraces}{\typename{frame\_descr} and \typename{frame\_debuginfo} in a nutshell}
  \begin{columns}[c]
    \begin{column}{0.5\textwidth}
      \begin{itemize}
        \item<1-> For every return address in an OCaml program, there is a associated \typename{frame\_descr}
        \item<2-> A \typename{frame\_descr} specifies
        \begin{itemize}
          \item<2-> The size of the function stack frame
          \item<3-> Where live values are on the stack (so that the GC can mark them as live and prevent from being swept)
        \end{itemize}
        \item<4-> When compiled with debug information, \typename{frame\_descr} will be linked to a \typename{frame\_debuginfo}
        \begin{itemize}
          \item<5-> Contains information about the position in the source code (file name, line and column number)
        \end{itemize}
      \end{itemize}
    \end{column}
    \begin{column}{0.5\textwidth}
        \onslide*<1>{\listFrameDescr[highlightlines={118-122}]{}}
        \onslide*<2>{\listFrameDescr[highlightlines={120}]{}}
        \onslide*<3>{\listFrameDescr[highlightlines={121}]{}}
        \onslide*<4->{\listFrameDescr[highlightlines={122}]{}}
        \medskip
        \onslide*<1-3>{\listDebugInfo{}}
        \onslide*<4>{\listDebugInfo[highlightlines={38,49}]{}}
        \onslide*<5>{\listDebugInfo[highlightlines={39-46}]{}}
    \end{column}
  \end{columns}
\end{frame}

\begin{frame}{Backtraces}{Scanning the stack with \typename{frame\_descr}}
  \begin{columns}[c]
    \begin{column}{0.5\textwidth}
      \begin{itemize}
        \item Given the return address of the code that raises the exception by calling \funcname{caml\_raise\_exn} (\funcarg{pc} argument of \funcname{caml\_stash\_backtrace})
        \item And the position of the stack pointer (\funcarg{sp} argument of \funcname{caml\_stash\_backtrace})
        \item The frame size given by \typename{frame\_descr}.\funcarg{frame\_size}, allows to find the end of the previous frame
        \item And the return address of the caller of this function
        \bigskip
        \onslide<3->{
        \item Note that traps (here the reddish \funcname{ExnA}, \funcname{ExnB} and \funcname{ExnC} blocks) belong to the frame that installed them, and so the frame size changes when traps are removed
        }
      \end{itemize}
    \end{column}
    \begin{column}{0.5\textwidth}
      \raggedleft
      \onslide*<1>{\providecommand\step{2}\input{diagrams/backtrace/backtrace.tex}}%
      \onslide*<2>{\providecommand\step{1}\input{diagrams/backtrace/scanstack.tex}}%
      \onslide*<3>{\providecommand\step{2}\input{diagrams/backtrace/scanstack.tex}}%
      \onslide*<4>{\providecommand\step{3}\input{diagrams/backtrace/scanstack.tex}}%
      \onslide*<5>{\providecommand\step{4}\input{diagrams/backtrace/scanstack.tex}}%
      \onslide*<6>{\providecommand\step{5}\input{diagrams/backtrace/scanstack.tex}}%
    \end{column}
  \end{columns}
\end{frame}

% Duplicate of previous-previous slide
\begin{frame}{Backtraces}{Continuing on \funcname{caml\_stash\_backtrace}}
  \begin{columns}[c]
    \begin{column}{0.5\textwidth}
      \minipage[c][0.75\textheight][s]{\columnwidth}
      \begin{itemize}
          \color{gray}
        \item A C function provided by the runtime (specific to native code)
        \item It scans the stack, between the stack pointer at the raising point up to the next trap, in order to collect the names of the detected function frames
        \item It uses the same technique and information as the Garbage Collector (GC) uses to scan the values live on the stack
      \end{itemize}
      \begin{itemize}
        \item For each function frame detected, store a pointer to the \typename{frame\_descr} in the \localname{domain\_state->backtrace\_buffer} array, effectively constructing the backtrace
      \end{itemize}
      \onslide<2->{
        \begin{itemize}
            \smallskip
          \item Constructed backtrace:
            \funcname{definitely\_raise},
            \onslide<3->{\funcname{probably\_raise}}
        \end{itemize}
      }
      \vfill
      \mintocaml[firstline=9,lastline=13,highlightlines=12]{../src/backtrace/backtrace.ml}
      \endminipage
    \end{column}
    \begin{column}{0.5\textwidth}
      \raggedleft
      \onslide*<1>{\listStashBacktrace[highlightlines={114-115}]{}}
      \onslide*<2>{\providecommand\step{3}\input{diagrams/backtrace/backtrace.tex}}%
      \onslide*<3>{\providecommand\step{4}\input{diagrams/backtrace/backtrace.tex}}%
    \end{column}
  \end{columns}
\end{frame}

\begin{frame}{Backtraces}{Back to \funcname{caml\_raise\_exn}}
  \begin{columns}[c]
    \begin{column}{0.5\textwidth}
      \minipage[c][0.66\textheight][s]{\columnwidth}
      \begin{itemize}\color{gray}
        \item Checks wheteher exceptions backtrace should be recorded, by checking the \funcarg{domain\_state->backtrace\_active} value
        \item Switches to the C stack to be able to execute C code
        \item Calls \funcname{caml\_stash\_backtrace}, to collect the backtrace up to the next trap
      \end{itemize}
      \onslide*<2->{
        \begin{itemize}
          \item<2-> Executes the exception handler in \funcname{probably\_raise} with \funcname{RESTORE\_EXN\_HANDLER\_OCAML}
            \begin{itemize}
              \item Set \regname{RSP} to \localname{domain\_state->exn\_handler} value
              \item Restore \localname{domain\_state->exn\_handler} from trap
              \item Jump to the handler code with \funcname{ret}
            \end{itemize}
        \end{itemize}
      }
      \vfill
      \onslide*<1>{\mintocaml[firstline=9,lastline=13,highlightlines=12]{../src/backtrace/backtrace.ml}}
      \onslide*<2->{\mintocaml[firstline=15,lastline=21,highlightlines={19-21}]{../src/backtrace/backtrace.ml}}
      \endminipage
    \end{column}
    \begin{column}{0.5\textwidth}
      \centering
      \onslide*<1>{
        \mintc[firstline=190,lastline=192]{../ocaml/runtime/amd64.S}
        \mintc[firstline=234,lastline=237]{../ocaml/runtime/amd64.S}
      }
      \onslide*<2>{
        \mintc[firstline=190,lastline=192,highlightlines={191-192}]{../ocaml/runtime/amd64.S}
        \mintc[firstline=234,lastline=237,highlightlines={235-237}]{../ocaml/runtime/amd64.S}
      }
      \onslide*<1>{\listCamlRaiseExn[highlightlines=720]{}}
      \onslide*<2>{\listCamlRaiseExn[highlightlines=722]{}}
      \onslide*<3->{\providecommand\step{5}\input{diagrams/backtrace/backtrace.tex}}
    \end{column}
  \end{columns}
\end{frame}

\begin{frame}{Backtraces}{\funcname{caml\_probably\_raise}'s handler doesn't catch \typename{ExnC}}
  \begin{columns}[c]
    \begin{column}{0.45\textwidth}
      \minipage[c][0.75\textheight][s]{\columnwidth}
      \begin{itemize}
        \item<1-> \funcname{match \dots with}\footnotemark pattern matching can also match exceptions
        \item<1-> The CMM of \funcname{probably\_raise} shows that this \funcname{match \dots with} is implemented with a pattern matching and a \funcname{try \dots with}
        \item<1-> But this one only matches on \typename{ExnA} exception
        \item<2-> Since the pattern matching fails to catch the exception, \funcname{reraise} is called with our current \typename{ExnC} exception
        \item<3-> Disassembly shows that \funcname{reraise} is actually a call to \funcname{caml\_reraise\_exn}, which is also an assembly function provided by the runtime
      \end{itemize}
      \vfill
      \mintocaml[firstline=15,lastline=21,highlightlines={18-21}]{../src/backtrace/backtrace.ml}
      \endminipage
    \end{column}
    \begin{column}{0.55\textwidth}
      \centering
      \onslide*<1>{\mintcmm[highlightlines={10-18}]{../src/backtrace/camlBacktrace\_\_probably\_raise\_351.cmm}}
      \onslide*<2>{\listcmm[highlightlines=18]{../src/backtrace/camlBacktrace\_\_probably\_raise_351.cmm}{CMM of probably\_raise}}
      \onslide*<3>{\mintobjdump[firstline=5,lastline=35,highlightlines=31]{../src/backtrace/camlBacktrace\_\_probably\_raise_351.objdump}}
    \end{column}
  \end{columns}
  \only<1>{\footnotetext[1]{https://blog.janestreet.com/pattern-matching-and-exception-handling-unite/}}
\end{frame}

\newcommand\listCamlReraiseExn[1][]{
  \listasm[firstline=727,lastline=734,#1]{../ocaml/runtime/amd64.S}{runtime/amd64.S:727}}

\begin{frame}{Backtraces}{Introducing \funcname{caml\_reraise\_exn}}
  \begin{columns}[c]
    \begin{column}{0.5\textwidth}
      \minipage[c][0.66\textheight][s]{\columnwidth}
      \begin{itemize}
        \item<1-> The exception have to be reraised to the next handler
        \item<2-> First, checks if the backtrace should be collected with \funcarg{domain\_state->backtrace\_active}
        \only<3>{
        \begin{itemize}
            \item If not, behaves like a \funcname{raise\_notrace} with \funcname{RESTORE\_EXN\_HANDLER\_OCAML}
        \end{itemize}
        }
        \item<4-> Jumps into \funcname{caml\_raise\_exn}
        \item<5-> Calls \funcname{caml\_stash\_backtrace} again, to scan the next part of the stack: from the trap just pop-ed to the next trap
      \end{itemize}
      \vfill
      \mintocaml[firstline=15,lastline=21,highlightlines={18-21}]{../src/backtrace/backtrace.ml}
      \endminipage
    \end{column}
    \begin{column}{0.5\textwidth}
      \raggedleft
      \onslide*<1>{\listCamlReraiseExn{}}
      \onslide*<2>{\listCamlReraiseExn[highlightlines=729]{}}
      \onslide*<3>{
        \mintc[firstline=190,lastline=192,highlightlines={191-192}]{../ocaml/runtime/amd64.S}
        \mintc[firstline=234,lastline=237,highlightlines={235-237}]{../ocaml/runtime/amd64.S}
        \medskip
        \listCamlReraiseExn[highlightlines=731]{}
      }
      \onslide*<4-5>{
        % TODO refactor with \listCamlReraiseExn
        \mintasm[firstline=727,lastline=734,highlightlines=730]{../ocaml/runtime/amd64.S}
        \medskip
      }
      \onslide*<4>{\listCamlRaiseExn[highlightlines=711]{}}
      \onslide*<5>{\listCamlRaiseExn[highlightlines=720]{}}
    \end{column}
  \end{columns}
\end{frame}

\begin{frame}{Backtraces}{Backtrace collection continues}
  \begin{columns}[c]
    \begin{column}{0.55\textwidth}
      \minipage[c][0.75\textheight][s]{\columnwidth}
      \begin{itemize}
        \item<1-> \funcname{caml\_stash\_backtrace} scans the older part of the stack, up to the next trap
        \item<6-> Controls jumps to the nested exception handler in \funcname{maybe\_raise}, which matches only on \typename{ExnB}
        \item<7-> \funcname{caml\_reraise\_exn} is called again, backtrace collection resumes with \funcname{caml\_stash\_backtrace}
      \end{itemize}
      \medskip
      \begin{itemize}
        \item<2-> Constructed backtrace:
        \begin{itemize}
          \item <2-> \funcname{definitely\_raise}
          \item <2-> \funcname{probably\_raise}
          \item <3-> \funcname{probably\_raise} (re-raise)
          \item <4-> \funcname{maybe\_raise}
          \item <5-> \funcname{main}
          \item <8-> \funcname{main} (re-raise)
        \end{itemize}
      \end{itemize}
      \vfill
      \onslide*<1-5>{\mintocaml[firstline=15,lastline=21]{../src/backtrace/backtrace.ml}}
      \onslide*<6>{\mintocaml[firstline=28,lastline=34,highlightlines=31]{../src/backtrace/backtrace.ml}}
      \onslide*<7->{\mintocaml[firstline=28,lastline=34,highlightlines=33]{../src/backtrace/backtrace.ml}}
      \endminipage
    \end{column}
    \begin{column}{0.45\textwidth}
      \raggedleft
      \onslide*<1>{\providecommand\step{6}\input{diagrams/backtrace/backtrace.tex}}%
      \onslide*<2>{\providecommand\step{6}\input{diagrams/backtrace/backtrace.tex}}%
      \onslide*<3>{\providecommand\step{7}\input{diagrams/backtrace/backtrace.tex}}%
      \onslide*<4>{\providecommand\step{8}\input{diagrams/backtrace/backtrace.tex}}%
      \onslide*<5>{\providecommand\step{9}\input{diagrams/backtrace/backtrace.tex}}%
      \onslide*<6>{\providecommand\step{10}\input{diagrams/backtrace/backtrace.tex}}%
      \onslide*<7>{\providecommand\step{11}\input{diagrams/backtrace/backtrace.tex}}%
      \onslide*<8>{\providecommand\step{12}\input{diagrams/backtrace/backtrace.tex}}%
      \onslide*<9>{\providecommand\step{13}\input{diagrams/backtrace/backtrace.tex}}%
    \end{column}
  \end{columns}
\end{frame}

\begin{frame}{Backtraces}{Printing the backtrace}
  \begin{columns}[c]
    \begin{column}{0.45\textwidth}
      \minipage[c][0.66\textheight][s]{\columnwidth}
      \begin{itemize}
        \item<1-> The exception backtrace can now be printed to \funcarg{stdout} with \funcname{Printexc.print_backtrace}
        \item<2-> First the exception backtrace is retrieved into an OCaml array with \funcname{get_raw_backtrace }
        \item<3-> Which is implemented as the \funcname{caml_get_exception_raw_backtrace} C function in the runtime
        \item<4-> And finally printed with \funcname{print_exception_backtrace}
      \end{itemize}
      \vfill
      \mintocaml[firstline=28,lastline=34,highlightlines=33]{../src/backtrace/backtrace.ml}
      \endminipage
    \end{column}
    \begin{column}{0.55\textwidth}
      \onslide*<1,2>{
        \mintocaml[firstline=100,lastline=101]{../ocaml/stdlib/printexc.ml}
      }
      \only<1>{
        \listocaml[firstline=170,lastline=172]{../ocaml/stdlib/printexc.ml}{stdlib/printexc.ml:170}
      }
      \only<2>{
        \listocaml[firstline=170,lastline=172,highlightlines=172]{../ocaml/stdlib/printexc.ml}{stdlib/printexc.ml:170}
      }
      \only<3>{
        \mintc[firstline=154,lastline=156]{../ocaml/runtime/backtrace.c}
        \listc[firstline=171,lastline=192,highlightlines={184-187}]{../ocaml/runtime/backtrace.c}{runtime/backtrace.c:154}
      }
      \only<4>{
        \listocaml[firstline=155,lastline=165]{../ocaml/stdlib/printexc.ml}{stdlib/printexc.ml:55}
      }
    \end{column}
  \end{columns}
\end{frame}


\subsection{The multiple kinds of raise}
\frameSubsection{}{}

\begin{frame}{Backtraces}{When backtraces are not needed}
  \begin{columns}[c]
    \begin{column}{0.45\textwidth}
      \minipage[c][0.66\textheight][s]{\columnwidth}
      \begin{itemize}
        \item<1-> What if the backtrace for an exception is never needed?
        \item<2-> It's possible to raise the exception explicitly with \funcname{raise\_notrace} to make raising as cheap as possible
        \item<3-> An example of use in the \funcname{dynlink} library of the OCaml compiler
      \end{itemize}
      \vfill
      \onslide<3->{
      \listocaml[firstline=205,lastline=209,highlightlines=207]{../ocaml/otherlibs/dynlink/dynlink\_compilerlibs/misc.ml}{otherlibs/dynlink/dynlink\_compilerlibs/misc.ml:205}
      }
      \endminipage
    \end{column}
    \begin{column}{0.55\textwidth}
    \only<4>{
      \mintcmm[highlightlines=3]{../src/notrace/camlNotrace\_\_fun_347.cmm}
      \listcmm{../src/notrace/camlNotrace\_\_all\_somes\_327.cmm}{all\_somes}
    }
    \onslide*<5->{
      \listobjdump[highlightlines={6-9}]{../src/notrace/camlNotrace\_\_fun_347.objdump}{anonymous function from \funcname{all\_somes}}
    }
    \end{column}
  \end{columns}
\end{frame}

\begin{frame}{Backtraces}{Summary table of the multiple kinds of raise}
  \begin{itemize}
    \item When debugging information is enabled and if the backtrace of an exception isn't needed, it's possible to raise with \funcname{raise\_notrace} for performance
    \item When debugging information is \emph{not} enabled, the compiler optimises \funcname{raise} calls to inline assembly (same effect as using \funcname{raise\_notrace})
  \end{itemize}
  \bigskip
  \centering
  \begin{tabular}{| l | c | c | c | c |}
    \hline
    \parbox{10em}{Debug information \\ (\funcarg{-g} flag)}  & \multicolumn{2}{|c|}{Disabled}                                     & \multicolumn{2}{|c|}{Enabled} \\
    \hline
    Raise kind                                               & \funcname{raise} & \funcname{raise\_notrace}                       & \funcname{raise} & \funcname{raise\_notrace} \\
    \hline
    Compilation                                              & \parbox{8em}{inline assembly\\ (optimization)} & inline assembly   & \funcname{caml\_raise\_exn} & inline assembly \\
    \hline
  \end{tabular}
\end{frame}


%
% Takeaway #5
%
\subsection*{Takeaway \#5}
\frameSubsectionTakeaway{}
\againframe<5>{takeaway}
}%
      \onslide*<5>{\providecommand\step{9}%
% Backtraces
%
\subsection{One advantage of raising with debugging information}
\frameSubsection{}{}

\begin{frame}{Backtraces}{Debugging information \& source code}
  \begin{columns}[c]
    \begin{column}{0.5\textwidth}
      \begin{itemize}
        \item Raising an exception is fun and games, but how do we know where the exception originated? $\rightarrow$ With the backtrace associated with the exception!
        \item In order to get a backtrace from the point where an exception was raised, it's necessary to compile with debugging information enabled (\funcarg{-g} flag for the compiler)
        \item Same example as in the `Nested handlers' section
      \end{itemize}
    \end{column}
    \begin{column}{0.5\textwidth}
      \listocaml[firstline=9,lastline=34]{../src/backtrace/backtrace.ml}{backtrace/backtrace.ml}
    \end{column}
  \end{columns}
\end{frame}

\begin{frame}{Backtraces}{CMM and disassembly of \funcname{definitely\_raise}}
  \begin{columns}[c]
    \begin{column}{0.5\textwidth}
      \minipage[c][0.66\textheight][s]{\columnwidth}
      \begin{itemize}
        \item<1-> An effect of compiling with debugging information (\funcarg{-g}): the CMM no longer uses \funcname{raise\_notrace} to raise the exception but \funcname{raise} instead
        \item<2-> Disassembly shows that CMM \funcname{raise} is actually a call to \funcname{caml\_raise\_exn}, a function in assembler provided by the runtime
      \end{itemize}
      \vfill
      \mintocaml[firstline=9,lastline=13,highlightlines=12]{../src/backtrace/backtrace.ml}
      \endminipage
    \end{column}
    \begin{column}{0.5\textwidth}
      \onslide*<1>{
        \mintcmm[highlightlines=5]{../src/backtrace/camlBacktrace\_\_definitely\_raise\_347.cmm}
      }
      \onslide*<2>{
        \mintobjdump[firstline=2,highlightlines=14]{../src/backtrace/camlBacktrace\_\_definitely\_raise\_347.objdump}
      }
    \end{column}
  \end{columns}
\end{frame}

\begin{frame}{Backtraces}{Introducing \funcname{caml\_raise\_exn}}
  \begin{columns}[c]
    \begin{column}{0.5\textwidth}
      \minipage[c][0.66\textheight][s]{\columnwidth}
      \onslide*<1>{
        \begin{itemize}
          \item Raising the exception is performed using \funcname{caml\_raise\_exn} which is
            \begin{itemize}
              \item An assembly function provided by the runtime
              \item Architecture specific
            \end{itemize}
          \item Let's see what it does differently from \funcname{raise\_notrace}\dots
          \item We are about to raise \typename{ExnC} with \funcname{caml\_raise\_exn} from \funcname{definitely\_raise}
        \end{itemize}
      }
      \onslide*<2>{
        \mintobjdump[firstline=2,highlightlines=14]{../src/backtrace/camlBacktrace\_\_definitely\_raise\_347.objdump}
      }
      \vfill
      \mintocaml[firstline=9,lastline=13,highlightlines=12]{../src/backtrace/backtrace.ml}
      \endminipage
    \end{column}
    \begin{column}{0.5\textwidth}
      \centering
      \providecommand\step{1}\input{diagrams/backtrace/backtrace.tex}
    \end{column}
  \end{columns}
\end{frame}

\begin{frame}{Backtraces}{But before that, we need more fields from \typename{caml\_domain\_state}}
  \begin{columns}
    \begin{column}{0.5\textwidth}
      \begin{itemize}
        \item Remember \typename{caml\_domain\_state} the per-domain runtime state?
        \item Some other members are of interest to us now\dots
        \item \localname{backtrace\_active} is used to know if the runtime should collect backtraces
        \item \localname{backtrace\_active} can be set by
          \begin{itemize}
            \item The \funcarg{b} flag in the \funcname{OCAMLRUNPARAM} environment variable
            \item \mintinline{ocaml}{Printexc.record_backtrace : bool -> unit} \footnotemark
          \end{itemize}
      \end{itemize}
    \end{column}
    \begin{column}{0.5\textwidth}
      \begin{listing}
        \mintc[highlightlines={9-11}]{src/ocaml/domain\_state\_backtrace.h}
        \caption{runtime/caml/domain\_state.h}
      \end{listing}
    \end{column}
  \end{columns}
  \footnotetext[1]{https://v2.ocaml.org/api/Printexc.html}
\end{frame}

\newcommand\listCamlRaiseExn[1][]{
  \listasm[firstline=702,lastline=725,#1]{../ocaml/runtime/amd64.S}{runtime/amd64.S:702}}

\begin{frame}{Backtraces}{Walk through \funcname{caml\_raise\_exn}}
  \begin{columns}[T]
    \begin{column}{0.5\textwidth}
      \begin{itemize}
        \item<1-> First checks whether exceptions backtrace should be recorded
        \item<1-> By checking the \funcarg{domain\_state->backtrace\_active} value
        \item<2-> If not, acts as \funcname{raise\_notrace} with \funcname{RESTORE\_EXN\_HANDLER\_OCAML}
          \begin{itemize}
            \item Set \regname{RSP} to \localname{domain\_state->exn\_handler} value
            \item Restore \localname{domain\_state->exn\_handler} from trap
            \item Jump with \funcname{ret} (same effect as using \funcname{pop} and \funcname{jmp})
          \end{itemize}
        \item<3-> Otherwise, switches to the C stack to be able to execute C code
        \item<4-> Calls \funcname{caml\_stash\_backtrace}, a C function provided by the runtime\ignorespaces
          \onslide<6->{, with
            \begin{itemize}
              \item \funcarg{exn}: exception value
              \item \funcarg{pc}: code address of the raising point
              \item \funcarg{sp}: stack address of the raising function
              \item \funcarg{trapsp}: stack address of the next handler
            \end{itemize}
          }
      \end{itemize}
      \bigskip
      \mintocaml[firstline=9,lastline=13,highlightlines=12]{../src/backtrace/backtrace.ml}
    \end{column}
    \begin{column}{0.5\textwidth}
      \raggedleft
      \onslide*<1,3-4>{
        \mintc[firstline=190,lastline=192]{../ocaml/runtime/amd64.S}
        \mintc[firstline=234,lastline=237]{../ocaml/runtime/amd64.S}
      }
      \onslide*<2>{
        \mintc[firstline=190,lastline=192,highlightlines={191-192}]{../ocaml/runtime/amd64.S}
        \mintc[firstline=234,lastline=237,highlightlines={235-237}]{../ocaml/runtime/amd64.S}
      }
      \onslide*<1>{\listCamlRaiseExn[highlightlines=705]{}}%
      \onslide*<2>{\listCamlRaiseExn[highlightlines={707-708}]{}}%
      \onslide*<3>{\listCamlRaiseExn[highlightlines={712-714}]{}}%
      \onslide*<4>{\listCamlRaiseExn[highlightlines={715-720}]{}}%
      \onslide*<5>{\providecommand\step{1}\input{diagrams/backtrace/backtrace.tex}}%
      \onslide*<6>{\providecommand\step{2}\input{diagrams/backtrace/backtrace.tex}}%
    \end{column}
  \end{columns}
\end{frame}

\newcommand\listStashBacktrace[1][]{
  \listc[firstline=91,lastline=120,#1]{../ocaml/runtime/backtrace\_nat.c}{runtime/backtrace\_nat.c:91}}

\begin{frame}{Backtraces}{Introducing \funcname{caml\_stash\_backtrace}}
  \begin{columns}[c]
    \begin{column}{0.5\textwidth}
      \minipage[c][0.66\textheight][s]{\columnwidth}
      \begin{itemize}
        \item<1-> A C function provided by the runtime (specific to native code)
        \item<2-> It scans the stack, between the stack pointer at the raising point up to the next trap, in order to collect the names of the detected function frames
          \onslide*<2>{
            \begin{itemize}
              \item And more information, like line number, column number...
            \end{itemize}
          }
        \item<3-> It uses the same technique and information as the Garbage Collector (GC) uses to scan the values live on the stack
          \onslide*<4>{
            \begin{itemize}
              \item \typename{frame\_descr} are at the core of stack unwinding
            \end{itemize}
          }
      \end{itemize}
      \vfill
      \mintocaml[firstline=9,lastline=13,highlightlines=12]{../src/backtrace/backtrace.ml}
      \endminipage
    \end{column}
    \begin{column}{0.5\textwidth}
      \onslide*<1>{\listStashBacktrace[highlightlines=91]{}}
      \onslide*<2>{\listStashBacktrace[highlightlines={91,117-118}]{}}
      \onslide*<3->{\listStashBacktrace[highlightlines={94,106,109-110}]{}}
    \end{column}
  \end{columns}
\end{frame}

\newcommand\listFrameDescr[1][]{
  \listocaml[firstline=111,lastline=124,#1]{../ocaml/asmcomp/emitaux.ml}{asmcomp/emitaux.ml:111}}
\newcommand\listDebugInfo[1][]{
  \listocaml[firstline=38,lastline=49,#1]{../ocaml/lambda/debuginfo.mli}{lambda/debuginfo.mli:38}}

\begin{frame}{Backtraces}{\typename{frame\_descr} and \typename{frame\_debuginfo} in a nutshell}
  \begin{columns}[c]
    \begin{column}{0.5\textwidth}
      \begin{itemize}
        \item<1-> For every return address in an OCaml program, there is a associated \typename{frame\_descr}
        \item<2-> A \typename{frame\_descr} specifies
        \begin{itemize}
          \item<2-> The size of the function stack frame
          \item<3-> Where live values are on the stack (so that the GC can mark them as live and prevent from being swept)
        \end{itemize}
        \item<4-> When compiled with debug information, \typename{frame\_descr} will be linked to a \typename{frame\_debuginfo}
        \begin{itemize}
          \item<5-> Contains information about the position in the source code (file name, line and column number)
        \end{itemize}
      \end{itemize}
    \end{column}
    \begin{column}{0.5\textwidth}
        \onslide*<1>{\listFrameDescr[highlightlines={118-122}]{}}
        \onslide*<2>{\listFrameDescr[highlightlines={120}]{}}
        \onslide*<3>{\listFrameDescr[highlightlines={121}]{}}
        \onslide*<4->{\listFrameDescr[highlightlines={122}]{}}
        \medskip
        \onslide*<1-3>{\listDebugInfo{}}
        \onslide*<4>{\listDebugInfo[highlightlines={38,49}]{}}
        \onslide*<5>{\listDebugInfo[highlightlines={39-46}]{}}
    \end{column}
  \end{columns}
\end{frame}

\begin{frame}{Backtraces}{Scanning the stack with \typename{frame\_descr}}
  \begin{columns}[c]
    \begin{column}{0.5\textwidth}
      \begin{itemize}
        \item Given the return address of the code that raises the exception by calling \funcname{caml\_raise\_exn} (\funcarg{pc} argument of \funcname{caml\_stash\_backtrace})
        \item And the position of the stack pointer (\funcarg{sp} argument of \funcname{caml\_stash\_backtrace})
        \item The frame size given by \typename{frame\_descr}.\funcarg{frame\_size}, allows to find the end of the previous frame
        \item And the return address of the caller of this function
        \bigskip
        \onslide<3->{
        \item Note that traps (here the reddish \funcname{ExnA}, \funcname{ExnB} and \funcname{ExnC} blocks) belong to the frame that installed them, and so the frame size changes when traps are removed
        }
      \end{itemize}
    \end{column}
    \begin{column}{0.5\textwidth}
      \raggedleft
      \onslide*<1>{\providecommand\step{2}\input{diagrams/backtrace/backtrace.tex}}%
      \onslide*<2>{\providecommand\step{1}\input{diagrams/backtrace/scanstack.tex}}%
      \onslide*<3>{\providecommand\step{2}\input{diagrams/backtrace/scanstack.tex}}%
      \onslide*<4>{\providecommand\step{3}\input{diagrams/backtrace/scanstack.tex}}%
      \onslide*<5>{\providecommand\step{4}\input{diagrams/backtrace/scanstack.tex}}%
      \onslide*<6>{\providecommand\step{5}\input{diagrams/backtrace/scanstack.tex}}%
    \end{column}
  \end{columns}
\end{frame}

% Duplicate of previous-previous slide
\begin{frame}{Backtraces}{Continuing on \funcname{caml\_stash\_backtrace}}
  \begin{columns}[c]
    \begin{column}{0.5\textwidth}
      \minipage[c][0.75\textheight][s]{\columnwidth}
      \begin{itemize}
          \color{gray}
        \item A C function provided by the runtime (specific to native code)
        \item It scans the stack, between the stack pointer at the raising point up to the next trap, in order to collect the names of the detected function frames
        \item It uses the same technique and information as the Garbage Collector (GC) uses to scan the values live on the stack
      \end{itemize}
      \begin{itemize}
        \item For each function frame detected, store a pointer to the \typename{frame\_descr} in the \localname{domain\_state->backtrace\_buffer} array, effectively constructing the backtrace
      \end{itemize}
      \onslide<2->{
        \begin{itemize}
            \smallskip
          \item Constructed backtrace:
            \funcname{definitely\_raise},
            \onslide<3->{\funcname{probably\_raise}}
        \end{itemize}
      }
      \vfill
      \mintocaml[firstline=9,lastline=13,highlightlines=12]{../src/backtrace/backtrace.ml}
      \endminipage
    \end{column}
    \begin{column}{0.5\textwidth}
      \raggedleft
      \onslide*<1>{\listStashBacktrace[highlightlines={114-115}]{}}
      \onslide*<2>{\providecommand\step{3}\input{diagrams/backtrace/backtrace.tex}}%
      \onslide*<3>{\providecommand\step{4}\input{diagrams/backtrace/backtrace.tex}}%
    \end{column}
  \end{columns}
\end{frame}

\begin{frame}{Backtraces}{Back to \funcname{caml\_raise\_exn}}
  \begin{columns}[c]
    \begin{column}{0.5\textwidth}
      \minipage[c][0.66\textheight][s]{\columnwidth}
      \begin{itemize}\color{gray}
        \item Checks wheteher exceptions backtrace should be recorded, by checking the \funcarg{domain\_state->backtrace\_active} value
        \item Switches to the C stack to be able to execute C code
        \item Calls \funcname{caml\_stash\_backtrace}, to collect the backtrace up to the next trap
      \end{itemize}
      \onslide*<2->{
        \begin{itemize}
          \item<2-> Executes the exception handler in \funcname{probably\_raise} with \funcname{RESTORE\_EXN\_HANDLER\_OCAML}
            \begin{itemize}
              \item Set \regname{RSP} to \localname{domain\_state->exn\_handler} value
              \item Restore \localname{domain\_state->exn\_handler} from trap
              \item Jump to the handler code with \funcname{ret}
            \end{itemize}
        \end{itemize}
      }
      \vfill
      \onslide*<1>{\mintocaml[firstline=9,lastline=13,highlightlines=12]{../src/backtrace/backtrace.ml}}
      \onslide*<2->{\mintocaml[firstline=15,lastline=21,highlightlines={19-21}]{../src/backtrace/backtrace.ml}}
      \endminipage
    \end{column}
    \begin{column}{0.5\textwidth}
      \centering
      \onslide*<1>{
        \mintc[firstline=190,lastline=192]{../ocaml/runtime/amd64.S}
        \mintc[firstline=234,lastline=237]{../ocaml/runtime/amd64.S}
      }
      \onslide*<2>{
        \mintc[firstline=190,lastline=192,highlightlines={191-192}]{../ocaml/runtime/amd64.S}
        \mintc[firstline=234,lastline=237,highlightlines={235-237}]{../ocaml/runtime/amd64.S}
      }
      \onslide*<1>{\listCamlRaiseExn[highlightlines=720]{}}
      \onslide*<2>{\listCamlRaiseExn[highlightlines=722]{}}
      \onslide*<3->{\providecommand\step{5}\input{diagrams/backtrace/backtrace.tex}}
    \end{column}
  \end{columns}
\end{frame}

\begin{frame}{Backtraces}{\funcname{caml\_probably\_raise}'s handler doesn't catch \typename{ExnC}}
  \begin{columns}[c]
    \begin{column}{0.45\textwidth}
      \minipage[c][0.75\textheight][s]{\columnwidth}
      \begin{itemize}
        \item<1-> \funcname{match \dots with}\footnotemark pattern matching can also match exceptions
        \item<1-> The CMM of \funcname{probably\_raise} shows that this \funcname{match \dots with} is implemented with a pattern matching and a \funcname{try \dots with}
        \item<1-> But this one only matches on \typename{ExnA} exception
        \item<2-> Since the pattern matching fails to catch the exception, \funcname{reraise} is called with our current \typename{ExnC} exception
        \item<3-> Disassembly shows that \funcname{reraise} is actually a call to \funcname{caml\_reraise\_exn}, which is also an assembly function provided by the runtime
      \end{itemize}
      \vfill
      \mintocaml[firstline=15,lastline=21,highlightlines={18-21}]{../src/backtrace/backtrace.ml}
      \endminipage
    \end{column}
    \begin{column}{0.55\textwidth}
      \centering
      \onslide*<1>{\mintcmm[highlightlines={10-18}]{../src/backtrace/camlBacktrace\_\_probably\_raise\_351.cmm}}
      \onslide*<2>{\listcmm[highlightlines=18]{../src/backtrace/camlBacktrace\_\_probably\_raise_351.cmm}{CMM of probably\_raise}}
      \onslide*<3>{\mintobjdump[firstline=5,lastline=35,highlightlines=31]{../src/backtrace/camlBacktrace\_\_probably\_raise_351.objdump}}
    \end{column}
  \end{columns}
  \only<1>{\footnotetext[1]{https://blog.janestreet.com/pattern-matching-and-exception-handling-unite/}}
\end{frame}

\newcommand\listCamlReraiseExn[1][]{
  \listasm[firstline=727,lastline=734,#1]{../ocaml/runtime/amd64.S}{runtime/amd64.S:727}}

\begin{frame}{Backtraces}{Introducing \funcname{caml\_reraise\_exn}}
  \begin{columns}[c]
    \begin{column}{0.5\textwidth}
      \minipage[c][0.66\textheight][s]{\columnwidth}
      \begin{itemize}
        \item<1-> The exception have to be reraised to the next handler
        \item<2-> First, checks if the backtrace should be collected with \funcarg{domain\_state->backtrace\_active}
        \only<3>{
        \begin{itemize}
            \item If not, behaves like a \funcname{raise\_notrace} with \funcname{RESTORE\_EXN\_HANDLER\_OCAML}
        \end{itemize}
        }
        \item<4-> Jumps into \funcname{caml\_raise\_exn}
        \item<5-> Calls \funcname{caml\_stash\_backtrace} again, to scan the next part of the stack: from the trap just pop-ed to the next trap
      \end{itemize}
      \vfill
      \mintocaml[firstline=15,lastline=21,highlightlines={18-21}]{../src/backtrace/backtrace.ml}
      \endminipage
    \end{column}
    \begin{column}{0.5\textwidth}
      \raggedleft
      \onslide*<1>{\listCamlReraiseExn{}}
      \onslide*<2>{\listCamlReraiseExn[highlightlines=729]{}}
      \onslide*<3>{
        \mintc[firstline=190,lastline=192,highlightlines={191-192}]{../ocaml/runtime/amd64.S}
        \mintc[firstline=234,lastline=237,highlightlines={235-237}]{../ocaml/runtime/amd64.S}
        \medskip
        \listCamlReraiseExn[highlightlines=731]{}
      }
      \onslide*<4-5>{
        % TODO refactor with \listCamlReraiseExn
        \mintasm[firstline=727,lastline=734,highlightlines=730]{../ocaml/runtime/amd64.S}
        \medskip
      }
      \onslide*<4>{\listCamlRaiseExn[highlightlines=711]{}}
      \onslide*<5>{\listCamlRaiseExn[highlightlines=720]{}}
    \end{column}
  \end{columns}
\end{frame}

\begin{frame}{Backtraces}{Backtrace collection continues}
  \begin{columns}[c]
    \begin{column}{0.55\textwidth}
      \minipage[c][0.75\textheight][s]{\columnwidth}
      \begin{itemize}
        \item<1-> \funcname{caml\_stash\_backtrace} scans the older part of the stack, up to the next trap
        \item<6-> Controls jumps to the nested exception handler in \funcname{maybe\_raise}, which matches only on \typename{ExnB}
        \item<7-> \funcname{caml\_reraise\_exn} is called again, backtrace collection resumes with \funcname{caml\_stash\_backtrace}
      \end{itemize}
      \medskip
      \begin{itemize}
        \item<2-> Constructed backtrace:
        \begin{itemize}
          \item <2-> \funcname{definitely\_raise}
          \item <2-> \funcname{probably\_raise}
          \item <3-> \funcname{probably\_raise} (re-raise)
          \item <4-> \funcname{maybe\_raise}
          \item <5-> \funcname{main}
          \item <8-> \funcname{main} (re-raise)
        \end{itemize}
      \end{itemize}
      \vfill
      \onslide*<1-5>{\mintocaml[firstline=15,lastline=21]{../src/backtrace/backtrace.ml}}
      \onslide*<6>{\mintocaml[firstline=28,lastline=34,highlightlines=31]{../src/backtrace/backtrace.ml}}
      \onslide*<7->{\mintocaml[firstline=28,lastline=34,highlightlines=33]{../src/backtrace/backtrace.ml}}
      \endminipage
    \end{column}
    \begin{column}{0.45\textwidth}
      \raggedleft
      \onslide*<1>{\providecommand\step{6}\input{diagrams/backtrace/backtrace.tex}}%
      \onslide*<2>{\providecommand\step{6}\input{diagrams/backtrace/backtrace.tex}}%
      \onslide*<3>{\providecommand\step{7}\input{diagrams/backtrace/backtrace.tex}}%
      \onslide*<4>{\providecommand\step{8}\input{diagrams/backtrace/backtrace.tex}}%
      \onslide*<5>{\providecommand\step{9}\input{diagrams/backtrace/backtrace.tex}}%
      \onslide*<6>{\providecommand\step{10}\input{diagrams/backtrace/backtrace.tex}}%
      \onslide*<7>{\providecommand\step{11}\input{diagrams/backtrace/backtrace.tex}}%
      \onslide*<8>{\providecommand\step{12}\input{diagrams/backtrace/backtrace.tex}}%
      \onslide*<9>{\providecommand\step{13}\input{diagrams/backtrace/backtrace.tex}}%
    \end{column}
  \end{columns}
\end{frame}

\begin{frame}{Backtraces}{Printing the backtrace}
  \begin{columns}[c]
    \begin{column}{0.45\textwidth}
      \minipage[c][0.66\textheight][s]{\columnwidth}
      \begin{itemize}
        \item<1-> The exception backtrace can now be printed to \funcarg{stdout} with \funcname{Printexc.print_backtrace}
        \item<2-> First the exception backtrace is retrieved into an OCaml array with \funcname{get_raw_backtrace }
        \item<3-> Which is implemented as the \funcname{caml_get_exception_raw_backtrace} C function in the runtime
        \item<4-> And finally printed with \funcname{print_exception_backtrace}
      \end{itemize}
      \vfill
      \mintocaml[firstline=28,lastline=34,highlightlines=33]{../src/backtrace/backtrace.ml}
      \endminipage
    \end{column}
    \begin{column}{0.55\textwidth}
      \onslide*<1,2>{
        \mintocaml[firstline=100,lastline=101]{../ocaml/stdlib/printexc.ml}
      }
      \only<1>{
        \listocaml[firstline=170,lastline=172]{../ocaml/stdlib/printexc.ml}{stdlib/printexc.ml:170}
      }
      \only<2>{
        \listocaml[firstline=170,lastline=172,highlightlines=172]{../ocaml/stdlib/printexc.ml}{stdlib/printexc.ml:170}
      }
      \only<3>{
        \mintc[firstline=154,lastline=156]{../ocaml/runtime/backtrace.c}
        \listc[firstline=171,lastline=192,highlightlines={184-187}]{../ocaml/runtime/backtrace.c}{runtime/backtrace.c:154}
      }
      \only<4>{
        \listocaml[firstline=155,lastline=165]{../ocaml/stdlib/printexc.ml}{stdlib/printexc.ml:55}
      }
    \end{column}
  \end{columns}
\end{frame}


\subsection{The multiple kinds of raise}
\frameSubsection{}{}

\begin{frame}{Backtraces}{When backtraces are not needed}
  \begin{columns}[c]
    \begin{column}{0.45\textwidth}
      \minipage[c][0.66\textheight][s]{\columnwidth}
      \begin{itemize}
        \item<1-> What if the backtrace for an exception is never needed?
        \item<2-> It's possible to raise the exception explicitly with \funcname{raise\_notrace} to make raising as cheap as possible
        \item<3-> An example of use in the \funcname{dynlink} library of the OCaml compiler
      \end{itemize}
      \vfill
      \onslide<3->{
      \listocaml[firstline=205,lastline=209,highlightlines=207]{../ocaml/otherlibs/dynlink/dynlink\_compilerlibs/misc.ml}{otherlibs/dynlink/dynlink\_compilerlibs/misc.ml:205}
      }
      \endminipage
    \end{column}
    \begin{column}{0.55\textwidth}
    \only<4>{
      \mintcmm[highlightlines=3]{../src/notrace/camlNotrace\_\_fun_347.cmm}
      \listcmm{../src/notrace/camlNotrace\_\_all\_somes\_327.cmm}{all\_somes}
    }
    \onslide*<5->{
      \listobjdump[highlightlines={6-9}]{../src/notrace/camlNotrace\_\_fun_347.objdump}{anonymous function from \funcname{all\_somes}}
    }
    \end{column}
  \end{columns}
\end{frame}

\begin{frame}{Backtraces}{Summary table of the multiple kinds of raise}
  \begin{itemize}
    \item When debugging information is enabled and if the backtrace of an exception isn't needed, it's possible to raise with \funcname{raise\_notrace} for performance
    \item When debugging information is \emph{not} enabled, the compiler optimises \funcname{raise} calls to inline assembly (same effect as using \funcname{raise\_notrace})
  \end{itemize}
  \bigskip
  \centering
  \begin{tabular}{| l | c | c | c | c |}
    \hline
    \parbox{10em}{Debug information \\ (\funcarg{-g} flag)}  & \multicolumn{2}{|c|}{Disabled}                                     & \multicolumn{2}{|c|}{Enabled} \\
    \hline
    Raise kind                                               & \funcname{raise} & \funcname{raise\_notrace}                       & \funcname{raise} & \funcname{raise\_notrace} \\
    \hline
    Compilation                                              & \parbox{8em}{inline assembly\\ (optimization)} & inline assembly   & \funcname{caml\_raise\_exn} & inline assembly \\
    \hline
  \end{tabular}
\end{frame}


%
% Takeaway #5
%
\subsection*{Takeaway \#5}
\frameSubsectionTakeaway{}
\againframe<5>{takeaway}
}%
      \onslide*<6>{\providecommand\step{10}%
% Backtraces
%
\subsection{One advantage of raising with debugging information}
\frameSubsection{}{}

\begin{frame}{Backtraces}{Debugging information \& source code}
  \begin{columns}[c]
    \begin{column}{0.5\textwidth}
      \begin{itemize}
        \item Raising an exception is fun and games, but how do we know where the exception originated? $\rightarrow$ With the backtrace associated with the exception!
        \item In order to get a backtrace from the point where an exception was raised, it's necessary to compile with debugging information enabled (\funcarg{-g} flag for the compiler)
        \item Same example as in the `Nested handlers' section
      \end{itemize}
    \end{column}
    \begin{column}{0.5\textwidth}
      \listocaml[firstline=9,lastline=34]{../src/backtrace/backtrace.ml}{backtrace/backtrace.ml}
    \end{column}
  \end{columns}
\end{frame}

\begin{frame}{Backtraces}{CMM and disassembly of \funcname{definitely\_raise}}
  \begin{columns}[c]
    \begin{column}{0.5\textwidth}
      \minipage[c][0.66\textheight][s]{\columnwidth}
      \begin{itemize}
        \item<1-> An effect of compiling with debugging information (\funcarg{-g}): the CMM no longer uses \funcname{raise\_notrace} to raise the exception but \funcname{raise} instead
        \item<2-> Disassembly shows that CMM \funcname{raise} is actually a call to \funcname{caml\_raise\_exn}, a function in assembler provided by the runtime
      \end{itemize}
      \vfill
      \mintocaml[firstline=9,lastline=13,highlightlines=12]{../src/backtrace/backtrace.ml}
      \endminipage
    \end{column}
    \begin{column}{0.5\textwidth}
      \onslide*<1>{
        \mintcmm[highlightlines=5]{../src/backtrace/camlBacktrace\_\_definitely\_raise\_347.cmm}
      }
      \onslide*<2>{
        \mintobjdump[firstline=2,highlightlines=14]{../src/backtrace/camlBacktrace\_\_definitely\_raise\_347.objdump}
      }
    \end{column}
  \end{columns}
\end{frame}

\begin{frame}{Backtraces}{Introducing \funcname{caml\_raise\_exn}}
  \begin{columns}[c]
    \begin{column}{0.5\textwidth}
      \minipage[c][0.66\textheight][s]{\columnwidth}
      \onslide*<1>{
        \begin{itemize}
          \item Raising the exception is performed using \funcname{caml\_raise\_exn} which is
            \begin{itemize}
              \item An assembly function provided by the runtime
              \item Architecture specific
            \end{itemize}
          \item Let's see what it does differently from \funcname{raise\_notrace}\dots
          \item We are about to raise \typename{ExnC} with \funcname{caml\_raise\_exn} from \funcname{definitely\_raise}
        \end{itemize}
      }
      \onslide*<2>{
        \mintobjdump[firstline=2,highlightlines=14]{../src/backtrace/camlBacktrace\_\_definitely\_raise\_347.objdump}
      }
      \vfill
      \mintocaml[firstline=9,lastline=13,highlightlines=12]{../src/backtrace/backtrace.ml}
      \endminipage
    \end{column}
    \begin{column}{0.5\textwidth}
      \centering
      \providecommand\step{1}\input{diagrams/backtrace/backtrace.tex}
    \end{column}
  \end{columns}
\end{frame}

\begin{frame}{Backtraces}{But before that, we need more fields from \typename{caml\_domain\_state}}
  \begin{columns}
    \begin{column}{0.5\textwidth}
      \begin{itemize}
        \item Remember \typename{caml\_domain\_state} the per-domain runtime state?
        \item Some other members are of interest to us now\dots
        \item \localname{backtrace\_active} is used to know if the runtime should collect backtraces
        \item \localname{backtrace\_active} can be set by
          \begin{itemize}
            \item The \funcarg{b} flag in the \funcname{OCAMLRUNPARAM} environment variable
            \item \mintinline{ocaml}{Printexc.record_backtrace : bool -> unit} \footnotemark
          \end{itemize}
      \end{itemize}
    \end{column}
    \begin{column}{0.5\textwidth}
      \begin{listing}
        \mintc[highlightlines={9-11}]{src/ocaml/domain\_state\_backtrace.h}
        \caption{runtime/caml/domain\_state.h}
      \end{listing}
    \end{column}
  \end{columns}
  \footnotetext[1]{https://v2.ocaml.org/api/Printexc.html}
\end{frame}

\newcommand\listCamlRaiseExn[1][]{
  \listasm[firstline=702,lastline=725,#1]{../ocaml/runtime/amd64.S}{runtime/amd64.S:702}}

\begin{frame}{Backtraces}{Walk through \funcname{caml\_raise\_exn}}
  \begin{columns}[T]
    \begin{column}{0.5\textwidth}
      \begin{itemize}
        \item<1-> First checks whether exceptions backtrace should be recorded
        \item<1-> By checking the \funcarg{domain\_state->backtrace\_active} value
        \item<2-> If not, acts as \funcname{raise\_notrace} with \funcname{RESTORE\_EXN\_HANDLER\_OCAML}
          \begin{itemize}
            \item Set \regname{RSP} to \localname{domain\_state->exn\_handler} value
            \item Restore \localname{domain\_state->exn\_handler} from trap
            \item Jump with \funcname{ret} (same effect as using \funcname{pop} and \funcname{jmp})
          \end{itemize}
        \item<3-> Otherwise, switches to the C stack to be able to execute C code
        \item<4-> Calls \funcname{caml\_stash\_backtrace}, a C function provided by the runtime\ignorespaces
          \onslide<6->{, with
            \begin{itemize}
              \item \funcarg{exn}: exception value
              \item \funcarg{pc}: code address of the raising point
              \item \funcarg{sp}: stack address of the raising function
              \item \funcarg{trapsp}: stack address of the next handler
            \end{itemize}
          }
      \end{itemize}
      \bigskip
      \mintocaml[firstline=9,lastline=13,highlightlines=12]{../src/backtrace/backtrace.ml}
    \end{column}
    \begin{column}{0.5\textwidth}
      \raggedleft
      \onslide*<1,3-4>{
        \mintc[firstline=190,lastline=192]{../ocaml/runtime/amd64.S}
        \mintc[firstline=234,lastline=237]{../ocaml/runtime/amd64.S}
      }
      \onslide*<2>{
        \mintc[firstline=190,lastline=192,highlightlines={191-192}]{../ocaml/runtime/amd64.S}
        \mintc[firstline=234,lastline=237,highlightlines={235-237}]{../ocaml/runtime/amd64.S}
      }
      \onslide*<1>{\listCamlRaiseExn[highlightlines=705]{}}%
      \onslide*<2>{\listCamlRaiseExn[highlightlines={707-708}]{}}%
      \onslide*<3>{\listCamlRaiseExn[highlightlines={712-714}]{}}%
      \onslide*<4>{\listCamlRaiseExn[highlightlines={715-720}]{}}%
      \onslide*<5>{\providecommand\step{1}\input{diagrams/backtrace/backtrace.tex}}%
      \onslide*<6>{\providecommand\step{2}\input{diagrams/backtrace/backtrace.tex}}%
    \end{column}
  \end{columns}
\end{frame}

\newcommand\listStashBacktrace[1][]{
  \listc[firstline=91,lastline=120,#1]{../ocaml/runtime/backtrace\_nat.c}{runtime/backtrace\_nat.c:91}}

\begin{frame}{Backtraces}{Introducing \funcname{caml\_stash\_backtrace}}
  \begin{columns}[c]
    \begin{column}{0.5\textwidth}
      \minipage[c][0.66\textheight][s]{\columnwidth}
      \begin{itemize}
        \item<1-> A C function provided by the runtime (specific to native code)
        \item<2-> It scans the stack, between the stack pointer at the raising point up to the next trap, in order to collect the names of the detected function frames
          \onslide*<2>{
            \begin{itemize}
              \item And more information, like line number, column number...
            \end{itemize}
          }
        \item<3-> It uses the same technique and information as the Garbage Collector (GC) uses to scan the values live on the stack
          \onslide*<4>{
            \begin{itemize}
              \item \typename{frame\_descr} are at the core of stack unwinding
            \end{itemize}
          }
      \end{itemize}
      \vfill
      \mintocaml[firstline=9,lastline=13,highlightlines=12]{../src/backtrace/backtrace.ml}
      \endminipage
    \end{column}
    \begin{column}{0.5\textwidth}
      \onslide*<1>{\listStashBacktrace[highlightlines=91]{}}
      \onslide*<2>{\listStashBacktrace[highlightlines={91,117-118}]{}}
      \onslide*<3->{\listStashBacktrace[highlightlines={94,106,109-110}]{}}
    \end{column}
  \end{columns}
\end{frame}

\newcommand\listFrameDescr[1][]{
  \listocaml[firstline=111,lastline=124,#1]{../ocaml/asmcomp/emitaux.ml}{asmcomp/emitaux.ml:111}}
\newcommand\listDebugInfo[1][]{
  \listocaml[firstline=38,lastline=49,#1]{../ocaml/lambda/debuginfo.mli}{lambda/debuginfo.mli:38}}

\begin{frame}{Backtraces}{\typename{frame\_descr} and \typename{frame\_debuginfo} in a nutshell}
  \begin{columns}[c]
    \begin{column}{0.5\textwidth}
      \begin{itemize}
        \item<1-> For every return address in an OCaml program, there is a associated \typename{frame\_descr}
        \item<2-> A \typename{frame\_descr} specifies
        \begin{itemize}
          \item<2-> The size of the function stack frame
          \item<3-> Where live values are on the stack (so that the GC can mark them as live and prevent from being swept)
        \end{itemize}
        \item<4-> When compiled with debug information, \typename{frame\_descr} will be linked to a \typename{frame\_debuginfo}
        \begin{itemize}
          \item<5-> Contains information about the position in the source code (file name, line and column number)
        \end{itemize}
      \end{itemize}
    \end{column}
    \begin{column}{0.5\textwidth}
        \onslide*<1>{\listFrameDescr[highlightlines={118-122}]{}}
        \onslide*<2>{\listFrameDescr[highlightlines={120}]{}}
        \onslide*<3>{\listFrameDescr[highlightlines={121}]{}}
        \onslide*<4->{\listFrameDescr[highlightlines={122}]{}}
        \medskip
        \onslide*<1-3>{\listDebugInfo{}}
        \onslide*<4>{\listDebugInfo[highlightlines={38,49}]{}}
        \onslide*<5>{\listDebugInfo[highlightlines={39-46}]{}}
    \end{column}
  \end{columns}
\end{frame}

\begin{frame}{Backtraces}{Scanning the stack with \typename{frame\_descr}}
  \begin{columns}[c]
    \begin{column}{0.5\textwidth}
      \begin{itemize}
        \item Given the return address of the code that raises the exception by calling \funcname{caml\_raise\_exn} (\funcarg{pc} argument of \funcname{caml\_stash\_backtrace})
        \item And the position of the stack pointer (\funcarg{sp} argument of \funcname{caml\_stash\_backtrace})
        \item The frame size given by \typename{frame\_descr}.\funcarg{frame\_size}, allows to find the end of the previous frame
        \item And the return address of the caller of this function
        \bigskip
        \onslide<3->{
        \item Note that traps (here the reddish \funcname{ExnA}, \funcname{ExnB} and \funcname{ExnC} blocks) belong to the frame that installed them, and so the frame size changes when traps are removed
        }
      \end{itemize}
    \end{column}
    \begin{column}{0.5\textwidth}
      \raggedleft
      \onslide*<1>{\providecommand\step{2}\input{diagrams/backtrace/backtrace.tex}}%
      \onslide*<2>{\providecommand\step{1}\input{diagrams/backtrace/scanstack.tex}}%
      \onslide*<3>{\providecommand\step{2}\input{diagrams/backtrace/scanstack.tex}}%
      \onslide*<4>{\providecommand\step{3}\input{diagrams/backtrace/scanstack.tex}}%
      \onslide*<5>{\providecommand\step{4}\input{diagrams/backtrace/scanstack.tex}}%
      \onslide*<6>{\providecommand\step{5}\input{diagrams/backtrace/scanstack.tex}}%
    \end{column}
  \end{columns}
\end{frame}

% Duplicate of previous-previous slide
\begin{frame}{Backtraces}{Continuing on \funcname{caml\_stash\_backtrace}}
  \begin{columns}[c]
    \begin{column}{0.5\textwidth}
      \minipage[c][0.75\textheight][s]{\columnwidth}
      \begin{itemize}
          \color{gray}
        \item A C function provided by the runtime (specific to native code)
        \item It scans the stack, between the stack pointer at the raising point up to the next trap, in order to collect the names of the detected function frames
        \item It uses the same technique and information as the Garbage Collector (GC) uses to scan the values live on the stack
      \end{itemize}
      \begin{itemize}
        \item For each function frame detected, store a pointer to the \typename{frame\_descr} in the \localname{domain\_state->backtrace\_buffer} array, effectively constructing the backtrace
      \end{itemize}
      \onslide<2->{
        \begin{itemize}
            \smallskip
          \item Constructed backtrace:
            \funcname{definitely\_raise},
            \onslide<3->{\funcname{probably\_raise}}
        \end{itemize}
      }
      \vfill
      \mintocaml[firstline=9,lastline=13,highlightlines=12]{../src/backtrace/backtrace.ml}
      \endminipage
    \end{column}
    \begin{column}{0.5\textwidth}
      \raggedleft
      \onslide*<1>{\listStashBacktrace[highlightlines={114-115}]{}}
      \onslide*<2>{\providecommand\step{3}\input{diagrams/backtrace/backtrace.tex}}%
      \onslide*<3>{\providecommand\step{4}\input{diagrams/backtrace/backtrace.tex}}%
    \end{column}
  \end{columns}
\end{frame}

\begin{frame}{Backtraces}{Back to \funcname{caml\_raise\_exn}}
  \begin{columns}[c]
    \begin{column}{0.5\textwidth}
      \minipage[c][0.66\textheight][s]{\columnwidth}
      \begin{itemize}\color{gray}
        \item Checks wheteher exceptions backtrace should be recorded, by checking the \funcarg{domain\_state->backtrace\_active} value
        \item Switches to the C stack to be able to execute C code
        \item Calls \funcname{caml\_stash\_backtrace}, to collect the backtrace up to the next trap
      \end{itemize}
      \onslide*<2->{
        \begin{itemize}
          \item<2-> Executes the exception handler in \funcname{probably\_raise} with \funcname{RESTORE\_EXN\_HANDLER\_OCAML}
            \begin{itemize}
              \item Set \regname{RSP} to \localname{domain\_state->exn\_handler} value
              \item Restore \localname{domain\_state->exn\_handler} from trap
              \item Jump to the handler code with \funcname{ret}
            \end{itemize}
        \end{itemize}
      }
      \vfill
      \onslide*<1>{\mintocaml[firstline=9,lastline=13,highlightlines=12]{../src/backtrace/backtrace.ml}}
      \onslide*<2->{\mintocaml[firstline=15,lastline=21,highlightlines={19-21}]{../src/backtrace/backtrace.ml}}
      \endminipage
    \end{column}
    \begin{column}{0.5\textwidth}
      \centering
      \onslide*<1>{
        \mintc[firstline=190,lastline=192]{../ocaml/runtime/amd64.S}
        \mintc[firstline=234,lastline=237]{../ocaml/runtime/amd64.S}
      }
      \onslide*<2>{
        \mintc[firstline=190,lastline=192,highlightlines={191-192}]{../ocaml/runtime/amd64.S}
        \mintc[firstline=234,lastline=237,highlightlines={235-237}]{../ocaml/runtime/amd64.S}
      }
      \onslide*<1>{\listCamlRaiseExn[highlightlines=720]{}}
      \onslide*<2>{\listCamlRaiseExn[highlightlines=722]{}}
      \onslide*<3->{\providecommand\step{5}\input{diagrams/backtrace/backtrace.tex}}
    \end{column}
  \end{columns}
\end{frame}

\begin{frame}{Backtraces}{\funcname{caml\_probably\_raise}'s handler doesn't catch \typename{ExnC}}
  \begin{columns}[c]
    \begin{column}{0.45\textwidth}
      \minipage[c][0.75\textheight][s]{\columnwidth}
      \begin{itemize}
        \item<1-> \funcname{match \dots with}\footnotemark pattern matching can also match exceptions
        \item<1-> The CMM of \funcname{probably\_raise} shows that this \funcname{match \dots with} is implemented with a pattern matching and a \funcname{try \dots with}
        \item<1-> But this one only matches on \typename{ExnA} exception
        \item<2-> Since the pattern matching fails to catch the exception, \funcname{reraise} is called with our current \typename{ExnC} exception
        \item<3-> Disassembly shows that \funcname{reraise} is actually a call to \funcname{caml\_reraise\_exn}, which is also an assembly function provided by the runtime
      \end{itemize}
      \vfill
      \mintocaml[firstline=15,lastline=21,highlightlines={18-21}]{../src/backtrace/backtrace.ml}
      \endminipage
    \end{column}
    \begin{column}{0.55\textwidth}
      \centering
      \onslide*<1>{\mintcmm[highlightlines={10-18}]{../src/backtrace/camlBacktrace\_\_probably\_raise\_351.cmm}}
      \onslide*<2>{\listcmm[highlightlines=18]{../src/backtrace/camlBacktrace\_\_probably\_raise_351.cmm}{CMM of probably\_raise}}
      \onslide*<3>{\mintobjdump[firstline=5,lastline=35,highlightlines=31]{../src/backtrace/camlBacktrace\_\_probably\_raise_351.objdump}}
    \end{column}
  \end{columns}
  \only<1>{\footnotetext[1]{https://blog.janestreet.com/pattern-matching-and-exception-handling-unite/}}
\end{frame}

\newcommand\listCamlReraiseExn[1][]{
  \listasm[firstline=727,lastline=734,#1]{../ocaml/runtime/amd64.S}{runtime/amd64.S:727}}

\begin{frame}{Backtraces}{Introducing \funcname{caml\_reraise\_exn}}
  \begin{columns}[c]
    \begin{column}{0.5\textwidth}
      \minipage[c][0.66\textheight][s]{\columnwidth}
      \begin{itemize}
        \item<1-> The exception have to be reraised to the next handler
        \item<2-> First, checks if the backtrace should be collected with \funcarg{domain\_state->backtrace\_active}
        \only<3>{
        \begin{itemize}
            \item If not, behaves like a \funcname{raise\_notrace} with \funcname{RESTORE\_EXN\_HANDLER\_OCAML}
        \end{itemize}
        }
        \item<4-> Jumps into \funcname{caml\_raise\_exn}
        \item<5-> Calls \funcname{caml\_stash\_backtrace} again, to scan the next part of the stack: from the trap just pop-ed to the next trap
      \end{itemize}
      \vfill
      \mintocaml[firstline=15,lastline=21,highlightlines={18-21}]{../src/backtrace/backtrace.ml}
      \endminipage
    \end{column}
    \begin{column}{0.5\textwidth}
      \raggedleft
      \onslide*<1>{\listCamlReraiseExn{}}
      \onslide*<2>{\listCamlReraiseExn[highlightlines=729]{}}
      \onslide*<3>{
        \mintc[firstline=190,lastline=192,highlightlines={191-192}]{../ocaml/runtime/amd64.S}
        \mintc[firstline=234,lastline=237,highlightlines={235-237}]{../ocaml/runtime/amd64.S}
        \medskip
        \listCamlReraiseExn[highlightlines=731]{}
      }
      \onslide*<4-5>{
        % TODO refactor with \listCamlReraiseExn
        \mintasm[firstline=727,lastline=734,highlightlines=730]{../ocaml/runtime/amd64.S}
        \medskip
      }
      \onslide*<4>{\listCamlRaiseExn[highlightlines=711]{}}
      \onslide*<5>{\listCamlRaiseExn[highlightlines=720]{}}
    \end{column}
  \end{columns}
\end{frame}

\begin{frame}{Backtraces}{Backtrace collection continues}
  \begin{columns}[c]
    \begin{column}{0.55\textwidth}
      \minipage[c][0.75\textheight][s]{\columnwidth}
      \begin{itemize}
        \item<1-> \funcname{caml\_stash\_backtrace} scans the older part of the stack, up to the next trap
        \item<6-> Controls jumps to the nested exception handler in \funcname{maybe\_raise}, which matches only on \typename{ExnB}
        \item<7-> \funcname{caml\_reraise\_exn} is called again, backtrace collection resumes with \funcname{caml\_stash\_backtrace}
      \end{itemize}
      \medskip
      \begin{itemize}
        \item<2-> Constructed backtrace:
        \begin{itemize}
          \item <2-> \funcname{definitely\_raise}
          \item <2-> \funcname{probably\_raise}
          \item <3-> \funcname{probably\_raise} (re-raise)
          \item <4-> \funcname{maybe\_raise}
          \item <5-> \funcname{main}
          \item <8-> \funcname{main} (re-raise)
        \end{itemize}
      \end{itemize}
      \vfill
      \onslide*<1-5>{\mintocaml[firstline=15,lastline=21]{../src/backtrace/backtrace.ml}}
      \onslide*<6>{\mintocaml[firstline=28,lastline=34,highlightlines=31]{../src/backtrace/backtrace.ml}}
      \onslide*<7->{\mintocaml[firstline=28,lastline=34,highlightlines=33]{../src/backtrace/backtrace.ml}}
      \endminipage
    \end{column}
    \begin{column}{0.45\textwidth}
      \raggedleft
      \onslide*<1>{\providecommand\step{6}\input{diagrams/backtrace/backtrace.tex}}%
      \onslide*<2>{\providecommand\step{6}\input{diagrams/backtrace/backtrace.tex}}%
      \onslide*<3>{\providecommand\step{7}\input{diagrams/backtrace/backtrace.tex}}%
      \onslide*<4>{\providecommand\step{8}\input{diagrams/backtrace/backtrace.tex}}%
      \onslide*<5>{\providecommand\step{9}\input{diagrams/backtrace/backtrace.tex}}%
      \onslide*<6>{\providecommand\step{10}\input{diagrams/backtrace/backtrace.tex}}%
      \onslide*<7>{\providecommand\step{11}\input{diagrams/backtrace/backtrace.tex}}%
      \onslide*<8>{\providecommand\step{12}\input{diagrams/backtrace/backtrace.tex}}%
      \onslide*<9>{\providecommand\step{13}\input{diagrams/backtrace/backtrace.tex}}%
    \end{column}
  \end{columns}
\end{frame}

\begin{frame}{Backtraces}{Printing the backtrace}
  \begin{columns}[c]
    \begin{column}{0.45\textwidth}
      \minipage[c][0.66\textheight][s]{\columnwidth}
      \begin{itemize}
        \item<1-> The exception backtrace can now be printed to \funcarg{stdout} with \funcname{Printexc.print_backtrace}
        \item<2-> First the exception backtrace is retrieved into an OCaml array with \funcname{get_raw_backtrace }
        \item<3-> Which is implemented as the \funcname{caml_get_exception_raw_backtrace} C function in the runtime
        \item<4-> And finally printed with \funcname{print_exception_backtrace}
      \end{itemize}
      \vfill
      \mintocaml[firstline=28,lastline=34,highlightlines=33]{../src/backtrace/backtrace.ml}
      \endminipage
    \end{column}
    \begin{column}{0.55\textwidth}
      \onslide*<1,2>{
        \mintocaml[firstline=100,lastline=101]{../ocaml/stdlib/printexc.ml}
      }
      \only<1>{
        \listocaml[firstline=170,lastline=172]{../ocaml/stdlib/printexc.ml}{stdlib/printexc.ml:170}
      }
      \only<2>{
        \listocaml[firstline=170,lastline=172,highlightlines=172]{../ocaml/stdlib/printexc.ml}{stdlib/printexc.ml:170}
      }
      \only<3>{
        \mintc[firstline=154,lastline=156]{../ocaml/runtime/backtrace.c}
        \listc[firstline=171,lastline=192,highlightlines={184-187}]{../ocaml/runtime/backtrace.c}{runtime/backtrace.c:154}
      }
      \only<4>{
        \listocaml[firstline=155,lastline=165]{../ocaml/stdlib/printexc.ml}{stdlib/printexc.ml:55}
      }
    \end{column}
  \end{columns}
\end{frame}


\subsection{The multiple kinds of raise}
\frameSubsection{}{}

\begin{frame}{Backtraces}{When backtraces are not needed}
  \begin{columns}[c]
    \begin{column}{0.45\textwidth}
      \minipage[c][0.66\textheight][s]{\columnwidth}
      \begin{itemize}
        \item<1-> What if the backtrace for an exception is never needed?
        \item<2-> It's possible to raise the exception explicitly with \funcname{raise\_notrace} to make raising as cheap as possible
        \item<3-> An example of use in the \funcname{dynlink} library of the OCaml compiler
      \end{itemize}
      \vfill
      \onslide<3->{
      \listocaml[firstline=205,lastline=209,highlightlines=207]{../ocaml/otherlibs/dynlink/dynlink\_compilerlibs/misc.ml}{otherlibs/dynlink/dynlink\_compilerlibs/misc.ml:205}
      }
      \endminipage
    \end{column}
    \begin{column}{0.55\textwidth}
    \only<4>{
      \mintcmm[highlightlines=3]{../src/notrace/camlNotrace\_\_fun_347.cmm}
      \listcmm{../src/notrace/camlNotrace\_\_all\_somes\_327.cmm}{all\_somes}
    }
    \onslide*<5->{
      \listobjdump[highlightlines={6-9}]{../src/notrace/camlNotrace\_\_fun_347.objdump}{anonymous function from \funcname{all\_somes}}
    }
    \end{column}
  \end{columns}
\end{frame}

\begin{frame}{Backtraces}{Summary table of the multiple kinds of raise}
  \begin{itemize}
    \item When debugging information is enabled and if the backtrace of an exception isn't needed, it's possible to raise with \funcname{raise\_notrace} for performance
    \item When debugging information is \emph{not} enabled, the compiler optimises \funcname{raise} calls to inline assembly (same effect as using \funcname{raise\_notrace})
  \end{itemize}
  \bigskip
  \centering
  \begin{tabular}{| l | c | c | c | c |}
    \hline
    \parbox{10em}{Debug information \\ (\funcarg{-g} flag)}  & \multicolumn{2}{|c|}{Disabled}                                     & \multicolumn{2}{|c|}{Enabled} \\
    \hline
    Raise kind                                               & \funcname{raise} & \funcname{raise\_notrace}                       & \funcname{raise} & \funcname{raise\_notrace} \\
    \hline
    Compilation                                              & \parbox{8em}{inline assembly\\ (optimization)} & inline assembly   & \funcname{caml\_raise\_exn} & inline assembly \\
    \hline
  \end{tabular}
\end{frame}


%
% Takeaway #5
%
\subsection*{Takeaway \#5}
\frameSubsectionTakeaway{}
\againframe<5>{takeaway}
}%
      \onslide*<7>{\providecommand\step{11}%
% Backtraces
%
\subsection{One advantage of raising with debugging information}
\frameSubsection{}{}

\begin{frame}{Backtraces}{Debugging information \& source code}
  \begin{columns}[c]
    \begin{column}{0.5\textwidth}
      \begin{itemize}
        \item Raising an exception is fun and games, but how do we know where the exception originated? $\rightarrow$ With the backtrace associated with the exception!
        \item In order to get a backtrace from the point where an exception was raised, it's necessary to compile with debugging information enabled (\funcarg{-g} flag for the compiler)
        \item Same example as in the `Nested handlers' section
      \end{itemize}
    \end{column}
    \begin{column}{0.5\textwidth}
      \listocaml[firstline=9,lastline=34]{../src/backtrace/backtrace.ml}{backtrace/backtrace.ml}
    \end{column}
  \end{columns}
\end{frame}

\begin{frame}{Backtraces}{CMM and disassembly of \funcname{definitely\_raise}}
  \begin{columns}[c]
    \begin{column}{0.5\textwidth}
      \minipage[c][0.66\textheight][s]{\columnwidth}
      \begin{itemize}
        \item<1-> An effect of compiling with debugging information (\funcarg{-g}): the CMM no longer uses \funcname{raise\_notrace} to raise the exception but \funcname{raise} instead
        \item<2-> Disassembly shows that CMM \funcname{raise} is actually a call to \funcname{caml\_raise\_exn}, a function in assembler provided by the runtime
      \end{itemize}
      \vfill
      \mintocaml[firstline=9,lastline=13,highlightlines=12]{../src/backtrace/backtrace.ml}
      \endminipage
    \end{column}
    \begin{column}{0.5\textwidth}
      \onslide*<1>{
        \mintcmm[highlightlines=5]{../src/backtrace/camlBacktrace\_\_definitely\_raise\_347.cmm}
      }
      \onslide*<2>{
        \mintobjdump[firstline=2,highlightlines=14]{../src/backtrace/camlBacktrace\_\_definitely\_raise\_347.objdump}
      }
    \end{column}
  \end{columns}
\end{frame}

\begin{frame}{Backtraces}{Introducing \funcname{caml\_raise\_exn}}
  \begin{columns}[c]
    \begin{column}{0.5\textwidth}
      \minipage[c][0.66\textheight][s]{\columnwidth}
      \onslide*<1>{
        \begin{itemize}
          \item Raising the exception is performed using \funcname{caml\_raise\_exn} which is
            \begin{itemize}
              \item An assembly function provided by the runtime
              \item Architecture specific
            \end{itemize}
          \item Let's see what it does differently from \funcname{raise\_notrace}\dots
          \item We are about to raise \typename{ExnC} with \funcname{caml\_raise\_exn} from \funcname{definitely\_raise}
        \end{itemize}
      }
      \onslide*<2>{
        \mintobjdump[firstline=2,highlightlines=14]{../src/backtrace/camlBacktrace\_\_definitely\_raise\_347.objdump}
      }
      \vfill
      \mintocaml[firstline=9,lastline=13,highlightlines=12]{../src/backtrace/backtrace.ml}
      \endminipage
    \end{column}
    \begin{column}{0.5\textwidth}
      \centering
      \providecommand\step{1}\input{diagrams/backtrace/backtrace.tex}
    \end{column}
  \end{columns}
\end{frame}

\begin{frame}{Backtraces}{But before that, we need more fields from \typename{caml\_domain\_state}}
  \begin{columns}
    \begin{column}{0.5\textwidth}
      \begin{itemize}
        \item Remember \typename{caml\_domain\_state} the per-domain runtime state?
        \item Some other members are of interest to us now\dots
        \item \localname{backtrace\_active} is used to know if the runtime should collect backtraces
        \item \localname{backtrace\_active} can be set by
          \begin{itemize}
            \item The \funcarg{b} flag in the \funcname{OCAMLRUNPARAM} environment variable
            \item \mintinline{ocaml}{Printexc.record_backtrace : bool -> unit} \footnotemark
          \end{itemize}
      \end{itemize}
    \end{column}
    \begin{column}{0.5\textwidth}
      \begin{listing}
        \mintc[highlightlines={9-11}]{src/ocaml/domain\_state\_backtrace.h}
        \caption{runtime/caml/domain\_state.h}
      \end{listing}
    \end{column}
  \end{columns}
  \footnotetext[1]{https://v2.ocaml.org/api/Printexc.html}
\end{frame}

\newcommand\listCamlRaiseExn[1][]{
  \listasm[firstline=702,lastline=725,#1]{../ocaml/runtime/amd64.S}{runtime/amd64.S:702}}

\begin{frame}{Backtraces}{Walk through \funcname{caml\_raise\_exn}}
  \begin{columns}[T]
    \begin{column}{0.5\textwidth}
      \begin{itemize}
        \item<1-> First checks whether exceptions backtrace should be recorded
        \item<1-> By checking the \funcarg{domain\_state->backtrace\_active} value
        \item<2-> If not, acts as \funcname{raise\_notrace} with \funcname{RESTORE\_EXN\_HANDLER\_OCAML}
          \begin{itemize}
            \item Set \regname{RSP} to \localname{domain\_state->exn\_handler} value
            \item Restore \localname{domain\_state->exn\_handler} from trap
            \item Jump with \funcname{ret} (same effect as using \funcname{pop} and \funcname{jmp})
          \end{itemize}
        \item<3-> Otherwise, switches to the C stack to be able to execute C code
        \item<4-> Calls \funcname{caml\_stash\_backtrace}, a C function provided by the runtime\ignorespaces
          \onslide<6->{, with
            \begin{itemize}
              \item \funcarg{exn}: exception value
              \item \funcarg{pc}: code address of the raising point
              \item \funcarg{sp}: stack address of the raising function
              \item \funcarg{trapsp}: stack address of the next handler
            \end{itemize}
          }
      \end{itemize}
      \bigskip
      \mintocaml[firstline=9,lastline=13,highlightlines=12]{../src/backtrace/backtrace.ml}
    \end{column}
    \begin{column}{0.5\textwidth}
      \raggedleft
      \onslide*<1,3-4>{
        \mintc[firstline=190,lastline=192]{../ocaml/runtime/amd64.S}
        \mintc[firstline=234,lastline=237]{../ocaml/runtime/amd64.S}
      }
      \onslide*<2>{
        \mintc[firstline=190,lastline=192,highlightlines={191-192}]{../ocaml/runtime/amd64.S}
        \mintc[firstline=234,lastline=237,highlightlines={235-237}]{../ocaml/runtime/amd64.S}
      }
      \onslide*<1>{\listCamlRaiseExn[highlightlines=705]{}}%
      \onslide*<2>{\listCamlRaiseExn[highlightlines={707-708}]{}}%
      \onslide*<3>{\listCamlRaiseExn[highlightlines={712-714}]{}}%
      \onslide*<4>{\listCamlRaiseExn[highlightlines={715-720}]{}}%
      \onslide*<5>{\providecommand\step{1}\input{diagrams/backtrace/backtrace.tex}}%
      \onslide*<6>{\providecommand\step{2}\input{diagrams/backtrace/backtrace.tex}}%
    \end{column}
  \end{columns}
\end{frame}

\newcommand\listStashBacktrace[1][]{
  \listc[firstline=91,lastline=120,#1]{../ocaml/runtime/backtrace\_nat.c}{runtime/backtrace\_nat.c:91}}

\begin{frame}{Backtraces}{Introducing \funcname{caml\_stash\_backtrace}}
  \begin{columns}[c]
    \begin{column}{0.5\textwidth}
      \minipage[c][0.66\textheight][s]{\columnwidth}
      \begin{itemize}
        \item<1-> A C function provided by the runtime (specific to native code)
        \item<2-> It scans the stack, between the stack pointer at the raising point up to the next trap, in order to collect the names of the detected function frames
          \onslide*<2>{
            \begin{itemize}
              \item And more information, like line number, column number...
            \end{itemize}
          }
        \item<3-> It uses the same technique and information as the Garbage Collector (GC) uses to scan the values live on the stack
          \onslide*<4>{
            \begin{itemize}
              \item \typename{frame\_descr} are at the core of stack unwinding
            \end{itemize}
          }
      \end{itemize}
      \vfill
      \mintocaml[firstline=9,lastline=13,highlightlines=12]{../src/backtrace/backtrace.ml}
      \endminipage
    \end{column}
    \begin{column}{0.5\textwidth}
      \onslide*<1>{\listStashBacktrace[highlightlines=91]{}}
      \onslide*<2>{\listStashBacktrace[highlightlines={91,117-118}]{}}
      \onslide*<3->{\listStashBacktrace[highlightlines={94,106,109-110}]{}}
    \end{column}
  \end{columns}
\end{frame}

\newcommand\listFrameDescr[1][]{
  \listocaml[firstline=111,lastline=124,#1]{../ocaml/asmcomp/emitaux.ml}{asmcomp/emitaux.ml:111}}
\newcommand\listDebugInfo[1][]{
  \listocaml[firstline=38,lastline=49,#1]{../ocaml/lambda/debuginfo.mli}{lambda/debuginfo.mli:38}}

\begin{frame}{Backtraces}{\typename{frame\_descr} and \typename{frame\_debuginfo} in a nutshell}
  \begin{columns}[c]
    \begin{column}{0.5\textwidth}
      \begin{itemize}
        \item<1-> For every return address in an OCaml program, there is a associated \typename{frame\_descr}
        \item<2-> A \typename{frame\_descr} specifies
        \begin{itemize}
          \item<2-> The size of the function stack frame
          \item<3-> Where live values are on the stack (so that the GC can mark them as live and prevent from being swept)
        \end{itemize}
        \item<4-> When compiled with debug information, \typename{frame\_descr} will be linked to a \typename{frame\_debuginfo}
        \begin{itemize}
          \item<5-> Contains information about the position in the source code (file name, line and column number)
        \end{itemize}
      \end{itemize}
    \end{column}
    \begin{column}{0.5\textwidth}
        \onslide*<1>{\listFrameDescr[highlightlines={118-122}]{}}
        \onslide*<2>{\listFrameDescr[highlightlines={120}]{}}
        \onslide*<3>{\listFrameDescr[highlightlines={121}]{}}
        \onslide*<4->{\listFrameDescr[highlightlines={122}]{}}
        \medskip
        \onslide*<1-3>{\listDebugInfo{}}
        \onslide*<4>{\listDebugInfo[highlightlines={38,49}]{}}
        \onslide*<5>{\listDebugInfo[highlightlines={39-46}]{}}
    \end{column}
  \end{columns}
\end{frame}

\begin{frame}{Backtraces}{Scanning the stack with \typename{frame\_descr}}
  \begin{columns}[c]
    \begin{column}{0.5\textwidth}
      \begin{itemize}
        \item Given the return address of the code that raises the exception by calling \funcname{caml\_raise\_exn} (\funcarg{pc} argument of \funcname{caml\_stash\_backtrace})
        \item And the position of the stack pointer (\funcarg{sp} argument of \funcname{caml\_stash\_backtrace})
        \item The frame size given by \typename{frame\_descr}.\funcarg{frame\_size}, allows to find the end of the previous frame
        \item And the return address of the caller of this function
        \bigskip
        \onslide<3->{
        \item Note that traps (here the reddish \funcname{ExnA}, \funcname{ExnB} and \funcname{ExnC} blocks) belong to the frame that installed them, and so the frame size changes when traps are removed
        }
      \end{itemize}
    \end{column}
    \begin{column}{0.5\textwidth}
      \raggedleft
      \onslide*<1>{\providecommand\step{2}\input{diagrams/backtrace/backtrace.tex}}%
      \onslide*<2>{\providecommand\step{1}\input{diagrams/backtrace/scanstack.tex}}%
      \onslide*<3>{\providecommand\step{2}\input{diagrams/backtrace/scanstack.tex}}%
      \onslide*<4>{\providecommand\step{3}\input{diagrams/backtrace/scanstack.tex}}%
      \onslide*<5>{\providecommand\step{4}\input{diagrams/backtrace/scanstack.tex}}%
      \onslide*<6>{\providecommand\step{5}\input{diagrams/backtrace/scanstack.tex}}%
    \end{column}
  \end{columns}
\end{frame}

% Duplicate of previous-previous slide
\begin{frame}{Backtraces}{Continuing on \funcname{caml\_stash\_backtrace}}
  \begin{columns}[c]
    \begin{column}{0.5\textwidth}
      \minipage[c][0.75\textheight][s]{\columnwidth}
      \begin{itemize}
          \color{gray}
        \item A C function provided by the runtime (specific to native code)
        \item It scans the stack, between the stack pointer at the raising point up to the next trap, in order to collect the names of the detected function frames
        \item It uses the same technique and information as the Garbage Collector (GC) uses to scan the values live on the stack
      \end{itemize}
      \begin{itemize}
        \item For each function frame detected, store a pointer to the \typename{frame\_descr} in the \localname{domain\_state->backtrace\_buffer} array, effectively constructing the backtrace
      \end{itemize}
      \onslide<2->{
        \begin{itemize}
            \smallskip
          \item Constructed backtrace:
            \funcname{definitely\_raise},
            \onslide<3->{\funcname{probably\_raise}}
        \end{itemize}
      }
      \vfill
      \mintocaml[firstline=9,lastline=13,highlightlines=12]{../src/backtrace/backtrace.ml}
      \endminipage
    \end{column}
    \begin{column}{0.5\textwidth}
      \raggedleft
      \onslide*<1>{\listStashBacktrace[highlightlines={114-115}]{}}
      \onslide*<2>{\providecommand\step{3}\input{diagrams/backtrace/backtrace.tex}}%
      \onslide*<3>{\providecommand\step{4}\input{diagrams/backtrace/backtrace.tex}}%
    \end{column}
  \end{columns}
\end{frame}

\begin{frame}{Backtraces}{Back to \funcname{caml\_raise\_exn}}
  \begin{columns}[c]
    \begin{column}{0.5\textwidth}
      \minipage[c][0.66\textheight][s]{\columnwidth}
      \begin{itemize}\color{gray}
        \item Checks wheteher exceptions backtrace should be recorded, by checking the \funcarg{domain\_state->backtrace\_active} value
        \item Switches to the C stack to be able to execute C code
        \item Calls \funcname{caml\_stash\_backtrace}, to collect the backtrace up to the next trap
      \end{itemize}
      \onslide*<2->{
        \begin{itemize}
          \item<2-> Executes the exception handler in \funcname{probably\_raise} with \funcname{RESTORE\_EXN\_HANDLER\_OCAML}
            \begin{itemize}
              \item Set \regname{RSP} to \localname{domain\_state->exn\_handler} value
              \item Restore \localname{domain\_state->exn\_handler} from trap
              \item Jump to the handler code with \funcname{ret}
            \end{itemize}
        \end{itemize}
      }
      \vfill
      \onslide*<1>{\mintocaml[firstline=9,lastline=13,highlightlines=12]{../src/backtrace/backtrace.ml}}
      \onslide*<2->{\mintocaml[firstline=15,lastline=21,highlightlines={19-21}]{../src/backtrace/backtrace.ml}}
      \endminipage
    \end{column}
    \begin{column}{0.5\textwidth}
      \centering
      \onslide*<1>{
        \mintc[firstline=190,lastline=192]{../ocaml/runtime/amd64.S}
        \mintc[firstline=234,lastline=237]{../ocaml/runtime/amd64.S}
      }
      \onslide*<2>{
        \mintc[firstline=190,lastline=192,highlightlines={191-192}]{../ocaml/runtime/amd64.S}
        \mintc[firstline=234,lastline=237,highlightlines={235-237}]{../ocaml/runtime/amd64.S}
      }
      \onslide*<1>{\listCamlRaiseExn[highlightlines=720]{}}
      \onslide*<2>{\listCamlRaiseExn[highlightlines=722]{}}
      \onslide*<3->{\providecommand\step{5}\input{diagrams/backtrace/backtrace.tex}}
    \end{column}
  \end{columns}
\end{frame}

\begin{frame}{Backtraces}{\funcname{caml\_probably\_raise}'s handler doesn't catch \typename{ExnC}}
  \begin{columns}[c]
    \begin{column}{0.45\textwidth}
      \minipage[c][0.75\textheight][s]{\columnwidth}
      \begin{itemize}
        \item<1-> \funcname{match \dots with}\footnotemark pattern matching can also match exceptions
        \item<1-> The CMM of \funcname{probably\_raise} shows that this \funcname{match \dots with} is implemented with a pattern matching and a \funcname{try \dots with}
        \item<1-> But this one only matches on \typename{ExnA} exception
        \item<2-> Since the pattern matching fails to catch the exception, \funcname{reraise} is called with our current \typename{ExnC} exception
        \item<3-> Disassembly shows that \funcname{reraise} is actually a call to \funcname{caml\_reraise\_exn}, which is also an assembly function provided by the runtime
      \end{itemize}
      \vfill
      \mintocaml[firstline=15,lastline=21,highlightlines={18-21}]{../src/backtrace/backtrace.ml}
      \endminipage
    \end{column}
    \begin{column}{0.55\textwidth}
      \centering
      \onslide*<1>{\mintcmm[highlightlines={10-18}]{../src/backtrace/camlBacktrace\_\_probably\_raise\_351.cmm}}
      \onslide*<2>{\listcmm[highlightlines=18]{../src/backtrace/camlBacktrace\_\_probably\_raise_351.cmm}{CMM of probably\_raise}}
      \onslide*<3>{\mintobjdump[firstline=5,lastline=35,highlightlines=31]{../src/backtrace/camlBacktrace\_\_probably\_raise_351.objdump}}
    \end{column}
  \end{columns}
  \only<1>{\footnotetext[1]{https://blog.janestreet.com/pattern-matching-and-exception-handling-unite/}}
\end{frame}

\newcommand\listCamlReraiseExn[1][]{
  \listasm[firstline=727,lastline=734,#1]{../ocaml/runtime/amd64.S}{runtime/amd64.S:727}}

\begin{frame}{Backtraces}{Introducing \funcname{caml\_reraise\_exn}}
  \begin{columns}[c]
    \begin{column}{0.5\textwidth}
      \minipage[c][0.66\textheight][s]{\columnwidth}
      \begin{itemize}
        \item<1-> The exception have to be reraised to the next handler
        \item<2-> First, checks if the backtrace should be collected with \funcarg{domain\_state->backtrace\_active}
        \only<3>{
        \begin{itemize}
            \item If not, behaves like a \funcname{raise\_notrace} with \funcname{RESTORE\_EXN\_HANDLER\_OCAML}
        \end{itemize}
        }
        \item<4-> Jumps into \funcname{caml\_raise\_exn}
        \item<5-> Calls \funcname{caml\_stash\_backtrace} again, to scan the next part of the stack: from the trap just pop-ed to the next trap
      \end{itemize}
      \vfill
      \mintocaml[firstline=15,lastline=21,highlightlines={18-21}]{../src/backtrace/backtrace.ml}
      \endminipage
    \end{column}
    \begin{column}{0.5\textwidth}
      \raggedleft
      \onslide*<1>{\listCamlReraiseExn{}}
      \onslide*<2>{\listCamlReraiseExn[highlightlines=729]{}}
      \onslide*<3>{
        \mintc[firstline=190,lastline=192,highlightlines={191-192}]{../ocaml/runtime/amd64.S}
        \mintc[firstline=234,lastline=237,highlightlines={235-237}]{../ocaml/runtime/amd64.S}
        \medskip
        \listCamlReraiseExn[highlightlines=731]{}
      }
      \onslide*<4-5>{
        % TODO refactor with \listCamlReraiseExn
        \mintasm[firstline=727,lastline=734,highlightlines=730]{../ocaml/runtime/amd64.S}
        \medskip
      }
      \onslide*<4>{\listCamlRaiseExn[highlightlines=711]{}}
      \onslide*<5>{\listCamlRaiseExn[highlightlines=720]{}}
    \end{column}
  \end{columns}
\end{frame}

\begin{frame}{Backtraces}{Backtrace collection continues}
  \begin{columns}[c]
    \begin{column}{0.55\textwidth}
      \minipage[c][0.75\textheight][s]{\columnwidth}
      \begin{itemize}
        \item<1-> \funcname{caml\_stash\_backtrace} scans the older part of the stack, up to the next trap
        \item<6-> Controls jumps to the nested exception handler in \funcname{maybe\_raise}, which matches only on \typename{ExnB}
        \item<7-> \funcname{caml\_reraise\_exn} is called again, backtrace collection resumes with \funcname{caml\_stash\_backtrace}
      \end{itemize}
      \medskip
      \begin{itemize}
        \item<2-> Constructed backtrace:
        \begin{itemize}
          \item <2-> \funcname{definitely\_raise}
          \item <2-> \funcname{probably\_raise}
          \item <3-> \funcname{probably\_raise} (re-raise)
          \item <4-> \funcname{maybe\_raise}
          \item <5-> \funcname{main}
          \item <8-> \funcname{main} (re-raise)
        \end{itemize}
      \end{itemize}
      \vfill
      \onslide*<1-5>{\mintocaml[firstline=15,lastline=21]{../src/backtrace/backtrace.ml}}
      \onslide*<6>{\mintocaml[firstline=28,lastline=34,highlightlines=31]{../src/backtrace/backtrace.ml}}
      \onslide*<7->{\mintocaml[firstline=28,lastline=34,highlightlines=33]{../src/backtrace/backtrace.ml}}
      \endminipage
    \end{column}
    \begin{column}{0.45\textwidth}
      \raggedleft
      \onslide*<1>{\providecommand\step{6}\input{diagrams/backtrace/backtrace.tex}}%
      \onslide*<2>{\providecommand\step{6}\input{diagrams/backtrace/backtrace.tex}}%
      \onslide*<3>{\providecommand\step{7}\input{diagrams/backtrace/backtrace.tex}}%
      \onslide*<4>{\providecommand\step{8}\input{diagrams/backtrace/backtrace.tex}}%
      \onslide*<5>{\providecommand\step{9}\input{diagrams/backtrace/backtrace.tex}}%
      \onslide*<6>{\providecommand\step{10}\input{diagrams/backtrace/backtrace.tex}}%
      \onslide*<7>{\providecommand\step{11}\input{diagrams/backtrace/backtrace.tex}}%
      \onslide*<8>{\providecommand\step{12}\input{diagrams/backtrace/backtrace.tex}}%
      \onslide*<9>{\providecommand\step{13}\input{diagrams/backtrace/backtrace.tex}}%
    \end{column}
  \end{columns}
\end{frame}

\begin{frame}{Backtraces}{Printing the backtrace}
  \begin{columns}[c]
    \begin{column}{0.45\textwidth}
      \minipage[c][0.66\textheight][s]{\columnwidth}
      \begin{itemize}
        \item<1-> The exception backtrace can now be printed to \funcarg{stdout} with \funcname{Printexc.print_backtrace}
        \item<2-> First the exception backtrace is retrieved into an OCaml array with \funcname{get_raw_backtrace }
        \item<3-> Which is implemented as the \funcname{caml_get_exception_raw_backtrace} C function in the runtime
        \item<4-> And finally printed with \funcname{print_exception_backtrace}
      \end{itemize}
      \vfill
      \mintocaml[firstline=28,lastline=34,highlightlines=33]{../src/backtrace/backtrace.ml}
      \endminipage
    \end{column}
    \begin{column}{0.55\textwidth}
      \onslide*<1,2>{
        \mintocaml[firstline=100,lastline=101]{../ocaml/stdlib/printexc.ml}
      }
      \only<1>{
        \listocaml[firstline=170,lastline=172]{../ocaml/stdlib/printexc.ml}{stdlib/printexc.ml:170}
      }
      \only<2>{
        \listocaml[firstline=170,lastline=172,highlightlines=172]{../ocaml/stdlib/printexc.ml}{stdlib/printexc.ml:170}
      }
      \only<3>{
        \mintc[firstline=154,lastline=156]{../ocaml/runtime/backtrace.c}
        \listc[firstline=171,lastline=192,highlightlines={184-187}]{../ocaml/runtime/backtrace.c}{runtime/backtrace.c:154}
      }
      \only<4>{
        \listocaml[firstline=155,lastline=165]{../ocaml/stdlib/printexc.ml}{stdlib/printexc.ml:55}
      }
    \end{column}
  \end{columns}
\end{frame}


\subsection{The multiple kinds of raise}
\frameSubsection{}{}

\begin{frame}{Backtraces}{When backtraces are not needed}
  \begin{columns}[c]
    \begin{column}{0.45\textwidth}
      \minipage[c][0.66\textheight][s]{\columnwidth}
      \begin{itemize}
        \item<1-> What if the backtrace for an exception is never needed?
        \item<2-> It's possible to raise the exception explicitly with \funcname{raise\_notrace} to make raising as cheap as possible
        \item<3-> An example of use in the \funcname{dynlink} library of the OCaml compiler
      \end{itemize}
      \vfill
      \onslide<3->{
      \listocaml[firstline=205,lastline=209,highlightlines=207]{../ocaml/otherlibs/dynlink/dynlink\_compilerlibs/misc.ml}{otherlibs/dynlink/dynlink\_compilerlibs/misc.ml:205}
      }
      \endminipage
    \end{column}
    \begin{column}{0.55\textwidth}
    \only<4>{
      \mintcmm[highlightlines=3]{../src/notrace/camlNotrace\_\_fun_347.cmm}
      \listcmm{../src/notrace/camlNotrace\_\_all\_somes\_327.cmm}{all\_somes}
    }
    \onslide*<5->{
      \listobjdump[highlightlines={6-9}]{../src/notrace/camlNotrace\_\_fun_347.objdump}{anonymous function from \funcname{all\_somes}}
    }
    \end{column}
  \end{columns}
\end{frame}

\begin{frame}{Backtraces}{Summary table of the multiple kinds of raise}
  \begin{itemize}
    \item When debugging information is enabled and if the backtrace of an exception isn't needed, it's possible to raise with \funcname{raise\_notrace} for performance
    \item When debugging information is \emph{not} enabled, the compiler optimises \funcname{raise} calls to inline assembly (same effect as using \funcname{raise\_notrace})
  \end{itemize}
  \bigskip
  \centering
  \begin{tabular}{| l | c | c | c | c |}
    \hline
    \parbox{10em}{Debug information \\ (\funcarg{-g} flag)}  & \multicolumn{2}{|c|}{Disabled}                                     & \multicolumn{2}{|c|}{Enabled} \\
    \hline
    Raise kind                                               & \funcname{raise} & \funcname{raise\_notrace}                       & \funcname{raise} & \funcname{raise\_notrace} \\
    \hline
    Compilation                                              & \parbox{8em}{inline assembly\\ (optimization)} & inline assembly   & \funcname{caml\_raise\_exn} & inline assembly \\
    \hline
  \end{tabular}
\end{frame}


%
% Takeaway #5
%
\subsection*{Takeaway \#5}
\frameSubsectionTakeaway{}
\againframe<5>{takeaway}
}%
      \onslide*<8>{\providecommand\step{12}%
% Backtraces
%
\subsection{One advantage of raising with debugging information}
\frameSubsection{}{}

\begin{frame}{Backtraces}{Debugging information \& source code}
  \begin{columns}[c]
    \begin{column}{0.5\textwidth}
      \begin{itemize}
        \item Raising an exception is fun and games, but how do we know where the exception originated? $\rightarrow$ With the backtrace associated with the exception!
        \item In order to get a backtrace from the point where an exception was raised, it's necessary to compile with debugging information enabled (\funcarg{-g} flag for the compiler)
        \item Same example as in the `Nested handlers' section
      \end{itemize}
    \end{column}
    \begin{column}{0.5\textwidth}
      \listocaml[firstline=9,lastline=34]{../src/backtrace/backtrace.ml}{backtrace/backtrace.ml}
    \end{column}
  \end{columns}
\end{frame}

\begin{frame}{Backtraces}{CMM and disassembly of \funcname{definitely\_raise}}
  \begin{columns}[c]
    \begin{column}{0.5\textwidth}
      \minipage[c][0.66\textheight][s]{\columnwidth}
      \begin{itemize}
        \item<1-> An effect of compiling with debugging information (\funcarg{-g}): the CMM no longer uses \funcname{raise\_notrace} to raise the exception but \funcname{raise} instead
        \item<2-> Disassembly shows that CMM \funcname{raise} is actually a call to \funcname{caml\_raise\_exn}, a function in assembler provided by the runtime
      \end{itemize}
      \vfill
      \mintocaml[firstline=9,lastline=13,highlightlines=12]{../src/backtrace/backtrace.ml}
      \endminipage
    \end{column}
    \begin{column}{0.5\textwidth}
      \onslide*<1>{
        \mintcmm[highlightlines=5]{../src/backtrace/camlBacktrace\_\_definitely\_raise\_347.cmm}
      }
      \onslide*<2>{
        \mintobjdump[firstline=2,highlightlines=14]{../src/backtrace/camlBacktrace\_\_definitely\_raise\_347.objdump}
      }
    \end{column}
  \end{columns}
\end{frame}

\begin{frame}{Backtraces}{Introducing \funcname{caml\_raise\_exn}}
  \begin{columns}[c]
    \begin{column}{0.5\textwidth}
      \minipage[c][0.66\textheight][s]{\columnwidth}
      \onslide*<1>{
        \begin{itemize}
          \item Raising the exception is performed using \funcname{caml\_raise\_exn} which is
            \begin{itemize}
              \item An assembly function provided by the runtime
              \item Architecture specific
            \end{itemize}
          \item Let's see what it does differently from \funcname{raise\_notrace}\dots
          \item We are about to raise \typename{ExnC} with \funcname{caml\_raise\_exn} from \funcname{definitely\_raise}
        \end{itemize}
      }
      \onslide*<2>{
        \mintobjdump[firstline=2,highlightlines=14]{../src/backtrace/camlBacktrace\_\_definitely\_raise\_347.objdump}
      }
      \vfill
      \mintocaml[firstline=9,lastline=13,highlightlines=12]{../src/backtrace/backtrace.ml}
      \endminipage
    \end{column}
    \begin{column}{0.5\textwidth}
      \centering
      \providecommand\step{1}\input{diagrams/backtrace/backtrace.tex}
    \end{column}
  \end{columns}
\end{frame}

\begin{frame}{Backtraces}{But before that, we need more fields from \typename{caml\_domain\_state}}
  \begin{columns}
    \begin{column}{0.5\textwidth}
      \begin{itemize}
        \item Remember \typename{caml\_domain\_state} the per-domain runtime state?
        \item Some other members are of interest to us now\dots
        \item \localname{backtrace\_active} is used to know if the runtime should collect backtraces
        \item \localname{backtrace\_active} can be set by
          \begin{itemize}
            \item The \funcarg{b} flag in the \funcname{OCAMLRUNPARAM} environment variable
            \item \mintinline{ocaml}{Printexc.record_backtrace : bool -> unit} \footnotemark
          \end{itemize}
      \end{itemize}
    \end{column}
    \begin{column}{0.5\textwidth}
      \begin{listing}
        \mintc[highlightlines={9-11}]{src/ocaml/domain\_state\_backtrace.h}
        \caption{runtime/caml/domain\_state.h}
      \end{listing}
    \end{column}
  \end{columns}
  \footnotetext[1]{https://v2.ocaml.org/api/Printexc.html}
\end{frame}

\newcommand\listCamlRaiseExn[1][]{
  \listasm[firstline=702,lastline=725,#1]{../ocaml/runtime/amd64.S}{runtime/amd64.S:702}}

\begin{frame}{Backtraces}{Walk through \funcname{caml\_raise\_exn}}
  \begin{columns}[T]
    \begin{column}{0.5\textwidth}
      \begin{itemize}
        \item<1-> First checks whether exceptions backtrace should be recorded
        \item<1-> By checking the \funcarg{domain\_state->backtrace\_active} value
        \item<2-> If not, acts as \funcname{raise\_notrace} with \funcname{RESTORE\_EXN\_HANDLER\_OCAML}
          \begin{itemize}
            \item Set \regname{RSP} to \localname{domain\_state->exn\_handler} value
            \item Restore \localname{domain\_state->exn\_handler} from trap
            \item Jump with \funcname{ret} (same effect as using \funcname{pop} and \funcname{jmp})
          \end{itemize}
        \item<3-> Otherwise, switches to the C stack to be able to execute C code
        \item<4-> Calls \funcname{caml\_stash\_backtrace}, a C function provided by the runtime\ignorespaces
          \onslide<6->{, with
            \begin{itemize}
              \item \funcarg{exn}: exception value
              \item \funcarg{pc}: code address of the raising point
              \item \funcarg{sp}: stack address of the raising function
              \item \funcarg{trapsp}: stack address of the next handler
            \end{itemize}
          }
      \end{itemize}
      \bigskip
      \mintocaml[firstline=9,lastline=13,highlightlines=12]{../src/backtrace/backtrace.ml}
    \end{column}
    \begin{column}{0.5\textwidth}
      \raggedleft
      \onslide*<1,3-4>{
        \mintc[firstline=190,lastline=192]{../ocaml/runtime/amd64.S}
        \mintc[firstline=234,lastline=237]{../ocaml/runtime/amd64.S}
      }
      \onslide*<2>{
        \mintc[firstline=190,lastline=192,highlightlines={191-192}]{../ocaml/runtime/amd64.S}
        \mintc[firstline=234,lastline=237,highlightlines={235-237}]{../ocaml/runtime/amd64.S}
      }
      \onslide*<1>{\listCamlRaiseExn[highlightlines=705]{}}%
      \onslide*<2>{\listCamlRaiseExn[highlightlines={707-708}]{}}%
      \onslide*<3>{\listCamlRaiseExn[highlightlines={712-714}]{}}%
      \onslide*<4>{\listCamlRaiseExn[highlightlines={715-720}]{}}%
      \onslide*<5>{\providecommand\step{1}\input{diagrams/backtrace/backtrace.tex}}%
      \onslide*<6>{\providecommand\step{2}\input{diagrams/backtrace/backtrace.tex}}%
    \end{column}
  \end{columns}
\end{frame}

\newcommand\listStashBacktrace[1][]{
  \listc[firstline=91,lastline=120,#1]{../ocaml/runtime/backtrace\_nat.c}{runtime/backtrace\_nat.c:91}}

\begin{frame}{Backtraces}{Introducing \funcname{caml\_stash\_backtrace}}
  \begin{columns}[c]
    \begin{column}{0.5\textwidth}
      \minipage[c][0.66\textheight][s]{\columnwidth}
      \begin{itemize}
        \item<1-> A C function provided by the runtime (specific to native code)
        \item<2-> It scans the stack, between the stack pointer at the raising point up to the next trap, in order to collect the names of the detected function frames
          \onslide*<2>{
            \begin{itemize}
              \item And more information, like line number, column number...
            \end{itemize}
          }
        \item<3-> It uses the same technique and information as the Garbage Collector (GC) uses to scan the values live on the stack
          \onslide*<4>{
            \begin{itemize}
              \item \typename{frame\_descr} are at the core of stack unwinding
            \end{itemize}
          }
      \end{itemize}
      \vfill
      \mintocaml[firstline=9,lastline=13,highlightlines=12]{../src/backtrace/backtrace.ml}
      \endminipage
    \end{column}
    \begin{column}{0.5\textwidth}
      \onslide*<1>{\listStashBacktrace[highlightlines=91]{}}
      \onslide*<2>{\listStashBacktrace[highlightlines={91,117-118}]{}}
      \onslide*<3->{\listStashBacktrace[highlightlines={94,106,109-110}]{}}
    \end{column}
  \end{columns}
\end{frame}

\newcommand\listFrameDescr[1][]{
  \listocaml[firstline=111,lastline=124,#1]{../ocaml/asmcomp/emitaux.ml}{asmcomp/emitaux.ml:111}}
\newcommand\listDebugInfo[1][]{
  \listocaml[firstline=38,lastline=49,#1]{../ocaml/lambda/debuginfo.mli}{lambda/debuginfo.mli:38}}

\begin{frame}{Backtraces}{\typename{frame\_descr} and \typename{frame\_debuginfo} in a nutshell}
  \begin{columns}[c]
    \begin{column}{0.5\textwidth}
      \begin{itemize}
        \item<1-> For every return address in an OCaml program, there is a associated \typename{frame\_descr}
        \item<2-> A \typename{frame\_descr} specifies
        \begin{itemize}
          \item<2-> The size of the function stack frame
          \item<3-> Where live values are on the stack (so that the GC can mark them as live and prevent from being swept)
        \end{itemize}
        \item<4-> When compiled with debug information, \typename{frame\_descr} will be linked to a \typename{frame\_debuginfo}
        \begin{itemize}
          \item<5-> Contains information about the position in the source code (file name, line and column number)
        \end{itemize}
      \end{itemize}
    \end{column}
    \begin{column}{0.5\textwidth}
        \onslide*<1>{\listFrameDescr[highlightlines={118-122}]{}}
        \onslide*<2>{\listFrameDescr[highlightlines={120}]{}}
        \onslide*<3>{\listFrameDescr[highlightlines={121}]{}}
        \onslide*<4->{\listFrameDescr[highlightlines={122}]{}}
        \medskip
        \onslide*<1-3>{\listDebugInfo{}}
        \onslide*<4>{\listDebugInfo[highlightlines={38,49}]{}}
        \onslide*<5>{\listDebugInfo[highlightlines={39-46}]{}}
    \end{column}
  \end{columns}
\end{frame}

\begin{frame}{Backtraces}{Scanning the stack with \typename{frame\_descr}}
  \begin{columns}[c]
    \begin{column}{0.5\textwidth}
      \begin{itemize}
        \item Given the return address of the code that raises the exception by calling \funcname{caml\_raise\_exn} (\funcarg{pc} argument of \funcname{caml\_stash\_backtrace})
        \item And the position of the stack pointer (\funcarg{sp} argument of \funcname{caml\_stash\_backtrace})
        \item The frame size given by \typename{frame\_descr}.\funcarg{frame\_size}, allows to find the end of the previous frame
        \item And the return address of the caller of this function
        \bigskip
        \onslide<3->{
        \item Note that traps (here the reddish \funcname{ExnA}, \funcname{ExnB} and \funcname{ExnC} blocks) belong to the frame that installed them, and so the frame size changes when traps are removed
        }
      \end{itemize}
    \end{column}
    \begin{column}{0.5\textwidth}
      \raggedleft
      \onslide*<1>{\providecommand\step{2}\input{diagrams/backtrace/backtrace.tex}}%
      \onslide*<2>{\providecommand\step{1}\input{diagrams/backtrace/scanstack.tex}}%
      \onslide*<3>{\providecommand\step{2}\input{diagrams/backtrace/scanstack.tex}}%
      \onslide*<4>{\providecommand\step{3}\input{diagrams/backtrace/scanstack.tex}}%
      \onslide*<5>{\providecommand\step{4}\input{diagrams/backtrace/scanstack.tex}}%
      \onslide*<6>{\providecommand\step{5}\input{diagrams/backtrace/scanstack.tex}}%
    \end{column}
  \end{columns}
\end{frame}

% Duplicate of previous-previous slide
\begin{frame}{Backtraces}{Continuing on \funcname{caml\_stash\_backtrace}}
  \begin{columns}[c]
    \begin{column}{0.5\textwidth}
      \minipage[c][0.75\textheight][s]{\columnwidth}
      \begin{itemize}
          \color{gray}
        \item A C function provided by the runtime (specific to native code)
        \item It scans the stack, between the stack pointer at the raising point up to the next trap, in order to collect the names of the detected function frames
        \item It uses the same technique and information as the Garbage Collector (GC) uses to scan the values live on the stack
      \end{itemize}
      \begin{itemize}
        \item For each function frame detected, store a pointer to the \typename{frame\_descr} in the \localname{domain\_state->backtrace\_buffer} array, effectively constructing the backtrace
      \end{itemize}
      \onslide<2->{
        \begin{itemize}
            \smallskip
          \item Constructed backtrace:
            \funcname{definitely\_raise},
            \onslide<3->{\funcname{probably\_raise}}
        \end{itemize}
      }
      \vfill
      \mintocaml[firstline=9,lastline=13,highlightlines=12]{../src/backtrace/backtrace.ml}
      \endminipage
    \end{column}
    \begin{column}{0.5\textwidth}
      \raggedleft
      \onslide*<1>{\listStashBacktrace[highlightlines={114-115}]{}}
      \onslide*<2>{\providecommand\step{3}\input{diagrams/backtrace/backtrace.tex}}%
      \onslide*<3>{\providecommand\step{4}\input{diagrams/backtrace/backtrace.tex}}%
    \end{column}
  \end{columns}
\end{frame}

\begin{frame}{Backtraces}{Back to \funcname{caml\_raise\_exn}}
  \begin{columns}[c]
    \begin{column}{0.5\textwidth}
      \minipage[c][0.66\textheight][s]{\columnwidth}
      \begin{itemize}\color{gray}
        \item Checks wheteher exceptions backtrace should be recorded, by checking the \funcarg{domain\_state->backtrace\_active} value
        \item Switches to the C stack to be able to execute C code
        \item Calls \funcname{caml\_stash\_backtrace}, to collect the backtrace up to the next trap
      \end{itemize}
      \onslide*<2->{
        \begin{itemize}
          \item<2-> Executes the exception handler in \funcname{probably\_raise} with \funcname{RESTORE\_EXN\_HANDLER\_OCAML}
            \begin{itemize}
              \item Set \regname{RSP} to \localname{domain\_state->exn\_handler} value
              \item Restore \localname{domain\_state->exn\_handler} from trap
              \item Jump to the handler code with \funcname{ret}
            \end{itemize}
        \end{itemize}
      }
      \vfill
      \onslide*<1>{\mintocaml[firstline=9,lastline=13,highlightlines=12]{../src/backtrace/backtrace.ml}}
      \onslide*<2->{\mintocaml[firstline=15,lastline=21,highlightlines={19-21}]{../src/backtrace/backtrace.ml}}
      \endminipage
    \end{column}
    \begin{column}{0.5\textwidth}
      \centering
      \onslide*<1>{
        \mintc[firstline=190,lastline=192]{../ocaml/runtime/amd64.S}
        \mintc[firstline=234,lastline=237]{../ocaml/runtime/amd64.S}
      }
      \onslide*<2>{
        \mintc[firstline=190,lastline=192,highlightlines={191-192}]{../ocaml/runtime/amd64.S}
        \mintc[firstline=234,lastline=237,highlightlines={235-237}]{../ocaml/runtime/amd64.S}
      }
      \onslide*<1>{\listCamlRaiseExn[highlightlines=720]{}}
      \onslide*<2>{\listCamlRaiseExn[highlightlines=722]{}}
      \onslide*<3->{\providecommand\step{5}\input{diagrams/backtrace/backtrace.tex}}
    \end{column}
  \end{columns}
\end{frame}

\begin{frame}{Backtraces}{\funcname{caml\_probably\_raise}'s handler doesn't catch \typename{ExnC}}
  \begin{columns}[c]
    \begin{column}{0.45\textwidth}
      \minipage[c][0.75\textheight][s]{\columnwidth}
      \begin{itemize}
        \item<1-> \funcname{match \dots with}\footnotemark pattern matching can also match exceptions
        \item<1-> The CMM of \funcname{probably\_raise} shows that this \funcname{match \dots with} is implemented with a pattern matching and a \funcname{try \dots with}
        \item<1-> But this one only matches on \typename{ExnA} exception
        \item<2-> Since the pattern matching fails to catch the exception, \funcname{reraise} is called with our current \typename{ExnC} exception
        \item<3-> Disassembly shows that \funcname{reraise} is actually a call to \funcname{caml\_reraise\_exn}, which is also an assembly function provided by the runtime
      \end{itemize}
      \vfill
      \mintocaml[firstline=15,lastline=21,highlightlines={18-21}]{../src/backtrace/backtrace.ml}
      \endminipage
    \end{column}
    \begin{column}{0.55\textwidth}
      \centering
      \onslide*<1>{\mintcmm[highlightlines={10-18}]{../src/backtrace/camlBacktrace\_\_probably\_raise\_351.cmm}}
      \onslide*<2>{\listcmm[highlightlines=18]{../src/backtrace/camlBacktrace\_\_probably\_raise_351.cmm}{CMM of probably\_raise}}
      \onslide*<3>{\mintobjdump[firstline=5,lastline=35,highlightlines=31]{../src/backtrace/camlBacktrace\_\_probably\_raise_351.objdump}}
    \end{column}
  \end{columns}
  \only<1>{\footnotetext[1]{https://blog.janestreet.com/pattern-matching-and-exception-handling-unite/}}
\end{frame}

\newcommand\listCamlReraiseExn[1][]{
  \listasm[firstline=727,lastline=734,#1]{../ocaml/runtime/amd64.S}{runtime/amd64.S:727}}

\begin{frame}{Backtraces}{Introducing \funcname{caml\_reraise\_exn}}
  \begin{columns}[c]
    \begin{column}{0.5\textwidth}
      \minipage[c][0.66\textheight][s]{\columnwidth}
      \begin{itemize}
        \item<1-> The exception have to be reraised to the next handler
        \item<2-> First, checks if the backtrace should be collected with \funcarg{domain\_state->backtrace\_active}
        \only<3>{
        \begin{itemize}
            \item If not, behaves like a \funcname{raise\_notrace} with \funcname{RESTORE\_EXN\_HANDLER\_OCAML}
        \end{itemize}
        }
        \item<4-> Jumps into \funcname{caml\_raise\_exn}
        \item<5-> Calls \funcname{caml\_stash\_backtrace} again, to scan the next part of the stack: from the trap just pop-ed to the next trap
      \end{itemize}
      \vfill
      \mintocaml[firstline=15,lastline=21,highlightlines={18-21}]{../src/backtrace/backtrace.ml}
      \endminipage
    \end{column}
    \begin{column}{0.5\textwidth}
      \raggedleft
      \onslide*<1>{\listCamlReraiseExn{}}
      \onslide*<2>{\listCamlReraiseExn[highlightlines=729]{}}
      \onslide*<3>{
        \mintc[firstline=190,lastline=192,highlightlines={191-192}]{../ocaml/runtime/amd64.S}
        \mintc[firstline=234,lastline=237,highlightlines={235-237}]{../ocaml/runtime/amd64.S}
        \medskip
        \listCamlReraiseExn[highlightlines=731]{}
      }
      \onslide*<4-5>{
        % TODO refactor with \listCamlReraiseExn
        \mintasm[firstline=727,lastline=734,highlightlines=730]{../ocaml/runtime/amd64.S}
        \medskip
      }
      \onslide*<4>{\listCamlRaiseExn[highlightlines=711]{}}
      \onslide*<5>{\listCamlRaiseExn[highlightlines=720]{}}
    \end{column}
  \end{columns}
\end{frame}

\begin{frame}{Backtraces}{Backtrace collection continues}
  \begin{columns}[c]
    \begin{column}{0.55\textwidth}
      \minipage[c][0.75\textheight][s]{\columnwidth}
      \begin{itemize}
        \item<1-> \funcname{caml\_stash\_backtrace} scans the older part of the stack, up to the next trap
        \item<6-> Controls jumps to the nested exception handler in \funcname{maybe\_raise}, which matches only on \typename{ExnB}
        \item<7-> \funcname{caml\_reraise\_exn} is called again, backtrace collection resumes with \funcname{caml\_stash\_backtrace}
      \end{itemize}
      \medskip
      \begin{itemize}
        \item<2-> Constructed backtrace:
        \begin{itemize}
          \item <2-> \funcname{definitely\_raise}
          \item <2-> \funcname{probably\_raise}
          \item <3-> \funcname{probably\_raise} (re-raise)
          \item <4-> \funcname{maybe\_raise}
          \item <5-> \funcname{main}
          \item <8-> \funcname{main} (re-raise)
        \end{itemize}
      \end{itemize}
      \vfill
      \onslide*<1-5>{\mintocaml[firstline=15,lastline=21]{../src/backtrace/backtrace.ml}}
      \onslide*<6>{\mintocaml[firstline=28,lastline=34,highlightlines=31]{../src/backtrace/backtrace.ml}}
      \onslide*<7->{\mintocaml[firstline=28,lastline=34,highlightlines=33]{../src/backtrace/backtrace.ml}}
      \endminipage
    \end{column}
    \begin{column}{0.45\textwidth}
      \raggedleft
      \onslide*<1>{\providecommand\step{6}\input{diagrams/backtrace/backtrace.tex}}%
      \onslide*<2>{\providecommand\step{6}\input{diagrams/backtrace/backtrace.tex}}%
      \onslide*<3>{\providecommand\step{7}\input{diagrams/backtrace/backtrace.tex}}%
      \onslide*<4>{\providecommand\step{8}\input{diagrams/backtrace/backtrace.tex}}%
      \onslide*<5>{\providecommand\step{9}\input{diagrams/backtrace/backtrace.tex}}%
      \onslide*<6>{\providecommand\step{10}\input{diagrams/backtrace/backtrace.tex}}%
      \onslide*<7>{\providecommand\step{11}\input{diagrams/backtrace/backtrace.tex}}%
      \onslide*<8>{\providecommand\step{12}\input{diagrams/backtrace/backtrace.tex}}%
      \onslide*<9>{\providecommand\step{13}\input{diagrams/backtrace/backtrace.tex}}%
    \end{column}
  \end{columns}
\end{frame}

\begin{frame}{Backtraces}{Printing the backtrace}
  \begin{columns}[c]
    \begin{column}{0.45\textwidth}
      \minipage[c][0.66\textheight][s]{\columnwidth}
      \begin{itemize}
        \item<1-> The exception backtrace can now be printed to \funcarg{stdout} with \funcname{Printexc.print_backtrace}
        \item<2-> First the exception backtrace is retrieved into an OCaml array with \funcname{get_raw_backtrace }
        \item<3-> Which is implemented as the \funcname{caml_get_exception_raw_backtrace} C function in the runtime
        \item<4-> And finally printed with \funcname{print_exception_backtrace}
      \end{itemize}
      \vfill
      \mintocaml[firstline=28,lastline=34,highlightlines=33]{../src/backtrace/backtrace.ml}
      \endminipage
    \end{column}
    \begin{column}{0.55\textwidth}
      \onslide*<1,2>{
        \mintocaml[firstline=100,lastline=101]{../ocaml/stdlib/printexc.ml}
      }
      \only<1>{
        \listocaml[firstline=170,lastline=172]{../ocaml/stdlib/printexc.ml}{stdlib/printexc.ml:170}
      }
      \only<2>{
        \listocaml[firstline=170,lastline=172,highlightlines=172]{../ocaml/stdlib/printexc.ml}{stdlib/printexc.ml:170}
      }
      \only<3>{
        \mintc[firstline=154,lastline=156]{../ocaml/runtime/backtrace.c}
        \listc[firstline=171,lastline=192,highlightlines={184-187}]{../ocaml/runtime/backtrace.c}{runtime/backtrace.c:154}
      }
      \only<4>{
        \listocaml[firstline=155,lastline=165]{../ocaml/stdlib/printexc.ml}{stdlib/printexc.ml:55}
      }
    \end{column}
  \end{columns}
\end{frame}


\subsection{The multiple kinds of raise}
\frameSubsection{}{}

\begin{frame}{Backtraces}{When backtraces are not needed}
  \begin{columns}[c]
    \begin{column}{0.45\textwidth}
      \minipage[c][0.66\textheight][s]{\columnwidth}
      \begin{itemize}
        \item<1-> What if the backtrace for an exception is never needed?
        \item<2-> It's possible to raise the exception explicitly with \funcname{raise\_notrace} to make raising as cheap as possible
        \item<3-> An example of use in the \funcname{dynlink} library of the OCaml compiler
      \end{itemize}
      \vfill
      \onslide<3->{
      \listocaml[firstline=205,lastline=209,highlightlines=207]{../ocaml/otherlibs/dynlink/dynlink\_compilerlibs/misc.ml}{otherlibs/dynlink/dynlink\_compilerlibs/misc.ml:205}
      }
      \endminipage
    \end{column}
    \begin{column}{0.55\textwidth}
    \only<4>{
      \mintcmm[highlightlines=3]{../src/notrace/camlNotrace\_\_fun_347.cmm}
      \listcmm{../src/notrace/camlNotrace\_\_all\_somes\_327.cmm}{all\_somes}
    }
    \onslide*<5->{
      \listobjdump[highlightlines={6-9}]{../src/notrace/camlNotrace\_\_fun_347.objdump}{anonymous function from \funcname{all\_somes}}
    }
    \end{column}
  \end{columns}
\end{frame}

\begin{frame}{Backtraces}{Summary table of the multiple kinds of raise}
  \begin{itemize}
    \item When debugging information is enabled and if the backtrace of an exception isn't needed, it's possible to raise with \funcname{raise\_notrace} for performance
    \item When debugging information is \emph{not} enabled, the compiler optimises \funcname{raise} calls to inline assembly (same effect as using \funcname{raise\_notrace})
  \end{itemize}
  \bigskip
  \centering
  \begin{tabular}{| l | c | c | c | c |}
    \hline
    \parbox{10em}{Debug information \\ (\funcarg{-g} flag)}  & \multicolumn{2}{|c|}{Disabled}                                     & \multicolumn{2}{|c|}{Enabled} \\
    \hline
    Raise kind                                               & \funcname{raise} & \funcname{raise\_notrace}                       & \funcname{raise} & \funcname{raise\_notrace} \\
    \hline
    Compilation                                              & \parbox{8em}{inline assembly\\ (optimization)} & inline assembly   & \funcname{caml\_raise\_exn} & inline assembly \\
    \hline
  \end{tabular}
\end{frame}


%
% Takeaway #5
%
\subsection*{Takeaway \#5}
\frameSubsectionTakeaway{}
\againframe<5>{takeaway}
}%
      \onslide*<9>{\providecommand\step{13}%
% Backtraces
%
\subsection{One advantage of raising with debugging information}
\frameSubsection{}{}

\begin{frame}{Backtraces}{Debugging information \& source code}
  \begin{columns}[c]
    \begin{column}{0.5\textwidth}
      \begin{itemize}
        \item Raising an exception is fun and games, but how do we know where the exception originated? $\rightarrow$ With the backtrace associated with the exception!
        \item In order to get a backtrace from the point where an exception was raised, it's necessary to compile with debugging information enabled (\funcarg{-g} flag for the compiler)
        \item Same example as in the `Nested handlers' section
      \end{itemize}
    \end{column}
    \begin{column}{0.5\textwidth}
      \listocaml[firstline=9,lastline=34]{../src/backtrace/backtrace.ml}{backtrace/backtrace.ml}
    \end{column}
  \end{columns}
\end{frame}

\begin{frame}{Backtraces}{CMM and disassembly of \funcname{definitely\_raise}}
  \begin{columns}[c]
    \begin{column}{0.5\textwidth}
      \minipage[c][0.66\textheight][s]{\columnwidth}
      \begin{itemize}
        \item<1-> An effect of compiling with debugging information (\funcarg{-g}): the CMM no longer uses \funcname{raise\_notrace} to raise the exception but \funcname{raise} instead
        \item<2-> Disassembly shows that CMM \funcname{raise} is actually a call to \funcname{caml\_raise\_exn}, a function in assembler provided by the runtime
      \end{itemize}
      \vfill
      \mintocaml[firstline=9,lastline=13,highlightlines=12]{../src/backtrace/backtrace.ml}
      \endminipage
    \end{column}
    \begin{column}{0.5\textwidth}
      \onslide*<1>{
        \mintcmm[highlightlines=5]{../src/backtrace/camlBacktrace\_\_definitely\_raise\_347.cmm}
      }
      \onslide*<2>{
        \mintobjdump[firstline=2,highlightlines=14]{../src/backtrace/camlBacktrace\_\_definitely\_raise\_347.objdump}
      }
    \end{column}
  \end{columns}
\end{frame}

\begin{frame}{Backtraces}{Introducing \funcname{caml\_raise\_exn}}
  \begin{columns}[c]
    \begin{column}{0.5\textwidth}
      \minipage[c][0.66\textheight][s]{\columnwidth}
      \onslide*<1>{
        \begin{itemize}
          \item Raising the exception is performed using \funcname{caml\_raise\_exn} which is
            \begin{itemize}
              \item An assembly function provided by the runtime
              \item Architecture specific
            \end{itemize}
          \item Let's see what it does differently from \funcname{raise\_notrace}\dots
          \item We are about to raise \typename{ExnC} with \funcname{caml\_raise\_exn} from \funcname{definitely\_raise}
        \end{itemize}
      }
      \onslide*<2>{
        \mintobjdump[firstline=2,highlightlines=14]{../src/backtrace/camlBacktrace\_\_definitely\_raise\_347.objdump}
      }
      \vfill
      \mintocaml[firstline=9,lastline=13,highlightlines=12]{../src/backtrace/backtrace.ml}
      \endminipage
    \end{column}
    \begin{column}{0.5\textwidth}
      \centering
      \providecommand\step{1}\input{diagrams/backtrace/backtrace.tex}
    \end{column}
  \end{columns}
\end{frame}

\begin{frame}{Backtraces}{But before that, we need more fields from \typename{caml\_domain\_state}}
  \begin{columns}
    \begin{column}{0.5\textwidth}
      \begin{itemize}
        \item Remember \typename{caml\_domain\_state} the per-domain runtime state?
        \item Some other members are of interest to us now\dots
        \item \localname{backtrace\_active} is used to know if the runtime should collect backtraces
        \item \localname{backtrace\_active} can be set by
          \begin{itemize}
            \item The \funcarg{b} flag in the \funcname{OCAMLRUNPARAM} environment variable
            \item \mintinline{ocaml}{Printexc.record_backtrace : bool -> unit} \footnotemark
          \end{itemize}
      \end{itemize}
    \end{column}
    \begin{column}{0.5\textwidth}
      \begin{listing}
        \mintc[highlightlines={9-11}]{src/ocaml/domain\_state\_backtrace.h}
        \caption{runtime/caml/domain\_state.h}
      \end{listing}
    \end{column}
  \end{columns}
  \footnotetext[1]{https://v2.ocaml.org/api/Printexc.html}
\end{frame}

\newcommand\listCamlRaiseExn[1][]{
  \listasm[firstline=702,lastline=725,#1]{../ocaml/runtime/amd64.S}{runtime/amd64.S:702}}

\begin{frame}{Backtraces}{Walk through \funcname{caml\_raise\_exn}}
  \begin{columns}[T]
    \begin{column}{0.5\textwidth}
      \begin{itemize}
        \item<1-> First checks whether exceptions backtrace should be recorded
        \item<1-> By checking the \funcarg{domain\_state->backtrace\_active} value
        \item<2-> If not, acts as \funcname{raise\_notrace} with \funcname{RESTORE\_EXN\_HANDLER\_OCAML}
          \begin{itemize}
            \item Set \regname{RSP} to \localname{domain\_state->exn\_handler} value
            \item Restore \localname{domain\_state->exn\_handler} from trap
            \item Jump with \funcname{ret} (same effect as using \funcname{pop} and \funcname{jmp})
          \end{itemize}
        \item<3-> Otherwise, switches to the C stack to be able to execute C code
        \item<4-> Calls \funcname{caml\_stash\_backtrace}, a C function provided by the runtime\ignorespaces
          \onslide<6->{, with
            \begin{itemize}
              \item \funcarg{exn}: exception value
              \item \funcarg{pc}: code address of the raising point
              \item \funcarg{sp}: stack address of the raising function
              \item \funcarg{trapsp}: stack address of the next handler
            \end{itemize}
          }
      \end{itemize}
      \bigskip
      \mintocaml[firstline=9,lastline=13,highlightlines=12]{../src/backtrace/backtrace.ml}
    \end{column}
    \begin{column}{0.5\textwidth}
      \raggedleft
      \onslide*<1,3-4>{
        \mintc[firstline=190,lastline=192]{../ocaml/runtime/amd64.S}
        \mintc[firstline=234,lastline=237]{../ocaml/runtime/amd64.S}
      }
      \onslide*<2>{
        \mintc[firstline=190,lastline=192,highlightlines={191-192}]{../ocaml/runtime/amd64.S}
        \mintc[firstline=234,lastline=237,highlightlines={235-237}]{../ocaml/runtime/amd64.S}
      }
      \onslide*<1>{\listCamlRaiseExn[highlightlines=705]{}}%
      \onslide*<2>{\listCamlRaiseExn[highlightlines={707-708}]{}}%
      \onslide*<3>{\listCamlRaiseExn[highlightlines={712-714}]{}}%
      \onslide*<4>{\listCamlRaiseExn[highlightlines={715-720}]{}}%
      \onslide*<5>{\providecommand\step{1}\input{diagrams/backtrace/backtrace.tex}}%
      \onslide*<6>{\providecommand\step{2}\input{diagrams/backtrace/backtrace.tex}}%
    \end{column}
  \end{columns}
\end{frame}

\newcommand\listStashBacktrace[1][]{
  \listc[firstline=91,lastline=120,#1]{../ocaml/runtime/backtrace\_nat.c}{runtime/backtrace\_nat.c:91}}

\begin{frame}{Backtraces}{Introducing \funcname{caml\_stash\_backtrace}}
  \begin{columns}[c]
    \begin{column}{0.5\textwidth}
      \minipage[c][0.66\textheight][s]{\columnwidth}
      \begin{itemize}
        \item<1-> A C function provided by the runtime (specific to native code)
        \item<2-> It scans the stack, between the stack pointer at the raising point up to the next trap, in order to collect the names of the detected function frames
          \onslide*<2>{
            \begin{itemize}
              \item And more information, like line number, column number...
            \end{itemize}
          }
        \item<3-> It uses the same technique and information as the Garbage Collector (GC) uses to scan the values live on the stack
          \onslide*<4>{
            \begin{itemize}
              \item \typename{frame\_descr} are at the core of stack unwinding
            \end{itemize}
          }
      \end{itemize}
      \vfill
      \mintocaml[firstline=9,lastline=13,highlightlines=12]{../src/backtrace/backtrace.ml}
      \endminipage
    \end{column}
    \begin{column}{0.5\textwidth}
      \onslide*<1>{\listStashBacktrace[highlightlines=91]{}}
      \onslide*<2>{\listStashBacktrace[highlightlines={91,117-118}]{}}
      \onslide*<3->{\listStashBacktrace[highlightlines={94,106,109-110}]{}}
    \end{column}
  \end{columns}
\end{frame}

\newcommand\listFrameDescr[1][]{
  \listocaml[firstline=111,lastline=124,#1]{../ocaml/asmcomp/emitaux.ml}{asmcomp/emitaux.ml:111}}
\newcommand\listDebugInfo[1][]{
  \listocaml[firstline=38,lastline=49,#1]{../ocaml/lambda/debuginfo.mli}{lambda/debuginfo.mli:38}}

\begin{frame}{Backtraces}{\typename{frame\_descr} and \typename{frame\_debuginfo} in a nutshell}
  \begin{columns}[c]
    \begin{column}{0.5\textwidth}
      \begin{itemize}
        \item<1-> For every return address in an OCaml program, there is a associated \typename{frame\_descr}
        \item<2-> A \typename{frame\_descr} specifies
        \begin{itemize}
          \item<2-> The size of the function stack frame
          \item<3-> Where live values are on the stack (so that the GC can mark them as live and prevent from being swept)
        \end{itemize}
        \item<4-> When compiled with debug information, \typename{frame\_descr} will be linked to a \typename{frame\_debuginfo}
        \begin{itemize}
          \item<5-> Contains information about the position in the source code (file name, line and column number)
        \end{itemize}
      \end{itemize}
    \end{column}
    \begin{column}{0.5\textwidth}
        \onslide*<1>{\listFrameDescr[highlightlines={118-122}]{}}
        \onslide*<2>{\listFrameDescr[highlightlines={120}]{}}
        \onslide*<3>{\listFrameDescr[highlightlines={121}]{}}
        \onslide*<4->{\listFrameDescr[highlightlines={122}]{}}
        \medskip
        \onslide*<1-3>{\listDebugInfo{}}
        \onslide*<4>{\listDebugInfo[highlightlines={38,49}]{}}
        \onslide*<5>{\listDebugInfo[highlightlines={39-46}]{}}
    \end{column}
  \end{columns}
\end{frame}

\begin{frame}{Backtraces}{Scanning the stack with \typename{frame\_descr}}
  \begin{columns}[c]
    \begin{column}{0.5\textwidth}
      \begin{itemize}
        \item Given the return address of the code that raises the exception by calling \funcname{caml\_raise\_exn} (\funcarg{pc} argument of \funcname{caml\_stash\_backtrace})
        \item And the position of the stack pointer (\funcarg{sp} argument of \funcname{caml\_stash\_backtrace})
        \item The frame size given by \typename{frame\_descr}.\funcarg{frame\_size}, allows to find the end of the previous frame
        \item And the return address of the caller of this function
        \bigskip
        \onslide<3->{
        \item Note that traps (here the reddish \funcname{ExnA}, \funcname{ExnB} and \funcname{ExnC} blocks) belong to the frame that installed them, and so the frame size changes when traps are removed
        }
      \end{itemize}
    \end{column}
    \begin{column}{0.5\textwidth}
      \raggedleft
      \onslide*<1>{\providecommand\step{2}\input{diagrams/backtrace/backtrace.tex}}%
      \onslide*<2>{\providecommand\step{1}\input{diagrams/backtrace/scanstack.tex}}%
      \onslide*<3>{\providecommand\step{2}\input{diagrams/backtrace/scanstack.tex}}%
      \onslide*<4>{\providecommand\step{3}\input{diagrams/backtrace/scanstack.tex}}%
      \onslide*<5>{\providecommand\step{4}\input{diagrams/backtrace/scanstack.tex}}%
      \onslide*<6>{\providecommand\step{5}\input{diagrams/backtrace/scanstack.tex}}%
    \end{column}
  \end{columns}
\end{frame}

% Duplicate of previous-previous slide
\begin{frame}{Backtraces}{Continuing on \funcname{caml\_stash\_backtrace}}
  \begin{columns}[c]
    \begin{column}{0.5\textwidth}
      \minipage[c][0.75\textheight][s]{\columnwidth}
      \begin{itemize}
          \color{gray}
        \item A C function provided by the runtime (specific to native code)
        \item It scans the stack, between the stack pointer at the raising point up to the next trap, in order to collect the names of the detected function frames
        \item It uses the same technique and information as the Garbage Collector (GC) uses to scan the values live on the stack
      \end{itemize}
      \begin{itemize}
        \item For each function frame detected, store a pointer to the \typename{frame\_descr} in the \localname{domain\_state->backtrace\_buffer} array, effectively constructing the backtrace
      \end{itemize}
      \onslide<2->{
        \begin{itemize}
            \smallskip
          \item Constructed backtrace:
            \funcname{definitely\_raise},
            \onslide<3->{\funcname{probably\_raise}}
        \end{itemize}
      }
      \vfill
      \mintocaml[firstline=9,lastline=13,highlightlines=12]{../src/backtrace/backtrace.ml}
      \endminipage
    \end{column}
    \begin{column}{0.5\textwidth}
      \raggedleft
      \onslide*<1>{\listStashBacktrace[highlightlines={114-115}]{}}
      \onslide*<2>{\providecommand\step{3}\input{diagrams/backtrace/backtrace.tex}}%
      \onslide*<3>{\providecommand\step{4}\input{diagrams/backtrace/backtrace.tex}}%
    \end{column}
  \end{columns}
\end{frame}

\begin{frame}{Backtraces}{Back to \funcname{caml\_raise\_exn}}
  \begin{columns}[c]
    \begin{column}{0.5\textwidth}
      \minipage[c][0.66\textheight][s]{\columnwidth}
      \begin{itemize}\color{gray}
        \item Checks wheteher exceptions backtrace should be recorded, by checking the \funcarg{domain\_state->backtrace\_active} value
        \item Switches to the C stack to be able to execute C code
        \item Calls \funcname{caml\_stash\_backtrace}, to collect the backtrace up to the next trap
      \end{itemize}
      \onslide*<2->{
        \begin{itemize}
          \item<2-> Executes the exception handler in \funcname{probably\_raise} with \funcname{RESTORE\_EXN\_HANDLER\_OCAML}
            \begin{itemize}
              \item Set \regname{RSP} to \localname{domain\_state->exn\_handler} value
              \item Restore \localname{domain\_state->exn\_handler} from trap
              \item Jump to the handler code with \funcname{ret}
            \end{itemize}
        \end{itemize}
      }
      \vfill
      \onslide*<1>{\mintocaml[firstline=9,lastline=13,highlightlines=12]{../src/backtrace/backtrace.ml}}
      \onslide*<2->{\mintocaml[firstline=15,lastline=21,highlightlines={19-21}]{../src/backtrace/backtrace.ml}}
      \endminipage
    \end{column}
    \begin{column}{0.5\textwidth}
      \centering
      \onslide*<1>{
        \mintc[firstline=190,lastline=192]{../ocaml/runtime/amd64.S}
        \mintc[firstline=234,lastline=237]{../ocaml/runtime/amd64.S}
      }
      \onslide*<2>{
        \mintc[firstline=190,lastline=192,highlightlines={191-192}]{../ocaml/runtime/amd64.S}
        \mintc[firstline=234,lastline=237,highlightlines={235-237}]{../ocaml/runtime/amd64.S}
      }
      \onslide*<1>{\listCamlRaiseExn[highlightlines=720]{}}
      \onslide*<2>{\listCamlRaiseExn[highlightlines=722]{}}
      \onslide*<3->{\providecommand\step{5}\input{diagrams/backtrace/backtrace.tex}}
    \end{column}
  \end{columns}
\end{frame}

\begin{frame}{Backtraces}{\funcname{caml\_probably\_raise}'s handler doesn't catch \typename{ExnC}}
  \begin{columns}[c]
    \begin{column}{0.45\textwidth}
      \minipage[c][0.75\textheight][s]{\columnwidth}
      \begin{itemize}
        \item<1-> \funcname{match \dots with}\footnotemark pattern matching can also match exceptions
        \item<1-> The CMM of \funcname{probably\_raise} shows that this \funcname{match \dots with} is implemented with a pattern matching and a \funcname{try \dots with}
        \item<1-> But this one only matches on \typename{ExnA} exception
        \item<2-> Since the pattern matching fails to catch the exception, \funcname{reraise} is called with our current \typename{ExnC} exception
        \item<3-> Disassembly shows that \funcname{reraise} is actually a call to \funcname{caml\_reraise\_exn}, which is also an assembly function provided by the runtime
      \end{itemize}
      \vfill
      \mintocaml[firstline=15,lastline=21,highlightlines={18-21}]{../src/backtrace/backtrace.ml}
      \endminipage
    \end{column}
    \begin{column}{0.55\textwidth}
      \centering
      \onslide*<1>{\mintcmm[highlightlines={10-18}]{../src/backtrace/camlBacktrace\_\_probably\_raise\_351.cmm}}
      \onslide*<2>{\listcmm[highlightlines=18]{../src/backtrace/camlBacktrace\_\_probably\_raise_351.cmm}{CMM of probably\_raise}}
      \onslide*<3>{\mintobjdump[firstline=5,lastline=35,highlightlines=31]{../src/backtrace/camlBacktrace\_\_probably\_raise_351.objdump}}
    \end{column}
  \end{columns}
  \only<1>{\footnotetext[1]{https://blog.janestreet.com/pattern-matching-and-exception-handling-unite/}}
\end{frame}

\newcommand\listCamlReraiseExn[1][]{
  \listasm[firstline=727,lastline=734,#1]{../ocaml/runtime/amd64.S}{runtime/amd64.S:727}}

\begin{frame}{Backtraces}{Introducing \funcname{caml\_reraise\_exn}}
  \begin{columns}[c]
    \begin{column}{0.5\textwidth}
      \minipage[c][0.66\textheight][s]{\columnwidth}
      \begin{itemize}
        \item<1-> The exception have to be reraised to the next handler
        \item<2-> First, checks if the backtrace should be collected with \funcarg{domain\_state->backtrace\_active}
        \only<3>{
        \begin{itemize}
            \item If not, behaves like a \funcname{raise\_notrace} with \funcname{RESTORE\_EXN\_HANDLER\_OCAML}
        \end{itemize}
        }
        \item<4-> Jumps into \funcname{caml\_raise\_exn}
        \item<5-> Calls \funcname{caml\_stash\_backtrace} again, to scan the next part of the stack: from the trap just pop-ed to the next trap
      \end{itemize}
      \vfill
      \mintocaml[firstline=15,lastline=21,highlightlines={18-21}]{../src/backtrace/backtrace.ml}
      \endminipage
    \end{column}
    \begin{column}{0.5\textwidth}
      \raggedleft
      \onslide*<1>{\listCamlReraiseExn{}}
      \onslide*<2>{\listCamlReraiseExn[highlightlines=729]{}}
      \onslide*<3>{
        \mintc[firstline=190,lastline=192,highlightlines={191-192}]{../ocaml/runtime/amd64.S}
        \mintc[firstline=234,lastline=237,highlightlines={235-237}]{../ocaml/runtime/amd64.S}
        \medskip
        \listCamlReraiseExn[highlightlines=731]{}
      }
      \onslide*<4-5>{
        % TODO refactor with \listCamlReraiseExn
        \mintasm[firstline=727,lastline=734,highlightlines=730]{../ocaml/runtime/amd64.S}
        \medskip
      }
      \onslide*<4>{\listCamlRaiseExn[highlightlines=711]{}}
      \onslide*<5>{\listCamlRaiseExn[highlightlines=720]{}}
    \end{column}
  \end{columns}
\end{frame}

\begin{frame}{Backtraces}{Backtrace collection continues}
  \begin{columns}[c]
    \begin{column}{0.55\textwidth}
      \minipage[c][0.75\textheight][s]{\columnwidth}
      \begin{itemize}
        \item<1-> \funcname{caml\_stash\_backtrace} scans the older part of the stack, up to the next trap
        \item<6-> Controls jumps to the nested exception handler in \funcname{maybe\_raise}, which matches only on \typename{ExnB}
        \item<7-> \funcname{caml\_reraise\_exn} is called again, backtrace collection resumes with \funcname{caml\_stash\_backtrace}
      \end{itemize}
      \medskip
      \begin{itemize}
        \item<2-> Constructed backtrace:
        \begin{itemize}
          \item <2-> \funcname{definitely\_raise}
          \item <2-> \funcname{probably\_raise}
          \item <3-> \funcname{probably\_raise} (re-raise)
          \item <4-> \funcname{maybe\_raise}
          \item <5-> \funcname{main}
          \item <8-> \funcname{main} (re-raise)
        \end{itemize}
      \end{itemize}
      \vfill
      \onslide*<1-5>{\mintocaml[firstline=15,lastline=21]{../src/backtrace/backtrace.ml}}
      \onslide*<6>{\mintocaml[firstline=28,lastline=34,highlightlines=31]{../src/backtrace/backtrace.ml}}
      \onslide*<7->{\mintocaml[firstline=28,lastline=34,highlightlines=33]{../src/backtrace/backtrace.ml}}
      \endminipage
    \end{column}
    \begin{column}{0.45\textwidth}
      \raggedleft
      \onslide*<1>{\providecommand\step{6}\input{diagrams/backtrace/backtrace.tex}}%
      \onslide*<2>{\providecommand\step{6}\input{diagrams/backtrace/backtrace.tex}}%
      \onslide*<3>{\providecommand\step{7}\input{diagrams/backtrace/backtrace.tex}}%
      \onslide*<4>{\providecommand\step{8}\input{diagrams/backtrace/backtrace.tex}}%
      \onslide*<5>{\providecommand\step{9}\input{diagrams/backtrace/backtrace.tex}}%
      \onslide*<6>{\providecommand\step{10}\input{diagrams/backtrace/backtrace.tex}}%
      \onslide*<7>{\providecommand\step{11}\input{diagrams/backtrace/backtrace.tex}}%
      \onslide*<8>{\providecommand\step{12}\input{diagrams/backtrace/backtrace.tex}}%
      \onslide*<9>{\providecommand\step{13}\input{diagrams/backtrace/backtrace.tex}}%
    \end{column}
  \end{columns}
\end{frame}

\begin{frame}{Backtraces}{Printing the backtrace}
  \begin{columns}[c]
    \begin{column}{0.45\textwidth}
      \minipage[c][0.66\textheight][s]{\columnwidth}
      \begin{itemize}
        \item<1-> The exception backtrace can now be printed to \funcarg{stdout} with \funcname{Printexc.print_backtrace}
        \item<2-> First the exception backtrace is retrieved into an OCaml array with \funcname{get_raw_backtrace }
        \item<3-> Which is implemented as the \funcname{caml_get_exception_raw_backtrace} C function in the runtime
        \item<4-> And finally printed with \funcname{print_exception_backtrace}
      \end{itemize}
      \vfill
      \mintocaml[firstline=28,lastline=34,highlightlines=33]{../src/backtrace/backtrace.ml}
      \endminipage
    \end{column}
    \begin{column}{0.55\textwidth}
      \onslide*<1,2>{
        \mintocaml[firstline=100,lastline=101]{../ocaml/stdlib/printexc.ml}
      }
      \only<1>{
        \listocaml[firstline=170,lastline=172]{../ocaml/stdlib/printexc.ml}{stdlib/printexc.ml:170}
      }
      \only<2>{
        \listocaml[firstline=170,lastline=172,highlightlines=172]{../ocaml/stdlib/printexc.ml}{stdlib/printexc.ml:170}
      }
      \only<3>{
        \mintc[firstline=154,lastline=156]{../ocaml/runtime/backtrace.c}
        \listc[firstline=171,lastline=192,highlightlines={184-187}]{../ocaml/runtime/backtrace.c}{runtime/backtrace.c:154}
      }
      \only<4>{
        \listocaml[firstline=155,lastline=165]{../ocaml/stdlib/printexc.ml}{stdlib/printexc.ml:55}
      }
    \end{column}
  \end{columns}
\end{frame}


\subsection{The multiple kinds of raise}
\frameSubsection{}{}

\begin{frame}{Backtraces}{When backtraces are not needed}
  \begin{columns}[c]
    \begin{column}{0.45\textwidth}
      \minipage[c][0.66\textheight][s]{\columnwidth}
      \begin{itemize}
        \item<1-> What if the backtrace for an exception is never needed?
        \item<2-> It's possible to raise the exception explicitly with \funcname{raise\_notrace} to make raising as cheap as possible
        \item<3-> An example of use in the \funcname{dynlink} library of the OCaml compiler
      \end{itemize}
      \vfill
      \onslide<3->{
      \listocaml[firstline=205,lastline=209,highlightlines=207]{../ocaml/otherlibs/dynlink/dynlink\_compilerlibs/misc.ml}{otherlibs/dynlink/dynlink\_compilerlibs/misc.ml:205}
      }
      \endminipage
    \end{column}
    \begin{column}{0.55\textwidth}
    \only<4>{
      \mintcmm[highlightlines=3]{../src/notrace/camlNotrace\_\_fun_347.cmm}
      \listcmm{../src/notrace/camlNotrace\_\_all\_somes\_327.cmm}{all\_somes}
    }
    \onslide*<5->{
      \listobjdump[highlightlines={6-9}]{../src/notrace/camlNotrace\_\_fun_347.objdump}{anonymous function from \funcname{all\_somes}}
    }
    \end{column}
  \end{columns}
\end{frame}

\begin{frame}{Backtraces}{Summary table of the multiple kinds of raise}
  \begin{itemize}
    \item When debugging information is enabled and if the backtrace of an exception isn't needed, it's possible to raise with \funcname{raise\_notrace} for performance
    \item When debugging information is \emph{not} enabled, the compiler optimises \funcname{raise} calls to inline assembly (same effect as using \funcname{raise\_notrace})
  \end{itemize}
  \bigskip
  \centering
  \begin{tabular}{| l | c | c | c | c |}
    \hline
    \parbox{10em}{Debug information \\ (\funcarg{-g} flag)}  & \multicolumn{2}{|c|}{Disabled}                                     & \multicolumn{2}{|c|}{Enabled} \\
    \hline
    Raise kind                                               & \funcname{raise} & \funcname{raise\_notrace}                       & \funcname{raise} & \funcname{raise\_notrace} \\
    \hline
    Compilation                                              & \parbox{8em}{inline assembly\\ (optimization)} & inline assembly   & \funcname{caml\_raise\_exn} & inline assembly \\
    \hline
  \end{tabular}
\end{frame}


%
% Takeaway #5
%
\subsection*{Takeaway \#5}
\frameSubsectionTakeaway{}
\againframe<5>{takeaway}
}%
    \end{column}
  \end{columns}
\end{frame}

\begin{frame}{Backtraces}{Printing the backtrace}
  \begin{columns}[c]
    \begin{column}{0.45\textwidth}
      \minipage[c][0.66\textheight][s]{\columnwidth}
      \begin{itemize}
        \item<1-> The exception backtrace can now be printed to \funcarg{stdout} with \funcname{Printexc.print_backtrace}
        \item<2-> First the exception backtrace is retrieved into an OCaml array with \funcname{get_raw_backtrace }
        \item<3-> Which is implemented as the \funcname{caml_get_exception_raw_backtrace} C function in the runtime
        \item<4-> And finally printed with \funcname{print_exception_backtrace}
      \end{itemize}
      \vfill
      \mintocaml[firstline=28,lastline=34,highlightlines=33]{../src/backtrace/backtrace.ml}
      \endminipage
    \end{column}
    \begin{column}{0.55\textwidth}
      \onslide*<1,2>{
        \mintocaml[firstline=100,lastline=101]{../ocaml/stdlib/printexc.ml}
      }
      \only<1>{
        \listocaml[firstline=170,lastline=172]{../ocaml/stdlib/printexc.ml}{stdlib/printexc.ml:170}
      }
      \only<2>{
        \listocaml[firstline=170,lastline=172,highlightlines=172]{../ocaml/stdlib/printexc.ml}{stdlib/printexc.ml:170}
      }
      \only<3>{
        \mintc[firstline=154,lastline=156]{../ocaml/runtime/backtrace.c}
        \listc[firstline=171,lastline=192,highlightlines={184-187}]{../ocaml/runtime/backtrace.c}{runtime/backtrace.c:154}
      }
      \only<4>{
        \listocaml[firstline=155,lastline=165]{../ocaml/stdlib/printexc.ml}{stdlib/printexc.ml:55}
      }
    \end{column}
  \end{columns}
\end{frame}


\subsection{The multiple kinds of raise}
\frameSubsection{}{}

\begin{frame}{Backtraces}{When backtraces are not needed}
  \begin{columns}[c]
    \begin{column}{0.45\textwidth}
      \minipage[c][0.66\textheight][s]{\columnwidth}
      \begin{itemize}
        \item<1-> What if the backtrace for an exception is never needed?
        \item<2-> It's possible to raise the exception explicitly with \funcname{raise\_notrace} to make raising as cheap as possible
        \item<3-> An example of use in the \funcname{dynlink} library of the OCaml compiler
      \end{itemize}
      \vfill
      \onslide<3->{
      \listocaml[firstline=205,lastline=209,highlightlines=207]{../ocaml/otherlibs/dynlink/dynlink\_compilerlibs/misc.ml}{otherlibs/dynlink/dynlink\_compilerlibs/misc.ml:205}
      }
      \endminipage
    \end{column}
    \begin{column}{0.55\textwidth}
    \only<4>{
      \mintcmm[highlightlines=3]{../src/notrace/camlNotrace\_\_fun_347.cmm}
      \listcmm{../src/notrace/camlNotrace\_\_all\_somes\_327.cmm}{all\_somes}
    }
    \onslide*<5->{
      \listobjdump[highlightlines={6-9}]{../src/notrace/camlNotrace\_\_fun_347.objdump}{anonymous function from \funcname{all\_somes}}
    }
    \end{column}
  \end{columns}
\end{frame}

\begin{frame}{Backtraces}{Summary table of the multiple kinds of raise}
  \begin{itemize}
    \item When debugging information is enabled and if the backtrace of an exception isn't needed, it's possible to raise with \funcname{raise\_notrace} for performance
    \item When debugging information is \emph{not} enabled, the compiler optimises \funcname{raise} calls to inline assembly (same effect as using \funcname{raise\_notrace})
  \end{itemize}
  \bigskip
  \centering
  \begin{tabular}{| l | c | c | c | c |}
    \hline
    \parbox{10em}{Debug information \\ (\funcarg{-g} flag)}  & \multicolumn{2}{|c|}{Disabled}                                     & \multicolumn{2}{|c|}{Enabled} \\
    \hline
    Raise kind                                               & \funcname{raise} & \funcname{raise\_notrace}                       & \funcname{raise} & \funcname{raise\_notrace} \\
    \hline
    Compilation                                              & \parbox{8em}{inline assembly\\ (optimization)} & inline assembly   & \funcname{caml\_raise\_exn} & inline assembly \\
    \hline
  \end{tabular}
\end{frame}


%
% Takeaway #5
%
\subsection*{Takeaway \#5}
\frameSubsectionTakeaway{}
\againframe<5>{takeaway}
}%
      \onslide*<2>{\providecommand\step{6}%
% Backtraces
%
\subsection{One advantage of raising with debugging information}
\frameSubsection{}{}

\begin{frame}{Backtraces}{Debugging information \& source code}
  \begin{columns}[c]
    \begin{column}{0.5\textwidth}
      \begin{itemize}
        \item Raising an exception is fun and games, but how do we know where the exception originated? $\rightarrow$ With the backtrace associated with the exception!
        \item In order to get a backtrace from the point where an exception was raised, it's necessary to compile with debugging information enabled (\funcarg{-g} flag for the compiler)
        \item Same example as in the `Nested handlers' section
      \end{itemize}
    \end{column}
    \begin{column}{0.5\textwidth}
      \listocaml[firstline=9,lastline=34]{../src/backtrace/backtrace.ml}{backtrace/backtrace.ml}
    \end{column}
  \end{columns}
\end{frame}

\begin{frame}{Backtraces}{CMM and disassembly of \funcname{definitely\_raise}}
  \begin{columns}[c]
    \begin{column}{0.5\textwidth}
      \minipage[c][0.66\textheight][s]{\columnwidth}
      \begin{itemize}
        \item<1-> An effect of compiling with debugging information (\funcarg{-g}): the CMM no longer uses \funcname{raise\_notrace} to raise the exception but \funcname{raise} instead
        \item<2-> Disassembly shows that CMM \funcname{raise} is actually a call to \funcname{caml\_raise\_exn}, a function in assembler provided by the runtime
      \end{itemize}
      \vfill
      \mintocaml[firstline=9,lastline=13,highlightlines=12]{../src/backtrace/backtrace.ml}
      \endminipage
    \end{column}
    \begin{column}{0.5\textwidth}
      \onslide*<1>{
        \mintcmm[highlightlines=5]{../src/backtrace/camlBacktrace\_\_definitely\_raise\_347.cmm}
      }
      \onslide*<2>{
        \mintobjdump[firstline=2,highlightlines=14]{../src/backtrace/camlBacktrace\_\_definitely\_raise\_347.objdump}
      }
    \end{column}
  \end{columns}
\end{frame}

\begin{frame}{Backtraces}{Introducing \funcname{caml\_raise\_exn}}
  \begin{columns}[c]
    \begin{column}{0.5\textwidth}
      \minipage[c][0.66\textheight][s]{\columnwidth}
      \onslide*<1>{
        \begin{itemize}
          \item Raising the exception is performed using \funcname{caml\_raise\_exn} which is
            \begin{itemize}
              \item An assembly function provided by the runtime
              \item Architecture specific
            \end{itemize}
          \item Let's see what it does differently from \funcname{raise\_notrace}\dots
          \item We are about to raise \typename{ExnC} with \funcname{caml\_raise\_exn} from \funcname{definitely\_raise}
        \end{itemize}
      }
      \onslide*<2>{
        \mintobjdump[firstline=2,highlightlines=14]{../src/backtrace/camlBacktrace\_\_definitely\_raise\_347.objdump}
      }
      \vfill
      \mintocaml[firstline=9,lastline=13,highlightlines=12]{../src/backtrace/backtrace.ml}
      \endminipage
    \end{column}
    \begin{column}{0.5\textwidth}
      \centering
      \providecommand\step{1}%
% Backtraces
%
\subsection{One advantage of raising with debugging information}
\frameSubsection{}{}

\begin{frame}{Backtraces}{Debugging information \& source code}
  \begin{columns}[c]
    \begin{column}{0.5\textwidth}
      \begin{itemize}
        \item Raising an exception is fun and games, but how do we know where the exception originated? $\rightarrow$ With the backtrace associated with the exception!
        \item In order to get a backtrace from the point where an exception was raised, it's necessary to compile with debugging information enabled (\funcarg{-g} flag for the compiler)
        \item Same example as in the `Nested handlers' section
      \end{itemize}
    \end{column}
    \begin{column}{0.5\textwidth}
      \listocaml[firstline=9,lastline=34]{../src/backtrace/backtrace.ml}{backtrace/backtrace.ml}
    \end{column}
  \end{columns}
\end{frame}

\begin{frame}{Backtraces}{CMM and disassembly of \funcname{definitely\_raise}}
  \begin{columns}[c]
    \begin{column}{0.5\textwidth}
      \minipage[c][0.66\textheight][s]{\columnwidth}
      \begin{itemize}
        \item<1-> An effect of compiling with debugging information (\funcarg{-g}): the CMM no longer uses \funcname{raise\_notrace} to raise the exception but \funcname{raise} instead
        \item<2-> Disassembly shows that CMM \funcname{raise} is actually a call to \funcname{caml\_raise\_exn}, a function in assembler provided by the runtime
      \end{itemize}
      \vfill
      \mintocaml[firstline=9,lastline=13,highlightlines=12]{../src/backtrace/backtrace.ml}
      \endminipage
    \end{column}
    \begin{column}{0.5\textwidth}
      \onslide*<1>{
        \mintcmm[highlightlines=5]{../src/backtrace/camlBacktrace\_\_definitely\_raise\_347.cmm}
      }
      \onslide*<2>{
        \mintobjdump[firstline=2,highlightlines=14]{../src/backtrace/camlBacktrace\_\_definitely\_raise\_347.objdump}
      }
    \end{column}
  \end{columns}
\end{frame}

\begin{frame}{Backtraces}{Introducing \funcname{caml\_raise\_exn}}
  \begin{columns}[c]
    \begin{column}{0.5\textwidth}
      \minipage[c][0.66\textheight][s]{\columnwidth}
      \onslide*<1>{
        \begin{itemize}
          \item Raising the exception is performed using \funcname{caml\_raise\_exn} which is
            \begin{itemize}
              \item An assembly function provided by the runtime
              \item Architecture specific
            \end{itemize}
          \item Let's see what it does differently from \funcname{raise\_notrace}\dots
          \item We are about to raise \typename{ExnC} with \funcname{caml\_raise\_exn} from \funcname{definitely\_raise}
        \end{itemize}
      }
      \onslide*<2>{
        \mintobjdump[firstline=2,highlightlines=14]{../src/backtrace/camlBacktrace\_\_definitely\_raise\_347.objdump}
      }
      \vfill
      \mintocaml[firstline=9,lastline=13,highlightlines=12]{../src/backtrace/backtrace.ml}
      \endminipage
    \end{column}
    \begin{column}{0.5\textwidth}
      \centering
      \providecommand\step{1}\input{diagrams/backtrace/backtrace.tex}
    \end{column}
  \end{columns}
\end{frame}

\begin{frame}{Backtraces}{But before that, we need more fields from \typename{caml\_domain\_state}}
  \begin{columns}
    \begin{column}{0.5\textwidth}
      \begin{itemize}
        \item Remember \typename{caml\_domain\_state} the per-domain runtime state?
        \item Some other members are of interest to us now\dots
        \item \localname{backtrace\_active} is used to know if the runtime should collect backtraces
        \item \localname{backtrace\_active} can be set by
          \begin{itemize}
            \item The \funcarg{b} flag in the \funcname{OCAMLRUNPARAM} environment variable
            \item \mintinline{ocaml}{Printexc.record_backtrace : bool -> unit} \footnotemark
          \end{itemize}
      \end{itemize}
    \end{column}
    \begin{column}{0.5\textwidth}
      \begin{listing}
        \mintc[highlightlines={9-11}]{src/ocaml/domain\_state\_backtrace.h}
        \caption{runtime/caml/domain\_state.h}
      \end{listing}
    \end{column}
  \end{columns}
  \footnotetext[1]{https://v2.ocaml.org/api/Printexc.html}
\end{frame}

\newcommand\listCamlRaiseExn[1][]{
  \listasm[firstline=702,lastline=725,#1]{../ocaml/runtime/amd64.S}{runtime/amd64.S:702}}

\begin{frame}{Backtraces}{Walk through \funcname{caml\_raise\_exn}}
  \begin{columns}[T]
    \begin{column}{0.5\textwidth}
      \begin{itemize}
        \item<1-> First checks whether exceptions backtrace should be recorded
        \item<1-> By checking the \funcarg{domain\_state->backtrace\_active} value
        \item<2-> If not, acts as \funcname{raise\_notrace} with \funcname{RESTORE\_EXN\_HANDLER\_OCAML}
          \begin{itemize}
            \item Set \regname{RSP} to \localname{domain\_state->exn\_handler} value
            \item Restore \localname{domain\_state->exn\_handler} from trap
            \item Jump with \funcname{ret} (same effect as using \funcname{pop} and \funcname{jmp})
          \end{itemize}
        \item<3-> Otherwise, switches to the C stack to be able to execute C code
        \item<4-> Calls \funcname{caml\_stash\_backtrace}, a C function provided by the runtime\ignorespaces
          \onslide<6->{, with
            \begin{itemize}
              \item \funcarg{exn}: exception value
              \item \funcarg{pc}: code address of the raising point
              \item \funcarg{sp}: stack address of the raising function
              \item \funcarg{trapsp}: stack address of the next handler
            \end{itemize}
          }
      \end{itemize}
      \bigskip
      \mintocaml[firstline=9,lastline=13,highlightlines=12]{../src/backtrace/backtrace.ml}
    \end{column}
    \begin{column}{0.5\textwidth}
      \raggedleft
      \onslide*<1,3-4>{
        \mintc[firstline=190,lastline=192]{../ocaml/runtime/amd64.S}
        \mintc[firstline=234,lastline=237]{../ocaml/runtime/amd64.S}
      }
      \onslide*<2>{
        \mintc[firstline=190,lastline=192,highlightlines={191-192}]{../ocaml/runtime/amd64.S}
        \mintc[firstline=234,lastline=237,highlightlines={235-237}]{../ocaml/runtime/amd64.S}
      }
      \onslide*<1>{\listCamlRaiseExn[highlightlines=705]{}}%
      \onslide*<2>{\listCamlRaiseExn[highlightlines={707-708}]{}}%
      \onslide*<3>{\listCamlRaiseExn[highlightlines={712-714}]{}}%
      \onslide*<4>{\listCamlRaiseExn[highlightlines={715-720}]{}}%
      \onslide*<5>{\providecommand\step{1}\input{diagrams/backtrace/backtrace.tex}}%
      \onslide*<6>{\providecommand\step{2}\input{diagrams/backtrace/backtrace.tex}}%
    \end{column}
  \end{columns}
\end{frame}

\newcommand\listStashBacktrace[1][]{
  \listc[firstline=91,lastline=120,#1]{../ocaml/runtime/backtrace\_nat.c}{runtime/backtrace\_nat.c:91}}

\begin{frame}{Backtraces}{Introducing \funcname{caml\_stash\_backtrace}}
  \begin{columns}[c]
    \begin{column}{0.5\textwidth}
      \minipage[c][0.66\textheight][s]{\columnwidth}
      \begin{itemize}
        \item<1-> A C function provided by the runtime (specific to native code)
        \item<2-> It scans the stack, between the stack pointer at the raising point up to the next trap, in order to collect the names of the detected function frames
          \onslide*<2>{
            \begin{itemize}
              \item And more information, like line number, column number...
            \end{itemize}
          }
        \item<3-> It uses the same technique and information as the Garbage Collector (GC) uses to scan the values live on the stack
          \onslide*<4>{
            \begin{itemize}
              \item \typename{frame\_descr} are at the core of stack unwinding
            \end{itemize}
          }
      \end{itemize}
      \vfill
      \mintocaml[firstline=9,lastline=13,highlightlines=12]{../src/backtrace/backtrace.ml}
      \endminipage
    \end{column}
    \begin{column}{0.5\textwidth}
      \onslide*<1>{\listStashBacktrace[highlightlines=91]{}}
      \onslide*<2>{\listStashBacktrace[highlightlines={91,117-118}]{}}
      \onslide*<3->{\listStashBacktrace[highlightlines={94,106,109-110}]{}}
    \end{column}
  \end{columns}
\end{frame}

\newcommand\listFrameDescr[1][]{
  \listocaml[firstline=111,lastline=124,#1]{../ocaml/asmcomp/emitaux.ml}{asmcomp/emitaux.ml:111}}
\newcommand\listDebugInfo[1][]{
  \listocaml[firstline=38,lastline=49,#1]{../ocaml/lambda/debuginfo.mli}{lambda/debuginfo.mli:38}}

\begin{frame}{Backtraces}{\typename{frame\_descr} and \typename{frame\_debuginfo} in a nutshell}
  \begin{columns}[c]
    \begin{column}{0.5\textwidth}
      \begin{itemize}
        \item<1-> For every return address in an OCaml program, there is a associated \typename{frame\_descr}
        \item<2-> A \typename{frame\_descr} specifies
        \begin{itemize}
          \item<2-> The size of the function stack frame
          \item<3-> Where live values are on the stack (so that the GC can mark them as live and prevent from being swept)
        \end{itemize}
        \item<4-> When compiled with debug information, \typename{frame\_descr} will be linked to a \typename{frame\_debuginfo}
        \begin{itemize}
          \item<5-> Contains information about the position in the source code (file name, line and column number)
        \end{itemize}
      \end{itemize}
    \end{column}
    \begin{column}{0.5\textwidth}
        \onslide*<1>{\listFrameDescr[highlightlines={118-122}]{}}
        \onslide*<2>{\listFrameDescr[highlightlines={120}]{}}
        \onslide*<3>{\listFrameDescr[highlightlines={121}]{}}
        \onslide*<4->{\listFrameDescr[highlightlines={122}]{}}
        \medskip
        \onslide*<1-3>{\listDebugInfo{}}
        \onslide*<4>{\listDebugInfo[highlightlines={38,49}]{}}
        \onslide*<5>{\listDebugInfo[highlightlines={39-46}]{}}
    \end{column}
  \end{columns}
\end{frame}

\begin{frame}{Backtraces}{Scanning the stack with \typename{frame\_descr}}
  \begin{columns}[c]
    \begin{column}{0.5\textwidth}
      \begin{itemize}
        \item Given the return address of the code that raises the exception by calling \funcname{caml\_raise\_exn} (\funcarg{pc} argument of \funcname{caml\_stash\_backtrace})
        \item And the position of the stack pointer (\funcarg{sp} argument of \funcname{caml\_stash\_backtrace})
        \item The frame size given by \typename{frame\_descr}.\funcarg{frame\_size}, allows to find the end of the previous frame
        \item And the return address of the caller of this function
        \bigskip
        \onslide<3->{
        \item Note that traps (here the reddish \funcname{ExnA}, \funcname{ExnB} and \funcname{ExnC} blocks) belong to the frame that installed them, and so the frame size changes when traps are removed
        }
      \end{itemize}
    \end{column}
    \begin{column}{0.5\textwidth}
      \raggedleft
      \onslide*<1>{\providecommand\step{2}\input{diagrams/backtrace/backtrace.tex}}%
      \onslide*<2>{\providecommand\step{1}\input{diagrams/backtrace/scanstack.tex}}%
      \onslide*<3>{\providecommand\step{2}\input{diagrams/backtrace/scanstack.tex}}%
      \onslide*<4>{\providecommand\step{3}\input{diagrams/backtrace/scanstack.tex}}%
      \onslide*<5>{\providecommand\step{4}\input{diagrams/backtrace/scanstack.tex}}%
      \onslide*<6>{\providecommand\step{5}\input{diagrams/backtrace/scanstack.tex}}%
    \end{column}
  \end{columns}
\end{frame}

% Duplicate of previous-previous slide
\begin{frame}{Backtraces}{Continuing on \funcname{caml\_stash\_backtrace}}
  \begin{columns}[c]
    \begin{column}{0.5\textwidth}
      \minipage[c][0.75\textheight][s]{\columnwidth}
      \begin{itemize}
          \color{gray}
        \item A C function provided by the runtime (specific to native code)
        \item It scans the stack, between the stack pointer at the raising point up to the next trap, in order to collect the names of the detected function frames
        \item It uses the same technique and information as the Garbage Collector (GC) uses to scan the values live on the stack
      \end{itemize}
      \begin{itemize}
        \item For each function frame detected, store a pointer to the \typename{frame\_descr} in the \localname{domain\_state->backtrace\_buffer} array, effectively constructing the backtrace
      \end{itemize}
      \onslide<2->{
        \begin{itemize}
            \smallskip
          \item Constructed backtrace:
            \funcname{definitely\_raise},
            \onslide<3->{\funcname{probably\_raise}}
        \end{itemize}
      }
      \vfill
      \mintocaml[firstline=9,lastline=13,highlightlines=12]{../src/backtrace/backtrace.ml}
      \endminipage
    \end{column}
    \begin{column}{0.5\textwidth}
      \raggedleft
      \onslide*<1>{\listStashBacktrace[highlightlines={114-115}]{}}
      \onslide*<2>{\providecommand\step{3}\input{diagrams/backtrace/backtrace.tex}}%
      \onslide*<3>{\providecommand\step{4}\input{diagrams/backtrace/backtrace.tex}}%
    \end{column}
  \end{columns}
\end{frame}

\begin{frame}{Backtraces}{Back to \funcname{caml\_raise\_exn}}
  \begin{columns}[c]
    \begin{column}{0.5\textwidth}
      \minipage[c][0.66\textheight][s]{\columnwidth}
      \begin{itemize}\color{gray}
        \item Checks wheteher exceptions backtrace should be recorded, by checking the \funcarg{domain\_state->backtrace\_active} value
        \item Switches to the C stack to be able to execute C code
        \item Calls \funcname{caml\_stash\_backtrace}, to collect the backtrace up to the next trap
      \end{itemize}
      \onslide*<2->{
        \begin{itemize}
          \item<2-> Executes the exception handler in \funcname{probably\_raise} with \funcname{RESTORE\_EXN\_HANDLER\_OCAML}
            \begin{itemize}
              \item Set \regname{RSP} to \localname{domain\_state->exn\_handler} value
              \item Restore \localname{domain\_state->exn\_handler} from trap
              \item Jump to the handler code with \funcname{ret}
            \end{itemize}
        \end{itemize}
      }
      \vfill
      \onslide*<1>{\mintocaml[firstline=9,lastline=13,highlightlines=12]{../src/backtrace/backtrace.ml}}
      \onslide*<2->{\mintocaml[firstline=15,lastline=21,highlightlines={19-21}]{../src/backtrace/backtrace.ml}}
      \endminipage
    \end{column}
    \begin{column}{0.5\textwidth}
      \centering
      \onslide*<1>{
        \mintc[firstline=190,lastline=192]{../ocaml/runtime/amd64.S}
        \mintc[firstline=234,lastline=237]{../ocaml/runtime/amd64.S}
      }
      \onslide*<2>{
        \mintc[firstline=190,lastline=192,highlightlines={191-192}]{../ocaml/runtime/amd64.S}
        \mintc[firstline=234,lastline=237,highlightlines={235-237}]{../ocaml/runtime/amd64.S}
      }
      \onslide*<1>{\listCamlRaiseExn[highlightlines=720]{}}
      \onslide*<2>{\listCamlRaiseExn[highlightlines=722]{}}
      \onslide*<3->{\providecommand\step{5}\input{diagrams/backtrace/backtrace.tex}}
    \end{column}
  \end{columns}
\end{frame}

\begin{frame}{Backtraces}{\funcname{caml\_probably\_raise}'s handler doesn't catch \typename{ExnC}}
  \begin{columns}[c]
    \begin{column}{0.45\textwidth}
      \minipage[c][0.75\textheight][s]{\columnwidth}
      \begin{itemize}
        \item<1-> \funcname{match \dots with}\footnotemark pattern matching can also match exceptions
        \item<1-> The CMM of \funcname{probably\_raise} shows that this \funcname{match \dots with} is implemented with a pattern matching and a \funcname{try \dots with}
        \item<1-> But this one only matches on \typename{ExnA} exception
        \item<2-> Since the pattern matching fails to catch the exception, \funcname{reraise} is called with our current \typename{ExnC} exception
        \item<3-> Disassembly shows that \funcname{reraise} is actually a call to \funcname{caml\_reraise\_exn}, which is also an assembly function provided by the runtime
      \end{itemize}
      \vfill
      \mintocaml[firstline=15,lastline=21,highlightlines={18-21}]{../src/backtrace/backtrace.ml}
      \endminipage
    \end{column}
    \begin{column}{0.55\textwidth}
      \centering
      \onslide*<1>{\mintcmm[highlightlines={10-18}]{../src/backtrace/camlBacktrace\_\_probably\_raise\_351.cmm}}
      \onslide*<2>{\listcmm[highlightlines=18]{../src/backtrace/camlBacktrace\_\_probably\_raise_351.cmm}{CMM of probably\_raise}}
      \onslide*<3>{\mintobjdump[firstline=5,lastline=35,highlightlines=31]{../src/backtrace/camlBacktrace\_\_probably\_raise_351.objdump}}
    \end{column}
  \end{columns}
  \only<1>{\footnotetext[1]{https://blog.janestreet.com/pattern-matching-and-exception-handling-unite/}}
\end{frame}

\newcommand\listCamlReraiseExn[1][]{
  \listasm[firstline=727,lastline=734,#1]{../ocaml/runtime/amd64.S}{runtime/amd64.S:727}}

\begin{frame}{Backtraces}{Introducing \funcname{caml\_reraise\_exn}}
  \begin{columns}[c]
    \begin{column}{0.5\textwidth}
      \minipage[c][0.66\textheight][s]{\columnwidth}
      \begin{itemize}
        \item<1-> The exception have to be reraised to the next handler
        \item<2-> First, checks if the backtrace should be collected with \funcarg{domain\_state->backtrace\_active}
        \only<3>{
        \begin{itemize}
            \item If not, behaves like a \funcname{raise\_notrace} with \funcname{RESTORE\_EXN\_HANDLER\_OCAML}
        \end{itemize}
        }
        \item<4-> Jumps into \funcname{caml\_raise\_exn}
        \item<5-> Calls \funcname{caml\_stash\_backtrace} again, to scan the next part of the stack: from the trap just pop-ed to the next trap
      \end{itemize}
      \vfill
      \mintocaml[firstline=15,lastline=21,highlightlines={18-21}]{../src/backtrace/backtrace.ml}
      \endminipage
    \end{column}
    \begin{column}{0.5\textwidth}
      \raggedleft
      \onslide*<1>{\listCamlReraiseExn{}}
      \onslide*<2>{\listCamlReraiseExn[highlightlines=729]{}}
      \onslide*<3>{
        \mintc[firstline=190,lastline=192,highlightlines={191-192}]{../ocaml/runtime/amd64.S}
        \mintc[firstline=234,lastline=237,highlightlines={235-237}]{../ocaml/runtime/amd64.S}
        \medskip
        \listCamlReraiseExn[highlightlines=731]{}
      }
      \onslide*<4-5>{
        % TODO refactor with \listCamlReraiseExn
        \mintasm[firstline=727,lastline=734,highlightlines=730]{../ocaml/runtime/amd64.S}
        \medskip
      }
      \onslide*<4>{\listCamlRaiseExn[highlightlines=711]{}}
      \onslide*<5>{\listCamlRaiseExn[highlightlines=720]{}}
    \end{column}
  \end{columns}
\end{frame}

\begin{frame}{Backtraces}{Backtrace collection continues}
  \begin{columns}[c]
    \begin{column}{0.55\textwidth}
      \minipage[c][0.75\textheight][s]{\columnwidth}
      \begin{itemize}
        \item<1-> \funcname{caml\_stash\_backtrace} scans the older part of the stack, up to the next trap
        \item<6-> Controls jumps to the nested exception handler in \funcname{maybe\_raise}, which matches only on \typename{ExnB}
        \item<7-> \funcname{caml\_reraise\_exn} is called again, backtrace collection resumes with \funcname{caml\_stash\_backtrace}
      \end{itemize}
      \medskip
      \begin{itemize}
        \item<2-> Constructed backtrace:
        \begin{itemize}
          \item <2-> \funcname{definitely\_raise}
          \item <2-> \funcname{probably\_raise}
          \item <3-> \funcname{probably\_raise} (re-raise)
          \item <4-> \funcname{maybe\_raise}
          \item <5-> \funcname{main}
          \item <8-> \funcname{main} (re-raise)
        \end{itemize}
      \end{itemize}
      \vfill
      \onslide*<1-5>{\mintocaml[firstline=15,lastline=21]{../src/backtrace/backtrace.ml}}
      \onslide*<6>{\mintocaml[firstline=28,lastline=34,highlightlines=31]{../src/backtrace/backtrace.ml}}
      \onslide*<7->{\mintocaml[firstline=28,lastline=34,highlightlines=33]{../src/backtrace/backtrace.ml}}
      \endminipage
    \end{column}
    \begin{column}{0.45\textwidth}
      \raggedleft
      \onslide*<1>{\providecommand\step{6}\input{diagrams/backtrace/backtrace.tex}}%
      \onslide*<2>{\providecommand\step{6}\input{diagrams/backtrace/backtrace.tex}}%
      \onslide*<3>{\providecommand\step{7}\input{diagrams/backtrace/backtrace.tex}}%
      \onslide*<4>{\providecommand\step{8}\input{diagrams/backtrace/backtrace.tex}}%
      \onslide*<5>{\providecommand\step{9}\input{diagrams/backtrace/backtrace.tex}}%
      \onslide*<6>{\providecommand\step{10}\input{diagrams/backtrace/backtrace.tex}}%
      \onslide*<7>{\providecommand\step{11}\input{diagrams/backtrace/backtrace.tex}}%
      \onslide*<8>{\providecommand\step{12}\input{diagrams/backtrace/backtrace.tex}}%
      \onslide*<9>{\providecommand\step{13}\input{diagrams/backtrace/backtrace.tex}}%
    \end{column}
  \end{columns}
\end{frame}

\begin{frame}{Backtraces}{Printing the backtrace}
  \begin{columns}[c]
    \begin{column}{0.45\textwidth}
      \minipage[c][0.66\textheight][s]{\columnwidth}
      \begin{itemize}
        \item<1-> The exception backtrace can now be printed to \funcarg{stdout} with \funcname{Printexc.print_backtrace}
        \item<2-> First the exception backtrace is retrieved into an OCaml array with \funcname{get_raw_backtrace }
        \item<3-> Which is implemented as the \funcname{caml_get_exception_raw_backtrace} C function in the runtime
        \item<4-> And finally printed with \funcname{print_exception_backtrace}
      \end{itemize}
      \vfill
      \mintocaml[firstline=28,lastline=34,highlightlines=33]{../src/backtrace/backtrace.ml}
      \endminipage
    \end{column}
    \begin{column}{0.55\textwidth}
      \onslide*<1,2>{
        \mintocaml[firstline=100,lastline=101]{../ocaml/stdlib/printexc.ml}
      }
      \only<1>{
        \listocaml[firstline=170,lastline=172]{../ocaml/stdlib/printexc.ml}{stdlib/printexc.ml:170}
      }
      \only<2>{
        \listocaml[firstline=170,lastline=172,highlightlines=172]{../ocaml/stdlib/printexc.ml}{stdlib/printexc.ml:170}
      }
      \only<3>{
        \mintc[firstline=154,lastline=156]{../ocaml/runtime/backtrace.c}
        \listc[firstline=171,lastline=192,highlightlines={184-187}]{../ocaml/runtime/backtrace.c}{runtime/backtrace.c:154}
      }
      \only<4>{
        \listocaml[firstline=155,lastline=165]{../ocaml/stdlib/printexc.ml}{stdlib/printexc.ml:55}
      }
    \end{column}
  \end{columns}
\end{frame}


\subsection{The multiple kinds of raise}
\frameSubsection{}{}

\begin{frame}{Backtraces}{When backtraces are not needed}
  \begin{columns}[c]
    \begin{column}{0.45\textwidth}
      \minipage[c][0.66\textheight][s]{\columnwidth}
      \begin{itemize}
        \item<1-> What if the backtrace for an exception is never needed?
        \item<2-> It's possible to raise the exception explicitly with \funcname{raise\_notrace} to make raising as cheap as possible
        \item<3-> An example of use in the \funcname{dynlink} library of the OCaml compiler
      \end{itemize}
      \vfill
      \onslide<3->{
      \listocaml[firstline=205,lastline=209,highlightlines=207]{../ocaml/otherlibs/dynlink/dynlink\_compilerlibs/misc.ml}{otherlibs/dynlink/dynlink\_compilerlibs/misc.ml:205}
      }
      \endminipage
    \end{column}
    \begin{column}{0.55\textwidth}
    \only<4>{
      \mintcmm[highlightlines=3]{../src/notrace/camlNotrace\_\_fun_347.cmm}
      \listcmm{../src/notrace/camlNotrace\_\_all\_somes\_327.cmm}{all\_somes}
    }
    \onslide*<5->{
      \listobjdump[highlightlines={6-9}]{../src/notrace/camlNotrace\_\_fun_347.objdump}{anonymous function from \funcname{all\_somes}}
    }
    \end{column}
  \end{columns}
\end{frame}

\begin{frame}{Backtraces}{Summary table of the multiple kinds of raise}
  \begin{itemize}
    \item When debugging information is enabled and if the backtrace of an exception isn't needed, it's possible to raise with \funcname{raise\_notrace} for performance
    \item When debugging information is \emph{not} enabled, the compiler optimises \funcname{raise} calls to inline assembly (same effect as using \funcname{raise\_notrace})
  \end{itemize}
  \bigskip
  \centering
  \begin{tabular}{| l | c | c | c | c |}
    \hline
    \parbox{10em}{Debug information \\ (\funcarg{-g} flag)}  & \multicolumn{2}{|c|}{Disabled}                                     & \multicolumn{2}{|c|}{Enabled} \\
    \hline
    Raise kind                                               & \funcname{raise} & \funcname{raise\_notrace}                       & \funcname{raise} & \funcname{raise\_notrace} \\
    \hline
    Compilation                                              & \parbox{8em}{inline assembly\\ (optimization)} & inline assembly   & \funcname{caml\_raise\_exn} & inline assembly \\
    \hline
  \end{tabular}
\end{frame}


%
% Takeaway #5
%
\subsection*{Takeaway \#5}
\frameSubsectionTakeaway{}
\againframe<5>{takeaway}

    \end{column}
  \end{columns}
\end{frame}

\begin{frame}{Backtraces}{But before that, we need more fields from \typename{caml\_domain\_state}}
  \begin{columns}
    \begin{column}{0.5\textwidth}
      \begin{itemize}
        \item Remember \typename{caml\_domain\_state} the per-domain runtime state?
        \item Some other members are of interest to us now\dots
        \item \localname{backtrace\_active} is used to know if the runtime should collect backtraces
        \item \localname{backtrace\_active} can be set by
          \begin{itemize}
            \item The \funcarg{b} flag in the \funcname{OCAMLRUNPARAM} environment variable
            \item \mintinline{ocaml}{Printexc.record_backtrace : bool -> unit} \footnotemark
          \end{itemize}
      \end{itemize}
    \end{column}
    \begin{column}{0.5\textwidth}
      \begin{listing}
        \mintc[highlightlines={9-11}]{src/ocaml/domain\_state\_backtrace.h}
        \caption{runtime/caml/domain\_state.h}
      \end{listing}
    \end{column}
  \end{columns}
  \footnotetext[1]{https://v2.ocaml.org/api/Printexc.html}
\end{frame}

\newcommand\listCamlRaiseExn[1][]{
  \listasm[firstline=702,lastline=725,#1]{../ocaml/runtime/amd64.S}{runtime/amd64.S:702}}

\begin{frame}{Backtraces}{Walk through \funcname{caml\_raise\_exn}}
  \begin{columns}[T]
    \begin{column}{0.5\textwidth}
      \begin{itemize}
        \item<1-> First checks whether exceptions backtrace should be recorded
        \item<1-> By checking the \funcarg{domain\_state->backtrace\_active} value
        \item<2-> If not, acts as \funcname{raise\_notrace} with \funcname{RESTORE\_EXN\_HANDLER\_OCAML}
          \begin{itemize}
            \item Set \regname{RSP} to \localname{domain\_state->exn\_handler} value
            \item Restore \localname{domain\_state->exn\_handler} from trap
            \item Jump with \funcname{ret} (same effect as using \funcname{pop} and \funcname{jmp})
          \end{itemize}
        \item<3-> Otherwise, switches to the C stack to be able to execute C code
        \item<4-> Calls \funcname{caml\_stash\_backtrace}, a C function provided by the runtime\ignorespaces
          \onslide<6->{, with
            \begin{itemize}
              \item \funcarg{exn}: exception value
              \item \funcarg{pc}: code address of the raising point
              \item \funcarg{sp}: stack address of the raising function
              \item \funcarg{trapsp}: stack address of the next handler
            \end{itemize}
          }
      \end{itemize}
      \bigskip
      \mintocaml[firstline=9,lastline=13,highlightlines=12]{../src/backtrace/backtrace.ml}
    \end{column}
    \begin{column}{0.5\textwidth}
      \raggedleft
      \onslide*<1,3-4>{
        \mintc[firstline=190,lastline=192]{../ocaml/runtime/amd64.S}
        \mintc[firstline=234,lastline=237]{../ocaml/runtime/amd64.S}
      }
      \onslide*<2>{
        \mintc[firstline=190,lastline=192,highlightlines={191-192}]{../ocaml/runtime/amd64.S}
        \mintc[firstline=234,lastline=237,highlightlines={235-237}]{../ocaml/runtime/amd64.S}
      }
      \onslide*<1>{\listCamlRaiseExn[highlightlines=705]{}}%
      \onslide*<2>{\listCamlRaiseExn[highlightlines={707-708}]{}}%
      \onslide*<3>{\listCamlRaiseExn[highlightlines={712-714}]{}}%
      \onslide*<4>{\listCamlRaiseExn[highlightlines={715-720}]{}}%
      \onslide*<5>{\providecommand\step{1}%
% Backtraces
%
\subsection{One advantage of raising with debugging information}
\frameSubsection{}{}

\begin{frame}{Backtraces}{Debugging information \& source code}
  \begin{columns}[c]
    \begin{column}{0.5\textwidth}
      \begin{itemize}
        \item Raising an exception is fun and games, but how do we know where the exception originated? $\rightarrow$ With the backtrace associated with the exception!
        \item In order to get a backtrace from the point where an exception was raised, it's necessary to compile with debugging information enabled (\funcarg{-g} flag for the compiler)
        \item Same example as in the `Nested handlers' section
      \end{itemize}
    \end{column}
    \begin{column}{0.5\textwidth}
      \listocaml[firstline=9,lastline=34]{../src/backtrace/backtrace.ml}{backtrace/backtrace.ml}
    \end{column}
  \end{columns}
\end{frame}

\begin{frame}{Backtraces}{CMM and disassembly of \funcname{definitely\_raise}}
  \begin{columns}[c]
    \begin{column}{0.5\textwidth}
      \minipage[c][0.66\textheight][s]{\columnwidth}
      \begin{itemize}
        \item<1-> An effect of compiling with debugging information (\funcarg{-g}): the CMM no longer uses \funcname{raise\_notrace} to raise the exception but \funcname{raise} instead
        \item<2-> Disassembly shows that CMM \funcname{raise} is actually a call to \funcname{caml\_raise\_exn}, a function in assembler provided by the runtime
      \end{itemize}
      \vfill
      \mintocaml[firstline=9,lastline=13,highlightlines=12]{../src/backtrace/backtrace.ml}
      \endminipage
    \end{column}
    \begin{column}{0.5\textwidth}
      \onslide*<1>{
        \mintcmm[highlightlines=5]{../src/backtrace/camlBacktrace\_\_definitely\_raise\_347.cmm}
      }
      \onslide*<2>{
        \mintobjdump[firstline=2,highlightlines=14]{../src/backtrace/camlBacktrace\_\_definitely\_raise\_347.objdump}
      }
    \end{column}
  \end{columns}
\end{frame}

\begin{frame}{Backtraces}{Introducing \funcname{caml\_raise\_exn}}
  \begin{columns}[c]
    \begin{column}{0.5\textwidth}
      \minipage[c][0.66\textheight][s]{\columnwidth}
      \onslide*<1>{
        \begin{itemize}
          \item Raising the exception is performed using \funcname{caml\_raise\_exn} which is
            \begin{itemize}
              \item An assembly function provided by the runtime
              \item Architecture specific
            \end{itemize}
          \item Let's see what it does differently from \funcname{raise\_notrace}\dots
          \item We are about to raise \typename{ExnC} with \funcname{caml\_raise\_exn} from \funcname{definitely\_raise}
        \end{itemize}
      }
      \onslide*<2>{
        \mintobjdump[firstline=2,highlightlines=14]{../src/backtrace/camlBacktrace\_\_definitely\_raise\_347.objdump}
      }
      \vfill
      \mintocaml[firstline=9,lastline=13,highlightlines=12]{../src/backtrace/backtrace.ml}
      \endminipage
    \end{column}
    \begin{column}{0.5\textwidth}
      \centering
      \providecommand\step{1}\input{diagrams/backtrace/backtrace.tex}
    \end{column}
  \end{columns}
\end{frame}

\begin{frame}{Backtraces}{But before that, we need more fields from \typename{caml\_domain\_state}}
  \begin{columns}
    \begin{column}{0.5\textwidth}
      \begin{itemize}
        \item Remember \typename{caml\_domain\_state} the per-domain runtime state?
        \item Some other members are of interest to us now\dots
        \item \localname{backtrace\_active} is used to know if the runtime should collect backtraces
        \item \localname{backtrace\_active} can be set by
          \begin{itemize}
            \item The \funcarg{b} flag in the \funcname{OCAMLRUNPARAM} environment variable
            \item \mintinline{ocaml}{Printexc.record_backtrace : bool -> unit} \footnotemark
          \end{itemize}
      \end{itemize}
    \end{column}
    \begin{column}{0.5\textwidth}
      \begin{listing}
        \mintc[highlightlines={9-11}]{src/ocaml/domain\_state\_backtrace.h}
        \caption{runtime/caml/domain\_state.h}
      \end{listing}
    \end{column}
  \end{columns}
  \footnotetext[1]{https://v2.ocaml.org/api/Printexc.html}
\end{frame}

\newcommand\listCamlRaiseExn[1][]{
  \listasm[firstline=702,lastline=725,#1]{../ocaml/runtime/amd64.S}{runtime/amd64.S:702}}

\begin{frame}{Backtraces}{Walk through \funcname{caml\_raise\_exn}}
  \begin{columns}[T]
    \begin{column}{0.5\textwidth}
      \begin{itemize}
        \item<1-> First checks whether exceptions backtrace should be recorded
        \item<1-> By checking the \funcarg{domain\_state->backtrace\_active} value
        \item<2-> If not, acts as \funcname{raise\_notrace} with \funcname{RESTORE\_EXN\_HANDLER\_OCAML}
          \begin{itemize}
            \item Set \regname{RSP} to \localname{domain\_state->exn\_handler} value
            \item Restore \localname{domain\_state->exn\_handler} from trap
            \item Jump with \funcname{ret} (same effect as using \funcname{pop} and \funcname{jmp})
          \end{itemize}
        \item<3-> Otherwise, switches to the C stack to be able to execute C code
        \item<4-> Calls \funcname{caml\_stash\_backtrace}, a C function provided by the runtime\ignorespaces
          \onslide<6->{, with
            \begin{itemize}
              \item \funcarg{exn}: exception value
              \item \funcarg{pc}: code address of the raising point
              \item \funcarg{sp}: stack address of the raising function
              \item \funcarg{trapsp}: stack address of the next handler
            \end{itemize}
          }
      \end{itemize}
      \bigskip
      \mintocaml[firstline=9,lastline=13,highlightlines=12]{../src/backtrace/backtrace.ml}
    \end{column}
    \begin{column}{0.5\textwidth}
      \raggedleft
      \onslide*<1,3-4>{
        \mintc[firstline=190,lastline=192]{../ocaml/runtime/amd64.S}
        \mintc[firstline=234,lastline=237]{../ocaml/runtime/amd64.S}
      }
      \onslide*<2>{
        \mintc[firstline=190,lastline=192,highlightlines={191-192}]{../ocaml/runtime/amd64.S}
        \mintc[firstline=234,lastline=237,highlightlines={235-237}]{../ocaml/runtime/amd64.S}
      }
      \onslide*<1>{\listCamlRaiseExn[highlightlines=705]{}}%
      \onslide*<2>{\listCamlRaiseExn[highlightlines={707-708}]{}}%
      \onslide*<3>{\listCamlRaiseExn[highlightlines={712-714}]{}}%
      \onslide*<4>{\listCamlRaiseExn[highlightlines={715-720}]{}}%
      \onslide*<5>{\providecommand\step{1}\input{diagrams/backtrace/backtrace.tex}}%
      \onslide*<6>{\providecommand\step{2}\input{diagrams/backtrace/backtrace.tex}}%
    \end{column}
  \end{columns}
\end{frame}

\newcommand\listStashBacktrace[1][]{
  \listc[firstline=91,lastline=120,#1]{../ocaml/runtime/backtrace\_nat.c}{runtime/backtrace\_nat.c:91}}

\begin{frame}{Backtraces}{Introducing \funcname{caml\_stash\_backtrace}}
  \begin{columns}[c]
    \begin{column}{0.5\textwidth}
      \minipage[c][0.66\textheight][s]{\columnwidth}
      \begin{itemize}
        \item<1-> A C function provided by the runtime (specific to native code)
        \item<2-> It scans the stack, between the stack pointer at the raising point up to the next trap, in order to collect the names of the detected function frames
          \onslide*<2>{
            \begin{itemize}
              \item And more information, like line number, column number...
            \end{itemize}
          }
        \item<3-> It uses the same technique and information as the Garbage Collector (GC) uses to scan the values live on the stack
          \onslide*<4>{
            \begin{itemize}
              \item \typename{frame\_descr} are at the core of stack unwinding
            \end{itemize}
          }
      \end{itemize}
      \vfill
      \mintocaml[firstline=9,lastline=13,highlightlines=12]{../src/backtrace/backtrace.ml}
      \endminipage
    \end{column}
    \begin{column}{0.5\textwidth}
      \onslide*<1>{\listStashBacktrace[highlightlines=91]{}}
      \onslide*<2>{\listStashBacktrace[highlightlines={91,117-118}]{}}
      \onslide*<3->{\listStashBacktrace[highlightlines={94,106,109-110}]{}}
    \end{column}
  \end{columns}
\end{frame}

\newcommand\listFrameDescr[1][]{
  \listocaml[firstline=111,lastline=124,#1]{../ocaml/asmcomp/emitaux.ml}{asmcomp/emitaux.ml:111}}
\newcommand\listDebugInfo[1][]{
  \listocaml[firstline=38,lastline=49,#1]{../ocaml/lambda/debuginfo.mli}{lambda/debuginfo.mli:38}}

\begin{frame}{Backtraces}{\typename{frame\_descr} and \typename{frame\_debuginfo} in a nutshell}
  \begin{columns}[c]
    \begin{column}{0.5\textwidth}
      \begin{itemize}
        \item<1-> For every return address in an OCaml program, there is a associated \typename{frame\_descr}
        \item<2-> A \typename{frame\_descr} specifies
        \begin{itemize}
          \item<2-> The size of the function stack frame
          \item<3-> Where live values are on the stack (so that the GC can mark them as live and prevent from being swept)
        \end{itemize}
        \item<4-> When compiled with debug information, \typename{frame\_descr} will be linked to a \typename{frame\_debuginfo}
        \begin{itemize}
          \item<5-> Contains information about the position in the source code (file name, line and column number)
        \end{itemize}
      \end{itemize}
    \end{column}
    \begin{column}{0.5\textwidth}
        \onslide*<1>{\listFrameDescr[highlightlines={118-122}]{}}
        \onslide*<2>{\listFrameDescr[highlightlines={120}]{}}
        \onslide*<3>{\listFrameDescr[highlightlines={121}]{}}
        \onslide*<4->{\listFrameDescr[highlightlines={122}]{}}
        \medskip
        \onslide*<1-3>{\listDebugInfo{}}
        \onslide*<4>{\listDebugInfo[highlightlines={38,49}]{}}
        \onslide*<5>{\listDebugInfo[highlightlines={39-46}]{}}
    \end{column}
  \end{columns}
\end{frame}

\begin{frame}{Backtraces}{Scanning the stack with \typename{frame\_descr}}
  \begin{columns}[c]
    \begin{column}{0.5\textwidth}
      \begin{itemize}
        \item Given the return address of the code that raises the exception by calling \funcname{caml\_raise\_exn} (\funcarg{pc} argument of \funcname{caml\_stash\_backtrace})
        \item And the position of the stack pointer (\funcarg{sp} argument of \funcname{caml\_stash\_backtrace})
        \item The frame size given by \typename{frame\_descr}.\funcarg{frame\_size}, allows to find the end of the previous frame
        \item And the return address of the caller of this function
        \bigskip
        \onslide<3->{
        \item Note that traps (here the reddish \funcname{ExnA}, \funcname{ExnB} and \funcname{ExnC} blocks) belong to the frame that installed them, and so the frame size changes when traps are removed
        }
      \end{itemize}
    \end{column}
    \begin{column}{0.5\textwidth}
      \raggedleft
      \onslide*<1>{\providecommand\step{2}\input{diagrams/backtrace/backtrace.tex}}%
      \onslide*<2>{\providecommand\step{1}\input{diagrams/backtrace/scanstack.tex}}%
      \onslide*<3>{\providecommand\step{2}\input{diagrams/backtrace/scanstack.tex}}%
      \onslide*<4>{\providecommand\step{3}\input{diagrams/backtrace/scanstack.tex}}%
      \onslide*<5>{\providecommand\step{4}\input{diagrams/backtrace/scanstack.tex}}%
      \onslide*<6>{\providecommand\step{5}\input{diagrams/backtrace/scanstack.tex}}%
    \end{column}
  \end{columns}
\end{frame}

% Duplicate of previous-previous slide
\begin{frame}{Backtraces}{Continuing on \funcname{caml\_stash\_backtrace}}
  \begin{columns}[c]
    \begin{column}{0.5\textwidth}
      \minipage[c][0.75\textheight][s]{\columnwidth}
      \begin{itemize}
          \color{gray}
        \item A C function provided by the runtime (specific to native code)
        \item It scans the stack, between the stack pointer at the raising point up to the next trap, in order to collect the names of the detected function frames
        \item It uses the same technique and information as the Garbage Collector (GC) uses to scan the values live on the stack
      \end{itemize}
      \begin{itemize}
        \item For each function frame detected, store a pointer to the \typename{frame\_descr} in the \localname{domain\_state->backtrace\_buffer} array, effectively constructing the backtrace
      \end{itemize}
      \onslide<2->{
        \begin{itemize}
            \smallskip
          \item Constructed backtrace:
            \funcname{definitely\_raise},
            \onslide<3->{\funcname{probably\_raise}}
        \end{itemize}
      }
      \vfill
      \mintocaml[firstline=9,lastline=13,highlightlines=12]{../src/backtrace/backtrace.ml}
      \endminipage
    \end{column}
    \begin{column}{0.5\textwidth}
      \raggedleft
      \onslide*<1>{\listStashBacktrace[highlightlines={114-115}]{}}
      \onslide*<2>{\providecommand\step{3}\input{diagrams/backtrace/backtrace.tex}}%
      \onslide*<3>{\providecommand\step{4}\input{diagrams/backtrace/backtrace.tex}}%
    \end{column}
  \end{columns}
\end{frame}

\begin{frame}{Backtraces}{Back to \funcname{caml\_raise\_exn}}
  \begin{columns}[c]
    \begin{column}{0.5\textwidth}
      \minipage[c][0.66\textheight][s]{\columnwidth}
      \begin{itemize}\color{gray}
        \item Checks wheteher exceptions backtrace should be recorded, by checking the \funcarg{domain\_state->backtrace\_active} value
        \item Switches to the C stack to be able to execute C code
        \item Calls \funcname{caml\_stash\_backtrace}, to collect the backtrace up to the next trap
      \end{itemize}
      \onslide*<2->{
        \begin{itemize}
          \item<2-> Executes the exception handler in \funcname{probably\_raise} with \funcname{RESTORE\_EXN\_HANDLER\_OCAML}
            \begin{itemize}
              \item Set \regname{RSP} to \localname{domain\_state->exn\_handler} value
              \item Restore \localname{domain\_state->exn\_handler} from trap
              \item Jump to the handler code with \funcname{ret}
            \end{itemize}
        \end{itemize}
      }
      \vfill
      \onslide*<1>{\mintocaml[firstline=9,lastline=13,highlightlines=12]{../src/backtrace/backtrace.ml}}
      \onslide*<2->{\mintocaml[firstline=15,lastline=21,highlightlines={19-21}]{../src/backtrace/backtrace.ml}}
      \endminipage
    \end{column}
    \begin{column}{0.5\textwidth}
      \centering
      \onslide*<1>{
        \mintc[firstline=190,lastline=192]{../ocaml/runtime/amd64.S}
        \mintc[firstline=234,lastline=237]{../ocaml/runtime/amd64.S}
      }
      \onslide*<2>{
        \mintc[firstline=190,lastline=192,highlightlines={191-192}]{../ocaml/runtime/amd64.S}
        \mintc[firstline=234,lastline=237,highlightlines={235-237}]{../ocaml/runtime/amd64.S}
      }
      \onslide*<1>{\listCamlRaiseExn[highlightlines=720]{}}
      \onslide*<2>{\listCamlRaiseExn[highlightlines=722]{}}
      \onslide*<3->{\providecommand\step{5}\input{diagrams/backtrace/backtrace.tex}}
    \end{column}
  \end{columns}
\end{frame}

\begin{frame}{Backtraces}{\funcname{caml\_probably\_raise}'s handler doesn't catch \typename{ExnC}}
  \begin{columns}[c]
    \begin{column}{0.45\textwidth}
      \minipage[c][0.75\textheight][s]{\columnwidth}
      \begin{itemize}
        \item<1-> \funcname{match \dots with}\footnotemark pattern matching can also match exceptions
        \item<1-> The CMM of \funcname{probably\_raise} shows that this \funcname{match \dots with} is implemented with a pattern matching and a \funcname{try \dots with}
        \item<1-> But this one only matches on \typename{ExnA} exception
        \item<2-> Since the pattern matching fails to catch the exception, \funcname{reraise} is called with our current \typename{ExnC} exception
        \item<3-> Disassembly shows that \funcname{reraise} is actually a call to \funcname{caml\_reraise\_exn}, which is also an assembly function provided by the runtime
      \end{itemize}
      \vfill
      \mintocaml[firstline=15,lastline=21,highlightlines={18-21}]{../src/backtrace/backtrace.ml}
      \endminipage
    \end{column}
    \begin{column}{0.55\textwidth}
      \centering
      \onslide*<1>{\mintcmm[highlightlines={10-18}]{../src/backtrace/camlBacktrace\_\_probably\_raise\_351.cmm}}
      \onslide*<2>{\listcmm[highlightlines=18]{../src/backtrace/camlBacktrace\_\_probably\_raise_351.cmm}{CMM of probably\_raise}}
      \onslide*<3>{\mintobjdump[firstline=5,lastline=35,highlightlines=31]{../src/backtrace/camlBacktrace\_\_probably\_raise_351.objdump}}
    \end{column}
  \end{columns}
  \only<1>{\footnotetext[1]{https://blog.janestreet.com/pattern-matching-and-exception-handling-unite/}}
\end{frame}

\newcommand\listCamlReraiseExn[1][]{
  \listasm[firstline=727,lastline=734,#1]{../ocaml/runtime/amd64.S}{runtime/amd64.S:727}}

\begin{frame}{Backtraces}{Introducing \funcname{caml\_reraise\_exn}}
  \begin{columns}[c]
    \begin{column}{0.5\textwidth}
      \minipage[c][0.66\textheight][s]{\columnwidth}
      \begin{itemize}
        \item<1-> The exception have to be reraised to the next handler
        \item<2-> First, checks if the backtrace should be collected with \funcarg{domain\_state->backtrace\_active}
        \only<3>{
        \begin{itemize}
            \item If not, behaves like a \funcname{raise\_notrace} with \funcname{RESTORE\_EXN\_HANDLER\_OCAML}
        \end{itemize}
        }
        \item<4-> Jumps into \funcname{caml\_raise\_exn}
        \item<5-> Calls \funcname{caml\_stash\_backtrace} again, to scan the next part of the stack: from the trap just pop-ed to the next trap
      \end{itemize}
      \vfill
      \mintocaml[firstline=15,lastline=21,highlightlines={18-21}]{../src/backtrace/backtrace.ml}
      \endminipage
    \end{column}
    \begin{column}{0.5\textwidth}
      \raggedleft
      \onslide*<1>{\listCamlReraiseExn{}}
      \onslide*<2>{\listCamlReraiseExn[highlightlines=729]{}}
      \onslide*<3>{
        \mintc[firstline=190,lastline=192,highlightlines={191-192}]{../ocaml/runtime/amd64.S}
        \mintc[firstline=234,lastline=237,highlightlines={235-237}]{../ocaml/runtime/amd64.S}
        \medskip
        \listCamlReraiseExn[highlightlines=731]{}
      }
      \onslide*<4-5>{
        % TODO refactor with \listCamlReraiseExn
        \mintasm[firstline=727,lastline=734,highlightlines=730]{../ocaml/runtime/amd64.S}
        \medskip
      }
      \onslide*<4>{\listCamlRaiseExn[highlightlines=711]{}}
      \onslide*<5>{\listCamlRaiseExn[highlightlines=720]{}}
    \end{column}
  \end{columns}
\end{frame}

\begin{frame}{Backtraces}{Backtrace collection continues}
  \begin{columns}[c]
    \begin{column}{0.55\textwidth}
      \minipage[c][0.75\textheight][s]{\columnwidth}
      \begin{itemize}
        \item<1-> \funcname{caml\_stash\_backtrace} scans the older part of the stack, up to the next trap
        \item<6-> Controls jumps to the nested exception handler in \funcname{maybe\_raise}, which matches only on \typename{ExnB}
        \item<7-> \funcname{caml\_reraise\_exn} is called again, backtrace collection resumes with \funcname{caml\_stash\_backtrace}
      \end{itemize}
      \medskip
      \begin{itemize}
        \item<2-> Constructed backtrace:
        \begin{itemize}
          \item <2-> \funcname{definitely\_raise}
          \item <2-> \funcname{probably\_raise}
          \item <3-> \funcname{probably\_raise} (re-raise)
          \item <4-> \funcname{maybe\_raise}
          \item <5-> \funcname{main}
          \item <8-> \funcname{main} (re-raise)
        \end{itemize}
      \end{itemize}
      \vfill
      \onslide*<1-5>{\mintocaml[firstline=15,lastline=21]{../src/backtrace/backtrace.ml}}
      \onslide*<6>{\mintocaml[firstline=28,lastline=34,highlightlines=31]{../src/backtrace/backtrace.ml}}
      \onslide*<7->{\mintocaml[firstline=28,lastline=34,highlightlines=33]{../src/backtrace/backtrace.ml}}
      \endminipage
    \end{column}
    \begin{column}{0.45\textwidth}
      \raggedleft
      \onslide*<1>{\providecommand\step{6}\input{diagrams/backtrace/backtrace.tex}}%
      \onslide*<2>{\providecommand\step{6}\input{diagrams/backtrace/backtrace.tex}}%
      \onslide*<3>{\providecommand\step{7}\input{diagrams/backtrace/backtrace.tex}}%
      \onslide*<4>{\providecommand\step{8}\input{diagrams/backtrace/backtrace.tex}}%
      \onslide*<5>{\providecommand\step{9}\input{diagrams/backtrace/backtrace.tex}}%
      \onslide*<6>{\providecommand\step{10}\input{diagrams/backtrace/backtrace.tex}}%
      \onslide*<7>{\providecommand\step{11}\input{diagrams/backtrace/backtrace.tex}}%
      \onslide*<8>{\providecommand\step{12}\input{diagrams/backtrace/backtrace.tex}}%
      \onslide*<9>{\providecommand\step{13}\input{diagrams/backtrace/backtrace.tex}}%
    \end{column}
  \end{columns}
\end{frame}

\begin{frame}{Backtraces}{Printing the backtrace}
  \begin{columns}[c]
    \begin{column}{0.45\textwidth}
      \minipage[c][0.66\textheight][s]{\columnwidth}
      \begin{itemize}
        \item<1-> The exception backtrace can now be printed to \funcarg{stdout} with \funcname{Printexc.print_backtrace}
        \item<2-> First the exception backtrace is retrieved into an OCaml array with \funcname{get_raw_backtrace }
        \item<3-> Which is implemented as the \funcname{caml_get_exception_raw_backtrace} C function in the runtime
        \item<4-> And finally printed with \funcname{print_exception_backtrace}
      \end{itemize}
      \vfill
      \mintocaml[firstline=28,lastline=34,highlightlines=33]{../src/backtrace/backtrace.ml}
      \endminipage
    \end{column}
    \begin{column}{0.55\textwidth}
      \onslide*<1,2>{
        \mintocaml[firstline=100,lastline=101]{../ocaml/stdlib/printexc.ml}
      }
      \only<1>{
        \listocaml[firstline=170,lastline=172]{../ocaml/stdlib/printexc.ml}{stdlib/printexc.ml:170}
      }
      \only<2>{
        \listocaml[firstline=170,lastline=172,highlightlines=172]{../ocaml/stdlib/printexc.ml}{stdlib/printexc.ml:170}
      }
      \only<3>{
        \mintc[firstline=154,lastline=156]{../ocaml/runtime/backtrace.c}
        \listc[firstline=171,lastline=192,highlightlines={184-187}]{../ocaml/runtime/backtrace.c}{runtime/backtrace.c:154}
      }
      \only<4>{
        \listocaml[firstline=155,lastline=165]{../ocaml/stdlib/printexc.ml}{stdlib/printexc.ml:55}
      }
    \end{column}
  \end{columns}
\end{frame}


\subsection{The multiple kinds of raise}
\frameSubsection{}{}

\begin{frame}{Backtraces}{When backtraces are not needed}
  \begin{columns}[c]
    \begin{column}{0.45\textwidth}
      \minipage[c][0.66\textheight][s]{\columnwidth}
      \begin{itemize}
        \item<1-> What if the backtrace for an exception is never needed?
        \item<2-> It's possible to raise the exception explicitly with \funcname{raise\_notrace} to make raising as cheap as possible
        \item<3-> An example of use in the \funcname{dynlink} library of the OCaml compiler
      \end{itemize}
      \vfill
      \onslide<3->{
      \listocaml[firstline=205,lastline=209,highlightlines=207]{../ocaml/otherlibs/dynlink/dynlink\_compilerlibs/misc.ml}{otherlibs/dynlink/dynlink\_compilerlibs/misc.ml:205}
      }
      \endminipage
    \end{column}
    \begin{column}{0.55\textwidth}
    \only<4>{
      \mintcmm[highlightlines=3]{../src/notrace/camlNotrace\_\_fun_347.cmm}
      \listcmm{../src/notrace/camlNotrace\_\_all\_somes\_327.cmm}{all\_somes}
    }
    \onslide*<5->{
      \listobjdump[highlightlines={6-9}]{../src/notrace/camlNotrace\_\_fun_347.objdump}{anonymous function from \funcname{all\_somes}}
    }
    \end{column}
  \end{columns}
\end{frame}

\begin{frame}{Backtraces}{Summary table of the multiple kinds of raise}
  \begin{itemize}
    \item When debugging information is enabled and if the backtrace of an exception isn't needed, it's possible to raise with \funcname{raise\_notrace} for performance
    \item When debugging information is \emph{not} enabled, the compiler optimises \funcname{raise} calls to inline assembly (same effect as using \funcname{raise\_notrace})
  \end{itemize}
  \bigskip
  \centering
  \begin{tabular}{| l | c | c | c | c |}
    \hline
    \parbox{10em}{Debug information \\ (\funcarg{-g} flag)}  & \multicolumn{2}{|c|}{Disabled}                                     & \multicolumn{2}{|c|}{Enabled} \\
    \hline
    Raise kind                                               & \funcname{raise} & \funcname{raise\_notrace}                       & \funcname{raise} & \funcname{raise\_notrace} \\
    \hline
    Compilation                                              & \parbox{8em}{inline assembly\\ (optimization)} & inline assembly   & \funcname{caml\_raise\_exn} & inline assembly \\
    \hline
  \end{tabular}
\end{frame}


%
% Takeaway #5
%
\subsection*{Takeaway \#5}
\frameSubsectionTakeaway{}
\againframe<5>{takeaway}
}%
      \onslide*<6>{\providecommand\step{2}%
% Backtraces
%
\subsection{One advantage of raising with debugging information}
\frameSubsection{}{}

\begin{frame}{Backtraces}{Debugging information \& source code}
  \begin{columns}[c]
    \begin{column}{0.5\textwidth}
      \begin{itemize}
        \item Raising an exception is fun and games, but how do we know where the exception originated? $\rightarrow$ With the backtrace associated with the exception!
        \item In order to get a backtrace from the point where an exception was raised, it's necessary to compile with debugging information enabled (\funcarg{-g} flag for the compiler)
        \item Same example as in the `Nested handlers' section
      \end{itemize}
    \end{column}
    \begin{column}{0.5\textwidth}
      \listocaml[firstline=9,lastline=34]{../src/backtrace/backtrace.ml}{backtrace/backtrace.ml}
    \end{column}
  \end{columns}
\end{frame}

\begin{frame}{Backtraces}{CMM and disassembly of \funcname{definitely\_raise}}
  \begin{columns}[c]
    \begin{column}{0.5\textwidth}
      \minipage[c][0.66\textheight][s]{\columnwidth}
      \begin{itemize}
        \item<1-> An effect of compiling with debugging information (\funcarg{-g}): the CMM no longer uses \funcname{raise\_notrace} to raise the exception but \funcname{raise} instead
        \item<2-> Disassembly shows that CMM \funcname{raise} is actually a call to \funcname{caml\_raise\_exn}, a function in assembler provided by the runtime
      \end{itemize}
      \vfill
      \mintocaml[firstline=9,lastline=13,highlightlines=12]{../src/backtrace/backtrace.ml}
      \endminipage
    \end{column}
    \begin{column}{0.5\textwidth}
      \onslide*<1>{
        \mintcmm[highlightlines=5]{../src/backtrace/camlBacktrace\_\_definitely\_raise\_347.cmm}
      }
      \onslide*<2>{
        \mintobjdump[firstline=2,highlightlines=14]{../src/backtrace/camlBacktrace\_\_definitely\_raise\_347.objdump}
      }
    \end{column}
  \end{columns}
\end{frame}

\begin{frame}{Backtraces}{Introducing \funcname{caml\_raise\_exn}}
  \begin{columns}[c]
    \begin{column}{0.5\textwidth}
      \minipage[c][0.66\textheight][s]{\columnwidth}
      \onslide*<1>{
        \begin{itemize}
          \item Raising the exception is performed using \funcname{caml\_raise\_exn} which is
            \begin{itemize}
              \item An assembly function provided by the runtime
              \item Architecture specific
            \end{itemize}
          \item Let's see what it does differently from \funcname{raise\_notrace}\dots
          \item We are about to raise \typename{ExnC} with \funcname{caml\_raise\_exn} from \funcname{definitely\_raise}
        \end{itemize}
      }
      \onslide*<2>{
        \mintobjdump[firstline=2,highlightlines=14]{../src/backtrace/camlBacktrace\_\_definitely\_raise\_347.objdump}
      }
      \vfill
      \mintocaml[firstline=9,lastline=13,highlightlines=12]{../src/backtrace/backtrace.ml}
      \endminipage
    \end{column}
    \begin{column}{0.5\textwidth}
      \centering
      \providecommand\step{1}\input{diagrams/backtrace/backtrace.tex}
    \end{column}
  \end{columns}
\end{frame}

\begin{frame}{Backtraces}{But before that, we need more fields from \typename{caml\_domain\_state}}
  \begin{columns}
    \begin{column}{0.5\textwidth}
      \begin{itemize}
        \item Remember \typename{caml\_domain\_state} the per-domain runtime state?
        \item Some other members are of interest to us now\dots
        \item \localname{backtrace\_active} is used to know if the runtime should collect backtraces
        \item \localname{backtrace\_active} can be set by
          \begin{itemize}
            \item The \funcarg{b} flag in the \funcname{OCAMLRUNPARAM} environment variable
            \item \mintinline{ocaml}{Printexc.record_backtrace : bool -> unit} \footnotemark
          \end{itemize}
      \end{itemize}
    \end{column}
    \begin{column}{0.5\textwidth}
      \begin{listing}
        \mintc[highlightlines={9-11}]{src/ocaml/domain\_state\_backtrace.h}
        \caption{runtime/caml/domain\_state.h}
      \end{listing}
    \end{column}
  \end{columns}
  \footnotetext[1]{https://v2.ocaml.org/api/Printexc.html}
\end{frame}

\newcommand\listCamlRaiseExn[1][]{
  \listasm[firstline=702,lastline=725,#1]{../ocaml/runtime/amd64.S}{runtime/amd64.S:702}}

\begin{frame}{Backtraces}{Walk through \funcname{caml\_raise\_exn}}
  \begin{columns}[T]
    \begin{column}{0.5\textwidth}
      \begin{itemize}
        \item<1-> First checks whether exceptions backtrace should be recorded
        \item<1-> By checking the \funcarg{domain\_state->backtrace\_active} value
        \item<2-> If not, acts as \funcname{raise\_notrace} with \funcname{RESTORE\_EXN\_HANDLER\_OCAML}
          \begin{itemize}
            \item Set \regname{RSP} to \localname{domain\_state->exn\_handler} value
            \item Restore \localname{domain\_state->exn\_handler} from trap
            \item Jump with \funcname{ret} (same effect as using \funcname{pop} and \funcname{jmp})
          \end{itemize}
        \item<3-> Otherwise, switches to the C stack to be able to execute C code
        \item<4-> Calls \funcname{caml\_stash\_backtrace}, a C function provided by the runtime\ignorespaces
          \onslide<6->{, with
            \begin{itemize}
              \item \funcarg{exn}: exception value
              \item \funcarg{pc}: code address of the raising point
              \item \funcarg{sp}: stack address of the raising function
              \item \funcarg{trapsp}: stack address of the next handler
            \end{itemize}
          }
      \end{itemize}
      \bigskip
      \mintocaml[firstline=9,lastline=13,highlightlines=12]{../src/backtrace/backtrace.ml}
    \end{column}
    \begin{column}{0.5\textwidth}
      \raggedleft
      \onslide*<1,3-4>{
        \mintc[firstline=190,lastline=192]{../ocaml/runtime/amd64.S}
        \mintc[firstline=234,lastline=237]{../ocaml/runtime/amd64.S}
      }
      \onslide*<2>{
        \mintc[firstline=190,lastline=192,highlightlines={191-192}]{../ocaml/runtime/amd64.S}
        \mintc[firstline=234,lastline=237,highlightlines={235-237}]{../ocaml/runtime/amd64.S}
      }
      \onslide*<1>{\listCamlRaiseExn[highlightlines=705]{}}%
      \onslide*<2>{\listCamlRaiseExn[highlightlines={707-708}]{}}%
      \onslide*<3>{\listCamlRaiseExn[highlightlines={712-714}]{}}%
      \onslide*<4>{\listCamlRaiseExn[highlightlines={715-720}]{}}%
      \onslide*<5>{\providecommand\step{1}\input{diagrams/backtrace/backtrace.tex}}%
      \onslide*<6>{\providecommand\step{2}\input{diagrams/backtrace/backtrace.tex}}%
    \end{column}
  \end{columns}
\end{frame}

\newcommand\listStashBacktrace[1][]{
  \listc[firstline=91,lastline=120,#1]{../ocaml/runtime/backtrace\_nat.c}{runtime/backtrace\_nat.c:91}}

\begin{frame}{Backtraces}{Introducing \funcname{caml\_stash\_backtrace}}
  \begin{columns}[c]
    \begin{column}{0.5\textwidth}
      \minipage[c][0.66\textheight][s]{\columnwidth}
      \begin{itemize}
        \item<1-> A C function provided by the runtime (specific to native code)
        \item<2-> It scans the stack, between the stack pointer at the raising point up to the next trap, in order to collect the names of the detected function frames
          \onslide*<2>{
            \begin{itemize}
              \item And more information, like line number, column number...
            \end{itemize}
          }
        \item<3-> It uses the same technique and information as the Garbage Collector (GC) uses to scan the values live on the stack
          \onslide*<4>{
            \begin{itemize}
              \item \typename{frame\_descr} are at the core of stack unwinding
            \end{itemize}
          }
      \end{itemize}
      \vfill
      \mintocaml[firstline=9,lastline=13,highlightlines=12]{../src/backtrace/backtrace.ml}
      \endminipage
    \end{column}
    \begin{column}{0.5\textwidth}
      \onslide*<1>{\listStashBacktrace[highlightlines=91]{}}
      \onslide*<2>{\listStashBacktrace[highlightlines={91,117-118}]{}}
      \onslide*<3->{\listStashBacktrace[highlightlines={94,106,109-110}]{}}
    \end{column}
  \end{columns}
\end{frame}

\newcommand\listFrameDescr[1][]{
  \listocaml[firstline=111,lastline=124,#1]{../ocaml/asmcomp/emitaux.ml}{asmcomp/emitaux.ml:111}}
\newcommand\listDebugInfo[1][]{
  \listocaml[firstline=38,lastline=49,#1]{../ocaml/lambda/debuginfo.mli}{lambda/debuginfo.mli:38}}

\begin{frame}{Backtraces}{\typename{frame\_descr} and \typename{frame\_debuginfo} in a nutshell}
  \begin{columns}[c]
    \begin{column}{0.5\textwidth}
      \begin{itemize}
        \item<1-> For every return address in an OCaml program, there is a associated \typename{frame\_descr}
        \item<2-> A \typename{frame\_descr} specifies
        \begin{itemize}
          \item<2-> The size of the function stack frame
          \item<3-> Where live values are on the stack (so that the GC can mark them as live and prevent from being swept)
        \end{itemize}
        \item<4-> When compiled with debug information, \typename{frame\_descr} will be linked to a \typename{frame\_debuginfo}
        \begin{itemize}
          \item<5-> Contains information about the position in the source code (file name, line and column number)
        \end{itemize}
      \end{itemize}
    \end{column}
    \begin{column}{0.5\textwidth}
        \onslide*<1>{\listFrameDescr[highlightlines={118-122}]{}}
        \onslide*<2>{\listFrameDescr[highlightlines={120}]{}}
        \onslide*<3>{\listFrameDescr[highlightlines={121}]{}}
        \onslide*<4->{\listFrameDescr[highlightlines={122}]{}}
        \medskip
        \onslide*<1-3>{\listDebugInfo{}}
        \onslide*<4>{\listDebugInfo[highlightlines={38,49}]{}}
        \onslide*<5>{\listDebugInfo[highlightlines={39-46}]{}}
    \end{column}
  \end{columns}
\end{frame}

\begin{frame}{Backtraces}{Scanning the stack with \typename{frame\_descr}}
  \begin{columns}[c]
    \begin{column}{0.5\textwidth}
      \begin{itemize}
        \item Given the return address of the code that raises the exception by calling \funcname{caml\_raise\_exn} (\funcarg{pc} argument of \funcname{caml\_stash\_backtrace})
        \item And the position of the stack pointer (\funcarg{sp} argument of \funcname{caml\_stash\_backtrace})
        \item The frame size given by \typename{frame\_descr}.\funcarg{frame\_size}, allows to find the end of the previous frame
        \item And the return address of the caller of this function
        \bigskip
        \onslide<3->{
        \item Note that traps (here the reddish \funcname{ExnA}, \funcname{ExnB} and \funcname{ExnC} blocks) belong to the frame that installed them, and so the frame size changes when traps are removed
        }
      \end{itemize}
    \end{column}
    \begin{column}{0.5\textwidth}
      \raggedleft
      \onslide*<1>{\providecommand\step{2}\input{diagrams/backtrace/backtrace.tex}}%
      \onslide*<2>{\providecommand\step{1}\input{diagrams/backtrace/scanstack.tex}}%
      \onslide*<3>{\providecommand\step{2}\input{diagrams/backtrace/scanstack.tex}}%
      \onslide*<4>{\providecommand\step{3}\input{diagrams/backtrace/scanstack.tex}}%
      \onslide*<5>{\providecommand\step{4}\input{diagrams/backtrace/scanstack.tex}}%
      \onslide*<6>{\providecommand\step{5}\input{diagrams/backtrace/scanstack.tex}}%
    \end{column}
  \end{columns}
\end{frame}

% Duplicate of previous-previous slide
\begin{frame}{Backtraces}{Continuing on \funcname{caml\_stash\_backtrace}}
  \begin{columns}[c]
    \begin{column}{0.5\textwidth}
      \minipage[c][0.75\textheight][s]{\columnwidth}
      \begin{itemize}
          \color{gray}
        \item A C function provided by the runtime (specific to native code)
        \item It scans the stack, between the stack pointer at the raising point up to the next trap, in order to collect the names of the detected function frames
        \item It uses the same technique and information as the Garbage Collector (GC) uses to scan the values live on the stack
      \end{itemize}
      \begin{itemize}
        \item For each function frame detected, store a pointer to the \typename{frame\_descr} in the \localname{domain\_state->backtrace\_buffer} array, effectively constructing the backtrace
      \end{itemize}
      \onslide<2->{
        \begin{itemize}
            \smallskip
          \item Constructed backtrace:
            \funcname{definitely\_raise},
            \onslide<3->{\funcname{probably\_raise}}
        \end{itemize}
      }
      \vfill
      \mintocaml[firstline=9,lastline=13,highlightlines=12]{../src/backtrace/backtrace.ml}
      \endminipage
    \end{column}
    \begin{column}{0.5\textwidth}
      \raggedleft
      \onslide*<1>{\listStashBacktrace[highlightlines={114-115}]{}}
      \onslide*<2>{\providecommand\step{3}\input{diagrams/backtrace/backtrace.tex}}%
      \onslide*<3>{\providecommand\step{4}\input{diagrams/backtrace/backtrace.tex}}%
    \end{column}
  \end{columns}
\end{frame}

\begin{frame}{Backtraces}{Back to \funcname{caml\_raise\_exn}}
  \begin{columns}[c]
    \begin{column}{0.5\textwidth}
      \minipage[c][0.66\textheight][s]{\columnwidth}
      \begin{itemize}\color{gray}
        \item Checks wheteher exceptions backtrace should be recorded, by checking the \funcarg{domain\_state->backtrace\_active} value
        \item Switches to the C stack to be able to execute C code
        \item Calls \funcname{caml\_stash\_backtrace}, to collect the backtrace up to the next trap
      \end{itemize}
      \onslide*<2->{
        \begin{itemize}
          \item<2-> Executes the exception handler in \funcname{probably\_raise} with \funcname{RESTORE\_EXN\_HANDLER\_OCAML}
            \begin{itemize}
              \item Set \regname{RSP} to \localname{domain\_state->exn\_handler} value
              \item Restore \localname{domain\_state->exn\_handler} from trap
              \item Jump to the handler code with \funcname{ret}
            \end{itemize}
        \end{itemize}
      }
      \vfill
      \onslide*<1>{\mintocaml[firstline=9,lastline=13,highlightlines=12]{../src/backtrace/backtrace.ml}}
      \onslide*<2->{\mintocaml[firstline=15,lastline=21,highlightlines={19-21}]{../src/backtrace/backtrace.ml}}
      \endminipage
    \end{column}
    \begin{column}{0.5\textwidth}
      \centering
      \onslide*<1>{
        \mintc[firstline=190,lastline=192]{../ocaml/runtime/amd64.S}
        \mintc[firstline=234,lastline=237]{../ocaml/runtime/amd64.S}
      }
      \onslide*<2>{
        \mintc[firstline=190,lastline=192,highlightlines={191-192}]{../ocaml/runtime/amd64.S}
        \mintc[firstline=234,lastline=237,highlightlines={235-237}]{../ocaml/runtime/amd64.S}
      }
      \onslide*<1>{\listCamlRaiseExn[highlightlines=720]{}}
      \onslide*<2>{\listCamlRaiseExn[highlightlines=722]{}}
      \onslide*<3->{\providecommand\step{5}\input{diagrams/backtrace/backtrace.tex}}
    \end{column}
  \end{columns}
\end{frame}

\begin{frame}{Backtraces}{\funcname{caml\_probably\_raise}'s handler doesn't catch \typename{ExnC}}
  \begin{columns}[c]
    \begin{column}{0.45\textwidth}
      \minipage[c][0.75\textheight][s]{\columnwidth}
      \begin{itemize}
        \item<1-> \funcname{match \dots with}\footnotemark pattern matching can also match exceptions
        \item<1-> The CMM of \funcname{probably\_raise} shows that this \funcname{match \dots with} is implemented with a pattern matching and a \funcname{try \dots with}
        \item<1-> But this one only matches on \typename{ExnA} exception
        \item<2-> Since the pattern matching fails to catch the exception, \funcname{reraise} is called with our current \typename{ExnC} exception
        \item<3-> Disassembly shows that \funcname{reraise} is actually a call to \funcname{caml\_reraise\_exn}, which is also an assembly function provided by the runtime
      \end{itemize}
      \vfill
      \mintocaml[firstline=15,lastline=21,highlightlines={18-21}]{../src/backtrace/backtrace.ml}
      \endminipage
    \end{column}
    \begin{column}{0.55\textwidth}
      \centering
      \onslide*<1>{\mintcmm[highlightlines={10-18}]{../src/backtrace/camlBacktrace\_\_probably\_raise\_351.cmm}}
      \onslide*<2>{\listcmm[highlightlines=18]{../src/backtrace/camlBacktrace\_\_probably\_raise_351.cmm}{CMM of probably\_raise}}
      \onslide*<3>{\mintobjdump[firstline=5,lastline=35,highlightlines=31]{../src/backtrace/camlBacktrace\_\_probably\_raise_351.objdump}}
    \end{column}
  \end{columns}
  \only<1>{\footnotetext[1]{https://blog.janestreet.com/pattern-matching-and-exception-handling-unite/}}
\end{frame}

\newcommand\listCamlReraiseExn[1][]{
  \listasm[firstline=727,lastline=734,#1]{../ocaml/runtime/amd64.S}{runtime/amd64.S:727}}

\begin{frame}{Backtraces}{Introducing \funcname{caml\_reraise\_exn}}
  \begin{columns}[c]
    \begin{column}{0.5\textwidth}
      \minipage[c][0.66\textheight][s]{\columnwidth}
      \begin{itemize}
        \item<1-> The exception have to be reraised to the next handler
        \item<2-> First, checks if the backtrace should be collected with \funcarg{domain\_state->backtrace\_active}
        \only<3>{
        \begin{itemize}
            \item If not, behaves like a \funcname{raise\_notrace} with \funcname{RESTORE\_EXN\_HANDLER\_OCAML}
        \end{itemize}
        }
        \item<4-> Jumps into \funcname{caml\_raise\_exn}
        \item<5-> Calls \funcname{caml\_stash\_backtrace} again, to scan the next part of the stack: from the trap just pop-ed to the next trap
      \end{itemize}
      \vfill
      \mintocaml[firstline=15,lastline=21,highlightlines={18-21}]{../src/backtrace/backtrace.ml}
      \endminipage
    \end{column}
    \begin{column}{0.5\textwidth}
      \raggedleft
      \onslide*<1>{\listCamlReraiseExn{}}
      \onslide*<2>{\listCamlReraiseExn[highlightlines=729]{}}
      \onslide*<3>{
        \mintc[firstline=190,lastline=192,highlightlines={191-192}]{../ocaml/runtime/amd64.S}
        \mintc[firstline=234,lastline=237,highlightlines={235-237}]{../ocaml/runtime/amd64.S}
        \medskip
        \listCamlReraiseExn[highlightlines=731]{}
      }
      \onslide*<4-5>{
        % TODO refactor with \listCamlReraiseExn
        \mintasm[firstline=727,lastline=734,highlightlines=730]{../ocaml/runtime/amd64.S}
        \medskip
      }
      \onslide*<4>{\listCamlRaiseExn[highlightlines=711]{}}
      \onslide*<5>{\listCamlRaiseExn[highlightlines=720]{}}
    \end{column}
  \end{columns}
\end{frame}

\begin{frame}{Backtraces}{Backtrace collection continues}
  \begin{columns}[c]
    \begin{column}{0.55\textwidth}
      \minipage[c][0.75\textheight][s]{\columnwidth}
      \begin{itemize}
        \item<1-> \funcname{caml\_stash\_backtrace} scans the older part of the stack, up to the next trap
        \item<6-> Controls jumps to the nested exception handler in \funcname{maybe\_raise}, which matches only on \typename{ExnB}
        \item<7-> \funcname{caml\_reraise\_exn} is called again, backtrace collection resumes with \funcname{caml\_stash\_backtrace}
      \end{itemize}
      \medskip
      \begin{itemize}
        \item<2-> Constructed backtrace:
        \begin{itemize}
          \item <2-> \funcname{definitely\_raise}
          \item <2-> \funcname{probably\_raise}
          \item <3-> \funcname{probably\_raise} (re-raise)
          \item <4-> \funcname{maybe\_raise}
          \item <5-> \funcname{main}
          \item <8-> \funcname{main} (re-raise)
        \end{itemize}
      \end{itemize}
      \vfill
      \onslide*<1-5>{\mintocaml[firstline=15,lastline=21]{../src/backtrace/backtrace.ml}}
      \onslide*<6>{\mintocaml[firstline=28,lastline=34,highlightlines=31]{../src/backtrace/backtrace.ml}}
      \onslide*<7->{\mintocaml[firstline=28,lastline=34,highlightlines=33]{../src/backtrace/backtrace.ml}}
      \endminipage
    \end{column}
    \begin{column}{0.45\textwidth}
      \raggedleft
      \onslide*<1>{\providecommand\step{6}\input{diagrams/backtrace/backtrace.tex}}%
      \onslide*<2>{\providecommand\step{6}\input{diagrams/backtrace/backtrace.tex}}%
      \onslide*<3>{\providecommand\step{7}\input{diagrams/backtrace/backtrace.tex}}%
      \onslide*<4>{\providecommand\step{8}\input{diagrams/backtrace/backtrace.tex}}%
      \onslide*<5>{\providecommand\step{9}\input{diagrams/backtrace/backtrace.tex}}%
      \onslide*<6>{\providecommand\step{10}\input{diagrams/backtrace/backtrace.tex}}%
      \onslide*<7>{\providecommand\step{11}\input{diagrams/backtrace/backtrace.tex}}%
      \onslide*<8>{\providecommand\step{12}\input{diagrams/backtrace/backtrace.tex}}%
      \onslide*<9>{\providecommand\step{13}\input{diagrams/backtrace/backtrace.tex}}%
    \end{column}
  \end{columns}
\end{frame}

\begin{frame}{Backtraces}{Printing the backtrace}
  \begin{columns}[c]
    \begin{column}{0.45\textwidth}
      \minipage[c][0.66\textheight][s]{\columnwidth}
      \begin{itemize}
        \item<1-> The exception backtrace can now be printed to \funcarg{stdout} with \funcname{Printexc.print_backtrace}
        \item<2-> First the exception backtrace is retrieved into an OCaml array with \funcname{get_raw_backtrace }
        \item<3-> Which is implemented as the \funcname{caml_get_exception_raw_backtrace} C function in the runtime
        \item<4-> And finally printed with \funcname{print_exception_backtrace}
      \end{itemize}
      \vfill
      \mintocaml[firstline=28,lastline=34,highlightlines=33]{../src/backtrace/backtrace.ml}
      \endminipage
    \end{column}
    \begin{column}{0.55\textwidth}
      \onslide*<1,2>{
        \mintocaml[firstline=100,lastline=101]{../ocaml/stdlib/printexc.ml}
      }
      \only<1>{
        \listocaml[firstline=170,lastline=172]{../ocaml/stdlib/printexc.ml}{stdlib/printexc.ml:170}
      }
      \only<2>{
        \listocaml[firstline=170,lastline=172,highlightlines=172]{../ocaml/stdlib/printexc.ml}{stdlib/printexc.ml:170}
      }
      \only<3>{
        \mintc[firstline=154,lastline=156]{../ocaml/runtime/backtrace.c}
        \listc[firstline=171,lastline=192,highlightlines={184-187}]{../ocaml/runtime/backtrace.c}{runtime/backtrace.c:154}
      }
      \only<4>{
        \listocaml[firstline=155,lastline=165]{../ocaml/stdlib/printexc.ml}{stdlib/printexc.ml:55}
      }
    \end{column}
  \end{columns}
\end{frame}


\subsection{The multiple kinds of raise}
\frameSubsection{}{}

\begin{frame}{Backtraces}{When backtraces are not needed}
  \begin{columns}[c]
    \begin{column}{0.45\textwidth}
      \minipage[c][0.66\textheight][s]{\columnwidth}
      \begin{itemize}
        \item<1-> What if the backtrace for an exception is never needed?
        \item<2-> It's possible to raise the exception explicitly with \funcname{raise\_notrace} to make raising as cheap as possible
        \item<3-> An example of use in the \funcname{dynlink} library of the OCaml compiler
      \end{itemize}
      \vfill
      \onslide<3->{
      \listocaml[firstline=205,lastline=209,highlightlines=207]{../ocaml/otherlibs/dynlink/dynlink\_compilerlibs/misc.ml}{otherlibs/dynlink/dynlink\_compilerlibs/misc.ml:205}
      }
      \endminipage
    \end{column}
    \begin{column}{0.55\textwidth}
    \only<4>{
      \mintcmm[highlightlines=3]{../src/notrace/camlNotrace\_\_fun_347.cmm}
      \listcmm{../src/notrace/camlNotrace\_\_all\_somes\_327.cmm}{all\_somes}
    }
    \onslide*<5->{
      \listobjdump[highlightlines={6-9}]{../src/notrace/camlNotrace\_\_fun_347.objdump}{anonymous function from \funcname{all\_somes}}
    }
    \end{column}
  \end{columns}
\end{frame}

\begin{frame}{Backtraces}{Summary table of the multiple kinds of raise}
  \begin{itemize}
    \item When debugging information is enabled and if the backtrace of an exception isn't needed, it's possible to raise with \funcname{raise\_notrace} for performance
    \item When debugging information is \emph{not} enabled, the compiler optimises \funcname{raise} calls to inline assembly (same effect as using \funcname{raise\_notrace})
  \end{itemize}
  \bigskip
  \centering
  \begin{tabular}{| l | c | c | c | c |}
    \hline
    \parbox{10em}{Debug information \\ (\funcarg{-g} flag)}  & \multicolumn{2}{|c|}{Disabled}                                     & \multicolumn{2}{|c|}{Enabled} \\
    \hline
    Raise kind                                               & \funcname{raise} & \funcname{raise\_notrace}                       & \funcname{raise} & \funcname{raise\_notrace} \\
    \hline
    Compilation                                              & \parbox{8em}{inline assembly\\ (optimization)} & inline assembly   & \funcname{caml\_raise\_exn} & inline assembly \\
    \hline
  \end{tabular}
\end{frame}


%
% Takeaway #5
%
\subsection*{Takeaway \#5}
\frameSubsectionTakeaway{}
\againframe<5>{takeaway}
}%
    \end{column}
  \end{columns}
\end{frame}

\newcommand\listStashBacktrace[1][]{
  \listc[firstline=91,lastline=120,#1]{../ocaml/runtime/backtrace\_nat.c}{runtime/backtrace\_nat.c:91}}

\begin{frame}{Backtraces}{Introducing \funcname{caml\_stash\_backtrace}}
  \begin{columns}[c]
    \begin{column}{0.5\textwidth}
      \minipage[c][0.66\textheight][s]{\columnwidth}
      \begin{itemize}
        \item<1-> A C function provided by the runtime (specific to native code)
        \item<2-> It scans the stack, between the stack pointer at the raising point up to the next trap, in order to collect the names of the detected function frames
          \onslide*<2>{
            \begin{itemize}
              \item And more information, like line number, column number...
            \end{itemize}
          }
        \item<3-> It uses the same technique and information as the Garbage Collector (GC) uses to scan the values live on the stack
          \onslide*<4>{
            \begin{itemize}
              \item \typename{frame\_descr} are at the core of stack unwinding
            \end{itemize}
          }
      \end{itemize}
      \vfill
      \mintocaml[firstline=9,lastline=13,highlightlines=12]{../src/backtrace/backtrace.ml}
      \endminipage
    \end{column}
    \begin{column}{0.5\textwidth}
      \onslide*<1>{\listStashBacktrace[highlightlines=91]{}}
      \onslide*<2>{\listStashBacktrace[highlightlines={91,117-118}]{}}
      \onslide*<3->{\listStashBacktrace[highlightlines={94,106,109-110}]{}}
    \end{column}
  \end{columns}
\end{frame}

\newcommand\listFrameDescr[1][]{
  \listocaml[firstline=111,lastline=124,#1]{../ocaml/asmcomp/emitaux.ml}{asmcomp/emitaux.ml:111}}
\newcommand\listDebugInfo[1][]{
  \listocaml[firstline=38,lastline=49,#1]{../ocaml/lambda/debuginfo.mli}{lambda/debuginfo.mli:38}}

\begin{frame}{Backtraces}{\typename{frame\_descr} and \typename{frame\_debuginfo} in a nutshell}
  \begin{columns}[c]
    \begin{column}{0.5\textwidth}
      \begin{itemize}
        \item<1-> For every return address in an OCaml program, there is a associated \typename{frame\_descr}
        \item<2-> A \typename{frame\_descr} specifies
        \begin{itemize}
          \item<2-> The size of the function stack frame
          \item<3-> Where live values are on the stack (so that the GC can mark them as live and prevent from being swept)
        \end{itemize}
        \item<4-> When compiled with debug information, \typename{frame\_descr} will be linked to a \typename{frame\_debuginfo}
        \begin{itemize}
          \item<5-> Contains information about the position in the source code (file name, line and column number)
        \end{itemize}
      \end{itemize}
    \end{column}
    \begin{column}{0.5\textwidth}
        \onslide*<1>{\listFrameDescr[highlightlines={118-122}]{}}
        \onslide*<2>{\listFrameDescr[highlightlines={120}]{}}
        \onslide*<3>{\listFrameDescr[highlightlines={121}]{}}
        \onslide*<4->{\listFrameDescr[highlightlines={122}]{}}
        \medskip
        \onslide*<1-3>{\listDebugInfo{}}
        \onslide*<4>{\listDebugInfo[highlightlines={38,49}]{}}
        \onslide*<5>{\listDebugInfo[highlightlines={39-46}]{}}
    \end{column}
  \end{columns}
\end{frame}

\begin{frame}{Backtraces}{Scanning the stack with \typename{frame\_descr}}
  \begin{columns}[c]
    \begin{column}{0.5\textwidth}
      \begin{itemize}
        \item Given the return address of the code that raises the exception by calling \funcname{caml\_raise\_exn} (\funcarg{pc} argument of \funcname{caml\_stash\_backtrace})
        \item And the position of the stack pointer (\funcarg{sp} argument of \funcname{caml\_stash\_backtrace})
        \item The frame size given by \typename{frame\_descr}.\funcarg{frame\_size}, allows to find the end of the previous frame
        \item And the return address of the caller of this function
        \bigskip
        \onslide<3->{
        \item Note that traps (here the reddish \funcname{ExnA}, \funcname{ExnB} and \funcname{ExnC} blocks) belong to the frame that installed them, and so the frame size changes when traps are removed
        }
      \end{itemize}
    \end{column}
    \begin{column}{0.5\textwidth}
      \raggedleft
      \onslide*<1>{\providecommand\step{2}%
% Backtraces
%
\subsection{One advantage of raising with debugging information}
\frameSubsection{}{}

\begin{frame}{Backtraces}{Debugging information \& source code}
  \begin{columns}[c]
    \begin{column}{0.5\textwidth}
      \begin{itemize}
        \item Raising an exception is fun and games, but how do we know where the exception originated? $\rightarrow$ With the backtrace associated with the exception!
        \item In order to get a backtrace from the point where an exception was raised, it's necessary to compile with debugging information enabled (\funcarg{-g} flag for the compiler)
        \item Same example as in the `Nested handlers' section
      \end{itemize}
    \end{column}
    \begin{column}{0.5\textwidth}
      \listocaml[firstline=9,lastline=34]{../src/backtrace/backtrace.ml}{backtrace/backtrace.ml}
    \end{column}
  \end{columns}
\end{frame}

\begin{frame}{Backtraces}{CMM and disassembly of \funcname{definitely\_raise}}
  \begin{columns}[c]
    \begin{column}{0.5\textwidth}
      \minipage[c][0.66\textheight][s]{\columnwidth}
      \begin{itemize}
        \item<1-> An effect of compiling with debugging information (\funcarg{-g}): the CMM no longer uses \funcname{raise\_notrace} to raise the exception but \funcname{raise} instead
        \item<2-> Disassembly shows that CMM \funcname{raise} is actually a call to \funcname{caml\_raise\_exn}, a function in assembler provided by the runtime
      \end{itemize}
      \vfill
      \mintocaml[firstline=9,lastline=13,highlightlines=12]{../src/backtrace/backtrace.ml}
      \endminipage
    \end{column}
    \begin{column}{0.5\textwidth}
      \onslide*<1>{
        \mintcmm[highlightlines=5]{../src/backtrace/camlBacktrace\_\_definitely\_raise\_347.cmm}
      }
      \onslide*<2>{
        \mintobjdump[firstline=2,highlightlines=14]{../src/backtrace/camlBacktrace\_\_definitely\_raise\_347.objdump}
      }
    \end{column}
  \end{columns}
\end{frame}

\begin{frame}{Backtraces}{Introducing \funcname{caml\_raise\_exn}}
  \begin{columns}[c]
    \begin{column}{0.5\textwidth}
      \minipage[c][0.66\textheight][s]{\columnwidth}
      \onslide*<1>{
        \begin{itemize}
          \item Raising the exception is performed using \funcname{caml\_raise\_exn} which is
            \begin{itemize}
              \item An assembly function provided by the runtime
              \item Architecture specific
            \end{itemize}
          \item Let's see what it does differently from \funcname{raise\_notrace}\dots
          \item We are about to raise \typename{ExnC} with \funcname{caml\_raise\_exn} from \funcname{definitely\_raise}
        \end{itemize}
      }
      \onslide*<2>{
        \mintobjdump[firstline=2,highlightlines=14]{../src/backtrace/camlBacktrace\_\_definitely\_raise\_347.objdump}
      }
      \vfill
      \mintocaml[firstline=9,lastline=13,highlightlines=12]{../src/backtrace/backtrace.ml}
      \endminipage
    \end{column}
    \begin{column}{0.5\textwidth}
      \centering
      \providecommand\step{1}\input{diagrams/backtrace/backtrace.tex}
    \end{column}
  \end{columns}
\end{frame}

\begin{frame}{Backtraces}{But before that, we need more fields from \typename{caml\_domain\_state}}
  \begin{columns}
    \begin{column}{0.5\textwidth}
      \begin{itemize}
        \item Remember \typename{caml\_domain\_state} the per-domain runtime state?
        \item Some other members are of interest to us now\dots
        \item \localname{backtrace\_active} is used to know if the runtime should collect backtraces
        \item \localname{backtrace\_active} can be set by
          \begin{itemize}
            \item The \funcarg{b} flag in the \funcname{OCAMLRUNPARAM} environment variable
            \item \mintinline{ocaml}{Printexc.record_backtrace : bool -> unit} \footnotemark
          \end{itemize}
      \end{itemize}
    \end{column}
    \begin{column}{0.5\textwidth}
      \begin{listing}
        \mintc[highlightlines={9-11}]{src/ocaml/domain\_state\_backtrace.h}
        \caption{runtime/caml/domain\_state.h}
      \end{listing}
    \end{column}
  \end{columns}
  \footnotetext[1]{https://v2.ocaml.org/api/Printexc.html}
\end{frame}

\newcommand\listCamlRaiseExn[1][]{
  \listasm[firstline=702,lastline=725,#1]{../ocaml/runtime/amd64.S}{runtime/amd64.S:702}}

\begin{frame}{Backtraces}{Walk through \funcname{caml\_raise\_exn}}
  \begin{columns}[T]
    \begin{column}{0.5\textwidth}
      \begin{itemize}
        \item<1-> First checks whether exceptions backtrace should be recorded
        \item<1-> By checking the \funcarg{domain\_state->backtrace\_active} value
        \item<2-> If not, acts as \funcname{raise\_notrace} with \funcname{RESTORE\_EXN\_HANDLER\_OCAML}
          \begin{itemize}
            \item Set \regname{RSP} to \localname{domain\_state->exn\_handler} value
            \item Restore \localname{domain\_state->exn\_handler} from trap
            \item Jump with \funcname{ret} (same effect as using \funcname{pop} and \funcname{jmp})
          \end{itemize}
        \item<3-> Otherwise, switches to the C stack to be able to execute C code
        \item<4-> Calls \funcname{caml\_stash\_backtrace}, a C function provided by the runtime\ignorespaces
          \onslide<6->{, with
            \begin{itemize}
              \item \funcarg{exn}: exception value
              \item \funcarg{pc}: code address of the raising point
              \item \funcarg{sp}: stack address of the raising function
              \item \funcarg{trapsp}: stack address of the next handler
            \end{itemize}
          }
      \end{itemize}
      \bigskip
      \mintocaml[firstline=9,lastline=13,highlightlines=12]{../src/backtrace/backtrace.ml}
    \end{column}
    \begin{column}{0.5\textwidth}
      \raggedleft
      \onslide*<1,3-4>{
        \mintc[firstline=190,lastline=192]{../ocaml/runtime/amd64.S}
        \mintc[firstline=234,lastline=237]{../ocaml/runtime/amd64.S}
      }
      \onslide*<2>{
        \mintc[firstline=190,lastline=192,highlightlines={191-192}]{../ocaml/runtime/amd64.S}
        \mintc[firstline=234,lastline=237,highlightlines={235-237}]{../ocaml/runtime/amd64.S}
      }
      \onslide*<1>{\listCamlRaiseExn[highlightlines=705]{}}%
      \onslide*<2>{\listCamlRaiseExn[highlightlines={707-708}]{}}%
      \onslide*<3>{\listCamlRaiseExn[highlightlines={712-714}]{}}%
      \onslide*<4>{\listCamlRaiseExn[highlightlines={715-720}]{}}%
      \onslide*<5>{\providecommand\step{1}\input{diagrams/backtrace/backtrace.tex}}%
      \onslide*<6>{\providecommand\step{2}\input{diagrams/backtrace/backtrace.tex}}%
    \end{column}
  \end{columns}
\end{frame}

\newcommand\listStashBacktrace[1][]{
  \listc[firstline=91,lastline=120,#1]{../ocaml/runtime/backtrace\_nat.c}{runtime/backtrace\_nat.c:91}}

\begin{frame}{Backtraces}{Introducing \funcname{caml\_stash\_backtrace}}
  \begin{columns}[c]
    \begin{column}{0.5\textwidth}
      \minipage[c][0.66\textheight][s]{\columnwidth}
      \begin{itemize}
        \item<1-> A C function provided by the runtime (specific to native code)
        \item<2-> It scans the stack, between the stack pointer at the raising point up to the next trap, in order to collect the names of the detected function frames
          \onslide*<2>{
            \begin{itemize}
              \item And more information, like line number, column number...
            \end{itemize}
          }
        \item<3-> It uses the same technique and information as the Garbage Collector (GC) uses to scan the values live on the stack
          \onslide*<4>{
            \begin{itemize}
              \item \typename{frame\_descr} are at the core of stack unwinding
            \end{itemize}
          }
      \end{itemize}
      \vfill
      \mintocaml[firstline=9,lastline=13,highlightlines=12]{../src/backtrace/backtrace.ml}
      \endminipage
    \end{column}
    \begin{column}{0.5\textwidth}
      \onslide*<1>{\listStashBacktrace[highlightlines=91]{}}
      \onslide*<2>{\listStashBacktrace[highlightlines={91,117-118}]{}}
      \onslide*<3->{\listStashBacktrace[highlightlines={94,106,109-110}]{}}
    \end{column}
  \end{columns}
\end{frame}

\newcommand\listFrameDescr[1][]{
  \listocaml[firstline=111,lastline=124,#1]{../ocaml/asmcomp/emitaux.ml}{asmcomp/emitaux.ml:111}}
\newcommand\listDebugInfo[1][]{
  \listocaml[firstline=38,lastline=49,#1]{../ocaml/lambda/debuginfo.mli}{lambda/debuginfo.mli:38}}

\begin{frame}{Backtraces}{\typename{frame\_descr} and \typename{frame\_debuginfo} in a nutshell}
  \begin{columns}[c]
    \begin{column}{0.5\textwidth}
      \begin{itemize}
        \item<1-> For every return address in an OCaml program, there is a associated \typename{frame\_descr}
        \item<2-> A \typename{frame\_descr} specifies
        \begin{itemize}
          \item<2-> The size of the function stack frame
          \item<3-> Where live values are on the stack (so that the GC can mark them as live and prevent from being swept)
        \end{itemize}
        \item<4-> When compiled with debug information, \typename{frame\_descr} will be linked to a \typename{frame\_debuginfo}
        \begin{itemize}
          \item<5-> Contains information about the position in the source code (file name, line and column number)
        \end{itemize}
      \end{itemize}
    \end{column}
    \begin{column}{0.5\textwidth}
        \onslide*<1>{\listFrameDescr[highlightlines={118-122}]{}}
        \onslide*<2>{\listFrameDescr[highlightlines={120}]{}}
        \onslide*<3>{\listFrameDescr[highlightlines={121}]{}}
        \onslide*<4->{\listFrameDescr[highlightlines={122}]{}}
        \medskip
        \onslide*<1-3>{\listDebugInfo{}}
        \onslide*<4>{\listDebugInfo[highlightlines={38,49}]{}}
        \onslide*<5>{\listDebugInfo[highlightlines={39-46}]{}}
    \end{column}
  \end{columns}
\end{frame}

\begin{frame}{Backtraces}{Scanning the stack with \typename{frame\_descr}}
  \begin{columns}[c]
    \begin{column}{0.5\textwidth}
      \begin{itemize}
        \item Given the return address of the code that raises the exception by calling \funcname{caml\_raise\_exn} (\funcarg{pc} argument of \funcname{caml\_stash\_backtrace})
        \item And the position of the stack pointer (\funcarg{sp} argument of \funcname{caml\_stash\_backtrace})
        \item The frame size given by \typename{frame\_descr}.\funcarg{frame\_size}, allows to find the end of the previous frame
        \item And the return address of the caller of this function
        \bigskip
        \onslide<3->{
        \item Note that traps (here the reddish \funcname{ExnA}, \funcname{ExnB} and \funcname{ExnC} blocks) belong to the frame that installed them, and so the frame size changes when traps are removed
        }
      \end{itemize}
    \end{column}
    \begin{column}{0.5\textwidth}
      \raggedleft
      \onslide*<1>{\providecommand\step{2}\input{diagrams/backtrace/backtrace.tex}}%
      \onslide*<2>{\providecommand\step{1}\input{diagrams/backtrace/scanstack.tex}}%
      \onslide*<3>{\providecommand\step{2}\input{diagrams/backtrace/scanstack.tex}}%
      \onslide*<4>{\providecommand\step{3}\input{diagrams/backtrace/scanstack.tex}}%
      \onslide*<5>{\providecommand\step{4}\input{diagrams/backtrace/scanstack.tex}}%
      \onslide*<6>{\providecommand\step{5}\input{diagrams/backtrace/scanstack.tex}}%
    \end{column}
  \end{columns}
\end{frame}

% Duplicate of previous-previous slide
\begin{frame}{Backtraces}{Continuing on \funcname{caml\_stash\_backtrace}}
  \begin{columns}[c]
    \begin{column}{0.5\textwidth}
      \minipage[c][0.75\textheight][s]{\columnwidth}
      \begin{itemize}
          \color{gray}
        \item A C function provided by the runtime (specific to native code)
        \item It scans the stack, between the stack pointer at the raising point up to the next trap, in order to collect the names of the detected function frames
        \item It uses the same technique and information as the Garbage Collector (GC) uses to scan the values live on the stack
      \end{itemize}
      \begin{itemize}
        \item For each function frame detected, store a pointer to the \typename{frame\_descr} in the \localname{domain\_state->backtrace\_buffer} array, effectively constructing the backtrace
      \end{itemize}
      \onslide<2->{
        \begin{itemize}
            \smallskip
          \item Constructed backtrace:
            \funcname{definitely\_raise},
            \onslide<3->{\funcname{probably\_raise}}
        \end{itemize}
      }
      \vfill
      \mintocaml[firstline=9,lastline=13,highlightlines=12]{../src/backtrace/backtrace.ml}
      \endminipage
    \end{column}
    \begin{column}{0.5\textwidth}
      \raggedleft
      \onslide*<1>{\listStashBacktrace[highlightlines={114-115}]{}}
      \onslide*<2>{\providecommand\step{3}\input{diagrams/backtrace/backtrace.tex}}%
      \onslide*<3>{\providecommand\step{4}\input{diagrams/backtrace/backtrace.tex}}%
    \end{column}
  \end{columns}
\end{frame}

\begin{frame}{Backtraces}{Back to \funcname{caml\_raise\_exn}}
  \begin{columns}[c]
    \begin{column}{0.5\textwidth}
      \minipage[c][0.66\textheight][s]{\columnwidth}
      \begin{itemize}\color{gray}
        \item Checks wheteher exceptions backtrace should be recorded, by checking the \funcarg{domain\_state->backtrace\_active} value
        \item Switches to the C stack to be able to execute C code
        \item Calls \funcname{caml\_stash\_backtrace}, to collect the backtrace up to the next trap
      \end{itemize}
      \onslide*<2->{
        \begin{itemize}
          \item<2-> Executes the exception handler in \funcname{probably\_raise} with \funcname{RESTORE\_EXN\_HANDLER\_OCAML}
            \begin{itemize}
              \item Set \regname{RSP} to \localname{domain\_state->exn\_handler} value
              \item Restore \localname{domain\_state->exn\_handler} from trap
              \item Jump to the handler code with \funcname{ret}
            \end{itemize}
        \end{itemize}
      }
      \vfill
      \onslide*<1>{\mintocaml[firstline=9,lastline=13,highlightlines=12]{../src/backtrace/backtrace.ml}}
      \onslide*<2->{\mintocaml[firstline=15,lastline=21,highlightlines={19-21}]{../src/backtrace/backtrace.ml}}
      \endminipage
    \end{column}
    \begin{column}{0.5\textwidth}
      \centering
      \onslide*<1>{
        \mintc[firstline=190,lastline=192]{../ocaml/runtime/amd64.S}
        \mintc[firstline=234,lastline=237]{../ocaml/runtime/amd64.S}
      }
      \onslide*<2>{
        \mintc[firstline=190,lastline=192,highlightlines={191-192}]{../ocaml/runtime/amd64.S}
        \mintc[firstline=234,lastline=237,highlightlines={235-237}]{../ocaml/runtime/amd64.S}
      }
      \onslide*<1>{\listCamlRaiseExn[highlightlines=720]{}}
      \onslide*<2>{\listCamlRaiseExn[highlightlines=722]{}}
      \onslide*<3->{\providecommand\step{5}\input{diagrams/backtrace/backtrace.tex}}
    \end{column}
  \end{columns}
\end{frame}

\begin{frame}{Backtraces}{\funcname{caml\_probably\_raise}'s handler doesn't catch \typename{ExnC}}
  \begin{columns}[c]
    \begin{column}{0.45\textwidth}
      \minipage[c][0.75\textheight][s]{\columnwidth}
      \begin{itemize}
        \item<1-> \funcname{match \dots with}\footnotemark pattern matching can also match exceptions
        \item<1-> The CMM of \funcname{probably\_raise} shows that this \funcname{match \dots with} is implemented with a pattern matching and a \funcname{try \dots with}
        \item<1-> But this one only matches on \typename{ExnA} exception
        \item<2-> Since the pattern matching fails to catch the exception, \funcname{reraise} is called with our current \typename{ExnC} exception
        \item<3-> Disassembly shows that \funcname{reraise} is actually a call to \funcname{caml\_reraise\_exn}, which is also an assembly function provided by the runtime
      \end{itemize}
      \vfill
      \mintocaml[firstline=15,lastline=21,highlightlines={18-21}]{../src/backtrace/backtrace.ml}
      \endminipage
    \end{column}
    \begin{column}{0.55\textwidth}
      \centering
      \onslide*<1>{\mintcmm[highlightlines={10-18}]{../src/backtrace/camlBacktrace\_\_probably\_raise\_351.cmm}}
      \onslide*<2>{\listcmm[highlightlines=18]{../src/backtrace/camlBacktrace\_\_probably\_raise_351.cmm}{CMM of probably\_raise}}
      \onslide*<3>{\mintobjdump[firstline=5,lastline=35,highlightlines=31]{../src/backtrace/camlBacktrace\_\_probably\_raise_351.objdump}}
    \end{column}
  \end{columns}
  \only<1>{\footnotetext[1]{https://blog.janestreet.com/pattern-matching-and-exception-handling-unite/}}
\end{frame}

\newcommand\listCamlReraiseExn[1][]{
  \listasm[firstline=727,lastline=734,#1]{../ocaml/runtime/amd64.S}{runtime/amd64.S:727}}

\begin{frame}{Backtraces}{Introducing \funcname{caml\_reraise\_exn}}
  \begin{columns}[c]
    \begin{column}{0.5\textwidth}
      \minipage[c][0.66\textheight][s]{\columnwidth}
      \begin{itemize}
        \item<1-> The exception have to be reraised to the next handler
        \item<2-> First, checks if the backtrace should be collected with \funcarg{domain\_state->backtrace\_active}
        \only<3>{
        \begin{itemize}
            \item If not, behaves like a \funcname{raise\_notrace} with \funcname{RESTORE\_EXN\_HANDLER\_OCAML}
        \end{itemize}
        }
        \item<4-> Jumps into \funcname{caml\_raise\_exn}
        \item<5-> Calls \funcname{caml\_stash\_backtrace} again, to scan the next part of the stack: from the trap just pop-ed to the next trap
      \end{itemize}
      \vfill
      \mintocaml[firstline=15,lastline=21,highlightlines={18-21}]{../src/backtrace/backtrace.ml}
      \endminipage
    \end{column}
    \begin{column}{0.5\textwidth}
      \raggedleft
      \onslide*<1>{\listCamlReraiseExn{}}
      \onslide*<2>{\listCamlReraiseExn[highlightlines=729]{}}
      \onslide*<3>{
        \mintc[firstline=190,lastline=192,highlightlines={191-192}]{../ocaml/runtime/amd64.S}
        \mintc[firstline=234,lastline=237,highlightlines={235-237}]{../ocaml/runtime/amd64.S}
        \medskip
        \listCamlReraiseExn[highlightlines=731]{}
      }
      \onslide*<4-5>{
        % TODO refactor with \listCamlReraiseExn
        \mintasm[firstline=727,lastline=734,highlightlines=730]{../ocaml/runtime/amd64.S}
        \medskip
      }
      \onslide*<4>{\listCamlRaiseExn[highlightlines=711]{}}
      \onslide*<5>{\listCamlRaiseExn[highlightlines=720]{}}
    \end{column}
  \end{columns}
\end{frame}

\begin{frame}{Backtraces}{Backtrace collection continues}
  \begin{columns}[c]
    \begin{column}{0.55\textwidth}
      \minipage[c][0.75\textheight][s]{\columnwidth}
      \begin{itemize}
        \item<1-> \funcname{caml\_stash\_backtrace} scans the older part of the stack, up to the next trap
        \item<6-> Controls jumps to the nested exception handler in \funcname{maybe\_raise}, which matches only on \typename{ExnB}
        \item<7-> \funcname{caml\_reraise\_exn} is called again, backtrace collection resumes with \funcname{caml\_stash\_backtrace}
      \end{itemize}
      \medskip
      \begin{itemize}
        \item<2-> Constructed backtrace:
        \begin{itemize}
          \item <2-> \funcname{definitely\_raise}
          \item <2-> \funcname{probably\_raise}
          \item <3-> \funcname{probably\_raise} (re-raise)
          \item <4-> \funcname{maybe\_raise}
          \item <5-> \funcname{main}
          \item <8-> \funcname{main} (re-raise)
        \end{itemize}
      \end{itemize}
      \vfill
      \onslide*<1-5>{\mintocaml[firstline=15,lastline=21]{../src/backtrace/backtrace.ml}}
      \onslide*<6>{\mintocaml[firstline=28,lastline=34,highlightlines=31]{../src/backtrace/backtrace.ml}}
      \onslide*<7->{\mintocaml[firstline=28,lastline=34,highlightlines=33]{../src/backtrace/backtrace.ml}}
      \endminipage
    \end{column}
    \begin{column}{0.45\textwidth}
      \raggedleft
      \onslide*<1>{\providecommand\step{6}\input{diagrams/backtrace/backtrace.tex}}%
      \onslide*<2>{\providecommand\step{6}\input{diagrams/backtrace/backtrace.tex}}%
      \onslide*<3>{\providecommand\step{7}\input{diagrams/backtrace/backtrace.tex}}%
      \onslide*<4>{\providecommand\step{8}\input{diagrams/backtrace/backtrace.tex}}%
      \onslide*<5>{\providecommand\step{9}\input{diagrams/backtrace/backtrace.tex}}%
      \onslide*<6>{\providecommand\step{10}\input{diagrams/backtrace/backtrace.tex}}%
      \onslide*<7>{\providecommand\step{11}\input{diagrams/backtrace/backtrace.tex}}%
      \onslide*<8>{\providecommand\step{12}\input{diagrams/backtrace/backtrace.tex}}%
      \onslide*<9>{\providecommand\step{13}\input{diagrams/backtrace/backtrace.tex}}%
    \end{column}
  \end{columns}
\end{frame}

\begin{frame}{Backtraces}{Printing the backtrace}
  \begin{columns}[c]
    \begin{column}{0.45\textwidth}
      \minipage[c][0.66\textheight][s]{\columnwidth}
      \begin{itemize}
        \item<1-> The exception backtrace can now be printed to \funcarg{stdout} with \funcname{Printexc.print_backtrace}
        \item<2-> First the exception backtrace is retrieved into an OCaml array with \funcname{get_raw_backtrace }
        \item<3-> Which is implemented as the \funcname{caml_get_exception_raw_backtrace} C function in the runtime
        \item<4-> And finally printed with \funcname{print_exception_backtrace}
      \end{itemize}
      \vfill
      \mintocaml[firstline=28,lastline=34,highlightlines=33]{../src/backtrace/backtrace.ml}
      \endminipage
    \end{column}
    \begin{column}{0.55\textwidth}
      \onslide*<1,2>{
        \mintocaml[firstline=100,lastline=101]{../ocaml/stdlib/printexc.ml}
      }
      \only<1>{
        \listocaml[firstline=170,lastline=172]{../ocaml/stdlib/printexc.ml}{stdlib/printexc.ml:170}
      }
      \only<2>{
        \listocaml[firstline=170,lastline=172,highlightlines=172]{../ocaml/stdlib/printexc.ml}{stdlib/printexc.ml:170}
      }
      \only<3>{
        \mintc[firstline=154,lastline=156]{../ocaml/runtime/backtrace.c}
        \listc[firstline=171,lastline=192,highlightlines={184-187}]{../ocaml/runtime/backtrace.c}{runtime/backtrace.c:154}
      }
      \only<4>{
        \listocaml[firstline=155,lastline=165]{../ocaml/stdlib/printexc.ml}{stdlib/printexc.ml:55}
      }
    \end{column}
  \end{columns}
\end{frame}


\subsection{The multiple kinds of raise}
\frameSubsection{}{}

\begin{frame}{Backtraces}{When backtraces are not needed}
  \begin{columns}[c]
    \begin{column}{0.45\textwidth}
      \minipage[c][0.66\textheight][s]{\columnwidth}
      \begin{itemize}
        \item<1-> What if the backtrace for an exception is never needed?
        \item<2-> It's possible to raise the exception explicitly with \funcname{raise\_notrace} to make raising as cheap as possible
        \item<3-> An example of use in the \funcname{dynlink} library of the OCaml compiler
      \end{itemize}
      \vfill
      \onslide<3->{
      \listocaml[firstline=205,lastline=209,highlightlines=207]{../ocaml/otherlibs/dynlink/dynlink\_compilerlibs/misc.ml}{otherlibs/dynlink/dynlink\_compilerlibs/misc.ml:205}
      }
      \endminipage
    \end{column}
    \begin{column}{0.55\textwidth}
    \only<4>{
      \mintcmm[highlightlines=3]{../src/notrace/camlNotrace\_\_fun_347.cmm}
      \listcmm{../src/notrace/camlNotrace\_\_all\_somes\_327.cmm}{all\_somes}
    }
    \onslide*<5->{
      \listobjdump[highlightlines={6-9}]{../src/notrace/camlNotrace\_\_fun_347.objdump}{anonymous function from \funcname{all\_somes}}
    }
    \end{column}
  \end{columns}
\end{frame}

\begin{frame}{Backtraces}{Summary table of the multiple kinds of raise}
  \begin{itemize}
    \item When debugging information is enabled and if the backtrace of an exception isn't needed, it's possible to raise with \funcname{raise\_notrace} for performance
    \item When debugging information is \emph{not} enabled, the compiler optimises \funcname{raise} calls to inline assembly (same effect as using \funcname{raise\_notrace})
  \end{itemize}
  \bigskip
  \centering
  \begin{tabular}{| l | c | c | c | c |}
    \hline
    \parbox{10em}{Debug information \\ (\funcarg{-g} flag)}  & \multicolumn{2}{|c|}{Disabled}                                     & \multicolumn{2}{|c|}{Enabled} \\
    \hline
    Raise kind                                               & \funcname{raise} & \funcname{raise\_notrace}                       & \funcname{raise} & \funcname{raise\_notrace} \\
    \hline
    Compilation                                              & \parbox{8em}{inline assembly\\ (optimization)} & inline assembly   & \funcname{caml\_raise\_exn} & inline assembly \\
    \hline
  \end{tabular}
\end{frame}


%
% Takeaway #5
%
\subsection*{Takeaway \#5}
\frameSubsectionTakeaway{}
\againframe<5>{takeaway}
}%
      \onslide*<2>{\providecommand\step{1}\input{diagrams/backtrace/scanstack.tex}}%
      \onslide*<3>{\providecommand\step{2}\input{diagrams/backtrace/scanstack.tex}}%
      \onslide*<4>{\providecommand\step{3}\input{diagrams/backtrace/scanstack.tex}}%
      \onslide*<5>{\providecommand\step{4}\input{diagrams/backtrace/scanstack.tex}}%
      \onslide*<6>{\providecommand\step{5}\input{diagrams/backtrace/scanstack.tex}}%
    \end{column}
  \end{columns}
\end{frame}

% Duplicate of previous-previous slide
\begin{frame}{Backtraces}{Continuing on \funcname{caml\_stash\_backtrace}}
  \begin{columns}[c]
    \begin{column}{0.5\textwidth}
      \minipage[c][0.75\textheight][s]{\columnwidth}
      \begin{itemize}
          \color{gray}
        \item A C function provided by the runtime (specific to native code)
        \item It scans the stack, between the stack pointer at the raising point up to the next trap, in order to collect the names of the detected function frames
        \item It uses the same technique and information as the Garbage Collector (GC) uses to scan the values live on the stack
      \end{itemize}
      \begin{itemize}
        \item For each function frame detected, store a pointer to the \typename{frame\_descr} in the \localname{domain\_state->backtrace\_buffer} array, effectively constructing the backtrace
      \end{itemize}
      \onslide<2->{
        \begin{itemize}
            \smallskip
          \item Constructed backtrace:
            \funcname{definitely\_raise},
            \onslide<3->{\funcname{probably\_raise}}
        \end{itemize}
      }
      \vfill
      \mintocaml[firstline=9,lastline=13,highlightlines=12]{../src/backtrace/backtrace.ml}
      \endminipage
    \end{column}
    \begin{column}{0.5\textwidth}
      \raggedleft
      \onslide*<1>{\listStashBacktrace[highlightlines={114-115}]{}}
      \onslide*<2>{\providecommand\step{3}%
% Backtraces
%
\subsection{One advantage of raising with debugging information}
\frameSubsection{}{}

\begin{frame}{Backtraces}{Debugging information \& source code}
  \begin{columns}[c]
    \begin{column}{0.5\textwidth}
      \begin{itemize}
        \item Raising an exception is fun and games, but how do we know where the exception originated? $\rightarrow$ With the backtrace associated with the exception!
        \item In order to get a backtrace from the point where an exception was raised, it's necessary to compile with debugging information enabled (\funcarg{-g} flag for the compiler)
        \item Same example as in the `Nested handlers' section
      \end{itemize}
    \end{column}
    \begin{column}{0.5\textwidth}
      \listocaml[firstline=9,lastline=34]{../src/backtrace/backtrace.ml}{backtrace/backtrace.ml}
    \end{column}
  \end{columns}
\end{frame}

\begin{frame}{Backtraces}{CMM and disassembly of \funcname{definitely\_raise}}
  \begin{columns}[c]
    \begin{column}{0.5\textwidth}
      \minipage[c][0.66\textheight][s]{\columnwidth}
      \begin{itemize}
        \item<1-> An effect of compiling with debugging information (\funcarg{-g}): the CMM no longer uses \funcname{raise\_notrace} to raise the exception but \funcname{raise} instead
        \item<2-> Disassembly shows that CMM \funcname{raise} is actually a call to \funcname{caml\_raise\_exn}, a function in assembler provided by the runtime
      \end{itemize}
      \vfill
      \mintocaml[firstline=9,lastline=13,highlightlines=12]{../src/backtrace/backtrace.ml}
      \endminipage
    \end{column}
    \begin{column}{0.5\textwidth}
      \onslide*<1>{
        \mintcmm[highlightlines=5]{../src/backtrace/camlBacktrace\_\_definitely\_raise\_347.cmm}
      }
      \onslide*<2>{
        \mintobjdump[firstline=2,highlightlines=14]{../src/backtrace/camlBacktrace\_\_definitely\_raise\_347.objdump}
      }
    \end{column}
  \end{columns}
\end{frame}

\begin{frame}{Backtraces}{Introducing \funcname{caml\_raise\_exn}}
  \begin{columns}[c]
    \begin{column}{0.5\textwidth}
      \minipage[c][0.66\textheight][s]{\columnwidth}
      \onslide*<1>{
        \begin{itemize}
          \item Raising the exception is performed using \funcname{caml\_raise\_exn} which is
            \begin{itemize}
              \item An assembly function provided by the runtime
              \item Architecture specific
            \end{itemize}
          \item Let's see what it does differently from \funcname{raise\_notrace}\dots
          \item We are about to raise \typename{ExnC} with \funcname{caml\_raise\_exn} from \funcname{definitely\_raise}
        \end{itemize}
      }
      \onslide*<2>{
        \mintobjdump[firstline=2,highlightlines=14]{../src/backtrace/camlBacktrace\_\_definitely\_raise\_347.objdump}
      }
      \vfill
      \mintocaml[firstline=9,lastline=13,highlightlines=12]{../src/backtrace/backtrace.ml}
      \endminipage
    \end{column}
    \begin{column}{0.5\textwidth}
      \centering
      \providecommand\step{1}\input{diagrams/backtrace/backtrace.tex}
    \end{column}
  \end{columns}
\end{frame}

\begin{frame}{Backtraces}{But before that, we need more fields from \typename{caml\_domain\_state}}
  \begin{columns}
    \begin{column}{0.5\textwidth}
      \begin{itemize}
        \item Remember \typename{caml\_domain\_state} the per-domain runtime state?
        \item Some other members are of interest to us now\dots
        \item \localname{backtrace\_active} is used to know if the runtime should collect backtraces
        \item \localname{backtrace\_active} can be set by
          \begin{itemize}
            \item The \funcarg{b} flag in the \funcname{OCAMLRUNPARAM} environment variable
            \item \mintinline{ocaml}{Printexc.record_backtrace : bool -> unit} \footnotemark
          \end{itemize}
      \end{itemize}
    \end{column}
    \begin{column}{0.5\textwidth}
      \begin{listing}
        \mintc[highlightlines={9-11}]{src/ocaml/domain\_state\_backtrace.h}
        \caption{runtime/caml/domain\_state.h}
      \end{listing}
    \end{column}
  \end{columns}
  \footnotetext[1]{https://v2.ocaml.org/api/Printexc.html}
\end{frame}

\newcommand\listCamlRaiseExn[1][]{
  \listasm[firstline=702,lastline=725,#1]{../ocaml/runtime/amd64.S}{runtime/amd64.S:702}}

\begin{frame}{Backtraces}{Walk through \funcname{caml\_raise\_exn}}
  \begin{columns}[T]
    \begin{column}{0.5\textwidth}
      \begin{itemize}
        \item<1-> First checks whether exceptions backtrace should be recorded
        \item<1-> By checking the \funcarg{domain\_state->backtrace\_active} value
        \item<2-> If not, acts as \funcname{raise\_notrace} with \funcname{RESTORE\_EXN\_HANDLER\_OCAML}
          \begin{itemize}
            \item Set \regname{RSP} to \localname{domain\_state->exn\_handler} value
            \item Restore \localname{domain\_state->exn\_handler} from trap
            \item Jump with \funcname{ret} (same effect as using \funcname{pop} and \funcname{jmp})
          \end{itemize}
        \item<3-> Otherwise, switches to the C stack to be able to execute C code
        \item<4-> Calls \funcname{caml\_stash\_backtrace}, a C function provided by the runtime\ignorespaces
          \onslide<6->{, with
            \begin{itemize}
              \item \funcarg{exn}: exception value
              \item \funcarg{pc}: code address of the raising point
              \item \funcarg{sp}: stack address of the raising function
              \item \funcarg{trapsp}: stack address of the next handler
            \end{itemize}
          }
      \end{itemize}
      \bigskip
      \mintocaml[firstline=9,lastline=13,highlightlines=12]{../src/backtrace/backtrace.ml}
    \end{column}
    \begin{column}{0.5\textwidth}
      \raggedleft
      \onslide*<1,3-4>{
        \mintc[firstline=190,lastline=192]{../ocaml/runtime/amd64.S}
        \mintc[firstline=234,lastline=237]{../ocaml/runtime/amd64.S}
      }
      \onslide*<2>{
        \mintc[firstline=190,lastline=192,highlightlines={191-192}]{../ocaml/runtime/amd64.S}
        \mintc[firstline=234,lastline=237,highlightlines={235-237}]{../ocaml/runtime/amd64.S}
      }
      \onslide*<1>{\listCamlRaiseExn[highlightlines=705]{}}%
      \onslide*<2>{\listCamlRaiseExn[highlightlines={707-708}]{}}%
      \onslide*<3>{\listCamlRaiseExn[highlightlines={712-714}]{}}%
      \onslide*<4>{\listCamlRaiseExn[highlightlines={715-720}]{}}%
      \onslide*<5>{\providecommand\step{1}\input{diagrams/backtrace/backtrace.tex}}%
      \onslide*<6>{\providecommand\step{2}\input{diagrams/backtrace/backtrace.tex}}%
    \end{column}
  \end{columns}
\end{frame}

\newcommand\listStashBacktrace[1][]{
  \listc[firstline=91,lastline=120,#1]{../ocaml/runtime/backtrace\_nat.c}{runtime/backtrace\_nat.c:91}}

\begin{frame}{Backtraces}{Introducing \funcname{caml\_stash\_backtrace}}
  \begin{columns}[c]
    \begin{column}{0.5\textwidth}
      \minipage[c][0.66\textheight][s]{\columnwidth}
      \begin{itemize}
        \item<1-> A C function provided by the runtime (specific to native code)
        \item<2-> It scans the stack, between the stack pointer at the raising point up to the next trap, in order to collect the names of the detected function frames
          \onslide*<2>{
            \begin{itemize}
              \item And more information, like line number, column number...
            \end{itemize}
          }
        \item<3-> It uses the same technique and information as the Garbage Collector (GC) uses to scan the values live on the stack
          \onslide*<4>{
            \begin{itemize}
              \item \typename{frame\_descr} are at the core of stack unwinding
            \end{itemize}
          }
      \end{itemize}
      \vfill
      \mintocaml[firstline=9,lastline=13,highlightlines=12]{../src/backtrace/backtrace.ml}
      \endminipage
    \end{column}
    \begin{column}{0.5\textwidth}
      \onslide*<1>{\listStashBacktrace[highlightlines=91]{}}
      \onslide*<2>{\listStashBacktrace[highlightlines={91,117-118}]{}}
      \onslide*<3->{\listStashBacktrace[highlightlines={94,106,109-110}]{}}
    \end{column}
  \end{columns}
\end{frame}

\newcommand\listFrameDescr[1][]{
  \listocaml[firstline=111,lastline=124,#1]{../ocaml/asmcomp/emitaux.ml}{asmcomp/emitaux.ml:111}}
\newcommand\listDebugInfo[1][]{
  \listocaml[firstline=38,lastline=49,#1]{../ocaml/lambda/debuginfo.mli}{lambda/debuginfo.mli:38}}

\begin{frame}{Backtraces}{\typename{frame\_descr} and \typename{frame\_debuginfo} in a nutshell}
  \begin{columns}[c]
    \begin{column}{0.5\textwidth}
      \begin{itemize}
        \item<1-> For every return address in an OCaml program, there is a associated \typename{frame\_descr}
        \item<2-> A \typename{frame\_descr} specifies
        \begin{itemize}
          \item<2-> The size of the function stack frame
          \item<3-> Where live values are on the stack (so that the GC can mark them as live and prevent from being swept)
        \end{itemize}
        \item<4-> When compiled with debug information, \typename{frame\_descr} will be linked to a \typename{frame\_debuginfo}
        \begin{itemize}
          \item<5-> Contains information about the position in the source code (file name, line and column number)
        \end{itemize}
      \end{itemize}
    \end{column}
    \begin{column}{0.5\textwidth}
        \onslide*<1>{\listFrameDescr[highlightlines={118-122}]{}}
        \onslide*<2>{\listFrameDescr[highlightlines={120}]{}}
        \onslide*<3>{\listFrameDescr[highlightlines={121}]{}}
        \onslide*<4->{\listFrameDescr[highlightlines={122}]{}}
        \medskip
        \onslide*<1-3>{\listDebugInfo{}}
        \onslide*<4>{\listDebugInfo[highlightlines={38,49}]{}}
        \onslide*<5>{\listDebugInfo[highlightlines={39-46}]{}}
    \end{column}
  \end{columns}
\end{frame}

\begin{frame}{Backtraces}{Scanning the stack with \typename{frame\_descr}}
  \begin{columns}[c]
    \begin{column}{0.5\textwidth}
      \begin{itemize}
        \item Given the return address of the code that raises the exception by calling \funcname{caml\_raise\_exn} (\funcarg{pc} argument of \funcname{caml\_stash\_backtrace})
        \item And the position of the stack pointer (\funcarg{sp} argument of \funcname{caml\_stash\_backtrace})
        \item The frame size given by \typename{frame\_descr}.\funcarg{frame\_size}, allows to find the end of the previous frame
        \item And the return address of the caller of this function
        \bigskip
        \onslide<3->{
        \item Note that traps (here the reddish \funcname{ExnA}, \funcname{ExnB} and \funcname{ExnC} blocks) belong to the frame that installed them, and so the frame size changes when traps are removed
        }
      \end{itemize}
    \end{column}
    \begin{column}{0.5\textwidth}
      \raggedleft
      \onslide*<1>{\providecommand\step{2}\input{diagrams/backtrace/backtrace.tex}}%
      \onslide*<2>{\providecommand\step{1}\input{diagrams/backtrace/scanstack.tex}}%
      \onslide*<3>{\providecommand\step{2}\input{diagrams/backtrace/scanstack.tex}}%
      \onslide*<4>{\providecommand\step{3}\input{diagrams/backtrace/scanstack.tex}}%
      \onslide*<5>{\providecommand\step{4}\input{diagrams/backtrace/scanstack.tex}}%
      \onslide*<6>{\providecommand\step{5}\input{diagrams/backtrace/scanstack.tex}}%
    \end{column}
  \end{columns}
\end{frame}

% Duplicate of previous-previous slide
\begin{frame}{Backtraces}{Continuing on \funcname{caml\_stash\_backtrace}}
  \begin{columns}[c]
    \begin{column}{0.5\textwidth}
      \minipage[c][0.75\textheight][s]{\columnwidth}
      \begin{itemize}
          \color{gray}
        \item A C function provided by the runtime (specific to native code)
        \item It scans the stack, between the stack pointer at the raising point up to the next trap, in order to collect the names of the detected function frames
        \item It uses the same technique and information as the Garbage Collector (GC) uses to scan the values live on the stack
      \end{itemize}
      \begin{itemize}
        \item For each function frame detected, store a pointer to the \typename{frame\_descr} in the \localname{domain\_state->backtrace\_buffer} array, effectively constructing the backtrace
      \end{itemize}
      \onslide<2->{
        \begin{itemize}
            \smallskip
          \item Constructed backtrace:
            \funcname{definitely\_raise},
            \onslide<3->{\funcname{probably\_raise}}
        \end{itemize}
      }
      \vfill
      \mintocaml[firstline=9,lastline=13,highlightlines=12]{../src/backtrace/backtrace.ml}
      \endminipage
    \end{column}
    \begin{column}{0.5\textwidth}
      \raggedleft
      \onslide*<1>{\listStashBacktrace[highlightlines={114-115}]{}}
      \onslide*<2>{\providecommand\step{3}\input{diagrams/backtrace/backtrace.tex}}%
      \onslide*<3>{\providecommand\step{4}\input{diagrams/backtrace/backtrace.tex}}%
    \end{column}
  \end{columns}
\end{frame}

\begin{frame}{Backtraces}{Back to \funcname{caml\_raise\_exn}}
  \begin{columns}[c]
    \begin{column}{0.5\textwidth}
      \minipage[c][0.66\textheight][s]{\columnwidth}
      \begin{itemize}\color{gray}
        \item Checks wheteher exceptions backtrace should be recorded, by checking the \funcarg{domain\_state->backtrace\_active} value
        \item Switches to the C stack to be able to execute C code
        \item Calls \funcname{caml\_stash\_backtrace}, to collect the backtrace up to the next trap
      \end{itemize}
      \onslide*<2->{
        \begin{itemize}
          \item<2-> Executes the exception handler in \funcname{probably\_raise} with \funcname{RESTORE\_EXN\_HANDLER\_OCAML}
            \begin{itemize}
              \item Set \regname{RSP} to \localname{domain\_state->exn\_handler} value
              \item Restore \localname{domain\_state->exn\_handler} from trap
              \item Jump to the handler code with \funcname{ret}
            \end{itemize}
        \end{itemize}
      }
      \vfill
      \onslide*<1>{\mintocaml[firstline=9,lastline=13,highlightlines=12]{../src/backtrace/backtrace.ml}}
      \onslide*<2->{\mintocaml[firstline=15,lastline=21,highlightlines={19-21}]{../src/backtrace/backtrace.ml}}
      \endminipage
    \end{column}
    \begin{column}{0.5\textwidth}
      \centering
      \onslide*<1>{
        \mintc[firstline=190,lastline=192]{../ocaml/runtime/amd64.S}
        \mintc[firstline=234,lastline=237]{../ocaml/runtime/amd64.S}
      }
      \onslide*<2>{
        \mintc[firstline=190,lastline=192,highlightlines={191-192}]{../ocaml/runtime/amd64.S}
        \mintc[firstline=234,lastline=237,highlightlines={235-237}]{../ocaml/runtime/amd64.S}
      }
      \onslide*<1>{\listCamlRaiseExn[highlightlines=720]{}}
      \onslide*<2>{\listCamlRaiseExn[highlightlines=722]{}}
      \onslide*<3->{\providecommand\step{5}\input{diagrams/backtrace/backtrace.tex}}
    \end{column}
  \end{columns}
\end{frame}

\begin{frame}{Backtraces}{\funcname{caml\_probably\_raise}'s handler doesn't catch \typename{ExnC}}
  \begin{columns}[c]
    \begin{column}{0.45\textwidth}
      \minipage[c][0.75\textheight][s]{\columnwidth}
      \begin{itemize}
        \item<1-> \funcname{match \dots with}\footnotemark pattern matching can also match exceptions
        \item<1-> The CMM of \funcname{probably\_raise} shows that this \funcname{match \dots with} is implemented with a pattern matching and a \funcname{try \dots with}
        \item<1-> But this one only matches on \typename{ExnA} exception
        \item<2-> Since the pattern matching fails to catch the exception, \funcname{reraise} is called with our current \typename{ExnC} exception
        \item<3-> Disassembly shows that \funcname{reraise} is actually a call to \funcname{caml\_reraise\_exn}, which is also an assembly function provided by the runtime
      \end{itemize}
      \vfill
      \mintocaml[firstline=15,lastline=21,highlightlines={18-21}]{../src/backtrace/backtrace.ml}
      \endminipage
    \end{column}
    \begin{column}{0.55\textwidth}
      \centering
      \onslide*<1>{\mintcmm[highlightlines={10-18}]{../src/backtrace/camlBacktrace\_\_probably\_raise\_351.cmm}}
      \onslide*<2>{\listcmm[highlightlines=18]{../src/backtrace/camlBacktrace\_\_probably\_raise_351.cmm}{CMM of probably\_raise}}
      \onslide*<3>{\mintobjdump[firstline=5,lastline=35,highlightlines=31]{../src/backtrace/camlBacktrace\_\_probably\_raise_351.objdump}}
    \end{column}
  \end{columns}
  \only<1>{\footnotetext[1]{https://blog.janestreet.com/pattern-matching-and-exception-handling-unite/}}
\end{frame}

\newcommand\listCamlReraiseExn[1][]{
  \listasm[firstline=727,lastline=734,#1]{../ocaml/runtime/amd64.S}{runtime/amd64.S:727}}

\begin{frame}{Backtraces}{Introducing \funcname{caml\_reraise\_exn}}
  \begin{columns}[c]
    \begin{column}{0.5\textwidth}
      \minipage[c][0.66\textheight][s]{\columnwidth}
      \begin{itemize}
        \item<1-> The exception have to be reraised to the next handler
        \item<2-> First, checks if the backtrace should be collected with \funcarg{domain\_state->backtrace\_active}
        \only<3>{
        \begin{itemize}
            \item If not, behaves like a \funcname{raise\_notrace} with \funcname{RESTORE\_EXN\_HANDLER\_OCAML}
        \end{itemize}
        }
        \item<4-> Jumps into \funcname{caml\_raise\_exn}
        \item<5-> Calls \funcname{caml\_stash\_backtrace} again, to scan the next part of the stack: from the trap just pop-ed to the next trap
      \end{itemize}
      \vfill
      \mintocaml[firstline=15,lastline=21,highlightlines={18-21}]{../src/backtrace/backtrace.ml}
      \endminipage
    \end{column}
    \begin{column}{0.5\textwidth}
      \raggedleft
      \onslide*<1>{\listCamlReraiseExn{}}
      \onslide*<2>{\listCamlReraiseExn[highlightlines=729]{}}
      \onslide*<3>{
        \mintc[firstline=190,lastline=192,highlightlines={191-192}]{../ocaml/runtime/amd64.S}
        \mintc[firstline=234,lastline=237,highlightlines={235-237}]{../ocaml/runtime/amd64.S}
        \medskip
        \listCamlReraiseExn[highlightlines=731]{}
      }
      \onslide*<4-5>{
        % TODO refactor with \listCamlReraiseExn
        \mintasm[firstline=727,lastline=734,highlightlines=730]{../ocaml/runtime/amd64.S}
        \medskip
      }
      \onslide*<4>{\listCamlRaiseExn[highlightlines=711]{}}
      \onslide*<5>{\listCamlRaiseExn[highlightlines=720]{}}
    \end{column}
  \end{columns}
\end{frame}

\begin{frame}{Backtraces}{Backtrace collection continues}
  \begin{columns}[c]
    \begin{column}{0.55\textwidth}
      \minipage[c][0.75\textheight][s]{\columnwidth}
      \begin{itemize}
        \item<1-> \funcname{caml\_stash\_backtrace} scans the older part of the stack, up to the next trap
        \item<6-> Controls jumps to the nested exception handler in \funcname{maybe\_raise}, which matches only on \typename{ExnB}
        \item<7-> \funcname{caml\_reraise\_exn} is called again, backtrace collection resumes with \funcname{caml\_stash\_backtrace}
      \end{itemize}
      \medskip
      \begin{itemize}
        \item<2-> Constructed backtrace:
        \begin{itemize}
          \item <2-> \funcname{definitely\_raise}
          \item <2-> \funcname{probably\_raise}
          \item <3-> \funcname{probably\_raise} (re-raise)
          \item <4-> \funcname{maybe\_raise}
          \item <5-> \funcname{main}
          \item <8-> \funcname{main} (re-raise)
        \end{itemize}
      \end{itemize}
      \vfill
      \onslide*<1-5>{\mintocaml[firstline=15,lastline=21]{../src/backtrace/backtrace.ml}}
      \onslide*<6>{\mintocaml[firstline=28,lastline=34,highlightlines=31]{../src/backtrace/backtrace.ml}}
      \onslide*<7->{\mintocaml[firstline=28,lastline=34,highlightlines=33]{../src/backtrace/backtrace.ml}}
      \endminipage
    \end{column}
    \begin{column}{0.45\textwidth}
      \raggedleft
      \onslide*<1>{\providecommand\step{6}\input{diagrams/backtrace/backtrace.tex}}%
      \onslide*<2>{\providecommand\step{6}\input{diagrams/backtrace/backtrace.tex}}%
      \onslide*<3>{\providecommand\step{7}\input{diagrams/backtrace/backtrace.tex}}%
      \onslide*<4>{\providecommand\step{8}\input{diagrams/backtrace/backtrace.tex}}%
      \onslide*<5>{\providecommand\step{9}\input{diagrams/backtrace/backtrace.tex}}%
      \onslide*<6>{\providecommand\step{10}\input{diagrams/backtrace/backtrace.tex}}%
      \onslide*<7>{\providecommand\step{11}\input{diagrams/backtrace/backtrace.tex}}%
      \onslide*<8>{\providecommand\step{12}\input{diagrams/backtrace/backtrace.tex}}%
      \onslide*<9>{\providecommand\step{13}\input{diagrams/backtrace/backtrace.tex}}%
    \end{column}
  \end{columns}
\end{frame}

\begin{frame}{Backtraces}{Printing the backtrace}
  \begin{columns}[c]
    \begin{column}{0.45\textwidth}
      \minipage[c][0.66\textheight][s]{\columnwidth}
      \begin{itemize}
        \item<1-> The exception backtrace can now be printed to \funcarg{stdout} with \funcname{Printexc.print_backtrace}
        \item<2-> First the exception backtrace is retrieved into an OCaml array with \funcname{get_raw_backtrace }
        \item<3-> Which is implemented as the \funcname{caml_get_exception_raw_backtrace} C function in the runtime
        \item<4-> And finally printed with \funcname{print_exception_backtrace}
      \end{itemize}
      \vfill
      \mintocaml[firstline=28,lastline=34,highlightlines=33]{../src/backtrace/backtrace.ml}
      \endminipage
    \end{column}
    \begin{column}{0.55\textwidth}
      \onslide*<1,2>{
        \mintocaml[firstline=100,lastline=101]{../ocaml/stdlib/printexc.ml}
      }
      \only<1>{
        \listocaml[firstline=170,lastline=172]{../ocaml/stdlib/printexc.ml}{stdlib/printexc.ml:170}
      }
      \only<2>{
        \listocaml[firstline=170,lastline=172,highlightlines=172]{../ocaml/stdlib/printexc.ml}{stdlib/printexc.ml:170}
      }
      \only<3>{
        \mintc[firstline=154,lastline=156]{../ocaml/runtime/backtrace.c}
        \listc[firstline=171,lastline=192,highlightlines={184-187}]{../ocaml/runtime/backtrace.c}{runtime/backtrace.c:154}
      }
      \only<4>{
        \listocaml[firstline=155,lastline=165]{../ocaml/stdlib/printexc.ml}{stdlib/printexc.ml:55}
      }
    \end{column}
  \end{columns}
\end{frame}


\subsection{The multiple kinds of raise}
\frameSubsection{}{}

\begin{frame}{Backtraces}{When backtraces are not needed}
  \begin{columns}[c]
    \begin{column}{0.45\textwidth}
      \minipage[c][0.66\textheight][s]{\columnwidth}
      \begin{itemize}
        \item<1-> What if the backtrace for an exception is never needed?
        \item<2-> It's possible to raise the exception explicitly with \funcname{raise\_notrace} to make raising as cheap as possible
        \item<3-> An example of use in the \funcname{dynlink} library of the OCaml compiler
      \end{itemize}
      \vfill
      \onslide<3->{
      \listocaml[firstline=205,lastline=209,highlightlines=207]{../ocaml/otherlibs/dynlink/dynlink\_compilerlibs/misc.ml}{otherlibs/dynlink/dynlink\_compilerlibs/misc.ml:205}
      }
      \endminipage
    \end{column}
    \begin{column}{0.55\textwidth}
    \only<4>{
      \mintcmm[highlightlines=3]{../src/notrace/camlNotrace\_\_fun_347.cmm}
      \listcmm{../src/notrace/camlNotrace\_\_all\_somes\_327.cmm}{all\_somes}
    }
    \onslide*<5->{
      \listobjdump[highlightlines={6-9}]{../src/notrace/camlNotrace\_\_fun_347.objdump}{anonymous function from \funcname{all\_somes}}
    }
    \end{column}
  \end{columns}
\end{frame}

\begin{frame}{Backtraces}{Summary table of the multiple kinds of raise}
  \begin{itemize}
    \item When debugging information is enabled and if the backtrace of an exception isn't needed, it's possible to raise with \funcname{raise\_notrace} for performance
    \item When debugging information is \emph{not} enabled, the compiler optimises \funcname{raise} calls to inline assembly (same effect as using \funcname{raise\_notrace})
  \end{itemize}
  \bigskip
  \centering
  \begin{tabular}{| l | c | c | c | c |}
    \hline
    \parbox{10em}{Debug information \\ (\funcarg{-g} flag)}  & \multicolumn{2}{|c|}{Disabled}                                     & \multicolumn{2}{|c|}{Enabled} \\
    \hline
    Raise kind                                               & \funcname{raise} & \funcname{raise\_notrace}                       & \funcname{raise} & \funcname{raise\_notrace} \\
    \hline
    Compilation                                              & \parbox{8em}{inline assembly\\ (optimization)} & inline assembly   & \funcname{caml\_raise\_exn} & inline assembly \\
    \hline
  \end{tabular}
\end{frame}


%
% Takeaway #5
%
\subsection*{Takeaway \#5}
\frameSubsectionTakeaway{}
\againframe<5>{takeaway}
}%
      \onslide*<3>{\providecommand\step{4}%
% Backtraces
%
\subsection{One advantage of raising with debugging information}
\frameSubsection{}{}

\begin{frame}{Backtraces}{Debugging information \& source code}
  \begin{columns}[c]
    \begin{column}{0.5\textwidth}
      \begin{itemize}
        \item Raising an exception is fun and games, but how do we know where the exception originated? $\rightarrow$ With the backtrace associated with the exception!
        \item In order to get a backtrace from the point where an exception was raised, it's necessary to compile with debugging information enabled (\funcarg{-g} flag for the compiler)
        \item Same example as in the `Nested handlers' section
      \end{itemize}
    \end{column}
    \begin{column}{0.5\textwidth}
      \listocaml[firstline=9,lastline=34]{../src/backtrace/backtrace.ml}{backtrace/backtrace.ml}
    \end{column}
  \end{columns}
\end{frame}

\begin{frame}{Backtraces}{CMM and disassembly of \funcname{definitely\_raise}}
  \begin{columns}[c]
    \begin{column}{0.5\textwidth}
      \minipage[c][0.66\textheight][s]{\columnwidth}
      \begin{itemize}
        \item<1-> An effect of compiling with debugging information (\funcarg{-g}): the CMM no longer uses \funcname{raise\_notrace} to raise the exception but \funcname{raise} instead
        \item<2-> Disassembly shows that CMM \funcname{raise} is actually a call to \funcname{caml\_raise\_exn}, a function in assembler provided by the runtime
      \end{itemize}
      \vfill
      \mintocaml[firstline=9,lastline=13,highlightlines=12]{../src/backtrace/backtrace.ml}
      \endminipage
    \end{column}
    \begin{column}{0.5\textwidth}
      \onslide*<1>{
        \mintcmm[highlightlines=5]{../src/backtrace/camlBacktrace\_\_definitely\_raise\_347.cmm}
      }
      \onslide*<2>{
        \mintobjdump[firstline=2,highlightlines=14]{../src/backtrace/camlBacktrace\_\_definitely\_raise\_347.objdump}
      }
    \end{column}
  \end{columns}
\end{frame}

\begin{frame}{Backtraces}{Introducing \funcname{caml\_raise\_exn}}
  \begin{columns}[c]
    \begin{column}{0.5\textwidth}
      \minipage[c][0.66\textheight][s]{\columnwidth}
      \onslide*<1>{
        \begin{itemize}
          \item Raising the exception is performed using \funcname{caml\_raise\_exn} which is
            \begin{itemize}
              \item An assembly function provided by the runtime
              \item Architecture specific
            \end{itemize}
          \item Let's see what it does differently from \funcname{raise\_notrace}\dots
          \item We are about to raise \typename{ExnC} with \funcname{caml\_raise\_exn} from \funcname{definitely\_raise}
        \end{itemize}
      }
      \onslide*<2>{
        \mintobjdump[firstline=2,highlightlines=14]{../src/backtrace/camlBacktrace\_\_definitely\_raise\_347.objdump}
      }
      \vfill
      \mintocaml[firstline=9,lastline=13,highlightlines=12]{../src/backtrace/backtrace.ml}
      \endminipage
    \end{column}
    \begin{column}{0.5\textwidth}
      \centering
      \providecommand\step{1}\input{diagrams/backtrace/backtrace.tex}
    \end{column}
  \end{columns}
\end{frame}

\begin{frame}{Backtraces}{But before that, we need more fields from \typename{caml\_domain\_state}}
  \begin{columns}
    \begin{column}{0.5\textwidth}
      \begin{itemize}
        \item Remember \typename{caml\_domain\_state} the per-domain runtime state?
        \item Some other members are of interest to us now\dots
        \item \localname{backtrace\_active} is used to know if the runtime should collect backtraces
        \item \localname{backtrace\_active} can be set by
          \begin{itemize}
            \item The \funcarg{b} flag in the \funcname{OCAMLRUNPARAM} environment variable
            \item \mintinline{ocaml}{Printexc.record_backtrace : bool -> unit} \footnotemark
          \end{itemize}
      \end{itemize}
    \end{column}
    \begin{column}{0.5\textwidth}
      \begin{listing}
        \mintc[highlightlines={9-11}]{src/ocaml/domain\_state\_backtrace.h}
        \caption{runtime/caml/domain\_state.h}
      \end{listing}
    \end{column}
  \end{columns}
  \footnotetext[1]{https://v2.ocaml.org/api/Printexc.html}
\end{frame}

\newcommand\listCamlRaiseExn[1][]{
  \listasm[firstline=702,lastline=725,#1]{../ocaml/runtime/amd64.S}{runtime/amd64.S:702}}

\begin{frame}{Backtraces}{Walk through \funcname{caml\_raise\_exn}}
  \begin{columns}[T]
    \begin{column}{0.5\textwidth}
      \begin{itemize}
        \item<1-> First checks whether exceptions backtrace should be recorded
        \item<1-> By checking the \funcarg{domain\_state->backtrace\_active} value
        \item<2-> If not, acts as \funcname{raise\_notrace} with \funcname{RESTORE\_EXN\_HANDLER\_OCAML}
          \begin{itemize}
            \item Set \regname{RSP} to \localname{domain\_state->exn\_handler} value
            \item Restore \localname{domain\_state->exn\_handler} from trap
            \item Jump with \funcname{ret} (same effect as using \funcname{pop} and \funcname{jmp})
          \end{itemize}
        \item<3-> Otherwise, switches to the C stack to be able to execute C code
        \item<4-> Calls \funcname{caml\_stash\_backtrace}, a C function provided by the runtime\ignorespaces
          \onslide<6->{, with
            \begin{itemize}
              \item \funcarg{exn}: exception value
              \item \funcarg{pc}: code address of the raising point
              \item \funcarg{sp}: stack address of the raising function
              \item \funcarg{trapsp}: stack address of the next handler
            \end{itemize}
          }
      \end{itemize}
      \bigskip
      \mintocaml[firstline=9,lastline=13,highlightlines=12]{../src/backtrace/backtrace.ml}
    \end{column}
    \begin{column}{0.5\textwidth}
      \raggedleft
      \onslide*<1,3-4>{
        \mintc[firstline=190,lastline=192]{../ocaml/runtime/amd64.S}
        \mintc[firstline=234,lastline=237]{../ocaml/runtime/amd64.S}
      }
      \onslide*<2>{
        \mintc[firstline=190,lastline=192,highlightlines={191-192}]{../ocaml/runtime/amd64.S}
        \mintc[firstline=234,lastline=237,highlightlines={235-237}]{../ocaml/runtime/amd64.S}
      }
      \onslide*<1>{\listCamlRaiseExn[highlightlines=705]{}}%
      \onslide*<2>{\listCamlRaiseExn[highlightlines={707-708}]{}}%
      \onslide*<3>{\listCamlRaiseExn[highlightlines={712-714}]{}}%
      \onslide*<4>{\listCamlRaiseExn[highlightlines={715-720}]{}}%
      \onslide*<5>{\providecommand\step{1}\input{diagrams/backtrace/backtrace.tex}}%
      \onslide*<6>{\providecommand\step{2}\input{diagrams/backtrace/backtrace.tex}}%
    \end{column}
  \end{columns}
\end{frame}

\newcommand\listStashBacktrace[1][]{
  \listc[firstline=91,lastline=120,#1]{../ocaml/runtime/backtrace\_nat.c}{runtime/backtrace\_nat.c:91}}

\begin{frame}{Backtraces}{Introducing \funcname{caml\_stash\_backtrace}}
  \begin{columns}[c]
    \begin{column}{0.5\textwidth}
      \minipage[c][0.66\textheight][s]{\columnwidth}
      \begin{itemize}
        \item<1-> A C function provided by the runtime (specific to native code)
        \item<2-> It scans the stack, between the stack pointer at the raising point up to the next trap, in order to collect the names of the detected function frames
          \onslide*<2>{
            \begin{itemize}
              \item And more information, like line number, column number...
            \end{itemize}
          }
        \item<3-> It uses the same technique and information as the Garbage Collector (GC) uses to scan the values live on the stack
          \onslide*<4>{
            \begin{itemize}
              \item \typename{frame\_descr} are at the core of stack unwinding
            \end{itemize}
          }
      \end{itemize}
      \vfill
      \mintocaml[firstline=9,lastline=13,highlightlines=12]{../src/backtrace/backtrace.ml}
      \endminipage
    \end{column}
    \begin{column}{0.5\textwidth}
      \onslide*<1>{\listStashBacktrace[highlightlines=91]{}}
      \onslide*<2>{\listStashBacktrace[highlightlines={91,117-118}]{}}
      \onslide*<3->{\listStashBacktrace[highlightlines={94,106,109-110}]{}}
    \end{column}
  \end{columns}
\end{frame}

\newcommand\listFrameDescr[1][]{
  \listocaml[firstline=111,lastline=124,#1]{../ocaml/asmcomp/emitaux.ml}{asmcomp/emitaux.ml:111}}
\newcommand\listDebugInfo[1][]{
  \listocaml[firstline=38,lastline=49,#1]{../ocaml/lambda/debuginfo.mli}{lambda/debuginfo.mli:38}}

\begin{frame}{Backtraces}{\typename{frame\_descr} and \typename{frame\_debuginfo} in a nutshell}
  \begin{columns}[c]
    \begin{column}{0.5\textwidth}
      \begin{itemize}
        \item<1-> For every return address in an OCaml program, there is a associated \typename{frame\_descr}
        \item<2-> A \typename{frame\_descr} specifies
        \begin{itemize}
          \item<2-> The size of the function stack frame
          \item<3-> Where live values are on the stack (so that the GC can mark them as live and prevent from being swept)
        \end{itemize}
        \item<4-> When compiled with debug information, \typename{frame\_descr} will be linked to a \typename{frame\_debuginfo}
        \begin{itemize}
          \item<5-> Contains information about the position in the source code (file name, line and column number)
        \end{itemize}
      \end{itemize}
    \end{column}
    \begin{column}{0.5\textwidth}
        \onslide*<1>{\listFrameDescr[highlightlines={118-122}]{}}
        \onslide*<2>{\listFrameDescr[highlightlines={120}]{}}
        \onslide*<3>{\listFrameDescr[highlightlines={121}]{}}
        \onslide*<4->{\listFrameDescr[highlightlines={122}]{}}
        \medskip
        \onslide*<1-3>{\listDebugInfo{}}
        \onslide*<4>{\listDebugInfo[highlightlines={38,49}]{}}
        \onslide*<5>{\listDebugInfo[highlightlines={39-46}]{}}
    \end{column}
  \end{columns}
\end{frame}

\begin{frame}{Backtraces}{Scanning the stack with \typename{frame\_descr}}
  \begin{columns}[c]
    \begin{column}{0.5\textwidth}
      \begin{itemize}
        \item Given the return address of the code that raises the exception by calling \funcname{caml\_raise\_exn} (\funcarg{pc} argument of \funcname{caml\_stash\_backtrace})
        \item And the position of the stack pointer (\funcarg{sp} argument of \funcname{caml\_stash\_backtrace})
        \item The frame size given by \typename{frame\_descr}.\funcarg{frame\_size}, allows to find the end of the previous frame
        \item And the return address of the caller of this function
        \bigskip
        \onslide<3->{
        \item Note that traps (here the reddish \funcname{ExnA}, \funcname{ExnB} and \funcname{ExnC} blocks) belong to the frame that installed them, and so the frame size changes when traps are removed
        }
      \end{itemize}
    \end{column}
    \begin{column}{0.5\textwidth}
      \raggedleft
      \onslide*<1>{\providecommand\step{2}\input{diagrams/backtrace/backtrace.tex}}%
      \onslide*<2>{\providecommand\step{1}\input{diagrams/backtrace/scanstack.tex}}%
      \onslide*<3>{\providecommand\step{2}\input{diagrams/backtrace/scanstack.tex}}%
      \onslide*<4>{\providecommand\step{3}\input{diagrams/backtrace/scanstack.tex}}%
      \onslide*<5>{\providecommand\step{4}\input{diagrams/backtrace/scanstack.tex}}%
      \onslide*<6>{\providecommand\step{5}\input{diagrams/backtrace/scanstack.tex}}%
    \end{column}
  \end{columns}
\end{frame}

% Duplicate of previous-previous slide
\begin{frame}{Backtraces}{Continuing on \funcname{caml\_stash\_backtrace}}
  \begin{columns}[c]
    \begin{column}{0.5\textwidth}
      \minipage[c][0.75\textheight][s]{\columnwidth}
      \begin{itemize}
          \color{gray}
        \item A C function provided by the runtime (specific to native code)
        \item It scans the stack, between the stack pointer at the raising point up to the next trap, in order to collect the names of the detected function frames
        \item It uses the same technique and information as the Garbage Collector (GC) uses to scan the values live on the stack
      \end{itemize}
      \begin{itemize}
        \item For each function frame detected, store a pointer to the \typename{frame\_descr} in the \localname{domain\_state->backtrace\_buffer} array, effectively constructing the backtrace
      \end{itemize}
      \onslide<2->{
        \begin{itemize}
            \smallskip
          \item Constructed backtrace:
            \funcname{definitely\_raise},
            \onslide<3->{\funcname{probably\_raise}}
        \end{itemize}
      }
      \vfill
      \mintocaml[firstline=9,lastline=13,highlightlines=12]{../src/backtrace/backtrace.ml}
      \endminipage
    \end{column}
    \begin{column}{0.5\textwidth}
      \raggedleft
      \onslide*<1>{\listStashBacktrace[highlightlines={114-115}]{}}
      \onslide*<2>{\providecommand\step{3}\input{diagrams/backtrace/backtrace.tex}}%
      \onslide*<3>{\providecommand\step{4}\input{diagrams/backtrace/backtrace.tex}}%
    \end{column}
  \end{columns}
\end{frame}

\begin{frame}{Backtraces}{Back to \funcname{caml\_raise\_exn}}
  \begin{columns}[c]
    \begin{column}{0.5\textwidth}
      \minipage[c][0.66\textheight][s]{\columnwidth}
      \begin{itemize}\color{gray}
        \item Checks wheteher exceptions backtrace should be recorded, by checking the \funcarg{domain\_state->backtrace\_active} value
        \item Switches to the C stack to be able to execute C code
        \item Calls \funcname{caml\_stash\_backtrace}, to collect the backtrace up to the next trap
      \end{itemize}
      \onslide*<2->{
        \begin{itemize}
          \item<2-> Executes the exception handler in \funcname{probably\_raise} with \funcname{RESTORE\_EXN\_HANDLER\_OCAML}
            \begin{itemize}
              \item Set \regname{RSP} to \localname{domain\_state->exn\_handler} value
              \item Restore \localname{domain\_state->exn\_handler} from trap
              \item Jump to the handler code with \funcname{ret}
            \end{itemize}
        \end{itemize}
      }
      \vfill
      \onslide*<1>{\mintocaml[firstline=9,lastline=13,highlightlines=12]{../src/backtrace/backtrace.ml}}
      \onslide*<2->{\mintocaml[firstline=15,lastline=21,highlightlines={19-21}]{../src/backtrace/backtrace.ml}}
      \endminipage
    \end{column}
    \begin{column}{0.5\textwidth}
      \centering
      \onslide*<1>{
        \mintc[firstline=190,lastline=192]{../ocaml/runtime/amd64.S}
        \mintc[firstline=234,lastline=237]{../ocaml/runtime/amd64.S}
      }
      \onslide*<2>{
        \mintc[firstline=190,lastline=192,highlightlines={191-192}]{../ocaml/runtime/amd64.S}
        \mintc[firstline=234,lastline=237,highlightlines={235-237}]{../ocaml/runtime/amd64.S}
      }
      \onslide*<1>{\listCamlRaiseExn[highlightlines=720]{}}
      \onslide*<2>{\listCamlRaiseExn[highlightlines=722]{}}
      \onslide*<3->{\providecommand\step{5}\input{diagrams/backtrace/backtrace.tex}}
    \end{column}
  \end{columns}
\end{frame}

\begin{frame}{Backtraces}{\funcname{caml\_probably\_raise}'s handler doesn't catch \typename{ExnC}}
  \begin{columns}[c]
    \begin{column}{0.45\textwidth}
      \minipage[c][0.75\textheight][s]{\columnwidth}
      \begin{itemize}
        \item<1-> \funcname{match \dots with}\footnotemark pattern matching can also match exceptions
        \item<1-> The CMM of \funcname{probably\_raise} shows that this \funcname{match \dots with} is implemented with a pattern matching and a \funcname{try \dots with}
        \item<1-> But this one only matches on \typename{ExnA} exception
        \item<2-> Since the pattern matching fails to catch the exception, \funcname{reraise} is called with our current \typename{ExnC} exception
        \item<3-> Disassembly shows that \funcname{reraise} is actually a call to \funcname{caml\_reraise\_exn}, which is also an assembly function provided by the runtime
      \end{itemize}
      \vfill
      \mintocaml[firstline=15,lastline=21,highlightlines={18-21}]{../src/backtrace/backtrace.ml}
      \endminipage
    \end{column}
    \begin{column}{0.55\textwidth}
      \centering
      \onslide*<1>{\mintcmm[highlightlines={10-18}]{../src/backtrace/camlBacktrace\_\_probably\_raise\_351.cmm}}
      \onslide*<2>{\listcmm[highlightlines=18]{../src/backtrace/camlBacktrace\_\_probably\_raise_351.cmm}{CMM of probably\_raise}}
      \onslide*<3>{\mintobjdump[firstline=5,lastline=35,highlightlines=31]{../src/backtrace/camlBacktrace\_\_probably\_raise_351.objdump}}
    \end{column}
  \end{columns}
  \only<1>{\footnotetext[1]{https://blog.janestreet.com/pattern-matching-and-exception-handling-unite/}}
\end{frame}

\newcommand\listCamlReraiseExn[1][]{
  \listasm[firstline=727,lastline=734,#1]{../ocaml/runtime/amd64.S}{runtime/amd64.S:727}}

\begin{frame}{Backtraces}{Introducing \funcname{caml\_reraise\_exn}}
  \begin{columns}[c]
    \begin{column}{0.5\textwidth}
      \minipage[c][0.66\textheight][s]{\columnwidth}
      \begin{itemize}
        \item<1-> The exception have to be reraised to the next handler
        \item<2-> First, checks if the backtrace should be collected with \funcarg{domain\_state->backtrace\_active}
        \only<3>{
        \begin{itemize}
            \item If not, behaves like a \funcname{raise\_notrace} with \funcname{RESTORE\_EXN\_HANDLER\_OCAML}
        \end{itemize}
        }
        \item<4-> Jumps into \funcname{caml\_raise\_exn}
        \item<5-> Calls \funcname{caml\_stash\_backtrace} again, to scan the next part of the stack: from the trap just pop-ed to the next trap
      \end{itemize}
      \vfill
      \mintocaml[firstline=15,lastline=21,highlightlines={18-21}]{../src/backtrace/backtrace.ml}
      \endminipage
    \end{column}
    \begin{column}{0.5\textwidth}
      \raggedleft
      \onslide*<1>{\listCamlReraiseExn{}}
      \onslide*<2>{\listCamlReraiseExn[highlightlines=729]{}}
      \onslide*<3>{
        \mintc[firstline=190,lastline=192,highlightlines={191-192}]{../ocaml/runtime/amd64.S}
        \mintc[firstline=234,lastline=237,highlightlines={235-237}]{../ocaml/runtime/amd64.S}
        \medskip
        \listCamlReraiseExn[highlightlines=731]{}
      }
      \onslide*<4-5>{
        % TODO refactor with \listCamlReraiseExn
        \mintasm[firstline=727,lastline=734,highlightlines=730]{../ocaml/runtime/amd64.S}
        \medskip
      }
      \onslide*<4>{\listCamlRaiseExn[highlightlines=711]{}}
      \onslide*<5>{\listCamlRaiseExn[highlightlines=720]{}}
    \end{column}
  \end{columns}
\end{frame}

\begin{frame}{Backtraces}{Backtrace collection continues}
  \begin{columns}[c]
    \begin{column}{0.55\textwidth}
      \minipage[c][0.75\textheight][s]{\columnwidth}
      \begin{itemize}
        \item<1-> \funcname{caml\_stash\_backtrace} scans the older part of the stack, up to the next trap
        \item<6-> Controls jumps to the nested exception handler in \funcname{maybe\_raise}, which matches only on \typename{ExnB}
        \item<7-> \funcname{caml\_reraise\_exn} is called again, backtrace collection resumes with \funcname{caml\_stash\_backtrace}
      \end{itemize}
      \medskip
      \begin{itemize}
        \item<2-> Constructed backtrace:
        \begin{itemize}
          \item <2-> \funcname{definitely\_raise}
          \item <2-> \funcname{probably\_raise}
          \item <3-> \funcname{probably\_raise} (re-raise)
          \item <4-> \funcname{maybe\_raise}
          \item <5-> \funcname{main}
          \item <8-> \funcname{main} (re-raise)
        \end{itemize}
      \end{itemize}
      \vfill
      \onslide*<1-5>{\mintocaml[firstline=15,lastline=21]{../src/backtrace/backtrace.ml}}
      \onslide*<6>{\mintocaml[firstline=28,lastline=34,highlightlines=31]{../src/backtrace/backtrace.ml}}
      \onslide*<7->{\mintocaml[firstline=28,lastline=34,highlightlines=33]{../src/backtrace/backtrace.ml}}
      \endminipage
    \end{column}
    \begin{column}{0.45\textwidth}
      \raggedleft
      \onslide*<1>{\providecommand\step{6}\input{diagrams/backtrace/backtrace.tex}}%
      \onslide*<2>{\providecommand\step{6}\input{diagrams/backtrace/backtrace.tex}}%
      \onslide*<3>{\providecommand\step{7}\input{diagrams/backtrace/backtrace.tex}}%
      \onslide*<4>{\providecommand\step{8}\input{diagrams/backtrace/backtrace.tex}}%
      \onslide*<5>{\providecommand\step{9}\input{diagrams/backtrace/backtrace.tex}}%
      \onslide*<6>{\providecommand\step{10}\input{diagrams/backtrace/backtrace.tex}}%
      \onslide*<7>{\providecommand\step{11}\input{diagrams/backtrace/backtrace.tex}}%
      \onslide*<8>{\providecommand\step{12}\input{diagrams/backtrace/backtrace.tex}}%
      \onslide*<9>{\providecommand\step{13}\input{diagrams/backtrace/backtrace.tex}}%
    \end{column}
  \end{columns}
\end{frame}

\begin{frame}{Backtraces}{Printing the backtrace}
  \begin{columns}[c]
    \begin{column}{0.45\textwidth}
      \minipage[c][0.66\textheight][s]{\columnwidth}
      \begin{itemize}
        \item<1-> The exception backtrace can now be printed to \funcarg{stdout} with \funcname{Printexc.print_backtrace}
        \item<2-> First the exception backtrace is retrieved into an OCaml array with \funcname{get_raw_backtrace }
        \item<3-> Which is implemented as the \funcname{caml_get_exception_raw_backtrace} C function in the runtime
        \item<4-> And finally printed with \funcname{print_exception_backtrace}
      \end{itemize}
      \vfill
      \mintocaml[firstline=28,lastline=34,highlightlines=33]{../src/backtrace/backtrace.ml}
      \endminipage
    \end{column}
    \begin{column}{0.55\textwidth}
      \onslide*<1,2>{
        \mintocaml[firstline=100,lastline=101]{../ocaml/stdlib/printexc.ml}
      }
      \only<1>{
        \listocaml[firstline=170,lastline=172]{../ocaml/stdlib/printexc.ml}{stdlib/printexc.ml:170}
      }
      \only<2>{
        \listocaml[firstline=170,lastline=172,highlightlines=172]{../ocaml/stdlib/printexc.ml}{stdlib/printexc.ml:170}
      }
      \only<3>{
        \mintc[firstline=154,lastline=156]{../ocaml/runtime/backtrace.c}
        \listc[firstline=171,lastline=192,highlightlines={184-187}]{../ocaml/runtime/backtrace.c}{runtime/backtrace.c:154}
      }
      \only<4>{
        \listocaml[firstline=155,lastline=165]{../ocaml/stdlib/printexc.ml}{stdlib/printexc.ml:55}
      }
    \end{column}
  \end{columns}
\end{frame}


\subsection{The multiple kinds of raise}
\frameSubsection{}{}

\begin{frame}{Backtraces}{When backtraces are not needed}
  \begin{columns}[c]
    \begin{column}{0.45\textwidth}
      \minipage[c][0.66\textheight][s]{\columnwidth}
      \begin{itemize}
        \item<1-> What if the backtrace for an exception is never needed?
        \item<2-> It's possible to raise the exception explicitly with \funcname{raise\_notrace} to make raising as cheap as possible
        \item<3-> An example of use in the \funcname{dynlink} library of the OCaml compiler
      \end{itemize}
      \vfill
      \onslide<3->{
      \listocaml[firstline=205,lastline=209,highlightlines=207]{../ocaml/otherlibs/dynlink/dynlink\_compilerlibs/misc.ml}{otherlibs/dynlink/dynlink\_compilerlibs/misc.ml:205}
      }
      \endminipage
    \end{column}
    \begin{column}{0.55\textwidth}
    \only<4>{
      \mintcmm[highlightlines=3]{../src/notrace/camlNotrace\_\_fun_347.cmm}
      \listcmm{../src/notrace/camlNotrace\_\_all\_somes\_327.cmm}{all\_somes}
    }
    \onslide*<5->{
      \listobjdump[highlightlines={6-9}]{../src/notrace/camlNotrace\_\_fun_347.objdump}{anonymous function from \funcname{all\_somes}}
    }
    \end{column}
  \end{columns}
\end{frame}

\begin{frame}{Backtraces}{Summary table of the multiple kinds of raise}
  \begin{itemize}
    \item When debugging information is enabled and if the backtrace of an exception isn't needed, it's possible to raise with \funcname{raise\_notrace} for performance
    \item When debugging information is \emph{not} enabled, the compiler optimises \funcname{raise} calls to inline assembly (same effect as using \funcname{raise\_notrace})
  \end{itemize}
  \bigskip
  \centering
  \begin{tabular}{| l | c | c | c | c |}
    \hline
    \parbox{10em}{Debug information \\ (\funcarg{-g} flag)}  & \multicolumn{2}{|c|}{Disabled}                                     & \multicolumn{2}{|c|}{Enabled} \\
    \hline
    Raise kind                                               & \funcname{raise} & \funcname{raise\_notrace}                       & \funcname{raise} & \funcname{raise\_notrace} \\
    \hline
    Compilation                                              & \parbox{8em}{inline assembly\\ (optimization)} & inline assembly   & \funcname{caml\_raise\_exn} & inline assembly \\
    \hline
  \end{tabular}
\end{frame}


%
% Takeaway #5
%
\subsection*{Takeaway \#5}
\frameSubsectionTakeaway{}
\againframe<5>{takeaway}
}%
    \end{column}
  \end{columns}
\end{frame}

\begin{frame}{Backtraces}{Back to \funcname{caml\_raise\_exn}}
  \begin{columns}[c]
    \begin{column}{0.5\textwidth}
      \minipage[c][0.66\textheight][s]{\columnwidth}
      \begin{itemize}\color{gray}
        \item Checks wheteher exceptions backtrace should be recorded, by checking the \funcarg{domain\_state->backtrace\_active} value
        \item Switches to the C stack to be able to execute C code
        \item Calls \funcname{caml\_stash\_backtrace}, to collect the backtrace up to the next trap
      \end{itemize}
      \onslide*<2->{
        \begin{itemize}
          \item<2-> Executes the exception handler in \funcname{probably\_raise} with \funcname{RESTORE\_EXN\_HANDLER\_OCAML}
            \begin{itemize}
              \item Set \regname{RSP} to \localname{domain\_state->exn\_handler} value
              \item Restore \localname{domain\_state->exn\_handler} from trap
              \item Jump to the handler code with \funcname{ret}
            \end{itemize}
        \end{itemize}
      }
      \vfill
      \onslide*<1>{\mintocaml[firstline=9,lastline=13,highlightlines=12]{../src/backtrace/backtrace.ml}}
      \onslide*<2->{\mintocaml[firstline=15,lastline=21,highlightlines={19-21}]{../src/backtrace/backtrace.ml}}
      \endminipage
    \end{column}
    \begin{column}{0.5\textwidth}
      \centering
      \onslide*<1>{
        \mintc[firstline=190,lastline=192]{../ocaml/runtime/amd64.S}
        \mintc[firstline=234,lastline=237]{../ocaml/runtime/amd64.S}
      }
      \onslide*<2>{
        \mintc[firstline=190,lastline=192,highlightlines={191-192}]{../ocaml/runtime/amd64.S}
        \mintc[firstline=234,lastline=237,highlightlines={235-237}]{../ocaml/runtime/amd64.S}
      }
      \onslide*<1>{\listCamlRaiseExn[highlightlines=720]{}}
      \onslide*<2>{\listCamlRaiseExn[highlightlines=722]{}}
      \onslide*<3->{\providecommand\step{5}%
% Backtraces
%
\subsection{One advantage of raising with debugging information}
\frameSubsection{}{}

\begin{frame}{Backtraces}{Debugging information \& source code}
  \begin{columns}[c]
    \begin{column}{0.5\textwidth}
      \begin{itemize}
        \item Raising an exception is fun and games, but how do we know where the exception originated? $\rightarrow$ With the backtrace associated with the exception!
        \item In order to get a backtrace from the point where an exception was raised, it's necessary to compile with debugging information enabled (\funcarg{-g} flag for the compiler)
        \item Same example as in the `Nested handlers' section
      \end{itemize}
    \end{column}
    \begin{column}{0.5\textwidth}
      \listocaml[firstline=9,lastline=34]{../src/backtrace/backtrace.ml}{backtrace/backtrace.ml}
    \end{column}
  \end{columns}
\end{frame}

\begin{frame}{Backtraces}{CMM and disassembly of \funcname{definitely\_raise}}
  \begin{columns}[c]
    \begin{column}{0.5\textwidth}
      \minipage[c][0.66\textheight][s]{\columnwidth}
      \begin{itemize}
        \item<1-> An effect of compiling with debugging information (\funcarg{-g}): the CMM no longer uses \funcname{raise\_notrace} to raise the exception but \funcname{raise} instead
        \item<2-> Disassembly shows that CMM \funcname{raise} is actually a call to \funcname{caml\_raise\_exn}, a function in assembler provided by the runtime
      \end{itemize}
      \vfill
      \mintocaml[firstline=9,lastline=13,highlightlines=12]{../src/backtrace/backtrace.ml}
      \endminipage
    \end{column}
    \begin{column}{0.5\textwidth}
      \onslide*<1>{
        \mintcmm[highlightlines=5]{../src/backtrace/camlBacktrace\_\_definitely\_raise\_347.cmm}
      }
      \onslide*<2>{
        \mintobjdump[firstline=2,highlightlines=14]{../src/backtrace/camlBacktrace\_\_definitely\_raise\_347.objdump}
      }
    \end{column}
  \end{columns}
\end{frame}

\begin{frame}{Backtraces}{Introducing \funcname{caml\_raise\_exn}}
  \begin{columns}[c]
    \begin{column}{0.5\textwidth}
      \minipage[c][0.66\textheight][s]{\columnwidth}
      \onslide*<1>{
        \begin{itemize}
          \item Raising the exception is performed using \funcname{caml\_raise\_exn} which is
            \begin{itemize}
              \item An assembly function provided by the runtime
              \item Architecture specific
            \end{itemize}
          \item Let's see what it does differently from \funcname{raise\_notrace}\dots
          \item We are about to raise \typename{ExnC} with \funcname{caml\_raise\_exn} from \funcname{definitely\_raise}
        \end{itemize}
      }
      \onslide*<2>{
        \mintobjdump[firstline=2,highlightlines=14]{../src/backtrace/camlBacktrace\_\_definitely\_raise\_347.objdump}
      }
      \vfill
      \mintocaml[firstline=9,lastline=13,highlightlines=12]{../src/backtrace/backtrace.ml}
      \endminipage
    \end{column}
    \begin{column}{0.5\textwidth}
      \centering
      \providecommand\step{1}\input{diagrams/backtrace/backtrace.tex}
    \end{column}
  \end{columns}
\end{frame}

\begin{frame}{Backtraces}{But before that, we need more fields from \typename{caml\_domain\_state}}
  \begin{columns}
    \begin{column}{0.5\textwidth}
      \begin{itemize}
        \item Remember \typename{caml\_domain\_state} the per-domain runtime state?
        \item Some other members are of interest to us now\dots
        \item \localname{backtrace\_active} is used to know if the runtime should collect backtraces
        \item \localname{backtrace\_active} can be set by
          \begin{itemize}
            \item The \funcarg{b} flag in the \funcname{OCAMLRUNPARAM} environment variable
            \item \mintinline{ocaml}{Printexc.record_backtrace : bool -> unit} \footnotemark
          \end{itemize}
      \end{itemize}
    \end{column}
    \begin{column}{0.5\textwidth}
      \begin{listing}
        \mintc[highlightlines={9-11}]{src/ocaml/domain\_state\_backtrace.h}
        \caption{runtime/caml/domain\_state.h}
      \end{listing}
    \end{column}
  \end{columns}
  \footnotetext[1]{https://v2.ocaml.org/api/Printexc.html}
\end{frame}

\newcommand\listCamlRaiseExn[1][]{
  \listasm[firstline=702,lastline=725,#1]{../ocaml/runtime/amd64.S}{runtime/amd64.S:702}}

\begin{frame}{Backtraces}{Walk through \funcname{caml\_raise\_exn}}
  \begin{columns}[T]
    \begin{column}{0.5\textwidth}
      \begin{itemize}
        \item<1-> First checks whether exceptions backtrace should be recorded
        \item<1-> By checking the \funcarg{domain\_state->backtrace\_active} value
        \item<2-> If not, acts as \funcname{raise\_notrace} with \funcname{RESTORE\_EXN\_HANDLER\_OCAML}
          \begin{itemize}
            \item Set \regname{RSP} to \localname{domain\_state->exn\_handler} value
            \item Restore \localname{domain\_state->exn\_handler} from trap
            \item Jump with \funcname{ret} (same effect as using \funcname{pop} and \funcname{jmp})
          \end{itemize}
        \item<3-> Otherwise, switches to the C stack to be able to execute C code
        \item<4-> Calls \funcname{caml\_stash\_backtrace}, a C function provided by the runtime\ignorespaces
          \onslide<6->{, with
            \begin{itemize}
              \item \funcarg{exn}: exception value
              \item \funcarg{pc}: code address of the raising point
              \item \funcarg{sp}: stack address of the raising function
              \item \funcarg{trapsp}: stack address of the next handler
            \end{itemize}
          }
      \end{itemize}
      \bigskip
      \mintocaml[firstline=9,lastline=13,highlightlines=12]{../src/backtrace/backtrace.ml}
    \end{column}
    \begin{column}{0.5\textwidth}
      \raggedleft
      \onslide*<1,3-4>{
        \mintc[firstline=190,lastline=192]{../ocaml/runtime/amd64.S}
        \mintc[firstline=234,lastline=237]{../ocaml/runtime/amd64.S}
      }
      \onslide*<2>{
        \mintc[firstline=190,lastline=192,highlightlines={191-192}]{../ocaml/runtime/amd64.S}
        \mintc[firstline=234,lastline=237,highlightlines={235-237}]{../ocaml/runtime/amd64.S}
      }
      \onslide*<1>{\listCamlRaiseExn[highlightlines=705]{}}%
      \onslide*<2>{\listCamlRaiseExn[highlightlines={707-708}]{}}%
      \onslide*<3>{\listCamlRaiseExn[highlightlines={712-714}]{}}%
      \onslide*<4>{\listCamlRaiseExn[highlightlines={715-720}]{}}%
      \onslide*<5>{\providecommand\step{1}\input{diagrams/backtrace/backtrace.tex}}%
      \onslide*<6>{\providecommand\step{2}\input{diagrams/backtrace/backtrace.tex}}%
    \end{column}
  \end{columns}
\end{frame}

\newcommand\listStashBacktrace[1][]{
  \listc[firstline=91,lastline=120,#1]{../ocaml/runtime/backtrace\_nat.c}{runtime/backtrace\_nat.c:91}}

\begin{frame}{Backtraces}{Introducing \funcname{caml\_stash\_backtrace}}
  \begin{columns}[c]
    \begin{column}{0.5\textwidth}
      \minipage[c][0.66\textheight][s]{\columnwidth}
      \begin{itemize}
        \item<1-> A C function provided by the runtime (specific to native code)
        \item<2-> It scans the stack, between the stack pointer at the raising point up to the next trap, in order to collect the names of the detected function frames
          \onslide*<2>{
            \begin{itemize}
              \item And more information, like line number, column number...
            \end{itemize}
          }
        \item<3-> It uses the same technique and information as the Garbage Collector (GC) uses to scan the values live on the stack
          \onslide*<4>{
            \begin{itemize}
              \item \typename{frame\_descr} are at the core of stack unwinding
            \end{itemize}
          }
      \end{itemize}
      \vfill
      \mintocaml[firstline=9,lastline=13,highlightlines=12]{../src/backtrace/backtrace.ml}
      \endminipage
    \end{column}
    \begin{column}{0.5\textwidth}
      \onslide*<1>{\listStashBacktrace[highlightlines=91]{}}
      \onslide*<2>{\listStashBacktrace[highlightlines={91,117-118}]{}}
      \onslide*<3->{\listStashBacktrace[highlightlines={94,106,109-110}]{}}
    \end{column}
  \end{columns}
\end{frame}

\newcommand\listFrameDescr[1][]{
  \listocaml[firstline=111,lastline=124,#1]{../ocaml/asmcomp/emitaux.ml}{asmcomp/emitaux.ml:111}}
\newcommand\listDebugInfo[1][]{
  \listocaml[firstline=38,lastline=49,#1]{../ocaml/lambda/debuginfo.mli}{lambda/debuginfo.mli:38}}

\begin{frame}{Backtraces}{\typename{frame\_descr} and \typename{frame\_debuginfo} in a nutshell}
  \begin{columns}[c]
    \begin{column}{0.5\textwidth}
      \begin{itemize}
        \item<1-> For every return address in an OCaml program, there is a associated \typename{frame\_descr}
        \item<2-> A \typename{frame\_descr} specifies
        \begin{itemize}
          \item<2-> The size of the function stack frame
          \item<3-> Where live values are on the stack (so that the GC can mark them as live and prevent from being swept)
        \end{itemize}
        \item<4-> When compiled with debug information, \typename{frame\_descr} will be linked to a \typename{frame\_debuginfo}
        \begin{itemize}
          \item<5-> Contains information about the position in the source code (file name, line and column number)
        \end{itemize}
      \end{itemize}
    \end{column}
    \begin{column}{0.5\textwidth}
        \onslide*<1>{\listFrameDescr[highlightlines={118-122}]{}}
        \onslide*<2>{\listFrameDescr[highlightlines={120}]{}}
        \onslide*<3>{\listFrameDescr[highlightlines={121}]{}}
        \onslide*<4->{\listFrameDescr[highlightlines={122}]{}}
        \medskip
        \onslide*<1-3>{\listDebugInfo{}}
        \onslide*<4>{\listDebugInfo[highlightlines={38,49}]{}}
        \onslide*<5>{\listDebugInfo[highlightlines={39-46}]{}}
    \end{column}
  \end{columns}
\end{frame}

\begin{frame}{Backtraces}{Scanning the stack with \typename{frame\_descr}}
  \begin{columns}[c]
    \begin{column}{0.5\textwidth}
      \begin{itemize}
        \item Given the return address of the code that raises the exception by calling \funcname{caml\_raise\_exn} (\funcarg{pc} argument of \funcname{caml\_stash\_backtrace})
        \item And the position of the stack pointer (\funcarg{sp} argument of \funcname{caml\_stash\_backtrace})
        \item The frame size given by \typename{frame\_descr}.\funcarg{frame\_size}, allows to find the end of the previous frame
        \item And the return address of the caller of this function
        \bigskip
        \onslide<3->{
        \item Note that traps (here the reddish \funcname{ExnA}, \funcname{ExnB} and \funcname{ExnC} blocks) belong to the frame that installed them, and so the frame size changes when traps are removed
        }
      \end{itemize}
    \end{column}
    \begin{column}{0.5\textwidth}
      \raggedleft
      \onslide*<1>{\providecommand\step{2}\input{diagrams/backtrace/backtrace.tex}}%
      \onslide*<2>{\providecommand\step{1}\input{diagrams/backtrace/scanstack.tex}}%
      \onslide*<3>{\providecommand\step{2}\input{diagrams/backtrace/scanstack.tex}}%
      \onslide*<4>{\providecommand\step{3}\input{diagrams/backtrace/scanstack.tex}}%
      \onslide*<5>{\providecommand\step{4}\input{diagrams/backtrace/scanstack.tex}}%
      \onslide*<6>{\providecommand\step{5}\input{diagrams/backtrace/scanstack.tex}}%
    \end{column}
  \end{columns}
\end{frame}

% Duplicate of previous-previous slide
\begin{frame}{Backtraces}{Continuing on \funcname{caml\_stash\_backtrace}}
  \begin{columns}[c]
    \begin{column}{0.5\textwidth}
      \minipage[c][0.75\textheight][s]{\columnwidth}
      \begin{itemize}
          \color{gray}
        \item A C function provided by the runtime (specific to native code)
        \item It scans the stack, between the stack pointer at the raising point up to the next trap, in order to collect the names of the detected function frames
        \item It uses the same technique and information as the Garbage Collector (GC) uses to scan the values live on the stack
      \end{itemize}
      \begin{itemize}
        \item For each function frame detected, store a pointer to the \typename{frame\_descr} in the \localname{domain\_state->backtrace\_buffer} array, effectively constructing the backtrace
      \end{itemize}
      \onslide<2->{
        \begin{itemize}
            \smallskip
          \item Constructed backtrace:
            \funcname{definitely\_raise},
            \onslide<3->{\funcname{probably\_raise}}
        \end{itemize}
      }
      \vfill
      \mintocaml[firstline=9,lastline=13,highlightlines=12]{../src/backtrace/backtrace.ml}
      \endminipage
    \end{column}
    \begin{column}{0.5\textwidth}
      \raggedleft
      \onslide*<1>{\listStashBacktrace[highlightlines={114-115}]{}}
      \onslide*<2>{\providecommand\step{3}\input{diagrams/backtrace/backtrace.tex}}%
      \onslide*<3>{\providecommand\step{4}\input{diagrams/backtrace/backtrace.tex}}%
    \end{column}
  \end{columns}
\end{frame}

\begin{frame}{Backtraces}{Back to \funcname{caml\_raise\_exn}}
  \begin{columns}[c]
    \begin{column}{0.5\textwidth}
      \minipage[c][0.66\textheight][s]{\columnwidth}
      \begin{itemize}\color{gray}
        \item Checks wheteher exceptions backtrace should be recorded, by checking the \funcarg{domain\_state->backtrace\_active} value
        \item Switches to the C stack to be able to execute C code
        \item Calls \funcname{caml\_stash\_backtrace}, to collect the backtrace up to the next trap
      \end{itemize}
      \onslide*<2->{
        \begin{itemize}
          \item<2-> Executes the exception handler in \funcname{probably\_raise} with \funcname{RESTORE\_EXN\_HANDLER\_OCAML}
            \begin{itemize}
              \item Set \regname{RSP} to \localname{domain\_state->exn\_handler} value
              \item Restore \localname{domain\_state->exn\_handler} from trap
              \item Jump to the handler code with \funcname{ret}
            \end{itemize}
        \end{itemize}
      }
      \vfill
      \onslide*<1>{\mintocaml[firstline=9,lastline=13,highlightlines=12]{../src/backtrace/backtrace.ml}}
      \onslide*<2->{\mintocaml[firstline=15,lastline=21,highlightlines={19-21}]{../src/backtrace/backtrace.ml}}
      \endminipage
    \end{column}
    \begin{column}{0.5\textwidth}
      \centering
      \onslide*<1>{
        \mintc[firstline=190,lastline=192]{../ocaml/runtime/amd64.S}
        \mintc[firstline=234,lastline=237]{../ocaml/runtime/amd64.S}
      }
      \onslide*<2>{
        \mintc[firstline=190,lastline=192,highlightlines={191-192}]{../ocaml/runtime/amd64.S}
        \mintc[firstline=234,lastline=237,highlightlines={235-237}]{../ocaml/runtime/amd64.S}
      }
      \onslide*<1>{\listCamlRaiseExn[highlightlines=720]{}}
      \onslide*<2>{\listCamlRaiseExn[highlightlines=722]{}}
      \onslide*<3->{\providecommand\step{5}\input{diagrams/backtrace/backtrace.tex}}
    \end{column}
  \end{columns}
\end{frame}

\begin{frame}{Backtraces}{\funcname{caml\_probably\_raise}'s handler doesn't catch \typename{ExnC}}
  \begin{columns}[c]
    \begin{column}{0.45\textwidth}
      \minipage[c][0.75\textheight][s]{\columnwidth}
      \begin{itemize}
        \item<1-> \funcname{match \dots with}\footnotemark pattern matching can also match exceptions
        \item<1-> The CMM of \funcname{probably\_raise} shows that this \funcname{match \dots with} is implemented with a pattern matching and a \funcname{try \dots with}
        \item<1-> But this one only matches on \typename{ExnA} exception
        \item<2-> Since the pattern matching fails to catch the exception, \funcname{reraise} is called with our current \typename{ExnC} exception
        \item<3-> Disassembly shows that \funcname{reraise} is actually a call to \funcname{caml\_reraise\_exn}, which is also an assembly function provided by the runtime
      \end{itemize}
      \vfill
      \mintocaml[firstline=15,lastline=21,highlightlines={18-21}]{../src/backtrace/backtrace.ml}
      \endminipage
    \end{column}
    \begin{column}{0.55\textwidth}
      \centering
      \onslide*<1>{\mintcmm[highlightlines={10-18}]{../src/backtrace/camlBacktrace\_\_probably\_raise\_351.cmm}}
      \onslide*<2>{\listcmm[highlightlines=18]{../src/backtrace/camlBacktrace\_\_probably\_raise_351.cmm}{CMM of probably\_raise}}
      \onslide*<3>{\mintobjdump[firstline=5,lastline=35,highlightlines=31]{../src/backtrace/camlBacktrace\_\_probably\_raise_351.objdump}}
    \end{column}
  \end{columns}
  \only<1>{\footnotetext[1]{https://blog.janestreet.com/pattern-matching-and-exception-handling-unite/}}
\end{frame}

\newcommand\listCamlReraiseExn[1][]{
  \listasm[firstline=727,lastline=734,#1]{../ocaml/runtime/amd64.S}{runtime/amd64.S:727}}

\begin{frame}{Backtraces}{Introducing \funcname{caml\_reraise\_exn}}
  \begin{columns}[c]
    \begin{column}{0.5\textwidth}
      \minipage[c][0.66\textheight][s]{\columnwidth}
      \begin{itemize}
        \item<1-> The exception have to be reraised to the next handler
        \item<2-> First, checks if the backtrace should be collected with \funcarg{domain\_state->backtrace\_active}
        \only<3>{
        \begin{itemize}
            \item If not, behaves like a \funcname{raise\_notrace} with \funcname{RESTORE\_EXN\_HANDLER\_OCAML}
        \end{itemize}
        }
        \item<4-> Jumps into \funcname{caml\_raise\_exn}
        \item<5-> Calls \funcname{caml\_stash\_backtrace} again, to scan the next part of the stack: from the trap just pop-ed to the next trap
      \end{itemize}
      \vfill
      \mintocaml[firstline=15,lastline=21,highlightlines={18-21}]{../src/backtrace/backtrace.ml}
      \endminipage
    \end{column}
    \begin{column}{0.5\textwidth}
      \raggedleft
      \onslide*<1>{\listCamlReraiseExn{}}
      \onslide*<2>{\listCamlReraiseExn[highlightlines=729]{}}
      \onslide*<3>{
        \mintc[firstline=190,lastline=192,highlightlines={191-192}]{../ocaml/runtime/amd64.S}
        \mintc[firstline=234,lastline=237,highlightlines={235-237}]{../ocaml/runtime/amd64.S}
        \medskip
        \listCamlReraiseExn[highlightlines=731]{}
      }
      \onslide*<4-5>{
        % TODO refactor with \listCamlReraiseExn
        \mintasm[firstline=727,lastline=734,highlightlines=730]{../ocaml/runtime/amd64.S}
        \medskip
      }
      \onslide*<4>{\listCamlRaiseExn[highlightlines=711]{}}
      \onslide*<5>{\listCamlRaiseExn[highlightlines=720]{}}
    \end{column}
  \end{columns}
\end{frame}

\begin{frame}{Backtraces}{Backtrace collection continues}
  \begin{columns}[c]
    \begin{column}{0.55\textwidth}
      \minipage[c][0.75\textheight][s]{\columnwidth}
      \begin{itemize}
        \item<1-> \funcname{caml\_stash\_backtrace} scans the older part of the stack, up to the next trap
        \item<6-> Controls jumps to the nested exception handler in \funcname{maybe\_raise}, which matches only on \typename{ExnB}
        \item<7-> \funcname{caml\_reraise\_exn} is called again, backtrace collection resumes with \funcname{caml\_stash\_backtrace}
      \end{itemize}
      \medskip
      \begin{itemize}
        \item<2-> Constructed backtrace:
        \begin{itemize}
          \item <2-> \funcname{definitely\_raise}
          \item <2-> \funcname{probably\_raise}
          \item <3-> \funcname{probably\_raise} (re-raise)
          \item <4-> \funcname{maybe\_raise}
          \item <5-> \funcname{main}
          \item <8-> \funcname{main} (re-raise)
        \end{itemize}
      \end{itemize}
      \vfill
      \onslide*<1-5>{\mintocaml[firstline=15,lastline=21]{../src/backtrace/backtrace.ml}}
      \onslide*<6>{\mintocaml[firstline=28,lastline=34,highlightlines=31]{../src/backtrace/backtrace.ml}}
      \onslide*<7->{\mintocaml[firstline=28,lastline=34,highlightlines=33]{../src/backtrace/backtrace.ml}}
      \endminipage
    \end{column}
    \begin{column}{0.45\textwidth}
      \raggedleft
      \onslide*<1>{\providecommand\step{6}\input{diagrams/backtrace/backtrace.tex}}%
      \onslide*<2>{\providecommand\step{6}\input{diagrams/backtrace/backtrace.tex}}%
      \onslide*<3>{\providecommand\step{7}\input{diagrams/backtrace/backtrace.tex}}%
      \onslide*<4>{\providecommand\step{8}\input{diagrams/backtrace/backtrace.tex}}%
      \onslide*<5>{\providecommand\step{9}\input{diagrams/backtrace/backtrace.tex}}%
      \onslide*<6>{\providecommand\step{10}\input{diagrams/backtrace/backtrace.tex}}%
      \onslide*<7>{\providecommand\step{11}\input{diagrams/backtrace/backtrace.tex}}%
      \onslide*<8>{\providecommand\step{12}\input{diagrams/backtrace/backtrace.tex}}%
      \onslide*<9>{\providecommand\step{13}\input{diagrams/backtrace/backtrace.tex}}%
    \end{column}
  \end{columns}
\end{frame}

\begin{frame}{Backtraces}{Printing the backtrace}
  \begin{columns}[c]
    \begin{column}{0.45\textwidth}
      \minipage[c][0.66\textheight][s]{\columnwidth}
      \begin{itemize}
        \item<1-> The exception backtrace can now be printed to \funcarg{stdout} with \funcname{Printexc.print_backtrace}
        \item<2-> First the exception backtrace is retrieved into an OCaml array with \funcname{get_raw_backtrace }
        \item<3-> Which is implemented as the \funcname{caml_get_exception_raw_backtrace} C function in the runtime
        \item<4-> And finally printed with \funcname{print_exception_backtrace}
      \end{itemize}
      \vfill
      \mintocaml[firstline=28,lastline=34,highlightlines=33]{../src/backtrace/backtrace.ml}
      \endminipage
    \end{column}
    \begin{column}{0.55\textwidth}
      \onslide*<1,2>{
        \mintocaml[firstline=100,lastline=101]{../ocaml/stdlib/printexc.ml}
      }
      \only<1>{
        \listocaml[firstline=170,lastline=172]{../ocaml/stdlib/printexc.ml}{stdlib/printexc.ml:170}
      }
      \only<2>{
        \listocaml[firstline=170,lastline=172,highlightlines=172]{../ocaml/stdlib/printexc.ml}{stdlib/printexc.ml:170}
      }
      \only<3>{
        \mintc[firstline=154,lastline=156]{../ocaml/runtime/backtrace.c}
        \listc[firstline=171,lastline=192,highlightlines={184-187}]{../ocaml/runtime/backtrace.c}{runtime/backtrace.c:154}
      }
      \only<4>{
        \listocaml[firstline=155,lastline=165]{../ocaml/stdlib/printexc.ml}{stdlib/printexc.ml:55}
      }
    \end{column}
  \end{columns}
\end{frame}


\subsection{The multiple kinds of raise}
\frameSubsection{}{}

\begin{frame}{Backtraces}{When backtraces are not needed}
  \begin{columns}[c]
    \begin{column}{0.45\textwidth}
      \minipage[c][0.66\textheight][s]{\columnwidth}
      \begin{itemize}
        \item<1-> What if the backtrace for an exception is never needed?
        \item<2-> It's possible to raise the exception explicitly with \funcname{raise\_notrace} to make raising as cheap as possible
        \item<3-> An example of use in the \funcname{dynlink} library of the OCaml compiler
      \end{itemize}
      \vfill
      \onslide<3->{
      \listocaml[firstline=205,lastline=209,highlightlines=207]{../ocaml/otherlibs/dynlink/dynlink\_compilerlibs/misc.ml}{otherlibs/dynlink/dynlink\_compilerlibs/misc.ml:205}
      }
      \endminipage
    \end{column}
    \begin{column}{0.55\textwidth}
    \only<4>{
      \mintcmm[highlightlines=3]{../src/notrace/camlNotrace\_\_fun_347.cmm}
      \listcmm{../src/notrace/camlNotrace\_\_all\_somes\_327.cmm}{all\_somes}
    }
    \onslide*<5->{
      \listobjdump[highlightlines={6-9}]{../src/notrace/camlNotrace\_\_fun_347.objdump}{anonymous function from \funcname{all\_somes}}
    }
    \end{column}
  \end{columns}
\end{frame}

\begin{frame}{Backtraces}{Summary table of the multiple kinds of raise}
  \begin{itemize}
    \item When debugging information is enabled and if the backtrace of an exception isn't needed, it's possible to raise with \funcname{raise\_notrace} for performance
    \item When debugging information is \emph{not} enabled, the compiler optimises \funcname{raise} calls to inline assembly (same effect as using \funcname{raise\_notrace})
  \end{itemize}
  \bigskip
  \centering
  \begin{tabular}{| l | c | c | c | c |}
    \hline
    \parbox{10em}{Debug information \\ (\funcarg{-g} flag)}  & \multicolumn{2}{|c|}{Disabled}                                     & \multicolumn{2}{|c|}{Enabled} \\
    \hline
    Raise kind                                               & \funcname{raise} & \funcname{raise\_notrace}                       & \funcname{raise} & \funcname{raise\_notrace} \\
    \hline
    Compilation                                              & \parbox{8em}{inline assembly\\ (optimization)} & inline assembly   & \funcname{caml\_raise\_exn} & inline assembly \\
    \hline
  \end{tabular}
\end{frame}


%
% Takeaway #5
%
\subsection*{Takeaway \#5}
\frameSubsectionTakeaway{}
\againframe<5>{takeaway}
}
    \end{column}
  \end{columns}
\end{frame}

\begin{frame}{Backtraces}{\funcname{caml\_probably\_raise}'s handler doesn't catch \typename{ExnC}}
  \begin{columns}[c]
    \begin{column}{0.45\textwidth}
      \minipage[c][0.75\textheight][s]{\columnwidth}
      \begin{itemize}
        \item<1-> \funcname{match \dots with}\footnotemark pattern matching can also match exceptions
        \item<1-> The CMM of \funcname{probably\_raise} shows that this \funcname{match \dots with} is implemented with a pattern matching and a \funcname{try \dots with}
        \item<1-> But this one only matches on \typename{ExnA} exception
        \item<2-> Since the pattern matching fails to catch the exception, \funcname{reraise} is called with our current \typename{ExnC} exception
        \item<3-> Disassembly shows that \funcname{reraise} is actually a call to \funcname{caml\_reraise\_exn}, which is also an assembly function provided by the runtime
      \end{itemize}
      \vfill
      \mintocaml[firstline=15,lastline=21,highlightlines={18-21}]{../src/backtrace/backtrace.ml}
      \endminipage
    \end{column}
    \begin{column}{0.55\textwidth}
      \centering
      \onslide*<1>{\mintcmm[highlightlines={10-18}]{../src/backtrace/camlBacktrace\_\_probably\_raise\_351.cmm}}
      \onslide*<2>{\listcmm[highlightlines=18]{../src/backtrace/camlBacktrace\_\_probably\_raise_351.cmm}{CMM of probably\_raise}}
      \onslide*<3>{\mintobjdump[firstline=5,lastline=35,highlightlines=31]{../src/backtrace/camlBacktrace\_\_probably\_raise_351.objdump}}
    \end{column}
  \end{columns}
  \only<1>{\footnotetext[1]{https://blog.janestreet.com/pattern-matching-and-exception-handling-unite/}}
\end{frame}

\newcommand\listCamlReraiseExn[1][]{
  \listasm[firstline=727,lastline=734,#1]{../ocaml/runtime/amd64.S}{runtime/amd64.S:727}}

\begin{frame}{Backtraces}{Introducing \funcname{caml\_reraise\_exn}}
  \begin{columns}[c]
    \begin{column}{0.5\textwidth}
      \minipage[c][0.66\textheight][s]{\columnwidth}
      \begin{itemize}
        \item<1-> The exception have to be reraised to the next handler
        \item<2-> First, checks if the backtrace should be collected with \funcarg{domain\_state->backtrace\_active}
        \only<3>{
        \begin{itemize}
            \item If not, behaves like a \funcname{raise\_notrace} with \funcname{RESTORE\_EXN\_HANDLER\_OCAML}
        \end{itemize}
        }
        \item<4-> Jumps into \funcname{caml\_raise\_exn}
        \item<5-> Calls \funcname{caml\_stash\_backtrace} again, to scan the next part of the stack: from the trap just pop-ed to the next trap
      \end{itemize}
      \vfill
      \mintocaml[firstline=15,lastline=21,highlightlines={18-21}]{../src/backtrace/backtrace.ml}
      \endminipage
    \end{column}
    \begin{column}{0.5\textwidth}
      \raggedleft
      \onslide*<1>{\listCamlReraiseExn{}}
      \onslide*<2>{\listCamlReraiseExn[highlightlines=729]{}}
      \onslide*<3>{
        \mintc[firstline=190,lastline=192,highlightlines={191-192}]{../ocaml/runtime/amd64.S}
        \mintc[firstline=234,lastline=237,highlightlines={235-237}]{../ocaml/runtime/amd64.S}
        \medskip
        \listCamlReraiseExn[highlightlines=731]{}
      }
      \onslide*<4-5>{
        % TODO refactor with \listCamlReraiseExn
        \mintasm[firstline=727,lastline=734,highlightlines=730]{../ocaml/runtime/amd64.S}
        \medskip
      }
      \onslide*<4>{\listCamlRaiseExn[highlightlines=711]{}}
      \onslide*<5>{\listCamlRaiseExn[highlightlines=720]{}}
    \end{column}
  \end{columns}
\end{frame}

\begin{frame}{Backtraces}{Backtrace collection continues}
  \begin{columns}[c]
    \begin{column}{0.55\textwidth}
      \minipage[c][0.75\textheight][s]{\columnwidth}
      \begin{itemize}
        \item<1-> \funcname{caml\_stash\_backtrace} scans the older part of the stack, up to the next trap
        \item<6-> Controls jumps to the nested exception handler in \funcname{maybe\_raise}, which matches only on \typename{ExnB}
        \item<7-> \funcname{caml\_reraise\_exn} is called again, backtrace collection resumes with \funcname{caml\_stash\_backtrace}
      \end{itemize}
      \medskip
      \begin{itemize}
        \item<2-> Constructed backtrace:
        \begin{itemize}
          \item <2-> \funcname{definitely\_raise}
          \item <2-> \funcname{probably\_raise}
          \item <3-> \funcname{probably\_raise} (re-raise)
          \item <4-> \funcname{maybe\_raise}
          \item <5-> \funcname{main}
          \item <8-> \funcname{main} (re-raise)
        \end{itemize}
      \end{itemize}
      \vfill
      \onslide*<1-5>{\mintocaml[firstline=15,lastline=21]{../src/backtrace/backtrace.ml}}
      \onslide*<6>{\mintocaml[firstline=28,lastline=34,highlightlines=31]{../src/backtrace/backtrace.ml}}
      \onslide*<7->{\mintocaml[firstline=28,lastline=34,highlightlines=33]{../src/backtrace/backtrace.ml}}
      \endminipage
    \end{column}
    \begin{column}{0.45\textwidth}
      \raggedleft
      \onslide*<1>{\providecommand\step{6}%
% Backtraces
%
\subsection{One advantage of raising with debugging information}
\frameSubsection{}{}

\begin{frame}{Backtraces}{Debugging information \& source code}
  \begin{columns}[c]
    \begin{column}{0.5\textwidth}
      \begin{itemize}
        \item Raising an exception is fun and games, but how do we know where the exception originated? $\rightarrow$ With the backtrace associated with the exception!
        \item In order to get a backtrace from the point where an exception was raised, it's necessary to compile with debugging information enabled (\funcarg{-g} flag for the compiler)
        \item Same example as in the `Nested handlers' section
      \end{itemize}
    \end{column}
    \begin{column}{0.5\textwidth}
      \listocaml[firstline=9,lastline=34]{../src/backtrace/backtrace.ml}{backtrace/backtrace.ml}
    \end{column}
  \end{columns}
\end{frame}

\begin{frame}{Backtraces}{CMM and disassembly of \funcname{definitely\_raise}}
  \begin{columns}[c]
    \begin{column}{0.5\textwidth}
      \minipage[c][0.66\textheight][s]{\columnwidth}
      \begin{itemize}
        \item<1-> An effect of compiling with debugging information (\funcarg{-g}): the CMM no longer uses \funcname{raise\_notrace} to raise the exception but \funcname{raise} instead
        \item<2-> Disassembly shows that CMM \funcname{raise} is actually a call to \funcname{caml\_raise\_exn}, a function in assembler provided by the runtime
      \end{itemize}
      \vfill
      \mintocaml[firstline=9,lastline=13,highlightlines=12]{../src/backtrace/backtrace.ml}
      \endminipage
    \end{column}
    \begin{column}{0.5\textwidth}
      \onslide*<1>{
        \mintcmm[highlightlines=5]{../src/backtrace/camlBacktrace\_\_definitely\_raise\_347.cmm}
      }
      \onslide*<2>{
        \mintobjdump[firstline=2,highlightlines=14]{../src/backtrace/camlBacktrace\_\_definitely\_raise\_347.objdump}
      }
    \end{column}
  \end{columns}
\end{frame}

\begin{frame}{Backtraces}{Introducing \funcname{caml\_raise\_exn}}
  \begin{columns}[c]
    \begin{column}{0.5\textwidth}
      \minipage[c][0.66\textheight][s]{\columnwidth}
      \onslide*<1>{
        \begin{itemize}
          \item Raising the exception is performed using \funcname{caml\_raise\_exn} which is
            \begin{itemize}
              \item An assembly function provided by the runtime
              \item Architecture specific
            \end{itemize}
          \item Let's see what it does differently from \funcname{raise\_notrace}\dots
          \item We are about to raise \typename{ExnC} with \funcname{caml\_raise\_exn} from \funcname{definitely\_raise}
        \end{itemize}
      }
      \onslide*<2>{
        \mintobjdump[firstline=2,highlightlines=14]{../src/backtrace/camlBacktrace\_\_definitely\_raise\_347.objdump}
      }
      \vfill
      \mintocaml[firstline=9,lastline=13,highlightlines=12]{../src/backtrace/backtrace.ml}
      \endminipage
    \end{column}
    \begin{column}{0.5\textwidth}
      \centering
      \providecommand\step{1}\input{diagrams/backtrace/backtrace.tex}
    \end{column}
  \end{columns}
\end{frame}

\begin{frame}{Backtraces}{But before that, we need more fields from \typename{caml\_domain\_state}}
  \begin{columns}
    \begin{column}{0.5\textwidth}
      \begin{itemize}
        \item Remember \typename{caml\_domain\_state} the per-domain runtime state?
        \item Some other members are of interest to us now\dots
        \item \localname{backtrace\_active} is used to know if the runtime should collect backtraces
        \item \localname{backtrace\_active} can be set by
          \begin{itemize}
            \item The \funcarg{b} flag in the \funcname{OCAMLRUNPARAM} environment variable
            \item \mintinline{ocaml}{Printexc.record_backtrace : bool -> unit} \footnotemark
          \end{itemize}
      \end{itemize}
    \end{column}
    \begin{column}{0.5\textwidth}
      \begin{listing}
        \mintc[highlightlines={9-11}]{src/ocaml/domain\_state\_backtrace.h}
        \caption{runtime/caml/domain\_state.h}
      \end{listing}
    \end{column}
  \end{columns}
  \footnotetext[1]{https://v2.ocaml.org/api/Printexc.html}
\end{frame}

\newcommand\listCamlRaiseExn[1][]{
  \listasm[firstline=702,lastline=725,#1]{../ocaml/runtime/amd64.S}{runtime/amd64.S:702}}

\begin{frame}{Backtraces}{Walk through \funcname{caml\_raise\_exn}}
  \begin{columns}[T]
    \begin{column}{0.5\textwidth}
      \begin{itemize}
        \item<1-> First checks whether exceptions backtrace should be recorded
        \item<1-> By checking the \funcarg{domain\_state->backtrace\_active} value
        \item<2-> If not, acts as \funcname{raise\_notrace} with \funcname{RESTORE\_EXN\_HANDLER\_OCAML}
          \begin{itemize}
            \item Set \regname{RSP} to \localname{domain\_state->exn\_handler} value
            \item Restore \localname{domain\_state->exn\_handler} from trap
            \item Jump with \funcname{ret} (same effect as using \funcname{pop} and \funcname{jmp})
          \end{itemize}
        \item<3-> Otherwise, switches to the C stack to be able to execute C code
        \item<4-> Calls \funcname{caml\_stash\_backtrace}, a C function provided by the runtime\ignorespaces
          \onslide<6->{, with
            \begin{itemize}
              \item \funcarg{exn}: exception value
              \item \funcarg{pc}: code address of the raising point
              \item \funcarg{sp}: stack address of the raising function
              \item \funcarg{trapsp}: stack address of the next handler
            \end{itemize}
          }
      \end{itemize}
      \bigskip
      \mintocaml[firstline=9,lastline=13,highlightlines=12]{../src/backtrace/backtrace.ml}
    \end{column}
    \begin{column}{0.5\textwidth}
      \raggedleft
      \onslide*<1,3-4>{
        \mintc[firstline=190,lastline=192]{../ocaml/runtime/amd64.S}
        \mintc[firstline=234,lastline=237]{../ocaml/runtime/amd64.S}
      }
      \onslide*<2>{
        \mintc[firstline=190,lastline=192,highlightlines={191-192}]{../ocaml/runtime/amd64.S}
        \mintc[firstline=234,lastline=237,highlightlines={235-237}]{../ocaml/runtime/amd64.S}
      }
      \onslide*<1>{\listCamlRaiseExn[highlightlines=705]{}}%
      \onslide*<2>{\listCamlRaiseExn[highlightlines={707-708}]{}}%
      \onslide*<3>{\listCamlRaiseExn[highlightlines={712-714}]{}}%
      \onslide*<4>{\listCamlRaiseExn[highlightlines={715-720}]{}}%
      \onslide*<5>{\providecommand\step{1}\input{diagrams/backtrace/backtrace.tex}}%
      \onslide*<6>{\providecommand\step{2}\input{diagrams/backtrace/backtrace.tex}}%
    \end{column}
  \end{columns}
\end{frame}

\newcommand\listStashBacktrace[1][]{
  \listc[firstline=91,lastline=120,#1]{../ocaml/runtime/backtrace\_nat.c}{runtime/backtrace\_nat.c:91}}

\begin{frame}{Backtraces}{Introducing \funcname{caml\_stash\_backtrace}}
  \begin{columns}[c]
    \begin{column}{0.5\textwidth}
      \minipage[c][0.66\textheight][s]{\columnwidth}
      \begin{itemize}
        \item<1-> A C function provided by the runtime (specific to native code)
        \item<2-> It scans the stack, between the stack pointer at the raising point up to the next trap, in order to collect the names of the detected function frames
          \onslide*<2>{
            \begin{itemize}
              \item And more information, like line number, column number...
            \end{itemize}
          }
        \item<3-> It uses the same technique and information as the Garbage Collector (GC) uses to scan the values live on the stack
          \onslide*<4>{
            \begin{itemize}
              \item \typename{frame\_descr} are at the core of stack unwinding
            \end{itemize}
          }
      \end{itemize}
      \vfill
      \mintocaml[firstline=9,lastline=13,highlightlines=12]{../src/backtrace/backtrace.ml}
      \endminipage
    \end{column}
    \begin{column}{0.5\textwidth}
      \onslide*<1>{\listStashBacktrace[highlightlines=91]{}}
      \onslide*<2>{\listStashBacktrace[highlightlines={91,117-118}]{}}
      \onslide*<3->{\listStashBacktrace[highlightlines={94,106,109-110}]{}}
    \end{column}
  \end{columns}
\end{frame}

\newcommand\listFrameDescr[1][]{
  \listocaml[firstline=111,lastline=124,#1]{../ocaml/asmcomp/emitaux.ml}{asmcomp/emitaux.ml:111}}
\newcommand\listDebugInfo[1][]{
  \listocaml[firstline=38,lastline=49,#1]{../ocaml/lambda/debuginfo.mli}{lambda/debuginfo.mli:38}}

\begin{frame}{Backtraces}{\typename{frame\_descr} and \typename{frame\_debuginfo} in a nutshell}
  \begin{columns}[c]
    \begin{column}{0.5\textwidth}
      \begin{itemize}
        \item<1-> For every return address in an OCaml program, there is a associated \typename{frame\_descr}
        \item<2-> A \typename{frame\_descr} specifies
        \begin{itemize}
          \item<2-> The size of the function stack frame
          \item<3-> Where live values are on the stack (so that the GC can mark them as live and prevent from being swept)
        \end{itemize}
        \item<4-> When compiled with debug information, \typename{frame\_descr} will be linked to a \typename{frame\_debuginfo}
        \begin{itemize}
          \item<5-> Contains information about the position in the source code (file name, line and column number)
        \end{itemize}
      \end{itemize}
    \end{column}
    \begin{column}{0.5\textwidth}
        \onslide*<1>{\listFrameDescr[highlightlines={118-122}]{}}
        \onslide*<2>{\listFrameDescr[highlightlines={120}]{}}
        \onslide*<3>{\listFrameDescr[highlightlines={121}]{}}
        \onslide*<4->{\listFrameDescr[highlightlines={122}]{}}
        \medskip
        \onslide*<1-3>{\listDebugInfo{}}
        \onslide*<4>{\listDebugInfo[highlightlines={38,49}]{}}
        \onslide*<5>{\listDebugInfo[highlightlines={39-46}]{}}
    \end{column}
  \end{columns}
\end{frame}

\begin{frame}{Backtraces}{Scanning the stack with \typename{frame\_descr}}
  \begin{columns}[c]
    \begin{column}{0.5\textwidth}
      \begin{itemize}
        \item Given the return address of the code that raises the exception by calling \funcname{caml\_raise\_exn} (\funcarg{pc} argument of \funcname{caml\_stash\_backtrace})
        \item And the position of the stack pointer (\funcarg{sp} argument of \funcname{caml\_stash\_backtrace})
        \item The frame size given by \typename{frame\_descr}.\funcarg{frame\_size}, allows to find the end of the previous frame
        \item And the return address of the caller of this function
        \bigskip
        \onslide<3->{
        \item Note that traps (here the reddish \funcname{ExnA}, \funcname{ExnB} and \funcname{ExnC} blocks) belong to the frame that installed them, and so the frame size changes when traps are removed
        }
      \end{itemize}
    \end{column}
    \begin{column}{0.5\textwidth}
      \raggedleft
      \onslide*<1>{\providecommand\step{2}\input{diagrams/backtrace/backtrace.tex}}%
      \onslide*<2>{\providecommand\step{1}\input{diagrams/backtrace/scanstack.tex}}%
      \onslide*<3>{\providecommand\step{2}\input{diagrams/backtrace/scanstack.tex}}%
      \onslide*<4>{\providecommand\step{3}\input{diagrams/backtrace/scanstack.tex}}%
      \onslide*<5>{\providecommand\step{4}\input{diagrams/backtrace/scanstack.tex}}%
      \onslide*<6>{\providecommand\step{5}\input{diagrams/backtrace/scanstack.tex}}%
    \end{column}
  \end{columns}
\end{frame}

% Duplicate of previous-previous slide
\begin{frame}{Backtraces}{Continuing on \funcname{caml\_stash\_backtrace}}
  \begin{columns}[c]
    \begin{column}{0.5\textwidth}
      \minipage[c][0.75\textheight][s]{\columnwidth}
      \begin{itemize}
          \color{gray}
        \item A C function provided by the runtime (specific to native code)
        \item It scans the stack, between the stack pointer at the raising point up to the next trap, in order to collect the names of the detected function frames
        \item It uses the same technique and information as the Garbage Collector (GC) uses to scan the values live on the stack
      \end{itemize}
      \begin{itemize}
        \item For each function frame detected, store a pointer to the \typename{frame\_descr} in the \localname{domain\_state->backtrace\_buffer} array, effectively constructing the backtrace
      \end{itemize}
      \onslide<2->{
        \begin{itemize}
            \smallskip
          \item Constructed backtrace:
            \funcname{definitely\_raise},
            \onslide<3->{\funcname{probably\_raise}}
        \end{itemize}
      }
      \vfill
      \mintocaml[firstline=9,lastline=13,highlightlines=12]{../src/backtrace/backtrace.ml}
      \endminipage
    \end{column}
    \begin{column}{0.5\textwidth}
      \raggedleft
      \onslide*<1>{\listStashBacktrace[highlightlines={114-115}]{}}
      \onslide*<2>{\providecommand\step{3}\input{diagrams/backtrace/backtrace.tex}}%
      \onslide*<3>{\providecommand\step{4}\input{diagrams/backtrace/backtrace.tex}}%
    \end{column}
  \end{columns}
\end{frame}

\begin{frame}{Backtraces}{Back to \funcname{caml\_raise\_exn}}
  \begin{columns}[c]
    \begin{column}{0.5\textwidth}
      \minipage[c][0.66\textheight][s]{\columnwidth}
      \begin{itemize}\color{gray}
        \item Checks wheteher exceptions backtrace should be recorded, by checking the \funcarg{domain\_state->backtrace\_active} value
        \item Switches to the C stack to be able to execute C code
        \item Calls \funcname{caml\_stash\_backtrace}, to collect the backtrace up to the next trap
      \end{itemize}
      \onslide*<2->{
        \begin{itemize}
          \item<2-> Executes the exception handler in \funcname{probably\_raise} with \funcname{RESTORE\_EXN\_HANDLER\_OCAML}
            \begin{itemize}
              \item Set \regname{RSP} to \localname{domain\_state->exn\_handler} value
              \item Restore \localname{domain\_state->exn\_handler} from trap
              \item Jump to the handler code with \funcname{ret}
            \end{itemize}
        \end{itemize}
      }
      \vfill
      \onslide*<1>{\mintocaml[firstline=9,lastline=13,highlightlines=12]{../src/backtrace/backtrace.ml}}
      \onslide*<2->{\mintocaml[firstline=15,lastline=21,highlightlines={19-21}]{../src/backtrace/backtrace.ml}}
      \endminipage
    \end{column}
    \begin{column}{0.5\textwidth}
      \centering
      \onslide*<1>{
        \mintc[firstline=190,lastline=192]{../ocaml/runtime/amd64.S}
        \mintc[firstline=234,lastline=237]{../ocaml/runtime/amd64.S}
      }
      \onslide*<2>{
        \mintc[firstline=190,lastline=192,highlightlines={191-192}]{../ocaml/runtime/amd64.S}
        \mintc[firstline=234,lastline=237,highlightlines={235-237}]{../ocaml/runtime/amd64.S}
      }
      \onslide*<1>{\listCamlRaiseExn[highlightlines=720]{}}
      \onslide*<2>{\listCamlRaiseExn[highlightlines=722]{}}
      \onslide*<3->{\providecommand\step{5}\input{diagrams/backtrace/backtrace.tex}}
    \end{column}
  \end{columns}
\end{frame}

\begin{frame}{Backtraces}{\funcname{caml\_probably\_raise}'s handler doesn't catch \typename{ExnC}}
  \begin{columns}[c]
    \begin{column}{0.45\textwidth}
      \minipage[c][0.75\textheight][s]{\columnwidth}
      \begin{itemize}
        \item<1-> \funcname{match \dots with}\footnotemark pattern matching can also match exceptions
        \item<1-> The CMM of \funcname{probably\_raise} shows that this \funcname{match \dots with} is implemented with a pattern matching and a \funcname{try \dots with}
        \item<1-> But this one only matches on \typename{ExnA} exception
        \item<2-> Since the pattern matching fails to catch the exception, \funcname{reraise} is called with our current \typename{ExnC} exception
        \item<3-> Disassembly shows that \funcname{reraise} is actually a call to \funcname{caml\_reraise\_exn}, which is also an assembly function provided by the runtime
      \end{itemize}
      \vfill
      \mintocaml[firstline=15,lastline=21,highlightlines={18-21}]{../src/backtrace/backtrace.ml}
      \endminipage
    \end{column}
    \begin{column}{0.55\textwidth}
      \centering
      \onslide*<1>{\mintcmm[highlightlines={10-18}]{../src/backtrace/camlBacktrace\_\_probably\_raise\_351.cmm}}
      \onslide*<2>{\listcmm[highlightlines=18]{../src/backtrace/camlBacktrace\_\_probably\_raise_351.cmm}{CMM of probably\_raise}}
      \onslide*<3>{\mintobjdump[firstline=5,lastline=35,highlightlines=31]{../src/backtrace/camlBacktrace\_\_probably\_raise_351.objdump}}
    \end{column}
  \end{columns}
  \only<1>{\footnotetext[1]{https://blog.janestreet.com/pattern-matching-and-exception-handling-unite/}}
\end{frame}

\newcommand\listCamlReraiseExn[1][]{
  \listasm[firstline=727,lastline=734,#1]{../ocaml/runtime/amd64.S}{runtime/amd64.S:727}}

\begin{frame}{Backtraces}{Introducing \funcname{caml\_reraise\_exn}}
  \begin{columns}[c]
    \begin{column}{0.5\textwidth}
      \minipage[c][0.66\textheight][s]{\columnwidth}
      \begin{itemize}
        \item<1-> The exception have to be reraised to the next handler
        \item<2-> First, checks if the backtrace should be collected with \funcarg{domain\_state->backtrace\_active}
        \only<3>{
        \begin{itemize}
            \item If not, behaves like a \funcname{raise\_notrace} with \funcname{RESTORE\_EXN\_HANDLER\_OCAML}
        \end{itemize}
        }
        \item<4-> Jumps into \funcname{caml\_raise\_exn}
        \item<5-> Calls \funcname{caml\_stash\_backtrace} again, to scan the next part of the stack: from the trap just pop-ed to the next trap
      \end{itemize}
      \vfill
      \mintocaml[firstline=15,lastline=21,highlightlines={18-21}]{../src/backtrace/backtrace.ml}
      \endminipage
    \end{column}
    \begin{column}{0.5\textwidth}
      \raggedleft
      \onslide*<1>{\listCamlReraiseExn{}}
      \onslide*<2>{\listCamlReraiseExn[highlightlines=729]{}}
      \onslide*<3>{
        \mintc[firstline=190,lastline=192,highlightlines={191-192}]{../ocaml/runtime/amd64.S}
        \mintc[firstline=234,lastline=237,highlightlines={235-237}]{../ocaml/runtime/amd64.S}
        \medskip
        \listCamlReraiseExn[highlightlines=731]{}
      }
      \onslide*<4-5>{
        % TODO refactor with \listCamlReraiseExn
        \mintasm[firstline=727,lastline=734,highlightlines=730]{../ocaml/runtime/amd64.S}
        \medskip
      }
      \onslide*<4>{\listCamlRaiseExn[highlightlines=711]{}}
      \onslide*<5>{\listCamlRaiseExn[highlightlines=720]{}}
    \end{column}
  \end{columns}
\end{frame}

\begin{frame}{Backtraces}{Backtrace collection continues}
  \begin{columns}[c]
    \begin{column}{0.55\textwidth}
      \minipage[c][0.75\textheight][s]{\columnwidth}
      \begin{itemize}
        \item<1-> \funcname{caml\_stash\_backtrace} scans the older part of the stack, up to the next trap
        \item<6-> Controls jumps to the nested exception handler in \funcname{maybe\_raise}, which matches only on \typename{ExnB}
        \item<7-> \funcname{caml\_reraise\_exn} is called again, backtrace collection resumes with \funcname{caml\_stash\_backtrace}
      \end{itemize}
      \medskip
      \begin{itemize}
        \item<2-> Constructed backtrace:
        \begin{itemize}
          \item <2-> \funcname{definitely\_raise}
          \item <2-> \funcname{probably\_raise}
          \item <3-> \funcname{probably\_raise} (re-raise)
          \item <4-> \funcname{maybe\_raise}
          \item <5-> \funcname{main}
          \item <8-> \funcname{main} (re-raise)
        \end{itemize}
      \end{itemize}
      \vfill
      \onslide*<1-5>{\mintocaml[firstline=15,lastline=21]{../src/backtrace/backtrace.ml}}
      \onslide*<6>{\mintocaml[firstline=28,lastline=34,highlightlines=31]{../src/backtrace/backtrace.ml}}
      \onslide*<7->{\mintocaml[firstline=28,lastline=34,highlightlines=33]{../src/backtrace/backtrace.ml}}
      \endminipage
    \end{column}
    \begin{column}{0.45\textwidth}
      \raggedleft
      \onslide*<1>{\providecommand\step{6}\input{diagrams/backtrace/backtrace.tex}}%
      \onslide*<2>{\providecommand\step{6}\input{diagrams/backtrace/backtrace.tex}}%
      \onslide*<3>{\providecommand\step{7}\input{diagrams/backtrace/backtrace.tex}}%
      \onslide*<4>{\providecommand\step{8}\input{diagrams/backtrace/backtrace.tex}}%
      \onslide*<5>{\providecommand\step{9}\input{diagrams/backtrace/backtrace.tex}}%
      \onslide*<6>{\providecommand\step{10}\input{diagrams/backtrace/backtrace.tex}}%
      \onslide*<7>{\providecommand\step{11}\input{diagrams/backtrace/backtrace.tex}}%
      \onslide*<8>{\providecommand\step{12}\input{diagrams/backtrace/backtrace.tex}}%
      \onslide*<9>{\providecommand\step{13}\input{diagrams/backtrace/backtrace.tex}}%
    \end{column}
  \end{columns}
\end{frame}

\begin{frame}{Backtraces}{Printing the backtrace}
  \begin{columns}[c]
    \begin{column}{0.45\textwidth}
      \minipage[c][0.66\textheight][s]{\columnwidth}
      \begin{itemize}
        \item<1-> The exception backtrace can now be printed to \funcarg{stdout} with \funcname{Printexc.print_backtrace}
        \item<2-> First the exception backtrace is retrieved into an OCaml array with \funcname{get_raw_backtrace }
        \item<3-> Which is implemented as the \funcname{caml_get_exception_raw_backtrace} C function in the runtime
        \item<4-> And finally printed with \funcname{print_exception_backtrace}
      \end{itemize}
      \vfill
      \mintocaml[firstline=28,lastline=34,highlightlines=33]{../src/backtrace/backtrace.ml}
      \endminipage
    \end{column}
    \begin{column}{0.55\textwidth}
      \onslide*<1,2>{
        \mintocaml[firstline=100,lastline=101]{../ocaml/stdlib/printexc.ml}
      }
      \only<1>{
        \listocaml[firstline=170,lastline=172]{../ocaml/stdlib/printexc.ml}{stdlib/printexc.ml:170}
      }
      \only<2>{
        \listocaml[firstline=170,lastline=172,highlightlines=172]{../ocaml/stdlib/printexc.ml}{stdlib/printexc.ml:170}
      }
      \only<3>{
        \mintc[firstline=154,lastline=156]{../ocaml/runtime/backtrace.c}
        \listc[firstline=171,lastline=192,highlightlines={184-187}]{../ocaml/runtime/backtrace.c}{runtime/backtrace.c:154}
      }
      \only<4>{
        \listocaml[firstline=155,lastline=165]{../ocaml/stdlib/printexc.ml}{stdlib/printexc.ml:55}
      }
    \end{column}
  \end{columns}
\end{frame}


\subsection{The multiple kinds of raise}
\frameSubsection{}{}

\begin{frame}{Backtraces}{When backtraces are not needed}
  \begin{columns}[c]
    \begin{column}{0.45\textwidth}
      \minipage[c][0.66\textheight][s]{\columnwidth}
      \begin{itemize}
        \item<1-> What if the backtrace for an exception is never needed?
        \item<2-> It's possible to raise the exception explicitly with \funcname{raise\_notrace} to make raising as cheap as possible
        \item<3-> An example of use in the \funcname{dynlink} library of the OCaml compiler
      \end{itemize}
      \vfill
      \onslide<3->{
      \listocaml[firstline=205,lastline=209,highlightlines=207]{../ocaml/otherlibs/dynlink/dynlink\_compilerlibs/misc.ml}{otherlibs/dynlink/dynlink\_compilerlibs/misc.ml:205}
      }
      \endminipage
    \end{column}
    \begin{column}{0.55\textwidth}
    \only<4>{
      \mintcmm[highlightlines=3]{../src/notrace/camlNotrace\_\_fun_347.cmm}
      \listcmm{../src/notrace/camlNotrace\_\_all\_somes\_327.cmm}{all\_somes}
    }
    \onslide*<5->{
      \listobjdump[highlightlines={6-9}]{../src/notrace/camlNotrace\_\_fun_347.objdump}{anonymous function from \funcname{all\_somes}}
    }
    \end{column}
  \end{columns}
\end{frame}

\begin{frame}{Backtraces}{Summary table of the multiple kinds of raise}
  \begin{itemize}
    \item When debugging information is enabled and if the backtrace of an exception isn't needed, it's possible to raise with \funcname{raise\_notrace} for performance
    \item When debugging information is \emph{not} enabled, the compiler optimises \funcname{raise} calls to inline assembly (same effect as using \funcname{raise\_notrace})
  \end{itemize}
  \bigskip
  \centering
  \begin{tabular}{| l | c | c | c | c |}
    \hline
    \parbox{10em}{Debug information \\ (\funcarg{-g} flag)}  & \multicolumn{2}{|c|}{Disabled}                                     & \multicolumn{2}{|c|}{Enabled} \\
    \hline
    Raise kind                                               & \funcname{raise} & \funcname{raise\_notrace}                       & \funcname{raise} & \funcname{raise\_notrace} \\
    \hline
    Compilation                                              & \parbox{8em}{inline assembly\\ (optimization)} & inline assembly   & \funcname{caml\_raise\_exn} & inline assembly \\
    \hline
  \end{tabular}
\end{frame}


%
% Takeaway #5
%
\subsection*{Takeaway \#5}
\frameSubsectionTakeaway{}
\againframe<5>{takeaway}
}%
      \onslide*<2>{\providecommand\step{6}%
% Backtraces
%
\subsection{One advantage of raising with debugging information}
\frameSubsection{}{}

\begin{frame}{Backtraces}{Debugging information \& source code}
  \begin{columns}[c]
    \begin{column}{0.5\textwidth}
      \begin{itemize}
        \item Raising an exception is fun and games, but how do we know where the exception originated? $\rightarrow$ With the backtrace associated with the exception!
        \item In order to get a backtrace from the point where an exception was raised, it's necessary to compile with debugging information enabled (\funcarg{-g} flag for the compiler)
        \item Same example as in the `Nested handlers' section
      \end{itemize}
    \end{column}
    \begin{column}{0.5\textwidth}
      \listocaml[firstline=9,lastline=34]{../src/backtrace/backtrace.ml}{backtrace/backtrace.ml}
    \end{column}
  \end{columns}
\end{frame}

\begin{frame}{Backtraces}{CMM and disassembly of \funcname{definitely\_raise}}
  \begin{columns}[c]
    \begin{column}{0.5\textwidth}
      \minipage[c][0.66\textheight][s]{\columnwidth}
      \begin{itemize}
        \item<1-> An effect of compiling with debugging information (\funcarg{-g}): the CMM no longer uses \funcname{raise\_notrace} to raise the exception but \funcname{raise} instead
        \item<2-> Disassembly shows that CMM \funcname{raise} is actually a call to \funcname{caml\_raise\_exn}, a function in assembler provided by the runtime
      \end{itemize}
      \vfill
      \mintocaml[firstline=9,lastline=13,highlightlines=12]{../src/backtrace/backtrace.ml}
      \endminipage
    \end{column}
    \begin{column}{0.5\textwidth}
      \onslide*<1>{
        \mintcmm[highlightlines=5]{../src/backtrace/camlBacktrace\_\_definitely\_raise\_347.cmm}
      }
      \onslide*<2>{
        \mintobjdump[firstline=2,highlightlines=14]{../src/backtrace/camlBacktrace\_\_definitely\_raise\_347.objdump}
      }
    \end{column}
  \end{columns}
\end{frame}

\begin{frame}{Backtraces}{Introducing \funcname{caml\_raise\_exn}}
  \begin{columns}[c]
    \begin{column}{0.5\textwidth}
      \minipage[c][0.66\textheight][s]{\columnwidth}
      \onslide*<1>{
        \begin{itemize}
          \item Raising the exception is performed using \funcname{caml\_raise\_exn} which is
            \begin{itemize}
              \item An assembly function provided by the runtime
              \item Architecture specific
            \end{itemize}
          \item Let's see what it does differently from \funcname{raise\_notrace}\dots
          \item We are about to raise \typename{ExnC} with \funcname{caml\_raise\_exn} from \funcname{definitely\_raise}
        \end{itemize}
      }
      \onslide*<2>{
        \mintobjdump[firstline=2,highlightlines=14]{../src/backtrace/camlBacktrace\_\_definitely\_raise\_347.objdump}
      }
      \vfill
      \mintocaml[firstline=9,lastline=13,highlightlines=12]{../src/backtrace/backtrace.ml}
      \endminipage
    \end{column}
    \begin{column}{0.5\textwidth}
      \centering
      \providecommand\step{1}\input{diagrams/backtrace/backtrace.tex}
    \end{column}
  \end{columns}
\end{frame}

\begin{frame}{Backtraces}{But before that, we need more fields from \typename{caml\_domain\_state}}
  \begin{columns}
    \begin{column}{0.5\textwidth}
      \begin{itemize}
        \item Remember \typename{caml\_domain\_state} the per-domain runtime state?
        \item Some other members are of interest to us now\dots
        \item \localname{backtrace\_active} is used to know if the runtime should collect backtraces
        \item \localname{backtrace\_active} can be set by
          \begin{itemize}
            \item The \funcarg{b} flag in the \funcname{OCAMLRUNPARAM} environment variable
            \item \mintinline{ocaml}{Printexc.record_backtrace : bool -> unit} \footnotemark
          \end{itemize}
      \end{itemize}
    \end{column}
    \begin{column}{0.5\textwidth}
      \begin{listing}
        \mintc[highlightlines={9-11}]{src/ocaml/domain\_state\_backtrace.h}
        \caption{runtime/caml/domain\_state.h}
      \end{listing}
    \end{column}
  \end{columns}
  \footnotetext[1]{https://v2.ocaml.org/api/Printexc.html}
\end{frame}

\newcommand\listCamlRaiseExn[1][]{
  \listasm[firstline=702,lastline=725,#1]{../ocaml/runtime/amd64.S}{runtime/amd64.S:702}}

\begin{frame}{Backtraces}{Walk through \funcname{caml\_raise\_exn}}
  \begin{columns}[T]
    \begin{column}{0.5\textwidth}
      \begin{itemize}
        \item<1-> First checks whether exceptions backtrace should be recorded
        \item<1-> By checking the \funcarg{domain\_state->backtrace\_active} value
        \item<2-> If not, acts as \funcname{raise\_notrace} with \funcname{RESTORE\_EXN\_HANDLER\_OCAML}
          \begin{itemize}
            \item Set \regname{RSP} to \localname{domain\_state->exn\_handler} value
            \item Restore \localname{domain\_state->exn\_handler} from trap
            \item Jump with \funcname{ret} (same effect as using \funcname{pop} and \funcname{jmp})
          \end{itemize}
        \item<3-> Otherwise, switches to the C stack to be able to execute C code
        \item<4-> Calls \funcname{caml\_stash\_backtrace}, a C function provided by the runtime\ignorespaces
          \onslide<6->{, with
            \begin{itemize}
              \item \funcarg{exn}: exception value
              \item \funcarg{pc}: code address of the raising point
              \item \funcarg{sp}: stack address of the raising function
              \item \funcarg{trapsp}: stack address of the next handler
            \end{itemize}
          }
      \end{itemize}
      \bigskip
      \mintocaml[firstline=9,lastline=13,highlightlines=12]{../src/backtrace/backtrace.ml}
    \end{column}
    \begin{column}{0.5\textwidth}
      \raggedleft
      \onslide*<1,3-4>{
        \mintc[firstline=190,lastline=192]{../ocaml/runtime/amd64.S}
        \mintc[firstline=234,lastline=237]{../ocaml/runtime/amd64.S}
      }
      \onslide*<2>{
        \mintc[firstline=190,lastline=192,highlightlines={191-192}]{../ocaml/runtime/amd64.S}
        \mintc[firstline=234,lastline=237,highlightlines={235-237}]{../ocaml/runtime/amd64.S}
      }
      \onslide*<1>{\listCamlRaiseExn[highlightlines=705]{}}%
      \onslide*<2>{\listCamlRaiseExn[highlightlines={707-708}]{}}%
      \onslide*<3>{\listCamlRaiseExn[highlightlines={712-714}]{}}%
      \onslide*<4>{\listCamlRaiseExn[highlightlines={715-720}]{}}%
      \onslide*<5>{\providecommand\step{1}\input{diagrams/backtrace/backtrace.tex}}%
      \onslide*<6>{\providecommand\step{2}\input{diagrams/backtrace/backtrace.tex}}%
    \end{column}
  \end{columns}
\end{frame}

\newcommand\listStashBacktrace[1][]{
  \listc[firstline=91,lastline=120,#1]{../ocaml/runtime/backtrace\_nat.c}{runtime/backtrace\_nat.c:91}}

\begin{frame}{Backtraces}{Introducing \funcname{caml\_stash\_backtrace}}
  \begin{columns}[c]
    \begin{column}{0.5\textwidth}
      \minipage[c][0.66\textheight][s]{\columnwidth}
      \begin{itemize}
        \item<1-> A C function provided by the runtime (specific to native code)
        \item<2-> It scans the stack, between the stack pointer at the raising point up to the next trap, in order to collect the names of the detected function frames
          \onslide*<2>{
            \begin{itemize}
              \item And more information, like line number, column number...
            \end{itemize}
          }
        \item<3-> It uses the same technique and information as the Garbage Collector (GC) uses to scan the values live on the stack
          \onslide*<4>{
            \begin{itemize}
              \item \typename{frame\_descr} are at the core of stack unwinding
            \end{itemize}
          }
      \end{itemize}
      \vfill
      \mintocaml[firstline=9,lastline=13,highlightlines=12]{../src/backtrace/backtrace.ml}
      \endminipage
    \end{column}
    \begin{column}{0.5\textwidth}
      \onslide*<1>{\listStashBacktrace[highlightlines=91]{}}
      \onslide*<2>{\listStashBacktrace[highlightlines={91,117-118}]{}}
      \onslide*<3->{\listStashBacktrace[highlightlines={94,106,109-110}]{}}
    \end{column}
  \end{columns}
\end{frame}

\newcommand\listFrameDescr[1][]{
  \listocaml[firstline=111,lastline=124,#1]{../ocaml/asmcomp/emitaux.ml}{asmcomp/emitaux.ml:111}}
\newcommand\listDebugInfo[1][]{
  \listocaml[firstline=38,lastline=49,#1]{../ocaml/lambda/debuginfo.mli}{lambda/debuginfo.mli:38}}

\begin{frame}{Backtraces}{\typename{frame\_descr} and \typename{frame\_debuginfo} in a nutshell}
  \begin{columns}[c]
    \begin{column}{0.5\textwidth}
      \begin{itemize}
        \item<1-> For every return address in an OCaml program, there is a associated \typename{frame\_descr}
        \item<2-> A \typename{frame\_descr} specifies
        \begin{itemize}
          \item<2-> The size of the function stack frame
          \item<3-> Where live values are on the stack (so that the GC can mark them as live and prevent from being swept)
        \end{itemize}
        \item<4-> When compiled with debug information, \typename{frame\_descr} will be linked to a \typename{frame\_debuginfo}
        \begin{itemize}
          \item<5-> Contains information about the position in the source code (file name, line and column number)
        \end{itemize}
      \end{itemize}
    \end{column}
    \begin{column}{0.5\textwidth}
        \onslide*<1>{\listFrameDescr[highlightlines={118-122}]{}}
        \onslide*<2>{\listFrameDescr[highlightlines={120}]{}}
        \onslide*<3>{\listFrameDescr[highlightlines={121}]{}}
        \onslide*<4->{\listFrameDescr[highlightlines={122}]{}}
        \medskip
        \onslide*<1-3>{\listDebugInfo{}}
        \onslide*<4>{\listDebugInfo[highlightlines={38,49}]{}}
        \onslide*<5>{\listDebugInfo[highlightlines={39-46}]{}}
    \end{column}
  \end{columns}
\end{frame}

\begin{frame}{Backtraces}{Scanning the stack with \typename{frame\_descr}}
  \begin{columns}[c]
    \begin{column}{0.5\textwidth}
      \begin{itemize}
        \item Given the return address of the code that raises the exception by calling \funcname{caml\_raise\_exn} (\funcarg{pc} argument of \funcname{caml\_stash\_backtrace})
        \item And the position of the stack pointer (\funcarg{sp} argument of \funcname{caml\_stash\_backtrace})
        \item The frame size given by \typename{frame\_descr}.\funcarg{frame\_size}, allows to find the end of the previous frame
        \item And the return address of the caller of this function
        \bigskip
        \onslide<3->{
        \item Note that traps (here the reddish \funcname{ExnA}, \funcname{ExnB} and \funcname{ExnC} blocks) belong to the frame that installed them, and so the frame size changes when traps are removed
        }
      \end{itemize}
    \end{column}
    \begin{column}{0.5\textwidth}
      \raggedleft
      \onslide*<1>{\providecommand\step{2}\input{diagrams/backtrace/backtrace.tex}}%
      \onslide*<2>{\providecommand\step{1}\input{diagrams/backtrace/scanstack.tex}}%
      \onslide*<3>{\providecommand\step{2}\input{diagrams/backtrace/scanstack.tex}}%
      \onslide*<4>{\providecommand\step{3}\input{diagrams/backtrace/scanstack.tex}}%
      \onslide*<5>{\providecommand\step{4}\input{diagrams/backtrace/scanstack.tex}}%
      \onslide*<6>{\providecommand\step{5}\input{diagrams/backtrace/scanstack.tex}}%
    \end{column}
  \end{columns}
\end{frame}

% Duplicate of previous-previous slide
\begin{frame}{Backtraces}{Continuing on \funcname{caml\_stash\_backtrace}}
  \begin{columns}[c]
    \begin{column}{0.5\textwidth}
      \minipage[c][0.75\textheight][s]{\columnwidth}
      \begin{itemize}
          \color{gray}
        \item A C function provided by the runtime (specific to native code)
        \item It scans the stack, between the stack pointer at the raising point up to the next trap, in order to collect the names of the detected function frames
        \item It uses the same technique and information as the Garbage Collector (GC) uses to scan the values live on the stack
      \end{itemize}
      \begin{itemize}
        \item For each function frame detected, store a pointer to the \typename{frame\_descr} in the \localname{domain\_state->backtrace\_buffer} array, effectively constructing the backtrace
      \end{itemize}
      \onslide<2->{
        \begin{itemize}
            \smallskip
          \item Constructed backtrace:
            \funcname{definitely\_raise},
            \onslide<3->{\funcname{probably\_raise}}
        \end{itemize}
      }
      \vfill
      \mintocaml[firstline=9,lastline=13,highlightlines=12]{../src/backtrace/backtrace.ml}
      \endminipage
    \end{column}
    \begin{column}{0.5\textwidth}
      \raggedleft
      \onslide*<1>{\listStashBacktrace[highlightlines={114-115}]{}}
      \onslide*<2>{\providecommand\step{3}\input{diagrams/backtrace/backtrace.tex}}%
      \onslide*<3>{\providecommand\step{4}\input{diagrams/backtrace/backtrace.tex}}%
    \end{column}
  \end{columns}
\end{frame}

\begin{frame}{Backtraces}{Back to \funcname{caml\_raise\_exn}}
  \begin{columns}[c]
    \begin{column}{0.5\textwidth}
      \minipage[c][0.66\textheight][s]{\columnwidth}
      \begin{itemize}\color{gray}
        \item Checks wheteher exceptions backtrace should be recorded, by checking the \funcarg{domain\_state->backtrace\_active} value
        \item Switches to the C stack to be able to execute C code
        \item Calls \funcname{caml\_stash\_backtrace}, to collect the backtrace up to the next trap
      \end{itemize}
      \onslide*<2->{
        \begin{itemize}
          \item<2-> Executes the exception handler in \funcname{probably\_raise} with \funcname{RESTORE\_EXN\_HANDLER\_OCAML}
            \begin{itemize}
              \item Set \regname{RSP} to \localname{domain\_state->exn\_handler} value
              \item Restore \localname{domain\_state->exn\_handler} from trap
              \item Jump to the handler code with \funcname{ret}
            \end{itemize}
        \end{itemize}
      }
      \vfill
      \onslide*<1>{\mintocaml[firstline=9,lastline=13,highlightlines=12]{../src/backtrace/backtrace.ml}}
      \onslide*<2->{\mintocaml[firstline=15,lastline=21,highlightlines={19-21}]{../src/backtrace/backtrace.ml}}
      \endminipage
    \end{column}
    \begin{column}{0.5\textwidth}
      \centering
      \onslide*<1>{
        \mintc[firstline=190,lastline=192]{../ocaml/runtime/amd64.S}
        \mintc[firstline=234,lastline=237]{../ocaml/runtime/amd64.S}
      }
      \onslide*<2>{
        \mintc[firstline=190,lastline=192,highlightlines={191-192}]{../ocaml/runtime/amd64.S}
        \mintc[firstline=234,lastline=237,highlightlines={235-237}]{../ocaml/runtime/amd64.S}
      }
      \onslide*<1>{\listCamlRaiseExn[highlightlines=720]{}}
      \onslide*<2>{\listCamlRaiseExn[highlightlines=722]{}}
      \onslide*<3->{\providecommand\step{5}\input{diagrams/backtrace/backtrace.tex}}
    \end{column}
  \end{columns}
\end{frame}

\begin{frame}{Backtraces}{\funcname{caml\_probably\_raise}'s handler doesn't catch \typename{ExnC}}
  \begin{columns}[c]
    \begin{column}{0.45\textwidth}
      \minipage[c][0.75\textheight][s]{\columnwidth}
      \begin{itemize}
        \item<1-> \funcname{match \dots with}\footnotemark pattern matching can also match exceptions
        \item<1-> The CMM of \funcname{probably\_raise} shows that this \funcname{match \dots with} is implemented with a pattern matching and a \funcname{try \dots with}
        \item<1-> But this one only matches on \typename{ExnA} exception
        \item<2-> Since the pattern matching fails to catch the exception, \funcname{reraise} is called with our current \typename{ExnC} exception
        \item<3-> Disassembly shows that \funcname{reraise} is actually a call to \funcname{caml\_reraise\_exn}, which is also an assembly function provided by the runtime
      \end{itemize}
      \vfill
      \mintocaml[firstline=15,lastline=21,highlightlines={18-21}]{../src/backtrace/backtrace.ml}
      \endminipage
    \end{column}
    \begin{column}{0.55\textwidth}
      \centering
      \onslide*<1>{\mintcmm[highlightlines={10-18}]{../src/backtrace/camlBacktrace\_\_probably\_raise\_351.cmm}}
      \onslide*<2>{\listcmm[highlightlines=18]{../src/backtrace/camlBacktrace\_\_probably\_raise_351.cmm}{CMM of probably\_raise}}
      \onslide*<3>{\mintobjdump[firstline=5,lastline=35,highlightlines=31]{../src/backtrace/camlBacktrace\_\_probably\_raise_351.objdump}}
    \end{column}
  \end{columns}
  \only<1>{\footnotetext[1]{https://blog.janestreet.com/pattern-matching-and-exception-handling-unite/}}
\end{frame}

\newcommand\listCamlReraiseExn[1][]{
  \listasm[firstline=727,lastline=734,#1]{../ocaml/runtime/amd64.S}{runtime/amd64.S:727}}

\begin{frame}{Backtraces}{Introducing \funcname{caml\_reraise\_exn}}
  \begin{columns}[c]
    \begin{column}{0.5\textwidth}
      \minipage[c][0.66\textheight][s]{\columnwidth}
      \begin{itemize}
        \item<1-> The exception have to be reraised to the next handler
        \item<2-> First, checks if the backtrace should be collected with \funcarg{domain\_state->backtrace\_active}
        \only<3>{
        \begin{itemize}
            \item If not, behaves like a \funcname{raise\_notrace} with \funcname{RESTORE\_EXN\_HANDLER\_OCAML}
        \end{itemize}
        }
        \item<4-> Jumps into \funcname{caml\_raise\_exn}
        \item<5-> Calls \funcname{caml\_stash\_backtrace} again, to scan the next part of the stack: from the trap just pop-ed to the next trap
      \end{itemize}
      \vfill
      \mintocaml[firstline=15,lastline=21,highlightlines={18-21}]{../src/backtrace/backtrace.ml}
      \endminipage
    \end{column}
    \begin{column}{0.5\textwidth}
      \raggedleft
      \onslide*<1>{\listCamlReraiseExn{}}
      \onslide*<2>{\listCamlReraiseExn[highlightlines=729]{}}
      \onslide*<3>{
        \mintc[firstline=190,lastline=192,highlightlines={191-192}]{../ocaml/runtime/amd64.S}
        \mintc[firstline=234,lastline=237,highlightlines={235-237}]{../ocaml/runtime/amd64.S}
        \medskip
        \listCamlReraiseExn[highlightlines=731]{}
      }
      \onslide*<4-5>{
        % TODO refactor with \listCamlReraiseExn
        \mintasm[firstline=727,lastline=734,highlightlines=730]{../ocaml/runtime/amd64.S}
        \medskip
      }
      \onslide*<4>{\listCamlRaiseExn[highlightlines=711]{}}
      \onslide*<5>{\listCamlRaiseExn[highlightlines=720]{}}
    \end{column}
  \end{columns}
\end{frame}

\begin{frame}{Backtraces}{Backtrace collection continues}
  \begin{columns}[c]
    \begin{column}{0.55\textwidth}
      \minipage[c][0.75\textheight][s]{\columnwidth}
      \begin{itemize}
        \item<1-> \funcname{caml\_stash\_backtrace} scans the older part of the stack, up to the next trap
        \item<6-> Controls jumps to the nested exception handler in \funcname{maybe\_raise}, which matches only on \typename{ExnB}
        \item<7-> \funcname{caml\_reraise\_exn} is called again, backtrace collection resumes with \funcname{caml\_stash\_backtrace}
      \end{itemize}
      \medskip
      \begin{itemize}
        \item<2-> Constructed backtrace:
        \begin{itemize}
          \item <2-> \funcname{definitely\_raise}
          \item <2-> \funcname{probably\_raise}
          \item <3-> \funcname{probably\_raise} (re-raise)
          \item <4-> \funcname{maybe\_raise}
          \item <5-> \funcname{main}
          \item <8-> \funcname{main} (re-raise)
        \end{itemize}
      \end{itemize}
      \vfill
      \onslide*<1-5>{\mintocaml[firstline=15,lastline=21]{../src/backtrace/backtrace.ml}}
      \onslide*<6>{\mintocaml[firstline=28,lastline=34,highlightlines=31]{../src/backtrace/backtrace.ml}}
      \onslide*<7->{\mintocaml[firstline=28,lastline=34,highlightlines=33]{../src/backtrace/backtrace.ml}}
      \endminipage
    \end{column}
    \begin{column}{0.45\textwidth}
      \raggedleft
      \onslide*<1>{\providecommand\step{6}\input{diagrams/backtrace/backtrace.tex}}%
      \onslide*<2>{\providecommand\step{6}\input{diagrams/backtrace/backtrace.tex}}%
      \onslide*<3>{\providecommand\step{7}\input{diagrams/backtrace/backtrace.tex}}%
      \onslide*<4>{\providecommand\step{8}\input{diagrams/backtrace/backtrace.tex}}%
      \onslide*<5>{\providecommand\step{9}\input{diagrams/backtrace/backtrace.tex}}%
      \onslide*<6>{\providecommand\step{10}\input{diagrams/backtrace/backtrace.tex}}%
      \onslide*<7>{\providecommand\step{11}\input{diagrams/backtrace/backtrace.tex}}%
      \onslide*<8>{\providecommand\step{12}\input{diagrams/backtrace/backtrace.tex}}%
      \onslide*<9>{\providecommand\step{13}\input{diagrams/backtrace/backtrace.tex}}%
    \end{column}
  \end{columns}
\end{frame}

\begin{frame}{Backtraces}{Printing the backtrace}
  \begin{columns}[c]
    \begin{column}{0.45\textwidth}
      \minipage[c][0.66\textheight][s]{\columnwidth}
      \begin{itemize}
        \item<1-> The exception backtrace can now be printed to \funcarg{stdout} with \funcname{Printexc.print_backtrace}
        \item<2-> First the exception backtrace is retrieved into an OCaml array with \funcname{get_raw_backtrace }
        \item<3-> Which is implemented as the \funcname{caml_get_exception_raw_backtrace} C function in the runtime
        \item<4-> And finally printed with \funcname{print_exception_backtrace}
      \end{itemize}
      \vfill
      \mintocaml[firstline=28,lastline=34,highlightlines=33]{../src/backtrace/backtrace.ml}
      \endminipage
    \end{column}
    \begin{column}{0.55\textwidth}
      \onslide*<1,2>{
        \mintocaml[firstline=100,lastline=101]{../ocaml/stdlib/printexc.ml}
      }
      \only<1>{
        \listocaml[firstline=170,lastline=172]{../ocaml/stdlib/printexc.ml}{stdlib/printexc.ml:170}
      }
      \only<2>{
        \listocaml[firstline=170,lastline=172,highlightlines=172]{../ocaml/stdlib/printexc.ml}{stdlib/printexc.ml:170}
      }
      \only<3>{
        \mintc[firstline=154,lastline=156]{../ocaml/runtime/backtrace.c}
        \listc[firstline=171,lastline=192,highlightlines={184-187}]{../ocaml/runtime/backtrace.c}{runtime/backtrace.c:154}
      }
      \only<4>{
        \listocaml[firstline=155,lastline=165]{../ocaml/stdlib/printexc.ml}{stdlib/printexc.ml:55}
      }
    \end{column}
  \end{columns}
\end{frame}


\subsection{The multiple kinds of raise}
\frameSubsection{}{}

\begin{frame}{Backtraces}{When backtraces are not needed}
  \begin{columns}[c]
    \begin{column}{0.45\textwidth}
      \minipage[c][0.66\textheight][s]{\columnwidth}
      \begin{itemize}
        \item<1-> What if the backtrace for an exception is never needed?
        \item<2-> It's possible to raise the exception explicitly with \funcname{raise\_notrace} to make raising as cheap as possible
        \item<3-> An example of use in the \funcname{dynlink} library of the OCaml compiler
      \end{itemize}
      \vfill
      \onslide<3->{
      \listocaml[firstline=205,lastline=209,highlightlines=207]{../ocaml/otherlibs/dynlink/dynlink\_compilerlibs/misc.ml}{otherlibs/dynlink/dynlink\_compilerlibs/misc.ml:205}
      }
      \endminipage
    \end{column}
    \begin{column}{0.55\textwidth}
    \only<4>{
      \mintcmm[highlightlines=3]{../src/notrace/camlNotrace\_\_fun_347.cmm}
      \listcmm{../src/notrace/camlNotrace\_\_all\_somes\_327.cmm}{all\_somes}
    }
    \onslide*<5->{
      \listobjdump[highlightlines={6-9}]{../src/notrace/camlNotrace\_\_fun_347.objdump}{anonymous function from \funcname{all\_somes}}
    }
    \end{column}
  \end{columns}
\end{frame}

\begin{frame}{Backtraces}{Summary table of the multiple kinds of raise}
  \begin{itemize}
    \item When debugging information is enabled and if the backtrace of an exception isn't needed, it's possible to raise with \funcname{raise\_notrace} for performance
    \item When debugging information is \emph{not} enabled, the compiler optimises \funcname{raise} calls to inline assembly (same effect as using \funcname{raise\_notrace})
  \end{itemize}
  \bigskip
  \centering
  \begin{tabular}{| l | c | c | c | c |}
    \hline
    \parbox{10em}{Debug information \\ (\funcarg{-g} flag)}  & \multicolumn{2}{|c|}{Disabled}                                     & \multicolumn{2}{|c|}{Enabled} \\
    \hline
    Raise kind                                               & \funcname{raise} & \funcname{raise\_notrace}                       & \funcname{raise} & \funcname{raise\_notrace} \\
    \hline
    Compilation                                              & \parbox{8em}{inline assembly\\ (optimization)} & inline assembly   & \funcname{caml\_raise\_exn} & inline assembly \\
    \hline
  \end{tabular}
\end{frame}


%
% Takeaway #5
%
\subsection*{Takeaway \#5}
\frameSubsectionTakeaway{}
\againframe<5>{takeaway}
}%
      \onslide*<3>{\providecommand\step{7}%
% Backtraces
%
\subsection{One advantage of raising with debugging information}
\frameSubsection{}{}

\begin{frame}{Backtraces}{Debugging information \& source code}
  \begin{columns}[c]
    \begin{column}{0.5\textwidth}
      \begin{itemize}
        \item Raising an exception is fun and games, but how do we know where the exception originated? $\rightarrow$ With the backtrace associated with the exception!
        \item In order to get a backtrace from the point where an exception was raised, it's necessary to compile with debugging information enabled (\funcarg{-g} flag for the compiler)
        \item Same example as in the `Nested handlers' section
      \end{itemize}
    \end{column}
    \begin{column}{0.5\textwidth}
      \listocaml[firstline=9,lastline=34]{../src/backtrace/backtrace.ml}{backtrace/backtrace.ml}
    \end{column}
  \end{columns}
\end{frame}

\begin{frame}{Backtraces}{CMM and disassembly of \funcname{definitely\_raise}}
  \begin{columns}[c]
    \begin{column}{0.5\textwidth}
      \minipage[c][0.66\textheight][s]{\columnwidth}
      \begin{itemize}
        \item<1-> An effect of compiling with debugging information (\funcarg{-g}): the CMM no longer uses \funcname{raise\_notrace} to raise the exception but \funcname{raise} instead
        \item<2-> Disassembly shows that CMM \funcname{raise} is actually a call to \funcname{caml\_raise\_exn}, a function in assembler provided by the runtime
      \end{itemize}
      \vfill
      \mintocaml[firstline=9,lastline=13,highlightlines=12]{../src/backtrace/backtrace.ml}
      \endminipage
    \end{column}
    \begin{column}{0.5\textwidth}
      \onslide*<1>{
        \mintcmm[highlightlines=5]{../src/backtrace/camlBacktrace\_\_definitely\_raise\_347.cmm}
      }
      \onslide*<2>{
        \mintobjdump[firstline=2,highlightlines=14]{../src/backtrace/camlBacktrace\_\_definitely\_raise\_347.objdump}
      }
    \end{column}
  \end{columns}
\end{frame}

\begin{frame}{Backtraces}{Introducing \funcname{caml\_raise\_exn}}
  \begin{columns}[c]
    \begin{column}{0.5\textwidth}
      \minipage[c][0.66\textheight][s]{\columnwidth}
      \onslide*<1>{
        \begin{itemize}
          \item Raising the exception is performed using \funcname{caml\_raise\_exn} which is
            \begin{itemize}
              \item An assembly function provided by the runtime
              \item Architecture specific
            \end{itemize}
          \item Let's see what it does differently from \funcname{raise\_notrace}\dots
          \item We are about to raise \typename{ExnC} with \funcname{caml\_raise\_exn} from \funcname{definitely\_raise}
        \end{itemize}
      }
      \onslide*<2>{
        \mintobjdump[firstline=2,highlightlines=14]{../src/backtrace/camlBacktrace\_\_definitely\_raise\_347.objdump}
      }
      \vfill
      \mintocaml[firstline=9,lastline=13,highlightlines=12]{../src/backtrace/backtrace.ml}
      \endminipage
    \end{column}
    \begin{column}{0.5\textwidth}
      \centering
      \providecommand\step{1}\input{diagrams/backtrace/backtrace.tex}
    \end{column}
  \end{columns}
\end{frame}

\begin{frame}{Backtraces}{But before that, we need more fields from \typename{caml\_domain\_state}}
  \begin{columns}
    \begin{column}{0.5\textwidth}
      \begin{itemize}
        \item Remember \typename{caml\_domain\_state} the per-domain runtime state?
        \item Some other members are of interest to us now\dots
        \item \localname{backtrace\_active} is used to know if the runtime should collect backtraces
        \item \localname{backtrace\_active} can be set by
          \begin{itemize}
            \item The \funcarg{b} flag in the \funcname{OCAMLRUNPARAM} environment variable
            \item \mintinline{ocaml}{Printexc.record_backtrace : bool -> unit} \footnotemark
          \end{itemize}
      \end{itemize}
    \end{column}
    \begin{column}{0.5\textwidth}
      \begin{listing}
        \mintc[highlightlines={9-11}]{src/ocaml/domain\_state\_backtrace.h}
        \caption{runtime/caml/domain\_state.h}
      \end{listing}
    \end{column}
  \end{columns}
  \footnotetext[1]{https://v2.ocaml.org/api/Printexc.html}
\end{frame}

\newcommand\listCamlRaiseExn[1][]{
  \listasm[firstline=702,lastline=725,#1]{../ocaml/runtime/amd64.S}{runtime/amd64.S:702}}

\begin{frame}{Backtraces}{Walk through \funcname{caml\_raise\_exn}}
  \begin{columns}[T]
    \begin{column}{0.5\textwidth}
      \begin{itemize}
        \item<1-> First checks whether exceptions backtrace should be recorded
        \item<1-> By checking the \funcarg{domain\_state->backtrace\_active} value
        \item<2-> If not, acts as \funcname{raise\_notrace} with \funcname{RESTORE\_EXN\_HANDLER\_OCAML}
          \begin{itemize}
            \item Set \regname{RSP} to \localname{domain\_state->exn\_handler} value
            \item Restore \localname{domain\_state->exn\_handler} from trap
            \item Jump with \funcname{ret} (same effect as using \funcname{pop} and \funcname{jmp})
          \end{itemize}
        \item<3-> Otherwise, switches to the C stack to be able to execute C code
        \item<4-> Calls \funcname{caml\_stash\_backtrace}, a C function provided by the runtime\ignorespaces
          \onslide<6->{, with
            \begin{itemize}
              \item \funcarg{exn}: exception value
              \item \funcarg{pc}: code address of the raising point
              \item \funcarg{sp}: stack address of the raising function
              \item \funcarg{trapsp}: stack address of the next handler
            \end{itemize}
          }
      \end{itemize}
      \bigskip
      \mintocaml[firstline=9,lastline=13,highlightlines=12]{../src/backtrace/backtrace.ml}
    \end{column}
    \begin{column}{0.5\textwidth}
      \raggedleft
      \onslide*<1,3-4>{
        \mintc[firstline=190,lastline=192]{../ocaml/runtime/amd64.S}
        \mintc[firstline=234,lastline=237]{../ocaml/runtime/amd64.S}
      }
      \onslide*<2>{
        \mintc[firstline=190,lastline=192,highlightlines={191-192}]{../ocaml/runtime/amd64.S}
        \mintc[firstline=234,lastline=237,highlightlines={235-237}]{../ocaml/runtime/amd64.S}
      }
      \onslide*<1>{\listCamlRaiseExn[highlightlines=705]{}}%
      \onslide*<2>{\listCamlRaiseExn[highlightlines={707-708}]{}}%
      \onslide*<3>{\listCamlRaiseExn[highlightlines={712-714}]{}}%
      \onslide*<4>{\listCamlRaiseExn[highlightlines={715-720}]{}}%
      \onslide*<5>{\providecommand\step{1}\input{diagrams/backtrace/backtrace.tex}}%
      \onslide*<6>{\providecommand\step{2}\input{diagrams/backtrace/backtrace.tex}}%
    \end{column}
  \end{columns}
\end{frame}

\newcommand\listStashBacktrace[1][]{
  \listc[firstline=91,lastline=120,#1]{../ocaml/runtime/backtrace\_nat.c}{runtime/backtrace\_nat.c:91}}

\begin{frame}{Backtraces}{Introducing \funcname{caml\_stash\_backtrace}}
  \begin{columns}[c]
    \begin{column}{0.5\textwidth}
      \minipage[c][0.66\textheight][s]{\columnwidth}
      \begin{itemize}
        \item<1-> A C function provided by the runtime (specific to native code)
        \item<2-> It scans the stack, between the stack pointer at the raising point up to the next trap, in order to collect the names of the detected function frames
          \onslide*<2>{
            \begin{itemize}
              \item And more information, like line number, column number...
            \end{itemize}
          }
        \item<3-> It uses the same technique and information as the Garbage Collector (GC) uses to scan the values live on the stack
          \onslide*<4>{
            \begin{itemize}
              \item \typename{frame\_descr} are at the core of stack unwinding
            \end{itemize}
          }
      \end{itemize}
      \vfill
      \mintocaml[firstline=9,lastline=13,highlightlines=12]{../src/backtrace/backtrace.ml}
      \endminipage
    \end{column}
    \begin{column}{0.5\textwidth}
      \onslide*<1>{\listStashBacktrace[highlightlines=91]{}}
      \onslide*<2>{\listStashBacktrace[highlightlines={91,117-118}]{}}
      \onslide*<3->{\listStashBacktrace[highlightlines={94,106,109-110}]{}}
    \end{column}
  \end{columns}
\end{frame}

\newcommand\listFrameDescr[1][]{
  \listocaml[firstline=111,lastline=124,#1]{../ocaml/asmcomp/emitaux.ml}{asmcomp/emitaux.ml:111}}
\newcommand\listDebugInfo[1][]{
  \listocaml[firstline=38,lastline=49,#1]{../ocaml/lambda/debuginfo.mli}{lambda/debuginfo.mli:38}}

\begin{frame}{Backtraces}{\typename{frame\_descr} and \typename{frame\_debuginfo} in a nutshell}
  \begin{columns}[c]
    \begin{column}{0.5\textwidth}
      \begin{itemize}
        \item<1-> For every return address in an OCaml program, there is a associated \typename{frame\_descr}
        \item<2-> A \typename{frame\_descr} specifies
        \begin{itemize}
          \item<2-> The size of the function stack frame
          \item<3-> Where live values are on the stack (so that the GC can mark them as live and prevent from being swept)
        \end{itemize}
        \item<4-> When compiled with debug information, \typename{frame\_descr} will be linked to a \typename{frame\_debuginfo}
        \begin{itemize}
          \item<5-> Contains information about the position in the source code (file name, line and column number)
        \end{itemize}
      \end{itemize}
    \end{column}
    \begin{column}{0.5\textwidth}
        \onslide*<1>{\listFrameDescr[highlightlines={118-122}]{}}
        \onslide*<2>{\listFrameDescr[highlightlines={120}]{}}
        \onslide*<3>{\listFrameDescr[highlightlines={121}]{}}
        \onslide*<4->{\listFrameDescr[highlightlines={122}]{}}
        \medskip
        \onslide*<1-3>{\listDebugInfo{}}
        \onslide*<4>{\listDebugInfo[highlightlines={38,49}]{}}
        \onslide*<5>{\listDebugInfo[highlightlines={39-46}]{}}
    \end{column}
  \end{columns}
\end{frame}

\begin{frame}{Backtraces}{Scanning the stack with \typename{frame\_descr}}
  \begin{columns}[c]
    \begin{column}{0.5\textwidth}
      \begin{itemize}
        \item Given the return address of the code that raises the exception by calling \funcname{caml\_raise\_exn} (\funcarg{pc} argument of \funcname{caml\_stash\_backtrace})
        \item And the position of the stack pointer (\funcarg{sp} argument of \funcname{caml\_stash\_backtrace})
        \item The frame size given by \typename{frame\_descr}.\funcarg{frame\_size}, allows to find the end of the previous frame
        \item And the return address of the caller of this function
        \bigskip
        \onslide<3->{
        \item Note that traps (here the reddish \funcname{ExnA}, \funcname{ExnB} and \funcname{ExnC} blocks) belong to the frame that installed them, and so the frame size changes when traps are removed
        }
      \end{itemize}
    \end{column}
    \begin{column}{0.5\textwidth}
      \raggedleft
      \onslide*<1>{\providecommand\step{2}\input{diagrams/backtrace/backtrace.tex}}%
      \onslide*<2>{\providecommand\step{1}\input{diagrams/backtrace/scanstack.tex}}%
      \onslide*<3>{\providecommand\step{2}\input{diagrams/backtrace/scanstack.tex}}%
      \onslide*<4>{\providecommand\step{3}\input{diagrams/backtrace/scanstack.tex}}%
      \onslide*<5>{\providecommand\step{4}\input{diagrams/backtrace/scanstack.tex}}%
      \onslide*<6>{\providecommand\step{5}\input{diagrams/backtrace/scanstack.tex}}%
    \end{column}
  \end{columns}
\end{frame}

% Duplicate of previous-previous slide
\begin{frame}{Backtraces}{Continuing on \funcname{caml\_stash\_backtrace}}
  \begin{columns}[c]
    \begin{column}{0.5\textwidth}
      \minipage[c][0.75\textheight][s]{\columnwidth}
      \begin{itemize}
          \color{gray}
        \item A C function provided by the runtime (specific to native code)
        \item It scans the stack, between the stack pointer at the raising point up to the next trap, in order to collect the names of the detected function frames
        \item It uses the same technique and information as the Garbage Collector (GC) uses to scan the values live on the stack
      \end{itemize}
      \begin{itemize}
        \item For each function frame detected, store a pointer to the \typename{frame\_descr} in the \localname{domain\_state->backtrace\_buffer} array, effectively constructing the backtrace
      \end{itemize}
      \onslide<2->{
        \begin{itemize}
            \smallskip
          \item Constructed backtrace:
            \funcname{definitely\_raise},
            \onslide<3->{\funcname{probably\_raise}}
        \end{itemize}
      }
      \vfill
      \mintocaml[firstline=9,lastline=13,highlightlines=12]{../src/backtrace/backtrace.ml}
      \endminipage
    \end{column}
    \begin{column}{0.5\textwidth}
      \raggedleft
      \onslide*<1>{\listStashBacktrace[highlightlines={114-115}]{}}
      \onslide*<2>{\providecommand\step{3}\input{diagrams/backtrace/backtrace.tex}}%
      \onslide*<3>{\providecommand\step{4}\input{diagrams/backtrace/backtrace.tex}}%
    \end{column}
  \end{columns}
\end{frame}

\begin{frame}{Backtraces}{Back to \funcname{caml\_raise\_exn}}
  \begin{columns}[c]
    \begin{column}{0.5\textwidth}
      \minipage[c][0.66\textheight][s]{\columnwidth}
      \begin{itemize}\color{gray}
        \item Checks wheteher exceptions backtrace should be recorded, by checking the \funcarg{domain\_state->backtrace\_active} value
        \item Switches to the C stack to be able to execute C code
        \item Calls \funcname{caml\_stash\_backtrace}, to collect the backtrace up to the next trap
      \end{itemize}
      \onslide*<2->{
        \begin{itemize}
          \item<2-> Executes the exception handler in \funcname{probably\_raise} with \funcname{RESTORE\_EXN\_HANDLER\_OCAML}
            \begin{itemize}
              \item Set \regname{RSP} to \localname{domain\_state->exn\_handler} value
              \item Restore \localname{domain\_state->exn\_handler} from trap
              \item Jump to the handler code with \funcname{ret}
            \end{itemize}
        \end{itemize}
      }
      \vfill
      \onslide*<1>{\mintocaml[firstline=9,lastline=13,highlightlines=12]{../src/backtrace/backtrace.ml}}
      \onslide*<2->{\mintocaml[firstline=15,lastline=21,highlightlines={19-21}]{../src/backtrace/backtrace.ml}}
      \endminipage
    \end{column}
    \begin{column}{0.5\textwidth}
      \centering
      \onslide*<1>{
        \mintc[firstline=190,lastline=192]{../ocaml/runtime/amd64.S}
        \mintc[firstline=234,lastline=237]{../ocaml/runtime/amd64.S}
      }
      \onslide*<2>{
        \mintc[firstline=190,lastline=192,highlightlines={191-192}]{../ocaml/runtime/amd64.S}
        \mintc[firstline=234,lastline=237,highlightlines={235-237}]{../ocaml/runtime/amd64.S}
      }
      \onslide*<1>{\listCamlRaiseExn[highlightlines=720]{}}
      \onslide*<2>{\listCamlRaiseExn[highlightlines=722]{}}
      \onslide*<3->{\providecommand\step{5}\input{diagrams/backtrace/backtrace.tex}}
    \end{column}
  \end{columns}
\end{frame}

\begin{frame}{Backtraces}{\funcname{caml\_probably\_raise}'s handler doesn't catch \typename{ExnC}}
  \begin{columns}[c]
    \begin{column}{0.45\textwidth}
      \minipage[c][0.75\textheight][s]{\columnwidth}
      \begin{itemize}
        \item<1-> \funcname{match \dots with}\footnotemark pattern matching can also match exceptions
        \item<1-> The CMM of \funcname{probably\_raise} shows that this \funcname{match \dots with} is implemented with a pattern matching and a \funcname{try \dots with}
        \item<1-> But this one only matches on \typename{ExnA} exception
        \item<2-> Since the pattern matching fails to catch the exception, \funcname{reraise} is called with our current \typename{ExnC} exception
        \item<3-> Disassembly shows that \funcname{reraise} is actually a call to \funcname{caml\_reraise\_exn}, which is also an assembly function provided by the runtime
      \end{itemize}
      \vfill
      \mintocaml[firstline=15,lastline=21,highlightlines={18-21}]{../src/backtrace/backtrace.ml}
      \endminipage
    \end{column}
    \begin{column}{0.55\textwidth}
      \centering
      \onslide*<1>{\mintcmm[highlightlines={10-18}]{../src/backtrace/camlBacktrace\_\_probably\_raise\_351.cmm}}
      \onslide*<2>{\listcmm[highlightlines=18]{../src/backtrace/camlBacktrace\_\_probably\_raise_351.cmm}{CMM of probably\_raise}}
      \onslide*<3>{\mintobjdump[firstline=5,lastline=35,highlightlines=31]{../src/backtrace/camlBacktrace\_\_probably\_raise_351.objdump}}
    \end{column}
  \end{columns}
  \only<1>{\footnotetext[1]{https://blog.janestreet.com/pattern-matching-and-exception-handling-unite/}}
\end{frame}

\newcommand\listCamlReraiseExn[1][]{
  \listasm[firstline=727,lastline=734,#1]{../ocaml/runtime/amd64.S}{runtime/amd64.S:727}}

\begin{frame}{Backtraces}{Introducing \funcname{caml\_reraise\_exn}}
  \begin{columns}[c]
    \begin{column}{0.5\textwidth}
      \minipage[c][0.66\textheight][s]{\columnwidth}
      \begin{itemize}
        \item<1-> The exception have to be reraised to the next handler
        \item<2-> First, checks if the backtrace should be collected with \funcarg{domain\_state->backtrace\_active}
        \only<3>{
        \begin{itemize}
            \item If not, behaves like a \funcname{raise\_notrace} with \funcname{RESTORE\_EXN\_HANDLER\_OCAML}
        \end{itemize}
        }
        \item<4-> Jumps into \funcname{caml\_raise\_exn}
        \item<5-> Calls \funcname{caml\_stash\_backtrace} again, to scan the next part of the stack: from the trap just pop-ed to the next trap
      \end{itemize}
      \vfill
      \mintocaml[firstline=15,lastline=21,highlightlines={18-21}]{../src/backtrace/backtrace.ml}
      \endminipage
    \end{column}
    \begin{column}{0.5\textwidth}
      \raggedleft
      \onslide*<1>{\listCamlReraiseExn{}}
      \onslide*<2>{\listCamlReraiseExn[highlightlines=729]{}}
      \onslide*<3>{
        \mintc[firstline=190,lastline=192,highlightlines={191-192}]{../ocaml/runtime/amd64.S}
        \mintc[firstline=234,lastline=237,highlightlines={235-237}]{../ocaml/runtime/amd64.S}
        \medskip
        \listCamlReraiseExn[highlightlines=731]{}
      }
      \onslide*<4-5>{
        % TODO refactor with \listCamlReraiseExn
        \mintasm[firstline=727,lastline=734,highlightlines=730]{../ocaml/runtime/amd64.S}
        \medskip
      }
      \onslide*<4>{\listCamlRaiseExn[highlightlines=711]{}}
      \onslide*<5>{\listCamlRaiseExn[highlightlines=720]{}}
    \end{column}
  \end{columns}
\end{frame}

\begin{frame}{Backtraces}{Backtrace collection continues}
  \begin{columns}[c]
    \begin{column}{0.55\textwidth}
      \minipage[c][0.75\textheight][s]{\columnwidth}
      \begin{itemize}
        \item<1-> \funcname{caml\_stash\_backtrace} scans the older part of the stack, up to the next trap
        \item<6-> Controls jumps to the nested exception handler in \funcname{maybe\_raise}, which matches only on \typename{ExnB}
        \item<7-> \funcname{caml\_reraise\_exn} is called again, backtrace collection resumes with \funcname{caml\_stash\_backtrace}
      \end{itemize}
      \medskip
      \begin{itemize}
        \item<2-> Constructed backtrace:
        \begin{itemize}
          \item <2-> \funcname{definitely\_raise}
          \item <2-> \funcname{probably\_raise}
          \item <3-> \funcname{probably\_raise} (re-raise)
          \item <4-> \funcname{maybe\_raise}
          \item <5-> \funcname{main}
          \item <8-> \funcname{main} (re-raise)
        \end{itemize}
      \end{itemize}
      \vfill
      \onslide*<1-5>{\mintocaml[firstline=15,lastline=21]{../src/backtrace/backtrace.ml}}
      \onslide*<6>{\mintocaml[firstline=28,lastline=34,highlightlines=31]{../src/backtrace/backtrace.ml}}
      \onslide*<7->{\mintocaml[firstline=28,lastline=34,highlightlines=33]{../src/backtrace/backtrace.ml}}
      \endminipage
    \end{column}
    \begin{column}{0.45\textwidth}
      \raggedleft
      \onslide*<1>{\providecommand\step{6}\input{diagrams/backtrace/backtrace.tex}}%
      \onslide*<2>{\providecommand\step{6}\input{diagrams/backtrace/backtrace.tex}}%
      \onslide*<3>{\providecommand\step{7}\input{diagrams/backtrace/backtrace.tex}}%
      \onslide*<4>{\providecommand\step{8}\input{diagrams/backtrace/backtrace.tex}}%
      \onslide*<5>{\providecommand\step{9}\input{diagrams/backtrace/backtrace.tex}}%
      \onslide*<6>{\providecommand\step{10}\input{diagrams/backtrace/backtrace.tex}}%
      \onslide*<7>{\providecommand\step{11}\input{diagrams/backtrace/backtrace.tex}}%
      \onslide*<8>{\providecommand\step{12}\input{diagrams/backtrace/backtrace.tex}}%
      \onslide*<9>{\providecommand\step{13}\input{diagrams/backtrace/backtrace.tex}}%
    \end{column}
  \end{columns}
\end{frame}

\begin{frame}{Backtraces}{Printing the backtrace}
  \begin{columns}[c]
    \begin{column}{0.45\textwidth}
      \minipage[c][0.66\textheight][s]{\columnwidth}
      \begin{itemize}
        \item<1-> The exception backtrace can now be printed to \funcarg{stdout} with \funcname{Printexc.print_backtrace}
        \item<2-> First the exception backtrace is retrieved into an OCaml array with \funcname{get_raw_backtrace }
        \item<3-> Which is implemented as the \funcname{caml_get_exception_raw_backtrace} C function in the runtime
        \item<4-> And finally printed with \funcname{print_exception_backtrace}
      \end{itemize}
      \vfill
      \mintocaml[firstline=28,lastline=34,highlightlines=33]{../src/backtrace/backtrace.ml}
      \endminipage
    \end{column}
    \begin{column}{0.55\textwidth}
      \onslide*<1,2>{
        \mintocaml[firstline=100,lastline=101]{../ocaml/stdlib/printexc.ml}
      }
      \only<1>{
        \listocaml[firstline=170,lastline=172]{../ocaml/stdlib/printexc.ml}{stdlib/printexc.ml:170}
      }
      \only<2>{
        \listocaml[firstline=170,lastline=172,highlightlines=172]{../ocaml/stdlib/printexc.ml}{stdlib/printexc.ml:170}
      }
      \only<3>{
        \mintc[firstline=154,lastline=156]{../ocaml/runtime/backtrace.c}
        \listc[firstline=171,lastline=192,highlightlines={184-187}]{../ocaml/runtime/backtrace.c}{runtime/backtrace.c:154}
      }
      \only<4>{
        \listocaml[firstline=155,lastline=165]{../ocaml/stdlib/printexc.ml}{stdlib/printexc.ml:55}
      }
    \end{column}
  \end{columns}
\end{frame}


\subsection{The multiple kinds of raise}
\frameSubsection{}{}

\begin{frame}{Backtraces}{When backtraces are not needed}
  \begin{columns}[c]
    \begin{column}{0.45\textwidth}
      \minipage[c][0.66\textheight][s]{\columnwidth}
      \begin{itemize}
        \item<1-> What if the backtrace for an exception is never needed?
        \item<2-> It's possible to raise the exception explicitly with \funcname{raise\_notrace} to make raising as cheap as possible
        \item<3-> An example of use in the \funcname{dynlink} library of the OCaml compiler
      \end{itemize}
      \vfill
      \onslide<3->{
      \listocaml[firstline=205,lastline=209,highlightlines=207]{../ocaml/otherlibs/dynlink/dynlink\_compilerlibs/misc.ml}{otherlibs/dynlink/dynlink\_compilerlibs/misc.ml:205}
      }
      \endminipage
    \end{column}
    \begin{column}{0.55\textwidth}
    \only<4>{
      \mintcmm[highlightlines=3]{../src/notrace/camlNotrace\_\_fun_347.cmm}
      \listcmm{../src/notrace/camlNotrace\_\_all\_somes\_327.cmm}{all\_somes}
    }
    \onslide*<5->{
      \listobjdump[highlightlines={6-9}]{../src/notrace/camlNotrace\_\_fun_347.objdump}{anonymous function from \funcname{all\_somes}}
    }
    \end{column}
  \end{columns}
\end{frame}

\begin{frame}{Backtraces}{Summary table of the multiple kinds of raise}
  \begin{itemize}
    \item When debugging information is enabled and if the backtrace of an exception isn't needed, it's possible to raise with \funcname{raise\_notrace} for performance
    \item When debugging information is \emph{not} enabled, the compiler optimises \funcname{raise} calls to inline assembly (same effect as using \funcname{raise\_notrace})
  \end{itemize}
  \bigskip
  \centering
  \begin{tabular}{| l | c | c | c | c |}
    \hline
    \parbox{10em}{Debug information \\ (\funcarg{-g} flag)}  & \multicolumn{2}{|c|}{Disabled}                                     & \multicolumn{2}{|c|}{Enabled} \\
    \hline
    Raise kind                                               & \funcname{raise} & \funcname{raise\_notrace}                       & \funcname{raise} & \funcname{raise\_notrace} \\
    \hline
    Compilation                                              & \parbox{8em}{inline assembly\\ (optimization)} & inline assembly   & \funcname{caml\_raise\_exn} & inline assembly \\
    \hline
  \end{tabular}
\end{frame}


%
% Takeaway #5
%
\subsection*{Takeaway \#5}
\frameSubsectionTakeaway{}
\againframe<5>{takeaway}
}%
      \onslide*<4>{\providecommand\step{8}%
% Backtraces
%
\subsection{One advantage of raising with debugging information}
\frameSubsection{}{}

\begin{frame}{Backtraces}{Debugging information \& source code}
  \begin{columns}[c]
    \begin{column}{0.5\textwidth}
      \begin{itemize}
        \item Raising an exception is fun and games, but how do we know where the exception originated? $\rightarrow$ With the backtrace associated with the exception!
        \item In order to get a backtrace from the point where an exception was raised, it's necessary to compile with debugging information enabled (\funcarg{-g} flag for the compiler)
        \item Same example as in the `Nested handlers' section
      \end{itemize}
    \end{column}
    \begin{column}{0.5\textwidth}
      \listocaml[firstline=9,lastline=34]{../src/backtrace/backtrace.ml}{backtrace/backtrace.ml}
    \end{column}
  \end{columns}
\end{frame}

\begin{frame}{Backtraces}{CMM and disassembly of \funcname{definitely\_raise}}
  \begin{columns}[c]
    \begin{column}{0.5\textwidth}
      \minipage[c][0.66\textheight][s]{\columnwidth}
      \begin{itemize}
        \item<1-> An effect of compiling with debugging information (\funcarg{-g}): the CMM no longer uses \funcname{raise\_notrace} to raise the exception but \funcname{raise} instead
        \item<2-> Disassembly shows that CMM \funcname{raise} is actually a call to \funcname{caml\_raise\_exn}, a function in assembler provided by the runtime
      \end{itemize}
      \vfill
      \mintocaml[firstline=9,lastline=13,highlightlines=12]{../src/backtrace/backtrace.ml}
      \endminipage
    \end{column}
    \begin{column}{0.5\textwidth}
      \onslide*<1>{
        \mintcmm[highlightlines=5]{../src/backtrace/camlBacktrace\_\_definitely\_raise\_347.cmm}
      }
      \onslide*<2>{
        \mintobjdump[firstline=2,highlightlines=14]{../src/backtrace/camlBacktrace\_\_definitely\_raise\_347.objdump}
      }
    \end{column}
  \end{columns}
\end{frame}

\begin{frame}{Backtraces}{Introducing \funcname{caml\_raise\_exn}}
  \begin{columns}[c]
    \begin{column}{0.5\textwidth}
      \minipage[c][0.66\textheight][s]{\columnwidth}
      \onslide*<1>{
        \begin{itemize}
          \item Raising the exception is performed using \funcname{caml\_raise\_exn} which is
            \begin{itemize}
              \item An assembly function provided by the runtime
              \item Architecture specific
            \end{itemize}
          \item Let's see what it does differently from \funcname{raise\_notrace}\dots
          \item We are about to raise \typename{ExnC} with \funcname{caml\_raise\_exn} from \funcname{definitely\_raise}
        \end{itemize}
      }
      \onslide*<2>{
        \mintobjdump[firstline=2,highlightlines=14]{../src/backtrace/camlBacktrace\_\_definitely\_raise\_347.objdump}
      }
      \vfill
      \mintocaml[firstline=9,lastline=13,highlightlines=12]{../src/backtrace/backtrace.ml}
      \endminipage
    \end{column}
    \begin{column}{0.5\textwidth}
      \centering
      \providecommand\step{1}\input{diagrams/backtrace/backtrace.tex}
    \end{column}
  \end{columns}
\end{frame}

\begin{frame}{Backtraces}{But before that, we need more fields from \typename{caml\_domain\_state}}
  \begin{columns}
    \begin{column}{0.5\textwidth}
      \begin{itemize}
        \item Remember \typename{caml\_domain\_state} the per-domain runtime state?
        \item Some other members are of interest to us now\dots
        \item \localname{backtrace\_active} is used to know if the runtime should collect backtraces
        \item \localname{backtrace\_active} can be set by
          \begin{itemize}
            \item The \funcarg{b} flag in the \funcname{OCAMLRUNPARAM} environment variable
            \item \mintinline{ocaml}{Printexc.record_backtrace : bool -> unit} \footnotemark
          \end{itemize}
      \end{itemize}
    \end{column}
    \begin{column}{0.5\textwidth}
      \begin{listing}
        \mintc[highlightlines={9-11}]{src/ocaml/domain\_state\_backtrace.h}
        \caption{runtime/caml/domain\_state.h}
      \end{listing}
    \end{column}
  \end{columns}
  \footnotetext[1]{https://v2.ocaml.org/api/Printexc.html}
\end{frame}

\newcommand\listCamlRaiseExn[1][]{
  \listasm[firstline=702,lastline=725,#1]{../ocaml/runtime/amd64.S}{runtime/amd64.S:702}}

\begin{frame}{Backtraces}{Walk through \funcname{caml\_raise\_exn}}
  \begin{columns}[T]
    \begin{column}{0.5\textwidth}
      \begin{itemize}
        \item<1-> First checks whether exceptions backtrace should be recorded
        \item<1-> By checking the \funcarg{domain\_state->backtrace\_active} value
        \item<2-> If not, acts as \funcname{raise\_notrace} with \funcname{RESTORE\_EXN\_HANDLER\_OCAML}
          \begin{itemize}
            \item Set \regname{RSP} to \localname{domain\_state->exn\_handler} value
            \item Restore \localname{domain\_state->exn\_handler} from trap
            \item Jump with \funcname{ret} (same effect as using \funcname{pop} and \funcname{jmp})
          \end{itemize}
        \item<3-> Otherwise, switches to the C stack to be able to execute C code
        \item<4-> Calls \funcname{caml\_stash\_backtrace}, a C function provided by the runtime\ignorespaces
          \onslide<6->{, with
            \begin{itemize}
              \item \funcarg{exn}: exception value
              \item \funcarg{pc}: code address of the raising point
              \item \funcarg{sp}: stack address of the raising function
              \item \funcarg{trapsp}: stack address of the next handler
            \end{itemize}
          }
      \end{itemize}
      \bigskip
      \mintocaml[firstline=9,lastline=13,highlightlines=12]{../src/backtrace/backtrace.ml}
    \end{column}
    \begin{column}{0.5\textwidth}
      \raggedleft
      \onslide*<1,3-4>{
        \mintc[firstline=190,lastline=192]{../ocaml/runtime/amd64.S}
        \mintc[firstline=234,lastline=237]{../ocaml/runtime/amd64.S}
      }
      \onslide*<2>{
        \mintc[firstline=190,lastline=192,highlightlines={191-192}]{../ocaml/runtime/amd64.S}
        \mintc[firstline=234,lastline=237,highlightlines={235-237}]{../ocaml/runtime/amd64.S}
      }
      \onslide*<1>{\listCamlRaiseExn[highlightlines=705]{}}%
      \onslide*<2>{\listCamlRaiseExn[highlightlines={707-708}]{}}%
      \onslide*<3>{\listCamlRaiseExn[highlightlines={712-714}]{}}%
      \onslide*<4>{\listCamlRaiseExn[highlightlines={715-720}]{}}%
      \onslide*<5>{\providecommand\step{1}\input{diagrams/backtrace/backtrace.tex}}%
      \onslide*<6>{\providecommand\step{2}\input{diagrams/backtrace/backtrace.tex}}%
    \end{column}
  \end{columns}
\end{frame}

\newcommand\listStashBacktrace[1][]{
  \listc[firstline=91,lastline=120,#1]{../ocaml/runtime/backtrace\_nat.c}{runtime/backtrace\_nat.c:91}}

\begin{frame}{Backtraces}{Introducing \funcname{caml\_stash\_backtrace}}
  \begin{columns}[c]
    \begin{column}{0.5\textwidth}
      \minipage[c][0.66\textheight][s]{\columnwidth}
      \begin{itemize}
        \item<1-> A C function provided by the runtime (specific to native code)
        \item<2-> It scans the stack, between the stack pointer at the raising point up to the next trap, in order to collect the names of the detected function frames
          \onslide*<2>{
            \begin{itemize}
              \item And more information, like line number, column number...
            \end{itemize}
          }
        \item<3-> It uses the same technique and information as the Garbage Collector (GC) uses to scan the values live on the stack
          \onslide*<4>{
            \begin{itemize}
              \item \typename{frame\_descr} are at the core of stack unwinding
            \end{itemize}
          }
      \end{itemize}
      \vfill
      \mintocaml[firstline=9,lastline=13,highlightlines=12]{../src/backtrace/backtrace.ml}
      \endminipage
    \end{column}
    \begin{column}{0.5\textwidth}
      \onslide*<1>{\listStashBacktrace[highlightlines=91]{}}
      \onslide*<2>{\listStashBacktrace[highlightlines={91,117-118}]{}}
      \onslide*<3->{\listStashBacktrace[highlightlines={94,106,109-110}]{}}
    \end{column}
  \end{columns}
\end{frame}

\newcommand\listFrameDescr[1][]{
  \listocaml[firstline=111,lastline=124,#1]{../ocaml/asmcomp/emitaux.ml}{asmcomp/emitaux.ml:111}}
\newcommand\listDebugInfo[1][]{
  \listocaml[firstline=38,lastline=49,#1]{../ocaml/lambda/debuginfo.mli}{lambda/debuginfo.mli:38}}

\begin{frame}{Backtraces}{\typename{frame\_descr} and \typename{frame\_debuginfo} in a nutshell}
  \begin{columns}[c]
    \begin{column}{0.5\textwidth}
      \begin{itemize}
        \item<1-> For every return address in an OCaml program, there is a associated \typename{frame\_descr}
        \item<2-> A \typename{frame\_descr} specifies
        \begin{itemize}
          \item<2-> The size of the function stack frame
          \item<3-> Where live values are on the stack (so that the GC can mark them as live and prevent from being swept)
        \end{itemize}
        \item<4-> When compiled with debug information, \typename{frame\_descr} will be linked to a \typename{frame\_debuginfo}
        \begin{itemize}
          \item<5-> Contains information about the position in the source code (file name, line and column number)
        \end{itemize}
      \end{itemize}
    \end{column}
    \begin{column}{0.5\textwidth}
        \onslide*<1>{\listFrameDescr[highlightlines={118-122}]{}}
        \onslide*<2>{\listFrameDescr[highlightlines={120}]{}}
        \onslide*<3>{\listFrameDescr[highlightlines={121}]{}}
        \onslide*<4->{\listFrameDescr[highlightlines={122}]{}}
        \medskip
        \onslide*<1-3>{\listDebugInfo{}}
        \onslide*<4>{\listDebugInfo[highlightlines={38,49}]{}}
        \onslide*<5>{\listDebugInfo[highlightlines={39-46}]{}}
    \end{column}
  \end{columns}
\end{frame}

\begin{frame}{Backtraces}{Scanning the stack with \typename{frame\_descr}}
  \begin{columns}[c]
    \begin{column}{0.5\textwidth}
      \begin{itemize}
        \item Given the return address of the code that raises the exception by calling \funcname{caml\_raise\_exn} (\funcarg{pc} argument of \funcname{caml\_stash\_backtrace})
        \item And the position of the stack pointer (\funcarg{sp} argument of \funcname{caml\_stash\_backtrace})
        \item The frame size given by \typename{frame\_descr}.\funcarg{frame\_size}, allows to find the end of the previous frame
        \item And the return address of the caller of this function
        \bigskip
        \onslide<3->{
        \item Note that traps (here the reddish \funcname{ExnA}, \funcname{ExnB} and \funcname{ExnC} blocks) belong to the frame that installed them, and so the frame size changes when traps are removed
        }
      \end{itemize}
    \end{column}
    \begin{column}{0.5\textwidth}
      \raggedleft
      \onslide*<1>{\providecommand\step{2}\input{diagrams/backtrace/backtrace.tex}}%
      \onslide*<2>{\providecommand\step{1}\input{diagrams/backtrace/scanstack.tex}}%
      \onslide*<3>{\providecommand\step{2}\input{diagrams/backtrace/scanstack.tex}}%
      \onslide*<4>{\providecommand\step{3}\input{diagrams/backtrace/scanstack.tex}}%
      \onslide*<5>{\providecommand\step{4}\input{diagrams/backtrace/scanstack.tex}}%
      \onslide*<6>{\providecommand\step{5}\input{diagrams/backtrace/scanstack.tex}}%
    \end{column}
  \end{columns}
\end{frame}

% Duplicate of previous-previous slide
\begin{frame}{Backtraces}{Continuing on \funcname{caml\_stash\_backtrace}}
  \begin{columns}[c]
    \begin{column}{0.5\textwidth}
      \minipage[c][0.75\textheight][s]{\columnwidth}
      \begin{itemize}
          \color{gray}
        \item A C function provided by the runtime (specific to native code)
        \item It scans the stack, between the stack pointer at the raising point up to the next trap, in order to collect the names of the detected function frames
        \item It uses the same technique and information as the Garbage Collector (GC) uses to scan the values live on the stack
      \end{itemize}
      \begin{itemize}
        \item For each function frame detected, store a pointer to the \typename{frame\_descr} in the \localname{domain\_state->backtrace\_buffer} array, effectively constructing the backtrace
      \end{itemize}
      \onslide<2->{
        \begin{itemize}
            \smallskip
          \item Constructed backtrace:
            \funcname{definitely\_raise},
            \onslide<3->{\funcname{probably\_raise}}
        \end{itemize}
      }
      \vfill
      \mintocaml[firstline=9,lastline=13,highlightlines=12]{../src/backtrace/backtrace.ml}
      \endminipage
    \end{column}
    \begin{column}{0.5\textwidth}
      \raggedleft
      \onslide*<1>{\listStashBacktrace[highlightlines={114-115}]{}}
      \onslide*<2>{\providecommand\step{3}\input{diagrams/backtrace/backtrace.tex}}%
      \onslide*<3>{\providecommand\step{4}\input{diagrams/backtrace/backtrace.tex}}%
    \end{column}
  \end{columns}
\end{frame}

\begin{frame}{Backtraces}{Back to \funcname{caml\_raise\_exn}}
  \begin{columns}[c]
    \begin{column}{0.5\textwidth}
      \minipage[c][0.66\textheight][s]{\columnwidth}
      \begin{itemize}\color{gray}
        \item Checks wheteher exceptions backtrace should be recorded, by checking the \funcarg{domain\_state->backtrace\_active} value
        \item Switches to the C stack to be able to execute C code
        \item Calls \funcname{caml\_stash\_backtrace}, to collect the backtrace up to the next trap
      \end{itemize}
      \onslide*<2->{
        \begin{itemize}
          \item<2-> Executes the exception handler in \funcname{probably\_raise} with \funcname{RESTORE\_EXN\_HANDLER\_OCAML}
            \begin{itemize}
              \item Set \regname{RSP} to \localname{domain\_state->exn\_handler} value
              \item Restore \localname{domain\_state->exn\_handler} from trap
              \item Jump to the handler code with \funcname{ret}
            \end{itemize}
        \end{itemize}
      }
      \vfill
      \onslide*<1>{\mintocaml[firstline=9,lastline=13,highlightlines=12]{../src/backtrace/backtrace.ml}}
      \onslide*<2->{\mintocaml[firstline=15,lastline=21,highlightlines={19-21}]{../src/backtrace/backtrace.ml}}
      \endminipage
    \end{column}
    \begin{column}{0.5\textwidth}
      \centering
      \onslide*<1>{
        \mintc[firstline=190,lastline=192]{../ocaml/runtime/amd64.S}
        \mintc[firstline=234,lastline=237]{../ocaml/runtime/amd64.S}
      }
      \onslide*<2>{
        \mintc[firstline=190,lastline=192,highlightlines={191-192}]{../ocaml/runtime/amd64.S}
        \mintc[firstline=234,lastline=237,highlightlines={235-237}]{../ocaml/runtime/amd64.S}
      }
      \onslide*<1>{\listCamlRaiseExn[highlightlines=720]{}}
      \onslide*<2>{\listCamlRaiseExn[highlightlines=722]{}}
      \onslide*<3->{\providecommand\step{5}\input{diagrams/backtrace/backtrace.tex}}
    \end{column}
  \end{columns}
\end{frame}

\begin{frame}{Backtraces}{\funcname{caml\_probably\_raise}'s handler doesn't catch \typename{ExnC}}
  \begin{columns}[c]
    \begin{column}{0.45\textwidth}
      \minipage[c][0.75\textheight][s]{\columnwidth}
      \begin{itemize}
        \item<1-> \funcname{match \dots with}\footnotemark pattern matching can also match exceptions
        \item<1-> The CMM of \funcname{probably\_raise} shows that this \funcname{match \dots with} is implemented with a pattern matching and a \funcname{try \dots with}
        \item<1-> But this one only matches on \typename{ExnA} exception
        \item<2-> Since the pattern matching fails to catch the exception, \funcname{reraise} is called with our current \typename{ExnC} exception
        \item<3-> Disassembly shows that \funcname{reraise} is actually a call to \funcname{caml\_reraise\_exn}, which is also an assembly function provided by the runtime
      \end{itemize}
      \vfill
      \mintocaml[firstline=15,lastline=21,highlightlines={18-21}]{../src/backtrace/backtrace.ml}
      \endminipage
    \end{column}
    \begin{column}{0.55\textwidth}
      \centering
      \onslide*<1>{\mintcmm[highlightlines={10-18}]{../src/backtrace/camlBacktrace\_\_probably\_raise\_351.cmm}}
      \onslide*<2>{\listcmm[highlightlines=18]{../src/backtrace/camlBacktrace\_\_probably\_raise_351.cmm}{CMM of probably\_raise}}
      \onslide*<3>{\mintobjdump[firstline=5,lastline=35,highlightlines=31]{../src/backtrace/camlBacktrace\_\_probably\_raise_351.objdump}}
    \end{column}
  \end{columns}
  \only<1>{\footnotetext[1]{https://blog.janestreet.com/pattern-matching-and-exception-handling-unite/}}
\end{frame}

\newcommand\listCamlReraiseExn[1][]{
  \listasm[firstline=727,lastline=734,#1]{../ocaml/runtime/amd64.S}{runtime/amd64.S:727}}

\begin{frame}{Backtraces}{Introducing \funcname{caml\_reraise\_exn}}
  \begin{columns}[c]
    \begin{column}{0.5\textwidth}
      \minipage[c][0.66\textheight][s]{\columnwidth}
      \begin{itemize}
        \item<1-> The exception have to be reraised to the next handler
        \item<2-> First, checks if the backtrace should be collected with \funcarg{domain\_state->backtrace\_active}
        \only<3>{
        \begin{itemize}
            \item If not, behaves like a \funcname{raise\_notrace} with \funcname{RESTORE\_EXN\_HANDLER\_OCAML}
        \end{itemize}
        }
        \item<4-> Jumps into \funcname{caml\_raise\_exn}
        \item<5-> Calls \funcname{caml\_stash\_backtrace} again, to scan the next part of the stack: from the trap just pop-ed to the next trap
      \end{itemize}
      \vfill
      \mintocaml[firstline=15,lastline=21,highlightlines={18-21}]{../src/backtrace/backtrace.ml}
      \endminipage
    \end{column}
    \begin{column}{0.5\textwidth}
      \raggedleft
      \onslide*<1>{\listCamlReraiseExn{}}
      \onslide*<2>{\listCamlReraiseExn[highlightlines=729]{}}
      \onslide*<3>{
        \mintc[firstline=190,lastline=192,highlightlines={191-192}]{../ocaml/runtime/amd64.S}
        \mintc[firstline=234,lastline=237,highlightlines={235-237}]{../ocaml/runtime/amd64.S}
        \medskip
        \listCamlReraiseExn[highlightlines=731]{}
      }
      \onslide*<4-5>{
        % TODO refactor with \listCamlReraiseExn
        \mintasm[firstline=727,lastline=734,highlightlines=730]{../ocaml/runtime/amd64.S}
        \medskip
      }
      \onslide*<4>{\listCamlRaiseExn[highlightlines=711]{}}
      \onslide*<5>{\listCamlRaiseExn[highlightlines=720]{}}
    \end{column}
  \end{columns}
\end{frame}

\begin{frame}{Backtraces}{Backtrace collection continues}
  \begin{columns}[c]
    \begin{column}{0.55\textwidth}
      \minipage[c][0.75\textheight][s]{\columnwidth}
      \begin{itemize}
        \item<1-> \funcname{caml\_stash\_backtrace} scans the older part of the stack, up to the next trap
        \item<6-> Controls jumps to the nested exception handler in \funcname{maybe\_raise}, which matches only on \typename{ExnB}
        \item<7-> \funcname{caml\_reraise\_exn} is called again, backtrace collection resumes with \funcname{caml\_stash\_backtrace}
      \end{itemize}
      \medskip
      \begin{itemize}
        \item<2-> Constructed backtrace:
        \begin{itemize}
          \item <2-> \funcname{definitely\_raise}
          \item <2-> \funcname{probably\_raise}
          \item <3-> \funcname{probably\_raise} (re-raise)
          \item <4-> \funcname{maybe\_raise}
          \item <5-> \funcname{main}
          \item <8-> \funcname{main} (re-raise)
        \end{itemize}
      \end{itemize}
      \vfill
      \onslide*<1-5>{\mintocaml[firstline=15,lastline=21]{../src/backtrace/backtrace.ml}}
      \onslide*<6>{\mintocaml[firstline=28,lastline=34,highlightlines=31]{../src/backtrace/backtrace.ml}}
      \onslide*<7->{\mintocaml[firstline=28,lastline=34,highlightlines=33]{../src/backtrace/backtrace.ml}}
      \endminipage
    \end{column}
    \begin{column}{0.45\textwidth}
      \raggedleft
      \onslide*<1>{\providecommand\step{6}\input{diagrams/backtrace/backtrace.tex}}%
      \onslide*<2>{\providecommand\step{6}\input{diagrams/backtrace/backtrace.tex}}%
      \onslide*<3>{\providecommand\step{7}\input{diagrams/backtrace/backtrace.tex}}%
      \onslide*<4>{\providecommand\step{8}\input{diagrams/backtrace/backtrace.tex}}%
      \onslide*<5>{\providecommand\step{9}\input{diagrams/backtrace/backtrace.tex}}%
      \onslide*<6>{\providecommand\step{10}\input{diagrams/backtrace/backtrace.tex}}%
      \onslide*<7>{\providecommand\step{11}\input{diagrams/backtrace/backtrace.tex}}%
      \onslide*<8>{\providecommand\step{12}\input{diagrams/backtrace/backtrace.tex}}%
      \onslide*<9>{\providecommand\step{13}\input{diagrams/backtrace/backtrace.tex}}%
    \end{column}
  \end{columns}
\end{frame}

\begin{frame}{Backtraces}{Printing the backtrace}
  \begin{columns}[c]
    \begin{column}{0.45\textwidth}
      \minipage[c][0.66\textheight][s]{\columnwidth}
      \begin{itemize}
        \item<1-> The exception backtrace can now be printed to \funcarg{stdout} with \funcname{Printexc.print_backtrace}
        \item<2-> First the exception backtrace is retrieved into an OCaml array with \funcname{get_raw_backtrace }
        \item<3-> Which is implemented as the \funcname{caml_get_exception_raw_backtrace} C function in the runtime
        \item<4-> And finally printed with \funcname{print_exception_backtrace}
      \end{itemize}
      \vfill
      \mintocaml[firstline=28,lastline=34,highlightlines=33]{../src/backtrace/backtrace.ml}
      \endminipage
    \end{column}
    \begin{column}{0.55\textwidth}
      \onslide*<1,2>{
        \mintocaml[firstline=100,lastline=101]{../ocaml/stdlib/printexc.ml}
      }
      \only<1>{
        \listocaml[firstline=170,lastline=172]{../ocaml/stdlib/printexc.ml}{stdlib/printexc.ml:170}
      }
      \only<2>{
        \listocaml[firstline=170,lastline=172,highlightlines=172]{../ocaml/stdlib/printexc.ml}{stdlib/printexc.ml:170}
      }
      \only<3>{
        \mintc[firstline=154,lastline=156]{../ocaml/runtime/backtrace.c}
        \listc[firstline=171,lastline=192,highlightlines={184-187}]{../ocaml/runtime/backtrace.c}{runtime/backtrace.c:154}
      }
      \only<4>{
        \listocaml[firstline=155,lastline=165]{../ocaml/stdlib/printexc.ml}{stdlib/printexc.ml:55}
      }
    \end{column}
  \end{columns}
\end{frame}


\subsection{The multiple kinds of raise}
\frameSubsection{}{}

\begin{frame}{Backtraces}{When backtraces are not needed}
  \begin{columns}[c]
    \begin{column}{0.45\textwidth}
      \minipage[c][0.66\textheight][s]{\columnwidth}
      \begin{itemize}
        \item<1-> What if the backtrace for an exception is never needed?
        \item<2-> It's possible to raise the exception explicitly with \funcname{raise\_notrace} to make raising as cheap as possible
        \item<3-> An example of use in the \funcname{dynlink} library of the OCaml compiler
      \end{itemize}
      \vfill
      \onslide<3->{
      \listocaml[firstline=205,lastline=209,highlightlines=207]{../ocaml/otherlibs/dynlink/dynlink\_compilerlibs/misc.ml}{otherlibs/dynlink/dynlink\_compilerlibs/misc.ml:205}
      }
      \endminipage
    \end{column}
    \begin{column}{0.55\textwidth}
    \only<4>{
      \mintcmm[highlightlines=3]{../src/notrace/camlNotrace\_\_fun_347.cmm}
      \listcmm{../src/notrace/camlNotrace\_\_all\_somes\_327.cmm}{all\_somes}
    }
    \onslide*<5->{
      \listobjdump[highlightlines={6-9}]{../src/notrace/camlNotrace\_\_fun_347.objdump}{anonymous function from \funcname{all\_somes}}
    }
    \end{column}
  \end{columns}
\end{frame}

\begin{frame}{Backtraces}{Summary table of the multiple kinds of raise}
  \begin{itemize}
    \item When debugging information is enabled and if the backtrace of an exception isn't needed, it's possible to raise with \funcname{raise\_notrace} for performance
    \item When debugging information is \emph{not} enabled, the compiler optimises \funcname{raise} calls to inline assembly (same effect as using \funcname{raise\_notrace})
  \end{itemize}
  \bigskip
  \centering
  \begin{tabular}{| l | c | c | c | c |}
    \hline
    \parbox{10em}{Debug information \\ (\funcarg{-g} flag)}  & \multicolumn{2}{|c|}{Disabled}                                     & \multicolumn{2}{|c|}{Enabled} \\
    \hline
    Raise kind                                               & \funcname{raise} & \funcname{raise\_notrace}                       & \funcname{raise} & \funcname{raise\_notrace} \\
    \hline
    Compilation                                              & \parbox{8em}{inline assembly\\ (optimization)} & inline assembly   & \funcname{caml\_raise\_exn} & inline assembly \\
    \hline
  \end{tabular}
\end{frame}


%
% Takeaway #5
%
\subsection*{Takeaway \#5}
\frameSubsectionTakeaway{}
\againframe<5>{takeaway}
}%
      \onslide*<5>{\providecommand\step{9}%
% Backtraces
%
\subsection{One advantage of raising with debugging information}
\frameSubsection{}{}

\begin{frame}{Backtraces}{Debugging information \& source code}
  \begin{columns}[c]
    \begin{column}{0.5\textwidth}
      \begin{itemize}
        \item Raising an exception is fun and games, but how do we know where the exception originated? $\rightarrow$ With the backtrace associated with the exception!
        \item In order to get a backtrace from the point where an exception was raised, it's necessary to compile with debugging information enabled (\funcarg{-g} flag for the compiler)
        \item Same example as in the `Nested handlers' section
      \end{itemize}
    \end{column}
    \begin{column}{0.5\textwidth}
      \listocaml[firstline=9,lastline=34]{../src/backtrace/backtrace.ml}{backtrace/backtrace.ml}
    \end{column}
  \end{columns}
\end{frame}

\begin{frame}{Backtraces}{CMM and disassembly of \funcname{definitely\_raise}}
  \begin{columns}[c]
    \begin{column}{0.5\textwidth}
      \minipage[c][0.66\textheight][s]{\columnwidth}
      \begin{itemize}
        \item<1-> An effect of compiling with debugging information (\funcarg{-g}): the CMM no longer uses \funcname{raise\_notrace} to raise the exception but \funcname{raise} instead
        \item<2-> Disassembly shows that CMM \funcname{raise} is actually a call to \funcname{caml\_raise\_exn}, a function in assembler provided by the runtime
      \end{itemize}
      \vfill
      \mintocaml[firstline=9,lastline=13,highlightlines=12]{../src/backtrace/backtrace.ml}
      \endminipage
    \end{column}
    \begin{column}{0.5\textwidth}
      \onslide*<1>{
        \mintcmm[highlightlines=5]{../src/backtrace/camlBacktrace\_\_definitely\_raise\_347.cmm}
      }
      \onslide*<2>{
        \mintobjdump[firstline=2,highlightlines=14]{../src/backtrace/camlBacktrace\_\_definitely\_raise\_347.objdump}
      }
    \end{column}
  \end{columns}
\end{frame}

\begin{frame}{Backtraces}{Introducing \funcname{caml\_raise\_exn}}
  \begin{columns}[c]
    \begin{column}{0.5\textwidth}
      \minipage[c][0.66\textheight][s]{\columnwidth}
      \onslide*<1>{
        \begin{itemize}
          \item Raising the exception is performed using \funcname{caml\_raise\_exn} which is
            \begin{itemize}
              \item An assembly function provided by the runtime
              \item Architecture specific
            \end{itemize}
          \item Let's see what it does differently from \funcname{raise\_notrace}\dots
          \item We are about to raise \typename{ExnC} with \funcname{caml\_raise\_exn} from \funcname{definitely\_raise}
        \end{itemize}
      }
      \onslide*<2>{
        \mintobjdump[firstline=2,highlightlines=14]{../src/backtrace/camlBacktrace\_\_definitely\_raise\_347.objdump}
      }
      \vfill
      \mintocaml[firstline=9,lastline=13,highlightlines=12]{../src/backtrace/backtrace.ml}
      \endminipage
    \end{column}
    \begin{column}{0.5\textwidth}
      \centering
      \providecommand\step{1}\input{diagrams/backtrace/backtrace.tex}
    \end{column}
  \end{columns}
\end{frame}

\begin{frame}{Backtraces}{But before that, we need more fields from \typename{caml\_domain\_state}}
  \begin{columns}
    \begin{column}{0.5\textwidth}
      \begin{itemize}
        \item Remember \typename{caml\_domain\_state} the per-domain runtime state?
        \item Some other members are of interest to us now\dots
        \item \localname{backtrace\_active} is used to know if the runtime should collect backtraces
        \item \localname{backtrace\_active} can be set by
          \begin{itemize}
            \item The \funcarg{b} flag in the \funcname{OCAMLRUNPARAM} environment variable
            \item \mintinline{ocaml}{Printexc.record_backtrace : bool -> unit} \footnotemark
          \end{itemize}
      \end{itemize}
    \end{column}
    \begin{column}{0.5\textwidth}
      \begin{listing}
        \mintc[highlightlines={9-11}]{src/ocaml/domain\_state\_backtrace.h}
        \caption{runtime/caml/domain\_state.h}
      \end{listing}
    \end{column}
  \end{columns}
  \footnotetext[1]{https://v2.ocaml.org/api/Printexc.html}
\end{frame}

\newcommand\listCamlRaiseExn[1][]{
  \listasm[firstline=702,lastline=725,#1]{../ocaml/runtime/amd64.S}{runtime/amd64.S:702}}

\begin{frame}{Backtraces}{Walk through \funcname{caml\_raise\_exn}}
  \begin{columns}[T]
    \begin{column}{0.5\textwidth}
      \begin{itemize}
        \item<1-> First checks whether exceptions backtrace should be recorded
        \item<1-> By checking the \funcarg{domain\_state->backtrace\_active} value
        \item<2-> If not, acts as \funcname{raise\_notrace} with \funcname{RESTORE\_EXN\_HANDLER\_OCAML}
          \begin{itemize}
            \item Set \regname{RSP} to \localname{domain\_state->exn\_handler} value
            \item Restore \localname{domain\_state->exn\_handler} from trap
            \item Jump with \funcname{ret} (same effect as using \funcname{pop} and \funcname{jmp})
          \end{itemize}
        \item<3-> Otherwise, switches to the C stack to be able to execute C code
        \item<4-> Calls \funcname{caml\_stash\_backtrace}, a C function provided by the runtime\ignorespaces
          \onslide<6->{, with
            \begin{itemize}
              \item \funcarg{exn}: exception value
              \item \funcarg{pc}: code address of the raising point
              \item \funcarg{sp}: stack address of the raising function
              \item \funcarg{trapsp}: stack address of the next handler
            \end{itemize}
          }
      \end{itemize}
      \bigskip
      \mintocaml[firstline=9,lastline=13,highlightlines=12]{../src/backtrace/backtrace.ml}
    \end{column}
    \begin{column}{0.5\textwidth}
      \raggedleft
      \onslide*<1,3-4>{
        \mintc[firstline=190,lastline=192]{../ocaml/runtime/amd64.S}
        \mintc[firstline=234,lastline=237]{../ocaml/runtime/amd64.S}
      }
      \onslide*<2>{
        \mintc[firstline=190,lastline=192,highlightlines={191-192}]{../ocaml/runtime/amd64.S}
        \mintc[firstline=234,lastline=237,highlightlines={235-237}]{../ocaml/runtime/amd64.S}
      }
      \onslide*<1>{\listCamlRaiseExn[highlightlines=705]{}}%
      \onslide*<2>{\listCamlRaiseExn[highlightlines={707-708}]{}}%
      \onslide*<3>{\listCamlRaiseExn[highlightlines={712-714}]{}}%
      \onslide*<4>{\listCamlRaiseExn[highlightlines={715-720}]{}}%
      \onslide*<5>{\providecommand\step{1}\input{diagrams/backtrace/backtrace.tex}}%
      \onslide*<6>{\providecommand\step{2}\input{diagrams/backtrace/backtrace.tex}}%
    \end{column}
  \end{columns}
\end{frame}

\newcommand\listStashBacktrace[1][]{
  \listc[firstline=91,lastline=120,#1]{../ocaml/runtime/backtrace\_nat.c}{runtime/backtrace\_nat.c:91}}

\begin{frame}{Backtraces}{Introducing \funcname{caml\_stash\_backtrace}}
  \begin{columns}[c]
    \begin{column}{0.5\textwidth}
      \minipage[c][0.66\textheight][s]{\columnwidth}
      \begin{itemize}
        \item<1-> A C function provided by the runtime (specific to native code)
        \item<2-> It scans the stack, between the stack pointer at the raising point up to the next trap, in order to collect the names of the detected function frames
          \onslide*<2>{
            \begin{itemize}
              \item And more information, like line number, column number...
            \end{itemize}
          }
        \item<3-> It uses the same technique and information as the Garbage Collector (GC) uses to scan the values live on the stack
          \onslide*<4>{
            \begin{itemize}
              \item \typename{frame\_descr} are at the core of stack unwinding
            \end{itemize}
          }
      \end{itemize}
      \vfill
      \mintocaml[firstline=9,lastline=13,highlightlines=12]{../src/backtrace/backtrace.ml}
      \endminipage
    \end{column}
    \begin{column}{0.5\textwidth}
      \onslide*<1>{\listStashBacktrace[highlightlines=91]{}}
      \onslide*<2>{\listStashBacktrace[highlightlines={91,117-118}]{}}
      \onslide*<3->{\listStashBacktrace[highlightlines={94,106,109-110}]{}}
    \end{column}
  \end{columns}
\end{frame}

\newcommand\listFrameDescr[1][]{
  \listocaml[firstline=111,lastline=124,#1]{../ocaml/asmcomp/emitaux.ml}{asmcomp/emitaux.ml:111}}
\newcommand\listDebugInfo[1][]{
  \listocaml[firstline=38,lastline=49,#1]{../ocaml/lambda/debuginfo.mli}{lambda/debuginfo.mli:38}}

\begin{frame}{Backtraces}{\typename{frame\_descr} and \typename{frame\_debuginfo} in a nutshell}
  \begin{columns}[c]
    \begin{column}{0.5\textwidth}
      \begin{itemize}
        \item<1-> For every return address in an OCaml program, there is a associated \typename{frame\_descr}
        \item<2-> A \typename{frame\_descr} specifies
        \begin{itemize}
          \item<2-> The size of the function stack frame
          \item<3-> Where live values are on the stack (so that the GC can mark them as live and prevent from being swept)
        \end{itemize}
        \item<4-> When compiled with debug information, \typename{frame\_descr} will be linked to a \typename{frame\_debuginfo}
        \begin{itemize}
          \item<5-> Contains information about the position in the source code (file name, line and column number)
        \end{itemize}
      \end{itemize}
    \end{column}
    \begin{column}{0.5\textwidth}
        \onslide*<1>{\listFrameDescr[highlightlines={118-122}]{}}
        \onslide*<2>{\listFrameDescr[highlightlines={120}]{}}
        \onslide*<3>{\listFrameDescr[highlightlines={121}]{}}
        \onslide*<4->{\listFrameDescr[highlightlines={122}]{}}
        \medskip
        \onslide*<1-3>{\listDebugInfo{}}
        \onslide*<4>{\listDebugInfo[highlightlines={38,49}]{}}
        \onslide*<5>{\listDebugInfo[highlightlines={39-46}]{}}
    \end{column}
  \end{columns}
\end{frame}

\begin{frame}{Backtraces}{Scanning the stack with \typename{frame\_descr}}
  \begin{columns}[c]
    \begin{column}{0.5\textwidth}
      \begin{itemize}
        \item Given the return address of the code that raises the exception by calling \funcname{caml\_raise\_exn} (\funcarg{pc} argument of \funcname{caml\_stash\_backtrace})
        \item And the position of the stack pointer (\funcarg{sp} argument of \funcname{caml\_stash\_backtrace})
        \item The frame size given by \typename{frame\_descr}.\funcarg{frame\_size}, allows to find the end of the previous frame
        \item And the return address of the caller of this function
        \bigskip
        \onslide<3->{
        \item Note that traps (here the reddish \funcname{ExnA}, \funcname{ExnB} and \funcname{ExnC} blocks) belong to the frame that installed them, and so the frame size changes when traps are removed
        }
      \end{itemize}
    \end{column}
    \begin{column}{0.5\textwidth}
      \raggedleft
      \onslide*<1>{\providecommand\step{2}\input{diagrams/backtrace/backtrace.tex}}%
      \onslide*<2>{\providecommand\step{1}\input{diagrams/backtrace/scanstack.tex}}%
      \onslide*<3>{\providecommand\step{2}\input{diagrams/backtrace/scanstack.tex}}%
      \onslide*<4>{\providecommand\step{3}\input{diagrams/backtrace/scanstack.tex}}%
      \onslide*<5>{\providecommand\step{4}\input{diagrams/backtrace/scanstack.tex}}%
      \onslide*<6>{\providecommand\step{5}\input{diagrams/backtrace/scanstack.tex}}%
    \end{column}
  \end{columns}
\end{frame}

% Duplicate of previous-previous slide
\begin{frame}{Backtraces}{Continuing on \funcname{caml\_stash\_backtrace}}
  \begin{columns}[c]
    \begin{column}{0.5\textwidth}
      \minipage[c][0.75\textheight][s]{\columnwidth}
      \begin{itemize}
          \color{gray}
        \item A C function provided by the runtime (specific to native code)
        \item It scans the stack, between the stack pointer at the raising point up to the next trap, in order to collect the names of the detected function frames
        \item It uses the same technique and information as the Garbage Collector (GC) uses to scan the values live on the stack
      \end{itemize}
      \begin{itemize}
        \item For each function frame detected, store a pointer to the \typename{frame\_descr} in the \localname{domain\_state->backtrace\_buffer} array, effectively constructing the backtrace
      \end{itemize}
      \onslide<2->{
        \begin{itemize}
            \smallskip
          \item Constructed backtrace:
            \funcname{definitely\_raise},
            \onslide<3->{\funcname{probably\_raise}}
        \end{itemize}
      }
      \vfill
      \mintocaml[firstline=9,lastline=13,highlightlines=12]{../src/backtrace/backtrace.ml}
      \endminipage
    \end{column}
    \begin{column}{0.5\textwidth}
      \raggedleft
      \onslide*<1>{\listStashBacktrace[highlightlines={114-115}]{}}
      \onslide*<2>{\providecommand\step{3}\input{diagrams/backtrace/backtrace.tex}}%
      \onslide*<3>{\providecommand\step{4}\input{diagrams/backtrace/backtrace.tex}}%
    \end{column}
  \end{columns}
\end{frame}

\begin{frame}{Backtraces}{Back to \funcname{caml\_raise\_exn}}
  \begin{columns}[c]
    \begin{column}{0.5\textwidth}
      \minipage[c][0.66\textheight][s]{\columnwidth}
      \begin{itemize}\color{gray}
        \item Checks wheteher exceptions backtrace should be recorded, by checking the \funcarg{domain\_state->backtrace\_active} value
        \item Switches to the C stack to be able to execute C code
        \item Calls \funcname{caml\_stash\_backtrace}, to collect the backtrace up to the next trap
      \end{itemize}
      \onslide*<2->{
        \begin{itemize}
          \item<2-> Executes the exception handler in \funcname{probably\_raise} with \funcname{RESTORE\_EXN\_HANDLER\_OCAML}
            \begin{itemize}
              \item Set \regname{RSP} to \localname{domain\_state->exn\_handler} value
              \item Restore \localname{domain\_state->exn\_handler} from trap
              \item Jump to the handler code with \funcname{ret}
            \end{itemize}
        \end{itemize}
      }
      \vfill
      \onslide*<1>{\mintocaml[firstline=9,lastline=13,highlightlines=12]{../src/backtrace/backtrace.ml}}
      \onslide*<2->{\mintocaml[firstline=15,lastline=21,highlightlines={19-21}]{../src/backtrace/backtrace.ml}}
      \endminipage
    \end{column}
    \begin{column}{0.5\textwidth}
      \centering
      \onslide*<1>{
        \mintc[firstline=190,lastline=192]{../ocaml/runtime/amd64.S}
        \mintc[firstline=234,lastline=237]{../ocaml/runtime/amd64.S}
      }
      \onslide*<2>{
        \mintc[firstline=190,lastline=192,highlightlines={191-192}]{../ocaml/runtime/amd64.S}
        \mintc[firstline=234,lastline=237,highlightlines={235-237}]{../ocaml/runtime/amd64.S}
      }
      \onslide*<1>{\listCamlRaiseExn[highlightlines=720]{}}
      \onslide*<2>{\listCamlRaiseExn[highlightlines=722]{}}
      \onslide*<3->{\providecommand\step{5}\input{diagrams/backtrace/backtrace.tex}}
    \end{column}
  \end{columns}
\end{frame}

\begin{frame}{Backtraces}{\funcname{caml\_probably\_raise}'s handler doesn't catch \typename{ExnC}}
  \begin{columns}[c]
    \begin{column}{0.45\textwidth}
      \minipage[c][0.75\textheight][s]{\columnwidth}
      \begin{itemize}
        \item<1-> \funcname{match \dots with}\footnotemark pattern matching can also match exceptions
        \item<1-> The CMM of \funcname{probably\_raise} shows that this \funcname{match \dots with} is implemented with a pattern matching and a \funcname{try \dots with}
        \item<1-> But this one only matches on \typename{ExnA} exception
        \item<2-> Since the pattern matching fails to catch the exception, \funcname{reraise} is called with our current \typename{ExnC} exception
        \item<3-> Disassembly shows that \funcname{reraise} is actually a call to \funcname{caml\_reraise\_exn}, which is also an assembly function provided by the runtime
      \end{itemize}
      \vfill
      \mintocaml[firstline=15,lastline=21,highlightlines={18-21}]{../src/backtrace/backtrace.ml}
      \endminipage
    \end{column}
    \begin{column}{0.55\textwidth}
      \centering
      \onslide*<1>{\mintcmm[highlightlines={10-18}]{../src/backtrace/camlBacktrace\_\_probably\_raise\_351.cmm}}
      \onslide*<2>{\listcmm[highlightlines=18]{../src/backtrace/camlBacktrace\_\_probably\_raise_351.cmm}{CMM of probably\_raise}}
      \onslide*<3>{\mintobjdump[firstline=5,lastline=35,highlightlines=31]{../src/backtrace/camlBacktrace\_\_probably\_raise_351.objdump}}
    \end{column}
  \end{columns}
  \only<1>{\footnotetext[1]{https://blog.janestreet.com/pattern-matching-and-exception-handling-unite/}}
\end{frame}

\newcommand\listCamlReraiseExn[1][]{
  \listasm[firstline=727,lastline=734,#1]{../ocaml/runtime/amd64.S}{runtime/amd64.S:727}}

\begin{frame}{Backtraces}{Introducing \funcname{caml\_reraise\_exn}}
  \begin{columns}[c]
    \begin{column}{0.5\textwidth}
      \minipage[c][0.66\textheight][s]{\columnwidth}
      \begin{itemize}
        \item<1-> The exception have to be reraised to the next handler
        \item<2-> First, checks if the backtrace should be collected with \funcarg{domain\_state->backtrace\_active}
        \only<3>{
        \begin{itemize}
            \item If not, behaves like a \funcname{raise\_notrace} with \funcname{RESTORE\_EXN\_HANDLER\_OCAML}
        \end{itemize}
        }
        \item<4-> Jumps into \funcname{caml\_raise\_exn}
        \item<5-> Calls \funcname{caml\_stash\_backtrace} again, to scan the next part of the stack: from the trap just pop-ed to the next trap
      \end{itemize}
      \vfill
      \mintocaml[firstline=15,lastline=21,highlightlines={18-21}]{../src/backtrace/backtrace.ml}
      \endminipage
    \end{column}
    \begin{column}{0.5\textwidth}
      \raggedleft
      \onslide*<1>{\listCamlReraiseExn{}}
      \onslide*<2>{\listCamlReraiseExn[highlightlines=729]{}}
      \onslide*<3>{
        \mintc[firstline=190,lastline=192,highlightlines={191-192}]{../ocaml/runtime/amd64.S}
        \mintc[firstline=234,lastline=237,highlightlines={235-237}]{../ocaml/runtime/amd64.S}
        \medskip
        \listCamlReraiseExn[highlightlines=731]{}
      }
      \onslide*<4-5>{
        % TODO refactor with \listCamlReraiseExn
        \mintasm[firstline=727,lastline=734,highlightlines=730]{../ocaml/runtime/amd64.S}
        \medskip
      }
      \onslide*<4>{\listCamlRaiseExn[highlightlines=711]{}}
      \onslide*<5>{\listCamlRaiseExn[highlightlines=720]{}}
    \end{column}
  \end{columns}
\end{frame}

\begin{frame}{Backtraces}{Backtrace collection continues}
  \begin{columns}[c]
    \begin{column}{0.55\textwidth}
      \minipage[c][0.75\textheight][s]{\columnwidth}
      \begin{itemize}
        \item<1-> \funcname{caml\_stash\_backtrace} scans the older part of the stack, up to the next trap
        \item<6-> Controls jumps to the nested exception handler in \funcname{maybe\_raise}, which matches only on \typename{ExnB}
        \item<7-> \funcname{caml\_reraise\_exn} is called again, backtrace collection resumes with \funcname{caml\_stash\_backtrace}
      \end{itemize}
      \medskip
      \begin{itemize}
        \item<2-> Constructed backtrace:
        \begin{itemize}
          \item <2-> \funcname{definitely\_raise}
          \item <2-> \funcname{probably\_raise}
          \item <3-> \funcname{probably\_raise} (re-raise)
          \item <4-> \funcname{maybe\_raise}
          \item <5-> \funcname{main}
          \item <8-> \funcname{main} (re-raise)
        \end{itemize}
      \end{itemize}
      \vfill
      \onslide*<1-5>{\mintocaml[firstline=15,lastline=21]{../src/backtrace/backtrace.ml}}
      \onslide*<6>{\mintocaml[firstline=28,lastline=34,highlightlines=31]{../src/backtrace/backtrace.ml}}
      \onslide*<7->{\mintocaml[firstline=28,lastline=34,highlightlines=33]{../src/backtrace/backtrace.ml}}
      \endminipage
    \end{column}
    \begin{column}{0.45\textwidth}
      \raggedleft
      \onslide*<1>{\providecommand\step{6}\input{diagrams/backtrace/backtrace.tex}}%
      \onslide*<2>{\providecommand\step{6}\input{diagrams/backtrace/backtrace.tex}}%
      \onslide*<3>{\providecommand\step{7}\input{diagrams/backtrace/backtrace.tex}}%
      \onslide*<4>{\providecommand\step{8}\input{diagrams/backtrace/backtrace.tex}}%
      \onslide*<5>{\providecommand\step{9}\input{diagrams/backtrace/backtrace.tex}}%
      \onslide*<6>{\providecommand\step{10}\input{diagrams/backtrace/backtrace.tex}}%
      \onslide*<7>{\providecommand\step{11}\input{diagrams/backtrace/backtrace.tex}}%
      \onslide*<8>{\providecommand\step{12}\input{diagrams/backtrace/backtrace.tex}}%
      \onslide*<9>{\providecommand\step{13}\input{diagrams/backtrace/backtrace.tex}}%
    \end{column}
  \end{columns}
\end{frame}

\begin{frame}{Backtraces}{Printing the backtrace}
  \begin{columns}[c]
    \begin{column}{0.45\textwidth}
      \minipage[c][0.66\textheight][s]{\columnwidth}
      \begin{itemize}
        \item<1-> The exception backtrace can now be printed to \funcarg{stdout} with \funcname{Printexc.print_backtrace}
        \item<2-> First the exception backtrace is retrieved into an OCaml array with \funcname{get_raw_backtrace }
        \item<3-> Which is implemented as the \funcname{caml_get_exception_raw_backtrace} C function in the runtime
        \item<4-> And finally printed with \funcname{print_exception_backtrace}
      \end{itemize}
      \vfill
      \mintocaml[firstline=28,lastline=34,highlightlines=33]{../src/backtrace/backtrace.ml}
      \endminipage
    \end{column}
    \begin{column}{0.55\textwidth}
      \onslide*<1,2>{
        \mintocaml[firstline=100,lastline=101]{../ocaml/stdlib/printexc.ml}
      }
      \only<1>{
        \listocaml[firstline=170,lastline=172]{../ocaml/stdlib/printexc.ml}{stdlib/printexc.ml:170}
      }
      \only<2>{
        \listocaml[firstline=170,lastline=172,highlightlines=172]{../ocaml/stdlib/printexc.ml}{stdlib/printexc.ml:170}
      }
      \only<3>{
        \mintc[firstline=154,lastline=156]{../ocaml/runtime/backtrace.c}
        \listc[firstline=171,lastline=192,highlightlines={184-187}]{../ocaml/runtime/backtrace.c}{runtime/backtrace.c:154}
      }
      \only<4>{
        \listocaml[firstline=155,lastline=165]{../ocaml/stdlib/printexc.ml}{stdlib/printexc.ml:55}
      }
    \end{column}
  \end{columns}
\end{frame}


\subsection{The multiple kinds of raise}
\frameSubsection{}{}

\begin{frame}{Backtraces}{When backtraces are not needed}
  \begin{columns}[c]
    \begin{column}{0.45\textwidth}
      \minipage[c][0.66\textheight][s]{\columnwidth}
      \begin{itemize}
        \item<1-> What if the backtrace for an exception is never needed?
        \item<2-> It's possible to raise the exception explicitly with \funcname{raise\_notrace} to make raising as cheap as possible
        \item<3-> An example of use in the \funcname{dynlink} library of the OCaml compiler
      \end{itemize}
      \vfill
      \onslide<3->{
      \listocaml[firstline=205,lastline=209,highlightlines=207]{../ocaml/otherlibs/dynlink/dynlink\_compilerlibs/misc.ml}{otherlibs/dynlink/dynlink\_compilerlibs/misc.ml:205}
      }
      \endminipage
    \end{column}
    \begin{column}{0.55\textwidth}
    \only<4>{
      \mintcmm[highlightlines=3]{../src/notrace/camlNotrace\_\_fun_347.cmm}
      \listcmm{../src/notrace/camlNotrace\_\_all\_somes\_327.cmm}{all\_somes}
    }
    \onslide*<5->{
      \listobjdump[highlightlines={6-9}]{../src/notrace/camlNotrace\_\_fun_347.objdump}{anonymous function from \funcname{all\_somes}}
    }
    \end{column}
  \end{columns}
\end{frame}

\begin{frame}{Backtraces}{Summary table of the multiple kinds of raise}
  \begin{itemize}
    \item When debugging information is enabled and if the backtrace of an exception isn't needed, it's possible to raise with \funcname{raise\_notrace} for performance
    \item When debugging information is \emph{not} enabled, the compiler optimises \funcname{raise} calls to inline assembly (same effect as using \funcname{raise\_notrace})
  \end{itemize}
  \bigskip
  \centering
  \begin{tabular}{| l | c | c | c | c |}
    \hline
    \parbox{10em}{Debug information \\ (\funcarg{-g} flag)}  & \multicolumn{2}{|c|}{Disabled}                                     & \multicolumn{2}{|c|}{Enabled} \\
    \hline
    Raise kind                                               & \funcname{raise} & \funcname{raise\_notrace}                       & \funcname{raise} & \funcname{raise\_notrace} \\
    \hline
    Compilation                                              & \parbox{8em}{inline assembly\\ (optimization)} & inline assembly   & \funcname{caml\_raise\_exn} & inline assembly \\
    \hline
  \end{tabular}
\end{frame}


%
% Takeaway #5
%
\subsection*{Takeaway \#5}
\frameSubsectionTakeaway{}
\againframe<5>{takeaway}
}%
      \onslide*<6>{\providecommand\step{10}%
% Backtraces
%
\subsection{One advantage of raising with debugging information}
\frameSubsection{}{}

\begin{frame}{Backtraces}{Debugging information \& source code}
  \begin{columns}[c]
    \begin{column}{0.5\textwidth}
      \begin{itemize}
        \item Raising an exception is fun and games, but how do we know where the exception originated? $\rightarrow$ With the backtrace associated with the exception!
        \item In order to get a backtrace from the point where an exception was raised, it's necessary to compile with debugging information enabled (\funcarg{-g} flag for the compiler)
        \item Same example as in the `Nested handlers' section
      \end{itemize}
    \end{column}
    \begin{column}{0.5\textwidth}
      \listocaml[firstline=9,lastline=34]{../src/backtrace/backtrace.ml}{backtrace/backtrace.ml}
    \end{column}
  \end{columns}
\end{frame}

\begin{frame}{Backtraces}{CMM and disassembly of \funcname{definitely\_raise}}
  \begin{columns}[c]
    \begin{column}{0.5\textwidth}
      \minipage[c][0.66\textheight][s]{\columnwidth}
      \begin{itemize}
        \item<1-> An effect of compiling with debugging information (\funcarg{-g}): the CMM no longer uses \funcname{raise\_notrace} to raise the exception but \funcname{raise} instead
        \item<2-> Disassembly shows that CMM \funcname{raise} is actually a call to \funcname{caml\_raise\_exn}, a function in assembler provided by the runtime
      \end{itemize}
      \vfill
      \mintocaml[firstline=9,lastline=13,highlightlines=12]{../src/backtrace/backtrace.ml}
      \endminipage
    \end{column}
    \begin{column}{0.5\textwidth}
      \onslide*<1>{
        \mintcmm[highlightlines=5]{../src/backtrace/camlBacktrace\_\_definitely\_raise\_347.cmm}
      }
      \onslide*<2>{
        \mintobjdump[firstline=2,highlightlines=14]{../src/backtrace/camlBacktrace\_\_definitely\_raise\_347.objdump}
      }
    \end{column}
  \end{columns}
\end{frame}

\begin{frame}{Backtraces}{Introducing \funcname{caml\_raise\_exn}}
  \begin{columns}[c]
    \begin{column}{0.5\textwidth}
      \minipage[c][0.66\textheight][s]{\columnwidth}
      \onslide*<1>{
        \begin{itemize}
          \item Raising the exception is performed using \funcname{caml\_raise\_exn} which is
            \begin{itemize}
              \item An assembly function provided by the runtime
              \item Architecture specific
            \end{itemize}
          \item Let's see what it does differently from \funcname{raise\_notrace}\dots
          \item We are about to raise \typename{ExnC} with \funcname{caml\_raise\_exn} from \funcname{definitely\_raise}
        \end{itemize}
      }
      \onslide*<2>{
        \mintobjdump[firstline=2,highlightlines=14]{../src/backtrace/camlBacktrace\_\_definitely\_raise\_347.objdump}
      }
      \vfill
      \mintocaml[firstline=9,lastline=13,highlightlines=12]{../src/backtrace/backtrace.ml}
      \endminipage
    \end{column}
    \begin{column}{0.5\textwidth}
      \centering
      \providecommand\step{1}\input{diagrams/backtrace/backtrace.tex}
    \end{column}
  \end{columns}
\end{frame}

\begin{frame}{Backtraces}{But before that, we need more fields from \typename{caml\_domain\_state}}
  \begin{columns}
    \begin{column}{0.5\textwidth}
      \begin{itemize}
        \item Remember \typename{caml\_domain\_state} the per-domain runtime state?
        \item Some other members are of interest to us now\dots
        \item \localname{backtrace\_active} is used to know if the runtime should collect backtraces
        \item \localname{backtrace\_active} can be set by
          \begin{itemize}
            \item The \funcarg{b} flag in the \funcname{OCAMLRUNPARAM} environment variable
            \item \mintinline{ocaml}{Printexc.record_backtrace : bool -> unit} \footnotemark
          \end{itemize}
      \end{itemize}
    \end{column}
    \begin{column}{0.5\textwidth}
      \begin{listing}
        \mintc[highlightlines={9-11}]{src/ocaml/domain\_state\_backtrace.h}
        \caption{runtime/caml/domain\_state.h}
      \end{listing}
    \end{column}
  \end{columns}
  \footnotetext[1]{https://v2.ocaml.org/api/Printexc.html}
\end{frame}

\newcommand\listCamlRaiseExn[1][]{
  \listasm[firstline=702,lastline=725,#1]{../ocaml/runtime/amd64.S}{runtime/amd64.S:702}}

\begin{frame}{Backtraces}{Walk through \funcname{caml\_raise\_exn}}
  \begin{columns}[T]
    \begin{column}{0.5\textwidth}
      \begin{itemize}
        \item<1-> First checks whether exceptions backtrace should be recorded
        \item<1-> By checking the \funcarg{domain\_state->backtrace\_active} value
        \item<2-> If not, acts as \funcname{raise\_notrace} with \funcname{RESTORE\_EXN\_HANDLER\_OCAML}
          \begin{itemize}
            \item Set \regname{RSP} to \localname{domain\_state->exn\_handler} value
            \item Restore \localname{domain\_state->exn\_handler} from trap
            \item Jump with \funcname{ret} (same effect as using \funcname{pop} and \funcname{jmp})
          \end{itemize}
        \item<3-> Otherwise, switches to the C stack to be able to execute C code
        \item<4-> Calls \funcname{caml\_stash\_backtrace}, a C function provided by the runtime\ignorespaces
          \onslide<6->{, with
            \begin{itemize}
              \item \funcarg{exn}: exception value
              \item \funcarg{pc}: code address of the raising point
              \item \funcarg{sp}: stack address of the raising function
              \item \funcarg{trapsp}: stack address of the next handler
            \end{itemize}
          }
      \end{itemize}
      \bigskip
      \mintocaml[firstline=9,lastline=13,highlightlines=12]{../src/backtrace/backtrace.ml}
    \end{column}
    \begin{column}{0.5\textwidth}
      \raggedleft
      \onslide*<1,3-4>{
        \mintc[firstline=190,lastline=192]{../ocaml/runtime/amd64.S}
        \mintc[firstline=234,lastline=237]{../ocaml/runtime/amd64.S}
      }
      \onslide*<2>{
        \mintc[firstline=190,lastline=192,highlightlines={191-192}]{../ocaml/runtime/amd64.S}
        \mintc[firstline=234,lastline=237,highlightlines={235-237}]{../ocaml/runtime/amd64.S}
      }
      \onslide*<1>{\listCamlRaiseExn[highlightlines=705]{}}%
      \onslide*<2>{\listCamlRaiseExn[highlightlines={707-708}]{}}%
      \onslide*<3>{\listCamlRaiseExn[highlightlines={712-714}]{}}%
      \onslide*<4>{\listCamlRaiseExn[highlightlines={715-720}]{}}%
      \onslide*<5>{\providecommand\step{1}\input{diagrams/backtrace/backtrace.tex}}%
      \onslide*<6>{\providecommand\step{2}\input{diagrams/backtrace/backtrace.tex}}%
    \end{column}
  \end{columns}
\end{frame}

\newcommand\listStashBacktrace[1][]{
  \listc[firstline=91,lastline=120,#1]{../ocaml/runtime/backtrace\_nat.c}{runtime/backtrace\_nat.c:91}}

\begin{frame}{Backtraces}{Introducing \funcname{caml\_stash\_backtrace}}
  \begin{columns}[c]
    \begin{column}{0.5\textwidth}
      \minipage[c][0.66\textheight][s]{\columnwidth}
      \begin{itemize}
        \item<1-> A C function provided by the runtime (specific to native code)
        \item<2-> It scans the stack, between the stack pointer at the raising point up to the next trap, in order to collect the names of the detected function frames
          \onslide*<2>{
            \begin{itemize}
              \item And more information, like line number, column number...
            \end{itemize}
          }
        \item<3-> It uses the same technique and information as the Garbage Collector (GC) uses to scan the values live on the stack
          \onslide*<4>{
            \begin{itemize}
              \item \typename{frame\_descr} are at the core of stack unwinding
            \end{itemize}
          }
      \end{itemize}
      \vfill
      \mintocaml[firstline=9,lastline=13,highlightlines=12]{../src/backtrace/backtrace.ml}
      \endminipage
    \end{column}
    \begin{column}{0.5\textwidth}
      \onslide*<1>{\listStashBacktrace[highlightlines=91]{}}
      \onslide*<2>{\listStashBacktrace[highlightlines={91,117-118}]{}}
      \onslide*<3->{\listStashBacktrace[highlightlines={94,106,109-110}]{}}
    \end{column}
  \end{columns}
\end{frame}

\newcommand\listFrameDescr[1][]{
  \listocaml[firstline=111,lastline=124,#1]{../ocaml/asmcomp/emitaux.ml}{asmcomp/emitaux.ml:111}}
\newcommand\listDebugInfo[1][]{
  \listocaml[firstline=38,lastline=49,#1]{../ocaml/lambda/debuginfo.mli}{lambda/debuginfo.mli:38}}

\begin{frame}{Backtraces}{\typename{frame\_descr} and \typename{frame\_debuginfo} in a nutshell}
  \begin{columns}[c]
    \begin{column}{0.5\textwidth}
      \begin{itemize}
        \item<1-> For every return address in an OCaml program, there is a associated \typename{frame\_descr}
        \item<2-> A \typename{frame\_descr} specifies
        \begin{itemize}
          \item<2-> The size of the function stack frame
          \item<3-> Where live values are on the stack (so that the GC can mark them as live and prevent from being swept)
        \end{itemize}
        \item<4-> When compiled with debug information, \typename{frame\_descr} will be linked to a \typename{frame\_debuginfo}
        \begin{itemize}
          \item<5-> Contains information about the position in the source code (file name, line and column number)
        \end{itemize}
      \end{itemize}
    \end{column}
    \begin{column}{0.5\textwidth}
        \onslide*<1>{\listFrameDescr[highlightlines={118-122}]{}}
        \onslide*<2>{\listFrameDescr[highlightlines={120}]{}}
        \onslide*<3>{\listFrameDescr[highlightlines={121}]{}}
        \onslide*<4->{\listFrameDescr[highlightlines={122}]{}}
        \medskip
        \onslide*<1-3>{\listDebugInfo{}}
        \onslide*<4>{\listDebugInfo[highlightlines={38,49}]{}}
        \onslide*<5>{\listDebugInfo[highlightlines={39-46}]{}}
    \end{column}
  \end{columns}
\end{frame}

\begin{frame}{Backtraces}{Scanning the stack with \typename{frame\_descr}}
  \begin{columns}[c]
    \begin{column}{0.5\textwidth}
      \begin{itemize}
        \item Given the return address of the code that raises the exception by calling \funcname{caml\_raise\_exn} (\funcarg{pc} argument of \funcname{caml\_stash\_backtrace})
        \item And the position of the stack pointer (\funcarg{sp} argument of \funcname{caml\_stash\_backtrace})
        \item The frame size given by \typename{frame\_descr}.\funcarg{frame\_size}, allows to find the end of the previous frame
        \item And the return address of the caller of this function
        \bigskip
        \onslide<3->{
        \item Note that traps (here the reddish \funcname{ExnA}, \funcname{ExnB} and \funcname{ExnC} blocks) belong to the frame that installed them, and so the frame size changes when traps are removed
        }
      \end{itemize}
    \end{column}
    \begin{column}{0.5\textwidth}
      \raggedleft
      \onslide*<1>{\providecommand\step{2}\input{diagrams/backtrace/backtrace.tex}}%
      \onslide*<2>{\providecommand\step{1}\input{diagrams/backtrace/scanstack.tex}}%
      \onslide*<3>{\providecommand\step{2}\input{diagrams/backtrace/scanstack.tex}}%
      \onslide*<4>{\providecommand\step{3}\input{diagrams/backtrace/scanstack.tex}}%
      \onslide*<5>{\providecommand\step{4}\input{diagrams/backtrace/scanstack.tex}}%
      \onslide*<6>{\providecommand\step{5}\input{diagrams/backtrace/scanstack.tex}}%
    \end{column}
  \end{columns}
\end{frame}

% Duplicate of previous-previous slide
\begin{frame}{Backtraces}{Continuing on \funcname{caml\_stash\_backtrace}}
  \begin{columns}[c]
    \begin{column}{0.5\textwidth}
      \minipage[c][0.75\textheight][s]{\columnwidth}
      \begin{itemize}
          \color{gray}
        \item A C function provided by the runtime (specific to native code)
        \item It scans the stack, between the stack pointer at the raising point up to the next trap, in order to collect the names of the detected function frames
        \item It uses the same technique and information as the Garbage Collector (GC) uses to scan the values live on the stack
      \end{itemize}
      \begin{itemize}
        \item For each function frame detected, store a pointer to the \typename{frame\_descr} in the \localname{domain\_state->backtrace\_buffer} array, effectively constructing the backtrace
      \end{itemize}
      \onslide<2->{
        \begin{itemize}
            \smallskip
          \item Constructed backtrace:
            \funcname{definitely\_raise},
            \onslide<3->{\funcname{probably\_raise}}
        \end{itemize}
      }
      \vfill
      \mintocaml[firstline=9,lastline=13,highlightlines=12]{../src/backtrace/backtrace.ml}
      \endminipage
    \end{column}
    \begin{column}{0.5\textwidth}
      \raggedleft
      \onslide*<1>{\listStashBacktrace[highlightlines={114-115}]{}}
      \onslide*<2>{\providecommand\step{3}\input{diagrams/backtrace/backtrace.tex}}%
      \onslide*<3>{\providecommand\step{4}\input{diagrams/backtrace/backtrace.tex}}%
    \end{column}
  \end{columns}
\end{frame}

\begin{frame}{Backtraces}{Back to \funcname{caml\_raise\_exn}}
  \begin{columns}[c]
    \begin{column}{0.5\textwidth}
      \minipage[c][0.66\textheight][s]{\columnwidth}
      \begin{itemize}\color{gray}
        \item Checks wheteher exceptions backtrace should be recorded, by checking the \funcarg{domain\_state->backtrace\_active} value
        \item Switches to the C stack to be able to execute C code
        \item Calls \funcname{caml\_stash\_backtrace}, to collect the backtrace up to the next trap
      \end{itemize}
      \onslide*<2->{
        \begin{itemize}
          \item<2-> Executes the exception handler in \funcname{probably\_raise} with \funcname{RESTORE\_EXN\_HANDLER\_OCAML}
            \begin{itemize}
              \item Set \regname{RSP} to \localname{domain\_state->exn\_handler} value
              \item Restore \localname{domain\_state->exn\_handler} from trap
              \item Jump to the handler code with \funcname{ret}
            \end{itemize}
        \end{itemize}
      }
      \vfill
      \onslide*<1>{\mintocaml[firstline=9,lastline=13,highlightlines=12]{../src/backtrace/backtrace.ml}}
      \onslide*<2->{\mintocaml[firstline=15,lastline=21,highlightlines={19-21}]{../src/backtrace/backtrace.ml}}
      \endminipage
    \end{column}
    \begin{column}{0.5\textwidth}
      \centering
      \onslide*<1>{
        \mintc[firstline=190,lastline=192]{../ocaml/runtime/amd64.S}
        \mintc[firstline=234,lastline=237]{../ocaml/runtime/amd64.S}
      }
      \onslide*<2>{
        \mintc[firstline=190,lastline=192,highlightlines={191-192}]{../ocaml/runtime/amd64.S}
        \mintc[firstline=234,lastline=237,highlightlines={235-237}]{../ocaml/runtime/amd64.S}
      }
      \onslide*<1>{\listCamlRaiseExn[highlightlines=720]{}}
      \onslide*<2>{\listCamlRaiseExn[highlightlines=722]{}}
      \onslide*<3->{\providecommand\step{5}\input{diagrams/backtrace/backtrace.tex}}
    \end{column}
  \end{columns}
\end{frame}

\begin{frame}{Backtraces}{\funcname{caml\_probably\_raise}'s handler doesn't catch \typename{ExnC}}
  \begin{columns}[c]
    \begin{column}{0.45\textwidth}
      \minipage[c][0.75\textheight][s]{\columnwidth}
      \begin{itemize}
        \item<1-> \funcname{match \dots with}\footnotemark pattern matching can also match exceptions
        \item<1-> The CMM of \funcname{probably\_raise} shows that this \funcname{match \dots with} is implemented with a pattern matching and a \funcname{try \dots with}
        \item<1-> But this one only matches on \typename{ExnA} exception
        \item<2-> Since the pattern matching fails to catch the exception, \funcname{reraise} is called with our current \typename{ExnC} exception
        \item<3-> Disassembly shows that \funcname{reraise} is actually a call to \funcname{caml\_reraise\_exn}, which is also an assembly function provided by the runtime
      \end{itemize}
      \vfill
      \mintocaml[firstline=15,lastline=21,highlightlines={18-21}]{../src/backtrace/backtrace.ml}
      \endminipage
    \end{column}
    \begin{column}{0.55\textwidth}
      \centering
      \onslide*<1>{\mintcmm[highlightlines={10-18}]{../src/backtrace/camlBacktrace\_\_probably\_raise\_351.cmm}}
      \onslide*<2>{\listcmm[highlightlines=18]{../src/backtrace/camlBacktrace\_\_probably\_raise_351.cmm}{CMM of probably\_raise}}
      \onslide*<3>{\mintobjdump[firstline=5,lastline=35,highlightlines=31]{../src/backtrace/camlBacktrace\_\_probably\_raise_351.objdump}}
    \end{column}
  \end{columns}
  \only<1>{\footnotetext[1]{https://blog.janestreet.com/pattern-matching-and-exception-handling-unite/}}
\end{frame}

\newcommand\listCamlReraiseExn[1][]{
  \listasm[firstline=727,lastline=734,#1]{../ocaml/runtime/amd64.S}{runtime/amd64.S:727}}

\begin{frame}{Backtraces}{Introducing \funcname{caml\_reraise\_exn}}
  \begin{columns}[c]
    \begin{column}{0.5\textwidth}
      \minipage[c][0.66\textheight][s]{\columnwidth}
      \begin{itemize}
        \item<1-> The exception have to be reraised to the next handler
        \item<2-> First, checks if the backtrace should be collected with \funcarg{domain\_state->backtrace\_active}
        \only<3>{
        \begin{itemize}
            \item If not, behaves like a \funcname{raise\_notrace} with \funcname{RESTORE\_EXN\_HANDLER\_OCAML}
        \end{itemize}
        }
        \item<4-> Jumps into \funcname{caml\_raise\_exn}
        \item<5-> Calls \funcname{caml\_stash\_backtrace} again, to scan the next part of the stack: from the trap just pop-ed to the next trap
      \end{itemize}
      \vfill
      \mintocaml[firstline=15,lastline=21,highlightlines={18-21}]{../src/backtrace/backtrace.ml}
      \endminipage
    \end{column}
    \begin{column}{0.5\textwidth}
      \raggedleft
      \onslide*<1>{\listCamlReraiseExn{}}
      \onslide*<2>{\listCamlReraiseExn[highlightlines=729]{}}
      \onslide*<3>{
        \mintc[firstline=190,lastline=192,highlightlines={191-192}]{../ocaml/runtime/amd64.S}
        \mintc[firstline=234,lastline=237,highlightlines={235-237}]{../ocaml/runtime/amd64.S}
        \medskip
        \listCamlReraiseExn[highlightlines=731]{}
      }
      \onslide*<4-5>{
        % TODO refactor with \listCamlReraiseExn
        \mintasm[firstline=727,lastline=734,highlightlines=730]{../ocaml/runtime/amd64.S}
        \medskip
      }
      \onslide*<4>{\listCamlRaiseExn[highlightlines=711]{}}
      \onslide*<5>{\listCamlRaiseExn[highlightlines=720]{}}
    \end{column}
  \end{columns}
\end{frame}

\begin{frame}{Backtraces}{Backtrace collection continues}
  \begin{columns}[c]
    \begin{column}{0.55\textwidth}
      \minipage[c][0.75\textheight][s]{\columnwidth}
      \begin{itemize}
        \item<1-> \funcname{caml\_stash\_backtrace} scans the older part of the stack, up to the next trap
        \item<6-> Controls jumps to the nested exception handler in \funcname{maybe\_raise}, which matches only on \typename{ExnB}
        \item<7-> \funcname{caml\_reraise\_exn} is called again, backtrace collection resumes with \funcname{caml\_stash\_backtrace}
      \end{itemize}
      \medskip
      \begin{itemize}
        \item<2-> Constructed backtrace:
        \begin{itemize}
          \item <2-> \funcname{definitely\_raise}
          \item <2-> \funcname{probably\_raise}
          \item <3-> \funcname{probably\_raise} (re-raise)
          \item <4-> \funcname{maybe\_raise}
          \item <5-> \funcname{main}
          \item <8-> \funcname{main} (re-raise)
        \end{itemize}
      \end{itemize}
      \vfill
      \onslide*<1-5>{\mintocaml[firstline=15,lastline=21]{../src/backtrace/backtrace.ml}}
      \onslide*<6>{\mintocaml[firstline=28,lastline=34,highlightlines=31]{../src/backtrace/backtrace.ml}}
      \onslide*<7->{\mintocaml[firstline=28,lastline=34,highlightlines=33]{../src/backtrace/backtrace.ml}}
      \endminipage
    \end{column}
    \begin{column}{0.45\textwidth}
      \raggedleft
      \onslide*<1>{\providecommand\step{6}\input{diagrams/backtrace/backtrace.tex}}%
      \onslide*<2>{\providecommand\step{6}\input{diagrams/backtrace/backtrace.tex}}%
      \onslide*<3>{\providecommand\step{7}\input{diagrams/backtrace/backtrace.tex}}%
      \onslide*<4>{\providecommand\step{8}\input{diagrams/backtrace/backtrace.tex}}%
      \onslide*<5>{\providecommand\step{9}\input{diagrams/backtrace/backtrace.tex}}%
      \onslide*<6>{\providecommand\step{10}\input{diagrams/backtrace/backtrace.tex}}%
      \onslide*<7>{\providecommand\step{11}\input{diagrams/backtrace/backtrace.tex}}%
      \onslide*<8>{\providecommand\step{12}\input{diagrams/backtrace/backtrace.tex}}%
      \onslide*<9>{\providecommand\step{13}\input{diagrams/backtrace/backtrace.tex}}%
    \end{column}
  \end{columns}
\end{frame}

\begin{frame}{Backtraces}{Printing the backtrace}
  \begin{columns}[c]
    \begin{column}{0.45\textwidth}
      \minipage[c][0.66\textheight][s]{\columnwidth}
      \begin{itemize}
        \item<1-> The exception backtrace can now be printed to \funcarg{stdout} with \funcname{Printexc.print_backtrace}
        \item<2-> First the exception backtrace is retrieved into an OCaml array with \funcname{get_raw_backtrace }
        \item<3-> Which is implemented as the \funcname{caml_get_exception_raw_backtrace} C function in the runtime
        \item<4-> And finally printed with \funcname{print_exception_backtrace}
      \end{itemize}
      \vfill
      \mintocaml[firstline=28,lastline=34,highlightlines=33]{../src/backtrace/backtrace.ml}
      \endminipage
    \end{column}
    \begin{column}{0.55\textwidth}
      \onslide*<1,2>{
        \mintocaml[firstline=100,lastline=101]{../ocaml/stdlib/printexc.ml}
      }
      \only<1>{
        \listocaml[firstline=170,lastline=172]{../ocaml/stdlib/printexc.ml}{stdlib/printexc.ml:170}
      }
      \only<2>{
        \listocaml[firstline=170,lastline=172,highlightlines=172]{../ocaml/stdlib/printexc.ml}{stdlib/printexc.ml:170}
      }
      \only<3>{
        \mintc[firstline=154,lastline=156]{../ocaml/runtime/backtrace.c}
        \listc[firstline=171,lastline=192,highlightlines={184-187}]{../ocaml/runtime/backtrace.c}{runtime/backtrace.c:154}
      }
      \only<4>{
        \listocaml[firstline=155,lastline=165]{../ocaml/stdlib/printexc.ml}{stdlib/printexc.ml:55}
      }
    \end{column}
  \end{columns}
\end{frame}


\subsection{The multiple kinds of raise}
\frameSubsection{}{}

\begin{frame}{Backtraces}{When backtraces are not needed}
  \begin{columns}[c]
    \begin{column}{0.45\textwidth}
      \minipage[c][0.66\textheight][s]{\columnwidth}
      \begin{itemize}
        \item<1-> What if the backtrace for an exception is never needed?
        \item<2-> It's possible to raise the exception explicitly with \funcname{raise\_notrace} to make raising as cheap as possible
        \item<3-> An example of use in the \funcname{dynlink} library of the OCaml compiler
      \end{itemize}
      \vfill
      \onslide<3->{
      \listocaml[firstline=205,lastline=209,highlightlines=207]{../ocaml/otherlibs/dynlink/dynlink\_compilerlibs/misc.ml}{otherlibs/dynlink/dynlink\_compilerlibs/misc.ml:205}
      }
      \endminipage
    \end{column}
    \begin{column}{0.55\textwidth}
    \only<4>{
      \mintcmm[highlightlines=3]{../src/notrace/camlNotrace\_\_fun_347.cmm}
      \listcmm{../src/notrace/camlNotrace\_\_all\_somes\_327.cmm}{all\_somes}
    }
    \onslide*<5->{
      \listobjdump[highlightlines={6-9}]{../src/notrace/camlNotrace\_\_fun_347.objdump}{anonymous function from \funcname{all\_somes}}
    }
    \end{column}
  \end{columns}
\end{frame}

\begin{frame}{Backtraces}{Summary table of the multiple kinds of raise}
  \begin{itemize}
    \item When debugging information is enabled and if the backtrace of an exception isn't needed, it's possible to raise with \funcname{raise\_notrace} for performance
    \item When debugging information is \emph{not} enabled, the compiler optimises \funcname{raise} calls to inline assembly (same effect as using \funcname{raise\_notrace})
  \end{itemize}
  \bigskip
  \centering
  \begin{tabular}{| l | c | c | c | c |}
    \hline
    \parbox{10em}{Debug information \\ (\funcarg{-g} flag)}  & \multicolumn{2}{|c|}{Disabled}                                     & \multicolumn{2}{|c|}{Enabled} \\
    \hline
    Raise kind                                               & \funcname{raise} & \funcname{raise\_notrace}                       & \funcname{raise} & \funcname{raise\_notrace} \\
    \hline
    Compilation                                              & \parbox{8em}{inline assembly\\ (optimization)} & inline assembly   & \funcname{caml\_raise\_exn} & inline assembly \\
    \hline
  \end{tabular}
\end{frame}


%
% Takeaway #5
%
\subsection*{Takeaway \#5}
\frameSubsectionTakeaway{}
\againframe<5>{takeaway}
}%
      \onslide*<7>{\providecommand\step{11}%
% Backtraces
%
\subsection{One advantage of raising with debugging information}
\frameSubsection{}{}

\begin{frame}{Backtraces}{Debugging information \& source code}
  \begin{columns}[c]
    \begin{column}{0.5\textwidth}
      \begin{itemize}
        \item Raising an exception is fun and games, but how do we know where the exception originated? $\rightarrow$ With the backtrace associated with the exception!
        \item In order to get a backtrace from the point where an exception was raised, it's necessary to compile with debugging information enabled (\funcarg{-g} flag for the compiler)
        \item Same example as in the `Nested handlers' section
      \end{itemize}
    \end{column}
    \begin{column}{0.5\textwidth}
      \listocaml[firstline=9,lastline=34]{../src/backtrace/backtrace.ml}{backtrace/backtrace.ml}
    \end{column}
  \end{columns}
\end{frame}

\begin{frame}{Backtraces}{CMM and disassembly of \funcname{definitely\_raise}}
  \begin{columns}[c]
    \begin{column}{0.5\textwidth}
      \minipage[c][0.66\textheight][s]{\columnwidth}
      \begin{itemize}
        \item<1-> An effect of compiling with debugging information (\funcarg{-g}): the CMM no longer uses \funcname{raise\_notrace} to raise the exception but \funcname{raise} instead
        \item<2-> Disassembly shows that CMM \funcname{raise} is actually a call to \funcname{caml\_raise\_exn}, a function in assembler provided by the runtime
      \end{itemize}
      \vfill
      \mintocaml[firstline=9,lastline=13,highlightlines=12]{../src/backtrace/backtrace.ml}
      \endminipage
    \end{column}
    \begin{column}{0.5\textwidth}
      \onslide*<1>{
        \mintcmm[highlightlines=5]{../src/backtrace/camlBacktrace\_\_definitely\_raise\_347.cmm}
      }
      \onslide*<2>{
        \mintobjdump[firstline=2,highlightlines=14]{../src/backtrace/camlBacktrace\_\_definitely\_raise\_347.objdump}
      }
    \end{column}
  \end{columns}
\end{frame}

\begin{frame}{Backtraces}{Introducing \funcname{caml\_raise\_exn}}
  \begin{columns}[c]
    \begin{column}{0.5\textwidth}
      \minipage[c][0.66\textheight][s]{\columnwidth}
      \onslide*<1>{
        \begin{itemize}
          \item Raising the exception is performed using \funcname{caml\_raise\_exn} which is
            \begin{itemize}
              \item An assembly function provided by the runtime
              \item Architecture specific
            \end{itemize}
          \item Let's see what it does differently from \funcname{raise\_notrace}\dots
          \item We are about to raise \typename{ExnC} with \funcname{caml\_raise\_exn} from \funcname{definitely\_raise}
        \end{itemize}
      }
      \onslide*<2>{
        \mintobjdump[firstline=2,highlightlines=14]{../src/backtrace/camlBacktrace\_\_definitely\_raise\_347.objdump}
      }
      \vfill
      \mintocaml[firstline=9,lastline=13,highlightlines=12]{../src/backtrace/backtrace.ml}
      \endminipage
    \end{column}
    \begin{column}{0.5\textwidth}
      \centering
      \providecommand\step{1}\input{diagrams/backtrace/backtrace.tex}
    \end{column}
  \end{columns}
\end{frame}

\begin{frame}{Backtraces}{But before that, we need more fields from \typename{caml\_domain\_state}}
  \begin{columns}
    \begin{column}{0.5\textwidth}
      \begin{itemize}
        \item Remember \typename{caml\_domain\_state} the per-domain runtime state?
        \item Some other members are of interest to us now\dots
        \item \localname{backtrace\_active} is used to know if the runtime should collect backtraces
        \item \localname{backtrace\_active} can be set by
          \begin{itemize}
            \item The \funcarg{b} flag in the \funcname{OCAMLRUNPARAM} environment variable
            \item \mintinline{ocaml}{Printexc.record_backtrace : bool -> unit} \footnotemark
          \end{itemize}
      \end{itemize}
    \end{column}
    \begin{column}{0.5\textwidth}
      \begin{listing}
        \mintc[highlightlines={9-11}]{src/ocaml/domain\_state\_backtrace.h}
        \caption{runtime/caml/domain\_state.h}
      \end{listing}
    \end{column}
  \end{columns}
  \footnotetext[1]{https://v2.ocaml.org/api/Printexc.html}
\end{frame}

\newcommand\listCamlRaiseExn[1][]{
  \listasm[firstline=702,lastline=725,#1]{../ocaml/runtime/amd64.S}{runtime/amd64.S:702}}

\begin{frame}{Backtraces}{Walk through \funcname{caml\_raise\_exn}}
  \begin{columns}[T]
    \begin{column}{0.5\textwidth}
      \begin{itemize}
        \item<1-> First checks whether exceptions backtrace should be recorded
        \item<1-> By checking the \funcarg{domain\_state->backtrace\_active} value
        \item<2-> If not, acts as \funcname{raise\_notrace} with \funcname{RESTORE\_EXN\_HANDLER\_OCAML}
          \begin{itemize}
            \item Set \regname{RSP} to \localname{domain\_state->exn\_handler} value
            \item Restore \localname{domain\_state->exn\_handler} from trap
            \item Jump with \funcname{ret} (same effect as using \funcname{pop} and \funcname{jmp})
          \end{itemize}
        \item<3-> Otherwise, switches to the C stack to be able to execute C code
        \item<4-> Calls \funcname{caml\_stash\_backtrace}, a C function provided by the runtime\ignorespaces
          \onslide<6->{, with
            \begin{itemize}
              \item \funcarg{exn}: exception value
              \item \funcarg{pc}: code address of the raising point
              \item \funcarg{sp}: stack address of the raising function
              \item \funcarg{trapsp}: stack address of the next handler
            \end{itemize}
          }
      \end{itemize}
      \bigskip
      \mintocaml[firstline=9,lastline=13,highlightlines=12]{../src/backtrace/backtrace.ml}
    \end{column}
    \begin{column}{0.5\textwidth}
      \raggedleft
      \onslide*<1,3-4>{
        \mintc[firstline=190,lastline=192]{../ocaml/runtime/amd64.S}
        \mintc[firstline=234,lastline=237]{../ocaml/runtime/amd64.S}
      }
      \onslide*<2>{
        \mintc[firstline=190,lastline=192,highlightlines={191-192}]{../ocaml/runtime/amd64.S}
        \mintc[firstline=234,lastline=237,highlightlines={235-237}]{../ocaml/runtime/amd64.S}
      }
      \onslide*<1>{\listCamlRaiseExn[highlightlines=705]{}}%
      \onslide*<2>{\listCamlRaiseExn[highlightlines={707-708}]{}}%
      \onslide*<3>{\listCamlRaiseExn[highlightlines={712-714}]{}}%
      \onslide*<4>{\listCamlRaiseExn[highlightlines={715-720}]{}}%
      \onslide*<5>{\providecommand\step{1}\input{diagrams/backtrace/backtrace.tex}}%
      \onslide*<6>{\providecommand\step{2}\input{diagrams/backtrace/backtrace.tex}}%
    \end{column}
  \end{columns}
\end{frame}

\newcommand\listStashBacktrace[1][]{
  \listc[firstline=91,lastline=120,#1]{../ocaml/runtime/backtrace\_nat.c}{runtime/backtrace\_nat.c:91}}

\begin{frame}{Backtraces}{Introducing \funcname{caml\_stash\_backtrace}}
  \begin{columns}[c]
    \begin{column}{0.5\textwidth}
      \minipage[c][0.66\textheight][s]{\columnwidth}
      \begin{itemize}
        \item<1-> A C function provided by the runtime (specific to native code)
        \item<2-> It scans the stack, between the stack pointer at the raising point up to the next trap, in order to collect the names of the detected function frames
          \onslide*<2>{
            \begin{itemize}
              \item And more information, like line number, column number...
            \end{itemize}
          }
        \item<3-> It uses the same technique and information as the Garbage Collector (GC) uses to scan the values live on the stack
          \onslide*<4>{
            \begin{itemize}
              \item \typename{frame\_descr} are at the core of stack unwinding
            \end{itemize}
          }
      \end{itemize}
      \vfill
      \mintocaml[firstline=9,lastline=13,highlightlines=12]{../src/backtrace/backtrace.ml}
      \endminipage
    \end{column}
    \begin{column}{0.5\textwidth}
      \onslide*<1>{\listStashBacktrace[highlightlines=91]{}}
      \onslide*<2>{\listStashBacktrace[highlightlines={91,117-118}]{}}
      \onslide*<3->{\listStashBacktrace[highlightlines={94,106,109-110}]{}}
    \end{column}
  \end{columns}
\end{frame}

\newcommand\listFrameDescr[1][]{
  \listocaml[firstline=111,lastline=124,#1]{../ocaml/asmcomp/emitaux.ml}{asmcomp/emitaux.ml:111}}
\newcommand\listDebugInfo[1][]{
  \listocaml[firstline=38,lastline=49,#1]{../ocaml/lambda/debuginfo.mli}{lambda/debuginfo.mli:38}}

\begin{frame}{Backtraces}{\typename{frame\_descr} and \typename{frame\_debuginfo} in a nutshell}
  \begin{columns}[c]
    \begin{column}{0.5\textwidth}
      \begin{itemize}
        \item<1-> For every return address in an OCaml program, there is a associated \typename{frame\_descr}
        \item<2-> A \typename{frame\_descr} specifies
        \begin{itemize}
          \item<2-> The size of the function stack frame
          \item<3-> Where live values are on the stack (so that the GC can mark them as live and prevent from being swept)
        \end{itemize}
        \item<4-> When compiled with debug information, \typename{frame\_descr} will be linked to a \typename{frame\_debuginfo}
        \begin{itemize}
          \item<5-> Contains information about the position in the source code (file name, line and column number)
        \end{itemize}
      \end{itemize}
    \end{column}
    \begin{column}{0.5\textwidth}
        \onslide*<1>{\listFrameDescr[highlightlines={118-122}]{}}
        \onslide*<2>{\listFrameDescr[highlightlines={120}]{}}
        \onslide*<3>{\listFrameDescr[highlightlines={121}]{}}
        \onslide*<4->{\listFrameDescr[highlightlines={122}]{}}
        \medskip
        \onslide*<1-3>{\listDebugInfo{}}
        \onslide*<4>{\listDebugInfo[highlightlines={38,49}]{}}
        \onslide*<5>{\listDebugInfo[highlightlines={39-46}]{}}
    \end{column}
  \end{columns}
\end{frame}

\begin{frame}{Backtraces}{Scanning the stack with \typename{frame\_descr}}
  \begin{columns}[c]
    \begin{column}{0.5\textwidth}
      \begin{itemize}
        \item Given the return address of the code that raises the exception by calling \funcname{caml\_raise\_exn} (\funcarg{pc} argument of \funcname{caml\_stash\_backtrace})
        \item And the position of the stack pointer (\funcarg{sp} argument of \funcname{caml\_stash\_backtrace})
        \item The frame size given by \typename{frame\_descr}.\funcarg{frame\_size}, allows to find the end of the previous frame
        \item And the return address of the caller of this function
        \bigskip
        \onslide<3->{
        \item Note that traps (here the reddish \funcname{ExnA}, \funcname{ExnB} and \funcname{ExnC} blocks) belong to the frame that installed them, and so the frame size changes when traps are removed
        }
      \end{itemize}
    \end{column}
    \begin{column}{0.5\textwidth}
      \raggedleft
      \onslide*<1>{\providecommand\step{2}\input{diagrams/backtrace/backtrace.tex}}%
      \onslide*<2>{\providecommand\step{1}\input{diagrams/backtrace/scanstack.tex}}%
      \onslide*<3>{\providecommand\step{2}\input{diagrams/backtrace/scanstack.tex}}%
      \onslide*<4>{\providecommand\step{3}\input{diagrams/backtrace/scanstack.tex}}%
      \onslide*<5>{\providecommand\step{4}\input{diagrams/backtrace/scanstack.tex}}%
      \onslide*<6>{\providecommand\step{5}\input{diagrams/backtrace/scanstack.tex}}%
    \end{column}
  \end{columns}
\end{frame}

% Duplicate of previous-previous slide
\begin{frame}{Backtraces}{Continuing on \funcname{caml\_stash\_backtrace}}
  \begin{columns}[c]
    \begin{column}{0.5\textwidth}
      \minipage[c][0.75\textheight][s]{\columnwidth}
      \begin{itemize}
          \color{gray}
        \item A C function provided by the runtime (specific to native code)
        \item It scans the stack, between the stack pointer at the raising point up to the next trap, in order to collect the names of the detected function frames
        \item It uses the same technique and information as the Garbage Collector (GC) uses to scan the values live on the stack
      \end{itemize}
      \begin{itemize}
        \item For each function frame detected, store a pointer to the \typename{frame\_descr} in the \localname{domain\_state->backtrace\_buffer} array, effectively constructing the backtrace
      \end{itemize}
      \onslide<2->{
        \begin{itemize}
            \smallskip
          \item Constructed backtrace:
            \funcname{definitely\_raise},
            \onslide<3->{\funcname{probably\_raise}}
        \end{itemize}
      }
      \vfill
      \mintocaml[firstline=9,lastline=13,highlightlines=12]{../src/backtrace/backtrace.ml}
      \endminipage
    \end{column}
    \begin{column}{0.5\textwidth}
      \raggedleft
      \onslide*<1>{\listStashBacktrace[highlightlines={114-115}]{}}
      \onslide*<2>{\providecommand\step{3}\input{diagrams/backtrace/backtrace.tex}}%
      \onslide*<3>{\providecommand\step{4}\input{diagrams/backtrace/backtrace.tex}}%
    \end{column}
  \end{columns}
\end{frame}

\begin{frame}{Backtraces}{Back to \funcname{caml\_raise\_exn}}
  \begin{columns}[c]
    \begin{column}{0.5\textwidth}
      \minipage[c][0.66\textheight][s]{\columnwidth}
      \begin{itemize}\color{gray}
        \item Checks wheteher exceptions backtrace should be recorded, by checking the \funcarg{domain\_state->backtrace\_active} value
        \item Switches to the C stack to be able to execute C code
        \item Calls \funcname{caml\_stash\_backtrace}, to collect the backtrace up to the next trap
      \end{itemize}
      \onslide*<2->{
        \begin{itemize}
          \item<2-> Executes the exception handler in \funcname{probably\_raise} with \funcname{RESTORE\_EXN\_HANDLER\_OCAML}
            \begin{itemize}
              \item Set \regname{RSP} to \localname{domain\_state->exn\_handler} value
              \item Restore \localname{domain\_state->exn\_handler} from trap
              \item Jump to the handler code with \funcname{ret}
            \end{itemize}
        \end{itemize}
      }
      \vfill
      \onslide*<1>{\mintocaml[firstline=9,lastline=13,highlightlines=12]{../src/backtrace/backtrace.ml}}
      \onslide*<2->{\mintocaml[firstline=15,lastline=21,highlightlines={19-21}]{../src/backtrace/backtrace.ml}}
      \endminipage
    \end{column}
    \begin{column}{0.5\textwidth}
      \centering
      \onslide*<1>{
        \mintc[firstline=190,lastline=192]{../ocaml/runtime/amd64.S}
        \mintc[firstline=234,lastline=237]{../ocaml/runtime/amd64.S}
      }
      \onslide*<2>{
        \mintc[firstline=190,lastline=192,highlightlines={191-192}]{../ocaml/runtime/amd64.S}
        \mintc[firstline=234,lastline=237,highlightlines={235-237}]{../ocaml/runtime/amd64.S}
      }
      \onslide*<1>{\listCamlRaiseExn[highlightlines=720]{}}
      \onslide*<2>{\listCamlRaiseExn[highlightlines=722]{}}
      \onslide*<3->{\providecommand\step{5}\input{diagrams/backtrace/backtrace.tex}}
    \end{column}
  \end{columns}
\end{frame}

\begin{frame}{Backtraces}{\funcname{caml\_probably\_raise}'s handler doesn't catch \typename{ExnC}}
  \begin{columns}[c]
    \begin{column}{0.45\textwidth}
      \minipage[c][0.75\textheight][s]{\columnwidth}
      \begin{itemize}
        \item<1-> \funcname{match \dots with}\footnotemark pattern matching can also match exceptions
        \item<1-> The CMM of \funcname{probably\_raise} shows that this \funcname{match \dots with} is implemented with a pattern matching and a \funcname{try \dots with}
        \item<1-> But this one only matches on \typename{ExnA} exception
        \item<2-> Since the pattern matching fails to catch the exception, \funcname{reraise} is called with our current \typename{ExnC} exception
        \item<3-> Disassembly shows that \funcname{reraise} is actually a call to \funcname{caml\_reraise\_exn}, which is also an assembly function provided by the runtime
      \end{itemize}
      \vfill
      \mintocaml[firstline=15,lastline=21,highlightlines={18-21}]{../src/backtrace/backtrace.ml}
      \endminipage
    \end{column}
    \begin{column}{0.55\textwidth}
      \centering
      \onslide*<1>{\mintcmm[highlightlines={10-18}]{../src/backtrace/camlBacktrace\_\_probably\_raise\_351.cmm}}
      \onslide*<2>{\listcmm[highlightlines=18]{../src/backtrace/camlBacktrace\_\_probably\_raise_351.cmm}{CMM of probably\_raise}}
      \onslide*<3>{\mintobjdump[firstline=5,lastline=35,highlightlines=31]{../src/backtrace/camlBacktrace\_\_probably\_raise_351.objdump}}
    \end{column}
  \end{columns}
  \only<1>{\footnotetext[1]{https://blog.janestreet.com/pattern-matching-and-exception-handling-unite/}}
\end{frame}

\newcommand\listCamlReraiseExn[1][]{
  \listasm[firstline=727,lastline=734,#1]{../ocaml/runtime/amd64.S}{runtime/amd64.S:727}}

\begin{frame}{Backtraces}{Introducing \funcname{caml\_reraise\_exn}}
  \begin{columns}[c]
    \begin{column}{0.5\textwidth}
      \minipage[c][0.66\textheight][s]{\columnwidth}
      \begin{itemize}
        \item<1-> The exception have to be reraised to the next handler
        \item<2-> First, checks if the backtrace should be collected with \funcarg{domain\_state->backtrace\_active}
        \only<3>{
        \begin{itemize}
            \item If not, behaves like a \funcname{raise\_notrace} with \funcname{RESTORE\_EXN\_HANDLER\_OCAML}
        \end{itemize}
        }
        \item<4-> Jumps into \funcname{caml\_raise\_exn}
        \item<5-> Calls \funcname{caml\_stash\_backtrace} again, to scan the next part of the stack: from the trap just pop-ed to the next trap
      \end{itemize}
      \vfill
      \mintocaml[firstline=15,lastline=21,highlightlines={18-21}]{../src/backtrace/backtrace.ml}
      \endminipage
    \end{column}
    \begin{column}{0.5\textwidth}
      \raggedleft
      \onslide*<1>{\listCamlReraiseExn{}}
      \onslide*<2>{\listCamlReraiseExn[highlightlines=729]{}}
      \onslide*<3>{
        \mintc[firstline=190,lastline=192,highlightlines={191-192}]{../ocaml/runtime/amd64.S}
        \mintc[firstline=234,lastline=237,highlightlines={235-237}]{../ocaml/runtime/amd64.S}
        \medskip
        \listCamlReraiseExn[highlightlines=731]{}
      }
      \onslide*<4-5>{
        % TODO refactor with \listCamlReraiseExn
        \mintasm[firstline=727,lastline=734,highlightlines=730]{../ocaml/runtime/amd64.S}
        \medskip
      }
      \onslide*<4>{\listCamlRaiseExn[highlightlines=711]{}}
      \onslide*<5>{\listCamlRaiseExn[highlightlines=720]{}}
    \end{column}
  \end{columns}
\end{frame}

\begin{frame}{Backtraces}{Backtrace collection continues}
  \begin{columns}[c]
    \begin{column}{0.55\textwidth}
      \minipage[c][0.75\textheight][s]{\columnwidth}
      \begin{itemize}
        \item<1-> \funcname{caml\_stash\_backtrace} scans the older part of the stack, up to the next trap
        \item<6-> Controls jumps to the nested exception handler in \funcname{maybe\_raise}, which matches only on \typename{ExnB}
        \item<7-> \funcname{caml\_reraise\_exn} is called again, backtrace collection resumes with \funcname{caml\_stash\_backtrace}
      \end{itemize}
      \medskip
      \begin{itemize}
        \item<2-> Constructed backtrace:
        \begin{itemize}
          \item <2-> \funcname{definitely\_raise}
          \item <2-> \funcname{probably\_raise}
          \item <3-> \funcname{probably\_raise} (re-raise)
          \item <4-> \funcname{maybe\_raise}
          \item <5-> \funcname{main}
          \item <8-> \funcname{main} (re-raise)
        \end{itemize}
      \end{itemize}
      \vfill
      \onslide*<1-5>{\mintocaml[firstline=15,lastline=21]{../src/backtrace/backtrace.ml}}
      \onslide*<6>{\mintocaml[firstline=28,lastline=34,highlightlines=31]{../src/backtrace/backtrace.ml}}
      \onslide*<7->{\mintocaml[firstline=28,lastline=34,highlightlines=33]{../src/backtrace/backtrace.ml}}
      \endminipage
    \end{column}
    \begin{column}{0.45\textwidth}
      \raggedleft
      \onslide*<1>{\providecommand\step{6}\input{diagrams/backtrace/backtrace.tex}}%
      \onslide*<2>{\providecommand\step{6}\input{diagrams/backtrace/backtrace.tex}}%
      \onslide*<3>{\providecommand\step{7}\input{diagrams/backtrace/backtrace.tex}}%
      \onslide*<4>{\providecommand\step{8}\input{diagrams/backtrace/backtrace.tex}}%
      \onslide*<5>{\providecommand\step{9}\input{diagrams/backtrace/backtrace.tex}}%
      \onslide*<6>{\providecommand\step{10}\input{diagrams/backtrace/backtrace.tex}}%
      \onslide*<7>{\providecommand\step{11}\input{diagrams/backtrace/backtrace.tex}}%
      \onslide*<8>{\providecommand\step{12}\input{diagrams/backtrace/backtrace.tex}}%
      \onslide*<9>{\providecommand\step{13}\input{diagrams/backtrace/backtrace.tex}}%
    \end{column}
  \end{columns}
\end{frame}

\begin{frame}{Backtraces}{Printing the backtrace}
  \begin{columns}[c]
    \begin{column}{0.45\textwidth}
      \minipage[c][0.66\textheight][s]{\columnwidth}
      \begin{itemize}
        \item<1-> The exception backtrace can now be printed to \funcarg{stdout} with \funcname{Printexc.print_backtrace}
        \item<2-> First the exception backtrace is retrieved into an OCaml array with \funcname{get_raw_backtrace }
        \item<3-> Which is implemented as the \funcname{caml_get_exception_raw_backtrace} C function in the runtime
        \item<4-> And finally printed with \funcname{print_exception_backtrace}
      \end{itemize}
      \vfill
      \mintocaml[firstline=28,lastline=34,highlightlines=33]{../src/backtrace/backtrace.ml}
      \endminipage
    \end{column}
    \begin{column}{0.55\textwidth}
      \onslide*<1,2>{
        \mintocaml[firstline=100,lastline=101]{../ocaml/stdlib/printexc.ml}
      }
      \only<1>{
        \listocaml[firstline=170,lastline=172]{../ocaml/stdlib/printexc.ml}{stdlib/printexc.ml:170}
      }
      \only<2>{
        \listocaml[firstline=170,lastline=172,highlightlines=172]{../ocaml/stdlib/printexc.ml}{stdlib/printexc.ml:170}
      }
      \only<3>{
        \mintc[firstline=154,lastline=156]{../ocaml/runtime/backtrace.c}
        \listc[firstline=171,lastline=192,highlightlines={184-187}]{../ocaml/runtime/backtrace.c}{runtime/backtrace.c:154}
      }
      \only<4>{
        \listocaml[firstline=155,lastline=165]{../ocaml/stdlib/printexc.ml}{stdlib/printexc.ml:55}
      }
    \end{column}
  \end{columns}
\end{frame}


\subsection{The multiple kinds of raise}
\frameSubsection{}{}

\begin{frame}{Backtraces}{When backtraces are not needed}
  \begin{columns}[c]
    \begin{column}{0.45\textwidth}
      \minipage[c][0.66\textheight][s]{\columnwidth}
      \begin{itemize}
        \item<1-> What if the backtrace for an exception is never needed?
        \item<2-> It's possible to raise the exception explicitly with \funcname{raise\_notrace} to make raising as cheap as possible
        \item<3-> An example of use in the \funcname{dynlink} library of the OCaml compiler
      \end{itemize}
      \vfill
      \onslide<3->{
      \listocaml[firstline=205,lastline=209,highlightlines=207]{../ocaml/otherlibs/dynlink/dynlink\_compilerlibs/misc.ml}{otherlibs/dynlink/dynlink\_compilerlibs/misc.ml:205}
      }
      \endminipage
    \end{column}
    \begin{column}{0.55\textwidth}
    \only<4>{
      \mintcmm[highlightlines=3]{../src/notrace/camlNotrace\_\_fun_347.cmm}
      \listcmm{../src/notrace/camlNotrace\_\_all\_somes\_327.cmm}{all\_somes}
    }
    \onslide*<5->{
      \listobjdump[highlightlines={6-9}]{../src/notrace/camlNotrace\_\_fun_347.objdump}{anonymous function from \funcname{all\_somes}}
    }
    \end{column}
  \end{columns}
\end{frame}

\begin{frame}{Backtraces}{Summary table of the multiple kinds of raise}
  \begin{itemize}
    \item When debugging information is enabled and if the backtrace of an exception isn't needed, it's possible to raise with \funcname{raise\_notrace} for performance
    \item When debugging information is \emph{not} enabled, the compiler optimises \funcname{raise} calls to inline assembly (same effect as using \funcname{raise\_notrace})
  \end{itemize}
  \bigskip
  \centering
  \begin{tabular}{| l | c | c | c | c |}
    \hline
    \parbox{10em}{Debug information \\ (\funcarg{-g} flag)}  & \multicolumn{2}{|c|}{Disabled}                                     & \multicolumn{2}{|c|}{Enabled} \\
    \hline
    Raise kind                                               & \funcname{raise} & \funcname{raise\_notrace}                       & \funcname{raise} & \funcname{raise\_notrace} \\
    \hline
    Compilation                                              & \parbox{8em}{inline assembly\\ (optimization)} & inline assembly   & \funcname{caml\_raise\_exn} & inline assembly \\
    \hline
  \end{tabular}
\end{frame}


%
% Takeaway #5
%
\subsection*{Takeaway \#5}
\frameSubsectionTakeaway{}
\againframe<5>{takeaway}
}%
      \onslide*<8>{\providecommand\step{12}%
% Backtraces
%
\subsection{One advantage of raising with debugging information}
\frameSubsection{}{}

\begin{frame}{Backtraces}{Debugging information \& source code}
  \begin{columns}[c]
    \begin{column}{0.5\textwidth}
      \begin{itemize}
        \item Raising an exception is fun and games, but how do we know where the exception originated? $\rightarrow$ With the backtrace associated with the exception!
        \item In order to get a backtrace from the point where an exception was raised, it's necessary to compile with debugging information enabled (\funcarg{-g} flag for the compiler)
        \item Same example as in the `Nested handlers' section
      \end{itemize}
    \end{column}
    \begin{column}{0.5\textwidth}
      \listocaml[firstline=9,lastline=34]{../src/backtrace/backtrace.ml}{backtrace/backtrace.ml}
    \end{column}
  \end{columns}
\end{frame}

\begin{frame}{Backtraces}{CMM and disassembly of \funcname{definitely\_raise}}
  \begin{columns}[c]
    \begin{column}{0.5\textwidth}
      \minipage[c][0.66\textheight][s]{\columnwidth}
      \begin{itemize}
        \item<1-> An effect of compiling with debugging information (\funcarg{-g}): the CMM no longer uses \funcname{raise\_notrace} to raise the exception but \funcname{raise} instead
        \item<2-> Disassembly shows that CMM \funcname{raise} is actually a call to \funcname{caml\_raise\_exn}, a function in assembler provided by the runtime
      \end{itemize}
      \vfill
      \mintocaml[firstline=9,lastline=13,highlightlines=12]{../src/backtrace/backtrace.ml}
      \endminipage
    \end{column}
    \begin{column}{0.5\textwidth}
      \onslide*<1>{
        \mintcmm[highlightlines=5]{../src/backtrace/camlBacktrace\_\_definitely\_raise\_347.cmm}
      }
      \onslide*<2>{
        \mintobjdump[firstline=2,highlightlines=14]{../src/backtrace/camlBacktrace\_\_definitely\_raise\_347.objdump}
      }
    \end{column}
  \end{columns}
\end{frame}

\begin{frame}{Backtraces}{Introducing \funcname{caml\_raise\_exn}}
  \begin{columns}[c]
    \begin{column}{0.5\textwidth}
      \minipage[c][0.66\textheight][s]{\columnwidth}
      \onslide*<1>{
        \begin{itemize}
          \item Raising the exception is performed using \funcname{caml\_raise\_exn} which is
            \begin{itemize}
              \item An assembly function provided by the runtime
              \item Architecture specific
            \end{itemize}
          \item Let's see what it does differently from \funcname{raise\_notrace}\dots
          \item We are about to raise \typename{ExnC} with \funcname{caml\_raise\_exn} from \funcname{definitely\_raise}
        \end{itemize}
      }
      \onslide*<2>{
        \mintobjdump[firstline=2,highlightlines=14]{../src/backtrace/camlBacktrace\_\_definitely\_raise\_347.objdump}
      }
      \vfill
      \mintocaml[firstline=9,lastline=13,highlightlines=12]{../src/backtrace/backtrace.ml}
      \endminipage
    \end{column}
    \begin{column}{0.5\textwidth}
      \centering
      \providecommand\step{1}\input{diagrams/backtrace/backtrace.tex}
    \end{column}
  \end{columns}
\end{frame}

\begin{frame}{Backtraces}{But before that, we need more fields from \typename{caml\_domain\_state}}
  \begin{columns}
    \begin{column}{0.5\textwidth}
      \begin{itemize}
        \item Remember \typename{caml\_domain\_state} the per-domain runtime state?
        \item Some other members are of interest to us now\dots
        \item \localname{backtrace\_active} is used to know if the runtime should collect backtraces
        \item \localname{backtrace\_active} can be set by
          \begin{itemize}
            \item The \funcarg{b} flag in the \funcname{OCAMLRUNPARAM} environment variable
            \item \mintinline{ocaml}{Printexc.record_backtrace : bool -> unit} \footnotemark
          \end{itemize}
      \end{itemize}
    \end{column}
    \begin{column}{0.5\textwidth}
      \begin{listing}
        \mintc[highlightlines={9-11}]{src/ocaml/domain\_state\_backtrace.h}
        \caption{runtime/caml/domain\_state.h}
      \end{listing}
    \end{column}
  \end{columns}
  \footnotetext[1]{https://v2.ocaml.org/api/Printexc.html}
\end{frame}

\newcommand\listCamlRaiseExn[1][]{
  \listasm[firstline=702,lastline=725,#1]{../ocaml/runtime/amd64.S}{runtime/amd64.S:702}}

\begin{frame}{Backtraces}{Walk through \funcname{caml\_raise\_exn}}
  \begin{columns}[T]
    \begin{column}{0.5\textwidth}
      \begin{itemize}
        \item<1-> First checks whether exceptions backtrace should be recorded
        \item<1-> By checking the \funcarg{domain\_state->backtrace\_active} value
        \item<2-> If not, acts as \funcname{raise\_notrace} with \funcname{RESTORE\_EXN\_HANDLER\_OCAML}
          \begin{itemize}
            \item Set \regname{RSP} to \localname{domain\_state->exn\_handler} value
            \item Restore \localname{domain\_state->exn\_handler} from trap
            \item Jump with \funcname{ret} (same effect as using \funcname{pop} and \funcname{jmp})
          \end{itemize}
        \item<3-> Otherwise, switches to the C stack to be able to execute C code
        \item<4-> Calls \funcname{caml\_stash\_backtrace}, a C function provided by the runtime\ignorespaces
          \onslide<6->{, with
            \begin{itemize}
              \item \funcarg{exn}: exception value
              \item \funcarg{pc}: code address of the raising point
              \item \funcarg{sp}: stack address of the raising function
              \item \funcarg{trapsp}: stack address of the next handler
            \end{itemize}
          }
      \end{itemize}
      \bigskip
      \mintocaml[firstline=9,lastline=13,highlightlines=12]{../src/backtrace/backtrace.ml}
    \end{column}
    \begin{column}{0.5\textwidth}
      \raggedleft
      \onslide*<1,3-4>{
        \mintc[firstline=190,lastline=192]{../ocaml/runtime/amd64.S}
        \mintc[firstline=234,lastline=237]{../ocaml/runtime/amd64.S}
      }
      \onslide*<2>{
        \mintc[firstline=190,lastline=192,highlightlines={191-192}]{../ocaml/runtime/amd64.S}
        \mintc[firstline=234,lastline=237,highlightlines={235-237}]{../ocaml/runtime/amd64.S}
      }
      \onslide*<1>{\listCamlRaiseExn[highlightlines=705]{}}%
      \onslide*<2>{\listCamlRaiseExn[highlightlines={707-708}]{}}%
      \onslide*<3>{\listCamlRaiseExn[highlightlines={712-714}]{}}%
      \onslide*<4>{\listCamlRaiseExn[highlightlines={715-720}]{}}%
      \onslide*<5>{\providecommand\step{1}\input{diagrams/backtrace/backtrace.tex}}%
      \onslide*<6>{\providecommand\step{2}\input{diagrams/backtrace/backtrace.tex}}%
    \end{column}
  \end{columns}
\end{frame}

\newcommand\listStashBacktrace[1][]{
  \listc[firstline=91,lastline=120,#1]{../ocaml/runtime/backtrace\_nat.c}{runtime/backtrace\_nat.c:91}}

\begin{frame}{Backtraces}{Introducing \funcname{caml\_stash\_backtrace}}
  \begin{columns}[c]
    \begin{column}{0.5\textwidth}
      \minipage[c][0.66\textheight][s]{\columnwidth}
      \begin{itemize}
        \item<1-> A C function provided by the runtime (specific to native code)
        \item<2-> It scans the stack, between the stack pointer at the raising point up to the next trap, in order to collect the names of the detected function frames
          \onslide*<2>{
            \begin{itemize}
              \item And more information, like line number, column number...
            \end{itemize}
          }
        \item<3-> It uses the same technique and information as the Garbage Collector (GC) uses to scan the values live on the stack
          \onslide*<4>{
            \begin{itemize}
              \item \typename{frame\_descr} are at the core of stack unwinding
            \end{itemize}
          }
      \end{itemize}
      \vfill
      \mintocaml[firstline=9,lastline=13,highlightlines=12]{../src/backtrace/backtrace.ml}
      \endminipage
    \end{column}
    \begin{column}{0.5\textwidth}
      \onslide*<1>{\listStashBacktrace[highlightlines=91]{}}
      \onslide*<2>{\listStashBacktrace[highlightlines={91,117-118}]{}}
      \onslide*<3->{\listStashBacktrace[highlightlines={94,106,109-110}]{}}
    \end{column}
  \end{columns}
\end{frame}

\newcommand\listFrameDescr[1][]{
  \listocaml[firstline=111,lastline=124,#1]{../ocaml/asmcomp/emitaux.ml}{asmcomp/emitaux.ml:111}}
\newcommand\listDebugInfo[1][]{
  \listocaml[firstline=38,lastline=49,#1]{../ocaml/lambda/debuginfo.mli}{lambda/debuginfo.mli:38}}

\begin{frame}{Backtraces}{\typename{frame\_descr} and \typename{frame\_debuginfo} in a nutshell}
  \begin{columns}[c]
    \begin{column}{0.5\textwidth}
      \begin{itemize}
        \item<1-> For every return address in an OCaml program, there is a associated \typename{frame\_descr}
        \item<2-> A \typename{frame\_descr} specifies
        \begin{itemize}
          \item<2-> The size of the function stack frame
          \item<3-> Where live values are on the stack (so that the GC can mark them as live and prevent from being swept)
        \end{itemize}
        \item<4-> When compiled with debug information, \typename{frame\_descr} will be linked to a \typename{frame\_debuginfo}
        \begin{itemize}
          \item<5-> Contains information about the position in the source code (file name, line and column number)
        \end{itemize}
      \end{itemize}
    \end{column}
    \begin{column}{0.5\textwidth}
        \onslide*<1>{\listFrameDescr[highlightlines={118-122}]{}}
        \onslide*<2>{\listFrameDescr[highlightlines={120}]{}}
        \onslide*<3>{\listFrameDescr[highlightlines={121}]{}}
        \onslide*<4->{\listFrameDescr[highlightlines={122}]{}}
        \medskip
        \onslide*<1-3>{\listDebugInfo{}}
        \onslide*<4>{\listDebugInfo[highlightlines={38,49}]{}}
        \onslide*<5>{\listDebugInfo[highlightlines={39-46}]{}}
    \end{column}
  \end{columns}
\end{frame}

\begin{frame}{Backtraces}{Scanning the stack with \typename{frame\_descr}}
  \begin{columns}[c]
    \begin{column}{0.5\textwidth}
      \begin{itemize}
        \item Given the return address of the code that raises the exception by calling \funcname{caml\_raise\_exn} (\funcarg{pc} argument of \funcname{caml\_stash\_backtrace})
        \item And the position of the stack pointer (\funcarg{sp} argument of \funcname{caml\_stash\_backtrace})
        \item The frame size given by \typename{frame\_descr}.\funcarg{frame\_size}, allows to find the end of the previous frame
        \item And the return address of the caller of this function
        \bigskip
        \onslide<3->{
        \item Note that traps (here the reddish \funcname{ExnA}, \funcname{ExnB} and \funcname{ExnC} blocks) belong to the frame that installed them, and so the frame size changes when traps are removed
        }
      \end{itemize}
    \end{column}
    \begin{column}{0.5\textwidth}
      \raggedleft
      \onslide*<1>{\providecommand\step{2}\input{diagrams/backtrace/backtrace.tex}}%
      \onslide*<2>{\providecommand\step{1}\input{diagrams/backtrace/scanstack.tex}}%
      \onslide*<3>{\providecommand\step{2}\input{diagrams/backtrace/scanstack.tex}}%
      \onslide*<4>{\providecommand\step{3}\input{diagrams/backtrace/scanstack.tex}}%
      \onslide*<5>{\providecommand\step{4}\input{diagrams/backtrace/scanstack.tex}}%
      \onslide*<6>{\providecommand\step{5}\input{diagrams/backtrace/scanstack.tex}}%
    \end{column}
  \end{columns}
\end{frame}

% Duplicate of previous-previous slide
\begin{frame}{Backtraces}{Continuing on \funcname{caml\_stash\_backtrace}}
  \begin{columns}[c]
    \begin{column}{0.5\textwidth}
      \minipage[c][0.75\textheight][s]{\columnwidth}
      \begin{itemize}
          \color{gray}
        \item A C function provided by the runtime (specific to native code)
        \item It scans the stack, between the stack pointer at the raising point up to the next trap, in order to collect the names of the detected function frames
        \item It uses the same technique and information as the Garbage Collector (GC) uses to scan the values live on the stack
      \end{itemize}
      \begin{itemize}
        \item For each function frame detected, store a pointer to the \typename{frame\_descr} in the \localname{domain\_state->backtrace\_buffer} array, effectively constructing the backtrace
      \end{itemize}
      \onslide<2->{
        \begin{itemize}
            \smallskip
          \item Constructed backtrace:
            \funcname{definitely\_raise},
            \onslide<3->{\funcname{probably\_raise}}
        \end{itemize}
      }
      \vfill
      \mintocaml[firstline=9,lastline=13,highlightlines=12]{../src/backtrace/backtrace.ml}
      \endminipage
    \end{column}
    \begin{column}{0.5\textwidth}
      \raggedleft
      \onslide*<1>{\listStashBacktrace[highlightlines={114-115}]{}}
      \onslide*<2>{\providecommand\step{3}\input{diagrams/backtrace/backtrace.tex}}%
      \onslide*<3>{\providecommand\step{4}\input{diagrams/backtrace/backtrace.tex}}%
    \end{column}
  \end{columns}
\end{frame}

\begin{frame}{Backtraces}{Back to \funcname{caml\_raise\_exn}}
  \begin{columns}[c]
    \begin{column}{0.5\textwidth}
      \minipage[c][0.66\textheight][s]{\columnwidth}
      \begin{itemize}\color{gray}
        \item Checks wheteher exceptions backtrace should be recorded, by checking the \funcarg{domain\_state->backtrace\_active} value
        \item Switches to the C stack to be able to execute C code
        \item Calls \funcname{caml\_stash\_backtrace}, to collect the backtrace up to the next trap
      \end{itemize}
      \onslide*<2->{
        \begin{itemize}
          \item<2-> Executes the exception handler in \funcname{probably\_raise} with \funcname{RESTORE\_EXN\_HANDLER\_OCAML}
            \begin{itemize}
              \item Set \regname{RSP} to \localname{domain\_state->exn\_handler} value
              \item Restore \localname{domain\_state->exn\_handler} from trap
              \item Jump to the handler code with \funcname{ret}
            \end{itemize}
        \end{itemize}
      }
      \vfill
      \onslide*<1>{\mintocaml[firstline=9,lastline=13,highlightlines=12]{../src/backtrace/backtrace.ml}}
      \onslide*<2->{\mintocaml[firstline=15,lastline=21,highlightlines={19-21}]{../src/backtrace/backtrace.ml}}
      \endminipage
    \end{column}
    \begin{column}{0.5\textwidth}
      \centering
      \onslide*<1>{
        \mintc[firstline=190,lastline=192]{../ocaml/runtime/amd64.S}
        \mintc[firstline=234,lastline=237]{../ocaml/runtime/amd64.S}
      }
      \onslide*<2>{
        \mintc[firstline=190,lastline=192,highlightlines={191-192}]{../ocaml/runtime/amd64.S}
        \mintc[firstline=234,lastline=237,highlightlines={235-237}]{../ocaml/runtime/amd64.S}
      }
      \onslide*<1>{\listCamlRaiseExn[highlightlines=720]{}}
      \onslide*<2>{\listCamlRaiseExn[highlightlines=722]{}}
      \onslide*<3->{\providecommand\step{5}\input{diagrams/backtrace/backtrace.tex}}
    \end{column}
  \end{columns}
\end{frame}

\begin{frame}{Backtraces}{\funcname{caml\_probably\_raise}'s handler doesn't catch \typename{ExnC}}
  \begin{columns}[c]
    \begin{column}{0.45\textwidth}
      \minipage[c][0.75\textheight][s]{\columnwidth}
      \begin{itemize}
        \item<1-> \funcname{match \dots with}\footnotemark pattern matching can also match exceptions
        \item<1-> The CMM of \funcname{probably\_raise} shows that this \funcname{match \dots with} is implemented with a pattern matching and a \funcname{try \dots with}
        \item<1-> But this one only matches on \typename{ExnA} exception
        \item<2-> Since the pattern matching fails to catch the exception, \funcname{reraise} is called with our current \typename{ExnC} exception
        \item<3-> Disassembly shows that \funcname{reraise} is actually a call to \funcname{caml\_reraise\_exn}, which is also an assembly function provided by the runtime
      \end{itemize}
      \vfill
      \mintocaml[firstline=15,lastline=21,highlightlines={18-21}]{../src/backtrace/backtrace.ml}
      \endminipage
    \end{column}
    \begin{column}{0.55\textwidth}
      \centering
      \onslide*<1>{\mintcmm[highlightlines={10-18}]{../src/backtrace/camlBacktrace\_\_probably\_raise\_351.cmm}}
      \onslide*<2>{\listcmm[highlightlines=18]{../src/backtrace/camlBacktrace\_\_probably\_raise_351.cmm}{CMM of probably\_raise}}
      \onslide*<3>{\mintobjdump[firstline=5,lastline=35,highlightlines=31]{../src/backtrace/camlBacktrace\_\_probably\_raise_351.objdump}}
    \end{column}
  \end{columns}
  \only<1>{\footnotetext[1]{https://blog.janestreet.com/pattern-matching-and-exception-handling-unite/}}
\end{frame}

\newcommand\listCamlReraiseExn[1][]{
  \listasm[firstline=727,lastline=734,#1]{../ocaml/runtime/amd64.S}{runtime/amd64.S:727}}

\begin{frame}{Backtraces}{Introducing \funcname{caml\_reraise\_exn}}
  \begin{columns}[c]
    \begin{column}{0.5\textwidth}
      \minipage[c][0.66\textheight][s]{\columnwidth}
      \begin{itemize}
        \item<1-> The exception have to be reraised to the next handler
        \item<2-> First, checks if the backtrace should be collected with \funcarg{domain\_state->backtrace\_active}
        \only<3>{
        \begin{itemize}
            \item If not, behaves like a \funcname{raise\_notrace} with \funcname{RESTORE\_EXN\_HANDLER\_OCAML}
        \end{itemize}
        }
        \item<4-> Jumps into \funcname{caml\_raise\_exn}
        \item<5-> Calls \funcname{caml\_stash\_backtrace} again, to scan the next part of the stack: from the trap just pop-ed to the next trap
      \end{itemize}
      \vfill
      \mintocaml[firstline=15,lastline=21,highlightlines={18-21}]{../src/backtrace/backtrace.ml}
      \endminipage
    \end{column}
    \begin{column}{0.5\textwidth}
      \raggedleft
      \onslide*<1>{\listCamlReraiseExn{}}
      \onslide*<2>{\listCamlReraiseExn[highlightlines=729]{}}
      \onslide*<3>{
        \mintc[firstline=190,lastline=192,highlightlines={191-192}]{../ocaml/runtime/amd64.S}
        \mintc[firstline=234,lastline=237,highlightlines={235-237}]{../ocaml/runtime/amd64.S}
        \medskip
        \listCamlReraiseExn[highlightlines=731]{}
      }
      \onslide*<4-5>{
        % TODO refactor with \listCamlReraiseExn
        \mintasm[firstline=727,lastline=734,highlightlines=730]{../ocaml/runtime/amd64.S}
        \medskip
      }
      \onslide*<4>{\listCamlRaiseExn[highlightlines=711]{}}
      \onslide*<5>{\listCamlRaiseExn[highlightlines=720]{}}
    \end{column}
  \end{columns}
\end{frame}

\begin{frame}{Backtraces}{Backtrace collection continues}
  \begin{columns}[c]
    \begin{column}{0.55\textwidth}
      \minipage[c][0.75\textheight][s]{\columnwidth}
      \begin{itemize}
        \item<1-> \funcname{caml\_stash\_backtrace} scans the older part of the stack, up to the next trap
        \item<6-> Controls jumps to the nested exception handler in \funcname{maybe\_raise}, which matches only on \typename{ExnB}
        \item<7-> \funcname{caml\_reraise\_exn} is called again, backtrace collection resumes with \funcname{caml\_stash\_backtrace}
      \end{itemize}
      \medskip
      \begin{itemize}
        \item<2-> Constructed backtrace:
        \begin{itemize}
          \item <2-> \funcname{definitely\_raise}
          \item <2-> \funcname{probably\_raise}
          \item <3-> \funcname{probably\_raise} (re-raise)
          \item <4-> \funcname{maybe\_raise}
          \item <5-> \funcname{main}
          \item <8-> \funcname{main} (re-raise)
        \end{itemize}
      \end{itemize}
      \vfill
      \onslide*<1-5>{\mintocaml[firstline=15,lastline=21]{../src/backtrace/backtrace.ml}}
      \onslide*<6>{\mintocaml[firstline=28,lastline=34,highlightlines=31]{../src/backtrace/backtrace.ml}}
      \onslide*<7->{\mintocaml[firstline=28,lastline=34,highlightlines=33]{../src/backtrace/backtrace.ml}}
      \endminipage
    \end{column}
    \begin{column}{0.45\textwidth}
      \raggedleft
      \onslide*<1>{\providecommand\step{6}\input{diagrams/backtrace/backtrace.tex}}%
      \onslide*<2>{\providecommand\step{6}\input{diagrams/backtrace/backtrace.tex}}%
      \onslide*<3>{\providecommand\step{7}\input{diagrams/backtrace/backtrace.tex}}%
      \onslide*<4>{\providecommand\step{8}\input{diagrams/backtrace/backtrace.tex}}%
      \onslide*<5>{\providecommand\step{9}\input{diagrams/backtrace/backtrace.tex}}%
      \onslide*<6>{\providecommand\step{10}\input{diagrams/backtrace/backtrace.tex}}%
      \onslide*<7>{\providecommand\step{11}\input{diagrams/backtrace/backtrace.tex}}%
      \onslide*<8>{\providecommand\step{12}\input{diagrams/backtrace/backtrace.tex}}%
      \onslide*<9>{\providecommand\step{13}\input{diagrams/backtrace/backtrace.tex}}%
    \end{column}
  \end{columns}
\end{frame}

\begin{frame}{Backtraces}{Printing the backtrace}
  \begin{columns}[c]
    \begin{column}{0.45\textwidth}
      \minipage[c][0.66\textheight][s]{\columnwidth}
      \begin{itemize}
        \item<1-> The exception backtrace can now be printed to \funcarg{stdout} with \funcname{Printexc.print_backtrace}
        \item<2-> First the exception backtrace is retrieved into an OCaml array with \funcname{get_raw_backtrace }
        \item<3-> Which is implemented as the \funcname{caml_get_exception_raw_backtrace} C function in the runtime
        \item<4-> And finally printed with \funcname{print_exception_backtrace}
      \end{itemize}
      \vfill
      \mintocaml[firstline=28,lastline=34,highlightlines=33]{../src/backtrace/backtrace.ml}
      \endminipage
    \end{column}
    \begin{column}{0.55\textwidth}
      \onslide*<1,2>{
        \mintocaml[firstline=100,lastline=101]{../ocaml/stdlib/printexc.ml}
      }
      \only<1>{
        \listocaml[firstline=170,lastline=172]{../ocaml/stdlib/printexc.ml}{stdlib/printexc.ml:170}
      }
      \only<2>{
        \listocaml[firstline=170,lastline=172,highlightlines=172]{../ocaml/stdlib/printexc.ml}{stdlib/printexc.ml:170}
      }
      \only<3>{
        \mintc[firstline=154,lastline=156]{../ocaml/runtime/backtrace.c}
        \listc[firstline=171,lastline=192,highlightlines={184-187}]{../ocaml/runtime/backtrace.c}{runtime/backtrace.c:154}
      }
      \only<4>{
        \listocaml[firstline=155,lastline=165]{../ocaml/stdlib/printexc.ml}{stdlib/printexc.ml:55}
      }
    \end{column}
  \end{columns}
\end{frame}


\subsection{The multiple kinds of raise}
\frameSubsection{}{}

\begin{frame}{Backtraces}{When backtraces are not needed}
  \begin{columns}[c]
    \begin{column}{0.45\textwidth}
      \minipage[c][0.66\textheight][s]{\columnwidth}
      \begin{itemize}
        \item<1-> What if the backtrace for an exception is never needed?
        \item<2-> It's possible to raise the exception explicitly with \funcname{raise\_notrace} to make raising as cheap as possible
        \item<3-> An example of use in the \funcname{dynlink} library of the OCaml compiler
      \end{itemize}
      \vfill
      \onslide<3->{
      \listocaml[firstline=205,lastline=209,highlightlines=207]{../ocaml/otherlibs/dynlink/dynlink\_compilerlibs/misc.ml}{otherlibs/dynlink/dynlink\_compilerlibs/misc.ml:205}
      }
      \endminipage
    \end{column}
    \begin{column}{0.55\textwidth}
    \only<4>{
      \mintcmm[highlightlines=3]{../src/notrace/camlNotrace\_\_fun_347.cmm}
      \listcmm{../src/notrace/camlNotrace\_\_all\_somes\_327.cmm}{all\_somes}
    }
    \onslide*<5->{
      \listobjdump[highlightlines={6-9}]{../src/notrace/camlNotrace\_\_fun_347.objdump}{anonymous function from \funcname{all\_somes}}
    }
    \end{column}
  \end{columns}
\end{frame}

\begin{frame}{Backtraces}{Summary table of the multiple kinds of raise}
  \begin{itemize}
    \item When debugging information is enabled and if the backtrace of an exception isn't needed, it's possible to raise with \funcname{raise\_notrace} for performance
    \item When debugging information is \emph{not} enabled, the compiler optimises \funcname{raise} calls to inline assembly (same effect as using \funcname{raise\_notrace})
  \end{itemize}
  \bigskip
  \centering
  \begin{tabular}{| l | c | c | c | c |}
    \hline
    \parbox{10em}{Debug information \\ (\funcarg{-g} flag)}  & \multicolumn{2}{|c|}{Disabled}                                     & \multicolumn{2}{|c|}{Enabled} \\
    \hline
    Raise kind                                               & \funcname{raise} & \funcname{raise\_notrace}                       & \funcname{raise} & \funcname{raise\_notrace} \\
    \hline
    Compilation                                              & \parbox{8em}{inline assembly\\ (optimization)} & inline assembly   & \funcname{caml\_raise\_exn} & inline assembly \\
    \hline
  \end{tabular}
\end{frame}


%
% Takeaway #5
%
\subsection*{Takeaway \#5}
\frameSubsectionTakeaway{}
\againframe<5>{takeaway}
}%
      \onslide*<9>{\providecommand\step{13}%
% Backtraces
%
\subsection{One advantage of raising with debugging information}
\frameSubsection{}{}

\begin{frame}{Backtraces}{Debugging information \& source code}
  \begin{columns}[c]
    \begin{column}{0.5\textwidth}
      \begin{itemize}
        \item Raising an exception is fun and games, but how do we know where the exception originated? $\rightarrow$ With the backtrace associated with the exception!
        \item In order to get a backtrace from the point where an exception was raised, it's necessary to compile with debugging information enabled (\funcarg{-g} flag for the compiler)
        \item Same example as in the `Nested handlers' section
      \end{itemize}
    \end{column}
    \begin{column}{0.5\textwidth}
      \listocaml[firstline=9,lastline=34]{../src/backtrace/backtrace.ml}{backtrace/backtrace.ml}
    \end{column}
  \end{columns}
\end{frame}

\begin{frame}{Backtraces}{CMM and disassembly of \funcname{definitely\_raise}}
  \begin{columns}[c]
    \begin{column}{0.5\textwidth}
      \minipage[c][0.66\textheight][s]{\columnwidth}
      \begin{itemize}
        \item<1-> An effect of compiling with debugging information (\funcarg{-g}): the CMM no longer uses \funcname{raise\_notrace} to raise the exception but \funcname{raise} instead
        \item<2-> Disassembly shows that CMM \funcname{raise} is actually a call to \funcname{caml\_raise\_exn}, a function in assembler provided by the runtime
      \end{itemize}
      \vfill
      \mintocaml[firstline=9,lastline=13,highlightlines=12]{../src/backtrace/backtrace.ml}
      \endminipage
    \end{column}
    \begin{column}{0.5\textwidth}
      \onslide*<1>{
        \mintcmm[highlightlines=5]{../src/backtrace/camlBacktrace\_\_definitely\_raise\_347.cmm}
      }
      \onslide*<2>{
        \mintobjdump[firstline=2,highlightlines=14]{../src/backtrace/camlBacktrace\_\_definitely\_raise\_347.objdump}
      }
    \end{column}
  \end{columns}
\end{frame}

\begin{frame}{Backtraces}{Introducing \funcname{caml\_raise\_exn}}
  \begin{columns}[c]
    \begin{column}{0.5\textwidth}
      \minipage[c][0.66\textheight][s]{\columnwidth}
      \onslide*<1>{
        \begin{itemize}
          \item Raising the exception is performed using \funcname{caml\_raise\_exn} which is
            \begin{itemize}
              \item An assembly function provided by the runtime
              \item Architecture specific
            \end{itemize}
          \item Let's see what it does differently from \funcname{raise\_notrace}\dots
          \item We are about to raise \typename{ExnC} with \funcname{caml\_raise\_exn} from \funcname{definitely\_raise}
        \end{itemize}
      }
      \onslide*<2>{
        \mintobjdump[firstline=2,highlightlines=14]{../src/backtrace/camlBacktrace\_\_definitely\_raise\_347.objdump}
      }
      \vfill
      \mintocaml[firstline=9,lastline=13,highlightlines=12]{../src/backtrace/backtrace.ml}
      \endminipage
    \end{column}
    \begin{column}{0.5\textwidth}
      \centering
      \providecommand\step{1}\input{diagrams/backtrace/backtrace.tex}
    \end{column}
  \end{columns}
\end{frame}

\begin{frame}{Backtraces}{But before that, we need more fields from \typename{caml\_domain\_state}}
  \begin{columns}
    \begin{column}{0.5\textwidth}
      \begin{itemize}
        \item Remember \typename{caml\_domain\_state} the per-domain runtime state?
        \item Some other members are of interest to us now\dots
        \item \localname{backtrace\_active} is used to know if the runtime should collect backtraces
        \item \localname{backtrace\_active} can be set by
          \begin{itemize}
            \item The \funcarg{b} flag in the \funcname{OCAMLRUNPARAM} environment variable
            \item \mintinline{ocaml}{Printexc.record_backtrace : bool -> unit} \footnotemark
          \end{itemize}
      \end{itemize}
    \end{column}
    \begin{column}{0.5\textwidth}
      \begin{listing}
        \mintc[highlightlines={9-11}]{src/ocaml/domain\_state\_backtrace.h}
        \caption{runtime/caml/domain\_state.h}
      \end{listing}
    \end{column}
  \end{columns}
  \footnotetext[1]{https://v2.ocaml.org/api/Printexc.html}
\end{frame}

\newcommand\listCamlRaiseExn[1][]{
  \listasm[firstline=702,lastline=725,#1]{../ocaml/runtime/amd64.S}{runtime/amd64.S:702}}

\begin{frame}{Backtraces}{Walk through \funcname{caml\_raise\_exn}}
  \begin{columns}[T]
    \begin{column}{0.5\textwidth}
      \begin{itemize}
        \item<1-> First checks whether exceptions backtrace should be recorded
        \item<1-> By checking the \funcarg{domain\_state->backtrace\_active} value
        \item<2-> If not, acts as \funcname{raise\_notrace} with \funcname{RESTORE\_EXN\_HANDLER\_OCAML}
          \begin{itemize}
            \item Set \regname{RSP} to \localname{domain\_state->exn\_handler} value
            \item Restore \localname{domain\_state->exn\_handler} from trap
            \item Jump with \funcname{ret} (same effect as using \funcname{pop} and \funcname{jmp})
          \end{itemize}
        \item<3-> Otherwise, switches to the C stack to be able to execute C code
        \item<4-> Calls \funcname{caml\_stash\_backtrace}, a C function provided by the runtime\ignorespaces
          \onslide<6->{, with
            \begin{itemize}
              \item \funcarg{exn}: exception value
              \item \funcarg{pc}: code address of the raising point
              \item \funcarg{sp}: stack address of the raising function
              \item \funcarg{trapsp}: stack address of the next handler
            \end{itemize}
          }
      \end{itemize}
      \bigskip
      \mintocaml[firstline=9,lastline=13,highlightlines=12]{../src/backtrace/backtrace.ml}
    \end{column}
    \begin{column}{0.5\textwidth}
      \raggedleft
      \onslide*<1,3-4>{
        \mintc[firstline=190,lastline=192]{../ocaml/runtime/amd64.S}
        \mintc[firstline=234,lastline=237]{../ocaml/runtime/amd64.S}
      }
      \onslide*<2>{
        \mintc[firstline=190,lastline=192,highlightlines={191-192}]{../ocaml/runtime/amd64.S}
        \mintc[firstline=234,lastline=237,highlightlines={235-237}]{../ocaml/runtime/amd64.S}
      }
      \onslide*<1>{\listCamlRaiseExn[highlightlines=705]{}}%
      \onslide*<2>{\listCamlRaiseExn[highlightlines={707-708}]{}}%
      \onslide*<3>{\listCamlRaiseExn[highlightlines={712-714}]{}}%
      \onslide*<4>{\listCamlRaiseExn[highlightlines={715-720}]{}}%
      \onslide*<5>{\providecommand\step{1}\input{diagrams/backtrace/backtrace.tex}}%
      \onslide*<6>{\providecommand\step{2}\input{diagrams/backtrace/backtrace.tex}}%
    \end{column}
  \end{columns}
\end{frame}

\newcommand\listStashBacktrace[1][]{
  \listc[firstline=91,lastline=120,#1]{../ocaml/runtime/backtrace\_nat.c}{runtime/backtrace\_nat.c:91}}

\begin{frame}{Backtraces}{Introducing \funcname{caml\_stash\_backtrace}}
  \begin{columns}[c]
    \begin{column}{0.5\textwidth}
      \minipage[c][0.66\textheight][s]{\columnwidth}
      \begin{itemize}
        \item<1-> A C function provided by the runtime (specific to native code)
        \item<2-> It scans the stack, between the stack pointer at the raising point up to the next trap, in order to collect the names of the detected function frames
          \onslide*<2>{
            \begin{itemize}
              \item And more information, like line number, column number...
            \end{itemize}
          }
        \item<3-> It uses the same technique and information as the Garbage Collector (GC) uses to scan the values live on the stack
          \onslide*<4>{
            \begin{itemize}
              \item \typename{frame\_descr} are at the core of stack unwinding
            \end{itemize}
          }
      \end{itemize}
      \vfill
      \mintocaml[firstline=9,lastline=13,highlightlines=12]{../src/backtrace/backtrace.ml}
      \endminipage
    \end{column}
    \begin{column}{0.5\textwidth}
      \onslide*<1>{\listStashBacktrace[highlightlines=91]{}}
      \onslide*<2>{\listStashBacktrace[highlightlines={91,117-118}]{}}
      \onslide*<3->{\listStashBacktrace[highlightlines={94,106,109-110}]{}}
    \end{column}
  \end{columns}
\end{frame}

\newcommand\listFrameDescr[1][]{
  \listocaml[firstline=111,lastline=124,#1]{../ocaml/asmcomp/emitaux.ml}{asmcomp/emitaux.ml:111}}
\newcommand\listDebugInfo[1][]{
  \listocaml[firstline=38,lastline=49,#1]{../ocaml/lambda/debuginfo.mli}{lambda/debuginfo.mli:38}}

\begin{frame}{Backtraces}{\typename{frame\_descr} and \typename{frame\_debuginfo} in a nutshell}
  \begin{columns}[c]
    \begin{column}{0.5\textwidth}
      \begin{itemize}
        \item<1-> For every return address in an OCaml program, there is a associated \typename{frame\_descr}
        \item<2-> A \typename{frame\_descr} specifies
        \begin{itemize}
          \item<2-> The size of the function stack frame
          \item<3-> Where live values are on the stack (so that the GC can mark them as live and prevent from being swept)
        \end{itemize}
        \item<4-> When compiled with debug information, \typename{frame\_descr} will be linked to a \typename{frame\_debuginfo}
        \begin{itemize}
          \item<5-> Contains information about the position in the source code (file name, line and column number)
        \end{itemize}
      \end{itemize}
    \end{column}
    \begin{column}{0.5\textwidth}
        \onslide*<1>{\listFrameDescr[highlightlines={118-122}]{}}
        \onslide*<2>{\listFrameDescr[highlightlines={120}]{}}
        \onslide*<3>{\listFrameDescr[highlightlines={121}]{}}
        \onslide*<4->{\listFrameDescr[highlightlines={122}]{}}
        \medskip
        \onslide*<1-3>{\listDebugInfo{}}
        \onslide*<4>{\listDebugInfo[highlightlines={38,49}]{}}
        \onslide*<5>{\listDebugInfo[highlightlines={39-46}]{}}
    \end{column}
  \end{columns}
\end{frame}

\begin{frame}{Backtraces}{Scanning the stack with \typename{frame\_descr}}
  \begin{columns}[c]
    \begin{column}{0.5\textwidth}
      \begin{itemize}
        \item Given the return address of the code that raises the exception by calling \funcname{caml\_raise\_exn} (\funcarg{pc} argument of \funcname{caml\_stash\_backtrace})
        \item And the position of the stack pointer (\funcarg{sp} argument of \funcname{caml\_stash\_backtrace})
        \item The frame size given by \typename{frame\_descr}.\funcarg{frame\_size}, allows to find the end of the previous frame
        \item And the return address of the caller of this function
        \bigskip
        \onslide<3->{
        \item Note that traps (here the reddish \funcname{ExnA}, \funcname{ExnB} and \funcname{ExnC} blocks) belong to the frame that installed them, and so the frame size changes when traps are removed
        }
      \end{itemize}
    \end{column}
    \begin{column}{0.5\textwidth}
      \raggedleft
      \onslide*<1>{\providecommand\step{2}\input{diagrams/backtrace/backtrace.tex}}%
      \onslide*<2>{\providecommand\step{1}\input{diagrams/backtrace/scanstack.tex}}%
      \onslide*<3>{\providecommand\step{2}\input{diagrams/backtrace/scanstack.tex}}%
      \onslide*<4>{\providecommand\step{3}\input{diagrams/backtrace/scanstack.tex}}%
      \onslide*<5>{\providecommand\step{4}\input{diagrams/backtrace/scanstack.tex}}%
      \onslide*<6>{\providecommand\step{5}\input{diagrams/backtrace/scanstack.tex}}%
    \end{column}
  \end{columns}
\end{frame}

% Duplicate of previous-previous slide
\begin{frame}{Backtraces}{Continuing on \funcname{caml\_stash\_backtrace}}
  \begin{columns}[c]
    \begin{column}{0.5\textwidth}
      \minipage[c][0.75\textheight][s]{\columnwidth}
      \begin{itemize}
          \color{gray}
        \item A C function provided by the runtime (specific to native code)
        \item It scans the stack, between the stack pointer at the raising point up to the next trap, in order to collect the names of the detected function frames
        \item It uses the same technique and information as the Garbage Collector (GC) uses to scan the values live on the stack
      \end{itemize}
      \begin{itemize}
        \item For each function frame detected, store a pointer to the \typename{frame\_descr} in the \localname{domain\_state->backtrace\_buffer} array, effectively constructing the backtrace
      \end{itemize}
      \onslide<2->{
        \begin{itemize}
            \smallskip
          \item Constructed backtrace:
            \funcname{definitely\_raise},
            \onslide<3->{\funcname{probably\_raise}}
        \end{itemize}
      }
      \vfill
      \mintocaml[firstline=9,lastline=13,highlightlines=12]{../src/backtrace/backtrace.ml}
      \endminipage
    \end{column}
    \begin{column}{0.5\textwidth}
      \raggedleft
      \onslide*<1>{\listStashBacktrace[highlightlines={114-115}]{}}
      \onslide*<2>{\providecommand\step{3}\input{diagrams/backtrace/backtrace.tex}}%
      \onslide*<3>{\providecommand\step{4}\input{diagrams/backtrace/backtrace.tex}}%
    \end{column}
  \end{columns}
\end{frame}

\begin{frame}{Backtraces}{Back to \funcname{caml\_raise\_exn}}
  \begin{columns}[c]
    \begin{column}{0.5\textwidth}
      \minipage[c][0.66\textheight][s]{\columnwidth}
      \begin{itemize}\color{gray}
        \item Checks wheteher exceptions backtrace should be recorded, by checking the \funcarg{domain\_state->backtrace\_active} value
        \item Switches to the C stack to be able to execute C code
        \item Calls \funcname{caml\_stash\_backtrace}, to collect the backtrace up to the next trap
      \end{itemize}
      \onslide*<2->{
        \begin{itemize}
          \item<2-> Executes the exception handler in \funcname{probably\_raise} with \funcname{RESTORE\_EXN\_HANDLER\_OCAML}
            \begin{itemize}
              \item Set \regname{RSP} to \localname{domain\_state->exn\_handler} value
              \item Restore \localname{domain\_state->exn\_handler} from trap
              \item Jump to the handler code with \funcname{ret}
            \end{itemize}
        \end{itemize}
      }
      \vfill
      \onslide*<1>{\mintocaml[firstline=9,lastline=13,highlightlines=12]{../src/backtrace/backtrace.ml}}
      \onslide*<2->{\mintocaml[firstline=15,lastline=21,highlightlines={19-21}]{../src/backtrace/backtrace.ml}}
      \endminipage
    \end{column}
    \begin{column}{0.5\textwidth}
      \centering
      \onslide*<1>{
        \mintc[firstline=190,lastline=192]{../ocaml/runtime/amd64.S}
        \mintc[firstline=234,lastline=237]{../ocaml/runtime/amd64.S}
      }
      \onslide*<2>{
        \mintc[firstline=190,lastline=192,highlightlines={191-192}]{../ocaml/runtime/amd64.S}
        \mintc[firstline=234,lastline=237,highlightlines={235-237}]{../ocaml/runtime/amd64.S}
      }
      \onslide*<1>{\listCamlRaiseExn[highlightlines=720]{}}
      \onslide*<2>{\listCamlRaiseExn[highlightlines=722]{}}
      \onslide*<3->{\providecommand\step{5}\input{diagrams/backtrace/backtrace.tex}}
    \end{column}
  \end{columns}
\end{frame}

\begin{frame}{Backtraces}{\funcname{caml\_probably\_raise}'s handler doesn't catch \typename{ExnC}}
  \begin{columns}[c]
    \begin{column}{0.45\textwidth}
      \minipage[c][0.75\textheight][s]{\columnwidth}
      \begin{itemize}
        \item<1-> \funcname{match \dots with}\footnotemark pattern matching can also match exceptions
        \item<1-> The CMM of \funcname{probably\_raise} shows that this \funcname{match \dots with} is implemented with a pattern matching and a \funcname{try \dots with}
        \item<1-> But this one only matches on \typename{ExnA} exception
        \item<2-> Since the pattern matching fails to catch the exception, \funcname{reraise} is called with our current \typename{ExnC} exception
        \item<3-> Disassembly shows that \funcname{reraise} is actually a call to \funcname{caml\_reraise\_exn}, which is also an assembly function provided by the runtime
      \end{itemize}
      \vfill
      \mintocaml[firstline=15,lastline=21,highlightlines={18-21}]{../src/backtrace/backtrace.ml}
      \endminipage
    \end{column}
    \begin{column}{0.55\textwidth}
      \centering
      \onslide*<1>{\mintcmm[highlightlines={10-18}]{../src/backtrace/camlBacktrace\_\_probably\_raise\_351.cmm}}
      \onslide*<2>{\listcmm[highlightlines=18]{../src/backtrace/camlBacktrace\_\_probably\_raise_351.cmm}{CMM of probably\_raise}}
      \onslide*<3>{\mintobjdump[firstline=5,lastline=35,highlightlines=31]{../src/backtrace/camlBacktrace\_\_probably\_raise_351.objdump}}
    \end{column}
  \end{columns}
  \only<1>{\footnotetext[1]{https://blog.janestreet.com/pattern-matching-and-exception-handling-unite/}}
\end{frame}

\newcommand\listCamlReraiseExn[1][]{
  \listasm[firstline=727,lastline=734,#1]{../ocaml/runtime/amd64.S}{runtime/amd64.S:727}}

\begin{frame}{Backtraces}{Introducing \funcname{caml\_reraise\_exn}}
  \begin{columns}[c]
    \begin{column}{0.5\textwidth}
      \minipage[c][0.66\textheight][s]{\columnwidth}
      \begin{itemize}
        \item<1-> The exception have to be reraised to the next handler
        \item<2-> First, checks if the backtrace should be collected with \funcarg{domain\_state->backtrace\_active}
        \only<3>{
        \begin{itemize}
            \item If not, behaves like a \funcname{raise\_notrace} with \funcname{RESTORE\_EXN\_HANDLER\_OCAML}
        \end{itemize}
        }
        \item<4-> Jumps into \funcname{caml\_raise\_exn}
        \item<5-> Calls \funcname{caml\_stash\_backtrace} again, to scan the next part of the stack: from the trap just pop-ed to the next trap
      \end{itemize}
      \vfill
      \mintocaml[firstline=15,lastline=21,highlightlines={18-21}]{../src/backtrace/backtrace.ml}
      \endminipage
    \end{column}
    \begin{column}{0.5\textwidth}
      \raggedleft
      \onslide*<1>{\listCamlReraiseExn{}}
      \onslide*<2>{\listCamlReraiseExn[highlightlines=729]{}}
      \onslide*<3>{
        \mintc[firstline=190,lastline=192,highlightlines={191-192}]{../ocaml/runtime/amd64.S}
        \mintc[firstline=234,lastline=237,highlightlines={235-237}]{../ocaml/runtime/amd64.S}
        \medskip
        \listCamlReraiseExn[highlightlines=731]{}
      }
      \onslide*<4-5>{
        % TODO refactor with \listCamlReraiseExn
        \mintasm[firstline=727,lastline=734,highlightlines=730]{../ocaml/runtime/amd64.S}
        \medskip
      }
      \onslide*<4>{\listCamlRaiseExn[highlightlines=711]{}}
      \onslide*<5>{\listCamlRaiseExn[highlightlines=720]{}}
    \end{column}
  \end{columns}
\end{frame}

\begin{frame}{Backtraces}{Backtrace collection continues}
  \begin{columns}[c]
    \begin{column}{0.55\textwidth}
      \minipage[c][0.75\textheight][s]{\columnwidth}
      \begin{itemize}
        \item<1-> \funcname{caml\_stash\_backtrace} scans the older part of the stack, up to the next trap
        \item<6-> Controls jumps to the nested exception handler in \funcname{maybe\_raise}, which matches only on \typename{ExnB}
        \item<7-> \funcname{caml\_reraise\_exn} is called again, backtrace collection resumes with \funcname{caml\_stash\_backtrace}
      \end{itemize}
      \medskip
      \begin{itemize}
        \item<2-> Constructed backtrace:
        \begin{itemize}
          \item <2-> \funcname{definitely\_raise}
          \item <2-> \funcname{probably\_raise}
          \item <3-> \funcname{probably\_raise} (re-raise)
          \item <4-> \funcname{maybe\_raise}
          \item <5-> \funcname{main}
          \item <8-> \funcname{main} (re-raise)
        \end{itemize}
      \end{itemize}
      \vfill
      \onslide*<1-5>{\mintocaml[firstline=15,lastline=21]{../src/backtrace/backtrace.ml}}
      \onslide*<6>{\mintocaml[firstline=28,lastline=34,highlightlines=31]{../src/backtrace/backtrace.ml}}
      \onslide*<7->{\mintocaml[firstline=28,lastline=34,highlightlines=33]{../src/backtrace/backtrace.ml}}
      \endminipage
    \end{column}
    \begin{column}{0.45\textwidth}
      \raggedleft
      \onslide*<1>{\providecommand\step{6}\input{diagrams/backtrace/backtrace.tex}}%
      \onslide*<2>{\providecommand\step{6}\input{diagrams/backtrace/backtrace.tex}}%
      \onslide*<3>{\providecommand\step{7}\input{diagrams/backtrace/backtrace.tex}}%
      \onslide*<4>{\providecommand\step{8}\input{diagrams/backtrace/backtrace.tex}}%
      \onslide*<5>{\providecommand\step{9}\input{diagrams/backtrace/backtrace.tex}}%
      \onslide*<6>{\providecommand\step{10}\input{diagrams/backtrace/backtrace.tex}}%
      \onslide*<7>{\providecommand\step{11}\input{diagrams/backtrace/backtrace.tex}}%
      \onslide*<8>{\providecommand\step{12}\input{diagrams/backtrace/backtrace.tex}}%
      \onslide*<9>{\providecommand\step{13}\input{diagrams/backtrace/backtrace.tex}}%
    \end{column}
  \end{columns}
\end{frame}

\begin{frame}{Backtraces}{Printing the backtrace}
  \begin{columns}[c]
    \begin{column}{0.45\textwidth}
      \minipage[c][0.66\textheight][s]{\columnwidth}
      \begin{itemize}
        \item<1-> The exception backtrace can now be printed to \funcarg{stdout} with \funcname{Printexc.print_backtrace}
        \item<2-> First the exception backtrace is retrieved into an OCaml array with \funcname{get_raw_backtrace }
        \item<3-> Which is implemented as the \funcname{caml_get_exception_raw_backtrace} C function in the runtime
        \item<4-> And finally printed with \funcname{print_exception_backtrace}
      \end{itemize}
      \vfill
      \mintocaml[firstline=28,lastline=34,highlightlines=33]{../src/backtrace/backtrace.ml}
      \endminipage
    \end{column}
    \begin{column}{0.55\textwidth}
      \onslide*<1,2>{
        \mintocaml[firstline=100,lastline=101]{../ocaml/stdlib/printexc.ml}
      }
      \only<1>{
        \listocaml[firstline=170,lastline=172]{../ocaml/stdlib/printexc.ml}{stdlib/printexc.ml:170}
      }
      \only<2>{
        \listocaml[firstline=170,lastline=172,highlightlines=172]{../ocaml/stdlib/printexc.ml}{stdlib/printexc.ml:170}
      }
      \only<3>{
        \mintc[firstline=154,lastline=156]{../ocaml/runtime/backtrace.c}
        \listc[firstline=171,lastline=192,highlightlines={184-187}]{../ocaml/runtime/backtrace.c}{runtime/backtrace.c:154}
      }
      \only<4>{
        \listocaml[firstline=155,lastline=165]{../ocaml/stdlib/printexc.ml}{stdlib/printexc.ml:55}
      }
    \end{column}
  \end{columns}
\end{frame}


\subsection{The multiple kinds of raise}
\frameSubsection{}{}

\begin{frame}{Backtraces}{When backtraces are not needed}
  \begin{columns}[c]
    \begin{column}{0.45\textwidth}
      \minipage[c][0.66\textheight][s]{\columnwidth}
      \begin{itemize}
        \item<1-> What if the backtrace for an exception is never needed?
        \item<2-> It's possible to raise the exception explicitly with \funcname{raise\_notrace} to make raising as cheap as possible
        \item<3-> An example of use in the \funcname{dynlink} library of the OCaml compiler
      \end{itemize}
      \vfill
      \onslide<3->{
      \listocaml[firstline=205,lastline=209,highlightlines=207]{../ocaml/otherlibs/dynlink/dynlink\_compilerlibs/misc.ml}{otherlibs/dynlink/dynlink\_compilerlibs/misc.ml:205}
      }
      \endminipage
    \end{column}
    \begin{column}{0.55\textwidth}
    \only<4>{
      \mintcmm[highlightlines=3]{../src/notrace/camlNotrace\_\_fun_347.cmm}
      \listcmm{../src/notrace/camlNotrace\_\_all\_somes\_327.cmm}{all\_somes}
    }
    \onslide*<5->{
      \listobjdump[highlightlines={6-9}]{../src/notrace/camlNotrace\_\_fun_347.objdump}{anonymous function from \funcname{all\_somes}}
    }
    \end{column}
  \end{columns}
\end{frame}

\begin{frame}{Backtraces}{Summary table of the multiple kinds of raise}
  \begin{itemize}
    \item When debugging information is enabled and if the backtrace of an exception isn't needed, it's possible to raise with \funcname{raise\_notrace} for performance
    \item When debugging information is \emph{not} enabled, the compiler optimises \funcname{raise} calls to inline assembly (same effect as using \funcname{raise\_notrace})
  \end{itemize}
  \bigskip
  \centering
  \begin{tabular}{| l | c | c | c | c |}
    \hline
    \parbox{10em}{Debug information \\ (\funcarg{-g} flag)}  & \multicolumn{2}{|c|}{Disabled}                                     & \multicolumn{2}{|c|}{Enabled} \\
    \hline
    Raise kind                                               & \funcname{raise} & \funcname{raise\_notrace}                       & \funcname{raise} & \funcname{raise\_notrace} \\
    \hline
    Compilation                                              & \parbox{8em}{inline assembly\\ (optimization)} & inline assembly   & \funcname{caml\_raise\_exn} & inline assembly \\
    \hline
  \end{tabular}
\end{frame}


%
% Takeaway #5
%
\subsection*{Takeaway \#5}
\frameSubsectionTakeaway{}
\againframe<5>{takeaway}
}%
    \end{column}
  \end{columns}
\end{frame}

\begin{frame}{Backtraces}{Printing the backtrace}
  \begin{columns}[c]
    \begin{column}{0.45\textwidth}
      \minipage[c][0.66\textheight][s]{\columnwidth}
      \begin{itemize}
        \item<1-> The exception backtrace can now be printed to \funcarg{stdout} with \funcname{Printexc.print_backtrace}
        \item<2-> First the exception backtrace is retrieved into an OCaml array with \funcname{get_raw_backtrace }
        \item<3-> Which is implemented as the \funcname{caml_get_exception_raw_backtrace} C function in the runtime
        \item<4-> And finally printed with \funcname{print_exception_backtrace}
      \end{itemize}
      \vfill
      \mintocaml[firstline=28,lastline=34,highlightlines=33]{../src/backtrace/backtrace.ml}
      \endminipage
    \end{column}
    \begin{column}{0.55\textwidth}
      \onslide*<1,2>{
        \mintocaml[firstline=100,lastline=101]{../ocaml/stdlib/printexc.ml}
      }
      \only<1>{
        \listocaml[firstline=170,lastline=172]{../ocaml/stdlib/printexc.ml}{stdlib/printexc.ml:170}
      }
      \only<2>{
        \listocaml[firstline=170,lastline=172,highlightlines=172]{../ocaml/stdlib/printexc.ml}{stdlib/printexc.ml:170}
      }
      \only<3>{
        \mintc[firstline=154,lastline=156]{../ocaml/runtime/backtrace.c}
        \listc[firstline=171,lastline=192,highlightlines={184-187}]{../ocaml/runtime/backtrace.c}{runtime/backtrace.c:154}
      }
      \only<4>{
        \listocaml[firstline=155,lastline=165]{../ocaml/stdlib/printexc.ml}{stdlib/printexc.ml:55}
      }
    \end{column}
  \end{columns}
\end{frame}


\subsection{The multiple kinds of raise}
\frameSubsection{}{}

\begin{frame}{Backtraces}{When backtraces are not needed}
  \begin{columns}[c]
    \begin{column}{0.45\textwidth}
      \minipage[c][0.66\textheight][s]{\columnwidth}
      \begin{itemize}
        \item<1-> What if the backtrace for an exception is never needed?
        \item<2-> It's possible to raise the exception explicitly with \funcname{raise\_notrace} to make raising as cheap as possible
        \item<3-> An example of use in the \funcname{dynlink} library of the OCaml compiler
      \end{itemize}
      \vfill
      \onslide<3->{
      \listocaml[firstline=205,lastline=209,highlightlines=207]{../ocaml/otherlibs/dynlink/dynlink\_compilerlibs/misc.ml}{otherlibs/dynlink/dynlink\_compilerlibs/misc.ml:205}
      }
      \endminipage
    \end{column}
    \begin{column}{0.55\textwidth}
    \only<4>{
      \mintcmm[highlightlines=3]{../src/notrace/camlNotrace\_\_fun_347.cmm}
      \listcmm{../src/notrace/camlNotrace\_\_all\_somes\_327.cmm}{all\_somes}
    }
    \onslide*<5->{
      \listobjdump[highlightlines={6-9}]{../src/notrace/camlNotrace\_\_fun_347.objdump}{anonymous function from \funcname{all\_somes}}
    }
    \end{column}
  \end{columns}
\end{frame}

\begin{frame}{Backtraces}{Summary table of the multiple kinds of raise}
  \begin{itemize}
    \item When debugging information is enabled and if the backtrace of an exception isn't needed, it's possible to raise with \funcname{raise\_notrace} for performance
    \item When debugging information is \emph{not} enabled, the compiler optimises \funcname{raise} calls to inline assembly (same effect as using \funcname{raise\_notrace})
  \end{itemize}
  \bigskip
  \centering
  \begin{tabular}{| l | c | c | c | c |}
    \hline
    \parbox{10em}{Debug information \\ (\funcarg{-g} flag)}  & \multicolumn{2}{|c|}{Disabled}                                     & \multicolumn{2}{|c|}{Enabled} \\
    \hline
    Raise kind                                               & \funcname{raise} & \funcname{raise\_notrace}                       & \funcname{raise} & \funcname{raise\_notrace} \\
    \hline
    Compilation                                              & \parbox{8em}{inline assembly\\ (optimization)} & inline assembly   & \funcname{caml\_raise\_exn} & inline assembly \\
    \hline
  \end{tabular}
\end{frame}


%
% Takeaway #5
%
\subsection*{Takeaway \#5}
\frameSubsectionTakeaway{}
\againframe<5>{takeaway}
}%
      \onslide*<3>{\providecommand\step{7}%
% Backtraces
%
\subsection{One advantage of raising with debugging information}
\frameSubsection{}{}

\begin{frame}{Backtraces}{Debugging information \& source code}
  \begin{columns}[c]
    \begin{column}{0.5\textwidth}
      \begin{itemize}
        \item Raising an exception is fun and games, but how do we know where the exception originated? $\rightarrow$ With the backtrace associated with the exception!
        \item In order to get a backtrace from the point where an exception was raised, it's necessary to compile with debugging information enabled (\funcarg{-g} flag for the compiler)
        \item Same example as in the `Nested handlers' section
      \end{itemize}
    \end{column}
    \begin{column}{0.5\textwidth}
      \listocaml[firstline=9,lastline=34]{../src/backtrace/backtrace.ml}{backtrace/backtrace.ml}
    \end{column}
  \end{columns}
\end{frame}

\begin{frame}{Backtraces}{CMM and disassembly of \funcname{definitely\_raise}}
  \begin{columns}[c]
    \begin{column}{0.5\textwidth}
      \minipage[c][0.66\textheight][s]{\columnwidth}
      \begin{itemize}
        \item<1-> An effect of compiling with debugging information (\funcarg{-g}): the CMM no longer uses \funcname{raise\_notrace} to raise the exception but \funcname{raise} instead
        \item<2-> Disassembly shows that CMM \funcname{raise} is actually a call to \funcname{caml\_raise\_exn}, a function in assembler provided by the runtime
      \end{itemize}
      \vfill
      \mintocaml[firstline=9,lastline=13,highlightlines=12]{../src/backtrace/backtrace.ml}
      \endminipage
    \end{column}
    \begin{column}{0.5\textwidth}
      \onslide*<1>{
        \mintcmm[highlightlines=5]{../src/backtrace/camlBacktrace\_\_definitely\_raise\_347.cmm}
      }
      \onslide*<2>{
        \mintobjdump[firstline=2,highlightlines=14]{../src/backtrace/camlBacktrace\_\_definitely\_raise\_347.objdump}
      }
    \end{column}
  \end{columns}
\end{frame}

\begin{frame}{Backtraces}{Introducing \funcname{caml\_raise\_exn}}
  \begin{columns}[c]
    \begin{column}{0.5\textwidth}
      \minipage[c][0.66\textheight][s]{\columnwidth}
      \onslide*<1>{
        \begin{itemize}
          \item Raising the exception is performed using \funcname{caml\_raise\_exn} which is
            \begin{itemize}
              \item An assembly function provided by the runtime
              \item Architecture specific
            \end{itemize}
          \item Let's see what it does differently from \funcname{raise\_notrace}\dots
          \item We are about to raise \typename{ExnC} with \funcname{caml\_raise\_exn} from \funcname{definitely\_raise}
        \end{itemize}
      }
      \onslide*<2>{
        \mintobjdump[firstline=2,highlightlines=14]{../src/backtrace/camlBacktrace\_\_definitely\_raise\_347.objdump}
      }
      \vfill
      \mintocaml[firstline=9,lastline=13,highlightlines=12]{../src/backtrace/backtrace.ml}
      \endminipage
    \end{column}
    \begin{column}{0.5\textwidth}
      \centering
      \providecommand\step{1}%
% Backtraces
%
\subsection{One advantage of raising with debugging information}
\frameSubsection{}{}

\begin{frame}{Backtraces}{Debugging information \& source code}
  \begin{columns}[c]
    \begin{column}{0.5\textwidth}
      \begin{itemize}
        \item Raising an exception is fun and games, but how do we know where the exception originated? $\rightarrow$ With the backtrace associated with the exception!
        \item In order to get a backtrace from the point where an exception was raised, it's necessary to compile with debugging information enabled (\funcarg{-g} flag for the compiler)
        \item Same example as in the `Nested handlers' section
      \end{itemize}
    \end{column}
    \begin{column}{0.5\textwidth}
      \listocaml[firstline=9,lastline=34]{../src/backtrace/backtrace.ml}{backtrace/backtrace.ml}
    \end{column}
  \end{columns}
\end{frame}

\begin{frame}{Backtraces}{CMM and disassembly of \funcname{definitely\_raise}}
  \begin{columns}[c]
    \begin{column}{0.5\textwidth}
      \minipage[c][0.66\textheight][s]{\columnwidth}
      \begin{itemize}
        \item<1-> An effect of compiling with debugging information (\funcarg{-g}): the CMM no longer uses \funcname{raise\_notrace} to raise the exception but \funcname{raise} instead
        \item<2-> Disassembly shows that CMM \funcname{raise} is actually a call to \funcname{caml\_raise\_exn}, a function in assembler provided by the runtime
      \end{itemize}
      \vfill
      \mintocaml[firstline=9,lastline=13,highlightlines=12]{../src/backtrace/backtrace.ml}
      \endminipage
    \end{column}
    \begin{column}{0.5\textwidth}
      \onslide*<1>{
        \mintcmm[highlightlines=5]{../src/backtrace/camlBacktrace\_\_definitely\_raise\_347.cmm}
      }
      \onslide*<2>{
        \mintobjdump[firstline=2,highlightlines=14]{../src/backtrace/camlBacktrace\_\_definitely\_raise\_347.objdump}
      }
    \end{column}
  \end{columns}
\end{frame}

\begin{frame}{Backtraces}{Introducing \funcname{caml\_raise\_exn}}
  \begin{columns}[c]
    \begin{column}{0.5\textwidth}
      \minipage[c][0.66\textheight][s]{\columnwidth}
      \onslide*<1>{
        \begin{itemize}
          \item Raising the exception is performed using \funcname{caml\_raise\_exn} which is
            \begin{itemize}
              \item An assembly function provided by the runtime
              \item Architecture specific
            \end{itemize}
          \item Let's see what it does differently from \funcname{raise\_notrace}\dots
          \item We are about to raise \typename{ExnC} with \funcname{caml\_raise\_exn} from \funcname{definitely\_raise}
        \end{itemize}
      }
      \onslide*<2>{
        \mintobjdump[firstline=2,highlightlines=14]{../src/backtrace/camlBacktrace\_\_definitely\_raise\_347.objdump}
      }
      \vfill
      \mintocaml[firstline=9,lastline=13,highlightlines=12]{../src/backtrace/backtrace.ml}
      \endminipage
    \end{column}
    \begin{column}{0.5\textwidth}
      \centering
      \providecommand\step{1}\input{diagrams/backtrace/backtrace.tex}
    \end{column}
  \end{columns}
\end{frame}

\begin{frame}{Backtraces}{But before that, we need more fields from \typename{caml\_domain\_state}}
  \begin{columns}
    \begin{column}{0.5\textwidth}
      \begin{itemize}
        \item Remember \typename{caml\_domain\_state} the per-domain runtime state?
        \item Some other members are of interest to us now\dots
        \item \localname{backtrace\_active} is used to know if the runtime should collect backtraces
        \item \localname{backtrace\_active} can be set by
          \begin{itemize}
            \item The \funcarg{b} flag in the \funcname{OCAMLRUNPARAM} environment variable
            \item \mintinline{ocaml}{Printexc.record_backtrace : bool -> unit} \footnotemark
          \end{itemize}
      \end{itemize}
    \end{column}
    \begin{column}{0.5\textwidth}
      \begin{listing}
        \mintc[highlightlines={9-11}]{src/ocaml/domain\_state\_backtrace.h}
        \caption{runtime/caml/domain\_state.h}
      \end{listing}
    \end{column}
  \end{columns}
  \footnotetext[1]{https://v2.ocaml.org/api/Printexc.html}
\end{frame}

\newcommand\listCamlRaiseExn[1][]{
  \listasm[firstline=702,lastline=725,#1]{../ocaml/runtime/amd64.S}{runtime/amd64.S:702}}

\begin{frame}{Backtraces}{Walk through \funcname{caml\_raise\_exn}}
  \begin{columns}[T]
    \begin{column}{0.5\textwidth}
      \begin{itemize}
        \item<1-> First checks whether exceptions backtrace should be recorded
        \item<1-> By checking the \funcarg{domain\_state->backtrace\_active} value
        \item<2-> If not, acts as \funcname{raise\_notrace} with \funcname{RESTORE\_EXN\_HANDLER\_OCAML}
          \begin{itemize}
            \item Set \regname{RSP} to \localname{domain\_state->exn\_handler} value
            \item Restore \localname{domain\_state->exn\_handler} from trap
            \item Jump with \funcname{ret} (same effect as using \funcname{pop} and \funcname{jmp})
          \end{itemize}
        \item<3-> Otherwise, switches to the C stack to be able to execute C code
        \item<4-> Calls \funcname{caml\_stash\_backtrace}, a C function provided by the runtime\ignorespaces
          \onslide<6->{, with
            \begin{itemize}
              \item \funcarg{exn}: exception value
              \item \funcarg{pc}: code address of the raising point
              \item \funcarg{sp}: stack address of the raising function
              \item \funcarg{trapsp}: stack address of the next handler
            \end{itemize}
          }
      \end{itemize}
      \bigskip
      \mintocaml[firstline=9,lastline=13,highlightlines=12]{../src/backtrace/backtrace.ml}
    \end{column}
    \begin{column}{0.5\textwidth}
      \raggedleft
      \onslide*<1,3-4>{
        \mintc[firstline=190,lastline=192]{../ocaml/runtime/amd64.S}
        \mintc[firstline=234,lastline=237]{../ocaml/runtime/amd64.S}
      }
      \onslide*<2>{
        \mintc[firstline=190,lastline=192,highlightlines={191-192}]{../ocaml/runtime/amd64.S}
        \mintc[firstline=234,lastline=237,highlightlines={235-237}]{../ocaml/runtime/amd64.S}
      }
      \onslide*<1>{\listCamlRaiseExn[highlightlines=705]{}}%
      \onslide*<2>{\listCamlRaiseExn[highlightlines={707-708}]{}}%
      \onslide*<3>{\listCamlRaiseExn[highlightlines={712-714}]{}}%
      \onslide*<4>{\listCamlRaiseExn[highlightlines={715-720}]{}}%
      \onslide*<5>{\providecommand\step{1}\input{diagrams/backtrace/backtrace.tex}}%
      \onslide*<6>{\providecommand\step{2}\input{diagrams/backtrace/backtrace.tex}}%
    \end{column}
  \end{columns}
\end{frame}

\newcommand\listStashBacktrace[1][]{
  \listc[firstline=91,lastline=120,#1]{../ocaml/runtime/backtrace\_nat.c}{runtime/backtrace\_nat.c:91}}

\begin{frame}{Backtraces}{Introducing \funcname{caml\_stash\_backtrace}}
  \begin{columns}[c]
    \begin{column}{0.5\textwidth}
      \minipage[c][0.66\textheight][s]{\columnwidth}
      \begin{itemize}
        \item<1-> A C function provided by the runtime (specific to native code)
        \item<2-> It scans the stack, between the stack pointer at the raising point up to the next trap, in order to collect the names of the detected function frames
          \onslide*<2>{
            \begin{itemize}
              \item And more information, like line number, column number...
            \end{itemize}
          }
        \item<3-> It uses the same technique and information as the Garbage Collector (GC) uses to scan the values live on the stack
          \onslide*<4>{
            \begin{itemize}
              \item \typename{frame\_descr} are at the core of stack unwinding
            \end{itemize}
          }
      \end{itemize}
      \vfill
      \mintocaml[firstline=9,lastline=13,highlightlines=12]{../src/backtrace/backtrace.ml}
      \endminipage
    \end{column}
    \begin{column}{0.5\textwidth}
      \onslide*<1>{\listStashBacktrace[highlightlines=91]{}}
      \onslide*<2>{\listStashBacktrace[highlightlines={91,117-118}]{}}
      \onslide*<3->{\listStashBacktrace[highlightlines={94,106,109-110}]{}}
    \end{column}
  \end{columns}
\end{frame}

\newcommand\listFrameDescr[1][]{
  \listocaml[firstline=111,lastline=124,#1]{../ocaml/asmcomp/emitaux.ml}{asmcomp/emitaux.ml:111}}
\newcommand\listDebugInfo[1][]{
  \listocaml[firstline=38,lastline=49,#1]{../ocaml/lambda/debuginfo.mli}{lambda/debuginfo.mli:38}}

\begin{frame}{Backtraces}{\typename{frame\_descr} and \typename{frame\_debuginfo} in a nutshell}
  \begin{columns}[c]
    \begin{column}{0.5\textwidth}
      \begin{itemize}
        \item<1-> For every return address in an OCaml program, there is a associated \typename{frame\_descr}
        \item<2-> A \typename{frame\_descr} specifies
        \begin{itemize}
          \item<2-> The size of the function stack frame
          \item<3-> Where live values are on the stack (so that the GC can mark them as live and prevent from being swept)
        \end{itemize}
        \item<4-> When compiled with debug information, \typename{frame\_descr} will be linked to a \typename{frame\_debuginfo}
        \begin{itemize}
          \item<5-> Contains information about the position in the source code (file name, line and column number)
        \end{itemize}
      \end{itemize}
    \end{column}
    \begin{column}{0.5\textwidth}
        \onslide*<1>{\listFrameDescr[highlightlines={118-122}]{}}
        \onslide*<2>{\listFrameDescr[highlightlines={120}]{}}
        \onslide*<3>{\listFrameDescr[highlightlines={121}]{}}
        \onslide*<4->{\listFrameDescr[highlightlines={122}]{}}
        \medskip
        \onslide*<1-3>{\listDebugInfo{}}
        \onslide*<4>{\listDebugInfo[highlightlines={38,49}]{}}
        \onslide*<5>{\listDebugInfo[highlightlines={39-46}]{}}
    \end{column}
  \end{columns}
\end{frame}

\begin{frame}{Backtraces}{Scanning the stack with \typename{frame\_descr}}
  \begin{columns}[c]
    \begin{column}{0.5\textwidth}
      \begin{itemize}
        \item Given the return address of the code that raises the exception by calling \funcname{caml\_raise\_exn} (\funcarg{pc} argument of \funcname{caml\_stash\_backtrace})
        \item And the position of the stack pointer (\funcarg{sp} argument of \funcname{caml\_stash\_backtrace})
        \item The frame size given by \typename{frame\_descr}.\funcarg{frame\_size}, allows to find the end of the previous frame
        \item And the return address of the caller of this function
        \bigskip
        \onslide<3->{
        \item Note that traps (here the reddish \funcname{ExnA}, \funcname{ExnB} and \funcname{ExnC} blocks) belong to the frame that installed them, and so the frame size changes when traps are removed
        }
      \end{itemize}
    \end{column}
    \begin{column}{0.5\textwidth}
      \raggedleft
      \onslide*<1>{\providecommand\step{2}\input{diagrams/backtrace/backtrace.tex}}%
      \onslide*<2>{\providecommand\step{1}\input{diagrams/backtrace/scanstack.tex}}%
      \onslide*<3>{\providecommand\step{2}\input{diagrams/backtrace/scanstack.tex}}%
      \onslide*<4>{\providecommand\step{3}\input{diagrams/backtrace/scanstack.tex}}%
      \onslide*<5>{\providecommand\step{4}\input{diagrams/backtrace/scanstack.tex}}%
      \onslide*<6>{\providecommand\step{5}\input{diagrams/backtrace/scanstack.tex}}%
    \end{column}
  \end{columns}
\end{frame}

% Duplicate of previous-previous slide
\begin{frame}{Backtraces}{Continuing on \funcname{caml\_stash\_backtrace}}
  \begin{columns}[c]
    \begin{column}{0.5\textwidth}
      \minipage[c][0.75\textheight][s]{\columnwidth}
      \begin{itemize}
          \color{gray}
        \item A C function provided by the runtime (specific to native code)
        \item It scans the stack, between the stack pointer at the raising point up to the next trap, in order to collect the names of the detected function frames
        \item It uses the same technique and information as the Garbage Collector (GC) uses to scan the values live on the stack
      \end{itemize}
      \begin{itemize}
        \item For each function frame detected, store a pointer to the \typename{frame\_descr} in the \localname{domain\_state->backtrace\_buffer} array, effectively constructing the backtrace
      \end{itemize}
      \onslide<2->{
        \begin{itemize}
            \smallskip
          \item Constructed backtrace:
            \funcname{definitely\_raise},
            \onslide<3->{\funcname{probably\_raise}}
        \end{itemize}
      }
      \vfill
      \mintocaml[firstline=9,lastline=13,highlightlines=12]{../src/backtrace/backtrace.ml}
      \endminipage
    \end{column}
    \begin{column}{0.5\textwidth}
      \raggedleft
      \onslide*<1>{\listStashBacktrace[highlightlines={114-115}]{}}
      \onslide*<2>{\providecommand\step{3}\input{diagrams/backtrace/backtrace.tex}}%
      \onslide*<3>{\providecommand\step{4}\input{diagrams/backtrace/backtrace.tex}}%
    \end{column}
  \end{columns}
\end{frame}

\begin{frame}{Backtraces}{Back to \funcname{caml\_raise\_exn}}
  \begin{columns}[c]
    \begin{column}{0.5\textwidth}
      \minipage[c][0.66\textheight][s]{\columnwidth}
      \begin{itemize}\color{gray}
        \item Checks wheteher exceptions backtrace should be recorded, by checking the \funcarg{domain\_state->backtrace\_active} value
        \item Switches to the C stack to be able to execute C code
        \item Calls \funcname{caml\_stash\_backtrace}, to collect the backtrace up to the next trap
      \end{itemize}
      \onslide*<2->{
        \begin{itemize}
          \item<2-> Executes the exception handler in \funcname{probably\_raise} with \funcname{RESTORE\_EXN\_HANDLER\_OCAML}
            \begin{itemize}
              \item Set \regname{RSP} to \localname{domain\_state->exn\_handler} value
              \item Restore \localname{domain\_state->exn\_handler} from trap
              \item Jump to the handler code with \funcname{ret}
            \end{itemize}
        \end{itemize}
      }
      \vfill
      \onslide*<1>{\mintocaml[firstline=9,lastline=13,highlightlines=12]{../src/backtrace/backtrace.ml}}
      \onslide*<2->{\mintocaml[firstline=15,lastline=21,highlightlines={19-21}]{../src/backtrace/backtrace.ml}}
      \endminipage
    \end{column}
    \begin{column}{0.5\textwidth}
      \centering
      \onslide*<1>{
        \mintc[firstline=190,lastline=192]{../ocaml/runtime/amd64.S}
        \mintc[firstline=234,lastline=237]{../ocaml/runtime/amd64.S}
      }
      \onslide*<2>{
        \mintc[firstline=190,lastline=192,highlightlines={191-192}]{../ocaml/runtime/amd64.S}
        \mintc[firstline=234,lastline=237,highlightlines={235-237}]{../ocaml/runtime/amd64.S}
      }
      \onslide*<1>{\listCamlRaiseExn[highlightlines=720]{}}
      \onslide*<2>{\listCamlRaiseExn[highlightlines=722]{}}
      \onslide*<3->{\providecommand\step{5}\input{diagrams/backtrace/backtrace.tex}}
    \end{column}
  \end{columns}
\end{frame}

\begin{frame}{Backtraces}{\funcname{caml\_probably\_raise}'s handler doesn't catch \typename{ExnC}}
  \begin{columns}[c]
    \begin{column}{0.45\textwidth}
      \minipage[c][0.75\textheight][s]{\columnwidth}
      \begin{itemize}
        \item<1-> \funcname{match \dots with}\footnotemark pattern matching can also match exceptions
        \item<1-> The CMM of \funcname{probably\_raise} shows that this \funcname{match \dots with} is implemented with a pattern matching and a \funcname{try \dots with}
        \item<1-> But this one only matches on \typename{ExnA} exception
        \item<2-> Since the pattern matching fails to catch the exception, \funcname{reraise} is called with our current \typename{ExnC} exception
        \item<3-> Disassembly shows that \funcname{reraise} is actually a call to \funcname{caml\_reraise\_exn}, which is also an assembly function provided by the runtime
      \end{itemize}
      \vfill
      \mintocaml[firstline=15,lastline=21,highlightlines={18-21}]{../src/backtrace/backtrace.ml}
      \endminipage
    \end{column}
    \begin{column}{0.55\textwidth}
      \centering
      \onslide*<1>{\mintcmm[highlightlines={10-18}]{../src/backtrace/camlBacktrace\_\_probably\_raise\_351.cmm}}
      \onslide*<2>{\listcmm[highlightlines=18]{../src/backtrace/camlBacktrace\_\_probably\_raise_351.cmm}{CMM of probably\_raise}}
      \onslide*<3>{\mintobjdump[firstline=5,lastline=35,highlightlines=31]{../src/backtrace/camlBacktrace\_\_probably\_raise_351.objdump}}
    \end{column}
  \end{columns}
  \only<1>{\footnotetext[1]{https://blog.janestreet.com/pattern-matching-and-exception-handling-unite/}}
\end{frame}

\newcommand\listCamlReraiseExn[1][]{
  \listasm[firstline=727,lastline=734,#1]{../ocaml/runtime/amd64.S}{runtime/amd64.S:727}}

\begin{frame}{Backtraces}{Introducing \funcname{caml\_reraise\_exn}}
  \begin{columns}[c]
    \begin{column}{0.5\textwidth}
      \minipage[c][0.66\textheight][s]{\columnwidth}
      \begin{itemize}
        \item<1-> The exception have to be reraised to the next handler
        \item<2-> First, checks if the backtrace should be collected with \funcarg{domain\_state->backtrace\_active}
        \only<3>{
        \begin{itemize}
            \item If not, behaves like a \funcname{raise\_notrace} with \funcname{RESTORE\_EXN\_HANDLER\_OCAML}
        \end{itemize}
        }
        \item<4-> Jumps into \funcname{caml\_raise\_exn}
        \item<5-> Calls \funcname{caml\_stash\_backtrace} again, to scan the next part of the stack: from the trap just pop-ed to the next trap
      \end{itemize}
      \vfill
      \mintocaml[firstline=15,lastline=21,highlightlines={18-21}]{../src/backtrace/backtrace.ml}
      \endminipage
    \end{column}
    \begin{column}{0.5\textwidth}
      \raggedleft
      \onslide*<1>{\listCamlReraiseExn{}}
      \onslide*<2>{\listCamlReraiseExn[highlightlines=729]{}}
      \onslide*<3>{
        \mintc[firstline=190,lastline=192,highlightlines={191-192}]{../ocaml/runtime/amd64.S}
        \mintc[firstline=234,lastline=237,highlightlines={235-237}]{../ocaml/runtime/amd64.S}
        \medskip
        \listCamlReraiseExn[highlightlines=731]{}
      }
      \onslide*<4-5>{
        % TODO refactor with \listCamlReraiseExn
        \mintasm[firstline=727,lastline=734,highlightlines=730]{../ocaml/runtime/amd64.S}
        \medskip
      }
      \onslide*<4>{\listCamlRaiseExn[highlightlines=711]{}}
      \onslide*<5>{\listCamlRaiseExn[highlightlines=720]{}}
    \end{column}
  \end{columns}
\end{frame}

\begin{frame}{Backtraces}{Backtrace collection continues}
  \begin{columns}[c]
    \begin{column}{0.55\textwidth}
      \minipage[c][0.75\textheight][s]{\columnwidth}
      \begin{itemize}
        \item<1-> \funcname{caml\_stash\_backtrace} scans the older part of the stack, up to the next trap
        \item<6-> Controls jumps to the nested exception handler in \funcname{maybe\_raise}, which matches only on \typename{ExnB}
        \item<7-> \funcname{caml\_reraise\_exn} is called again, backtrace collection resumes with \funcname{caml\_stash\_backtrace}
      \end{itemize}
      \medskip
      \begin{itemize}
        \item<2-> Constructed backtrace:
        \begin{itemize}
          \item <2-> \funcname{definitely\_raise}
          \item <2-> \funcname{probably\_raise}
          \item <3-> \funcname{probably\_raise} (re-raise)
          \item <4-> \funcname{maybe\_raise}
          \item <5-> \funcname{main}
          \item <8-> \funcname{main} (re-raise)
        \end{itemize}
      \end{itemize}
      \vfill
      \onslide*<1-5>{\mintocaml[firstline=15,lastline=21]{../src/backtrace/backtrace.ml}}
      \onslide*<6>{\mintocaml[firstline=28,lastline=34,highlightlines=31]{../src/backtrace/backtrace.ml}}
      \onslide*<7->{\mintocaml[firstline=28,lastline=34,highlightlines=33]{../src/backtrace/backtrace.ml}}
      \endminipage
    \end{column}
    \begin{column}{0.45\textwidth}
      \raggedleft
      \onslide*<1>{\providecommand\step{6}\input{diagrams/backtrace/backtrace.tex}}%
      \onslide*<2>{\providecommand\step{6}\input{diagrams/backtrace/backtrace.tex}}%
      \onslide*<3>{\providecommand\step{7}\input{diagrams/backtrace/backtrace.tex}}%
      \onslide*<4>{\providecommand\step{8}\input{diagrams/backtrace/backtrace.tex}}%
      \onslide*<5>{\providecommand\step{9}\input{diagrams/backtrace/backtrace.tex}}%
      \onslide*<6>{\providecommand\step{10}\input{diagrams/backtrace/backtrace.tex}}%
      \onslide*<7>{\providecommand\step{11}\input{diagrams/backtrace/backtrace.tex}}%
      \onslide*<8>{\providecommand\step{12}\input{diagrams/backtrace/backtrace.tex}}%
      \onslide*<9>{\providecommand\step{13}\input{diagrams/backtrace/backtrace.tex}}%
    \end{column}
  \end{columns}
\end{frame}

\begin{frame}{Backtraces}{Printing the backtrace}
  \begin{columns}[c]
    \begin{column}{0.45\textwidth}
      \minipage[c][0.66\textheight][s]{\columnwidth}
      \begin{itemize}
        \item<1-> The exception backtrace can now be printed to \funcarg{stdout} with \funcname{Printexc.print_backtrace}
        \item<2-> First the exception backtrace is retrieved into an OCaml array with \funcname{get_raw_backtrace }
        \item<3-> Which is implemented as the \funcname{caml_get_exception_raw_backtrace} C function in the runtime
        \item<4-> And finally printed with \funcname{print_exception_backtrace}
      \end{itemize}
      \vfill
      \mintocaml[firstline=28,lastline=34,highlightlines=33]{../src/backtrace/backtrace.ml}
      \endminipage
    \end{column}
    \begin{column}{0.55\textwidth}
      \onslide*<1,2>{
        \mintocaml[firstline=100,lastline=101]{../ocaml/stdlib/printexc.ml}
      }
      \only<1>{
        \listocaml[firstline=170,lastline=172]{../ocaml/stdlib/printexc.ml}{stdlib/printexc.ml:170}
      }
      \only<2>{
        \listocaml[firstline=170,lastline=172,highlightlines=172]{../ocaml/stdlib/printexc.ml}{stdlib/printexc.ml:170}
      }
      \only<3>{
        \mintc[firstline=154,lastline=156]{../ocaml/runtime/backtrace.c}
        \listc[firstline=171,lastline=192,highlightlines={184-187}]{../ocaml/runtime/backtrace.c}{runtime/backtrace.c:154}
      }
      \only<4>{
        \listocaml[firstline=155,lastline=165]{../ocaml/stdlib/printexc.ml}{stdlib/printexc.ml:55}
      }
    \end{column}
  \end{columns}
\end{frame}


\subsection{The multiple kinds of raise}
\frameSubsection{}{}

\begin{frame}{Backtraces}{When backtraces are not needed}
  \begin{columns}[c]
    \begin{column}{0.45\textwidth}
      \minipage[c][0.66\textheight][s]{\columnwidth}
      \begin{itemize}
        \item<1-> What if the backtrace for an exception is never needed?
        \item<2-> It's possible to raise the exception explicitly with \funcname{raise\_notrace} to make raising as cheap as possible
        \item<3-> An example of use in the \funcname{dynlink} library of the OCaml compiler
      \end{itemize}
      \vfill
      \onslide<3->{
      \listocaml[firstline=205,lastline=209,highlightlines=207]{../ocaml/otherlibs/dynlink/dynlink\_compilerlibs/misc.ml}{otherlibs/dynlink/dynlink\_compilerlibs/misc.ml:205}
      }
      \endminipage
    \end{column}
    \begin{column}{0.55\textwidth}
    \only<4>{
      \mintcmm[highlightlines=3]{../src/notrace/camlNotrace\_\_fun_347.cmm}
      \listcmm{../src/notrace/camlNotrace\_\_all\_somes\_327.cmm}{all\_somes}
    }
    \onslide*<5->{
      \listobjdump[highlightlines={6-9}]{../src/notrace/camlNotrace\_\_fun_347.objdump}{anonymous function from \funcname{all\_somes}}
    }
    \end{column}
  \end{columns}
\end{frame}

\begin{frame}{Backtraces}{Summary table of the multiple kinds of raise}
  \begin{itemize}
    \item When debugging information is enabled and if the backtrace of an exception isn't needed, it's possible to raise with \funcname{raise\_notrace} for performance
    \item When debugging information is \emph{not} enabled, the compiler optimises \funcname{raise} calls to inline assembly (same effect as using \funcname{raise\_notrace})
  \end{itemize}
  \bigskip
  \centering
  \begin{tabular}{| l | c | c | c | c |}
    \hline
    \parbox{10em}{Debug information \\ (\funcarg{-g} flag)}  & \multicolumn{2}{|c|}{Disabled}                                     & \multicolumn{2}{|c|}{Enabled} \\
    \hline
    Raise kind                                               & \funcname{raise} & \funcname{raise\_notrace}                       & \funcname{raise} & \funcname{raise\_notrace} \\
    \hline
    Compilation                                              & \parbox{8em}{inline assembly\\ (optimization)} & inline assembly   & \funcname{caml\_raise\_exn} & inline assembly \\
    \hline
  \end{tabular}
\end{frame}


%
% Takeaway #5
%
\subsection*{Takeaway \#5}
\frameSubsectionTakeaway{}
\againframe<5>{takeaway}

    \end{column}
  \end{columns}
\end{frame}

\begin{frame}{Backtraces}{But before that, we need more fields from \typename{caml\_domain\_state}}
  \begin{columns}
    \begin{column}{0.5\textwidth}
      \begin{itemize}
        \item Remember \typename{caml\_domain\_state} the per-domain runtime state?
        \item Some other members are of interest to us now\dots
        \item \localname{backtrace\_active} is used to know if the runtime should collect backtraces
        \item \localname{backtrace\_active} can be set by
          \begin{itemize}
            \item The \funcarg{b} flag in the \funcname{OCAMLRUNPARAM} environment variable
            \item \mintinline{ocaml}{Printexc.record_backtrace : bool -> unit} \footnotemark
          \end{itemize}
      \end{itemize}
    \end{column}
    \begin{column}{0.5\textwidth}
      \begin{listing}
        \mintc[highlightlines={9-11}]{src/ocaml/domain\_state\_backtrace.h}
        \caption{runtime/caml/domain\_state.h}
      \end{listing}
    \end{column}
  \end{columns}
  \footnotetext[1]{https://v2.ocaml.org/api/Printexc.html}
\end{frame}

\newcommand\listCamlRaiseExn[1][]{
  \listasm[firstline=702,lastline=725,#1]{../ocaml/runtime/amd64.S}{runtime/amd64.S:702}}

\begin{frame}{Backtraces}{Walk through \funcname{caml\_raise\_exn}}
  \begin{columns}[T]
    \begin{column}{0.5\textwidth}
      \begin{itemize}
        \item<1-> First checks whether exceptions backtrace should be recorded
        \item<1-> By checking the \funcarg{domain\_state->backtrace\_active} value
        \item<2-> If not, acts as \funcname{raise\_notrace} with \funcname{RESTORE\_EXN\_HANDLER\_OCAML}
          \begin{itemize}
            \item Set \regname{RSP} to \localname{domain\_state->exn\_handler} value
            \item Restore \localname{domain\_state->exn\_handler} from trap
            \item Jump with \funcname{ret} (same effect as using \funcname{pop} and \funcname{jmp})
          \end{itemize}
        \item<3-> Otherwise, switches to the C stack to be able to execute C code
        \item<4-> Calls \funcname{caml\_stash\_backtrace}, a C function provided by the runtime\ignorespaces
          \onslide<6->{, with
            \begin{itemize}
              \item \funcarg{exn}: exception value
              \item \funcarg{pc}: code address of the raising point
              \item \funcarg{sp}: stack address of the raising function
              \item \funcarg{trapsp}: stack address of the next handler
            \end{itemize}
          }
      \end{itemize}
      \bigskip
      \mintocaml[firstline=9,lastline=13,highlightlines=12]{../src/backtrace/backtrace.ml}
    \end{column}
    \begin{column}{0.5\textwidth}
      \raggedleft
      \onslide*<1,3-4>{
        \mintc[firstline=190,lastline=192]{../ocaml/runtime/amd64.S}
        \mintc[firstline=234,lastline=237]{../ocaml/runtime/amd64.S}
      }
      \onslide*<2>{
        \mintc[firstline=190,lastline=192,highlightlines={191-192}]{../ocaml/runtime/amd64.S}
        \mintc[firstline=234,lastline=237,highlightlines={235-237}]{../ocaml/runtime/amd64.S}
      }
      \onslide*<1>{\listCamlRaiseExn[highlightlines=705]{}}%
      \onslide*<2>{\listCamlRaiseExn[highlightlines={707-708}]{}}%
      \onslide*<3>{\listCamlRaiseExn[highlightlines={712-714}]{}}%
      \onslide*<4>{\listCamlRaiseExn[highlightlines={715-720}]{}}%
      \onslide*<5>{\providecommand\step{1}%
% Backtraces
%
\subsection{One advantage of raising with debugging information}
\frameSubsection{}{}

\begin{frame}{Backtraces}{Debugging information \& source code}
  \begin{columns}[c]
    \begin{column}{0.5\textwidth}
      \begin{itemize}
        \item Raising an exception is fun and games, but how do we know where the exception originated? $\rightarrow$ With the backtrace associated with the exception!
        \item In order to get a backtrace from the point where an exception was raised, it's necessary to compile with debugging information enabled (\funcarg{-g} flag for the compiler)
        \item Same example as in the `Nested handlers' section
      \end{itemize}
    \end{column}
    \begin{column}{0.5\textwidth}
      \listocaml[firstline=9,lastline=34]{../src/backtrace/backtrace.ml}{backtrace/backtrace.ml}
    \end{column}
  \end{columns}
\end{frame}

\begin{frame}{Backtraces}{CMM and disassembly of \funcname{definitely\_raise}}
  \begin{columns}[c]
    \begin{column}{0.5\textwidth}
      \minipage[c][0.66\textheight][s]{\columnwidth}
      \begin{itemize}
        \item<1-> An effect of compiling with debugging information (\funcarg{-g}): the CMM no longer uses \funcname{raise\_notrace} to raise the exception but \funcname{raise} instead
        \item<2-> Disassembly shows that CMM \funcname{raise} is actually a call to \funcname{caml\_raise\_exn}, a function in assembler provided by the runtime
      \end{itemize}
      \vfill
      \mintocaml[firstline=9,lastline=13,highlightlines=12]{../src/backtrace/backtrace.ml}
      \endminipage
    \end{column}
    \begin{column}{0.5\textwidth}
      \onslide*<1>{
        \mintcmm[highlightlines=5]{../src/backtrace/camlBacktrace\_\_definitely\_raise\_347.cmm}
      }
      \onslide*<2>{
        \mintobjdump[firstline=2,highlightlines=14]{../src/backtrace/camlBacktrace\_\_definitely\_raise\_347.objdump}
      }
    \end{column}
  \end{columns}
\end{frame}

\begin{frame}{Backtraces}{Introducing \funcname{caml\_raise\_exn}}
  \begin{columns}[c]
    \begin{column}{0.5\textwidth}
      \minipage[c][0.66\textheight][s]{\columnwidth}
      \onslide*<1>{
        \begin{itemize}
          \item Raising the exception is performed using \funcname{caml\_raise\_exn} which is
            \begin{itemize}
              \item An assembly function provided by the runtime
              \item Architecture specific
            \end{itemize}
          \item Let's see what it does differently from \funcname{raise\_notrace}\dots
          \item We are about to raise \typename{ExnC} with \funcname{caml\_raise\_exn} from \funcname{definitely\_raise}
        \end{itemize}
      }
      \onslide*<2>{
        \mintobjdump[firstline=2,highlightlines=14]{../src/backtrace/camlBacktrace\_\_definitely\_raise\_347.objdump}
      }
      \vfill
      \mintocaml[firstline=9,lastline=13,highlightlines=12]{../src/backtrace/backtrace.ml}
      \endminipage
    \end{column}
    \begin{column}{0.5\textwidth}
      \centering
      \providecommand\step{1}\input{diagrams/backtrace/backtrace.tex}
    \end{column}
  \end{columns}
\end{frame}

\begin{frame}{Backtraces}{But before that, we need more fields from \typename{caml\_domain\_state}}
  \begin{columns}
    \begin{column}{0.5\textwidth}
      \begin{itemize}
        \item Remember \typename{caml\_domain\_state} the per-domain runtime state?
        \item Some other members are of interest to us now\dots
        \item \localname{backtrace\_active} is used to know if the runtime should collect backtraces
        \item \localname{backtrace\_active} can be set by
          \begin{itemize}
            \item The \funcarg{b} flag in the \funcname{OCAMLRUNPARAM} environment variable
            \item \mintinline{ocaml}{Printexc.record_backtrace : bool -> unit} \footnotemark
          \end{itemize}
      \end{itemize}
    \end{column}
    \begin{column}{0.5\textwidth}
      \begin{listing}
        \mintc[highlightlines={9-11}]{src/ocaml/domain\_state\_backtrace.h}
        \caption{runtime/caml/domain\_state.h}
      \end{listing}
    \end{column}
  \end{columns}
  \footnotetext[1]{https://v2.ocaml.org/api/Printexc.html}
\end{frame}

\newcommand\listCamlRaiseExn[1][]{
  \listasm[firstline=702,lastline=725,#1]{../ocaml/runtime/amd64.S}{runtime/amd64.S:702}}

\begin{frame}{Backtraces}{Walk through \funcname{caml\_raise\_exn}}
  \begin{columns}[T]
    \begin{column}{0.5\textwidth}
      \begin{itemize}
        \item<1-> First checks whether exceptions backtrace should be recorded
        \item<1-> By checking the \funcarg{domain\_state->backtrace\_active} value
        \item<2-> If not, acts as \funcname{raise\_notrace} with \funcname{RESTORE\_EXN\_HANDLER\_OCAML}
          \begin{itemize}
            \item Set \regname{RSP} to \localname{domain\_state->exn\_handler} value
            \item Restore \localname{domain\_state->exn\_handler} from trap
            \item Jump with \funcname{ret} (same effect as using \funcname{pop} and \funcname{jmp})
          \end{itemize}
        \item<3-> Otherwise, switches to the C stack to be able to execute C code
        \item<4-> Calls \funcname{caml\_stash\_backtrace}, a C function provided by the runtime\ignorespaces
          \onslide<6->{, with
            \begin{itemize}
              \item \funcarg{exn}: exception value
              \item \funcarg{pc}: code address of the raising point
              \item \funcarg{sp}: stack address of the raising function
              \item \funcarg{trapsp}: stack address of the next handler
            \end{itemize}
          }
      \end{itemize}
      \bigskip
      \mintocaml[firstline=9,lastline=13,highlightlines=12]{../src/backtrace/backtrace.ml}
    \end{column}
    \begin{column}{0.5\textwidth}
      \raggedleft
      \onslide*<1,3-4>{
        \mintc[firstline=190,lastline=192]{../ocaml/runtime/amd64.S}
        \mintc[firstline=234,lastline=237]{../ocaml/runtime/amd64.S}
      }
      \onslide*<2>{
        \mintc[firstline=190,lastline=192,highlightlines={191-192}]{../ocaml/runtime/amd64.S}
        \mintc[firstline=234,lastline=237,highlightlines={235-237}]{../ocaml/runtime/amd64.S}
      }
      \onslide*<1>{\listCamlRaiseExn[highlightlines=705]{}}%
      \onslide*<2>{\listCamlRaiseExn[highlightlines={707-708}]{}}%
      \onslide*<3>{\listCamlRaiseExn[highlightlines={712-714}]{}}%
      \onslide*<4>{\listCamlRaiseExn[highlightlines={715-720}]{}}%
      \onslide*<5>{\providecommand\step{1}\input{diagrams/backtrace/backtrace.tex}}%
      \onslide*<6>{\providecommand\step{2}\input{diagrams/backtrace/backtrace.tex}}%
    \end{column}
  \end{columns}
\end{frame}

\newcommand\listStashBacktrace[1][]{
  \listc[firstline=91,lastline=120,#1]{../ocaml/runtime/backtrace\_nat.c}{runtime/backtrace\_nat.c:91}}

\begin{frame}{Backtraces}{Introducing \funcname{caml\_stash\_backtrace}}
  \begin{columns}[c]
    \begin{column}{0.5\textwidth}
      \minipage[c][0.66\textheight][s]{\columnwidth}
      \begin{itemize}
        \item<1-> A C function provided by the runtime (specific to native code)
        \item<2-> It scans the stack, between the stack pointer at the raising point up to the next trap, in order to collect the names of the detected function frames
          \onslide*<2>{
            \begin{itemize}
              \item And more information, like line number, column number...
            \end{itemize}
          }
        \item<3-> It uses the same technique and information as the Garbage Collector (GC) uses to scan the values live on the stack
          \onslide*<4>{
            \begin{itemize}
              \item \typename{frame\_descr} are at the core of stack unwinding
            \end{itemize}
          }
      \end{itemize}
      \vfill
      \mintocaml[firstline=9,lastline=13,highlightlines=12]{../src/backtrace/backtrace.ml}
      \endminipage
    \end{column}
    \begin{column}{0.5\textwidth}
      \onslide*<1>{\listStashBacktrace[highlightlines=91]{}}
      \onslide*<2>{\listStashBacktrace[highlightlines={91,117-118}]{}}
      \onslide*<3->{\listStashBacktrace[highlightlines={94,106,109-110}]{}}
    \end{column}
  \end{columns}
\end{frame}

\newcommand\listFrameDescr[1][]{
  \listocaml[firstline=111,lastline=124,#1]{../ocaml/asmcomp/emitaux.ml}{asmcomp/emitaux.ml:111}}
\newcommand\listDebugInfo[1][]{
  \listocaml[firstline=38,lastline=49,#1]{../ocaml/lambda/debuginfo.mli}{lambda/debuginfo.mli:38}}

\begin{frame}{Backtraces}{\typename{frame\_descr} and \typename{frame\_debuginfo} in a nutshell}
  \begin{columns}[c]
    \begin{column}{0.5\textwidth}
      \begin{itemize}
        \item<1-> For every return address in an OCaml program, there is a associated \typename{frame\_descr}
        \item<2-> A \typename{frame\_descr} specifies
        \begin{itemize}
          \item<2-> The size of the function stack frame
          \item<3-> Where live values are on the stack (so that the GC can mark them as live and prevent from being swept)
        \end{itemize}
        \item<4-> When compiled with debug information, \typename{frame\_descr} will be linked to a \typename{frame\_debuginfo}
        \begin{itemize}
          \item<5-> Contains information about the position in the source code (file name, line and column number)
        \end{itemize}
      \end{itemize}
    \end{column}
    \begin{column}{0.5\textwidth}
        \onslide*<1>{\listFrameDescr[highlightlines={118-122}]{}}
        \onslide*<2>{\listFrameDescr[highlightlines={120}]{}}
        \onslide*<3>{\listFrameDescr[highlightlines={121}]{}}
        \onslide*<4->{\listFrameDescr[highlightlines={122}]{}}
        \medskip
        \onslide*<1-3>{\listDebugInfo{}}
        \onslide*<4>{\listDebugInfo[highlightlines={38,49}]{}}
        \onslide*<5>{\listDebugInfo[highlightlines={39-46}]{}}
    \end{column}
  \end{columns}
\end{frame}

\begin{frame}{Backtraces}{Scanning the stack with \typename{frame\_descr}}
  \begin{columns}[c]
    \begin{column}{0.5\textwidth}
      \begin{itemize}
        \item Given the return address of the code that raises the exception by calling \funcname{caml\_raise\_exn} (\funcarg{pc} argument of \funcname{caml\_stash\_backtrace})
        \item And the position of the stack pointer (\funcarg{sp} argument of \funcname{caml\_stash\_backtrace})
        \item The frame size given by \typename{frame\_descr}.\funcarg{frame\_size}, allows to find the end of the previous frame
        \item And the return address of the caller of this function
        \bigskip
        \onslide<3->{
        \item Note that traps (here the reddish \funcname{ExnA}, \funcname{ExnB} and \funcname{ExnC} blocks) belong to the frame that installed them, and so the frame size changes when traps are removed
        }
      \end{itemize}
    \end{column}
    \begin{column}{0.5\textwidth}
      \raggedleft
      \onslide*<1>{\providecommand\step{2}\input{diagrams/backtrace/backtrace.tex}}%
      \onslide*<2>{\providecommand\step{1}\input{diagrams/backtrace/scanstack.tex}}%
      \onslide*<3>{\providecommand\step{2}\input{diagrams/backtrace/scanstack.tex}}%
      \onslide*<4>{\providecommand\step{3}\input{diagrams/backtrace/scanstack.tex}}%
      \onslide*<5>{\providecommand\step{4}\input{diagrams/backtrace/scanstack.tex}}%
      \onslide*<6>{\providecommand\step{5}\input{diagrams/backtrace/scanstack.tex}}%
    \end{column}
  \end{columns}
\end{frame}

% Duplicate of previous-previous slide
\begin{frame}{Backtraces}{Continuing on \funcname{caml\_stash\_backtrace}}
  \begin{columns}[c]
    \begin{column}{0.5\textwidth}
      \minipage[c][0.75\textheight][s]{\columnwidth}
      \begin{itemize}
          \color{gray}
        \item A C function provided by the runtime (specific to native code)
        \item It scans the stack, between the stack pointer at the raising point up to the next trap, in order to collect the names of the detected function frames
        \item It uses the same technique and information as the Garbage Collector (GC) uses to scan the values live on the stack
      \end{itemize}
      \begin{itemize}
        \item For each function frame detected, store a pointer to the \typename{frame\_descr} in the \localname{domain\_state->backtrace\_buffer} array, effectively constructing the backtrace
      \end{itemize}
      \onslide<2->{
        \begin{itemize}
            \smallskip
          \item Constructed backtrace:
            \funcname{definitely\_raise},
            \onslide<3->{\funcname{probably\_raise}}
        \end{itemize}
      }
      \vfill
      \mintocaml[firstline=9,lastline=13,highlightlines=12]{../src/backtrace/backtrace.ml}
      \endminipage
    \end{column}
    \begin{column}{0.5\textwidth}
      \raggedleft
      \onslide*<1>{\listStashBacktrace[highlightlines={114-115}]{}}
      \onslide*<2>{\providecommand\step{3}\input{diagrams/backtrace/backtrace.tex}}%
      \onslide*<3>{\providecommand\step{4}\input{diagrams/backtrace/backtrace.tex}}%
    \end{column}
  \end{columns}
\end{frame}

\begin{frame}{Backtraces}{Back to \funcname{caml\_raise\_exn}}
  \begin{columns}[c]
    \begin{column}{0.5\textwidth}
      \minipage[c][0.66\textheight][s]{\columnwidth}
      \begin{itemize}\color{gray}
        \item Checks wheteher exceptions backtrace should be recorded, by checking the \funcarg{domain\_state->backtrace\_active} value
        \item Switches to the C stack to be able to execute C code
        \item Calls \funcname{caml\_stash\_backtrace}, to collect the backtrace up to the next trap
      \end{itemize}
      \onslide*<2->{
        \begin{itemize}
          \item<2-> Executes the exception handler in \funcname{probably\_raise} with \funcname{RESTORE\_EXN\_HANDLER\_OCAML}
            \begin{itemize}
              \item Set \regname{RSP} to \localname{domain\_state->exn\_handler} value
              \item Restore \localname{domain\_state->exn\_handler} from trap
              \item Jump to the handler code with \funcname{ret}
            \end{itemize}
        \end{itemize}
      }
      \vfill
      \onslide*<1>{\mintocaml[firstline=9,lastline=13,highlightlines=12]{../src/backtrace/backtrace.ml}}
      \onslide*<2->{\mintocaml[firstline=15,lastline=21,highlightlines={19-21}]{../src/backtrace/backtrace.ml}}
      \endminipage
    \end{column}
    \begin{column}{0.5\textwidth}
      \centering
      \onslide*<1>{
        \mintc[firstline=190,lastline=192]{../ocaml/runtime/amd64.S}
        \mintc[firstline=234,lastline=237]{../ocaml/runtime/amd64.S}
      }
      \onslide*<2>{
        \mintc[firstline=190,lastline=192,highlightlines={191-192}]{../ocaml/runtime/amd64.S}
        \mintc[firstline=234,lastline=237,highlightlines={235-237}]{../ocaml/runtime/amd64.S}
      }
      \onslide*<1>{\listCamlRaiseExn[highlightlines=720]{}}
      \onslide*<2>{\listCamlRaiseExn[highlightlines=722]{}}
      \onslide*<3->{\providecommand\step{5}\input{diagrams/backtrace/backtrace.tex}}
    \end{column}
  \end{columns}
\end{frame}

\begin{frame}{Backtraces}{\funcname{caml\_probably\_raise}'s handler doesn't catch \typename{ExnC}}
  \begin{columns}[c]
    \begin{column}{0.45\textwidth}
      \minipage[c][0.75\textheight][s]{\columnwidth}
      \begin{itemize}
        \item<1-> \funcname{match \dots with}\footnotemark pattern matching can also match exceptions
        \item<1-> The CMM of \funcname{probably\_raise} shows that this \funcname{match \dots with} is implemented with a pattern matching and a \funcname{try \dots with}
        \item<1-> But this one only matches on \typename{ExnA} exception
        \item<2-> Since the pattern matching fails to catch the exception, \funcname{reraise} is called with our current \typename{ExnC} exception
        \item<3-> Disassembly shows that \funcname{reraise} is actually a call to \funcname{caml\_reraise\_exn}, which is also an assembly function provided by the runtime
      \end{itemize}
      \vfill
      \mintocaml[firstline=15,lastline=21,highlightlines={18-21}]{../src/backtrace/backtrace.ml}
      \endminipage
    \end{column}
    \begin{column}{0.55\textwidth}
      \centering
      \onslide*<1>{\mintcmm[highlightlines={10-18}]{../src/backtrace/camlBacktrace\_\_probably\_raise\_351.cmm}}
      \onslide*<2>{\listcmm[highlightlines=18]{../src/backtrace/camlBacktrace\_\_probably\_raise_351.cmm}{CMM of probably\_raise}}
      \onslide*<3>{\mintobjdump[firstline=5,lastline=35,highlightlines=31]{../src/backtrace/camlBacktrace\_\_probably\_raise_351.objdump}}
    \end{column}
  \end{columns}
  \only<1>{\footnotetext[1]{https://blog.janestreet.com/pattern-matching-and-exception-handling-unite/}}
\end{frame}

\newcommand\listCamlReraiseExn[1][]{
  \listasm[firstline=727,lastline=734,#1]{../ocaml/runtime/amd64.S}{runtime/amd64.S:727}}

\begin{frame}{Backtraces}{Introducing \funcname{caml\_reraise\_exn}}
  \begin{columns}[c]
    \begin{column}{0.5\textwidth}
      \minipage[c][0.66\textheight][s]{\columnwidth}
      \begin{itemize}
        \item<1-> The exception have to be reraised to the next handler
        \item<2-> First, checks if the backtrace should be collected with \funcarg{domain\_state->backtrace\_active}
        \only<3>{
        \begin{itemize}
            \item If not, behaves like a \funcname{raise\_notrace} with \funcname{RESTORE\_EXN\_HANDLER\_OCAML}
        \end{itemize}
        }
        \item<4-> Jumps into \funcname{caml\_raise\_exn}
        \item<5-> Calls \funcname{caml\_stash\_backtrace} again, to scan the next part of the stack: from the trap just pop-ed to the next trap
      \end{itemize}
      \vfill
      \mintocaml[firstline=15,lastline=21,highlightlines={18-21}]{../src/backtrace/backtrace.ml}
      \endminipage
    \end{column}
    \begin{column}{0.5\textwidth}
      \raggedleft
      \onslide*<1>{\listCamlReraiseExn{}}
      \onslide*<2>{\listCamlReraiseExn[highlightlines=729]{}}
      \onslide*<3>{
        \mintc[firstline=190,lastline=192,highlightlines={191-192}]{../ocaml/runtime/amd64.S}
        \mintc[firstline=234,lastline=237,highlightlines={235-237}]{../ocaml/runtime/amd64.S}
        \medskip
        \listCamlReraiseExn[highlightlines=731]{}
      }
      \onslide*<4-5>{
        % TODO refactor with \listCamlReraiseExn
        \mintasm[firstline=727,lastline=734,highlightlines=730]{../ocaml/runtime/amd64.S}
        \medskip
      }
      \onslide*<4>{\listCamlRaiseExn[highlightlines=711]{}}
      \onslide*<5>{\listCamlRaiseExn[highlightlines=720]{}}
    \end{column}
  \end{columns}
\end{frame}

\begin{frame}{Backtraces}{Backtrace collection continues}
  \begin{columns}[c]
    \begin{column}{0.55\textwidth}
      \minipage[c][0.75\textheight][s]{\columnwidth}
      \begin{itemize}
        \item<1-> \funcname{caml\_stash\_backtrace} scans the older part of the stack, up to the next trap
        \item<6-> Controls jumps to the nested exception handler in \funcname{maybe\_raise}, which matches only on \typename{ExnB}
        \item<7-> \funcname{caml\_reraise\_exn} is called again, backtrace collection resumes with \funcname{caml\_stash\_backtrace}
      \end{itemize}
      \medskip
      \begin{itemize}
        \item<2-> Constructed backtrace:
        \begin{itemize}
          \item <2-> \funcname{definitely\_raise}
          \item <2-> \funcname{probably\_raise}
          \item <3-> \funcname{probably\_raise} (re-raise)
          \item <4-> \funcname{maybe\_raise}
          \item <5-> \funcname{main}
          \item <8-> \funcname{main} (re-raise)
        \end{itemize}
      \end{itemize}
      \vfill
      \onslide*<1-5>{\mintocaml[firstline=15,lastline=21]{../src/backtrace/backtrace.ml}}
      \onslide*<6>{\mintocaml[firstline=28,lastline=34,highlightlines=31]{../src/backtrace/backtrace.ml}}
      \onslide*<7->{\mintocaml[firstline=28,lastline=34,highlightlines=33]{../src/backtrace/backtrace.ml}}
      \endminipage
    \end{column}
    \begin{column}{0.45\textwidth}
      \raggedleft
      \onslide*<1>{\providecommand\step{6}\input{diagrams/backtrace/backtrace.tex}}%
      \onslide*<2>{\providecommand\step{6}\input{diagrams/backtrace/backtrace.tex}}%
      \onslide*<3>{\providecommand\step{7}\input{diagrams/backtrace/backtrace.tex}}%
      \onslide*<4>{\providecommand\step{8}\input{diagrams/backtrace/backtrace.tex}}%
      \onslide*<5>{\providecommand\step{9}\input{diagrams/backtrace/backtrace.tex}}%
      \onslide*<6>{\providecommand\step{10}\input{diagrams/backtrace/backtrace.tex}}%
      \onslide*<7>{\providecommand\step{11}\input{diagrams/backtrace/backtrace.tex}}%
      \onslide*<8>{\providecommand\step{12}\input{diagrams/backtrace/backtrace.tex}}%
      \onslide*<9>{\providecommand\step{13}\input{diagrams/backtrace/backtrace.tex}}%
    \end{column}
  \end{columns}
\end{frame}

\begin{frame}{Backtraces}{Printing the backtrace}
  \begin{columns}[c]
    \begin{column}{0.45\textwidth}
      \minipage[c][0.66\textheight][s]{\columnwidth}
      \begin{itemize}
        \item<1-> The exception backtrace can now be printed to \funcarg{stdout} with \funcname{Printexc.print_backtrace}
        \item<2-> First the exception backtrace is retrieved into an OCaml array with \funcname{get_raw_backtrace }
        \item<3-> Which is implemented as the \funcname{caml_get_exception_raw_backtrace} C function in the runtime
        \item<4-> And finally printed with \funcname{print_exception_backtrace}
      \end{itemize}
      \vfill
      \mintocaml[firstline=28,lastline=34,highlightlines=33]{../src/backtrace/backtrace.ml}
      \endminipage
    \end{column}
    \begin{column}{0.55\textwidth}
      \onslide*<1,2>{
        \mintocaml[firstline=100,lastline=101]{../ocaml/stdlib/printexc.ml}
      }
      \only<1>{
        \listocaml[firstline=170,lastline=172]{../ocaml/stdlib/printexc.ml}{stdlib/printexc.ml:170}
      }
      \only<2>{
        \listocaml[firstline=170,lastline=172,highlightlines=172]{../ocaml/stdlib/printexc.ml}{stdlib/printexc.ml:170}
      }
      \only<3>{
        \mintc[firstline=154,lastline=156]{../ocaml/runtime/backtrace.c}
        \listc[firstline=171,lastline=192,highlightlines={184-187}]{../ocaml/runtime/backtrace.c}{runtime/backtrace.c:154}
      }
      \only<4>{
        \listocaml[firstline=155,lastline=165]{../ocaml/stdlib/printexc.ml}{stdlib/printexc.ml:55}
      }
    \end{column}
  \end{columns}
\end{frame}


\subsection{The multiple kinds of raise}
\frameSubsection{}{}

\begin{frame}{Backtraces}{When backtraces are not needed}
  \begin{columns}[c]
    \begin{column}{0.45\textwidth}
      \minipage[c][0.66\textheight][s]{\columnwidth}
      \begin{itemize}
        \item<1-> What if the backtrace for an exception is never needed?
        \item<2-> It's possible to raise the exception explicitly with \funcname{raise\_notrace} to make raising as cheap as possible
        \item<3-> An example of use in the \funcname{dynlink} library of the OCaml compiler
      \end{itemize}
      \vfill
      \onslide<3->{
      \listocaml[firstline=205,lastline=209,highlightlines=207]{../ocaml/otherlibs/dynlink/dynlink\_compilerlibs/misc.ml}{otherlibs/dynlink/dynlink\_compilerlibs/misc.ml:205}
      }
      \endminipage
    \end{column}
    \begin{column}{0.55\textwidth}
    \only<4>{
      \mintcmm[highlightlines=3]{../src/notrace/camlNotrace\_\_fun_347.cmm}
      \listcmm{../src/notrace/camlNotrace\_\_all\_somes\_327.cmm}{all\_somes}
    }
    \onslide*<5->{
      \listobjdump[highlightlines={6-9}]{../src/notrace/camlNotrace\_\_fun_347.objdump}{anonymous function from \funcname{all\_somes}}
    }
    \end{column}
  \end{columns}
\end{frame}

\begin{frame}{Backtraces}{Summary table of the multiple kinds of raise}
  \begin{itemize}
    \item When debugging information is enabled and if the backtrace of an exception isn't needed, it's possible to raise with \funcname{raise\_notrace} for performance
    \item When debugging information is \emph{not} enabled, the compiler optimises \funcname{raise} calls to inline assembly (same effect as using \funcname{raise\_notrace})
  \end{itemize}
  \bigskip
  \centering
  \begin{tabular}{| l | c | c | c | c |}
    \hline
    \parbox{10em}{Debug information \\ (\funcarg{-g} flag)}  & \multicolumn{2}{|c|}{Disabled}                                     & \multicolumn{2}{|c|}{Enabled} \\
    \hline
    Raise kind                                               & \funcname{raise} & \funcname{raise\_notrace}                       & \funcname{raise} & \funcname{raise\_notrace} \\
    \hline
    Compilation                                              & \parbox{8em}{inline assembly\\ (optimization)} & inline assembly   & \funcname{caml\_raise\_exn} & inline assembly \\
    \hline
  \end{tabular}
\end{frame}


%
% Takeaway #5
%
\subsection*{Takeaway \#5}
\frameSubsectionTakeaway{}
\againframe<5>{takeaway}
}%
      \onslide*<6>{\providecommand\step{2}%
% Backtraces
%
\subsection{One advantage of raising with debugging information}
\frameSubsection{}{}

\begin{frame}{Backtraces}{Debugging information \& source code}
  \begin{columns}[c]
    \begin{column}{0.5\textwidth}
      \begin{itemize}
        \item Raising an exception is fun and games, but how do we know where the exception originated? $\rightarrow$ With the backtrace associated with the exception!
        \item In order to get a backtrace from the point where an exception was raised, it's necessary to compile with debugging information enabled (\funcarg{-g} flag for the compiler)
        \item Same example as in the `Nested handlers' section
      \end{itemize}
    \end{column}
    \begin{column}{0.5\textwidth}
      \listocaml[firstline=9,lastline=34]{../src/backtrace/backtrace.ml}{backtrace/backtrace.ml}
    \end{column}
  \end{columns}
\end{frame}

\begin{frame}{Backtraces}{CMM and disassembly of \funcname{definitely\_raise}}
  \begin{columns}[c]
    \begin{column}{0.5\textwidth}
      \minipage[c][0.66\textheight][s]{\columnwidth}
      \begin{itemize}
        \item<1-> An effect of compiling with debugging information (\funcarg{-g}): the CMM no longer uses \funcname{raise\_notrace} to raise the exception but \funcname{raise} instead
        \item<2-> Disassembly shows that CMM \funcname{raise} is actually a call to \funcname{caml\_raise\_exn}, a function in assembler provided by the runtime
      \end{itemize}
      \vfill
      \mintocaml[firstline=9,lastline=13,highlightlines=12]{../src/backtrace/backtrace.ml}
      \endminipage
    \end{column}
    \begin{column}{0.5\textwidth}
      \onslide*<1>{
        \mintcmm[highlightlines=5]{../src/backtrace/camlBacktrace\_\_definitely\_raise\_347.cmm}
      }
      \onslide*<2>{
        \mintobjdump[firstline=2,highlightlines=14]{../src/backtrace/camlBacktrace\_\_definitely\_raise\_347.objdump}
      }
    \end{column}
  \end{columns}
\end{frame}

\begin{frame}{Backtraces}{Introducing \funcname{caml\_raise\_exn}}
  \begin{columns}[c]
    \begin{column}{0.5\textwidth}
      \minipage[c][0.66\textheight][s]{\columnwidth}
      \onslide*<1>{
        \begin{itemize}
          \item Raising the exception is performed using \funcname{caml\_raise\_exn} which is
            \begin{itemize}
              \item An assembly function provided by the runtime
              \item Architecture specific
            \end{itemize}
          \item Let's see what it does differently from \funcname{raise\_notrace}\dots
          \item We are about to raise \typename{ExnC} with \funcname{caml\_raise\_exn} from \funcname{definitely\_raise}
        \end{itemize}
      }
      \onslide*<2>{
        \mintobjdump[firstline=2,highlightlines=14]{../src/backtrace/camlBacktrace\_\_definitely\_raise\_347.objdump}
      }
      \vfill
      \mintocaml[firstline=9,lastline=13,highlightlines=12]{../src/backtrace/backtrace.ml}
      \endminipage
    \end{column}
    \begin{column}{0.5\textwidth}
      \centering
      \providecommand\step{1}\input{diagrams/backtrace/backtrace.tex}
    \end{column}
  \end{columns}
\end{frame}

\begin{frame}{Backtraces}{But before that, we need more fields from \typename{caml\_domain\_state}}
  \begin{columns}
    \begin{column}{0.5\textwidth}
      \begin{itemize}
        \item Remember \typename{caml\_domain\_state} the per-domain runtime state?
        \item Some other members are of interest to us now\dots
        \item \localname{backtrace\_active} is used to know if the runtime should collect backtraces
        \item \localname{backtrace\_active} can be set by
          \begin{itemize}
            \item The \funcarg{b} flag in the \funcname{OCAMLRUNPARAM} environment variable
            \item \mintinline{ocaml}{Printexc.record_backtrace : bool -> unit} \footnotemark
          \end{itemize}
      \end{itemize}
    \end{column}
    \begin{column}{0.5\textwidth}
      \begin{listing}
        \mintc[highlightlines={9-11}]{src/ocaml/domain\_state\_backtrace.h}
        \caption{runtime/caml/domain\_state.h}
      \end{listing}
    \end{column}
  \end{columns}
  \footnotetext[1]{https://v2.ocaml.org/api/Printexc.html}
\end{frame}

\newcommand\listCamlRaiseExn[1][]{
  \listasm[firstline=702,lastline=725,#1]{../ocaml/runtime/amd64.S}{runtime/amd64.S:702}}

\begin{frame}{Backtraces}{Walk through \funcname{caml\_raise\_exn}}
  \begin{columns}[T]
    \begin{column}{0.5\textwidth}
      \begin{itemize}
        \item<1-> First checks whether exceptions backtrace should be recorded
        \item<1-> By checking the \funcarg{domain\_state->backtrace\_active} value
        \item<2-> If not, acts as \funcname{raise\_notrace} with \funcname{RESTORE\_EXN\_HANDLER\_OCAML}
          \begin{itemize}
            \item Set \regname{RSP} to \localname{domain\_state->exn\_handler} value
            \item Restore \localname{domain\_state->exn\_handler} from trap
            \item Jump with \funcname{ret} (same effect as using \funcname{pop} and \funcname{jmp})
          \end{itemize}
        \item<3-> Otherwise, switches to the C stack to be able to execute C code
        \item<4-> Calls \funcname{caml\_stash\_backtrace}, a C function provided by the runtime\ignorespaces
          \onslide<6->{, with
            \begin{itemize}
              \item \funcarg{exn}: exception value
              \item \funcarg{pc}: code address of the raising point
              \item \funcarg{sp}: stack address of the raising function
              \item \funcarg{trapsp}: stack address of the next handler
            \end{itemize}
          }
      \end{itemize}
      \bigskip
      \mintocaml[firstline=9,lastline=13,highlightlines=12]{../src/backtrace/backtrace.ml}
    \end{column}
    \begin{column}{0.5\textwidth}
      \raggedleft
      \onslide*<1,3-4>{
        \mintc[firstline=190,lastline=192]{../ocaml/runtime/amd64.S}
        \mintc[firstline=234,lastline=237]{../ocaml/runtime/amd64.S}
      }
      \onslide*<2>{
        \mintc[firstline=190,lastline=192,highlightlines={191-192}]{../ocaml/runtime/amd64.S}
        \mintc[firstline=234,lastline=237,highlightlines={235-237}]{../ocaml/runtime/amd64.S}
      }
      \onslide*<1>{\listCamlRaiseExn[highlightlines=705]{}}%
      \onslide*<2>{\listCamlRaiseExn[highlightlines={707-708}]{}}%
      \onslide*<3>{\listCamlRaiseExn[highlightlines={712-714}]{}}%
      \onslide*<4>{\listCamlRaiseExn[highlightlines={715-720}]{}}%
      \onslide*<5>{\providecommand\step{1}\input{diagrams/backtrace/backtrace.tex}}%
      \onslide*<6>{\providecommand\step{2}\input{diagrams/backtrace/backtrace.tex}}%
    \end{column}
  \end{columns}
\end{frame}

\newcommand\listStashBacktrace[1][]{
  \listc[firstline=91,lastline=120,#1]{../ocaml/runtime/backtrace\_nat.c}{runtime/backtrace\_nat.c:91}}

\begin{frame}{Backtraces}{Introducing \funcname{caml\_stash\_backtrace}}
  \begin{columns}[c]
    \begin{column}{0.5\textwidth}
      \minipage[c][0.66\textheight][s]{\columnwidth}
      \begin{itemize}
        \item<1-> A C function provided by the runtime (specific to native code)
        \item<2-> It scans the stack, between the stack pointer at the raising point up to the next trap, in order to collect the names of the detected function frames
          \onslide*<2>{
            \begin{itemize}
              \item And more information, like line number, column number...
            \end{itemize}
          }
        \item<3-> It uses the same technique and information as the Garbage Collector (GC) uses to scan the values live on the stack
          \onslide*<4>{
            \begin{itemize}
              \item \typename{frame\_descr} are at the core of stack unwinding
            \end{itemize}
          }
      \end{itemize}
      \vfill
      \mintocaml[firstline=9,lastline=13,highlightlines=12]{../src/backtrace/backtrace.ml}
      \endminipage
    \end{column}
    \begin{column}{0.5\textwidth}
      \onslide*<1>{\listStashBacktrace[highlightlines=91]{}}
      \onslide*<2>{\listStashBacktrace[highlightlines={91,117-118}]{}}
      \onslide*<3->{\listStashBacktrace[highlightlines={94,106,109-110}]{}}
    \end{column}
  \end{columns}
\end{frame}

\newcommand\listFrameDescr[1][]{
  \listocaml[firstline=111,lastline=124,#1]{../ocaml/asmcomp/emitaux.ml}{asmcomp/emitaux.ml:111}}
\newcommand\listDebugInfo[1][]{
  \listocaml[firstline=38,lastline=49,#1]{../ocaml/lambda/debuginfo.mli}{lambda/debuginfo.mli:38}}

\begin{frame}{Backtraces}{\typename{frame\_descr} and \typename{frame\_debuginfo} in a nutshell}
  \begin{columns}[c]
    \begin{column}{0.5\textwidth}
      \begin{itemize}
        \item<1-> For every return address in an OCaml program, there is a associated \typename{frame\_descr}
        \item<2-> A \typename{frame\_descr} specifies
        \begin{itemize}
          \item<2-> The size of the function stack frame
          \item<3-> Where live values are on the stack (so that the GC can mark them as live and prevent from being swept)
        \end{itemize}
        \item<4-> When compiled with debug information, \typename{frame\_descr} will be linked to a \typename{frame\_debuginfo}
        \begin{itemize}
          \item<5-> Contains information about the position in the source code (file name, line and column number)
        \end{itemize}
      \end{itemize}
    \end{column}
    \begin{column}{0.5\textwidth}
        \onslide*<1>{\listFrameDescr[highlightlines={118-122}]{}}
        \onslide*<2>{\listFrameDescr[highlightlines={120}]{}}
        \onslide*<3>{\listFrameDescr[highlightlines={121}]{}}
        \onslide*<4->{\listFrameDescr[highlightlines={122}]{}}
        \medskip
        \onslide*<1-3>{\listDebugInfo{}}
        \onslide*<4>{\listDebugInfo[highlightlines={38,49}]{}}
        \onslide*<5>{\listDebugInfo[highlightlines={39-46}]{}}
    \end{column}
  \end{columns}
\end{frame}

\begin{frame}{Backtraces}{Scanning the stack with \typename{frame\_descr}}
  \begin{columns}[c]
    \begin{column}{0.5\textwidth}
      \begin{itemize}
        \item Given the return address of the code that raises the exception by calling \funcname{caml\_raise\_exn} (\funcarg{pc} argument of \funcname{caml\_stash\_backtrace})
        \item And the position of the stack pointer (\funcarg{sp} argument of \funcname{caml\_stash\_backtrace})
        \item The frame size given by \typename{frame\_descr}.\funcarg{frame\_size}, allows to find the end of the previous frame
        \item And the return address of the caller of this function
        \bigskip
        \onslide<3->{
        \item Note that traps (here the reddish \funcname{ExnA}, \funcname{ExnB} and \funcname{ExnC} blocks) belong to the frame that installed them, and so the frame size changes when traps are removed
        }
      \end{itemize}
    \end{column}
    \begin{column}{0.5\textwidth}
      \raggedleft
      \onslide*<1>{\providecommand\step{2}\input{diagrams/backtrace/backtrace.tex}}%
      \onslide*<2>{\providecommand\step{1}\input{diagrams/backtrace/scanstack.tex}}%
      \onslide*<3>{\providecommand\step{2}\input{diagrams/backtrace/scanstack.tex}}%
      \onslide*<4>{\providecommand\step{3}\input{diagrams/backtrace/scanstack.tex}}%
      \onslide*<5>{\providecommand\step{4}\input{diagrams/backtrace/scanstack.tex}}%
      \onslide*<6>{\providecommand\step{5}\input{diagrams/backtrace/scanstack.tex}}%
    \end{column}
  \end{columns}
\end{frame}

% Duplicate of previous-previous slide
\begin{frame}{Backtraces}{Continuing on \funcname{caml\_stash\_backtrace}}
  \begin{columns}[c]
    \begin{column}{0.5\textwidth}
      \minipage[c][0.75\textheight][s]{\columnwidth}
      \begin{itemize}
          \color{gray}
        \item A C function provided by the runtime (specific to native code)
        \item It scans the stack, between the stack pointer at the raising point up to the next trap, in order to collect the names of the detected function frames
        \item It uses the same technique and information as the Garbage Collector (GC) uses to scan the values live on the stack
      \end{itemize}
      \begin{itemize}
        \item For each function frame detected, store a pointer to the \typename{frame\_descr} in the \localname{domain\_state->backtrace\_buffer} array, effectively constructing the backtrace
      \end{itemize}
      \onslide<2->{
        \begin{itemize}
            \smallskip
          \item Constructed backtrace:
            \funcname{definitely\_raise},
            \onslide<3->{\funcname{probably\_raise}}
        \end{itemize}
      }
      \vfill
      \mintocaml[firstline=9,lastline=13,highlightlines=12]{../src/backtrace/backtrace.ml}
      \endminipage
    \end{column}
    \begin{column}{0.5\textwidth}
      \raggedleft
      \onslide*<1>{\listStashBacktrace[highlightlines={114-115}]{}}
      \onslide*<2>{\providecommand\step{3}\input{diagrams/backtrace/backtrace.tex}}%
      \onslide*<3>{\providecommand\step{4}\input{diagrams/backtrace/backtrace.tex}}%
    \end{column}
  \end{columns}
\end{frame}

\begin{frame}{Backtraces}{Back to \funcname{caml\_raise\_exn}}
  \begin{columns}[c]
    \begin{column}{0.5\textwidth}
      \minipage[c][0.66\textheight][s]{\columnwidth}
      \begin{itemize}\color{gray}
        \item Checks wheteher exceptions backtrace should be recorded, by checking the \funcarg{domain\_state->backtrace\_active} value
        \item Switches to the C stack to be able to execute C code
        \item Calls \funcname{caml\_stash\_backtrace}, to collect the backtrace up to the next trap
      \end{itemize}
      \onslide*<2->{
        \begin{itemize}
          \item<2-> Executes the exception handler in \funcname{probably\_raise} with \funcname{RESTORE\_EXN\_HANDLER\_OCAML}
            \begin{itemize}
              \item Set \regname{RSP} to \localname{domain\_state->exn\_handler} value
              \item Restore \localname{domain\_state->exn\_handler} from trap
              \item Jump to the handler code with \funcname{ret}
            \end{itemize}
        \end{itemize}
      }
      \vfill
      \onslide*<1>{\mintocaml[firstline=9,lastline=13,highlightlines=12]{../src/backtrace/backtrace.ml}}
      \onslide*<2->{\mintocaml[firstline=15,lastline=21,highlightlines={19-21}]{../src/backtrace/backtrace.ml}}
      \endminipage
    \end{column}
    \begin{column}{0.5\textwidth}
      \centering
      \onslide*<1>{
        \mintc[firstline=190,lastline=192]{../ocaml/runtime/amd64.S}
        \mintc[firstline=234,lastline=237]{../ocaml/runtime/amd64.S}
      }
      \onslide*<2>{
        \mintc[firstline=190,lastline=192,highlightlines={191-192}]{../ocaml/runtime/amd64.S}
        \mintc[firstline=234,lastline=237,highlightlines={235-237}]{../ocaml/runtime/amd64.S}
      }
      \onslide*<1>{\listCamlRaiseExn[highlightlines=720]{}}
      \onslide*<2>{\listCamlRaiseExn[highlightlines=722]{}}
      \onslide*<3->{\providecommand\step{5}\input{diagrams/backtrace/backtrace.tex}}
    \end{column}
  \end{columns}
\end{frame}

\begin{frame}{Backtraces}{\funcname{caml\_probably\_raise}'s handler doesn't catch \typename{ExnC}}
  \begin{columns}[c]
    \begin{column}{0.45\textwidth}
      \minipage[c][0.75\textheight][s]{\columnwidth}
      \begin{itemize}
        \item<1-> \funcname{match \dots with}\footnotemark pattern matching can also match exceptions
        \item<1-> The CMM of \funcname{probably\_raise} shows that this \funcname{match \dots with} is implemented with a pattern matching and a \funcname{try \dots with}
        \item<1-> But this one only matches on \typename{ExnA} exception
        \item<2-> Since the pattern matching fails to catch the exception, \funcname{reraise} is called with our current \typename{ExnC} exception
        \item<3-> Disassembly shows that \funcname{reraise} is actually a call to \funcname{caml\_reraise\_exn}, which is also an assembly function provided by the runtime
      \end{itemize}
      \vfill
      \mintocaml[firstline=15,lastline=21,highlightlines={18-21}]{../src/backtrace/backtrace.ml}
      \endminipage
    \end{column}
    \begin{column}{0.55\textwidth}
      \centering
      \onslide*<1>{\mintcmm[highlightlines={10-18}]{../src/backtrace/camlBacktrace\_\_probably\_raise\_351.cmm}}
      \onslide*<2>{\listcmm[highlightlines=18]{../src/backtrace/camlBacktrace\_\_probably\_raise_351.cmm}{CMM of probably\_raise}}
      \onslide*<3>{\mintobjdump[firstline=5,lastline=35,highlightlines=31]{../src/backtrace/camlBacktrace\_\_probably\_raise_351.objdump}}
    \end{column}
  \end{columns}
  \only<1>{\footnotetext[1]{https://blog.janestreet.com/pattern-matching-and-exception-handling-unite/}}
\end{frame}

\newcommand\listCamlReraiseExn[1][]{
  \listasm[firstline=727,lastline=734,#1]{../ocaml/runtime/amd64.S}{runtime/amd64.S:727}}

\begin{frame}{Backtraces}{Introducing \funcname{caml\_reraise\_exn}}
  \begin{columns}[c]
    \begin{column}{0.5\textwidth}
      \minipage[c][0.66\textheight][s]{\columnwidth}
      \begin{itemize}
        \item<1-> The exception have to be reraised to the next handler
        \item<2-> First, checks if the backtrace should be collected with \funcarg{domain\_state->backtrace\_active}
        \only<3>{
        \begin{itemize}
            \item If not, behaves like a \funcname{raise\_notrace} with \funcname{RESTORE\_EXN\_HANDLER\_OCAML}
        \end{itemize}
        }
        \item<4-> Jumps into \funcname{caml\_raise\_exn}
        \item<5-> Calls \funcname{caml\_stash\_backtrace} again, to scan the next part of the stack: from the trap just pop-ed to the next trap
      \end{itemize}
      \vfill
      \mintocaml[firstline=15,lastline=21,highlightlines={18-21}]{../src/backtrace/backtrace.ml}
      \endminipage
    \end{column}
    \begin{column}{0.5\textwidth}
      \raggedleft
      \onslide*<1>{\listCamlReraiseExn{}}
      \onslide*<2>{\listCamlReraiseExn[highlightlines=729]{}}
      \onslide*<3>{
        \mintc[firstline=190,lastline=192,highlightlines={191-192}]{../ocaml/runtime/amd64.S}
        \mintc[firstline=234,lastline=237,highlightlines={235-237}]{../ocaml/runtime/amd64.S}
        \medskip
        \listCamlReraiseExn[highlightlines=731]{}
      }
      \onslide*<4-5>{
        % TODO refactor with \listCamlReraiseExn
        \mintasm[firstline=727,lastline=734,highlightlines=730]{../ocaml/runtime/amd64.S}
        \medskip
      }
      \onslide*<4>{\listCamlRaiseExn[highlightlines=711]{}}
      \onslide*<5>{\listCamlRaiseExn[highlightlines=720]{}}
    \end{column}
  \end{columns}
\end{frame}

\begin{frame}{Backtraces}{Backtrace collection continues}
  \begin{columns}[c]
    \begin{column}{0.55\textwidth}
      \minipage[c][0.75\textheight][s]{\columnwidth}
      \begin{itemize}
        \item<1-> \funcname{caml\_stash\_backtrace} scans the older part of the stack, up to the next trap
        \item<6-> Controls jumps to the nested exception handler in \funcname{maybe\_raise}, which matches only on \typename{ExnB}
        \item<7-> \funcname{caml\_reraise\_exn} is called again, backtrace collection resumes with \funcname{caml\_stash\_backtrace}
      \end{itemize}
      \medskip
      \begin{itemize}
        \item<2-> Constructed backtrace:
        \begin{itemize}
          \item <2-> \funcname{definitely\_raise}
          \item <2-> \funcname{probably\_raise}
          \item <3-> \funcname{probably\_raise} (re-raise)
          \item <4-> \funcname{maybe\_raise}
          \item <5-> \funcname{main}
          \item <8-> \funcname{main} (re-raise)
        \end{itemize}
      \end{itemize}
      \vfill
      \onslide*<1-5>{\mintocaml[firstline=15,lastline=21]{../src/backtrace/backtrace.ml}}
      \onslide*<6>{\mintocaml[firstline=28,lastline=34,highlightlines=31]{../src/backtrace/backtrace.ml}}
      \onslide*<7->{\mintocaml[firstline=28,lastline=34,highlightlines=33]{../src/backtrace/backtrace.ml}}
      \endminipage
    \end{column}
    \begin{column}{0.45\textwidth}
      \raggedleft
      \onslide*<1>{\providecommand\step{6}\input{diagrams/backtrace/backtrace.tex}}%
      \onslide*<2>{\providecommand\step{6}\input{diagrams/backtrace/backtrace.tex}}%
      \onslide*<3>{\providecommand\step{7}\input{diagrams/backtrace/backtrace.tex}}%
      \onslide*<4>{\providecommand\step{8}\input{diagrams/backtrace/backtrace.tex}}%
      \onslide*<5>{\providecommand\step{9}\input{diagrams/backtrace/backtrace.tex}}%
      \onslide*<6>{\providecommand\step{10}\input{diagrams/backtrace/backtrace.tex}}%
      \onslide*<7>{\providecommand\step{11}\input{diagrams/backtrace/backtrace.tex}}%
      \onslide*<8>{\providecommand\step{12}\input{diagrams/backtrace/backtrace.tex}}%
      \onslide*<9>{\providecommand\step{13}\input{diagrams/backtrace/backtrace.tex}}%
    \end{column}
  \end{columns}
\end{frame}

\begin{frame}{Backtraces}{Printing the backtrace}
  \begin{columns}[c]
    \begin{column}{0.45\textwidth}
      \minipage[c][0.66\textheight][s]{\columnwidth}
      \begin{itemize}
        \item<1-> The exception backtrace can now be printed to \funcarg{stdout} with \funcname{Printexc.print_backtrace}
        \item<2-> First the exception backtrace is retrieved into an OCaml array with \funcname{get_raw_backtrace }
        \item<3-> Which is implemented as the \funcname{caml_get_exception_raw_backtrace} C function in the runtime
        \item<4-> And finally printed with \funcname{print_exception_backtrace}
      \end{itemize}
      \vfill
      \mintocaml[firstline=28,lastline=34,highlightlines=33]{../src/backtrace/backtrace.ml}
      \endminipage
    \end{column}
    \begin{column}{0.55\textwidth}
      \onslide*<1,2>{
        \mintocaml[firstline=100,lastline=101]{../ocaml/stdlib/printexc.ml}
      }
      \only<1>{
        \listocaml[firstline=170,lastline=172]{../ocaml/stdlib/printexc.ml}{stdlib/printexc.ml:170}
      }
      \only<2>{
        \listocaml[firstline=170,lastline=172,highlightlines=172]{../ocaml/stdlib/printexc.ml}{stdlib/printexc.ml:170}
      }
      \only<3>{
        \mintc[firstline=154,lastline=156]{../ocaml/runtime/backtrace.c}
        \listc[firstline=171,lastline=192,highlightlines={184-187}]{../ocaml/runtime/backtrace.c}{runtime/backtrace.c:154}
      }
      \only<4>{
        \listocaml[firstline=155,lastline=165]{../ocaml/stdlib/printexc.ml}{stdlib/printexc.ml:55}
      }
    \end{column}
  \end{columns}
\end{frame}


\subsection{The multiple kinds of raise}
\frameSubsection{}{}

\begin{frame}{Backtraces}{When backtraces are not needed}
  \begin{columns}[c]
    \begin{column}{0.45\textwidth}
      \minipage[c][0.66\textheight][s]{\columnwidth}
      \begin{itemize}
        \item<1-> What if the backtrace for an exception is never needed?
        \item<2-> It's possible to raise the exception explicitly with \funcname{raise\_notrace} to make raising as cheap as possible
        \item<3-> An example of use in the \funcname{dynlink} library of the OCaml compiler
      \end{itemize}
      \vfill
      \onslide<3->{
      \listocaml[firstline=205,lastline=209,highlightlines=207]{../ocaml/otherlibs/dynlink/dynlink\_compilerlibs/misc.ml}{otherlibs/dynlink/dynlink\_compilerlibs/misc.ml:205}
      }
      \endminipage
    \end{column}
    \begin{column}{0.55\textwidth}
    \only<4>{
      \mintcmm[highlightlines=3]{../src/notrace/camlNotrace\_\_fun_347.cmm}
      \listcmm{../src/notrace/camlNotrace\_\_all\_somes\_327.cmm}{all\_somes}
    }
    \onslide*<5->{
      \listobjdump[highlightlines={6-9}]{../src/notrace/camlNotrace\_\_fun_347.objdump}{anonymous function from \funcname{all\_somes}}
    }
    \end{column}
  \end{columns}
\end{frame}

\begin{frame}{Backtraces}{Summary table of the multiple kinds of raise}
  \begin{itemize}
    \item When debugging information is enabled and if the backtrace of an exception isn't needed, it's possible to raise with \funcname{raise\_notrace} for performance
    \item When debugging information is \emph{not} enabled, the compiler optimises \funcname{raise} calls to inline assembly (same effect as using \funcname{raise\_notrace})
  \end{itemize}
  \bigskip
  \centering
  \begin{tabular}{| l | c | c | c | c |}
    \hline
    \parbox{10em}{Debug information \\ (\funcarg{-g} flag)}  & \multicolumn{2}{|c|}{Disabled}                                     & \multicolumn{2}{|c|}{Enabled} \\
    \hline
    Raise kind                                               & \funcname{raise} & \funcname{raise\_notrace}                       & \funcname{raise} & \funcname{raise\_notrace} \\
    \hline
    Compilation                                              & \parbox{8em}{inline assembly\\ (optimization)} & inline assembly   & \funcname{caml\_raise\_exn} & inline assembly \\
    \hline
  \end{tabular}
\end{frame}


%
% Takeaway #5
%
\subsection*{Takeaway \#5}
\frameSubsectionTakeaway{}
\againframe<5>{takeaway}
}%
    \end{column}
  \end{columns}
\end{frame}

\newcommand\listStashBacktrace[1][]{
  \listc[firstline=91,lastline=120,#1]{../ocaml/runtime/backtrace\_nat.c}{runtime/backtrace\_nat.c:91}}

\begin{frame}{Backtraces}{Introducing \funcname{caml\_stash\_backtrace}}
  \begin{columns}[c]
    \begin{column}{0.5\textwidth}
      \minipage[c][0.66\textheight][s]{\columnwidth}
      \begin{itemize}
        \item<1-> A C function provided by the runtime (specific to native code)
        \item<2-> It scans the stack, between the stack pointer at the raising point up to the next trap, in order to collect the names of the detected function frames
          \onslide*<2>{
            \begin{itemize}
              \item And more information, like line number, column number...
            \end{itemize}
          }
        \item<3-> It uses the same technique and information as the Garbage Collector (GC) uses to scan the values live on the stack
          \onslide*<4>{
            \begin{itemize}
              \item \typename{frame\_descr} are at the core of stack unwinding
            \end{itemize}
          }
      \end{itemize}
      \vfill
      \mintocaml[firstline=9,lastline=13,highlightlines=12]{../src/backtrace/backtrace.ml}
      \endminipage
    \end{column}
    \begin{column}{0.5\textwidth}
      \onslide*<1>{\listStashBacktrace[highlightlines=91]{}}
      \onslide*<2>{\listStashBacktrace[highlightlines={91,117-118}]{}}
      \onslide*<3->{\listStashBacktrace[highlightlines={94,106,109-110}]{}}
    \end{column}
  \end{columns}
\end{frame}

\newcommand\listFrameDescr[1][]{
  \listocaml[firstline=111,lastline=124,#1]{../ocaml/asmcomp/emitaux.ml}{asmcomp/emitaux.ml:111}}
\newcommand\listDebugInfo[1][]{
  \listocaml[firstline=38,lastline=49,#1]{../ocaml/lambda/debuginfo.mli}{lambda/debuginfo.mli:38}}

\begin{frame}{Backtraces}{\typename{frame\_descr} and \typename{frame\_debuginfo} in a nutshell}
  \begin{columns}[c]
    \begin{column}{0.5\textwidth}
      \begin{itemize}
        \item<1-> For every return address in an OCaml program, there is a associated \typename{frame\_descr}
        \item<2-> A \typename{frame\_descr} specifies
        \begin{itemize}
          \item<2-> The size of the function stack frame
          \item<3-> Where live values are on the stack (so that the GC can mark them as live and prevent from being swept)
        \end{itemize}
        \item<4-> When compiled with debug information, \typename{frame\_descr} will be linked to a \typename{frame\_debuginfo}
        \begin{itemize}
          \item<5-> Contains information about the position in the source code (file name, line and column number)
        \end{itemize}
      \end{itemize}
    \end{column}
    \begin{column}{0.5\textwidth}
        \onslide*<1>{\listFrameDescr[highlightlines={118-122}]{}}
        \onslide*<2>{\listFrameDescr[highlightlines={120}]{}}
        \onslide*<3>{\listFrameDescr[highlightlines={121}]{}}
        \onslide*<4->{\listFrameDescr[highlightlines={122}]{}}
        \medskip
        \onslide*<1-3>{\listDebugInfo{}}
        \onslide*<4>{\listDebugInfo[highlightlines={38,49}]{}}
        \onslide*<5>{\listDebugInfo[highlightlines={39-46}]{}}
    \end{column}
  \end{columns}
\end{frame}

\begin{frame}{Backtraces}{Scanning the stack with \typename{frame\_descr}}
  \begin{columns}[c]
    \begin{column}{0.5\textwidth}
      \begin{itemize}
        \item Given the return address of the code that raises the exception by calling \funcname{caml\_raise\_exn} (\funcarg{pc} argument of \funcname{caml\_stash\_backtrace})
        \item And the position of the stack pointer (\funcarg{sp} argument of \funcname{caml\_stash\_backtrace})
        \item The frame size given by \typename{frame\_descr}.\funcarg{frame\_size}, allows to find the end of the previous frame
        \item And the return address of the caller of this function
        \bigskip
        \onslide<3->{
        \item Note that traps (here the reddish \funcname{ExnA}, \funcname{ExnB} and \funcname{ExnC} blocks) belong to the frame that installed them, and so the frame size changes when traps are removed
        }
      \end{itemize}
    \end{column}
    \begin{column}{0.5\textwidth}
      \raggedleft
      \onslide*<1>{\providecommand\step{2}%
% Backtraces
%
\subsection{One advantage of raising with debugging information}
\frameSubsection{}{}

\begin{frame}{Backtraces}{Debugging information \& source code}
  \begin{columns}[c]
    \begin{column}{0.5\textwidth}
      \begin{itemize}
        \item Raising an exception is fun and games, but how do we know where the exception originated? $\rightarrow$ With the backtrace associated with the exception!
        \item In order to get a backtrace from the point where an exception was raised, it's necessary to compile with debugging information enabled (\funcarg{-g} flag for the compiler)
        \item Same example as in the `Nested handlers' section
      \end{itemize}
    \end{column}
    \begin{column}{0.5\textwidth}
      \listocaml[firstline=9,lastline=34]{../src/backtrace/backtrace.ml}{backtrace/backtrace.ml}
    \end{column}
  \end{columns}
\end{frame}

\begin{frame}{Backtraces}{CMM and disassembly of \funcname{definitely\_raise}}
  \begin{columns}[c]
    \begin{column}{0.5\textwidth}
      \minipage[c][0.66\textheight][s]{\columnwidth}
      \begin{itemize}
        \item<1-> An effect of compiling with debugging information (\funcarg{-g}): the CMM no longer uses \funcname{raise\_notrace} to raise the exception but \funcname{raise} instead
        \item<2-> Disassembly shows that CMM \funcname{raise} is actually a call to \funcname{caml\_raise\_exn}, a function in assembler provided by the runtime
      \end{itemize}
      \vfill
      \mintocaml[firstline=9,lastline=13,highlightlines=12]{../src/backtrace/backtrace.ml}
      \endminipage
    \end{column}
    \begin{column}{0.5\textwidth}
      \onslide*<1>{
        \mintcmm[highlightlines=5]{../src/backtrace/camlBacktrace\_\_definitely\_raise\_347.cmm}
      }
      \onslide*<2>{
        \mintobjdump[firstline=2,highlightlines=14]{../src/backtrace/camlBacktrace\_\_definitely\_raise\_347.objdump}
      }
    \end{column}
  \end{columns}
\end{frame}

\begin{frame}{Backtraces}{Introducing \funcname{caml\_raise\_exn}}
  \begin{columns}[c]
    \begin{column}{0.5\textwidth}
      \minipage[c][0.66\textheight][s]{\columnwidth}
      \onslide*<1>{
        \begin{itemize}
          \item Raising the exception is performed using \funcname{caml\_raise\_exn} which is
            \begin{itemize}
              \item An assembly function provided by the runtime
              \item Architecture specific
            \end{itemize}
          \item Let's see what it does differently from \funcname{raise\_notrace}\dots
          \item We are about to raise \typename{ExnC} with \funcname{caml\_raise\_exn} from \funcname{definitely\_raise}
        \end{itemize}
      }
      \onslide*<2>{
        \mintobjdump[firstline=2,highlightlines=14]{../src/backtrace/camlBacktrace\_\_definitely\_raise\_347.objdump}
      }
      \vfill
      \mintocaml[firstline=9,lastline=13,highlightlines=12]{../src/backtrace/backtrace.ml}
      \endminipage
    \end{column}
    \begin{column}{0.5\textwidth}
      \centering
      \providecommand\step{1}\input{diagrams/backtrace/backtrace.tex}
    \end{column}
  \end{columns}
\end{frame}

\begin{frame}{Backtraces}{But before that, we need more fields from \typename{caml\_domain\_state}}
  \begin{columns}
    \begin{column}{0.5\textwidth}
      \begin{itemize}
        \item Remember \typename{caml\_domain\_state} the per-domain runtime state?
        \item Some other members are of interest to us now\dots
        \item \localname{backtrace\_active} is used to know if the runtime should collect backtraces
        \item \localname{backtrace\_active} can be set by
          \begin{itemize}
            \item The \funcarg{b} flag in the \funcname{OCAMLRUNPARAM} environment variable
            \item \mintinline{ocaml}{Printexc.record_backtrace : bool -> unit} \footnotemark
          \end{itemize}
      \end{itemize}
    \end{column}
    \begin{column}{0.5\textwidth}
      \begin{listing}
        \mintc[highlightlines={9-11}]{src/ocaml/domain\_state\_backtrace.h}
        \caption{runtime/caml/domain\_state.h}
      \end{listing}
    \end{column}
  \end{columns}
  \footnotetext[1]{https://v2.ocaml.org/api/Printexc.html}
\end{frame}

\newcommand\listCamlRaiseExn[1][]{
  \listasm[firstline=702,lastline=725,#1]{../ocaml/runtime/amd64.S}{runtime/amd64.S:702}}

\begin{frame}{Backtraces}{Walk through \funcname{caml\_raise\_exn}}
  \begin{columns}[T]
    \begin{column}{0.5\textwidth}
      \begin{itemize}
        \item<1-> First checks whether exceptions backtrace should be recorded
        \item<1-> By checking the \funcarg{domain\_state->backtrace\_active} value
        \item<2-> If not, acts as \funcname{raise\_notrace} with \funcname{RESTORE\_EXN\_HANDLER\_OCAML}
          \begin{itemize}
            \item Set \regname{RSP} to \localname{domain\_state->exn\_handler} value
            \item Restore \localname{domain\_state->exn\_handler} from trap
            \item Jump with \funcname{ret} (same effect as using \funcname{pop} and \funcname{jmp})
          \end{itemize}
        \item<3-> Otherwise, switches to the C stack to be able to execute C code
        \item<4-> Calls \funcname{caml\_stash\_backtrace}, a C function provided by the runtime\ignorespaces
          \onslide<6->{, with
            \begin{itemize}
              \item \funcarg{exn}: exception value
              \item \funcarg{pc}: code address of the raising point
              \item \funcarg{sp}: stack address of the raising function
              \item \funcarg{trapsp}: stack address of the next handler
            \end{itemize}
          }
      \end{itemize}
      \bigskip
      \mintocaml[firstline=9,lastline=13,highlightlines=12]{../src/backtrace/backtrace.ml}
    \end{column}
    \begin{column}{0.5\textwidth}
      \raggedleft
      \onslide*<1,3-4>{
        \mintc[firstline=190,lastline=192]{../ocaml/runtime/amd64.S}
        \mintc[firstline=234,lastline=237]{../ocaml/runtime/amd64.S}
      }
      \onslide*<2>{
        \mintc[firstline=190,lastline=192,highlightlines={191-192}]{../ocaml/runtime/amd64.S}
        \mintc[firstline=234,lastline=237,highlightlines={235-237}]{../ocaml/runtime/amd64.S}
      }
      \onslide*<1>{\listCamlRaiseExn[highlightlines=705]{}}%
      \onslide*<2>{\listCamlRaiseExn[highlightlines={707-708}]{}}%
      \onslide*<3>{\listCamlRaiseExn[highlightlines={712-714}]{}}%
      \onslide*<4>{\listCamlRaiseExn[highlightlines={715-720}]{}}%
      \onslide*<5>{\providecommand\step{1}\input{diagrams/backtrace/backtrace.tex}}%
      \onslide*<6>{\providecommand\step{2}\input{diagrams/backtrace/backtrace.tex}}%
    \end{column}
  \end{columns}
\end{frame}

\newcommand\listStashBacktrace[1][]{
  \listc[firstline=91,lastline=120,#1]{../ocaml/runtime/backtrace\_nat.c}{runtime/backtrace\_nat.c:91}}

\begin{frame}{Backtraces}{Introducing \funcname{caml\_stash\_backtrace}}
  \begin{columns}[c]
    \begin{column}{0.5\textwidth}
      \minipage[c][0.66\textheight][s]{\columnwidth}
      \begin{itemize}
        \item<1-> A C function provided by the runtime (specific to native code)
        \item<2-> It scans the stack, between the stack pointer at the raising point up to the next trap, in order to collect the names of the detected function frames
          \onslide*<2>{
            \begin{itemize}
              \item And more information, like line number, column number...
            \end{itemize}
          }
        \item<3-> It uses the same technique and information as the Garbage Collector (GC) uses to scan the values live on the stack
          \onslide*<4>{
            \begin{itemize}
              \item \typename{frame\_descr} are at the core of stack unwinding
            \end{itemize}
          }
      \end{itemize}
      \vfill
      \mintocaml[firstline=9,lastline=13,highlightlines=12]{../src/backtrace/backtrace.ml}
      \endminipage
    \end{column}
    \begin{column}{0.5\textwidth}
      \onslide*<1>{\listStashBacktrace[highlightlines=91]{}}
      \onslide*<2>{\listStashBacktrace[highlightlines={91,117-118}]{}}
      \onslide*<3->{\listStashBacktrace[highlightlines={94,106,109-110}]{}}
    \end{column}
  \end{columns}
\end{frame}

\newcommand\listFrameDescr[1][]{
  \listocaml[firstline=111,lastline=124,#1]{../ocaml/asmcomp/emitaux.ml}{asmcomp/emitaux.ml:111}}
\newcommand\listDebugInfo[1][]{
  \listocaml[firstline=38,lastline=49,#1]{../ocaml/lambda/debuginfo.mli}{lambda/debuginfo.mli:38}}

\begin{frame}{Backtraces}{\typename{frame\_descr} and \typename{frame\_debuginfo} in a nutshell}
  \begin{columns}[c]
    \begin{column}{0.5\textwidth}
      \begin{itemize}
        \item<1-> For every return address in an OCaml program, there is a associated \typename{frame\_descr}
        \item<2-> A \typename{frame\_descr} specifies
        \begin{itemize}
          \item<2-> The size of the function stack frame
          \item<3-> Where live values are on the stack (so that the GC can mark them as live and prevent from being swept)
        \end{itemize}
        \item<4-> When compiled with debug information, \typename{frame\_descr} will be linked to a \typename{frame\_debuginfo}
        \begin{itemize}
          \item<5-> Contains information about the position in the source code (file name, line and column number)
        \end{itemize}
      \end{itemize}
    \end{column}
    \begin{column}{0.5\textwidth}
        \onslide*<1>{\listFrameDescr[highlightlines={118-122}]{}}
        \onslide*<2>{\listFrameDescr[highlightlines={120}]{}}
        \onslide*<3>{\listFrameDescr[highlightlines={121}]{}}
        \onslide*<4->{\listFrameDescr[highlightlines={122}]{}}
        \medskip
        \onslide*<1-3>{\listDebugInfo{}}
        \onslide*<4>{\listDebugInfo[highlightlines={38,49}]{}}
        \onslide*<5>{\listDebugInfo[highlightlines={39-46}]{}}
    \end{column}
  \end{columns}
\end{frame}

\begin{frame}{Backtraces}{Scanning the stack with \typename{frame\_descr}}
  \begin{columns}[c]
    \begin{column}{0.5\textwidth}
      \begin{itemize}
        \item Given the return address of the code that raises the exception by calling \funcname{caml\_raise\_exn} (\funcarg{pc} argument of \funcname{caml\_stash\_backtrace})
        \item And the position of the stack pointer (\funcarg{sp} argument of \funcname{caml\_stash\_backtrace})
        \item The frame size given by \typename{frame\_descr}.\funcarg{frame\_size}, allows to find the end of the previous frame
        \item And the return address of the caller of this function
        \bigskip
        \onslide<3->{
        \item Note that traps (here the reddish \funcname{ExnA}, \funcname{ExnB} and \funcname{ExnC} blocks) belong to the frame that installed them, and so the frame size changes when traps are removed
        }
      \end{itemize}
    \end{column}
    \begin{column}{0.5\textwidth}
      \raggedleft
      \onslide*<1>{\providecommand\step{2}\input{diagrams/backtrace/backtrace.tex}}%
      \onslide*<2>{\providecommand\step{1}\input{diagrams/backtrace/scanstack.tex}}%
      \onslide*<3>{\providecommand\step{2}\input{diagrams/backtrace/scanstack.tex}}%
      \onslide*<4>{\providecommand\step{3}\input{diagrams/backtrace/scanstack.tex}}%
      \onslide*<5>{\providecommand\step{4}\input{diagrams/backtrace/scanstack.tex}}%
      \onslide*<6>{\providecommand\step{5}\input{diagrams/backtrace/scanstack.tex}}%
    \end{column}
  \end{columns}
\end{frame}

% Duplicate of previous-previous slide
\begin{frame}{Backtraces}{Continuing on \funcname{caml\_stash\_backtrace}}
  \begin{columns}[c]
    \begin{column}{0.5\textwidth}
      \minipage[c][0.75\textheight][s]{\columnwidth}
      \begin{itemize}
          \color{gray}
        \item A C function provided by the runtime (specific to native code)
        \item It scans the stack, between the stack pointer at the raising point up to the next trap, in order to collect the names of the detected function frames
        \item It uses the same technique and information as the Garbage Collector (GC) uses to scan the values live on the stack
      \end{itemize}
      \begin{itemize}
        \item For each function frame detected, store a pointer to the \typename{frame\_descr} in the \localname{domain\_state->backtrace\_buffer} array, effectively constructing the backtrace
      \end{itemize}
      \onslide<2->{
        \begin{itemize}
            \smallskip
          \item Constructed backtrace:
            \funcname{definitely\_raise},
            \onslide<3->{\funcname{probably\_raise}}
        \end{itemize}
      }
      \vfill
      \mintocaml[firstline=9,lastline=13,highlightlines=12]{../src/backtrace/backtrace.ml}
      \endminipage
    \end{column}
    \begin{column}{0.5\textwidth}
      \raggedleft
      \onslide*<1>{\listStashBacktrace[highlightlines={114-115}]{}}
      \onslide*<2>{\providecommand\step{3}\input{diagrams/backtrace/backtrace.tex}}%
      \onslide*<3>{\providecommand\step{4}\input{diagrams/backtrace/backtrace.tex}}%
    \end{column}
  \end{columns}
\end{frame}

\begin{frame}{Backtraces}{Back to \funcname{caml\_raise\_exn}}
  \begin{columns}[c]
    \begin{column}{0.5\textwidth}
      \minipage[c][0.66\textheight][s]{\columnwidth}
      \begin{itemize}\color{gray}
        \item Checks wheteher exceptions backtrace should be recorded, by checking the \funcarg{domain\_state->backtrace\_active} value
        \item Switches to the C stack to be able to execute C code
        \item Calls \funcname{caml\_stash\_backtrace}, to collect the backtrace up to the next trap
      \end{itemize}
      \onslide*<2->{
        \begin{itemize}
          \item<2-> Executes the exception handler in \funcname{probably\_raise} with \funcname{RESTORE\_EXN\_HANDLER\_OCAML}
            \begin{itemize}
              \item Set \regname{RSP} to \localname{domain\_state->exn\_handler} value
              \item Restore \localname{domain\_state->exn\_handler} from trap
              \item Jump to the handler code with \funcname{ret}
            \end{itemize}
        \end{itemize}
      }
      \vfill
      \onslide*<1>{\mintocaml[firstline=9,lastline=13,highlightlines=12]{../src/backtrace/backtrace.ml}}
      \onslide*<2->{\mintocaml[firstline=15,lastline=21,highlightlines={19-21}]{../src/backtrace/backtrace.ml}}
      \endminipage
    \end{column}
    \begin{column}{0.5\textwidth}
      \centering
      \onslide*<1>{
        \mintc[firstline=190,lastline=192]{../ocaml/runtime/amd64.S}
        \mintc[firstline=234,lastline=237]{../ocaml/runtime/amd64.S}
      }
      \onslide*<2>{
        \mintc[firstline=190,lastline=192,highlightlines={191-192}]{../ocaml/runtime/amd64.S}
        \mintc[firstline=234,lastline=237,highlightlines={235-237}]{../ocaml/runtime/amd64.S}
      }
      \onslide*<1>{\listCamlRaiseExn[highlightlines=720]{}}
      \onslide*<2>{\listCamlRaiseExn[highlightlines=722]{}}
      \onslide*<3->{\providecommand\step{5}\input{diagrams/backtrace/backtrace.tex}}
    \end{column}
  \end{columns}
\end{frame}

\begin{frame}{Backtraces}{\funcname{caml\_probably\_raise}'s handler doesn't catch \typename{ExnC}}
  \begin{columns}[c]
    \begin{column}{0.45\textwidth}
      \minipage[c][0.75\textheight][s]{\columnwidth}
      \begin{itemize}
        \item<1-> \funcname{match \dots with}\footnotemark pattern matching can also match exceptions
        \item<1-> The CMM of \funcname{probably\_raise} shows that this \funcname{match \dots with} is implemented with a pattern matching and a \funcname{try \dots with}
        \item<1-> But this one only matches on \typename{ExnA} exception
        \item<2-> Since the pattern matching fails to catch the exception, \funcname{reraise} is called with our current \typename{ExnC} exception
        \item<3-> Disassembly shows that \funcname{reraise} is actually a call to \funcname{caml\_reraise\_exn}, which is also an assembly function provided by the runtime
      \end{itemize}
      \vfill
      \mintocaml[firstline=15,lastline=21,highlightlines={18-21}]{../src/backtrace/backtrace.ml}
      \endminipage
    \end{column}
    \begin{column}{0.55\textwidth}
      \centering
      \onslide*<1>{\mintcmm[highlightlines={10-18}]{../src/backtrace/camlBacktrace\_\_probably\_raise\_351.cmm}}
      \onslide*<2>{\listcmm[highlightlines=18]{../src/backtrace/camlBacktrace\_\_probably\_raise_351.cmm}{CMM of probably\_raise}}
      \onslide*<3>{\mintobjdump[firstline=5,lastline=35,highlightlines=31]{../src/backtrace/camlBacktrace\_\_probably\_raise_351.objdump}}
    \end{column}
  \end{columns}
  \only<1>{\footnotetext[1]{https://blog.janestreet.com/pattern-matching-and-exception-handling-unite/}}
\end{frame}

\newcommand\listCamlReraiseExn[1][]{
  \listasm[firstline=727,lastline=734,#1]{../ocaml/runtime/amd64.S}{runtime/amd64.S:727}}

\begin{frame}{Backtraces}{Introducing \funcname{caml\_reraise\_exn}}
  \begin{columns}[c]
    \begin{column}{0.5\textwidth}
      \minipage[c][0.66\textheight][s]{\columnwidth}
      \begin{itemize}
        \item<1-> The exception have to be reraised to the next handler
        \item<2-> First, checks if the backtrace should be collected with \funcarg{domain\_state->backtrace\_active}
        \only<3>{
        \begin{itemize}
            \item If not, behaves like a \funcname{raise\_notrace} with \funcname{RESTORE\_EXN\_HANDLER\_OCAML}
        \end{itemize}
        }
        \item<4-> Jumps into \funcname{caml\_raise\_exn}
        \item<5-> Calls \funcname{caml\_stash\_backtrace} again, to scan the next part of the stack: from the trap just pop-ed to the next trap
      \end{itemize}
      \vfill
      \mintocaml[firstline=15,lastline=21,highlightlines={18-21}]{../src/backtrace/backtrace.ml}
      \endminipage
    \end{column}
    \begin{column}{0.5\textwidth}
      \raggedleft
      \onslide*<1>{\listCamlReraiseExn{}}
      \onslide*<2>{\listCamlReraiseExn[highlightlines=729]{}}
      \onslide*<3>{
        \mintc[firstline=190,lastline=192,highlightlines={191-192}]{../ocaml/runtime/amd64.S}
        \mintc[firstline=234,lastline=237,highlightlines={235-237}]{../ocaml/runtime/amd64.S}
        \medskip
        \listCamlReraiseExn[highlightlines=731]{}
      }
      \onslide*<4-5>{
        % TODO refactor with \listCamlReraiseExn
        \mintasm[firstline=727,lastline=734,highlightlines=730]{../ocaml/runtime/amd64.S}
        \medskip
      }
      \onslide*<4>{\listCamlRaiseExn[highlightlines=711]{}}
      \onslide*<5>{\listCamlRaiseExn[highlightlines=720]{}}
    \end{column}
  \end{columns}
\end{frame}

\begin{frame}{Backtraces}{Backtrace collection continues}
  \begin{columns}[c]
    \begin{column}{0.55\textwidth}
      \minipage[c][0.75\textheight][s]{\columnwidth}
      \begin{itemize}
        \item<1-> \funcname{caml\_stash\_backtrace} scans the older part of the stack, up to the next trap
        \item<6-> Controls jumps to the nested exception handler in \funcname{maybe\_raise}, which matches only on \typename{ExnB}
        \item<7-> \funcname{caml\_reraise\_exn} is called again, backtrace collection resumes with \funcname{caml\_stash\_backtrace}
      \end{itemize}
      \medskip
      \begin{itemize}
        \item<2-> Constructed backtrace:
        \begin{itemize}
          \item <2-> \funcname{definitely\_raise}
          \item <2-> \funcname{probably\_raise}
          \item <3-> \funcname{probably\_raise} (re-raise)
          \item <4-> \funcname{maybe\_raise}
          \item <5-> \funcname{main}
          \item <8-> \funcname{main} (re-raise)
        \end{itemize}
      \end{itemize}
      \vfill
      \onslide*<1-5>{\mintocaml[firstline=15,lastline=21]{../src/backtrace/backtrace.ml}}
      \onslide*<6>{\mintocaml[firstline=28,lastline=34,highlightlines=31]{../src/backtrace/backtrace.ml}}
      \onslide*<7->{\mintocaml[firstline=28,lastline=34,highlightlines=33]{../src/backtrace/backtrace.ml}}
      \endminipage
    \end{column}
    \begin{column}{0.45\textwidth}
      \raggedleft
      \onslide*<1>{\providecommand\step{6}\input{diagrams/backtrace/backtrace.tex}}%
      \onslide*<2>{\providecommand\step{6}\input{diagrams/backtrace/backtrace.tex}}%
      \onslide*<3>{\providecommand\step{7}\input{diagrams/backtrace/backtrace.tex}}%
      \onslide*<4>{\providecommand\step{8}\input{diagrams/backtrace/backtrace.tex}}%
      \onslide*<5>{\providecommand\step{9}\input{diagrams/backtrace/backtrace.tex}}%
      \onslide*<6>{\providecommand\step{10}\input{diagrams/backtrace/backtrace.tex}}%
      \onslide*<7>{\providecommand\step{11}\input{diagrams/backtrace/backtrace.tex}}%
      \onslide*<8>{\providecommand\step{12}\input{diagrams/backtrace/backtrace.tex}}%
      \onslide*<9>{\providecommand\step{13}\input{diagrams/backtrace/backtrace.tex}}%
    \end{column}
  \end{columns}
\end{frame}

\begin{frame}{Backtraces}{Printing the backtrace}
  \begin{columns}[c]
    \begin{column}{0.45\textwidth}
      \minipage[c][0.66\textheight][s]{\columnwidth}
      \begin{itemize}
        \item<1-> The exception backtrace can now be printed to \funcarg{stdout} with \funcname{Printexc.print_backtrace}
        \item<2-> First the exception backtrace is retrieved into an OCaml array with \funcname{get_raw_backtrace }
        \item<3-> Which is implemented as the \funcname{caml_get_exception_raw_backtrace} C function in the runtime
        \item<4-> And finally printed with \funcname{print_exception_backtrace}
      \end{itemize}
      \vfill
      \mintocaml[firstline=28,lastline=34,highlightlines=33]{../src/backtrace/backtrace.ml}
      \endminipage
    \end{column}
    \begin{column}{0.55\textwidth}
      \onslide*<1,2>{
        \mintocaml[firstline=100,lastline=101]{../ocaml/stdlib/printexc.ml}
      }
      \only<1>{
        \listocaml[firstline=170,lastline=172]{../ocaml/stdlib/printexc.ml}{stdlib/printexc.ml:170}
      }
      \only<2>{
        \listocaml[firstline=170,lastline=172,highlightlines=172]{../ocaml/stdlib/printexc.ml}{stdlib/printexc.ml:170}
      }
      \only<3>{
        \mintc[firstline=154,lastline=156]{../ocaml/runtime/backtrace.c}
        \listc[firstline=171,lastline=192,highlightlines={184-187}]{../ocaml/runtime/backtrace.c}{runtime/backtrace.c:154}
      }
      \only<4>{
        \listocaml[firstline=155,lastline=165]{../ocaml/stdlib/printexc.ml}{stdlib/printexc.ml:55}
      }
    \end{column}
  \end{columns}
\end{frame}


\subsection{The multiple kinds of raise}
\frameSubsection{}{}

\begin{frame}{Backtraces}{When backtraces are not needed}
  \begin{columns}[c]
    \begin{column}{0.45\textwidth}
      \minipage[c][0.66\textheight][s]{\columnwidth}
      \begin{itemize}
        \item<1-> What if the backtrace for an exception is never needed?
        \item<2-> It's possible to raise the exception explicitly with \funcname{raise\_notrace} to make raising as cheap as possible
        \item<3-> An example of use in the \funcname{dynlink} library of the OCaml compiler
      \end{itemize}
      \vfill
      \onslide<3->{
      \listocaml[firstline=205,lastline=209,highlightlines=207]{../ocaml/otherlibs/dynlink/dynlink\_compilerlibs/misc.ml}{otherlibs/dynlink/dynlink\_compilerlibs/misc.ml:205}
      }
      \endminipage
    \end{column}
    \begin{column}{0.55\textwidth}
    \only<4>{
      \mintcmm[highlightlines=3]{../src/notrace/camlNotrace\_\_fun_347.cmm}
      \listcmm{../src/notrace/camlNotrace\_\_all\_somes\_327.cmm}{all\_somes}
    }
    \onslide*<5->{
      \listobjdump[highlightlines={6-9}]{../src/notrace/camlNotrace\_\_fun_347.objdump}{anonymous function from \funcname{all\_somes}}
    }
    \end{column}
  \end{columns}
\end{frame}

\begin{frame}{Backtraces}{Summary table of the multiple kinds of raise}
  \begin{itemize}
    \item When debugging information is enabled and if the backtrace of an exception isn't needed, it's possible to raise with \funcname{raise\_notrace} for performance
    \item When debugging information is \emph{not} enabled, the compiler optimises \funcname{raise} calls to inline assembly (same effect as using \funcname{raise\_notrace})
  \end{itemize}
  \bigskip
  \centering
  \begin{tabular}{| l | c | c | c | c |}
    \hline
    \parbox{10em}{Debug information \\ (\funcarg{-g} flag)}  & \multicolumn{2}{|c|}{Disabled}                                     & \multicolumn{2}{|c|}{Enabled} \\
    \hline
    Raise kind                                               & \funcname{raise} & \funcname{raise\_notrace}                       & \funcname{raise} & \funcname{raise\_notrace} \\
    \hline
    Compilation                                              & \parbox{8em}{inline assembly\\ (optimization)} & inline assembly   & \funcname{caml\_raise\_exn} & inline assembly \\
    \hline
  \end{tabular}
\end{frame}


%
% Takeaway #5
%
\subsection*{Takeaway \#5}
\frameSubsectionTakeaway{}
\againframe<5>{takeaway}
}%
      \onslide*<2>{\providecommand\step{1}\input{diagrams/backtrace/scanstack.tex}}%
      \onslide*<3>{\providecommand\step{2}\input{diagrams/backtrace/scanstack.tex}}%
      \onslide*<4>{\providecommand\step{3}\input{diagrams/backtrace/scanstack.tex}}%
      \onslide*<5>{\providecommand\step{4}\input{diagrams/backtrace/scanstack.tex}}%
      \onslide*<6>{\providecommand\step{5}\input{diagrams/backtrace/scanstack.tex}}%
    \end{column}
  \end{columns}
\end{frame}

% Duplicate of previous-previous slide
\begin{frame}{Backtraces}{Continuing on \funcname{caml\_stash\_backtrace}}
  \begin{columns}[c]
    \begin{column}{0.5\textwidth}
      \minipage[c][0.75\textheight][s]{\columnwidth}
      \begin{itemize}
          \color{gray}
        \item A C function provided by the runtime (specific to native code)
        \item It scans the stack, between the stack pointer at the raising point up to the next trap, in order to collect the names of the detected function frames
        \item It uses the same technique and information as the Garbage Collector (GC) uses to scan the values live on the stack
      \end{itemize}
      \begin{itemize}
        \item For each function frame detected, store a pointer to the \typename{frame\_descr} in the \localname{domain\_state->backtrace\_buffer} array, effectively constructing the backtrace
      \end{itemize}
      \onslide<2->{
        \begin{itemize}
            \smallskip
          \item Constructed backtrace:
            \funcname{definitely\_raise},
            \onslide<3->{\funcname{probably\_raise}}
        \end{itemize}
      }
      \vfill
      \mintocaml[firstline=9,lastline=13,highlightlines=12]{../src/backtrace/backtrace.ml}
      \endminipage
    \end{column}
    \begin{column}{0.5\textwidth}
      \raggedleft
      \onslide*<1>{\listStashBacktrace[highlightlines={114-115}]{}}
      \onslide*<2>{\providecommand\step{3}%
% Backtraces
%
\subsection{One advantage of raising with debugging information}
\frameSubsection{}{}

\begin{frame}{Backtraces}{Debugging information \& source code}
  \begin{columns}[c]
    \begin{column}{0.5\textwidth}
      \begin{itemize}
        \item Raising an exception is fun and games, but how do we know where the exception originated? $\rightarrow$ With the backtrace associated with the exception!
        \item In order to get a backtrace from the point where an exception was raised, it's necessary to compile with debugging information enabled (\funcarg{-g} flag for the compiler)
        \item Same example as in the `Nested handlers' section
      \end{itemize}
    \end{column}
    \begin{column}{0.5\textwidth}
      \listocaml[firstline=9,lastline=34]{../src/backtrace/backtrace.ml}{backtrace/backtrace.ml}
    \end{column}
  \end{columns}
\end{frame}

\begin{frame}{Backtraces}{CMM and disassembly of \funcname{definitely\_raise}}
  \begin{columns}[c]
    \begin{column}{0.5\textwidth}
      \minipage[c][0.66\textheight][s]{\columnwidth}
      \begin{itemize}
        \item<1-> An effect of compiling with debugging information (\funcarg{-g}): the CMM no longer uses \funcname{raise\_notrace} to raise the exception but \funcname{raise} instead
        \item<2-> Disassembly shows that CMM \funcname{raise} is actually a call to \funcname{caml\_raise\_exn}, a function in assembler provided by the runtime
      \end{itemize}
      \vfill
      \mintocaml[firstline=9,lastline=13,highlightlines=12]{../src/backtrace/backtrace.ml}
      \endminipage
    \end{column}
    \begin{column}{0.5\textwidth}
      \onslide*<1>{
        \mintcmm[highlightlines=5]{../src/backtrace/camlBacktrace\_\_definitely\_raise\_347.cmm}
      }
      \onslide*<2>{
        \mintobjdump[firstline=2,highlightlines=14]{../src/backtrace/camlBacktrace\_\_definitely\_raise\_347.objdump}
      }
    \end{column}
  \end{columns}
\end{frame}

\begin{frame}{Backtraces}{Introducing \funcname{caml\_raise\_exn}}
  \begin{columns}[c]
    \begin{column}{0.5\textwidth}
      \minipage[c][0.66\textheight][s]{\columnwidth}
      \onslide*<1>{
        \begin{itemize}
          \item Raising the exception is performed using \funcname{caml\_raise\_exn} which is
            \begin{itemize}
              \item An assembly function provided by the runtime
              \item Architecture specific
            \end{itemize}
          \item Let's see what it does differently from \funcname{raise\_notrace}\dots
          \item We are about to raise \typename{ExnC} with \funcname{caml\_raise\_exn} from \funcname{definitely\_raise}
        \end{itemize}
      }
      \onslide*<2>{
        \mintobjdump[firstline=2,highlightlines=14]{../src/backtrace/camlBacktrace\_\_definitely\_raise\_347.objdump}
      }
      \vfill
      \mintocaml[firstline=9,lastline=13,highlightlines=12]{../src/backtrace/backtrace.ml}
      \endminipage
    \end{column}
    \begin{column}{0.5\textwidth}
      \centering
      \providecommand\step{1}\input{diagrams/backtrace/backtrace.tex}
    \end{column}
  \end{columns}
\end{frame}

\begin{frame}{Backtraces}{But before that, we need more fields from \typename{caml\_domain\_state}}
  \begin{columns}
    \begin{column}{0.5\textwidth}
      \begin{itemize}
        \item Remember \typename{caml\_domain\_state} the per-domain runtime state?
        \item Some other members are of interest to us now\dots
        \item \localname{backtrace\_active} is used to know if the runtime should collect backtraces
        \item \localname{backtrace\_active} can be set by
          \begin{itemize}
            \item The \funcarg{b} flag in the \funcname{OCAMLRUNPARAM} environment variable
            \item \mintinline{ocaml}{Printexc.record_backtrace : bool -> unit} \footnotemark
          \end{itemize}
      \end{itemize}
    \end{column}
    \begin{column}{0.5\textwidth}
      \begin{listing}
        \mintc[highlightlines={9-11}]{src/ocaml/domain\_state\_backtrace.h}
        \caption{runtime/caml/domain\_state.h}
      \end{listing}
    \end{column}
  \end{columns}
  \footnotetext[1]{https://v2.ocaml.org/api/Printexc.html}
\end{frame}

\newcommand\listCamlRaiseExn[1][]{
  \listasm[firstline=702,lastline=725,#1]{../ocaml/runtime/amd64.S}{runtime/amd64.S:702}}

\begin{frame}{Backtraces}{Walk through \funcname{caml\_raise\_exn}}
  \begin{columns}[T]
    \begin{column}{0.5\textwidth}
      \begin{itemize}
        \item<1-> First checks whether exceptions backtrace should be recorded
        \item<1-> By checking the \funcarg{domain\_state->backtrace\_active} value
        \item<2-> If not, acts as \funcname{raise\_notrace} with \funcname{RESTORE\_EXN\_HANDLER\_OCAML}
          \begin{itemize}
            \item Set \regname{RSP} to \localname{domain\_state->exn\_handler} value
            \item Restore \localname{domain\_state->exn\_handler} from trap
            \item Jump with \funcname{ret} (same effect as using \funcname{pop} and \funcname{jmp})
          \end{itemize}
        \item<3-> Otherwise, switches to the C stack to be able to execute C code
        \item<4-> Calls \funcname{caml\_stash\_backtrace}, a C function provided by the runtime\ignorespaces
          \onslide<6->{, with
            \begin{itemize}
              \item \funcarg{exn}: exception value
              \item \funcarg{pc}: code address of the raising point
              \item \funcarg{sp}: stack address of the raising function
              \item \funcarg{trapsp}: stack address of the next handler
            \end{itemize}
          }
      \end{itemize}
      \bigskip
      \mintocaml[firstline=9,lastline=13,highlightlines=12]{../src/backtrace/backtrace.ml}
    \end{column}
    \begin{column}{0.5\textwidth}
      \raggedleft
      \onslide*<1,3-4>{
        \mintc[firstline=190,lastline=192]{../ocaml/runtime/amd64.S}
        \mintc[firstline=234,lastline=237]{../ocaml/runtime/amd64.S}
      }
      \onslide*<2>{
        \mintc[firstline=190,lastline=192,highlightlines={191-192}]{../ocaml/runtime/amd64.S}
        \mintc[firstline=234,lastline=237,highlightlines={235-237}]{../ocaml/runtime/amd64.S}
      }
      \onslide*<1>{\listCamlRaiseExn[highlightlines=705]{}}%
      \onslide*<2>{\listCamlRaiseExn[highlightlines={707-708}]{}}%
      \onslide*<3>{\listCamlRaiseExn[highlightlines={712-714}]{}}%
      \onslide*<4>{\listCamlRaiseExn[highlightlines={715-720}]{}}%
      \onslide*<5>{\providecommand\step{1}\input{diagrams/backtrace/backtrace.tex}}%
      \onslide*<6>{\providecommand\step{2}\input{diagrams/backtrace/backtrace.tex}}%
    \end{column}
  \end{columns}
\end{frame}

\newcommand\listStashBacktrace[1][]{
  \listc[firstline=91,lastline=120,#1]{../ocaml/runtime/backtrace\_nat.c}{runtime/backtrace\_nat.c:91}}

\begin{frame}{Backtraces}{Introducing \funcname{caml\_stash\_backtrace}}
  \begin{columns}[c]
    \begin{column}{0.5\textwidth}
      \minipage[c][0.66\textheight][s]{\columnwidth}
      \begin{itemize}
        \item<1-> A C function provided by the runtime (specific to native code)
        \item<2-> It scans the stack, between the stack pointer at the raising point up to the next trap, in order to collect the names of the detected function frames
          \onslide*<2>{
            \begin{itemize}
              \item And more information, like line number, column number...
            \end{itemize}
          }
        \item<3-> It uses the same technique and information as the Garbage Collector (GC) uses to scan the values live on the stack
          \onslide*<4>{
            \begin{itemize}
              \item \typename{frame\_descr} are at the core of stack unwinding
            \end{itemize}
          }
      \end{itemize}
      \vfill
      \mintocaml[firstline=9,lastline=13,highlightlines=12]{../src/backtrace/backtrace.ml}
      \endminipage
    \end{column}
    \begin{column}{0.5\textwidth}
      \onslide*<1>{\listStashBacktrace[highlightlines=91]{}}
      \onslide*<2>{\listStashBacktrace[highlightlines={91,117-118}]{}}
      \onslide*<3->{\listStashBacktrace[highlightlines={94,106,109-110}]{}}
    \end{column}
  \end{columns}
\end{frame}

\newcommand\listFrameDescr[1][]{
  \listocaml[firstline=111,lastline=124,#1]{../ocaml/asmcomp/emitaux.ml}{asmcomp/emitaux.ml:111}}
\newcommand\listDebugInfo[1][]{
  \listocaml[firstline=38,lastline=49,#1]{../ocaml/lambda/debuginfo.mli}{lambda/debuginfo.mli:38}}

\begin{frame}{Backtraces}{\typename{frame\_descr} and \typename{frame\_debuginfo} in a nutshell}
  \begin{columns}[c]
    \begin{column}{0.5\textwidth}
      \begin{itemize}
        \item<1-> For every return address in an OCaml program, there is a associated \typename{frame\_descr}
        \item<2-> A \typename{frame\_descr} specifies
        \begin{itemize}
          \item<2-> The size of the function stack frame
          \item<3-> Where live values are on the stack (so that the GC can mark them as live and prevent from being swept)
        \end{itemize}
        \item<4-> When compiled with debug information, \typename{frame\_descr} will be linked to a \typename{frame\_debuginfo}
        \begin{itemize}
          \item<5-> Contains information about the position in the source code (file name, line and column number)
        \end{itemize}
      \end{itemize}
    \end{column}
    \begin{column}{0.5\textwidth}
        \onslide*<1>{\listFrameDescr[highlightlines={118-122}]{}}
        \onslide*<2>{\listFrameDescr[highlightlines={120}]{}}
        \onslide*<3>{\listFrameDescr[highlightlines={121}]{}}
        \onslide*<4->{\listFrameDescr[highlightlines={122}]{}}
        \medskip
        \onslide*<1-3>{\listDebugInfo{}}
        \onslide*<4>{\listDebugInfo[highlightlines={38,49}]{}}
        \onslide*<5>{\listDebugInfo[highlightlines={39-46}]{}}
    \end{column}
  \end{columns}
\end{frame}

\begin{frame}{Backtraces}{Scanning the stack with \typename{frame\_descr}}
  \begin{columns}[c]
    \begin{column}{0.5\textwidth}
      \begin{itemize}
        \item Given the return address of the code that raises the exception by calling \funcname{caml\_raise\_exn} (\funcarg{pc} argument of \funcname{caml\_stash\_backtrace})
        \item And the position of the stack pointer (\funcarg{sp} argument of \funcname{caml\_stash\_backtrace})
        \item The frame size given by \typename{frame\_descr}.\funcarg{frame\_size}, allows to find the end of the previous frame
        \item And the return address of the caller of this function
        \bigskip
        \onslide<3->{
        \item Note that traps (here the reddish \funcname{ExnA}, \funcname{ExnB} and \funcname{ExnC} blocks) belong to the frame that installed them, and so the frame size changes when traps are removed
        }
      \end{itemize}
    \end{column}
    \begin{column}{0.5\textwidth}
      \raggedleft
      \onslide*<1>{\providecommand\step{2}\input{diagrams/backtrace/backtrace.tex}}%
      \onslide*<2>{\providecommand\step{1}\input{diagrams/backtrace/scanstack.tex}}%
      \onslide*<3>{\providecommand\step{2}\input{diagrams/backtrace/scanstack.tex}}%
      \onslide*<4>{\providecommand\step{3}\input{diagrams/backtrace/scanstack.tex}}%
      \onslide*<5>{\providecommand\step{4}\input{diagrams/backtrace/scanstack.tex}}%
      \onslide*<6>{\providecommand\step{5}\input{diagrams/backtrace/scanstack.tex}}%
    \end{column}
  \end{columns}
\end{frame}

% Duplicate of previous-previous slide
\begin{frame}{Backtraces}{Continuing on \funcname{caml\_stash\_backtrace}}
  \begin{columns}[c]
    \begin{column}{0.5\textwidth}
      \minipage[c][0.75\textheight][s]{\columnwidth}
      \begin{itemize}
          \color{gray}
        \item A C function provided by the runtime (specific to native code)
        \item It scans the stack, between the stack pointer at the raising point up to the next trap, in order to collect the names of the detected function frames
        \item It uses the same technique and information as the Garbage Collector (GC) uses to scan the values live on the stack
      \end{itemize}
      \begin{itemize}
        \item For each function frame detected, store a pointer to the \typename{frame\_descr} in the \localname{domain\_state->backtrace\_buffer} array, effectively constructing the backtrace
      \end{itemize}
      \onslide<2->{
        \begin{itemize}
            \smallskip
          \item Constructed backtrace:
            \funcname{definitely\_raise},
            \onslide<3->{\funcname{probably\_raise}}
        \end{itemize}
      }
      \vfill
      \mintocaml[firstline=9,lastline=13,highlightlines=12]{../src/backtrace/backtrace.ml}
      \endminipage
    \end{column}
    \begin{column}{0.5\textwidth}
      \raggedleft
      \onslide*<1>{\listStashBacktrace[highlightlines={114-115}]{}}
      \onslide*<2>{\providecommand\step{3}\input{diagrams/backtrace/backtrace.tex}}%
      \onslide*<3>{\providecommand\step{4}\input{diagrams/backtrace/backtrace.tex}}%
    \end{column}
  \end{columns}
\end{frame}

\begin{frame}{Backtraces}{Back to \funcname{caml\_raise\_exn}}
  \begin{columns}[c]
    \begin{column}{0.5\textwidth}
      \minipage[c][0.66\textheight][s]{\columnwidth}
      \begin{itemize}\color{gray}
        \item Checks wheteher exceptions backtrace should be recorded, by checking the \funcarg{domain\_state->backtrace\_active} value
        \item Switches to the C stack to be able to execute C code
        \item Calls \funcname{caml\_stash\_backtrace}, to collect the backtrace up to the next trap
      \end{itemize}
      \onslide*<2->{
        \begin{itemize}
          \item<2-> Executes the exception handler in \funcname{probably\_raise} with \funcname{RESTORE\_EXN\_HANDLER\_OCAML}
            \begin{itemize}
              \item Set \regname{RSP} to \localname{domain\_state->exn\_handler} value
              \item Restore \localname{domain\_state->exn\_handler} from trap
              \item Jump to the handler code with \funcname{ret}
            \end{itemize}
        \end{itemize}
      }
      \vfill
      \onslide*<1>{\mintocaml[firstline=9,lastline=13,highlightlines=12]{../src/backtrace/backtrace.ml}}
      \onslide*<2->{\mintocaml[firstline=15,lastline=21,highlightlines={19-21}]{../src/backtrace/backtrace.ml}}
      \endminipage
    \end{column}
    \begin{column}{0.5\textwidth}
      \centering
      \onslide*<1>{
        \mintc[firstline=190,lastline=192]{../ocaml/runtime/amd64.S}
        \mintc[firstline=234,lastline=237]{../ocaml/runtime/amd64.S}
      }
      \onslide*<2>{
        \mintc[firstline=190,lastline=192,highlightlines={191-192}]{../ocaml/runtime/amd64.S}
        \mintc[firstline=234,lastline=237,highlightlines={235-237}]{../ocaml/runtime/amd64.S}
      }
      \onslide*<1>{\listCamlRaiseExn[highlightlines=720]{}}
      \onslide*<2>{\listCamlRaiseExn[highlightlines=722]{}}
      \onslide*<3->{\providecommand\step{5}\input{diagrams/backtrace/backtrace.tex}}
    \end{column}
  \end{columns}
\end{frame}

\begin{frame}{Backtraces}{\funcname{caml\_probably\_raise}'s handler doesn't catch \typename{ExnC}}
  \begin{columns}[c]
    \begin{column}{0.45\textwidth}
      \minipage[c][0.75\textheight][s]{\columnwidth}
      \begin{itemize}
        \item<1-> \funcname{match \dots with}\footnotemark pattern matching can also match exceptions
        \item<1-> The CMM of \funcname{probably\_raise} shows that this \funcname{match \dots with} is implemented with a pattern matching and a \funcname{try \dots with}
        \item<1-> But this one only matches on \typename{ExnA} exception
        \item<2-> Since the pattern matching fails to catch the exception, \funcname{reraise} is called with our current \typename{ExnC} exception
        \item<3-> Disassembly shows that \funcname{reraise} is actually a call to \funcname{caml\_reraise\_exn}, which is also an assembly function provided by the runtime
      \end{itemize}
      \vfill
      \mintocaml[firstline=15,lastline=21,highlightlines={18-21}]{../src/backtrace/backtrace.ml}
      \endminipage
    \end{column}
    \begin{column}{0.55\textwidth}
      \centering
      \onslide*<1>{\mintcmm[highlightlines={10-18}]{../src/backtrace/camlBacktrace\_\_probably\_raise\_351.cmm}}
      \onslide*<2>{\listcmm[highlightlines=18]{../src/backtrace/camlBacktrace\_\_probably\_raise_351.cmm}{CMM of probably\_raise}}
      \onslide*<3>{\mintobjdump[firstline=5,lastline=35,highlightlines=31]{../src/backtrace/camlBacktrace\_\_probably\_raise_351.objdump}}
    \end{column}
  \end{columns}
  \only<1>{\footnotetext[1]{https://blog.janestreet.com/pattern-matching-and-exception-handling-unite/}}
\end{frame}

\newcommand\listCamlReraiseExn[1][]{
  \listasm[firstline=727,lastline=734,#1]{../ocaml/runtime/amd64.S}{runtime/amd64.S:727}}

\begin{frame}{Backtraces}{Introducing \funcname{caml\_reraise\_exn}}
  \begin{columns}[c]
    \begin{column}{0.5\textwidth}
      \minipage[c][0.66\textheight][s]{\columnwidth}
      \begin{itemize}
        \item<1-> The exception have to be reraised to the next handler
        \item<2-> First, checks if the backtrace should be collected with \funcarg{domain\_state->backtrace\_active}
        \only<3>{
        \begin{itemize}
            \item If not, behaves like a \funcname{raise\_notrace} with \funcname{RESTORE\_EXN\_HANDLER\_OCAML}
        \end{itemize}
        }
        \item<4-> Jumps into \funcname{caml\_raise\_exn}
        \item<5-> Calls \funcname{caml\_stash\_backtrace} again, to scan the next part of the stack: from the trap just pop-ed to the next trap
      \end{itemize}
      \vfill
      \mintocaml[firstline=15,lastline=21,highlightlines={18-21}]{../src/backtrace/backtrace.ml}
      \endminipage
    \end{column}
    \begin{column}{0.5\textwidth}
      \raggedleft
      \onslide*<1>{\listCamlReraiseExn{}}
      \onslide*<2>{\listCamlReraiseExn[highlightlines=729]{}}
      \onslide*<3>{
        \mintc[firstline=190,lastline=192,highlightlines={191-192}]{../ocaml/runtime/amd64.S}
        \mintc[firstline=234,lastline=237,highlightlines={235-237}]{../ocaml/runtime/amd64.S}
        \medskip
        \listCamlReraiseExn[highlightlines=731]{}
      }
      \onslide*<4-5>{
        % TODO refactor with \listCamlReraiseExn
        \mintasm[firstline=727,lastline=734,highlightlines=730]{../ocaml/runtime/amd64.S}
        \medskip
      }
      \onslide*<4>{\listCamlRaiseExn[highlightlines=711]{}}
      \onslide*<5>{\listCamlRaiseExn[highlightlines=720]{}}
    \end{column}
  \end{columns}
\end{frame}

\begin{frame}{Backtraces}{Backtrace collection continues}
  \begin{columns}[c]
    \begin{column}{0.55\textwidth}
      \minipage[c][0.75\textheight][s]{\columnwidth}
      \begin{itemize}
        \item<1-> \funcname{caml\_stash\_backtrace} scans the older part of the stack, up to the next trap
        \item<6-> Controls jumps to the nested exception handler in \funcname{maybe\_raise}, which matches only on \typename{ExnB}
        \item<7-> \funcname{caml\_reraise\_exn} is called again, backtrace collection resumes with \funcname{caml\_stash\_backtrace}
      \end{itemize}
      \medskip
      \begin{itemize}
        \item<2-> Constructed backtrace:
        \begin{itemize}
          \item <2-> \funcname{definitely\_raise}
          \item <2-> \funcname{probably\_raise}
          \item <3-> \funcname{probably\_raise} (re-raise)
          \item <4-> \funcname{maybe\_raise}
          \item <5-> \funcname{main}
          \item <8-> \funcname{main} (re-raise)
        \end{itemize}
      \end{itemize}
      \vfill
      \onslide*<1-5>{\mintocaml[firstline=15,lastline=21]{../src/backtrace/backtrace.ml}}
      \onslide*<6>{\mintocaml[firstline=28,lastline=34,highlightlines=31]{../src/backtrace/backtrace.ml}}
      \onslide*<7->{\mintocaml[firstline=28,lastline=34,highlightlines=33]{../src/backtrace/backtrace.ml}}
      \endminipage
    \end{column}
    \begin{column}{0.45\textwidth}
      \raggedleft
      \onslide*<1>{\providecommand\step{6}\input{diagrams/backtrace/backtrace.tex}}%
      \onslide*<2>{\providecommand\step{6}\input{diagrams/backtrace/backtrace.tex}}%
      \onslide*<3>{\providecommand\step{7}\input{diagrams/backtrace/backtrace.tex}}%
      \onslide*<4>{\providecommand\step{8}\input{diagrams/backtrace/backtrace.tex}}%
      \onslide*<5>{\providecommand\step{9}\input{diagrams/backtrace/backtrace.tex}}%
      \onslide*<6>{\providecommand\step{10}\input{diagrams/backtrace/backtrace.tex}}%
      \onslide*<7>{\providecommand\step{11}\input{diagrams/backtrace/backtrace.tex}}%
      \onslide*<8>{\providecommand\step{12}\input{diagrams/backtrace/backtrace.tex}}%
      \onslide*<9>{\providecommand\step{13}\input{diagrams/backtrace/backtrace.tex}}%
    \end{column}
  \end{columns}
\end{frame}

\begin{frame}{Backtraces}{Printing the backtrace}
  \begin{columns}[c]
    \begin{column}{0.45\textwidth}
      \minipage[c][0.66\textheight][s]{\columnwidth}
      \begin{itemize}
        \item<1-> The exception backtrace can now be printed to \funcarg{stdout} with \funcname{Printexc.print_backtrace}
        \item<2-> First the exception backtrace is retrieved into an OCaml array with \funcname{get_raw_backtrace }
        \item<3-> Which is implemented as the \funcname{caml_get_exception_raw_backtrace} C function in the runtime
        \item<4-> And finally printed with \funcname{print_exception_backtrace}
      \end{itemize}
      \vfill
      \mintocaml[firstline=28,lastline=34,highlightlines=33]{../src/backtrace/backtrace.ml}
      \endminipage
    \end{column}
    \begin{column}{0.55\textwidth}
      \onslide*<1,2>{
        \mintocaml[firstline=100,lastline=101]{../ocaml/stdlib/printexc.ml}
      }
      \only<1>{
        \listocaml[firstline=170,lastline=172]{../ocaml/stdlib/printexc.ml}{stdlib/printexc.ml:170}
      }
      \only<2>{
        \listocaml[firstline=170,lastline=172,highlightlines=172]{../ocaml/stdlib/printexc.ml}{stdlib/printexc.ml:170}
      }
      \only<3>{
        \mintc[firstline=154,lastline=156]{../ocaml/runtime/backtrace.c}
        \listc[firstline=171,lastline=192,highlightlines={184-187}]{../ocaml/runtime/backtrace.c}{runtime/backtrace.c:154}
      }
      \only<4>{
        \listocaml[firstline=155,lastline=165]{../ocaml/stdlib/printexc.ml}{stdlib/printexc.ml:55}
      }
    \end{column}
  \end{columns}
\end{frame}


\subsection{The multiple kinds of raise}
\frameSubsection{}{}

\begin{frame}{Backtraces}{When backtraces are not needed}
  \begin{columns}[c]
    \begin{column}{0.45\textwidth}
      \minipage[c][0.66\textheight][s]{\columnwidth}
      \begin{itemize}
        \item<1-> What if the backtrace for an exception is never needed?
        \item<2-> It's possible to raise the exception explicitly with \funcname{raise\_notrace} to make raising as cheap as possible
        \item<3-> An example of use in the \funcname{dynlink} library of the OCaml compiler
      \end{itemize}
      \vfill
      \onslide<3->{
      \listocaml[firstline=205,lastline=209,highlightlines=207]{../ocaml/otherlibs/dynlink/dynlink\_compilerlibs/misc.ml}{otherlibs/dynlink/dynlink\_compilerlibs/misc.ml:205}
      }
      \endminipage
    \end{column}
    \begin{column}{0.55\textwidth}
    \only<4>{
      \mintcmm[highlightlines=3]{../src/notrace/camlNotrace\_\_fun_347.cmm}
      \listcmm{../src/notrace/camlNotrace\_\_all\_somes\_327.cmm}{all\_somes}
    }
    \onslide*<5->{
      \listobjdump[highlightlines={6-9}]{../src/notrace/camlNotrace\_\_fun_347.objdump}{anonymous function from \funcname{all\_somes}}
    }
    \end{column}
  \end{columns}
\end{frame}

\begin{frame}{Backtraces}{Summary table of the multiple kinds of raise}
  \begin{itemize}
    \item When debugging information is enabled and if the backtrace of an exception isn't needed, it's possible to raise with \funcname{raise\_notrace} for performance
    \item When debugging information is \emph{not} enabled, the compiler optimises \funcname{raise} calls to inline assembly (same effect as using \funcname{raise\_notrace})
  \end{itemize}
  \bigskip
  \centering
  \begin{tabular}{| l | c | c | c | c |}
    \hline
    \parbox{10em}{Debug information \\ (\funcarg{-g} flag)}  & \multicolumn{2}{|c|}{Disabled}                                     & \multicolumn{2}{|c|}{Enabled} \\
    \hline
    Raise kind                                               & \funcname{raise} & \funcname{raise\_notrace}                       & \funcname{raise} & \funcname{raise\_notrace} \\
    \hline
    Compilation                                              & \parbox{8em}{inline assembly\\ (optimization)} & inline assembly   & \funcname{caml\_raise\_exn} & inline assembly \\
    \hline
  \end{tabular}
\end{frame}


%
% Takeaway #5
%
\subsection*{Takeaway \#5}
\frameSubsectionTakeaway{}
\againframe<5>{takeaway}
}%
      \onslide*<3>{\providecommand\step{4}%
% Backtraces
%
\subsection{One advantage of raising with debugging information}
\frameSubsection{}{}

\begin{frame}{Backtraces}{Debugging information \& source code}
  \begin{columns}[c]
    \begin{column}{0.5\textwidth}
      \begin{itemize}
        \item Raising an exception is fun and games, but how do we know where the exception originated? $\rightarrow$ With the backtrace associated with the exception!
        \item In order to get a backtrace from the point where an exception was raised, it's necessary to compile with debugging information enabled (\funcarg{-g} flag for the compiler)
        \item Same example as in the `Nested handlers' section
      \end{itemize}
    \end{column}
    \begin{column}{0.5\textwidth}
      \listocaml[firstline=9,lastline=34]{../src/backtrace/backtrace.ml}{backtrace/backtrace.ml}
    \end{column}
  \end{columns}
\end{frame}

\begin{frame}{Backtraces}{CMM and disassembly of \funcname{definitely\_raise}}
  \begin{columns}[c]
    \begin{column}{0.5\textwidth}
      \minipage[c][0.66\textheight][s]{\columnwidth}
      \begin{itemize}
        \item<1-> An effect of compiling with debugging information (\funcarg{-g}): the CMM no longer uses \funcname{raise\_notrace} to raise the exception but \funcname{raise} instead
        \item<2-> Disassembly shows that CMM \funcname{raise} is actually a call to \funcname{caml\_raise\_exn}, a function in assembler provided by the runtime
      \end{itemize}
      \vfill
      \mintocaml[firstline=9,lastline=13,highlightlines=12]{../src/backtrace/backtrace.ml}
      \endminipage
    \end{column}
    \begin{column}{0.5\textwidth}
      \onslide*<1>{
        \mintcmm[highlightlines=5]{../src/backtrace/camlBacktrace\_\_definitely\_raise\_347.cmm}
      }
      \onslide*<2>{
        \mintobjdump[firstline=2,highlightlines=14]{../src/backtrace/camlBacktrace\_\_definitely\_raise\_347.objdump}
      }
    \end{column}
  \end{columns}
\end{frame}

\begin{frame}{Backtraces}{Introducing \funcname{caml\_raise\_exn}}
  \begin{columns}[c]
    \begin{column}{0.5\textwidth}
      \minipage[c][0.66\textheight][s]{\columnwidth}
      \onslide*<1>{
        \begin{itemize}
          \item Raising the exception is performed using \funcname{caml\_raise\_exn} which is
            \begin{itemize}
              \item An assembly function provided by the runtime
              \item Architecture specific
            \end{itemize}
          \item Let's see what it does differently from \funcname{raise\_notrace}\dots
          \item We are about to raise \typename{ExnC} with \funcname{caml\_raise\_exn} from \funcname{definitely\_raise}
        \end{itemize}
      }
      \onslide*<2>{
        \mintobjdump[firstline=2,highlightlines=14]{../src/backtrace/camlBacktrace\_\_definitely\_raise\_347.objdump}
      }
      \vfill
      \mintocaml[firstline=9,lastline=13,highlightlines=12]{../src/backtrace/backtrace.ml}
      \endminipage
    \end{column}
    \begin{column}{0.5\textwidth}
      \centering
      \providecommand\step{1}\input{diagrams/backtrace/backtrace.tex}
    \end{column}
  \end{columns}
\end{frame}

\begin{frame}{Backtraces}{But before that, we need more fields from \typename{caml\_domain\_state}}
  \begin{columns}
    \begin{column}{0.5\textwidth}
      \begin{itemize}
        \item Remember \typename{caml\_domain\_state} the per-domain runtime state?
        \item Some other members are of interest to us now\dots
        \item \localname{backtrace\_active} is used to know if the runtime should collect backtraces
        \item \localname{backtrace\_active} can be set by
          \begin{itemize}
            \item The \funcarg{b} flag in the \funcname{OCAMLRUNPARAM} environment variable
            \item \mintinline{ocaml}{Printexc.record_backtrace : bool -> unit} \footnotemark
          \end{itemize}
      \end{itemize}
    \end{column}
    \begin{column}{0.5\textwidth}
      \begin{listing}
        \mintc[highlightlines={9-11}]{src/ocaml/domain\_state\_backtrace.h}
        \caption{runtime/caml/domain\_state.h}
      \end{listing}
    \end{column}
  \end{columns}
  \footnotetext[1]{https://v2.ocaml.org/api/Printexc.html}
\end{frame}

\newcommand\listCamlRaiseExn[1][]{
  \listasm[firstline=702,lastline=725,#1]{../ocaml/runtime/amd64.S}{runtime/amd64.S:702}}

\begin{frame}{Backtraces}{Walk through \funcname{caml\_raise\_exn}}
  \begin{columns}[T]
    \begin{column}{0.5\textwidth}
      \begin{itemize}
        \item<1-> First checks whether exceptions backtrace should be recorded
        \item<1-> By checking the \funcarg{domain\_state->backtrace\_active} value
        \item<2-> If not, acts as \funcname{raise\_notrace} with \funcname{RESTORE\_EXN\_HANDLER\_OCAML}
          \begin{itemize}
            \item Set \regname{RSP} to \localname{domain\_state->exn\_handler} value
            \item Restore \localname{domain\_state->exn\_handler} from trap
            \item Jump with \funcname{ret} (same effect as using \funcname{pop} and \funcname{jmp})
          \end{itemize}
        \item<3-> Otherwise, switches to the C stack to be able to execute C code
        \item<4-> Calls \funcname{caml\_stash\_backtrace}, a C function provided by the runtime\ignorespaces
          \onslide<6->{, with
            \begin{itemize}
              \item \funcarg{exn}: exception value
              \item \funcarg{pc}: code address of the raising point
              \item \funcarg{sp}: stack address of the raising function
              \item \funcarg{trapsp}: stack address of the next handler
            \end{itemize}
          }
      \end{itemize}
      \bigskip
      \mintocaml[firstline=9,lastline=13,highlightlines=12]{../src/backtrace/backtrace.ml}
    \end{column}
    \begin{column}{0.5\textwidth}
      \raggedleft
      \onslide*<1,3-4>{
        \mintc[firstline=190,lastline=192]{../ocaml/runtime/amd64.S}
        \mintc[firstline=234,lastline=237]{../ocaml/runtime/amd64.S}
      }
      \onslide*<2>{
        \mintc[firstline=190,lastline=192,highlightlines={191-192}]{../ocaml/runtime/amd64.S}
        \mintc[firstline=234,lastline=237,highlightlines={235-237}]{../ocaml/runtime/amd64.S}
      }
      \onslide*<1>{\listCamlRaiseExn[highlightlines=705]{}}%
      \onslide*<2>{\listCamlRaiseExn[highlightlines={707-708}]{}}%
      \onslide*<3>{\listCamlRaiseExn[highlightlines={712-714}]{}}%
      \onslide*<4>{\listCamlRaiseExn[highlightlines={715-720}]{}}%
      \onslide*<5>{\providecommand\step{1}\input{diagrams/backtrace/backtrace.tex}}%
      \onslide*<6>{\providecommand\step{2}\input{diagrams/backtrace/backtrace.tex}}%
    \end{column}
  \end{columns}
\end{frame}

\newcommand\listStashBacktrace[1][]{
  \listc[firstline=91,lastline=120,#1]{../ocaml/runtime/backtrace\_nat.c}{runtime/backtrace\_nat.c:91}}

\begin{frame}{Backtraces}{Introducing \funcname{caml\_stash\_backtrace}}
  \begin{columns}[c]
    \begin{column}{0.5\textwidth}
      \minipage[c][0.66\textheight][s]{\columnwidth}
      \begin{itemize}
        \item<1-> A C function provided by the runtime (specific to native code)
        \item<2-> It scans the stack, between the stack pointer at the raising point up to the next trap, in order to collect the names of the detected function frames
          \onslide*<2>{
            \begin{itemize}
              \item And more information, like line number, column number...
            \end{itemize}
          }
        \item<3-> It uses the same technique and information as the Garbage Collector (GC) uses to scan the values live on the stack
          \onslide*<4>{
            \begin{itemize}
              \item \typename{frame\_descr} are at the core of stack unwinding
            \end{itemize}
          }
      \end{itemize}
      \vfill
      \mintocaml[firstline=9,lastline=13,highlightlines=12]{../src/backtrace/backtrace.ml}
      \endminipage
    \end{column}
    \begin{column}{0.5\textwidth}
      \onslide*<1>{\listStashBacktrace[highlightlines=91]{}}
      \onslide*<2>{\listStashBacktrace[highlightlines={91,117-118}]{}}
      \onslide*<3->{\listStashBacktrace[highlightlines={94,106,109-110}]{}}
    \end{column}
  \end{columns}
\end{frame}

\newcommand\listFrameDescr[1][]{
  \listocaml[firstline=111,lastline=124,#1]{../ocaml/asmcomp/emitaux.ml}{asmcomp/emitaux.ml:111}}
\newcommand\listDebugInfo[1][]{
  \listocaml[firstline=38,lastline=49,#1]{../ocaml/lambda/debuginfo.mli}{lambda/debuginfo.mli:38}}

\begin{frame}{Backtraces}{\typename{frame\_descr} and \typename{frame\_debuginfo} in a nutshell}
  \begin{columns}[c]
    \begin{column}{0.5\textwidth}
      \begin{itemize}
        \item<1-> For every return address in an OCaml program, there is a associated \typename{frame\_descr}
        \item<2-> A \typename{frame\_descr} specifies
        \begin{itemize}
          \item<2-> The size of the function stack frame
          \item<3-> Where live values are on the stack (so that the GC can mark them as live and prevent from being swept)
        \end{itemize}
        \item<4-> When compiled with debug information, \typename{frame\_descr} will be linked to a \typename{frame\_debuginfo}
        \begin{itemize}
          \item<5-> Contains information about the position in the source code (file name, line and column number)
        \end{itemize}
      \end{itemize}
    \end{column}
    \begin{column}{0.5\textwidth}
        \onslide*<1>{\listFrameDescr[highlightlines={118-122}]{}}
        \onslide*<2>{\listFrameDescr[highlightlines={120}]{}}
        \onslide*<3>{\listFrameDescr[highlightlines={121}]{}}
        \onslide*<4->{\listFrameDescr[highlightlines={122}]{}}
        \medskip
        \onslide*<1-3>{\listDebugInfo{}}
        \onslide*<4>{\listDebugInfo[highlightlines={38,49}]{}}
        \onslide*<5>{\listDebugInfo[highlightlines={39-46}]{}}
    \end{column}
  \end{columns}
\end{frame}

\begin{frame}{Backtraces}{Scanning the stack with \typename{frame\_descr}}
  \begin{columns}[c]
    \begin{column}{0.5\textwidth}
      \begin{itemize}
        \item Given the return address of the code that raises the exception by calling \funcname{caml\_raise\_exn} (\funcarg{pc} argument of \funcname{caml\_stash\_backtrace})
        \item And the position of the stack pointer (\funcarg{sp} argument of \funcname{caml\_stash\_backtrace})
        \item The frame size given by \typename{frame\_descr}.\funcarg{frame\_size}, allows to find the end of the previous frame
        \item And the return address of the caller of this function
        \bigskip
        \onslide<3->{
        \item Note that traps (here the reddish \funcname{ExnA}, \funcname{ExnB} and \funcname{ExnC} blocks) belong to the frame that installed them, and so the frame size changes when traps are removed
        }
      \end{itemize}
    \end{column}
    \begin{column}{0.5\textwidth}
      \raggedleft
      \onslide*<1>{\providecommand\step{2}\input{diagrams/backtrace/backtrace.tex}}%
      \onslide*<2>{\providecommand\step{1}\input{diagrams/backtrace/scanstack.tex}}%
      \onslide*<3>{\providecommand\step{2}\input{diagrams/backtrace/scanstack.tex}}%
      \onslide*<4>{\providecommand\step{3}\input{diagrams/backtrace/scanstack.tex}}%
      \onslide*<5>{\providecommand\step{4}\input{diagrams/backtrace/scanstack.tex}}%
      \onslide*<6>{\providecommand\step{5}\input{diagrams/backtrace/scanstack.tex}}%
    \end{column}
  \end{columns}
\end{frame}

% Duplicate of previous-previous slide
\begin{frame}{Backtraces}{Continuing on \funcname{caml\_stash\_backtrace}}
  \begin{columns}[c]
    \begin{column}{0.5\textwidth}
      \minipage[c][0.75\textheight][s]{\columnwidth}
      \begin{itemize}
          \color{gray}
        \item A C function provided by the runtime (specific to native code)
        \item It scans the stack, between the stack pointer at the raising point up to the next trap, in order to collect the names of the detected function frames
        \item It uses the same technique and information as the Garbage Collector (GC) uses to scan the values live on the stack
      \end{itemize}
      \begin{itemize}
        \item For each function frame detected, store a pointer to the \typename{frame\_descr} in the \localname{domain\_state->backtrace\_buffer} array, effectively constructing the backtrace
      \end{itemize}
      \onslide<2->{
        \begin{itemize}
            \smallskip
          \item Constructed backtrace:
            \funcname{definitely\_raise},
            \onslide<3->{\funcname{probably\_raise}}
        \end{itemize}
      }
      \vfill
      \mintocaml[firstline=9,lastline=13,highlightlines=12]{../src/backtrace/backtrace.ml}
      \endminipage
    \end{column}
    \begin{column}{0.5\textwidth}
      \raggedleft
      \onslide*<1>{\listStashBacktrace[highlightlines={114-115}]{}}
      \onslide*<2>{\providecommand\step{3}\input{diagrams/backtrace/backtrace.tex}}%
      \onslide*<3>{\providecommand\step{4}\input{diagrams/backtrace/backtrace.tex}}%
    \end{column}
  \end{columns}
\end{frame}

\begin{frame}{Backtraces}{Back to \funcname{caml\_raise\_exn}}
  \begin{columns}[c]
    \begin{column}{0.5\textwidth}
      \minipage[c][0.66\textheight][s]{\columnwidth}
      \begin{itemize}\color{gray}
        \item Checks wheteher exceptions backtrace should be recorded, by checking the \funcarg{domain\_state->backtrace\_active} value
        \item Switches to the C stack to be able to execute C code
        \item Calls \funcname{caml\_stash\_backtrace}, to collect the backtrace up to the next trap
      \end{itemize}
      \onslide*<2->{
        \begin{itemize}
          \item<2-> Executes the exception handler in \funcname{probably\_raise} with \funcname{RESTORE\_EXN\_HANDLER\_OCAML}
            \begin{itemize}
              \item Set \regname{RSP} to \localname{domain\_state->exn\_handler} value
              \item Restore \localname{domain\_state->exn\_handler} from trap
              \item Jump to the handler code with \funcname{ret}
            \end{itemize}
        \end{itemize}
      }
      \vfill
      \onslide*<1>{\mintocaml[firstline=9,lastline=13,highlightlines=12]{../src/backtrace/backtrace.ml}}
      \onslide*<2->{\mintocaml[firstline=15,lastline=21,highlightlines={19-21}]{../src/backtrace/backtrace.ml}}
      \endminipage
    \end{column}
    \begin{column}{0.5\textwidth}
      \centering
      \onslide*<1>{
        \mintc[firstline=190,lastline=192]{../ocaml/runtime/amd64.S}
        \mintc[firstline=234,lastline=237]{../ocaml/runtime/amd64.S}
      }
      \onslide*<2>{
        \mintc[firstline=190,lastline=192,highlightlines={191-192}]{../ocaml/runtime/amd64.S}
        \mintc[firstline=234,lastline=237,highlightlines={235-237}]{../ocaml/runtime/amd64.S}
      }
      \onslide*<1>{\listCamlRaiseExn[highlightlines=720]{}}
      \onslide*<2>{\listCamlRaiseExn[highlightlines=722]{}}
      \onslide*<3->{\providecommand\step{5}\input{diagrams/backtrace/backtrace.tex}}
    \end{column}
  \end{columns}
\end{frame}

\begin{frame}{Backtraces}{\funcname{caml\_probably\_raise}'s handler doesn't catch \typename{ExnC}}
  \begin{columns}[c]
    \begin{column}{0.45\textwidth}
      \minipage[c][0.75\textheight][s]{\columnwidth}
      \begin{itemize}
        \item<1-> \funcname{match \dots with}\footnotemark pattern matching can also match exceptions
        \item<1-> The CMM of \funcname{probably\_raise} shows that this \funcname{match \dots with} is implemented with a pattern matching and a \funcname{try \dots with}
        \item<1-> But this one only matches on \typename{ExnA} exception
        \item<2-> Since the pattern matching fails to catch the exception, \funcname{reraise} is called with our current \typename{ExnC} exception
        \item<3-> Disassembly shows that \funcname{reraise} is actually a call to \funcname{caml\_reraise\_exn}, which is also an assembly function provided by the runtime
      \end{itemize}
      \vfill
      \mintocaml[firstline=15,lastline=21,highlightlines={18-21}]{../src/backtrace/backtrace.ml}
      \endminipage
    \end{column}
    \begin{column}{0.55\textwidth}
      \centering
      \onslide*<1>{\mintcmm[highlightlines={10-18}]{../src/backtrace/camlBacktrace\_\_probably\_raise\_351.cmm}}
      \onslide*<2>{\listcmm[highlightlines=18]{../src/backtrace/camlBacktrace\_\_probably\_raise_351.cmm}{CMM of probably\_raise}}
      \onslide*<3>{\mintobjdump[firstline=5,lastline=35,highlightlines=31]{../src/backtrace/camlBacktrace\_\_probably\_raise_351.objdump}}
    \end{column}
  \end{columns}
  \only<1>{\footnotetext[1]{https://blog.janestreet.com/pattern-matching-and-exception-handling-unite/}}
\end{frame}

\newcommand\listCamlReraiseExn[1][]{
  \listasm[firstline=727,lastline=734,#1]{../ocaml/runtime/amd64.S}{runtime/amd64.S:727}}

\begin{frame}{Backtraces}{Introducing \funcname{caml\_reraise\_exn}}
  \begin{columns}[c]
    \begin{column}{0.5\textwidth}
      \minipage[c][0.66\textheight][s]{\columnwidth}
      \begin{itemize}
        \item<1-> The exception have to be reraised to the next handler
        \item<2-> First, checks if the backtrace should be collected with \funcarg{domain\_state->backtrace\_active}
        \only<3>{
        \begin{itemize}
            \item If not, behaves like a \funcname{raise\_notrace} with \funcname{RESTORE\_EXN\_HANDLER\_OCAML}
        \end{itemize}
        }
        \item<4-> Jumps into \funcname{caml\_raise\_exn}
        \item<5-> Calls \funcname{caml\_stash\_backtrace} again, to scan the next part of the stack: from the trap just pop-ed to the next trap
      \end{itemize}
      \vfill
      \mintocaml[firstline=15,lastline=21,highlightlines={18-21}]{../src/backtrace/backtrace.ml}
      \endminipage
    \end{column}
    \begin{column}{0.5\textwidth}
      \raggedleft
      \onslide*<1>{\listCamlReraiseExn{}}
      \onslide*<2>{\listCamlReraiseExn[highlightlines=729]{}}
      \onslide*<3>{
        \mintc[firstline=190,lastline=192,highlightlines={191-192}]{../ocaml/runtime/amd64.S}
        \mintc[firstline=234,lastline=237,highlightlines={235-237}]{../ocaml/runtime/amd64.S}
        \medskip
        \listCamlReraiseExn[highlightlines=731]{}
      }
      \onslide*<4-5>{
        % TODO refactor with \listCamlReraiseExn
        \mintasm[firstline=727,lastline=734,highlightlines=730]{../ocaml/runtime/amd64.S}
        \medskip
      }
      \onslide*<4>{\listCamlRaiseExn[highlightlines=711]{}}
      \onslide*<5>{\listCamlRaiseExn[highlightlines=720]{}}
    \end{column}
  \end{columns}
\end{frame}

\begin{frame}{Backtraces}{Backtrace collection continues}
  \begin{columns}[c]
    \begin{column}{0.55\textwidth}
      \minipage[c][0.75\textheight][s]{\columnwidth}
      \begin{itemize}
        \item<1-> \funcname{caml\_stash\_backtrace} scans the older part of the stack, up to the next trap
        \item<6-> Controls jumps to the nested exception handler in \funcname{maybe\_raise}, which matches only on \typename{ExnB}
        \item<7-> \funcname{caml\_reraise\_exn} is called again, backtrace collection resumes with \funcname{caml\_stash\_backtrace}
      \end{itemize}
      \medskip
      \begin{itemize}
        \item<2-> Constructed backtrace:
        \begin{itemize}
          \item <2-> \funcname{definitely\_raise}
          \item <2-> \funcname{probably\_raise}
          \item <3-> \funcname{probably\_raise} (re-raise)
          \item <4-> \funcname{maybe\_raise}
          \item <5-> \funcname{main}
          \item <8-> \funcname{main} (re-raise)
        \end{itemize}
      \end{itemize}
      \vfill
      \onslide*<1-5>{\mintocaml[firstline=15,lastline=21]{../src/backtrace/backtrace.ml}}
      \onslide*<6>{\mintocaml[firstline=28,lastline=34,highlightlines=31]{../src/backtrace/backtrace.ml}}
      \onslide*<7->{\mintocaml[firstline=28,lastline=34,highlightlines=33]{../src/backtrace/backtrace.ml}}
      \endminipage
    \end{column}
    \begin{column}{0.45\textwidth}
      \raggedleft
      \onslide*<1>{\providecommand\step{6}\input{diagrams/backtrace/backtrace.tex}}%
      \onslide*<2>{\providecommand\step{6}\input{diagrams/backtrace/backtrace.tex}}%
      \onslide*<3>{\providecommand\step{7}\input{diagrams/backtrace/backtrace.tex}}%
      \onslide*<4>{\providecommand\step{8}\input{diagrams/backtrace/backtrace.tex}}%
      \onslide*<5>{\providecommand\step{9}\input{diagrams/backtrace/backtrace.tex}}%
      \onslide*<6>{\providecommand\step{10}\input{diagrams/backtrace/backtrace.tex}}%
      \onslide*<7>{\providecommand\step{11}\input{diagrams/backtrace/backtrace.tex}}%
      \onslide*<8>{\providecommand\step{12}\input{diagrams/backtrace/backtrace.tex}}%
      \onslide*<9>{\providecommand\step{13}\input{diagrams/backtrace/backtrace.tex}}%
    \end{column}
  \end{columns}
\end{frame}

\begin{frame}{Backtraces}{Printing the backtrace}
  \begin{columns}[c]
    \begin{column}{0.45\textwidth}
      \minipage[c][0.66\textheight][s]{\columnwidth}
      \begin{itemize}
        \item<1-> The exception backtrace can now be printed to \funcarg{stdout} with \funcname{Printexc.print_backtrace}
        \item<2-> First the exception backtrace is retrieved into an OCaml array with \funcname{get_raw_backtrace }
        \item<3-> Which is implemented as the \funcname{caml_get_exception_raw_backtrace} C function in the runtime
        \item<4-> And finally printed with \funcname{print_exception_backtrace}
      \end{itemize}
      \vfill
      \mintocaml[firstline=28,lastline=34,highlightlines=33]{../src/backtrace/backtrace.ml}
      \endminipage
    \end{column}
    \begin{column}{0.55\textwidth}
      \onslide*<1,2>{
        \mintocaml[firstline=100,lastline=101]{../ocaml/stdlib/printexc.ml}
      }
      \only<1>{
        \listocaml[firstline=170,lastline=172]{../ocaml/stdlib/printexc.ml}{stdlib/printexc.ml:170}
      }
      \only<2>{
        \listocaml[firstline=170,lastline=172,highlightlines=172]{../ocaml/stdlib/printexc.ml}{stdlib/printexc.ml:170}
      }
      \only<3>{
        \mintc[firstline=154,lastline=156]{../ocaml/runtime/backtrace.c}
        \listc[firstline=171,lastline=192,highlightlines={184-187}]{../ocaml/runtime/backtrace.c}{runtime/backtrace.c:154}
      }
      \only<4>{
        \listocaml[firstline=155,lastline=165]{../ocaml/stdlib/printexc.ml}{stdlib/printexc.ml:55}
      }
    \end{column}
  \end{columns}
\end{frame}


\subsection{The multiple kinds of raise}
\frameSubsection{}{}

\begin{frame}{Backtraces}{When backtraces are not needed}
  \begin{columns}[c]
    \begin{column}{0.45\textwidth}
      \minipage[c][0.66\textheight][s]{\columnwidth}
      \begin{itemize}
        \item<1-> What if the backtrace for an exception is never needed?
        \item<2-> It's possible to raise the exception explicitly with \funcname{raise\_notrace} to make raising as cheap as possible
        \item<3-> An example of use in the \funcname{dynlink} library of the OCaml compiler
      \end{itemize}
      \vfill
      \onslide<3->{
      \listocaml[firstline=205,lastline=209,highlightlines=207]{../ocaml/otherlibs/dynlink/dynlink\_compilerlibs/misc.ml}{otherlibs/dynlink/dynlink\_compilerlibs/misc.ml:205}
      }
      \endminipage
    \end{column}
    \begin{column}{0.55\textwidth}
    \only<4>{
      \mintcmm[highlightlines=3]{../src/notrace/camlNotrace\_\_fun_347.cmm}
      \listcmm{../src/notrace/camlNotrace\_\_all\_somes\_327.cmm}{all\_somes}
    }
    \onslide*<5->{
      \listobjdump[highlightlines={6-9}]{../src/notrace/camlNotrace\_\_fun_347.objdump}{anonymous function from \funcname{all\_somes}}
    }
    \end{column}
  \end{columns}
\end{frame}

\begin{frame}{Backtraces}{Summary table of the multiple kinds of raise}
  \begin{itemize}
    \item When debugging information is enabled and if the backtrace of an exception isn't needed, it's possible to raise with \funcname{raise\_notrace} for performance
    \item When debugging information is \emph{not} enabled, the compiler optimises \funcname{raise} calls to inline assembly (same effect as using \funcname{raise\_notrace})
  \end{itemize}
  \bigskip
  \centering
  \begin{tabular}{| l | c | c | c | c |}
    \hline
    \parbox{10em}{Debug information \\ (\funcarg{-g} flag)}  & \multicolumn{2}{|c|}{Disabled}                                     & \multicolumn{2}{|c|}{Enabled} \\
    \hline
    Raise kind                                               & \funcname{raise} & \funcname{raise\_notrace}                       & \funcname{raise} & \funcname{raise\_notrace} \\
    \hline
    Compilation                                              & \parbox{8em}{inline assembly\\ (optimization)} & inline assembly   & \funcname{caml\_raise\_exn} & inline assembly \\
    \hline
  \end{tabular}
\end{frame}


%
% Takeaway #5
%
\subsection*{Takeaway \#5}
\frameSubsectionTakeaway{}
\againframe<5>{takeaway}
}%
    \end{column}
  \end{columns}
\end{frame}

\begin{frame}{Backtraces}{Back to \funcname{caml\_raise\_exn}}
  \begin{columns}[c]
    \begin{column}{0.5\textwidth}
      \minipage[c][0.66\textheight][s]{\columnwidth}
      \begin{itemize}\color{gray}
        \item Checks wheteher exceptions backtrace should be recorded, by checking the \funcarg{domain\_state->backtrace\_active} value
        \item Switches to the C stack to be able to execute C code
        \item Calls \funcname{caml\_stash\_backtrace}, to collect the backtrace up to the next trap
      \end{itemize}
      \onslide*<2->{
        \begin{itemize}
          \item<2-> Executes the exception handler in \funcname{probably\_raise} with \funcname{RESTORE\_EXN\_HANDLER\_OCAML}
            \begin{itemize}
              \item Set \regname{RSP} to \localname{domain\_state->exn\_handler} value
              \item Restore \localname{domain\_state->exn\_handler} from trap
              \item Jump to the handler code with \funcname{ret}
            \end{itemize}
        \end{itemize}
      }
      \vfill
      \onslide*<1>{\mintocaml[firstline=9,lastline=13,highlightlines=12]{../src/backtrace/backtrace.ml}}
      \onslide*<2->{\mintocaml[firstline=15,lastline=21,highlightlines={19-21}]{../src/backtrace/backtrace.ml}}
      \endminipage
    \end{column}
    \begin{column}{0.5\textwidth}
      \centering
      \onslide*<1>{
        \mintc[firstline=190,lastline=192]{../ocaml/runtime/amd64.S}
        \mintc[firstline=234,lastline=237]{../ocaml/runtime/amd64.S}
      }
      \onslide*<2>{
        \mintc[firstline=190,lastline=192,highlightlines={191-192}]{../ocaml/runtime/amd64.S}
        \mintc[firstline=234,lastline=237,highlightlines={235-237}]{../ocaml/runtime/amd64.S}
      }
      \onslide*<1>{\listCamlRaiseExn[highlightlines=720]{}}
      \onslide*<2>{\listCamlRaiseExn[highlightlines=722]{}}
      \onslide*<3->{\providecommand\step{5}%
% Backtraces
%
\subsection{One advantage of raising with debugging information}
\frameSubsection{}{}

\begin{frame}{Backtraces}{Debugging information \& source code}
  \begin{columns}[c]
    \begin{column}{0.5\textwidth}
      \begin{itemize}
        \item Raising an exception is fun and games, but how do we know where the exception originated? $\rightarrow$ With the backtrace associated with the exception!
        \item In order to get a backtrace from the point where an exception was raised, it's necessary to compile with debugging information enabled (\funcarg{-g} flag for the compiler)
        \item Same example as in the `Nested handlers' section
      \end{itemize}
    \end{column}
    \begin{column}{0.5\textwidth}
      \listocaml[firstline=9,lastline=34]{../src/backtrace/backtrace.ml}{backtrace/backtrace.ml}
    \end{column}
  \end{columns}
\end{frame}

\begin{frame}{Backtraces}{CMM and disassembly of \funcname{definitely\_raise}}
  \begin{columns}[c]
    \begin{column}{0.5\textwidth}
      \minipage[c][0.66\textheight][s]{\columnwidth}
      \begin{itemize}
        \item<1-> An effect of compiling with debugging information (\funcarg{-g}): the CMM no longer uses \funcname{raise\_notrace} to raise the exception but \funcname{raise} instead
        \item<2-> Disassembly shows that CMM \funcname{raise} is actually a call to \funcname{caml\_raise\_exn}, a function in assembler provided by the runtime
      \end{itemize}
      \vfill
      \mintocaml[firstline=9,lastline=13,highlightlines=12]{../src/backtrace/backtrace.ml}
      \endminipage
    \end{column}
    \begin{column}{0.5\textwidth}
      \onslide*<1>{
        \mintcmm[highlightlines=5]{../src/backtrace/camlBacktrace\_\_definitely\_raise\_347.cmm}
      }
      \onslide*<2>{
        \mintobjdump[firstline=2,highlightlines=14]{../src/backtrace/camlBacktrace\_\_definitely\_raise\_347.objdump}
      }
    \end{column}
  \end{columns}
\end{frame}

\begin{frame}{Backtraces}{Introducing \funcname{caml\_raise\_exn}}
  \begin{columns}[c]
    \begin{column}{0.5\textwidth}
      \minipage[c][0.66\textheight][s]{\columnwidth}
      \onslide*<1>{
        \begin{itemize}
          \item Raising the exception is performed using \funcname{caml\_raise\_exn} which is
            \begin{itemize}
              \item An assembly function provided by the runtime
              \item Architecture specific
            \end{itemize}
          \item Let's see what it does differently from \funcname{raise\_notrace}\dots
          \item We are about to raise \typename{ExnC} with \funcname{caml\_raise\_exn} from \funcname{definitely\_raise}
        \end{itemize}
      }
      \onslide*<2>{
        \mintobjdump[firstline=2,highlightlines=14]{../src/backtrace/camlBacktrace\_\_definitely\_raise\_347.objdump}
      }
      \vfill
      \mintocaml[firstline=9,lastline=13,highlightlines=12]{../src/backtrace/backtrace.ml}
      \endminipage
    \end{column}
    \begin{column}{0.5\textwidth}
      \centering
      \providecommand\step{1}\input{diagrams/backtrace/backtrace.tex}
    \end{column}
  \end{columns}
\end{frame}

\begin{frame}{Backtraces}{But before that, we need more fields from \typename{caml\_domain\_state}}
  \begin{columns}
    \begin{column}{0.5\textwidth}
      \begin{itemize}
        \item Remember \typename{caml\_domain\_state} the per-domain runtime state?
        \item Some other members are of interest to us now\dots
        \item \localname{backtrace\_active} is used to know if the runtime should collect backtraces
        \item \localname{backtrace\_active} can be set by
          \begin{itemize}
            \item The \funcarg{b} flag in the \funcname{OCAMLRUNPARAM} environment variable
            \item \mintinline{ocaml}{Printexc.record_backtrace : bool -> unit} \footnotemark
          \end{itemize}
      \end{itemize}
    \end{column}
    \begin{column}{0.5\textwidth}
      \begin{listing}
        \mintc[highlightlines={9-11}]{src/ocaml/domain\_state\_backtrace.h}
        \caption{runtime/caml/domain\_state.h}
      \end{listing}
    \end{column}
  \end{columns}
  \footnotetext[1]{https://v2.ocaml.org/api/Printexc.html}
\end{frame}

\newcommand\listCamlRaiseExn[1][]{
  \listasm[firstline=702,lastline=725,#1]{../ocaml/runtime/amd64.S}{runtime/amd64.S:702}}

\begin{frame}{Backtraces}{Walk through \funcname{caml\_raise\_exn}}
  \begin{columns}[T]
    \begin{column}{0.5\textwidth}
      \begin{itemize}
        \item<1-> First checks whether exceptions backtrace should be recorded
        \item<1-> By checking the \funcarg{domain\_state->backtrace\_active} value
        \item<2-> If not, acts as \funcname{raise\_notrace} with \funcname{RESTORE\_EXN\_HANDLER\_OCAML}
          \begin{itemize}
            \item Set \regname{RSP} to \localname{domain\_state->exn\_handler} value
            \item Restore \localname{domain\_state->exn\_handler} from trap
            \item Jump with \funcname{ret} (same effect as using \funcname{pop} and \funcname{jmp})
          \end{itemize}
        \item<3-> Otherwise, switches to the C stack to be able to execute C code
        \item<4-> Calls \funcname{caml\_stash\_backtrace}, a C function provided by the runtime\ignorespaces
          \onslide<6->{, with
            \begin{itemize}
              \item \funcarg{exn}: exception value
              \item \funcarg{pc}: code address of the raising point
              \item \funcarg{sp}: stack address of the raising function
              \item \funcarg{trapsp}: stack address of the next handler
            \end{itemize}
          }
      \end{itemize}
      \bigskip
      \mintocaml[firstline=9,lastline=13,highlightlines=12]{../src/backtrace/backtrace.ml}
    \end{column}
    \begin{column}{0.5\textwidth}
      \raggedleft
      \onslide*<1,3-4>{
        \mintc[firstline=190,lastline=192]{../ocaml/runtime/amd64.S}
        \mintc[firstline=234,lastline=237]{../ocaml/runtime/amd64.S}
      }
      \onslide*<2>{
        \mintc[firstline=190,lastline=192,highlightlines={191-192}]{../ocaml/runtime/amd64.S}
        \mintc[firstline=234,lastline=237,highlightlines={235-237}]{../ocaml/runtime/amd64.S}
      }
      \onslide*<1>{\listCamlRaiseExn[highlightlines=705]{}}%
      \onslide*<2>{\listCamlRaiseExn[highlightlines={707-708}]{}}%
      \onslide*<3>{\listCamlRaiseExn[highlightlines={712-714}]{}}%
      \onslide*<4>{\listCamlRaiseExn[highlightlines={715-720}]{}}%
      \onslide*<5>{\providecommand\step{1}\input{diagrams/backtrace/backtrace.tex}}%
      \onslide*<6>{\providecommand\step{2}\input{diagrams/backtrace/backtrace.tex}}%
    \end{column}
  \end{columns}
\end{frame}

\newcommand\listStashBacktrace[1][]{
  \listc[firstline=91,lastline=120,#1]{../ocaml/runtime/backtrace\_nat.c}{runtime/backtrace\_nat.c:91}}

\begin{frame}{Backtraces}{Introducing \funcname{caml\_stash\_backtrace}}
  \begin{columns}[c]
    \begin{column}{0.5\textwidth}
      \minipage[c][0.66\textheight][s]{\columnwidth}
      \begin{itemize}
        \item<1-> A C function provided by the runtime (specific to native code)
        \item<2-> It scans the stack, between the stack pointer at the raising point up to the next trap, in order to collect the names of the detected function frames
          \onslide*<2>{
            \begin{itemize}
              \item And more information, like line number, column number...
            \end{itemize}
          }
        \item<3-> It uses the same technique and information as the Garbage Collector (GC) uses to scan the values live on the stack
          \onslide*<4>{
            \begin{itemize}
              \item \typename{frame\_descr} are at the core of stack unwinding
            \end{itemize}
          }
      \end{itemize}
      \vfill
      \mintocaml[firstline=9,lastline=13,highlightlines=12]{../src/backtrace/backtrace.ml}
      \endminipage
    \end{column}
    \begin{column}{0.5\textwidth}
      \onslide*<1>{\listStashBacktrace[highlightlines=91]{}}
      \onslide*<2>{\listStashBacktrace[highlightlines={91,117-118}]{}}
      \onslide*<3->{\listStashBacktrace[highlightlines={94,106,109-110}]{}}
    \end{column}
  \end{columns}
\end{frame}

\newcommand\listFrameDescr[1][]{
  \listocaml[firstline=111,lastline=124,#1]{../ocaml/asmcomp/emitaux.ml}{asmcomp/emitaux.ml:111}}
\newcommand\listDebugInfo[1][]{
  \listocaml[firstline=38,lastline=49,#1]{../ocaml/lambda/debuginfo.mli}{lambda/debuginfo.mli:38}}

\begin{frame}{Backtraces}{\typename{frame\_descr} and \typename{frame\_debuginfo} in a nutshell}
  \begin{columns}[c]
    \begin{column}{0.5\textwidth}
      \begin{itemize}
        \item<1-> For every return address in an OCaml program, there is a associated \typename{frame\_descr}
        \item<2-> A \typename{frame\_descr} specifies
        \begin{itemize}
          \item<2-> The size of the function stack frame
          \item<3-> Where live values are on the stack (so that the GC can mark them as live and prevent from being swept)
        \end{itemize}
        \item<4-> When compiled with debug information, \typename{frame\_descr} will be linked to a \typename{frame\_debuginfo}
        \begin{itemize}
          \item<5-> Contains information about the position in the source code (file name, line and column number)
        \end{itemize}
      \end{itemize}
    \end{column}
    \begin{column}{0.5\textwidth}
        \onslide*<1>{\listFrameDescr[highlightlines={118-122}]{}}
        \onslide*<2>{\listFrameDescr[highlightlines={120}]{}}
        \onslide*<3>{\listFrameDescr[highlightlines={121}]{}}
        \onslide*<4->{\listFrameDescr[highlightlines={122}]{}}
        \medskip
        \onslide*<1-3>{\listDebugInfo{}}
        \onslide*<4>{\listDebugInfo[highlightlines={38,49}]{}}
        \onslide*<5>{\listDebugInfo[highlightlines={39-46}]{}}
    \end{column}
  \end{columns}
\end{frame}

\begin{frame}{Backtraces}{Scanning the stack with \typename{frame\_descr}}
  \begin{columns}[c]
    \begin{column}{0.5\textwidth}
      \begin{itemize}
        \item Given the return address of the code that raises the exception by calling \funcname{caml\_raise\_exn} (\funcarg{pc} argument of \funcname{caml\_stash\_backtrace})
        \item And the position of the stack pointer (\funcarg{sp} argument of \funcname{caml\_stash\_backtrace})
        \item The frame size given by \typename{frame\_descr}.\funcarg{frame\_size}, allows to find the end of the previous frame
        \item And the return address of the caller of this function
        \bigskip
        \onslide<3->{
        \item Note that traps (here the reddish \funcname{ExnA}, \funcname{ExnB} and \funcname{ExnC} blocks) belong to the frame that installed them, and so the frame size changes when traps are removed
        }
      \end{itemize}
    \end{column}
    \begin{column}{0.5\textwidth}
      \raggedleft
      \onslide*<1>{\providecommand\step{2}\input{diagrams/backtrace/backtrace.tex}}%
      \onslide*<2>{\providecommand\step{1}\input{diagrams/backtrace/scanstack.tex}}%
      \onslide*<3>{\providecommand\step{2}\input{diagrams/backtrace/scanstack.tex}}%
      \onslide*<4>{\providecommand\step{3}\input{diagrams/backtrace/scanstack.tex}}%
      \onslide*<5>{\providecommand\step{4}\input{diagrams/backtrace/scanstack.tex}}%
      \onslide*<6>{\providecommand\step{5}\input{diagrams/backtrace/scanstack.tex}}%
    \end{column}
  \end{columns}
\end{frame}

% Duplicate of previous-previous slide
\begin{frame}{Backtraces}{Continuing on \funcname{caml\_stash\_backtrace}}
  \begin{columns}[c]
    \begin{column}{0.5\textwidth}
      \minipage[c][0.75\textheight][s]{\columnwidth}
      \begin{itemize}
          \color{gray}
        \item A C function provided by the runtime (specific to native code)
        \item It scans the stack, between the stack pointer at the raising point up to the next trap, in order to collect the names of the detected function frames
        \item It uses the same technique and information as the Garbage Collector (GC) uses to scan the values live on the stack
      \end{itemize}
      \begin{itemize}
        \item For each function frame detected, store a pointer to the \typename{frame\_descr} in the \localname{domain\_state->backtrace\_buffer} array, effectively constructing the backtrace
      \end{itemize}
      \onslide<2->{
        \begin{itemize}
            \smallskip
          \item Constructed backtrace:
            \funcname{definitely\_raise},
            \onslide<3->{\funcname{probably\_raise}}
        \end{itemize}
      }
      \vfill
      \mintocaml[firstline=9,lastline=13,highlightlines=12]{../src/backtrace/backtrace.ml}
      \endminipage
    \end{column}
    \begin{column}{0.5\textwidth}
      \raggedleft
      \onslide*<1>{\listStashBacktrace[highlightlines={114-115}]{}}
      \onslide*<2>{\providecommand\step{3}\input{diagrams/backtrace/backtrace.tex}}%
      \onslide*<3>{\providecommand\step{4}\input{diagrams/backtrace/backtrace.tex}}%
    \end{column}
  \end{columns}
\end{frame}

\begin{frame}{Backtraces}{Back to \funcname{caml\_raise\_exn}}
  \begin{columns}[c]
    \begin{column}{0.5\textwidth}
      \minipage[c][0.66\textheight][s]{\columnwidth}
      \begin{itemize}\color{gray}
        \item Checks wheteher exceptions backtrace should be recorded, by checking the \funcarg{domain\_state->backtrace\_active} value
        \item Switches to the C stack to be able to execute C code
        \item Calls \funcname{caml\_stash\_backtrace}, to collect the backtrace up to the next trap
      \end{itemize}
      \onslide*<2->{
        \begin{itemize}
          \item<2-> Executes the exception handler in \funcname{probably\_raise} with \funcname{RESTORE\_EXN\_HANDLER\_OCAML}
            \begin{itemize}
              \item Set \regname{RSP} to \localname{domain\_state->exn\_handler} value
              \item Restore \localname{domain\_state->exn\_handler} from trap
              \item Jump to the handler code with \funcname{ret}
            \end{itemize}
        \end{itemize}
      }
      \vfill
      \onslide*<1>{\mintocaml[firstline=9,lastline=13,highlightlines=12]{../src/backtrace/backtrace.ml}}
      \onslide*<2->{\mintocaml[firstline=15,lastline=21,highlightlines={19-21}]{../src/backtrace/backtrace.ml}}
      \endminipage
    \end{column}
    \begin{column}{0.5\textwidth}
      \centering
      \onslide*<1>{
        \mintc[firstline=190,lastline=192]{../ocaml/runtime/amd64.S}
        \mintc[firstline=234,lastline=237]{../ocaml/runtime/amd64.S}
      }
      \onslide*<2>{
        \mintc[firstline=190,lastline=192,highlightlines={191-192}]{../ocaml/runtime/amd64.S}
        \mintc[firstline=234,lastline=237,highlightlines={235-237}]{../ocaml/runtime/amd64.S}
      }
      \onslide*<1>{\listCamlRaiseExn[highlightlines=720]{}}
      \onslide*<2>{\listCamlRaiseExn[highlightlines=722]{}}
      \onslide*<3->{\providecommand\step{5}\input{diagrams/backtrace/backtrace.tex}}
    \end{column}
  \end{columns}
\end{frame}

\begin{frame}{Backtraces}{\funcname{caml\_probably\_raise}'s handler doesn't catch \typename{ExnC}}
  \begin{columns}[c]
    \begin{column}{0.45\textwidth}
      \minipage[c][0.75\textheight][s]{\columnwidth}
      \begin{itemize}
        \item<1-> \funcname{match \dots with}\footnotemark pattern matching can also match exceptions
        \item<1-> The CMM of \funcname{probably\_raise} shows that this \funcname{match \dots with} is implemented with a pattern matching and a \funcname{try \dots with}
        \item<1-> But this one only matches on \typename{ExnA} exception
        \item<2-> Since the pattern matching fails to catch the exception, \funcname{reraise} is called with our current \typename{ExnC} exception
        \item<3-> Disassembly shows that \funcname{reraise} is actually a call to \funcname{caml\_reraise\_exn}, which is also an assembly function provided by the runtime
      \end{itemize}
      \vfill
      \mintocaml[firstline=15,lastline=21,highlightlines={18-21}]{../src/backtrace/backtrace.ml}
      \endminipage
    \end{column}
    \begin{column}{0.55\textwidth}
      \centering
      \onslide*<1>{\mintcmm[highlightlines={10-18}]{../src/backtrace/camlBacktrace\_\_probably\_raise\_351.cmm}}
      \onslide*<2>{\listcmm[highlightlines=18]{../src/backtrace/camlBacktrace\_\_probably\_raise_351.cmm}{CMM of probably\_raise}}
      \onslide*<3>{\mintobjdump[firstline=5,lastline=35,highlightlines=31]{../src/backtrace/camlBacktrace\_\_probably\_raise_351.objdump}}
    \end{column}
  \end{columns}
  \only<1>{\footnotetext[1]{https://blog.janestreet.com/pattern-matching-and-exception-handling-unite/}}
\end{frame}

\newcommand\listCamlReraiseExn[1][]{
  \listasm[firstline=727,lastline=734,#1]{../ocaml/runtime/amd64.S}{runtime/amd64.S:727}}

\begin{frame}{Backtraces}{Introducing \funcname{caml\_reraise\_exn}}
  \begin{columns}[c]
    \begin{column}{0.5\textwidth}
      \minipage[c][0.66\textheight][s]{\columnwidth}
      \begin{itemize}
        \item<1-> The exception have to be reraised to the next handler
        \item<2-> First, checks if the backtrace should be collected with \funcarg{domain\_state->backtrace\_active}
        \only<3>{
        \begin{itemize}
            \item If not, behaves like a \funcname{raise\_notrace} with \funcname{RESTORE\_EXN\_HANDLER\_OCAML}
        \end{itemize}
        }
        \item<4-> Jumps into \funcname{caml\_raise\_exn}
        \item<5-> Calls \funcname{caml\_stash\_backtrace} again, to scan the next part of the stack: from the trap just pop-ed to the next trap
      \end{itemize}
      \vfill
      \mintocaml[firstline=15,lastline=21,highlightlines={18-21}]{../src/backtrace/backtrace.ml}
      \endminipage
    \end{column}
    \begin{column}{0.5\textwidth}
      \raggedleft
      \onslide*<1>{\listCamlReraiseExn{}}
      \onslide*<2>{\listCamlReraiseExn[highlightlines=729]{}}
      \onslide*<3>{
        \mintc[firstline=190,lastline=192,highlightlines={191-192}]{../ocaml/runtime/amd64.S}
        \mintc[firstline=234,lastline=237,highlightlines={235-237}]{../ocaml/runtime/amd64.S}
        \medskip
        \listCamlReraiseExn[highlightlines=731]{}
      }
      \onslide*<4-5>{
        % TODO refactor with \listCamlReraiseExn
        \mintasm[firstline=727,lastline=734,highlightlines=730]{../ocaml/runtime/amd64.S}
        \medskip
      }
      \onslide*<4>{\listCamlRaiseExn[highlightlines=711]{}}
      \onslide*<5>{\listCamlRaiseExn[highlightlines=720]{}}
    \end{column}
  \end{columns}
\end{frame}

\begin{frame}{Backtraces}{Backtrace collection continues}
  \begin{columns}[c]
    \begin{column}{0.55\textwidth}
      \minipage[c][0.75\textheight][s]{\columnwidth}
      \begin{itemize}
        \item<1-> \funcname{caml\_stash\_backtrace} scans the older part of the stack, up to the next trap
        \item<6-> Controls jumps to the nested exception handler in \funcname{maybe\_raise}, which matches only on \typename{ExnB}
        \item<7-> \funcname{caml\_reraise\_exn} is called again, backtrace collection resumes with \funcname{caml\_stash\_backtrace}
      \end{itemize}
      \medskip
      \begin{itemize}
        \item<2-> Constructed backtrace:
        \begin{itemize}
          \item <2-> \funcname{definitely\_raise}
          \item <2-> \funcname{probably\_raise}
          \item <3-> \funcname{probably\_raise} (re-raise)
          \item <4-> \funcname{maybe\_raise}
          \item <5-> \funcname{main}
          \item <8-> \funcname{main} (re-raise)
        \end{itemize}
      \end{itemize}
      \vfill
      \onslide*<1-5>{\mintocaml[firstline=15,lastline=21]{../src/backtrace/backtrace.ml}}
      \onslide*<6>{\mintocaml[firstline=28,lastline=34,highlightlines=31]{../src/backtrace/backtrace.ml}}
      \onslide*<7->{\mintocaml[firstline=28,lastline=34,highlightlines=33]{../src/backtrace/backtrace.ml}}
      \endminipage
    \end{column}
    \begin{column}{0.45\textwidth}
      \raggedleft
      \onslide*<1>{\providecommand\step{6}\input{diagrams/backtrace/backtrace.tex}}%
      \onslide*<2>{\providecommand\step{6}\input{diagrams/backtrace/backtrace.tex}}%
      \onslide*<3>{\providecommand\step{7}\input{diagrams/backtrace/backtrace.tex}}%
      \onslide*<4>{\providecommand\step{8}\input{diagrams/backtrace/backtrace.tex}}%
      \onslide*<5>{\providecommand\step{9}\input{diagrams/backtrace/backtrace.tex}}%
      \onslide*<6>{\providecommand\step{10}\input{diagrams/backtrace/backtrace.tex}}%
      \onslide*<7>{\providecommand\step{11}\input{diagrams/backtrace/backtrace.tex}}%
      \onslide*<8>{\providecommand\step{12}\input{diagrams/backtrace/backtrace.tex}}%
      \onslide*<9>{\providecommand\step{13}\input{diagrams/backtrace/backtrace.tex}}%
    \end{column}
  \end{columns}
\end{frame}

\begin{frame}{Backtraces}{Printing the backtrace}
  \begin{columns}[c]
    \begin{column}{0.45\textwidth}
      \minipage[c][0.66\textheight][s]{\columnwidth}
      \begin{itemize}
        \item<1-> The exception backtrace can now be printed to \funcarg{stdout} with \funcname{Printexc.print_backtrace}
        \item<2-> First the exception backtrace is retrieved into an OCaml array with \funcname{get_raw_backtrace }
        \item<3-> Which is implemented as the \funcname{caml_get_exception_raw_backtrace} C function in the runtime
        \item<4-> And finally printed with \funcname{print_exception_backtrace}
      \end{itemize}
      \vfill
      \mintocaml[firstline=28,lastline=34,highlightlines=33]{../src/backtrace/backtrace.ml}
      \endminipage
    \end{column}
    \begin{column}{0.55\textwidth}
      \onslide*<1,2>{
        \mintocaml[firstline=100,lastline=101]{../ocaml/stdlib/printexc.ml}
      }
      \only<1>{
        \listocaml[firstline=170,lastline=172]{../ocaml/stdlib/printexc.ml}{stdlib/printexc.ml:170}
      }
      \only<2>{
        \listocaml[firstline=170,lastline=172,highlightlines=172]{../ocaml/stdlib/printexc.ml}{stdlib/printexc.ml:170}
      }
      \only<3>{
        \mintc[firstline=154,lastline=156]{../ocaml/runtime/backtrace.c}
        \listc[firstline=171,lastline=192,highlightlines={184-187}]{../ocaml/runtime/backtrace.c}{runtime/backtrace.c:154}
      }
      \only<4>{
        \listocaml[firstline=155,lastline=165]{../ocaml/stdlib/printexc.ml}{stdlib/printexc.ml:55}
      }
    \end{column}
  \end{columns}
\end{frame}


\subsection{The multiple kinds of raise}
\frameSubsection{}{}

\begin{frame}{Backtraces}{When backtraces are not needed}
  \begin{columns}[c]
    \begin{column}{0.45\textwidth}
      \minipage[c][0.66\textheight][s]{\columnwidth}
      \begin{itemize}
        \item<1-> What if the backtrace for an exception is never needed?
        \item<2-> It's possible to raise the exception explicitly with \funcname{raise\_notrace} to make raising as cheap as possible
        \item<3-> An example of use in the \funcname{dynlink} library of the OCaml compiler
      \end{itemize}
      \vfill
      \onslide<3->{
      \listocaml[firstline=205,lastline=209,highlightlines=207]{../ocaml/otherlibs/dynlink/dynlink\_compilerlibs/misc.ml}{otherlibs/dynlink/dynlink\_compilerlibs/misc.ml:205}
      }
      \endminipage
    \end{column}
    \begin{column}{0.55\textwidth}
    \only<4>{
      \mintcmm[highlightlines=3]{../src/notrace/camlNotrace\_\_fun_347.cmm}
      \listcmm{../src/notrace/camlNotrace\_\_all\_somes\_327.cmm}{all\_somes}
    }
    \onslide*<5->{
      \listobjdump[highlightlines={6-9}]{../src/notrace/camlNotrace\_\_fun_347.objdump}{anonymous function from \funcname{all\_somes}}
    }
    \end{column}
  \end{columns}
\end{frame}

\begin{frame}{Backtraces}{Summary table of the multiple kinds of raise}
  \begin{itemize}
    \item When debugging information is enabled and if the backtrace of an exception isn't needed, it's possible to raise with \funcname{raise\_notrace} for performance
    \item When debugging information is \emph{not} enabled, the compiler optimises \funcname{raise} calls to inline assembly (same effect as using \funcname{raise\_notrace})
  \end{itemize}
  \bigskip
  \centering
  \begin{tabular}{| l | c | c | c | c |}
    \hline
    \parbox{10em}{Debug information \\ (\funcarg{-g} flag)}  & \multicolumn{2}{|c|}{Disabled}                                     & \multicolumn{2}{|c|}{Enabled} \\
    \hline
    Raise kind                                               & \funcname{raise} & \funcname{raise\_notrace}                       & \funcname{raise} & \funcname{raise\_notrace} \\
    \hline
    Compilation                                              & \parbox{8em}{inline assembly\\ (optimization)} & inline assembly   & \funcname{caml\_raise\_exn} & inline assembly \\
    \hline
  \end{tabular}
\end{frame}


%
% Takeaway #5
%
\subsection*{Takeaway \#5}
\frameSubsectionTakeaway{}
\againframe<5>{takeaway}
}
    \end{column}
  \end{columns}
\end{frame}

\begin{frame}{Backtraces}{\funcname{caml\_probably\_raise}'s handler doesn't catch \typename{ExnC}}
  \begin{columns}[c]
    \begin{column}{0.45\textwidth}
      \minipage[c][0.75\textheight][s]{\columnwidth}
      \begin{itemize}
        \item<1-> \funcname{match \dots with}\footnotemark pattern matching can also match exceptions
        \item<1-> The CMM of \funcname{probably\_raise} shows that this \funcname{match \dots with} is implemented with a pattern matching and a \funcname{try \dots with}
        \item<1-> But this one only matches on \typename{ExnA} exception
        \item<2-> Since the pattern matching fails to catch the exception, \funcname{reraise} is called with our current \typename{ExnC} exception
        \item<3-> Disassembly shows that \funcname{reraise} is actually a call to \funcname{caml\_reraise\_exn}, which is also an assembly function provided by the runtime
      \end{itemize}
      \vfill
      \mintocaml[firstline=15,lastline=21,highlightlines={18-21}]{../src/backtrace/backtrace.ml}
      \endminipage
    \end{column}
    \begin{column}{0.55\textwidth}
      \centering
      \onslide*<1>{\mintcmm[highlightlines={10-18}]{../src/backtrace/camlBacktrace\_\_probably\_raise\_351.cmm}}
      \onslide*<2>{\listcmm[highlightlines=18]{../src/backtrace/camlBacktrace\_\_probably\_raise_351.cmm}{CMM of probably\_raise}}
      \onslide*<3>{\mintobjdump[firstline=5,lastline=35,highlightlines=31]{../src/backtrace/camlBacktrace\_\_probably\_raise_351.objdump}}
    \end{column}
  \end{columns}
  \only<1>{\footnotetext[1]{https://blog.janestreet.com/pattern-matching-and-exception-handling-unite/}}
\end{frame}

\newcommand\listCamlReraiseExn[1][]{
  \listasm[firstline=727,lastline=734,#1]{../ocaml/runtime/amd64.S}{runtime/amd64.S:727}}

\begin{frame}{Backtraces}{Introducing \funcname{caml\_reraise\_exn}}
  \begin{columns}[c]
    \begin{column}{0.5\textwidth}
      \minipage[c][0.66\textheight][s]{\columnwidth}
      \begin{itemize}
        \item<1-> The exception have to be reraised to the next handler
        \item<2-> First, checks if the backtrace should be collected with \funcarg{domain\_state->backtrace\_active}
        \only<3>{
        \begin{itemize}
            \item If not, behaves like a \funcname{raise\_notrace} with \funcname{RESTORE\_EXN\_HANDLER\_OCAML}
        \end{itemize}
        }
        \item<4-> Jumps into \funcname{caml\_raise\_exn}
        \item<5-> Calls \funcname{caml\_stash\_backtrace} again, to scan the next part of the stack: from the trap just pop-ed to the next trap
      \end{itemize}
      \vfill
      \mintocaml[firstline=15,lastline=21,highlightlines={18-21}]{../src/backtrace/backtrace.ml}
      \endminipage
    \end{column}
    \begin{column}{0.5\textwidth}
      \raggedleft
      \onslide*<1>{\listCamlReraiseExn{}}
      \onslide*<2>{\listCamlReraiseExn[highlightlines=729]{}}
      \onslide*<3>{
        \mintc[firstline=190,lastline=192,highlightlines={191-192}]{../ocaml/runtime/amd64.S}
        \mintc[firstline=234,lastline=237,highlightlines={235-237}]{../ocaml/runtime/amd64.S}
        \medskip
        \listCamlReraiseExn[highlightlines=731]{}
      }
      \onslide*<4-5>{
        % TODO refactor with \listCamlReraiseExn
        \mintasm[firstline=727,lastline=734,highlightlines=730]{../ocaml/runtime/amd64.S}
        \medskip
      }
      \onslide*<4>{\listCamlRaiseExn[highlightlines=711]{}}
      \onslide*<5>{\listCamlRaiseExn[highlightlines=720]{}}
    \end{column}
  \end{columns}
\end{frame}

\begin{frame}{Backtraces}{Backtrace collection continues}
  \begin{columns}[c]
    \begin{column}{0.55\textwidth}
      \minipage[c][0.75\textheight][s]{\columnwidth}
      \begin{itemize}
        \item<1-> \funcname{caml\_stash\_backtrace} scans the older part of the stack, up to the next trap
        \item<6-> Controls jumps to the nested exception handler in \funcname{maybe\_raise}, which matches only on \typename{ExnB}
        \item<7-> \funcname{caml\_reraise\_exn} is called again, backtrace collection resumes with \funcname{caml\_stash\_backtrace}
      \end{itemize}
      \medskip
      \begin{itemize}
        \item<2-> Constructed backtrace:
        \begin{itemize}
          \item <2-> \funcname{definitely\_raise}
          \item <2-> \funcname{probably\_raise}
          \item <3-> \funcname{probably\_raise} (re-raise)
          \item <4-> \funcname{maybe\_raise}
          \item <5-> \funcname{main}
          \item <8-> \funcname{main} (re-raise)
        \end{itemize}
      \end{itemize}
      \vfill
      \onslide*<1-5>{\mintocaml[firstline=15,lastline=21]{../src/backtrace/backtrace.ml}}
      \onslide*<6>{\mintocaml[firstline=28,lastline=34,highlightlines=31]{../src/backtrace/backtrace.ml}}
      \onslide*<7->{\mintocaml[firstline=28,lastline=34,highlightlines=33]{../src/backtrace/backtrace.ml}}
      \endminipage
    \end{column}
    \begin{column}{0.45\textwidth}
      \raggedleft
      \onslide*<1>{\providecommand\step{6}%
% Backtraces
%
\subsection{One advantage of raising with debugging information}
\frameSubsection{}{}

\begin{frame}{Backtraces}{Debugging information \& source code}
  \begin{columns}[c]
    \begin{column}{0.5\textwidth}
      \begin{itemize}
        \item Raising an exception is fun and games, but how do we know where the exception originated? $\rightarrow$ With the backtrace associated with the exception!
        \item In order to get a backtrace from the point where an exception was raised, it's necessary to compile with debugging information enabled (\funcarg{-g} flag for the compiler)
        \item Same example as in the `Nested handlers' section
      \end{itemize}
    \end{column}
    \begin{column}{0.5\textwidth}
      \listocaml[firstline=9,lastline=34]{../src/backtrace/backtrace.ml}{backtrace/backtrace.ml}
    \end{column}
  \end{columns}
\end{frame}

\begin{frame}{Backtraces}{CMM and disassembly of \funcname{definitely\_raise}}
  \begin{columns}[c]
    \begin{column}{0.5\textwidth}
      \minipage[c][0.66\textheight][s]{\columnwidth}
      \begin{itemize}
        \item<1-> An effect of compiling with debugging information (\funcarg{-g}): the CMM no longer uses \funcname{raise\_notrace} to raise the exception but \funcname{raise} instead
        \item<2-> Disassembly shows that CMM \funcname{raise} is actually a call to \funcname{caml\_raise\_exn}, a function in assembler provided by the runtime
      \end{itemize}
      \vfill
      \mintocaml[firstline=9,lastline=13,highlightlines=12]{../src/backtrace/backtrace.ml}
      \endminipage
    \end{column}
    \begin{column}{0.5\textwidth}
      \onslide*<1>{
        \mintcmm[highlightlines=5]{../src/backtrace/camlBacktrace\_\_definitely\_raise\_347.cmm}
      }
      \onslide*<2>{
        \mintobjdump[firstline=2,highlightlines=14]{../src/backtrace/camlBacktrace\_\_definitely\_raise\_347.objdump}
      }
    \end{column}
  \end{columns}
\end{frame}

\begin{frame}{Backtraces}{Introducing \funcname{caml\_raise\_exn}}
  \begin{columns}[c]
    \begin{column}{0.5\textwidth}
      \minipage[c][0.66\textheight][s]{\columnwidth}
      \onslide*<1>{
        \begin{itemize}
          \item Raising the exception is performed using \funcname{caml\_raise\_exn} which is
            \begin{itemize}
              \item An assembly function provided by the runtime
              \item Architecture specific
            \end{itemize}
          \item Let's see what it does differently from \funcname{raise\_notrace}\dots
          \item We are about to raise \typename{ExnC} with \funcname{caml\_raise\_exn} from \funcname{definitely\_raise}
        \end{itemize}
      }
      \onslide*<2>{
        \mintobjdump[firstline=2,highlightlines=14]{../src/backtrace/camlBacktrace\_\_definitely\_raise\_347.objdump}
      }
      \vfill
      \mintocaml[firstline=9,lastline=13,highlightlines=12]{../src/backtrace/backtrace.ml}
      \endminipage
    \end{column}
    \begin{column}{0.5\textwidth}
      \centering
      \providecommand\step{1}\input{diagrams/backtrace/backtrace.tex}
    \end{column}
  \end{columns}
\end{frame}

\begin{frame}{Backtraces}{But before that, we need more fields from \typename{caml\_domain\_state}}
  \begin{columns}
    \begin{column}{0.5\textwidth}
      \begin{itemize}
        \item Remember \typename{caml\_domain\_state} the per-domain runtime state?
        \item Some other members are of interest to us now\dots
        \item \localname{backtrace\_active} is used to know if the runtime should collect backtraces
        \item \localname{backtrace\_active} can be set by
          \begin{itemize}
            \item The \funcarg{b} flag in the \funcname{OCAMLRUNPARAM} environment variable
            \item \mintinline{ocaml}{Printexc.record_backtrace : bool -> unit} \footnotemark
          \end{itemize}
      \end{itemize}
    \end{column}
    \begin{column}{0.5\textwidth}
      \begin{listing}
        \mintc[highlightlines={9-11}]{src/ocaml/domain\_state\_backtrace.h}
        \caption{runtime/caml/domain\_state.h}
      \end{listing}
    \end{column}
  \end{columns}
  \footnotetext[1]{https://v2.ocaml.org/api/Printexc.html}
\end{frame}

\newcommand\listCamlRaiseExn[1][]{
  \listasm[firstline=702,lastline=725,#1]{../ocaml/runtime/amd64.S}{runtime/amd64.S:702}}

\begin{frame}{Backtraces}{Walk through \funcname{caml\_raise\_exn}}
  \begin{columns}[T]
    \begin{column}{0.5\textwidth}
      \begin{itemize}
        \item<1-> First checks whether exceptions backtrace should be recorded
        \item<1-> By checking the \funcarg{domain\_state->backtrace\_active} value
        \item<2-> If not, acts as \funcname{raise\_notrace} with \funcname{RESTORE\_EXN\_HANDLER\_OCAML}
          \begin{itemize}
            \item Set \regname{RSP} to \localname{domain\_state->exn\_handler} value
            \item Restore \localname{domain\_state->exn\_handler} from trap
            \item Jump with \funcname{ret} (same effect as using \funcname{pop} and \funcname{jmp})
          \end{itemize}
        \item<3-> Otherwise, switches to the C stack to be able to execute C code
        \item<4-> Calls \funcname{caml\_stash\_backtrace}, a C function provided by the runtime\ignorespaces
          \onslide<6->{, with
            \begin{itemize}
              \item \funcarg{exn}: exception value
              \item \funcarg{pc}: code address of the raising point
              \item \funcarg{sp}: stack address of the raising function
              \item \funcarg{trapsp}: stack address of the next handler
            \end{itemize}
          }
      \end{itemize}
      \bigskip
      \mintocaml[firstline=9,lastline=13,highlightlines=12]{../src/backtrace/backtrace.ml}
    \end{column}
    \begin{column}{0.5\textwidth}
      \raggedleft
      \onslide*<1,3-4>{
        \mintc[firstline=190,lastline=192]{../ocaml/runtime/amd64.S}
        \mintc[firstline=234,lastline=237]{../ocaml/runtime/amd64.S}
      }
      \onslide*<2>{
        \mintc[firstline=190,lastline=192,highlightlines={191-192}]{../ocaml/runtime/amd64.S}
        \mintc[firstline=234,lastline=237,highlightlines={235-237}]{../ocaml/runtime/amd64.S}
      }
      \onslide*<1>{\listCamlRaiseExn[highlightlines=705]{}}%
      \onslide*<2>{\listCamlRaiseExn[highlightlines={707-708}]{}}%
      \onslide*<3>{\listCamlRaiseExn[highlightlines={712-714}]{}}%
      \onslide*<4>{\listCamlRaiseExn[highlightlines={715-720}]{}}%
      \onslide*<5>{\providecommand\step{1}\input{diagrams/backtrace/backtrace.tex}}%
      \onslide*<6>{\providecommand\step{2}\input{diagrams/backtrace/backtrace.tex}}%
    \end{column}
  \end{columns}
\end{frame}

\newcommand\listStashBacktrace[1][]{
  \listc[firstline=91,lastline=120,#1]{../ocaml/runtime/backtrace\_nat.c}{runtime/backtrace\_nat.c:91}}

\begin{frame}{Backtraces}{Introducing \funcname{caml\_stash\_backtrace}}
  \begin{columns}[c]
    \begin{column}{0.5\textwidth}
      \minipage[c][0.66\textheight][s]{\columnwidth}
      \begin{itemize}
        \item<1-> A C function provided by the runtime (specific to native code)
        \item<2-> It scans the stack, between the stack pointer at the raising point up to the next trap, in order to collect the names of the detected function frames
          \onslide*<2>{
            \begin{itemize}
              \item And more information, like line number, column number...
            \end{itemize}
          }
        \item<3-> It uses the same technique and information as the Garbage Collector (GC) uses to scan the values live on the stack
          \onslide*<4>{
            \begin{itemize}
              \item \typename{frame\_descr} are at the core of stack unwinding
            \end{itemize}
          }
      \end{itemize}
      \vfill
      \mintocaml[firstline=9,lastline=13,highlightlines=12]{../src/backtrace/backtrace.ml}
      \endminipage
    \end{column}
    \begin{column}{0.5\textwidth}
      \onslide*<1>{\listStashBacktrace[highlightlines=91]{}}
      \onslide*<2>{\listStashBacktrace[highlightlines={91,117-118}]{}}
      \onslide*<3->{\listStashBacktrace[highlightlines={94,106,109-110}]{}}
    \end{column}
  \end{columns}
\end{frame}

\newcommand\listFrameDescr[1][]{
  \listocaml[firstline=111,lastline=124,#1]{../ocaml/asmcomp/emitaux.ml}{asmcomp/emitaux.ml:111}}
\newcommand\listDebugInfo[1][]{
  \listocaml[firstline=38,lastline=49,#1]{../ocaml/lambda/debuginfo.mli}{lambda/debuginfo.mli:38}}

\begin{frame}{Backtraces}{\typename{frame\_descr} and \typename{frame\_debuginfo} in a nutshell}
  \begin{columns}[c]
    \begin{column}{0.5\textwidth}
      \begin{itemize}
        \item<1-> For every return address in an OCaml program, there is a associated \typename{frame\_descr}
        \item<2-> A \typename{frame\_descr} specifies
        \begin{itemize}
          \item<2-> The size of the function stack frame
          \item<3-> Where live values are on the stack (so that the GC can mark them as live and prevent from being swept)
        \end{itemize}
        \item<4-> When compiled with debug information, \typename{frame\_descr} will be linked to a \typename{frame\_debuginfo}
        \begin{itemize}
          \item<5-> Contains information about the position in the source code (file name, line and column number)
        \end{itemize}
      \end{itemize}
    \end{column}
    \begin{column}{0.5\textwidth}
        \onslide*<1>{\listFrameDescr[highlightlines={118-122}]{}}
        \onslide*<2>{\listFrameDescr[highlightlines={120}]{}}
        \onslide*<3>{\listFrameDescr[highlightlines={121}]{}}
        \onslide*<4->{\listFrameDescr[highlightlines={122}]{}}
        \medskip
        \onslide*<1-3>{\listDebugInfo{}}
        \onslide*<4>{\listDebugInfo[highlightlines={38,49}]{}}
        \onslide*<5>{\listDebugInfo[highlightlines={39-46}]{}}
    \end{column}
  \end{columns}
\end{frame}

\begin{frame}{Backtraces}{Scanning the stack with \typename{frame\_descr}}
  \begin{columns}[c]
    \begin{column}{0.5\textwidth}
      \begin{itemize}
        \item Given the return address of the code that raises the exception by calling \funcname{caml\_raise\_exn} (\funcarg{pc} argument of \funcname{caml\_stash\_backtrace})
        \item And the position of the stack pointer (\funcarg{sp} argument of \funcname{caml\_stash\_backtrace})
        \item The frame size given by \typename{frame\_descr}.\funcarg{frame\_size}, allows to find the end of the previous frame
        \item And the return address of the caller of this function
        \bigskip
        \onslide<3->{
        \item Note that traps (here the reddish \funcname{ExnA}, \funcname{ExnB} and \funcname{ExnC} blocks) belong to the frame that installed them, and so the frame size changes when traps are removed
        }
      \end{itemize}
    \end{column}
    \begin{column}{0.5\textwidth}
      \raggedleft
      \onslide*<1>{\providecommand\step{2}\input{diagrams/backtrace/backtrace.tex}}%
      \onslide*<2>{\providecommand\step{1}\input{diagrams/backtrace/scanstack.tex}}%
      \onslide*<3>{\providecommand\step{2}\input{diagrams/backtrace/scanstack.tex}}%
      \onslide*<4>{\providecommand\step{3}\input{diagrams/backtrace/scanstack.tex}}%
      \onslide*<5>{\providecommand\step{4}\input{diagrams/backtrace/scanstack.tex}}%
      \onslide*<6>{\providecommand\step{5}\input{diagrams/backtrace/scanstack.tex}}%
    \end{column}
  \end{columns}
\end{frame}

% Duplicate of previous-previous slide
\begin{frame}{Backtraces}{Continuing on \funcname{caml\_stash\_backtrace}}
  \begin{columns}[c]
    \begin{column}{0.5\textwidth}
      \minipage[c][0.75\textheight][s]{\columnwidth}
      \begin{itemize}
          \color{gray}
        \item A C function provided by the runtime (specific to native code)
        \item It scans the stack, between the stack pointer at the raising point up to the next trap, in order to collect the names of the detected function frames
        \item It uses the same technique and information as the Garbage Collector (GC) uses to scan the values live on the stack
      \end{itemize}
      \begin{itemize}
        \item For each function frame detected, store a pointer to the \typename{frame\_descr} in the \localname{domain\_state->backtrace\_buffer} array, effectively constructing the backtrace
      \end{itemize}
      \onslide<2->{
        \begin{itemize}
            \smallskip
          \item Constructed backtrace:
            \funcname{definitely\_raise},
            \onslide<3->{\funcname{probably\_raise}}
        \end{itemize}
      }
      \vfill
      \mintocaml[firstline=9,lastline=13,highlightlines=12]{../src/backtrace/backtrace.ml}
      \endminipage
    \end{column}
    \begin{column}{0.5\textwidth}
      \raggedleft
      \onslide*<1>{\listStashBacktrace[highlightlines={114-115}]{}}
      \onslide*<2>{\providecommand\step{3}\input{diagrams/backtrace/backtrace.tex}}%
      \onslide*<3>{\providecommand\step{4}\input{diagrams/backtrace/backtrace.tex}}%
    \end{column}
  \end{columns}
\end{frame}

\begin{frame}{Backtraces}{Back to \funcname{caml\_raise\_exn}}
  \begin{columns}[c]
    \begin{column}{0.5\textwidth}
      \minipage[c][0.66\textheight][s]{\columnwidth}
      \begin{itemize}\color{gray}
        \item Checks wheteher exceptions backtrace should be recorded, by checking the \funcarg{domain\_state->backtrace\_active} value
        \item Switches to the C stack to be able to execute C code
        \item Calls \funcname{caml\_stash\_backtrace}, to collect the backtrace up to the next trap
      \end{itemize}
      \onslide*<2->{
        \begin{itemize}
          \item<2-> Executes the exception handler in \funcname{probably\_raise} with \funcname{RESTORE\_EXN\_HANDLER\_OCAML}
            \begin{itemize}
              \item Set \regname{RSP} to \localname{domain\_state->exn\_handler} value
              \item Restore \localname{domain\_state->exn\_handler} from trap
              \item Jump to the handler code with \funcname{ret}
            \end{itemize}
        \end{itemize}
      }
      \vfill
      \onslide*<1>{\mintocaml[firstline=9,lastline=13,highlightlines=12]{../src/backtrace/backtrace.ml}}
      \onslide*<2->{\mintocaml[firstline=15,lastline=21,highlightlines={19-21}]{../src/backtrace/backtrace.ml}}
      \endminipage
    \end{column}
    \begin{column}{0.5\textwidth}
      \centering
      \onslide*<1>{
        \mintc[firstline=190,lastline=192]{../ocaml/runtime/amd64.S}
        \mintc[firstline=234,lastline=237]{../ocaml/runtime/amd64.S}
      }
      \onslide*<2>{
        \mintc[firstline=190,lastline=192,highlightlines={191-192}]{../ocaml/runtime/amd64.S}
        \mintc[firstline=234,lastline=237,highlightlines={235-237}]{../ocaml/runtime/amd64.S}
      }
      \onslide*<1>{\listCamlRaiseExn[highlightlines=720]{}}
      \onslide*<2>{\listCamlRaiseExn[highlightlines=722]{}}
      \onslide*<3->{\providecommand\step{5}\input{diagrams/backtrace/backtrace.tex}}
    \end{column}
  \end{columns}
\end{frame}

\begin{frame}{Backtraces}{\funcname{caml\_probably\_raise}'s handler doesn't catch \typename{ExnC}}
  \begin{columns}[c]
    \begin{column}{0.45\textwidth}
      \minipage[c][0.75\textheight][s]{\columnwidth}
      \begin{itemize}
        \item<1-> \funcname{match \dots with}\footnotemark pattern matching can also match exceptions
        \item<1-> The CMM of \funcname{probably\_raise} shows that this \funcname{match \dots with} is implemented with a pattern matching and a \funcname{try \dots with}
        \item<1-> But this one only matches on \typename{ExnA} exception
        \item<2-> Since the pattern matching fails to catch the exception, \funcname{reraise} is called with our current \typename{ExnC} exception
        \item<3-> Disassembly shows that \funcname{reraise} is actually a call to \funcname{caml\_reraise\_exn}, which is also an assembly function provided by the runtime
      \end{itemize}
      \vfill
      \mintocaml[firstline=15,lastline=21,highlightlines={18-21}]{../src/backtrace/backtrace.ml}
      \endminipage
    \end{column}
    \begin{column}{0.55\textwidth}
      \centering
      \onslide*<1>{\mintcmm[highlightlines={10-18}]{../src/backtrace/camlBacktrace\_\_probably\_raise\_351.cmm}}
      \onslide*<2>{\listcmm[highlightlines=18]{../src/backtrace/camlBacktrace\_\_probably\_raise_351.cmm}{CMM of probably\_raise}}
      \onslide*<3>{\mintobjdump[firstline=5,lastline=35,highlightlines=31]{../src/backtrace/camlBacktrace\_\_probably\_raise_351.objdump}}
    \end{column}
  \end{columns}
  \only<1>{\footnotetext[1]{https://blog.janestreet.com/pattern-matching-and-exception-handling-unite/}}
\end{frame}

\newcommand\listCamlReraiseExn[1][]{
  \listasm[firstline=727,lastline=734,#1]{../ocaml/runtime/amd64.S}{runtime/amd64.S:727}}

\begin{frame}{Backtraces}{Introducing \funcname{caml\_reraise\_exn}}
  \begin{columns}[c]
    \begin{column}{0.5\textwidth}
      \minipage[c][0.66\textheight][s]{\columnwidth}
      \begin{itemize}
        \item<1-> The exception have to be reraised to the next handler
        \item<2-> First, checks if the backtrace should be collected with \funcarg{domain\_state->backtrace\_active}
        \only<3>{
        \begin{itemize}
            \item If not, behaves like a \funcname{raise\_notrace} with \funcname{RESTORE\_EXN\_HANDLER\_OCAML}
        \end{itemize}
        }
        \item<4-> Jumps into \funcname{caml\_raise\_exn}
        \item<5-> Calls \funcname{caml\_stash\_backtrace} again, to scan the next part of the stack: from the trap just pop-ed to the next trap
      \end{itemize}
      \vfill
      \mintocaml[firstline=15,lastline=21,highlightlines={18-21}]{../src/backtrace/backtrace.ml}
      \endminipage
    \end{column}
    \begin{column}{0.5\textwidth}
      \raggedleft
      \onslide*<1>{\listCamlReraiseExn{}}
      \onslide*<2>{\listCamlReraiseExn[highlightlines=729]{}}
      \onslide*<3>{
        \mintc[firstline=190,lastline=192,highlightlines={191-192}]{../ocaml/runtime/amd64.S}
        \mintc[firstline=234,lastline=237,highlightlines={235-237}]{../ocaml/runtime/amd64.S}
        \medskip
        \listCamlReraiseExn[highlightlines=731]{}
      }
      \onslide*<4-5>{
        % TODO refactor with \listCamlReraiseExn
        \mintasm[firstline=727,lastline=734,highlightlines=730]{../ocaml/runtime/amd64.S}
        \medskip
      }
      \onslide*<4>{\listCamlRaiseExn[highlightlines=711]{}}
      \onslide*<5>{\listCamlRaiseExn[highlightlines=720]{}}
    \end{column}
  \end{columns}
\end{frame}

\begin{frame}{Backtraces}{Backtrace collection continues}
  \begin{columns}[c]
    \begin{column}{0.55\textwidth}
      \minipage[c][0.75\textheight][s]{\columnwidth}
      \begin{itemize}
        \item<1-> \funcname{caml\_stash\_backtrace} scans the older part of the stack, up to the next trap
        \item<6-> Controls jumps to the nested exception handler in \funcname{maybe\_raise}, which matches only on \typename{ExnB}
        \item<7-> \funcname{caml\_reraise\_exn} is called again, backtrace collection resumes with \funcname{caml\_stash\_backtrace}
      \end{itemize}
      \medskip
      \begin{itemize}
        \item<2-> Constructed backtrace:
        \begin{itemize}
          \item <2-> \funcname{definitely\_raise}
          \item <2-> \funcname{probably\_raise}
          \item <3-> \funcname{probably\_raise} (re-raise)
          \item <4-> \funcname{maybe\_raise}
          \item <5-> \funcname{main}
          \item <8-> \funcname{main} (re-raise)
        \end{itemize}
      \end{itemize}
      \vfill
      \onslide*<1-5>{\mintocaml[firstline=15,lastline=21]{../src/backtrace/backtrace.ml}}
      \onslide*<6>{\mintocaml[firstline=28,lastline=34,highlightlines=31]{../src/backtrace/backtrace.ml}}
      \onslide*<7->{\mintocaml[firstline=28,lastline=34,highlightlines=33]{../src/backtrace/backtrace.ml}}
      \endminipage
    \end{column}
    \begin{column}{0.45\textwidth}
      \raggedleft
      \onslide*<1>{\providecommand\step{6}\input{diagrams/backtrace/backtrace.tex}}%
      \onslide*<2>{\providecommand\step{6}\input{diagrams/backtrace/backtrace.tex}}%
      \onslide*<3>{\providecommand\step{7}\input{diagrams/backtrace/backtrace.tex}}%
      \onslide*<4>{\providecommand\step{8}\input{diagrams/backtrace/backtrace.tex}}%
      \onslide*<5>{\providecommand\step{9}\input{diagrams/backtrace/backtrace.tex}}%
      \onslide*<6>{\providecommand\step{10}\input{diagrams/backtrace/backtrace.tex}}%
      \onslide*<7>{\providecommand\step{11}\input{diagrams/backtrace/backtrace.tex}}%
      \onslide*<8>{\providecommand\step{12}\input{diagrams/backtrace/backtrace.tex}}%
      \onslide*<9>{\providecommand\step{13}\input{diagrams/backtrace/backtrace.tex}}%
    \end{column}
  \end{columns}
\end{frame}

\begin{frame}{Backtraces}{Printing the backtrace}
  \begin{columns}[c]
    \begin{column}{0.45\textwidth}
      \minipage[c][0.66\textheight][s]{\columnwidth}
      \begin{itemize}
        \item<1-> The exception backtrace can now be printed to \funcarg{stdout} with \funcname{Printexc.print_backtrace}
        \item<2-> First the exception backtrace is retrieved into an OCaml array with \funcname{get_raw_backtrace }
        \item<3-> Which is implemented as the \funcname{caml_get_exception_raw_backtrace} C function in the runtime
        \item<4-> And finally printed with \funcname{print_exception_backtrace}
      \end{itemize}
      \vfill
      \mintocaml[firstline=28,lastline=34,highlightlines=33]{../src/backtrace/backtrace.ml}
      \endminipage
    \end{column}
    \begin{column}{0.55\textwidth}
      \onslide*<1,2>{
        \mintocaml[firstline=100,lastline=101]{../ocaml/stdlib/printexc.ml}
      }
      \only<1>{
        \listocaml[firstline=170,lastline=172]{../ocaml/stdlib/printexc.ml}{stdlib/printexc.ml:170}
      }
      \only<2>{
        \listocaml[firstline=170,lastline=172,highlightlines=172]{../ocaml/stdlib/printexc.ml}{stdlib/printexc.ml:170}
      }
      \only<3>{
        \mintc[firstline=154,lastline=156]{../ocaml/runtime/backtrace.c}
        \listc[firstline=171,lastline=192,highlightlines={184-187}]{../ocaml/runtime/backtrace.c}{runtime/backtrace.c:154}
      }
      \only<4>{
        \listocaml[firstline=155,lastline=165]{../ocaml/stdlib/printexc.ml}{stdlib/printexc.ml:55}
      }
    \end{column}
  \end{columns}
\end{frame}


\subsection{The multiple kinds of raise}
\frameSubsection{}{}

\begin{frame}{Backtraces}{When backtraces are not needed}
  \begin{columns}[c]
    \begin{column}{0.45\textwidth}
      \minipage[c][0.66\textheight][s]{\columnwidth}
      \begin{itemize}
        \item<1-> What if the backtrace for an exception is never needed?
        \item<2-> It's possible to raise the exception explicitly with \funcname{raise\_notrace} to make raising as cheap as possible
        \item<3-> An example of use in the \funcname{dynlink} library of the OCaml compiler
      \end{itemize}
      \vfill
      \onslide<3->{
      \listocaml[firstline=205,lastline=209,highlightlines=207]{../ocaml/otherlibs/dynlink/dynlink\_compilerlibs/misc.ml}{otherlibs/dynlink/dynlink\_compilerlibs/misc.ml:205}
      }
      \endminipage
    \end{column}
    \begin{column}{0.55\textwidth}
    \only<4>{
      \mintcmm[highlightlines=3]{../src/notrace/camlNotrace\_\_fun_347.cmm}
      \listcmm{../src/notrace/camlNotrace\_\_all\_somes\_327.cmm}{all\_somes}
    }
    \onslide*<5->{
      \listobjdump[highlightlines={6-9}]{../src/notrace/camlNotrace\_\_fun_347.objdump}{anonymous function from \funcname{all\_somes}}
    }
    \end{column}
  \end{columns}
\end{frame}

\begin{frame}{Backtraces}{Summary table of the multiple kinds of raise}
  \begin{itemize}
    \item When debugging information is enabled and if the backtrace of an exception isn't needed, it's possible to raise with \funcname{raise\_notrace} for performance
    \item When debugging information is \emph{not} enabled, the compiler optimises \funcname{raise} calls to inline assembly (same effect as using \funcname{raise\_notrace})
  \end{itemize}
  \bigskip
  \centering
  \begin{tabular}{| l | c | c | c | c |}
    \hline
    \parbox{10em}{Debug information \\ (\funcarg{-g} flag)}  & \multicolumn{2}{|c|}{Disabled}                                     & \multicolumn{2}{|c|}{Enabled} \\
    \hline
    Raise kind                                               & \funcname{raise} & \funcname{raise\_notrace}                       & \funcname{raise} & \funcname{raise\_notrace} \\
    \hline
    Compilation                                              & \parbox{8em}{inline assembly\\ (optimization)} & inline assembly   & \funcname{caml\_raise\_exn} & inline assembly \\
    \hline
  \end{tabular}
\end{frame}


%
% Takeaway #5
%
\subsection*{Takeaway \#5}
\frameSubsectionTakeaway{}
\againframe<5>{takeaway}
}%
      \onslide*<2>{\providecommand\step{6}%
% Backtraces
%
\subsection{One advantage of raising with debugging information}
\frameSubsection{}{}

\begin{frame}{Backtraces}{Debugging information \& source code}
  \begin{columns}[c]
    \begin{column}{0.5\textwidth}
      \begin{itemize}
        \item Raising an exception is fun and games, but how do we know where the exception originated? $\rightarrow$ With the backtrace associated with the exception!
        \item In order to get a backtrace from the point where an exception was raised, it's necessary to compile with debugging information enabled (\funcarg{-g} flag for the compiler)
        \item Same example as in the `Nested handlers' section
      \end{itemize}
    \end{column}
    \begin{column}{0.5\textwidth}
      \listocaml[firstline=9,lastline=34]{../src/backtrace/backtrace.ml}{backtrace/backtrace.ml}
    \end{column}
  \end{columns}
\end{frame}

\begin{frame}{Backtraces}{CMM and disassembly of \funcname{definitely\_raise}}
  \begin{columns}[c]
    \begin{column}{0.5\textwidth}
      \minipage[c][0.66\textheight][s]{\columnwidth}
      \begin{itemize}
        \item<1-> An effect of compiling with debugging information (\funcarg{-g}): the CMM no longer uses \funcname{raise\_notrace} to raise the exception but \funcname{raise} instead
        \item<2-> Disassembly shows that CMM \funcname{raise} is actually a call to \funcname{caml\_raise\_exn}, a function in assembler provided by the runtime
      \end{itemize}
      \vfill
      \mintocaml[firstline=9,lastline=13,highlightlines=12]{../src/backtrace/backtrace.ml}
      \endminipage
    \end{column}
    \begin{column}{0.5\textwidth}
      \onslide*<1>{
        \mintcmm[highlightlines=5]{../src/backtrace/camlBacktrace\_\_definitely\_raise\_347.cmm}
      }
      \onslide*<2>{
        \mintobjdump[firstline=2,highlightlines=14]{../src/backtrace/camlBacktrace\_\_definitely\_raise\_347.objdump}
      }
    \end{column}
  \end{columns}
\end{frame}

\begin{frame}{Backtraces}{Introducing \funcname{caml\_raise\_exn}}
  \begin{columns}[c]
    \begin{column}{0.5\textwidth}
      \minipage[c][0.66\textheight][s]{\columnwidth}
      \onslide*<1>{
        \begin{itemize}
          \item Raising the exception is performed using \funcname{caml\_raise\_exn} which is
            \begin{itemize}
              \item An assembly function provided by the runtime
              \item Architecture specific
            \end{itemize}
          \item Let's see what it does differently from \funcname{raise\_notrace}\dots
          \item We are about to raise \typename{ExnC} with \funcname{caml\_raise\_exn} from \funcname{definitely\_raise}
        \end{itemize}
      }
      \onslide*<2>{
        \mintobjdump[firstline=2,highlightlines=14]{../src/backtrace/camlBacktrace\_\_definitely\_raise\_347.objdump}
      }
      \vfill
      \mintocaml[firstline=9,lastline=13,highlightlines=12]{../src/backtrace/backtrace.ml}
      \endminipage
    \end{column}
    \begin{column}{0.5\textwidth}
      \centering
      \providecommand\step{1}\input{diagrams/backtrace/backtrace.tex}
    \end{column}
  \end{columns}
\end{frame}

\begin{frame}{Backtraces}{But before that, we need more fields from \typename{caml\_domain\_state}}
  \begin{columns}
    \begin{column}{0.5\textwidth}
      \begin{itemize}
        \item Remember \typename{caml\_domain\_state} the per-domain runtime state?
        \item Some other members are of interest to us now\dots
        \item \localname{backtrace\_active} is used to know if the runtime should collect backtraces
        \item \localname{backtrace\_active} can be set by
          \begin{itemize}
            \item The \funcarg{b} flag in the \funcname{OCAMLRUNPARAM} environment variable
            \item \mintinline{ocaml}{Printexc.record_backtrace : bool -> unit} \footnotemark
          \end{itemize}
      \end{itemize}
    \end{column}
    \begin{column}{0.5\textwidth}
      \begin{listing}
        \mintc[highlightlines={9-11}]{src/ocaml/domain\_state\_backtrace.h}
        \caption{runtime/caml/domain\_state.h}
      \end{listing}
    \end{column}
  \end{columns}
  \footnotetext[1]{https://v2.ocaml.org/api/Printexc.html}
\end{frame}

\newcommand\listCamlRaiseExn[1][]{
  \listasm[firstline=702,lastline=725,#1]{../ocaml/runtime/amd64.S}{runtime/amd64.S:702}}

\begin{frame}{Backtraces}{Walk through \funcname{caml\_raise\_exn}}
  \begin{columns}[T]
    \begin{column}{0.5\textwidth}
      \begin{itemize}
        \item<1-> First checks whether exceptions backtrace should be recorded
        \item<1-> By checking the \funcarg{domain\_state->backtrace\_active} value
        \item<2-> If not, acts as \funcname{raise\_notrace} with \funcname{RESTORE\_EXN\_HANDLER\_OCAML}
          \begin{itemize}
            \item Set \regname{RSP} to \localname{domain\_state->exn\_handler} value
            \item Restore \localname{domain\_state->exn\_handler} from trap
            \item Jump with \funcname{ret} (same effect as using \funcname{pop} and \funcname{jmp})
          \end{itemize}
        \item<3-> Otherwise, switches to the C stack to be able to execute C code
        \item<4-> Calls \funcname{caml\_stash\_backtrace}, a C function provided by the runtime\ignorespaces
          \onslide<6->{, with
            \begin{itemize}
              \item \funcarg{exn}: exception value
              \item \funcarg{pc}: code address of the raising point
              \item \funcarg{sp}: stack address of the raising function
              \item \funcarg{trapsp}: stack address of the next handler
            \end{itemize}
          }
      \end{itemize}
      \bigskip
      \mintocaml[firstline=9,lastline=13,highlightlines=12]{../src/backtrace/backtrace.ml}
    \end{column}
    \begin{column}{0.5\textwidth}
      \raggedleft
      \onslide*<1,3-4>{
        \mintc[firstline=190,lastline=192]{../ocaml/runtime/amd64.S}
        \mintc[firstline=234,lastline=237]{../ocaml/runtime/amd64.S}
      }
      \onslide*<2>{
        \mintc[firstline=190,lastline=192,highlightlines={191-192}]{../ocaml/runtime/amd64.S}
        \mintc[firstline=234,lastline=237,highlightlines={235-237}]{../ocaml/runtime/amd64.S}
      }
      \onslide*<1>{\listCamlRaiseExn[highlightlines=705]{}}%
      \onslide*<2>{\listCamlRaiseExn[highlightlines={707-708}]{}}%
      \onslide*<3>{\listCamlRaiseExn[highlightlines={712-714}]{}}%
      \onslide*<4>{\listCamlRaiseExn[highlightlines={715-720}]{}}%
      \onslide*<5>{\providecommand\step{1}\input{diagrams/backtrace/backtrace.tex}}%
      \onslide*<6>{\providecommand\step{2}\input{diagrams/backtrace/backtrace.tex}}%
    \end{column}
  \end{columns}
\end{frame}

\newcommand\listStashBacktrace[1][]{
  \listc[firstline=91,lastline=120,#1]{../ocaml/runtime/backtrace\_nat.c}{runtime/backtrace\_nat.c:91}}

\begin{frame}{Backtraces}{Introducing \funcname{caml\_stash\_backtrace}}
  \begin{columns}[c]
    \begin{column}{0.5\textwidth}
      \minipage[c][0.66\textheight][s]{\columnwidth}
      \begin{itemize}
        \item<1-> A C function provided by the runtime (specific to native code)
        \item<2-> It scans the stack, between the stack pointer at the raising point up to the next trap, in order to collect the names of the detected function frames
          \onslide*<2>{
            \begin{itemize}
              \item And more information, like line number, column number...
            \end{itemize}
          }
        \item<3-> It uses the same technique and information as the Garbage Collector (GC) uses to scan the values live on the stack
          \onslide*<4>{
            \begin{itemize}
              \item \typename{frame\_descr} are at the core of stack unwinding
            \end{itemize}
          }
      \end{itemize}
      \vfill
      \mintocaml[firstline=9,lastline=13,highlightlines=12]{../src/backtrace/backtrace.ml}
      \endminipage
    \end{column}
    \begin{column}{0.5\textwidth}
      \onslide*<1>{\listStashBacktrace[highlightlines=91]{}}
      \onslide*<2>{\listStashBacktrace[highlightlines={91,117-118}]{}}
      \onslide*<3->{\listStashBacktrace[highlightlines={94,106,109-110}]{}}
    \end{column}
  \end{columns}
\end{frame}

\newcommand\listFrameDescr[1][]{
  \listocaml[firstline=111,lastline=124,#1]{../ocaml/asmcomp/emitaux.ml}{asmcomp/emitaux.ml:111}}
\newcommand\listDebugInfo[1][]{
  \listocaml[firstline=38,lastline=49,#1]{../ocaml/lambda/debuginfo.mli}{lambda/debuginfo.mli:38}}

\begin{frame}{Backtraces}{\typename{frame\_descr} and \typename{frame\_debuginfo} in a nutshell}
  \begin{columns}[c]
    \begin{column}{0.5\textwidth}
      \begin{itemize}
        \item<1-> For every return address in an OCaml program, there is a associated \typename{frame\_descr}
        \item<2-> A \typename{frame\_descr} specifies
        \begin{itemize}
          \item<2-> The size of the function stack frame
          \item<3-> Where live values are on the stack (so that the GC can mark them as live and prevent from being swept)
        \end{itemize}
        \item<4-> When compiled with debug information, \typename{frame\_descr} will be linked to a \typename{frame\_debuginfo}
        \begin{itemize}
          \item<5-> Contains information about the position in the source code (file name, line and column number)
        \end{itemize}
      \end{itemize}
    \end{column}
    \begin{column}{0.5\textwidth}
        \onslide*<1>{\listFrameDescr[highlightlines={118-122}]{}}
        \onslide*<2>{\listFrameDescr[highlightlines={120}]{}}
        \onslide*<3>{\listFrameDescr[highlightlines={121}]{}}
        \onslide*<4->{\listFrameDescr[highlightlines={122}]{}}
        \medskip
        \onslide*<1-3>{\listDebugInfo{}}
        \onslide*<4>{\listDebugInfo[highlightlines={38,49}]{}}
        \onslide*<5>{\listDebugInfo[highlightlines={39-46}]{}}
    \end{column}
  \end{columns}
\end{frame}

\begin{frame}{Backtraces}{Scanning the stack with \typename{frame\_descr}}
  \begin{columns}[c]
    \begin{column}{0.5\textwidth}
      \begin{itemize}
        \item Given the return address of the code that raises the exception by calling \funcname{caml\_raise\_exn} (\funcarg{pc} argument of \funcname{caml\_stash\_backtrace})
        \item And the position of the stack pointer (\funcarg{sp} argument of \funcname{caml\_stash\_backtrace})
        \item The frame size given by \typename{frame\_descr}.\funcarg{frame\_size}, allows to find the end of the previous frame
        \item And the return address of the caller of this function
        \bigskip
        \onslide<3->{
        \item Note that traps (here the reddish \funcname{ExnA}, \funcname{ExnB} and \funcname{ExnC} blocks) belong to the frame that installed them, and so the frame size changes when traps are removed
        }
      \end{itemize}
    \end{column}
    \begin{column}{0.5\textwidth}
      \raggedleft
      \onslide*<1>{\providecommand\step{2}\input{diagrams/backtrace/backtrace.tex}}%
      \onslide*<2>{\providecommand\step{1}\input{diagrams/backtrace/scanstack.tex}}%
      \onslide*<3>{\providecommand\step{2}\input{diagrams/backtrace/scanstack.tex}}%
      \onslide*<4>{\providecommand\step{3}\input{diagrams/backtrace/scanstack.tex}}%
      \onslide*<5>{\providecommand\step{4}\input{diagrams/backtrace/scanstack.tex}}%
      \onslide*<6>{\providecommand\step{5}\input{diagrams/backtrace/scanstack.tex}}%
    \end{column}
  \end{columns}
\end{frame}

% Duplicate of previous-previous slide
\begin{frame}{Backtraces}{Continuing on \funcname{caml\_stash\_backtrace}}
  \begin{columns}[c]
    \begin{column}{0.5\textwidth}
      \minipage[c][0.75\textheight][s]{\columnwidth}
      \begin{itemize}
          \color{gray}
        \item A C function provided by the runtime (specific to native code)
        \item It scans the stack, between the stack pointer at the raising point up to the next trap, in order to collect the names of the detected function frames
        \item It uses the same technique and information as the Garbage Collector (GC) uses to scan the values live on the stack
      \end{itemize}
      \begin{itemize}
        \item For each function frame detected, store a pointer to the \typename{frame\_descr} in the \localname{domain\_state->backtrace\_buffer} array, effectively constructing the backtrace
      \end{itemize}
      \onslide<2->{
        \begin{itemize}
            \smallskip
          \item Constructed backtrace:
            \funcname{definitely\_raise},
            \onslide<3->{\funcname{probably\_raise}}
        \end{itemize}
      }
      \vfill
      \mintocaml[firstline=9,lastline=13,highlightlines=12]{../src/backtrace/backtrace.ml}
      \endminipage
    \end{column}
    \begin{column}{0.5\textwidth}
      \raggedleft
      \onslide*<1>{\listStashBacktrace[highlightlines={114-115}]{}}
      \onslide*<2>{\providecommand\step{3}\input{diagrams/backtrace/backtrace.tex}}%
      \onslide*<3>{\providecommand\step{4}\input{diagrams/backtrace/backtrace.tex}}%
    \end{column}
  \end{columns}
\end{frame}

\begin{frame}{Backtraces}{Back to \funcname{caml\_raise\_exn}}
  \begin{columns}[c]
    \begin{column}{0.5\textwidth}
      \minipage[c][0.66\textheight][s]{\columnwidth}
      \begin{itemize}\color{gray}
        \item Checks wheteher exceptions backtrace should be recorded, by checking the \funcarg{domain\_state->backtrace\_active} value
        \item Switches to the C stack to be able to execute C code
        \item Calls \funcname{caml\_stash\_backtrace}, to collect the backtrace up to the next trap
      \end{itemize}
      \onslide*<2->{
        \begin{itemize}
          \item<2-> Executes the exception handler in \funcname{probably\_raise} with \funcname{RESTORE\_EXN\_HANDLER\_OCAML}
            \begin{itemize}
              \item Set \regname{RSP} to \localname{domain\_state->exn\_handler} value
              \item Restore \localname{domain\_state->exn\_handler} from trap
              \item Jump to the handler code with \funcname{ret}
            \end{itemize}
        \end{itemize}
      }
      \vfill
      \onslide*<1>{\mintocaml[firstline=9,lastline=13,highlightlines=12]{../src/backtrace/backtrace.ml}}
      \onslide*<2->{\mintocaml[firstline=15,lastline=21,highlightlines={19-21}]{../src/backtrace/backtrace.ml}}
      \endminipage
    \end{column}
    \begin{column}{0.5\textwidth}
      \centering
      \onslide*<1>{
        \mintc[firstline=190,lastline=192]{../ocaml/runtime/amd64.S}
        \mintc[firstline=234,lastline=237]{../ocaml/runtime/amd64.S}
      }
      \onslide*<2>{
        \mintc[firstline=190,lastline=192,highlightlines={191-192}]{../ocaml/runtime/amd64.S}
        \mintc[firstline=234,lastline=237,highlightlines={235-237}]{../ocaml/runtime/amd64.S}
      }
      \onslide*<1>{\listCamlRaiseExn[highlightlines=720]{}}
      \onslide*<2>{\listCamlRaiseExn[highlightlines=722]{}}
      \onslide*<3->{\providecommand\step{5}\input{diagrams/backtrace/backtrace.tex}}
    \end{column}
  \end{columns}
\end{frame}

\begin{frame}{Backtraces}{\funcname{caml\_probably\_raise}'s handler doesn't catch \typename{ExnC}}
  \begin{columns}[c]
    \begin{column}{0.45\textwidth}
      \minipage[c][0.75\textheight][s]{\columnwidth}
      \begin{itemize}
        \item<1-> \funcname{match \dots with}\footnotemark pattern matching can also match exceptions
        \item<1-> The CMM of \funcname{probably\_raise} shows that this \funcname{match \dots with} is implemented with a pattern matching and a \funcname{try \dots with}
        \item<1-> But this one only matches on \typename{ExnA} exception
        \item<2-> Since the pattern matching fails to catch the exception, \funcname{reraise} is called with our current \typename{ExnC} exception
        \item<3-> Disassembly shows that \funcname{reraise} is actually a call to \funcname{caml\_reraise\_exn}, which is also an assembly function provided by the runtime
      \end{itemize}
      \vfill
      \mintocaml[firstline=15,lastline=21,highlightlines={18-21}]{../src/backtrace/backtrace.ml}
      \endminipage
    \end{column}
    \begin{column}{0.55\textwidth}
      \centering
      \onslide*<1>{\mintcmm[highlightlines={10-18}]{../src/backtrace/camlBacktrace\_\_probably\_raise\_351.cmm}}
      \onslide*<2>{\listcmm[highlightlines=18]{../src/backtrace/camlBacktrace\_\_probably\_raise_351.cmm}{CMM of probably\_raise}}
      \onslide*<3>{\mintobjdump[firstline=5,lastline=35,highlightlines=31]{../src/backtrace/camlBacktrace\_\_probably\_raise_351.objdump}}
    \end{column}
  \end{columns}
  \only<1>{\footnotetext[1]{https://blog.janestreet.com/pattern-matching-and-exception-handling-unite/}}
\end{frame}

\newcommand\listCamlReraiseExn[1][]{
  \listasm[firstline=727,lastline=734,#1]{../ocaml/runtime/amd64.S}{runtime/amd64.S:727}}

\begin{frame}{Backtraces}{Introducing \funcname{caml\_reraise\_exn}}
  \begin{columns}[c]
    \begin{column}{0.5\textwidth}
      \minipage[c][0.66\textheight][s]{\columnwidth}
      \begin{itemize}
        \item<1-> The exception have to be reraised to the next handler
        \item<2-> First, checks if the backtrace should be collected with \funcarg{domain\_state->backtrace\_active}
        \only<3>{
        \begin{itemize}
            \item If not, behaves like a \funcname{raise\_notrace} with \funcname{RESTORE\_EXN\_HANDLER\_OCAML}
        \end{itemize}
        }
        \item<4-> Jumps into \funcname{caml\_raise\_exn}
        \item<5-> Calls \funcname{caml\_stash\_backtrace} again, to scan the next part of the stack: from the trap just pop-ed to the next trap
      \end{itemize}
      \vfill
      \mintocaml[firstline=15,lastline=21,highlightlines={18-21}]{../src/backtrace/backtrace.ml}
      \endminipage
    \end{column}
    \begin{column}{0.5\textwidth}
      \raggedleft
      \onslide*<1>{\listCamlReraiseExn{}}
      \onslide*<2>{\listCamlReraiseExn[highlightlines=729]{}}
      \onslide*<3>{
        \mintc[firstline=190,lastline=192,highlightlines={191-192}]{../ocaml/runtime/amd64.S}
        \mintc[firstline=234,lastline=237,highlightlines={235-237}]{../ocaml/runtime/amd64.S}
        \medskip
        \listCamlReraiseExn[highlightlines=731]{}
      }
      \onslide*<4-5>{
        % TODO refactor with \listCamlReraiseExn
        \mintasm[firstline=727,lastline=734,highlightlines=730]{../ocaml/runtime/amd64.S}
        \medskip
      }
      \onslide*<4>{\listCamlRaiseExn[highlightlines=711]{}}
      \onslide*<5>{\listCamlRaiseExn[highlightlines=720]{}}
    \end{column}
  \end{columns}
\end{frame}

\begin{frame}{Backtraces}{Backtrace collection continues}
  \begin{columns}[c]
    \begin{column}{0.55\textwidth}
      \minipage[c][0.75\textheight][s]{\columnwidth}
      \begin{itemize}
        \item<1-> \funcname{caml\_stash\_backtrace} scans the older part of the stack, up to the next trap
        \item<6-> Controls jumps to the nested exception handler in \funcname{maybe\_raise}, which matches only on \typename{ExnB}
        \item<7-> \funcname{caml\_reraise\_exn} is called again, backtrace collection resumes with \funcname{caml\_stash\_backtrace}
      \end{itemize}
      \medskip
      \begin{itemize}
        \item<2-> Constructed backtrace:
        \begin{itemize}
          \item <2-> \funcname{definitely\_raise}
          \item <2-> \funcname{probably\_raise}
          \item <3-> \funcname{probably\_raise} (re-raise)
          \item <4-> \funcname{maybe\_raise}
          \item <5-> \funcname{main}
          \item <8-> \funcname{main} (re-raise)
        \end{itemize}
      \end{itemize}
      \vfill
      \onslide*<1-5>{\mintocaml[firstline=15,lastline=21]{../src/backtrace/backtrace.ml}}
      \onslide*<6>{\mintocaml[firstline=28,lastline=34,highlightlines=31]{../src/backtrace/backtrace.ml}}
      \onslide*<7->{\mintocaml[firstline=28,lastline=34,highlightlines=33]{../src/backtrace/backtrace.ml}}
      \endminipage
    \end{column}
    \begin{column}{0.45\textwidth}
      \raggedleft
      \onslide*<1>{\providecommand\step{6}\input{diagrams/backtrace/backtrace.tex}}%
      \onslide*<2>{\providecommand\step{6}\input{diagrams/backtrace/backtrace.tex}}%
      \onslide*<3>{\providecommand\step{7}\input{diagrams/backtrace/backtrace.tex}}%
      \onslide*<4>{\providecommand\step{8}\input{diagrams/backtrace/backtrace.tex}}%
      \onslide*<5>{\providecommand\step{9}\input{diagrams/backtrace/backtrace.tex}}%
      \onslide*<6>{\providecommand\step{10}\input{diagrams/backtrace/backtrace.tex}}%
      \onslide*<7>{\providecommand\step{11}\input{diagrams/backtrace/backtrace.tex}}%
      \onslide*<8>{\providecommand\step{12}\input{diagrams/backtrace/backtrace.tex}}%
      \onslide*<9>{\providecommand\step{13}\input{diagrams/backtrace/backtrace.tex}}%
    \end{column}
  \end{columns}
\end{frame}

\begin{frame}{Backtraces}{Printing the backtrace}
  \begin{columns}[c]
    \begin{column}{0.45\textwidth}
      \minipage[c][0.66\textheight][s]{\columnwidth}
      \begin{itemize}
        \item<1-> The exception backtrace can now be printed to \funcarg{stdout} with \funcname{Printexc.print_backtrace}
        \item<2-> First the exception backtrace is retrieved into an OCaml array with \funcname{get_raw_backtrace }
        \item<3-> Which is implemented as the \funcname{caml_get_exception_raw_backtrace} C function in the runtime
        \item<4-> And finally printed with \funcname{print_exception_backtrace}
      \end{itemize}
      \vfill
      \mintocaml[firstline=28,lastline=34,highlightlines=33]{../src/backtrace/backtrace.ml}
      \endminipage
    \end{column}
    \begin{column}{0.55\textwidth}
      \onslide*<1,2>{
        \mintocaml[firstline=100,lastline=101]{../ocaml/stdlib/printexc.ml}
      }
      \only<1>{
        \listocaml[firstline=170,lastline=172]{../ocaml/stdlib/printexc.ml}{stdlib/printexc.ml:170}
      }
      \only<2>{
        \listocaml[firstline=170,lastline=172,highlightlines=172]{../ocaml/stdlib/printexc.ml}{stdlib/printexc.ml:170}
      }
      \only<3>{
        \mintc[firstline=154,lastline=156]{../ocaml/runtime/backtrace.c}
        \listc[firstline=171,lastline=192,highlightlines={184-187}]{../ocaml/runtime/backtrace.c}{runtime/backtrace.c:154}
      }
      \only<4>{
        \listocaml[firstline=155,lastline=165]{../ocaml/stdlib/printexc.ml}{stdlib/printexc.ml:55}
      }
    \end{column}
  \end{columns}
\end{frame}


\subsection{The multiple kinds of raise}
\frameSubsection{}{}

\begin{frame}{Backtraces}{When backtraces are not needed}
  \begin{columns}[c]
    \begin{column}{0.45\textwidth}
      \minipage[c][0.66\textheight][s]{\columnwidth}
      \begin{itemize}
        \item<1-> What if the backtrace for an exception is never needed?
        \item<2-> It's possible to raise the exception explicitly with \funcname{raise\_notrace} to make raising as cheap as possible
        \item<3-> An example of use in the \funcname{dynlink} library of the OCaml compiler
      \end{itemize}
      \vfill
      \onslide<3->{
      \listocaml[firstline=205,lastline=209,highlightlines=207]{../ocaml/otherlibs/dynlink/dynlink\_compilerlibs/misc.ml}{otherlibs/dynlink/dynlink\_compilerlibs/misc.ml:205}
      }
      \endminipage
    \end{column}
    \begin{column}{0.55\textwidth}
    \only<4>{
      \mintcmm[highlightlines=3]{../src/notrace/camlNotrace\_\_fun_347.cmm}
      \listcmm{../src/notrace/camlNotrace\_\_all\_somes\_327.cmm}{all\_somes}
    }
    \onslide*<5->{
      \listobjdump[highlightlines={6-9}]{../src/notrace/camlNotrace\_\_fun_347.objdump}{anonymous function from \funcname{all\_somes}}
    }
    \end{column}
  \end{columns}
\end{frame}

\begin{frame}{Backtraces}{Summary table of the multiple kinds of raise}
  \begin{itemize}
    \item When debugging information is enabled and if the backtrace of an exception isn't needed, it's possible to raise with \funcname{raise\_notrace} for performance
    \item When debugging information is \emph{not} enabled, the compiler optimises \funcname{raise} calls to inline assembly (same effect as using \funcname{raise\_notrace})
  \end{itemize}
  \bigskip
  \centering
  \begin{tabular}{| l | c | c | c | c |}
    \hline
    \parbox{10em}{Debug information \\ (\funcarg{-g} flag)}  & \multicolumn{2}{|c|}{Disabled}                                     & \multicolumn{2}{|c|}{Enabled} \\
    \hline
    Raise kind                                               & \funcname{raise} & \funcname{raise\_notrace}                       & \funcname{raise} & \funcname{raise\_notrace} \\
    \hline
    Compilation                                              & \parbox{8em}{inline assembly\\ (optimization)} & inline assembly   & \funcname{caml\_raise\_exn} & inline assembly \\
    \hline
  \end{tabular}
\end{frame}


%
% Takeaway #5
%
\subsection*{Takeaway \#5}
\frameSubsectionTakeaway{}
\againframe<5>{takeaway}
}%
      \onslide*<3>{\providecommand\step{7}%
% Backtraces
%
\subsection{One advantage of raising with debugging information}
\frameSubsection{}{}

\begin{frame}{Backtraces}{Debugging information \& source code}
  \begin{columns}[c]
    \begin{column}{0.5\textwidth}
      \begin{itemize}
        \item Raising an exception is fun and games, but how do we know where the exception originated? $\rightarrow$ With the backtrace associated with the exception!
        \item In order to get a backtrace from the point where an exception was raised, it's necessary to compile with debugging information enabled (\funcarg{-g} flag for the compiler)
        \item Same example as in the `Nested handlers' section
      \end{itemize}
    \end{column}
    \begin{column}{0.5\textwidth}
      \listocaml[firstline=9,lastline=34]{../src/backtrace/backtrace.ml}{backtrace/backtrace.ml}
    \end{column}
  \end{columns}
\end{frame}

\begin{frame}{Backtraces}{CMM and disassembly of \funcname{definitely\_raise}}
  \begin{columns}[c]
    \begin{column}{0.5\textwidth}
      \minipage[c][0.66\textheight][s]{\columnwidth}
      \begin{itemize}
        \item<1-> An effect of compiling with debugging information (\funcarg{-g}): the CMM no longer uses \funcname{raise\_notrace} to raise the exception but \funcname{raise} instead
        \item<2-> Disassembly shows that CMM \funcname{raise} is actually a call to \funcname{caml\_raise\_exn}, a function in assembler provided by the runtime
      \end{itemize}
      \vfill
      \mintocaml[firstline=9,lastline=13,highlightlines=12]{../src/backtrace/backtrace.ml}
      \endminipage
    \end{column}
    \begin{column}{0.5\textwidth}
      \onslide*<1>{
        \mintcmm[highlightlines=5]{../src/backtrace/camlBacktrace\_\_definitely\_raise\_347.cmm}
      }
      \onslide*<2>{
        \mintobjdump[firstline=2,highlightlines=14]{../src/backtrace/camlBacktrace\_\_definitely\_raise\_347.objdump}
      }
    \end{column}
  \end{columns}
\end{frame}

\begin{frame}{Backtraces}{Introducing \funcname{caml\_raise\_exn}}
  \begin{columns}[c]
    \begin{column}{0.5\textwidth}
      \minipage[c][0.66\textheight][s]{\columnwidth}
      \onslide*<1>{
        \begin{itemize}
          \item Raising the exception is performed using \funcname{caml\_raise\_exn} which is
            \begin{itemize}
              \item An assembly function provided by the runtime
              \item Architecture specific
            \end{itemize}
          \item Let's see what it does differently from \funcname{raise\_notrace}\dots
          \item We are about to raise \typename{ExnC} with \funcname{caml\_raise\_exn} from \funcname{definitely\_raise}
        \end{itemize}
      }
      \onslide*<2>{
        \mintobjdump[firstline=2,highlightlines=14]{../src/backtrace/camlBacktrace\_\_definitely\_raise\_347.objdump}
      }
      \vfill
      \mintocaml[firstline=9,lastline=13,highlightlines=12]{../src/backtrace/backtrace.ml}
      \endminipage
    \end{column}
    \begin{column}{0.5\textwidth}
      \centering
      \providecommand\step{1}\input{diagrams/backtrace/backtrace.tex}
    \end{column}
  \end{columns}
\end{frame}

\begin{frame}{Backtraces}{But before that, we need more fields from \typename{caml\_domain\_state}}
  \begin{columns}
    \begin{column}{0.5\textwidth}
      \begin{itemize}
        \item Remember \typename{caml\_domain\_state} the per-domain runtime state?
        \item Some other members are of interest to us now\dots
        \item \localname{backtrace\_active} is used to know if the runtime should collect backtraces
        \item \localname{backtrace\_active} can be set by
          \begin{itemize}
            \item The \funcarg{b} flag in the \funcname{OCAMLRUNPARAM} environment variable
            \item \mintinline{ocaml}{Printexc.record_backtrace : bool -> unit} \footnotemark
          \end{itemize}
      \end{itemize}
    \end{column}
    \begin{column}{0.5\textwidth}
      \begin{listing}
        \mintc[highlightlines={9-11}]{src/ocaml/domain\_state\_backtrace.h}
        \caption{runtime/caml/domain\_state.h}
      \end{listing}
    \end{column}
  \end{columns}
  \footnotetext[1]{https://v2.ocaml.org/api/Printexc.html}
\end{frame}

\newcommand\listCamlRaiseExn[1][]{
  \listasm[firstline=702,lastline=725,#1]{../ocaml/runtime/amd64.S}{runtime/amd64.S:702}}

\begin{frame}{Backtraces}{Walk through \funcname{caml\_raise\_exn}}
  \begin{columns}[T]
    \begin{column}{0.5\textwidth}
      \begin{itemize}
        \item<1-> First checks whether exceptions backtrace should be recorded
        \item<1-> By checking the \funcarg{domain\_state->backtrace\_active} value
        \item<2-> If not, acts as \funcname{raise\_notrace} with \funcname{RESTORE\_EXN\_HANDLER\_OCAML}
          \begin{itemize}
            \item Set \regname{RSP} to \localname{domain\_state->exn\_handler} value
            \item Restore \localname{domain\_state->exn\_handler} from trap
            \item Jump with \funcname{ret} (same effect as using \funcname{pop} and \funcname{jmp})
          \end{itemize}
        \item<3-> Otherwise, switches to the C stack to be able to execute C code
        \item<4-> Calls \funcname{caml\_stash\_backtrace}, a C function provided by the runtime\ignorespaces
          \onslide<6->{, with
            \begin{itemize}
              \item \funcarg{exn}: exception value
              \item \funcarg{pc}: code address of the raising point
              \item \funcarg{sp}: stack address of the raising function
              \item \funcarg{trapsp}: stack address of the next handler
            \end{itemize}
          }
      \end{itemize}
      \bigskip
      \mintocaml[firstline=9,lastline=13,highlightlines=12]{../src/backtrace/backtrace.ml}
    \end{column}
    \begin{column}{0.5\textwidth}
      \raggedleft
      \onslide*<1,3-4>{
        \mintc[firstline=190,lastline=192]{../ocaml/runtime/amd64.S}
        \mintc[firstline=234,lastline=237]{../ocaml/runtime/amd64.S}
      }
      \onslide*<2>{
        \mintc[firstline=190,lastline=192,highlightlines={191-192}]{../ocaml/runtime/amd64.S}
        \mintc[firstline=234,lastline=237,highlightlines={235-237}]{../ocaml/runtime/amd64.S}
      }
      \onslide*<1>{\listCamlRaiseExn[highlightlines=705]{}}%
      \onslide*<2>{\listCamlRaiseExn[highlightlines={707-708}]{}}%
      \onslide*<3>{\listCamlRaiseExn[highlightlines={712-714}]{}}%
      \onslide*<4>{\listCamlRaiseExn[highlightlines={715-720}]{}}%
      \onslide*<5>{\providecommand\step{1}\input{diagrams/backtrace/backtrace.tex}}%
      \onslide*<6>{\providecommand\step{2}\input{diagrams/backtrace/backtrace.tex}}%
    \end{column}
  \end{columns}
\end{frame}

\newcommand\listStashBacktrace[1][]{
  \listc[firstline=91,lastline=120,#1]{../ocaml/runtime/backtrace\_nat.c}{runtime/backtrace\_nat.c:91}}

\begin{frame}{Backtraces}{Introducing \funcname{caml\_stash\_backtrace}}
  \begin{columns}[c]
    \begin{column}{0.5\textwidth}
      \minipage[c][0.66\textheight][s]{\columnwidth}
      \begin{itemize}
        \item<1-> A C function provided by the runtime (specific to native code)
        \item<2-> It scans the stack, between the stack pointer at the raising point up to the next trap, in order to collect the names of the detected function frames
          \onslide*<2>{
            \begin{itemize}
              \item And more information, like line number, column number...
            \end{itemize}
          }
        \item<3-> It uses the same technique and information as the Garbage Collector (GC) uses to scan the values live on the stack
          \onslide*<4>{
            \begin{itemize}
              \item \typename{frame\_descr} are at the core of stack unwinding
            \end{itemize}
          }
      \end{itemize}
      \vfill
      \mintocaml[firstline=9,lastline=13,highlightlines=12]{../src/backtrace/backtrace.ml}
      \endminipage
    \end{column}
    \begin{column}{0.5\textwidth}
      \onslide*<1>{\listStashBacktrace[highlightlines=91]{}}
      \onslide*<2>{\listStashBacktrace[highlightlines={91,117-118}]{}}
      \onslide*<3->{\listStashBacktrace[highlightlines={94,106,109-110}]{}}
    \end{column}
  \end{columns}
\end{frame}

\newcommand\listFrameDescr[1][]{
  \listocaml[firstline=111,lastline=124,#1]{../ocaml/asmcomp/emitaux.ml}{asmcomp/emitaux.ml:111}}
\newcommand\listDebugInfo[1][]{
  \listocaml[firstline=38,lastline=49,#1]{../ocaml/lambda/debuginfo.mli}{lambda/debuginfo.mli:38}}

\begin{frame}{Backtraces}{\typename{frame\_descr} and \typename{frame\_debuginfo} in a nutshell}
  \begin{columns}[c]
    \begin{column}{0.5\textwidth}
      \begin{itemize}
        \item<1-> For every return address in an OCaml program, there is a associated \typename{frame\_descr}
        \item<2-> A \typename{frame\_descr} specifies
        \begin{itemize}
          \item<2-> The size of the function stack frame
          \item<3-> Where live values are on the stack (so that the GC can mark them as live and prevent from being swept)
        \end{itemize}
        \item<4-> When compiled with debug information, \typename{frame\_descr} will be linked to a \typename{frame\_debuginfo}
        \begin{itemize}
          \item<5-> Contains information about the position in the source code (file name, line and column number)
        \end{itemize}
      \end{itemize}
    \end{column}
    \begin{column}{0.5\textwidth}
        \onslide*<1>{\listFrameDescr[highlightlines={118-122}]{}}
        \onslide*<2>{\listFrameDescr[highlightlines={120}]{}}
        \onslide*<3>{\listFrameDescr[highlightlines={121}]{}}
        \onslide*<4->{\listFrameDescr[highlightlines={122}]{}}
        \medskip
        \onslide*<1-3>{\listDebugInfo{}}
        \onslide*<4>{\listDebugInfo[highlightlines={38,49}]{}}
        \onslide*<5>{\listDebugInfo[highlightlines={39-46}]{}}
    \end{column}
  \end{columns}
\end{frame}

\begin{frame}{Backtraces}{Scanning the stack with \typename{frame\_descr}}
  \begin{columns}[c]
    \begin{column}{0.5\textwidth}
      \begin{itemize}
        \item Given the return address of the code that raises the exception by calling \funcname{caml\_raise\_exn} (\funcarg{pc} argument of \funcname{caml\_stash\_backtrace})
        \item And the position of the stack pointer (\funcarg{sp} argument of \funcname{caml\_stash\_backtrace})
        \item The frame size given by \typename{frame\_descr}.\funcarg{frame\_size}, allows to find the end of the previous frame
        \item And the return address of the caller of this function
        \bigskip
        \onslide<3->{
        \item Note that traps (here the reddish \funcname{ExnA}, \funcname{ExnB} and \funcname{ExnC} blocks) belong to the frame that installed them, and so the frame size changes when traps are removed
        }
      \end{itemize}
    \end{column}
    \begin{column}{0.5\textwidth}
      \raggedleft
      \onslide*<1>{\providecommand\step{2}\input{diagrams/backtrace/backtrace.tex}}%
      \onslide*<2>{\providecommand\step{1}\input{diagrams/backtrace/scanstack.tex}}%
      \onslide*<3>{\providecommand\step{2}\input{diagrams/backtrace/scanstack.tex}}%
      \onslide*<4>{\providecommand\step{3}\input{diagrams/backtrace/scanstack.tex}}%
      \onslide*<5>{\providecommand\step{4}\input{diagrams/backtrace/scanstack.tex}}%
      \onslide*<6>{\providecommand\step{5}\input{diagrams/backtrace/scanstack.tex}}%
    \end{column}
  \end{columns}
\end{frame}

% Duplicate of previous-previous slide
\begin{frame}{Backtraces}{Continuing on \funcname{caml\_stash\_backtrace}}
  \begin{columns}[c]
    \begin{column}{0.5\textwidth}
      \minipage[c][0.75\textheight][s]{\columnwidth}
      \begin{itemize}
          \color{gray}
        \item A C function provided by the runtime (specific to native code)
        \item It scans the stack, between the stack pointer at the raising point up to the next trap, in order to collect the names of the detected function frames
        \item It uses the same technique and information as the Garbage Collector (GC) uses to scan the values live on the stack
      \end{itemize}
      \begin{itemize}
        \item For each function frame detected, store a pointer to the \typename{frame\_descr} in the \localname{domain\_state->backtrace\_buffer} array, effectively constructing the backtrace
      \end{itemize}
      \onslide<2->{
        \begin{itemize}
            \smallskip
          \item Constructed backtrace:
            \funcname{definitely\_raise},
            \onslide<3->{\funcname{probably\_raise}}
        \end{itemize}
      }
      \vfill
      \mintocaml[firstline=9,lastline=13,highlightlines=12]{../src/backtrace/backtrace.ml}
      \endminipage
    \end{column}
    \begin{column}{0.5\textwidth}
      \raggedleft
      \onslide*<1>{\listStashBacktrace[highlightlines={114-115}]{}}
      \onslide*<2>{\providecommand\step{3}\input{diagrams/backtrace/backtrace.tex}}%
      \onslide*<3>{\providecommand\step{4}\input{diagrams/backtrace/backtrace.tex}}%
    \end{column}
  \end{columns}
\end{frame}

\begin{frame}{Backtraces}{Back to \funcname{caml\_raise\_exn}}
  \begin{columns}[c]
    \begin{column}{0.5\textwidth}
      \minipage[c][0.66\textheight][s]{\columnwidth}
      \begin{itemize}\color{gray}
        \item Checks wheteher exceptions backtrace should be recorded, by checking the \funcarg{domain\_state->backtrace\_active} value
        \item Switches to the C stack to be able to execute C code
        \item Calls \funcname{caml\_stash\_backtrace}, to collect the backtrace up to the next trap
      \end{itemize}
      \onslide*<2->{
        \begin{itemize}
          \item<2-> Executes the exception handler in \funcname{probably\_raise} with \funcname{RESTORE\_EXN\_HANDLER\_OCAML}
            \begin{itemize}
              \item Set \regname{RSP} to \localname{domain\_state->exn\_handler} value
              \item Restore \localname{domain\_state->exn\_handler} from trap
              \item Jump to the handler code with \funcname{ret}
            \end{itemize}
        \end{itemize}
      }
      \vfill
      \onslide*<1>{\mintocaml[firstline=9,lastline=13,highlightlines=12]{../src/backtrace/backtrace.ml}}
      \onslide*<2->{\mintocaml[firstline=15,lastline=21,highlightlines={19-21}]{../src/backtrace/backtrace.ml}}
      \endminipage
    \end{column}
    \begin{column}{0.5\textwidth}
      \centering
      \onslide*<1>{
        \mintc[firstline=190,lastline=192]{../ocaml/runtime/amd64.S}
        \mintc[firstline=234,lastline=237]{../ocaml/runtime/amd64.S}
      }
      \onslide*<2>{
        \mintc[firstline=190,lastline=192,highlightlines={191-192}]{../ocaml/runtime/amd64.S}
        \mintc[firstline=234,lastline=237,highlightlines={235-237}]{../ocaml/runtime/amd64.S}
      }
      \onslide*<1>{\listCamlRaiseExn[highlightlines=720]{}}
      \onslide*<2>{\listCamlRaiseExn[highlightlines=722]{}}
      \onslide*<3->{\providecommand\step{5}\input{diagrams/backtrace/backtrace.tex}}
    \end{column}
  \end{columns}
\end{frame}

\begin{frame}{Backtraces}{\funcname{caml\_probably\_raise}'s handler doesn't catch \typename{ExnC}}
  \begin{columns}[c]
    \begin{column}{0.45\textwidth}
      \minipage[c][0.75\textheight][s]{\columnwidth}
      \begin{itemize}
        \item<1-> \funcname{match \dots with}\footnotemark pattern matching can also match exceptions
        \item<1-> The CMM of \funcname{probably\_raise} shows that this \funcname{match \dots with} is implemented with a pattern matching and a \funcname{try \dots with}
        \item<1-> But this one only matches on \typename{ExnA} exception
        \item<2-> Since the pattern matching fails to catch the exception, \funcname{reraise} is called with our current \typename{ExnC} exception
        \item<3-> Disassembly shows that \funcname{reraise} is actually a call to \funcname{caml\_reraise\_exn}, which is also an assembly function provided by the runtime
      \end{itemize}
      \vfill
      \mintocaml[firstline=15,lastline=21,highlightlines={18-21}]{../src/backtrace/backtrace.ml}
      \endminipage
    \end{column}
    \begin{column}{0.55\textwidth}
      \centering
      \onslide*<1>{\mintcmm[highlightlines={10-18}]{../src/backtrace/camlBacktrace\_\_probably\_raise\_351.cmm}}
      \onslide*<2>{\listcmm[highlightlines=18]{../src/backtrace/camlBacktrace\_\_probably\_raise_351.cmm}{CMM of probably\_raise}}
      \onslide*<3>{\mintobjdump[firstline=5,lastline=35,highlightlines=31]{../src/backtrace/camlBacktrace\_\_probably\_raise_351.objdump}}
    \end{column}
  \end{columns}
  \only<1>{\footnotetext[1]{https://blog.janestreet.com/pattern-matching-and-exception-handling-unite/}}
\end{frame}

\newcommand\listCamlReraiseExn[1][]{
  \listasm[firstline=727,lastline=734,#1]{../ocaml/runtime/amd64.S}{runtime/amd64.S:727}}

\begin{frame}{Backtraces}{Introducing \funcname{caml\_reraise\_exn}}
  \begin{columns}[c]
    \begin{column}{0.5\textwidth}
      \minipage[c][0.66\textheight][s]{\columnwidth}
      \begin{itemize}
        \item<1-> The exception have to be reraised to the next handler
        \item<2-> First, checks if the backtrace should be collected with \funcarg{domain\_state->backtrace\_active}
        \only<3>{
        \begin{itemize}
            \item If not, behaves like a \funcname{raise\_notrace} with \funcname{RESTORE\_EXN\_HANDLER\_OCAML}
        \end{itemize}
        }
        \item<4-> Jumps into \funcname{caml\_raise\_exn}
        \item<5-> Calls \funcname{caml\_stash\_backtrace} again, to scan the next part of the stack: from the trap just pop-ed to the next trap
      \end{itemize}
      \vfill
      \mintocaml[firstline=15,lastline=21,highlightlines={18-21}]{../src/backtrace/backtrace.ml}
      \endminipage
    \end{column}
    \begin{column}{0.5\textwidth}
      \raggedleft
      \onslide*<1>{\listCamlReraiseExn{}}
      \onslide*<2>{\listCamlReraiseExn[highlightlines=729]{}}
      \onslide*<3>{
        \mintc[firstline=190,lastline=192,highlightlines={191-192}]{../ocaml/runtime/amd64.S}
        \mintc[firstline=234,lastline=237,highlightlines={235-237}]{../ocaml/runtime/amd64.S}
        \medskip
        \listCamlReraiseExn[highlightlines=731]{}
      }
      \onslide*<4-5>{
        % TODO refactor with \listCamlReraiseExn
        \mintasm[firstline=727,lastline=734,highlightlines=730]{../ocaml/runtime/amd64.S}
        \medskip
      }
      \onslide*<4>{\listCamlRaiseExn[highlightlines=711]{}}
      \onslide*<5>{\listCamlRaiseExn[highlightlines=720]{}}
    \end{column}
  \end{columns}
\end{frame}

\begin{frame}{Backtraces}{Backtrace collection continues}
  \begin{columns}[c]
    \begin{column}{0.55\textwidth}
      \minipage[c][0.75\textheight][s]{\columnwidth}
      \begin{itemize}
        \item<1-> \funcname{caml\_stash\_backtrace} scans the older part of the stack, up to the next trap
        \item<6-> Controls jumps to the nested exception handler in \funcname{maybe\_raise}, which matches only on \typename{ExnB}
        \item<7-> \funcname{caml\_reraise\_exn} is called again, backtrace collection resumes with \funcname{caml\_stash\_backtrace}
      \end{itemize}
      \medskip
      \begin{itemize}
        \item<2-> Constructed backtrace:
        \begin{itemize}
          \item <2-> \funcname{definitely\_raise}
          \item <2-> \funcname{probably\_raise}
          \item <3-> \funcname{probably\_raise} (re-raise)
          \item <4-> \funcname{maybe\_raise}
          \item <5-> \funcname{main}
          \item <8-> \funcname{main} (re-raise)
        \end{itemize}
      \end{itemize}
      \vfill
      \onslide*<1-5>{\mintocaml[firstline=15,lastline=21]{../src/backtrace/backtrace.ml}}
      \onslide*<6>{\mintocaml[firstline=28,lastline=34,highlightlines=31]{../src/backtrace/backtrace.ml}}
      \onslide*<7->{\mintocaml[firstline=28,lastline=34,highlightlines=33]{../src/backtrace/backtrace.ml}}
      \endminipage
    \end{column}
    \begin{column}{0.45\textwidth}
      \raggedleft
      \onslide*<1>{\providecommand\step{6}\input{diagrams/backtrace/backtrace.tex}}%
      \onslide*<2>{\providecommand\step{6}\input{diagrams/backtrace/backtrace.tex}}%
      \onslide*<3>{\providecommand\step{7}\input{diagrams/backtrace/backtrace.tex}}%
      \onslide*<4>{\providecommand\step{8}\input{diagrams/backtrace/backtrace.tex}}%
      \onslide*<5>{\providecommand\step{9}\input{diagrams/backtrace/backtrace.tex}}%
      \onslide*<6>{\providecommand\step{10}\input{diagrams/backtrace/backtrace.tex}}%
      \onslide*<7>{\providecommand\step{11}\input{diagrams/backtrace/backtrace.tex}}%
      \onslide*<8>{\providecommand\step{12}\input{diagrams/backtrace/backtrace.tex}}%
      \onslide*<9>{\providecommand\step{13}\input{diagrams/backtrace/backtrace.tex}}%
    \end{column}
  \end{columns}
\end{frame}

\begin{frame}{Backtraces}{Printing the backtrace}
  \begin{columns}[c]
    \begin{column}{0.45\textwidth}
      \minipage[c][0.66\textheight][s]{\columnwidth}
      \begin{itemize}
        \item<1-> The exception backtrace can now be printed to \funcarg{stdout} with \funcname{Printexc.print_backtrace}
        \item<2-> First the exception backtrace is retrieved into an OCaml array with \funcname{get_raw_backtrace }
        \item<3-> Which is implemented as the \funcname{caml_get_exception_raw_backtrace} C function in the runtime
        \item<4-> And finally printed with \funcname{print_exception_backtrace}
      \end{itemize}
      \vfill
      \mintocaml[firstline=28,lastline=34,highlightlines=33]{../src/backtrace/backtrace.ml}
      \endminipage
    \end{column}
    \begin{column}{0.55\textwidth}
      \onslide*<1,2>{
        \mintocaml[firstline=100,lastline=101]{../ocaml/stdlib/printexc.ml}
      }
      \only<1>{
        \listocaml[firstline=170,lastline=172]{../ocaml/stdlib/printexc.ml}{stdlib/printexc.ml:170}
      }
      \only<2>{
        \listocaml[firstline=170,lastline=172,highlightlines=172]{../ocaml/stdlib/printexc.ml}{stdlib/printexc.ml:170}
      }
      \only<3>{
        \mintc[firstline=154,lastline=156]{../ocaml/runtime/backtrace.c}
        \listc[firstline=171,lastline=192,highlightlines={184-187}]{../ocaml/runtime/backtrace.c}{runtime/backtrace.c:154}
      }
      \only<4>{
        \listocaml[firstline=155,lastline=165]{../ocaml/stdlib/printexc.ml}{stdlib/printexc.ml:55}
      }
    \end{column}
  \end{columns}
\end{frame}


\subsection{The multiple kinds of raise}
\frameSubsection{}{}

\begin{frame}{Backtraces}{When backtraces are not needed}
  \begin{columns}[c]
    \begin{column}{0.45\textwidth}
      \minipage[c][0.66\textheight][s]{\columnwidth}
      \begin{itemize}
        \item<1-> What if the backtrace for an exception is never needed?
        \item<2-> It's possible to raise the exception explicitly with \funcname{raise\_notrace} to make raising as cheap as possible
        \item<3-> An example of use in the \funcname{dynlink} library of the OCaml compiler
      \end{itemize}
      \vfill
      \onslide<3->{
      \listocaml[firstline=205,lastline=209,highlightlines=207]{../ocaml/otherlibs/dynlink/dynlink\_compilerlibs/misc.ml}{otherlibs/dynlink/dynlink\_compilerlibs/misc.ml:205}
      }
      \endminipage
    \end{column}
    \begin{column}{0.55\textwidth}
    \only<4>{
      \mintcmm[highlightlines=3]{../src/notrace/camlNotrace\_\_fun_347.cmm}
      \listcmm{../src/notrace/camlNotrace\_\_all\_somes\_327.cmm}{all\_somes}
    }
    \onslide*<5->{
      \listobjdump[highlightlines={6-9}]{../src/notrace/camlNotrace\_\_fun_347.objdump}{anonymous function from \funcname{all\_somes}}
    }
    \end{column}
  \end{columns}
\end{frame}

\begin{frame}{Backtraces}{Summary table of the multiple kinds of raise}
  \begin{itemize}
    \item When debugging information is enabled and if the backtrace of an exception isn't needed, it's possible to raise with \funcname{raise\_notrace} for performance
    \item When debugging information is \emph{not} enabled, the compiler optimises \funcname{raise} calls to inline assembly (same effect as using \funcname{raise\_notrace})
  \end{itemize}
  \bigskip
  \centering
  \begin{tabular}{| l | c | c | c | c |}
    \hline
    \parbox{10em}{Debug information \\ (\funcarg{-g} flag)}  & \multicolumn{2}{|c|}{Disabled}                                     & \multicolumn{2}{|c|}{Enabled} \\
    \hline
    Raise kind                                               & \funcname{raise} & \funcname{raise\_notrace}                       & \funcname{raise} & \funcname{raise\_notrace} \\
    \hline
    Compilation                                              & \parbox{8em}{inline assembly\\ (optimization)} & inline assembly   & \funcname{caml\_raise\_exn} & inline assembly \\
    \hline
  \end{tabular}
\end{frame}


%
% Takeaway #5
%
\subsection*{Takeaway \#5}
\frameSubsectionTakeaway{}
\againframe<5>{takeaway}
}%
      \onslide*<4>{\providecommand\step{8}%
% Backtraces
%
\subsection{One advantage of raising with debugging information}
\frameSubsection{}{}

\begin{frame}{Backtraces}{Debugging information \& source code}
  \begin{columns}[c]
    \begin{column}{0.5\textwidth}
      \begin{itemize}
        \item Raising an exception is fun and games, but how do we know where the exception originated? $\rightarrow$ With the backtrace associated with the exception!
        \item In order to get a backtrace from the point where an exception was raised, it's necessary to compile with debugging information enabled (\funcarg{-g} flag for the compiler)
        \item Same example as in the `Nested handlers' section
      \end{itemize}
    \end{column}
    \begin{column}{0.5\textwidth}
      \listocaml[firstline=9,lastline=34]{../src/backtrace/backtrace.ml}{backtrace/backtrace.ml}
    \end{column}
  \end{columns}
\end{frame}

\begin{frame}{Backtraces}{CMM and disassembly of \funcname{definitely\_raise}}
  \begin{columns}[c]
    \begin{column}{0.5\textwidth}
      \minipage[c][0.66\textheight][s]{\columnwidth}
      \begin{itemize}
        \item<1-> An effect of compiling with debugging information (\funcarg{-g}): the CMM no longer uses \funcname{raise\_notrace} to raise the exception but \funcname{raise} instead
        \item<2-> Disassembly shows that CMM \funcname{raise} is actually a call to \funcname{caml\_raise\_exn}, a function in assembler provided by the runtime
      \end{itemize}
      \vfill
      \mintocaml[firstline=9,lastline=13,highlightlines=12]{../src/backtrace/backtrace.ml}
      \endminipage
    \end{column}
    \begin{column}{0.5\textwidth}
      \onslide*<1>{
        \mintcmm[highlightlines=5]{../src/backtrace/camlBacktrace\_\_definitely\_raise\_347.cmm}
      }
      \onslide*<2>{
        \mintobjdump[firstline=2,highlightlines=14]{../src/backtrace/camlBacktrace\_\_definitely\_raise\_347.objdump}
      }
    \end{column}
  \end{columns}
\end{frame}

\begin{frame}{Backtraces}{Introducing \funcname{caml\_raise\_exn}}
  \begin{columns}[c]
    \begin{column}{0.5\textwidth}
      \minipage[c][0.66\textheight][s]{\columnwidth}
      \onslide*<1>{
        \begin{itemize}
          \item Raising the exception is performed using \funcname{caml\_raise\_exn} which is
            \begin{itemize}
              \item An assembly function provided by the runtime
              \item Architecture specific
            \end{itemize}
          \item Let's see what it does differently from \funcname{raise\_notrace}\dots
          \item We are about to raise \typename{ExnC} with \funcname{caml\_raise\_exn} from \funcname{definitely\_raise}
        \end{itemize}
      }
      \onslide*<2>{
        \mintobjdump[firstline=2,highlightlines=14]{../src/backtrace/camlBacktrace\_\_definitely\_raise\_347.objdump}
      }
      \vfill
      \mintocaml[firstline=9,lastline=13,highlightlines=12]{../src/backtrace/backtrace.ml}
      \endminipage
    \end{column}
    \begin{column}{0.5\textwidth}
      \centering
      \providecommand\step{1}\input{diagrams/backtrace/backtrace.tex}
    \end{column}
  \end{columns}
\end{frame}

\begin{frame}{Backtraces}{But before that, we need more fields from \typename{caml\_domain\_state}}
  \begin{columns}
    \begin{column}{0.5\textwidth}
      \begin{itemize}
        \item Remember \typename{caml\_domain\_state} the per-domain runtime state?
        \item Some other members are of interest to us now\dots
        \item \localname{backtrace\_active} is used to know if the runtime should collect backtraces
        \item \localname{backtrace\_active} can be set by
          \begin{itemize}
            \item The \funcarg{b} flag in the \funcname{OCAMLRUNPARAM} environment variable
            \item \mintinline{ocaml}{Printexc.record_backtrace : bool -> unit} \footnotemark
          \end{itemize}
      \end{itemize}
    \end{column}
    \begin{column}{0.5\textwidth}
      \begin{listing}
        \mintc[highlightlines={9-11}]{src/ocaml/domain\_state\_backtrace.h}
        \caption{runtime/caml/domain\_state.h}
      \end{listing}
    \end{column}
  \end{columns}
  \footnotetext[1]{https://v2.ocaml.org/api/Printexc.html}
\end{frame}

\newcommand\listCamlRaiseExn[1][]{
  \listasm[firstline=702,lastline=725,#1]{../ocaml/runtime/amd64.S}{runtime/amd64.S:702}}

\begin{frame}{Backtraces}{Walk through \funcname{caml\_raise\_exn}}
  \begin{columns}[T]
    \begin{column}{0.5\textwidth}
      \begin{itemize}
        \item<1-> First checks whether exceptions backtrace should be recorded
        \item<1-> By checking the \funcarg{domain\_state->backtrace\_active} value
        \item<2-> If not, acts as \funcname{raise\_notrace} with \funcname{RESTORE\_EXN\_HANDLER\_OCAML}
          \begin{itemize}
            \item Set \regname{RSP} to \localname{domain\_state->exn\_handler} value
            \item Restore \localname{domain\_state->exn\_handler} from trap
            \item Jump with \funcname{ret} (same effect as using \funcname{pop} and \funcname{jmp})
          \end{itemize}
        \item<3-> Otherwise, switches to the C stack to be able to execute C code
        \item<4-> Calls \funcname{caml\_stash\_backtrace}, a C function provided by the runtime\ignorespaces
          \onslide<6->{, with
            \begin{itemize}
              \item \funcarg{exn}: exception value
              \item \funcarg{pc}: code address of the raising point
              \item \funcarg{sp}: stack address of the raising function
              \item \funcarg{trapsp}: stack address of the next handler
            \end{itemize}
          }
      \end{itemize}
      \bigskip
      \mintocaml[firstline=9,lastline=13,highlightlines=12]{../src/backtrace/backtrace.ml}
    \end{column}
    \begin{column}{0.5\textwidth}
      \raggedleft
      \onslide*<1,3-4>{
        \mintc[firstline=190,lastline=192]{../ocaml/runtime/amd64.S}
        \mintc[firstline=234,lastline=237]{../ocaml/runtime/amd64.S}
      }
      \onslide*<2>{
        \mintc[firstline=190,lastline=192,highlightlines={191-192}]{../ocaml/runtime/amd64.S}
        \mintc[firstline=234,lastline=237,highlightlines={235-237}]{../ocaml/runtime/amd64.S}
      }
      \onslide*<1>{\listCamlRaiseExn[highlightlines=705]{}}%
      \onslide*<2>{\listCamlRaiseExn[highlightlines={707-708}]{}}%
      \onslide*<3>{\listCamlRaiseExn[highlightlines={712-714}]{}}%
      \onslide*<4>{\listCamlRaiseExn[highlightlines={715-720}]{}}%
      \onslide*<5>{\providecommand\step{1}\input{diagrams/backtrace/backtrace.tex}}%
      \onslide*<6>{\providecommand\step{2}\input{diagrams/backtrace/backtrace.tex}}%
    \end{column}
  \end{columns}
\end{frame}

\newcommand\listStashBacktrace[1][]{
  \listc[firstline=91,lastline=120,#1]{../ocaml/runtime/backtrace\_nat.c}{runtime/backtrace\_nat.c:91}}

\begin{frame}{Backtraces}{Introducing \funcname{caml\_stash\_backtrace}}
  \begin{columns}[c]
    \begin{column}{0.5\textwidth}
      \minipage[c][0.66\textheight][s]{\columnwidth}
      \begin{itemize}
        \item<1-> A C function provided by the runtime (specific to native code)
        \item<2-> It scans the stack, between the stack pointer at the raising point up to the next trap, in order to collect the names of the detected function frames
          \onslide*<2>{
            \begin{itemize}
              \item And more information, like line number, column number...
            \end{itemize}
          }
        \item<3-> It uses the same technique and information as the Garbage Collector (GC) uses to scan the values live on the stack
          \onslide*<4>{
            \begin{itemize}
              \item \typename{frame\_descr} are at the core of stack unwinding
            \end{itemize}
          }
      \end{itemize}
      \vfill
      \mintocaml[firstline=9,lastline=13,highlightlines=12]{../src/backtrace/backtrace.ml}
      \endminipage
    \end{column}
    \begin{column}{0.5\textwidth}
      \onslide*<1>{\listStashBacktrace[highlightlines=91]{}}
      \onslide*<2>{\listStashBacktrace[highlightlines={91,117-118}]{}}
      \onslide*<3->{\listStashBacktrace[highlightlines={94,106,109-110}]{}}
    \end{column}
  \end{columns}
\end{frame}

\newcommand\listFrameDescr[1][]{
  \listocaml[firstline=111,lastline=124,#1]{../ocaml/asmcomp/emitaux.ml}{asmcomp/emitaux.ml:111}}
\newcommand\listDebugInfo[1][]{
  \listocaml[firstline=38,lastline=49,#1]{../ocaml/lambda/debuginfo.mli}{lambda/debuginfo.mli:38}}

\begin{frame}{Backtraces}{\typename{frame\_descr} and \typename{frame\_debuginfo} in a nutshell}
  \begin{columns}[c]
    \begin{column}{0.5\textwidth}
      \begin{itemize}
        \item<1-> For every return address in an OCaml program, there is a associated \typename{frame\_descr}
        \item<2-> A \typename{frame\_descr} specifies
        \begin{itemize}
          \item<2-> The size of the function stack frame
          \item<3-> Where live values are on the stack (so that the GC can mark them as live and prevent from being swept)
        \end{itemize}
        \item<4-> When compiled with debug information, \typename{frame\_descr} will be linked to a \typename{frame\_debuginfo}
        \begin{itemize}
          \item<5-> Contains information about the position in the source code (file name, line and column number)
        \end{itemize}
      \end{itemize}
    \end{column}
    \begin{column}{0.5\textwidth}
        \onslide*<1>{\listFrameDescr[highlightlines={118-122}]{}}
        \onslide*<2>{\listFrameDescr[highlightlines={120}]{}}
        \onslide*<3>{\listFrameDescr[highlightlines={121}]{}}
        \onslide*<4->{\listFrameDescr[highlightlines={122}]{}}
        \medskip
        \onslide*<1-3>{\listDebugInfo{}}
        \onslide*<4>{\listDebugInfo[highlightlines={38,49}]{}}
        \onslide*<5>{\listDebugInfo[highlightlines={39-46}]{}}
    \end{column}
  \end{columns}
\end{frame}

\begin{frame}{Backtraces}{Scanning the stack with \typename{frame\_descr}}
  \begin{columns}[c]
    \begin{column}{0.5\textwidth}
      \begin{itemize}
        \item Given the return address of the code that raises the exception by calling \funcname{caml\_raise\_exn} (\funcarg{pc} argument of \funcname{caml\_stash\_backtrace})
        \item And the position of the stack pointer (\funcarg{sp} argument of \funcname{caml\_stash\_backtrace})
        \item The frame size given by \typename{frame\_descr}.\funcarg{frame\_size}, allows to find the end of the previous frame
        \item And the return address of the caller of this function
        \bigskip
        \onslide<3->{
        \item Note that traps (here the reddish \funcname{ExnA}, \funcname{ExnB} and \funcname{ExnC} blocks) belong to the frame that installed them, and so the frame size changes when traps are removed
        }
      \end{itemize}
    \end{column}
    \begin{column}{0.5\textwidth}
      \raggedleft
      \onslide*<1>{\providecommand\step{2}\input{diagrams/backtrace/backtrace.tex}}%
      \onslide*<2>{\providecommand\step{1}\input{diagrams/backtrace/scanstack.tex}}%
      \onslide*<3>{\providecommand\step{2}\input{diagrams/backtrace/scanstack.tex}}%
      \onslide*<4>{\providecommand\step{3}\input{diagrams/backtrace/scanstack.tex}}%
      \onslide*<5>{\providecommand\step{4}\input{diagrams/backtrace/scanstack.tex}}%
      \onslide*<6>{\providecommand\step{5}\input{diagrams/backtrace/scanstack.tex}}%
    \end{column}
  \end{columns}
\end{frame}

% Duplicate of previous-previous slide
\begin{frame}{Backtraces}{Continuing on \funcname{caml\_stash\_backtrace}}
  \begin{columns}[c]
    \begin{column}{0.5\textwidth}
      \minipage[c][0.75\textheight][s]{\columnwidth}
      \begin{itemize}
          \color{gray}
        \item A C function provided by the runtime (specific to native code)
        \item It scans the stack, between the stack pointer at the raising point up to the next trap, in order to collect the names of the detected function frames
        \item It uses the same technique and information as the Garbage Collector (GC) uses to scan the values live on the stack
      \end{itemize}
      \begin{itemize}
        \item For each function frame detected, store a pointer to the \typename{frame\_descr} in the \localname{domain\_state->backtrace\_buffer} array, effectively constructing the backtrace
      \end{itemize}
      \onslide<2->{
        \begin{itemize}
            \smallskip
          \item Constructed backtrace:
            \funcname{definitely\_raise},
            \onslide<3->{\funcname{probably\_raise}}
        \end{itemize}
      }
      \vfill
      \mintocaml[firstline=9,lastline=13,highlightlines=12]{../src/backtrace/backtrace.ml}
      \endminipage
    \end{column}
    \begin{column}{0.5\textwidth}
      \raggedleft
      \onslide*<1>{\listStashBacktrace[highlightlines={114-115}]{}}
      \onslide*<2>{\providecommand\step{3}\input{diagrams/backtrace/backtrace.tex}}%
      \onslide*<3>{\providecommand\step{4}\input{diagrams/backtrace/backtrace.tex}}%
    \end{column}
  \end{columns}
\end{frame}

\begin{frame}{Backtraces}{Back to \funcname{caml\_raise\_exn}}
  \begin{columns}[c]
    \begin{column}{0.5\textwidth}
      \minipage[c][0.66\textheight][s]{\columnwidth}
      \begin{itemize}\color{gray}
        \item Checks wheteher exceptions backtrace should be recorded, by checking the \funcarg{domain\_state->backtrace\_active} value
        \item Switches to the C stack to be able to execute C code
        \item Calls \funcname{caml\_stash\_backtrace}, to collect the backtrace up to the next trap
      \end{itemize}
      \onslide*<2->{
        \begin{itemize}
          \item<2-> Executes the exception handler in \funcname{probably\_raise} with \funcname{RESTORE\_EXN\_HANDLER\_OCAML}
            \begin{itemize}
              \item Set \regname{RSP} to \localname{domain\_state->exn\_handler} value
              \item Restore \localname{domain\_state->exn\_handler} from trap
              \item Jump to the handler code with \funcname{ret}
            \end{itemize}
        \end{itemize}
      }
      \vfill
      \onslide*<1>{\mintocaml[firstline=9,lastline=13,highlightlines=12]{../src/backtrace/backtrace.ml}}
      \onslide*<2->{\mintocaml[firstline=15,lastline=21,highlightlines={19-21}]{../src/backtrace/backtrace.ml}}
      \endminipage
    \end{column}
    \begin{column}{0.5\textwidth}
      \centering
      \onslide*<1>{
        \mintc[firstline=190,lastline=192]{../ocaml/runtime/amd64.S}
        \mintc[firstline=234,lastline=237]{../ocaml/runtime/amd64.S}
      }
      \onslide*<2>{
        \mintc[firstline=190,lastline=192,highlightlines={191-192}]{../ocaml/runtime/amd64.S}
        \mintc[firstline=234,lastline=237,highlightlines={235-237}]{../ocaml/runtime/amd64.S}
      }
      \onslide*<1>{\listCamlRaiseExn[highlightlines=720]{}}
      \onslide*<2>{\listCamlRaiseExn[highlightlines=722]{}}
      \onslide*<3->{\providecommand\step{5}\input{diagrams/backtrace/backtrace.tex}}
    \end{column}
  \end{columns}
\end{frame}

\begin{frame}{Backtraces}{\funcname{caml\_probably\_raise}'s handler doesn't catch \typename{ExnC}}
  \begin{columns}[c]
    \begin{column}{0.45\textwidth}
      \minipage[c][0.75\textheight][s]{\columnwidth}
      \begin{itemize}
        \item<1-> \funcname{match \dots with}\footnotemark pattern matching can also match exceptions
        \item<1-> The CMM of \funcname{probably\_raise} shows that this \funcname{match \dots with} is implemented with a pattern matching and a \funcname{try \dots with}
        \item<1-> But this one only matches on \typename{ExnA} exception
        \item<2-> Since the pattern matching fails to catch the exception, \funcname{reraise} is called with our current \typename{ExnC} exception
        \item<3-> Disassembly shows that \funcname{reraise} is actually a call to \funcname{caml\_reraise\_exn}, which is also an assembly function provided by the runtime
      \end{itemize}
      \vfill
      \mintocaml[firstline=15,lastline=21,highlightlines={18-21}]{../src/backtrace/backtrace.ml}
      \endminipage
    \end{column}
    \begin{column}{0.55\textwidth}
      \centering
      \onslide*<1>{\mintcmm[highlightlines={10-18}]{../src/backtrace/camlBacktrace\_\_probably\_raise\_351.cmm}}
      \onslide*<2>{\listcmm[highlightlines=18]{../src/backtrace/camlBacktrace\_\_probably\_raise_351.cmm}{CMM of probably\_raise}}
      \onslide*<3>{\mintobjdump[firstline=5,lastline=35,highlightlines=31]{../src/backtrace/camlBacktrace\_\_probably\_raise_351.objdump}}
    \end{column}
  \end{columns}
  \only<1>{\footnotetext[1]{https://blog.janestreet.com/pattern-matching-and-exception-handling-unite/}}
\end{frame}

\newcommand\listCamlReraiseExn[1][]{
  \listasm[firstline=727,lastline=734,#1]{../ocaml/runtime/amd64.S}{runtime/amd64.S:727}}

\begin{frame}{Backtraces}{Introducing \funcname{caml\_reraise\_exn}}
  \begin{columns}[c]
    \begin{column}{0.5\textwidth}
      \minipage[c][0.66\textheight][s]{\columnwidth}
      \begin{itemize}
        \item<1-> The exception have to be reraised to the next handler
        \item<2-> First, checks if the backtrace should be collected with \funcarg{domain\_state->backtrace\_active}
        \only<3>{
        \begin{itemize}
            \item If not, behaves like a \funcname{raise\_notrace} with \funcname{RESTORE\_EXN\_HANDLER\_OCAML}
        \end{itemize}
        }
        \item<4-> Jumps into \funcname{caml\_raise\_exn}
        \item<5-> Calls \funcname{caml\_stash\_backtrace} again, to scan the next part of the stack: from the trap just pop-ed to the next trap
      \end{itemize}
      \vfill
      \mintocaml[firstline=15,lastline=21,highlightlines={18-21}]{../src/backtrace/backtrace.ml}
      \endminipage
    \end{column}
    \begin{column}{0.5\textwidth}
      \raggedleft
      \onslide*<1>{\listCamlReraiseExn{}}
      \onslide*<2>{\listCamlReraiseExn[highlightlines=729]{}}
      \onslide*<3>{
        \mintc[firstline=190,lastline=192,highlightlines={191-192}]{../ocaml/runtime/amd64.S}
        \mintc[firstline=234,lastline=237,highlightlines={235-237}]{../ocaml/runtime/amd64.S}
        \medskip
        \listCamlReraiseExn[highlightlines=731]{}
      }
      \onslide*<4-5>{
        % TODO refactor with \listCamlReraiseExn
        \mintasm[firstline=727,lastline=734,highlightlines=730]{../ocaml/runtime/amd64.S}
        \medskip
      }
      \onslide*<4>{\listCamlRaiseExn[highlightlines=711]{}}
      \onslide*<5>{\listCamlRaiseExn[highlightlines=720]{}}
    \end{column}
  \end{columns}
\end{frame}

\begin{frame}{Backtraces}{Backtrace collection continues}
  \begin{columns}[c]
    \begin{column}{0.55\textwidth}
      \minipage[c][0.75\textheight][s]{\columnwidth}
      \begin{itemize}
        \item<1-> \funcname{caml\_stash\_backtrace} scans the older part of the stack, up to the next trap
        \item<6-> Controls jumps to the nested exception handler in \funcname{maybe\_raise}, which matches only on \typename{ExnB}
        \item<7-> \funcname{caml\_reraise\_exn} is called again, backtrace collection resumes with \funcname{caml\_stash\_backtrace}
      \end{itemize}
      \medskip
      \begin{itemize}
        \item<2-> Constructed backtrace:
        \begin{itemize}
          \item <2-> \funcname{definitely\_raise}
          \item <2-> \funcname{probably\_raise}
          \item <3-> \funcname{probably\_raise} (re-raise)
          \item <4-> \funcname{maybe\_raise}
          \item <5-> \funcname{main}
          \item <8-> \funcname{main} (re-raise)
        \end{itemize}
      \end{itemize}
      \vfill
      \onslide*<1-5>{\mintocaml[firstline=15,lastline=21]{../src/backtrace/backtrace.ml}}
      \onslide*<6>{\mintocaml[firstline=28,lastline=34,highlightlines=31]{../src/backtrace/backtrace.ml}}
      \onslide*<7->{\mintocaml[firstline=28,lastline=34,highlightlines=33]{../src/backtrace/backtrace.ml}}
      \endminipage
    \end{column}
    \begin{column}{0.45\textwidth}
      \raggedleft
      \onslide*<1>{\providecommand\step{6}\input{diagrams/backtrace/backtrace.tex}}%
      \onslide*<2>{\providecommand\step{6}\input{diagrams/backtrace/backtrace.tex}}%
      \onslide*<3>{\providecommand\step{7}\input{diagrams/backtrace/backtrace.tex}}%
      \onslide*<4>{\providecommand\step{8}\input{diagrams/backtrace/backtrace.tex}}%
      \onslide*<5>{\providecommand\step{9}\input{diagrams/backtrace/backtrace.tex}}%
      \onslide*<6>{\providecommand\step{10}\input{diagrams/backtrace/backtrace.tex}}%
      \onslide*<7>{\providecommand\step{11}\input{diagrams/backtrace/backtrace.tex}}%
      \onslide*<8>{\providecommand\step{12}\input{diagrams/backtrace/backtrace.tex}}%
      \onslide*<9>{\providecommand\step{13}\input{diagrams/backtrace/backtrace.tex}}%
    \end{column}
  \end{columns}
\end{frame}

\begin{frame}{Backtraces}{Printing the backtrace}
  \begin{columns}[c]
    \begin{column}{0.45\textwidth}
      \minipage[c][0.66\textheight][s]{\columnwidth}
      \begin{itemize}
        \item<1-> The exception backtrace can now be printed to \funcarg{stdout} with \funcname{Printexc.print_backtrace}
        \item<2-> First the exception backtrace is retrieved into an OCaml array with \funcname{get_raw_backtrace }
        \item<3-> Which is implemented as the \funcname{caml_get_exception_raw_backtrace} C function in the runtime
        \item<4-> And finally printed with \funcname{print_exception_backtrace}
      \end{itemize}
      \vfill
      \mintocaml[firstline=28,lastline=34,highlightlines=33]{../src/backtrace/backtrace.ml}
      \endminipage
    \end{column}
    \begin{column}{0.55\textwidth}
      \onslide*<1,2>{
        \mintocaml[firstline=100,lastline=101]{../ocaml/stdlib/printexc.ml}
      }
      \only<1>{
        \listocaml[firstline=170,lastline=172]{../ocaml/stdlib/printexc.ml}{stdlib/printexc.ml:170}
      }
      \only<2>{
        \listocaml[firstline=170,lastline=172,highlightlines=172]{../ocaml/stdlib/printexc.ml}{stdlib/printexc.ml:170}
      }
      \only<3>{
        \mintc[firstline=154,lastline=156]{../ocaml/runtime/backtrace.c}
        \listc[firstline=171,lastline=192,highlightlines={184-187}]{../ocaml/runtime/backtrace.c}{runtime/backtrace.c:154}
      }
      \only<4>{
        \listocaml[firstline=155,lastline=165]{../ocaml/stdlib/printexc.ml}{stdlib/printexc.ml:55}
      }
    \end{column}
  \end{columns}
\end{frame}


\subsection{The multiple kinds of raise}
\frameSubsection{}{}

\begin{frame}{Backtraces}{When backtraces are not needed}
  \begin{columns}[c]
    \begin{column}{0.45\textwidth}
      \minipage[c][0.66\textheight][s]{\columnwidth}
      \begin{itemize}
        \item<1-> What if the backtrace for an exception is never needed?
        \item<2-> It's possible to raise the exception explicitly with \funcname{raise\_notrace} to make raising as cheap as possible
        \item<3-> An example of use in the \funcname{dynlink} library of the OCaml compiler
      \end{itemize}
      \vfill
      \onslide<3->{
      \listocaml[firstline=205,lastline=209,highlightlines=207]{../ocaml/otherlibs/dynlink/dynlink\_compilerlibs/misc.ml}{otherlibs/dynlink/dynlink\_compilerlibs/misc.ml:205}
      }
      \endminipage
    \end{column}
    \begin{column}{0.55\textwidth}
    \only<4>{
      \mintcmm[highlightlines=3]{../src/notrace/camlNotrace\_\_fun_347.cmm}
      \listcmm{../src/notrace/camlNotrace\_\_all\_somes\_327.cmm}{all\_somes}
    }
    \onslide*<5->{
      \listobjdump[highlightlines={6-9}]{../src/notrace/camlNotrace\_\_fun_347.objdump}{anonymous function from \funcname{all\_somes}}
    }
    \end{column}
  \end{columns}
\end{frame}

\begin{frame}{Backtraces}{Summary table of the multiple kinds of raise}
  \begin{itemize}
    \item When debugging information is enabled and if the backtrace of an exception isn't needed, it's possible to raise with \funcname{raise\_notrace} for performance
    \item When debugging information is \emph{not} enabled, the compiler optimises \funcname{raise} calls to inline assembly (same effect as using \funcname{raise\_notrace})
  \end{itemize}
  \bigskip
  \centering
  \begin{tabular}{| l | c | c | c | c |}
    \hline
    \parbox{10em}{Debug information \\ (\funcarg{-g} flag)}  & \multicolumn{2}{|c|}{Disabled}                                     & \multicolumn{2}{|c|}{Enabled} \\
    \hline
    Raise kind                                               & \funcname{raise} & \funcname{raise\_notrace}                       & \funcname{raise} & \funcname{raise\_notrace} \\
    \hline
    Compilation                                              & \parbox{8em}{inline assembly\\ (optimization)} & inline assembly   & \funcname{caml\_raise\_exn} & inline assembly \\
    \hline
  \end{tabular}
\end{frame}


%
% Takeaway #5
%
\subsection*{Takeaway \#5}
\frameSubsectionTakeaway{}
\againframe<5>{takeaway}
}%
      \onslide*<5>{\providecommand\step{9}%
% Backtraces
%
\subsection{One advantage of raising with debugging information}
\frameSubsection{}{}

\begin{frame}{Backtraces}{Debugging information \& source code}
  \begin{columns}[c]
    \begin{column}{0.5\textwidth}
      \begin{itemize}
        \item Raising an exception is fun and games, but how do we know where the exception originated? $\rightarrow$ With the backtrace associated with the exception!
        \item In order to get a backtrace from the point where an exception was raised, it's necessary to compile with debugging information enabled (\funcarg{-g} flag for the compiler)
        \item Same example as in the `Nested handlers' section
      \end{itemize}
    \end{column}
    \begin{column}{0.5\textwidth}
      \listocaml[firstline=9,lastline=34]{../src/backtrace/backtrace.ml}{backtrace/backtrace.ml}
    \end{column}
  \end{columns}
\end{frame}

\begin{frame}{Backtraces}{CMM and disassembly of \funcname{definitely\_raise}}
  \begin{columns}[c]
    \begin{column}{0.5\textwidth}
      \minipage[c][0.66\textheight][s]{\columnwidth}
      \begin{itemize}
        \item<1-> An effect of compiling with debugging information (\funcarg{-g}): the CMM no longer uses \funcname{raise\_notrace} to raise the exception but \funcname{raise} instead
        \item<2-> Disassembly shows that CMM \funcname{raise} is actually a call to \funcname{caml\_raise\_exn}, a function in assembler provided by the runtime
      \end{itemize}
      \vfill
      \mintocaml[firstline=9,lastline=13,highlightlines=12]{../src/backtrace/backtrace.ml}
      \endminipage
    \end{column}
    \begin{column}{0.5\textwidth}
      \onslide*<1>{
        \mintcmm[highlightlines=5]{../src/backtrace/camlBacktrace\_\_definitely\_raise\_347.cmm}
      }
      \onslide*<2>{
        \mintobjdump[firstline=2,highlightlines=14]{../src/backtrace/camlBacktrace\_\_definitely\_raise\_347.objdump}
      }
    \end{column}
  \end{columns}
\end{frame}

\begin{frame}{Backtraces}{Introducing \funcname{caml\_raise\_exn}}
  \begin{columns}[c]
    \begin{column}{0.5\textwidth}
      \minipage[c][0.66\textheight][s]{\columnwidth}
      \onslide*<1>{
        \begin{itemize}
          \item Raising the exception is performed using \funcname{caml\_raise\_exn} which is
            \begin{itemize}
              \item An assembly function provided by the runtime
              \item Architecture specific
            \end{itemize}
          \item Let's see what it does differently from \funcname{raise\_notrace}\dots
          \item We are about to raise \typename{ExnC} with \funcname{caml\_raise\_exn} from \funcname{definitely\_raise}
        \end{itemize}
      }
      \onslide*<2>{
        \mintobjdump[firstline=2,highlightlines=14]{../src/backtrace/camlBacktrace\_\_definitely\_raise\_347.objdump}
      }
      \vfill
      \mintocaml[firstline=9,lastline=13,highlightlines=12]{../src/backtrace/backtrace.ml}
      \endminipage
    \end{column}
    \begin{column}{0.5\textwidth}
      \centering
      \providecommand\step{1}\input{diagrams/backtrace/backtrace.tex}
    \end{column}
  \end{columns}
\end{frame}

\begin{frame}{Backtraces}{But before that, we need more fields from \typename{caml\_domain\_state}}
  \begin{columns}
    \begin{column}{0.5\textwidth}
      \begin{itemize}
        \item Remember \typename{caml\_domain\_state} the per-domain runtime state?
        \item Some other members are of interest to us now\dots
        \item \localname{backtrace\_active} is used to know if the runtime should collect backtraces
        \item \localname{backtrace\_active} can be set by
          \begin{itemize}
            \item The \funcarg{b} flag in the \funcname{OCAMLRUNPARAM} environment variable
            \item \mintinline{ocaml}{Printexc.record_backtrace : bool -> unit} \footnotemark
          \end{itemize}
      \end{itemize}
    \end{column}
    \begin{column}{0.5\textwidth}
      \begin{listing}
        \mintc[highlightlines={9-11}]{src/ocaml/domain\_state\_backtrace.h}
        \caption{runtime/caml/domain\_state.h}
      \end{listing}
    \end{column}
  \end{columns}
  \footnotetext[1]{https://v2.ocaml.org/api/Printexc.html}
\end{frame}

\newcommand\listCamlRaiseExn[1][]{
  \listasm[firstline=702,lastline=725,#1]{../ocaml/runtime/amd64.S}{runtime/amd64.S:702}}

\begin{frame}{Backtraces}{Walk through \funcname{caml\_raise\_exn}}
  \begin{columns}[T]
    \begin{column}{0.5\textwidth}
      \begin{itemize}
        \item<1-> First checks whether exceptions backtrace should be recorded
        \item<1-> By checking the \funcarg{domain\_state->backtrace\_active} value
        \item<2-> If not, acts as \funcname{raise\_notrace} with \funcname{RESTORE\_EXN\_HANDLER\_OCAML}
          \begin{itemize}
            \item Set \regname{RSP} to \localname{domain\_state->exn\_handler} value
            \item Restore \localname{domain\_state->exn\_handler} from trap
            \item Jump with \funcname{ret} (same effect as using \funcname{pop} and \funcname{jmp})
          \end{itemize}
        \item<3-> Otherwise, switches to the C stack to be able to execute C code
        \item<4-> Calls \funcname{caml\_stash\_backtrace}, a C function provided by the runtime\ignorespaces
          \onslide<6->{, with
            \begin{itemize}
              \item \funcarg{exn}: exception value
              \item \funcarg{pc}: code address of the raising point
              \item \funcarg{sp}: stack address of the raising function
              \item \funcarg{trapsp}: stack address of the next handler
            \end{itemize}
          }
      \end{itemize}
      \bigskip
      \mintocaml[firstline=9,lastline=13,highlightlines=12]{../src/backtrace/backtrace.ml}
    \end{column}
    \begin{column}{0.5\textwidth}
      \raggedleft
      \onslide*<1,3-4>{
        \mintc[firstline=190,lastline=192]{../ocaml/runtime/amd64.S}
        \mintc[firstline=234,lastline=237]{../ocaml/runtime/amd64.S}
      }
      \onslide*<2>{
        \mintc[firstline=190,lastline=192,highlightlines={191-192}]{../ocaml/runtime/amd64.S}
        \mintc[firstline=234,lastline=237,highlightlines={235-237}]{../ocaml/runtime/amd64.S}
      }
      \onslide*<1>{\listCamlRaiseExn[highlightlines=705]{}}%
      \onslide*<2>{\listCamlRaiseExn[highlightlines={707-708}]{}}%
      \onslide*<3>{\listCamlRaiseExn[highlightlines={712-714}]{}}%
      \onslide*<4>{\listCamlRaiseExn[highlightlines={715-720}]{}}%
      \onslide*<5>{\providecommand\step{1}\input{diagrams/backtrace/backtrace.tex}}%
      \onslide*<6>{\providecommand\step{2}\input{diagrams/backtrace/backtrace.tex}}%
    \end{column}
  \end{columns}
\end{frame}

\newcommand\listStashBacktrace[1][]{
  \listc[firstline=91,lastline=120,#1]{../ocaml/runtime/backtrace\_nat.c}{runtime/backtrace\_nat.c:91}}

\begin{frame}{Backtraces}{Introducing \funcname{caml\_stash\_backtrace}}
  \begin{columns}[c]
    \begin{column}{0.5\textwidth}
      \minipage[c][0.66\textheight][s]{\columnwidth}
      \begin{itemize}
        \item<1-> A C function provided by the runtime (specific to native code)
        \item<2-> It scans the stack, between the stack pointer at the raising point up to the next trap, in order to collect the names of the detected function frames
          \onslide*<2>{
            \begin{itemize}
              \item And more information, like line number, column number...
            \end{itemize}
          }
        \item<3-> It uses the same technique and information as the Garbage Collector (GC) uses to scan the values live on the stack
          \onslide*<4>{
            \begin{itemize}
              \item \typename{frame\_descr} are at the core of stack unwinding
            \end{itemize}
          }
      \end{itemize}
      \vfill
      \mintocaml[firstline=9,lastline=13,highlightlines=12]{../src/backtrace/backtrace.ml}
      \endminipage
    \end{column}
    \begin{column}{0.5\textwidth}
      \onslide*<1>{\listStashBacktrace[highlightlines=91]{}}
      \onslide*<2>{\listStashBacktrace[highlightlines={91,117-118}]{}}
      \onslide*<3->{\listStashBacktrace[highlightlines={94,106,109-110}]{}}
    \end{column}
  \end{columns}
\end{frame}

\newcommand\listFrameDescr[1][]{
  \listocaml[firstline=111,lastline=124,#1]{../ocaml/asmcomp/emitaux.ml}{asmcomp/emitaux.ml:111}}
\newcommand\listDebugInfo[1][]{
  \listocaml[firstline=38,lastline=49,#1]{../ocaml/lambda/debuginfo.mli}{lambda/debuginfo.mli:38}}

\begin{frame}{Backtraces}{\typename{frame\_descr} and \typename{frame\_debuginfo} in a nutshell}
  \begin{columns}[c]
    \begin{column}{0.5\textwidth}
      \begin{itemize}
        \item<1-> For every return address in an OCaml program, there is a associated \typename{frame\_descr}
        \item<2-> A \typename{frame\_descr} specifies
        \begin{itemize}
          \item<2-> The size of the function stack frame
          \item<3-> Where live values are on the stack (so that the GC can mark them as live and prevent from being swept)
        \end{itemize}
        \item<4-> When compiled with debug information, \typename{frame\_descr} will be linked to a \typename{frame\_debuginfo}
        \begin{itemize}
          \item<5-> Contains information about the position in the source code (file name, line and column number)
        \end{itemize}
      \end{itemize}
    \end{column}
    \begin{column}{0.5\textwidth}
        \onslide*<1>{\listFrameDescr[highlightlines={118-122}]{}}
        \onslide*<2>{\listFrameDescr[highlightlines={120}]{}}
        \onslide*<3>{\listFrameDescr[highlightlines={121}]{}}
        \onslide*<4->{\listFrameDescr[highlightlines={122}]{}}
        \medskip
        \onslide*<1-3>{\listDebugInfo{}}
        \onslide*<4>{\listDebugInfo[highlightlines={38,49}]{}}
        \onslide*<5>{\listDebugInfo[highlightlines={39-46}]{}}
    \end{column}
  \end{columns}
\end{frame}

\begin{frame}{Backtraces}{Scanning the stack with \typename{frame\_descr}}
  \begin{columns}[c]
    \begin{column}{0.5\textwidth}
      \begin{itemize}
        \item Given the return address of the code that raises the exception by calling \funcname{caml\_raise\_exn} (\funcarg{pc} argument of \funcname{caml\_stash\_backtrace})
        \item And the position of the stack pointer (\funcarg{sp} argument of \funcname{caml\_stash\_backtrace})
        \item The frame size given by \typename{frame\_descr}.\funcarg{frame\_size}, allows to find the end of the previous frame
        \item And the return address of the caller of this function
        \bigskip
        \onslide<3->{
        \item Note that traps (here the reddish \funcname{ExnA}, \funcname{ExnB} and \funcname{ExnC} blocks) belong to the frame that installed them, and so the frame size changes when traps are removed
        }
      \end{itemize}
    \end{column}
    \begin{column}{0.5\textwidth}
      \raggedleft
      \onslide*<1>{\providecommand\step{2}\input{diagrams/backtrace/backtrace.tex}}%
      \onslide*<2>{\providecommand\step{1}\input{diagrams/backtrace/scanstack.tex}}%
      \onslide*<3>{\providecommand\step{2}\input{diagrams/backtrace/scanstack.tex}}%
      \onslide*<4>{\providecommand\step{3}\input{diagrams/backtrace/scanstack.tex}}%
      \onslide*<5>{\providecommand\step{4}\input{diagrams/backtrace/scanstack.tex}}%
      \onslide*<6>{\providecommand\step{5}\input{diagrams/backtrace/scanstack.tex}}%
    \end{column}
  \end{columns}
\end{frame}

% Duplicate of previous-previous slide
\begin{frame}{Backtraces}{Continuing on \funcname{caml\_stash\_backtrace}}
  \begin{columns}[c]
    \begin{column}{0.5\textwidth}
      \minipage[c][0.75\textheight][s]{\columnwidth}
      \begin{itemize}
          \color{gray}
        \item A C function provided by the runtime (specific to native code)
        \item It scans the stack, between the stack pointer at the raising point up to the next trap, in order to collect the names of the detected function frames
        \item It uses the same technique and information as the Garbage Collector (GC) uses to scan the values live on the stack
      \end{itemize}
      \begin{itemize}
        \item For each function frame detected, store a pointer to the \typename{frame\_descr} in the \localname{domain\_state->backtrace\_buffer} array, effectively constructing the backtrace
      \end{itemize}
      \onslide<2->{
        \begin{itemize}
            \smallskip
          \item Constructed backtrace:
            \funcname{definitely\_raise},
            \onslide<3->{\funcname{probably\_raise}}
        \end{itemize}
      }
      \vfill
      \mintocaml[firstline=9,lastline=13,highlightlines=12]{../src/backtrace/backtrace.ml}
      \endminipage
    \end{column}
    \begin{column}{0.5\textwidth}
      \raggedleft
      \onslide*<1>{\listStashBacktrace[highlightlines={114-115}]{}}
      \onslide*<2>{\providecommand\step{3}\input{diagrams/backtrace/backtrace.tex}}%
      \onslide*<3>{\providecommand\step{4}\input{diagrams/backtrace/backtrace.tex}}%
    \end{column}
  \end{columns}
\end{frame}

\begin{frame}{Backtraces}{Back to \funcname{caml\_raise\_exn}}
  \begin{columns}[c]
    \begin{column}{0.5\textwidth}
      \minipage[c][0.66\textheight][s]{\columnwidth}
      \begin{itemize}\color{gray}
        \item Checks wheteher exceptions backtrace should be recorded, by checking the \funcarg{domain\_state->backtrace\_active} value
        \item Switches to the C stack to be able to execute C code
        \item Calls \funcname{caml\_stash\_backtrace}, to collect the backtrace up to the next trap
      \end{itemize}
      \onslide*<2->{
        \begin{itemize}
          \item<2-> Executes the exception handler in \funcname{probably\_raise} with \funcname{RESTORE\_EXN\_HANDLER\_OCAML}
            \begin{itemize}
              \item Set \regname{RSP} to \localname{domain\_state->exn\_handler} value
              \item Restore \localname{domain\_state->exn\_handler} from trap
              \item Jump to the handler code with \funcname{ret}
            \end{itemize}
        \end{itemize}
      }
      \vfill
      \onslide*<1>{\mintocaml[firstline=9,lastline=13,highlightlines=12]{../src/backtrace/backtrace.ml}}
      \onslide*<2->{\mintocaml[firstline=15,lastline=21,highlightlines={19-21}]{../src/backtrace/backtrace.ml}}
      \endminipage
    \end{column}
    \begin{column}{0.5\textwidth}
      \centering
      \onslide*<1>{
        \mintc[firstline=190,lastline=192]{../ocaml/runtime/amd64.S}
        \mintc[firstline=234,lastline=237]{../ocaml/runtime/amd64.S}
      }
      \onslide*<2>{
        \mintc[firstline=190,lastline=192,highlightlines={191-192}]{../ocaml/runtime/amd64.S}
        \mintc[firstline=234,lastline=237,highlightlines={235-237}]{../ocaml/runtime/amd64.S}
      }
      \onslide*<1>{\listCamlRaiseExn[highlightlines=720]{}}
      \onslide*<2>{\listCamlRaiseExn[highlightlines=722]{}}
      \onslide*<3->{\providecommand\step{5}\input{diagrams/backtrace/backtrace.tex}}
    \end{column}
  \end{columns}
\end{frame}

\begin{frame}{Backtraces}{\funcname{caml\_probably\_raise}'s handler doesn't catch \typename{ExnC}}
  \begin{columns}[c]
    \begin{column}{0.45\textwidth}
      \minipage[c][0.75\textheight][s]{\columnwidth}
      \begin{itemize}
        \item<1-> \funcname{match \dots with}\footnotemark pattern matching can also match exceptions
        \item<1-> The CMM of \funcname{probably\_raise} shows that this \funcname{match \dots with} is implemented with a pattern matching and a \funcname{try \dots with}
        \item<1-> But this one only matches on \typename{ExnA} exception
        \item<2-> Since the pattern matching fails to catch the exception, \funcname{reraise} is called with our current \typename{ExnC} exception
        \item<3-> Disassembly shows that \funcname{reraise} is actually a call to \funcname{caml\_reraise\_exn}, which is also an assembly function provided by the runtime
      \end{itemize}
      \vfill
      \mintocaml[firstline=15,lastline=21,highlightlines={18-21}]{../src/backtrace/backtrace.ml}
      \endminipage
    \end{column}
    \begin{column}{0.55\textwidth}
      \centering
      \onslide*<1>{\mintcmm[highlightlines={10-18}]{../src/backtrace/camlBacktrace\_\_probably\_raise\_351.cmm}}
      \onslide*<2>{\listcmm[highlightlines=18]{../src/backtrace/camlBacktrace\_\_probably\_raise_351.cmm}{CMM of probably\_raise}}
      \onslide*<3>{\mintobjdump[firstline=5,lastline=35,highlightlines=31]{../src/backtrace/camlBacktrace\_\_probably\_raise_351.objdump}}
    \end{column}
  \end{columns}
  \only<1>{\footnotetext[1]{https://blog.janestreet.com/pattern-matching-and-exception-handling-unite/}}
\end{frame}

\newcommand\listCamlReraiseExn[1][]{
  \listasm[firstline=727,lastline=734,#1]{../ocaml/runtime/amd64.S}{runtime/amd64.S:727}}

\begin{frame}{Backtraces}{Introducing \funcname{caml\_reraise\_exn}}
  \begin{columns}[c]
    \begin{column}{0.5\textwidth}
      \minipage[c][0.66\textheight][s]{\columnwidth}
      \begin{itemize}
        \item<1-> The exception have to be reraised to the next handler
        \item<2-> First, checks if the backtrace should be collected with \funcarg{domain\_state->backtrace\_active}
        \only<3>{
        \begin{itemize}
            \item If not, behaves like a \funcname{raise\_notrace} with \funcname{RESTORE\_EXN\_HANDLER\_OCAML}
        \end{itemize}
        }
        \item<4-> Jumps into \funcname{caml\_raise\_exn}
        \item<5-> Calls \funcname{caml\_stash\_backtrace} again, to scan the next part of the stack: from the trap just pop-ed to the next trap
      \end{itemize}
      \vfill
      \mintocaml[firstline=15,lastline=21,highlightlines={18-21}]{../src/backtrace/backtrace.ml}
      \endminipage
    \end{column}
    \begin{column}{0.5\textwidth}
      \raggedleft
      \onslide*<1>{\listCamlReraiseExn{}}
      \onslide*<2>{\listCamlReraiseExn[highlightlines=729]{}}
      \onslide*<3>{
        \mintc[firstline=190,lastline=192,highlightlines={191-192}]{../ocaml/runtime/amd64.S}
        \mintc[firstline=234,lastline=237,highlightlines={235-237}]{../ocaml/runtime/amd64.S}
        \medskip
        \listCamlReraiseExn[highlightlines=731]{}
      }
      \onslide*<4-5>{
        % TODO refactor with \listCamlReraiseExn
        \mintasm[firstline=727,lastline=734,highlightlines=730]{../ocaml/runtime/amd64.S}
        \medskip
      }
      \onslide*<4>{\listCamlRaiseExn[highlightlines=711]{}}
      \onslide*<5>{\listCamlRaiseExn[highlightlines=720]{}}
    \end{column}
  \end{columns}
\end{frame}

\begin{frame}{Backtraces}{Backtrace collection continues}
  \begin{columns}[c]
    \begin{column}{0.55\textwidth}
      \minipage[c][0.75\textheight][s]{\columnwidth}
      \begin{itemize}
        \item<1-> \funcname{caml\_stash\_backtrace} scans the older part of the stack, up to the next trap
        \item<6-> Controls jumps to the nested exception handler in \funcname{maybe\_raise}, which matches only on \typename{ExnB}
        \item<7-> \funcname{caml\_reraise\_exn} is called again, backtrace collection resumes with \funcname{caml\_stash\_backtrace}
      \end{itemize}
      \medskip
      \begin{itemize}
        \item<2-> Constructed backtrace:
        \begin{itemize}
          \item <2-> \funcname{definitely\_raise}
          \item <2-> \funcname{probably\_raise}
          \item <3-> \funcname{probably\_raise} (re-raise)
          \item <4-> \funcname{maybe\_raise}
          \item <5-> \funcname{main}
          \item <8-> \funcname{main} (re-raise)
        \end{itemize}
      \end{itemize}
      \vfill
      \onslide*<1-5>{\mintocaml[firstline=15,lastline=21]{../src/backtrace/backtrace.ml}}
      \onslide*<6>{\mintocaml[firstline=28,lastline=34,highlightlines=31]{../src/backtrace/backtrace.ml}}
      \onslide*<7->{\mintocaml[firstline=28,lastline=34,highlightlines=33]{../src/backtrace/backtrace.ml}}
      \endminipage
    \end{column}
    \begin{column}{0.45\textwidth}
      \raggedleft
      \onslide*<1>{\providecommand\step{6}\input{diagrams/backtrace/backtrace.tex}}%
      \onslide*<2>{\providecommand\step{6}\input{diagrams/backtrace/backtrace.tex}}%
      \onslide*<3>{\providecommand\step{7}\input{diagrams/backtrace/backtrace.tex}}%
      \onslide*<4>{\providecommand\step{8}\input{diagrams/backtrace/backtrace.tex}}%
      \onslide*<5>{\providecommand\step{9}\input{diagrams/backtrace/backtrace.tex}}%
      \onslide*<6>{\providecommand\step{10}\input{diagrams/backtrace/backtrace.tex}}%
      \onslide*<7>{\providecommand\step{11}\input{diagrams/backtrace/backtrace.tex}}%
      \onslide*<8>{\providecommand\step{12}\input{diagrams/backtrace/backtrace.tex}}%
      \onslide*<9>{\providecommand\step{13}\input{diagrams/backtrace/backtrace.tex}}%
    \end{column}
  \end{columns}
\end{frame}

\begin{frame}{Backtraces}{Printing the backtrace}
  \begin{columns}[c]
    \begin{column}{0.45\textwidth}
      \minipage[c][0.66\textheight][s]{\columnwidth}
      \begin{itemize}
        \item<1-> The exception backtrace can now be printed to \funcarg{stdout} with \funcname{Printexc.print_backtrace}
        \item<2-> First the exception backtrace is retrieved into an OCaml array with \funcname{get_raw_backtrace }
        \item<3-> Which is implemented as the \funcname{caml_get_exception_raw_backtrace} C function in the runtime
        \item<4-> And finally printed with \funcname{print_exception_backtrace}
      \end{itemize}
      \vfill
      \mintocaml[firstline=28,lastline=34,highlightlines=33]{../src/backtrace/backtrace.ml}
      \endminipage
    \end{column}
    \begin{column}{0.55\textwidth}
      \onslide*<1,2>{
        \mintocaml[firstline=100,lastline=101]{../ocaml/stdlib/printexc.ml}
      }
      \only<1>{
        \listocaml[firstline=170,lastline=172]{../ocaml/stdlib/printexc.ml}{stdlib/printexc.ml:170}
      }
      \only<2>{
        \listocaml[firstline=170,lastline=172,highlightlines=172]{../ocaml/stdlib/printexc.ml}{stdlib/printexc.ml:170}
      }
      \only<3>{
        \mintc[firstline=154,lastline=156]{../ocaml/runtime/backtrace.c}
        \listc[firstline=171,lastline=192,highlightlines={184-187}]{../ocaml/runtime/backtrace.c}{runtime/backtrace.c:154}
      }
      \only<4>{
        \listocaml[firstline=155,lastline=165]{../ocaml/stdlib/printexc.ml}{stdlib/printexc.ml:55}
      }
    \end{column}
  \end{columns}
\end{frame}


\subsection{The multiple kinds of raise}
\frameSubsection{}{}

\begin{frame}{Backtraces}{When backtraces are not needed}
  \begin{columns}[c]
    \begin{column}{0.45\textwidth}
      \minipage[c][0.66\textheight][s]{\columnwidth}
      \begin{itemize}
        \item<1-> What if the backtrace for an exception is never needed?
        \item<2-> It's possible to raise the exception explicitly with \funcname{raise\_notrace} to make raising as cheap as possible
        \item<3-> An example of use in the \funcname{dynlink} library of the OCaml compiler
      \end{itemize}
      \vfill
      \onslide<3->{
      \listocaml[firstline=205,lastline=209,highlightlines=207]{../ocaml/otherlibs/dynlink/dynlink\_compilerlibs/misc.ml}{otherlibs/dynlink/dynlink\_compilerlibs/misc.ml:205}
      }
      \endminipage
    \end{column}
    \begin{column}{0.55\textwidth}
    \only<4>{
      \mintcmm[highlightlines=3]{../src/notrace/camlNotrace\_\_fun_347.cmm}
      \listcmm{../src/notrace/camlNotrace\_\_all\_somes\_327.cmm}{all\_somes}
    }
    \onslide*<5->{
      \listobjdump[highlightlines={6-9}]{../src/notrace/camlNotrace\_\_fun_347.objdump}{anonymous function from \funcname{all\_somes}}
    }
    \end{column}
  \end{columns}
\end{frame}

\begin{frame}{Backtraces}{Summary table of the multiple kinds of raise}
  \begin{itemize}
    \item When debugging information is enabled and if the backtrace of an exception isn't needed, it's possible to raise with \funcname{raise\_notrace} for performance
    \item When debugging information is \emph{not} enabled, the compiler optimises \funcname{raise} calls to inline assembly (same effect as using \funcname{raise\_notrace})
  \end{itemize}
  \bigskip
  \centering
  \begin{tabular}{| l | c | c | c | c |}
    \hline
    \parbox{10em}{Debug information \\ (\funcarg{-g} flag)}  & \multicolumn{2}{|c|}{Disabled}                                     & \multicolumn{2}{|c|}{Enabled} \\
    \hline
    Raise kind                                               & \funcname{raise} & \funcname{raise\_notrace}                       & \funcname{raise} & \funcname{raise\_notrace} \\
    \hline
    Compilation                                              & \parbox{8em}{inline assembly\\ (optimization)} & inline assembly   & \funcname{caml\_raise\_exn} & inline assembly \\
    \hline
  \end{tabular}
\end{frame}


%
% Takeaway #5
%
\subsection*{Takeaway \#5}
\frameSubsectionTakeaway{}
\againframe<5>{takeaway}
}%
      \onslide*<6>{\providecommand\step{10}%
% Backtraces
%
\subsection{One advantage of raising with debugging information}
\frameSubsection{}{}

\begin{frame}{Backtraces}{Debugging information \& source code}
  \begin{columns}[c]
    \begin{column}{0.5\textwidth}
      \begin{itemize}
        \item Raising an exception is fun and games, but how do we know where the exception originated? $\rightarrow$ With the backtrace associated with the exception!
        \item In order to get a backtrace from the point where an exception was raised, it's necessary to compile with debugging information enabled (\funcarg{-g} flag for the compiler)
        \item Same example as in the `Nested handlers' section
      \end{itemize}
    \end{column}
    \begin{column}{0.5\textwidth}
      \listocaml[firstline=9,lastline=34]{../src/backtrace/backtrace.ml}{backtrace/backtrace.ml}
    \end{column}
  \end{columns}
\end{frame}

\begin{frame}{Backtraces}{CMM and disassembly of \funcname{definitely\_raise}}
  \begin{columns}[c]
    \begin{column}{0.5\textwidth}
      \minipage[c][0.66\textheight][s]{\columnwidth}
      \begin{itemize}
        \item<1-> An effect of compiling with debugging information (\funcarg{-g}): the CMM no longer uses \funcname{raise\_notrace} to raise the exception but \funcname{raise} instead
        \item<2-> Disassembly shows that CMM \funcname{raise} is actually a call to \funcname{caml\_raise\_exn}, a function in assembler provided by the runtime
      \end{itemize}
      \vfill
      \mintocaml[firstline=9,lastline=13,highlightlines=12]{../src/backtrace/backtrace.ml}
      \endminipage
    \end{column}
    \begin{column}{0.5\textwidth}
      \onslide*<1>{
        \mintcmm[highlightlines=5]{../src/backtrace/camlBacktrace\_\_definitely\_raise\_347.cmm}
      }
      \onslide*<2>{
        \mintobjdump[firstline=2,highlightlines=14]{../src/backtrace/camlBacktrace\_\_definitely\_raise\_347.objdump}
      }
    \end{column}
  \end{columns}
\end{frame}

\begin{frame}{Backtraces}{Introducing \funcname{caml\_raise\_exn}}
  \begin{columns}[c]
    \begin{column}{0.5\textwidth}
      \minipage[c][0.66\textheight][s]{\columnwidth}
      \onslide*<1>{
        \begin{itemize}
          \item Raising the exception is performed using \funcname{caml\_raise\_exn} which is
            \begin{itemize}
              \item An assembly function provided by the runtime
              \item Architecture specific
            \end{itemize}
          \item Let's see what it does differently from \funcname{raise\_notrace}\dots
          \item We are about to raise \typename{ExnC} with \funcname{caml\_raise\_exn} from \funcname{definitely\_raise}
        \end{itemize}
      }
      \onslide*<2>{
        \mintobjdump[firstline=2,highlightlines=14]{../src/backtrace/camlBacktrace\_\_definitely\_raise\_347.objdump}
      }
      \vfill
      \mintocaml[firstline=9,lastline=13,highlightlines=12]{../src/backtrace/backtrace.ml}
      \endminipage
    \end{column}
    \begin{column}{0.5\textwidth}
      \centering
      \providecommand\step{1}\input{diagrams/backtrace/backtrace.tex}
    \end{column}
  \end{columns}
\end{frame}

\begin{frame}{Backtraces}{But before that, we need more fields from \typename{caml\_domain\_state}}
  \begin{columns}
    \begin{column}{0.5\textwidth}
      \begin{itemize}
        \item Remember \typename{caml\_domain\_state} the per-domain runtime state?
        \item Some other members are of interest to us now\dots
        \item \localname{backtrace\_active} is used to know if the runtime should collect backtraces
        \item \localname{backtrace\_active} can be set by
          \begin{itemize}
            \item The \funcarg{b} flag in the \funcname{OCAMLRUNPARAM} environment variable
            \item \mintinline{ocaml}{Printexc.record_backtrace : bool -> unit} \footnotemark
          \end{itemize}
      \end{itemize}
    \end{column}
    \begin{column}{0.5\textwidth}
      \begin{listing}
        \mintc[highlightlines={9-11}]{src/ocaml/domain\_state\_backtrace.h}
        \caption{runtime/caml/domain\_state.h}
      \end{listing}
    \end{column}
  \end{columns}
  \footnotetext[1]{https://v2.ocaml.org/api/Printexc.html}
\end{frame}

\newcommand\listCamlRaiseExn[1][]{
  \listasm[firstline=702,lastline=725,#1]{../ocaml/runtime/amd64.S}{runtime/amd64.S:702}}

\begin{frame}{Backtraces}{Walk through \funcname{caml\_raise\_exn}}
  \begin{columns}[T]
    \begin{column}{0.5\textwidth}
      \begin{itemize}
        \item<1-> First checks whether exceptions backtrace should be recorded
        \item<1-> By checking the \funcarg{domain\_state->backtrace\_active} value
        \item<2-> If not, acts as \funcname{raise\_notrace} with \funcname{RESTORE\_EXN\_HANDLER\_OCAML}
          \begin{itemize}
            \item Set \regname{RSP} to \localname{domain\_state->exn\_handler} value
            \item Restore \localname{domain\_state->exn\_handler} from trap
            \item Jump with \funcname{ret} (same effect as using \funcname{pop} and \funcname{jmp})
          \end{itemize}
        \item<3-> Otherwise, switches to the C stack to be able to execute C code
        \item<4-> Calls \funcname{caml\_stash\_backtrace}, a C function provided by the runtime\ignorespaces
          \onslide<6->{, with
            \begin{itemize}
              \item \funcarg{exn}: exception value
              \item \funcarg{pc}: code address of the raising point
              \item \funcarg{sp}: stack address of the raising function
              \item \funcarg{trapsp}: stack address of the next handler
            \end{itemize}
          }
      \end{itemize}
      \bigskip
      \mintocaml[firstline=9,lastline=13,highlightlines=12]{../src/backtrace/backtrace.ml}
    \end{column}
    \begin{column}{0.5\textwidth}
      \raggedleft
      \onslide*<1,3-4>{
        \mintc[firstline=190,lastline=192]{../ocaml/runtime/amd64.S}
        \mintc[firstline=234,lastline=237]{../ocaml/runtime/amd64.S}
      }
      \onslide*<2>{
        \mintc[firstline=190,lastline=192,highlightlines={191-192}]{../ocaml/runtime/amd64.S}
        \mintc[firstline=234,lastline=237,highlightlines={235-237}]{../ocaml/runtime/amd64.S}
      }
      \onslide*<1>{\listCamlRaiseExn[highlightlines=705]{}}%
      \onslide*<2>{\listCamlRaiseExn[highlightlines={707-708}]{}}%
      \onslide*<3>{\listCamlRaiseExn[highlightlines={712-714}]{}}%
      \onslide*<4>{\listCamlRaiseExn[highlightlines={715-720}]{}}%
      \onslide*<5>{\providecommand\step{1}\input{diagrams/backtrace/backtrace.tex}}%
      \onslide*<6>{\providecommand\step{2}\input{diagrams/backtrace/backtrace.tex}}%
    \end{column}
  \end{columns}
\end{frame}

\newcommand\listStashBacktrace[1][]{
  \listc[firstline=91,lastline=120,#1]{../ocaml/runtime/backtrace\_nat.c}{runtime/backtrace\_nat.c:91}}

\begin{frame}{Backtraces}{Introducing \funcname{caml\_stash\_backtrace}}
  \begin{columns}[c]
    \begin{column}{0.5\textwidth}
      \minipage[c][0.66\textheight][s]{\columnwidth}
      \begin{itemize}
        \item<1-> A C function provided by the runtime (specific to native code)
        \item<2-> It scans the stack, between the stack pointer at the raising point up to the next trap, in order to collect the names of the detected function frames
          \onslide*<2>{
            \begin{itemize}
              \item And more information, like line number, column number...
            \end{itemize}
          }
        \item<3-> It uses the same technique and information as the Garbage Collector (GC) uses to scan the values live on the stack
          \onslide*<4>{
            \begin{itemize}
              \item \typename{frame\_descr} are at the core of stack unwinding
            \end{itemize}
          }
      \end{itemize}
      \vfill
      \mintocaml[firstline=9,lastline=13,highlightlines=12]{../src/backtrace/backtrace.ml}
      \endminipage
    \end{column}
    \begin{column}{0.5\textwidth}
      \onslide*<1>{\listStashBacktrace[highlightlines=91]{}}
      \onslide*<2>{\listStashBacktrace[highlightlines={91,117-118}]{}}
      \onslide*<3->{\listStashBacktrace[highlightlines={94,106,109-110}]{}}
    \end{column}
  \end{columns}
\end{frame}

\newcommand\listFrameDescr[1][]{
  \listocaml[firstline=111,lastline=124,#1]{../ocaml/asmcomp/emitaux.ml}{asmcomp/emitaux.ml:111}}
\newcommand\listDebugInfo[1][]{
  \listocaml[firstline=38,lastline=49,#1]{../ocaml/lambda/debuginfo.mli}{lambda/debuginfo.mli:38}}

\begin{frame}{Backtraces}{\typename{frame\_descr} and \typename{frame\_debuginfo} in a nutshell}
  \begin{columns}[c]
    \begin{column}{0.5\textwidth}
      \begin{itemize}
        \item<1-> For every return address in an OCaml program, there is a associated \typename{frame\_descr}
        \item<2-> A \typename{frame\_descr} specifies
        \begin{itemize}
          \item<2-> The size of the function stack frame
          \item<3-> Where live values are on the stack (so that the GC can mark them as live and prevent from being swept)
        \end{itemize}
        \item<4-> When compiled with debug information, \typename{frame\_descr} will be linked to a \typename{frame\_debuginfo}
        \begin{itemize}
          \item<5-> Contains information about the position in the source code (file name, line and column number)
        \end{itemize}
      \end{itemize}
    \end{column}
    \begin{column}{0.5\textwidth}
        \onslide*<1>{\listFrameDescr[highlightlines={118-122}]{}}
        \onslide*<2>{\listFrameDescr[highlightlines={120}]{}}
        \onslide*<3>{\listFrameDescr[highlightlines={121}]{}}
        \onslide*<4->{\listFrameDescr[highlightlines={122}]{}}
        \medskip
        \onslide*<1-3>{\listDebugInfo{}}
        \onslide*<4>{\listDebugInfo[highlightlines={38,49}]{}}
        \onslide*<5>{\listDebugInfo[highlightlines={39-46}]{}}
    \end{column}
  \end{columns}
\end{frame}

\begin{frame}{Backtraces}{Scanning the stack with \typename{frame\_descr}}
  \begin{columns}[c]
    \begin{column}{0.5\textwidth}
      \begin{itemize}
        \item Given the return address of the code that raises the exception by calling \funcname{caml\_raise\_exn} (\funcarg{pc} argument of \funcname{caml\_stash\_backtrace})
        \item And the position of the stack pointer (\funcarg{sp} argument of \funcname{caml\_stash\_backtrace})
        \item The frame size given by \typename{frame\_descr}.\funcarg{frame\_size}, allows to find the end of the previous frame
        \item And the return address of the caller of this function
        \bigskip
        \onslide<3->{
        \item Note that traps (here the reddish \funcname{ExnA}, \funcname{ExnB} and \funcname{ExnC} blocks) belong to the frame that installed them, and so the frame size changes when traps are removed
        }
      \end{itemize}
    \end{column}
    \begin{column}{0.5\textwidth}
      \raggedleft
      \onslide*<1>{\providecommand\step{2}\input{diagrams/backtrace/backtrace.tex}}%
      \onslide*<2>{\providecommand\step{1}\input{diagrams/backtrace/scanstack.tex}}%
      \onslide*<3>{\providecommand\step{2}\input{diagrams/backtrace/scanstack.tex}}%
      \onslide*<4>{\providecommand\step{3}\input{diagrams/backtrace/scanstack.tex}}%
      \onslide*<5>{\providecommand\step{4}\input{diagrams/backtrace/scanstack.tex}}%
      \onslide*<6>{\providecommand\step{5}\input{diagrams/backtrace/scanstack.tex}}%
    \end{column}
  \end{columns}
\end{frame}

% Duplicate of previous-previous slide
\begin{frame}{Backtraces}{Continuing on \funcname{caml\_stash\_backtrace}}
  \begin{columns}[c]
    \begin{column}{0.5\textwidth}
      \minipage[c][0.75\textheight][s]{\columnwidth}
      \begin{itemize}
          \color{gray}
        \item A C function provided by the runtime (specific to native code)
        \item It scans the stack, between the stack pointer at the raising point up to the next trap, in order to collect the names of the detected function frames
        \item It uses the same technique and information as the Garbage Collector (GC) uses to scan the values live on the stack
      \end{itemize}
      \begin{itemize}
        \item For each function frame detected, store a pointer to the \typename{frame\_descr} in the \localname{domain\_state->backtrace\_buffer} array, effectively constructing the backtrace
      \end{itemize}
      \onslide<2->{
        \begin{itemize}
            \smallskip
          \item Constructed backtrace:
            \funcname{definitely\_raise},
            \onslide<3->{\funcname{probably\_raise}}
        \end{itemize}
      }
      \vfill
      \mintocaml[firstline=9,lastline=13,highlightlines=12]{../src/backtrace/backtrace.ml}
      \endminipage
    \end{column}
    \begin{column}{0.5\textwidth}
      \raggedleft
      \onslide*<1>{\listStashBacktrace[highlightlines={114-115}]{}}
      \onslide*<2>{\providecommand\step{3}\input{diagrams/backtrace/backtrace.tex}}%
      \onslide*<3>{\providecommand\step{4}\input{diagrams/backtrace/backtrace.tex}}%
    \end{column}
  \end{columns}
\end{frame}

\begin{frame}{Backtraces}{Back to \funcname{caml\_raise\_exn}}
  \begin{columns}[c]
    \begin{column}{0.5\textwidth}
      \minipage[c][0.66\textheight][s]{\columnwidth}
      \begin{itemize}\color{gray}
        \item Checks wheteher exceptions backtrace should be recorded, by checking the \funcarg{domain\_state->backtrace\_active} value
        \item Switches to the C stack to be able to execute C code
        \item Calls \funcname{caml\_stash\_backtrace}, to collect the backtrace up to the next trap
      \end{itemize}
      \onslide*<2->{
        \begin{itemize}
          \item<2-> Executes the exception handler in \funcname{probably\_raise} with \funcname{RESTORE\_EXN\_HANDLER\_OCAML}
            \begin{itemize}
              \item Set \regname{RSP} to \localname{domain\_state->exn\_handler} value
              \item Restore \localname{domain\_state->exn\_handler} from trap
              \item Jump to the handler code with \funcname{ret}
            \end{itemize}
        \end{itemize}
      }
      \vfill
      \onslide*<1>{\mintocaml[firstline=9,lastline=13,highlightlines=12]{../src/backtrace/backtrace.ml}}
      \onslide*<2->{\mintocaml[firstline=15,lastline=21,highlightlines={19-21}]{../src/backtrace/backtrace.ml}}
      \endminipage
    \end{column}
    \begin{column}{0.5\textwidth}
      \centering
      \onslide*<1>{
        \mintc[firstline=190,lastline=192]{../ocaml/runtime/amd64.S}
        \mintc[firstline=234,lastline=237]{../ocaml/runtime/amd64.S}
      }
      \onslide*<2>{
        \mintc[firstline=190,lastline=192,highlightlines={191-192}]{../ocaml/runtime/amd64.S}
        \mintc[firstline=234,lastline=237,highlightlines={235-237}]{../ocaml/runtime/amd64.S}
      }
      \onslide*<1>{\listCamlRaiseExn[highlightlines=720]{}}
      \onslide*<2>{\listCamlRaiseExn[highlightlines=722]{}}
      \onslide*<3->{\providecommand\step{5}\input{diagrams/backtrace/backtrace.tex}}
    \end{column}
  \end{columns}
\end{frame}

\begin{frame}{Backtraces}{\funcname{caml\_probably\_raise}'s handler doesn't catch \typename{ExnC}}
  \begin{columns}[c]
    \begin{column}{0.45\textwidth}
      \minipage[c][0.75\textheight][s]{\columnwidth}
      \begin{itemize}
        \item<1-> \funcname{match \dots with}\footnotemark pattern matching can also match exceptions
        \item<1-> The CMM of \funcname{probably\_raise} shows that this \funcname{match \dots with} is implemented with a pattern matching and a \funcname{try \dots with}
        \item<1-> But this one only matches on \typename{ExnA} exception
        \item<2-> Since the pattern matching fails to catch the exception, \funcname{reraise} is called with our current \typename{ExnC} exception
        \item<3-> Disassembly shows that \funcname{reraise} is actually a call to \funcname{caml\_reraise\_exn}, which is also an assembly function provided by the runtime
      \end{itemize}
      \vfill
      \mintocaml[firstline=15,lastline=21,highlightlines={18-21}]{../src/backtrace/backtrace.ml}
      \endminipage
    \end{column}
    \begin{column}{0.55\textwidth}
      \centering
      \onslide*<1>{\mintcmm[highlightlines={10-18}]{../src/backtrace/camlBacktrace\_\_probably\_raise\_351.cmm}}
      \onslide*<2>{\listcmm[highlightlines=18]{../src/backtrace/camlBacktrace\_\_probably\_raise_351.cmm}{CMM of probably\_raise}}
      \onslide*<3>{\mintobjdump[firstline=5,lastline=35,highlightlines=31]{../src/backtrace/camlBacktrace\_\_probably\_raise_351.objdump}}
    \end{column}
  \end{columns}
  \only<1>{\footnotetext[1]{https://blog.janestreet.com/pattern-matching-and-exception-handling-unite/}}
\end{frame}

\newcommand\listCamlReraiseExn[1][]{
  \listasm[firstline=727,lastline=734,#1]{../ocaml/runtime/amd64.S}{runtime/amd64.S:727}}

\begin{frame}{Backtraces}{Introducing \funcname{caml\_reraise\_exn}}
  \begin{columns}[c]
    \begin{column}{0.5\textwidth}
      \minipage[c][0.66\textheight][s]{\columnwidth}
      \begin{itemize}
        \item<1-> The exception have to be reraised to the next handler
        \item<2-> First, checks if the backtrace should be collected with \funcarg{domain\_state->backtrace\_active}
        \only<3>{
        \begin{itemize}
            \item If not, behaves like a \funcname{raise\_notrace} with \funcname{RESTORE\_EXN\_HANDLER\_OCAML}
        \end{itemize}
        }
        \item<4-> Jumps into \funcname{caml\_raise\_exn}
        \item<5-> Calls \funcname{caml\_stash\_backtrace} again, to scan the next part of the stack: from the trap just pop-ed to the next trap
      \end{itemize}
      \vfill
      \mintocaml[firstline=15,lastline=21,highlightlines={18-21}]{../src/backtrace/backtrace.ml}
      \endminipage
    \end{column}
    \begin{column}{0.5\textwidth}
      \raggedleft
      \onslide*<1>{\listCamlReraiseExn{}}
      \onslide*<2>{\listCamlReraiseExn[highlightlines=729]{}}
      \onslide*<3>{
        \mintc[firstline=190,lastline=192,highlightlines={191-192}]{../ocaml/runtime/amd64.S}
        \mintc[firstline=234,lastline=237,highlightlines={235-237}]{../ocaml/runtime/amd64.S}
        \medskip
        \listCamlReraiseExn[highlightlines=731]{}
      }
      \onslide*<4-5>{
        % TODO refactor with \listCamlReraiseExn
        \mintasm[firstline=727,lastline=734,highlightlines=730]{../ocaml/runtime/amd64.S}
        \medskip
      }
      \onslide*<4>{\listCamlRaiseExn[highlightlines=711]{}}
      \onslide*<5>{\listCamlRaiseExn[highlightlines=720]{}}
    \end{column}
  \end{columns}
\end{frame}

\begin{frame}{Backtraces}{Backtrace collection continues}
  \begin{columns}[c]
    \begin{column}{0.55\textwidth}
      \minipage[c][0.75\textheight][s]{\columnwidth}
      \begin{itemize}
        \item<1-> \funcname{caml\_stash\_backtrace} scans the older part of the stack, up to the next trap
        \item<6-> Controls jumps to the nested exception handler in \funcname{maybe\_raise}, which matches only on \typename{ExnB}
        \item<7-> \funcname{caml\_reraise\_exn} is called again, backtrace collection resumes with \funcname{caml\_stash\_backtrace}
      \end{itemize}
      \medskip
      \begin{itemize}
        \item<2-> Constructed backtrace:
        \begin{itemize}
          \item <2-> \funcname{definitely\_raise}
          \item <2-> \funcname{probably\_raise}
          \item <3-> \funcname{probably\_raise} (re-raise)
          \item <4-> \funcname{maybe\_raise}
          \item <5-> \funcname{main}
          \item <8-> \funcname{main} (re-raise)
        \end{itemize}
      \end{itemize}
      \vfill
      \onslide*<1-5>{\mintocaml[firstline=15,lastline=21]{../src/backtrace/backtrace.ml}}
      \onslide*<6>{\mintocaml[firstline=28,lastline=34,highlightlines=31]{../src/backtrace/backtrace.ml}}
      \onslide*<7->{\mintocaml[firstline=28,lastline=34,highlightlines=33]{../src/backtrace/backtrace.ml}}
      \endminipage
    \end{column}
    \begin{column}{0.45\textwidth}
      \raggedleft
      \onslide*<1>{\providecommand\step{6}\input{diagrams/backtrace/backtrace.tex}}%
      \onslide*<2>{\providecommand\step{6}\input{diagrams/backtrace/backtrace.tex}}%
      \onslide*<3>{\providecommand\step{7}\input{diagrams/backtrace/backtrace.tex}}%
      \onslide*<4>{\providecommand\step{8}\input{diagrams/backtrace/backtrace.tex}}%
      \onslide*<5>{\providecommand\step{9}\input{diagrams/backtrace/backtrace.tex}}%
      \onslide*<6>{\providecommand\step{10}\input{diagrams/backtrace/backtrace.tex}}%
      \onslide*<7>{\providecommand\step{11}\input{diagrams/backtrace/backtrace.tex}}%
      \onslide*<8>{\providecommand\step{12}\input{diagrams/backtrace/backtrace.tex}}%
      \onslide*<9>{\providecommand\step{13}\input{diagrams/backtrace/backtrace.tex}}%
    \end{column}
  \end{columns}
\end{frame}

\begin{frame}{Backtraces}{Printing the backtrace}
  \begin{columns}[c]
    \begin{column}{0.45\textwidth}
      \minipage[c][0.66\textheight][s]{\columnwidth}
      \begin{itemize}
        \item<1-> The exception backtrace can now be printed to \funcarg{stdout} with \funcname{Printexc.print_backtrace}
        \item<2-> First the exception backtrace is retrieved into an OCaml array with \funcname{get_raw_backtrace }
        \item<3-> Which is implemented as the \funcname{caml_get_exception_raw_backtrace} C function in the runtime
        \item<4-> And finally printed with \funcname{print_exception_backtrace}
      \end{itemize}
      \vfill
      \mintocaml[firstline=28,lastline=34,highlightlines=33]{../src/backtrace/backtrace.ml}
      \endminipage
    \end{column}
    \begin{column}{0.55\textwidth}
      \onslide*<1,2>{
        \mintocaml[firstline=100,lastline=101]{../ocaml/stdlib/printexc.ml}
      }
      \only<1>{
        \listocaml[firstline=170,lastline=172]{../ocaml/stdlib/printexc.ml}{stdlib/printexc.ml:170}
      }
      \only<2>{
        \listocaml[firstline=170,lastline=172,highlightlines=172]{../ocaml/stdlib/printexc.ml}{stdlib/printexc.ml:170}
      }
      \only<3>{
        \mintc[firstline=154,lastline=156]{../ocaml/runtime/backtrace.c}
        \listc[firstline=171,lastline=192,highlightlines={184-187}]{../ocaml/runtime/backtrace.c}{runtime/backtrace.c:154}
      }
      \only<4>{
        \listocaml[firstline=155,lastline=165]{../ocaml/stdlib/printexc.ml}{stdlib/printexc.ml:55}
      }
    \end{column}
  \end{columns}
\end{frame}


\subsection{The multiple kinds of raise}
\frameSubsection{}{}

\begin{frame}{Backtraces}{When backtraces are not needed}
  \begin{columns}[c]
    \begin{column}{0.45\textwidth}
      \minipage[c][0.66\textheight][s]{\columnwidth}
      \begin{itemize}
        \item<1-> What if the backtrace for an exception is never needed?
        \item<2-> It's possible to raise the exception explicitly with \funcname{raise\_notrace} to make raising as cheap as possible
        \item<3-> An example of use in the \funcname{dynlink} library of the OCaml compiler
      \end{itemize}
      \vfill
      \onslide<3->{
      \listocaml[firstline=205,lastline=209,highlightlines=207]{../ocaml/otherlibs/dynlink/dynlink\_compilerlibs/misc.ml}{otherlibs/dynlink/dynlink\_compilerlibs/misc.ml:205}
      }
      \endminipage
    \end{column}
    \begin{column}{0.55\textwidth}
    \only<4>{
      \mintcmm[highlightlines=3]{../src/notrace/camlNotrace\_\_fun_347.cmm}
      \listcmm{../src/notrace/camlNotrace\_\_all\_somes\_327.cmm}{all\_somes}
    }
    \onslide*<5->{
      \listobjdump[highlightlines={6-9}]{../src/notrace/camlNotrace\_\_fun_347.objdump}{anonymous function from \funcname{all\_somes}}
    }
    \end{column}
  \end{columns}
\end{frame}

\begin{frame}{Backtraces}{Summary table of the multiple kinds of raise}
  \begin{itemize}
    \item When debugging information is enabled and if the backtrace of an exception isn't needed, it's possible to raise with \funcname{raise\_notrace} for performance
    \item When debugging information is \emph{not} enabled, the compiler optimises \funcname{raise} calls to inline assembly (same effect as using \funcname{raise\_notrace})
  \end{itemize}
  \bigskip
  \centering
  \begin{tabular}{| l | c | c | c | c |}
    \hline
    \parbox{10em}{Debug information \\ (\funcarg{-g} flag)}  & \multicolumn{2}{|c|}{Disabled}                                     & \multicolumn{2}{|c|}{Enabled} \\
    \hline
    Raise kind                                               & \funcname{raise} & \funcname{raise\_notrace}                       & \funcname{raise} & \funcname{raise\_notrace} \\
    \hline
    Compilation                                              & \parbox{8em}{inline assembly\\ (optimization)} & inline assembly   & \funcname{caml\_raise\_exn} & inline assembly \\
    \hline
  \end{tabular}
\end{frame}


%
% Takeaway #5
%
\subsection*{Takeaway \#5}
\frameSubsectionTakeaway{}
\againframe<5>{takeaway}
}%
      \onslide*<7>{\providecommand\step{11}%
% Backtraces
%
\subsection{One advantage of raising with debugging information}
\frameSubsection{}{}

\begin{frame}{Backtraces}{Debugging information \& source code}
  \begin{columns}[c]
    \begin{column}{0.5\textwidth}
      \begin{itemize}
        \item Raising an exception is fun and games, but how do we know where the exception originated? $\rightarrow$ With the backtrace associated with the exception!
        \item In order to get a backtrace from the point where an exception was raised, it's necessary to compile with debugging information enabled (\funcarg{-g} flag for the compiler)
        \item Same example as in the `Nested handlers' section
      \end{itemize}
    \end{column}
    \begin{column}{0.5\textwidth}
      \listocaml[firstline=9,lastline=34]{../src/backtrace/backtrace.ml}{backtrace/backtrace.ml}
    \end{column}
  \end{columns}
\end{frame}

\begin{frame}{Backtraces}{CMM and disassembly of \funcname{definitely\_raise}}
  \begin{columns}[c]
    \begin{column}{0.5\textwidth}
      \minipage[c][0.66\textheight][s]{\columnwidth}
      \begin{itemize}
        \item<1-> An effect of compiling with debugging information (\funcarg{-g}): the CMM no longer uses \funcname{raise\_notrace} to raise the exception but \funcname{raise} instead
        \item<2-> Disassembly shows that CMM \funcname{raise} is actually a call to \funcname{caml\_raise\_exn}, a function in assembler provided by the runtime
      \end{itemize}
      \vfill
      \mintocaml[firstline=9,lastline=13,highlightlines=12]{../src/backtrace/backtrace.ml}
      \endminipage
    \end{column}
    \begin{column}{0.5\textwidth}
      \onslide*<1>{
        \mintcmm[highlightlines=5]{../src/backtrace/camlBacktrace\_\_definitely\_raise\_347.cmm}
      }
      \onslide*<2>{
        \mintobjdump[firstline=2,highlightlines=14]{../src/backtrace/camlBacktrace\_\_definitely\_raise\_347.objdump}
      }
    \end{column}
  \end{columns}
\end{frame}

\begin{frame}{Backtraces}{Introducing \funcname{caml\_raise\_exn}}
  \begin{columns}[c]
    \begin{column}{0.5\textwidth}
      \minipage[c][0.66\textheight][s]{\columnwidth}
      \onslide*<1>{
        \begin{itemize}
          \item Raising the exception is performed using \funcname{caml\_raise\_exn} which is
            \begin{itemize}
              \item An assembly function provided by the runtime
              \item Architecture specific
            \end{itemize}
          \item Let's see what it does differently from \funcname{raise\_notrace}\dots
          \item We are about to raise \typename{ExnC} with \funcname{caml\_raise\_exn} from \funcname{definitely\_raise}
        \end{itemize}
      }
      \onslide*<2>{
        \mintobjdump[firstline=2,highlightlines=14]{../src/backtrace/camlBacktrace\_\_definitely\_raise\_347.objdump}
      }
      \vfill
      \mintocaml[firstline=9,lastline=13,highlightlines=12]{../src/backtrace/backtrace.ml}
      \endminipage
    \end{column}
    \begin{column}{0.5\textwidth}
      \centering
      \providecommand\step{1}\input{diagrams/backtrace/backtrace.tex}
    \end{column}
  \end{columns}
\end{frame}

\begin{frame}{Backtraces}{But before that, we need more fields from \typename{caml\_domain\_state}}
  \begin{columns}
    \begin{column}{0.5\textwidth}
      \begin{itemize}
        \item Remember \typename{caml\_domain\_state} the per-domain runtime state?
        \item Some other members are of interest to us now\dots
        \item \localname{backtrace\_active} is used to know if the runtime should collect backtraces
        \item \localname{backtrace\_active} can be set by
          \begin{itemize}
            \item The \funcarg{b} flag in the \funcname{OCAMLRUNPARAM} environment variable
            \item \mintinline{ocaml}{Printexc.record_backtrace : bool -> unit} \footnotemark
          \end{itemize}
      \end{itemize}
    \end{column}
    \begin{column}{0.5\textwidth}
      \begin{listing}
        \mintc[highlightlines={9-11}]{src/ocaml/domain\_state\_backtrace.h}
        \caption{runtime/caml/domain\_state.h}
      \end{listing}
    \end{column}
  \end{columns}
  \footnotetext[1]{https://v2.ocaml.org/api/Printexc.html}
\end{frame}

\newcommand\listCamlRaiseExn[1][]{
  \listasm[firstline=702,lastline=725,#1]{../ocaml/runtime/amd64.S}{runtime/amd64.S:702}}

\begin{frame}{Backtraces}{Walk through \funcname{caml\_raise\_exn}}
  \begin{columns}[T]
    \begin{column}{0.5\textwidth}
      \begin{itemize}
        \item<1-> First checks whether exceptions backtrace should be recorded
        \item<1-> By checking the \funcarg{domain\_state->backtrace\_active} value
        \item<2-> If not, acts as \funcname{raise\_notrace} with \funcname{RESTORE\_EXN\_HANDLER\_OCAML}
          \begin{itemize}
            \item Set \regname{RSP} to \localname{domain\_state->exn\_handler} value
            \item Restore \localname{domain\_state->exn\_handler} from trap
            \item Jump with \funcname{ret} (same effect as using \funcname{pop} and \funcname{jmp})
          \end{itemize}
        \item<3-> Otherwise, switches to the C stack to be able to execute C code
        \item<4-> Calls \funcname{caml\_stash\_backtrace}, a C function provided by the runtime\ignorespaces
          \onslide<6->{, with
            \begin{itemize}
              \item \funcarg{exn}: exception value
              \item \funcarg{pc}: code address of the raising point
              \item \funcarg{sp}: stack address of the raising function
              \item \funcarg{trapsp}: stack address of the next handler
            \end{itemize}
          }
      \end{itemize}
      \bigskip
      \mintocaml[firstline=9,lastline=13,highlightlines=12]{../src/backtrace/backtrace.ml}
    \end{column}
    \begin{column}{0.5\textwidth}
      \raggedleft
      \onslide*<1,3-4>{
        \mintc[firstline=190,lastline=192]{../ocaml/runtime/amd64.S}
        \mintc[firstline=234,lastline=237]{../ocaml/runtime/amd64.S}
      }
      \onslide*<2>{
        \mintc[firstline=190,lastline=192,highlightlines={191-192}]{../ocaml/runtime/amd64.S}
        \mintc[firstline=234,lastline=237,highlightlines={235-237}]{../ocaml/runtime/amd64.S}
      }
      \onslide*<1>{\listCamlRaiseExn[highlightlines=705]{}}%
      \onslide*<2>{\listCamlRaiseExn[highlightlines={707-708}]{}}%
      \onslide*<3>{\listCamlRaiseExn[highlightlines={712-714}]{}}%
      \onslide*<4>{\listCamlRaiseExn[highlightlines={715-720}]{}}%
      \onslide*<5>{\providecommand\step{1}\input{diagrams/backtrace/backtrace.tex}}%
      \onslide*<6>{\providecommand\step{2}\input{diagrams/backtrace/backtrace.tex}}%
    \end{column}
  \end{columns}
\end{frame}

\newcommand\listStashBacktrace[1][]{
  \listc[firstline=91,lastline=120,#1]{../ocaml/runtime/backtrace\_nat.c}{runtime/backtrace\_nat.c:91}}

\begin{frame}{Backtraces}{Introducing \funcname{caml\_stash\_backtrace}}
  \begin{columns}[c]
    \begin{column}{0.5\textwidth}
      \minipage[c][0.66\textheight][s]{\columnwidth}
      \begin{itemize}
        \item<1-> A C function provided by the runtime (specific to native code)
        \item<2-> It scans the stack, between the stack pointer at the raising point up to the next trap, in order to collect the names of the detected function frames
          \onslide*<2>{
            \begin{itemize}
              \item And more information, like line number, column number...
            \end{itemize}
          }
        \item<3-> It uses the same technique and information as the Garbage Collector (GC) uses to scan the values live on the stack
          \onslide*<4>{
            \begin{itemize}
              \item \typename{frame\_descr} are at the core of stack unwinding
            \end{itemize}
          }
      \end{itemize}
      \vfill
      \mintocaml[firstline=9,lastline=13,highlightlines=12]{../src/backtrace/backtrace.ml}
      \endminipage
    \end{column}
    \begin{column}{0.5\textwidth}
      \onslide*<1>{\listStashBacktrace[highlightlines=91]{}}
      \onslide*<2>{\listStashBacktrace[highlightlines={91,117-118}]{}}
      \onslide*<3->{\listStashBacktrace[highlightlines={94,106,109-110}]{}}
    \end{column}
  \end{columns}
\end{frame}

\newcommand\listFrameDescr[1][]{
  \listocaml[firstline=111,lastline=124,#1]{../ocaml/asmcomp/emitaux.ml}{asmcomp/emitaux.ml:111}}
\newcommand\listDebugInfo[1][]{
  \listocaml[firstline=38,lastline=49,#1]{../ocaml/lambda/debuginfo.mli}{lambda/debuginfo.mli:38}}

\begin{frame}{Backtraces}{\typename{frame\_descr} and \typename{frame\_debuginfo} in a nutshell}
  \begin{columns}[c]
    \begin{column}{0.5\textwidth}
      \begin{itemize}
        \item<1-> For every return address in an OCaml program, there is a associated \typename{frame\_descr}
        \item<2-> A \typename{frame\_descr} specifies
        \begin{itemize}
          \item<2-> The size of the function stack frame
          \item<3-> Where live values are on the stack (so that the GC can mark them as live and prevent from being swept)
        \end{itemize}
        \item<4-> When compiled with debug information, \typename{frame\_descr} will be linked to a \typename{frame\_debuginfo}
        \begin{itemize}
          \item<5-> Contains information about the position in the source code (file name, line and column number)
        \end{itemize}
      \end{itemize}
    \end{column}
    \begin{column}{0.5\textwidth}
        \onslide*<1>{\listFrameDescr[highlightlines={118-122}]{}}
        \onslide*<2>{\listFrameDescr[highlightlines={120}]{}}
        \onslide*<3>{\listFrameDescr[highlightlines={121}]{}}
        \onslide*<4->{\listFrameDescr[highlightlines={122}]{}}
        \medskip
        \onslide*<1-3>{\listDebugInfo{}}
        \onslide*<4>{\listDebugInfo[highlightlines={38,49}]{}}
        \onslide*<5>{\listDebugInfo[highlightlines={39-46}]{}}
    \end{column}
  \end{columns}
\end{frame}

\begin{frame}{Backtraces}{Scanning the stack with \typename{frame\_descr}}
  \begin{columns}[c]
    \begin{column}{0.5\textwidth}
      \begin{itemize}
        \item Given the return address of the code that raises the exception by calling \funcname{caml\_raise\_exn} (\funcarg{pc} argument of \funcname{caml\_stash\_backtrace})
        \item And the position of the stack pointer (\funcarg{sp} argument of \funcname{caml\_stash\_backtrace})
        \item The frame size given by \typename{frame\_descr}.\funcarg{frame\_size}, allows to find the end of the previous frame
        \item And the return address of the caller of this function
        \bigskip
        \onslide<3->{
        \item Note that traps (here the reddish \funcname{ExnA}, \funcname{ExnB} and \funcname{ExnC} blocks) belong to the frame that installed them, and so the frame size changes when traps are removed
        }
      \end{itemize}
    \end{column}
    \begin{column}{0.5\textwidth}
      \raggedleft
      \onslide*<1>{\providecommand\step{2}\input{diagrams/backtrace/backtrace.tex}}%
      \onslide*<2>{\providecommand\step{1}\input{diagrams/backtrace/scanstack.tex}}%
      \onslide*<3>{\providecommand\step{2}\input{diagrams/backtrace/scanstack.tex}}%
      \onslide*<4>{\providecommand\step{3}\input{diagrams/backtrace/scanstack.tex}}%
      \onslide*<5>{\providecommand\step{4}\input{diagrams/backtrace/scanstack.tex}}%
      \onslide*<6>{\providecommand\step{5}\input{diagrams/backtrace/scanstack.tex}}%
    \end{column}
  \end{columns}
\end{frame}

% Duplicate of previous-previous slide
\begin{frame}{Backtraces}{Continuing on \funcname{caml\_stash\_backtrace}}
  \begin{columns}[c]
    \begin{column}{0.5\textwidth}
      \minipage[c][0.75\textheight][s]{\columnwidth}
      \begin{itemize}
          \color{gray}
        \item A C function provided by the runtime (specific to native code)
        \item It scans the stack, between the stack pointer at the raising point up to the next trap, in order to collect the names of the detected function frames
        \item It uses the same technique and information as the Garbage Collector (GC) uses to scan the values live on the stack
      \end{itemize}
      \begin{itemize}
        \item For each function frame detected, store a pointer to the \typename{frame\_descr} in the \localname{domain\_state->backtrace\_buffer} array, effectively constructing the backtrace
      \end{itemize}
      \onslide<2->{
        \begin{itemize}
            \smallskip
          \item Constructed backtrace:
            \funcname{definitely\_raise},
            \onslide<3->{\funcname{probably\_raise}}
        \end{itemize}
      }
      \vfill
      \mintocaml[firstline=9,lastline=13,highlightlines=12]{../src/backtrace/backtrace.ml}
      \endminipage
    \end{column}
    \begin{column}{0.5\textwidth}
      \raggedleft
      \onslide*<1>{\listStashBacktrace[highlightlines={114-115}]{}}
      \onslide*<2>{\providecommand\step{3}\input{diagrams/backtrace/backtrace.tex}}%
      \onslide*<3>{\providecommand\step{4}\input{diagrams/backtrace/backtrace.tex}}%
    \end{column}
  \end{columns}
\end{frame}

\begin{frame}{Backtraces}{Back to \funcname{caml\_raise\_exn}}
  \begin{columns}[c]
    \begin{column}{0.5\textwidth}
      \minipage[c][0.66\textheight][s]{\columnwidth}
      \begin{itemize}\color{gray}
        \item Checks wheteher exceptions backtrace should be recorded, by checking the \funcarg{domain\_state->backtrace\_active} value
        \item Switches to the C stack to be able to execute C code
        \item Calls \funcname{caml\_stash\_backtrace}, to collect the backtrace up to the next trap
      \end{itemize}
      \onslide*<2->{
        \begin{itemize}
          \item<2-> Executes the exception handler in \funcname{probably\_raise} with \funcname{RESTORE\_EXN\_HANDLER\_OCAML}
            \begin{itemize}
              \item Set \regname{RSP} to \localname{domain\_state->exn\_handler} value
              \item Restore \localname{domain\_state->exn\_handler} from trap
              \item Jump to the handler code with \funcname{ret}
            \end{itemize}
        \end{itemize}
      }
      \vfill
      \onslide*<1>{\mintocaml[firstline=9,lastline=13,highlightlines=12]{../src/backtrace/backtrace.ml}}
      \onslide*<2->{\mintocaml[firstline=15,lastline=21,highlightlines={19-21}]{../src/backtrace/backtrace.ml}}
      \endminipage
    \end{column}
    \begin{column}{0.5\textwidth}
      \centering
      \onslide*<1>{
        \mintc[firstline=190,lastline=192]{../ocaml/runtime/amd64.S}
        \mintc[firstline=234,lastline=237]{../ocaml/runtime/amd64.S}
      }
      \onslide*<2>{
        \mintc[firstline=190,lastline=192,highlightlines={191-192}]{../ocaml/runtime/amd64.S}
        \mintc[firstline=234,lastline=237,highlightlines={235-237}]{../ocaml/runtime/amd64.S}
      }
      \onslide*<1>{\listCamlRaiseExn[highlightlines=720]{}}
      \onslide*<2>{\listCamlRaiseExn[highlightlines=722]{}}
      \onslide*<3->{\providecommand\step{5}\input{diagrams/backtrace/backtrace.tex}}
    \end{column}
  \end{columns}
\end{frame}

\begin{frame}{Backtraces}{\funcname{caml\_probably\_raise}'s handler doesn't catch \typename{ExnC}}
  \begin{columns}[c]
    \begin{column}{0.45\textwidth}
      \minipage[c][0.75\textheight][s]{\columnwidth}
      \begin{itemize}
        \item<1-> \funcname{match \dots with}\footnotemark pattern matching can also match exceptions
        \item<1-> The CMM of \funcname{probably\_raise} shows that this \funcname{match \dots with} is implemented with a pattern matching and a \funcname{try \dots with}
        \item<1-> But this one only matches on \typename{ExnA} exception
        \item<2-> Since the pattern matching fails to catch the exception, \funcname{reraise} is called with our current \typename{ExnC} exception
        \item<3-> Disassembly shows that \funcname{reraise} is actually a call to \funcname{caml\_reraise\_exn}, which is also an assembly function provided by the runtime
      \end{itemize}
      \vfill
      \mintocaml[firstline=15,lastline=21,highlightlines={18-21}]{../src/backtrace/backtrace.ml}
      \endminipage
    \end{column}
    \begin{column}{0.55\textwidth}
      \centering
      \onslide*<1>{\mintcmm[highlightlines={10-18}]{../src/backtrace/camlBacktrace\_\_probably\_raise\_351.cmm}}
      \onslide*<2>{\listcmm[highlightlines=18]{../src/backtrace/camlBacktrace\_\_probably\_raise_351.cmm}{CMM of probably\_raise}}
      \onslide*<3>{\mintobjdump[firstline=5,lastline=35,highlightlines=31]{../src/backtrace/camlBacktrace\_\_probably\_raise_351.objdump}}
    \end{column}
  \end{columns}
  \only<1>{\footnotetext[1]{https://blog.janestreet.com/pattern-matching-and-exception-handling-unite/}}
\end{frame}

\newcommand\listCamlReraiseExn[1][]{
  \listasm[firstline=727,lastline=734,#1]{../ocaml/runtime/amd64.S}{runtime/amd64.S:727}}

\begin{frame}{Backtraces}{Introducing \funcname{caml\_reraise\_exn}}
  \begin{columns}[c]
    \begin{column}{0.5\textwidth}
      \minipage[c][0.66\textheight][s]{\columnwidth}
      \begin{itemize}
        \item<1-> The exception have to be reraised to the next handler
        \item<2-> First, checks if the backtrace should be collected with \funcarg{domain\_state->backtrace\_active}
        \only<3>{
        \begin{itemize}
            \item If not, behaves like a \funcname{raise\_notrace} with \funcname{RESTORE\_EXN\_HANDLER\_OCAML}
        \end{itemize}
        }
        \item<4-> Jumps into \funcname{caml\_raise\_exn}
        \item<5-> Calls \funcname{caml\_stash\_backtrace} again, to scan the next part of the stack: from the trap just pop-ed to the next trap
      \end{itemize}
      \vfill
      \mintocaml[firstline=15,lastline=21,highlightlines={18-21}]{../src/backtrace/backtrace.ml}
      \endminipage
    \end{column}
    \begin{column}{0.5\textwidth}
      \raggedleft
      \onslide*<1>{\listCamlReraiseExn{}}
      \onslide*<2>{\listCamlReraiseExn[highlightlines=729]{}}
      \onslide*<3>{
        \mintc[firstline=190,lastline=192,highlightlines={191-192}]{../ocaml/runtime/amd64.S}
        \mintc[firstline=234,lastline=237,highlightlines={235-237}]{../ocaml/runtime/amd64.S}
        \medskip
        \listCamlReraiseExn[highlightlines=731]{}
      }
      \onslide*<4-5>{
        % TODO refactor with \listCamlReraiseExn
        \mintasm[firstline=727,lastline=734,highlightlines=730]{../ocaml/runtime/amd64.S}
        \medskip
      }
      \onslide*<4>{\listCamlRaiseExn[highlightlines=711]{}}
      \onslide*<5>{\listCamlRaiseExn[highlightlines=720]{}}
    \end{column}
  \end{columns}
\end{frame}

\begin{frame}{Backtraces}{Backtrace collection continues}
  \begin{columns}[c]
    \begin{column}{0.55\textwidth}
      \minipage[c][0.75\textheight][s]{\columnwidth}
      \begin{itemize}
        \item<1-> \funcname{caml\_stash\_backtrace} scans the older part of the stack, up to the next trap
        \item<6-> Controls jumps to the nested exception handler in \funcname{maybe\_raise}, which matches only on \typename{ExnB}
        \item<7-> \funcname{caml\_reraise\_exn} is called again, backtrace collection resumes with \funcname{caml\_stash\_backtrace}
      \end{itemize}
      \medskip
      \begin{itemize}
        \item<2-> Constructed backtrace:
        \begin{itemize}
          \item <2-> \funcname{definitely\_raise}
          \item <2-> \funcname{probably\_raise}
          \item <3-> \funcname{probably\_raise} (re-raise)
          \item <4-> \funcname{maybe\_raise}
          \item <5-> \funcname{main}
          \item <8-> \funcname{main} (re-raise)
        \end{itemize}
      \end{itemize}
      \vfill
      \onslide*<1-5>{\mintocaml[firstline=15,lastline=21]{../src/backtrace/backtrace.ml}}
      \onslide*<6>{\mintocaml[firstline=28,lastline=34,highlightlines=31]{../src/backtrace/backtrace.ml}}
      \onslide*<7->{\mintocaml[firstline=28,lastline=34,highlightlines=33]{../src/backtrace/backtrace.ml}}
      \endminipage
    \end{column}
    \begin{column}{0.45\textwidth}
      \raggedleft
      \onslide*<1>{\providecommand\step{6}\input{diagrams/backtrace/backtrace.tex}}%
      \onslide*<2>{\providecommand\step{6}\input{diagrams/backtrace/backtrace.tex}}%
      \onslide*<3>{\providecommand\step{7}\input{diagrams/backtrace/backtrace.tex}}%
      \onslide*<4>{\providecommand\step{8}\input{diagrams/backtrace/backtrace.tex}}%
      \onslide*<5>{\providecommand\step{9}\input{diagrams/backtrace/backtrace.tex}}%
      \onslide*<6>{\providecommand\step{10}\input{diagrams/backtrace/backtrace.tex}}%
      \onslide*<7>{\providecommand\step{11}\input{diagrams/backtrace/backtrace.tex}}%
      \onslide*<8>{\providecommand\step{12}\input{diagrams/backtrace/backtrace.tex}}%
      \onslide*<9>{\providecommand\step{13}\input{diagrams/backtrace/backtrace.tex}}%
    \end{column}
  \end{columns}
\end{frame}

\begin{frame}{Backtraces}{Printing the backtrace}
  \begin{columns}[c]
    \begin{column}{0.45\textwidth}
      \minipage[c][0.66\textheight][s]{\columnwidth}
      \begin{itemize}
        \item<1-> The exception backtrace can now be printed to \funcarg{stdout} with \funcname{Printexc.print_backtrace}
        \item<2-> First the exception backtrace is retrieved into an OCaml array with \funcname{get_raw_backtrace }
        \item<3-> Which is implemented as the \funcname{caml_get_exception_raw_backtrace} C function in the runtime
        \item<4-> And finally printed with \funcname{print_exception_backtrace}
      \end{itemize}
      \vfill
      \mintocaml[firstline=28,lastline=34,highlightlines=33]{../src/backtrace/backtrace.ml}
      \endminipage
    \end{column}
    \begin{column}{0.55\textwidth}
      \onslide*<1,2>{
        \mintocaml[firstline=100,lastline=101]{../ocaml/stdlib/printexc.ml}
      }
      \only<1>{
        \listocaml[firstline=170,lastline=172]{../ocaml/stdlib/printexc.ml}{stdlib/printexc.ml:170}
      }
      \only<2>{
        \listocaml[firstline=170,lastline=172,highlightlines=172]{../ocaml/stdlib/printexc.ml}{stdlib/printexc.ml:170}
      }
      \only<3>{
        \mintc[firstline=154,lastline=156]{../ocaml/runtime/backtrace.c}
        \listc[firstline=171,lastline=192,highlightlines={184-187}]{../ocaml/runtime/backtrace.c}{runtime/backtrace.c:154}
      }
      \only<4>{
        \listocaml[firstline=155,lastline=165]{../ocaml/stdlib/printexc.ml}{stdlib/printexc.ml:55}
      }
    \end{column}
  \end{columns}
\end{frame}


\subsection{The multiple kinds of raise}
\frameSubsection{}{}

\begin{frame}{Backtraces}{When backtraces are not needed}
  \begin{columns}[c]
    \begin{column}{0.45\textwidth}
      \minipage[c][0.66\textheight][s]{\columnwidth}
      \begin{itemize}
        \item<1-> What if the backtrace for an exception is never needed?
        \item<2-> It's possible to raise the exception explicitly with \funcname{raise\_notrace} to make raising as cheap as possible
        \item<3-> An example of use in the \funcname{dynlink} library of the OCaml compiler
      \end{itemize}
      \vfill
      \onslide<3->{
      \listocaml[firstline=205,lastline=209,highlightlines=207]{../ocaml/otherlibs/dynlink/dynlink\_compilerlibs/misc.ml}{otherlibs/dynlink/dynlink\_compilerlibs/misc.ml:205}
      }
      \endminipage
    \end{column}
    \begin{column}{0.55\textwidth}
    \only<4>{
      \mintcmm[highlightlines=3]{../src/notrace/camlNotrace\_\_fun_347.cmm}
      \listcmm{../src/notrace/camlNotrace\_\_all\_somes\_327.cmm}{all\_somes}
    }
    \onslide*<5->{
      \listobjdump[highlightlines={6-9}]{../src/notrace/camlNotrace\_\_fun_347.objdump}{anonymous function from \funcname{all\_somes}}
    }
    \end{column}
  \end{columns}
\end{frame}

\begin{frame}{Backtraces}{Summary table of the multiple kinds of raise}
  \begin{itemize}
    \item When debugging information is enabled and if the backtrace of an exception isn't needed, it's possible to raise with \funcname{raise\_notrace} for performance
    \item When debugging information is \emph{not} enabled, the compiler optimises \funcname{raise} calls to inline assembly (same effect as using \funcname{raise\_notrace})
  \end{itemize}
  \bigskip
  \centering
  \begin{tabular}{| l | c | c | c | c |}
    \hline
    \parbox{10em}{Debug information \\ (\funcarg{-g} flag)}  & \multicolumn{2}{|c|}{Disabled}                                     & \multicolumn{2}{|c|}{Enabled} \\
    \hline
    Raise kind                                               & \funcname{raise} & \funcname{raise\_notrace}                       & \funcname{raise} & \funcname{raise\_notrace} \\
    \hline
    Compilation                                              & \parbox{8em}{inline assembly\\ (optimization)} & inline assembly   & \funcname{caml\_raise\_exn} & inline assembly \\
    \hline
  \end{tabular}
\end{frame}


%
% Takeaway #5
%
\subsection*{Takeaway \#5}
\frameSubsectionTakeaway{}
\againframe<5>{takeaway}
}%
      \onslide*<8>{\providecommand\step{12}%
% Backtraces
%
\subsection{One advantage of raising with debugging information}
\frameSubsection{}{}

\begin{frame}{Backtraces}{Debugging information \& source code}
  \begin{columns}[c]
    \begin{column}{0.5\textwidth}
      \begin{itemize}
        \item Raising an exception is fun and games, but how do we know where the exception originated? $\rightarrow$ With the backtrace associated with the exception!
        \item In order to get a backtrace from the point where an exception was raised, it's necessary to compile with debugging information enabled (\funcarg{-g} flag for the compiler)
        \item Same example as in the `Nested handlers' section
      \end{itemize}
    \end{column}
    \begin{column}{0.5\textwidth}
      \listocaml[firstline=9,lastline=34]{../src/backtrace/backtrace.ml}{backtrace/backtrace.ml}
    \end{column}
  \end{columns}
\end{frame}

\begin{frame}{Backtraces}{CMM and disassembly of \funcname{definitely\_raise}}
  \begin{columns}[c]
    \begin{column}{0.5\textwidth}
      \minipage[c][0.66\textheight][s]{\columnwidth}
      \begin{itemize}
        \item<1-> An effect of compiling with debugging information (\funcarg{-g}): the CMM no longer uses \funcname{raise\_notrace} to raise the exception but \funcname{raise} instead
        \item<2-> Disassembly shows that CMM \funcname{raise} is actually a call to \funcname{caml\_raise\_exn}, a function in assembler provided by the runtime
      \end{itemize}
      \vfill
      \mintocaml[firstline=9,lastline=13,highlightlines=12]{../src/backtrace/backtrace.ml}
      \endminipage
    \end{column}
    \begin{column}{0.5\textwidth}
      \onslide*<1>{
        \mintcmm[highlightlines=5]{../src/backtrace/camlBacktrace\_\_definitely\_raise\_347.cmm}
      }
      \onslide*<2>{
        \mintobjdump[firstline=2,highlightlines=14]{../src/backtrace/camlBacktrace\_\_definitely\_raise\_347.objdump}
      }
    \end{column}
  \end{columns}
\end{frame}

\begin{frame}{Backtraces}{Introducing \funcname{caml\_raise\_exn}}
  \begin{columns}[c]
    \begin{column}{0.5\textwidth}
      \minipage[c][0.66\textheight][s]{\columnwidth}
      \onslide*<1>{
        \begin{itemize}
          \item Raising the exception is performed using \funcname{caml\_raise\_exn} which is
            \begin{itemize}
              \item An assembly function provided by the runtime
              \item Architecture specific
            \end{itemize}
          \item Let's see what it does differently from \funcname{raise\_notrace}\dots
          \item We are about to raise \typename{ExnC} with \funcname{caml\_raise\_exn} from \funcname{definitely\_raise}
        \end{itemize}
      }
      \onslide*<2>{
        \mintobjdump[firstline=2,highlightlines=14]{../src/backtrace/camlBacktrace\_\_definitely\_raise\_347.objdump}
      }
      \vfill
      \mintocaml[firstline=9,lastline=13,highlightlines=12]{../src/backtrace/backtrace.ml}
      \endminipage
    \end{column}
    \begin{column}{0.5\textwidth}
      \centering
      \providecommand\step{1}\input{diagrams/backtrace/backtrace.tex}
    \end{column}
  \end{columns}
\end{frame}

\begin{frame}{Backtraces}{But before that, we need more fields from \typename{caml\_domain\_state}}
  \begin{columns}
    \begin{column}{0.5\textwidth}
      \begin{itemize}
        \item Remember \typename{caml\_domain\_state} the per-domain runtime state?
        \item Some other members are of interest to us now\dots
        \item \localname{backtrace\_active} is used to know if the runtime should collect backtraces
        \item \localname{backtrace\_active} can be set by
          \begin{itemize}
            \item The \funcarg{b} flag in the \funcname{OCAMLRUNPARAM} environment variable
            \item \mintinline{ocaml}{Printexc.record_backtrace : bool -> unit} \footnotemark
          \end{itemize}
      \end{itemize}
    \end{column}
    \begin{column}{0.5\textwidth}
      \begin{listing}
        \mintc[highlightlines={9-11}]{src/ocaml/domain\_state\_backtrace.h}
        \caption{runtime/caml/domain\_state.h}
      \end{listing}
    \end{column}
  \end{columns}
  \footnotetext[1]{https://v2.ocaml.org/api/Printexc.html}
\end{frame}

\newcommand\listCamlRaiseExn[1][]{
  \listasm[firstline=702,lastline=725,#1]{../ocaml/runtime/amd64.S}{runtime/amd64.S:702}}

\begin{frame}{Backtraces}{Walk through \funcname{caml\_raise\_exn}}
  \begin{columns}[T]
    \begin{column}{0.5\textwidth}
      \begin{itemize}
        \item<1-> First checks whether exceptions backtrace should be recorded
        \item<1-> By checking the \funcarg{domain\_state->backtrace\_active} value
        \item<2-> If not, acts as \funcname{raise\_notrace} with \funcname{RESTORE\_EXN\_HANDLER\_OCAML}
          \begin{itemize}
            \item Set \regname{RSP} to \localname{domain\_state->exn\_handler} value
            \item Restore \localname{domain\_state->exn\_handler} from trap
            \item Jump with \funcname{ret} (same effect as using \funcname{pop} and \funcname{jmp})
          \end{itemize}
        \item<3-> Otherwise, switches to the C stack to be able to execute C code
        \item<4-> Calls \funcname{caml\_stash\_backtrace}, a C function provided by the runtime\ignorespaces
          \onslide<6->{, with
            \begin{itemize}
              \item \funcarg{exn}: exception value
              \item \funcarg{pc}: code address of the raising point
              \item \funcarg{sp}: stack address of the raising function
              \item \funcarg{trapsp}: stack address of the next handler
            \end{itemize}
          }
      \end{itemize}
      \bigskip
      \mintocaml[firstline=9,lastline=13,highlightlines=12]{../src/backtrace/backtrace.ml}
    \end{column}
    \begin{column}{0.5\textwidth}
      \raggedleft
      \onslide*<1,3-4>{
        \mintc[firstline=190,lastline=192]{../ocaml/runtime/amd64.S}
        \mintc[firstline=234,lastline=237]{../ocaml/runtime/amd64.S}
      }
      \onslide*<2>{
        \mintc[firstline=190,lastline=192,highlightlines={191-192}]{../ocaml/runtime/amd64.S}
        \mintc[firstline=234,lastline=237,highlightlines={235-237}]{../ocaml/runtime/amd64.S}
      }
      \onslide*<1>{\listCamlRaiseExn[highlightlines=705]{}}%
      \onslide*<2>{\listCamlRaiseExn[highlightlines={707-708}]{}}%
      \onslide*<3>{\listCamlRaiseExn[highlightlines={712-714}]{}}%
      \onslide*<4>{\listCamlRaiseExn[highlightlines={715-720}]{}}%
      \onslide*<5>{\providecommand\step{1}\input{diagrams/backtrace/backtrace.tex}}%
      \onslide*<6>{\providecommand\step{2}\input{diagrams/backtrace/backtrace.tex}}%
    \end{column}
  \end{columns}
\end{frame}

\newcommand\listStashBacktrace[1][]{
  \listc[firstline=91,lastline=120,#1]{../ocaml/runtime/backtrace\_nat.c}{runtime/backtrace\_nat.c:91}}

\begin{frame}{Backtraces}{Introducing \funcname{caml\_stash\_backtrace}}
  \begin{columns}[c]
    \begin{column}{0.5\textwidth}
      \minipage[c][0.66\textheight][s]{\columnwidth}
      \begin{itemize}
        \item<1-> A C function provided by the runtime (specific to native code)
        \item<2-> It scans the stack, between the stack pointer at the raising point up to the next trap, in order to collect the names of the detected function frames
          \onslide*<2>{
            \begin{itemize}
              \item And more information, like line number, column number...
            \end{itemize}
          }
        \item<3-> It uses the same technique and information as the Garbage Collector (GC) uses to scan the values live on the stack
          \onslide*<4>{
            \begin{itemize}
              \item \typename{frame\_descr} are at the core of stack unwinding
            \end{itemize}
          }
      \end{itemize}
      \vfill
      \mintocaml[firstline=9,lastline=13,highlightlines=12]{../src/backtrace/backtrace.ml}
      \endminipage
    \end{column}
    \begin{column}{0.5\textwidth}
      \onslide*<1>{\listStashBacktrace[highlightlines=91]{}}
      \onslide*<2>{\listStashBacktrace[highlightlines={91,117-118}]{}}
      \onslide*<3->{\listStashBacktrace[highlightlines={94,106,109-110}]{}}
    \end{column}
  \end{columns}
\end{frame}

\newcommand\listFrameDescr[1][]{
  \listocaml[firstline=111,lastline=124,#1]{../ocaml/asmcomp/emitaux.ml}{asmcomp/emitaux.ml:111}}
\newcommand\listDebugInfo[1][]{
  \listocaml[firstline=38,lastline=49,#1]{../ocaml/lambda/debuginfo.mli}{lambda/debuginfo.mli:38}}

\begin{frame}{Backtraces}{\typename{frame\_descr} and \typename{frame\_debuginfo} in a nutshell}
  \begin{columns}[c]
    \begin{column}{0.5\textwidth}
      \begin{itemize}
        \item<1-> For every return address in an OCaml program, there is a associated \typename{frame\_descr}
        \item<2-> A \typename{frame\_descr} specifies
        \begin{itemize}
          \item<2-> The size of the function stack frame
          \item<3-> Where live values are on the stack (so that the GC can mark them as live and prevent from being swept)
        \end{itemize}
        \item<4-> When compiled with debug information, \typename{frame\_descr} will be linked to a \typename{frame\_debuginfo}
        \begin{itemize}
          \item<5-> Contains information about the position in the source code (file name, line and column number)
        \end{itemize}
      \end{itemize}
    \end{column}
    \begin{column}{0.5\textwidth}
        \onslide*<1>{\listFrameDescr[highlightlines={118-122}]{}}
        \onslide*<2>{\listFrameDescr[highlightlines={120}]{}}
        \onslide*<3>{\listFrameDescr[highlightlines={121}]{}}
        \onslide*<4->{\listFrameDescr[highlightlines={122}]{}}
        \medskip
        \onslide*<1-3>{\listDebugInfo{}}
        \onslide*<4>{\listDebugInfo[highlightlines={38,49}]{}}
        \onslide*<5>{\listDebugInfo[highlightlines={39-46}]{}}
    \end{column}
  \end{columns}
\end{frame}

\begin{frame}{Backtraces}{Scanning the stack with \typename{frame\_descr}}
  \begin{columns}[c]
    \begin{column}{0.5\textwidth}
      \begin{itemize}
        \item Given the return address of the code that raises the exception by calling \funcname{caml\_raise\_exn} (\funcarg{pc} argument of \funcname{caml\_stash\_backtrace})
        \item And the position of the stack pointer (\funcarg{sp} argument of \funcname{caml\_stash\_backtrace})
        \item The frame size given by \typename{frame\_descr}.\funcarg{frame\_size}, allows to find the end of the previous frame
        \item And the return address of the caller of this function
        \bigskip
        \onslide<3->{
        \item Note that traps (here the reddish \funcname{ExnA}, \funcname{ExnB} and \funcname{ExnC} blocks) belong to the frame that installed them, and so the frame size changes when traps are removed
        }
      \end{itemize}
    \end{column}
    \begin{column}{0.5\textwidth}
      \raggedleft
      \onslide*<1>{\providecommand\step{2}\input{diagrams/backtrace/backtrace.tex}}%
      \onslide*<2>{\providecommand\step{1}\input{diagrams/backtrace/scanstack.tex}}%
      \onslide*<3>{\providecommand\step{2}\input{diagrams/backtrace/scanstack.tex}}%
      \onslide*<4>{\providecommand\step{3}\input{diagrams/backtrace/scanstack.tex}}%
      \onslide*<5>{\providecommand\step{4}\input{diagrams/backtrace/scanstack.tex}}%
      \onslide*<6>{\providecommand\step{5}\input{diagrams/backtrace/scanstack.tex}}%
    \end{column}
  \end{columns}
\end{frame}

% Duplicate of previous-previous slide
\begin{frame}{Backtraces}{Continuing on \funcname{caml\_stash\_backtrace}}
  \begin{columns}[c]
    \begin{column}{0.5\textwidth}
      \minipage[c][0.75\textheight][s]{\columnwidth}
      \begin{itemize}
          \color{gray}
        \item A C function provided by the runtime (specific to native code)
        \item It scans the stack, between the stack pointer at the raising point up to the next trap, in order to collect the names of the detected function frames
        \item It uses the same technique and information as the Garbage Collector (GC) uses to scan the values live on the stack
      \end{itemize}
      \begin{itemize}
        \item For each function frame detected, store a pointer to the \typename{frame\_descr} in the \localname{domain\_state->backtrace\_buffer} array, effectively constructing the backtrace
      \end{itemize}
      \onslide<2->{
        \begin{itemize}
            \smallskip
          \item Constructed backtrace:
            \funcname{definitely\_raise},
            \onslide<3->{\funcname{probably\_raise}}
        \end{itemize}
      }
      \vfill
      \mintocaml[firstline=9,lastline=13,highlightlines=12]{../src/backtrace/backtrace.ml}
      \endminipage
    \end{column}
    \begin{column}{0.5\textwidth}
      \raggedleft
      \onslide*<1>{\listStashBacktrace[highlightlines={114-115}]{}}
      \onslide*<2>{\providecommand\step{3}\input{diagrams/backtrace/backtrace.tex}}%
      \onslide*<3>{\providecommand\step{4}\input{diagrams/backtrace/backtrace.tex}}%
    \end{column}
  \end{columns}
\end{frame}

\begin{frame}{Backtraces}{Back to \funcname{caml\_raise\_exn}}
  \begin{columns}[c]
    \begin{column}{0.5\textwidth}
      \minipage[c][0.66\textheight][s]{\columnwidth}
      \begin{itemize}\color{gray}
        \item Checks wheteher exceptions backtrace should be recorded, by checking the \funcarg{domain\_state->backtrace\_active} value
        \item Switches to the C stack to be able to execute C code
        \item Calls \funcname{caml\_stash\_backtrace}, to collect the backtrace up to the next trap
      \end{itemize}
      \onslide*<2->{
        \begin{itemize}
          \item<2-> Executes the exception handler in \funcname{probably\_raise} with \funcname{RESTORE\_EXN\_HANDLER\_OCAML}
            \begin{itemize}
              \item Set \regname{RSP} to \localname{domain\_state->exn\_handler} value
              \item Restore \localname{domain\_state->exn\_handler} from trap
              \item Jump to the handler code with \funcname{ret}
            \end{itemize}
        \end{itemize}
      }
      \vfill
      \onslide*<1>{\mintocaml[firstline=9,lastline=13,highlightlines=12]{../src/backtrace/backtrace.ml}}
      \onslide*<2->{\mintocaml[firstline=15,lastline=21,highlightlines={19-21}]{../src/backtrace/backtrace.ml}}
      \endminipage
    \end{column}
    \begin{column}{0.5\textwidth}
      \centering
      \onslide*<1>{
        \mintc[firstline=190,lastline=192]{../ocaml/runtime/amd64.S}
        \mintc[firstline=234,lastline=237]{../ocaml/runtime/amd64.S}
      }
      \onslide*<2>{
        \mintc[firstline=190,lastline=192,highlightlines={191-192}]{../ocaml/runtime/amd64.S}
        \mintc[firstline=234,lastline=237,highlightlines={235-237}]{../ocaml/runtime/amd64.S}
      }
      \onslide*<1>{\listCamlRaiseExn[highlightlines=720]{}}
      \onslide*<2>{\listCamlRaiseExn[highlightlines=722]{}}
      \onslide*<3->{\providecommand\step{5}\input{diagrams/backtrace/backtrace.tex}}
    \end{column}
  \end{columns}
\end{frame}

\begin{frame}{Backtraces}{\funcname{caml\_probably\_raise}'s handler doesn't catch \typename{ExnC}}
  \begin{columns}[c]
    \begin{column}{0.45\textwidth}
      \minipage[c][0.75\textheight][s]{\columnwidth}
      \begin{itemize}
        \item<1-> \funcname{match \dots with}\footnotemark pattern matching can also match exceptions
        \item<1-> The CMM of \funcname{probably\_raise} shows that this \funcname{match \dots with} is implemented with a pattern matching and a \funcname{try \dots with}
        \item<1-> But this one only matches on \typename{ExnA} exception
        \item<2-> Since the pattern matching fails to catch the exception, \funcname{reraise} is called with our current \typename{ExnC} exception
        \item<3-> Disassembly shows that \funcname{reraise} is actually a call to \funcname{caml\_reraise\_exn}, which is also an assembly function provided by the runtime
      \end{itemize}
      \vfill
      \mintocaml[firstline=15,lastline=21,highlightlines={18-21}]{../src/backtrace/backtrace.ml}
      \endminipage
    \end{column}
    \begin{column}{0.55\textwidth}
      \centering
      \onslide*<1>{\mintcmm[highlightlines={10-18}]{../src/backtrace/camlBacktrace\_\_probably\_raise\_351.cmm}}
      \onslide*<2>{\listcmm[highlightlines=18]{../src/backtrace/camlBacktrace\_\_probably\_raise_351.cmm}{CMM of probably\_raise}}
      \onslide*<3>{\mintobjdump[firstline=5,lastline=35,highlightlines=31]{../src/backtrace/camlBacktrace\_\_probably\_raise_351.objdump}}
    \end{column}
  \end{columns}
  \only<1>{\footnotetext[1]{https://blog.janestreet.com/pattern-matching-and-exception-handling-unite/}}
\end{frame}

\newcommand\listCamlReraiseExn[1][]{
  \listasm[firstline=727,lastline=734,#1]{../ocaml/runtime/amd64.S}{runtime/amd64.S:727}}

\begin{frame}{Backtraces}{Introducing \funcname{caml\_reraise\_exn}}
  \begin{columns}[c]
    \begin{column}{0.5\textwidth}
      \minipage[c][0.66\textheight][s]{\columnwidth}
      \begin{itemize}
        \item<1-> The exception have to be reraised to the next handler
        \item<2-> First, checks if the backtrace should be collected with \funcarg{domain\_state->backtrace\_active}
        \only<3>{
        \begin{itemize}
            \item If not, behaves like a \funcname{raise\_notrace} with \funcname{RESTORE\_EXN\_HANDLER\_OCAML}
        \end{itemize}
        }
        \item<4-> Jumps into \funcname{caml\_raise\_exn}
        \item<5-> Calls \funcname{caml\_stash\_backtrace} again, to scan the next part of the stack: from the trap just pop-ed to the next trap
      \end{itemize}
      \vfill
      \mintocaml[firstline=15,lastline=21,highlightlines={18-21}]{../src/backtrace/backtrace.ml}
      \endminipage
    \end{column}
    \begin{column}{0.5\textwidth}
      \raggedleft
      \onslide*<1>{\listCamlReraiseExn{}}
      \onslide*<2>{\listCamlReraiseExn[highlightlines=729]{}}
      \onslide*<3>{
        \mintc[firstline=190,lastline=192,highlightlines={191-192}]{../ocaml/runtime/amd64.S}
        \mintc[firstline=234,lastline=237,highlightlines={235-237}]{../ocaml/runtime/amd64.S}
        \medskip
        \listCamlReraiseExn[highlightlines=731]{}
      }
      \onslide*<4-5>{
        % TODO refactor with \listCamlReraiseExn
        \mintasm[firstline=727,lastline=734,highlightlines=730]{../ocaml/runtime/amd64.S}
        \medskip
      }
      \onslide*<4>{\listCamlRaiseExn[highlightlines=711]{}}
      \onslide*<5>{\listCamlRaiseExn[highlightlines=720]{}}
    \end{column}
  \end{columns}
\end{frame}

\begin{frame}{Backtraces}{Backtrace collection continues}
  \begin{columns}[c]
    \begin{column}{0.55\textwidth}
      \minipage[c][0.75\textheight][s]{\columnwidth}
      \begin{itemize}
        \item<1-> \funcname{caml\_stash\_backtrace} scans the older part of the stack, up to the next trap
        \item<6-> Controls jumps to the nested exception handler in \funcname{maybe\_raise}, which matches only on \typename{ExnB}
        \item<7-> \funcname{caml\_reraise\_exn} is called again, backtrace collection resumes with \funcname{caml\_stash\_backtrace}
      \end{itemize}
      \medskip
      \begin{itemize}
        \item<2-> Constructed backtrace:
        \begin{itemize}
          \item <2-> \funcname{definitely\_raise}
          \item <2-> \funcname{probably\_raise}
          \item <3-> \funcname{probably\_raise} (re-raise)
          \item <4-> \funcname{maybe\_raise}
          \item <5-> \funcname{main}
          \item <8-> \funcname{main} (re-raise)
        \end{itemize}
      \end{itemize}
      \vfill
      \onslide*<1-5>{\mintocaml[firstline=15,lastline=21]{../src/backtrace/backtrace.ml}}
      \onslide*<6>{\mintocaml[firstline=28,lastline=34,highlightlines=31]{../src/backtrace/backtrace.ml}}
      \onslide*<7->{\mintocaml[firstline=28,lastline=34,highlightlines=33]{../src/backtrace/backtrace.ml}}
      \endminipage
    \end{column}
    \begin{column}{0.45\textwidth}
      \raggedleft
      \onslide*<1>{\providecommand\step{6}\input{diagrams/backtrace/backtrace.tex}}%
      \onslide*<2>{\providecommand\step{6}\input{diagrams/backtrace/backtrace.tex}}%
      \onslide*<3>{\providecommand\step{7}\input{diagrams/backtrace/backtrace.tex}}%
      \onslide*<4>{\providecommand\step{8}\input{diagrams/backtrace/backtrace.tex}}%
      \onslide*<5>{\providecommand\step{9}\input{diagrams/backtrace/backtrace.tex}}%
      \onslide*<6>{\providecommand\step{10}\input{diagrams/backtrace/backtrace.tex}}%
      \onslide*<7>{\providecommand\step{11}\input{diagrams/backtrace/backtrace.tex}}%
      \onslide*<8>{\providecommand\step{12}\input{diagrams/backtrace/backtrace.tex}}%
      \onslide*<9>{\providecommand\step{13}\input{diagrams/backtrace/backtrace.tex}}%
    \end{column}
  \end{columns}
\end{frame}

\begin{frame}{Backtraces}{Printing the backtrace}
  \begin{columns}[c]
    \begin{column}{0.45\textwidth}
      \minipage[c][0.66\textheight][s]{\columnwidth}
      \begin{itemize}
        \item<1-> The exception backtrace can now be printed to \funcarg{stdout} with \funcname{Printexc.print_backtrace}
        \item<2-> First the exception backtrace is retrieved into an OCaml array with \funcname{get_raw_backtrace }
        \item<3-> Which is implemented as the \funcname{caml_get_exception_raw_backtrace} C function in the runtime
        \item<4-> And finally printed with \funcname{print_exception_backtrace}
      \end{itemize}
      \vfill
      \mintocaml[firstline=28,lastline=34,highlightlines=33]{../src/backtrace/backtrace.ml}
      \endminipage
    \end{column}
    \begin{column}{0.55\textwidth}
      \onslide*<1,2>{
        \mintocaml[firstline=100,lastline=101]{../ocaml/stdlib/printexc.ml}
      }
      \only<1>{
        \listocaml[firstline=170,lastline=172]{../ocaml/stdlib/printexc.ml}{stdlib/printexc.ml:170}
      }
      \only<2>{
        \listocaml[firstline=170,lastline=172,highlightlines=172]{../ocaml/stdlib/printexc.ml}{stdlib/printexc.ml:170}
      }
      \only<3>{
        \mintc[firstline=154,lastline=156]{../ocaml/runtime/backtrace.c}
        \listc[firstline=171,lastline=192,highlightlines={184-187}]{../ocaml/runtime/backtrace.c}{runtime/backtrace.c:154}
      }
      \only<4>{
        \listocaml[firstline=155,lastline=165]{../ocaml/stdlib/printexc.ml}{stdlib/printexc.ml:55}
      }
    \end{column}
  \end{columns}
\end{frame}


\subsection{The multiple kinds of raise}
\frameSubsection{}{}

\begin{frame}{Backtraces}{When backtraces are not needed}
  \begin{columns}[c]
    \begin{column}{0.45\textwidth}
      \minipage[c][0.66\textheight][s]{\columnwidth}
      \begin{itemize}
        \item<1-> What if the backtrace for an exception is never needed?
        \item<2-> It's possible to raise the exception explicitly with \funcname{raise\_notrace} to make raising as cheap as possible
        \item<3-> An example of use in the \funcname{dynlink} library of the OCaml compiler
      \end{itemize}
      \vfill
      \onslide<3->{
      \listocaml[firstline=205,lastline=209,highlightlines=207]{../ocaml/otherlibs/dynlink/dynlink\_compilerlibs/misc.ml}{otherlibs/dynlink/dynlink\_compilerlibs/misc.ml:205}
      }
      \endminipage
    \end{column}
    \begin{column}{0.55\textwidth}
    \only<4>{
      \mintcmm[highlightlines=3]{../src/notrace/camlNotrace\_\_fun_347.cmm}
      \listcmm{../src/notrace/camlNotrace\_\_all\_somes\_327.cmm}{all\_somes}
    }
    \onslide*<5->{
      \listobjdump[highlightlines={6-9}]{../src/notrace/camlNotrace\_\_fun_347.objdump}{anonymous function from \funcname{all\_somes}}
    }
    \end{column}
  \end{columns}
\end{frame}

\begin{frame}{Backtraces}{Summary table of the multiple kinds of raise}
  \begin{itemize}
    \item When debugging information is enabled and if the backtrace of an exception isn't needed, it's possible to raise with \funcname{raise\_notrace} for performance
    \item When debugging information is \emph{not} enabled, the compiler optimises \funcname{raise} calls to inline assembly (same effect as using \funcname{raise\_notrace})
  \end{itemize}
  \bigskip
  \centering
  \begin{tabular}{| l | c | c | c | c |}
    \hline
    \parbox{10em}{Debug information \\ (\funcarg{-g} flag)}  & \multicolumn{2}{|c|}{Disabled}                                     & \multicolumn{2}{|c|}{Enabled} \\
    \hline
    Raise kind                                               & \funcname{raise} & \funcname{raise\_notrace}                       & \funcname{raise} & \funcname{raise\_notrace} \\
    \hline
    Compilation                                              & \parbox{8em}{inline assembly\\ (optimization)} & inline assembly   & \funcname{caml\_raise\_exn} & inline assembly \\
    \hline
  \end{tabular}
\end{frame}


%
% Takeaway #5
%
\subsection*{Takeaway \#5}
\frameSubsectionTakeaway{}
\againframe<5>{takeaway}
}%
      \onslide*<9>{\providecommand\step{13}%
% Backtraces
%
\subsection{One advantage of raising with debugging information}
\frameSubsection{}{}

\begin{frame}{Backtraces}{Debugging information \& source code}
  \begin{columns}[c]
    \begin{column}{0.5\textwidth}
      \begin{itemize}
        \item Raising an exception is fun and games, but how do we know where the exception originated? $\rightarrow$ With the backtrace associated with the exception!
        \item In order to get a backtrace from the point where an exception was raised, it's necessary to compile with debugging information enabled (\funcarg{-g} flag for the compiler)
        \item Same example as in the `Nested handlers' section
      \end{itemize}
    \end{column}
    \begin{column}{0.5\textwidth}
      \listocaml[firstline=9,lastline=34]{../src/backtrace/backtrace.ml}{backtrace/backtrace.ml}
    \end{column}
  \end{columns}
\end{frame}

\begin{frame}{Backtraces}{CMM and disassembly of \funcname{definitely\_raise}}
  \begin{columns}[c]
    \begin{column}{0.5\textwidth}
      \minipage[c][0.66\textheight][s]{\columnwidth}
      \begin{itemize}
        \item<1-> An effect of compiling with debugging information (\funcarg{-g}): the CMM no longer uses \funcname{raise\_notrace} to raise the exception but \funcname{raise} instead
        \item<2-> Disassembly shows that CMM \funcname{raise} is actually a call to \funcname{caml\_raise\_exn}, a function in assembler provided by the runtime
      \end{itemize}
      \vfill
      \mintocaml[firstline=9,lastline=13,highlightlines=12]{../src/backtrace/backtrace.ml}
      \endminipage
    \end{column}
    \begin{column}{0.5\textwidth}
      \onslide*<1>{
        \mintcmm[highlightlines=5]{../src/backtrace/camlBacktrace\_\_definitely\_raise\_347.cmm}
      }
      \onslide*<2>{
        \mintobjdump[firstline=2,highlightlines=14]{../src/backtrace/camlBacktrace\_\_definitely\_raise\_347.objdump}
      }
    \end{column}
  \end{columns}
\end{frame}

\begin{frame}{Backtraces}{Introducing \funcname{caml\_raise\_exn}}
  \begin{columns}[c]
    \begin{column}{0.5\textwidth}
      \minipage[c][0.66\textheight][s]{\columnwidth}
      \onslide*<1>{
        \begin{itemize}
          \item Raising the exception is performed using \funcname{caml\_raise\_exn} which is
            \begin{itemize}
              \item An assembly function provided by the runtime
              \item Architecture specific
            \end{itemize}
          \item Let's see what it does differently from \funcname{raise\_notrace}\dots
          \item We are about to raise \typename{ExnC} with \funcname{caml\_raise\_exn} from \funcname{definitely\_raise}
        \end{itemize}
      }
      \onslide*<2>{
        \mintobjdump[firstline=2,highlightlines=14]{../src/backtrace/camlBacktrace\_\_definitely\_raise\_347.objdump}
      }
      \vfill
      \mintocaml[firstline=9,lastline=13,highlightlines=12]{../src/backtrace/backtrace.ml}
      \endminipage
    \end{column}
    \begin{column}{0.5\textwidth}
      \centering
      \providecommand\step{1}\input{diagrams/backtrace/backtrace.tex}
    \end{column}
  \end{columns}
\end{frame}

\begin{frame}{Backtraces}{But before that, we need more fields from \typename{caml\_domain\_state}}
  \begin{columns}
    \begin{column}{0.5\textwidth}
      \begin{itemize}
        \item Remember \typename{caml\_domain\_state} the per-domain runtime state?
        \item Some other members are of interest to us now\dots
        \item \localname{backtrace\_active} is used to know if the runtime should collect backtraces
        \item \localname{backtrace\_active} can be set by
          \begin{itemize}
            \item The \funcarg{b} flag in the \funcname{OCAMLRUNPARAM} environment variable
            \item \mintinline{ocaml}{Printexc.record_backtrace : bool -> unit} \footnotemark
          \end{itemize}
      \end{itemize}
    \end{column}
    \begin{column}{0.5\textwidth}
      \begin{listing}
        \mintc[highlightlines={9-11}]{src/ocaml/domain\_state\_backtrace.h}
        \caption{runtime/caml/domain\_state.h}
      \end{listing}
    \end{column}
  \end{columns}
  \footnotetext[1]{https://v2.ocaml.org/api/Printexc.html}
\end{frame}

\newcommand\listCamlRaiseExn[1][]{
  \listasm[firstline=702,lastline=725,#1]{../ocaml/runtime/amd64.S}{runtime/amd64.S:702}}

\begin{frame}{Backtraces}{Walk through \funcname{caml\_raise\_exn}}
  \begin{columns}[T]
    \begin{column}{0.5\textwidth}
      \begin{itemize}
        \item<1-> First checks whether exceptions backtrace should be recorded
        \item<1-> By checking the \funcarg{domain\_state->backtrace\_active} value
        \item<2-> If not, acts as \funcname{raise\_notrace} with \funcname{RESTORE\_EXN\_HANDLER\_OCAML}
          \begin{itemize}
            \item Set \regname{RSP} to \localname{domain\_state->exn\_handler} value
            \item Restore \localname{domain\_state->exn\_handler} from trap
            \item Jump with \funcname{ret} (same effect as using \funcname{pop} and \funcname{jmp})
          \end{itemize}
        \item<3-> Otherwise, switches to the C stack to be able to execute C code
        \item<4-> Calls \funcname{caml\_stash\_backtrace}, a C function provided by the runtime\ignorespaces
          \onslide<6->{, with
            \begin{itemize}
              \item \funcarg{exn}: exception value
              \item \funcarg{pc}: code address of the raising point
              \item \funcarg{sp}: stack address of the raising function
              \item \funcarg{trapsp}: stack address of the next handler
            \end{itemize}
          }
      \end{itemize}
      \bigskip
      \mintocaml[firstline=9,lastline=13,highlightlines=12]{../src/backtrace/backtrace.ml}
    \end{column}
    \begin{column}{0.5\textwidth}
      \raggedleft
      \onslide*<1,3-4>{
        \mintc[firstline=190,lastline=192]{../ocaml/runtime/amd64.S}
        \mintc[firstline=234,lastline=237]{../ocaml/runtime/amd64.S}
      }
      \onslide*<2>{
        \mintc[firstline=190,lastline=192,highlightlines={191-192}]{../ocaml/runtime/amd64.S}
        \mintc[firstline=234,lastline=237,highlightlines={235-237}]{../ocaml/runtime/amd64.S}
      }
      \onslide*<1>{\listCamlRaiseExn[highlightlines=705]{}}%
      \onslide*<2>{\listCamlRaiseExn[highlightlines={707-708}]{}}%
      \onslide*<3>{\listCamlRaiseExn[highlightlines={712-714}]{}}%
      \onslide*<4>{\listCamlRaiseExn[highlightlines={715-720}]{}}%
      \onslide*<5>{\providecommand\step{1}\input{diagrams/backtrace/backtrace.tex}}%
      \onslide*<6>{\providecommand\step{2}\input{diagrams/backtrace/backtrace.tex}}%
    \end{column}
  \end{columns}
\end{frame}

\newcommand\listStashBacktrace[1][]{
  \listc[firstline=91,lastline=120,#1]{../ocaml/runtime/backtrace\_nat.c}{runtime/backtrace\_nat.c:91}}

\begin{frame}{Backtraces}{Introducing \funcname{caml\_stash\_backtrace}}
  \begin{columns}[c]
    \begin{column}{0.5\textwidth}
      \minipage[c][0.66\textheight][s]{\columnwidth}
      \begin{itemize}
        \item<1-> A C function provided by the runtime (specific to native code)
        \item<2-> It scans the stack, between the stack pointer at the raising point up to the next trap, in order to collect the names of the detected function frames
          \onslide*<2>{
            \begin{itemize}
              \item And more information, like line number, column number...
            \end{itemize}
          }
        \item<3-> It uses the same technique and information as the Garbage Collector (GC) uses to scan the values live on the stack
          \onslide*<4>{
            \begin{itemize}
              \item \typename{frame\_descr} are at the core of stack unwinding
            \end{itemize}
          }
      \end{itemize}
      \vfill
      \mintocaml[firstline=9,lastline=13,highlightlines=12]{../src/backtrace/backtrace.ml}
      \endminipage
    \end{column}
    \begin{column}{0.5\textwidth}
      \onslide*<1>{\listStashBacktrace[highlightlines=91]{}}
      \onslide*<2>{\listStashBacktrace[highlightlines={91,117-118}]{}}
      \onslide*<3->{\listStashBacktrace[highlightlines={94,106,109-110}]{}}
    \end{column}
  \end{columns}
\end{frame}

\newcommand\listFrameDescr[1][]{
  \listocaml[firstline=111,lastline=124,#1]{../ocaml/asmcomp/emitaux.ml}{asmcomp/emitaux.ml:111}}
\newcommand\listDebugInfo[1][]{
  \listocaml[firstline=38,lastline=49,#1]{../ocaml/lambda/debuginfo.mli}{lambda/debuginfo.mli:38}}

\begin{frame}{Backtraces}{\typename{frame\_descr} and \typename{frame\_debuginfo} in a nutshell}
  \begin{columns}[c]
    \begin{column}{0.5\textwidth}
      \begin{itemize}
        \item<1-> For every return address in an OCaml program, there is a associated \typename{frame\_descr}
        \item<2-> A \typename{frame\_descr} specifies
        \begin{itemize}
          \item<2-> The size of the function stack frame
          \item<3-> Where live values are on the stack (so that the GC can mark them as live and prevent from being swept)
        \end{itemize}
        \item<4-> When compiled with debug information, \typename{frame\_descr} will be linked to a \typename{frame\_debuginfo}
        \begin{itemize}
          \item<5-> Contains information about the position in the source code (file name, line and column number)
        \end{itemize}
      \end{itemize}
    \end{column}
    \begin{column}{0.5\textwidth}
        \onslide*<1>{\listFrameDescr[highlightlines={118-122}]{}}
        \onslide*<2>{\listFrameDescr[highlightlines={120}]{}}
        \onslide*<3>{\listFrameDescr[highlightlines={121}]{}}
        \onslide*<4->{\listFrameDescr[highlightlines={122}]{}}
        \medskip
        \onslide*<1-3>{\listDebugInfo{}}
        \onslide*<4>{\listDebugInfo[highlightlines={38,49}]{}}
        \onslide*<5>{\listDebugInfo[highlightlines={39-46}]{}}
    \end{column}
  \end{columns}
\end{frame}

\begin{frame}{Backtraces}{Scanning the stack with \typename{frame\_descr}}
  \begin{columns}[c]
    \begin{column}{0.5\textwidth}
      \begin{itemize}
        \item Given the return address of the code that raises the exception by calling \funcname{caml\_raise\_exn} (\funcarg{pc} argument of \funcname{caml\_stash\_backtrace})
        \item And the position of the stack pointer (\funcarg{sp} argument of \funcname{caml\_stash\_backtrace})
        \item The frame size given by \typename{frame\_descr}.\funcarg{frame\_size}, allows to find the end of the previous frame
        \item And the return address of the caller of this function
        \bigskip
        \onslide<3->{
        \item Note that traps (here the reddish \funcname{ExnA}, \funcname{ExnB} and \funcname{ExnC} blocks) belong to the frame that installed them, and so the frame size changes when traps are removed
        }
      \end{itemize}
    \end{column}
    \begin{column}{0.5\textwidth}
      \raggedleft
      \onslide*<1>{\providecommand\step{2}\input{diagrams/backtrace/backtrace.tex}}%
      \onslide*<2>{\providecommand\step{1}\input{diagrams/backtrace/scanstack.tex}}%
      \onslide*<3>{\providecommand\step{2}\input{diagrams/backtrace/scanstack.tex}}%
      \onslide*<4>{\providecommand\step{3}\input{diagrams/backtrace/scanstack.tex}}%
      \onslide*<5>{\providecommand\step{4}\input{diagrams/backtrace/scanstack.tex}}%
      \onslide*<6>{\providecommand\step{5}\input{diagrams/backtrace/scanstack.tex}}%
    \end{column}
  \end{columns}
\end{frame}

% Duplicate of previous-previous slide
\begin{frame}{Backtraces}{Continuing on \funcname{caml\_stash\_backtrace}}
  \begin{columns}[c]
    \begin{column}{0.5\textwidth}
      \minipage[c][0.75\textheight][s]{\columnwidth}
      \begin{itemize}
          \color{gray}
        \item A C function provided by the runtime (specific to native code)
        \item It scans the stack, between the stack pointer at the raising point up to the next trap, in order to collect the names of the detected function frames
        \item It uses the same technique and information as the Garbage Collector (GC) uses to scan the values live on the stack
      \end{itemize}
      \begin{itemize}
        \item For each function frame detected, store a pointer to the \typename{frame\_descr} in the \localname{domain\_state->backtrace\_buffer} array, effectively constructing the backtrace
      \end{itemize}
      \onslide<2->{
        \begin{itemize}
            \smallskip
          \item Constructed backtrace:
            \funcname{definitely\_raise},
            \onslide<3->{\funcname{probably\_raise}}
        \end{itemize}
      }
      \vfill
      \mintocaml[firstline=9,lastline=13,highlightlines=12]{../src/backtrace/backtrace.ml}
      \endminipage
    \end{column}
    \begin{column}{0.5\textwidth}
      \raggedleft
      \onslide*<1>{\listStashBacktrace[highlightlines={114-115}]{}}
      \onslide*<2>{\providecommand\step{3}\input{diagrams/backtrace/backtrace.tex}}%
      \onslide*<3>{\providecommand\step{4}\input{diagrams/backtrace/backtrace.tex}}%
    \end{column}
  \end{columns}
\end{frame}

\begin{frame}{Backtraces}{Back to \funcname{caml\_raise\_exn}}
  \begin{columns}[c]
    \begin{column}{0.5\textwidth}
      \minipage[c][0.66\textheight][s]{\columnwidth}
      \begin{itemize}\color{gray}
        \item Checks wheteher exceptions backtrace should be recorded, by checking the \funcarg{domain\_state->backtrace\_active} value
        \item Switches to the C stack to be able to execute C code
        \item Calls \funcname{caml\_stash\_backtrace}, to collect the backtrace up to the next trap
      \end{itemize}
      \onslide*<2->{
        \begin{itemize}
          \item<2-> Executes the exception handler in \funcname{probably\_raise} with \funcname{RESTORE\_EXN\_HANDLER\_OCAML}
            \begin{itemize}
              \item Set \regname{RSP} to \localname{domain\_state->exn\_handler} value
              \item Restore \localname{domain\_state->exn\_handler} from trap
              \item Jump to the handler code with \funcname{ret}
            \end{itemize}
        \end{itemize}
      }
      \vfill
      \onslide*<1>{\mintocaml[firstline=9,lastline=13,highlightlines=12]{../src/backtrace/backtrace.ml}}
      \onslide*<2->{\mintocaml[firstline=15,lastline=21,highlightlines={19-21}]{../src/backtrace/backtrace.ml}}
      \endminipage
    \end{column}
    \begin{column}{0.5\textwidth}
      \centering
      \onslide*<1>{
        \mintc[firstline=190,lastline=192]{../ocaml/runtime/amd64.S}
        \mintc[firstline=234,lastline=237]{../ocaml/runtime/amd64.S}
      }
      \onslide*<2>{
        \mintc[firstline=190,lastline=192,highlightlines={191-192}]{../ocaml/runtime/amd64.S}
        \mintc[firstline=234,lastline=237,highlightlines={235-237}]{../ocaml/runtime/amd64.S}
      }
      \onslide*<1>{\listCamlRaiseExn[highlightlines=720]{}}
      \onslide*<2>{\listCamlRaiseExn[highlightlines=722]{}}
      \onslide*<3->{\providecommand\step{5}\input{diagrams/backtrace/backtrace.tex}}
    \end{column}
  \end{columns}
\end{frame}

\begin{frame}{Backtraces}{\funcname{caml\_probably\_raise}'s handler doesn't catch \typename{ExnC}}
  \begin{columns}[c]
    \begin{column}{0.45\textwidth}
      \minipage[c][0.75\textheight][s]{\columnwidth}
      \begin{itemize}
        \item<1-> \funcname{match \dots with}\footnotemark pattern matching can also match exceptions
        \item<1-> The CMM of \funcname{probably\_raise} shows that this \funcname{match \dots with} is implemented with a pattern matching and a \funcname{try \dots with}
        \item<1-> But this one only matches on \typename{ExnA} exception
        \item<2-> Since the pattern matching fails to catch the exception, \funcname{reraise} is called with our current \typename{ExnC} exception
        \item<3-> Disassembly shows that \funcname{reraise} is actually a call to \funcname{caml\_reraise\_exn}, which is also an assembly function provided by the runtime
      \end{itemize}
      \vfill
      \mintocaml[firstline=15,lastline=21,highlightlines={18-21}]{../src/backtrace/backtrace.ml}
      \endminipage
    \end{column}
    \begin{column}{0.55\textwidth}
      \centering
      \onslide*<1>{\mintcmm[highlightlines={10-18}]{../src/backtrace/camlBacktrace\_\_probably\_raise\_351.cmm}}
      \onslide*<2>{\listcmm[highlightlines=18]{../src/backtrace/camlBacktrace\_\_probably\_raise_351.cmm}{CMM of probably\_raise}}
      \onslide*<3>{\mintobjdump[firstline=5,lastline=35,highlightlines=31]{../src/backtrace/camlBacktrace\_\_probably\_raise_351.objdump}}
    \end{column}
  \end{columns}
  \only<1>{\footnotetext[1]{https://blog.janestreet.com/pattern-matching-and-exception-handling-unite/}}
\end{frame}

\newcommand\listCamlReraiseExn[1][]{
  \listasm[firstline=727,lastline=734,#1]{../ocaml/runtime/amd64.S}{runtime/amd64.S:727}}

\begin{frame}{Backtraces}{Introducing \funcname{caml\_reraise\_exn}}
  \begin{columns}[c]
    \begin{column}{0.5\textwidth}
      \minipage[c][0.66\textheight][s]{\columnwidth}
      \begin{itemize}
        \item<1-> The exception have to be reraised to the next handler
        \item<2-> First, checks if the backtrace should be collected with \funcarg{domain\_state->backtrace\_active}
        \only<3>{
        \begin{itemize}
            \item If not, behaves like a \funcname{raise\_notrace} with \funcname{RESTORE\_EXN\_HANDLER\_OCAML}
        \end{itemize}
        }
        \item<4-> Jumps into \funcname{caml\_raise\_exn}
        \item<5-> Calls \funcname{caml\_stash\_backtrace} again, to scan the next part of the stack: from the trap just pop-ed to the next trap
      \end{itemize}
      \vfill
      \mintocaml[firstline=15,lastline=21,highlightlines={18-21}]{../src/backtrace/backtrace.ml}
      \endminipage
    \end{column}
    \begin{column}{0.5\textwidth}
      \raggedleft
      \onslide*<1>{\listCamlReraiseExn{}}
      \onslide*<2>{\listCamlReraiseExn[highlightlines=729]{}}
      \onslide*<3>{
        \mintc[firstline=190,lastline=192,highlightlines={191-192}]{../ocaml/runtime/amd64.S}
        \mintc[firstline=234,lastline=237,highlightlines={235-237}]{../ocaml/runtime/amd64.S}
        \medskip
        \listCamlReraiseExn[highlightlines=731]{}
      }
      \onslide*<4-5>{
        % TODO refactor with \listCamlReraiseExn
        \mintasm[firstline=727,lastline=734,highlightlines=730]{../ocaml/runtime/amd64.S}
        \medskip
      }
      \onslide*<4>{\listCamlRaiseExn[highlightlines=711]{}}
      \onslide*<5>{\listCamlRaiseExn[highlightlines=720]{}}
    \end{column}
  \end{columns}
\end{frame}

\begin{frame}{Backtraces}{Backtrace collection continues}
  \begin{columns}[c]
    \begin{column}{0.55\textwidth}
      \minipage[c][0.75\textheight][s]{\columnwidth}
      \begin{itemize}
        \item<1-> \funcname{caml\_stash\_backtrace} scans the older part of the stack, up to the next trap
        \item<6-> Controls jumps to the nested exception handler in \funcname{maybe\_raise}, which matches only on \typename{ExnB}
        \item<7-> \funcname{caml\_reraise\_exn} is called again, backtrace collection resumes with \funcname{caml\_stash\_backtrace}
      \end{itemize}
      \medskip
      \begin{itemize}
        \item<2-> Constructed backtrace:
        \begin{itemize}
          \item <2-> \funcname{definitely\_raise}
          \item <2-> \funcname{probably\_raise}
          \item <3-> \funcname{probably\_raise} (re-raise)
          \item <4-> \funcname{maybe\_raise}
          \item <5-> \funcname{main}
          \item <8-> \funcname{main} (re-raise)
        \end{itemize}
      \end{itemize}
      \vfill
      \onslide*<1-5>{\mintocaml[firstline=15,lastline=21]{../src/backtrace/backtrace.ml}}
      \onslide*<6>{\mintocaml[firstline=28,lastline=34,highlightlines=31]{../src/backtrace/backtrace.ml}}
      \onslide*<7->{\mintocaml[firstline=28,lastline=34,highlightlines=33]{../src/backtrace/backtrace.ml}}
      \endminipage
    \end{column}
    \begin{column}{0.45\textwidth}
      \raggedleft
      \onslide*<1>{\providecommand\step{6}\input{diagrams/backtrace/backtrace.tex}}%
      \onslide*<2>{\providecommand\step{6}\input{diagrams/backtrace/backtrace.tex}}%
      \onslide*<3>{\providecommand\step{7}\input{diagrams/backtrace/backtrace.tex}}%
      \onslide*<4>{\providecommand\step{8}\input{diagrams/backtrace/backtrace.tex}}%
      \onslide*<5>{\providecommand\step{9}\input{diagrams/backtrace/backtrace.tex}}%
      \onslide*<6>{\providecommand\step{10}\input{diagrams/backtrace/backtrace.tex}}%
      \onslide*<7>{\providecommand\step{11}\input{diagrams/backtrace/backtrace.tex}}%
      \onslide*<8>{\providecommand\step{12}\input{diagrams/backtrace/backtrace.tex}}%
      \onslide*<9>{\providecommand\step{13}\input{diagrams/backtrace/backtrace.tex}}%
    \end{column}
  \end{columns}
\end{frame}

\begin{frame}{Backtraces}{Printing the backtrace}
  \begin{columns}[c]
    \begin{column}{0.45\textwidth}
      \minipage[c][0.66\textheight][s]{\columnwidth}
      \begin{itemize}
        \item<1-> The exception backtrace can now be printed to \funcarg{stdout} with \funcname{Printexc.print_backtrace}
        \item<2-> First the exception backtrace is retrieved into an OCaml array with \funcname{get_raw_backtrace }
        \item<3-> Which is implemented as the \funcname{caml_get_exception_raw_backtrace} C function in the runtime
        \item<4-> And finally printed with \funcname{print_exception_backtrace}
      \end{itemize}
      \vfill
      \mintocaml[firstline=28,lastline=34,highlightlines=33]{../src/backtrace/backtrace.ml}
      \endminipage
    \end{column}
    \begin{column}{0.55\textwidth}
      \onslide*<1,2>{
        \mintocaml[firstline=100,lastline=101]{../ocaml/stdlib/printexc.ml}
      }
      \only<1>{
        \listocaml[firstline=170,lastline=172]{../ocaml/stdlib/printexc.ml}{stdlib/printexc.ml:170}
      }
      \only<2>{
        \listocaml[firstline=170,lastline=172,highlightlines=172]{../ocaml/stdlib/printexc.ml}{stdlib/printexc.ml:170}
      }
      \only<3>{
        \mintc[firstline=154,lastline=156]{../ocaml/runtime/backtrace.c}
        \listc[firstline=171,lastline=192,highlightlines={184-187}]{../ocaml/runtime/backtrace.c}{runtime/backtrace.c:154}
      }
      \only<4>{
        \listocaml[firstline=155,lastline=165]{../ocaml/stdlib/printexc.ml}{stdlib/printexc.ml:55}
      }
    \end{column}
  \end{columns}
\end{frame}


\subsection{The multiple kinds of raise}
\frameSubsection{}{}

\begin{frame}{Backtraces}{When backtraces are not needed}
  \begin{columns}[c]
    \begin{column}{0.45\textwidth}
      \minipage[c][0.66\textheight][s]{\columnwidth}
      \begin{itemize}
        \item<1-> What if the backtrace for an exception is never needed?
        \item<2-> It's possible to raise the exception explicitly with \funcname{raise\_notrace} to make raising as cheap as possible
        \item<3-> An example of use in the \funcname{dynlink} library of the OCaml compiler
      \end{itemize}
      \vfill
      \onslide<3->{
      \listocaml[firstline=205,lastline=209,highlightlines=207]{../ocaml/otherlibs/dynlink/dynlink\_compilerlibs/misc.ml}{otherlibs/dynlink/dynlink\_compilerlibs/misc.ml:205}
      }
      \endminipage
    \end{column}
    \begin{column}{0.55\textwidth}
    \only<4>{
      \mintcmm[highlightlines=3]{../src/notrace/camlNotrace\_\_fun_347.cmm}
      \listcmm{../src/notrace/camlNotrace\_\_all\_somes\_327.cmm}{all\_somes}
    }
    \onslide*<5->{
      \listobjdump[highlightlines={6-9}]{../src/notrace/camlNotrace\_\_fun_347.objdump}{anonymous function from \funcname{all\_somes}}
    }
    \end{column}
  \end{columns}
\end{frame}

\begin{frame}{Backtraces}{Summary table of the multiple kinds of raise}
  \begin{itemize}
    \item When debugging information is enabled and if the backtrace of an exception isn't needed, it's possible to raise with \funcname{raise\_notrace} for performance
    \item When debugging information is \emph{not} enabled, the compiler optimises \funcname{raise} calls to inline assembly (same effect as using \funcname{raise\_notrace})
  \end{itemize}
  \bigskip
  \centering
  \begin{tabular}{| l | c | c | c | c |}
    \hline
    \parbox{10em}{Debug information \\ (\funcarg{-g} flag)}  & \multicolumn{2}{|c|}{Disabled}                                     & \multicolumn{2}{|c|}{Enabled} \\
    \hline
    Raise kind                                               & \funcname{raise} & \funcname{raise\_notrace}                       & \funcname{raise} & \funcname{raise\_notrace} \\
    \hline
    Compilation                                              & \parbox{8em}{inline assembly\\ (optimization)} & inline assembly   & \funcname{caml\_raise\_exn} & inline assembly \\
    \hline
  \end{tabular}
\end{frame}


%
% Takeaway #5
%
\subsection*{Takeaway \#5}
\frameSubsectionTakeaway{}
\againframe<5>{takeaway}
}%
    \end{column}
  \end{columns}
\end{frame}

\begin{frame}{Backtraces}{Printing the backtrace}
  \begin{columns}[c]
    \begin{column}{0.45\textwidth}
      \minipage[c][0.66\textheight][s]{\columnwidth}
      \begin{itemize}
        \item<1-> The exception backtrace can now be printed to \funcarg{stdout} with \funcname{Printexc.print_backtrace}
        \item<2-> First the exception backtrace is retrieved into an OCaml array with \funcname{get_raw_backtrace }
        \item<3-> Which is implemented as the \funcname{caml_get_exception_raw_backtrace} C function in the runtime
        \item<4-> And finally printed with \funcname{print_exception_backtrace}
      \end{itemize}
      \vfill
      \mintocaml[firstline=28,lastline=34,highlightlines=33]{../src/backtrace/backtrace.ml}
      \endminipage
    \end{column}
    \begin{column}{0.55\textwidth}
      \onslide*<1,2>{
        \mintocaml[firstline=100,lastline=101]{../ocaml/stdlib/printexc.ml}
      }
      \only<1>{
        \listocaml[firstline=170,lastline=172]{../ocaml/stdlib/printexc.ml}{stdlib/printexc.ml:170}
      }
      \only<2>{
        \listocaml[firstline=170,lastline=172,highlightlines=172]{../ocaml/stdlib/printexc.ml}{stdlib/printexc.ml:170}
      }
      \only<3>{
        \mintc[firstline=154,lastline=156]{../ocaml/runtime/backtrace.c}
        \listc[firstline=171,lastline=192,highlightlines={184-187}]{../ocaml/runtime/backtrace.c}{runtime/backtrace.c:154}
      }
      \only<4>{
        \listocaml[firstline=155,lastline=165]{../ocaml/stdlib/printexc.ml}{stdlib/printexc.ml:55}
      }
    \end{column}
  \end{columns}
\end{frame}


\subsection{The multiple kinds of raise}
\frameSubsection{}{}

\begin{frame}{Backtraces}{When backtraces are not needed}
  \begin{columns}[c]
    \begin{column}{0.45\textwidth}
      \minipage[c][0.66\textheight][s]{\columnwidth}
      \begin{itemize}
        \item<1-> What if the backtrace for an exception is never needed?
        \item<2-> It's possible to raise the exception explicitly with \funcname{raise\_notrace} to make raising as cheap as possible
        \item<3-> An example of use in the \funcname{dynlink} library of the OCaml compiler
      \end{itemize}
      \vfill
      \onslide<3->{
      \listocaml[firstline=205,lastline=209,highlightlines=207]{../ocaml/otherlibs/dynlink/dynlink\_compilerlibs/misc.ml}{otherlibs/dynlink/dynlink\_compilerlibs/misc.ml:205}
      }
      \endminipage
    \end{column}
    \begin{column}{0.55\textwidth}
    \only<4>{
      \mintcmm[highlightlines=3]{../src/notrace/camlNotrace\_\_fun_347.cmm}
      \listcmm{../src/notrace/camlNotrace\_\_all\_somes\_327.cmm}{all\_somes}
    }
    \onslide*<5->{
      \listobjdump[highlightlines={6-9}]{../src/notrace/camlNotrace\_\_fun_347.objdump}{anonymous function from \funcname{all\_somes}}
    }
    \end{column}
  \end{columns}
\end{frame}

\begin{frame}{Backtraces}{Summary table of the multiple kinds of raise}
  \begin{itemize}
    \item When debugging information is enabled and if the backtrace of an exception isn't needed, it's possible to raise with \funcname{raise\_notrace} for performance
    \item When debugging information is \emph{not} enabled, the compiler optimises \funcname{raise} calls to inline assembly (same effect as using \funcname{raise\_notrace})
  \end{itemize}
  \bigskip
  \centering
  \begin{tabular}{| l | c | c | c | c |}
    \hline
    \parbox{10em}{Debug information \\ (\funcarg{-g} flag)}  & \multicolumn{2}{|c|}{Disabled}                                     & \multicolumn{2}{|c|}{Enabled} \\
    \hline
    Raise kind                                               & \funcname{raise} & \funcname{raise\_notrace}                       & \funcname{raise} & \funcname{raise\_notrace} \\
    \hline
    Compilation                                              & \parbox{8em}{inline assembly\\ (optimization)} & inline assembly   & \funcname{caml\_raise\_exn} & inline assembly \\
    \hline
  \end{tabular}
\end{frame}


%
% Takeaway #5
%
\subsection*{Takeaway \#5}
\frameSubsectionTakeaway{}
\againframe<5>{takeaway}
}%
      \onslide*<4>{\providecommand\step{8}%
% Backtraces
%
\subsection{One advantage of raising with debugging information}
\frameSubsection{}{}

\begin{frame}{Backtraces}{Debugging information \& source code}
  \begin{columns}[c]
    \begin{column}{0.5\textwidth}
      \begin{itemize}
        \item Raising an exception is fun and games, but how do we know where the exception originated? $\rightarrow$ With the backtrace associated with the exception!
        \item In order to get a backtrace from the point where an exception was raised, it's necessary to compile with debugging information enabled (\funcarg{-g} flag for the compiler)
        \item Same example as in the `Nested handlers' section
      \end{itemize}
    \end{column}
    \begin{column}{0.5\textwidth}
      \listocaml[firstline=9,lastline=34]{../src/backtrace/backtrace.ml}{backtrace/backtrace.ml}
    \end{column}
  \end{columns}
\end{frame}

\begin{frame}{Backtraces}{CMM and disassembly of \funcname{definitely\_raise}}
  \begin{columns}[c]
    \begin{column}{0.5\textwidth}
      \minipage[c][0.66\textheight][s]{\columnwidth}
      \begin{itemize}
        \item<1-> An effect of compiling with debugging information (\funcarg{-g}): the CMM no longer uses \funcname{raise\_notrace} to raise the exception but \funcname{raise} instead
        \item<2-> Disassembly shows that CMM \funcname{raise} is actually a call to \funcname{caml\_raise\_exn}, a function in assembler provided by the runtime
      \end{itemize}
      \vfill
      \mintocaml[firstline=9,lastline=13,highlightlines=12]{../src/backtrace/backtrace.ml}
      \endminipage
    \end{column}
    \begin{column}{0.5\textwidth}
      \onslide*<1>{
        \mintcmm[highlightlines=5]{../src/backtrace/camlBacktrace\_\_definitely\_raise\_347.cmm}
      }
      \onslide*<2>{
        \mintobjdump[firstline=2,highlightlines=14]{../src/backtrace/camlBacktrace\_\_definitely\_raise\_347.objdump}
      }
    \end{column}
  \end{columns}
\end{frame}

\begin{frame}{Backtraces}{Introducing \funcname{caml\_raise\_exn}}
  \begin{columns}[c]
    \begin{column}{0.5\textwidth}
      \minipage[c][0.66\textheight][s]{\columnwidth}
      \onslide*<1>{
        \begin{itemize}
          \item Raising the exception is performed using \funcname{caml\_raise\_exn} which is
            \begin{itemize}
              \item An assembly function provided by the runtime
              \item Architecture specific
            \end{itemize}
          \item Let's see what it does differently from \funcname{raise\_notrace}\dots
          \item We are about to raise \typename{ExnC} with \funcname{caml\_raise\_exn} from \funcname{definitely\_raise}
        \end{itemize}
      }
      \onslide*<2>{
        \mintobjdump[firstline=2,highlightlines=14]{../src/backtrace/camlBacktrace\_\_definitely\_raise\_347.objdump}
      }
      \vfill
      \mintocaml[firstline=9,lastline=13,highlightlines=12]{../src/backtrace/backtrace.ml}
      \endminipage
    \end{column}
    \begin{column}{0.5\textwidth}
      \centering
      \providecommand\step{1}%
% Backtraces
%
\subsection{One advantage of raising with debugging information}
\frameSubsection{}{}

\begin{frame}{Backtraces}{Debugging information \& source code}
  \begin{columns}[c]
    \begin{column}{0.5\textwidth}
      \begin{itemize}
        \item Raising an exception is fun and games, but how do we know where the exception originated? $\rightarrow$ With the backtrace associated with the exception!
        \item In order to get a backtrace from the point where an exception was raised, it's necessary to compile with debugging information enabled (\funcarg{-g} flag for the compiler)
        \item Same example as in the `Nested handlers' section
      \end{itemize}
    \end{column}
    \begin{column}{0.5\textwidth}
      \listocaml[firstline=9,lastline=34]{../src/backtrace/backtrace.ml}{backtrace/backtrace.ml}
    \end{column}
  \end{columns}
\end{frame}

\begin{frame}{Backtraces}{CMM and disassembly of \funcname{definitely\_raise}}
  \begin{columns}[c]
    \begin{column}{0.5\textwidth}
      \minipage[c][0.66\textheight][s]{\columnwidth}
      \begin{itemize}
        \item<1-> An effect of compiling with debugging information (\funcarg{-g}): the CMM no longer uses \funcname{raise\_notrace} to raise the exception but \funcname{raise} instead
        \item<2-> Disassembly shows that CMM \funcname{raise} is actually a call to \funcname{caml\_raise\_exn}, a function in assembler provided by the runtime
      \end{itemize}
      \vfill
      \mintocaml[firstline=9,lastline=13,highlightlines=12]{../src/backtrace/backtrace.ml}
      \endminipage
    \end{column}
    \begin{column}{0.5\textwidth}
      \onslide*<1>{
        \mintcmm[highlightlines=5]{../src/backtrace/camlBacktrace\_\_definitely\_raise\_347.cmm}
      }
      \onslide*<2>{
        \mintobjdump[firstline=2,highlightlines=14]{../src/backtrace/camlBacktrace\_\_definitely\_raise\_347.objdump}
      }
    \end{column}
  \end{columns}
\end{frame}

\begin{frame}{Backtraces}{Introducing \funcname{caml\_raise\_exn}}
  \begin{columns}[c]
    \begin{column}{0.5\textwidth}
      \minipage[c][0.66\textheight][s]{\columnwidth}
      \onslide*<1>{
        \begin{itemize}
          \item Raising the exception is performed using \funcname{caml\_raise\_exn} which is
            \begin{itemize}
              \item An assembly function provided by the runtime
              \item Architecture specific
            \end{itemize}
          \item Let's see what it does differently from \funcname{raise\_notrace}\dots
          \item We are about to raise \typename{ExnC} with \funcname{caml\_raise\_exn} from \funcname{definitely\_raise}
        \end{itemize}
      }
      \onslide*<2>{
        \mintobjdump[firstline=2,highlightlines=14]{../src/backtrace/camlBacktrace\_\_definitely\_raise\_347.objdump}
      }
      \vfill
      \mintocaml[firstline=9,lastline=13,highlightlines=12]{../src/backtrace/backtrace.ml}
      \endminipage
    \end{column}
    \begin{column}{0.5\textwidth}
      \centering
      \providecommand\step{1}\input{diagrams/backtrace/backtrace.tex}
    \end{column}
  \end{columns}
\end{frame}

\begin{frame}{Backtraces}{But before that, we need more fields from \typename{caml\_domain\_state}}
  \begin{columns}
    \begin{column}{0.5\textwidth}
      \begin{itemize}
        \item Remember \typename{caml\_domain\_state} the per-domain runtime state?
        \item Some other members are of interest to us now\dots
        \item \localname{backtrace\_active} is used to know if the runtime should collect backtraces
        \item \localname{backtrace\_active} can be set by
          \begin{itemize}
            \item The \funcarg{b} flag in the \funcname{OCAMLRUNPARAM} environment variable
            \item \mintinline{ocaml}{Printexc.record_backtrace : bool -> unit} \footnotemark
          \end{itemize}
      \end{itemize}
    \end{column}
    \begin{column}{0.5\textwidth}
      \begin{listing}
        \mintc[highlightlines={9-11}]{src/ocaml/domain\_state\_backtrace.h}
        \caption{runtime/caml/domain\_state.h}
      \end{listing}
    \end{column}
  \end{columns}
  \footnotetext[1]{https://v2.ocaml.org/api/Printexc.html}
\end{frame}

\newcommand\listCamlRaiseExn[1][]{
  \listasm[firstline=702,lastline=725,#1]{../ocaml/runtime/amd64.S}{runtime/amd64.S:702}}

\begin{frame}{Backtraces}{Walk through \funcname{caml\_raise\_exn}}
  \begin{columns}[T]
    \begin{column}{0.5\textwidth}
      \begin{itemize}
        \item<1-> First checks whether exceptions backtrace should be recorded
        \item<1-> By checking the \funcarg{domain\_state->backtrace\_active} value
        \item<2-> If not, acts as \funcname{raise\_notrace} with \funcname{RESTORE\_EXN\_HANDLER\_OCAML}
          \begin{itemize}
            \item Set \regname{RSP} to \localname{domain\_state->exn\_handler} value
            \item Restore \localname{domain\_state->exn\_handler} from trap
            \item Jump with \funcname{ret} (same effect as using \funcname{pop} and \funcname{jmp})
          \end{itemize}
        \item<3-> Otherwise, switches to the C stack to be able to execute C code
        \item<4-> Calls \funcname{caml\_stash\_backtrace}, a C function provided by the runtime\ignorespaces
          \onslide<6->{, with
            \begin{itemize}
              \item \funcarg{exn}: exception value
              \item \funcarg{pc}: code address of the raising point
              \item \funcarg{sp}: stack address of the raising function
              \item \funcarg{trapsp}: stack address of the next handler
            \end{itemize}
          }
      \end{itemize}
      \bigskip
      \mintocaml[firstline=9,lastline=13,highlightlines=12]{../src/backtrace/backtrace.ml}
    \end{column}
    \begin{column}{0.5\textwidth}
      \raggedleft
      \onslide*<1,3-4>{
        \mintc[firstline=190,lastline=192]{../ocaml/runtime/amd64.S}
        \mintc[firstline=234,lastline=237]{../ocaml/runtime/amd64.S}
      }
      \onslide*<2>{
        \mintc[firstline=190,lastline=192,highlightlines={191-192}]{../ocaml/runtime/amd64.S}
        \mintc[firstline=234,lastline=237,highlightlines={235-237}]{../ocaml/runtime/amd64.S}
      }
      \onslide*<1>{\listCamlRaiseExn[highlightlines=705]{}}%
      \onslide*<2>{\listCamlRaiseExn[highlightlines={707-708}]{}}%
      \onslide*<3>{\listCamlRaiseExn[highlightlines={712-714}]{}}%
      \onslide*<4>{\listCamlRaiseExn[highlightlines={715-720}]{}}%
      \onslide*<5>{\providecommand\step{1}\input{diagrams/backtrace/backtrace.tex}}%
      \onslide*<6>{\providecommand\step{2}\input{diagrams/backtrace/backtrace.tex}}%
    \end{column}
  \end{columns}
\end{frame}

\newcommand\listStashBacktrace[1][]{
  \listc[firstline=91,lastline=120,#1]{../ocaml/runtime/backtrace\_nat.c}{runtime/backtrace\_nat.c:91}}

\begin{frame}{Backtraces}{Introducing \funcname{caml\_stash\_backtrace}}
  \begin{columns}[c]
    \begin{column}{0.5\textwidth}
      \minipage[c][0.66\textheight][s]{\columnwidth}
      \begin{itemize}
        \item<1-> A C function provided by the runtime (specific to native code)
        \item<2-> It scans the stack, between the stack pointer at the raising point up to the next trap, in order to collect the names of the detected function frames
          \onslide*<2>{
            \begin{itemize}
              \item And more information, like line number, column number...
            \end{itemize}
          }
        \item<3-> It uses the same technique and information as the Garbage Collector (GC) uses to scan the values live on the stack
          \onslide*<4>{
            \begin{itemize}
              \item \typename{frame\_descr} are at the core of stack unwinding
            \end{itemize}
          }
      \end{itemize}
      \vfill
      \mintocaml[firstline=9,lastline=13,highlightlines=12]{../src/backtrace/backtrace.ml}
      \endminipage
    \end{column}
    \begin{column}{0.5\textwidth}
      \onslide*<1>{\listStashBacktrace[highlightlines=91]{}}
      \onslide*<2>{\listStashBacktrace[highlightlines={91,117-118}]{}}
      \onslide*<3->{\listStashBacktrace[highlightlines={94,106,109-110}]{}}
    \end{column}
  \end{columns}
\end{frame}

\newcommand\listFrameDescr[1][]{
  \listocaml[firstline=111,lastline=124,#1]{../ocaml/asmcomp/emitaux.ml}{asmcomp/emitaux.ml:111}}
\newcommand\listDebugInfo[1][]{
  \listocaml[firstline=38,lastline=49,#1]{../ocaml/lambda/debuginfo.mli}{lambda/debuginfo.mli:38}}

\begin{frame}{Backtraces}{\typename{frame\_descr} and \typename{frame\_debuginfo} in a nutshell}
  \begin{columns}[c]
    \begin{column}{0.5\textwidth}
      \begin{itemize}
        \item<1-> For every return address in an OCaml program, there is a associated \typename{frame\_descr}
        \item<2-> A \typename{frame\_descr} specifies
        \begin{itemize}
          \item<2-> The size of the function stack frame
          \item<3-> Where live values are on the stack (so that the GC can mark them as live and prevent from being swept)
        \end{itemize}
        \item<4-> When compiled with debug information, \typename{frame\_descr} will be linked to a \typename{frame\_debuginfo}
        \begin{itemize}
          \item<5-> Contains information about the position in the source code (file name, line and column number)
        \end{itemize}
      \end{itemize}
    \end{column}
    \begin{column}{0.5\textwidth}
        \onslide*<1>{\listFrameDescr[highlightlines={118-122}]{}}
        \onslide*<2>{\listFrameDescr[highlightlines={120}]{}}
        \onslide*<3>{\listFrameDescr[highlightlines={121}]{}}
        \onslide*<4->{\listFrameDescr[highlightlines={122}]{}}
        \medskip
        \onslide*<1-3>{\listDebugInfo{}}
        \onslide*<4>{\listDebugInfo[highlightlines={38,49}]{}}
        \onslide*<5>{\listDebugInfo[highlightlines={39-46}]{}}
    \end{column}
  \end{columns}
\end{frame}

\begin{frame}{Backtraces}{Scanning the stack with \typename{frame\_descr}}
  \begin{columns}[c]
    \begin{column}{0.5\textwidth}
      \begin{itemize}
        \item Given the return address of the code that raises the exception by calling \funcname{caml\_raise\_exn} (\funcarg{pc} argument of \funcname{caml\_stash\_backtrace})
        \item And the position of the stack pointer (\funcarg{sp} argument of \funcname{caml\_stash\_backtrace})
        \item The frame size given by \typename{frame\_descr}.\funcarg{frame\_size}, allows to find the end of the previous frame
        \item And the return address of the caller of this function
        \bigskip
        \onslide<3->{
        \item Note that traps (here the reddish \funcname{ExnA}, \funcname{ExnB} and \funcname{ExnC} blocks) belong to the frame that installed them, and so the frame size changes when traps are removed
        }
      \end{itemize}
    \end{column}
    \begin{column}{0.5\textwidth}
      \raggedleft
      \onslide*<1>{\providecommand\step{2}\input{diagrams/backtrace/backtrace.tex}}%
      \onslide*<2>{\providecommand\step{1}\input{diagrams/backtrace/scanstack.tex}}%
      \onslide*<3>{\providecommand\step{2}\input{diagrams/backtrace/scanstack.tex}}%
      \onslide*<4>{\providecommand\step{3}\input{diagrams/backtrace/scanstack.tex}}%
      \onslide*<5>{\providecommand\step{4}\input{diagrams/backtrace/scanstack.tex}}%
      \onslide*<6>{\providecommand\step{5}\input{diagrams/backtrace/scanstack.tex}}%
    \end{column}
  \end{columns}
\end{frame}

% Duplicate of previous-previous slide
\begin{frame}{Backtraces}{Continuing on \funcname{caml\_stash\_backtrace}}
  \begin{columns}[c]
    \begin{column}{0.5\textwidth}
      \minipage[c][0.75\textheight][s]{\columnwidth}
      \begin{itemize}
          \color{gray}
        \item A C function provided by the runtime (specific to native code)
        \item It scans the stack, between the stack pointer at the raising point up to the next trap, in order to collect the names of the detected function frames
        \item It uses the same technique and information as the Garbage Collector (GC) uses to scan the values live on the stack
      \end{itemize}
      \begin{itemize}
        \item For each function frame detected, store a pointer to the \typename{frame\_descr} in the \localname{domain\_state->backtrace\_buffer} array, effectively constructing the backtrace
      \end{itemize}
      \onslide<2->{
        \begin{itemize}
            \smallskip
          \item Constructed backtrace:
            \funcname{definitely\_raise},
            \onslide<3->{\funcname{probably\_raise}}
        \end{itemize}
      }
      \vfill
      \mintocaml[firstline=9,lastline=13,highlightlines=12]{../src/backtrace/backtrace.ml}
      \endminipage
    \end{column}
    \begin{column}{0.5\textwidth}
      \raggedleft
      \onslide*<1>{\listStashBacktrace[highlightlines={114-115}]{}}
      \onslide*<2>{\providecommand\step{3}\input{diagrams/backtrace/backtrace.tex}}%
      \onslide*<3>{\providecommand\step{4}\input{diagrams/backtrace/backtrace.tex}}%
    \end{column}
  \end{columns}
\end{frame}

\begin{frame}{Backtraces}{Back to \funcname{caml\_raise\_exn}}
  \begin{columns}[c]
    \begin{column}{0.5\textwidth}
      \minipage[c][0.66\textheight][s]{\columnwidth}
      \begin{itemize}\color{gray}
        \item Checks wheteher exceptions backtrace should be recorded, by checking the \funcarg{domain\_state->backtrace\_active} value
        \item Switches to the C stack to be able to execute C code
        \item Calls \funcname{caml\_stash\_backtrace}, to collect the backtrace up to the next trap
      \end{itemize}
      \onslide*<2->{
        \begin{itemize}
          \item<2-> Executes the exception handler in \funcname{probably\_raise} with \funcname{RESTORE\_EXN\_HANDLER\_OCAML}
            \begin{itemize}
              \item Set \regname{RSP} to \localname{domain\_state->exn\_handler} value
              \item Restore \localname{domain\_state->exn\_handler} from trap
              \item Jump to the handler code with \funcname{ret}
            \end{itemize}
        \end{itemize}
      }
      \vfill
      \onslide*<1>{\mintocaml[firstline=9,lastline=13,highlightlines=12]{../src/backtrace/backtrace.ml}}
      \onslide*<2->{\mintocaml[firstline=15,lastline=21,highlightlines={19-21}]{../src/backtrace/backtrace.ml}}
      \endminipage
    \end{column}
    \begin{column}{0.5\textwidth}
      \centering
      \onslide*<1>{
        \mintc[firstline=190,lastline=192]{../ocaml/runtime/amd64.S}
        \mintc[firstline=234,lastline=237]{../ocaml/runtime/amd64.S}
      }
      \onslide*<2>{
        \mintc[firstline=190,lastline=192,highlightlines={191-192}]{../ocaml/runtime/amd64.S}
        \mintc[firstline=234,lastline=237,highlightlines={235-237}]{../ocaml/runtime/amd64.S}
      }
      \onslide*<1>{\listCamlRaiseExn[highlightlines=720]{}}
      \onslide*<2>{\listCamlRaiseExn[highlightlines=722]{}}
      \onslide*<3->{\providecommand\step{5}\input{diagrams/backtrace/backtrace.tex}}
    \end{column}
  \end{columns}
\end{frame}

\begin{frame}{Backtraces}{\funcname{caml\_probably\_raise}'s handler doesn't catch \typename{ExnC}}
  \begin{columns}[c]
    \begin{column}{0.45\textwidth}
      \minipage[c][0.75\textheight][s]{\columnwidth}
      \begin{itemize}
        \item<1-> \funcname{match \dots with}\footnotemark pattern matching can also match exceptions
        \item<1-> The CMM of \funcname{probably\_raise} shows that this \funcname{match \dots with} is implemented with a pattern matching and a \funcname{try \dots with}
        \item<1-> But this one only matches on \typename{ExnA} exception
        \item<2-> Since the pattern matching fails to catch the exception, \funcname{reraise} is called with our current \typename{ExnC} exception
        \item<3-> Disassembly shows that \funcname{reraise} is actually a call to \funcname{caml\_reraise\_exn}, which is also an assembly function provided by the runtime
      \end{itemize}
      \vfill
      \mintocaml[firstline=15,lastline=21,highlightlines={18-21}]{../src/backtrace/backtrace.ml}
      \endminipage
    \end{column}
    \begin{column}{0.55\textwidth}
      \centering
      \onslide*<1>{\mintcmm[highlightlines={10-18}]{../src/backtrace/camlBacktrace\_\_probably\_raise\_351.cmm}}
      \onslide*<2>{\listcmm[highlightlines=18]{../src/backtrace/camlBacktrace\_\_probably\_raise_351.cmm}{CMM of probably\_raise}}
      \onslide*<3>{\mintobjdump[firstline=5,lastline=35,highlightlines=31]{../src/backtrace/camlBacktrace\_\_probably\_raise_351.objdump}}
    \end{column}
  \end{columns}
  \only<1>{\footnotetext[1]{https://blog.janestreet.com/pattern-matching-and-exception-handling-unite/}}
\end{frame}

\newcommand\listCamlReraiseExn[1][]{
  \listasm[firstline=727,lastline=734,#1]{../ocaml/runtime/amd64.S}{runtime/amd64.S:727}}

\begin{frame}{Backtraces}{Introducing \funcname{caml\_reraise\_exn}}
  \begin{columns}[c]
    \begin{column}{0.5\textwidth}
      \minipage[c][0.66\textheight][s]{\columnwidth}
      \begin{itemize}
        \item<1-> The exception have to be reraised to the next handler
        \item<2-> First, checks if the backtrace should be collected with \funcarg{domain\_state->backtrace\_active}
        \only<3>{
        \begin{itemize}
            \item If not, behaves like a \funcname{raise\_notrace} with \funcname{RESTORE\_EXN\_HANDLER\_OCAML}
        \end{itemize}
        }
        \item<4-> Jumps into \funcname{caml\_raise\_exn}
        \item<5-> Calls \funcname{caml\_stash\_backtrace} again, to scan the next part of the stack: from the trap just pop-ed to the next trap
      \end{itemize}
      \vfill
      \mintocaml[firstline=15,lastline=21,highlightlines={18-21}]{../src/backtrace/backtrace.ml}
      \endminipage
    \end{column}
    \begin{column}{0.5\textwidth}
      \raggedleft
      \onslide*<1>{\listCamlReraiseExn{}}
      \onslide*<2>{\listCamlReraiseExn[highlightlines=729]{}}
      \onslide*<3>{
        \mintc[firstline=190,lastline=192,highlightlines={191-192}]{../ocaml/runtime/amd64.S}
        \mintc[firstline=234,lastline=237,highlightlines={235-237}]{../ocaml/runtime/amd64.S}
        \medskip
        \listCamlReraiseExn[highlightlines=731]{}
      }
      \onslide*<4-5>{
        % TODO refactor with \listCamlReraiseExn
        \mintasm[firstline=727,lastline=734,highlightlines=730]{../ocaml/runtime/amd64.S}
        \medskip
      }
      \onslide*<4>{\listCamlRaiseExn[highlightlines=711]{}}
      \onslide*<5>{\listCamlRaiseExn[highlightlines=720]{}}
    \end{column}
  \end{columns}
\end{frame}

\begin{frame}{Backtraces}{Backtrace collection continues}
  \begin{columns}[c]
    \begin{column}{0.55\textwidth}
      \minipage[c][0.75\textheight][s]{\columnwidth}
      \begin{itemize}
        \item<1-> \funcname{caml\_stash\_backtrace} scans the older part of the stack, up to the next trap
        \item<6-> Controls jumps to the nested exception handler in \funcname{maybe\_raise}, which matches only on \typename{ExnB}
        \item<7-> \funcname{caml\_reraise\_exn} is called again, backtrace collection resumes with \funcname{caml\_stash\_backtrace}
      \end{itemize}
      \medskip
      \begin{itemize}
        \item<2-> Constructed backtrace:
        \begin{itemize}
          \item <2-> \funcname{definitely\_raise}
          \item <2-> \funcname{probably\_raise}
          \item <3-> \funcname{probably\_raise} (re-raise)
          \item <4-> \funcname{maybe\_raise}
          \item <5-> \funcname{main}
          \item <8-> \funcname{main} (re-raise)
        \end{itemize}
      \end{itemize}
      \vfill
      \onslide*<1-5>{\mintocaml[firstline=15,lastline=21]{../src/backtrace/backtrace.ml}}
      \onslide*<6>{\mintocaml[firstline=28,lastline=34,highlightlines=31]{../src/backtrace/backtrace.ml}}
      \onslide*<7->{\mintocaml[firstline=28,lastline=34,highlightlines=33]{../src/backtrace/backtrace.ml}}
      \endminipage
    \end{column}
    \begin{column}{0.45\textwidth}
      \raggedleft
      \onslide*<1>{\providecommand\step{6}\input{diagrams/backtrace/backtrace.tex}}%
      \onslide*<2>{\providecommand\step{6}\input{diagrams/backtrace/backtrace.tex}}%
      \onslide*<3>{\providecommand\step{7}\input{diagrams/backtrace/backtrace.tex}}%
      \onslide*<4>{\providecommand\step{8}\input{diagrams/backtrace/backtrace.tex}}%
      \onslide*<5>{\providecommand\step{9}\input{diagrams/backtrace/backtrace.tex}}%
      \onslide*<6>{\providecommand\step{10}\input{diagrams/backtrace/backtrace.tex}}%
      \onslide*<7>{\providecommand\step{11}\input{diagrams/backtrace/backtrace.tex}}%
      \onslide*<8>{\providecommand\step{12}\input{diagrams/backtrace/backtrace.tex}}%
      \onslide*<9>{\providecommand\step{13}\input{diagrams/backtrace/backtrace.tex}}%
    \end{column}
  \end{columns}
\end{frame}

\begin{frame}{Backtraces}{Printing the backtrace}
  \begin{columns}[c]
    \begin{column}{0.45\textwidth}
      \minipage[c][0.66\textheight][s]{\columnwidth}
      \begin{itemize}
        \item<1-> The exception backtrace can now be printed to \funcarg{stdout} with \funcname{Printexc.print_backtrace}
        \item<2-> First the exception backtrace is retrieved into an OCaml array with \funcname{get_raw_backtrace }
        \item<3-> Which is implemented as the \funcname{caml_get_exception_raw_backtrace} C function in the runtime
        \item<4-> And finally printed with \funcname{print_exception_backtrace}
      \end{itemize}
      \vfill
      \mintocaml[firstline=28,lastline=34,highlightlines=33]{../src/backtrace/backtrace.ml}
      \endminipage
    \end{column}
    \begin{column}{0.55\textwidth}
      \onslide*<1,2>{
        \mintocaml[firstline=100,lastline=101]{../ocaml/stdlib/printexc.ml}
      }
      \only<1>{
        \listocaml[firstline=170,lastline=172]{../ocaml/stdlib/printexc.ml}{stdlib/printexc.ml:170}
      }
      \only<2>{
        \listocaml[firstline=170,lastline=172,highlightlines=172]{../ocaml/stdlib/printexc.ml}{stdlib/printexc.ml:170}
      }
      \only<3>{
        \mintc[firstline=154,lastline=156]{../ocaml/runtime/backtrace.c}
        \listc[firstline=171,lastline=192,highlightlines={184-187}]{../ocaml/runtime/backtrace.c}{runtime/backtrace.c:154}
      }
      \only<4>{
        \listocaml[firstline=155,lastline=165]{../ocaml/stdlib/printexc.ml}{stdlib/printexc.ml:55}
      }
    \end{column}
  \end{columns}
\end{frame}


\subsection{The multiple kinds of raise}
\frameSubsection{}{}

\begin{frame}{Backtraces}{When backtraces are not needed}
  \begin{columns}[c]
    \begin{column}{0.45\textwidth}
      \minipage[c][0.66\textheight][s]{\columnwidth}
      \begin{itemize}
        \item<1-> What if the backtrace for an exception is never needed?
        \item<2-> It's possible to raise the exception explicitly with \funcname{raise\_notrace} to make raising as cheap as possible
        \item<3-> An example of use in the \funcname{dynlink} library of the OCaml compiler
      \end{itemize}
      \vfill
      \onslide<3->{
      \listocaml[firstline=205,lastline=209,highlightlines=207]{../ocaml/otherlibs/dynlink/dynlink\_compilerlibs/misc.ml}{otherlibs/dynlink/dynlink\_compilerlibs/misc.ml:205}
      }
      \endminipage
    \end{column}
    \begin{column}{0.55\textwidth}
    \only<4>{
      \mintcmm[highlightlines=3]{../src/notrace/camlNotrace\_\_fun_347.cmm}
      \listcmm{../src/notrace/camlNotrace\_\_all\_somes\_327.cmm}{all\_somes}
    }
    \onslide*<5->{
      \listobjdump[highlightlines={6-9}]{../src/notrace/camlNotrace\_\_fun_347.objdump}{anonymous function from \funcname{all\_somes}}
    }
    \end{column}
  \end{columns}
\end{frame}

\begin{frame}{Backtraces}{Summary table of the multiple kinds of raise}
  \begin{itemize}
    \item When debugging information is enabled and if the backtrace of an exception isn't needed, it's possible to raise with \funcname{raise\_notrace} for performance
    \item When debugging information is \emph{not} enabled, the compiler optimises \funcname{raise} calls to inline assembly (same effect as using \funcname{raise\_notrace})
  \end{itemize}
  \bigskip
  \centering
  \begin{tabular}{| l | c | c | c | c |}
    \hline
    \parbox{10em}{Debug information \\ (\funcarg{-g} flag)}  & \multicolumn{2}{|c|}{Disabled}                                     & \multicolumn{2}{|c|}{Enabled} \\
    \hline
    Raise kind                                               & \funcname{raise} & \funcname{raise\_notrace}                       & \funcname{raise} & \funcname{raise\_notrace} \\
    \hline
    Compilation                                              & \parbox{8em}{inline assembly\\ (optimization)} & inline assembly   & \funcname{caml\_raise\_exn} & inline assembly \\
    \hline
  \end{tabular}
\end{frame}


%
% Takeaway #5
%
\subsection*{Takeaway \#5}
\frameSubsectionTakeaway{}
\againframe<5>{takeaway}

    \end{column}
  \end{columns}
\end{frame}

\begin{frame}{Backtraces}{But before that, we need more fields from \typename{caml\_domain\_state}}
  \begin{columns}
    \begin{column}{0.5\textwidth}
      \begin{itemize}
        \item Remember \typename{caml\_domain\_state} the per-domain runtime state?
        \item Some other members are of interest to us now\dots
        \item \localname{backtrace\_active} is used to know if the runtime should collect backtraces
        \item \localname{backtrace\_active} can be set by
          \begin{itemize}
            \item The \funcarg{b} flag in the \funcname{OCAMLRUNPARAM} environment variable
            \item \mintinline{ocaml}{Printexc.record_backtrace : bool -> unit} \footnotemark
          \end{itemize}
      \end{itemize}
    \end{column}
    \begin{column}{0.5\textwidth}
      \begin{listing}
        \mintc[highlightlines={9-11}]{src/ocaml/domain\_state\_backtrace.h}
        \caption{runtime/caml/domain\_state.h}
      \end{listing}
    \end{column}
  \end{columns}
  \footnotetext[1]{https://v2.ocaml.org/api/Printexc.html}
\end{frame}

\newcommand\listCamlRaiseExn[1][]{
  \listasm[firstline=702,lastline=725,#1]{../ocaml/runtime/amd64.S}{runtime/amd64.S:702}}

\begin{frame}{Backtraces}{Walk through \funcname{caml\_raise\_exn}}
  \begin{columns}[T]
    \begin{column}{0.5\textwidth}
      \begin{itemize}
        \item<1-> First checks whether exceptions backtrace should be recorded
        \item<1-> By checking the \funcarg{domain\_state->backtrace\_active} value
        \item<2-> If not, acts as \funcname{raise\_notrace} with \funcname{RESTORE\_EXN\_HANDLER\_OCAML}
          \begin{itemize}
            \item Set \regname{RSP} to \localname{domain\_state->exn\_handler} value
            \item Restore \localname{domain\_state->exn\_handler} from trap
            \item Jump with \funcname{ret} (same effect as using \funcname{pop} and \funcname{jmp})
          \end{itemize}
        \item<3-> Otherwise, switches to the C stack to be able to execute C code
        \item<4-> Calls \funcname{caml\_stash\_backtrace}, a C function provided by the runtime\ignorespaces
          \onslide<6->{, with
            \begin{itemize}
              \item \funcarg{exn}: exception value
              \item \funcarg{pc}: code address of the raising point
              \item \funcarg{sp}: stack address of the raising function
              \item \funcarg{trapsp}: stack address of the next handler
            \end{itemize}
          }
      \end{itemize}
      \bigskip
      \mintocaml[firstline=9,lastline=13,highlightlines=12]{../src/backtrace/backtrace.ml}
    \end{column}
    \begin{column}{0.5\textwidth}
      \raggedleft
      \onslide*<1,3-4>{
        \mintc[firstline=190,lastline=192]{../ocaml/runtime/amd64.S}
        \mintc[firstline=234,lastline=237]{../ocaml/runtime/amd64.S}
      }
      \onslide*<2>{
        \mintc[firstline=190,lastline=192,highlightlines={191-192}]{../ocaml/runtime/amd64.S}
        \mintc[firstline=234,lastline=237,highlightlines={235-237}]{../ocaml/runtime/amd64.S}
      }
      \onslide*<1>{\listCamlRaiseExn[highlightlines=705]{}}%
      \onslide*<2>{\listCamlRaiseExn[highlightlines={707-708}]{}}%
      \onslide*<3>{\listCamlRaiseExn[highlightlines={712-714}]{}}%
      \onslide*<4>{\listCamlRaiseExn[highlightlines={715-720}]{}}%
      \onslide*<5>{\providecommand\step{1}%
% Backtraces
%
\subsection{One advantage of raising with debugging information}
\frameSubsection{}{}

\begin{frame}{Backtraces}{Debugging information \& source code}
  \begin{columns}[c]
    \begin{column}{0.5\textwidth}
      \begin{itemize}
        \item Raising an exception is fun and games, but how do we know where the exception originated? $\rightarrow$ With the backtrace associated with the exception!
        \item In order to get a backtrace from the point where an exception was raised, it's necessary to compile with debugging information enabled (\funcarg{-g} flag for the compiler)
        \item Same example as in the `Nested handlers' section
      \end{itemize}
    \end{column}
    \begin{column}{0.5\textwidth}
      \listocaml[firstline=9,lastline=34]{../src/backtrace/backtrace.ml}{backtrace/backtrace.ml}
    \end{column}
  \end{columns}
\end{frame}

\begin{frame}{Backtraces}{CMM and disassembly of \funcname{definitely\_raise}}
  \begin{columns}[c]
    \begin{column}{0.5\textwidth}
      \minipage[c][0.66\textheight][s]{\columnwidth}
      \begin{itemize}
        \item<1-> An effect of compiling with debugging information (\funcarg{-g}): the CMM no longer uses \funcname{raise\_notrace} to raise the exception but \funcname{raise} instead
        \item<2-> Disassembly shows that CMM \funcname{raise} is actually a call to \funcname{caml\_raise\_exn}, a function in assembler provided by the runtime
      \end{itemize}
      \vfill
      \mintocaml[firstline=9,lastline=13,highlightlines=12]{../src/backtrace/backtrace.ml}
      \endminipage
    \end{column}
    \begin{column}{0.5\textwidth}
      \onslide*<1>{
        \mintcmm[highlightlines=5]{../src/backtrace/camlBacktrace\_\_definitely\_raise\_347.cmm}
      }
      \onslide*<2>{
        \mintobjdump[firstline=2,highlightlines=14]{../src/backtrace/camlBacktrace\_\_definitely\_raise\_347.objdump}
      }
    \end{column}
  \end{columns}
\end{frame}

\begin{frame}{Backtraces}{Introducing \funcname{caml\_raise\_exn}}
  \begin{columns}[c]
    \begin{column}{0.5\textwidth}
      \minipage[c][0.66\textheight][s]{\columnwidth}
      \onslide*<1>{
        \begin{itemize}
          \item Raising the exception is performed using \funcname{caml\_raise\_exn} which is
            \begin{itemize}
              \item An assembly function provided by the runtime
              \item Architecture specific
            \end{itemize}
          \item Let's see what it does differently from \funcname{raise\_notrace}\dots
          \item We are about to raise \typename{ExnC} with \funcname{caml\_raise\_exn} from \funcname{definitely\_raise}
        \end{itemize}
      }
      \onslide*<2>{
        \mintobjdump[firstline=2,highlightlines=14]{../src/backtrace/camlBacktrace\_\_definitely\_raise\_347.objdump}
      }
      \vfill
      \mintocaml[firstline=9,lastline=13,highlightlines=12]{../src/backtrace/backtrace.ml}
      \endminipage
    \end{column}
    \begin{column}{0.5\textwidth}
      \centering
      \providecommand\step{1}\input{diagrams/backtrace/backtrace.tex}
    \end{column}
  \end{columns}
\end{frame}

\begin{frame}{Backtraces}{But before that, we need more fields from \typename{caml\_domain\_state}}
  \begin{columns}
    \begin{column}{0.5\textwidth}
      \begin{itemize}
        \item Remember \typename{caml\_domain\_state} the per-domain runtime state?
        \item Some other members are of interest to us now\dots
        \item \localname{backtrace\_active} is used to know if the runtime should collect backtraces
        \item \localname{backtrace\_active} can be set by
          \begin{itemize}
            \item The \funcarg{b} flag in the \funcname{OCAMLRUNPARAM} environment variable
            \item \mintinline{ocaml}{Printexc.record_backtrace : bool -> unit} \footnotemark
          \end{itemize}
      \end{itemize}
    \end{column}
    \begin{column}{0.5\textwidth}
      \begin{listing}
        \mintc[highlightlines={9-11}]{src/ocaml/domain\_state\_backtrace.h}
        \caption{runtime/caml/domain\_state.h}
      \end{listing}
    \end{column}
  \end{columns}
  \footnotetext[1]{https://v2.ocaml.org/api/Printexc.html}
\end{frame}

\newcommand\listCamlRaiseExn[1][]{
  \listasm[firstline=702,lastline=725,#1]{../ocaml/runtime/amd64.S}{runtime/amd64.S:702}}

\begin{frame}{Backtraces}{Walk through \funcname{caml\_raise\_exn}}
  \begin{columns}[T]
    \begin{column}{0.5\textwidth}
      \begin{itemize}
        \item<1-> First checks whether exceptions backtrace should be recorded
        \item<1-> By checking the \funcarg{domain\_state->backtrace\_active} value
        \item<2-> If not, acts as \funcname{raise\_notrace} with \funcname{RESTORE\_EXN\_HANDLER\_OCAML}
          \begin{itemize}
            \item Set \regname{RSP} to \localname{domain\_state->exn\_handler} value
            \item Restore \localname{domain\_state->exn\_handler} from trap
            \item Jump with \funcname{ret} (same effect as using \funcname{pop} and \funcname{jmp})
          \end{itemize}
        \item<3-> Otherwise, switches to the C stack to be able to execute C code
        \item<4-> Calls \funcname{caml\_stash\_backtrace}, a C function provided by the runtime\ignorespaces
          \onslide<6->{, with
            \begin{itemize}
              \item \funcarg{exn}: exception value
              \item \funcarg{pc}: code address of the raising point
              \item \funcarg{sp}: stack address of the raising function
              \item \funcarg{trapsp}: stack address of the next handler
            \end{itemize}
          }
      \end{itemize}
      \bigskip
      \mintocaml[firstline=9,lastline=13,highlightlines=12]{../src/backtrace/backtrace.ml}
    \end{column}
    \begin{column}{0.5\textwidth}
      \raggedleft
      \onslide*<1,3-4>{
        \mintc[firstline=190,lastline=192]{../ocaml/runtime/amd64.S}
        \mintc[firstline=234,lastline=237]{../ocaml/runtime/amd64.S}
      }
      \onslide*<2>{
        \mintc[firstline=190,lastline=192,highlightlines={191-192}]{../ocaml/runtime/amd64.S}
        \mintc[firstline=234,lastline=237,highlightlines={235-237}]{../ocaml/runtime/amd64.S}
      }
      \onslide*<1>{\listCamlRaiseExn[highlightlines=705]{}}%
      \onslide*<2>{\listCamlRaiseExn[highlightlines={707-708}]{}}%
      \onslide*<3>{\listCamlRaiseExn[highlightlines={712-714}]{}}%
      \onslide*<4>{\listCamlRaiseExn[highlightlines={715-720}]{}}%
      \onslide*<5>{\providecommand\step{1}\input{diagrams/backtrace/backtrace.tex}}%
      \onslide*<6>{\providecommand\step{2}\input{diagrams/backtrace/backtrace.tex}}%
    \end{column}
  \end{columns}
\end{frame}

\newcommand\listStashBacktrace[1][]{
  \listc[firstline=91,lastline=120,#1]{../ocaml/runtime/backtrace\_nat.c}{runtime/backtrace\_nat.c:91}}

\begin{frame}{Backtraces}{Introducing \funcname{caml\_stash\_backtrace}}
  \begin{columns}[c]
    \begin{column}{0.5\textwidth}
      \minipage[c][0.66\textheight][s]{\columnwidth}
      \begin{itemize}
        \item<1-> A C function provided by the runtime (specific to native code)
        \item<2-> It scans the stack, between the stack pointer at the raising point up to the next trap, in order to collect the names of the detected function frames
          \onslide*<2>{
            \begin{itemize}
              \item And more information, like line number, column number...
            \end{itemize}
          }
        \item<3-> It uses the same technique and information as the Garbage Collector (GC) uses to scan the values live on the stack
          \onslide*<4>{
            \begin{itemize}
              \item \typename{frame\_descr} are at the core of stack unwinding
            \end{itemize}
          }
      \end{itemize}
      \vfill
      \mintocaml[firstline=9,lastline=13,highlightlines=12]{../src/backtrace/backtrace.ml}
      \endminipage
    \end{column}
    \begin{column}{0.5\textwidth}
      \onslide*<1>{\listStashBacktrace[highlightlines=91]{}}
      \onslide*<2>{\listStashBacktrace[highlightlines={91,117-118}]{}}
      \onslide*<3->{\listStashBacktrace[highlightlines={94,106,109-110}]{}}
    \end{column}
  \end{columns}
\end{frame}

\newcommand\listFrameDescr[1][]{
  \listocaml[firstline=111,lastline=124,#1]{../ocaml/asmcomp/emitaux.ml}{asmcomp/emitaux.ml:111}}
\newcommand\listDebugInfo[1][]{
  \listocaml[firstline=38,lastline=49,#1]{../ocaml/lambda/debuginfo.mli}{lambda/debuginfo.mli:38}}

\begin{frame}{Backtraces}{\typename{frame\_descr} and \typename{frame\_debuginfo} in a nutshell}
  \begin{columns}[c]
    \begin{column}{0.5\textwidth}
      \begin{itemize}
        \item<1-> For every return address in an OCaml program, there is a associated \typename{frame\_descr}
        \item<2-> A \typename{frame\_descr} specifies
        \begin{itemize}
          \item<2-> The size of the function stack frame
          \item<3-> Where live values are on the stack (so that the GC can mark them as live and prevent from being swept)
        \end{itemize}
        \item<4-> When compiled with debug information, \typename{frame\_descr} will be linked to a \typename{frame\_debuginfo}
        \begin{itemize}
          \item<5-> Contains information about the position in the source code (file name, line and column number)
        \end{itemize}
      \end{itemize}
    \end{column}
    \begin{column}{0.5\textwidth}
        \onslide*<1>{\listFrameDescr[highlightlines={118-122}]{}}
        \onslide*<2>{\listFrameDescr[highlightlines={120}]{}}
        \onslide*<3>{\listFrameDescr[highlightlines={121}]{}}
        \onslide*<4->{\listFrameDescr[highlightlines={122}]{}}
        \medskip
        \onslide*<1-3>{\listDebugInfo{}}
        \onslide*<4>{\listDebugInfo[highlightlines={38,49}]{}}
        \onslide*<5>{\listDebugInfo[highlightlines={39-46}]{}}
    \end{column}
  \end{columns}
\end{frame}

\begin{frame}{Backtraces}{Scanning the stack with \typename{frame\_descr}}
  \begin{columns}[c]
    \begin{column}{0.5\textwidth}
      \begin{itemize}
        \item Given the return address of the code that raises the exception by calling \funcname{caml\_raise\_exn} (\funcarg{pc} argument of \funcname{caml\_stash\_backtrace})
        \item And the position of the stack pointer (\funcarg{sp} argument of \funcname{caml\_stash\_backtrace})
        \item The frame size given by \typename{frame\_descr}.\funcarg{frame\_size}, allows to find the end of the previous frame
        \item And the return address of the caller of this function
        \bigskip
        \onslide<3->{
        \item Note that traps (here the reddish \funcname{ExnA}, \funcname{ExnB} and \funcname{ExnC} blocks) belong to the frame that installed them, and so the frame size changes when traps are removed
        }
      \end{itemize}
    \end{column}
    \begin{column}{0.5\textwidth}
      \raggedleft
      \onslide*<1>{\providecommand\step{2}\input{diagrams/backtrace/backtrace.tex}}%
      \onslide*<2>{\providecommand\step{1}\input{diagrams/backtrace/scanstack.tex}}%
      \onslide*<3>{\providecommand\step{2}\input{diagrams/backtrace/scanstack.tex}}%
      \onslide*<4>{\providecommand\step{3}\input{diagrams/backtrace/scanstack.tex}}%
      \onslide*<5>{\providecommand\step{4}\input{diagrams/backtrace/scanstack.tex}}%
      \onslide*<6>{\providecommand\step{5}\input{diagrams/backtrace/scanstack.tex}}%
    \end{column}
  \end{columns}
\end{frame}

% Duplicate of previous-previous slide
\begin{frame}{Backtraces}{Continuing on \funcname{caml\_stash\_backtrace}}
  \begin{columns}[c]
    \begin{column}{0.5\textwidth}
      \minipage[c][0.75\textheight][s]{\columnwidth}
      \begin{itemize}
          \color{gray}
        \item A C function provided by the runtime (specific to native code)
        \item It scans the stack, between the stack pointer at the raising point up to the next trap, in order to collect the names of the detected function frames
        \item It uses the same technique and information as the Garbage Collector (GC) uses to scan the values live on the stack
      \end{itemize}
      \begin{itemize}
        \item For each function frame detected, store a pointer to the \typename{frame\_descr} in the \localname{domain\_state->backtrace\_buffer} array, effectively constructing the backtrace
      \end{itemize}
      \onslide<2->{
        \begin{itemize}
            \smallskip
          \item Constructed backtrace:
            \funcname{definitely\_raise},
            \onslide<3->{\funcname{probably\_raise}}
        \end{itemize}
      }
      \vfill
      \mintocaml[firstline=9,lastline=13,highlightlines=12]{../src/backtrace/backtrace.ml}
      \endminipage
    \end{column}
    \begin{column}{0.5\textwidth}
      \raggedleft
      \onslide*<1>{\listStashBacktrace[highlightlines={114-115}]{}}
      \onslide*<2>{\providecommand\step{3}\input{diagrams/backtrace/backtrace.tex}}%
      \onslide*<3>{\providecommand\step{4}\input{diagrams/backtrace/backtrace.tex}}%
    \end{column}
  \end{columns}
\end{frame}

\begin{frame}{Backtraces}{Back to \funcname{caml\_raise\_exn}}
  \begin{columns}[c]
    \begin{column}{0.5\textwidth}
      \minipage[c][0.66\textheight][s]{\columnwidth}
      \begin{itemize}\color{gray}
        \item Checks wheteher exceptions backtrace should be recorded, by checking the \funcarg{domain\_state->backtrace\_active} value
        \item Switches to the C stack to be able to execute C code
        \item Calls \funcname{caml\_stash\_backtrace}, to collect the backtrace up to the next trap
      \end{itemize}
      \onslide*<2->{
        \begin{itemize}
          \item<2-> Executes the exception handler in \funcname{probably\_raise} with \funcname{RESTORE\_EXN\_HANDLER\_OCAML}
            \begin{itemize}
              \item Set \regname{RSP} to \localname{domain\_state->exn\_handler} value
              \item Restore \localname{domain\_state->exn\_handler} from trap
              \item Jump to the handler code with \funcname{ret}
            \end{itemize}
        \end{itemize}
      }
      \vfill
      \onslide*<1>{\mintocaml[firstline=9,lastline=13,highlightlines=12]{../src/backtrace/backtrace.ml}}
      \onslide*<2->{\mintocaml[firstline=15,lastline=21,highlightlines={19-21}]{../src/backtrace/backtrace.ml}}
      \endminipage
    \end{column}
    \begin{column}{0.5\textwidth}
      \centering
      \onslide*<1>{
        \mintc[firstline=190,lastline=192]{../ocaml/runtime/amd64.S}
        \mintc[firstline=234,lastline=237]{../ocaml/runtime/amd64.S}
      }
      \onslide*<2>{
        \mintc[firstline=190,lastline=192,highlightlines={191-192}]{../ocaml/runtime/amd64.S}
        \mintc[firstline=234,lastline=237,highlightlines={235-237}]{../ocaml/runtime/amd64.S}
      }
      \onslide*<1>{\listCamlRaiseExn[highlightlines=720]{}}
      \onslide*<2>{\listCamlRaiseExn[highlightlines=722]{}}
      \onslide*<3->{\providecommand\step{5}\input{diagrams/backtrace/backtrace.tex}}
    \end{column}
  \end{columns}
\end{frame}

\begin{frame}{Backtraces}{\funcname{caml\_probably\_raise}'s handler doesn't catch \typename{ExnC}}
  \begin{columns}[c]
    \begin{column}{0.45\textwidth}
      \minipage[c][0.75\textheight][s]{\columnwidth}
      \begin{itemize}
        \item<1-> \funcname{match \dots with}\footnotemark pattern matching can also match exceptions
        \item<1-> The CMM of \funcname{probably\_raise} shows that this \funcname{match \dots with} is implemented with a pattern matching and a \funcname{try \dots with}
        \item<1-> But this one only matches on \typename{ExnA} exception
        \item<2-> Since the pattern matching fails to catch the exception, \funcname{reraise} is called with our current \typename{ExnC} exception
        \item<3-> Disassembly shows that \funcname{reraise} is actually a call to \funcname{caml\_reraise\_exn}, which is also an assembly function provided by the runtime
      \end{itemize}
      \vfill
      \mintocaml[firstline=15,lastline=21,highlightlines={18-21}]{../src/backtrace/backtrace.ml}
      \endminipage
    \end{column}
    \begin{column}{0.55\textwidth}
      \centering
      \onslide*<1>{\mintcmm[highlightlines={10-18}]{../src/backtrace/camlBacktrace\_\_probably\_raise\_351.cmm}}
      \onslide*<2>{\listcmm[highlightlines=18]{../src/backtrace/camlBacktrace\_\_probably\_raise_351.cmm}{CMM of probably\_raise}}
      \onslide*<3>{\mintobjdump[firstline=5,lastline=35,highlightlines=31]{../src/backtrace/camlBacktrace\_\_probably\_raise_351.objdump}}
    \end{column}
  \end{columns}
  \only<1>{\footnotetext[1]{https://blog.janestreet.com/pattern-matching-and-exception-handling-unite/}}
\end{frame}

\newcommand\listCamlReraiseExn[1][]{
  \listasm[firstline=727,lastline=734,#1]{../ocaml/runtime/amd64.S}{runtime/amd64.S:727}}

\begin{frame}{Backtraces}{Introducing \funcname{caml\_reraise\_exn}}
  \begin{columns}[c]
    \begin{column}{0.5\textwidth}
      \minipage[c][0.66\textheight][s]{\columnwidth}
      \begin{itemize}
        \item<1-> The exception have to be reraised to the next handler
        \item<2-> First, checks if the backtrace should be collected with \funcarg{domain\_state->backtrace\_active}
        \only<3>{
        \begin{itemize}
            \item If not, behaves like a \funcname{raise\_notrace} with \funcname{RESTORE\_EXN\_HANDLER\_OCAML}
        \end{itemize}
        }
        \item<4-> Jumps into \funcname{caml\_raise\_exn}
        \item<5-> Calls \funcname{caml\_stash\_backtrace} again, to scan the next part of the stack: from the trap just pop-ed to the next trap
      \end{itemize}
      \vfill
      \mintocaml[firstline=15,lastline=21,highlightlines={18-21}]{../src/backtrace/backtrace.ml}
      \endminipage
    \end{column}
    \begin{column}{0.5\textwidth}
      \raggedleft
      \onslide*<1>{\listCamlReraiseExn{}}
      \onslide*<2>{\listCamlReraiseExn[highlightlines=729]{}}
      \onslide*<3>{
        \mintc[firstline=190,lastline=192,highlightlines={191-192}]{../ocaml/runtime/amd64.S}
        \mintc[firstline=234,lastline=237,highlightlines={235-237}]{../ocaml/runtime/amd64.S}
        \medskip
        \listCamlReraiseExn[highlightlines=731]{}
      }
      \onslide*<4-5>{
        % TODO refactor with \listCamlReraiseExn
        \mintasm[firstline=727,lastline=734,highlightlines=730]{../ocaml/runtime/amd64.S}
        \medskip
      }
      \onslide*<4>{\listCamlRaiseExn[highlightlines=711]{}}
      \onslide*<5>{\listCamlRaiseExn[highlightlines=720]{}}
    \end{column}
  \end{columns}
\end{frame}

\begin{frame}{Backtraces}{Backtrace collection continues}
  \begin{columns}[c]
    \begin{column}{0.55\textwidth}
      \minipage[c][0.75\textheight][s]{\columnwidth}
      \begin{itemize}
        \item<1-> \funcname{caml\_stash\_backtrace} scans the older part of the stack, up to the next trap
        \item<6-> Controls jumps to the nested exception handler in \funcname{maybe\_raise}, which matches only on \typename{ExnB}
        \item<7-> \funcname{caml\_reraise\_exn} is called again, backtrace collection resumes with \funcname{caml\_stash\_backtrace}
      \end{itemize}
      \medskip
      \begin{itemize}
        \item<2-> Constructed backtrace:
        \begin{itemize}
          \item <2-> \funcname{definitely\_raise}
          \item <2-> \funcname{probably\_raise}
          \item <3-> \funcname{probably\_raise} (re-raise)
          \item <4-> \funcname{maybe\_raise}
          \item <5-> \funcname{main}
          \item <8-> \funcname{main} (re-raise)
        \end{itemize}
      \end{itemize}
      \vfill
      \onslide*<1-5>{\mintocaml[firstline=15,lastline=21]{../src/backtrace/backtrace.ml}}
      \onslide*<6>{\mintocaml[firstline=28,lastline=34,highlightlines=31]{../src/backtrace/backtrace.ml}}
      \onslide*<7->{\mintocaml[firstline=28,lastline=34,highlightlines=33]{../src/backtrace/backtrace.ml}}
      \endminipage
    \end{column}
    \begin{column}{0.45\textwidth}
      \raggedleft
      \onslide*<1>{\providecommand\step{6}\input{diagrams/backtrace/backtrace.tex}}%
      \onslide*<2>{\providecommand\step{6}\input{diagrams/backtrace/backtrace.tex}}%
      \onslide*<3>{\providecommand\step{7}\input{diagrams/backtrace/backtrace.tex}}%
      \onslide*<4>{\providecommand\step{8}\input{diagrams/backtrace/backtrace.tex}}%
      \onslide*<5>{\providecommand\step{9}\input{diagrams/backtrace/backtrace.tex}}%
      \onslide*<6>{\providecommand\step{10}\input{diagrams/backtrace/backtrace.tex}}%
      \onslide*<7>{\providecommand\step{11}\input{diagrams/backtrace/backtrace.tex}}%
      \onslide*<8>{\providecommand\step{12}\input{diagrams/backtrace/backtrace.tex}}%
      \onslide*<9>{\providecommand\step{13}\input{diagrams/backtrace/backtrace.tex}}%
    \end{column}
  \end{columns}
\end{frame}

\begin{frame}{Backtraces}{Printing the backtrace}
  \begin{columns}[c]
    \begin{column}{0.45\textwidth}
      \minipage[c][0.66\textheight][s]{\columnwidth}
      \begin{itemize}
        \item<1-> The exception backtrace can now be printed to \funcarg{stdout} with \funcname{Printexc.print_backtrace}
        \item<2-> First the exception backtrace is retrieved into an OCaml array with \funcname{get_raw_backtrace }
        \item<3-> Which is implemented as the \funcname{caml_get_exception_raw_backtrace} C function in the runtime
        \item<4-> And finally printed with \funcname{print_exception_backtrace}
      \end{itemize}
      \vfill
      \mintocaml[firstline=28,lastline=34,highlightlines=33]{../src/backtrace/backtrace.ml}
      \endminipage
    \end{column}
    \begin{column}{0.55\textwidth}
      \onslide*<1,2>{
        \mintocaml[firstline=100,lastline=101]{../ocaml/stdlib/printexc.ml}
      }
      \only<1>{
        \listocaml[firstline=170,lastline=172]{../ocaml/stdlib/printexc.ml}{stdlib/printexc.ml:170}
      }
      \only<2>{
        \listocaml[firstline=170,lastline=172,highlightlines=172]{../ocaml/stdlib/printexc.ml}{stdlib/printexc.ml:170}
      }
      \only<3>{
        \mintc[firstline=154,lastline=156]{../ocaml/runtime/backtrace.c}
        \listc[firstline=171,lastline=192,highlightlines={184-187}]{../ocaml/runtime/backtrace.c}{runtime/backtrace.c:154}
      }
      \only<4>{
        \listocaml[firstline=155,lastline=165]{../ocaml/stdlib/printexc.ml}{stdlib/printexc.ml:55}
      }
    \end{column}
  \end{columns}
\end{frame}


\subsection{The multiple kinds of raise}
\frameSubsection{}{}

\begin{frame}{Backtraces}{When backtraces are not needed}
  \begin{columns}[c]
    \begin{column}{0.45\textwidth}
      \minipage[c][0.66\textheight][s]{\columnwidth}
      \begin{itemize}
        \item<1-> What if the backtrace for an exception is never needed?
        \item<2-> It's possible to raise the exception explicitly with \funcname{raise\_notrace} to make raising as cheap as possible
        \item<3-> An example of use in the \funcname{dynlink} library of the OCaml compiler
      \end{itemize}
      \vfill
      \onslide<3->{
      \listocaml[firstline=205,lastline=209,highlightlines=207]{../ocaml/otherlibs/dynlink/dynlink\_compilerlibs/misc.ml}{otherlibs/dynlink/dynlink\_compilerlibs/misc.ml:205}
      }
      \endminipage
    \end{column}
    \begin{column}{0.55\textwidth}
    \only<4>{
      \mintcmm[highlightlines=3]{../src/notrace/camlNotrace\_\_fun_347.cmm}
      \listcmm{../src/notrace/camlNotrace\_\_all\_somes\_327.cmm}{all\_somes}
    }
    \onslide*<5->{
      \listobjdump[highlightlines={6-9}]{../src/notrace/camlNotrace\_\_fun_347.objdump}{anonymous function from \funcname{all\_somes}}
    }
    \end{column}
  \end{columns}
\end{frame}

\begin{frame}{Backtraces}{Summary table of the multiple kinds of raise}
  \begin{itemize}
    \item When debugging information is enabled and if the backtrace of an exception isn't needed, it's possible to raise with \funcname{raise\_notrace} for performance
    \item When debugging information is \emph{not} enabled, the compiler optimises \funcname{raise} calls to inline assembly (same effect as using \funcname{raise\_notrace})
  \end{itemize}
  \bigskip
  \centering
  \begin{tabular}{| l | c | c | c | c |}
    \hline
    \parbox{10em}{Debug information \\ (\funcarg{-g} flag)}  & \multicolumn{2}{|c|}{Disabled}                                     & \multicolumn{2}{|c|}{Enabled} \\
    \hline
    Raise kind                                               & \funcname{raise} & \funcname{raise\_notrace}                       & \funcname{raise} & \funcname{raise\_notrace} \\
    \hline
    Compilation                                              & \parbox{8em}{inline assembly\\ (optimization)} & inline assembly   & \funcname{caml\_raise\_exn} & inline assembly \\
    \hline
  \end{tabular}
\end{frame}


%
% Takeaway #5
%
\subsection*{Takeaway \#5}
\frameSubsectionTakeaway{}
\againframe<5>{takeaway}
}%
      \onslide*<6>{\providecommand\step{2}%
% Backtraces
%
\subsection{One advantage of raising with debugging information}
\frameSubsection{}{}

\begin{frame}{Backtraces}{Debugging information \& source code}
  \begin{columns}[c]
    \begin{column}{0.5\textwidth}
      \begin{itemize}
        \item Raising an exception is fun and games, but how do we know where the exception originated? $\rightarrow$ With the backtrace associated with the exception!
        \item In order to get a backtrace from the point where an exception was raised, it's necessary to compile with debugging information enabled (\funcarg{-g} flag for the compiler)
        \item Same example as in the `Nested handlers' section
      \end{itemize}
    \end{column}
    \begin{column}{0.5\textwidth}
      \listocaml[firstline=9,lastline=34]{../src/backtrace/backtrace.ml}{backtrace/backtrace.ml}
    \end{column}
  \end{columns}
\end{frame}

\begin{frame}{Backtraces}{CMM and disassembly of \funcname{definitely\_raise}}
  \begin{columns}[c]
    \begin{column}{0.5\textwidth}
      \minipage[c][0.66\textheight][s]{\columnwidth}
      \begin{itemize}
        \item<1-> An effect of compiling with debugging information (\funcarg{-g}): the CMM no longer uses \funcname{raise\_notrace} to raise the exception but \funcname{raise} instead
        \item<2-> Disassembly shows that CMM \funcname{raise} is actually a call to \funcname{caml\_raise\_exn}, a function in assembler provided by the runtime
      \end{itemize}
      \vfill
      \mintocaml[firstline=9,lastline=13,highlightlines=12]{../src/backtrace/backtrace.ml}
      \endminipage
    \end{column}
    \begin{column}{0.5\textwidth}
      \onslide*<1>{
        \mintcmm[highlightlines=5]{../src/backtrace/camlBacktrace\_\_definitely\_raise\_347.cmm}
      }
      \onslide*<2>{
        \mintobjdump[firstline=2,highlightlines=14]{../src/backtrace/camlBacktrace\_\_definitely\_raise\_347.objdump}
      }
    \end{column}
  \end{columns}
\end{frame}

\begin{frame}{Backtraces}{Introducing \funcname{caml\_raise\_exn}}
  \begin{columns}[c]
    \begin{column}{0.5\textwidth}
      \minipage[c][0.66\textheight][s]{\columnwidth}
      \onslide*<1>{
        \begin{itemize}
          \item Raising the exception is performed using \funcname{caml\_raise\_exn} which is
            \begin{itemize}
              \item An assembly function provided by the runtime
              \item Architecture specific
            \end{itemize}
          \item Let's see what it does differently from \funcname{raise\_notrace}\dots
          \item We are about to raise \typename{ExnC} with \funcname{caml\_raise\_exn} from \funcname{definitely\_raise}
        \end{itemize}
      }
      \onslide*<2>{
        \mintobjdump[firstline=2,highlightlines=14]{../src/backtrace/camlBacktrace\_\_definitely\_raise\_347.objdump}
      }
      \vfill
      \mintocaml[firstline=9,lastline=13,highlightlines=12]{../src/backtrace/backtrace.ml}
      \endminipage
    \end{column}
    \begin{column}{0.5\textwidth}
      \centering
      \providecommand\step{1}\input{diagrams/backtrace/backtrace.tex}
    \end{column}
  \end{columns}
\end{frame}

\begin{frame}{Backtraces}{But before that, we need more fields from \typename{caml\_domain\_state}}
  \begin{columns}
    \begin{column}{0.5\textwidth}
      \begin{itemize}
        \item Remember \typename{caml\_domain\_state} the per-domain runtime state?
        \item Some other members are of interest to us now\dots
        \item \localname{backtrace\_active} is used to know if the runtime should collect backtraces
        \item \localname{backtrace\_active} can be set by
          \begin{itemize}
            \item The \funcarg{b} flag in the \funcname{OCAMLRUNPARAM} environment variable
            \item \mintinline{ocaml}{Printexc.record_backtrace : bool -> unit} \footnotemark
          \end{itemize}
      \end{itemize}
    \end{column}
    \begin{column}{0.5\textwidth}
      \begin{listing}
        \mintc[highlightlines={9-11}]{src/ocaml/domain\_state\_backtrace.h}
        \caption{runtime/caml/domain\_state.h}
      \end{listing}
    \end{column}
  \end{columns}
  \footnotetext[1]{https://v2.ocaml.org/api/Printexc.html}
\end{frame}

\newcommand\listCamlRaiseExn[1][]{
  \listasm[firstline=702,lastline=725,#1]{../ocaml/runtime/amd64.S}{runtime/amd64.S:702}}

\begin{frame}{Backtraces}{Walk through \funcname{caml\_raise\_exn}}
  \begin{columns}[T]
    \begin{column}{0.5\textwidth}
      \begin{itemize}
        \item<1-> First checks whether exceptions backtrace should be recorded
        \item<1-> By checking the \funcarg{domain\_state->backtrace\_active} value
        \item<2-> If not, acts as \funcname{raise\_notrace} with \funcname{RESTORE\_EXN\_HANDLER\_OCAML}
          \begin{itemize}
            \item Set \regname{RSP} to \localname{domain\_state->exn\_handler} value
            \item Restore \localname{domain\_state->exn\_handler} from trap
            \item Jump with \funcname{ret} (same effect as using \funcname{pop} and \funcname{jmp})
          \end{itemize}
        \item<3-> Otherwise, switches to the C stack to be able to execute C code
        \item<4-> Calls \funcname{caml\_stash\_backtrace}, a C function provided by the runtime\ignorespaces
          \onslide<6->{, with
            \begin{itemize}
              \item \funcarg{exn}: exception value
              \item \funcarg{pc}: code address of the raising point
              \item \funcarg{sp}: stack address of the raising function
              \item \funcarg{trapsp}: stack address of the next handler
            \end{itemize}
          }
      \end{itemize}
      \bigskip
      \mintocaml[firstline=9,lastline=13,highlightlines=12]{../src/backtrace/backtrace.ml}
    \end{column}
    \begin{column}{0.5\textwidth}
      \raggedleft
      \onslide*<1,3-4>{
        \mintc[firstline=190,lastline=192]{../ocaml/runtime/amd64.S}
        \mintc[firstline=234,lastline=237]{../ocaml/runtime/amd64.S}
      }
      \onslide*<2>{
        \mintc[firstline=190,lastline=192,highlightlines={191-192}]{../ocaml/runtime/amd64.S}
        \mintc[firstline=234,lastline=237,highlightlines={235-237}]{../ocaml/runtime/amd64.S}
      }
      \onslide*<1>{\listCamlRaiseExn[highlightlines=705]{}}%
      \onslide*<2>{\listCamlRaiseExn[highlightlines={707-708}]{}}%
      \onslide*<3>{\listCamlRaiseExn[highlightlines={712-714}]{}}%
      \onslide*<4>{\listCamlRaiseExn[highlightlines={715-720}]{}}%
      \onslide*<5>{\providecommand\step{1}\input{diagrams/backtrace/backtrace.tex}}%
      \onslide*<6>{\providecommand\step{2}\input{diagrams/backtrace/backtrace.tex}}%
    \end{column}
  \end{columns}
\end{frame}

\newcommand\listStashBacktrace[1][]{
  \listc[firstline=91,lastline=120,#1]{../ocaml/runtime/backtrace\_nat.c}{runtime/backtrace\_nat.c:91}}

\begin{frame}{Backtraces}{Introducing \funcname{caml\_stash\_backtrace}}
  \begin{columns}[c]
    \begin{column}{0.5\textwidth}
      \minipage[c][0.66\textheight][s]{\columnwidth}
      \begin{itemize}
        \item<1-> A C function provided by the runtime (specific to native code)
        \item<2-> It scans the stack, between the stack pointer at the raising point up to the next trap, in order to collect the names of the detected function frames
          \onslide*<2>{
            \begin{itemize}
              \item And more information, like line number, column number...
            \end{itemize}
          }
        \item<3-> It uses the same technique and information as the Garbage Collector (GC) uses to scan the values live on the stack
          \onslide*<4>{
            \begin{itemize}
              \item \typename{frame\_descr} are at the core of stack unwinding
            \end{itemize}
          }
      \end{itemize}
      \vfill
      \mintocaml[firstline=9,lastline=13,highlightlines=12]{../src/backtrace/backtrace.ml}
      \endminipage
    \end{column}
    \begin{column}{0.5\textwidth}
      \onslide*<1>{\listStashBacktrace[highlightlines=91]{}}
      \onslide*<2>{\listStashBacktrace[highlightlines={91,117-118}]{}}
      \onslide*<3->{\listStashBacktrace[highlightlines={94,106,109-110}]{}}
    \end{column}
  \end{columns}
\end{frame}

\newcommand\listFrameDescr[1][]{
  \listocaml[firstline=111,lastline=124,#1]{../ocaml/asmcomp/emitaux.ml}{asmcomp/emitaux.ml:111}}
\newcommand\listDebugInfo[1][]{
  \listocaml[firstline=38,lastline=49,#1]{../ocaml/lambda/debuginfo.mli}{lambda/debuginfo.mli:38}}

\begin{frame}{Backtraces}{\typename{frame\_descr} and \typename{frame\_debuginfo} in a nutshell}
  \begin{columns}[c]
    \begin{column}{0.5\textwidth}
      \begin{itemize}
        \item<1-> For every return address in an OCaml program, there is a associated \typename{frame\_descr}
        \item<2-> A \typename{frame\_descr} specifies
        \begin{itemize}
          \item<2-> The size of the function stack frame
          \item<3-> Where live values are on the stack (so that the GC can mark them as live and prevent from being swept)
        \end{itemize}
        \item<4-> When compiled with debug information, \typename{frame\_descr} will be linked to a \typename{frame\_debuginfo}
        \begin{itemize}
          \item<5-> Contains information about the position in the source code (file name, line and column number)
        \end{itemize}
      \end{itemize}
    \end{column}
    \begin{column}{0.5\textwidth}
        \onslide*<1>{\listFrameDescr[highlightlines={118-122}]{}}
        \onslide*<2>{\listFrameDescr[highlightlines={120}]{}}
        \onslide*<3>{\listFrameDescr[highlightlines={121}]{}}
        \onslide*<4->{\listFrameDescr[highlightlines={122}]{}}
        \medskip
        \onslide*<1-3>{\listDebugInfo{}}
        \onslide*<4>{\listDebugInfo[highlightlines={38,49}]{}}
        \onslide*<5>{\listDebugInfo[highlightlines={39-46}]{}}
    \end{column}
  \end{columns}
\end{frame}

\begin{frame}{Backtraces}{Scanning the stack with \typename{frame\_descr}}
  \begin{columns}[c]
    \begin{column}{0.5\textwidth}
      \begin{itemize}
        \item Given the return address of the code that raises the exception by calling \funcname{caml\_raise\_exn} (\funcarg{pc} argument of \funcname{caml\_stash\_backtrace})
        \item And the position of the stack pointer (\funcarg{sp} argument of \funcname{caml\_stash\_backtrace})
        \item The frame size given by \typename{frame\_descr}.\funcarg{frame\_size}, allows to find the end of the previous frame
        \item And the return address of the caller of this function
        \bigskip
        \onslide<3->{
        \item Note that traps (here the reddish \funcname{ExnA}, \funcname{ExnB} and \funcname{ExnC} blocks) belong to the frame that installed them, and so the frame size changes when traps are removed
        }
      \end{itemize}
    \end{column}
    \begin{column}{0.5\textwidth}
      \raggedleft
      \onslide*<1>{\providecommand\step{2}\input{diagrams/backtrace/backtrace.tex}}%
      \onslide*<2>{\providecommand\step{1}\input{diagrams/backtrace/scanstack.tex}}%
      \onslide*<3>{\providecommand\step{2}\input{diagrams/backtrace/scanstack.tex}}%
      \onslide*<4>{\providecommand\step{3}\input{diagrams/backtrace/scanstack.tex}}%
      \onslide*<5>{\providecommand\step{4}\input{diagrams/backtrace/scanstack.tex}}%
      \onslide*<6>{\providecommand\step{5}\input{diagrams/backtrace/scanstack.tex}}%
    \end{column}
  \end{columns}
\end{frame}

% Duplicate of previous-previous slide
\begin{frame}{Backtraces}{Continuing on \funcname{caml\_stash\_backtrace}}
  \begin{columns}[c]
    \begin{column}{0.5\textwidth}
      \minipage[c][0.75\textheight][s]{\columnwidth}
      \begin{itemize}
          \color{gray}
        \item A C function provided by the runtime (specific to native code)
        \item It scans the stack, between the stack pointer at the raising point up to the next trap, in order to collect the names of the detected function frames
        \item It uses the same technique and information as the Garbage Collector (GC) uses to scan the values live on the stack
      \end{itemize}
      \begin{itemize}
        \item For each function frame detected, store a pointer to the \typename{frame\_descr} in the \localname{domain\_state->backtrace\_buffer} array, effectively constructing the backtrace
      \end{itemize}
      \onslide<2->{
        \begin{itemize}
            \smallskip
          \item Constructed backtrace:
            \funcname{definitely\_raise},
            \onslide<3->{\funcname{probably\_raise}}
        \end{itemize}
      }
      \vfill
      \mintocaml[firstline=9,lastline=13,highlightlines=12]{../src/backtrace/backtrace.ml}
      \endminipage
    \end{column}
    \begin{column}{0.5\textwidth}
      \raggedleft
      \onslide*<1>{\listStashBacktrace[highlightlines={114-115}]{}}
      \onslide*<2>{\providecommand\step{3}\input{diagrams/backtrace/backtrace.tex}}%
      \onslide*<3>{\providecommand\step{4}\input{diagrams/backtrace/backtrace.tex}}%
    \end{column}
  \end{columns}
\end{frame}

\begin{frame}{Backtraces}{Back to \funcname{caml\_raise\_exn}}
  \begin{columns}[c]
    \begin{column}{0.5\textwidth}
      \minipage[c][0.66\textheight][s]{\columnwidth}
      \begin{itemize}\color{gray}
        \item Checks wheteher exceptions backtrace should be recorded, by checking the \funcarg{domain\_state->backtrace\_active} value
        \item Switches to the C stack to be able to execute C code
        \item Calls \funcname{caml\_stash\_backtrace}, to collect the backtrace up to the next trap
      \end{itemize}
      \onslide*<2->{
        \begin{itemize}
          \item<2-> Executes the exception handler in \funcname{probably\_raise} with \funcname{RESTORE\_EXN\_HANDLER\_OCAML}
            \begin{itemize}
              \item Set \regname{RSP} to \localname{domain\_state->exn\_handler} value
              \item Restore \localname{domain\_state->exn\_handler} from trap
              \item Jump to the handler code with \funcname{ret}
            \end{itemize}
        \end{itemize}
      }
      \vfill
      \onslide*<1>{\mintocaml[firstline=9,lastline=13,highlightlines=12]{../src/backtrace/backtrace.ml}}
      \onslide*<2->{\mintocaml[firstline=15,lastline=21,highlightlines={19-21}]{../src/backtrace/backtrace.ml}}
      \endminipage
    \end{column}
    \begin{column}{0.5\textwidth}
      \centering
      \onslide*<1>{
        \mintc[firstline=190,lastline=192]{../ocaml/runtime/amd64.S}
        \mintc[firstline=234,lastline=237]{../ocaml/runtime/amd64.S}
      }
      \onslide*<2>{
        \mintc[firstline=190,lastline=192,highlightlines={191-192}]{../ocaml/runtime/amd64.S}
        \mintc[firstline=234,lastline=237,highlightlines={235-237}]{../ocaml/runtime/amd64.S}
      }
      \onslide*<1>{\listCamlRaiseExn[highlightlines=720]{}}
      \onslide*<2>{\listCamlRaiseExn[highlightlines=722]{}}
      \onslide*<3->{\providecommand\step{5}\input{diagrams/backtrace/backtrace.tex}}
    \end{column}
  \end{columns}
\end{frame}

\begin{frame}{Backtraces}{\funcname{caml\_probably\_raise}'s handler doesn't catch \typename{ExnC}}
  \begin{columns}[c]
    \begin{column}{0.45\textwidth}
      \minipage[c][0.75\textheight][s]{\columnwidth}
      \begin{itemize}
        \item<1-> \funcname{match \dots with}\footnotemark pattern matching can also match exceptions
        \item<1-> The CMM of \funcname{probably\_raise} shows that this \funcname{match \dots with} is implemented with a pattern matching and a \funcname{try \dots with}
        \item<1-> But this one only matches on \typename{ExnA} exception
        \item<2-> Since the pattern matching fails to catch the exception, \funcname{reraise} is called with our current \typename{ExnC} exception
        \item<3-> Disassembly shows that \funcname{reraise} is actually a call to \funcname{caml\_reraise\_exn}, which is also an assembly function provided by the runtime
      \end{itemize}
      \vfill
      \mintocaml[firstline=15,lastline=21,highlightlines={18-21}]{../src/backtrace/backtrace.ml}
      \endminipage
    \end{column}
    \begin{column}{0.55\textwidth}
      \centering
      \onslide*<1>{\mintcmm[highlightlines={10-18}]{../src/backtrace/camlBacktrace\_\_probably\_raise\_351.cmm}}
      \onslide*<2>{\listcmm[highlightlines=18]{../src/backtrace/camlBacktrace\_\_probably\_raise_351.cmm}{CMM of probably\_raise}}
      \onslide*<3>{\mintobjdump[firstline=5,lastline=35,highlightlines=31]{../src/backtrace/camlBacktrace\_\_probably\_raise_351.objdump}}
    \end{column}
  \end{columns}
  \only<1>{\footnotetext[1]{https://blog.janestreet.com/pattern-matching-and-exception-handling-unite/}}
\end{frame}

\newcommand\listCamlReraiseExn[1][]{
  \listasm[firstline=727,lastline=734,#1]{../ocaml/runtime/amd64.S}{runtime/amd64.S:727}}

\begin{frame}{Backtraces}{Introducing \funcname{caml\_reraise\_exn}}
  \begin{columns}[c]
    \begin{column}{0.5\textwidth}
      \minipage[c][0.66\textheight][s]{\columnwidth}
      \begin{itemize}
        \item<1-> The exception have to be reraised to the next handler
        \item<2-> First, checks if the backtrace should be collected with \funcarg{domain\_state->backtrace\_active}
        \only<3>{
        \begin{itemize}
            \item If not, behaves like a \funcname{raise\_notrace} with \funcname{RESTORE\_EXN\_HANDLER\_OCAML}
        \end{itemize}
        }
        \item<4-> Jumps into \funcname{caml\_raise\_exn}
        \item<5-> Calls \funcname{caml\_stash\_backtrace} again, to scan the next part of the stack: from the trap just pop-ed to the next trap
      \end{itemize}
      \vfill
      \mintocaml[firstline=15,lastline=21,highlightlines={18-21}]{../src/backtrace/backtrace.ml}
      \endminipage
    \end{column}
    \begin{column}{0.5\textwidth}
      \raggedleft
      \onslide*<1>{\listCamlReraiseExn{}}
      \onslide*<2>{\listCamlReraiseExn[highlightlines=729]{}}
      \onslide*<3>{
        \mintc[firstline=190,lastline=192,highlightlines={191-192}]{../ocaml/runtime/amd64.S}
        \mintc[firstline=234,lastline=237,highlightlines={235-237}]{../ocaml/runtime/amd64.S}
        \medskip
        \listCamlReraiseExn[highlightlines=731]{}
      }
      \onslide*<4-5>{
        % TODO refactor with \listCamlReraiseExn
        \mintasm[firstline=727,lastline=734,highlightlines=730]{../ocaml/runtime/amd64.S}
        \medskip
      }
      \onslide*<4>{\listCamlRaiseExn[highlightlines=711]{}}
      \onslide*<5>{\listCamlRaiseExn[highlightlines=720]{}}
    \end{column}
  \end{columns}
\end{frame}

\begin{frame}{Backtraces}{Backtrace collection continues}
  \begin{columns}[c]
    \begin{column}{0.55\textwidth}
      \minipage[c][0.75\textheight][s]{\columnwidth}
      \begin{itemize}
        \item<1-> \funcname{caml\_stash\_backtrace} scans the older part of the stack, up to the next trap
        \item<6-> Controls jumps to the nested exception handler in \funcname{maybe\_raise}, which matches only on \typename{ExnB}
        \item<7-> \funcname{caml\_reraise\_exn} is called again, backtrace collection resumes with \funcname{caml\_stash\_backtrace}
      \end{itemize}
      \medskip
      \begin{itemize}
        \item<2-> Constructed backtrace:
        \begin{itemize}
          \item <2-> \funcname{definitely\_raise}
          \item <2-> \funcname{probably\_raise}
          \item <3-> \funcname{probably\_raise} (re-raise)
          \item <4-> \funcname{maybe\_raise}
          \item <5-> \funcname{main}
          \item <8-> \funcname{main} (re-raise)
        \end{itemize}
      \end{itemize}
      \vfill
      \onslide*<1-5>{\mintocaml[firstline=15,lastline=21]{../src/backtrace/backtrace.ml}}
      \onslide*<6>{\mintocaml[firstline=28,lastline=34,highlightlines=31]{../src/backtrace/backtrace.ml}}
      \onslide*<7->{\mintocaml[firstline=28,lastline=34,highlightlines=33]{../src/backtrace/backtrace.ml}}
      \endminipage
    \end{column}
    \begin{column}{0.45\textwidth}
      \raggedleft
      \onslide*<1>{\providecommand\step{6}\input{diagrams/backtrace/backtrace.tex}}%
      \onslide*<2>{\providecommand\step{6}\input{diagrams/backtrace/backtrace.tex}}%
      \onslide*<3>{\providecommand\step{7}\input{diagrams/backtrace/backtrace.tex}}%
      \onslide*<4>{\providecommand\step{8}\input{diagrams/backtrace/backtrace.tex}}%
      \onslide*<5>{\providecommand\step{9}\input{diagrams/backtrace/backtrace.tex}}%
      \onslide*<6>{\providecommand\step{10}\input{diagrams/backtrace/backtrace.tex}}%
      \onslide*<7>{\providecommand\step{11}\input{diagrams/backtrace/backtrace.tex}}%
      \onslide*<8>{\providecommand\step{12}\input{diagrams/backtrace/backtrace.tex}}%
      \onslide*<9>{\providecommand\step{13}\input{diagrams/backtrace/backtrace.tex}}%
    \end{column}
  \end{columns}
\end{frame}

\begin{frame}{Backtraces}{Printing the backtrace}
  \begin{columns}[c]
    \begin{column}{0.45\textwidth}
      \minipage[c][0.66\textheight][s]{\columnwidth}
      \begin{itemize}
        \item<1-> The exception backtrace can now be printed to \funcarg{stdout} with \funcname{Printexc.print_backtrace}
        \item<2-> First the exception backtrace is retrieved into an OCaml array with \funcname{get_raw_backtrace }
        \item<3-> Which is implemented as the \funcname{caml_get_exception_raw_backtrace} C function in the runtime
        \item<4-> And finally printed with \funcname{print_exception_backtrace}
      \end{itemize}
      \vfill
      \mintocaml[firstline=28,lastline=34,highlightlines=33]{../src/backtrace/backtrace.ml}
      \endminipage
    \end{column}
    \begin{column}{0.55\textwidth}
      \onslide*<1,2>{
        \mintocaml[firstline=100,lastline=101]{../ocaml/stdlib/printexc.ml}
      }
      \only<1>{
        \listocaml[firstline=170,lastline=172]{../ocaml/stdlib/printexc.ml}{stdlib/printexc.ml:170}
      }
      \only<2>{
        \listocaml[firstline=170,lastline=172,highlightlines=172]{../ocaml/stdlib/printexc.ml}{stdlib/printexc.ml:170}
      }
      \only<3>{
        \mintc[firstline=154,lastline=156]{../ocaml/runtime/backtrace.c}
        \listc[firstline=171,lastline=192,highlightlines={184-187}]{../ocaml/runtime/backtrace.c}{runtime/backtrace.c:154}
      }
      \only<4>{
        \listocaml[firstline=155,lastline=165]{../ocaml/stdlib/printexc.ml}{stdlib/printexc.ml:55}
      }
    \end{column}
  \end{columns}
\end{frame}


\subsection{The multiple kinds of raise}
\frameSubsection{}{}

\begin{frame}{Backtraces}{When backtraces are not needed}
  \begin{columns}[c]
    \begin{column}{0.45\textwidth}
      \minipage[c][0.66\textheight][s]{\columnwidth}
      \begin{itemize}
        \item<1-> What if the backtrace for an exception is never needed?
        \item<2-> It's possible to raise the exception explicitly with \funcname{raise\_notrace} to make raising as cheap as possible
        \item<3-> An example of use in the \funcname{dynlink} library of the OCaml compiler
      \end{itemize}
      \vfill
      \onslide<3->{
      \listocaml[firstline=205,lastline=209,highlightlines=207]{../ocaml/otherlibs/dynlink/dynlink\_compilerlibs/misc.ml}{otherlibs/dynlink/dynlink\_compilerlibs/misc.ml:205}
      }
      \endminipage
    \end{column}
    \begin{column}{0.55\textwidth}
    \only<4>{
      \mintcmm[highlightlines=3]{../src/notrace/camlNotrace\_\_fun_347.cmm}
      \listcmm{../src/notrace/camlNotrace\_\_all\_somes\_327.cmm}{all\_somes}
    }
    \onslide*<5->{
      \listobjdump[highlightlines={6-9}]{../src/notrace/camlNotrace\_\_fun_347.objdump}{anonymous function from \funcname{all\_somes}}
    }
    \end{column}
  \end{columns}
\end{frame}

\begin{frame}{Backtraces}{Summary table of the multiple kinds of raise}
  \begin{itemize}
    \item When debugging information is enabled and if the backtrace of an exception isn't needed, it's possible to raise with \funcname{raise\_notrace} for performance
    \item When debugging information is \emph{not} enabled, the compiler optimises \funcname{raise} calls to inline assembly (same effect as using \funcname{raise\_notrace})
  \end{itemize}
  \bigskip
  \centering
  \begin{tabular}{| l | c | c | c | c |}
    \hline
    \parbox{10em}{Debug information \\ (\funcarg{-g} flag)}  & \multicolumn{2}{|c|}{Disabled}                                     & \multicolumn{2}{|c|}{Enabled} \\
    \hline
    Raise kind                                               & \funcname{raise} & \funcname{raise\_notrace}                       & \funcname{raise} & \funcname{raise\_notrace} \\
    \hline
    Compilation                                              & \parbox{8em}{inline assembly\\ (optimization)} & inline assembly   & \funcname{caml\_raise\_exn} & inline assembly \\
    \hline
  \end{tabular}
\end{frame}


%
% Takeaway #5
%
\subsection*{Takeaway \#5}
\frameSubsectionTakeaway{}
\againframe<5>{takeaway}
}%
    \end{column}
  \end{columns}
\end{frame}

\newcommand\listStashBacktrace[1][]{
  \listc[firstline=91,lastline=120,#1]{../ocaml/runtime/backtrace\_nat.c}{runtime/backtrace\_nat.c:91}}

\begin{frame}{Backtraces}{Introducing \funcname{caml\_stash\_backtrace}}
  \begin{columns}[c]
    \begin{column}{0.5\textwidth}
      \minipage[c][0.66\textheight][s]{\columnwidth}
      \begin{itemize}
        \item<1-> A C function provided by the runtime (specific to native code)
        \item<2-> It scans the stack, between the stack pointer at the raising point up to the next trap, in order to collect the names of the detected function frames
          \onslide*<2>{
            \begin{itemize}
              \item And more information, like line number, column number...
            \end{itemize}
          }
        \item<3-> It uses the same technique and information as the Garbage Collector (GC) uses to scan the values live on the stack
          \onslide*<4>{
            \begin{itemize}
              \item \typename{frame\_descr} are at the core of stack unwinding
            \end{itemize}
          }
      \end{itemize}
      \vfill
      \mintocaml[firstline=9,lastline=13,highlightlines=12]{../src/backtrace/backtrace.ml}
      \endminipage
    \end{column}
    \begin{column}{0.5\textwidth}
      \onslide*<1>{\listStashBacktrace[highlightlines=91]{}}
      \onslide*<2>{\listStashBacktrace[highlightlines={91,117-118}]{}}
      \onslide*<3->{\listStashBacktrace[highlightlines={94,106,109-110}]{}}
    \end{column}
  \end{columns}
\end{frame}

\newcommand\listFrameDescr[1][]{
  \listocaml[firstline=111,lastline=124,#1]{../ocaml/asmcomp/emitaux.ml}{asmcomp/emitaux.ml:111}}
\newcommand\listDebugInfo[1][]{
  \listocaml[firstline=38,lastline=49,#1]{../ocaml/lambda/debuginfo.mli}{lambda/debuginfo.mli:38}}

\begin{frame}{Backtraces}{\typename{frame\_descr} and \typename{frame\_debuginfo} in a nutshell}
  \begin{columns}[c]
    \begin{column}{0.5\textwidth}
      \begin{itemize}
        \item<1-> For every return address in an OCaml program, there is a associated \typename{frame\_descr}
        \item<2-> A \typename{frame\_descr} specifies
        \begin{itemize}
          \item<2-> The size of the function stack frame
          \item<3-> Where live values are on the stack (so that the GC can mark them as live and prevent from being swept)
        \end{itemize}
        \item<4-> When compiled with debug information, \typename{frame\_descr} will be linked to a \typename{frame\_debuginfo}
        \begin{itemize}
          \item<5-> Contains information about the position in the source code (file name, line and column number)
        \end{itemize}
      \end{itemize}
    \end{column}
    \begin{column}{0.5\textwidth}
        \onslide*<1>{\listFrameDescr[highlightlines={118-122}]{}}
        \onslide*<2>{\listFrameDescr[highlightlines={120}]{}}
        \onslide*<3>{\listFrameDescr[highlightlines={121}]{}}
        \onslide*<4->{\listFrameDescr[highlightlines={122}]{}}
        \medskip
        \onslide*<1-3>{\listDebugInfo{}}
        \onslide*<4>{\listDebugInfo[highlightlines={38,49}]{}}
        \onslide*<5>{\listDebugInfo[highlightlines={39-46}]{}}
    \end{column}
  \end{columns}
\end{frame}

\begin{frame}{Backtraces}{Scanning the stack with \typename{frame\_descr}}
  \begin{columns}[c]
    \begin{column}{0.5\textwidth}
      \begin{itemize}
        \item Given the return address of the code that raises the exception by calling \funcname{caml\_raise\_exn} (\funcarg{pc} argument of \funcname{caml\_stash\_backtrace})
        \item And the position of the stack pointer (\funcarg{sp} argument of \funcname{caml\_stash\_backtrace})
        \item The frame size given by \typename{frame\_descr}.\funcarg{frame\_size}, allows to find the end of the previous frame
        \item And the return address of the caller of this function
        \bigskip
        \onslide<3->{
        \item Note that traps (here the reddish \funcname{ExnA}, \funcname{ExnB} and \funcname{ExnC} blocks) belong to the frame that installed them, and so the frame size changes when traps are removed
        }
      \end{itemize}
    \end{column}
    \begin{column}{0.5\textwidth}
      \raggedleft
      \onslide*<1>{\providecommand\step{2}%
% Backtraces
%
\subsection{One advantage of raising with debugging information}
\frameSubsection{}{}

\begin{frame}{Backtraces}{Debugging information \& source code}
  \begin{columns}[c]
    \begin{column}{0.5\textwidth}
      \begin{itemize}
        \item Raising an exception is fun and games, but how do we know where the exception originated? $\rightarrow$ With the backtrace associated with the exception!
        \item In order to get a backtrace from the point where an exception was raised, it's necessary to compile with debugging information enabled (\funcarg{-g} flag for the compiler)
        \item Same example as in the `Nested handlers' section
      \end{itemize}
    \end{column}
    \begin{column}{0.5\textwidth}
      \listocaml[firstline=9,lastline=34]{../src/backtrace/backtrace.ml}{backtrace/backtrace.ml}
    \end{column}
  \end{columns}
\end{frame}

\begin{frame}{Backtraces}{CMM and disassembly of \funcname{definitely\_raise}}
  \begin{columns}[c]
    \begin{column}{0.5\textwidth}
      \minipage[c][0.66\textheight][s]{\columnwidth}
      \begin{itemize}
        \item<1-> An effect of compiling with debugging information (\funcarg{-g}): the CMM no longer uses \funcname{raise\_notrace} to raise the exception but \funcname{raise} instead
        \item<2-> Disassembly shows that CMM \funcname{raise} is actually a call to \funcname{caml\_raise\_exn}, a function in assembler provided by the runtime
      \end{itemize}
      \vfill
      \mintocaml[firstline=9,lastline=13,highlightlines=12]{../src/backtrace/backtrace.ml}
      \endminipage
    \end{column}
    \begin{column}{0.5\textwidth}
      \onslide*<1>{
        \mintcmm[highlightlines=5]{../src/backtrace/camlBacktrace\_\_definitely\_raise\_347.cmm}
      }
      \onslide*<2>{
        \mintobjdump[firstline=2,highlightlines=14]{../src/backtrace/camlBacktrace\_\_definitely\_raise\_347.objdump}
      }
    \end{column}
  \end{columns}
\end{frame}

\begin{frame}{Backtraces}{Introducing \funcname{caml\_raise\_exn}}
  \begin{columns}[c]
    \begin{column}{0.5\textwidth}
      \minipage[c][0.66\textheight][s]{\columnwidth}
      \onslide*<1>{
        \begin{itemize}
          \item Raising the exception is performed using \funcname{caml\_raise\_exn} which is
            \begin{itemize}
              \item An assembly function provided by the runtime
              \item Architecture specific
            \end{itemize}
          \item Let's see what it does differently from \funcname{raise\_notrace}\dots
          \item We are about to raise \typename{ExnC} with \funcname{caml\_raise\_exn} from \funcname{definitely\_raise}
        \end{itemize}
      }
      \onslide*<2>{
        \mintobjdump[firstline=2,highlightlines=14]{../src/backtrace/camlBacktrace\_\_definitely\_raise\_347.objdump}
      }
      \vfill
      \mintocaml[firstline=9,lastline=13,highlightlines=12]{../src/backtrace/backtrace.ml}
      \endminipage
    \end{column}
    \begin{column}{0.5\textwidth}
      \centering
      \providecommand\step{1}\input{diagrams/backtrace/backtrace.tex}
    \end{column}
  \end{columns}
\end{frame}

\begin{frame}{Backtraces}{But before that, we need more fields from \typename{caml\_domain\_state}}
  \begin{columns}
    \begin{column}{0.5\textwidth}
      \begin{itemize}
        \item Remember \typename{caml\_domain\_state} the per-domain runtime state?
        \item Some other members are of interest to us now\dots
        \item \localname{backtrace\_active} is used to know if the runtime should collect backtraces
        \item \localname{backtrace\_active} can be set by
          \begin{itemize}
            \item The \funcarg{b} flag in the \funcname{OCAMLRUNPARAM} environment variable
            \item \mintinline{ocaml}{Printexc.record_backtrace : bool -> unit} \footnotemark
          \end{itemize}
      \end{itemize}
    \end{column}
    \begin{column}{0.5\textwidth}
      \begin{listing}
        \mintc[highlightlines={9-11}]{src/ocaml/domain\_state\_backtrace.h}
        \caption{runtime/caml/domain\_state.h}
      \end{listing}
    \end{column}
  \end{columns}
  \footnotetext[1]{https://v2.ocaml.org/api/Printexc.html}
\end{frame}

\newcommand\listCamlRaiseExn[1][]{
  \listasm[firstline=702,lastline=725,#1]{../ocaml/runtime/amd64.S}{runtime/amd64.S:702}}

\begin{frame}{Backtraces}{Walk through \funcname{caml\_raise\_exn}}
  \begin{columns}[T]
    \begin{column}{0.5\textwidth}
      \begin{itemize}
        \item<1-> First checks whether exceptions backtrace should be recorded
        \item<1-> By checking the \funcarg{domain\_state->backtrace\_active} value
        \item<2-> If not, acts as \funcname{raise\_notrace} with \funcname{RESTORE\_EXN\_HANDLER\_OCAML}
          \begin{itemize}
            \item Set \regname{RSP} to \localname{domain\_state->exn\_handler} value
            \item Restore \localname{domain\_state->exn\_handler} from trap
            \item Jump with \funcname{ret} (same effect as using \funcname{pop} and \funcname{jmp})
          \end{itemize}
        \item<3-> Otherwise, switches to the C stack to be able to execute C code
        \item<4-> Calls \funcname{caml\_stash\_backtrace}, a C function provided by the runtime\ignorespaces
          \onslide<6->{, with
            \begin{itemize}
              \item \funcarg{exn}: exception value
              \item \funcarg{pc}: code address of the raising point
              \item \funcarg{sp}: stack address of the raising function
              \item \funcarg{trapsp}: stack address of the next handler
            \end{itemize}
          }
      \end{itemize}
      \bigskip
      \mintocaml[firstline=9,lastline=13,highlightlines=12]{../src/backtrace/backtrace.ml}
    \end{column}
    \begin{column}{0.5\textwidth}
      \raggedleft
      \onslide*<1,3-4>{
        \mintc[firstline=190,lastline=192]{../ocaml/runtime/amd64.S}
        \mintc[firstline=234,lastline=237]{../ocaml/runtime/amd64.S}
      }
      \onslide*<2>{
        \mintc[firstline=190,lastline=192,highlightlines={191-192}]{../ocaml/runtime/amd64.S}
        \mintc[firstline=234,lastline=237,highlightlines={235-237}]{../ocaml/runtime/amd64.S}
      }
      \onslide*<1>{\listCamlRaiseExn[highlightlines=705]{}}%
      \onslide*<2>{\listCamlRaiseExn[highlightlines={707-708}]{}}%
      \onslide*<3>{\listCamlRaiseExn[highlightlines={712-714}]{}}%
      \onslide*<4>{\listCamlRaiseExn[highlightlines={715-720}]{}}%
      \onslide*<5>{\providecommand\step{1}\input{diagrams/backtrace/backtrace.tex}}%
      \onslide*<6>{\providecommand\step{2}\input{diagrams/backtrace/backtrace.tex}}%
    \end{column}
  \end{columns}
\end{frame}

\newcommand\listStashBacktrace[1][]{
  \listc[firstline=91,lastline=120,#1]{../ocaml/runtime/backtrace\_nat.c}{runtime/backtrace\_nat.c:91}}

\begin{frame}{Backtraces}{Introducing \funcname{caml\_stash\_backtrace}}
  \begin{columns}[c]
    \begin{column}{0.5\textwidth}
      \minipage[c][0.66\textheight][s]{\columnwidth}
      \begin{itemize}
        \item<1-> A C function provided by the runtime (specific to native code)
        \item<2-> It scans the stack, between the stack pointer at the raising point up to the next trap, in order to collect the names of the detected function frames
          \onslide*<2>{
            \begin{itemize}
              \item And more information, like line number, column number...
            \end{itemize}
          }
        \item<3-> It uses the same technique and information as the Garbage Collector (GC) uses to scan the values live on the stack
          \onslide*<4>{
            \begin{itemize}
              \item \typename{frame\_descr} are at the core of stack unwinding
            \end{itemize}
          }
      \end{itemize}
      \vfill
      \mintocaml[firstline=9,lastline=13,highlightlines=12]{../src/backtrace/backtrace.ml}
      \endminipage
    \end{column}
    \begin{column}{0.5\textwidth}
      \onslide*<1>{\listStashBacktrace[highlightlines=91]{}}
      \onslide*<2>{\listStashBacktrace[highlightlines={91,117-118}]{}}
      \onslide*<3->{\listStashBacktrace[highlightlines={94,106,109-110}]{}}
    \end{column}
  \end{columns}
\end{frame}

\newcommand\listFrameDescr[1][]{
  \listocaml[firstline=111,lastline=124,#1]{../ocaml/asmcomp/emitaux.ml}{asmcomp/emitaux.ml:111}}
\newcommand\listDebugInfo[1][]{
  \listocaml[firstline=38,lastline=49,#1]{../ocaml/lambda/debuginfo.mli}{lambda/debuginfo.mli:38}}

\begin{frame}{Backtraces}{\typename{frame\_descr} and \typename{frame\_debuginfo} in a nutshell}
  \begin{columns}[c]
    \begin{column}{0.5\textwidth}
      \begin{itemize}
        \item<1-> For every return address in an OCaml program, there is a associated \typename{frame\_descr}
        \item<2-> A \typename{frame\_descr} specifies
        \begin{itemize}
          \item<2-> The size of the function stack frame
          \item<3-> Where live values are on the stack (so that the GC can mark them as live and prevent from being swept)
        \end{itemize}
        \item<4-> When compiled with debug information, \typename{frame\_descr} will be linked to a \typename{frame\_debuginfo}
        \begin{itemize}
          \item<5-> Contains information about the position in the source code (file name, line and column number)
        \end{itemize}
      \end{itemize}
    \end{column}
    \begin{column}{0.5\textwidth}
        \onslide*<1>{\listFrameDescr[highlightlines={118-122}]{}}
        \onslide*<2>{\listFrameDescr[highlightlines={120}]{}}
        \onslide*<3>{\listFrameDescr[highlightlines={121}]{}}
        \onslide*<4->{\listFrameDescr[highlightlines={122}]{}}
        \medskip
        \onslide*<1-3>{\listDebugInfo{}}
        \onslide*<4>{\listDebugInfo[highlightlines={38,49}]{}}
        \onslide*<5>{\listDebugInfo[highlightlines={39-46}]{}}
    \end{column}
  \end{columns}
\end{frame}

\begin{frame}{Backtraces}{Scanning the stack with \typename{frame\_descr}}
  \begin{columns}[c]
    \begin{column}{0.5\textwidth}
      \begin{itemize}
        \item Given the return address of the code that raises the exception by calling \funcname{caml\_raise\_exn} (\funcarg{pc} argument of \funcname{caml\_stash\_backtrace})
        \item And the position of the stack pointer (\funcarg{sp} argument of \funcname{caml\_stash\_backtrace})
        \item The frame size given by \typename{frame\_descr}.\funcarg{frame\_size}, allows to find the end of the previous frame
        \item And the return address of the caller of this function
        \bigskip
        \onslide<3->{
        \item Note that traps (here the reddish \funcname{ExnA}, \funcname{ExnB} and \funcname{ExnC} blocks) belong to the frame that installed them, and so the frame size changes when traps are removed
        }
      \end{itemize}
    \end{column}
    \begin{column}{0.5\textwidth}
      \raggedleft
      \onslide*<1>{\providecommand\step{2}\input{diagrams/backtrace/backtrace.tex}}%
      \onslide*<2>{\providecommand\step{1}\input{diagrams/backtrace/scanstack.tex}}%
      \onslide*<3>{\providecommand\step{2}\input{diagrams/backtrace/scanstack.tex}}%
      \onslide*<4>{\providecommand\step{3}\input{diagrams/backtrace/scanstack.tex}}%
      \onslide*<5>{\providecommand\step{4}\input{diagrams/backtrace/scanstack.tex}}%
      \onslide*<6>{\providecommand\step{5}\input{diagrams/backtrace/scanstack.tex}}%
    \end{column}
  \end{columns}
\end{frame}

% Duplicate of previous-previous slide
\begin{frame}{Backtraces}{Continuing on \funcname{caml\_stash\_backtrace}}
  \begin{columns}[c]
    \begin{column}{0.5\textwidth}
      \minipage[c][0.75\textheight][s]{\columnwidth}
      \begin{itemize}
          \color{gray}
        \item A C function provided by the runtime (specific to native code)
        \item It scans the stack, between the stack pointer at the raising point up to the next trap, in order to collect the names of the detected function frames
        \item It uses the same technique and information as the Garbage Collector (GC) uses to scan the values live on the stack
      \end{itemize}
      \begin{itemize}
        \item For each function frame detected, store a pointer to the \typename{frame\_descr} in the \localname{domain\_state->backtrace\_buffer} array, effectively constructing the backtrace
      \end{itemize}
      \onslide<2->{
        \begin{itemize}
            \smallskip
          \item Constructed backtrace:
            \funcname{definitely\_raise},
            \onslide<3->{\funcname{probably\_raise}}
        \end{itemize}
      }
      \vfill
      \mintocaml[firstline=9,lastline=13,highlightlines=12]{../src/backtrace/backtrace.ml}
      \endminipage
    \end{column}
    \begin{column}{0.5\textwidth}
      \raggedleft
      \onslide*<1>{\listStashBacktrace[highlightlines={114-115}]{}}
      \onslide*<2>{\providecommand\step{3}\input{diagrams/backtrace/backtrace.tex}}%
      \onslide*<3>{\providecommand\step{4}\input{diagrams/backtrace/backtrace.tex}}%
    \end{column}
  \end{columns}
\end{frame}

\begin{frame}{Backtraces}{Back to \funcname{caml\_raise\_exn}}
  \begin{columns}[c]
    \begin{column}{0.5\textwidth}
      \minipage[c][0.66\textheight][s]{\columnwidth}
      \begin{itemize}\color{gray}
        \item Checks wheteher exceptions backtrace should be recorded, by checking the \funcarg{domain\_state->backtrace\_active} value
        \item Switches to the C stack to be able to execute C code
        \item Calls \funcname{caml\_stash\_backtrace}, to collect the backtrace up to the next trap
      \end{itemize}
      \onslide*<2->{
        \begin{itemize}
          \item<2-> Executes the exception handler in \funcname{probably\_raise} with \funcname{RESTORE\_EXN\_HANDLER\_OCAML}
            \begin{itemize}
              \item Set \regname{RSP} to \localname{domain\_state->exn\_handler} value
              \item Restore \localname{domain\_state->exn\_handler} from trap
              \item Jump to the handler code with \funcname{ret}
            \end{itemize}
        \end{itemize}
      }
      \vfill
      \onslide*<1>{\mintocaml[firstline=9,lastline=13,highlightlines=12]{../src/backtrace/backtrace.ml}}
      \onslide*<2->{\mintocaml[firstline=15,lastline=21,highlightlines={19-21}]{../src/backtrace/backtrace.ml}}
      \endminipage
    \end{column}
    \begin{column}{0.5\textwidth}
      \centering
      \onslide*<1>{
        \mintc[firstline=190,lastline=192]{../ocaml/runtime/amd64.S}
        \mintc[firstline=234,lastline=237]{../ocaml/runtime/amd64.S}
      }
      \onslide*<2>{
        \mintc[firstline=190,lastline=192,highlightlines={191-192}]{../ocaml/runtime/amd64.S}
        \mintc[firstline=234,lastline=237,highlightlines={235-237}]{../ocaml/runtime/amd64.S}
      }
      \onslide*<1>{\listCamlRaiseExn[highlightlines=720]{}}
      \onslide*<2>{\listCamlRaiseExn[highlightlines=722]{}}
      \onslide*<3->{\providecommand\step{5}\input{diagrams/backtrace/backtrace.tex}}
    \end{column}
  \end{columns}
\end{frame}

\begin{frame}{Backtraces}{\funcname{caml\_probably\_raise}'s handler doesn't catch \typename{ExnC}}
  \begin{columns}[c]
    \begin{column}{0.45\textwidth}
      \minipage[c][0.75\textheight][s]{\columnwidth}
      \begin{itemize}
        \item<1-> \funcname{match \dots with}\footnotemark pattern matching can also match exceptions
        \item<1-> The CMM of \funcname{probably\_raise} shows that this \funcname{match \dots with} is implemented with a pattern matching and a \funcname{try \dots with}
        \item<1-> But this one only matches on \typename{ExnA} exception
        \item<2-> Since the pattern matching fails to catch the exception, \funcname{reraise} is called with our current \typename{ExnC} exception
        \item<3-> Disassembly shows that \funcname{reraise} is actually a call to \funcname{caml\_reraise\_exn}, which is also an assembly function provided by the runtime
      \end{itemize}
      \vfill
      \mintocaml[firstline=15,lastline=21,highlightlines={18-21}]{../src/backtrace/backtrace.ml}
      \endminipage
    \end{column}
    \begin{column}{0.55\textwidth}
      \centering
      \onslide*<1>{\mintcmm[highlightlines={10-18}]{../src/backtrace/camlBacktrace\_\_probably\_raise\_351.cmm}}
      \onslide*<2>{\listcmm[highlightlines=18]{../src/backtrace/camlBacktrace\_\_probably\_raise_351.cmm}{CMM of probably\_raise}}
      \onslide*<3>{\mintobjdump[firstline=5,lastline=35,highlightlines=31]{../src/backtrace/camlBacktrace\_\_probably\_raise_351.objdump}}
    \end{column}
  \end{columns}
  \only<1>{\footnotetext[1]{https://blog.janestreet.com/pattern-matching-and-exception-handling-unite/}}
\end{frame}

\newcommand\listCamlReraiseExn[1][]{
  \listasm[firstline=727,lastline=734,#1]{../ocaml/runtime/amd64.S}{runtime/amd64.S:727}}

\begin{frame}{Backtraces}{Introducing \funcname{caml\_reraise\_exn}}
  \begin{columns}[c]
    \begin{column}{0.5\textwidth}
      \minipage[c][0.66\textheight][s]{\columnwidth}
      \begin{itemize}
        \item<1-> The exception have to be reraised to the next handler
        \item<2-> First, checks if the backtrace should be collected with \funcarg{domain\_state->backtrace\_active}
        \only<3>{
        \begin{itemize}
            \item If not, behaves like a \funcname{raise\_notrace} with \funcname{RESTORE\_EXN\_HANDLER\_OCAML}
        \end{itemize}
        }
        \item<4-> Jumps into \funcname{caml\_raise\_exn}
        \item<5-> Calls \funcname{caml\_stash\_backtrace} again, to scan the next part of the stack: from the trap just pop-ed to the next trap
      \end{itemize}
      \vfill
      \mintocaml[firstline=15,lastline=21,highlightlines={18-21}]{../src/backtrace/backtrace.ml}
      \endminipage
    \end{column}
    \begin{column}{0.5\textwidth}
      \raggedleft
      \onslide*<1>{\listCamlReraiseExn{}}
      \onslide*<2>{\listCamlReraiseExn[highlightlines=729]{}}
      \onslide*<3>{
        \mintc[firstline=190,lastline=192,highlightlines={191-192}]{../ocaml/runtime/amd64.S}
        \mintc[firstline=234,lastline=237,highlightlines={235-237}]{../ocaml/runtime/amd64.S}
        \medskip
        \listCamlReraiseExn[highlightlines=731]{}
      }
      \onslide*<4-5>{
        % TODO refactor with \listCamlReraiseExn
        \mintasm[firstline=727,lastline=734,highlightlines=730]{../ocaml/runtime/amd64.S}
        \medskip
      }
      \onslide*<4>{\listCamlRaiseExn[highlightlines=711]{}}
      \onslide*<5>{\listCamlRaiseExn[highlightlines=720]{}}
    \end{column}
  \end{columns}
\end{frame}

\begin{frame}{Backtraces}{Backtrace collection continues}
  \begin{columns}[c]
    \begin{column}{0.55\textwidth}
      \minipage[c][0.75\textheight][s]{\columnwidth}
      \begin{itemize}
        \item<1-> \funcname{caml\_stash\_backtrace} scans the older part of the stack, up to the next trap
        \item<6-> Controls jumps to the nested exception handler in \funcname{maybe\_raise}, which matches only on \typename{ExnB}
        \item<7-> \funcname{caml\_reraise\_exn} is called again, backtrace collection resumes with \funcname{caml\_stash\_backtrace}
      \end{itemize}
      \medskip
      \begin{itemize}
        \item<2-> Constructed backtrace:
        \begin{itemize}
          \item <2-> \funcname{definitely\_raise}
          \item <2-> \funcname{probably\_raise}
          \item <3-> \funcname{probably\_raise} (re-raise)
          \item <4-> \funcname{maybe\_raise}
          \item <5-> \funcname{main}
          \item <8-> \funcname{main} (re-raise)
        \end{itemize}
      \end{itemize}
      \vfill
      \onslide*<1-5>{\mintocaml[firstline=15,lastline=21]{../src/backtrace/backtrace.ml}}
      \onslide*<6>{\mintocaml[firstline=28,lastline=34,highlightlines=31]{../src/backtrace/backtrace.ml}}
      \onslide*<7->{\mintocaml[firstline=28,lastline=34,highlightlines=33]{../src/backtrace/backtrace.ml}}
      \endminipage
    \end{column}
    \begin{column}{0.45\textwidth}
      \raggedleft
      \onslide*<1>{\providecommand\step{6}\input{diagrams/backtrace/backtrace.tex}}%
      \onslide*<2>{\providecommand\step{6}\input{diagrams/backtrace/backtrace.tex}}%
      \onslide*<3>{\providecommand\step{7}\input{diagrams/backtrace/backtrace.tex}}%
      \onslide*<4>{\providecommand\step{8}\input{diagrams/backtrace/backtrace.tex}}%
      \onslide*<5>{\providecommand\step{9}\input{diagrams/backtrace/backtrace.tex}}%
      \onslide*<6>{\providecommand\step{10}\input{diagrams/backtrace/backtrace.tex}}%
      \onslide*<7>{\providecommand\step{11}\input{diagrams/backtrace/backtrace.tex}}%
      \onslide*<8>{\providecommand\step{12}\input{diagrams/backtrace/backtrace.tex}}%
      \onslide*<9>{\providecommand\step{13}\input{diagrams/backtrace/backtrace.tex}}%
    \end{column}
  \end{columns}
\end{frame}

\begin{frame}{Backtraces}{Printing the backtrace}
  \begin{columns}[c]
    \begin{column}{0.45\textwidth}
      \minipage[c][0.66\textheight][s]{\columnwidth}
      \begin{itemize}
        \item<1-> The exception backtrace can now be printed to \funcarg{stdout} with \funcname{Printexc.print_backtrace}
        \item<2-> First the exception backtrace is retrieved into an OCaml array with \funcname{get_raw_backtrace }
        \item<3-> Which is implemented as the \funcname{caml_get_exception_raw_backtrace} C function in the runtime
        \item<4-> And finally printed with \funcname{print_exception_backtrace}
      \end{itemize}
      \vfill
      \mintocaml[firstline=28,lastline=34,highlightlines=33]{../src/backtrace/backtrace.ml}
      \endminipage
    \end{column}
    \begin{column}{0.55\textwidth}
      \onslide*<1,2>{
        \mintocaml[firstline=100,lastline=101]{../ocaml/stdlib/printexc.ml}
      }
      \only<1>{
        \listocaml[firstline=170,lastline=172]{../ocaml/stdlib/printexc.ml}{stdlib/printexc.ml:170}
      }
      \only<2>{
        \listocaml[firstline=170,lastline=172,highlightlines=172]{../ocaml/stdlib/printexc.ml}{stdlib/printexc.ml:170}
      }
      \only<3>{
        \mintc[firstline=154,lastline=156]{../ocaml/runtime/backtrace.c}
        \listc[firstline=171,lastline=192,highlightlines={184-187}]{../ocaml/runtime/backtrace.c}{runtime/backtrace.c:154}
      }
      \only<4>{
        \listocaml[firstline=155,lastline=165]{../ocaml/stdlib/printexc.ml}{stdlib/printexc.ml:55}
      }
    \end{column}
  \end{columns}
\end{frame}


\subsection{The multiple kinds of raise}
\frameSubsection{}{}

\begin{frame}{Backtraces}{When backtraces are not needed}
  \begin{columns}[c]
    \begin{column}{0.45\textwidth}
      \minipage[c][0.66\textheight][s]{\columnwidth}
      \begin{itemize}
        \item<1-> What if the backtrace for an exception is never needed?
        \item<2-> It's possible to raise the exception explicitly with \funcname{raise\_notrace} to make raising as cheap as possible
        \item<3-> An example of use in the \funcname{dynlink} library of the OCaml compiler
      \end{itemize}
      \vfill
      \onslide<3->{
      \listocaml[firstline=205,lastline=209,highlightlines=207]{../ocaml/otherlibs/dynlink/dynlink\_compilerlibs/misc.ml}{otherlibs/dynlink/dynlink\_compilerlibs/misc.ml:205}
      }
      \endminipage
    \end{column}
    \begin{column}{0.55\textwidth}
    \only<4>{
      \mintcmm[highlightlines=3]{../src/notrace/camlNotrace\_\_fun_347.cmm}
      \listcmm{../src/notrace/camlNotrace\_\_all\_somes\_327.cmm}{all\_somes}
    }
    \onslide*<5->{
      \listobjdump[highlightlines={6-9}]{../src/notrace/camlNotrace\_\_fun_347.objdump}{anonymous function from \funcname{all\_somes}}
    }
    \end{column}
  \end{columns}
\end{frame}

\begin{frame}{Backtraces}{Summary table of the multiple kinds of raise}
  \begin{itemize}
    \item When debugging information is enabled and if the backtrace of an exception isn't needed, it's possible to raise with \funcname{raise\_notrace} for performance
    \item When debugging information is \emph{not} enabled, the compiler optimises \funcname{raise} calls to inline assembly (same effect as using \funcname{raise\_notrace})
  \end{itemize}
  \bigskip
  \centering
  \begin{tabular}{| l | c | c | c | c |}
    \hline
    \parbox{10em}{Debug information \\ (\funcarg{-g} flag)}  & \multicolumn{2}{|c|}{Disabled}                                     & \multicolumn{2}{|c|}{Enabled} \\
    \hline
    Raise kind                                               & \funcname{raise} & \funcname{raise\_notrace}                       & \funcname{raise} & \funcname{raise\_notrace} \\
    \hline
    Compilation                                              & \parbox{8em}{inline assembly\\ (optimization)} & inline assembly   & \funcname{caml\_raise\_exn} & inline assembly \\
    \hline
  \end{tabular}
\end{frame}


%
% Takeaway #5
%
\subsection*{Takeaway \#5}
\frameSubsectionTakeaway{}
\againframe<5>{takeaway}
}%
      \onslide*<2>{\providecommand\step{1}\input{diagrams/backtrace/scanstack.tex}}%
      \onslide*<3>{\providecommand\step{2}\input{diagrams/backtrace/scanstack.tex}}%
      \onslide*<4>{\providecommand\step{3}\input{diagrams/backtrace/scanstack.tex}}%
      \onslide*<5>{\providecommand\step{4}\input{diagrams/backtrace/scanstack.tex}}%
      \onslide*<6>{\providecommand\step{5}\input{diagrams/backtrace/scanstack.tex}}%
    \end{column}
  \end{columns}
\end{frame}

% Duplicate of previous-previous slide
\begin{frame}{Backtraces}{Continuing on \funcname{caml\_stash\_backtrace}}
  \begin{columns}[c]
    \begin{column}{0.5\textwidth}
      \minipage[c][0.75\textheight][s]{\columnwidth}
      \begin{itemize}
          \color{gray}
        \item A C function provided by the runtime (specific to native code)
        \item It scans the stack, between the stack pointer at the raising point up to the next trap, in order to collect the names of the detected function frames
        \item It uses the same technique and information as the Garbage Collector (GC) uses to scan the values live on the stack
      \end{itemize}
      \begin{itemize}
        \item For each function frame detected, store a pointer to the \typename{frame\_descr} in the \localname{domain\_state->backtrace\_buffer} array, effectively constructing the backtrace
      \end{itemize}
      \onslide<2->{
        \begin{itemize}
            \smallskip
          \item Constructed backtrace:
            \funcname{definitely\_raise},
            \onslide<3->{\funcname{probably\_raise}}
        \end{itemize}
      }
      \vfill
      \mintocaml[firstline=9,lastline=13,highlightlines=12]{../src/backtrace/backtrace.ml}
      \endminipage
    \end{column}
    \begin{column}{0.5\textwidth}
      \raggedleft
      \onslide*<1>{\listStashBacktrace[highlightlines={114-115}]{}}
      \onslide*<2>{\providecommand\step{3}%
% Backtraces
%
\subsection{One advantage of raising with debugging information}
\frameSubsection{}{}

\begin{frame}{Backtraces}{Debugging information \& source code}
  \begin{columns}[c]
    \begin{column}{0.5\textwidth}
      \begin{itemize}
        \item Raising an exception is fun and games, but how do we know where the exception originated? $\rightarrow$ With the backtrace associated with the exception!
        \item In order to get a backtrace from the point where an exception was raised, it's necessary to compile with debugging information enabled (\funcarg{-g} flag for the compiler)
        \item Same example as in the `Nested handlers' section
      \end{itemize}
    \end{column}
    \begin{column}{0.5\textwidth}
      \listocaml[firstline=9,lastline=34]{../src/backtrace/backtrace.ml}{backtrace/backtrace.ml}
    \end{column}
  \end{columns}
\end{frame}

\begin{frame}{Backtraces}{CMM and disassembly of \funcname{definitely\_raise}}
  \begin{columns}[c]
    \begin{column}{0.5\textwidth}
      \minipage[c][0.66\textheight][s]{\columnwidth}
      \begin{itemize}
        \item<1-> An effect of compiling with debugging information (\funcarg{-g}): the CMM no longer uses \funcname{raise\_notrace} to raise the exception but \funcname{raise} instead
        \item<2-> Disassembly shows that CMM \funcname{raise} is actually a call to \funcname{caml\_raise\_exn}, a function in assembler provided by the runtime
      \end{itemize}
      \vfill
      \mintocaml[firstline=9,lastline=13,highlightlines=12]{../src/backtrace/backtrace.ml}
      \endminipage
    \end{column}
    \begin{column}{0.5\textwidth}
      \onslide*<1>{
        \mintcmm[highlightlines=5]{../src/backtrace/camlBacktrace\_\_definitely\_raise\_347.cmm}
      }
      \onslide*<2>{
        \mintobjdump[firstline=2,highlightlines=14]{../src/backtrace/camlBacktrace\_\_definitely\_raise\_347.objdump}
      }
    \end{column}
  \end{columns}
\end{frame}

\begin{frame}{Backtraces}{Introducing \funcname{caml\_raise\_exn}}
  \begin{columns}[c]
    \begin{column}{0.5\textwidth}
      \minipage[c][0.66\textheight][s]{\columnwidth}
      \onslide*<1>{
        \begin{itemize}
          \item Raising the exception is performed using \funcname{caml\_raise\_exn} which is
            \begin{itemize}
              \item An assembly function provided by the runtime
              \item Architecture specific
            \end{itemize}
          \item Let's see what it does differently from \funcname{raise\_notrace}\dots
          \item We are about to raise \typename{ExnC} with \funcname{caml\_raise\_exn} from \funcname{definitely\_raise}
        \end{itemize}
      }
      \onslide*<2>{
        \mintobjdump[firstline=2,highlightlines=14]{../src/backtrace/camlBacktrace\_\_definitely\_raise\_347.objdump}
      }
      \vfill
      \mintocaml[firstline=9,lastline=13,highlightlines=12]{../src/backtrace/backtrace.ml}
      \endminipage
    \end{column}
    \begin{column}{0.5\textwidth}
      \centering
      \providecommand\step{1}\input{diagrams/backtrace/backtrace.tex}
    \end{column}
  \end{columns}
\end{frame}

\begin{frame}{Backtraces}{But before that, we need more fields from \typename{caml\_domain\_state}}
  \begin{columns}
    \begin{column}{0.5\textwidth}
      \begin{itemize}
        \item Remember \typename{caml\_domain\_state} the per-domain runtime state?
        \item Some other members are of interest to us now\dots
        \item \localname{backtrace\_active} is used to know if the runtime should collect backtraces
        \item \localname{backtrace\_active} can be set by
          \begin{itemize}
            \item The \funcarg{b} flag in the \funcname{OCAMLRUNPARAM} environment variable
            \item \mintinline{ocaml}{Printexc.record_backtrace : bool -> unit} \footnotemark
          \end{itemize}
      \end{itemize}
    \end{column}
    \begin{column}{0.5\textwidth}
      \begin{listing}
        \mintc[highlightlines={9-11}]{src/ocaml/domain\_state\_backtrace.h}
        \caption{runtime/caml/domain\_state.h}
      \end{listing}
    \end{column}
  \end{columns}
  \footnotetext[1]{https://v2.ocaml.org/api/Printexc.html}
\end{frame}

\newcommand\listCamlRaiseExn[1][]{
  \listasm[firstline=702,lastline=725,#1]{../ocaml/runtime/amd64.S}{runtime/amd64.S:702}}

\begin{frame}{Backtraces}{Walk through \funcname{caml\_raise\_exn}}
  \begin{columns}[T]
    \begin{column}{0.5\textwidth}
      \begin{itemize}
        \item<1-> First checks whether exceptions backtrace should be recorded
        \item<1-> By checking the \funcarg{domain\_state->backtrace\_active} value
        \item<2-> If not, acts as \funcname{raise\_notrace} with \funcname{RESTORE\_EXN\_HANDLER\_OCAML}
          \begin{itemize}
            \item Set \regname{RSP} to \localname{domain\_state->exn\_handler} value
            \item Restore \localname{domain\_state->exn\_handler} from trap
            \item Jump with \funcname{ret} (same effect as using \funcname{pop} and \funcname{jmp})
          \end{itemize}
        \item<3-> Otherwise, switches to the C stack to be able to execute C code
        \item<4-> Calls \funcname{caml\_stash\_backtrace}, a C function provided by the runtime\ignorespaces
          \onslide<6->{, with
            \begin{itemize}
              \item \funcarg{exn}: exception value
              \item \funcarg{pc}: code address of the raising point
              \item \funcarg{sp}: stack address of the raising function
              \item \funcarg{trapsp}: stack address of the next handler
            \end{itemize}
          }
      \end{itemize}
      \bigskip
      \mintocaml[firstline=9,lastline=13,highlightlines=12]{../src/backtrace/backtrace.ml}
    \end{column}
    \begin{column}{0.5\textwidth}
      \raggedleft
      \onslide*<1,3-4>{
        \mintc[firstline=190,lastline=192]{../ocaml/runtime/amd64.S}
        \mintc[firstline=234,lastline=237]{../ocaml/runtime/amd64.S}
      }
      \onslide*<2>{
        \mintc[firstline=190,lastline=192,highlightlines={191-192}]{../ocaml/runtime/amd64.S}
        \mintc[firstline=234,lastline=237,highlightlines={235-237}]{../ocaml/runtime/amd64.S}
      }
      \onslide*<1>{\listCamlRaiseExn[highlightlines=705]{}}%
      \onslide*<2>{\listCamlRaiseExn[highlightlines={707-708}]{}}%
      \onslide*<3>{\listCamlRaiseExn[highlightlines={712-714}]{}}%
      \onslide*<4>{\listCamlRaiseExn[highlightlines={715-720}]{}}%
      \onslide*<5>{\providecommand\step{1}\input{diagrams/backtrace/backtrace.tex}}%
      \onslide*<6>{\providecommand\step{2}\input{diagrams/backtrace/backtrace.tex}}%
    \end{column}
  \end{columns}
\end{frame}

\newcommand\listStashBacktrace[1][]{
  \listc[firstline=91,lastline=120,#1]{../ocaml/runtime/backtrace\_nat.c}{runtime/backtrace\_nat.c:91}}

\begin{frame}{Backtraces}{Introducing \funcname{caml\_stash\_backtrace}}
  \begin{columns}[c]
    \begin{column}{0.5\textwidth}
      \minipage[c][0.66\textheight][s]{\columnwidth}
      \begin{itemize}
        \item<1-> A C function provided by the runtime (specific to native code)
        \item<2-> It scans the stack, between the stack pointer at the raising point up to the next trap, in order to collect the names of the detected function frames
          \onslide*<2>{
            \begin{itemize}
              \item And more information, like line number, column number...
            \end{itemize}
          }
        \item<3-> It uses the same technique and information as the Garbage Collector (GC) uses to scan the values live on the stack
          \onslide*<4>{
            \begin{itemize}
              \item \typename{frame\_descr} are at the core of stack unwinding
            \end{itemize}
          }
      \end{itemize}
      \vfill
      \mintocaml[firstline=9,lastline=13,highlightlines=12]{../src/backtrace/backtrace.ml}
      \endminipage
    \end{column}
    \begin{column}{0.5\textwidth}
      \onslide*<1>{\listStashBacktrace[highlightlines=91]{}}
      \onslide*<2>{\listStashBacktrace[highlightlines={91,117-118}]{}}
      \onslide*<3->{\listStashBacktrace[highlightlines={94,106,109-110}]{}}
    \end{column}
  \end{columns}
\end{frame}

\newcommand\listFrameDescr[1][]{
  \listocaml[firstline=111,lastline=124,#1]{../ocaml/asmcomp/emitaux.ml}{asmcomp/emitaux.ml:111}}
\newcommand\listDebugInfo[1][]{
  \listocaml[firstline=38,lastline=49,#1]{../ocaml/lambda/debuginfo.mli}{lambda/debuginfo.mli:38}}

\begin{frame}{Backtraces}{\typename{frame\_descr} and \typename{frame\_debuginfo} in a nutshell}
  \begin{columns}[c]
    \begin{column}{0.5\textwidth}
      \begin{itemize}
        \item<1-> For every return address in an OCaml program, there is a associated \typename{frame\_descr}
        \item<2-> A \typename{frame\_descr} specifies
        \begin{itemize}
          \item<2-> The size of the function stack frame
          \item<3-> Where live values are on the stack (so that the GC can mark them as live and prevent from being swept)
        \end{itemize}
        \item<4-> When compiled with debug information, \typename{frame\_descr} will be linked to a \typename{frame\_debuginfo}
        \begin{itemize}
          \item<5-> Contains information about the position in the source code (file name, line and column number)
        \end{itemize}
      \end{itemize}
    \end{column}
    \begin{column}{0.5\textwidth}
        \onslide*<1>{\listFrameDescr[highlightlines={118-122}]{}}
        \onslide*<2>{\listFrameDescr[highlightlines={120}]{}}
        \onslide*<3>{\listFrameDescr[highlightlines={121}]{}}
        \onslide*<4->{\listFrameDescr[highlightlines={122}]{}}
        \medskip
        \onslide*<1-3>{\listDebugInfo{}}
        \onslide*<4>{\listDebugInfo[highlightlines={38,49}]{}}
        \onslide*<5>{\listDebugInfo[highlightlines={39-46}]{}}
    \end{column}
  \end{columns}
\end{frame}

\begin{frame}{Backtraces}{Scanning the stack with \typename{frame\_descr}}
  \begin{columns}[c]
    \begin{column}{0.5\textwidth}
      \begin{itemize}
        \item Given the return address of the code that raises the exception by calling \funcname{caml\_raise\_exn} (\funcarg{pc} argument of \funcname{caml\_stash\_backtrace})
        \item And the position of the stack pointer (\funcarg{sp} argument of \funcname{caml\_stash\_backtrace})
        \item The frame size given by \typename{frame\_descr}.\funcarg{frame\_size}, allows to find the end of the previous frame
        \item And the return address of the caller of this function
        \bigskip
        \onslide<3->{
        \item Note that traps (here the reddish \funcname{ExnA}, \funcname{ExnB} and \funcname{ExnC} blocks) belong to the frame that installed them, and so the frame size changes when traps are removed
        }
      \end{itemize}
    \end{column}
    \begin{column}{0.5\textwidth}
      \raggedleft
      \onslide*<1>{\providecommand\step{2}\input{diagrams/backtrace/backtrace.tex}}%
      \onslide*<2>{\providecommand\step{1}\input{diagrams/backtrace/scanstack.tex}}%
      \onslide*<3>{\providecommand\step{2}\input{diagrams/backtrace/scanstack.tex}}%
      \onslide*<4>{\providecommand\step{3}\input{diagrams/backtrace/scanstack.tex}}%
      \onslide*<5>{\providecommand\step{4}\input{diagrams/backtrace/scanstack.tex}}%
      \onslide*<6>{\providecommand\step{5}\input{diagrams/backtrace/scanstack.tex}}%
    \end{column}
  \end{columns}
\end{frame}

% Duplicate of previous-previous slide
\begin{frame}{Backtraces}{Continuing on \funcname{caml\_stash\_backtrace}}
  \begin{columns}[c]
    \begin{column}{0.5\textwidth}
      \minipage[c][0.75\textheight][s]{\columnwidth}
      \begin{itemize}
          \color{gray}
        \item A C function provided by the runtime (specific to native code)
        \item It scans the stack, between the stack pointer at the raising point up to the next trap, in order to collect the names of the detected function frames
        \item It uses the same technique and information as the Garbage Collector (GC) uses to scan the values live on the stack
      \end{itemize}
      \begin{itemize}
        \item For each function frame detected, store a pointer to the \typename{frame\_descr} in the \localname{domain\_state->backtrace\_buffer} array, effectively constructing the backtrace
      \end{itemize}
      \onslide<2->{
        \begin{itemize}
            \smallskip
          \item Constructed backtrace:
            \funcname{definitely\_raise},
            \onslide<3->{\funcname{probably\_raise}}
        \end{itemize}
      }
      \vfill
      \mintocaml[firstline=9,lastline=13,highlightlines=12]{../src/backtrace/backtrace.ml}
      \endminipage
    \end{column}
    \begin{column}{0.5\textwidth}
      \raggedleft
      \onslide*<1>{\listStashBacktrace[highlightlines={114-115}]{}}
      \onslide*<2>{\providecommand\step{3}\input{diagrams/backtrace/backtrace.tex}}%
      \onslide*<3>{\providecommand\step{4}\input{diagrams/backtrace/backtrace.tex}}%
    \end{column}
  \end{columns}
\end{frame}

\begin{frame}{Backtraces}{Back to \funcname{caml\_raise\_exn}}
  \begin{columns}[c]
    \begin{column}{0.5\textwidth}
      \minipage[c][0.66\textheight][s]{\columnwidth}
      \begin{itemize}\color{gray}
        \item Checks wheteher exceptions backtrace should be recorded, by checking the \funcarg{domain\_state->backtrace\_active} value
        \item Switches to the C stack to be able to execute C code
        \item Calls \funcname{caml\_stash\_backtrace}, to collect the backtrace up to the next trap
      \end{itemize}
      \onslide*<2->{
        \begin{itemize}
          \item<2-> Executes the exception handler in \funcname{probably\_raise} with \funcname{RESTORE\_EXN\_HANDLER\_OCAML}
            \begin{itemize}
              \item Set \regname{RSP} to \localname{domain\_state->exn\_handler} value
              \item Restore \localname{domain\_state->exn\_handler} from trap
              \item Jump to the handler code with \funcname{ret}
            \end{itemize}
        \end{itemize}
      }
      \vfill
      \onslide*<1>{\mintocaml[firstline=9,lastline=13,highlightlines=12]{../src/backtrace/backtrace.ml}}
      \onslide*<2->{\mintocaml[firstline=15,lastline=21,highlightlines={19-21}]{../src/backtrace/backtrace.ml}}
      \endminipage
    \end{column}
    \begin{column}{0.5\textwidth}
      \centering
      \onslide*<1>{
        \mintc[firstline=190,lastline=192]{../ocaml/runtime/amd64.S}
        \mintc[firstline=234,lastline=237]{../ocaml/runtime/amd64.S}
      }
      \onslide*<2>{
        \mintc[firstline=190,lastline=192,highlightlines={191-192}]{../ocaml/runtime/amd64.S}
        \mintc[firstline=234,lastline=237,highlightlines={235-237}]{../ocaml/runtime/amd64.S}
      }
      \onslide*<1>{\listCamlRaiseExn[highlightlines=720]{}}
      \onslide*<2>{\listCamlRaiseExn[highlightlines=722]{}}
      \onslide*<3->{\providecommand\step{5}\input{diagrams/backtrace/backtrace.tex}}
    \end{column}
  \end{columns}
\end{frame}

\begin{frame}{Backtraces}{\funcname{caml\_probably\_raise}'s handler doesn't catch \typename{ExnC}}
  \begin{columns}[c]
    \begin{column}{0.45\textwidth}
      \minipage[c][0.75\textheight][s]{\columnwidth}
      \begin{itemize}
        \item<1-> \funcname{match \dots with}\footnotemark pattern matching can also match exceptions
        \item<1-> The CMM of \funcname{probably\_raise} shows that this \funcname{match \dots with} is implemented with a pattern matching and a \funcname{try \dots with}
        \item<1-> But this one only matches on \typename{ExnA} exception
        \item<2-> Since the pattern matching fails to catch the exception, \funcname{reraise} is called with our current \typename{ExnC} exception
        \item<3-> Disassembly shows that \funcname{reraise} is actually a call to \funcname{caml\_reraise\_exn}, which is also an assembly function provided by the runtime
      \end{itemize}
      \vfill
      \mintocaml[firstline=15,lastline=21,highlightlines={18-21}]{../src/backtrace/backtrace.ml}
      \endminipage
    \end{column}
    \begin{column}{0.55\textwidth}
      \centering
      \onslide*<1>{\mintcmm[highlightlines={10-18}]{../src/backtrace/camlBacktrace\_\_probably\_raise\_351.cmm}}
      \onslide*<2>{\listcmm[highlightlines=18]{../src/backtrace/camlBacktrace\_\_probably\_raise_351.cmm}{CMM of probably\_raise}}
      \onslide*<3>{\mintobjdump[firstline=5,lastline=35,highlightlines=31]{../src/backtrace/camlBacktrace\_\_probably\_raise_351.objdump}}
    \end{column}
  \end{columns}
  \only<1>{\footnotetext[1]{https://blog.janestreet.com/pattern-matching-and-exception-handling-unite/}}
\end{frame}

\newcommand\listCamlReraiseExn[1][]{
  \listasm[firstline=727,lastline=734,#1]{../ocaml/runtime/amd64.S}{runtime/amd64.S:727}}

\begin{frame}{Backtraces}{Introducing \funcname{caml\_reraise\_exn}}
  \begin{columns}[c]
    \begin{column}{0.5\textwidth}
      \minipage[c][0.66\textheight][s]{\columnwidth}
      \begin{itemize}
        \item<1-> The exception have to be reraised to the next handler
        \item<2-> First, checks if the backtrace should be collected with \funcarg{domain\_state->backtrace\_active}
        \only<3>{
        \begin{itemize}
            \item If not, behaves like a \funcname{raise\_notrace} with \funcname{RESTORE\_EXN\_HANDLER\_OCAML}
        \end{itemize}
        }
        \item<4-> Jumps into \funcname{caml\_raise\_exn}
        \item<5-> Calls \funcname{caml\_stash\_backtrace} again, to scan the next part of the stack: from the trap just pop-ed to the next trap
      \end{itemize}
      \vfill
      \mintocaml[firstline=15,lastline=21,highlightlines={18-21}]{../src/backtrace/backtrace.ml}
      \endminipage
    \end{column}
    \begin{column}{0.5\textwidth}
      \raggedleft
      \onslide*<1>{\listCamlReraiseExn{}}
      \onslide*<2>{\listCamlReraiseExn[highlightlines=729]{}}
      \onslide*<3>{
        \mintc[firstline=190,lastline=192,highlightlines={191-192}]{../ocaml/runtime/amd64.S}
        \mintc[firstline=234,lastline=237,highlightlines={235-237}]{../ocaml/runtime/amd64.S}
        \medskip
        \listCamlReraiseExn[highlightlines=731]{}
      }
      \onslide*<4-5>{
        % TODO refactor with \listCamlReraiseExn
        \mintasm[firstline=727,lastline=734,highlightlines=730]{../ocaml/runtime/amd64.S}
        \medskip
      }
      \onslide*<4>{\listCamlRaiseExn[highlightlines=711]{}}
      \onslide*<5>{\listCamlRaiseExn[highlightlines=720]{}}
    \end{column}
  \end{columns}
\end{frame}

\begin{frame}{Backtraces}{Backtrace collection continues}
  \begin{columns}[c]
    \begin{column}{0.55\textwidth}
      \minipage[c][0.75\textheight][s]{\columnwidth}
      \begin{itemize}
        \item<1-> \funcname{caml\_stash\_backtrace} scans the older part of the stack, up to the next trap
        \item<6-> Controls jumps to the nested exception handler in \funcname{maybe\_raise}, which matches only on \typename{ExnB}
        \item<7-> \funcname{caml\_reraise\_exn} is called again, backtrace collection resumes with \funcname{caml\_stash\_backtrace}
      \end{itemize}
      \medskip
      \begin{itemize}
        \item<2-> Constructed backtrace:
        \begin{itemize}
          \item <2-> \funcname{definitely\_raise}
          \item <2-> \funcname{probably\_raise}
          \item <3-> \funcname{probably\_raise} (re-raise)
          \item <4-> \funcname{maybe\_raise}
          \item <5-> \funcname{main}
          \item <8-> \funcname{main} (re-raise)
        \end{itemize}
      \end{itemize}
      \vfill
      \onslide*<1-5>{\mintocaml[firstline=15,lastline=21]{../src/backtrace/backtrace.ml}}
      \onslide*<6>{\mintocaml[firstline=28,lastline=34,highlightlines=31]{../src/backtrace/backtrace.ml}}
      \onslide*<7->{\mintocaml[firstline=28,lastline=34,highlightlines=33]{../src/backtrace/backtrace.ml}}
      \endminipage
    \end{column}
    \begin{column}{0.45\textwidth}
      \raggedleft
      \onslide*<1>{\providecommand\step{6}\input{diagrams/backtrace/backtrace.tex}}%
      \onslide*<2>{\providecommand\step{6}\input{diagrams/backtrace/backtrace.tex}}%
      \onslide*<3>{\providecommand\step{7}\input{diagrams/backtrace/backtrace.tex}}%
      \onslide*<4>{\providecommand\step{8}\input{diagrams/backtrace/backtrace.tex}}%
      \onslide*<5>{\providecommand\step{9}\input{diagrams/backtrace/backtrace.tex}}%
      \onslide*<6>{\providecommand\step{10}\input{diagrams/backtrace/backtrace.tex}}%
      \onslide*<7>{\providecommand\step{11}\input{diagrams/backtrace/backtrace.tex}}%
      \onslide*<8>{\providecommand\step{12}\input{diagrams/backtrace/backtrace.tex}}%
      \onslide*<9>{\providecommand\step{13}\input{diagrams/backtrace/backtrace.tex}}%
    \end{column}
  \end{columns}
\end{frame}

\begin{frame}{Backtraces}{Printing the backtrace}
  \begin{columns}[c]
    \begin{column}{0.45\textwidth}
      \minipage[c][0.66\textheight][s]{\columnwidth}
      \begin{itemize}
        \item<1-> The exception backtrace can now be printed to \funcarg{stdout} with \funcname{Printexc.print_backtrace}
        \item<2-> First the exception backtrace is retrieved into an OCaml array with \funcname{get_raw_backtrace }
        \item<3-> Which is implemented as the \funcname{caml_get_exception_raw_backtrace} C function in the runtime
        \item<4-> And finally printed with \funcname{print_exception_backtrace}
      \end{itemize}
      \vfill
      \mintocaml[firstline=28,lastline=34,highlightlines=33]{../src/backtrace/backtrace.ml}
      \endminipage
    \end{column}
    \begin{column}{0.55\textwidth}
      \onslide*<1,2>{
        \mintocaml[firstline=100,lastline=101]{../ocaml/stdlib/printexc.ml}
      }
      \only<1>{
        \listocaml[firstline=170,lastline=172]{../ocaml/stdlib/printexc.ml}{stdlib/printexc.ml:170}
      }
      \only<2>{
        \listocaml[firstline=170,lastline=172,highlightlines=172]{../ocaml/stdlib/printexc.ml}{stdlib/printexc.ml:170}
      }
      \only<3>{
        \mintc[firstline=154,lastline=156]{../ocaml/runtime/backtrace.c}
        \listc[firstline=171,lastline=192,highlightlines={184-187}]{../ocaml/runtime/backtrace.c}{runtime/backtrace.c:154}
      }
      \only<4>{
        \listocaml[firstline=155,lastline=165]{../ocaml/stdlib/printexc.ml}{stdlib/printexc.ml:55}
      }
    \end{column}
  \end{columns}
\end{frame}


\subsection{The multiple kinds of raise}
\frameSubsection{}{}

\begin{frame}{Backtraces}{When backtraces are not needed}
  \begin{columns}[c]
    \begin{column}{0.45\textwidth}
      \minipage[c][0.66\textheight][s]{\columnwidth}
      \begin{itemize}
        \item<1-> What if the backtrace for an exception is never needed?
        \item<2-> It's possible to raise the exception explicitly with \funcname{raise\_notrace} to make raising as cheap as possible
        \item<3-> An example of use in the \funcname{dynlink} library of the OCaml compiler
      \end{itemize}
      \vfill
      \onslide<3->{
      \listocaml[firstline=205,lastline=209,highlightlines=207]{../ocaml/otherlibs/dynlink/dynlink\_compilerlibs/misc.ml}{otherlibs/dynlink/dynlink\_compilerlibs/misc.ml:205}
      }
      \endminipage
    \end{column}
    \begin{column}{0.55\textwidth}
    \only<4>{
      \mintcmm[highlightlines=3]{../src/notrace/camlNotrace\_\_fun_347.cmm}
      \listcmm{../src/notrace/camlNotrace\_\_all\_somes\_327.cmm}{all\_somes}
    }
    \onslide*<5->{
      \listobjdump[highlightlines={6-9}]{../src/notrace/camlNotrace\_\_fun_347.objdump}{anonymous function from \funcname{all\_somes}}
    }
    \end{column}
  \end{columns}
\end{frame}

\begin{frame}{Backtraces}{Summary table of the multiple kinds of raise}
  \begin{itemize}
    \item When debugging information is enabled and if the backtrace of an exception isn't needed, it's possible to raise with \funcname{raise\_notrace} for performance
    \item When debugging information is \emph{not} enabled, the compiler optimises \funcname{raise} calls to inline assembly (same effect as using \funcname{raise\_notrace})
  \end{itemize}
  \bigskip
  \centering
  \begin{tabular}{| l | c | c | c | c |}
    \hline
    \parbox{10em}{Debug information \\ (\funcarg{-g} flag)}  & \multicolumn{2}{|c|}{Disabled}                                     & \multicolumn{2}{|c|}{Enabled} \\
    \hline
    Raise kind                                               & \funcname{raise} & \funcname{raise\_notrace}                       & \funcname{raise} & \funcname{raise\_notrace} \\
    \hline
    Compilation                                              & \parbox{8em}{inline assembly\\ (optimization)} & inline assembly   & \funcname{caml\_raise\_exn} & inline assembly \\
    \hline
  \end{tabular}
\end{frame}


%
% Takeaway #5
%
\subsection*{Takeaway \#5}
\frameSubsectionTakeaway{}
\againframe<5>{takeaway}
}%
      \onslide*<3>{\providecommand\step{4}%
% Backtraces
%
\subsection{One advantage of raising with debugging information}
\frameSubsection{}{}

\begin{frame}{Backtraces}{Debugging information \& source code}
  \begin{columns}[c]
    \begin{column}{0.5\textwidth}
      \begin{itemize}
        \item Raising an exception is fun and games, but how do we know where the exception originated? $\rightarrow$ With the backtrace associated with the exception!
        \item In order to get a backtrace from the point where an exception was raised, it's necessary to compile with debugging information enabled (\funcarg{-g} flag for the compiler)
        \item Same example as in the `Nested handlers' section
      \end{itemize}
    \end{column}
    \begin{column}{0.5\textwidth}
      \listocaml[firstline=9,lastline=34]{../src/backtrace/backtrace.ml}{backtrace/backtrace.ml}
    \end{column}
  \end{columns}
\end{frame}

\begin{frame}{Backtraces}{CMM and disassembly of \funcname{definitely\_raise}}
  \begin{columns}[c]
    \begin{column}{0.5\textwidth}
      \minipage[c][0.66\textheight][s]{\columnwidth}
      \begin{itemize}
        \item<1-> An effect of compiling with debugging information (\funcarg{-g}): the CMM no longer uses \funcname{raise\_notrace} to raise the exception but \funcname{raise} instead
        \item<2-> Disassembly shows that CMM \funcname{raise} is actually a call to \funcname{caml\_raise\_exn}, a function in assembler provided by the runtime
      \end{itemize}
      \vfill
      \mintocaml[firstline=9,lastline=13,highlightlines=12]{../src/backtrace/backtrace.ml}
      \endminipage
    \end{column}
    \begin{column}{0.5\textwidth}
      \onslide*<1>{
        \mintcmm[highlightlines=5]{../src/backtrace/camlBacktrace\_\_definitely\_raise\_347.cmm}
      }
      \onslide*<2>{
        \mintobjdump[firstline=2,highlightlines=14]{../src/backtrace/camlBacktrace\_\_definitely\_raise\_347.objdump}
      }
    \end{column}
  \end{columns}
\end{frame}

\begin{frame}{Backtraces}{Introducing \funcname{caml\_raise\_exn}}
  \begin{columns}[c]
    \begin{column}{0.5\textwidth}
      \minipage[c][0.66\textheight][s]{\columnwidth}
      \onslide*<1>{
        \begin{itemize}
          \item Raising the exception is performed using \funcname{caml\_raise\_exn} which is
            \begin{itemize}
              \item An assembly function provided by the runtime
              \item Architecture specific
            \end{itemize}
          \item Let's see what it does differently from \funcname{raise\_notrace}\dots
          \item We are about to raise \typename{ExnC} with \funcname{caml\_raise\_exn} from \funcname{definitely\_raise}
        \end{itemize}
      }
      \onslide*<2>{
        \mintobjdump[firstline=2,highlightlines=14]{../src/backtrace/camlBacktrace\_\_definitely\_raise\_347.objdump}
      }
      \vfill
      \mintocaml[firstline=9,lastline=13,highlightlines=12]{../src/backtrace/backtrace.ml}
      \endminipage
    \end{column}
    \begin{column}{0.5\textwidth}
      \centering
      \providecommand\step{1}\input{diagrams/backtrace/backtrace.tex}
    \end{column}
  \end{columns}
\end{frame}

\begin{frame}{Backtraces}{But before that, we need more fields from \typename{caml\_domain\_state}}
  \begin{columns}
    \begin{column}{0.5\textwidth}
      \begin{itemize}
        \item Remember \typename{caml\_domain\_state} the per-domain runtime state?
        \item Some other members are of interest to us now\dots
        \item \localname{backtrace\_active} is used to know if the runtime should collect backtraces
        \item \localname{backtrace\_active} can be set by
          \begin{itemize}
            \item The \funcarg{b} flag in the \funcname{OCAMLRUNPARAM} environment variable
            \item \mintinline{ocaml}{Printexc.record_backtrace : bool -> unit} \footnotemark
          \end{itemize}
      \end{itemize}
    \end{column}
    \begin{column}{0.5\textwidth}
      \begin{listing}
        \mintc[highlightlines={9-11}]{src/ocaml/domain\_state\_backtrace.h}
        \caption{runtime/caml/domain\_state.h}
      \end{listing}
    \end{column}
  \end{columns}
  \footnotetext[1]{https://v2.ocaml.org/api/Printexc.html}
\end{frame}

\newcommand\listCamlRaiseExn[1][]{
  \listasm[firstline=702,lastline=725,#1]{../ocaml/runtime/amd64.S}{runtime/amd64.S:702}}

\begin{frame}{Backtraces}{Walk through \funcname{caml\_raise\_exn}}
  \begin{columns}[T]
    \begin{column}{0.5\textwidth}
      \begin{itemize}
        \item<1-> First checks whether exceptions backtrace should be recorded
        \item<1-> By checking the \funcarg{domain\_state->backtrace\_active} value
        \item<2-> If not, acts as \funcname{raise\_notrace} with \funcname{RESTORE\_EXN\_HANDLER\_OCAML}
          \begin{itemize}
            \item Set \regname{RSP} to \localname{domain\_state->exn\_handler} value
            \item Restore \localname{domain\_state->exn\_handler} from trap
            \item Jump with \funcname{ret} (same effect as using \funcname{pop} and \funcname{jmp})
          \end{itemize}
        \item<3-> Otherwise, switches to the C stack to be able to execute C code
        \item<4-> Calls \funcname{caml\_stash\_backtrace}, a C function provided by the runtime\ignorespaces
          \onslide<6->{, with
            \begin{itemize}
              \item \funcarg{exn}: exception value
              \item \funcarg{pc}: code address of the raising point
              \item \funcarg{sp}: stack address of the raising function
              \item \funcarg{trapsp}: stack address of the next handler
            \end{itemize}
          }
      \end{itemize}
      \bigskip
      \mintocaml[firstline=9,lastline=13,highlightlines=12]{../src/backtrace/backtrace.ml}
    \end{column}
    \begin{column}{0.5\textwidth}
      \raggedleft
      \onslide*<1,3-4>{
        \mintc[firstline=190,lastline=192]{../ocaml/runtime/amd64.S}
        \mintc[firstline=234,lastline=237]{../ocaml/runtime/amd64.S}
      }
      \onslide*<2>{
        \mintc[firstline=190,lastline=192,highlightlines={191-192}]{../ocaml/runtime/amd64.S}
        \mintc[firstline=234,lastline=237,highlightlines={235-237}]{../ocaml/runtime/amd64.S}
      }
      \onslide*<1>{\listCamlRaiseExn[highlightlines=705]{}}%
      \onslide*<2>{\listCamlRaiseExn[highlightlines={707-708}]{}}%
      \onslide*<3>{\listCamlRaiseExn[highlightlines={712-714}]{}}%
      \onslide*<4>{\listCamlRaiseExn[highlightlines={715-720}]{}}%
      \onslide*<5>{\providecommand\step{1}\input{diagrams/backtrace/backtrace.tex}}%
      \onslide*<6>{\providecommand\step{2}\input{diagrams/backtrace/backtrace.tex}}%
    \end{column}
  \end{columns}
\end{frame}

\newcommand\listStashBacktrace[1][]{
  \listc[firstline=91,lastline=120,#1]{../ocaml/runtime/backtrace\_nat.c}{runtime/backtrace\_nat.c:91}}

\begin{frame}{Backtraces}{Introducing \funcname{caml\_stash\_backtrace}}
  \begin{columns}[c]
    \begin{column}{0.5\textwidth}
      \minipage[c][0.66\textheight][s]{\columnwidth}
      \begin{itemize}
        \item<1-> A C function provided by the runtime (specific to native code)
        \item<2-> It scans the stack, between the stack pointer at the raising point up to the next trap, in order to collect the names of the detected function frames
          \onslide*<2>{
            \begin{itemize}
              \item And more information, like line number, column number...
            \end{itemize}
          }
        \item<3-> It uses the same technique and information as the Garbage Collector (GC) uses to scan the values live on the stack
          \onslide*<4>{
            \begin{itemize}
              \item \typename{frame\_descr} are at the core of stack unwinding
            \end{itemize}
          }
      \end{itemize}
      \vfill
      \mintocaml[firstline=9,lastline=13,highlightlines=12]{../src/backtrace/backtrace.ml}
      \endminipage
    \end{column}
    \begin{column}{0.5\textwidth}
      \onslide*<1>{\listStashBacktrace[highlightlines=91]{}}
      \onslide*<2>{\listStashBacktrace[highlightlines={91,117-118}]{}}
      \onslide*<3->{\listStashBacktrace[highlightlines={94,106,109-110}]{}}
    \end{column}
  \end{columns}
\end{frame}

\newcommand\listFrameDescr[1][]{
  \listocaml[firstline=111,lastline=124,#1]{../ocaml/asmcomp/emitaux.ml}{asmcomp/emitaux.ml:111}}
\newcommand\listDebugInfo[1][]{
  \listocaml[firstline=38,lastline=49,#1]{../ocaml/lambda/debuginfo.mli}{lambda/debuginfo.mli:38}}

\begin{frame}{Backtraces}{\typename{frame\_descr} and \typename{frame\_debuginfo} in a nutshell}
  \begin{columns}[c]
    \begin{column}{0.5\textwidth}
      \begin{itemize}
        \item<1-> For every return address in an OCaml program, there is a associated \typename{frame\_descr}
        \item<2-> A \typename{frame\_descr} specifies
        \begin{itemize}
          \item<2-> The size of the function stack frame
          \item<3-> Where live values are on the stack (so that the GC can mark them as live and prevent from being swept)
        \end{itemize}
        \item<4-> When compiled with debug information, \typename{frame\_descr} will be linked to a \typename{frame\_debuginfo}
        \begin{itemize}
          \item<5-> Contains information about the position in the source code (file name, line and column number)
        \end{itemize}
      \end{itemize}
    \end{column}
    \begin{column}{0.5\textwidth}
        \onslide*<1>{\listFrameDescr[highlightlines={118-122}]{}}
        \onslide*<2>{\listFrameDescr[highlightlines={120}]{}}
        \onslide*<3>{\listFrameDescr[highlightlines={121}]{}}
        \onslide*<4->{\listFrameDescr[highlightlines={122}]{}}
        \medskip
        \onslide*<1-3>{\listDebugInfo{}}
        \onslide*<4>{\listDebugInfo[highlightlines={38,49}]{}}
        \onslide*<5>{\listDebugInfo[highlightlines={39-46}]{}}
    \end{column}
  \end{columns}
\end{frame}

\begin{frame}{Backtraces}{Scanning the stack with \typename{frame\_descr}}
  \begin{columns}[c]
    \begin{column}{0.5\textwidth}
      \begin{itemize}
        \item Given the return address of the code that raises the exception by calling \funcname{caml\_raise\_exn} (\funcarg{pc} argument of \funcname{caml\_stash\_backtrace})
        \item And the position of the stack pointer (\funcarg{sp} argument of \funcname{caml\_stash\_backtrace})
        \item The frame size given by \typename{frame\_descr}.\funcarg{frame\_size}, allows to find the end of the previous frame
        \item And the return address of the caller of this function
        \bigskip
        \onslide<3->{
        \item Note that traps (here the reddish \funcname{ExnA}, \funcname{ExnB} and \funcname{ExnC} blocks) belong to the frame that installed them, and so the frame size changes when traps are removed
        }
      \end{itemize}
    \end{column}
    \begin{column}{0.5\textwidth}
      \raggedleft
      \onslide*<1>{\providecommand\step{2}\input{diagrams/backtrace/backtrace.tex}}%
      \onslide*<2>{\providecommand\step{1}\input{diagrams/backtrace/scanstack.tex}}%
      \onslide*<3>{\providecommand\step{2}\input{diagrams/backtrace/scanstack.tex}}%
      \onslide*<4>{\providecommand\step{3}\input{diagrams/backtrace/scanstack.tex}}%
      \onslide*<5>{\providecommand\step{4}\input{diagrams/backtrace/scanstack.tex}}%
      \onslide*<6>{\providecommand\step{5}\input{diagrams/backtrace/scanstack.tex}}%
    \end{column}
  \end{columns}
\end{frame}

% Duplicate of previous-previous slide
\begin{frame}{Backtraces}{Continuing on \funcname{caml\_stash\_backtrace}}
  \begin{columns}[c]
    \begin{column}{0.5\textwidth}
      \minipage[c][0.75\textheight][s]{\columnwidth}
      \begin{itemize}
          \color{gray}
        \item A C function provided by the runtime (specific to native code)
        \item It scans the stack, between the stack pointer at the raising point up to the next trap, in order to collect the names of the detected function frames
        \item It uses the same technique and information as the Garbage Collector (GC) uses to scan the values live on the stack
      \end{itemize}
      \begin{itemize}
        \item For each function frame detected, store a pointer to the \typename{frame\_descr} in the \localname{domain\_state->backtrace\_buffer} array, effectively constructing the backtrace
      \end{itemize}
      \onslide<2->{
        \begin{itemize}
            \smallskip
          \item Constructed backtrace:
            \funcname{definitely\_raise},
            \onslide<3->{\funcname{probably\_raise}}
        \end{itemize}
      }
      \vfill
      \mintocaml[firstline=9,lastline=13,highlightlines=12]{../src/backtrace/backtrace.ml}
      \endminipage
    \end{column}
    \begin{column}{0.5\textwidth}
      \raggedleft
      \onslide*<1>{\listStashBacktrace[highlightlines={114-115}]{}}
      \onslide*<2>{\providecommand\step{3}\input{diagrams/backtrace/backtrace.tex}}%
      \onslide*<3>{\providecommand\step{4}\input{diagrams/backtrace/backtrace.tex}}%
    \end{column}
  \end{columns}
\end{frame}

\begin{frame}{Backtraces}{Back to \funcname{caml\_raise\_exn}}
  \begin{columns}[c]
    \begin{column}{0.5\textwidth}
      \minipage[c][0.66\textheight][s]{\columnwidth}
      \begin{itemize}\color{gray}
        \item Checks wheteher exceptions backtrace should be recorded, by checking the \funcarg{domain\_state->backtrace\_active} value
        \item Switches to the C stack to be able to execute C code
        \item Calls \funcname{caml\_stash\_backtrace}, to collect the backtrace up to the next trap
      \end{itemize}
      \onslide*<2->{
        \begin{itemize}
          \item<2-> Executes the exception handler in \funcname{probably\_raise} with \funcname{RESTORE\_EXN\_HANDLER\_OCAML}
            \begin{itemize}
              \item Set \regname{RSP} to \localname{domain\_state->exn\_handler} value
              \item Restore \localname{domain\_state->exn\_handler} from trap
              \item Jump to the handler code with \funcname{ret}
            \end{itemize}
        \end{itemize}
      }
      \vfill
      \onslide*<1>{\mintocaml[firstline=9,lastline=13,highlightlines=12]{../src/backtrace/backtrace.ml}}
      \onslide*<2->{\mintocaml[firstline=15,lastline=21,highlightlines={19-21}]{../src/backtrace/backtrace.ml}}
      \endminipage
    \end{column}
    \begin{column}{0.5\textwidth}
      \centering
      \onslide*<1>{
        \mintc[firstline=190,lastline=192]{../ocaml/runtime/amd64.S}
        \mintc[firstline=234,lastline=237]{../ocaml/runtime/amd64.S}
      }
      \onslide*<2>{
        \mintc[firstline=190,lastline=192,highlightlines={191-192}]{../ocaml/runtime/amd64.S}
        \mintc[firstline=234,lastline=237,highlightlines={235-237}]{../ocaml/runtime/amd64.S}
      }
      \onslide*<1>{\listCamlRaiseExn[highlightlines=720]{}}
      \onslide*<2>{\listCamlRaiseExn[highlightlines=722]{}}
      \onslide*<3->{\providecommand\step{5}\input{diagrams/backtrace/backtrace.tex}}
    \end{column}
  \end{columns}
\end{frame}

\begin{frame}{Backtraces}{\funcname{caml\_probably\_raise}'s handler doesn't catch \typename{ExnC}}
  \begin{columns}[c]
    \begin{column}{0.45\textwidth}
      \minipage[c][0.75\textheight][s]{\columnwidth}
      \begin{itemize}
        \item<1-> \funcname{match \dots with}\footnotemark pattern matching can also match exceptions
        \item<1-> The CMM of \funcname{probably\_raise} shows that this \funcname{match \dots with} is implemented with a pattern matching and a \funcname{try \dots with}
        \item<1-> But this one only matches on \typename{ExnA} exception
        \item<2-> Since the pattern matching fails to catch the exception, \funcname{reraise} is called with our current \typename{ExnC} exception
        \item<3-> Disassembly shows that \funcname{reraise} is actually a call to \funcname{caml\_reraise\_exn}, which is also an assembly function provided by the runtime
      \end{itemize}
      \vfill
      \mintocaml[firstline=15,lastline=21,highlightlines={18-21}]{../src/backtrace/backtrace.ml}
      \endminipage
    \end{column}
    \begin{column}{0.55\textwidth}
      \centering
      \onslide*<1>{\mintcmm[highlightlines={10-18}]{../src/backtrace/camlBacktrace\_\_probably\_raise\_351.cmm}}
      \onslide*<2>{\listcmm[highlightlines=18]{../src/backtrace/camlBacktrace\_\_probably\_raise_351.cmm}{CMM of probably\_raise}}
      \onslide*<3>{\mintobjdump[firstline=5,lastline=35,highlightlines=31]{../src/backtrace/camlBacktrace\_\_probably\_raise_351.objdump}}
    \end{column}
  \end{columns}
  \only<1>{\footnotetext[1]{https://blog.janestreet.com/pattern-matching-and-exception-handling-unite/}}
\end{frame}

\newcommand\listCamlReraiseExn[1][]{
  \listasm[firstline=727,lastline=734,#1]{../ocaml/runtime/amd64.S}{runtime/amd64.S:727}}

\begin{frame}{Backtraces}{Introducing \funcname{caml\_reraise\_exn}}
  \begin{columns}[c]
    \begin{column}{0.5\textwidth}
      \minipage[c][0.66\textheight][s]{\columnwidth}
      \begin{itemize}
        \item<1-> The exception have to be reraised to the next handler
        \item<2-> First, checks if the backtrace should be collected with \funcarg{domain\_state->backtrace\_active}
        \only<3>{
        \begin{itemize}
            \item If not, behaves like a \funcname{raise\_notrace} with \funcname{RESTORE\_EXN\_HANDLER\_OCAML}
        \end{itemize}
        }
        \item<4-> Jumps into \funcname{caml\_raise\_exn}
        \item<5-> Calls \funcname{caml\_stash\_backtrace} again, to scan the next part of the stack: from the trap just pop-ed to the next trap
      \end{itemize}
      \vfill
      \mintocaml[firstline=15,lastline=21,highlightlines={18-21}]{../src/backtrace/backtrace.ml}
      \endminipage
    \end{column}
    \begin{column}{0.5\textwidth}
      \raggedleft
      \onslide*<1>{\listCamlReraiseExn{}}
      \onslide*<2>{\listCamlReraiseExn[highlightlines=729]{}}
      \onslide*<3>{
        \mintc[firstline=190,lastline=192,highlightlines={191-192}]{../ocaml/runtime/amd64.S}
        \mintc[firstline=234,lastline=237,highlightlines={235-237}]{../ocaml/runtime/amd64.S}
        \medskip
        \listCamlReraiseExn[highlightlines=731]{}
      }
      \onslide*<4-5>{
        % TODO refactor with \listCamlReraiseExn
        \mintasm[firstline=727,lastline=734,highlightlines=730]{../ocaml/runtime/amd64.S}
        \medskip
      }
      \onslide*<4>{\listCamlRaiseExn[highlightlines=711]{}}
      \onslide*<5>{\listCamlRaiseExn[highlightlines=720]{}}
    \end{column}
  \end{columns}
\end{frame}

\begin{frame}{Backtraces}{Backtrace collection continues}
  \begin{columns}[c]
    \begin{column}{0.55\textwidth}
      \minipage[c][0.75\textheight][s]{\columnwidth}
      \begin{itemize}
        \item<1-> \funcname{caml\_stash\_backtrace} scans the older part of the stack, up to the next trap
        \item<6-> Controls jumps to the nested exception handler in \funcname{maybe\_raise}, which matches only on \typename{ExnB}
        \item<7-> \funcname{caml\_reraise\_exn} is called again, backtrace collection resumes with \funcname{caml\_stash\_backtrace}
      \end{itemize}
      \medskip
      \begin{itemize}
        \item<2-> Constructed backtrace:
        \begin{itemize}
          \item <2-> \funcname{definitely\_raise}
          \item <2-> \funcname{probably\_raise}
          \item <3-> \funcname{probably\_raise} (re-raise)
          \item <4-> \funcname{maybe\_raise}
          \item <5-> \funcname{main}
          \item <8-> \funcname{main} (re-raise)
        \end{itemize}
      \end{itemize}
      \vfill
      \onslide*<1-5>{\mintocaml[firstline=15,lastline=21]{../src/backtrace/backtrace.ml}}
      \onslide*<6>{\mintocaml[firstline=28,lastline=34,highlightlines=31]{../src/backtrace/backtrace.ml}}
      \onslide*<7->{\mintocaml[firstline=28,lastline=34,highlightlines=33]{../src/backtrace/backtrace.ml}}
      \endminipage
    \end{column}
    \begin{column}{0.45\textwidth}
      \raggedleft
      \onslide*<1>{\providecommand\step{6}\input{diagrams/backtrace/backtrace.tex}}%
      \onslide*<2>{\providecommand\step{6}\input{diagrams/backtrace/backtrace.tex}}%
      \onslide*<3>{\providecommand\step{7}\input{diagrams/backtrace/backtrace.tex}}%
      \onslide*<4>{\providecommand\step{8}\input{diagrams/backtrace/backtrace.tex}}%
      \onslide*<5>{\providecommand\step{9}\input{diagrams/backtrace/backtrace.tex}}%
      \onslide*<6>{\providecommand\step{10}\input{diagrams/backtrace/backtrace.tex}}%
      \onslide*<7>{\providecommand\step{11}\input{diagrams/backtrace/backtrace.tex}}%
      \onslide*<8>{\providecommand\step{12}\input{diagrams/backtrace/backtrace.tex}}%
      \onslide*<9>{\providecommand\step{13}\input{diagrams/backtrace/backtrace.tex}}%
    \end{column}
  \end{columns}
\end{frame}

\begin{frame}{Backtraces}{Printing the backtrace}
  \begin{columns}[c]
    \begin{column}{0.45\textwidth}
      \minipage[c][0.66\textheight][s]{\columnwidth}
      \begin{itemize}
        \item<1-> The exception backtrace can now be printed to \funcarg{stdout} with \funcname{Printexc.print_backtrace}
        \item<2-> First the exception backtrace is retrieved into an OCaml array with \funcname{get_raw_backtrace }
        \item<3-> Which is implemented as the \funcname{caml_get_exception_raw_backtrace} C function in the runtime
        \item<4-> And finally printed with \funcname{print_exception_backtrace}
      \end{itemize}
      \vfill
      \mintocaml[firstline=28,lastline=34,highlightlines=33]{../src/backtrace/backtrace.ml}
      \endminipage
    \end{column}
    \begin{column}{0.55\textwidth}
      \onslide*<1,2>{
        \mintocaml[firstline=100,lastline=101]{../ocaml/stdlib/printexc.ml}
      }
      \only<1>{
        \listocaml[firstline=170,lastline=172]{../ocaml/stdlib/printexc.ml}{stdlib/printexc.ml:170}
      }
      \only<2>{
        \listocaml[firstline=170,lastline=172,highlightlines=172]{../ocaml/stdlib/printexc.ml}{stdlib/printexc.ml:170}
      }
      \only<3>{
        \mintc[firstline=154,lastline=156]{../ocaml/runtime/backtrace.c}
        \listc[firstline=171,lastline=192,highlightlines={184-187}]{../ocaml/runtime/backtrace.c}{runtime/backtrace.c:154}
      }
      \only<4>{
        \listocaml[firstline=155,lastline=165]{../ocaml/stdlib/printexc.ml}{stdlib/printexc.ml:55}
      }
    \end{column}
  \end{columns}
\end{frame}


\subsection{The multiple kinds of raise}
\frameSubsection{}{}

\begin{frame}{Backtraces}{When backtraces are not needed}
  \begin{columns}[c]
    \begin{column}{0.45\textwidth}
      \minipage[c][0.66\textheight][s]{\columnwidth}
      \begin{itemize}
        \item<1-> What if the backtrace for an exception is never needed?
        \item<2-> It's possible to raise the exception explicitly with \funcname{raise\_notrace} to make raising as cheap as possible
        \item<3-> An example of use in the \funcname{dynlink} library of the OCaml compiler
      \end{itemize}
      \vfill
      \onslide<3->{
      \listocaml[firstline=205,lastline=209,highlightlines=207]{../ocaml/otherlibs/dynlink/dynlink\_compilerlibs/misc.ml}{otherlibs/dynlink/dynlink\_compilerlibs/misc.ml:205}
      }
      \endminipage
    \end{column}
    \begin{column}{0.55\textwidth}
    \only<4>{
      \mintcmm[highlightlines=3]{../src/notrace/camlNotrace\_\_fun_347.cmm}
      \listcmm{../src/notrace/camlNotrace\_\_all\_somes\_327.cmm}{all\_somes}
    }
    \onslide*<5->{
      \listobjdump[highlightlines={6-9}]{../src/notrace/camlNotrace\_\_fun_347.objdump}{anonymous function from \funcname{all\_somes}}
    }
    \end{column}
  \end{columns}
\end{frame}

\begin{frame}{Backtraces}{Summary table of the multiple kinds of raise}
  \begin{itemize}
    \item When debugging information is enabled and if the backtrace of an exception isn't needed, it's possible to raise with \funcname{raise\_notrace} for performance
    \item When debugging information is \emph{not} enabled, the compiler optimises \funcname{raise} calls to inline assembly (same effect as using \funcname{raise\_notrace})
  \end{itemize}
  \bigskip
  \centering
  \begin{tabular}{| l | c | c | c | c |}
    \hline
    \parbox{10em}{Debug information \\ (\funcarg{-g} flag)}  & \multicolumn{2}{|c|}{Disabled}                                     & \multicolumn{2}{|c|}{Enabled} \\
    \hline
    Raise kind                                               & \funcname{raise} & \funcname{raise\_notrace}                       & \funcname{raise} & \funcname{raise\_notrace} \\
    \hline
    Compilation                                              & \parbox{8em}{inline assembly\\ (optimization)} & inline assembly   & \funcname{caml\_raise\_exn} & inline assembly \\
    \hline
  \end{tabular}
\end{frame}


%
% Takeaway #5
%
\subsection*{Takeaway \#5}
\frameSubsectionTakeaway{}
\againframe<5>{takeaway}
}%
    \end{column}
  \end{columns}
\end{frame}

\begin{frame}{Backtraces}{Back to \funcname{caml\_raise\_exn}}
  \begin{columns}[c]
    \begin{column}{0.5\textwidth}
      \minipage[c][0.66\textheight][s]{\columnwidth}
      \begin{itemize}\color{gray}
        \item Checks wheteher exceptions backtrace should be recorded, by checking the \funcarg{domain\_state->backtrace\_active} value
        \item Switches to the C stack to be able to execute C code
        \item Calls \funcname{caml\_stash\_backtrace}, to collect the backtrace up to the next trap
      \end{itemize}
      \onslide*<2->{
        \begin{itemize}
          \item<2-> Executes the exception handler in \funcname{probably\_raise} with \funcname{RESTORE\_EXN\_HANDLER\_OCAML}
            \begin{itemize}
              \item Set \regname{RSP} to \localname{domain\_state->exn\_handler} value
              \item Restore \localname{domain\_state->exn\_handler} from trap
              \item Jump to the handler code with \funcname{ret}
            \end{itemize}
        \end{itemize}
      }
      \vfill
      \onslide*<1>{\mintocaml[firstline=9,lastline=13,highlightlines=12]{../src/backtrace/backtrace.ml}}
      \onslide*<2->{\mintocaml[firstline=15,lastline=21,highlightlines={19-21}]{../src/backtrace/backtrace.ml}}
      \endminipage
    \end{column}
    \begin{column}{0.5\textwidth}
      \centering
      \onslide*<1>{
        \mintc[firstline=190,lastline=192]{../ocaml/runtime/amd64.S}
        \mintc[firstline=234,lastline=237]{../ocaml/runtime/amd64.S}
      }
      \onslide*<2>{
        \mintc[firstline=190,lastline=192,highlightlines={191-192}]{../ocaml/runtime/amd64.S}
        \mintc[firstline=234,lastline=237,highlightlines={235-237}]{../ocaml/runtime/amd64.S}
      }
      \onslide*<1>{\listCamlRaiseExn[highlightlines=720]{}}
      \onslide*<2>{\listCamlRaiseExn[highlightlines=722]{}}
      \onslide*<3->{\providecommand\step{5}%
% Backtraces
%
\subsection{One advantage of raising with debugging information}
\frameSubsection{}{}

\begin{frame}{Backtraces}{Debugging information \& source code}
  \begin{columns}[c]
    \begin{column}{0.5\textwidth}
      \begin{itemize}
        \item Raising an exception is fun and games, but how do we know where the exception originated? $\rightarrow$ With the backtrace associated with the exception!
        \item In order to get a backtrace from the point where an exception was raised, it's necessary to compile with debugging information enabled (\funcarg{-g} flag for the compiler)
        \item Same example as in the `Nested handlers' section
      \end{itemize}
    \end{column}
    \begin{column}{0.5\textwidth}
      \listocaml[firstline=9,lastline=34]{../src/backtrace/backtrace.ml}{backtrace/backtrace.ml}
    \end{column}
  \end{columns}
\end{frame}

\begin{frame}{Backtraces}{CMM and disassembly of \funcname{definitely\_raise}}
  \begin{columns}[c]
    \begin{column}{0.5\textwidth}
      \minipage[c][0.66\textheight][s]{\columnwidth}
      \begin{itemize}
        \item<1-> An effect of compiling with debugging information (\funcarg{-g}): the CMM no longer uses \funcname{raise\_notrace} to raise the exception but \funcname{raise} instead
        \item<2-> Disassembly shows that CMM \funcname{raise} is actually a call to \funcname{caml\_raise\_exn}, a function in assembler provided by the runtime
      \end{itemize}
      \vfill
      \mintocaml[firstline=9,lastline=13,highlightlines=12]{../src/backtrace/backtrace.ml}
      \endminipage
    \end{column}
    \begin{column}{0.5\textwidth}
      \onslide*<1>{
        \mintcmm[highlightlines=5]{../src/backtrace/camlBacktrace\_\_definitely\_raise\_347.cmm}
      }
      \onslide*<2>{
        \mintobjdump[firstline=2,highlightlines=14]{../src/backtrace/camlBacktrace\_\_definitely\_raise\_347.objdump}
      }
    \end{column}
  \end{columns}
\end{frame}

\begin{frame}{Backtraces}{Introducing \funcname{caml\_raise\_exn}}
  \begin{columns}[c]
    \begin{column}{0.5\textwidth}
      \minipage[c][0.66\textheight][s]{\columnwidth}
      \onslide*<1>{
        \begin{itemize}
          \item Raising the exception is performed using \funcname{caml\_raise\_exn} which is
            \begin{itemize}
              \item An assembly function provided by the runtime
              \item Architecture specific
            \end{itemize}
          \item Let's see what it does differently from \funcname{raise\_notrace}\dots
          \item We are about to raise \typename{ExnC} with \funcname{caml\_raise\_exn} from \funcname{definitely\_raise}
        \end{itemize}
      }
      \onslide*<2>{
        \mintobjdump[firstline=2,highlightlines=14]{../src/backtrace/camlBacktrace\_\_definitely\_raise\_347.objdump}
      }
      \vfill
      \mintocaml[firstline=9,lastline=13,highlightlines=12]{../src/backtrace/backtrace.ml}
      \endminipage
    \end{column}
    \begin{column}{0.5\textwidth}
      \centering
      \providecommand\step{1}\input{diagrams/backtrace/backtrace.tex}
    \end{column}
  \end{columns}
\end{frame}

\begin{frame}{Backtraces}{But before that, we need more fields from \typename{caml\_domain\_state}}
  \begin{columns}
    \begin{column}{0.5\textwidth}
      \begin{itemize}
        \item Remember \typename{caml\_domain\_state} the per-domain runtime state?
        \item Some other members are of interest to us now\dots
        \item \localname{backtrace\_active} is used to know if the runtime should collect backtraces
        \item \localname{backtrace\_active} can be set by
          \begin{itemize}
            \item The \funcarg{b} flag in the \funcname{OCAMLRUNPARAM} environment variable
            \item \mintinline{ocaml}{Printexc.record_backtrace : bool -> unit} \footnotemark
          \end{itemize}
      \end{itemize}
    \end{column}
    \begin{column}{0.5\textwidth}
      \begin{listing}
        \mintc[highlightlines={9-11}]{src/ocaml/domain\_state\_backtrace.h}
        \caption{runtime/caml/domain\_state.h}
      \end{listing}
    \end{column}
  \end{columns}
  \footnotetext[1]{https://v2.ocaml.org/api/Printexc.html}
\end{frame}

\newcommand\listCamlRaiseExn[1][]{
  \listasm[firstline=702,lastline=725,#1]{../ocaml/runtime/amd64.S}{runtime/amd64.S:702}}

\begin{frame}{Backtraces}{Walk through \funcname{caml\_raise\_exn}}
  \begin{columns}[T]
    \begin{column}{0.5\textwidth}
      \begin{itemize}
        \item<1-> First checks whether exceptions backtrace should be recorded
        \item<1-> By checking the \funcarg{domain\_state->backtrace\_active} value
        \item<2-> If not, acts as \funcname{raise\_notrace} with \funcname{RESTORE\_EXN\_HANDLER\_OCAML}
          \begin{itemize}
            \item Set \regname{RSP} to \localname{domain\_state->exn\_handler} value
            \item Restore \localname{domain\_state->exn\_handler} from trap
            \item Jump with \funcname{ret} (same effect as using \funcname{pop} and \funcname{jmp})
          \end{itemize}
        \item<3-> Otherwise, switches to the C stack to be able to execute C code
        \item<4-> Calls \funcname{caml\_stash\_backtrace}, a C function provided by the runtime\ignorespaces
          \onslide<6->{, with
            \begin{itemize}
              \item \funcarg{exn}: exception value
              \item \funcarg{pc}: code address of the raising point
              \item \funcarg{sp}: stack address of the raising function
              \item \funcarg{trapsp}: stack address of the next handler
            \end{itemize}
          }
      \end{itemize}
      \bigskip
      \mintocaml[firstline=9,lastline=13,highlightlines=12]{../src/backtrace/backtrace.ml}
    \end{column}
    \begin{column}{0.5\textwidth}
      \raggedleft
      \onslide*<1,3-4>{
        \mintc[firstline=190,lastline=192]{../ocaml/runtime/amd64.S}
        \mintc[firstline=234,lastline=237]{../ocaml/runtime/amd64.S}
      }
      \onslide*<2>{
        \mintc[firstline=190,lastline=192,highlightlines={191-192}]{../ocaml/runtime/amd64.S}
        \mintc[firstline=234,lastline=237,highlightlines={235-237}]{../ocaml/runtime/amd64.S}
      }
      \onslide*<1>{\listCamlRaiseExn[highlightlines=705]{}}%
      \onslide*<2>{\listCamlRaiseExn[highlightlines={707-708}]{}}%
      \onslide*<3>{\listCamlRaiseExn[highlightlines={712-714}]{}}%
      \onslide*<4>{\listCamlRaiseExn[highlightlines={715-720}]{}}%
      \onslide*<5>{\providecommand\step{1}\input{diagrams/backtrace/backtrace.tex}}%
      \onslide*<6>{\providecommand\step{2}\input{diagrams/backtrace/backtrace.tex}}%
    \end{column}
  \end{columns}
\end{frame}

\newcommand\listStashBacktrace[1][]{
  \listc[firstline=91,lastline=120,#1]{../ocaml/runtime/backtrace\_nat.c}{runtime/backtrace\_nat.c:91}}

\begin{frame}{Backtraces}{Introducing \funcname{caml\_stash\_backtrace}}
  \begin{columns}[c]
    \begin{column}{0.5\textwidth}
      \minipage[c][0.66\textheight][s]{\columnwidth}
      \begin{itemize}
        \item<1-> A C function provided by the runtime (specific to native code)
        \item<2-> It scans the stack, between the stack pointer at the raising point up to the next trap, in order to collect the names of the detected function frames
          \onslide*<2>{
            \begin{itemize}
              \item And more information, like line number, column number...
            \end{itemize}
          }
        \item<3-> It uses the same technique and information as the Garbage Collector (GC) uses to scan the values live on the stack
          \onslide*<4>{
            \begin{itemize}
              \item \typename{frame\_descr} are at the core of stack unwinding
            \end{itemize}
          }
      \end{itemize}
      \vfill
      \mintocaml[firstline=9,lastline=13,highlightlines=12]{../src/backtrace/backtrace.ml}
      \endminipage
    \end{column}
    \begin{column}{0.5\textwidth}
      \onslide*<1>{\listStashBacktrace[highlightlines=91]{}}
      \onslide*<2>{\listStashBacktrace[highlightlines={91,117-118}]{}}
      \onslide*<3->{\listStashBacktrace[highlightlines={94,106,109-110}]{}}
    \end{column}
  \end{columns}
\end{frame}

\newcommand\listFrameDescr[1][]{
  \listocaml[firstline=111,lastline=124,#1]{../ocaml/asmcomp/emitaux.ml}{asmcomp/emitaux.ml:111}}
\newcommand\listDebugInfo[1][]{
  \listocaml[firstline=38,lastline=49,#1]{../ocaml/lambda/debuginfo.mli}{lambda/debuginfo.mli:38}}

\begin{frame}{Backtraces}{\typename{frame\_descr} and \typename{frame\_debuginfo} in a nutshell}
  \begin{columns}[c]
    \begin{column}{0.5\textwidth}
      \begin{itemize}
        \item<1-> For every return address in an OCaml program, there is a associated \typename{frame\_descr}
        \item<2-> A \typename{frame\_descr} specifies
        \begin{itemize}
          \item<2-> The size of the function stack frame
          \item<3-> Where live values are on the stack (so that the GC can mark them as live and prevent from being swept)
        \end{itemize}
        \item<4-> When compiled with debug information, \typename{frame\_descr} will be linked to a \typename{frame\_debuginfo}
        \begin{itemize}
          \item<5-> Contains information about the position in the source code (file name, line and column number)
        \end{itemize}
      \end{itemize}
    \end{column}
    \begin{column}{0.5\textwidth}
        \onslide*<1>{\listFrameDescr[highlightlines={118-122}]{}}
        \onslide*<2>{\listFrameDescr[highlightlines={120}]{}}
        \onslide*<3>{\listFrameDescr[highlightlines={121}]{}}
        \onslide*<4->{\listFrameDescr[highlightlines={122}]{}}
        \medskip
        \onslide*<1-3>{\listDebugInfo{}}
        \onslide*<4>{\listDebugInfo[highlightlines={38,49}]{}}
        \onslide*<5>{\listDebugInfo[highlightlines={39-46}]{}}
    \end{column}
  \end{columns}
\end{frame}

\begin{frame}{Backtraces}{Scanning the stack with \typename{frame\_descr}}
  \begin{columns}[c]
    \begin{column}{0.5\textwidth}
      \begin{itemize}
        \item Given the return address of the code that raises the exception by calling \funcname{caml\_raise\_exn} (\funcarg{pc} argument of \funcname{caml\_stash\_backtrace})
        \item And the position of the stack pointer (\funcarg{sp} argument of \funcname{caml\_stash\_backtrace})
        \item The frame size given by \typename{frame\_descr}.\funcarg{frame\_size}, allows to find the end of the previous frame
        \item And the return address of the caller of this function
        \bigskip
        \onslide<3->{
        \item Note that traps (here the reddish \funcname{ExnA}, \funcname{ExnB} and \funcname{ExnC} blocks) belong to the frame that installed them, and so the frame size changes when traps are removed
        }
      \end{itemize}
    \end{column}
    \begin{column}{0.5\textwidth}
      \raggedleft
      \onslide*<1>{\providecommand\step{2}\input{diagrams/backtrace/backtrace.tex}}%
      \onslide*<2>{\providecommand\step{1}\input{diagrams/backtrace/scanstack.tex}}%
      \onslide*<3>{\providecommand\step{2}\input{diagrams/backtrace/scanstack.tex}}%
      \onslide*<4>{\providecommand\step{3}\input{diagrams/backtrace/scanstack.tex}}%
      \onslide*<5>{\providecommand\step{4}\input{diagrams/backtrace/scanstack.tex}}%
      \onslide*<6>{\providecommand\step{5}\input{diagrams/backtrace/scanstack.tex}}%
    \end{column}
  \end{columns}
\end{frame}

% Duplicate of previous-previous slide
\begin{frame}{Backtraces}{Continuing on \funcname{caml\_stash\_backtrace}}
  \begin{columns}[c]
    \begin{column}{0.5\textwidth}
      \minipage[c][0.75\textheight][s]{\columnwidth}
      \begin{itemize}
          \color{gray}
        \item A C function provided by the runtime (specific to native code)
        \item It scans the stack, between the stack pointer at the raising point up to the next trap, in order to collect the names of the detected function frames
        \item It uses the same technique and information as the Garbage Collector (GC) uses to scan the values live on the stack
      \end{itemize}
      \begin{itemize}
        \item For each function frame detected, store a pointer to the \typename{frame\_descr} in the \localname{domain\_state->backtrace\_buffer} array, effectively constructing the backtrace
      \end{itemize}
      \onslide<2->{
        \begin{itemize}
            \smallskip
          \item Constructed backtrace:
            \funcname{definitely\_raise},
            \onslide<3->{\funcname{probably\_raise}}
        \end{itemize}
      }
      \vfill
      \mintocaml[firstline=9,lastline=13,highlightlines=12]{../src/backtrace/backtrace.ml}
      \endminipage
    \end{column}
    \begin{column}{0.5\textwidth}
      \raggedleft
      \onslide*<1>{\listStashBacktrace[highlightlines={114-115}]{}}
      \onslide*<2>{\providecommand\step{3}\input{diagrams/backtrace/backtrace.tex}}%
      \onslide*<3>{\providecommand\step{4}\input{diagrams/backtrace/backtrace.tex}}%
    \end{column}
  \end{columns}
\end{frame}

\begin{frame}{Backtraces}{Back to \funcname{caml\_raise\_exn}}
  \begin{columns}[c]
    \begin{column}{0.5\textwidth}
      \minipage[c][0.66\textheight][s]{\columnwidth}
      \begin{itemize}\color{gray}
        \item Checks wheteher exceptions backtrace should be recorded, by checking the \funcarg{domain\_state->backtrace\_active} value
        \item Switches to the C stack to be able to execute C code
        \item Calls \funcname{caml\_stash\_backtrace}, to collect the backtrace up to the next trap
      \end{itemize}
      \onslide*<2->{
        \begin{itemize}
          \item<2-> Executes the exception handler in \funcname{probably\_raise} with \funcname{RESTORE\_EXN\_HANDLER\_OCAML}
            \begin{itemize}
              \item Set \regname{RSP} to \localname{domain\_state->exn\_handler} value
              \item Restore \localname{domain\_state->exn\_handler} from trap
              \item Jump to the handler code with \funcname{ret}
            \end{itemize}
        \end{itemize}
      }
      \vfill
      \onslide*<1>{\mintocaml[firstline=9,lastline=13,highlightlines=12]{../src/backtrace/backtrace.ml}}
      \onslide*<2->{\mintocaml[firstline=15,lastline=21,highlightlines={19-21}]{../src/backtrace/backtrace.ml}}
      \endminipage
    \end{column}
    \begin{column}{0.5\textwidth}
      \centering
      \onslide*<1>{
        \mintc[firstline=190,lastline=192]{../ocaml/runtime/amd64.S}
        \mintc[firstline=234,lastline=237]{../ocaml/runtime/amd64.S}
      }
      \onslide*<2>{
        \mintc[firstline=190,lastline=192,highlightlines={191-192}]{../ocaml/runtime/amd64.S}
        \mintc[firstline=234,lastline=237,highlightlines={235-237}]{../ocaml/runtime/amd64.S}
      }
      \onslide*<1>{\listCamlRaiseExn[highlightlines=720]{}}
      \onslide*<2>{\listCamlRaiseExn[highlightlines=722]{}}
      \onslide*<3->{\providecommand\step{5}\input{diagrams/backtrace/backtrace.tex}}
    \end{column}
  \end{columns}
\end{frame}

\begin{frame}{Backtraces}{\funcname{caml\_probably\_raise}'s handler doesn't catch \typename{ExnC}}
  \begin{columns}[c]
    \begin{column}{0.45\textwidth}
      \minipage[c][0.75\textheight][s]{\columnwidth}
      \begin{itemize}
        \item<1-> \funcname{match \dots with}\footnotemark pattern matching can also match exceptions
        \item<1-> The CMM of \funcname{probably\_raise} shows that this \funcname{match \dots with} is implemented with a pattern matching and a \funcname{try \dots with}
        \item<1-> But this one only matches on \typename{ExnA} exception
        \item<2-> Since the pattern matching fails to catch the exception, \funcname{reraise} is called with our current \typename{ExnC} exception
        \item<3-> Disassembly shows that \funcname{reraise} is actually a call to \funcname{caml\_reraise\_exn}, which is also an assembly function provided by the runtime
      \end{itemize}
      \vfill
      \mintocaml[firstline=15,lastline=21,highlightlines={18-21}]{../src/backtrace/backtrace.ml}
      \endminipage
    \end{column}
    \begin{column}{0.55\textwidth}
      \centering
      \onslide*<1>{\mintcmm[highlightlines={10-18}]{../src/backtrace/camlBacktrace\_\_probably\_raise\_351.cmm}}
      \onslide*<2>{\listcmm[highlightlines=18]{../src/backtrace/camlBacktrace\_\_probably\_raise_351.cmm}{CMM of probably\_raise}}
      \onslide*<3>{\mintobjdump[firstline=5,lastline=35,highlightlines=31]{../src/backtrace/camlBacktrace\_\_probably\_raise_351.objdump}}
    \end{column}
  \end{columns}
  \only<1>{\footnotetext[1]{https://blog.janestreet.com/pattern-matching-and-exception-handling-unite/}}
\end{frame}

\newcommand\listCamlReraiseExn[1][]{
  \listasm[firstline=727,lastline=734,#1]{../ocaml/runtime/amd64.S}{runtime/amd64.S:727}}

\begin{frame}{Backtraces}{Introducing \funcname{caml\_reraise\_exn}}
  \begin{columns}[c]
    \begin{column}{0.5\textwidth}
      \minipage[c][0.66\textheight][s]{\columnwidth}
      \begin{itemize}
        \item<1-> The exception have to be reraised to the next handler
        \item<2-> First, checks if the backtrace should be collected with \funcarg{domain\_state->backtrace\_active}
        \only<3>{
        \begin{itemize}
            \item If not, behaves like a \funcname{raise\_notrace} with \funcname{RESTORE\_EXN\_HANDLER\_OCAML}
        \end{itemize}
        }
        \item<4-> Jumps into \funcname{caml\_raise\_exn}
        \item<5-> Calls \funcname{caml\_stash\_backtrace} again, to scan the next part of the stack: from the trap just pop-ed to the next trap
      \end{itemize}
      \vfill
      \mintocaml[firstline=15,lastline=21,highlightlines={18-21}]{../src/backtrace/backtrace.ml}
      \endminipage
    \end{column}
    \begin{column}{0.5\textwidth}
      \raggedleft
      \onslide*<1>{\listCamlReraiseExn{}}
      \onslide*<2>{\listCamlReraiseExn[highlightlines=729]{}}
      \onslide*<3>{
        \mintc[firstline=190,lastline=192,highlightlines={191-192}]{../ocaml/runtime/amd64.S}
        \mintc[firstline=234,lastline=237,highlightlines={235-237}]{../ocaml/runtime/amd64.S}
        \medskip
        \listCamlReraiseExn[highlightlines=731]{}
      }
      \onslide*<4-5>{
        % TODO refactor with \listCamlReraiseExn
        \mintasm[firstline=727,lastline=734,highlightlines=730]{../ocaml/runtime/amd64.S}
        \medskip
      }
      \onslide*<4>{\listCamlRaiseExn[highlightlines=711]{}}
      \onslide*<5>{\listCamlRaiseExn[highlightlines=720]{}}
    \end{column}
  \end{columns}
\end{frame}

\begin{frame}{Backtraces}{Backtrace collection continues}
  \begin{columns}[c]
    \begin{column}{0.55\textwidth}
      \minipage[c][0.75\textheight][s]{\columnwidth}
      \begin{itemize}
        \item<1-> \funcname{caml\_stash\_backtrace} scans the older part of the stack, up to the next trap
        \item<6-> Controls jumps to the nested exception handler in \funcname{maybe\_raise}, which matches only on \typename{ExnB}
        \item<7-> \funcname{caml\_reraise\_exn} is called again, backtrace collection resumes with \funcname{caml\_stash\_backtrace}
      \end{itemize}
      \medskip
      \begin{itemize}
        \item<2-> Constructed backtrace:
        \begin{itemize}
          \item <2-> \funcname{definitely\_raise}
          \item <2-> \funcname{probably\_raise}
          \item <3-> \funcname{probably\_raise} (re-raise)
          \item <4-> \funcname{maybe\_raise}
          \item <5-> \funcname{main}
          \item <8-> \funcname{main} (re-raise)
        \end{itemize}
      \end{itemize}
      \vfill
      \onslide*<1-5>{\mintocaml[firstline=15,lastline=21]{../src/backtrace/backtrace.ml}}
      \onslide*<6>{\mintocaml[firstline=28,lastline=34,highlightlines=31]{../src/backtrace/backtrace.ml}}
      \onslide*<7->{\mintocaml[firstline=28,lastline=34,highlightlines=33]{../src/backtrace/backtrace.ml}}
      \endminipage
    \end{column}
    \begin{column}{0.45\textwidth}
      \raggedleft
      \onslide*<1>{\providecommand\step{6}\input{diagrams/backtrace/backtrace.tex}}%
      \onslide*<2>{\providecommand\step{6}\input{diagrams/backtrace/backtrace.tex}}%
      \onslide*<3>{\providecommand\step{7}\input{diagrams/backtrace/backtrace.tex}}%
      \onslide*<4>{\providecommand\step{8}\input{diagrams/backtrace/backtrace.tex}}%
      \onslide*<5>{\providecommand\step{9}\input{diagrams/backtrace/backtrace.tex}}%
      \onslide*<6>{\providecommand\step{10}\input{diagrams/backtrace/backtrace.tex}}%
      \onslide*<7>{\providecommand\step{11}\input{diagrams/backtrace/backtrace.tex}}%
      \onslide*<8>{\providecommand\step{12}\input{diagrams/backtrace/backtrace.tex}}%
      \onslide*<9>{\providecommand\step{13}\input{diagrams/backtrace/backtrace.tex}}%
    \end{column}
  \end{columns}
\end{frame}

\begin{frame}{Backtraces}{Printing the backtrace}
  \begin{columns}[c]
    \begin{column}{0.45\textwidth}
      \minipage[c][0.66\textheight][s]{\columnwidth}
      \begin{itemize}
        \item<1-> The exception backtrace can now be printed to \funcarg{stdout} with \funcname{Printexc.print_backtrace}
        \item<2-> First the exception backtrace is retrieved into an OCaml array with \funcname{get_raw_backtrace }
        \item<3-> Which is implemented as the \funcname{caml_get_exception_raw_backtrace} C function in the runtime
        \item<4-> And finally printed with \funcname{print_exception_backtrace}
      \end{itemize}
      \vfill
      \mintocaml[firstline=28,lastline=34,highlightlines=33]{../src/backtrace/backtrace.ml}
      \endminipage
    \end{column}
    \begin{column}{0.55\textwidth}
      \onslide*<1,2>{
        \mintocaml[firstline=100,lastline=101]{../ocaml/stdlib/printexc.ml}
      }
      \only<1>{
        \listocaml[firstline=170,lastline=172]{../ocaml/stdlib/printexc.ml}{stdlib/printexc.ml:170}
      }
      \only<2>{
        \listocaml[firstline=170,lastline=172,highlightlines=172]{../ocaml/stdlib/printexc.ml}{stdlib/printexc.ml:170}
      }
      \only<3>{
        \mintc[firstline=154,lastline=156]{../ocaml/runtime/backtrace.c}
        \listc[firstline=171,lastline=192,highlightlines={184-187}]{../ocaml/runtime/backtrace.c}{runtime/backtrace.c:154}
      }
      \only<4>{
        \listocaml[firstline=155,lastline=165]{../ocaml/stdlib/printexc.ml}{stdlib/printexc.ml:55}
      }
    \end{column}
  \end{columns}
\end{frame}


\subsection{The multiple kinds of raise}
\frameSubsection{}{}

\begin{frame}{Backtraces}{When backtraces are not needed}
  \begin{columns}[c]
    \begin{column}{0.45\textwidth}
      \minipage[c][0.66\textheight][s]{\columnwidth}
      \begin{itemize}
        \item<1-> What if the backtrace for an exception is never needed?
        \item<2-> It's possible to raise the exception explicitly with \funcname{raise\_notrace} to make raising as cheap as possible
        \item<3-> An example of use in the \funcname{dynlink} library of the OCaml compiler
      \end{itemize}
      \vfill
      \onslide<3->{
      \listocaml[firstline=205,lastline=209,highlightlines=207]{../ocaml/otherlibs/dynlink/dynlink\_compilerlibs/misc.ml}{otherlibs/dynlink/dynlink\_compilerlibs/misc.ml:205}
      }
      \endminipage
    \end{column}
    \begin{column}{0.55\textwidth}
    \only<4>{
      \mintcmm[highlightlines=3]{../src/notrace/camlNotrace\_\_fun_347.cmm}
      \listcmm{../src/notrace/camlNotrace\_\_all\_somes\_327.cmm}{all\_somes}
    }
    \onslide*<5->{
      \listobjdump[highlightlines={6-9}]{../src/notrace/camlNotrace\_\_fun_347.objdump}{anonymous function from \funcname{all\_somes}}
    }
    \end{column}
  \end{columns}
\end{frame}

\begin{frame}{Backtraces}{Summary table of the multiple kinds of raise}
  \begin{itemize}
    \item When debugging information is enabled and if the backtrace of an exception isn't needed, it's possible to raise with \funcname{raise\_notrace} for performance
    \item When debugging information is \emph{not} enabled, the compiler optimises \funcname{raise} calls to inline assembly (same effect as using \funcname{raise\_notrace})
  \end{itemize}
  \bigskip
  \centering
  \begin{tabular}{| l | c | c | c | c |}
    \hline
    \parbox{10em}{Debug information \\ (\funcarg{-g} flag)}  & \multicolumn{2}{|c|}{Disabled}                                     & \multicolumn{2}{|c|}{Enabled} \\
    \hline
    Raise kind                                               & \funcname{raise} & \funcname{raise\_notrace}                       & \funcname{raise} & \funcname{raise\_notrace} \\
    \hline
    Compilation                                              & \parbox{8em}{inline assembly\\ (optimization)} & inline assembly   & \funcname{caml\_raise\_exn} & inline assembly \\
    \hline
  \end{tabular}
\end{frame}


%
% Takeaway #5
%
\subsection*{Takeaway \#5}
\frameSubsectionTakeaway{}
\againframe<5>{takeaway}
}
    \end{column}
  \end{columns}
\end{frame}

\begin{frame}{Backtraces}{\funcname{caml\_probably\_raise}'s handler doesn't catch \typename{ExnC}}
  \begin{columns}[c]
    \begin{column}{0.45\textwidth}
      \minipage[c][0.75\textheight][s]{\columnwidth}
      \begin{itemize}
        \item<1-> \funcname{match \dots with}\footnotemark pattern matching can also match exceptions
        \item<1-> The CMM of \funcname{probably\_raise} shows that this \funcname{match \dots with} is implemented with a pattern matching and a \funcname{try \dots with}
        \item<1-> But this one only matches on \typename{ExnA} exception
        \item<2-> Since the pattern matching fails to catch the exception, \funcname{reraise} is called with our current \typename{ExnC} exception
        \item<3-> Disassembly shows that \funcname{reraise} is actually a call to \funcname{caml\_reraise\_exn}, which is also an assembly function provided by the runtime
      \end{itemize}
      \vfill
      \mintocaml[firstline=15,lastline=21,highlightlines={18-21}]{../src/backtrace/backtrace.ml}
      \endminipage
    \end{column}
    \begin{column}{0.55\textwidth}
      \centering
      \onslide*<1>{\mintcmm[highlightlines={10-18}]{../src/backtrace/camlBacktrace\_\_probably\_raise\_351.cmm}}
      \onslide*<2>{\listcmm[highlightlines=18]{../src/backtrace/camlBacktrace\_\_probably\_raise_351.cmm}{CMM of probably\_raise}}
      \onslide*<3>{\mintobjdump[firstline=5,lastline=35,highlightlines=31]{../src/backtrace/camlBacktrace\_\_probably\_raise_351.objdump}}
    \end{column}
  \end{columns}
  \only<1>{\footnotetext[1]{https://blog.janestreet.com/pattern-matching-and-exception-handling-unite/}}
\end{frame}

\newcommand\listCamlReraiseExn[1][]{
  \listasm[firstline=727,lastline=734,#1]{../ocaml/runtime/amd64.S}{runtime/amd64.S:727}}

\begin{frame}{Backtraces}{Introducing \funcname{caml\_reraise\_exn}}
  \begin{columns}[c]
    \begin{column}{0.5\textwidth}
      \minipage[c][0.66\textheight][s]{\columnwidth}
      \begin{itemize}
        \item<1-> The exception have to be reraised to the next handler
        \item<2-> First, checks if the backtrace should be collected with \funcarg{domain\_state->backtrace\_active}
        \only<3>{
        \begin{itemize}
            \item If not, behaves like a \funcname{raise\_notrace} with \funcname{RESTORE\_EXN\_HANDLER\_OCAML}
        \end{itemize}
        }
        \item<4-> Jumps into \funcname{caml\_raise\_exn}
        \item<5-> Calls \funcname{caml\_stash\_backtrace} again, to scan the next part of the stack: from the trap just pop-ed to the next trap
      \end{itemize}
      \vfill
      \mintocaml[firstline=15,lastline=21,highlightlines={18-21}]{../src/backtrace/backtrace.ml}
      \endminipage
    \end{column}
    \begin{column}{0.5\textwidth}
      \raggedleft
      \onslide*<1>{\listCamlReraiseExn{}}
      \onslide*<2>{\listCamlReraiseExn[highlightlines=729]{}}
      \onslide*<3>{
        \mintc[firstline=190,lastline=192,highlightlines={191-192}]{../ocaml/runtime/amd64.S}
        \mintc[firstline=234,lastline=237,highlightlines={235-237}]{../ocaml/runtime/amd64.S}
        \medskip
        \listCamlReraiseExn[highlightlines=731]{}
      }
      \onslide*<4-5>{
        % TODO refactor with \listCamlReraiseExn
        \mintasm[firstline=727,lastline=734,highlightlines=730]{../ocaml/runtime/amd64.S}
        \medskip
      }
      \onslide*<4>{\listCamlRaiseExn[highlightlines=711]{}}
      \onslide*<5>{\listCamlRaiseExn[highlightlines=720]{}}
    \end{column}
  \end{columns}
\end{frame}

\begin{frame}{Backtraces}{Backtrace collection continues}
  \begin{columns}[c]
    \begin{column}{0.55\textwidth}
      \minipage[c][0.75\textheight][s]{\columnwidth}
      \begin{itemize}
        \item<1-> \funcname{caml\_stash\_backtrace} scans the older part of the stack, up to the next trap
        \item<6-> Controls jumps to the nested exception handler in \funcname{maybe\_raise}, which matches only on \typename{ExnB}
        \item<7-> \funcname{caml\_reraise\_exn} is called again, backtrace collection resumes with \funcname{caml\_stash\_backtrace}
      \end{itemize}
      \medskip
      \begin{itemize}
        \item<2-> Constructed backtrace:
        \begin{itemize}
          \item <2-> \funcname{definitely\_raise}
          \item <2-> \funcname{probably\_raise}
          \item <3-> \funcname{probably\_raise} (re-raise)
          \item <4-> \funcname{maybe\_raise}
          \item <5-> \funcname{main}
          \item <8-> \funcname{main} (re-raise)
        \end{itemize}
      \end{itemize}
      \vfill
      \onslide*<1-5>{\mintocaml[firstline=15,lastline=21]{../src/backtrace/backtrace.ml}}
      \onslide*<6>{\mintocaml[firstline=28,lastline=34,highlightlines=31]{../src/backtrace/backtrace.ml}}
      \onslide*<7->{\mintocaml[firstline=28,lastline=34,highlightlines=33]{../src/backtrace/backtrace.ml}}
      \endminipage
    \end{column}
    \begin{column}{0.45\textwidth}
      \raggedleft
      \onslide*<1>{\providecommand\step{6}%
% Backtraces
%
\subsection{One advantage of raising with debugging information}
\frameSubsection{}{}

\begin{frame}{Backtraces}{Debugging information \& source code}
  \begin{columns}[c]
    \begin{column}{0.5\textwidth}
      \begin{itemize}
        \item Raising an exception is fun and games, but how do we know where the exception originated? $\rightarrow$ With the backtrace associated with the exception!
        \item In order to get a backtrace from the point where an exception was raised, it's necessary to compile with debugging information enabled (\funcarg{-g} flag for the compiler)
        \item Same example as in the `Nested handlers' section
      \end{itemize}
    \end{column}
    \begin{column}{0.5\textwidth}
      \listocaml[firstline=9,lastline=34]{../src/backtrace/backtrace.ml}{backtrace/backtrace.ml}
    \end{column}
  \end{columns}
\end{frame}

\begin{frame}{Backtraces}{CMM and disassembly of \funcname{definitely\_raise}}
  \begin{columns}[c]
    \begin{column}{0.5\textwidth}
      \minipage[c][0.66\textheight][s]{\columnwidth}
      \begin{itemize}
        \item<1-> An effect of compiling with debugging information (\funcarg{-g}): the CMM no longer uses \funcname{raise\_notrace} to raise the exception but \funcname{raise} instead
        \item<2-> Disassembly shows that CMM \funcname{raise} is actually a call to \funcname{caml\_raise\_exn}, a function in assembler provided by the runtime
      \end{itemize}
      \vfill
      \mintocaml[firstline=9,lastline=13,highlightlines=12]{../src/backtrace/backtrace.ml}
      \endminipage
    \end{column}
    \begin{column}{0.5\textwidth}
      \onslide*<1>{
        \mintcmm[highlightlines=5]{../src/backtrace/camlBacktrace\_\_definitely\_raise\_347.cmm}
      }
      \onslide*<2>{
        \mintobjdump[firstline=2,highlightlines=14]{../src/backtrace/camlBacktrace\_\_definitely\_raise\_347.objdump}
      }
    \end{column}
  \end{columns}
\end{frame}

\begin{frame}{Backtraces}{Introducing \funcname{caml\_raise\_exn}}
  \begin{columns}[c]
    \begin{column}{0.5\textwidth}
      \minipage[c][0.66\textheight][s]{\columnwidth}
      \onslide*<1>{
        \begin{itemize}
          \item Raising the exception is performed using \funcname{caml\_raise\_exn} which is
            \begin{itemize}
              \item An assembly function provided by the runtime
              \item Architecture specific
            \end{itemize}
          \item Let's see what it does differently from \funcname{raise\_notrace}\dots
          \item We are about to raise \typename{ExnC} with \funcname{caml\_raise\_exn} from \funcname{definitely\_raise}
        \end{itemize}
      }
      \onslide*<2>{
        \mintobjdump[firstline=2,highlightlines=14]{../src/backtrace/camlBacktrace\_\_definitely\_raise\_347.objdump}
      }
      \vfill
      \mintocaml[firstline=9,lastline=13,highlightlines=12]{../src/backtrace/backtrace.ml}
      \endminipage
    \end{column}
    \begin{column}{0.5\textwidth}
      \centering
      \providecommand\step{1}\input{diagrams/backtrace/backtrace.tex}
    \end{column}
  \end{columns}
\end{frame}

\begin{frame}{Backtraces}{But before that, we need more fields from \typename{caml\_domain\_state}}
  \begin{columns}
    \begin{column}{0.5\textwidth}
      \begin{itemize}
        \item Remember \typename{caml\_domain\_state} the per-domain runtime state?
        \item Some other members are of interest to us now\dots
        \item \localname{backtrace\_active} is used to know if the runtime should collect backtraces
        \item \localname{backtrace\_active} can be set by
          \begin{itemize}
            \item The \funcarg{b} flag in the \funcname{OCAMLRUNPARAM} environment variable
            \item \mintinline{ocaml}{Printexc.record_backtrace : bool -> unit} \footnotemark
          \end{itemize}
      \end{itemize}
    \end{column}
    \begin{column}{0.5\textwidth}
      \begin{listing}
        \mintc[highlightlines={9-11}]{src/ocaml/domain\_state\_backtrace.h}
        \caption{runtime/caml/domain\_state.h}
      \end{listing}
    \end{column}
  \end{columns}
  \footnotetext[1]{https://v2.ocaml.org/api/Printexc.html}
\end{frame}

\newcommand\listCamlRaiseExn[1][]{
  \listasm[firstline=702,lastline=725,#1]{../ocaml/runtime/amd64.S}{runtime/amd64.S:702}}

\begin{frame}{Backtraces}{Walk through \funcname{caml\_raise\_exn}}
  \begin{columns}[T]
    \begin{column}{0.5\textwidth}
      \begin{itemize}
        \item<1-> First checks whether exceptions backtrace should be recorded
        \item<1-> By checking the \funcarg{domain\_state->backtrace\_active} value
        \item<2-> If not, acts as \funcname{raise\_notrace} with \funcname{RESTORE\_EXN\_HANDLER\_OCAML}
          \begin{itemize}
            \item Set \regname{RSP} to \localname{domain\_state->exn\_handler} value
            \item Restore \localname{domain\_state->exn\_handler} from trap
            \item Jump with \funcname{ret} (same effect as using \funcname{pop} and \funcname{jmp})
          \end{itemize}
        \item<3-> Otherwise, switches to the C stack to be able to execute C code
        \item<4-> Calls \funcname{caml\_stash\_backtrace}, a C function provided by the runtime\ignorespaces
          \onslide<6->{, with
            \begin{itemize}
              \item \funcarg{exn}: exception value
              \item \funcarg{pc}: code address of the raising point
              \item \funcarg{sp}: stack address of the raising function
              \item \funcarg{trapsp}: stack address of the next handler
            \end{itemize}
          }
      \end{itemize}
      \bigskip
      \mintocaml[firstline=9,lastline=13,highlightlines=12]{../src/backtrace/backtrace.ml}
    \end{column}
    \begin{column}{0.5\textwidth}
      \raggedleft
      \onslide*<1,3-4>{
        \mintc[firstline=190,lastline=192]{../ocaml/runtime/amd64.S}
        \mintc[firstline=234,lastline=237]{../ocaml/runtime/amd64.S}
      }
      \onslide*<2>{
        \mintc[firstline=190,lastline=192,highlightlines={191-192}]{../ocaml/runtime/amd64.S}
        \mintc[firstline=234,lastline=237,highlightlines={235-237}]{../ocaml/runtime/amd64.S}
      }
      \onslide*<1>{\listCamlRaiseExn[highlightlines=705]{}}%
      \onslide*<2>{\listCamlRaiseExn[highlightlines={707-708}]{}}%
      \onslide*<3>{\listCamlRaiseExn[highlightlines={712-714}]{}}%
      \onslide*<4>{\listCamlRaiseExn[highlightlines={715-720}]{}}%
      \onslide*<5>{\providecommand\step{1}\input{diagrams/backtrace/backtrace.tex}}%
      \onslide*<6>{\providecommand\step{2}\input{diagrams/backtrace/backtrace.tex}}%
    \end{column}
  \end{columns}
\end{frame}

\newcommand\listStashBacktrace[1][]{
  \listc[firstline=91,lastline=120,#1]{../ocaml/runtime/backtrace\_nat.c}{runtime/backtrace\_nat.c:91}}

\begin{frame}{Backtraces}{Introducing \funcname{caml\_stash\_backtrace}}
  \begin{columns}[c]
    \begin{column}{0.5\textwidth}
      \minipage[c][0.66\textheight][s]{\columnwidth}
      \begin{itemize}
        \item<1-> A C function provided by the runtime (specific to native code)
        \item<2-> It scans the stack, between the stack pointer at the raising point up to the next trap, in order to collect the names of the detected function frames
          \onslide*<2>{
            \begin{itemize}
              \item And more information, like line number, column number...
            \end{itemize}
          }
        \item<3-> It uses the same technique and information as the Garbage Collector (GC) uses to scan the values live on the stack
          \onslide*<4>{
            \begin{itemize}
              \item \typename{frame\_descr} are at the core of stack unwinding
            \end{itemize}
          }
      \end{itemize}
      \vfill
      \mintocaml[firstline=9,lastline=13,highlightlines=12]{../src/backtrace/backtrace.ml}
      \endminipage
    \end{column}
    \begin{column}{0.5\textwidth}
      \onslide*<1>{\listStashBacktrace[highlightlines=91]{}}
      \onslide*<2>{\listStashBacktrace[highlightlines={91,117-118}]{}}
      \onslide*<3->{\listStashBacktrace[highlightlines={94,106,109-110}]{}}
    \end{column}
  \end{columns}
\end{frame}

\newcommand\listFrameDescr[1][]{
  \listocaml[firstline=111,lastline=124,#1]{../ocaml/asmcomp/emitaux.ml}{asmcomp/emitaux.ml:111}}
\newcommand\listDebugInfo[1][]{
  \listocaml[firstline=38,lastline=49,#1]{../ocaml/lambda/debuginfo.mli}{lambda/debuginfo.mli:38}}

\begin{frame}{Backtraces}{\typename{frame\_descr} and \typename{frame\_debuginfo} in a nutshell}
  \begin{columns}[c]
    \begin{column}{0.5\textwidth}
      \begin{itemize}
        \item<1-> For every return address in an OCaml program, there is a associated \typename{frame\_descr}
        \item<2-> A \typename{frame\_descr} specifies
        \begin{itemize}
          \item<2-> The size of the function stack frame
          \item<3-> Where live values are on the stack (so that the GC can mark them as live and prevent from being swept)
        \end{itemize}
        \item<4-> When compiled with debug information, \typename{frame\_descr} will be linked to a \typename{frame\_debuginfo}
        \begin{itemize}
          \item<5-> Contains information about the position in the source code (file name, line and column number)
        \end{itemize}
      \end{itemize}
    \end{column}
    \begin{column}{0.5\textwidth}
        \onslide*<1>{\listFrameDescr[highlightlines={118-122}]{}}
        \onslide*<2>{\listFrameDescr[highlightlines={120}]{}}
        \onslide*<3>{\listFrameDescr[highlightlines={121}]{}}
        \onslide*<4->{\listFrameDescr[highlightlines={122}]{}}
        \medskip
        \onslide*<1-3>{\listDebugInfo{}}
        \onslide*<4>{\listDebugInfo[highlightlines={38,49}]{}}
        \onslide*<5>{\listDebugInfo[highlightlines={39-46}]{}}
    \end{column}
  \end{columns}
\end{frame}

\begin{frame}{Backtraces}{Scanning the stack with \typename{frame\_descr}}
  \begin{columns}[c]
    \begin{column}{0.5\textwidth}
      \begin{itemize}
        \item Given the return address of the code that raises the exception by calling \funcname{caml\_raise\_exn} (\funcarg{pc} argument of \funcname{caml\_stash\_backtrace})
        \item And the position of the stack pointer (\funcarg{sp} argument of \funcname{caml\_stash\_backtrace})
        \item The frame size given by \typename{frame\_descr}.\funcarg{frame\_size}, allows to find the end of the previous frame
        \item And the return address of the caller of this function
        \bigskip
        \onslide<3->{
        \item Note that traps (here the reddish \funcname{ExnA}, \funcname{ExnB} and \funcname{ExnC} blocks) belong to the frame that installed them, and so the frame size changes when traps are removed
        }
      \end{itemize}
    \end{column}
    \begin{column}{0.5\textwidth}
      \raggedleft
      \onslide*<1>{\providecommand\step{2}\input{diagrams/backtrace/backtrace.tex}}%
      \onslide*<2>{\providecommand\step{1}\input{diagrams/backtrace/scanstack.tex}}%
      \onslide*<3>{\providecommand\step{2}\input{diagrams/backtrace/scanstack.tex}}%
      \onslide*<4>{\providecommand\step{3}\input{diagrams/backtrace/scanstack.tex}}%
      \onslide*<5>{\providecommand\step{4}\input{diagrams/backtrace/scanstack.tex}}%
      \onslide*<6>{\providecommand\step{5}\input{diagrams/backtrace/scanstack.tex}}%
    \end{column}
  \end{columns}
\end{frame}

% Duplicate of previous-previous slide
\begin{frame}{Backtraces}{Continuing on \funcname{caml\_stash\_backtrace}}
  \begin{columns}[c]
    \begin{column}{0.5\textwidth}
      \minipage[c][0.75\textheight][s]{\columnwidth}
      \begin{itemize}
          \color{gray}
        \item A C function provided by the runtime (specific to native code)
        \item It scans the stack, between the stack pointer at the raising point up to the next trap, in order to collect the names of the detected function frames
        \item It uses the same technique and information as the Garbage Collector (GC) uses to scan the values live on the stack
      \end{itemize}
      \begin{itemize}
        \item For each function frame detected, store a pointer to the \typename{frame\_descr} in the \localname{domain\_state->backtrace\_buffer} array, effectively constructing the backtrace
      \end{itemize}
      \onslide<2->{
        \begin{itemize}
            \smallskip
          \item Constructed backtrace:
            \funcname{definitely\_raise},
            \onslide<3->{\funcname{probably\_raise}}
        \end{itemize}
      }
      \vfill
      \mintocaml[firstline=9,lastline=13,highlightlines=12]{../src/backtrace/backtrace.ml}
      \endminipage
    \end{column}
    \begin{column}{0.5\textwidth}
      \raggedleft
      \onslide*<1>{\listStashBacktrace[highlightlines={114-115}]{}}
      \onslide*<2>{\providecommand\step{3}\input{diagrams/backtrace/backtrace.tex}}%
      \onslide*<3>{\providecommand\step{4}\input{diagrams/backtrace/backtrace.tex}}%
    \end{column}
  \end{columns}
\end{frame}

\begin{frame}{Backtraces}{Back to \funcname{caml\_raise\_exn}}
  \begin{columns}[c]
    \begin{column}{0.5\textwidth}
      \minipage[c][0.66\textheight][s]{\columnwidth}
      \begin{itemize}\color{gray}
        \item Checks wheteher exceptions backtrace should be recorded, by checking the \funcarg{domain\_state->backtrace\_active} value
        \item Switches to the C stack to be able to execute C code
        \item Calls \funcname{caml\_stash\_backtrace}, to collect the backtrace up to the next trap
      \end{itemize}
      \onslide*<2->{
        \begin{itemize}
          \item<2-> Executes the exception handler in \funcname{probably\_raise} with \funcname{RESTORE\_EXN\_HANDLER\_OCAML}
            \begin{itemize}
              \item Set \regname{RSP} to \localname{domain\_state->exn\_handler} value
              \item Restore \localname{domain\_state->exn\_handler} from trap
              \item Jump to the handler code with \funcname{ret}
            \end{itemize}
        \end{itemize}
      }
      \vfill
      \onslide*<1>{\mintocaml[firstline=9,lastline=13,highlightlines=12]{../src/backtrace/backtrace.ml}}
      \onslide*<2->{\mintocaml[firstline=15,lastline=21,highlightlines={19-21}]{../src/backtrace/backtrace.ml}}
      \endminipage
    \end{column}
    \begin{column}{0.5\textwidth}
      \centering
      \onslide*<1>{
        \mintc[firstline=190,lastline=192]{../ocaml/runtime/amd64.S}
        \mintc[firstline=234,lastline=237]{../ocaml/runtime/amd64.S}
      }
      \onslide*<2>{
        \mintc[firstline=190,lastline=192,highlightlines={191-192}]{../ocaml/runtime/amd64.S}
        \mintc[firstline=234,lastline=237,highlightlines={235-237}]{../ocaml/runtime/amd64.S}
      }
      \onslide*<1>{\listCamlRaiseExn[highlightlines=720]{}}
      \onslide*<2>{\listCamlRaiseExn[highlightlines=722]{}}
      \onslide*<3->{\providecommand\step{5}\input{diagrams/backtrace/backtrace.tex}}
    \end{column}
  \end{columns}
\end{frame}

\begin{frame}{Backtraces}{\funcname{caml\_probably\_raise}'s handler doesn't catch \typename{ExnC}}
  \begin{columns}[c]
    \begin{column}{0.45\textwidth}
      \minipage[c][0.75\textheight][s]{\columnwidth}
      \begin{itemize}
        \item<1-> \funcname{match \dots with}\footnotemark pattern matching can also match exceptions
        \item<1-> The CMM of \funcname{probably\_raise} shows that this \funcname{match \dots with} is implemented with a pattern matching and a \funcname{try \dots with}
        \item<1-> But this one only matches on \typename{ExnA} exception
        \item<2-> Since the pattern matching fails to catch the exception, \funcname{reraise} is called with our current \typename{ExnC} exception
        \item<3-> Disassembly shows that \funcname{reraise} is actually a call to \funcname{caml\_reraise\_exn}, which is also an assembly function provided by the runtime
      \end{itemize}
      \vfill
      \mintocaml[firstline=15,lastline=21,highlightlines={18-21}]{../src/backtrace/backtrace.ml}
      \endminipage
    \end{column}
    \begin{column}{0.55\textwidth}
      \centering
      \onslide*<1>{\mintcmm[highlightlines={10-18}]{../src/backtrace/camlBacktrace\_\_probably\_raise\_351.cmm}}
      \onslide*<2>{\listcmm[highlightlines=18]{../src/backtrace/camlBacktrace\_\_probably\_raise_351.cmm}{CMM of probably\_raise}}
      \onslide*<3>{\mintobjdump[firstline=5,lastline=35,highlightlines=31]{../src/backtrace/camlBacktrace\_\_probably\_raise_351.objdump}}
    \end{column}
  \end{columns}
  \only<1>{\footnotetext[1]{https://blog.janestreet.com/pattern-matching-and-exception-handling-unite/}}
\end{frame}

\newcommand\listCamlReraiseExn[1][]{
  \listasm[firstline=727,lastline=734,#1]{../ocaml/runtime/amd64.S}{runtime/amd64.S:727}}

\begin{frame}{Backtraces}{Introducing \funcname{caml\_reraise\_exn}}
  \begin{columns}[c]
    \begin{column}{0.5\textwidth}
      \minipage[c][0.66\textheight][s]{\columnwidth}
      \begin{itemize}
        \item<1-> The exception have to be reraised to the next handler
        \item<2-> First, checks if the backtrace should be collected with \funcarg{domain\_state->backtrace\_active}
        \only<3>{
        \begin{itemize}
            \item If not, behaves like a \funcname{raise\_notrace} with \funcname{RESTORE\_EXN\_HANDLER\_OCAML}
        \end{itemize}
        }
        \item<4-> Jumps into \funcname{caml\_raise\_exn}
        \item<5-> Calls \funcname{caml\_stash\_backtrace} again, to scan the next part of the stack: from the trap just pop-ed to the next trap
      \end{itemize}
      \vfill
      \mintocaml[firstline=15,lastline=21,highlightlines={18-21}]{../src/backtrace/backtrace.ml}
      \endminipage
    \end{column}
    \begin{column}{0.5\textwidth}
      \raggedleft
      \onslide*<1>{\listCamlReraiseExn{}}
      \onslide*<2>{\listCamlReraiseExn[highlightlines=729]{}}
      \onslide*<3>{
        \mintc[firstline=190,lastline=192,highlightlines={191-192}]{../ocaml/runtime/amd64.S}
        \mintc[firstline=234,lastline=237,highlightlines={235-237}]{../ocaml/runtime/amd64.S}
        \medskip
        \listCamlReraiseExn[highlightlines=731]{}
      }
      \onslide*<4-5>{
        % TODO refactor with \listCamlReraiseExn
        \mintasm[firstline=727,lastline=734,highlightlines=730]{../ocaml/runtime/amd64.S}
        \medskip
      }
      \onslide*<4>{\listCamlRaiseExn[highlightlines=711]{}}
      \onslide*<5>{\listCamlRaiseExn[highlightlines=720]{}}
    \end{column}
  \end{columns}
\end{frame}

\begin{frame}{Backtraces}{Backtrace collection continues}
  \begin{columns}[c]
    \begin{column}{0.55\textwidth}
      \minipage[c][0.75\textheight][s]{\columnwidth}
      \begin{itemize}
        \item<1-> \funcname{caml\_stash\_backtrace} scans the older part of the stack, up to the next trap
        \item<6-> Controls jumps to the nested exception handler in \funcname{maybe\_raise}, which matches only on \typename{ExnB}
        \item<7-> \funcname{caml\_reraise\_exn} is called again, backtrace collection resumes with \funcname{caml\_stash\_backtrace}
      \end{itemize}
      \medskip
      \begin{itemize}
        \item<2-> Constructed backtrace:
        \begin{itemize}
          \item <2-> \funcname{definitely\_raise}
          \item <2-> \funcname{probably\_raise}
          \item <3-> \funcname{probably\_raise} (re-raise)
          \item <4-> \funcname{maybe\_raise}
          \item <5-> \funcname{main}
          \item <8-> \funcname{main} (re-raise)
        \end{itemize}
      \end{itemize}
      \vfill
      \onslide*<1-5>{\mintocaml[firstline=15,lastline=21]{../src/backtrace/backtrace.ml}}
      \onslide*<6>{\mintocaml[firstline=28,lastline=34,highlightlines=31]{../src/backtrace/backtrace.ml}}
      \onslide*<7->{\mintocaml[firstline=28,lastline=34,highlightlines=33]{../src/backtrace/backtrace.ml}}
      \endminipage
    \end{column}
    \begin{column}{0.45\textwidth}
      \raggedleft
      \onslide*<1>{\providecommand\step{6}\input{diagrams/backtrace/backtrace.tex}}%
      \onslide*<2>{\providecommand\step{6}\input{diagrams/backtrace/backtrace.tex}}%
      \onslide*<3>{\providecommand\step{7}\input{diagrams/backtrace/backtrace.tex}}%
      \onslide*<4>{\providecommand\step{8}\input{diagrams/backtrace/backtrace.tex}}%
      \onslide*<5>{\providecommand\step{9}\input{diagrams/backtrace/backtrace.tex}}%
      \onslide*<6>{\providecommand\step{10}\input{diagrams/backtrace/backtrace.tex}}%
      \onslide*<7>{\providecommand\step{11}\input{diagrams/backtrace/backtrace.tex}}%
      \onslide*<8>{\providecommand\step{12}\input{diagrams/backtrace/backtrace.tex}}%
      \onslide*<9>{\providecommand\step{13}\input{diagrams/backtrace/backtrace.tex}}%
    \end{column}
  \end{columns}
\end{frame}

\begin{frame}{Backtraces}{Printing the backtrace}
  \begin{columns}[c]
    \begin{column}{0.45\textwidth}
      \minipage[c][0.66\textheight][s]{\columnwidth}
      \begin{itemize}
        \item<1-> The exception backtrace can now be printed to \funcarg{stdout} with \funcname{Printexc.print_backtrace}
        \item<2-> First the exception backtrace is retrieved into an OCaml array with \funcname{get_raw_backtrace }
        \item<3-> Which is implemented as the \funcname{caml_get_exception_raw_backtrace} C function in the runtime
        \item<4-> And finally printed with \funcname{print_exception_backtrace}
      \end{itemize}
      \vfill
      \mintocaml[firstline=28,lastline=34,highlightlines=33]{../src/backtrace/backtrace.ml}
      \endminipage
    \end{column}
    \begin{column}{0.55\textwidth}
      \onslide*<1,2>{
        \mintocaml[firstline=100,lastline=101]{../ocaml/stdlib/printexc.ml}
      }
      \only<1>{
        \listocaml[firstline=170,lastline=172]{../ocaml/stdlib/printexc.ml}{stdlib/printexc.ml:170}
      }
      \only<2>{
        \listocaml[firstline=170,lastline=172,highlightlines=172]{../ocaml/stdlib/printexc.ml}{stdlib/printexc.ml:170}
      }
      \only<3>{
        \mintc[firstline=154,lastline=156]{../ocaml/runtime/backtrace.c}
        \listc[firstline=171,lastline=192,highlightlines={184-187}]{../ocaml/runtime/backtrace.c}{runtime/backtrace.c:154}
      }
      \only<4>{
        \listocaml[firstline=155,lastline=165]{../ocaml/stdlib/printexc.ml}{stdlib/printexc.ml:55}
      }
    \end{column}
  \end{columns}
\end{frame}


\subsection{The multiple kinds of raise}
\frameSubsection{}{}

\begin{frame}{Backtraces}{When backtraces are not needed}
  \begin{columns}[c]
    \begin{column}{0.45\textwidth}
      \minipage[c][0.66\textheight][s]{\columnwidth}
      \begin{itemize}
        \item<1-> What if the backtrace for an exception is never needed?
        \item<2-> It's possible to raise the exception explicitly with \funcname{raise\_notrace} to make raising as cheap as possible
        \item<3-> An example of use in the \funcname{dynlink} library of the OCaml compiler
      \end{itemize}
      \vfill
      \onslide<3->{
      \listocaml[firstline=205,lastline=209,highlightlines=207]{../ocaml/otherlibs/dynlink/dynlink\_compilerlibs/misc.ml}{otherlibs/dynlink/dynlink\_compilerlibs/misc.ml:205}
      }
      \endminipage
    \end{column}
    \begin{column}{0.55\textwidth}
    \only<4>{
      \mintcmm[highlightlines=3]{../src/notrace/camlNotrace\_\_fun_347.cmm}
      \listcmm{../src/notrace/camlNotrace\_\_all\_somes\_327.cmm}{all\_somes}
    }
    \onslide*<5->{
      \listobjdump[highlightlines={6-9}]{../src/notrace/camlNotrace\_\_fun_347.objdump}{anonymous function from \funcname{all\_somes}}
    }
    \end{column}
  \end{columns}
\end{frame}

\begin{frame}{Backtraces}{Summary table of the multiple kinds of raise}
  \begin{itemize}
    \item When debugging information is enabled and if the backtrace of an exception isn't needed, it's possible to raise with \funcname{raise\_notrace} for performance
    \item When debugging information is \emph{not} enabled, the compiler optimises \funcname{raise} calls to inline assembly (same effect as using \funcname{raise\_notrace})
  \end{itemize}
  \bigskip
  \centering
  \begin{tabular}{| l | c | c | c | c |}
    \hline
    \parbox{10em}{Debug information \\ (\funcarg{-g} flag)}  & \multicolumn{2}{|c|}{Disabled}                                     & \multicolumn{2}{|c|}{Enabled} \\
    \hline
    Raise kind                                               & \funcname{raise} & \funcname{raise\_notrace}                       & \funcname{raise} & \funcname{raise\_notrace} \\
    \hline
    Compilation                                              & \parbox{8em}{inline assembly\\ (optimization)} & inline assembly   & \funcname{caml\_raise\_exn} & inline assembly \\
    \hline
  \end{tabular}
\end{frame}


%
% Takeaway #5
%
\subsection*{Takeaway \#5}
\frameSubsectionTakeaway{}
\againframe<5>{takeaway}
}%
      \onslide*<2>{\providecommand\step{6}%
% Backtraces
%
\subsection{One advantage of raising with debugging information}
\frameSubsection{}{}

\begin{frame}{Backtraces}{Debugging information \& source code}
  \begin{columns}[c]
    \begin{column}{0.5\textwidth}
      \begin{itemize}
        \item Raising an exception is fun and games, but how do we know where the exception originated? $\rightarrow$ With the backtrace associated with the exception!
        \item In order to get a backtrace from the point where an exception was raised, it's necessary to compile with debugging information enabled (\funcarg{-g} flag for the compiler)
        \item Same example as in the `Nested handlers' section
      \end{itemize}
    \end{column}
    \begin{column}{0.5\textwidth}
      \listocaml[firstline=9,lastline=34]{../src/backtrace/backtrace.ml}{backtrace/backtrace.ml}
    \end{column}
  \end{columns}
\end{frame}

\begin{frame}{Backtraces}{CMM and disassembly of \funcname{definitely\_raise}}
  \begin{columns}[c]
    \begin{column}{0.5\textwidth}
      \minipage[c][0.66\textheight][s]{\columnwidth}
      \begin{itemize}
        \item<1-> An effect of compiling with debugging information (\funcarg{-g}): the CMM no longer uses \funcname{raise\_notrace} to raise the exception but \funcname{raise} instead
        \item<2-> Disassembly shows that CMM \funcname{raise} is actually a call to \funcname{caml\_raise\_exn}, a function in assembler provided by the runtime
      \end{itemize}
      \vfill
      \mintocaml[firstline=9,lastline=13,highlightlines=12]{../src/backtrace/backtrace.ml}
      \endminipage
    \end{column}
    \begin{column}{0.5\textwidth}
      \onslide*<1>{
        \mintcmm[highlightlines=5]{../src/backtrace/camlBacktrace\_\_definitely\_raise\_347.cmm}
      }
      \onslide*<2>{
        \mintobjdump[firstline=2,highlightlines=14]{../src/backtrace/camlBacktrace\_\_definitely\_raise\_347.objdump}
      }
    \end{column}
  \end{columns}
\end{frame}

\begin{frame}{Backtraces}{Introducing \funcname{caml\_raise\_exn}}
  \begin{columns}[c]
    \begin{column}{0.5\textwidth}
      \minipage[c][0.66\textheight][s]{\columnwidth}
      \onslide*<1>{
        \begin{itemize}
          \item Raising the exception is performed using \funcname{caml\_raise\_exn} which is
            \begin{itemize}
              \item An assembly function provided by the runtime
              \item Architecture specific
            \end{itemize}
          \item Let's see what it does differently from \funcname{raise\_notrace}\dots
          \item We are about to raise \typename{ExnC} with \funcname{caml\_raise\_exn} from \funcname{definitely\_raise}
        \end{itemize}
      }
      \onslide*<2>{
        \mintobjdump[firstline=2,highlightlines=14]{../src/backtrace/camlBacktrace\_\_definitely\_raise\_347.objdump}
      }
      \vfill
      \mintocaml[firstline=9,lastline=13,highlightlines=12]{../src/backtrace/backtrace.ml}
      \endminipage
    \end{column}
    \begin{column}{0.5\textwidth}
      \centering
      \providecommand\step{1}\input{diagrams/backtrace/backtrace.tex}
    \end{column}
  \end{columns}
\end{frame}

\begin{frame}{Backtraces}{But before that, we need more fields from \typename{caml\_domain\_state}}
  \begin{columns}
    \begin{column}{0.5\textwidth}
      \begin{itemize}
        \item Remember \typename{caml\_domain\_state} the per-domain runtime state?
        \item Some other members are of interest to us now\dots
        \item \localname{backtrace\_active} is used to know if the runtime should collect backtraces
        \item \localname{backtrace\_active} can be set by
          \begin{itemize}
            \item The \funcarg{b} flag in the \funcname{OCAMLRUNPARAM} environment variable
            \item \mintinline{ocaml}{Printexc.record_backtrace : bool -> unit} \footnotemark
          \end{itemize}
      \end{itemize}
    \end{column}
    \begin{column}{0.5\textwidth}
      \begin{listing}
        \mintc[highlightlines={9-11}]{src/ocaml/domain\_state\_backtrace.h}
        \caption{runtime/caml/domain\_state.h}
      \end{listing}
    \end{column}
  \end{columns}
  \footnotetext[1]{https://v2.ocaml.org/api/Printexc.html}
\end{frame}

\newcommand\listCamlRaiseExn[1][]{
  \listasm[firstline=702,lastline=725,#1]{../ocaml/runtime/amd64.S}{runtime/amd64.S:702}}

\begin{frame}{Backtraces}{Walk through \funcname{caml\_raise\_exn}}
  \begin{columns}[T]
    \begin{column}{0.5\textwidth}
      \begin{itemize}
        \item<1-> First checks whether exceptions backtrace should be recorded
        \item<1-> By checking the \funcarg{domain\_state->backtrace\_active} value
        \item<2-> If not, acts as \funcname{raise\_notrace} with \funcname{RESTORE\_EXN\_HANDLER\_OCAML}
          \begin{itemize}
            \item Set \regname{RSP} to \localname{domain\_state->exn\_handler} value
            \item Restore \localname{domain\_state->exn\_handler} from trap
            \item Jump with \funcname{ret} (same effect as using \funcname{pop} and \funcname{jmp})
          \end{itemize}
        \item<3-> Otherwise, switches to the C stack to be able to execute C code
        \item<4-> Calls \funcname{caml\_stash\_backtrace}, a C function provided by the runtime\ignorespaces
          \onslide<6->{, with
            \begin{itemize}
              \item \funcarg{exn}: exception value
              \item \funcarg{pc}: code address of the raising point
              \item \funcarg{sp}: stack address of the raising function
              \item \funcarg{trapsp}: stack address of the next handler
            \end{itemize}
          }
      \end{itemize}
      \bigskip
      \mintocaml[firstline=9,lastline=13,highlightlines=12]{../src/backtrace/backtrace.ml}
    \end{column}
    \begin{column}{0.5\textwidth}
      \raggedleft
      \onslide*<1,3-4>{
        \mintc[firstline=190,lastline=192]{../ocaml/runtime/amd64.S}
        \mintc[firstline=234,lastline=237]{../ocaml/runtime/amd64.S}
      }
      \onslide*<2>{
        \mintc[firstline=190,lastline=192,highlightlines={191-192}]{../ocaml/runtime/amd64.S}
        \mintc[firstline=234,lastline=237,highlightlines={235-237}]{../ocaml/runtime/amd64.S}
      }
      \onslide*<1>{\listCamlRaiseExn[highlightlines=705]{}}%
      \onslide*<2>{\listCamlRaiseExn[highlightlines={707-708}]{}}%
      \onslide*<3>{\listCamlRaiseExn[highlightlines={712-714}]{}}%
      \onslide*<4>{\listCamlRaiseExn[highlightlines={715-720}]{}}%
      \onslide*<5>{\providecommand\step{1}\input{diagrams/backtrace/backtrace.tex}}%
      \onslide*<6>{\providecommand\step{2}\input{diagrams/backtrace/backtrace.tex}}%
    \end{column}
  \end{columns}
\end{frame}

\newcommand\listStashBacktrace[1][]{
  \listc[firstline=91,lastline=120,#1]{../ocaml/runtime/backtrace\_nat.c}{runtime/backtrace\_nat.c:91}}

\begin{frame}{Backtraces}{Introducing \funcname{caml\_stash\_backtrace}}
  \begin{columns}[c]
    \begin{column}{0.5\textwidth}
      \minipage[c][0.66\textheight][s]{\columnwidth}
      \begin{itemize}
        \item<1-> A C function provided by the runtime (specific to native code)
        \item<2-> It scans the stack, between the stack pointer at the raising point up to the next trap, in order to collect the names of the detected function frames
          \onslide*<2>{
            \begin{itemize}
              \item And more information, like line number, column number...
            \end{itemize}
          }
        \item<3-> It uses the same technique and information as the Garbage Collector (GC) uses to scan the values live on the stack
          \onslide*<4>{
            \begin{itemize}
              \item \typename{frame\_descr} are at the core of stack unwinding
            \end{itemize}
          }
      \end{itemize}
      \vfill
      \mintocaml[firstline=9,lastline=13,highlightlines=12]{../src/backtrace/backtrace.ml}
      \endminipage
    \end{column}
    \begin{column}{0.5\textwidth}
      \onslide*<1>{\listStashBacktrace[highlightlines=91]{}}
      \onslide*<2>{\listStashBacktrace[highlightlines={91,117-118}]{}}
      \onslide*<3->{\listStashBacktrace[highlightlines={94,106,109-110}]{}}
    \end{column}
  \end{columns}
\end{frame}

\newcommand\listFrameDescr[1][]{
  \listocaml[firstline=111,lastline=124,#1]{../ocaml/asmcomp/emitaux.ml}{asmcomp/emitaux.ml:111}}
\newcommand\listDebugInfo[1][]{
  \listocaml[firstline=38,lastline=49,#1]{../ocaml/lambda/debuginfo.mli}{lambda/debuginfo.mli:38}}

\begin{frame}{Backtraces}{\typename{frame\_descr} and \typename{frame\_debuginfo} in a nutshell}
  \begin{columns}[c]
    \begin{column}{0.5\textwidth}
      \begin{itemize}
        \item<1-> For every return address in an OCaml program, there is a associated \typename{frame\_descr}
        \item<2-> A \typename{frame\_descr} specifies
        \begin{itemize}
          \item<2-> The size of the function stack frame
          \item<3-> Where live values are on the stack (so that the GC can mark them as live and prevent from being swept)
        \end{itemize}
        \item<4-> When compiled with debug information, \typename{frame\_descr} will be linked to a \typename{frame\_debuginfo}
        \begin{itemize}
          \item<5-> Contains information about the position in the source code (file name, line and column number)
        \end{itemize}
      \end{itemize}
    \end{column}
    \begin{column}{0.5\textwidth}
        \onslide*<1>{\listFrameDescr[highlightlines={118-122}]{}}
        \onslide*<2>{\listFrameDescr[highlightlines={120}]{}}
        \onslide*<3>{\listFrameDescr[highlightlines={121}]{}}
        \onslide*<4->{\listFrameDescr[highlightlines={122}]{}}
        \medskip
        \onslide*<1-3>{\listDebugInfo{}}
        \onslide*<4>{\listDebugInfo[highlightlines={38,49}]{}}
        \onslide*<5>{\listDebugInfo[highlightlines={39-46}]{}}
    \end{column}
  \end{columns}
\end{frame}

\begin{frame}{Backtraces}{Scanning the stack with \typename{frame\_descr}}
  \begin{columns}[c]
    \begin{column}{0.5\textwidth}
      \begin{itemize}
        \item Given the return address of the code that raises the exception by calling \funcname{caml\_raise\_exn} (\funcarg{pc} argument of \funcname{caml\_stash\_backtrace})
        \item And the position of the stack pointer (\funcarg{sp} argument of \funcname{caml\_stash\_backtrace})
        \item The frame size given by \typename{frame\_descr}.\funcarg{frame\_size}, allows to find the end of the previous frame
        \item And the return address of the caller of this function
        \bigskip
        \onslide<3->{
        \item Note that traps (here the reddish \funcname{ExnA}, \funcname{ExnB} and \funcname{ExnC} blocks) belong to the frame that installed them, and so the frame size changes when traps are removed
        }
      \end{itemize}
    \end{column}
    \begin{column}{0.5\textwidth}
      \raggedleft
      \onslide*<1>{\providecommand\step{2}\input{diagrams/backtrace/backtrace.tex}}%
      \onslide*<2>{\providecommand\step{1}\input{diagrams/backtrace/scanstack.tex}}%
      \onslide*<3>{\providecommand\step{2}\input{diagrams/backtrace/scanstack.tex}}%
      \onslide*<4>{\providecommand\step{3}\input{diagrams/backtrace/scanstack.tex}}%
      \onslide*<5>{\providecommand\step{4}\input{diagrams/backtrace/scanstack.tex}}%
      \onslide*<6>{\providecommand\step{5}\input{diagrams/backtrace/scanstack.tex}}%
    \end{column}
  \end{columns}
\end{frame}

% Duplicate of previous-previous slide
\begin{frame}{Backtraces}{Continuing on \funcname{caml\_stash\_backtrace}}
  \begin{columns}[c]
    \begin{column}{0.5\textwidth}
      \minipage[c][0.75\textheight][s]{\columnwidth}
      \begin{itemize}
          \color{gray}
        \item A C function provided by the runtime (specific to native code)
        \item It scans the stack, between the stack pointer at the raising point up to the next trap, in order to collect the names of the detected function frames
        \item It uses the same technique and information as the Garbage Collector (GC) uses to scan the values live on the stack
      \end{itemize}
      \begin{itemize}
        \item For each function frame detected, store a pointer to the \typename{frame\_descr} in the \localname{domain\_state->backtrace\_buffer} array, effectively constructing the backtrace
      \end{itemize}
      \onslide<2->{
        \begin{itemize}
            \smallskip
          \item Constructed backtrace:
            \funcname{definitely\_raise},
            \onslide<3->{\funcname{probably\_raise}}
        \end{itemize}
      }
      \vfill
      \mintocaml[firstline=9,lastline=13,highlightlines=12]{../src/backtrace/backtrace.ml}
      \endminipage
    \end{column}
    \begin{column}{0.5\textwidth}
      \raggedleft
      \onslide*<1>{\listStashBacktrace[highlightlines={114-115}]{}}
      \onslide*<2>{\providecommand\step{3}\input{diagrams/backtrace/backtrace.tex}}%
      \onslide*<3>{\providecommand\step{4}\input{diagrams/backtrace/backtrace.tex}}%
    \end{column}
  \end{columns}
\end{frame}

\begin{frame}{Backtraces}{Back to \funcname{caml\_raise\_exn}}
  \begin{columns}[c]
    \begin{column}{0.5\textwidth}
      \minipage[c][0.66\textheight][s]{\columnwidth}
      \begin{itemize}\color{gray}
        \item Checks wheteher exceptions backtrace should be recorded, by checking the \funcarg{domain\_state->backtrace\_active} value
        \item Switches to the C stack to be able to execute C code
        \item Calls \funcname{caml\_stash\_backtrace}, to collect the backtrace up to the next trap
      \end{itemize}
      \onslide*<2->{
        \begin{itemize}
          \item<2-> Executes the exception handler in \funcname{probably\_raise} with \funcname{RESTORE\_EXN\_HANDLER\_OCAML}
            \begin{itemize}
              \item Set \regname{RSP} to \localname{domain\_state->exn\_handler} value
              \item Restore \localname{domain\_state->exn\_handler} from trap
              \item Jump to the handler code with \funcname{ret}
            \end{itemize}
        \end{itemize}
      }
      \vfill
      \onslide*<1>{\mintocaml[firstline=9,lastline=13,highlightlines=12]{../src/backtrace/backtrace.ml}}
      \onslide*<2->{\mintocaml[firstline=15,lastline=21,highlightlines={19-21}]{../src/backtrace/backtrace.ml}}
      \endminipage
    \end{column}
    \begin{column}{0.5\textwidth}
      \centering
      \onslide*<1>{
        \mintc[firstline=190,lastline=192]{../ocaml/runtime/amd64.S}
        \mintc[firstline=234,lastline=237]{../ocaml/runtime/amd64.S}
      }
      \onslide*<2>{
        \mintc[firstline=190,lastline=192,highlightlines={191-192}]{../ocaml/runtime/amd64.S}
        \mintc[firstline=234,lastline=237,highlightlines={235-237}]{../ocaml/runtime/amd64.S}
      }
      \onslide*<1>{\listCamlRaiseExn[highlightlines=720]{}}
      \onslide*<2>{\listCamlRaiseExn[highlightlines=722]{}}
      \onslide*<3->{\providecommand\step{5}\input{diagrams/backtrace/backtrace.tex}}
    \end{column}
  \end{columns}
\end{frame}

\begin{frame}{Backtraces}{\funcname{caml\_probably\_raise}'s handler doesn't catch \typename{ExnC}}
  \begin{columns}[c]
    \begin{column}{0.45\textwidth}
      \minipage[c][0.75\textheight][s]{\columnwidth}
      \begin{itemize}
        \item<1-> \funcname{match \dots with}\footnotemark pattern matching can also match exceptions
        \item<1-> The CMM of \funcname{probably\_raise} shows that this \funcname{match \dots with} is implemented with a pattern matching and a \funcname{try \dots with}
        \item<1-> But this one only matches on \typename{ExnA} exception
        \item<2-> Since the pattern matching fails to catch the exception, \funcname{reraise} is called with our current \typename{ExnC} exception
        \item<3-> Disassembly shows that \funcname{reraise} is actually a call to \funcname{caml\_reraise\_exn}, which is also an assembly function provided by the runtime
      \end{itemize}
      \vfill
      \mintocaml[firstline=15,lastline=21,highlightlines={18-21}]{../src/backtrace/backtrace.ml}
      \endminipage
    \end{column}
    \begin{column}{0.55\textwidth}
      \centering
      \onslide*<1>{\mintcmm[highlightlines={10-18}]{../src/backtrace/camlBacktrace\_\_probably\_raise\_351.cmm}}
      \onslide*<2>{\listcmm[highlightlines=18]{../src/backtrace/camlBacktrace\_\_probably\_raise_351.cmm}{CMM of probably\_raise}}
      \onslide*<3>{\mintobjdump[firstline=5,lastline=35,highlightlines=31]{../src/backtrace/camlBacktrace\_\_probably\_raise_351.objdump}}
    \end{column}
  \end{columns}
  \only<1>{\footnotetext[1]{https://blog.janestreet.com/pattern-matching-and-exception-handling-unite/}}
\end{frame}

\newcommand\listCamlReraiseExn[1][]{
  \listasm[firstline=727,lastline=734,#1]{../ocaml/runtime/amd64.S}{runtime/amd64.S:727}}

\begin{frame}{Backtraces}{Introducing \funcname{caml\_reraise\_exn}}
  \begin{columns}[c]
    \begin{column}{0.5\textwidth}
      \minipage[c][0.66\textheight][s]{\columnwidth}
      \begin{itemize}
        \item<1-> The exception have to be reraised to the next handler
        \item<2-> First, checks if the backtrace should be collected with \funcarg{domain\_state->backtrace\_active}
        \only<3>{
        \begin{itemize}
            \item If not, behaves like a \funcname{raise\_notrace} with \funcname{RESTORE\_EXN\_HANDLER\_OCAML}
        \end{itemize}
        }
        \item<4-> Jumps into \funcname{caml\_raise\_exn}
        \item<5-> Calls \funcname{caml\_stash\_backtrace} again, to scan the next part of the stack: from the trap just pop-ed to the next trap
      \end{itemize}
      \vfill
      \mintocaml[firstline=15,lastline=21,highlightlines={18-21}]{../src/backtrace/backtrace.ml}
      \endminipage
    \end{column}
    \begin{column}{0.5\textwidth}
      \raggedleft
      \onslide*<1>{\listCamlReraiseExn{}}
      \onslide*<2>{\listCamlReraiseExn[highlightlines=729]{}}
      \onslide*<3>{
        \mintc[firstline=190,lastline=192,highlightlines={191-192}]{../ocaml/runtime/amd64.S}
        \mintc[firstline=234,lastline=237,highlightlines={235-237}]{../ocaml/runtime/amd64.S}
        \medskip
        \listCamlReraiseExn[highlightlines=731]{}
      }
      \onslide*<4-5>{
        % TODO refactor with \listCamlReraiseExn
        \mintasm[firstline=727,lastline=734,highlightlines=730]{../ocaml/runtime/amd64.S}
        \medskip
      }
      \onslide*<4>{\listCamlRaiseExn[highlightlines=711]{}}
      \onslide*<5>{\listCamlRaiseExn[highlightlines=720]{}}
    \end{column}
  \end{columns}
\end{frame}

\begin{frame}{Backtraces}{Backtrace collection continues}
  \begin{columns}[c]
    \begin{column}{0.55\textwidth}
      \minipage[c][0.75\textheight][s]{\columnwidth}
      \begin{itemize}
        \item<1-> \funcname{caml\_stash\_backtrace} scans the older part of the stack, up to the next trap
        \item<6-> Controls jumps to the nested exception handler in \funcname{maybe\_raise}, which matches only on \typename{ExnB}
        \item<7-> \funcname{caml\_reraise\_exn} is called again, backtrace collection resumes with \funcname{caml\_stash\_backtrace}
      \end{itemize}
      \medskip
      \begin{itemize}
        \item<2-> Constructed backtrace:
        \begin{itemize}
          \item <2-> \funcname{definitely\_raise}
          \item <2-> \funcname{probably\_raise}
          \item <3-> \funcname{probably\_raise} (re-raise)
          \item <4-> \funcname{maybe\_raise}
          \item <5-> \funcname{main}
          \item <8-> \funcname{main} (re-raise)
        \end{itemize}
      \end{itemize}
      \vfill
      \onslide*<1-5>{\mintocaml[firstline=15,lastline=21]{../src/backtrace/backtrace.ml}}
      \onslide*<6>{\mintocaml[firstline=28,lastline=34,highlightlines=31]{../src/backtrace/backtrace.ml}}
      \onslide*<7->{\mintocaml[firstline=28,lastline=34,highlightlines=33]{../src/backtrace/backtrace.ml}}
      \endminipage
    \end{column}
    \begin{column}{0.45\textwidth}
      \raggedleft
      \onslide*<1>{\providecommand\step{6}\input{diagrams/backtrace/backtrace.tex}}%
      \onslide*<2>{\providecommand\step{6}\input{diagrams/backtrace/backtrace.tex}}%
      \onslide*<3>{\providecommand\step{7}\input{diagrams/backtrace/backtrace.tex}}%
      \onslide*<4>{\providecommand\step{8}\input{diagrams/backtrace/backtrace.tex}}%
      \onslide*<5>{\providecommand\step{9}\input{diagrams/backtrace/backtrace.tex}}%
      \onslide*<6>{\providecommand\step{10}\input{diagrams/backtrace/backtrace.tex}}%
      \onslide*<7>{\providecommand\step{11}\input{diagrams/backtrace/backtrace.tex}}%
      \onslide*<8>{\providecommand\step{12}\input{diagrams/backtrace/backtrace.tex}}%
      \onslide*<9>{\providecommand\step{13}\input{diagrams/backtrace/backtrace.tex}}%
    \end{column}
  \end{columns}
\end{frame}

\begin{frame}{Backtraces}{Printing the backtrace}
  \begin{columns}[c]
    \begin{column}{0.45\textwidth}
      \minipage[c][0.66\textheight][s]{\columnwidth}
      \begin{itemize}
        \item<1-> The exception backtrace can now be printed to \funcarg{stdout} with \funcname{Printexc.print_backtrace}
        \item<2-> First the exception backtrace is retrieved into an OCaml array with \funcname{get_raw_backtrace }
        \item<3-> Which is implemented as the \funcname{caml_get_exception_raw_backtrace} C function in the runtime
        \item<4-> And finally printed with \funcname{print_exception_backtrace}
      \end{itemize}
      \vfill
      \mintocaml[firstline=28,lastline=34,highlightlines=33]{../src/backtrace/backtrace.ml}
      \endminipage
    \end{column}
    \begin{column}{0.55\textwidth}
      \onslide*<1,2>{
        \mintocaml[firstline=100,lastline=101]{../ocaml/stdlib/printexc.ml}
      }
      \only<1>{
        \listocaml[firstline=170,lastline=172]{../ocaml/stdlib/printexc.ml}{stdlib/printexc.ml:170}
      }
      \only<2>{
        \listocaml[firstline=170,lastline=172,highlightlines=172]{../ocaml/stdlib/printexc.ml}{stdlib/printexc.ml:170}
      }
      \only<3>{
        \mintc[firstline=154,lastline=156]{../ocaml/runtime/backtrace.c}
        \listc[firstline=171,lastline=192,highlightlines={184-187}]{../ocaml/runtime/backtrace.c}{runtime/backtrace.c:154}
      }
      \only<4>{
        \listocaml[firstline=155,lastline=165]{../ocaml/stdlib/printexc.ml}{stdlib/printexc.ml:55}
      }
    \end{column}
  \end{columns}
\end{frame}


\subsection{The multiple kinds of raise}
\frameSubsection{}{}

\begin{frame}{Backtraces}{When backtraces are not needed}
  \begin{columns}[c]
    \begin{column}{0.45\textwidth}
      \minipage[c][0.66\textheight][s]{\columnwidth}
      \begin{itemize}
        \item<1-> What if the backtrace for an exception is never needed?
        \item<2-> It's possible to raise the exception explicitly with \funcname{raise\_notrace} to make raising as cheap as possible
        \item<3-> An example of use in the \funcname{dynlink} library of the OCaml compiler
      \end{itemize}
      \vfill
      \onslide<3->{
      \listocaml[firstline=205,lastline=209,highlightlines=207]{../ocaml/otherlibs/dynlink/dynlink\_compilerlibs/misc.ml}{otherlibs/dynlink/dynlink\_compilerlibs/misc.ml:205}
      }
      \endminipage
    \end{column}
    \begin{column}{0.55\textwidth}
    \only<4>{
      \mintcmm[highlightlines=3]{../src/notrace/camlNotrace\_\_fun_347.cmm}
      \listcmm{../src/notrace/camlNotrace\_\_all\_somes\_327.cmm}{all\_somes}
    }
    \onslide*<5->{
      \listobjdump[highlightlines={6-9}]{../src/notrace/camlNotrace\_\_fun_347.objdump}{anonymous function from \funcname{all\_somes}}
    }
    \end{column}
  \end{columns}
\end{frame}

\begin{frame}{Backtraces}{Summary table of the multiple kinds of raise}
  \begin{itemize}
    \item When debugging information is enabled and if the backtrace of an exception isn't needed, it's possible to raise with \funcname{raise\_notrace} for performance
    \item When debugging information is \emph{not} enabled, the compiler optimises \funcname{raise} calls to inline assembly (same effect as using \funcname{raise\_notrace})
  \end{itemize}
  \bigskip
  \centering
  \begin{tabular}{| l | c | c | c | c |}
    \hline
    \parbox{10em}{Debug information \\ (\funcarg{-g} flag)}  & \multicolumn{2}{|c|}{Disabled}                                     & \multicolumn{2}{|c|}{Enabled} \\
    \hline
    Raise kind                                               & \funcname{raise} & \funcname{raise\_notrace}                       & \funcname{raise} & \funcname{raise\_notrace} \\
    \hline
    Compilation                                              & \parbox{8em}{inline assembly\\ (optimization)} & inline assembly   & \funcname{caml\_raise\_exn} & inline assembly \\
    \hline
  \end{tabular}
\end{frame}


%
% Takeaway #5
%
\subsection*{Takeaway \#5}
\frameSubsectionTakeaway{}
\againframe<5>{takeaway}
}%
      \onslide*<3>{\providecommand\step{7}%
% Backtraces
%
\subsection{One advantage of raising with debugging information}
\frameSubsection{}{}

\begin{frame}{Backtraces}{Debugging information \& source code}
  \begin{columns}[c]
    \begin{column}{0.5\textwidth}
      \begin{itemize}
        \item Raising an exception is fun and games, but how do we know where the exception originated? $\rightarrow$ With the backtrace associated with the exception!
        \item In order to get a backtrace from the point where an exception was raised, it's necessary to compile with debugging information enabled (\funcarg{-g} flag for the compiler)
        \item Same example as in the `Nested handlers' section
      \end{itemize}
    \end{column}
    \begin{column}{0.5\textwidth}
      \listocaml[firstline=9,lastline=34]{../src/backtrace/backtrace.ml}{backtrace/backtrace.ml}
    \end{column}
  \end{columns}
\end{frame}

\begin{frame}{Backtraces}{CMM and disassembly of \funcname{definitely\_raise}}
  \begin{columns}[c]
    \begin{column}{0.5\textwidth}
      \minipage[c][0.66\textheight][s]{\columnwidth}
      \begin{itemize}
        \item<1-> An effect of compiling with debugging information (\funcarg{-g}): the CMM no longer uses \funcname{raise\_notrace} to raise the exception but \funcname{raise} instead
        \item<2-> Disassembly shows that CMM \funcname{raise} is actually a call to \funcname{caml\_raise\_exn}, a function in assembler provided by the runtime
      \end{itemize}
      \vfill
      \mintocaml[firstline=9,lastline=13,highlightlines=12]{../src/backtrace/backtrace.ml}
      \endminipage
    \end{column}
    \begin{column}{0.5\textwidth}
      \onslide*<1>{
        \mintcmm[highlightlines=5]{../src/backtrace/camlBacktrace\_\_definitely\_raise\_347.cmm}
      }
      \onslide*<2>{
        \mintobjdump[firstline=2,highlightlines=14]{../src/backtrace/camlBacktrace\_\_definitely\_raise\_347.objdump}
      }
    \end{column}
  \end{columns}
\end{frame}

\begin{frame}{Backtraces}{Introducing \funcname{caml\_raise\_exn}}
  \begin{columns}[c]
    \begin{column}{0.5\textwidth}
      \minipage[c][0.66\textheight][s]{\columnwidth}
      \onslide*<1>{
        \begin{itemize}
          \item Raising the exception is performed using \funcname{caml\_raise\_exn} which is
            \begin{itemize}
              \item An assembly function provided by the runtime
              \item Architecture specific
            \end{itemize}
          \item Let's see what it does differently from \funcname{raise\_notrace}\dots
          \item We are about to raise \typename{ExnC} with \funcname{caml\_raise\_exn} from \funcname{definitely\_raise}
        \end{itemize}
      }
      \onslide*<2>{
        \mintobjdump[firstline=2,highlightlines=14]{../src/backtrace/camlBacktrace\_\_definitely\_raise\_347.objdump}
      }
      \vfill
      \mintocaml[firstline=9,lastline=13,highlightlines=12]{../src/backtrace/backtrace.ml}
      \endminipage
    \end{column}
    \begin{column}{0.5\textwidth}
      \centering
      \providecommand\step{1}\input{diagrams/backtrace/backtrace.tex}
    \end{column}
  \end{columns}
\end{frame}

\begin{frame}{Backtraces}{But before that, we need more fields from \typename{caml\_domain\_state}}
  \begin{columns}
    \begin{column}{0.5\textwidth}
      \begin{itemize}
        \item Remember \typename{caml\_domain\_state} the per-domain runtime state?
        \item Some other members are of interest to us now\dots
        \item \localname{backtrace\_active} is used to know if the runtime should collect backtraces
        \item \localname{backtrace\_active} can be set by
          \begin{itemize}
            \item The \funcarg{b} flag in the \funcname{OCAMLRUNPARAM} environment variable
            \item \mintinline{ocaml}{Printexc.record_backtrace : bool -> unit} \footnotemark
          \end{itemize}
      \end{itemize}
    \end{column}
    \begin{column}{0.5\textwidth}
      \begin{listing}
        \mintc[highlightlines={9-11}]{src/ocaml/domain\_state\_backtrace.h}
        \caption{runtime/caml/domain\_state.h}
      \end{listing}
    \end{column}
  \end{columns}
  \footnotetext[1]{https://v2.ocaml.org/api/Printexc.html}
\end{frame}

\newcommand\listCamlRaiseExn[1][]{
  \listasm[firstline=702,lastline=725,#1]{../ocaml/runtime/amd64.S}{runtime/amd64.S:702}}

\begin{frame}{Backtraces}{Walk through \funcname{caml\_raise\_exn}}
  \begin{columns}[T]
    \begin{column}{0.5\textwidth}
      \begin{itemize}
        \item<1-> First checks whether exceptions backtrace should be recorded
        \item<1-> By checking the \funcarg{domain\_state->backtrace\_active} value
        \item<2-> If not, acts as \funcname{raise\_notrace} with \funcname{RESTORE\_EXN\_HANDLER\_OCAML}
          \begin{itemize}
            \item Set \regname{RSP} to \localname{domain\_state->exn\_handler} value
            \item Restore \localname{domain\_state->exn\_handler} from trap
            \item Jump with \funcname{ret} (same effect as using \funcname{pop} and \funcname{jmp})
          \end{itemize}
        \item<3-> Otherwise, switches to the C stack to be able to execute C code
        \item<4-> Calls \funcname{caml\_stash\_backtrace}, a C function provided by the runtime\ignorespaces
          \onslide<6->{, with
            \begin{itemize}
              \item \funcarg{exn}: exception value
              \item \funcarg{pc}: code address of the raising point
              \item \funcarg{sp}: stack address of the raising function
              \item \funcarg{trapsp}: stack address of the next handler
            \end{itemize}
          }
      \end{itemize}
      \bigskip
      \mintocaml[firstline=9,lastline=13,highlightlines=12]{../src/backtrace/backtrace.ml}
    \end{column}
    \begin{column}{0.5\textwidth}
      \raggedleft
      \onslide*<1,3-4>{
        \mintc[firstline=190,lastline=192]{../ocaml/runtime/amd64.S}
        \mintc[firstline=234,lastline=237]{../ocaml/runtime/amd64.S}
      }
      \onslide*<2>{
        \mintc[firstline=190,lastline=192,highlightlines={191-192}]{../ocaml/runtime/amd64.S}
        \mintc[firstline=234,lastline=237,highlightlines={235-237}]{../ocaml/runtime/amd64.S}
      }
      \onslide*<1>{\listCamlRaiseExn[highlightlines=705]{}}%
      \onslide*<2>{\listCamlRaiseExn[highlightlines={707-708}]{}}%
      \onslide*<3>{\listCamlRaiseExn[highlightlines={712-714}]{}}%
      \onslide*<4>{\listCamlRaiseExn[highlightlines={715-720}]{}}%
      \onslide*<5>{\providecommand\step{1}\input{diagrams/backtrace/backtrace.tex}}%
      \onslide*<6>{\providecommand\step{2}\input{diagrams/backtrace/backtrace.tex}}%
    \end{column}
  \end{columns}
\end{frame}

\newcommand\listStashBacktrace[1][]{
  \listc[firstline=91,lastline=120,#1]{../ocaml/runtime/backtrace\_nat.c}{runtime/backtrace\_nat.c:91}}

\begin{frame}{Backtraces}{Introducing \funcname{caml\_stash\_backtrace}}
  \begin{columns}[c]
    \begin{column}{0.5\textwidth}
      \minipage[c][0.66\textheight][s]{\columnwidth}
      \begin{itemize}
        \item<1-> A C function provided by the runtime (specific to native code)
        \item<2-> It scans the stack, between the stack pointer at the raising point up to the next trap, in order to collect the names of the detected function frames
          \onslide*<2>{
            \begin{itemize}
              \item And more information, like line number, column number...
            \end{itemize}
          }
        \item<3-> It uses the same technique and information as the Garbage Collector (GC) uses to scan the values live on the stack
          \onslide*<4>{
            \begin{itemize}
              \item \typename{frame\_descr} are at the core of stack unwinding
            \end{itemize}
          }
      \end{itemize}
      \vfill
      \mintocaml[firstline=9,lastline=13,highlightlines=12]{../src/backtrace/backtrace.ml}
      \endminipage
    \end{column}
    \begin{column}{0.5\textwidth}
      \onslide*<1>{\listStashBacktrace[highlightlines=91]{}}
      \onslide*<2>{\listStashBacktrace[highlightlines={91,117-118}]{}}
      \onslide*<3->{\listStashBacktrace[highlightlines={94,106,109-110}]{}}
    \end{column}
  \end{columns}
\end{frame}

\newcommand\listFrameDescr[1][]{
  \listocaml[firstline=111,lastline=124,#1]{../ocaml/asmcomp/emitaux.ml}{asmcomp/emitaux.ml:111}}
\newcommand\listDebugInfo[1][]{
  \listocaml[firstline=38,lastline=49,#1]{../ocaml/lambda/debuginfo.mli}{lambda/debuginfo.mli:38}}

\begin{frame}{Backtraces}{\typename{frame\_descr} and \typename{frame\_debuginfo} in a nutshell}
  \begin{columns}[c]
    \begin{column}{0.5\textwidth}
      \begin{itemize}
        \item<1-> For every return address in an OCaml program, there is a associated \typename{frame\_descr}
        \item<2-> A \typename{frame\_descr} specifies
        \begin{itemize}
          \item<2-> The size of the function stack frame
          \item<3-> Where live values are on the stack (so that the GC can mark them as live and prevent from being swept)
        \end{itemize}
        \item<4-> When compiled with debug information, \typename{frame\_descr} will be linked to a \typename{frame\_debuginfo}
        \begin{itemize}
          \item<5-> Contains information about the position in the source code (file name, line and column number)
        \end{itemize}
      \end{itemize}
    \end{column}
    \begin{column}{0.5\textwidth}
        \onslide*<1>{\listFrameDescr[highlightlines={118-122}]{}}
        \onslide*<2>{\listFrameDescr[highlightlines={120}]{}}
        \onslide*<3>{\listFrameDescr[highlightlines={121}]{}}
        \onslide*<4->{\listFrameDescr[highlightlines={122}]{}}
        \medskip
        \onslide*<1-3>{\listDebugInfo{}}
        \onslide*<4>{\listDebugInfo[highlightlines={38,49}]{}}
        \onslide*<5>{\listDebugInfo[highlightlines={39-46}]{}}
    \end{column}
  \end{columns}
\end{frame}

\begin{frame}{Backtraces}{Scanning the stack with \typename{frame\_descr}}
  \begin{columns}[c]
    \begin{column}{0.5\textwidth}
      \begin{itemize}
        \item Given the return address of the code that raises the exception by calling \funcname{caml\_raise\_exn} (\funcarg{pc} argument of \funcname{caml\_stash\_backtrace})
        \item And the position of the stack pointer (\funcarg{sp} argument of \funcname{caml\_stash\_backtrace})
        \item The frame size given by \typename{frame\_descr}.\funcarg{frame\_size}, allows to find the end of the previous frame
        \item And the return address of the caller of this function
        \bigskip
        \onslide<3->{
        \item Note that traps (here the reddish \funcname{ExnA}, \funcname{ExnB} and \funcname{ExnC} blocks) belong to the frame that installed them, and so the frame size changes when traps are removed
        }
      \end{itemize}
    \end{column}
    \begin{column}{0.5\textwidth}
      \raggedleft
      \onslide*<1>{\providecommand\step{2}\input{diagrams/backtrace/backtrace.tex}}%
      \onslide*<2>{\providecommand\step{1}\input{diagrams/backtrace/scanstack.tex}}%
      \onslide*<3>{\providecommand\step{2}\input{diagrams/backtrace/scanstack.tex}}%
      \onslide*<4>{\providecommand\step{3}\input{diagrams/backtrace/scanstack.tex}}%
      \onslide*<5>{\providecommand\step{4}\input{diagrams/backtrace/scanstack.tex}}%
      \onslide*<6>{\providecommand\step{5}\input{diagrams/backtrace/scanstack.tex}}%
    \end{column}
  \end{columns}
\end{frame}

% Duplicate of previous-previous slide
\begin{frame}{Backtraces}{Continuing on \funcname{caml\_stash\_backtrace}}
  \begin{columns}[c]
    \begin{column}{0.5\textwidth}
      \minipage[c][0.75\textheight][s]{\columnwidth}
      \begin{itemize}
          \color{gray}
        \item A C function provided by the runtime (specific to native code)
        \item It scans the stack, between the stack pointer at the raising point up to the next trap, in order to collect the names of the detected function frames
        \item It uses the same technique and information as the Garbage Collector (GC) uses to scan the values live on the stack
      \end{itemize}
      \begin{itemize}
        \item For each function frame detected, store a pointer to the \typename{frame\_descr} in the \localname{domain\_state->backtrace\_buffer} array, effectively constructing the backtrace
      \end{itemize}
      \onslide<2->{
        \begin{itemize}
            \smallskip
          \item Constructed backtrace:
            \funcname{definitely\_raise},
            \onslide<3->{\funcname{probably\_raise}}
        \end{itemize}
      }
      \vfill
      \mintocaml[firstline=9,lastline=13,highlightlines=12]{../src/backtrace/backtrace.ml}
      \endminipage
    \end{column}
    \begin{column}{0.5\textwidth}
      \raggedleft
      \onslide*<1>{\listStashBacktrace[highlightlines={114-115}]{}}
      \onslide*<2>{\providecommand\step{3}\input{diagrams/backtrace/backtrace.tex}}%
      \onslide*<3>{\providecommand\step{4}\input{diagrams/backtrace/backtrace.tex}}%
    \end{column}
  \end{columns}
\end{frame}

\begin{frame}{Backtraces}{Back to \funcname{caml\_raise\_exn}}
  \begin{columns}[c]
    \begin{column}{0.5\textwidth}
      \minipage[c][0.66\textheight][s]{\columnwidth}
      \begin{itemize}\color{gray}
        \item Checks wheteher exceptions backtrace should be recorded, by checking the \funcarg{domain\_state->backtrace\_active} value
        \item Switches to the C stack to be able to execute C code
        \item Calls \funcname{caml\_stash\_backtrace}, to collect the backtrace up to the next trap
      \end{itemize}
      \onslide*<2->{
        \begin{itemize}
          \item<2-> Executes the exception handler in \funcname{probably\_raise} with \funcname{RESTORE\_EXN\_HANDLER\_OCAML}
            \begin{itemize}
              \item Set \regname{RSP} to \localname{domain\_state->exn\_handler} value
              \item Restore \localname{domain\_state->exn\_handler} from trap
              \item Jump to the handler code with \funcname{ret}
            \end{itemize}
        \end{itemize}
      }
      \vfill
      \onslide*<1>{\mintocaml[firstline=9,lastline=13,highlightlines=12]{../src/backtrace/backtrace.ml}}
      \onslide*<2->{\mintocaml[firstline=15,lastline=21,highlightlines={19-21}]{../src/backtrace/backtrace.ml}}
      \endminipage
    \end{column}
    \begin{column}{0.5\textwidth}
      \centering
      \onslide*<1>{
        \mintc[firstline=190,lastline=192]{../ocaml/runtime/amd64.S}
        \mintc[firstline=234,lastline=237]{../ocaml/runtime/amd64.S}
      }
      \onslide*<2>{
        \mintc[firstline=190,lastline=192,highlightlines={191-192}]{../ocaml/runtime/amd64.S}
        \mintc[firstline=234,lastline=237,highlightlines={235-237}]{../ocaml/runtime/amd64.S}
      }
      \onslide*<1>{\listCamlRaiseExn[highlightlines=720]{}}
      \onslide*<2>{\listCamlRaiseExn[highlightlines=722]{}}
      \onslide*<3->{\providecommand\step{5}\input{diagrams/backtrace/backtrace.tex}}
    \end{column}
  \end{columns}
\end{frame}

\begin{frame}{Backtraces}{\funcname{caml\_probably\_raise}'s handler doesn't catch \typename{ExnC}}
  \begin{columns}[c]
    \begin{column}{0.45\textwidth}
      \minipage[c][0.75\textheight][s]{\columnwidth}
      \begin{itemize}
        \item<1-> \funcname{match \dots with}\footnotemark pattern matching can also match exceptions
        \item<1-> The CMM of \funcname{probably\_raise} shows that this \funcname{match \dots with} is implemented with a pattern matching and a \funcname{try \dots with}
        \item<1-> But this one only matches on \typename{ExnA} exception
        \item<2-> Since the pattern matching fails to catch the exception, \funcname{reraise} is called with our current \typename{ExnC} exception
        \item<3-> Disassembly shows that \funcname{reraise} is actually a call to \funcname{caml\_reraise\_exn}, which is also an assembly function provided by the runtime
      \end{itemize}
      \vfill
      \mintocaml[firstline=15,lastline=21,highlightlines={18-21}]{../src/backtrace/backtrace.ml}
      \endminipage
    \end{column}
    \begin{column}{0.55\textwidth}
      \centering
      \onslide*<1>{\mintcmm[highlightlines={10-18}]{../src/backtrace/camlBacktrace\_\_probably\_raise\_351.cmm}}
      \onslide*<2>{\listcmm[highlightlines=18]{../src/backtrace/camlBacktrace\_\_probably\_raise_351.cmm}{CMM of probably\_raise}}
      \onslide*<3>{\mintobjdump[firstline=5,lastline=35,highlightlines=31]{../src/backtrace/camlBacktrace\_\_probably\_raise_351.objdump}}
    \end{column}
  \end{columns}
  \only<1>{\footnotetext[1]{https://blog.janestreet.com/pattern-matching-and-exception-handling-unite/}}
\end{frame}

\newcommand\listCamlReraiseExn[1][]{
  \listasm[firstline=727,lastline=734,#1]{../ocaml/runtime/amd64.S}{runtime/amd64.S:727}}

\begin{frame}{Backtraces}{Introducing \funcname{caml\_reraise\_exn}}
  \begin{columns}[c]
    \begin{column}{0.5\textwidth}
      \minipage[c][0.66\textheight][s]{\columnwidth}
      \begin{itemize}
        \item<1-> The exception have to be reraised to the next handler
        \item<2-> First, checks if the backtrace should be collected with \funcarg{domain\_state->backtrace\_active}
        \only<3>{
        \begin{itemize}
            \item If not, behaves like a \funcname{raise\_notrace} with \funcname{RESTORE\_EXN\_HANDLER\_OCAML}
        \end{itemize}
        }
        \item<4-> Jumps into \funcname{caml\_raise\_exn}
        \item<5-> Calls \funcname{caml\_stash\_backtrace} again, to scan the next part of the stack: from the trap just pop-ed to the next trap
      \end{itemize}
      \vfill
      \mintocaml[firstline=15,lastline=21,highlightlines={18-21}]{../src/backtrace/backtrace.ml}
      \endminipage
    \end{column}
    \begin{column}{0.5\textwidth}
      \raggedleft
      \onslide*<1>{\listCamlReraiseExn{}}
      \onslide*<2>{\listCamlReraiseExn[highlightlines=729]{}}
      \onslide*<3>{
        \mintc[firstline=190,lastline=192,highlightlines={191-192}]{../ocaml/runtime/amd64.S}
        \mintc[firstline=234,lastline=237,highlightlines={235-237}]{../ocaml/runtime/amd64.S}
        \medskip
        \listCamlReraiseExn[highlightlines=731]{}
      }
      \onslide*<4-5>{
        % TODO refactor with \listCamlReraiseExn
        \mintasm[firstline=727,lastline=734,highlightlines=730]{../ocaml/runtime/amd64.S}
        \medskip
      }
      \onslide*<4>{\listCamlRaiseExn[highlightlines=711]{}}
      \onslide*<5>{\listCamlRaiseExn[highlightlines=720]{}}
    \end{column}
  \end{columns}
\end{frame}

\begin{frame}{Backtraces}{Backtrace collection continues}
  \begin{columns}[c]
    \begin{column}{0.55\textwidth}
      \minipage[c][0.75\textheight][s]{\columnwidth}
      \begin{itemize}
        \item<1-> \funcname{caml\_stash\_backtrace} scans the older part of the stack, up to the next trap
        \item<6-> Controls jumps to the nested exception handler in \funcname{maybe\_raise}, which matches only on \typename{ExnB}
        \item<7-> \funcname{caml\_reraise\_exn} is called again, backtrace collection resumes with \funcname{caml\_stash\_backtrace}
      \end{itemize}
      \medskip
      \begin{itemize}
        \item<2-> Constructed backtrace:
        \begin{itemize}
          \item <2-> \funcname{definitely\_raise}
          \item <2-> \funcname{probably\_raise}
          \item <3-> \funcname{probably\_raise} (re-raise)
          \item <4-> \funcname{maybe\_raise}
          \item <5-> \funcname{main}
          \item <8-> \funcname{main} (re-raise)
        \end{itemize}
      \end{itemize}
      \vfill
      \onslide*<1-5>{\mintocaml[firstline=15,lastline=21]{../src/backtrace/backtrace.ml}}
      \onslide*<6>{\mintocaml[firstline=28,lastline=34,highlightlines=31]{../src/backtrace/backtrace.ml}}
      \onslide*<7->{\mintocaml[firstline=28,lastline=34,highlightlines=33]{../src/backtrace/backtrace.ml}}
      \endminipage
    \end{column}
    \begin{column}{0.45\textwidth}
      \raggedleft
      \onslide*<1>{\providecommand\step{6}\input{diagrams/backtrace/backtrace.tex}}%
      \onslide*<2>{\providecommand\step{6}\input{diagrams/backtrace/backtrace.tex}}%
      \onslide*<3>{\providecommand\step{7}\input{diagrams/backtrace/backtrace.tex}}%
      \onslide*<4>{\providecommand\step{8}\input{diagrams/backtrace/backtrace.tex}}%
      \onslide*<5>{\providecommand\step{9}\input{diagrams/backtrace/backtrace.tex}}%
      \onslide*<6>{\providecommand\step{10}\input{diagrams/backtrace/backtrace.tex}}%
      \onslide*<7>{\providecommand\step{11}\input{diagrams/backtrace/backtrace.tex}}%
      \onslide*<8>{\providecommand\step{12}\input{diagrams/backtrace/backtrace.tex}}%
      \onslide*<9>{\providecommand\step{13}\input{diagrams/backtrace/backtrace.tex}}%
    \end{column}
  \end{columns}
\end{frame}

\begin{frame}{Backtraces}{Printing the backtrace}
  \begin{columns}[c]
    \begin{column}{0.45\textwidth}
      \minipage[c][0.66\textheight][s]{\columnwidth}
      \begin{itemize}
        \item<1-> The exception backtrace can now be printed to \funcarg{stdout} with \funcname{Printexc.print_backtrace}
        \item<2-> First the exception backtrace is retrieved into an OCaml array with \funcname{get_raw_backtrace }
        \item<3-> Which is implemented as the \funcname{caml_get_exception_raw_backtrace} C function in the runtime
        \item<4-> And finally printed with \funcname{print_exception_backtrace}
      \end{itemize}
      \vfill
      \mintocaml[firstline=28,lastline=34,highlightlines=33]{../src/backtrace/backtrace.ml}
      \endminipage
    \end{column}
    \begin{column}{0.55\textwidth}
      \onslide*<1,2>{
        \mintocaml[firstline=100,lastline=101]{../ocaml/stdlib/printexc.ml}
      }
      \only<1>{
        \listocaml[firstline=170,lastline=172]{../ocaml/stdlib/printexc.ml}{stdlib/printexc.ml:170}
      }
      \only<2>{
        \listocaml[firstline=170,lastline=172,highlightlines=172]{../ocaml/stdlib/printexc.ml}{stdlib/printexc.ml:170}
      }
      \only<3>{
        \mintc[firstline=154,lastline=156]{../ocaml/runtime/backtrace.c}
        \listc[firstline=171,lastline=192,highlightlines={184-187}]{../ocaml/runtime/backtrace.c}{runtime/backtrace.c:154}
      }
      \only<4>{
        \listocaml[firstline=155,lastline=165]{../ocaml/stdlib/printexc.ml}{stdlib/printexc.ml:55}
      }
    \end{column}
  \end{columns}
\end{frame}


\subsection{The multiple kinds of raise}
\frameSubsection{}{}

\begin{frame}{Backtraces}{When backtraces are not needed}
  \begin{columns}[c]
    \begin{column}{0.45\textwidth}
      \minipage[c][0.66\textheight][s]{\columnwidth}
      \begin{itemize}
        \item<1-> What if the backtrace for an exception is never needed?
        \item<2-> It's possible to raise the exception explicitly with \funcname{raise\_notrace} to make raising as cheap as possible
        \item<3-> An example of use in the \funcname{dynlink} library of the OCaml compiler
      \end{itemize}
      \vfill
      \onslide<3->{
      \listocaml[firstline=205,lastline=209,highlightlines=207]{../ocaml/otherlibs/dynlink/dynlink\_compilerlibs/misc.ml}{otherlibs/dynlink/dynlink\_compilerlibs/misc.ml:205}
      }
      \endminipage
    \end{column}
    \begin{column}{0.55\textwidth}
    \only<4>{
      \mintcmm[highlightlines=3]{../src/notrace/camlNotrace\_\_fun_347.cmm}
      \listcmm{../src/notrace/camlNotrace\_\_all\_somes\_327.cmm}{all\_somes}
    }
    \onslide*<5->{
      \listobjdump[highlightlines={6-9}]{../src/notrace/camlNotrace\_\_fun_347.objdump}{anonymous function from \funcname{all\_somes}}
    }
    \end{column}
  \end{columns}
\end{frame}

\begin{frame}{Backtraces}{Summary table of the multiple kinds of raise}
  \begin{itemize}
    \item When debugging information is enabled and if the backtrace of an exception isn't needed, it's possible to raise with \funcname{raise\_notrace} for performance
    \item When debugging information is \emph{not} enabled, the compiler optimises \funcname{raise} calls to inline assembly (same effect as using \funcname{raise\_notrace})
  \end{itemize}
  \bigskip
  \centering
  \begin{tabular}{| l | c | c | c | c |}
    \hline
    \parbox{10em}{Debug information \\ (\funcarg{-g} flag)}  & \multicolumn{2}{|c|}{Disabled}                                     & \multicolumn{2}{|c|}{Enabled} \\
    \hline
    Raise kind                                               & \funcname{raise} & \funcname{raise\_notrace}                       & \funcname{raise} & \funcname{raise\_notrace} \\
    \hline
    Compilation                                              & \parbox{8em}{inline assembly\\ (optimization)} & inline assembly   & \funcname{caml\_raise\_exn} & inline assembly \\
    \hline
  \end{tabular}
\end{frame}


%
% Takeaway #5
%
\subsection*{Takeaway \#5}
\frameSubsectionTakeaway{}
\againframe<5>{takeaway}
}%
      \onslide*<4>{\providecommand\step{8}%
% Backtraces
%
\subsection{One advantage of raising with debugging information}
\frameSubsection{}{}

\begin{frame}{Backtraces}{Debugging information \& source code}
  \begin{columns}[c]
    \begin{column}{0.5\textwidth}
      \begin{itemize}
        \item Raising an exception is fun and games, but how do we know where the exception originated? $\rightarrow$ With the backtrace associated with the exception!
        \item In order to get a backtrace from the point where an exception was raised, it's necessary to compile with debugging information enabled (\funcarg{-g} flag for the compiler)
        \item Same example as in the `Nested handlers' section
      \end{itemize}
    \end{column}
    \begin{column}{0.5\textwidth}
      \listocaml[firstline=9,lastline=34]{../src/backtrace/backtrace.ml}{backtrace/backtrace.ml}
    \end{column}
  \end{columns}
\end{frame}

\begin{frame}{Backtraces}{CMM and disassembly of \funcname{definitely\_raise}}
  \begin{columns}[c]
    \begin{column}{0.5\textwidth}
      \minipage[c][0.66\textheight][s]{\columnwidth}
      \begin{itemize}
        \item<1-> An effect of compiling with debugging information (\funcarg{-g}): the CMM no longer uses \funcname{raise\_notrace} to raise the exception but \funcname{raise} instead
        \item<2-> Disassembly shows that CMM \funcname{raise} is actually a call to \funcname{caml\_raise\_exn}, a function in assembler provided by the runtime
      \end{itemize}
      \vfill
      \mintocaml[firstline=9,lastline=13,highlightlines=12]{../src/backtrace/backtrace.ml}
      \endminipage
    \end{column}
    \begin{column}{0.5\textwidth}
      \onslide*<1>{
        \mintcmm[highlightlines=5]{../src/backtrace/camlBacktrace\_\_definitely\_raise\_347.cmm}
      }
      \onslide*<2>{
        \mintobjdump[firstline=2,highlightlines=14]{../src/backtrace/camlBacktrace\_\_definitely\_raise\_347.objdump}
      }
    \end{column}
  \end{columns}
\end{frame}

\begin{frame}{Backtraces}{Introducing \funcname{caml\_raise\_exn}}
  \begin{columns}[c]
    \begin{column}{0.5\textwidth}
      \minipage[c][0.66\textheight][s]{\columnwidth}
      \onslide*<1>{
        \begin{itemize}
          \item Raising the exception is performed using \funcname{caml\_raise\_exn} which is
            \begin{itemize}
              \item An assembly function provided by the runtime
              \item Architecture specific
            \end{itemize}
          \item Let's see what it does differently from \funcname{raise\_notrace}\dots
          \item We are about to raise \typename{ExnC} with \funcname{caml\_raise\_exn} from \funcname{definitely\_raise}
        \end{itemize}
      }
      \onslide*<2>{
        \mintobjdump[firstline=2,highlightlines=14]{../src/backtrace/camlBacktrace\_\_definitely\_raise\_347.objdump}
      }
      \vfill
      \mintocaml[firstline=9,lastline=13,highlightlines=12]{../src/backtrace/backtrace.ml}
      \endminipage
    \end{column}
    \begin{column}{0.5\textwidth}
      \centering
      \providecommand\step{1}\input{diagrams/backtrace/backtrace.tex}
    \end{column}
  \end{columns}
\end{frame}

\begin{frame}{Backtraces}{But before that, we need more fields from \typename{caml\_domain\_state}}
  \begin{columns}
    \begin{column}{0.5\textwidth}
      \begin{itemize}
        \item Remember \typename{caml\_domain\_state} the per-domain runtime state?
        \item Some other members are of interest to us now\dots
        \item \localname{backtrace\_active} is used to know if the runtime should collect backtraces
        \item \localname{backtrace\_active} can be set by
          \begin{itemize}
            \item The \funcarg{b} flag in the \funcname{OCAMLRUNPARAM} environment variable
            \item \mintinline{ocaml}{Printexc.record_backtrace : bool -> unit} \footnotemark
          \end{itemize}
      \end{itemize}
    \end{column}
    \begin{column}{0.5\textwidth}
      \begin{listing}
        \mintc[highlightlines={9-11}]{src/ocaml/domain\_state\_backtrace.h}
        \caption{runtime/caml/domain\_state.h}
      \end{listing}
    \end{column}
  \end{columns}
  \footnotetext[1]{https://v2.ocaml.org/api/Printexc.html}
\end{frame}

\newcommand\listCamlRaiseExn[1][]{
  \listasm[firstline=702,lastline=725,#1]{../ocaml/runtime/amd64.S}{runtime/amd64.S:702}}

\begin{frame}{Backtraces}{Walk through \funcname{caml\_raise\_exn}}
  \begin{columns}[T]
    \begin{column}{0.5\textwidth}
      \begin{itemize}
        \item<1-> First checks whether exceptions backtrace should be recorded
        \item<1-> By checking the \funcarg{domain\_state->backtrace\_active} value
        \item<2-> If not, acts as \funcname{raise\_notrace} with \funcname{RESTORE\_EXN\_HANDLER\_OCAML}
          \begin{itemize}
            \item Set \regname{RSP} to \localname{domain\_state->exn\_handler} value
            \item Restore \localname{domain\_state->exn\_handler} from trap
            \item Jump with \funcname{ret} (same effect as using \funcname{pop} and \funcname{jmp})
          \end{itemize}
        \item<3-> Otherwise, switches to the C stack to be able to execute C code
        \item<4-> Calls \funcname{caml\_stash\_backtrace}, a C function provided by the runtime\ignorespaces
          \onslide<6->{, with
            \begin{itemize}
              \item \funcarg{exn}: exception value
              \item \funcarg{pc}: code address of the raising point
              \item \funcarg{sp}: stack address of the raising function
              \item \funcarg{trapsp}: stack address of the next handler
            \end{itemize}
          }
      \end{itemize}
      \bigskip
      \mintocaml[firstline=9,lastline=13,highlightlines=12]{../src/backtrace/backtrace.ml}
    \end{column}
    \begin{column}{0.5\textwidth}
      \raggedleft
      \onslide*<1,3-4>{
        \mintc[firstline=190,lastline=192]{../ocaml/runtime/amd64.S}
        \mintc[firstline=234,lastline=237]{../ocaml/runtime/amd64.S}
      }
      \onslide*<2>{
        \mintc[firstline=190,lastline=192,highlightlines={191-192}]{../ocaml/runtime/amd64.S}
        \mintc[firstline=234,lastline=237,highlightlines={235-237}]{../ocaml/runtime/amd64.S}
      }
      \onslide*<1>{\listCamlRaiseExn[highlightlines=705]{}}%
      \onslide*<2>{\listCamlRaiseExn[highlightlines={707-708}]{}}%
      \onslide*<3>{\listCamlRaiseExn[highlightlines={712-714}]{}}%
      \onslide*<4>{\listCamlRaiseExn[highlightlines={715-720}]{}}%
      \onslide*<5>{\providecommand\step{1}\input{diagrams/backtrace/backtrace.tex}}%
      \onslide*<6>{\providecommand\step{2}\input{diagrams/backtrace/backtrace.tex}}%
    \end{column}
  \end{columns}
\end{frame}

\newcommand\listStashBacktrace[1][]{
  \listc[firstline=91,lastline=120,#1]{../ocaml/runtime/backtrace\_nat.c}{runtime/backtrace\_nat.c:91}}

\begin{frame}{Backtraces}{Introducing \funcname{caml\_stash\_backtrace}}
  \begin{columns}[c]
    \begin{column}{0.5\textwidth}
      \minipage[c][0.66\textheight][s]{\columnwidth}
      \begin{itemize}
        \item<1-> A C function provided by the runtime (specific to native code)
        \item<2-> It scans the stack, between the stack pointer at the raising point up to the next trap, in order to collect the names of the detected function frames
          \onslide*<2>{
            \begin{itemize}
              \item And more information, like line number, column number...
            \end{itemize}
          }
        \item<3-> It uses the same technique and information as the Garbage Collector (GC) uses to scan the values live on the stack
          \onslide*<4>{
            \begin{itemize}
              \item \typename{frame\_descr} are at the core of stack unwinding
            \end{itemize}
          }
      \end{itemize}
      \vfill
      \mintocaml[firstline=9,lastline=13,highlightlines=12]{../src/backtrace/backtrace.ml}
      \endminipage
    \end{column}
    \begin{column}{0.5\textwidth}
      \onslide*<1>{\listStashBacktrace[highlightlines=91]{}}
      \onslide*<2>{\listStashBacktrace[highlightlines={91,117-118}]{}}
      \onslide*<3->{\listStashBacktrace[highlightlines={94,106,109-110}]{}}
    \end{column}
  \end{columns}
\end{frame}

\newcommand\listFrameDescr[1][]{
  \listocaml[firstline=111,lastline=124,#1]{../ocaml/asmcomp/emitaux.ml}{asmcomp/emitaux.ml:111}}
\newcommand\listDebugInfo[1][]{
  \listocaml[firstline=38,lastline=49,#1]{../ocaml/lambda/debuginfo.mli}{lambda/debuginfo.mli:38}}

\begin{frame}{Backtraces}{\typename{frame\_descr} and \typename{frame\_debuginfo} in a nutshell}
  \begin{columns}[c]
    \begin{column}{0.5\textwidth}
      \begin{itemize}
        \item<1-> For every return address in an OCaml program, there is a associated \typename{frame\_descr}
        \item<2-> A \typename{frame\_descr} specifies
        \begin{itemize}
          \item<2-> The size of the function stack frame
          \item<3-> Where live values are on the stack (so that the GC can mark them as live and prevent from being swept)
        \end{itemize}
        \item<4-> When compiled with debug information, \typename{frame\_descr} will be linked to a \typename{frame\_debuginfo}
        \begin{itemize}
          \item<5-> Contains information about the position in the source code (file name, line and column number)
        \end{itemize}
      \end{itemize}
    \end{column}
    \begin{column}{0.5\textwidth}
        \onslide*<1>{\listFrameDescr[highlightlines={118-122}]{}}
        \onslide*<2>{\listFrameDescr[highlightlines={120}]{}}
        \onslide*<3>{\listFrameDescr[highlightlines={121}]{}}
        \onslide*<4->{\listFrameDescr[highlightlines={122}]{}}
        \medskip
        \onslide*<1-3>{\listDebugInfo{}}
        \onslide*<4>{\listDebugInfo[highlightlines={38,49}]{}}
        \onslide*<5>{\listDebugInfo[highlightlines={39-46}]{}}
    \end{column}
  \end{columns}
\end{frame}

\begin{frame}{Backtraces}{Scanning the stack with \typename{frame\_descr}}
  \begin{columns}[c]
    \begin{column}{0.5\textwidth}
      \begin{itemize}
        \item Given the return address of the code that raises the exception by calling \funcname{caml\_raise\_exn} (\funcarg{pc} argument of \funcname{caml\_stash\_backtrace})
        \item And the position of the stack pointer (\funcarg{sp} argument of \funcname{caml\_stash\_backtrace})
        \item The frame size given by \typename{frame\_descr}.\funcarg{frame\_size}, allows to find the end of the previous frame
        \item And the return address of the caller of this function
        \bigskip
        \onslide<3->{
        \item Note that traps (here the reddish \funcname{ExnA}, \funcname{ExnB} and \funcname{ExnC} blocks) belong to the frame that installed them, and so the frame size changes when traps are removed
        }
      \end{itemize}
    \end{column}
    \begin{column}{0.5\textwidth}
      \raggedleft
      \onslide*<1>{\providecommand\step{2}\input{diagrams/backtrace/backtrace.tex}}%
      \onslide*<2>{\providecommand\step{1}\input{diagrams/backtrace/scanstack.tex}}%
      \onslide*<3>{\providecommand\step{2}\input{diagrams/backtrace/scanstack.tex}}%
      \onslide*<4>{\providecommand\step{3}\input{diagrams/backtrace/scanstack.tex}}%
      \onslide*<5>{\providecommand\step{4}\input{diagrams/backtrace/scanstack.tex}}%
      \onslide*<6>{\providecommand\step{5}\input{diagrams/backtrace/scanstack.tex}}%
    \end{column}
  \end{columns}
\end{frame}

% Duplicate of previous-previous slide
\begin{frame}{Backtraces}{Continuing on \funcname{caml\_stash\_backtrace}}
  \begin{columns}[c]
    \begin{column}{0.5\textwidth}
      \minipage[c][0.75\textheight][s]{\columnwidth}
      \begin{itemize}
          \color{gray}
        \item A C function provided by the runtime (specific to native code)
        \item It scans the stack, between the stack pointer at the raising point up to the next trap, in order to collect the names of the detected function frames
        \item It uses the same technique and information as the Garbage Collector (GC) uses to scan the values live on the stack
      \end{itemize}
      \begin{itemize}
        \item For each function frame detected, store a pointer to the \typename{frame\_descr} in the \localname{domain\_state->backtrace\_buffer} array, effectively constructing the backtrace
      \end{itemize}
      \onslide<2->{
        \begin{itemize}
            \smallskip
          \item Constructed backtrace:
            \funcname{definitely\_raise},
            \onslide<3->{\funcname{probably\_raise}}
        \end{itemize}
      }
      \vfill
      \mintocaml[firstline=9,lastline=13,highlightlines=12]{../src/backtrace/backtrace.ml}
      \endminipage
    \end{column}
    \begin{column}{0.5\textwidth}
      \raggedleft
      \onslide*<1>{\listStashBacktrace[highlightlines={114-115}]{}}
      \onslide*<2>{\providecommand\step{3}\input{diagrams/backtrace/backtrace.tex}}%
      \onslide*<3>{\providecommand\step{4}\input{diagrams/backtrace/backtrace.tex}}%
    \end{column}
  \end{columns}
\end{frame}

\begin{frame}{Backtraces}{Back to \funcname{caml\_raise\_exn}}
  \begin{columns}[c]
    \begin{column}{0.5\textwidth}
      \minipage[c][0.66\textheight][s]{\columnwidth}
      \begin{itemize}\color{gray}
        \item Checks wheteher exceptions backtrace should be recorded, by checking the \funcarg{domain\_state->backtrace\_active} value
        \item Switches to the C stack to be able to execute C code
        \item Calls \funcname{caml\_stash\_backtrace}, to collect the backtrace up to the next trap
      \end{itemize}
      \onslide*<2->{
        \begin{itemize}
          \item<2-> Executes the exception handler in \funcname{probably\_raise} with \funcname{RESTORE\_EXN\_HANDLER\_OCAML}
            \begin{itemize}
              \item Set \regname{RSP} to \localname{domain\_state->exn\_handler} value
              \item Restore \localname{domain\_state->exn\_handler} from trap
              \item Jump to the handler code with \funcname{ret}
            \end{itemize}
        \end{itemize}
      }
      \vfill
      \onslide*<1>{\mintocaml[firstline=9,lastline=13,highlightlines=12]{../src/backtrace/backtrace.ml}}
      \onslide*<2->{\mintocaml[firstline=15,lastline=21,highlightlines={19-21}]{../src/backtrace/backtrace.ml}}
      \endminipage
    \end{column}
    \begin{column}{0.5\textwidth}
      \centering
      \onslide*<1>{
        \mintc[firstline=190,lastline=192]{../ocaml/runtime/amd64.S}
        \mintc[firstline=234,lastline=237]{../ocaml/runtime/amd64.S}
      }
      \onslide*<2>{
        \mintc[firstline=190,lastline=192,highlightlines={191-192}]{../ocaml/runtime/amd64.S}
        \mintc[firstline=234,lastline=237,highlightlines={235-237}]{../ocaml/runtime/amd64.S}
      }
      \onslide*<1>{\listCamlRaiseExn[highlightlines=720]{}}
      \onslide*<2>{\listCamlRaiseExn[highlightlines=722]{}}
      \onslide*<3->{\providecommand\step{5}\input{diagrams/backtrace/backtrace.tex}}
    \end{column}
  \end{columns}
\end{frame}

\begin{frame}{Backtraces}{\funcname{caml\_probably\_raise}'s handler doesn't catch \typename{ExnC}}
  \begin{columns}[c]
    \begin{column}{0.45\textwidth}
      \minipage[c][0.75\textheight][s]{\columnwidth}
      \begin{itemize}
        \item<1-> \funcname{match \dots with}\footnotemark pattern matching can also match exceptions
        \item<1-> The CMM of \funcname{probably\_raise} shows that this \funcname{match \dots with} is implemented with a pattern matching and a \funcname{try \dots with}
        \item<1-> But this one only matches on \typename{ExnA} exception
        \item<2-> Since the pattern matching fails to catch the exception, \funcname{reraise} is called with our current \typename{ExnC} exception
        \item<3-> Disassembly shows that \funcname{reraise} is actually a call to \funcname{caml\_reraise\_exn}, which is also an assembly function provided by the runtime
      \end{itemize}
      \vfill
      \mintocaml[firstline=15,lastline=21,highlightlines={18-21}]{../src/backtrace/backtrace.ml}
      \endminipage
    \end{column}
    \begin{column}{0.55\textwidth}
      \centering
      \onslide*<1>{\mintcmm[highlightlines={10-18}]{../src/backtrace/camlBacktrace\_\_probably\_raise\_351.cmm}}
      \onslide*<2>{\listcmm[highlightlines=18]{../src/backtrace/camlBacktrace\_\_probably\_raise_351.cmm}{CMM of probably\_raise}}
      \onslide*<3>{\mintobjdump[firstline=5,lastline=35,highlightlines=31]{../src/backtrace/camlBacktrace\_\_probably\_raise_351.objdump}}
    \end{column}
  \end{columns}
  \only<1>{\footnotetext[1]{https://blog.janestreet.com/pattern-matching-and-exception-handling-unite/}}
\end{frame}

\newcommand\listCamlReraiseExn[1][]{
  \listasm[firstline=727,lastline=734,#1]{../ocaml/runtime/amd64.S}{runtime/amd64.S:727}}

\begin{frame}{Backtraces}{Introducing \funcname{caml\_reraise\_exn}}
  \begin{columns}[c]
    \begin{column}{0.5\textwidth}
      \minipage[c][0.66\textheight][s]{\columnwidth}
      \begin{itemize}
        \item<1-> The exception have to be reraised to the next handler
        \item<2-> First, checks if the backtrace should be collected with \funcarg{domain\_state->backtrace\_active}
        \only<3>{
        \begin{itemize}
            \item If not, behaves like a \funcname{raise\_notrace} with \funcname{RESTORE\_EXN\_HANDLER\_OCAML}
        \end{itemize}
        }
        \item<4-> Jumps into \funcname{caml\_raise\_exn}
        \item<5-> Calls \funcname{caml\_stash\_backtrace} again, to scan the next part of the stack: from the trap just pop-ed to the next trap
      \end{itemize}
      \vfill
      \mintocaml[firstline=15,lastline=21,highlightlines={18-21}]{../src/backtrace/backtrace.ml}
      \endminipage
    \end{column}
    \begin{column}{0.5\textwidth}
      \raggedleft
      \onslide*<1>{\listCamlReraiseExn{}}
      \onslide*<2>{\listCamlReraiseExn[highlightlines=729]{}}
      \onslide*<3>{
        \mintc[firstline=190,lastline=192,highlightlines={191-192}]{../ocaml/runtime/amd64.S}
        \mintc[firstline=234,lastline=237,highlightlines={235-237}]{../ocaml/runtime/amd64.S}
        \medskip
        \listCamlReraiseExn[highlightlines=731]{}
      }
      \onslide*<4-5>{
        % TODO refactor with \listCamlReraiseExn
        \mintasm[firstline=727,lastline=734,highlightlines=730]{../ocaml/runtime/amd64.S}
        \medskip
      }
      \onslide*<4>{\listCamlRaiseExn[highlightlines=711]{}}
      \onslide*<5>{\listCamlRaiseExn[highlightlines=720]{}}
    \end{column}
  \end{columns}
\end{frame}

\begin{frame}{Backtraces}{Backtrace collection continues}
  \begin{columns}[c]
    \begin{column}{0.55\textwidth}
      \minipage[c][0.75\textheight][s]{\columnwidth}
      \begin{itemize}
        \item<1-> \funcname{caml\_stash\_backtrace} scans the older part of the stack, up to the next trap
        \item<6-> Controls jumps to the nested exception handler in \funcname{maybe\_raise}, which matches only on \typename{ExnB}
        \item<7-> \funcname{caml\_reraise\_exn} is called again, backtrace collection resumes with \funcname{caml\_stash\_backtrace}
      \end{itemize}
      \medskip
      \begin{itemize}
        \item<2-> Constructed backtrace:
        \begin{itemize}
          \item <2-> \funcname{definitely\_raise}
          \item <2-> \funcname{probably\_raise}
          \item <3-> \funcname{probably\_raise} (re-raise)
          \item <4-> \funcname{maybe\_raise}
          \item <5-> \funcname{main}
          \item <8-> \funcname{main} (re-raise)
        \end{itemize}
      \end{itemize}
      \vfill
      \onslide*<1-5>{\mintocaml[firstline=15,lastline=21]{../src/backtrace/backtrace.ml}}
      \onslide*<6>{\mintocaml[firstline=28,lastline=34,highlightlines=31]{../src/backtrace/backtrace.ml}}
      \onslide*<7->{\mintocaml[firstline=28,lastline=34,highlightlines=33]{../src/backtrace/backtrace.ml}}
      \endminipage
    \end{column}
    \begin{column}{0.45\textwidth}
      \raggedleft
      \onslide*<1>{\providecommand\step{6}\input{diagrams/backtrace/backtrace.tex}}%
      \onslide*<2>{\providecommand\step{6}\input{diagrams/backtrace/backtrace.tex}}%
      \onslide*<3>{\providecommand\step{7}\input{diagrams/backtrace/backtrace.tex}}%
      \onslide*<4>{\providecommand\step{8}\input{diagrams/backtrace/backtrace.tex}}%
      \onslide*<5>{\providecommand\step{9}\input{diagrams/backtrace/backtrace.tex}}%
      \onslide*<6>{\providecommand\step{10}\input{diagrams/backtrace/backtrace.tex}}%
      \onslide*<7>{\providecommand\step{11}\input{diagrams/backtrace/backtrace.tex}}%
      \onslide*<8>{\providecommand\step{12}\input{diagrams/backtrace/backtrace.tex}}%
      \onslide*<9>{\providecommand\step{13}\input{diagrams/backtrace/backtrace.tex}}%
    \end{column}
  \end{columns}
\end{frame}

\begin{frame}{Backtraces}{Printing the backtrace}
  \begin{columns}[c]
    \begin{column}{0.45\textwidth}
      \minipage[c][0.66\textheight][s]{\columnwidth}
      \begin{itemize}
        \item<1-> The exception backtrace can now be printed to \funcarg{stdout} with \funcname{Printexc.print_backtrace}
        \item<2-> First the exception backtrace is retrieved into an OCaml array with \funcname{get_raw_backtrace }
        \item<3-> Which is implemented as the \funcname{caml_get_exception_raw_backtrace} C function in the runtime
        \item<4-> And finally printed with \funcname{print_exception_backtrace}
      \end{itemize}
      \vfill
      \mintocaml[firstline=28,lastline=34,highlightlines=33]{../src/backtrace/backtrace.ml}
      \endminipage
    \end{column}
    \begin{column}{0.55\textwidth}
      \onslide*<1,2>{
        \mintocaml[firstline=100,lastline=101]{../ocaml/stdlib/printexc.ml}
      }
      \only<1>{
        \listocaml[firstline=170,lastline=172]{../ocaml/stdlib/printexc.ml}{stdlib/printexc.ml:170}
      }
      \only<2>{
        \listocaml[firstline=170,lastline=172,highlightlines=172]{../ocaml/stdlib/printexc.ml}{stdlib/printexc.ml:170}
      }
      \only<3>{
        \mintc[firstline=154,lastline=156]{../ocaml/runtime/backtrace.c}
        \listc[firstline=171,lastline=192,highlightlines={184-187}]{../ocaml/runtime/backtrace.c}{runtime/backtrace.c:154}
      }
      \only<4>{
        \listocaml[firstline=155,lastline=165]{../ocaml/stdlib/printexc.ml}{stdlib/printexc.ml:55}
      }
    \end{column}
  \end{columns}
\end{frame}


\subsection{The multiple kinds of raise}
\frameSubsection{}{}

\begin{frame}{Backtraces}{When backtraces are not needed}
  \begin{columns}[c]
    \begin{column}{0.45\textwidth}
      \minipage[c][0.66\textheight][s]{\columnwidth}
      \begin{itemize}
        \item<1-> What if the backtrace for an exception is never needed?
        \item<2-> It's possible to raise the exception explicitly with \funcname{raise\_notrace} to make raising as cheap as possible
        \item<3-> An example of use in the \funcname{dynlink} library of the OCaml compiler
      \end{itemize}
      \vfill
      \onslide<3->{
      \listocaml[firstline=205,lastline=209,highlightlines=207]{../ocaml/otherlibs/dynlink/dynlink\_compilerlibs/misc.ml}{otherlibs/dynlink/dynlink\_compilerlibs/misc.ml:205}
      }
      \endminipage
    \end{column}
    \begin{column}{0.55\textwidth}
    \only<4>{
      \mintcmm[highlightlines=3]{../src/notrace/camlNotrace\_\_fun_347.cmm}
      \listcmm{../src/notrace/camlNotrace\_\_all\_somes\_327.cmm}{all\_somes}
    }
    \onslide*<5->{
      \listobjdump[highlightlines={6-9}]{../src/notrace/camlNotrace\_\_fun_347.objdump}{anonymous function from \funcname{all\_somes}}
    }
    \end{column}
  \end{columns}
\end{frame}

\begin{frame}{Backtraces}{Summary table of the multiple kinds of raise}
  \begin{itemize}
    \item When debugging information is enabled and if the backtrace of an exception isn't needed, it's possible to raise with \funcname{raise\_notrace} for performance
    \item When debugging information is \emph{not} enabled, the compiler optimises \funcname{raise} calls to inline assembly (same effect as using \funcname{raise\_notrace})
  \end{itemize}
  \bigskip
  \centering
  \begin{tabular}{| l | c | c | c | c |}
    \hline
    \parbox{10em}{Debug information \\ (\funcarg{-g} flag)}  & \multicolumn{2}{|c|}{Disabled}                                     & \multicolumn{2}{|c|}{Enabled} \\
    \hline
    Raise kind                                               & \funcname{raise} & \funcname{raise\_notrace}                       & \funcname{raise} & \funcname{raise\_notrace} \\
    \hline
    Compilation                                              & \parbox{8em}{inline assembly\\ (optimization)} & inline assembly   & \funcname{caml\_raise\_exn} & inline assembly \\
    \hline
  \end{tabular}
\end{frame}


%
% Takeaway #5
%
\subsection*{Takeaway \#5}
\frameSubsectionTakeaway{}
\againframe<5>{takeaway}
}%
      \onslide*<5>{\providecommand\step{9}%
% Backtraces
%
\subsection{One advantage of raising with debugging information}
\frameSubsection{}{}

\begin{frame}{Backtraces}{Debugging information \& source code}
  \begin{columns}[c]
    \begin{column}{0.5\textwidth}
      \begin{itemize}
        \item Raising an exception is fun and games, but how do we know where the exception originated? $\rightarrow$ With the backtrace associated with the exception!
        \item In order to get a backtrace from the point where an exception was raised, it's necessary to compile with debugging information enabled (\funcarg{-g} flag for the compiler)
        \item Same example as in the `Nested handlers' section
      \end{itemize}
    \end{column}
    \begin{column}{0.5\textwidth}
      \listocaml[firstline=9,lastline=34]{../src/backtrace/backtrace.ml}{backtrace/backtrace.ml}
    \end{column}
  \end{columns}
\end{frame}

\begin{frame}{Backtraces}{CMM and disassembly of \funcname{definitely\_raise}}
  \begin{columns}[c]
    \begin{column}{0.5\textwidth}
      \minipage[c][0.66\textheight][s]{\columnwidth}
      \begin{itemize}
        \item<1-> An effect of compiling with debugging information (\funcarg{-g}): the CMM no longer uses \funcname{raise\_notrace} to raise the exception but \funcname{raise} instead
        \item<2-> Disassembly shows that CMM \funcname{raise} is actually a call to \funcname{caml\_raise\_exn}, a function in assembler provided by the runtime
      \end{itemize}
      \vfill
      \mintocaml[firstline=9,lastline=13,highlightlines=12]{../src/backtrace/backtrace.ml}
      \endminipage
    \end{column}
    \begin{column}{0.5\textwidth}
      \onslide*<1>{
        \mintcmm[highlightlines=5]{../src/backtrace/camlBacktrace\_\_definitely\_raise\_347.cmm}
      }
      \onslide*<2>{
        \mintobjdump[firstline=2,highlightlines=14]{../src/backtrace/camlBacktrace\_\_definitely\_raise\_347.objdump}
      }
    \end{column}
  \end{columns}
\end{frame}

\begin{frame}{Backtraces}{Introducing \funcname{caml\_raise\_exn}}
  \begin{columns}[c]
    \begin{column}{0.5\textwidth}
      \minipage[c][0.66\textheight][s]{\columnwidth}
      \onslide*<1>{
        \begin{itemize}
          \item Raising the exception is performed using \funcname{caml\_raise\_exn} which is
            \begin{itemize}
              \item An assembly function provided by the runtime
              \item Architecture specific
            \end{itemize}
          \item Let's see what it does differently from \funcname{raise\_notrace}\dots
          \item We are about to raise \typename{ExnC} with \funcname{caml\_raise\_exn} from \funcname{definitely\_raise}
        \end{itemize}
      }
      \onslide*<2>{
        \mintobjdump[firstline=2,highlightlines=14]{../src/backtrace/camlBacktrace\_\_definitely\_raise\_347.objdump}
      }
      \vfill
      \mintocaml[firstline=9,lastline=13,highlightlines=12]{../src/backtrace/backtrace.ml}
      \endminipage
    \end{column}
    \begin{column}{0.5\textwidth}
      \centering
      \providecommand\step{1}\input{diagrams/backtrace/backtrace.tex}
    \end{column}
  \end{columns}
\end{frame}

\begin{frame}{Backtraces}{But before that, we need more fields from \typename{caml\_domain\_state}}
  \begin{columns}
    \begin{column}{0.5\textwidth}
      \begin{itemize}
        \item Remember \typename{caml\_domain\_state} the per-domain runtime state?
        \item Some other members are of interest to us now\dots
        \item \localname{backtrace\_active} is used to know if the runtime should collect backtraces
        \item \localname{backtrace\_active} can be set by
          \begin{itemize}
            \item The \funcarg{b} flag in the \funcname{OCAMLRUNPARAM} environment variable
            \item \mintinline{ocaml}{Printexc.record_backtrace : bool -> unit} \footnotemark
          \end{itemize}
      \end{itemize}
    \end{column}
    \begin{column}{0.5\textwidth}
      \begin{listing}
        \mintc[highlightlines={9-11}]{src/ocaml/domain\_state\_backtrace.h}
        \caption{runtime/caml/domain\_state.h}
      \end{listing}
    \end{column}
  \end{columns}
  \footnotetext[1]{https://v2.ocaml.org/api/Printexc.html}
\end{frame}

\newcommand\listCamlRaiseExn[1][]{
  \listasm[firstline=702,lastline=725,#1]{../ocaml/runtime/amd64.S}{runtime/amd64.S:702}}

\begin{frame}{Backtraces}{Walk through \funcname{caml\_raise\_exn}}
  \begin{columns}[T]
    \begin{column}{0.5\textwidth}
      \begin{itemize}
        \item<1-> First checks whether exceptions backtrace should be recorded
        \item<1-> By checking the \funcarg{domain\_state->backtrace\_active} value
        \item<2-> If not, acts as \funcname{raise\_notrace} with \funcname{RESTORE\_EXN\_HANDLER\_OCAML}
          \begin{itemize}
            \item Set \regname{RSP} to \localname{domain\_state->exn\_handler} value
            \item Restore \localname{domain\_state->exn\_handler} from trap
            \item Jump with \funcname{ret} (same effect as using \funcname{pop} and \funcname{jmp})
          \end{itemize}
        \item<3-> Otherwise, switches to the C stack to be able to execute C code
        \item<4-> Calls \funcname{caml\_stash\_backtrace}, a C function provided by the runtime\ignorespaces
          \onslide<6->{, with
            \begin{itemize}
              \item \funcarg{exn}: exception value
              \item \funcarg{pc}: code address of the raising point
              \item \funcarg{sp}: stack address of the raising function
              \item \funcarg{trapsp}: stack address of the next handler
            \end{itemize}
          }
      \end{itemize}
      \bigskip
      \mintocaml[firstline=9,lastline=13,highlightlines=12]{../src/backtrace/backtrace.ml}
    \end{column}
    \begin{column}{0.5\textwidth}
      \raggedleft
      \onslide*<1,3-4>{
        \mintc[firstline=190,lastline=192]{../ocaml/runtime/amd64.S}
        \mintc[firstline=234,lastline=237]{../ocaml/runtime/amd64.S}
      }
      \onslide*<2>{
        \mintc[firstline=190,lastline=192,highlightlines={191-192}]{../ocaml/runtime/amd64.S}
        \mintc[firstline=234,lastline=237,highlightlines={235-237}]{../ocaml/runtime/amd64.S}
      }
      \onslide*<1>{\listCamlRaiseExn[highlightlines=705]{}}%
      \onslide*<2>{\listCamlRaiseExn[highlightlines={707-708}]{}}%
      \onslide*<3>{\listCamlRaiseExn[highlightlines={712-714}]{}}%
      \onslide*<4>{\listCamlRaiseExn[highlightlines={715-720}]{}}%
      \onslide*<5>{\providecommand\step{1}\input{diagrams/backtrace/backtrace.tex}}%
      \onslide*<6>{\providecommand\step{2}\input{diagrams/backtrace/backtrace.tex}}%
    \end{column}
  \end{columns}
\end{frame}

\newcommand\listStashBacktrace[1][]{
  \listc[firstline=91,lastline=120,#1]{../ocaml/runtime/backtrace\_nat.c}{runtime/backtrace\_nat.c:91}}

\begin{frame}{Backtraces}{Introducing \funcname{caml\_stash\_backtrace}}
  \begin{columns}[c]
    \begin{column}{0.5\textwidth}
      \minipage[c][0.66\textheight][s]{\columnwidth}
      \begin{itemize}
        \item<1-> A C function provided by the runtime (specific to native code)
        \item<2-> It scans the stack, between the stack pointer at the raising point up to the next trap, in order to collect the names of the detected function frames
          \onslide*<2>{
            \begin{itemize}
              \item And more information, like line number, column number...
            \end{itemize}
          }
        \item<3-> It uses the same technique and information as the Garbage Collector (GC) uses to scan the values live on the stack
          \onslide*<4>{
            \begin{itemize}
              \item \typename{frame\_descr} are at the core of stack unwinding
            \end{itemize}
          }
      \end{itemize}
      \vfill
      \mintocaml[firstline=9,lastline=13,highlightlines=12]{../src/backtrace/backtrace.ml}
      \endminipage
    \end{column}
    \begin{column}{0.5\textwidth}
      \onslide*<1>{\listStashBacktrace[highlightlines=91]{}}
      \onslide*<2>{\listStashBacktrace[highlightlines={91,117-118}]{}}
      \onslide*<3->{\listStashBacktrace[highlightlines={94,106,109-110}]{}}
    \end{column}
  \end{columns}
\end{frame}

\newcommand\listFrameDescr[1][]{
  \listocaml[firstline=111,lastline=124,#1]{../ocaml/asmcomp/emitaux.ml}{asmcomp/emitaux.ml:111}}
\newcommand\listDebugInfo[1][]{
  \listocaml[firstline=38,lastline=49,#1]{../ocaml/lambda/debuginfo.mli}{lambda/debuginfo.mli:38}}

\begin{frame}{Backtraces}{\typename{frame\_descr} and \typename{frame\_debuginfo} in a nutshell}
  \begin{columns}[c]
    \begin{column}{0.5\textwidth}
      \begin{itemize}
        \item<1-> For every return address in an OCaml program, there is a associated \typename{frame\_descr}
        \item<2-> A \typename{frame\_descr} specifies
        \begin{itemize}
          \item<2-> The size of the function stack frame
          \item<3-> Where live values are on the stack (so that the GC can mark them as live and prevent from being swept)
        \end{itemize}
        \item<4-> When compiled with debug information, \typename{frame\_descr} will be linked to a \typename{frame\_debuginfo}
        \begin{itemize}
          \item<5-> Contains information about the position in the source code (file name, line and column number)
        \end{itemize}
      \end{itemize}
    \end{column}
    \begin{column}{0.5\textwidth}
        \onslide*<1>{\listFrameDescr[highlightlines={118-122}]{}}
        \onslide*<2>{\listFrameDescr[highlightlines={120}]{}}
        \onslide*<3>{\listFrameDescr[highlightlines={121}]{}}
        \onslide*<4->{\listFrameDescr[highlightlines={122}]{}}
        \medskip
        \onslide*<1-3>{\listDebugInfo{}}
        \onslide*<4>{\listDebugInfo[highlightlines={38,49}]{}}
        \onslide*<5>{\listDebugInfo[highlightlines={39-46}]{}}
    \end{column}
  \end{columns}
\end{frame}

\begin{frame}{Backtraces}{Scanning the stack with \typename{frame\_descr}}
  \begin{columns}[c]
    \begin{column}{0.5\textwidth}
      \begin{itemize}
        \item Given the return address of the code that raises the exception by calling \funcname{caml\_raise\_exn} (\funcarg{pc} argument of \funcname{caml\_stash\_backtrace})
        \item And the position of the stack pointer (\funcarg{sp} argument of \funcname{caml\_stash\_backtrace})
        \item The frame size given by \typename{frame\_descr}.\funcarg{frame\_size}, allows to find the end of the previous frame
        \item And the return address of the caller of this function
        \bigskip
        \onslide<3->{
        \item Note that traps (here the reddish \funcname{ExnA}, \funcname{ExnB} and \funcname{ExnC} blocks) belong to the frame that installed them, and so the frame size changes when traps are removed
        }
      \end{itemize}
    \end{column}
    \begin{column}{0.5\textwidth}
      \raggedleft
      \onslide*<1>{\providecommand\step{2}\input{diagrams/backtrace/backtrace.tex}}%
      \onslide*<2>{\providecommand\step{1}\input{diagrams/backtrace/scanstack.tex}}%
      \onslide*<3>{\providecommand\step{2}\input{diagrams/backtrace/scanstack.tex}}%
      \onslide*<4>{\providecommand\step{3}\input{diagrams/backtrace/scanstack.tex}}%
      \onslide*<5>{\providecommand\step{4}\input{diagrams/backtrace/scanstack.tex}}%
      \onslide*<6>{\providecommand\step{5}\input{diagrams/backtrace/scanstack.tex}}%
    \end{column}
  \end{columns}
\end{frame}

% Duplicate of previous-previous slide
\begin{frame}{Backtraces}{Continuing on \funcname{caml\_stash\_backtrace}}
  \begin{columns}[c]
    \begin{column}{0.5\textwidth}
      \minipage[c][0.75\textheight][s]{\columnwidth}
      \begin{itemize}
          \color{gray}
        \item A C function provided by the runtime (specific to native code)
        \item It scans the stack, between the stack pointer at the raising point up to the next trap, in order to collect the names of the detected function frames
        \item It uses the same technique and information as the Garbage Collector (GC) uses to scan the values live on the stack
      \end{itemize}
      \begin{itemize}
        \item For each function frame detected, store a pointer to the \typename{frame\_descr} in the \localname{domain\_state->backtrace\_buffer} array, effectively constructing the backtrace
      \end{itemize}
      \onslide<2->{
        \begin{itemize}
            \smallskip
          \item Constructed backtrace:
            \funcname{definitely\_raise},
            \onslide<3->{\funcname{probably\_raise}}
        \end{itemize}
      }
      \vfill
      \mintocaml[firstline=9,lastline=13,highlightlines=12]{../src/backtrace/backtrace.ml}
      \endminipage
    \end{column}
    \begin{column}{0.5\textwidth}
      \raggedleft
      \onslide*<1>{\listStashBacktrace[highlightlines={114-115}]{}}
      \onslide*<2>{\providecommand\step{3}\input{diagrams/backtrace/backtrace.tex}}%
      \onslide*<3>{\providecommand\step{4}\input{diagrams/backtrace/backtrace.tex}}%
    \end{column}
  \end{columns}
\end{frame}

\begin{frame}{Backtraces}{Back to \funcname{caml\_raise\_exn}}
  \begin{columns}[c]
    \begin{column}{0.5\textwidth}
      \minipage[c][0.66\textheight][s]{\columnwidth}
      \begin{itemize}\color{gray}
        \item Checks wheteher exceptions backtrace should be recorded, by checking the \funcarg{domain\_state->backtrace\_active} value
        \item Switches to the C stack to be able to execute C code
        \item Calls \funcname{caml\_stash\_backtrace}, to collect the backtrace up to the next trap
      \end{itemize}
      \onslide*<2->{
        \begin{itemize}
          \item<2-> Executes the exception handler in \funcname{probably\_raise} with \funcname{RESTORE\_EXN\_HANDLER\_OCAML}
            \begin{itemize}
              \item Set \regname{RSP} to \localname{domain\_state->exn\_handler} value
              \item Restore \localname{domain\_state->exn\_handler} from trap
              \item Jump to the handler code with \funcname{ret}
            \end{itemize}
        \end{itemize}
      }
      \vfill
      \onslide*<1>{\mintocaml[firstline=9,lastline=13,highlightlines=12]{../src/backtrace/backtrace.ml}}
      \onslide*<2->{\mintocaml[firstline=15,lastline=21,highlightlines={19-21}]{../src/backtrace/backtrace.ml}}
      \endminipage
    \end{column}
    \begin{column}{0.5\textwidth}
      \centering
      \onslide*<1>{
        \mintc[firstline=190,lastline=192]{../ocaml/runtime/amd64.S}
        \mintc[firstline=234,lastline=237]{../ocaml/runtime/amd64.S}
      }
      \onslide*<2>{
        \mintc[firstline=190,lastline=192,highlightlines={191-192}]{../ocaml/runtime/amd64.S}
        \mintc[firstline=234,lastline=237,highlightlines={235-237}]{../ocaml/runtime/amd64.S}
      }
      \onslide*<1>{\listCamlRaiseExn[highlightlines=720]{}}
      \onslide*<2>{\listCamlRaiseExn[highlightlines=722]{}}
      \onslide*<3->{\providecommand\step{5}\input{diagrams/backtrace/backtrace.tex}}
    \end{column}
  \end{columns}
\end{frame}

\begin{frame}{Backtraces}{\funcname{caml\_probably\_raise}'s handler doesn't catch \typename{ExnC}}
  \begin{columns}[c]
    \begin{column}{0.45\textwidth}
      \minipage[c][0.75\textheight][s]{\columnwidth}
      \begin{itemize}
        \item<1-> \funcname{match \dots with}\footnotemark pattern matching can also match exceptions
        \item<1-> The CMM of \funcname{probably\_raise} shows that this \funcname{match \dots with} is implemented with a pattern matching and a \funcname{try \dots with}
        \item<1-> But this one only matches on \typename{ExnA} exception
        \item<2-> Since the pattern matching fails to catch the exception, \funcname{reraise} is called with our current \typename{ExnC} exception
        \item<3-> Disassembly shows that \funcname{reraise} is actually a call to \funcname{caml\_reraise\_exn}, which is also an assembly function provided by the runtime
      \end{itemize}
      \vfill
      \mintocaml[firstline=15,lastline=21,highlightlines={18-21}]{../src/backtrace/backtrace.ml}
      \endminipage
    \end{column}
    \begin{column}{0.55\textwidth}
      \centering
      \onslide*<1>{\mintcmm[highlightlines={10-18}]{../src/backtrace/camlBacktrace\_\_probably\_raise\_351.cmm}}
      \onslide*<2>{\listcmm[highlightlines=18]{../src/backtrace/camlBacktrace\_\_probably\_raise_351.cmm}{CMM of probably\_raise}}
      \onslide*<3>{\mintobjdump[firstline=5,lastline=35,highlightlines=31]{../src/backtrace/camlBacktrace\_\_probably\_raise_351.objdump}}
    \end{column}
  \end{columns}
  \only<1>{\footnotetext[1]{https://blog.janestreet.com/pattern-matching-and-exception-handling-unite/}}
\end{frame}

\newcommand\listCamlReraiseExn[1][]{
  \listasm[firstline=727,lastline=734,#1]{../ocaml/runtime/amd64.S}{runtime/amd64.S:727}}

\begin{frame}{Backtraces}{Introducing \funcname{caml\_reraise\_exn}}
  \begin{columns}[c]
    \begin{column}{0.5\textwidth}
      \minipage[c][0.66\textheight][s]{\columnwidth}
      \begin{itemize}
        \item<1-> The exception have to be reraised to the next handler
        \item<2-> First, checks if the backtrace should be collected with \funcarg{domain\_state->backtrace\_active}
        \only<3>{
        \begin{itemize}
            \item If not, behaves like a \funcname{raise\_notrace} with \funcname{RESTORE\_EXN\_HANDLER\_OCAML}
        \end{itemize}
        }
        \item<4-> Jumps into \funcname{caml\_raise\_exn}
        \item<5-> Calls \funcname{caml\_stash\_backtrace} again, to scan the next part of the stack: from the trap just pop-ed to the next trap
      \end{itemize}
      \vfill
      \mintocaml[firstline=15,lastline=21,highlightlines={18-21}]{../src/backtrace/backtrace.ml}
      \endminipage
    \end{column}
    \begin{column}{0.5\textwidth}
      \raggedleft
      \onslide*<1>{\listCamlReraiseExn{}}
      \onslide*<2>{\listCamlReraiseExn[highlightlines=729]{}}
      \onslide*<3>{
        \mintc[firstline=190,lastline=192,highlightlines={191-192}]{../ocaml/runtime/amd64.S}
        \mintc[firstline=234,lastline=237,highlightlines={235-237}]{../ocaml/runtime/amd64.S}
        \medskip
        \listCamlReraiseExn[highlightlines=731]{}
      }
      \onslide*<4-5>{
        % TODO refactor with \listCamlReraiseExn
        \mintasm[firstline=727,lastline=734,highlightlines=730]{../ocaml/runtime/amd64.S}
        \medskip
      }
      \onslide*<4>{\listCamlRaiseExn[highlightlines=711]{}}
      \onslide*<5>{\listCamlRaiseExn[highlightlines=720]{}}
    \end{column}
  \end{columns}
\end{frame}

\begin{frame}{Backtraces}{Backtrace collection continues}
  \begin{columns}[c]
    \begin{column}{0.55\textwidth}
      \minipage[c][0.75\textheight][s]{\columnwidth}
      \begin{itemize}
        \item<1-> \funcname{caml\_stash\_backtrace} scans the older part of the stack, up to the next trap
        \item<6-> Controls jumps to the nested exception handler in \funcname{maybe\_raise}, which matches only on \typename{ExnB}
        \item<7-> \funcname{caml\_reraise\_exn} is called again, backtrace collection resumes with \funcname{caml\_stash\_backtrace}
      \end{itemize}
      \medskip
      \begin{itemize}
        \item<2-> Constructed backtrace:
        \begin{itemize}
          \item <2-> \funcname{definitely\_raise}
          \item <2-> \funcname{probably\_raise}
          \item <3-> \funcname{probably\_raise} (re-raise)
          \item <4-> \funcname{maybe\_raise}
          \item <5-> \funcname{main}
          \item <8-> \funcname{main} (re-raise)
        \end{itemize}
      \end{itemize}
      \vfill
      \onslide*<1-5>{\mintocaml[firstline=15,lastline=21]{../src/backtrace/backtrace.ml}}
      \onslide*<6>{\mintocaml[firstline=28,lastline=34,highlightlines=31]{../src/backtrace/backtrace.ml}}
      \onslide*<7->{\mintocaml[firstline=28,lastline=34,highlightlines=33]{../src/backtrace/backtrace.ml}}
      \endminipage
    \end{column}
    \begin{column}{0.45\textwidth}
      \raggedleft
      \onslide*<1>{\providecommand\step{6}\input{diagrams/backtrace/backtrace.tex}}%
      \onslide*<2>{\providecommand\step{6}\input{diagrams/backtrace/backtrace.tex}}%
      \onslide*<3>{\providecommand\step{7}\input{diagrams/backtrace/backtrace.tex}}%
      \onslide*<4>{\providecommand\step{8}\input{diagrams/backtrace/backtrace.tex}}%
      \onslide*<5>{\providecommand\step{9}\input{diagrams/backtrace/backtrace.tex}}%
      \onslide*<6>{\providecommand\step{10}\input{diagrams/backtrace/backtrace.tex}}%
      \onslide*<7>{\providecommand\step{11}\input{diagrams/backtrace/backtrace.tex}}%
      \onslide*<8>{\providecommand\step{12}\input{diagrams/backtrace/backtrace.tex}}%
      \onslide*<9>{\providecommand\step{13}\input{diagrams/backtrace/backtrace.tex}}%
    \end{column}
  \end{columns}
\end{frame}

\begin{frame}{Backtraces}{Printing the backtrace}
  \begin{columns}[c]
    \begin{column}{0.45\textwidth}
      \minipage[c][0.66\textheight][s]{\columnwidth}
      \begin{itemize}
        \item<1-> The exception backtrace can now be printed to \funcarg{stdout} with \funcname{Printexc.print_backtrace}
        \item<2-> First the exception backtrace is retrieved into an OCaml array with \funcname{get_raw_backtrace }
        \item<3-> Which is implemented as the \funcname{caml_get_exception_raw_backtrace} C function in the runtime
        \item<4-> And finally printed with \funcname{print_exception_backtrace}
      \end{itemize}
      \vfill
      \mintocaml[firstline=28,lastline=34,highlightlines=33]{../src/backtrace/backtrace.ml}
      \endminipage
    \end{column}
    \begin{column}{0.55\textwidth}
      \onslide*<1,2>{
        \mintocaml[firstline=100,lastline=101]{../ocaml/stdlib/printexc.ml}
      }
      \only<1>{
        \listocaml[firstline=170,lastline=172]{../ocaml/stdlib/printexc.ml}{stdlib/printexc.ml:170}
      }
      \only<2>{
        \listocaml[firstline=170,lastline=172,highlightlines=172]{../ocaml/stdlib/printexc.ml}{stdlib/printexc.ml:170}
      }
      \only<3>{
        \mintc[firstline=154,lastline=156]{../ocaml/runtime/backtrace.c}
        \listc[firstline=171,lastline=192,highlightlines={184-187}]{../ocaml/runtime/backtrace.c}{runtime/backtrace.c:154}
      }
      \only<4>{
        \listocaml[firstline=155,lastline=165]{../ocaml/stdlib/printexc.ml}{stdlib/printexc.ml:55}
      }
    \end{column}
  \end{columns}
\end{frame}


\subsection{The multiple kinds of raise}
\frameSubsection{}{}

\begin{frame}{Backtraces}{When backtraces are not needed}
  \begin{columns}[c]
    \begin{column}{0.45\textwidth}
      \minipage[c][0.66\textheight][s]{\columnwidth}
      \begin{itemize}
        \item<1-> What if the backtrace for an exception is never needed?
        \item<2-> It's possible to raise the exception explicitly with \funcname{raise\_notrace} to make raising as cheap as possible
        \item<3-> An example of use in the \funcname{dynlink} library of the OCaml compiler
      \end{itemize}
      \vfill
      \onslide<3->{
      \listocaml[firstline=205,lastline=209,highlightlines=207]{../ocaml/otherlibs/dynlink/dynlink\_compilerlibs/misc.ml}{otherlibs/dynlink/dynlink\_compilerlibs/misc.ml:205}
      }
      \endminipage
    \end{column}
    \begin{column}{0.55\textwidth}
    \only<4>{
      \mintcmm[highlightlines=3]{../src/notrace/camlNotrace\_\_fun_347.cmm}
      \listcmm{../src/notrace/camlNotrace\_\_all\_somes\_327.cmm}{all\_somes}
    }
    \onslide*<5->{
      \listobjdump[highlightlines={6-9}]{../src/notrace/camlNotrace\_\_fun_347.objdump}{anonymous function from \funcname{all\_somes}}
    }
    \end{column}
  \end{columns}
\end{frame}

\begin{frame}{Backtraces}{Summary table of the multiple kinds of raise}
  \begin{itemize}
    \item When debugging information is enabled and if the backtrace of an exception isn't needed, it's possible to raise with \funcname{raise\_notrace} for performance
    \item When debugging information is \emph{not} enabled, the compiler optimises \funcname{raise} calls to inline assembly (same effect as using \funcname{raise\_notrace})
  \end{itemize}
  \bigskip
  \centering
  \begin{tabular}{| l | c | c | c | c |}
    \hline
    \parbox{10em}{Debug information \\ (\funcarg{-g} flag)}  & \multicolumn{2}{|c|}{Disabled}                                     & \multicolumn{2}{|c|}{Enabled} \\
    \hline
    Raise kind                                               & \funcname{raise} & \funcname{raise\_notrace}                       & \funcname{raise} & \funcname{raise\_notrace} \\
    \hline
    Compilation                                              & \parbox{8em}{inline assembly\\ (optimization)} & inline assembly   & \funcname{caml\_raise\_exn} & inline assembly \\
    \hline
  \end{tabular}
\end{frame}


%
% Takeaway #5
%
\subsection*{Takeaway \#5}
\frameSubsectionTakeaway{}
\againframe<5>{takeaway}
}%
      \onslide*<6>{\providecommand\step{10}%
% Backtraces
%
\subsection{One advantage of raising with debugging information}
\frameSubsection{}{}

\begin{frame}{Backtraces}{Debugging information \& source code}
  \begin{columns}[c]
    \begin{column}{0.5\textwidth}
      \begin{itemize}
        \item Raising an exception is fun and games, but how do we know where the exception originated? $\rightarrow$ With the backtrace associated with the exception!
        \item In order to get a backtrace from the point where an exception was raised, it's necessary to compile with debugging information enabled (\funcarg{-g} flag for the compiler)
        \item Same example as in the `Nested handlers' section
      \end{itemize}
    \end{column}
    \begin{column}{0.5\textwidth}
      \listocaml[firstline=9,lastline=34]{../src/backtrace/backtrace.ml}{backtrace/backtrace.ml}
    \end{column}
  \end{columns}
\end{frame}

\begin{frame}{Backtraces}{CMM and disassembly of \funcname{definitely\_raise}}
  \begin{columns}[c]
    \begin{column}{0.5\textwidth}
      \minipage[c][0.66\textheight][s]{\columnwidth}
      \begin{itemize}
        \item<1-> An effect of compiling with debugging information (\funcarg{-g}): the CMM no longer uses \funcname{raise\_notrace} to raise the exception but \funcname{raise} instead
        \item<2-> Disassembly shows that CMM \funcname{raise} is actually a call to \funcname{caml\_raise\_exn}, a function in assembler provided by the runtime
      \end{itemize}
      \vfill
      \mintocaml[firstline=9,lastline=13,highlightlines=12]{../src/backtrace/backtrace.ml}
      \endminipage
    \end{column}
    \begin{column}{0.5\textwidth}
      \onslide*<1>{
        \mintcmm[highlightlines=5]{../src/backtrace/camlBacktrace\_\_definitely\_raise\_347.cmm}
      }
      \onslide*<2>{
        \mintobjdump[firstline=2,highlightlines=14]{../src/backtrace/camlBacktrace\_\_definitely\_raise\_347.objdump}
      }
    \end{column}
  \end{columns}
\end{frame}

\begin{frame}{Backtraces}{Introducing \funcname{caml\_raise\_exn}}
  \begin{columns}[c]
    \begin{column}{0.5\textwidth}
      \minipage[c][0.66\textheight][s]{\columnwidth}
      \onslide*<1>{
        \begin{itemize}
          \item Raising the exception is performed using \funcname{caml\_raise\_exn} which is
            \begin{itemize}
              \item An assembly function provided by the runtime
              \item Architecture specific
            \end{itemize}
          \item Let's see what it does differently from \funcname{raise\_notrace}\dots
          \item We are about to raise \typename{ExnC} with \funcname{caml\_raise\_exn} from \funcname{definitely\_raise}
        \end{itemize}
      }
      \onslide*<2>{
        \mintobjdump[firstline=2,highlightlines=14]{../src/backtrace/camlBacktrace\_\_definitely\_raise\_347.objdump}
      }
      \vfill
      \mintocaml[firstline=9,lastline=13,highlightlines=12]{../src/backtrace/backtrace.ml}
      \endminipage
    \end{column}
    \begin{column}{0.5\textwidth}
      \centering
      \providecommand\step{1}\input{diagrams/backtrace/backtrace.tex}
    \end{column}
  \end{columns}
\end{frame}

\begin{frame}{Backtraces}{But before that, we need more fields from \typename{caml\_domain\_state}}
  \begin{columns}
    \begin{column}{0.5\textwidth}
      \begin{itemize}
        \item Remember \typename{caml\_domain\_state} the per-domain runtime state?
        \item Some other members are of interest to us now\dots
        \item \localname{backtrace\_active} is used to know if the runtime should collect backtraces
        \item \localname{backtrace\_active} can be set by
          \begin{itemize}
            \item The \funcarg{b} flag in the \funcname{OCAMLRUNPARAM} environment variable
            \item \mintinline{ocaml}{Printexc.record_backtrace : bool -> unit} \footnotemark
          \end{itemize}
      \end{itemize}
    \end{column}
    \begin{column}{0.5\textwidth}
      \begin{listing}
        \mintc[highlightlines={9-11}]{src/ocaml/domain\_state\_backtrace.h}
        \caption{runtime/caml/domain\_state.h}
      \end{listing}
    \end{column}
  \end{columns}
  \footnotetext[1]{https://v2.ocaml.org/api/Printexc.html}
\end{frame}

\newcommand\listCamlRaiseExn[1][]{
  \listasm[firstline=702,lastline=725,#1]{../ocaml/runtime/amd64.S}{runtime/amd64.S:702}}

\begin{frame}{Backtraces}{Walk through \funcname{caml\_raise\_exn}}
  \begin{columns}[T]
    \begin{column}{0.5\textwidth}
      \begin{itemize}
        \item<1-> First checks whether exceptions backtrace should be recorded
        \item<1-> By checking the \funcarg{domain\_state->backtrace\_active} value
        \item<2-> If not, acts as \funcname{raise\_notrace} with \funcname{RESTORE\_EXN\_HANDLER\_OCAML}
          \begin{itemize}
            \item Set \regname{RSP} to \localname{domain\_state->exn\_handler} value
            \item Restore \localname{domain\_state->exn\_handler} from trap
            \item Jump with \funcname{ret} (same effect as using \funcname{pop} and \funcname{jmp})
          \end{itemize}
        \item<3-> Otherwise, switches to the C stack to be able to execute C code
        \item<4-> Calls \funcname{caml\_stash\_backtrace}, a C function provided by the runtime\ignorespaces
          \onslide<6->{, with
            \begin{itemize}
              \item \funcarg{exn}: exception value
              \item \funcarg{pc}: code address of the raising point
              \item \funcarg{sp}: stack address of the raising function
              \item \funcarg{trapsp}: stack address of the next handler
            \end{itemize}
          }
      \end{itemize}
      \bigskip
      \mintocaml[firstline=9,lastline=13,highlightlines=12]{../src/backtrace/backtrace.ml}
    \end{column}
    \begin{column}{0.5\textwidth}
      \raggedleft
      \onslide*<1,3-4>{
        \mintc[firstline=190,lastline=192]{../ocaml/runtime/amd64.S}
        \mintc[firstline=234,lastline=237]{../ocaml/runtime/amd64.S}
      }
      \onslide*<2>{
        \mintc[firstline=190,lastline=192,highlightlines={191-192}]{../ocaml/runtime/amd64.S}
        \mintc[firstline=234,lastline=237,highlightlines={235-237}]{../ocaml/runtime/amd64.S}
      }
      \onslide*<1>{\listCamlRaiseExn[highlightlines=705]{}}%
      \onslide*<2>{\listCamlRaiseExn[highlightlines={707-708}]{}}%
      \onslide*<3>{\listCamlRaiseExn[highlightlines={712-714}]{}}%
      \onslide*<4>{\listCamlRaiseExn[highlightlines={715-720}]{}}%
      \onslide*<5>{\providecommand\step{1}\input{diagrams/backtrace/backtrace.tex}}%
      \onslide*<6>{\providecommand\step{2}\input{diagrams/backtrace/backtrace.tex}}%
    \end{column}
  \end{columns}
\end{frame}

\newcommand\listStashBacktrace[1][]{
  \listc[firstline=91,lastline=120,#1]{../ocaml/runtime/backtrace\_nat.c}{runtime/backtrace\_nat.c:91}}

\begin{frame}{Backtraces}{Introducing \funcname{caml\_stash\_backtrace}}
  \begin{columns}[c]
    \begin{column}{0.5\textwidth}
      \minipage[c][0.66\textheight][s]{\columnwidth}
      \begin{itemize}
        \item<1-> A C function provided by the runtime (specific to native code)
        \item<2-> It scans the stack, between the stack pointer at the raising point up to the next trap, in order to collect the names of the detected function frames
          \onslide*<2>{
            \begin{itemize}
              \item And more information, like line number, column number...
            \end{itemize}
          }
        \item<3-> It uses the same technique and information as the Garbage Collector (GC) uses to scan the values live on the stack
          \onslide*<4>{
            \begin{itemize}
              \item \typename{frame\_descr} are at the core of stack unwinding
            \end{itemize}
          }
      \end{itemize}
      \vfill
      \mintocaml[firstline=9,lastline=13,highlightlines=12]{../src/backtrace/backtrace.ml}
      \endminipage
    \end{column}
    \begin{column}{0.5\textwidth}
      \onslide*<1>{\listStashBacktrace[highlightlines=91]{}}
      \onslide*<2>{\listStashBacktrace[highlightlines={91,117-118}]{}}
      \onslide*<3->{\listStashBacktrace[highlightlines={94,106,109-110}]{}}
    \end{column}
  \end{columns}
\end{frame}

\newcommand\listFrameDescr[1][]{
  \listocaml[firstline=111,lastline=124,#1]{../ocaml/asmcomp/emitaux.ml}{asmcomp/emitaux.ml:111}}
\newcommand\listDebugInfo[1][]{
  \listocaml[firstline=38,lastline=49,#1]{../ocaml/lambda/debuginfo.mli}{lambda/debuginfo.mli:38}}

\begin{frame}{Backtraces}{\typename{frame\_descr} and \typename{frame\_debuginfo} in a nutshell}
  \begin{columns}[c]
    \begin{column}{0.5\textwidth}
      \begin{itemize}
        \item<1-> For every return address in an OCaml program, there is a associated \typename{frame\_descr}
        \item<2-> A \typename{frame\_descr} specifies
        \begin{itemize}
          \item<2-> The size of the function stack frame
          \item<3-> Where live values are on the stack (so that the GC can mark them as live and prevent from being swept)
        \end{itemize}
        \item<4-> When compiled with debug information, \typename{frame\_descr} will be linked to a \typename{frame\_debuginfo}
        \begin{itemize}
          \item<5-> Contains information about the position in the source code (file name, line and column number)
        \end{itemize}
      \end{itemize}
    \end{column}
    \begin{column}{0.5\textwidth}
        \onslide*<1>{\listFrameDescr[highlightlines={118-122}]{}}
        \onslide*<2>{\listFrameDescr[highlightlines={120}]{}}
        \onslide*<3>{\listFrameDescr[highlightlines={121}]{}}
        \onslide*<4->{\listFrameDescr[highlightlines={122}]{}}
        \medskip
        \onslide*<1-3>{\listDebugInfo{}}
        \onslide*<4>{\listDebugInfo[highlightlines={38,49}]{}}
        \onslide*<5>{\listDebugInfo[highlightlines={39-46}]{}}
    \end{column}
  \end{columns}
\end{frame}

\begin{frame}{Backtraces}{Scanning the stack with \typename{frame\_descr}}
  \begin{columns}[c]
    \begin{column}{0.5\textwidth}
      \begin{itemize}
        \item Given the return address of the code that raises the exception by calling \funcname{caml\_raise\_exn} (\funcarg{pc} argument of \funcname{caml\_stash\_backtrace})
        \item And the position of the stack pointer (\funcarg{sp} argument of \funcname{caml\_stash\_backtrace})
        \item The frame size given by \typename{frame\_descr}.\funcarg{frame\_size}, allows to find the end of the previous frame
        \item And the return address of the caller of this function
        \bigskip
        \onslide<3->{
        \item Note that traps (here the reddish \funcname{ExnA}, \funcname{ExnB} and \funcname{ExnC} blocks) belong to the frame that installed them, and so the frame size changes when traps are removed
        }
      \end{itemize}
    \end{column}
    \begin{column}{0.5\textwidth}
      \raggedleft
      \onslide*<1>{\providecommand\step{2}\input{diagrams/backtrace/backtrace.tex}}%
      \onslide*<2>{\providecommand\step{1}\input{diagrams/backtrace/scanstack.tex}}%
      \onslide*<3>{\providecommand\step{2}\input{diagrams/backtrace/scanstack.tex}}%
      \onslide*<4>{\providecommand\step{3}\input{diagrams/backtrace/scanstack.tex}}%
      \onslide*<5>{\providecommand\step{4}\input{diagrams/backtrace/scanstack.tex}}%
      \onslide*<6>{\providecommand\step{5}\input{diagrams/backtrace/scanstack.tex}}%
    \end{column}
  \end{columns}
\end{frame}

% Duplicate of previous-previous slide
\begin{frame}{Backtraces}{Continuing on \funcname{caml\_stash\_backtrace}}
  \begin{columns}[c]
    \begin{column}{0.5\textwidth}
      \minipage[c][0.75\textheight][s]{\columnwidth}
      \begin{itemize}
          \color{gray}
        \item A C function provided by the runtime (specific to native code)
        \item It scans the stack, between the stack pointer at the raising point up to the next trap, in order to collect the names of the detected function frames
        \item It uses the same technique and information as the Garbage Collector (GC) uses to scan the values live on the stack
      \end{itemize}
      \begin{itemize}
        \item For each function frame detected, store a pointer to the \typename{frame\_descr} in the \localname{domain\_state->backtrace\_buffer} array, effectively constructing the backtrace
      \end{itemize}
      \onslide<2->{
        \begin{itemize}
            \smallskip
          \item Constructed backtrace:
            \funcname{definitely\_raise},
            \onslide<3->{\funcname{probably\_raise}}
        \end{itemize}
      }
      \vfill
      \mintocaml[firstline=9,lastline=13,highlightlines=12]{../src/backtrace/backtrace.ml}
      \endminipage
    \end{column}
    \begin{column}{0.5\textwidth}
      \raggedleft
      \onslide*<1>{\listStashBacktrace[highlightlines={114-115}]{}}
      \onslide*<2>{\providecommand\step{3}\input{diagrams/backtrace/backtrace.tex}}%
      \onslide*<3>{\providecommand\step{4}\input{diagrams/backtrace/backtrace.tex}}%
    \end{column}
  \end{columns}
\end{frame}

\begin{frame}{Backtraces}{Back to \funcname{caml\_raise\_exn}}
  \begin{columns}[c]
    \begin{column}{0.5\textwidth}
      \minipage[c][0.66\textheight][s]{\columnwidth}
      \begin{itemize}\color{gray}
        \item Checks wheteher exceptions backtrace should be recorded, by checking the \funcarg{domain\_state->backtrace\_active} value
        \item Switches to the C stack to be able to execute C code
        \item Calls \funcname{caml\_stash\_backtrace}, to collect the backtrace up to the next trap
      \end{itemize}
      \onslide*<2->{
        \begin{itemize}
          \item<2-> Executes the exception handler in \funcname{probably\_raise} with \funcname{RESTORE\_EXN\_HANDLER\_OCAML}
            \begin{itemize}
              \item Set \regname{RSP} to \localname{domain\_state->exn\_handler} value
              \item Restore \localname{domain\_state->exn\_handler} from trap
              \item Jump to the handler code with \funcname{ret}
            \end{itemize}
        \end{itemize}
      }
      \vfill
      \onslide*<1>{\mintocaml[firstline=9,lastline=13,highlightlines=12]{../src/backtrace/backtrace.ml}}
      \onslide*<2->{\mintocaml[firstline=15,lastline=21,highlightlines={19-21}]{../src/backtrace/backtrace.ml}}
      \endminipage
    \end{column}
    \begin{column}{0.5\textwidth}
      \centering
      \onslide*<1>{
        \mintc[firstline=190,lastline=192]{../ocaml/runtime/amd64.S}
        \mintc[firstline=234,lastline=237]{../ocaml/runtime/amd64.S}
      }
      \onslide*<2>{
        \mintc[firstline=190,lastline=192,highlightlines={191-192}]{../ocaml/runtime/amd64.S}
        \mintc[firstline=234,lastline=237,highlightlines={235-237}]{../ocaml/runtime/amd64.S}
      }
      \onslide*<1>{\listCamlRaiseExn[highlightlines=720]{}}
      \onslide*<2>{\listCamlRaiseExn[highlightlines=722]{}}
      \onslide*<3->{\providecommand\step{5}\input{diagrams/backtrace/backtrace.tex}}
    \end{column}
  \end{columns}
\end{frame}

\begin{frame}{Backtraces}{\funcname{caml\_probably\_raise}'s handler doesn't catch \typename{ExnC}}
  \begin{columns}[c]
    \begin{column}{0.45\textwidth}
      \minipage[c][0.75\textheight][s]{\columnwidth}
      \begin{itemize}
        \item<1-> \funcname{match \dots with}\footnotemark pattern matching can also match exceptions
        \item<1-> The CMM of \funcname{probably\_raise} shows that this \funcname{match \dots with} is implemented with a pattern matching and a \funcname{try \dots with}
        \item<1-> But this one only matches on \typename{ExnA} exception
        \item<2-> Since the pattern matching fails to catch the exception, \funcname{reraise} is called with our current \typename{ExnC} exception
        \item<3-> Disassembly shows that \funcname{reraise} is actually a call to \funcname{caml\_reraise\_exn}, which is also an assembly function provided by the runtime
      \end{itemize}
      \vfill
      \mintocaml[firstline=15,lastline=21,highlightlines={18-21}]{../src/backtrace/backtrace.ml}
      \endminipage
    \end{column}
    \begin{column}{0.55\textwidth}
      \centering
      \onslide*<1>{\mintcmm[highlightlines={10-18}]{../src/backtrace/camlBacktrace\_\_probably\_raise\_351.cmm}}
      \onslide*<2>{\listcmm[highlightlines=18]{../src/backtrace/camlBacktrace\_\_probably\_raise_351.cmm}{CMM of probably\_raise}}
      \onslide*<3>{\mintobjdump[firstline=5,lastline=35,highlightlines=31]{../src/backtrace/camlBacktrace\_\_probably\_raise_351.objdump}}
    \end{column}
  \end{columns}
  \only<1>{\footnotetext[1]{https://blog.janestreet.com/pattern-matching-and-exception-handling-unite/}}
\end{frame}

\newcommand\listCamlReraiseExn[1][]{
  \listasm[firstline=727,lastline=734,#1]{../ocaml/runtime/amd64.S}{runtime/amd64.S:727}}

\begin{frame}{Backtraces}{Introducing \funcname{caml\_reraise\_exn}}
  \begin{columns}[c]
    \begin{column}{0.5\textwidth}
      \minipage[c][0.66\textheight][s]{\columnwidth}
      \begin{itemize}
        \item<1-> The exception have to be reraised to the next handler
        \item<2-> First, checks if the backtrace should be collected with \funcarg{domain\_state->backtrace\_active}
        \only<3>{
        \begin{itemize}
            \item If not, behaves like a \funcname{raise\_notrace} with \funcname{RESTORE\_EXN\_HANDLER\_OCAML}
        \end{itemize}
        }
        \item<4-> Jumps into \funcname{caml\_raise\_exn}
        \item<5-> Calls \funcname{caml\_stash\_backtrace} again, to scan the next part of the stack: from the trap just pop-ed to the next trap
      \end{itemize}
      \vfill
      \mintocaml[firstline=15,lastline=21,highlightlines={18-21}]{../src/backtrace/backtrace.ml}
      \endminipage
    \end{column}
    \begin{column}{0.5\textwidth}
      \raggedleft
      \onslide*<1>{\listCamlReraiseExn{}}
      \onslide*<2>{\listCamlReraiseExn[highlightlines=729]{}}
      \onslide*<3>{
        \mintc[firstline=190,lastline=192,highlightlines={191-192}]{../ocaml/runtime/amd64.S}
        \mintc[firstline=234,lastline=237,highlightlines={235-237}]{../ocaml/runtime/amd64.S}
        \medskip
        \listCamlReraiseExn[highlightlines=731]{}
      }
      \onslide*<4-5>{
        % TODO refactor with \listCamlReraiseExn
        \mintasm[firstline=727,lastline=734,highlightlines=730]{../ocaml/runtime/amd64.S}
        \medskip
      }
      \onslide*<4>{\listCamlRaiseExn[highlightlines=711]{}}
      \onslide*<5>{\listCamlRaiseExn[highlightlines=720]{}}
    \end{column}
  \end{columns}
\end{frame}

\begin{frame}{Backtraces}{Backtrace collection continues}
  \begin{columns}[c]
    \begin{column}{0.55\textwidth}
      \minipage[c][0.75\textheight][s]{\columnwidth}
      \begin{itemize}
        \item<1-> \funcname{caml\_stash\_backtrace} scans the older part of the stack, up to the next trap
        \item<6-> Controls jumps to the nested exception handler in \funcname{maybe\_raise}, which matches only on \typename{ExnB}
        \item<7-> \funcname{caml\_reraise\_exn} is called again, backtrace collection resumes with \funcname{caml\_stash\_backtrace}
      \end{itemize}
      \medskip
      \begin{itemize}
        \item<2-> Constructed backtrace:
        \begin{itemize}
          \item <2-> \funcname{definitely\_raise}
          \item <2-> \funcname{probably\_raise}
          \item <3-> \funcname{probably\_raise} (re-raise)
          \item <4-> \funcname{maybe\_raise}
          \item <5-> \funcname{main}
          \item <8-> \funcname{main} (re-raise)
        \end{itemize}
      \end{itemize}
      \vfill
      \onslide*<1-5>{\mintocaml[firstline=15,lastline=21]{../src/backtrace/backtrace.ml}}
      \onslide*<6>{\mintocaml[firstline=28,lastline=34,highlightlines=31]{../src/backtrace/backtrace.ml}}
      \onslide*<7->{\mintocaml[firstline=28,lastline=34,highlightlines=33]{../src/backtrace/backtrace.ml}}
      \endminipage
    \end{column}
    \begin{column}{0.45\textwidth}
      \raggedleft
      \onslide*<1>{\providecommand\step{6}\input{diagrams/backtrace/backtrace.tex}}%
      \onslide*<2>{\providecommand\step{6}\input{diagrams/backtrace/backtrace.tex}}%
      \onslide*<3>{\providecommand\step{7}\input{diagrams/backtrace/backtrace.tex}}%
      \onslide*<4>{\providecommand\step{8}\input{diagrams/backtrace/backtrace.tex}}%
      \onslide*<5>{\providecommand\step{9}\input{diagrams/backtrace/backtrace.tex}}%
      \onslide*<6>{\providecommand\step{10}\input{diagrams/backtrace/backtrace.tex}}%
      \onslide*<7>{\providecommand\step{11}\input{diagrams/backtrace/backtrace.tex}}%
      \onslide*<8>{\providecommand\step{12}\input{diagrams/backtrace/backtrace.tex}}%
      \onslide*<9>{\providecommand\step{13}\input{diagrams/backtrace/backtrace.tex}}%
    \end{column}
  \end{columns}
\end{frame}

\begin{frame}{Backtraces}{Printing the backtrace}
  \begin{columns}[c]
    \begin{column}{0.45\textwidth}
      \minipage[c][0.66\textheight][s]{\columnwidth}
      \begin{itemize}
        \item<1-> The exception backtrace can now be printed to \funcarg{stdout} with \funcname{Printexc.print_backtrace}
        \item<2-> First the exception backtrace is retrieved into an OCaml array with \funcname{get_raw_backtrace }
        \item<3-> Which is implemented as the \funcname{caml_get_exception_raw_backtrace} C function in the runtime
        \item<4-> And finally printed with \funcname{print_exception_backtrace}
      \end{itemize}
      \vfill
      \mintocaml[firstline=28,lastline=34,highlightlines=33]{../src/backtrace/backtrace.ml}
      \endminipage
    \end{column}
    \begin{column}{0.55\textwidth}
      \onslide*<1,2>{
        \mintocaml[firstline=100,lastline=101]{../ocaml/stdlib/printexc.ml}
      }
      \only<1>{
        \listocaml[firstline=170,lastline=172]{../ocaml/stdlib/printexc.ml}{stdlib/printexc.ml:170}
      }
      \only<2>{
        \listocaml[firstline=170,lastline=172,highlightlines=172]{../ocaml/stdlib/printexc.ml}{stdlib/printexc.ml:170}
      }
      \only<3>{
        \mintc[firstline=154,lastline=156]{../ocaml/runtime/backtrace.c}
        \listc[firstline=171,lastline=192,highlightlines={184-187}]{../ocaml/runtime/backtrace.c}{runtime/backtrace.c:154}
      }
      \only<4>{
        \listocaml[firstline=155,lastline=165]{../ocaml/stdlib/printexc.ml}{stdlib/printexc.ml:55}
      }
    \end{column}
  \end{columns}
\end{frame}


\subsection{The multiple kinds of raise}
\frameSubsection{}{}

\begin{frame}{Backtraces}{When backtraces are not needed}
  \begin{columns}[c]
    \begin{column}{0.45\textwidth}
      \minipage[c][0.66\textheight][s]{\columnwidth}
      \begin{itemize}
        \item<1-> What if the backtrace for an exception is never needed?
        \item<2-> It's possible to raise the exception explicitly with \funcname{raise\_notrace} to make raising as cheap as possible
        \item<3-> An example of use in the \funcname{dynlink} library of the OCaml compiler
      \end{itemize}
      \vfill
      \onslide<3->{
      \listocaml[firstline=205,lastline=209,highlightlines=207]{../ocaml/otherlibs/dynlink/dynlink\_compilerlibs/misc.ml}{otherlibs/dynlink/dynlink\_compilerlibs/misc.ml:205}
      }
      \endminipage
    \end{column}
    \begin{column}{0.55\textwidth}
    \only<4>{
      \mintcmm[highlightlines=3]{../src/notrace/camlNotrace\_\_fun_347.cmm}
      \listcmm{../src/notrace/camlNotrace\_\_all\_somes\_327.cmm}{all\_somes}
    }
    \onslide*<5->{
      \listobjdump[highlightlines={6-9}]{../src/notrace/camlNotrace\_\_fun_347.objdump}{anonymous function from \funcname{all\_somes}}
    }
    \end{column}
  \end{columns}
\end{frame}

\begin{frame}{Backtraces}{Summary table of the multiple kinds of raise}
  \begin{itemize}
    \item When debugging information is enabled and if the backtrace of an exception isn't needed, it's possible to raise with \funcname{raise\_notrace} for performance
    \item When debugging information is \emph{not} enabled, the compiler optimises \funcname{raise} calls to inline assembly (same effect as using \funcname{raise\_notrace})
  \end{itemize}
  \bigskip
  \centering
  \begin{tabular}{| l | c | c | c | c |}
    \hline
    \parbox{10em}{Debug information \\ (\funcarg{-g} flag)}  & \multicolumn{2}{|c|}{Disabled}                                     & \multicolumn{2}{|c|}{Enabled} \\
    \hline
    Raise kind                                               & \funcname{raise} & \funcname{raise\_notrace}                       & \funcname{raise} & \funcname{raise\_notrace} \\
    \hline
    Compilation                                              & \parbox{8em}{inline assembly\\ (optimization)} & inline assembly   & \funcname{caml\_raise\_exn} & inline assembly \\
    \hline
  \end{tabular}
\end{frame}


%
% Takeaway #5
%
\subsection*{Takeaway \#5}
\frameSubsectionTakeaway{}
\againframe<5>{takeaway}
}%
      \onslide*<7>{\providecommand\step{11}%
% Backtraces
%
\subsection{One advantage of raising with debugging information}
\frameSubsection{}{}

\begin{frame}{Backtraces}{Debugging information \& source code}
  \begin{columns}[c]
    \begin{column}{0.5\textwidth}
      \begin{itemize}
        \item Raising an exception is fun and games, but how do we know where the exception originated? $\rightarrow$ With the backtrace associated with the exception!
        \item In order to get a backtrace from the point where an exception was raised, it's necessary to compile with debugging information enabled (\funcarg{-g} flag for the compiler)
        \item Same example as in the `Nested handlers' section
      \end{itemize}
    \end{column}
    \begin{column}{0.5\textwidth}
      \listocaml[firstline=9,lastline=34]{../src/backtrace/backtrace.ml}{backtrace/backtrace.ml}
    \end{column}
  \end{columns}
\end{frame}

\begin{frame}{Backtraces}{CMM and disassembly of \funcname{definitely\_raise}}
  \begin{columns}[c]
    \begin{column}{0.5\textwidth}
      \minipage[c][0.66\textheight][s]{\columnwidth}
      \begin{itemize}
        \item<1-> An effect of compiling with debugging information (\funcarg{-g}): the CMM no longer uses \funcname{raise\_notrace} to raise the exception but \funcname{raise} instead
        \item<2-> Disassembly shows that CMM \funcname{raise} is actually a call to \funcname{caml\_raise\_exn}, a function in assembler provided by the runtime
      \end{itemize}
      \vfill
      \mintocaml[firstline=9,lastline=13,highlightlines=12]{../src/backtrace/backtrace.ml}
      \endminipage
    \end{column}
    \begin{column}{0.5\textwidth}
      \onslide*<1>{
        \mintcmm[highlightlines=5]{../src/backtrace/camlBacktrace\_\_definitely\_raise\_347.cmm}
      }
      \onslide*<2>{
        \mintobjdump[firstline=2,highlightlines=14]{../src/backtrace/camlBacktrace\_\_definitely\_raise\_347.objdump}
      }
    \end{column}
  \end{columns}
\end{frame}

\begin{frame}{Backtraces}{Introducing \funcname{caml\_raise\_exn}}
  \begin{columns}[c]
    \begin{column}{0.5\textwidth}
      \minipage[c][0.66\textheight][s]{\columnwidth}
      \onslide*<1>{
        \begin{itemize}
          \item Raising the exception is performed using \funcname{caml\_raise\_exn} which is
            \begin{itemize}
              \item An assembly function provided by the runtime
              \item Architecture specific
            \end{itemize}
          \item Let's see what it does differently from \funcname{raise\_notrace}\dots
          \item We are about to raise \typename{ExnC} with \funcname{caml\_raise\_exn} from \funcname{definitely\_raise}
        \end{itemize}
      }
      \onslide*<2>{
        \mintobjdump[firstline=2,highlightlines=14]{../src/backtrace/camlBacktrace\_\_definitely\_raise\_347.objdump}
      }
      \vfill
      \mintocaml[firstline=9,lastline=13,highlightlines=12]{../src/backtrace/backtrace.ml}
      \endminipage
    \end{column}
    \begin{column}{0.5\textwidth}
      \centering
      \providecommand\step{1}\input{diagrams/backtrace/backtrace.tex}
    \end{column}
  \end{columns}
\end{frame}

\begin{frame}{Backtraces}{But before that, we need more fields from \typename{caml\_domain\_state}}
  \begin{columns}
    \begin{column}{0.5\textwidth}
      \begin{itemize}
        \item Remember \typename{caml\_domain\_state} the per-domain runtime state?
        \item Some other members are of interest to us now\dots
        \item \localname{backtrace\_active} is used to know if the runtime should collect backtraces
        \item \localname{backtrace\_active} can be set by
          \begin{itemize}
            \item The \funcarg{b} flag in the \funcname{OCAMLRUNPARAM} environment variable
            \item \mintinline{ocaml}{Printexc.record_backtrace : bool -> unit} \footnotemark
          \end{itemize}
      \end{itemize}
    \end{column}
    \begin{column}{0.5\textwidth}
      \begin{listing}
        \mintc[highlightlines={9-11}]{src/ocaml/domain\_state\_backtrace.h}
        \caption{runtime/caml/domain\_state.h}
      \end{listing}
    \end{column}
  \end{columns}
  \footnotetext[1]{https://v2.ocaml.org/api/Printexc.html}
\end{frame}

\newcommand\listCamlRaiseExn[1][]{
  \listasm[firstline=702,lastline=725,#1]{../ocaml/runtime/amd64.S}{runtime/amd64.S:702}}

\begin{frame}{Backtraces}{Walk through \funcname{caml\_raise\_exn}}
  \begin{columns}[T]
    \begin{column}{0.5\textwidth}
      \begin{itemize}
        \item<1-> First checks whether exceptions backtrace should be recorded
        \item<1-> By checking the \funcarg{domain\_state->backtrace\_active} value
        \item<2-> If not, acts as \funcname{raise\_notrace} with \funcname{RESTORE\_EXN\_HANDLER\_OCAML}
          \begin{itemize}
            \item Set \regname{RSP} to \localname{domain\_state->exn\_handler} value
            \item Restore \localname{domain\_state->exn\_handler} from trap
            \item Jump with \funcname{ret} (same effect as using \funcname{pop} and \funcname{jmp})
          \end{itemize}
        \item<3-> Otherwise, switches to the C stack to be able to execute C code
        \item<4-> Calls \funcname{caml\_stash\_backtrace}, a C function provided by the runtime\ignorespaces
          \onslide<6->{, with
            \begin{itemize}
              \item \funcarg{exn}: exception value
              \item \funcarg{pc}: code address of the raising point
              \item \funcarg{sp}: stack address of the raising function
              \item \funcarg{trapsp}: stack address of the next handler
            \end{itemize}
          }
      \end{itemize}
      \bigskip
      \mintocaml[firstline=9,lastline=13,highlightlines=12]{../src/backtrace/backtrace.ml}
    \end{column}
    \begin{column}{0.5\textwidth}
      \raggedleft
      \onslide*<1,3-4>{
        \mintc[firstline=190,lastline=192]{../ocaml/runtime/amd64.S}
        \mintc[firstline=234,lastline=237]{../ocaml/runtime/amd64.S}
      }
      \onslide*<2>{
        \mintc[firstline=190,lastline=192,highlightlines={191-192}]{../ocaml/runtime/amd64.S}
        \mintc[firstline=234,lastline=237,highlightlines={235-237}]{../ocaml/runtime/amd64.S}
      }
      \onslide*<1>{\listCamlRaiseExn[highlightlines=705]{}}%
      \onslide*<2>{\listCamlRaiseExn[highlightlines={707-708}]{}}%
      \onslide*<3>{\listCamlRaiseExn[highlightlines={712-714}]{}}%
      \onslide*<4>{\listCamlRaiseExn[highlightlines={715-720}]{}}%
      \onslide*<5>{\providecommand\step{1}\input{diagrams/backtrace/backtrace.tex}}%
      \onslide*<6>{\providecommand\step{2}\input{diagrams/backtrace/backtrace.tex}}%
    \end{column}
  \end{columns}
\end{frame}

\newcommand\listStashBacktrace[1][]{
  \listc[firstline=91,lastline=120,#1]{../ocaml/runtime/backtrace\_nat.c}{runtime/backtrace\_nat.c:91}}

\begin{frame}{Backtraces}{Introducing \funcname{caml\_stash\_backtrace}}
  \begin{columns}[c]
    \begin{column}{0.5\textwidth}
      \minipage[c][0.66\textheight][s]{\columnwidth}
      \begin{itemize}
        \item<1-> A C function provided by the runtime (specific to native code)
        \item<2-> It scans the stack, between the stack pointer at the raising point up to the next trap, in order to collect the names of the detected function frames
          \onslide*<2>{
            \begin{itemize}
              \item And more information, like line number, column number...
            \end{itemize}
          }
        \item<3-> It uses the same technique and information as the Garbage Collector (GC) uses to scan the values live on the stack
          \onslide*<4>{
            \begin{itemize}
              \item \typename{frame\_descr} are at the core of stack unwinding
            \end{itemize}
          }
      \end{itemize}
      \vfill
      \mintocaml[firstline=9,lastline=13,highlightlines=12]{../src/backtrace/backtrace.ml}
      \endminipage
    \end{column}
    \begin{column}{0.5\textwidth}
      \onslide*<1>{\listStashBacktrace[highlightlines=91]{}}
      \onslide*<2>{\listStashBacktrace[highlightlines={91,117-118}]{}}
      \onslide*<3->{\listStashBacktrace[highlightlines={94,106,109-110}]{}}
    \end{column}
  \end{columns}
\end{frame}

\newcommand\listFrameDescr[1][]{
  \listocaml[firstline=111,lastline=124,#1]{../ocaml/asmcomp/emitaux.ml}{asmcomp/emitaux.ml:111}}
\newcommand\listDebugInfo[1][]{
  \listocaml[firstline=38,lastline=49,#1]{../ocaml/lambda/debuginfo.mli}{lambda/debuginfo.mli:38}}

\begin{frame}{Backtraces}{\typename{frame\_descr} and \typename{frame\_debuginfo} in a nutshell}
  \begin{columns}[c]
    \begin{column}{0.5\textwidth}
      \begin{itemize}
        \item<1-> For every return address in an OCaml program, there is a associated \typename{frame\_descr}
        \item<2-> A \typename{frame\_descr} specifies
        \begin{itemize}
          \item<2-> The size of the function stack frame
          \item<3-> Where live values are on the stack (so that the GC can mark them as live and prevent from being swept)
        \end{itemize}
        \item<4-> When compiled with debug information, \typename{frame\_descr} will be linked to a \typename{frame\_debuginfo}
        \begin{itemize}
          \item<5-> Contains information about the position in the source code (file name, line and column number)
        \end{itemize}
      \end{itemize}
    \end{column}
    \begin{column}{0.5\textwidth}
        \onslide*<1>{\listFrameDescr[highlightlines={118-122}]{}}
        \onslide*<2>{\listFrameDescr[highlightlines={120}]{}}
        \onslide*<3>{\listFrameDescr[highlightlines={121}]{}}
        \onslide*<4->{\listFrameDescr[highlightlines={122}]{}}
        \medskip
        \onslide*<1-3>{\listDebugInfo{}}
        \onslide*<4>{\listDebugInfo[highlightlines={38,49}]{}}
        \onslide*<5>{\listDebugInfo[highlightlines={39-46}]{}}
    \end{column}
  \end{columns}
\end{frame}

\begin{frame}{Backtraces}{Scanning the stack with \typename{frame\_descr}}
  \begin{columns}[c]
    \begin{column}{0.5\textwidth}
      \begin{itemize}
        \item Given the return address of the code that raises the exception by calling \funcname{caml\_raise\_exn} (\funcarg{pc} argument of \funcname{caml\_stash\_backtrace})
        \item And the position of the stack pointer (\funcarg{sp} argument of \funcname{caml\_stash\_backtrace})
        \item The frame size given by \typename{frame\_descr}.\funcarg{frame\_size}, allows to find the end of the previous frame
        \item And the return address of the caller of this function
        \bigskip
        \onslide<3->{
        \item Note that traps (here the reddish \funcname{ExnA}, \funcname{ExnB} and \funcname{ExnC} blocks) belong to the frame that installed them, and so the frame size changes when traps are removed
        }
      \end{itemize}
    \end{column}
    \begin{column}{0.5\textwidth}
      \raggedleft
      \onslide*<1>{\providecommand\step{2}\input{diagrams/backtrace/backtrace.tex}}%
      \onslide*<2>{\providecommand\step{1}\input{diagrams/backtrace/scanstack.tex}}%
      \onslide*<3>{\providecommand\step{2}\input{diagrams/backtrace/scanstack.tex}}%
      \onslide*<4>{\providecommand\step{3}\input{diagrams/backtrace/scanstack.tex}}%
      \onslide*<5>{\providecommand\step{4}\input{diagrams/backtrace/scanstack.tex}}%
      \onslide*<6>{\providecommand\step{5}\input{diagrams/backtrace/scanstack.tex}}%
    \end{column}
  \end{columns}
\end{frame}

% Duplicate of previous-previous slide
\begin{frame}{Backtraces}{Continuing on \funcname{caml\_stash\_backtrace}}
  \begin{columns}[c]
    \begin{column}{0.5\textwidth}
      \minipage[c][0.75\textheight][s]{\columnwidth}
      \begin{itemize}
          \color{gray}
        \item A C function provided by the runtime (specific to native code)
        \item It scans the stack, between the stack pointer at the raising point up to the next trap, in order to collect the names of the detected function frames
        \item It uses the same technique and information as the Garbage Collector (GC) uses to scan the values live on the stack
      \end{itemize}
      \begin{itemize}
        \item For each function frame detected, store a pointer to the \typename{frame\_descr} in the \localname{domain\_state->backtrace\_buffer} array, effectively constructing the backtrace
      \end{itemize}
      \onslide<2->{
        \begin{itemize}
            \smallskip
          \item Constructed backtrace:
            \funcname{definitely\_raise},
            \onslide<3->{\funcname{probably\_raise}}
        \end{itemize}
      }
      \vfill
      \mintocaml[firstline=9,lastline=13,highlightlines=12]{../src/backtrace/backtrace.ml}
      \endminipage
    \end{column}
    \begin{column}{0.5\textwidth}
      \raggedleft
      \onslide*<1>{\listStashBacktrace[highlightlines={114-115}]{}}
      \onslide*<2>{\providecommand\step{3}\input{diagrams/backtrace/backtrace.tex}}%
      \onslide*<3>{\providecommand\step{4}\input{diagrams/backtrace/backtrace.tex}}%
    \end{column}
  \end{columns}
\end{frame}

\begin{frame}{Backtraces}{Back to \funcname{caml\_raise\_exn}}
  \begin{columns}[c]
    \begin{column}{0.5\textwidth}
      \minipage[c][0.66\textheight][s]{\columnwidth}
      \begin{itemize}\color{gray}
        \item Checks wheteher exceptions backtrace should be recorded, by checking the \funcarg{domain\_state->backtrace\_active} value
        \item Switches to the C stack to be able to execute C code
        \item Calls \funcname{caml\_stash\_backtrace}, to collect the backtrace up to the next trap
      \end{itemize}
      \onslide*<2->{
        \begin{itemize}
          \item<2-> Executes the exception handler in \funcname{probably\_raise} with \funcname{RESTORE\_EXN\_HANDLER\_OCAML}
            \begin{itemize}
              \item Set \regname{RSP} to \localname{domain\_state->exn\_handler} value
              \item Restore \localname{domain\_state->exn\_handler} from trap
              \item Jump to the handler code with \funcname{ret}
            \end{itemize}
        \end{itemize}
      }
      \vfill
      \onslide*<1>{\mintocaml[firstline=9,lastline=13,highlightlines=12]{../src/backtrace/backtrace.ml}}
      \onslide*<2->{\mintocaml[firstline=15,lastline=21,highlightlines={19-21}]{../src/backtrace/backtrace.ml}}
      \endminipage
    \end{column}
    \begin{column}{0.5\textwidth}
      \centering
      \onslide*<1>{
        \mintc[firstline=190,lastline=192]{../ocaml/runtime/amd64.S}
        \mintc[firstline=234,lastline=237]{../ocaml/runtime/amd64.S}
      }
      \onslide*<2>{
        \mintc[firstline=190,lastline=192,highlightlines={191-192}]{../ocaml/runtime/amd64.S}
        \mintc[firstline=234,lastline=237,highlightlines={235-237}]{../ocaml/runtime/amd64.S}
      }
      \onslide*<1>{\listCamlRaiseExn[highlightlines=720]{}}
      \onslide*<2>{\listCamlRaiseExn[highlightlines=722]{}}
      \onslide*<3->{\providecommand\step{5}\input{diagrams/backtrace/backtrace.tex}}
    \end{column}
  \end{columns}
\end{frame}

\begin{frame}{Backtraces}{\funcname{caml\_probably\_raise}'s handler doesn't catch \typename{ExnC}}
  \begin{columns}[c]
    \begin{column}{0.45\textwidth}
      \minipage[c][0.75\textheight][s]{\columnwidth}
      \begin{itemize}
        \item<1-> \funcname{match \dots with}\footnotemark pattern matching can also match exceptions
        \item<1-> The CMM of \funcname{probably\_raise} shows that this \funcname{match \dots with} is implemented with a pattern matching and a \funcname{try \dots with}
        \item<1-> But this one only matches on \typename{ExnA} exception
        \item<2-> Since the pattern matching fails to catch the exception, \funcname{reraise} is called with our current \typename{ExnC} exception
        \item<3-> Disassembly shows that \funcname{reraise} is actually a call to \funcname{caml\_reraise\_exn}, which is also an assembly function provided by the runtime
      \end{itemize}
      \vfill
      \mintocaml[firstline=15,lastline=21,highlightlines={18-21}]{../src/backtrace/backtrace.ml}
      \endminipage
    \end{column}
    \begin{column}{0.55\textwidth}
      \centering
      \onslide*<1>{\mintcmm[highlightlines={10-18}]{../src/backtrace/camlBacktrace\_\_probably\_raise\_351.cmm}}
      \onslide*<2>{\listcmm[highlightlines=18]{../src/backtrace/camlBacktrace\_\_probably\_raise_351.cmm}{CMM of probably\_raise}}
      \onslide*<3>{\mintobjdump[firstline=5,lastline=35,highlightlines=31]{../src/backtrace/camlBacktrace\_\_probably\_raise_351.objdump}}
    \end{column}
  \end{columns}
  \only<1>{\footnotetext[1]{https://blog.janestreet.com/pattern-matching-and-exception-handling-unite/}}
\end{frame}

\newcommand\listCamlReraiseExn[1][]{
  \listasm[firstline=727,lastline=734,#1]{../ocaml/runtime/amd64.S}{runtime/amd64.S:727}}

\begin{frame}{Backtraces}{Introducing \funcname{caml\_reraise\_exn}}
  \begin{columns}[c]
    \begin{column}{0.5\textwidth}
      \minipage[c][0.66\textheight][s]{\columnwidth}
      \begin{itemize}
        \item<1-> The exception have to be reraised to the next handler
        \item<2-> First, checks if the backtrace should be collected with \funcarg{domain\_state->backtrace\_active}
        \only<3>{
        \begin{itemize}
            \item If not, behaves like a \funcname{raise\_notrace} with \funcname{RESTORE\_EXN\_HANDLER\_OCAML}
        \end{itemize}
        }
        \item<4-> Jumps into \funcname{caml\_raise\_exn}
        \item<5-> Calls \funcname{caml\_stash\_backtrace} again, to scan the next part of the stack: from the trap just pop-ed to the next trap
      \end{itemize}
      \vfill
      \mintocaml[firstline=15,lastline=21,highlightlines={18-21}]{../src/backtrace/backtrace.ml}
      \endminipage
    \end{column}
    \begin{column}{0.5\textwidth}
      \raggedleft
      \onslide*<1>{\listCamlReraiseExn{}}
      \onslide*<2>{\listCamlReraiseExn[highlightlines=729]{}}
      \onslide*<3>{
        \mintc[firstline=190,lastline=192,highlightlines={191-192}]{../ocaml/runtime/amd64.S}
        \mintc[firstline=234,lastline=237,highlightlines={235-237}]{../ocaml/runtime/amd64.S}
        \medskip
        \listCamlReraiseExn[highlightlines=731]{}
      }
      \onslide*<4-5>{
        % TODO refactor with \listCamlReraiseExn
        \mintasm[firstline=727,lastline=734,highlightlines=730]{../ocaml/runtime/amd64.S}
        \medskip
      }
      \onslide*<4>{\listCamlRaiseExn[highlightlines=711]{}}
      \onslide*<5>{\listCamlRaiseExn[highlightlines=720]{}}
    \end{column}
  \end{columns}
\end{frame}

\begin{frame}{Backtraces}{Backtrace collection continues}
  \begin{columns}[c]
    \begin{column}{0.55\textwidth}
      \minipage[c][0.75\textheight][s]{\columnwidth}
      \begin{itemize}
        \item<1-> \funcname{caml\_stash\_backtrace} scans the older part of the stack, up to the next trap
        \item<6-> Controls jumps to the nested exception handler in \funcname{maybe\_raise}, which matches only on \typename{ExnB}
        \item<7-> \funcname{caml\_reraise\_exn} is called again, backtrace collection resumes with \funcname{caml\_stash\_backtrace}
      \end{itemize}
      \medskip
      \begin{itemize}
        \item<2-> Constructed backtrace:
        \begin{itemize}
          \item <2-> \funcname{definitely\_raise}
          \item <2-> \funcname{probably\_raise}
          \item <3-> \funcname{probably\_raise} (re-raise)
          \item <4-> \funcname{maybe\_raise}
          \item <5-> \funcname{main}
          \item <8-> \funcname{main} (re-raise)
        \end{itemize}
      \end{itemize}
      \vfill
      \onslide*<1-5>{\mintocaml[firstline=15,lastline=21]{../src/backtrace/backtrace.ml}}
      \onslide*<6>{\mintocaml[firstline=28,lastline=34,highlightlines=31]{../src/backtrace/backtrace.ml}}
      \onslide*<7->{\mintocaml[firstline=28,lastline=34,highlightlines=33]{../src/backtrace/backtrace.ml}}
      \endminipage
    \end{column}
    \begin{column}{0.45\textwidth}
      \raggedleft
      \onslide*<1>{\providecommand\step{6}\input{diagrams/backtrace/backtrace.tex}}%
      \onslide*<2>{\providecommand\step{6}\input{diagrams/backtrace/backtrace.tex}}%
      \onslide*<3>{\providecommand\step{7}\input{diagrams/backtrace/backtrace.tex}}%
      \onslide*<4>{\providecommand\step{8}\input{diagrams/backtrace/backtrace.tex}}%
      \onslide*<5>{\providecommand\step{9}\input{diagrams/backtrace/backtrace.tex}}%
      \onslide*<6>{\providecommand\step{10}\input{diagrams/backtrace/backtrace.tex}}%
      \onslide*<7>{\providecommand\step{11}\input{diagrams/backtrace/backtrace.tex}}%
      \onslide*<8>{\providecommand\step{12}\input{diagrams/backtrace/backtrace.tex}}%
      \onslide*<9>{\providecommand\step{13}\input{diagrams/backtrace/backtrace.tex}}%
    \end{column}
  \end{columns}
\end{frame}

\begin{frame}{Backtraces}{Printing the backtrace}
  \begin{columns}[c]
    \begin{column}{0.45\textwidth}
      \minipage[c][0.66\textheight][s]{\columnwidth}
      \begin{itemize}
        \item<1-> The exception backtrace can now be printed to \funcarg{stdout} with \funcname{Printexc.print_backtrace}
        \item<2-> First the exception backtrace is retrieved into an OCaml array with \funcname{get_raw_backtrace }
        \item<3-> Which is implemented as the \funcname{caml_get_exception_raw_backtrace} C function in the runtime
        \item<4-> And finally printed with \funcname{print_exception_backtrace}
      \end{itemize}
      \vfill
      \mintocaml[firstline=28,lastline=34,highlightlines=33]{../src/backtrace/backtrace.ml}
      \endminipage
    \end{column}
    \begin{column}{0.55\textwidth}
      \onslide*<1,2>{
        \mintocaml[firstline=100,lastline=101]{../ocaml/stdlib/printexc.ml}
      }
      \only<1>{
        \listocaml[firstline=170,lastline=172]{../ocaml/stdlib/printexc.ml}{stdlib/printexc.ml:170}
      }
      \only<2>{
        \listocaml[firstline=170,lastline=172,highlightlines=172]{../ocaml/stdlib/printexc.ml}{stdlib/printexc.ml:170}
      }
      \only<3>{
        \mintc[firstline=154,lastline=156]{../ocaml/runtime/backtrace.c}
        \listc[firstline=171,lastline=192,highlightlines={184-187}]{../ocaml/runtime/backtrace.c}{runtime/backtrace.c:154}
      }
      \only<4>{
        \listocaml[firstline=155,lastline=165]{../ocaml/stdlib/printexc.ml}{stdlib/printexc.ml:55}
      }
    \end{column}
  \end{columns}
\end{frame}


\subsection{The multiple kinds of raise}
\frameSubsection{}{}

\begin{frame}{Backtraces}{When backtraces are not needed}
  \begin{columns}[c]
    \begin{column}{0.45\textwidth}
      \minipage[c][0.66\textheight][s]{\columnwidth}
      \begin{itemize}
        \item<1-> What if the backtrace for an exception is never needed?
        \item<2-> It's possible to raise the exception explicitly with \funcname{raise\_notrace} to make raising as cheap as possible
        \item<3-> An example of use in the \funcname{dynlink} library of the OCaml compiler
      \end{itemize}
      \vfill
      \onslide<3->{
      \listocaml[firstline=205,lastline=209,highlightlines=207]{../ocaml/otherlibs/dynlink/dynlink\_compilerlibs/misc.ml}{otherlibs/dynlink/dynlink\_compilerlibs/misc.ml:205}
      }
      \endminipage
    \end{column}
    \begin{column}{0.55\textwidth}
    \only<4>{
      \mintcmm[highlightlines=3]{../src/notrace/camlNotrace\_\_fun_347.cmm}
      \listcmm{../src/notrace/camlNotrace\_\_all\_somes\_327.cmm}{all\_somes}
    }
    \onslide*<5->{
      \listobjdump[highlightlines={6-9}]{../src/notrace/camlNotrace\_\_fun_347.objdump}{anonymous function from \funcname{all\_somes}}
    }
    \end{column}
  \end{columns}
\end{frame}

\begin{frame}{Backtraces}{Summary table of the multiple kinds of raise}
  \begin{itemize}
    \item When debugging information is enabled and if the backtrace of an exception isn't needed, it's possible to raise with \funcname{raise\_notrace} for performance
    \item When debugging information is \emph{not} enabled, the compiler optimises \funcname{raise} calls to inline assembly (same effect as using \funcname{raise\_notrace})
  \end{itemize}
  \bigskip
  \centering
  \begin{tabular}{| l | c | c | c | c |}
    \hline
    \parbox{10em}{Debug information \\ (\funcarg{-g} flag)}  & \multicolumn{2}{|c|}{Disabled}                                     & \multicolumn{2}{|c|}{Enabled} \\
    \hline
    Raise kind                                               & \funcname{raise} & \funcname{raise\_notrace}                       & \funcname{raise} & \funcname{raise\_notrace} \\
    \hline
    Compilation                                              & \parbox{8em}{inline assembly\\ (optimization)} & inline assembly   & \funcname{caml\_raise\_exn} & inline assembly \\
    \hline
  \end{tabular}
\end{frame}


%
% Takeaway #5
%
\subsection*{Takeaway \#5}
\frameSubsectionTakeaway{}
\againframe<5>{takeaway}
}%
      \onslide*<8>{\providecommand\step{12}%
% Backtraces
%
\subsection{One advantage of raising with debugging information}
\frameSubsection{}{}

\begin{frame}{Backtraces}{Debugging information \& source code}
  \begin{columns}[c]
    \begin{column}{0.5\textwidth}
      \begin{itemize}
        \item Raising an exception is fun and games, but how do we know where the exception originated? $\rightarrow$ With the backtrace associated with the exception!
        \item In order to get a backtrace from the point where an exception was raised, it's necessary to compile with debugging information enabled (\funcarg{-g} flag for the compiler)
        \item Same example as in the `Nested handlers' section
      \end{itemize}
    \end{column}
    \begin{column}{0.5\textwidth}
      \listocaml[firstline=9,lastline=34]{../src/backtrace/backtrace.ml}{backtrace/backtrace.ml}
    \end{column}
  \end{columns}
\end{frame}

\begin{frame}{Backtraces}{CMM and disassembly of \funcname{definitely\_raise}}
  \begin{columns}[c]
    \begin{column}{0.5\textwidth}
      \minipage[c][0.66\textheight][s]{\columnwidth}
      \begin{itemize}
        \item<1-> An effect of compiling with debugging information (\funcarg{-g}): the CMM no longer uses \funcname{raise\_notrace} to raise the exception but \funcname{raise} instead
        \item<2-> Disassembly shows that CMM \funcname{raise} is actually a call to \funcname{caml\_raise\_exn}, a function in assembler provided by the runtime
      \end{itemize}
      \vfill
      \mintocaml[firstline=9,lastline=13,highlightlines=12]{../src/backtrace/backtrace.ml}
      \endminipage
    \end{column}
    \begin{column}{0.5\textwidth}
      \onslide*<1>{
        \mintcmm[highlightlines=5]{../src/backtrace/camlBacktrace\_\_definitely\_raise\_347.cmm}
      }
      \onslide*<2>{
        \mintobjdump[firstline=2,highlightlines=14]{../src/backtrace/camlBacktrace\_\_definitely\_raise\_347.objdump}
      }
    \end{column}
  \end{columns}
\end{frame}

\begin{frame}{Backtraces}{Introducing \funcname{caml\_raise\_exn}}
  \begin{columns}[c]
    \begin{column}{0.5\textwidth}
      \minipage[c][0.66\textheight][s]{\columnwidth}
      \onslide*<1>{
        \begin{itemize}
          \item Raising the exception is performed using \funcname{caml\_raise\_exn} which is
            \begin{itemize}
              \item An assembly function provided by the runtime
              \item Architecture specific
            \end{itemize}
          \item Let's see what it does differently from \funcname{raise\_notrace}\dots
          \item We are about to raise \typename{ExnC} with \funcname{caml\_raise\_exn} from \funcname{definitely\_raise}
        \end{itemize}
      }
      \onslide*<2>{
        \mintobjdump[firstline=2,highlightlines=14]{../src/backtrace/camlBacktrace\_\_definitely\_raise\_347.objdump}
      }
      \vfill
      \mintocaml[firstline=9,lastline=13,highlightlines=12]{../src/backtrace/backtrace.ml}
      \endminipage
    \end{column}
    \begin{column}{0.5\textwidth}
      \centering
      \providecommand\step{1}\input{diagrams/backtrace/backtrace.tex}
    \end{column}
  \end{columns}
\end{frame}

\begin{frame}{Backtraces}{But before that, we need more fields from \typename{caml\_domain\_state}}
  \begin{columns}
    \begin{column}{0.5\textwidth}
      \begin{itemize}
        \item Remember \typename{caml\_domain\_state} the per-domain runtime state?
        \item Some other members are of interest to us now\dots
        \item \localname{backtrace\_active} is used to know if the runtime should collect backtraces
        \item \localname{backtrace\_active} can be set by
          \begin{itemize}
            \item The \funcarg{b} flag in the \funcname{OCAMLRUNPARAM} environment variable
            \item \mintinline{ocaml}{Printexc.record_backtrace : bool -> unit} \footnotemark
          \end{itemize}
      \end{itemize}
    \end{column}
    \begin{column}{0.5\textwidth}
      \begin{listing}
        \mintc[highlightlines={9-11}]{src/ocaml/domain\_state\_backtrace.h}
        \caption{runtime/caml/domain\_state.h}
      \end{listing}
    \end{column}
  \end{columns}
  \footnotetext[1]{https://v2.ocaml.org/api/Printexc.html}
\end{frame}

\newcommand\listCamlRaiseExn[1][]{
  \listasm[firstline=702,lastline=725,#1]{../ocaml/runtime/amd64.S}{runtime/amd64.S:702}}

\begin{frame}{Backtraces}{Walk through \funcname{caml\_raise\_exn}}
  \begin{columns}[T]
    \begin{column}{0.5\textwidth}
      \begin{itemize}
        \item<1-> First checks whether exceptions backtrace should be recorded
        \item<1-> By checking the \funcarg{domain\_state->backtrace\_active} value
        \item<2-> If not, acts as \funcname{raise\_notrace} with \funcname{RESTORE\_EXN\_HANDLER\_OCAML}
          \begin{itemize}
            \item Set \regname{RSP} to \localname{domain\_state->exn\_handler} value
            \item Restore \localname{domain\_state->exn\_handler} from trap
            \item Jump with \funcname{ret} (same effect as using \funcname{pop} and \funcname{jmp})
          \end{itemize}
        \item<3-> Otherwise, switches to the C stack to be able to execute C code
        \item<4-> Calls \funcname{caml\_stash\_backtrace}, a C function provided by the runtime\ignorespaces
          \onslide<6->{, with
            \begin{itemize}
              \item \funcarg{exn}: exception value
              \item \funcarg{pc}: code address of the raising point
              \item \funcarg{sp}: stack address of the raising function
              \item \funcarg{trapsp}: stack address of the next handler
            \end{itemize}
          }
      \end{itemize}
      \bigskip
      \mintocaml[firstline=9,lastline=13,highlightlines=12]{../src/backtrace/backtrace.ml}
    \end{column}
    \begin{column}{0.5\textwidth}
      \raggedleft
      \onslide*<1,3-4>{
        \mintc[firstline=190,lastline=192]{../ocaml/runtime/amd64.S}
        \mintc[firstline=234,lastline=237]{../ocaml/runtime/amd64.S}
      }
      \onslide*<2>{
        \mintc[firstline=190,lastline=192,highlightlines={191-192}]{../ocaml/runtime/amd64.S}
        \mintc[firstline=234,lastline=237,highlightlines={235-237}]{../ocaml/runtime/amd64.S}
      }
      \onslide*<1>{\listCamlRaiseExn[highlightlines=705]{}}%
      \onslide*<2>{\listCamlRaiseExn[highlightlines={707-708}]{}}%
      \onslide*<3>{\listCamlRaiseExn[highlightlines={712-714}]{}}%
      \onslide*<4>{\listCamlRaiseExn[highlightlines={715-720}]{}}%
      \onslide*<5>{\providecommand\step{1}\input{diagrams/backtrace/backtrace.tex}}%
      \onslide*<6>{\providecommand\step{2}\input{diagrams/backtrace/backtrace.tex}}%
    \end{column}
  \end{columns}
\end{frame}

\newcommand\listStashBacktrace[1][]{
  \listc[firstline=91,lastline=120,#1]{../ocaml/runtime/backtrace\_nat.c}{runtime/backtrace\_nat.c:91}}

\begin{frame}{Backtraces}{Introducing \funcname{caml\_stash\_backtrace}}
  \begin{columns}[c]
    \begin{column}{0.5\textwidth}
      \minipage[c][0.66\textheight][s]{\columnwidth}
      \begin{itemize}
        \item<1-> A C function provided by the runtime (specific to native code)
        \item<2-> It scans the stack, between the stack pointer at the raising point up to the next trap, in order to collect the names of the detected function frames
          \onslide*<2>{
            \begin{itemize}
              \item And more information, like line number, column number...
            \end{itemize}
          }
        \item<3-> It uses the same technique and information as the Garbage Collector (GC) uses to scan the values live on the stack
          \onslide*<4>{
            \begin{itemize}
              \item \typename{frame\_descr} are at the core of stack unwinding
            \end{itemize}
          }
      \end{itemize}
      \vfill
      \mintocaml[firstline=9,lastline=13,highlightlines=12]{../src/backtrace/backtrace.ml}
      \endminipage
    \end{column}
    \begin{column}{0.5\textwidth}
      \onslide*<1>{\listStashBacktrace[highlightlines=91]{}}
      \onslide*<2>{\listStashBacktrace[highlightlines={91,117-118}]{}}
      \onslide*<3->{\listStashBacktrace[highlightlines={94,106,109-110}]{}}
    \end{column}
  \end{columns}
\end{frame}

\newcommand\listFrameDescr[1][]{
  \listocaml[firstline=111,lastline=124,#1]{../ocaml/asmcomp/emitaux.ml}{asmcomp/emitaux.ml:111}}
\newcommand\listDebugInfo[1][]{
  \listocaml[firstline=38,lastline=49,#1]{../ocaml/lambda/debuginfo.mli}{lambda/debuginfo.mli:38}}

\begin{frame}{Backtraces}{\typename{frame\_descr} and \typename{frame\_debuginfo} in a nutshell}
  \begin{columns}[c]
    \begin{column}{0.5\textwidth}
      \begin{itemize}
        \item<1-> For every return address in an OCaml program, there is a associated \typename{frame\_descr}
        \item<2-> A \typename{frame\_descr} specifies
        \begin{itemize}
          \item<2-> The size of the function stack frame
          \item<3-> Where live values are on the stack (so that the GC can mark them as live and prevent from being swept)
        \end{itemize}
        \item<4-> When compiled with debug information, \typename{frame\_descr} will be linked to a \typename{frame\_debuginfo}
        \begin{itemize}
          \item<5-> Contains information about the position in the source code (file name, line and column number)
        \end{itemize}
      \end{itemize}
    \end{column}
    \begin{column}{0.5\textwidth}
        \onslide*<1>{\listFrameDescr[highlightlines={118-122}]{}}
        \onslide*<2>{\listFrameDescr[highlightlines={120}]{}}
        \onslide*<3>{\listFrameDescr[highlightlines={121}]{}}
        \onslide*<4->{\listFrameDescr[highlightlines={122}]{}}
        \medskip
        \onslide*<1-3>{\listDebugInfo{}}
        \onslide*<4>{\listDebugInfo[highlightlines={38,49}]{}}
        \onslide*<5>{\listDebugInfo[highlightlines={39-46}]{}}
    \end{column}
  \end{columns}
\end{frame}

\begin{frame}{Backtraces}{Scanning the stack with \typename{frame\_descr}}
  \begin{columns}[c]
    \begin{column}{0.5\textwidth}
      \begin{itemize}
        \item Given the return address of the code that raises the exception by calling \funcname{caml\_raise\_exn} (\funcarg{pc} argument of \funcname{caml\_stash\_backtrace})
        \item And the position of the stack pointer (\funcarg{sp} argument of \funcname{caml\_stash\_backtrace})
        \item The frame size given by \typename{frame\_descr}.\funcarg{frame\_size}, allows to find the end of the previous frame
        \item And the return address of the caller of this function
        \bigskip
        \onslide<3->{
        \item Note that traps (here the reddish \funcname{ExnA}, \funcname{ExnB} and \funcname{ExnC} blocks) belong to the frame that installed them, and so the frame size changes when traps are removed
        }
      \end{itemize}
    \end{column}
    \begin{column}{0.5\textwidth}
      \raggedleft
      \onslide*<1>{\providecommand\step{2}\input{diagrams/backtrace/backtrace.tex}}%
      \onslide*<2>{\providecommand\step{1}\input{diagrams/backtrace/scanstack.tex}}%
      \onslide*<3>{\providecommand\step{2}\input{diagrams/backtrace/scanstack.tex}}%
      \onslide*<4>{\providecommand\step{3}\input{diagrams/backtrace/scanstack.tex}}%
      \onslide*<5>{\providecommand\step{4}\input{diagrams/backtrace/scanstack.tex}}%
      \onslide*<6>{\providecommand\step{5}\input{diagrams/backtrace/scanstack.tex}}%
    \end{column}
  \end{columns}
\end{frame}

% Duplicate of previous-previous slide
\begin{frame}{Backtraces}{Continuing on \funcname{caml\_stash\_backtrace}}
  \begin{columns}[c]
    \begin{column}{0.5\textwidth}
      \minipage[c][0.75\textheight][s]{\columnwidth}
      \begin{itemize}
          \color{gray}
        \item A C function provided by the runtime (specific to native code)
        \item It scans the stack, between the stack pointer at the raising point up to the next trap, in order to collect the names of the detected function frames
        \item It uses the same technique and information as the Garbage Collector (GC) uses to scan the values live on the stack
      \end{itemize}
      \begin{itemize}
        \item For each function frame detected, store a pointer to the \typename{frame\_descr} in the \localname{domain\_state->backtrace\_buffer} array, effectively constructing the backtrace
      \end{itemize}
      \onslide<2->{
        \begin{itemize}
            \smallskip
          \item Constructed backtrace:
            \funcname{definitely\_raise},
            \onslide<3->{\funcname{probably\_raise}}
        \end{itemize}
      }
      \vfill
      \mintocaml[firstline=9,lastline=13,highlightlines=12]{../src/backtrace/backtrace.ml}
      \endminipage
    \end{column}
    \begin{column}{0.5\textwidth}
      \raggedleft
      \onslide*<1>{\listStashBacktrace[highlightlines={114-115}]{}}
      \onslide*<2>{\providecommand\step{3}\input{diagrams/backtrace/backtrace.tex}}%
      \onslide*<3>{\providecommand\step{4}\input{diagrams/backtrace/backtrace.tex}}%
    \end{column}
  \end{columns}
\end{frame}

\begin{frame}{Backtraces}{Back to \funcname{caml\_raise\_exn}}
  \begin{columns}[c]
    \begin{column}{0.5\textwidth}
      \minipage[c][0.66\textheight][s]{\columnwidth}
      \begin{itemize}\color{gray}
        \item Checks wheteher exceptions backtrace should be recorded, by checking the \funcarg{domain\_state->backtrace\_active} value
        \item Switches to the C stack to be able to execute C code
        \item Calls \funcname{caml\_stash\_backtrace}, to collect the backtrace up to the next trap
      \end{itemize}
      \onslide*<2->{
        \begin{itemize}
          \item<2-> Executes the exception handler in \funcname{probably\_raise} with \funcname{RESTORE\_EXN\_HANDLER\_OCAML}
            \begin{itemize}
              \item Set \regname{RSP} to \localname{domain\_state->exn\_handler} value
              \item Restore \localname{domain\_state->exn\_handler} from trap
              \item Jump to the handler code with \funcname{ret}
            \end{itemize}
        \end{itemize}
      }
      \vfill
      \onslide*<1>{\mintocaml[firstline=9,lastline=13,highlightlines=12]{../src/backtrace/backtrace.ml}}
      \onslide*<2->{\mintocaml[firstline=15,lastline=21,highlightlines={19-21}]{../src/backtrace/backtrace.ml}}
      \endminipage
    \end{column}
    \begin{column}{0.5\textwidth}
      \centering
      \onslide*<1>{
        \mintc[firstline=190,lastline=192]{../ocaml/runtime/amd64.S}
        \mintc[firstline=234,lastline=237]{../ocaml/runtime/amd64.S}
      }
      \onslide*<2>{
        \mintc[firstline=190,lastline=192,highlightlines={191-192}]{../ocaml/runtime/amd64.S}
        \mintc[firstline=234,lastline=237,highlightlines={235-237}]{../ocaml/runtime/amd64.S}
      }
      \onslide*<1>{\listCamlRaiseExn[highlightlines=720]{}}
      \onslide*<2>{\listCamlRaiseExn[highlightlines=722]{}}
      \onslide*<3->{\providecommand\step{5}\input{diagrams/backtrace/backtrace.tex}}
    \end{column}
  \end{columns}
\end{frame}

\begin{frame}{Backtraces}{\funcname{caml\_probably\_raise}'s handler doesn't catch \typename{ExnC}}
  \begin{columns}[c]
    \begin{column}{0.45\textwidth}
      \minipage[c][0.75\textheight][s]{\columnwidth}
      \begin{itemize}
        \item<1-> \funcname{match \dots with}\footnotemark pattern matching can also match exceptions
        \item<1-> The CMM of \funcname{probably\_raise} shows that this \funcname{match \dots with} is implemented with a pattern matching and a \funcname{try \dots with}
        \item<1-> But this one only matches on \typename{ExnA} exception
        \item<2-> Since the pattern matching fails to catch the exception, \funcname{reraise} is called with our current \typename{ExnC} exception
        \item<3-> Disassembly shows that \funcname{reraise} is actually a call to \funcname{caml\_reraise\_exn}, which is also an assembly function provided by the runtime
      \end{itemize}
      \vfill
      \mintocaml[firstline=15,lastline=21,highlightlines={18-21}]{../src/backtrace/backtrace.ml}
      \endminipage
    \end{column}
    \begin{column}{0.55\textwidth}
      \centering
      \onslide*<1>{\mintcmm[highlightlines={10-18}]{../src/backtrace/camlBacktrace\_\_probably\_raise\_351.cmm}}
      \onslide*<2>{\listcmm[highlightlines=18]{../src/backtrace/camlBacktrace\_\_probably\_raise_351.cmm}{CMM of probably\_raise}}
      \onslide*<3>{\mintobjdump[firstline=5,lastline=35,highlightlines=31]{../src/backtrace/camlBacktrace\_\_probably\_raise_351.objdump}}
    \end{column}
  \end{columns}
  \only<1>{\footnotetext[1]{https://blog.janestreet.com/pattern-matching-and-exception-handling-unite/}}
\end{frame}

\newcommand\listCamlReraiseExn[1][]{
  \listasm[firstline=727,lastline=734,#1]{../ocaml/runtime/amd64.S}{runtime/amd64.S:727}}

\begin{frame}{Backtraces}{Introducing \funcname{caml\_reraise\_exn}}
  \begin{columns}[c]
    \begin{column}{0.5\textwidth}
      \minipage[c][0.66\textheight][s]{\columnwidth}
      \begin{itemize}
        \item<1-> The exception have to be reraised to the next handler
        \item<2-> First, checks if the backtrace should be collected with \funcarg{domain\_state->backtrace\_active}
        \only<3>{
        \begin{itemize}
            \item If not, behaves like a \funcname{raise\_notrace} with \funcname{RESTORE\_EXN\_HANDLER\_OCAML}
        \end{itemize}
        }
        \item<4-> Jumps into \funcname{caml\_raise\_exn}
        \item<5-> Calls \funcname{caml\_stash\_backtrace} again, to scan the next part of the stack: from the trap just pop-ed to the next trap
      \end{itemize}
      \vfill
      \mintocaml[firstline=15,lastline=21,highlightlines={18-21}]{../src/backtrace/backtrace.ml}
      \endminipage
    \end{column}
    \begin{column}{0.5\textwidth}
      \raggedleft
      \onslide*<1>{\listCamlReraiseExn{}}
      \onslide*<2>{\listCamlReraiseExn[highlightlines=729]{}}
      \onslide*<3>{
        \mintc[firstline=190,lastline=192,highlightlines={191-192}]{../ocaml/runtime/amd64.S}
        \mintc[firstline=234,lastline=237,highlightlines={235-237}]{../ocaml/runtime/amd64.S}
        \medskip
        \listCamlReraiseExn[highlightlines=731]{}
      }
      \onslide*<4-5>{
        % TODO refactor with \listCamlReraiseExn
        \mintasm[firstline=727,lastline=734,highlightlines=730]{../ocaml/runtime/amd64.S}
        \medskip
      }
      \onslide*<4>{\listCamlRaiseExn[highlightlines=711]{}}
      \onslide*<5>{\listCamlRaiseExn[highlightlines=720]{}}
    \end{column}
  \end{columns}
\end{frame}

\begin{frame}{Backtraces}{Backtrace collection continues}
  \begin{columns}[c]
    \begin{column}{0.55\textwidth}
      \minipage[c][0.75\textheight][s]{\columnwidth}
      \begin{itemize}
        \item<1-> \funcname{caml\_stash\_backtrace} scans the older part of the stack, up to the next trap
        \item<6-> Controls jumps to the nested exception handler in \funcname{maybe\_raise}, which matches only on \typename{ExnB}
        \item<7-> \funcname{caml\_reraise\_exn} is called again, backtrace collection resumes with \funcname{caml\_stash\_backtrace}
      \end{itemize}
      \medskip
      \begin{itemize}
        \item<2-> Constructed backtrace:
        \begin{itemize}
          \item <2-> \funcname{definitely\_raise}
          \item <2-> \funcname{probably\_raise}
          \item <3-> \funcname{probably\_raise} (re-raise)
          \item <4-> \funcname{maybe\_raise}
          \item <5-> \funcname{main}
          \item <8-> \funcname{main} (re-raise)
        \end{itemize}
      \end{itemize}
      \vfill
      \onslide*<1-5>{\mintocaml[firstline=15,lastline=21]{../src/backtrace/backtrace.ml}}
      \onslide*<6>{\mintocaml[firstline=28,lastline=34,highlightlines=31]{../src/backtrace/backtrace.ml}}
      \onslide*<7->{\mintocaml[firstline=28,lastline=34,highlightlines=33]{../src/backtrace/backtrace.ml}}
      \endminipage
    \end{column}
    \begin{column}{0.45\textwidth}
      \raggedleft
      \onslide*<1>{\providecommand\step{6}\input{diagrams/backtrace/backtrace.tex}}%
      \onslide*<2>{\providecommand\step{6}\input{diagrams/backtrace/backtrace.tex}}%
      \onslide*<3>{\providecommand\step{7}\input{diagrams/backtrace/backtrace.tex}}%
      \onslide*<4>{\providecommand\step{8}\input{diagrams/backtrace/backtrace.tex}}%
      \onslide*<5>{\providecommand\step{9}\input{diagrams/backtrace/backtrace.tex}}%
      \onslide*<6>{\providecommand\step{10}\input{diagrams/backtrace/backtrace.tex}}%
      \onslide*<7>{\providecommand\step{11}\input{diagrams/backtrace/backtrace.tex}}%
      \onslide*<8>{\providecommand\step{12}\input{diagrams/backtrace/backtrace.tex}}%
      \onslide*<9>{\providecommand\step{13}\input{diagrams/backtrace/backtrace.tex}}%
    \end{column}
  \end{columns}
\end{frame}

\begin{frame}{Backtraces}{Printing the backtrace}
  \begin{columns}[c]
    \begin{column}{0.45\textwidth}
      \minipage[c][0.66\textheight][s]{\columnwidth}
      \begin{itemize}
        \item<1-> The exception backtrace can now be printed to \funcarg{stdout} with \funcname{Printexc.print_backtrace}
        \item<2-> First the exception backtrace is retrieved into an OCaml array with \funcname{get_raw_backtrace }
        \item<3-> Which is implemented as the \funcname{caml_get_exception_raw_backtrace} C function in the runtime
        \item<4-> And finally printed with \funcname{print_exception_backtrace}
      \end{itemize}
      \vfill
      \mintocaml[firstline=28,lastline=34,highlightlines=33]{../src/backtrace/backtrace.ml}
      \endminipage
    \end{column}
    \begin{column}{0.55\textwidth}
      \onslide*<1,2>{
        \mintocaml[firstline=100,lastline=101]{../ocaml/stdlib/printexc.ml}
      }
      \only<1>{
        \listocaml[firstline=170,lastline=172]{../ocaml/stdlib/printexc.ml}{stdlib/printexc.ml:170}
      }
      \only<2>{
        \listocaml[firstline=170,lastline=172,highlightlines=172]{../ocaml/stdlib/printexc.ml}{stdlib/printexc.ml:170}
      }
      \only<3>{
        \mintc[firstline=154,lastline=156]{../ocaml/runtime/backtrace.c}
        \listc[firstline=171,lastline=192,highlightlines={184-187}]{../ocaml/runtime/backtrace.c}{runtime/backtrace.c:154}
      }
      \only<4>{
        \listocaml[firstline=155,lastline=165]{../ocaml/stdlib/printexc.ml}{stdlib/printexc.ml:55}
      }
    \end{column}
  \end{columns}
\end{frame}


\subsection{The multiple kinds of raise}
\frameSubsection{}{}

\begin{frame}{Backtraces}{When backtraces are not needed}
  \begin{columns}[c]
    \begin{column}{0.45\textwidth}
      \minipage[c][0.66\textheight][s]{\columnwidth}
      \begin{itemize}
        \item<1-> What if the backtrace for an exception is never needed?
        \item<2-> It's possible to raise the exception explicitly with \funcname{raise\_notrace} to make raising as cheap as possible
        \item<3-> An example of use in the \funcname{dynlink} library of the OCaml compiler
      \end{itemize}
      \vfill
      \onslide<3->{
      \listocaml[firstline=205,lastline=209,highlightlines=207]{../ocaml/otherlibs/dynlink/dynlink\_compilerlibs/misc.ml}{otherlibs/dynlink/dynlink\_compilerlibs/misc.ml:205}
      }
      \endminipage
    \end{column}
    \begin{column}{0.55\textwidth}
    \only<4>{
      \mintcmm[highlightlines=3]{../src/notrace/camlNotrace\_\_fun_347.cmm}
      \listcmm{../src/notrace/camlNotrace\_\_all\_somes\_327.cmm}{all\_somes}
    }
    \onslide*<5->{
      \listobjdump[highlightlines={6-9}]{../src/notrace/camlNotrace\_\_fun_347.objdump}{anonymous function from \funcname{all\_somes}}
    }
    \end{column}
  \end{columns}
\end{frame}

\begin{frame}{Backtraces}{Summary table of the multiple kinds of raise}
  \begin{itemize}
    \item When debugging information is enabled and if the backtrace of an exception isn't needed, it's possible to raise with \funcname{raise\_notrace} for performance
    \item When debugging information is \emph{not} enabled, the compiler optimises \funcname{raise} calls to inline assembly (same effect as using \funcname{raise\_notrace})
  \end{itemize}
  \bigskip
  \centering
  \begin{tabular}{| l | c | c | c | c |}
    \hline
    \parbox{10em}{Debug information \\ (\funcarg{-g} flag)}  & \multicolumn{2}{|c|}{Disabled}                                     & \multicolumn{2}{|c|}{Enabled} \\
    \hline
    Raise kind                                               & \funcname{raise} & \funcname{raise\_notrace}                       & \funcname{raise} & \funcname{raise\_notrace} \\
    \hline
    Compilation                                              & \parbox{8em}{inline assembly\\ (optimization)} & inline assembly   & \funcname{caml\_raise\_exn} & inline assembly \\
    \hline
  \end{tabular}
\end{frame}


%
% Takeaway #5
%
\subsection*{Takeaway \#5}
\frameSubsectionTakeaway{}
\againframe<5>{takeaway}
}%
      \onslide*<9>{\providecommand\step{13}%
% Backtraces
%
\subsection{One advantage of raising with debugging information}
\frameSubsection{}{}

\begin{frame}{Backtraces}{Debugging information \& source code}
  \begin{columns}[c]
    \begin{column}{0.5\textwidth}
      \begin{itemize}
        \item Raising an exception is fun and games, but how do we know where the exception originated? $\rightarrow$ With the backtrace associated with the exception!
        \item In order to get a backtrace from the point where an exception was raised, it's necessary to compile with debugging information enabled (\funcarg{-g} flag for the compiler)
        \item Same example as in the `Nested handlers' section
      \end{itemize}
    \end{column}
    \begin{column}{0.5\textwidth}
      \listocaml[firstline=9,lastline=34]{../src/backtrace/backtrace.ml}{backtrace/backtrace.ml}
    \end{column}
  \end{columns}
\end{frame}

\begin{frame}{Backtraces}{CMM and disassembly of \funcname{definitely\_raise}}
  \begin{columns}[c]
    \begin{column}{0.5\textwidth}
      \minipage[c][0.66\textheight][s]{\columnwidth}
      \begin{itemize}
        \item<1-> An effect of compiling with debugging information (\funcarg{-g}): the CMM no longer uses \funcname{raise\_notrace} to raise the exception but \funcname{raise} instead
        \item<2-> Disassembly shows that CMM \funcname{raise} is actually a call to \funcname{caml\_raise\_exn}, a function in assembler provided by the runtime
      \end{itemize}
      \vfill
      \mintocaml[firstline=9,lastline=13,highlightlines=12]{../src/backtrace/backtrace.ml}
      \endminipage
    \end{column}
    \begin{column}{0.5\textwidth}
      \onslide*<1>{
        \mintcmm[highlightlines=5]{../src/backtrace/camlBacktrace\_\_definitely\_raise\_347.cmm}
      }
      \onslide*<2>{
        \mintobjdump[firstline=2,highlightlines=14]{../src/backtrace/camlBacktrace\_\_definitely\_raise\_347.objdump}
      }
    \end{column}
  \end{columns}
\end{frame}

\begin{frame}{Backtraces}{Introducing \funcname{caml\_raise\_exn}}
  \begin{columns}[c]
    \begin{column}{0.5\textwidth}
      \minipage[c][0.66\textheight][s]{\columnwidth}
      \onslide*<1>{
        \begin{itemize}
          \item Raising the exception is performed using \funcname{caml\_raise\_exn} which is
            \begin{itemize}
              \item An assembly function provided by the runtime
              \item Architecture specific
            \end{itemize}
          \item Let's see what it does differently from \funcname{raise\_notrace}\dots
          \item We are about to raise \typename{ExnC} with \funcname{caml\_raise\_exn} from \funcname{definitely\_raise}
        \end{itemize}
      }
      \onslide*<2>{
        \mintobjdump[firstline=2,highlightlines=14]{../src/backtrace/camlBacktrace\_\_definitely\_raise\_347.objdump}
      }
      \vfill
      \mintocaml[firstline=9,lastline=13,highlightlines=12]{../src/backtrace/backtrace.ml}
      \endminipage
    \end{column}
    \begin{column}{0.5\textwidth}
      \centering
      \providecommand\step{1}\input{diagrams/backtrace/backtrace.tex}
    \end{column}
  \end{columns}
\end{frame}

\begin{frame}{Backtraces}{But before that, we need more fields from \typename{caml\_domain\_state}}
  \begin{columns}
    \begin{column}{0.5\textwidth}
      \begin{itemize}
        \item Remember \typename{caml\_domain\_state} the per-domain runtime state?
        \item Some other members are of interest to us now\dots
        \item \localname{backtrace\_active} is used to know if the runtime should collect backtraces
        \item \localname{backtrace\_active} can be set by
          \begin{itemize}
            \item The \funcarg{b} flag in the \funcname{OCAMLRUNPARAM} environment variable
            \item \mintinline{ocaml}{Printexc.record_backtrace : bool -> unit} \footnotemark
          \end{itemize}
      \end{itemize}
    \end{column}
    \begin{column}{0.5\textwidth}
      \begin{listing}
        \mintc[highlightlines={9-11}]{src/ocaml/domain\_state\_backtrace.h}
        \caption{runtime/caml/domain\_state.h}
      \end{listing}
    \end{column}
  \end{columns}
  \footnotetext[1]{https://v2.ocaml.org/api/Printexc.html}
\end{frame}

\newcommand\listCamlRaiseExn[1][]{
  \listasm[firstline=702,lastline=725,#1]{../ocaml/runtime/amd64.S}{runtime/amd64.S:702}}

\begin{frame}{Backtraces}{Walk through \funcname{caml\_raise\_exn}}
  \begin{columns}[T]
    \begin{column}{0.5\textwidth}
      \begin{itemize}
        \item<1-> First checks whether exceptions backtrace should be recorded
        \item<1-> By checking the \funcarg{domain\_state->backtrace\_active} value
        \item<2-> If not, acts as \funcname{raise\_notrace} with \funcname{RESTORE\_EXN\_HANDLER\_OCAML}
          \begin{itemize}
            \item Set \regname{RSP} to \localname{domain\_state->exn\_handler} value
            \item Restore \localname{domain\_state->exn\_handler} from trap
            \item Jump with \funcname{ret} (same effect as using \funcname{pop} and \funcname{jmp})
          \end{itemize}
        \item<3-> Otherwise, switches to the C stack to be able to execute C code
        \item<4-> Calls \funcname{caml\_stash\_backtrace}, a C function provided by the runtime\ignorespaces
          \onslide<6->{, with
            \begin{itemize}
              \item \funcarg{exn}: exception value
              \item \funcarg{pc}: code address of the raising point
              \item \funcarg{sp}: stack address of the raising function
              \item \funcarg{trapsp}: stack address of the next handler
            \end{itemize}
          }
      \end{itemize}
      \bigskip
      \mintocaml[firstline=9,lastline=13,highlightlines=12]{../src/backtrace/backtrace.ml}
    \end{column}
    \begin{column}{0.5\textwidth}
      \raggedleft
      \onslide*<1,3-4>{
        \mintc[firstline=190,lastline=192]{../ocaml/runtime/amd64.S}
        \mintc[firstline=234,lastline=237]{../ocaml/runtime/amd64.S}
      }
      \onslide*<2>{
        \mintc[firstline=190,lastline=192,highlightlines={191-192}]{../ocaml/runtime/amd64.S}
        \mintc[firstline=234,lastline=237,highlightlines={235-237}]{../ocaml/runtime/amd64.S}
      }
      \onslide*<1>{\listCamlRaiseExn[highlightlines=705]{}}%
      \onslide*<2>{\listCamlRaiseExn[highlightlines={707-708}]{}}%
      \onslide*<3>{\listCamlRaiseExn[highlightlines={712-714}]{}}%
      \onslide*<4>{\listCamlRaiseExn[highlightlines={715-720}]{}}%
      \onslide*<5>{\providecommand\step{1}\input{diagrams/backtrace/backtrace.tex}}%
      \onslide*<6>{\providecommand\step{2}\input{diagrams/backtrace/backtrace.tex}}%
    \end{column}
  \end{columns}
\end{frame}

\newcommand\listStashBacktrace[1][]{
  \listc[firstline=91,lastline=120,#1]{../ocaml/runtime/backtrace\_nat.c}{runtime/backtrace\_nat.c:91}}

\begin{frame}{Backtraces}{Introducing \funcname{caml\_stash\_backtrace}}
  \begin{columns}[c]
    \begin{column}{0.5\textwidth}
      \minipage[c][0.66\textheight][s]{\columnwidth}
      \begin{itemize}
        \item<1-> A C function provided by the runtime (specific to native code)
        \item<2-> It scans the stack, between the stack pointer at the raising point up to the next trap, in order to collect the names of the detected function frames
          \onslide*<2>{
            \begin{itemize}
              \item And more information, like line number, column number...
            \end{itemize}
          }
        \item<3-> It uses the same technique and information as the Garbage Collector (GC) uses to scan the values live on the stack
          \onslide*<4>{
            \begin{itemize}
              \item \typename{frame\_descr} are at the core of stack unwinding
            \end{itemize}
          }
      \end{itemize}
      \vfill
      \mintocaml[firstline=9,lastline=13,highlightlines=12]{../src/backtrace/backtrace.ml}
      \endminipage
    \end{column}
    \begin{column}{0.5\textwidth}
      \onslide*<1>{\listStashBacktrace[highlightlines=91]{}}
      \onslide*<2>{\listStashBacktrace[highlightlines={91,117-118}]{}}
      \onslide*<3->{\listStashBacktrace[highlightlines={94,106,109-110}]{}}
    \end{column}
  \end{columns}
\end{frame}

\newcommand\listFrameDescr[1][]{
  \listocaml[firstline=111,lastline=124,#1]{../ocaml/asmcomp/emitaux.ml}{asmcomp/emitaux.ml:111}}
\newcommand\listDebugInfo[1][]{
  \listocaml[firstline=38,lastline=49,#1]{../ocaml/lambda/debuginfo.mli}{lambda/debuginfo.mli:38}}

\begin{frame}{Backtraces}{\typename{frame\_descr} and \typename{frame\_debuginfo} in a nutshell}
  \begin{columns}[c]
    \begin{column}{0.5\textwidth}
      \begin{itemize}
        \item<1-> For every return address in an OCaml program, there is a associated \typename{frame\_descr}
        \item<2-> A \typename{frame\_descr} specifies
        \begin{itemize}
          \item<2-> The size of the function stack frame
          \item<3-> Where live values are on the stack (so that the GC can mark them as live and prevent from being swept)
        \end{itemize}
        \item<4-> When compiled with debug information, \typename{frame\_descr} will be linked to a \typename{frame\_debuginfo}
        \begin{itemize}
          \item<5-> Contains information about the position in the source code (file name, line and column number)
        \end{itemize}
      \end{itemize}
    \end{column}
    \begin{column}{0.5\textwidth}
        \onslide*<1>{\listFrameDescr[highlightlines={118-122}]{}}
        \onslide*<2>{\listFrameDescr[highlightlines={120}]{}}
        \onslide*<3>{\listFrameDescr[highlightlines={121}]{}}
        \onslide*<4->{\listFrameDescr[highlightlines={122}]{}}
        \medskip
        \onslide*<1-3>{\listDebugInfo{}}
        \onslide*<4>{\listDebugInfo[highlightlines={38,49}]{}}
        \onslide*<5>{\listDebugInfo[highlightlines={39-46}]{}}
    \end{column}
  \end{columns}
\end{frame}

\begin{frame}{Backtraces}{Scanning the stack with \typename{frame\_descr}}
  \begin{columns}[c]
    \begin{column}{0.5\textwidth}
      \begin{itemize}
        \item Given the return address of the code that raises the exception by calling \funcname{caml\_raise\_exn} (\funcarg{pc} argument of \funcname{caml\_stash\_backtrace})
        \item And the position of the stack pointer (\funcarg{sp} argument of \funcname{caml\_stash\_backtrace})
        \item The frame size given by \typename{frame\_descr}.\funcarg{frame\_size}, allows to find the end of the previous frame
        \item And the return address of the caller of this function
        \bigskip
        \onslide<3->{
        \item Note that traps (here the reddish \funcname{ExnA}, \funcname{ExnB} and \funcname{ExnC} blocks) belong to the frame that installed them, and so the frame size changes when traps are removed
        }
      \end{itemize}
    \end{column}
    \begin{column}{0.5\textwidth}
      \raggedleft
      \onslide*<1>{\providecommand\step{2}\input{diagrams/backtrace/backtrace.tex}}%
      \onslide*<2>{\providecommand\step{1}\input{diagrams/backtrace/scanstack.tex}}%
      \onslide*<3>{\providecommand\step{2}\input{diagrams/backtrace/scanstack.tex}}%
      \onslide*<4>{\providecommand\step{3}\input{diagrams/backtrace/scanstack.tex}}%
      \onslide*<5>{\providecommand\step{4}\input{diagrams/backtrace/scanstack.tex}}%
      \onslide*<6>{\providecommand\step{5}\input{diagrams/backtrace/scanstack.tex}}%
    \end{column}
  \end{columns}
\end{frame}

% Duplicate of previous-previous slide
\begin{frame}{Backtraces}{Continuing on \funcname{caml\_stash\_backtrace}}
  \begin{columns}[c]
    \begin{column}{0.5\textwidth}
      \minipage[c][0.75\textheight][s]{\columnwidth}
      \begin{itemize}
          \color{gray}
        \item A C function provided by the runtime (specific to native code)
        \item It scans the stack, between the stack pointer at the raising point up to the next trap, in order to collect the names of the detected function frames
        \item It uses the same technique and information as the Garbage Collector (GC) uses to scan the values live on the stack
      \end{itemize}
      \begin{itemize}
        \item For each function frame detected, store a pointer to the \typename{frame\_descr} in the \localname{domain\_state->backtrace\_buffer} array, effectively constructing the backtrace
      \end{itemize}
      \onslide<2->{
        \begin{itemize}
            \smallskip
          \item Constructed backtrace:
            \funcname{definitely\_raise},
            \onslide<3->{\funcname{probably\_raise}}
        \end{itemize}
      }
      \vfill
      \mintocaml[firstline=9,lastline=13,highlightlines=12]{../src/backtrace/backtrace.ml}
      \endminipage
    \end{column}
    \begin{column}{0.5\textwidth}
      \raggedleft
      \onslide*<1>{\listStashBacktrace[highlightlines={114-115}]{}}
      \onslide*<2>{\providecommand\step{3}\input{diagrams/backtrace/backtrace.tex}}%
      \onslide*<3>{\providecommand\step{4}\input{diagrams/backtrace/backtrace.tex}}%
    \end{column}
  \end{columns}
\end{frame}

\begin{frame}{Backtraces}{Back to \funcname{caml\_raise\_exn}}
  \begin{columns}[c]
    \begin{column}{0.5\textwidth}
      \minipage[c][0.66\textheight][s]{\columnwidth}
      \begin{itemize}\color{gray}
        \item Checks wheteher exceptions backtrace should be recorded, by checking the \funcarg{domain\_state->backtrace\_active} value
        \item Switches to the C stack to be able to execute C code
        \item Calls \funcname{caml\_stash\_backtrace}, to collect the backtrace up to the next trap
      \end{itemize}
      \onslide*<2->{
        \begin{itemize}
          \item<2-> Executes the exception handler in \funcname{probably\_raise} with \funcname{RESTORE\_EXN\_HANDLER\_OCAML}
            \begin{itemize}
              \item Set \regname{RSP} to \localname{domain\_state->exn\_handler} value
              \item Restore \localname{domain\_state->exn\_handler} from trap
              \item Jump to the handler code with \funcname{ret}
            \end{itemize}
        \end{itemize}
      }
      \vfill
      \onslide*<1>{\mintocaml[firstline=9,lastline=13,highlightlines=12]{../src/backtrace/backtrace.ml}}
      \onslide*<2->{\mintocaml[firstline=15,lastline=21,highlightlines={19-21}]{../src/backtrace/backtrace.ml}}
      \endminipage
    \end{column}
    \begin{column}{0.5\textwidth}
      \centering
      \onslide*<1>{
        \mintc[firstline=190,lastline=192]{../ocaml/runtime/amd64.S}
        \mintc[firstline=234,lastline=237]{../ocaml/runtime/amd64.S}
      }
      \onslide*<2>{
        \mintc[firstline=190,lastline=192,highlightlines={191-192}]{../ocaml/runtime/amd64.S}
        \mintc[firstline=234,lastline=237,highlightlines={235-237}]{../ocaml/runtime/amd64.S}
      }
      \onslide*<1>{\listCamlRaiseExn[highlightlines=720]{}}
      \onslide*<2>{\listCamlRaiseExn[highlightlines=722]{}}
      \onslide*<3->{\providecommand\step{5}\input{diagrams/backtrace/backtrace.tex}}
    \end{column}
  \end{columns}
\end{frame}

\begin{frame}{Backtraces}{\funcname{caml\_probably\_raise}'s handler doesn't catch \typename{ExnC}}
  \begin{columns}[c]
    \begin{column}{0.45\textwidth}
      \minipage[c][0.75\textheight][s]{\columnwidth}
      \begin{itemize}
        \item<1-> \funcname{match \dots with}\footnotemark pattern matching can also match exceptions
        \item<1-> The CMM of \funcname{probably\_raise} shows that this \funcname{match \dots with} is implemented with a pattern matching and a \funcname{try \dots with}
        \item<1-> But this one only matches on \typename{ExnA} exception
        \item<2-> Since the pattern matching fails to catch the exception, \funcname{reraise} is called with our current \typename{ExnC} exception
        \item<3-> Disassembly shows that \funcname{reraise} is actually a call to \funcname{caml\_reraise\_exn}, which is also an assembly function provided by the runtime
      \end{itemize}
      \vfill
      \mintocaml[firstline=15,lastline=21,highlightlines={18-21}]{../src/backtrace/backtrace.ml}
      \endminipage
    \end{column}
    \begin{column}{0.55\textwidth}
      \centering
      \onslide*<1>{\mintcmm[highlightlines={10-18}]{../src/backtrace/camlBacktrace\_\_probably\_raise\_351.cmm}}
      \onslide*<2>{\listcmm[highlightlines=18]{../src/backtrace/camlBacktrace\_\_probably\_raise_351.cmm}{CMM of probably\_raise}}
      \onslide*<3>{\mintobjdump[firstline=5,lastline=35,highlightlines=31]{../src/backtrace/camlBacktrace\_\_probably\_raise_351.objdump}}
    \end{column}
  \end{columns}
  \only<1>{\footnotetext[1]{https://blog.janestreet.com/pattern-matching-and-exception-handling-unite/}}
\end{frame}

\newcommand\listCamlReraiseExn[1][]{
  \listasm[firstline=727,lastline=734,#1]{../ocaml/runtime/amd64.S}{runtime/amd64.S:727}}

\begin{frame}{Backtraces}{Introducing \funcname{caml\_reraise\_exn}}
  \begin{columns}[c]
    \begin{column}{0.5\textwidth}
      \minipage[c][0.66\textheight][s]{\columnwidth}
      \begin{itemize}
        \item<1-> The exception have to be reraised to the next handler
        \item<2-> First, checks if the backtrace should be collected with \funcarg{domain\_state->backtrace\_active}
        \only<3>{
        \begin{itemize}
            \item If not, behaves like a \funcname{raise\_notrace} with \funcname{RESTORE\_EXN\_HANDLER\_OCAML}
        \end{itemize}
        }
        \item<4-> Jumps into \funcname{caml\_raise\_exn}
        \item<5-> Calls \funcname{caml\_stash\_backtrace} again, to scan the next part of the stack: from the trap just pop-ed to the next trap
      \end{itemize}
      \vfill
      \mintocaml[firstline=15,lastline=21,highlightlines={18-21}]{../src/backtrace/backtrace.ml}
      \endminipage
    \end{column}
    \begin{column}{0.5\textwidth}
      \raggedleft
      \onslide*<1>{\listCamlReraiseExn{}}
      \onslide*<2>{\listCamlReraiseExn[highlightlines=729]{}}
      \onslide*<3>{
        \mintc[firstline=190,lastline=192,highlightlines={191-192}]{../ocaml/runtime/amd64.S}
        \mintc[firstline=234,lastline=237,highlightlines={235-237}]{../ocaml/runtime/amd64.S}
        \medskip
        \listCamlReraiseExn[highlightlines=731]{}
      }
      \onslide*<4-5>{
        % TODO refactor with \listCamlReraiseExn
        \mintasm[firstline=727,lastline=734,highlightlines=730]{../ocaml/runtime/amd64.S}
        \medskip
      }
      \onslide*<4>{\listCamlRaiseExn[highlightlines=711]{}}
      \onslide*<5>{\listCamlRaiseExn[highlightlines=720]{}}
    \end{column}
  \end{columns}
\end{frame}

\begin{frame}{Backtraces}{Backtrace collection continues}
  \begin{columns}[c]
    \begin{column}{0.55\textwidth}
      \minipage[c][0.75\textheight][s]{\columnwidth}
      \begin{itemize}
        \item<1-> \funcname{caml\_stash\_backtrace} scans the older part of the stack, up to the next trap
        \item<6-> Controls jumps to the nested exception handler in \funcname{maybe\_raise}, which matches only on \typename{ExnB}
        \item<7-> \funcname{caml\_reraise\_exn} is called again, backtrace collection resumes with \funcname{caml\_stash\_backtrace}
      \end{itemize}
      \medskip
      \begin{itemize}
        \item<2-> Constructed backtrace:
        \begin{itemize}
          \item <2-> \funcname{definitely\_raise}
          \item <2-> \funcname{probably\_raise}
          \item <3-> \funcname{probably\_raise} (re-raise)
          \item <4-> \funcname{maybe\_raise}
          \item <5-> \funcname{main}
          \item <8-> \funcname{main} (re-raise)
        \end{itemize}
      \end{itemize}
      \vfill
      \onslide*<1-5>{\mintocaml[firstline=15,lastline=21]{../src/backtrace/backtrace.ml}}
      \onslide*<6>{\mintocaml[firstline=28,lastline=34,highlightlines=31]{../src/backtrace/backtrace.ml}}
      \onslide*<7->{\mintocaml[firstline=28,lastline=34,highlightlines=33]{../src/backtrace/backtrace.ml}}
      \endminipage
    \end{column}
    \begin{column}{0.45\textwidth}
      \raggedleft
      \onslide*<1>{\providecommand\step{6}\input{diagrams/backtrace/backtrace.tex}}%
      \onslide*<2>{\providecommand\step{6}\input{diagrams/backtrace/backtrace.tex}}%
      \onslide*<3>{\providecommand\step{7}\input{diagrams/backtrace/backtrace.tex}}%
      \onslide*<4>{\providecommand\step{8}\input{diagrams/backtrace/backtrace.tex}}%
      \onslide*<5>{\providecommand\step{9}\input{diagrams/backtrace/backtrace.tex}}%
      \onslide*<6>{\providecommand\step{10}\input{diagrams/backtrace/backtrace.tex}}%
      \onslide*<7>{\providecommand\step{11}\input{diagrams/backtrace/backtrace.tex}}%
      \onslide*<8>{\providecommand\step{12}\input{diagrams/backtrace/backtrace.tex}}%
      \onslide*<9>{\providecommand\step{13}\input{diagrams/backtrace/backtrace.tex}}%
    \end{column}
  \end{columns}
\end{frame}

\begin{frame}{Backtraces}{Printing the backtrace}
  \begin{columns}[c]
    \begin{column}{0.45\textwidth}
      \minipage[c][0.66\textheight][s]{\columnwidth}
      \begin{itemize}
        \item<1-> The exception backtrace can now be printed to \funcarg{stdout} with \funcname{Printexc.print_backtrace}
        \item<2-> First the exception backtrace is retrieved into an OCaml array with \funcname{get_raw_backtrace }
        \item<3-> Which is implemented as the \funcname{caml_get_exception_raw_backtrace} C function in the runtime
        \item<4-> And finally printed with \funcname{print_exception_backtrace}
      \end{itemize}
      \vfill
      \mintocaml[firstline=28,lastline=34,highlightlines=33]{../src/backtrace/backtrace.ml}
      \endminipage
    \end{column}
    \begin{column}{0.55\textwidth}
      \onslide*<1,2>{
        \mintocaml[firstline=100,lastline=101]{../ocaml/stdlib/printexc.ml}
      }
      \only<1>{
        \listocaml[firstline=170,lastline=172]{../ocaml/stdlib/printexc.ml}{stdlib/printexc.ml:170}
      }
      \only<2>{
        \listocaml[firstline=170,lastline=172,highlightlines=172]{../ocaml/stdlib/printexc.ml}{stdlib/printexc.ml:170}
      }
      \only<3>{
        \mintc[firstline=154,lastline=156]{../ocaml/runtime/backtrace.c}
        \listc[firstline=171,lastline=192,highlightlines={184-187}]{../ocaml/runtime/backtrace.c}{runtime/backtrace.c:154}
      }
      \only<4>{
        \listocaml[firstline=155,lastline=165]{../ocaml/stdlib/printexc.ml}{stdlib/printexc.ml:55}
      }
    \end{column}
  \end{columns}
\end{frame}


\subsection{The multiple kinds of raise}
\frameSubsection{}{}

\begin{frame}{Backtraces}{When backtraces are not needed}
  \begin{columns}[c]
    \begin{column}{0.45\textwidth}
      \minipage[c][0.66\textheight][s]{\columnwidth}
      \begin{itemize}
        \item<1-> What if the backtrace for an exception is never needed?
        \item<2-> It's possible to raise the exception explicitly with \funcname{raise\_notrace} to make raising as cheap as possible
        \item<3-> An example of use in the \funcname{dynlink} library of the OCaml compiler
      \end{itemize}
      \vfill
      \onslide<3->{
      \listocaml[firstline=205,lastline=209,highlightlines=207]{../ocaml/otherlibs/dynlink/dynlink\_compilerlibs/misc.ml}{otherlibs/dynlink/dynlink\_compilerlibs/misc.ml:205}
      }
      \endminipage
    \end{column}
    \begin{column}{0.55\textwidth}
    \only<4>{
      \mintcmm[highlightlines=3]{../src/notrace/camlNotrace\_\_fun_347.cmm}
      \listcmm{../src/notrace/camlNotrace\_\_all\_somes\_327.cmm}{all\_somes}
    }
    \onslide*<5->{
      \listobjdump[highlightlines={6-9}]{../src/notrace/camlNotrace\_\_fun_347.objdump}{anonymous function from \funcname{all\_somes}}
    }
    \end{column}
  \end{columns}
\end{frame}

\begin{frame}{Backtraces}{Summary table of the multiple kinds of raise}
  \begin{itemize}
    \item When debugging information is enabled and if the backtrace of an exception isn't needed, it's possible to raise with \funcname{raise\_notrace} for performance
    \item When debugging information is \emph{not} enabled, the compiler optimises \funcname{raise} calls to inline assembly (same effect as using \funcname{raise\_notrace})
  \end{itemize}
  \bigskip
  \centering
  \begin{tabular}{| l | c | c | c | c |}
    \hline
    \parbox{10em}{Debug information \\ (\funcarg{-g} flag)}  & \multicolumn{2}{|c|}{Disabled}                                     & \multicolumn{2}{|c|}{Enabled} \\
    \hline
    Raise kind                                               & \funcname{raise} & \funcname{raise\_notrace}                       & \funcname{raise} & \funcname{raise\_notrace} \\
    \hline
    Compilation                                              & \parbox{8em}{inline assembly\\ (optimization)} & inline assembly   & \funcname{caml\_raise\_exn} & inline assembly \\
    \hline
  \end{tabular}
\end{frame}


%
% Takeaway #5
%
\subsection*{Takeaway \#5}
\frameSubsectionTakeaway{}
\againframe<5>{takeaway}
}%
    \end{column}
  \end{columns}
\end{frame}

\begin{frame}{Backtraces}{Printing the backtrace}
  \begin{columns}[c]
    \begin{column}{0.45\textwidth}
      \minipage[c][0.66\textheight][s]{\columnwidth}
      \begin{itemize}
        \item<1-> The exception backtrace can now be printed to \funcarg{stdout} with \funcname{Printexc.print_backtrace}
        \item<2-> First the exception backtrace is retrieved into an OCaml array with \funcname{get_raw_backtrace }
        \item<3-> Which is implemented as the \funcname{caml_get_exception_raw_backtrace} C function in the runtime
        \item<4-> And finally printed with \funcname{print_exception_backtrace}
      \end{itemize}
      \vfill
      \mintocaml[firstline=28,lastline=34,highlightlines=33]{../src/backtrace/backtrace.ml}
      \endminipage
    \end{column}
    \begin{column}{0.55\textwidth}
      \onslide*<1,2>{
        \mintocaml[firstline=100,lastline=101]{../ocaml/stdlib/printexc.ml}
      }
      \only<1>{
        \listocaml[firstline=170,lastline=172]{../ocaml/stdlib/printexc.ml}{stdlib/printexc.ml:170}
      }
      \only<2>{
        \listocaml[firstline=170,lastline=172,highlightlines=172]{../ocaml/stdlib/printexc.ml}{stdlib/printexc.ml:170}
      }
      \only<3>{
        \mintc[firstline=154,lastline=156]{../ocaml/runtime/backtrace.c}
        \listc[firstline=171,lastline=192,highlightlines={184-187}]{../ocaml/runtime/backtrace.c}{runtime/backtrace.c:154}
      }
      \only<4>{
        \listocaml[firstline=155,lastline=165]{../ocaml/stdlib/printexc.ml}{stdlib/printexc.ml:55}
      }
    \end{column}
  \end{columns}
\end{frame}


\subsection{The multiple kinds of raise}
\frameSubsection{}{}

\begin{frame}{Backtraces}{When backtraces are not needed}
  \begin{columns}[c]
    \begin{column}{0.45\textwidth}
      \minipage[c][0.66\textheight][s]{\columnwidth}
      \begin{itemize}
        \item<1-> What if the backtrace for an exception is never needed?
        \item<2-> It's possible to raise the exception explicitly with \funcname{raise\_notrace} to make raising as cheap as possible
        \item<3-> An example of use in the \funcname{dynlink} library of the OCaml compiler
      \end{itemize}
      \vfill
      \onslide<3->{
      \listocaml[firstline=205,lastline=209,highlightlines=207]{../ocaml/otherlibs/dynlink/dynlink\_compilerlibs/misc.ml}{otherlibs/dynlink/dynlink\_compilerlibs/misc.ml:205}
      }
      \endminipage
    \end{column}
    \begin{column}{0.55\textwidth}
    \only<4>{
      \mintcmm[highlightlines=3]{../src/notrace/camlNotrace\_\_fun_347.cmm}
      \listcmm{../src/notrace/camlNotrace\_\_all\_somes\_327.cmm}{all\_somes}
    }
    \onslide*<5->{
      \listobjdump[highlightlines={6-9}]{../src/notrace/camlNotrace\_\_fun_347.objdump}{anonymous function from \funcname{all\_somes}}
    }
    \end{column}
  \end{columns}
\end{frame}

\begin{frame}{Backtraces}{Summary table of the multiple kinds of raise}
  \begin{itemize}
    \item When debugging information is enabled and if the backtrace of an exception isn't needed, it's possible to raise with \funcname{raise\_notrace} for performance
    \item When debugging information is \emph{not} enabled, the compiler optimises \funcname{raise} calls to inline assembly (same effect as using \funcname{raise\_notrace})
  \end{itemize}
  \bigskip
  \centering
  \begin{tabular}{| l | c | c | c | c |}
    \hline
    \parbox{10em}{Debug information \\ (\funcarg{-g} flag)}  & \multicolumn{2}{|c|}{Disabled}                                     & \multicolumn{2}{|c|}{Enabled} \\
    \hline
    Raise kind                                               & \funcname{raise} & \funcname{raise\_notrace}                       & \funcname{raise} & \funcname{raise\_notrace} \\
    \hline
    Compilation                                              & \parbox{8em}{inline assembly\\ (optimization)} & inline assembly   & \funcname{caml\_raise\_exn} & inline assembly \\
    \hline
  \end{tabular}
\end{frame}


%
% Takeaway #5
%
\subsection*{Takeaway \#5}
\frameSubsectionTakeaway{}
\againframe<5>{takeaway}
}%
      \onslide*<5>{\providecommand\step{9}%
% Backtraces
%
\subsection{One advantage of raising with debugging information}
\frameSubsection{}{}

\begin{frame}{Backtraces}{Debugging information \& source code}
  \begin{columns}[c]
    \begin{column}{0.5\textwidth}
      \begin{itemize}
        \item Raising an exception is fun and games, but how do we know where the exception originated? $\rightarrow$ With the backtrace associated with the exception!
        \item In order to get a backtrace from the point where an exception was raised, it's necessary to compile with debugging information enabled (\funcarg{-g} flag for the compiler)
        \item Same example as in the `Nested handlers' section
      \end{itemize}
    \end{column}
    \begin{column}{0.5\textwidth}
      \listocaml[firstline=9,lastline=34]{../src/backtrace/backtrace.ml}{backtrace/backtrace.ml}
    \end{column}
  \end{columns}
\end{frame}

\begin{frame}{Backtraces}{CMM and disassembly of \funcname{definitely\_raise}}
  \begin{columns}[c]
    \begin{column}{0.5\textwidth}
      \minipage[c][0.66\textheight][s]{\columnwidth}
      \begin{itemize}
        \item<1-> An effect of compiling with debugging information (\funcarg{-g}): the CMM no longer uses \funcname{raise\_notrace} to raise the exception but \funcname{raise} instead
        \item<2-> Disassembly shows that CMM \funcname{raise} is actually a call to \funcname{caml\_raise\_exn}, a function in assembler provided by the runtime
      \end{itemize}
      \vfill
      \mintocaml[firstline=9,lastline=13,highlightlines=12]{../src/backtrace/backtrace.ml}
      \endminipage
    \end{column}
    \begin{column}{0.5\textwidth}
      \onslide*<1>{
        \mintcmm[highlightlines=5]{../src/backtrace/camlBacktrace\_\_definitely\_raise\_347.cmm}
      }
      \onslide*<2>{
        \mintobjdump[firstline=2,highlightlines=14]{../src/backtrace/camlBacktrace\_\_definitely\_raise\_347.objdump}
      }
    \end{column}
  \end{columns}
\end{frame}

\begin{frame}{Backtraces}{Introducing \funcname{caml\_raise\_exn}}
  \begin{columns}[c]
    \begin{column}{0.5\textwidth}
      \minipage[c][0.66\textheight][s]{\columnwidth}
      \onslide*<1>{
        \begin{itemize}
          \item Raising the exception is performed using \funcname{caml\_raise\_exn} which is
            \begin{itemize}
              \item An assembly function provided by the runtime
              \item Architecture specific
            \end{itemize}
          \item Let's see what it does differently from \funcname{raise\_notrace}\dots
          \item We are about to raise \typename{ExnC} with \funcname{caml\_raise\_exn} from \funcname{definitely\_raise}
        \end{itemize}
      }
      \onslide*<2>{
        \mintobjdump[firstline=2,highlightlines=14]{../src/backtrace/camlBacktrace\_\_definitely\_raise\_347.objdump}
      }
      \vfill
      \mintocaml[firstline=9,lastline=13,highlightlines=12]{../src/backtrace/backtrace.ml}
      \endminipage
    \end{column}
    \begin{column}{0.5\textwidth}
      \centering
      \providecommand\step{1}%
% Backtraces
%
\subsection{One advantage of raising with debugging information}
\frameSubsection{}{}

\begin{frame}{Backtraces}{Debugging information \& source code}
  \begin{columns}[c]
    \begin{column}{0.5\textwidth}
      \begin{itemize}
        \item Raising an exception is fun and games, but how do we know where the exception originated? $\rightarrow$ With the backtrace associated with the exception!
        \item In order to get a backtrace from the point where an exception was raised, it's necessary to compile with debugging information enabled (\funcarg{-g} flag for the compiler)
        \item Same example as in the `Nested handlers' section
      \end{itemize}
    \end{column}
    \begin{column}{0.5\textwidth}
      \listocaml[firstline=9,lastline=34]{../src/backtrace/backtrace.ml}{backtrace/backtrace.ml}
    \end{column}
  \end{columns}
\end{frame}

\begin{frame}{Backtraces}{CMM and disassembly of \funcname{definitely\_raise}}
  \begin{columns}[c]
    \begin{column}{0.5\textwidth}
      \minipage[c][0.66\textheight][s]{\columnwidth}
      \begin{itemize}
        \item<1-> An effect of compiling with debugging information (\funcarg{-g}): the CMM no longer uses \funcname{raise\_notrace} to raise the exception but \funcname{raise} instead
        \item<2-> Disassembly shows that CMM \funcname{raise} is actually a call to \funcname{caml\_raise\_exn}, a function in assembler provided by the runtime
      \end{itemize}
      \vfill
      \mintocaml[firstline=9,lastline=13,highlightlines=12]{../src/backtrace/backtrace.ml}
      \endminipage
    \end{column}
    \begin{column}{0.5\textwidth}
      \onslide*<1>{
        \mintcmm[highlightlines=5]{../src/backtrace/camlBacktrace\_\_definitely\_raise\_347.cmm}
      }
      \onslide*<2>{
        \mintobjdump[firstline=2,highlightlines=14]{../src/backtrace/camlBacktrace\_\_definitely\_raise\_347.objdump}
      }
    \end{column}
  \end{columns}
\end{frame}

\begin{frame}{Backtraces}{Introducing \funcname{caml\_raise\_exn}}
  \begin{columns}[c]
    \begin{column}{0.5\textwidth}
      \minipage[c][0.66\textheight][s]{\columnwidth}
      \onslide*<1>{
        \begin{itemize}
          \item Raising the exception is performed using \funcname{caml\_raise\_exn} which is
            \begin{itemize}
              \item An assembly function provided by the runtime
              \item Architecture specific
            \end{itemize}
          \item Let's see what it does differently from \funcname{raise\_notrace}\dots
          \item We are about to raise \typename{ExnC} with \funcname{caml\_raise\_exn} from \funcname{definitely\_raise}
        \end{itemize}
      }
      \onslide*<2>{
        \mintobjdump[firstline=2,highlightlines=14]{../src/backtrace/camlBacktrace\_\_definitely\_raise\_347.objdump}
      }
      \vfill
      \mintocaml[firstline=9,lastline=13,highlightlines=12]{../src/backtrace/backtrace.ml}
      \endminipage
    \end{column}
    \begin{column}{0.5\textwidth}
      \centering
      \providecommand\step{1}\input{diagrams/backtrace/backtrace.tex}
    \end{column}
  \end{columns}
\end{frame}

\begin{frame}{Backtraces}{But before that, we need more fields from \typename{caml\_domain\_state}}
  \begin{columns}
    \begin{column}{0.5\textwidth}
      \begin{itemize}
        \item Remember \typename{caml\_domain\_state} the per-domain runtime state?
        \item Some other members are of interest to us now\dots
        \item \localname{backtrace\_active} is used to know if the runtime should collect backtraces
        \item \localname{backtrace\_active} can be set by
          \begin{itemize}
            \item The \funcarg{b} flag in the \funcname{OCAMLRUNPARAM} environment variable
            \item \mintinline{ocaml}{Printexc.record_backtrace : bool -> unit} \footnotemark
          \end{itemize}
      \end{itemize}
    \end{column}
    \begin{column}{0.5\textwidth}
      \begin{listing}
        \mintc[highlightlines={9-11}]{src/ocaml/domain\_state\_backtrace.h}
        \caption{runtime/caml/domain\_state.h}
      \end{listing}
    \end{column}
  \end{columns}
  \footnotetext[1]{https://v2.ocaml.org/api/Printexc.html}
\end{frame}

\newcommand\listCamlRaiseExn[1][]{
  \listasm[firstline=702,lastline=725,#1]{../ocaml/runtime/amd64.S}{runtime/amd64.S:702}}

\begin{frame}{Backtraces}{Walk through \funcname{caml\_raise\_exn}}
  \begin{columns}[T]
    \begin{column}{0.5\textwidth}
      \begin{itemize}
        \item<1-> First checks whether exceptions backtrace should be recorded
        \item<1-> By checking the \funcarg{domain\_state->backtrace\_active} value
        \item<2-> If not, acts as \funcname{raise\_notrace} with \funcname{RESTORE\_EXN\_HANDLER\_OCAML}
          \begin{itemize}
            \item Set \regname{RSP} to \localname{domain\_state->exn\_handler} value
            \item Restore \localname{domain\_state->exn\_handler} from trap
            \item Jump with \funcname{ret} (same effect as using \funcname{pop} and \funcname{jmp})
          \end{itemize}
        \item<3-> Otherwise, switches to the C stack to be able to execute C code
        \item<4-> Calls \funcname{caml\_stash\_backtrace}, a C function provided by the runtime\ignorespaces
          \onslide<6->{, with
            \begin{itemize}
              \item \funcarg{exn}: exception value
              \item \funcarg{pc}: code address of the raising point
              \item \funcarg{sp}: stack address of the raising function
              \item \funcarg{trapsp}: stack address of the next handler
            \end{itemize}
          }
      \end{itemize}
      \bigskip
      \mintocaml[firstline=9,lastline=13,highlightlines=12]{../src/backtrace/backtrace.ml}
    \end{column}
    \begin{column}{0.5\textwidth}
      \raggedleft
      \onslide*<1,3-4>{
        \mintc[firstline=190,lastline=192]{../ocaml/runtime/amd64.S}
        \mintc[firstline=234,lastline=237]{../ocaml/runtime/amd64.S}
      }
      \onslide*<2>{
        \mintc[firstline=190,lastline=192,highlightlines={191-192}]{../ocaml/runtime/amd64.S}
        \mintc[firstline=234,lastline=237,highlightlines={235-237}]{../ocaml/runtime/amd64.S}
      }
      \onslide*<1>{\listCamlRaiseExn[highlightlines=705]{}}%
      \onslide*<2>{\listCamlRaiseExn[highlightlines={707-708}]{}}%
      \onslide*<3>{\listCamlRaiseExn[highlightlines={712-714}]{}}%
      \onslide*<4>{\listCamlRaiseExn[highlightlines={715-720}]{}}%
      \onslide*<5>{\providecommand\step{1}\input{diagrams/backtrace/backtrace.tex}}%
      \onslide*<6>{\providecommand\step{2}\input{diagrams/backtrace/backtrace.tex}}%
    \end{column}
  \end{columns}
\end{frame}

\newcommand\listStashBacktrace[1][]{
  \listc[firstline=91,lastline=120,#1]{../ocaml/runtime/backtrace\_nat.c}{runtime/backtrace\_nat.c:91}}

\begin{frame}{Backtraces}{Introducing \funcname{caml\_stash\_backtrace}}
  \begin{columns}[c]
    \begin{column}{0.5\textwidth}
      \minipage[c][0.66\textheight][s]{\columnwidth}
      \begin{itemize}
        \item<1-> A C function provided by the runtime (specific to native code)
        \item<2-> It scans the stack, between the stack pointer at the raising point up to the next trap, in order to collect the names of the detected function frames
          \onslide*<2>{
            \begin{itemize}
              \item And more information, like line number, column number...
            \end{itemize}
          }
        \item<3-> It uses the same technique and information as the Garbage Collector (GC) uses to scan the values live on the stack
          \onslide*<4>{
            \begin{itemize}
              \item \typename{frame\_descr} are at the core of stack unwinding
            \end{itemize}
          }
      \end{itemize}
      \vfill
      \mintocaml[firstline=9,lastline=13,highlightlines=12]{../src/backtrace/backtrace.ml}
      \endminipage
    \end{column}
    \begin{column}{0.5\textwidth}
      \onslide*<1>{\listStashBacktrace[highlightlines=91]{}}
      \onslide*<2>{\listStashBacktrace[highlightlines={91,117-118}]{}}
      \onslide*<3->{\listStashBacktrace[highlightlines={94,106,109-110}]{}}
    \end{column}
  \end{columns}
\end{frame}

\newcommand\listFrameDescr[1][]{
  \listocaml[firstline=111,lastline=124,#1]{../ocaml/asmcomp/emitaux.ml}{asmcomp/emitaux.ml:111}}
\newcommand\listDebugInfo[1][]{
  \listocaml[firstline=38,lastline=49,#1]{../ocaml/lambda/debuginfo.mli}{lambda/debuginfo.mli:38}}

\begin{frame}{Backtraces}{\typename{frame\_descr} and \typename{frame\_debuginfo} in a nutshell}
  \begin{columns}[c]
    \begin{column}{0.5\textwidth}
      \begin{itemize}
        \item<1-> For every return address in an OCaml program, there is a associated \typename{frame\_descr}
        \item<2-> A \typename{frame\_descr} specifies
        \begin{itemize}
          \item<2-> The size of the function stack frame
          \item<3-> Where live values are on the stack (so that the GC can mark them as live and prevent from being swept)
        \end{itemize}
        \item<4-> When compiled with debug information, \typename{frame\_descr} will be linked to a \typename{frame\_debuginfo}
        \begin{itemize}
          \item<5-> Contains information about the position in the source code (file name, line and column number)
        \end{itemize}
      \end{itemize}
    \end{column}
    \begin{column}{0.5\textwidth}
        \onslide*<1>{\listFrameDescr[highlightlines={118-122}]{}}
        \onslide*<2>{\listFrameDescr[highlightlines={120}]{}}
        \onslide*<3>{\listFrameDescr[highlightlines={121}]{}}
        \onslide*<4->{\listFrameDescr[highlightlines={122}]{}}
        \medskip
        \onslide*<1-3>{\listDebugInfo{}}
        \onslide*<4>{\listDebugInfo[highlightlines={38,49}]{}}
        \onslide*<5>{\listDebugInfo[highlightlines={39-46}]{}}
    \end{column}
  \end{columns}
\end{frame}

\begin{frame}{Backtraces}{Scanning the stack with \typename{frame\_descr}}
  \begin{columns}[c]
    \begin{column}{0.5\textwidth}
      \begin{itemize}
        \item Given the return address of the code that raises the exception by calling \funcname{caml\_raise\_exn} (\funcarg{pc} argument of \funcname{caml\_stash\_backtrace})
        \item And the position of the stack pointer (\funcarg{sp} argument of \funcname{caml\_stash\_backtrace})
        \item The frame size given by \typename{frame\_descr}.\funcarg{frame\_size}, allows to find the end of the previous frame
        \item And the return address of the caller of this function
        \bigskip
        \onslide<3->{
        \item Note that traps (here the reddish \funcname{ExnA}, \funcname{ExnB} and \funcname{ExnC} blocks) belong to the frame that installed them, and so the frame size changes when traps are removed
        }
      \end{itemize}
    \end{column}
    \begin{column}{0.5\textwidth}
      \raggedleft
      \onslide*<1>{\providecommand\step{2}\input{diagrams/backtrace/backtrace.tex}}%
      \onslide*<2>{\providecommand\step{1}\input{diagrams/backtrace/scanstack.tex}}%
      \onslide*<3>{\providecommand\step{2}\input{diagrams/backtrace/scanstack.tex}}%
      \onslide*<4>{\providecommand\step{3}\input{diagrams/backtrace/scanstack.tex}}%
      \onslide*<5>{\providecommand\step{4}\input{diagrams/backtrace/scanstack.tex}}%
      \onslide*<6>{\providecommand\step{5}\input{diagrams/backtrace/scanstack.tex}}%
    \end{column}
  \end{columns}
\end{frame}

% Duplicate of previous-previous slide
\begin{frame}{Backtraces}{Continuing on \funcname{caml\_stash\_backtrace}}
  \begin{columns}[c]
    \begin{column}{0.5\textwidth}
      \minipage[c][0.75\textheight][s]{\columnwidth}
      \begin{itemize}
          \color{gray}
        \item A C function provided by the runtime (specific to native code)
        \item It scans the stack, between the stack pointer at the raising point up to the next trap, in order to collect the names of the detected function frames
        \item It uses the same technique and information as the Garbage Collector (GC) uses to scan the values live on the stack
      \end{itemize}
      \begin{itemize}
        \item For each function frame detected, store a pointer to the \typename{frame\_descr} in the \localname{domain\_state->backtrace\_buffer} array, effectively constructing the backtrace
      \end{itemize}
      \onslide<2->{
        \begin{itemize}
            \smallskip
          \item Constructed backtrace:
            \funcname{definitely\_raise},
            \onslide<3->{\funcname{probably\_raise}}
        \end{itemize}
      }
      \vfill
      \mintocaml[firstline=9,lastline=13,highlightlines=12]{../src/backtrace/backtrace.ml}
      \endminipage
    \end{column}
    \begin{column}{0.5\textwidth}
      \raggedleft
      \onslide*<1>{\listStashBacktrace[highlightlines={114-115}]{}}
      \onslide*<2>{\providecommand\step{3}\input{diagrams/backtrace/backtrace.tex}}%
      \onslide*<3>{\providecommand\step{4}\input{diagrams/backtrace/backtrace.tex}}%
    \end{column}
  \end{columns}
\end{frame}

\begin{frame}{Backtraces}{Back to \funcname{caml\_raise\_exn}}
  \begin{columns}[c]
    \begin{column}{0.5\textwidth}
      \minipage[c][0.66\textheight][s]{\columnwidth}
      \begin{itemize}\color{gray}
        \item Checks wheteher exceptions backtrace should be recorded, by checking the \funcarg{domain\_state->backtrace\_active} value
        \item Switches to the C stack to be able to execute C code
        \item Calls \funcname{caml\_stash\_backtrace}, to collect the backtrace up to the next trap
      \end{itemize}
      \onslide*<2->{
        \begin{itemize}
          \item<2-> Executes the exception handler in \funcname{probably\_raise} with \funcname{RESTORE\_EXN\_HANDLER\_OCAML}
            \begin{itemize}
              \item Set \regname{RSP} to \localname{domain\_state->exn\_handler} value
              \item Restore \localname{domain\_state->exn\_handler} from trap
              \item Jump to the handler code with \funcname{ret}
            \end{itemize}
        \end{itemize}
      }
      \vfill
      \onslide*<1>{\mintocaml[firstline=9,lastline=13,highlightlines=12]{../src/backtrace/backtrace.ml}}
      \onslide*<2->{\mintocaml[firstline=15,lastline=21,highlightlines={19-21}]{../src/backtrace/backtrace.ml}}
      \endminipage
    \end{column}
    \begin{column}{0.5\textwidth}
      \centering
      \onslide*<1>{
        \mintc[firstline=190,lastline=192]{../ocaml/runtime/amd64.S}
        \mintc[firstline=234,lastline=237]{../ocaml/runtime/amd64.S}
      }
      \onslide*<2>{
        \mintc[firstline=190,lastline=192,highlightlines={191-192}]{../ocaml/runtime/amd64.S}
        \mintc[firstline=234,lastline=237,highlightlines={235-237}]{../ocaml/runtime/amd64.S}
      }
      \onslide*<1>{\listCamlRaiseExn[highlightlines=720]{}}
      \onslide*<2>{\listCamlRaiseExn[highlightlines=722]{}}
      \onslide*<3->{\providecommand\step{5}\input{diagrams/backtrace/backtrace.tex}}
    \end{column}
  \end{columns}
\end{frame}

\begin{frame}{Backtraces}{\funcname{caml\_probably\_raise}'s handler doesn't catch \typename{ExnC}}
  \begin{columns}[c]
    \begin{column}{0.45\textwidth}
      \minipage[c][0.75\textheight][s]{\columnwidth}
      \begin{itemize}
        \item<1-> \funcname{match \dots with}\footnotemark pattern matching can also match exceptions
        \item<1-> The CMM of \funcname{probably\_raise} shows that this \funcname{match \dots with} is implemented with a pattern matching and a \funcname{try \dots with}
        \item<1-> But this one only matches on \typename{ExnA} exception
        \item<2-> Since the pattern matching fails to catch the exception, \funcname{reraise} is called with our current \typename{ExnC} exception
        \item<3-> Disassembly shows that \funcname{reraise} is actually a call to \funcname{caml\_reraise\_exn}, which is also an assembly function provided by the runtime
      \end{itemize}
      \vfill
      \mintocaml[firstline=15,lastline=21,highlightlines={18-21}]{../src/backtrace/backtrace.ml}
      \endminipage
    \end{column}
    \begin{column}{0.55\textwidth}
      \centering
      \onslide*<1>{\mintcmm[highlightlines={10-18}]{../src/backtrace/camlBacktrace\_\_probably\_raise\_351.cmm}}
      \onslide*<2>{\listcmm[highlightlines=18]{../src/backtrace/camlBacktrace\_\_probably\_raise_351.cmm}{CMM of probably\_raise}}
      \onslide*<3>{\mintobjdump[firstline=5,lastline=35,highlightlines=31]{../src/backtrace/camlBacktrace\_\_probably\_raise_351.objdump}}
    \end{column}
  \end{columns}
  \only<1>{\footnotetext[1]{https://blog.janestreet.com/pattern-matching-and-exception-handling-unite/}}
\end{frame}

\newcommand\listCamlReraiseExn[1][]{
  \listasm[firstline=727,lastline=734,#1]{../ocaml/runtime/amd64.S}{runtime/amd64.S:727}}

\begin{frame}{Backtraces}{Introducing \funcname{caml\_reraise\_exn}}
  \begin{columns}[c]
    \begin{column}{0.5\textwidth}
      \minipage[c][0.66\textheight][s]{\columnwidth}
      \begin{itemize}
        \item<1-> The exception have to be reraised to the next handler
        \item<2-> First, checks if the backtrace should be collected with \funcarg{domain\_state->backtrace\_active}
        \only<3>{
        \begin{itemize}
            \item If not, behaves like a \funcname{raise\_notrace} with \funcname{RESTORE\_EXN\_HANDLER\_OCAML}
        \end{itemize}
        }
        \item<4-> Jumps into \funcname{caml\_raise\_exn}
        \item<5-> Calls \funcname{caml\_stash\_backtrace} again, to scan the next part of the stack: from the trap just pop-ed to the next trap
      \end{itemize}
      \vfill
      \mintocaml[firstline=15,lastline=21,highlightlines={18-21}]{../src/backtrace/backtrace.ml}
      \endminipage
    \end{column}
    \begin{column}{0.5\textwidth}
      \raggedleft
      \onslide*<1>{\listCamlReraiseExn{}}
      \onslide*<2>{\listCamlReraiseExn[highlightlines=729]{}}
      \onslide*<3>{
        \mintc[firstline=190,lastline=192,highlightlines={191-192}]{../ocaml/runtime/amd64.S}
        \mintc[firstline=234,lastline=237,highlightlines={235-237}]{../ocaml/runtime/amd64.S}
        \medskip
        \listCamlReraiseExn[highlightlines=731]{}
      }
      \onslide*<4-5>{
        % TODO refactor with \listCamlReraiseExn
        \mintasm[firstline=727,lastline=734,highlightlines=730]{../ocaml/runtime/amd64.S}
        \medskip
      }
      \onslide*<4>{\listCamlRaiseExn[highlightlines=711]{}}
      \onslide*<5>{\listCamlRaiseExn[highlightlines=720]{}}
    \end{column}
  \end{columns}
\end{frame}

\begin{frame}{Backtraces}{Backtrace collection continues}
  \begin{columns}[c]
    \begin{column}{0.55\textwidth}
      \minipage[c][0.75\textheight][s]{\columnwidth}
      \begin{itemize}
        \item<1-> \funcname{caml\_stash\_backtrace} scans the older part of the stack, up to the next trap
        \item<6-> Controls jumps to the nested exception handler in \funcname{maybe\_raise}, which matches only on \typename{ExnB}
        \item<7-> \funcname{caml\_reraise\_exn} is called again, backtrace collection resumes with \funcname{caml\_stash\_backtrace}
      \end{itemize}
      \medskip
      \begin{itemize}
        \item<2-> Constructed backtrace:
        \begin{itemize}
          \item <2-> \funcname{definitely\_raise}
          \item <2-> \funcname{probably\_raise}
          \item <3-> \funcname{probably\_raise} (re-raise)
          \item <4-> \funcname{maybe\_raise}
          \item <5-> \funcname{main}
          \item <8-> \funcname{main} (re-raise)
        \end{itemize}
      \end{itemize}
      \vfill
      \onslide*<1-5>{\mintocaml[firstline=15,lastline=21]{../src/backtrace/backtrace.ml}}
      \onslide*<6>{\mintocaml[firstline=28,lastline=34,highlightlines=31]{../src/backtrace/backtrace.ml}}
      \onslide*<7->{\mintocaml[firstline=28,lastline=34,highlightlines=33]{../src/backtrace/backtrace.ml}}
      \endminipage
    \end{column}
    \begin{column}{0.45\textwidth}
      \raggedleft
      \onslide*<1>{\providecommand\step{6}\input{diagrams/backtrace/backtrace.tex}}%
      \onslide*<2>{\providecommand\step{6}\input{diagrams/backtrace/backtrace.tex}}%
      \onslide*<3>{\providecommand\step{7}\input{diagrams/backtrace/backtrace.tex}}%
      \onslide*<4>{\providecommand\step{8}\input{diagrams/backtrace/backtrace.tex}}%
      \onslide*<5>{\providecommand\step{9}\input{diagrams/backtrace/backtrace.tex}}%
      \onslide*<6>{\providecommand\step{10}\input{diagrams/backtrace/backtrace.tex}}%
      \onslide*<7>{\providecommand\step{11}\input{diagrams/backtrace/backtrace.tex}}%
      \onslide*<8>{\providecommand\step{12}\input{diagrams/backtrace/backtrace.tex}}%
      \onslide*<9>{\providecommand\step{13}\input{diagrams/backtrace/backtrace.tex}}%
    \end{column}
  \end{columns}
\end{frame}

\begin{frame}{Backtraces}{Printing the backtrace}
  \begin{columns}[c]
    \begin{column}{0.45\textwidth}
      \minipage[c][0.66\textheight][s]{\columnwidth}
      \begin{itemize}
        \item<1-> The exception backtrace can now be printed to \funcarg{stdout} with \funcname{Printexc.print_backtrace}
        \item<2-> First the exception backtrace is retrieved into an OCaml array with \funcname{get_raw_backtrace }
        \item<3-> Which is implemented as the \funcname{caml_get_exception_raw_backtrace} C function in the runtime
        \item<4-> And finally printed with \funcname{print_exception_backtrace}
      \end{itemize}
      \vfill
      \mintocaml[firstline=28,lastline=34,highlightlines=33]{../src/backtrace/backtrace.ml}
      \endminipage
    \end{column}
    \begin{column}{0.55\textwidth}
      \onslide*<1,2>{
        \mintocaml[firstline=100,lastline=101]{../ocaml/stdlib/printexc.ml}
      }
      \only<1>{
        \listocaml[firstline=170,lastline=172]{../ocaml/stdlib/printexc.ml}{stdlib/printexc.ml:170}
      }
      \only<2>{
        \listocaml[firstline=170,lastline=172,highlightlines=172]{../ocaml/stdlib/printexc.ml}{stdlib/printexc.ml:170}
      }
      \only<3>{
        \mintc[firstline=154,lastline=156]{../ocaml/runtime/backtrace.c}
        \listc[firstline=171,lastline=192,highlightlines={184-187}]{../ocaml/runtime/backtrace.c}{runtime/backtrace.c:154}
      }
      \only<4>{
        \listocaml[firstline=155,lastline=165]{../ocaml/stdlib/printexc.ml}{stdlib/printexc.ml:55}
      }
    \end{column}
  \end{columns}
\end{frame}


\subsection{The multiple kinds of raise}
\frameSubsection{}{}

\begin{frame}{Backtraces}{When backtraces are not needed}
  \begin{columns}[c]
    \begin{column}{0.45\textwidth}
      \minipage[c][0.66\textheight][s]{\columnwidth}
      \begin{itemize}
        \item<1-> What if the backtrace for an exception is never needed?
        \item<2-> It's possible to raise the exception explicitly with \funcname{raise\_notrace} to make raising as cheap as possible
        \item<3-> An example of use in the \funcname{dynlink} library of the OCaml compiler
      \end{itemize}
      \vfill
      \onslide<3->{
      \listocaml[firstline=205,lastline=209,highlightlines=207]{../ocaml/otherlibs/dynlink/dynlink\_compilerlibs/misc.ml}{otherlibs/dynlink/dynlink\_compilerlibs/misc.ml:205}
      }
      \endminipage
    \end{column}
    \begin{column}{0.55\textwidth}
    \only<4>{
      \mintcmm[highlightlines=3]{../src/notrace/camlNotrace\_\_fun_347.cmm}
      \listcmm{../src/notrace/camlNotrace\_\_all\_somes\_327.cmm}{all\_somes}
    }
    \onslide*<5->{
      \listobjdump[highlightlines={6-9}]{../src/notrace/camlNotrace\_\_fun_347.objdump}{anonymous function from \funcname{all\_somes}}
    }
    \end{column}
  \end{columns}
\end{frame}

\begin{frame}{Backtraces}{Summary table of the multiple kinds of raise}
  \begin{itemize}
    \item When debugging information is enabled and if the backtrace of an exception isn't needed, it's possible to raise with \funcname{raise\_notrace} for performance
    \item When debugging information is \emph{not} enabled, the compiler optimises \funcname{raise} calls to inline assembly (same effect as using \funcname{raise\_notrace})
  \end{itemize}
  \bigskip
  \centering
  \begin{tabular}{| l | c | c | c | c |}
    \hline
    \parbox{10em}{Debug information \\ (\funcarg{-g} flag)}  & \multicolumn{2}{|c|}{Disabled}                                     & \multicolumn{2}{|c|}{Enabled} \\
    \hline
    Raise kind                                               & \funcname{raise} & \funcname{raise\_notrace}                       & \funcname{raise} & \funcname{raise\_notrace} \\
    \hline
    Compilation                                              & \parbox{8em}{inline assembly\\ (optimization)} & inline assembly   & \funcname{caml\_raise\_exn} & inline assembly \\
    \hline
  \end{tabular}
\end{frame}


%
% Takeaway #5
%
\subsection*{Takeaway \#5}
\frameSubsectionTakeaway{}
\againframe<5>{takeaway}

    \end{column}
  \end{columns}
\end{frame}

\begin{frame}{Backtraces}{But before that, we need more fields from \typename{caml\_domain\_state}}
  \begin{columns}
    \begin{column}{0.5\textwidth}
      \begin{itemize}
        \item Remember \typename{caml\_domain\_state} the per-domain runtime state?
        \item Some other members are of interest to us now\dots
        \item \localname{backtrace\_active} is used to know if the runtime should collect backtraces
        \item \localname{backtrace\_active} can be set by
          \begin{itemize}
            \item The \funcarg{b} flag in the \funcname{OCAMLRUNPARAM} environment variable
            \item \mintinline{ocaml}{Printexc.record_backtrace : bool -> unit} \footnotemark
          \end{itemize}
      \end{itemize}
    \end{column}
    \begin{column}{0.5\textwidth}
      \begin{listing}
        \mintc[highlightlines={9-11}]{src/ocaml/domain\_state\_backtrace.h}
        \caption{runtime/caml/domain\_state.h}
      \end{listing}
    \end{column}
  \end{columns}
  \footnotetext[1]{https://v2.ocaml.org/api/Printexc.html}
\end{frame}

\newcommand\listCamlRaiseExn[1][]{
  \listasm[firstline=702,lastline=725,#1]{../ocaml/runtime/amd64.S}{runtime/amd64.S:702}}

\begin{frame}{Backtraces}{Walk through \funcname{caml\_raise\_exn}}
  \begin{columns}[T]
    \begin{column}{0.5\textwidth}
      \begin{itemize}
        \item<1-> First checks whether exceptions backtrace should be recorded
        \item<1-> By checking the \funcarg{domain\_state->backtrace\_active} value
        \item<2-> If not, acts as \funcname{raise\_notrace} with \funcname{RESTORE\_EXN\_HANDLER\_OCAML}
          \begin{itemize}
            \item Set \regname{RSP} to \localname{domain\_state->exn\_handler} value
            \item Restore \localname{domain\_state->exn\_handler} from trap
            \item Jump with \funcname{ret} (same effect as using \funcname{pop} and \funcname{jmp})
          \end{itemize}
        \item<3-> Otherwise, switches to the C stack to be able to execute C code
        \item<4-> Calls \funcname{caml\_stash\_backtrace}, a C function provided by the runtime\ignorespaces
          \onslide<6->{, with
            \begin{itemize}
              \item \funcarg{exn}: exception value
              \item \funcarg{pc}: code address of the raising point
              \item \funcarg{sp}: stack address of the raising function
              \item \funcarg{trapsp}: stack address of the next handler
            \end{itemize}
          }
      \end{itemize}
      \bigskip
      \mintocaml[firstline=9,lastline=13,highlightlines=12]{../src/backtrace/backtrace.ml}
    \end{column}
    \begin{column}{0.5\textwidth}
      \raggedleft
      \onslide*<1,3-4>{
        \mintc[firstline=190,lastline=192]{../ocaml/runtime/amd64.S}
        \mintc[firstline=234,lastline=237]{../ocaml/runtime/amd64.S}
      }
      \onslide*<2>{
        \mintc[firstline=190,lastline=192,highlightlines={191-192}]{../ocaml/runtime/amd64.S}
        \mintc[firstline=234,lastline=237,highlightlines={235-237}]{../ocaml/runtime/amd64.S}
      }
      \onslide*<1>{\listCamlRaiseExn[highlightlines=705]{}}%
      \onslide*<2>{\listCamlRaiseExn[highlightlines={707-708}]{}}%
      \onslide*<3>{\listCamlRaiseExn[highlightlines={712-714}]{}}%
      \onslide*<4>{\listCamlRaiseExn[highlightlines={715-720}]{}}%
      \onslide*<5>{\providecommand\step{1}%
% Backtraces
%
\subsection{One advantage of raising with debugging information}
\frameSubsection{}{}

\begin{frame}{Backtraces}{Debugging information \& source code}
  \begin{columns}[c]
    \begin{column}{0.5\textwidth}
      \begin{itemize}
        \item Raising an exception is fun and games, but how do we know where the exception originated? $\rightarrow$ With the backtrace associated with the exception!
        \item In order to get a backtrace from the point where an exception was raised, it's necessary to compile with debugging information enabled (\funcarg{-g} flag for the compiler)
        \item Same example as in the `Nested handlers' section
      \end{itemize}
    \end{column}
    \begin{column}{0.5\textwidth}
      \listocaml[firstline=9,lastline=34]{../src/backtrace/backtrace.ml}{backtrace/backtrace.ml}
    \end{column}
  \end{columns}
\end{frame}

\begin{frame}{Backtraces}{CMM and disassembly of \funcname{definitely\_raise}}
  \begin{columns}[c]
    \begin{column}{0.5\textwidth}
      \minipage[c][0.66\textheight][s]{\columnwidth}
      \begin{itemize}
        \item<1-> An effect of compiling with debugging information (\funcarg{-g}): the CMM no longer uses \funcname{raise\_notrace} to raise the exception but \funcname{raise} instead
        \item<2-> Disassembly shows that CMM \funcname{raise} is actually a call to \funcname{caml\_raise\_exn}, a function in assembler provided by the runtime
      \end{itemize}
      \vfill
      \mintocaml[firstline=9,lastline=13,highlightlines=12]{../src/backtrace/backtrace.ml}
      \endminipage
    \end{column}
    \begin{column}{0.5\textwidth}
      \onslide*<1>{
        \mintcmm[highlightlines=5]{../src/backtrace/camlBacktrace\_\_definitely\_raise\_347.cmm}
      }
      \onslide*<2>{
        \mintobjdump[firstline=2,highlightlines=14]{../src/backtrace/camlBacktrace\_\_definitely\_raise\_347.objdump}
      }
    \end{column}
  \end{columns}
\end{frame}

\begin{frame}{Backtraces}{Introducing \funcname{caml\_raise\_exn}}
  \begin{columns}[c]
    \begin{column}{0.5\textwidth}
      \minipage[c][0.66\textheight][s]{\columnwidth}
      \onslide*<1>{
        \begin{itemize}
          \item Raising the exception is performed using \funcname{caml\_raise\_exn} which is
            \begin{itemize}
              \item An assembly function provided by the runtime
              \item Architecture specific
            \end{itemize}
          \item Let's see what it does differently from \funcname{raise\_notrace}\dots
          \item We are about to raise \typename{ExnC} with \funcname{caml\_raise\_exn} from \funcname{definitely\_raise}
        \end{itemize}
      }
      \onslide*<2>{
        \mintobjdump[firstline=2,highlightlines=14]{../src/backtrace/camlBacktrace\_\_definitely\_raise\_347.objdump}
      }
      \vfill
      \mintocaml[firstline=9,lastline=13,highlightlines=12]{../src/backtrace/backtrace.ml}
      \endminipage
    \end{column}
    \begin{column}{0.5\textwidth}
      \centering
      \providecommand\step{1}\input{diagrams/backtrace/backtrace.tex}
    \end{column}
  \end{columns}
\end{frame}

\begin{frame}{Backtraces}{But before that, we need more fields from \typename{caml\_domain\_state}}
  \begin{columns}
    \begin{column}{0.5\textwidth}
      \begin{itemize}
        \item Remember \typename{caml\_domain\_state} the per-domain runtime state?
        \item Some other members are of interest to us now\dots
        \item \localname{backtrace\_active} is used to know if the runtime should collect backtraces
        \item \localname{backtrace\_active} can be set by
          \begin{itemize}
            \item The \funcarg{b} flag in the \funcname{OCAMLRUNPARAM} environment variable
            \item \mintinline{ocaml}{Printexc.record_backtrace : bool -> unit} \footnotemark
          \end{itemize}
      \end{itemize}
    \end{column}
    \begin{column}{0.5\textwidth}
      \begin{listing}
        \mintc[highlightlines={9-11}]{src/ocaml/domain\_state\_backtrace.h}
        \caption{runtime/caml/domain\_state.h}
      \end{listing}
    \end{column}
  \end{columns}
  \footnotetext[1]{https://v2.ocaml.org/api/Printexc.html}
\end{frame}

\newcommand\listCamlRaiseExn[1][]{
  \listasm[firstline=702,lastline=725,#1]{../ocaml/runtime/amd64.S}{runtime/amd64.S:702}}

\begin{frame}{Backtraces}{Walk through \funcname{caml\_raise\_exn}}
  \begin{columns}[T]
    \begin{column}{0.5\textwidth}
      \begin{itemize}
        \item<1-> First checks whether exceptions backtrace should be recorded
        \item<1-> By checking the \funcarg{domain\_state->backtrace\_active} value
        \item<2-> If not, acts as \funcname{raise\_notrace} with \funcname{RESTORE\_EXN\_HANDLER\_OCAML}
          \begin{itemize}
            \item Set \regname{RSP} to \localname{domain\_state->exn\_handler} value
            \item Restore \localname{domain\_state->exn\_handler} from trap
            \item Jump with \funcname{ret} (same effect as using \funcname{pop} and \funcname{jmp})
          \end{itemize}
        \item<3-> Otherwise, switches to the C stack to be able to execute C code
        \item<4-> Calls \funcname{caml\_stash\_backtrace}, a C function provided by the runtime\ignorespaces
          \onslide<6->{, with
            \begin{itemize}
              \item \funcarg{exn}: exception value
              \item \funcarg{pc}: code address of the raising point
              \item \funcarg{sp}: stack address of the raising function
              \item \funcarg{trapsp}: stack address of the next handler
            \end{itemize}
          }
      \end{itemize}
      \bigskip
      \mintocaml[firstline=9,lastline=13,highlightlines=12]{../src/backtrace/backtrace.ml}
    \end{column}
    \begin{column}{0.5\textwidth}
      \raggedleft
      \onslide*<1,3-4>{
        \mintc[firstline=190,lastline=192]{../ocaml/runtime/amd64.S}
        \mintc[firstline=234,lastline=237]{../ocaml/runtime/amd64.S}
      }
      \onslide*<2>{
        \mintc[firstline=190,lastline=192,highlightlines={191-192}]{../ocaml/runtime/amd64.S}
        \mintc[firstline=234,lastline=237,highlightlines={235-237}]{../ocaml/runtime/amd64.S}
      }
      \onslide*<1>{\listCamlRaiseExn[highlightlines=705]{}}%
      \onslide*<2>{\listCamlRaiseExn[highlightlines={707-708}]{}}%
      \onslide*<3>{\listCamlRaiseExn[highlightlines={712-714}]{}}%
      \onslide*<4>{\listCamlRaiseExn[highlightlines={715-720}]{}}%
      \onslide*<5>{\providecommand\step{1}\input{diagrams/backtrace/backtrace.tex}}%
      \onslide*<6>{\providecommand\step{2}\input{diagrams/backtrace/backtrace.tex}}%
    \end{column}
  \end{columns}
\end{frame}

\newcommand\listStashBacktrace[1][]{
  \listc[firstline=91,lastline=120,#1]{../ocaml/runtime/backtrace\_nat.c}{runtime/backtrace\_nat.c:91}}

\begin{frame}{Backtraces}{Introducing \funcname{caml\_stash\_backtrace}}
  \begin{columns}[c]
    \begin{column}{0.5\textwidth}
      \minipage[c][0.66\textheight][s]{\columnwidth}
      \begin{itemize}
        \item<1-> A C function provided by the runtime (specific to native code)
        \item<2-> It scans the stack, between the stack pointer at the raising point up to the next trap, in order to collect the names of the detected function frames
          \onslide*<2>{
            \begin{itemize}
              \item And more information, like line number, column number...
            \end{itemize}
          }
        \item<3-> It uses the same technique and information as the Garbage Collector (GC) uses to scan the values live on the stack
          \onslide*<4>{
            \begin{itemize}
              \item \typename{frame\_descr} are at the core of stack unwinding
            \end{itemize}
          }
      \end{itemize}
      \vfill
      \mintocaml[firstline=9,lastline=13,highlightlines=12]{../src/backtrace/backtrace.ml}
      \endminipage
    \end{column}
    \begin{column}{0.5\textwidth}
      \onslide*<1>{\listStashBacktrace[highlightlines=91]{}}
      \onslide*<2>{\listStashBacktrace[highlightlines={91,117-118}]{}}
      \onslide*<3->{\listStashBacktrace[highlightlines={94,106,109-110}]{}}
    \end{column}
  \end{columns}
\end{frame}

\newcommand\listFrameDescr[1][]{
  \listocaml[firstline=111,lastline=124,#1]{../ocaml/asmcomp/emitaux.ml}{asmcomp/emitaux.ml:111}}
\newcommand\listDebugInfo[1][]{
  \listocaml[firstline=38,lastline=49,#1]{../ocaml/lambda/debuginfo.mli}{lambda/debuginfo.mli:38}}

\begin{frame}{Backtraces}{\typename{frame\_descr} and \typename{frame\_debuginfo} in a nutshell}
  \begin{columns}[c]
    \begin{column}{0.5\textwidth}
      \begin{itemize}
        \item<1-> For every return address in an OCaml program, there is a associated \typename{frame\_descr}
        \item<2-> A \typename{frame\_descr} specifies
        \begin{itemize}
          \item<2-> The size of the function stack frame
          \item<3-> Where live values are on the stack (so that the GC can mark them as live and prevent from being swept)
        \end{itemize}
        \item<4-> When compiled with debug information, \typename{frame\_descr} will be linked to a \typename{frame\_debuginfo}
        \begin{itemize}
          \item<5-> Contains information about the position in the source code (file name, line and column number)
        \end{itemize}
      \end{itemize}
    \end{column}
    \begin{column}{0.5\textwidth}
        \onslide*<1>{\listFrameDescr[highlightlines={118-122}]{}}
        \onslide*<2>{\listFrameDescr[highlightlines={120}]{}}
        \onslide*<3>{\listFrameDescr[highlightlines={121}]{}}
        \onslide*<4->{\listFrameDescr[highlightlines={122}]{}}
        \medskip
        \onslide*<1-3>{\listDebugInfo{}}
        \onslide*<4>{\listDebugInfo[highlightlines={38,49}]{}}
        \onslide*<5>{\listDebugInfo[highlightlines={39-46}]{}}
    \end{column}
  \end{columns}
\end{frame}

\begin{frame}{Backtraces}{Scanning the stack with \typename{frame\_descr}}
  \begin{columns}[c]
    \begin{column}{0.5\textwidth}
      \begin{itemize}
        \item Given the return address of the code that raises the exception by calling \funcname{caml\_raise\_exn} (\funcarg{pc} argument of \funcname{caml\_stash\_backtrace})
        \item And the position of the stack pointer (\funcarg{sp} argument of \funcname{caml\_stash\_backtrace})
        \item The frame size given by \typename{frame\_descr}.\funcarg{frame\_size}, allows to find the end of the previous frame
        \item And the return address of the caller of this function
        \bigskip
        \onslide<3->{
        \item Note that traps (here the reddish \funcname{ExnA}, \funcname{ExnB} and \funcname{ExnC} blocks) belong to the frame that installed them, and so the frame size changes when traps are removed
        }
      \end{itemize}
    \end{column}
    \begin{column}{0.5\textwidth}
      \raggedleft
      \onslide*<1>{\providecommand\step{2}\input{diagrams/backtrace/backtrace.tex}}%
      \onslide*<2>{\providecommand\step{1}\input{diagrams/backtrace/scanstack.tex}}%
      \onslide*<3>{\providecommand\step{2}\input{diagrams/backtrace/scanstack.tex}}%
      \onslide*<4>{\providecommand\step{3}\input{diagrams/backtrace/scanstack.tex}}%
      \onslide*<5>{\providecommand\step{4}\input{diagrams/backtrace/scanstack.tex}}%
      \onslide*<6>{\providecommand\step{5}\input{diagrams/backtrace/scanstack.tex}}%
    \end{column}
  \end{columns}
\end{frame}

% Duplicate of previous-previous slide
\begin{frame}{Backtraces}{Continuing on \funcname{caml\_stash\_backtrace}}
  \begin{columns}[c]
    \begin{column}{0.5\textwidth}
      \minipage[c][0.75\textheight][s]{\columnwidth}
      \begin{itemize}
          \color{gray}
        \item A C function provided by the runtime (specific to native code)
        \item It scans the stack, between the stack pointer at the raising point up to the next trap, in order to collect the names of the detected function frames
        \item It uses the same technique and information as the Garbage Collector (GC) uses to scan the values live on the stack
      \end{itemize}
      \begin{itemize}
        \item For each function frame detected, store a pointer to the \typename{frame\_descr} in the \localname{domain\_state->backtrace\_buffer} array, effectively constructing the backtrace
      \end{itemize}
      \onslide<2->{
        \begin{itemize}
            \smallskip
          \item Constructed backtrace:
            \funcname{definitely\_raise},
            \onslide<3->{\funcname{probably\_raise}}
        \end{itemize}
      }
      \vfill
      \mintocaml[firstline=9,lastline=13,highlightlines=12]{../src/backtrace/backtrace.ml}
      \endminipage
    \end{column}
    \begin{column}{0.5\textwidth}
      \raggedleft
      \onslide*<1>{\listStashBacktrace[highlightlines={114-115}]{}}
      \onslide*<2>{\providecommand\step{3}\input{diagrams/backtrace/backtrace.tex}}%
      \onslide*<3>{\providecommand\step{4}\input{diagrams/backtrace/backtrace.tex}}%
    \end{column}
  \end{columns}
\end{frame}

\begin{frame}{Backtraces}{Back to \funcname{caml\_raise\_exn}}
  \begin{columns}[c]
    \begin{column}{0.5\textwidth}
      \minipage[c][0.66\textheight][s]{\columnwidth}
      \begin{itemize}\color{gray}
        \item Checks wheteher exceptions backtrace should be recorded, by checking the \funcarg{domain\_state->backtrace\_active} value
        \item Switches to the C stack to be able to execute C code
        \item Calls \funcname{caml\_stash\_backtrace}, to collect the backtrace up to the next trap
      \end{itemize}
      \onslide*<2->{
        \begin{itemize}
          \item<2-> Executes the exception handler in \funcname{probably\_raise} with \funcname{RESTORE\_EXN\_HANDLER\_OCAML}
            \begin{itemize}
              \item Set \regname{RSP} to \localname{domain\_state->exn\_handler} value
              \item Restore \localname{domain\_state->exn\_handler} from trap
              \item Jump to the handler code with \funcname{ret}
            \end{itemize}
        \end{itemize}
      }
      \vfill
      \onslide*<1>{\mintocaml[firstline=9,lastline=13,highlightlines=12]{../src/backtrace/backtrace.ml}}
      \onslide*<2->{\mintocaml[firstline=15,lastline=21,highlightlines={19-21}]{../src/backtrace/backtrace.ml}}
      \endminipage
    \end{column}
    \begin{column}{0.5\textwidth}
      \centering
      \onslide*<1>{
        \mintc[firstline=190,lastline=192]{../ocaml/runtime/amd64.S}
        \mintc[firstline=234,lastline=237]{../ocaml/runtime/amd64.S}
      }
      \onslide*<2>{
        \mintc[firstline=190,lastline=192,highlightlines={191-192}]{../ocaml/runtime/amd64.S}
        \mintc[firstline=234,lastline=237,highlightlines={235-237}]{../ocaml/runtime/amd64.S}
      }
      \onslide*<1>{\listCamlRaiseExn[highlightlines=720]{}}
      \onslide*<2>{\listCamlRaiseExn[highlightlines=722]{}}
      \onslide*<3->{\providecommand\step{5}\input{diagrams/backtrace/backtrace.tex}}
    \end{column}
  \end{columns}
\end{frame}

\begin{frame}{Backtraces}{\funcname{caml\_probably\_raise}'s handler doesn't catch \typename{ExnC}}
  \begin{columns}[c]
    \begin{column}{0.45\textwidth}
      \minipage[c][0.75\textheight][s]{\columnwidth}
      \begin{itemize}
        \item<1-> \funcname{match \dots with}\footnotemark pattern matching can also match exceptions
        \item<1-> The CMM of \funcname{probably\_raise} shows that this \funcname{match \dots with} is implemented with a pattern matching and a \funcname{try \dots with}
        \item<1-> But this one only matches on \typename{ExnA} exception
        \item<2-> Since the pattern matching fails to catch the exception, \funcname{reraise} is called with our current \typename{ExnC} exception
        \item<3-> Disassembly shows that \funcname{reraise} is actually a call to \funcname{caml\_reraise\_exn}, which is also an assembly function provided by the runtime
      \end{itemize}
      \vfill
      \mintocaml[firstline=15,lastline=21,highlightlines={18-21}]{../src/backtrace/backtrace.ml}
      \endminipage
    \end{column}
    \begin{column}{0.55\textwidth}
      \centering
      \onslide*<1>{\mintcmm[highlightlines={10-18}]{../src/backtrace/camlBacktrace\_\_probably\_raise\_351.cmm}}
      \onslide*<2>{\listcmm[highlightlines=18]{../src/backtrace/camlBacktrace\_\_probably\_raise_351.cmm}{CMM of probably\_raise}}
      \onslide*<3>{\mintobjdump[firstline=5,lastline=35,highlightlines=31]{../src/backtrace/camlBacktrace\_\_probably\_raise_351.objdump}}
    \end{column}
  \end{columns}
  \only<1>{\footnotetext[1]{https://blog.janestreet.com/pattern-matching-and-exception-handling-unite/}}
\end{frame}

\newcommand\listCamlReraiseExn[1][]{
  \listasm[firstline=727,lastline=734,#1]{../ocaml/runtime/amd64.S}{runtime/amd64.S:727}}

\begin{frame}{Backtraces}{Introducing \funcname{caml\_reraise\_exn}}
  \begin{columns}[c]
    \begin{column}{0.5\textwidth}
      \minipage[c][0.66\textheight][s]{\columnwidth}
      \begin{itemize}
        \item<1-> The exception have to be reraised to the next handler
        \item<2-> First, checks if the backtrace should be collected with \funcarg{domain\_state->backtrace\_active}
        \only<3>{
        \begin{itemize}
            \item If not, behaves like a \funcname{raise\_notrace} with \funcname{RESTORE\_EXN\_HANDLER\_OCAML}
        \end{itemize}
        }
        \item<4-> Jumps into \funcname{caml\_raise\_exn}
        \item<5-> Calls \funcname{caml\_stash\_backtrace} again, to scan the next part of the stack: from the trap just pop-ed to the next trap
      \end{itemize}
      \vfill
      \mintocaml[firstline=15,lastline=21,highlightlines={18-21}]{../src/backtrace/backtrace.ml}
      \endminipage
    \end{column}
    \begin{column}{0.5\textwidth}
      \raggedleft
      \onslide*<1>{\listCamlReraiseExn{}}
      \onslide*<2>{\listCamlReraiseExn[highlightlines=729]{}}
      \onslide*<3>{
        \mintc[firstline=190,lastline=192,highlightlines={191-192}]{../ocaml/runtime/amd64.S}
        \mintc[firstline=234,lastline=237,highlightlines={235-237}]{../ocaml/runtime/amd64.S}
        \medskip
        \listCamlReraiseExn[highlightlines=731]{}
      }
      \onslide*<4-5>{
        % TODO refactor with \listCamlReraiseExn
        \mintasm[firstline=727,lastline=734,highlightlines=730]{../ocaml/runtime/amd64.S}
        \medskip
      }
      \onslide*<4>{\listCamlRaiseExn[highlightlines=711]{}}
      \onslide*<5>{\listCamlRaiseExn[highlightlines=720]{}}
    \end{column}
  \end{columns}
\end{frame}

\begin{frame}{Backtraces}{Backtrace collection continues}
  \begin{columns}[c]
    \begin{column}{0.55\textwidth}
      \minipage[c][0.75\textheight][s]{\columnwidth}
      \begin{itemize}
        \item<1-> \funcname{caml\_stash\_backtrace} scans the older part of the stack, up to the next trap
        \item<6-> Controls jumps to the nested exception handler in \funcname{maybe\_raise}, which matches only on \typename{ExnB}
        \item<7-> \funcname{caml\_reraise\_exn} is called again, backtrace collection resumes with \funcname{caml\_stash\_backtrace}
      \end{itemize}
      \medskip
      \begin{itemize}
        \item<2-> Constructed backtrace:
        \begin{itemize}
          \item <2-> \funcname{definitely\_raise}
          \item <2-> \funcname{probably\_raise}
          \item <3-> \funcname{probably\_raise} (re-raise)
          \item <4-> \funcname{maybe\_raise}
          \item <5-> \funcname{main}
          \item <8-> \funcname{main} (re-raise)
        \end{itemize}
      \end{itemize}
      \vfill
      \onslide*<1-5>{\mintocaml[firstline=15,lastline=21]{../src/backtrace/backtrace.ml}}
      \onslide*<6>{\mintocaml[firstline=28,lastline=34,highlightlines=31]{../src/backtrace/backtrace.ml}}
      \onslide*<7->{\mintocaml[firstline=28,lastline=34,highlightlines=33]{../src/backtrace/backtrace.ml}}
      \endminipage
    \end{column}
    \begin{column}{0.45\textwidth}
      \raggedleft
      \onslide*<1>{\providecommand\step{6}\input{diagrams/backtrace/backtrace.tex}}%
      \onslide*<2>{\providecommand\step{6}\input{diagrams/backtrace/backtrace.tex}}%
      \onslide*<3>{\providecommand\step{7}\input{diagrams/backtrace/backtrace.tex}}%
      \onslide*<4>{\providecommand\step{8}\input{diagrams/backtrace/backtrace.tex}}%
      \onslide*<5>{\providecommand\step{9}\input{diagrams/backtrace/backtrace.tex}}%
      \onslide*<6>{\providecommand\step{10}\input{diagrams/backtrace/backtrace.tex}}%
      \onslide*<7>{\providecommand\step{11}\input{diagrams/backtrace/backtrace.tex}}%
      \onslide*<8>{\providecommand\step{12}\input{diagrams/backtrace/backtrace.tex}}%
      \onslide*<9>{\providecommand\step{13}\input{diagrams/backtrace/backtrace.tex}}%
    \end{column}
  \end{columns}
\end{frame}

\begin{frame}{Backtraces}{Printing the backtrace}
  \begin{columns}[c]
    \begin{column}{0.45\textwidth}
      \minipage[c][0.66\textheight][s]{\columnwidth}
      \begin{itemize}
        \item<1-> The exception backtrace can now be printed to \funcarg{stdout} with \funcname{Printexc.print_backtrace}
        \item<2-> First the exception backtrace is retrieved into an OCaml array with \funcname{get_raw_backtrace }
        \item<3-> Which is implemented as the \funcname{caml_get_exception_raw_backtrace} C function in the runtime
        \item<4-> And finally printed with \funcname{print_exception_backtrace}
      \end{itemize}
      \vfill
      \mintocaml[firstline=28,lastline=34,highlightlines=33]{../src/backtrace/backtrace.ml}
      \endminipage
    \end{column}
    \begin{column}{0.55\textwidth}
      \onslide*<1,2>{
        \mintocaml[firstline=100,lastline=101]{../ocaml/stdlib/printexc.ml}
      }
      \only<1>{
        \listocaml[firstline=170,lastline=172]{../ocaml/stdlib/printexc.ml}{stdlib/printexc.ml:170}
      }
      \only<2>{
        \listocaml[firstline=170,lastline=172,highlightlines=172]{../ocaml/stdlib/printexc.ml}{stdlib/printexc.ml:170}
      }
      \only<3>{
        \mintc[firstline=154,lastline=156]{../ocaml/runtime/backtrace.c}
        \listc[firstline=171,lastline=192,highlightlines={184-187}]{../ocaml/runtime/backtrace.c}{runtime/backtrace.c:154}
      }
      \only<4>{
        \listocaml[firstline=155,lastline=165]{../ocaml/stdlib/printexc.ml}{stdlib/printexc.ml:55}
      }
    \end{column}
  \end{columns}
\end{frame}


\subsection{The multiple kinds of raise}
\frameSubsection{}{}

\begin{frame}{Backtraces}{When backtraces are not needed}
  \begin{columns}[c]
    \begin{column}{0.45\textwidth}
      \minipage[c][0.66\textheight][s]{\columnwidth}
      \begin{itemize}
        \item<1-> What if the backtrace for an exception is never needed?
        \item<2-> It's possible to raise the exception explicitly with \funcname{raise\_notrace} to make raising as cheap as possible
        \item<3-> An example of use in the \funcname{dynlink} library of the OCaml compiler
      \end{itemize}
      \vfill
      \onslide<3->{
      \listocaml[firstline=205,lastline=209,highlightlines=207]{../ocaml/otherlibs/dynlink/dynlink\_compilerlibs/misc.ml}{otherlibs/dynlink/dynlink\_compilerlibs/misc.ml:205}
      }
      \endminipage
    \end{column}
    \begin{column}{0.55\textwidth}
    \only<4>{
      \mintcmm[highlightlines=3]{../src/notrace/camlNotrace\_\_fun_347.cmm}
      \listcmm{../src/notrace/camlNotrace\_\_all\_somes\_327.cmm}{all\_somes}
    }
    \onslide*<5->{
      \listobjdump[highlightlines={6-9}]{../src/notrace/camlNotrace\_\_fun_347.objdump}{anonymous function from \funcname{all\_somes}}
    }
    \end{column}
  \end{columns}
\end{frame}

\begin{frame}{Backtraces}{Summary table of the multiple kinds of raise}
  \begin{itemize}
    \item When debugging information is enabled and if the backtrace of an exception isn't needed, it's possible to raise with \funcname{raise\_notrace} for performance
    \item When debugging information is \emph{not} enabled, the compiler optimises \funcname{raise} calls to inline assembly (same effect as using \funcname{raise\_notrace})
  \end{itemize}
  \bigskip
  \centering
  \begin{tabular}{| l | c | c | c | c |}
    \hline
    \parbox{10em}{Debug information \\ (\funcarg{-g} flag)}  & \multicolumn{2}{|c|}{Disabled}                                     & \multicolumn{2}{|c|}{Enabled} \\
    \hline
    Raise kind                                               & \funcname{raise} & \funcname{raise\_notrace}                       & \funcname{raise} & \funcname{raise\_notrace} \\
    \hline
    Compilation                                              & \parbox{8em}{inline assembly\\ (optimization)} & inline assembly   & \funcname{caml\_raise\_exn} & inline assembly \\
    \hline
  \end{tabular}
\end{frame}


%
% Takeaway #5
%
\subsection*{Takeaway \#5}
\frameSubsectionTakeaway{}
\againframe<5>{takeaway}
}%
      \onslide*<6>{\providecommand\step{2}%
% Backtraces
%
\subsection{One advantage of raising with debugging information}
\frameSubsection{}{}

\begin{frame}{Backtraces}{Debugging information \& source code}
  \begin{columns}[c]
    \begin{column}{0.5\textwidth}
      \begin{itemize}
        \item Raising an exception is fun and games, but how do we know where the exception originated? $\rightarrow$ With the backtrace associated with the exception!
        \item In order to get a backtrace from the point where an exception was raised, it's necessary to compile with debugging information enabled (\funcarg{-g} flag for the compiler)
        \item Same example as in the `Nested handlers' section
      \end{itemize}
    \end{column}
    \begin{column}{0.5\textwidth}
      \listocaml[firstline=9,lastline=34]{../src/backtrace/backtrace.ml}{backtrace/backtrace.ml}
    \end{column}
  \end{columns}
\end{frame}

\begin{frame}{Backtraces}{CMM and disassembly of \funcname{definitely\_raise}}
  \begin{columns}[c]
    \begin{column}{0.5\textwidth}
      \minipage[c][0.66\textheight][s]{\columnwidth}
      \begin{itemize}
        \item<1-> An effect of compiling with debugging information (\funcarg{-g}): the CMM no longer uses \funcname{raise\_notrace} to raise the exception but \funcname{raise} instead
        \item<2-> Disassembly shows that CMM \funcname{raise} is actually a call to \funcname{caml\_raise\_exn}, a function in assembler provided by the runtime
      \end{itemize}
      \vfill
      \mintocaml[firstline=9,lastline=13,highlightlines=12]{../src/backtrace/backtrace.ml}
      \endminipage
    \end{column}
    \begin{column}{0.5\textwidth}
      \onslide*<1>{
        \mintcmm[highlightlines=5]{../src/backtrace/camlBacktrace\_\_definitely\_raise\_347.cmm}
      }
      \onslide*<2>{
        \mintobjdump[firstline=2,highlightlines=14]{../src/backtrace/camlBacktrace\_\_definitely\_raise\_347.objdump}
      }
    \end{column}
  \end{columns}
\end{frame}

\begin{frame}{Backtraces}{Introducing \funcname{caml\_raise\_exn}}
  \begin{columns}[c]
    \begin{column}{0.5\textwidth}
      \minipage[c][0.66\textheight][s]{\columnwidth}
      \onslide*<1>{
        \begin{itemize}
          \item Raising the exception is performed using \funcname{caml\_raise\_exn} which is
            \begin{itemize}
              \item An assembly function provided by the runtime
              \item Architecture specific
            \end{itemize}
          \item Let's see what it does differently from \funcname{raise\_notrace}\dots
          \item We are about to raise \typename{ExnC} with \funcname{caml\_raise\_exn} from \funcname{definitely\_raise}
        \end{itemize}
      }
      \onslide*<2>{
        \mintobjdump[firstline=2,highlightlines=14]{../src/backtrace/camlBacktrace\_\_definitely\_raise\_347.objdump}
      }
      \vfill
      \mintocaml[firstline=9,lastline=13,highlightlines=12]{../src/backtrace/backtrace.ml}
      \endminipage
    \end{column}
    \begin{column}{0.5\textwidth}
      \centering
      \providecommand\step{1}\input{diagrams/backtrace/backtrace.tex}
    \end{column}
  \end{columns}
\end{frame}

\begin{frame}{Backtraces}{But before that, we need more fields from \typename{caml\_domain\_state}}
  \begin{columns}
    \begin{column}{0.5\textwidth}
      \begin{itemize}
        \item Remember \typename{caml\_domain\_state} the per-domain runtime state?
        \item Some other members are of interest to us now\dots
        \item \localname{backtrace\_active} is used to know if the runtime should collect backtraces
        \item \localname{backtrace\_active} can be set by
          \begin{itemize}
            \item The \funcarg{b} flag in the \funcname{OCAMLRUNPARAM} environment variable
            \item \mintinline{ocaml}{Printexc.record_backtrace : bool -> unit} \footnotemark
          \end{itemize}
      \end{itemize}
    \end{column}
    \begin{column}{0.5\textwidth}
      \begin{listing}
        \mintc[highlightlines={9-11}]{src/ocaml/domain\_state\_backtrace.h}
        \caption{runtime/caml/domain\_state.h}
      \end{listing}
    \end{column}
  \end{columns}
  \footnotetext[1]{https://v2.ocaml.org/api/Printexc.html}
\end{frame}

\newcommand\listCamlRaiseExn[1][]{
  \listasm[firstline=702,lastline=725,#1]{../ocaml/runtime/amd64.S}{runtime/amd64.S:702}}

\begin{frame}{Backtraces}{Walk through \funcname{caml\_raise\_exn}}
  \begin{columns}[T]
    \begin{column}{0.5\textwidth}
      \begin{itemize}
        \item<1-> First checks whether exceptions backtrace should be recorded
        \item<1-> By checking the \funcarg{domain\_state->backtrace\_active} value
        \item<2-> If not, acts as \funcname{raise\_notrace} with \funcname{RESTORE\_EXN\_HANDLER\_OCAML}
          \begin{itemize}
            \item Set \regname{RSP} to \localname{domain\_state->exn\_handler} value
            \item Restore \localname{domain\_state->exn\_handler} from trap
            \item Jump with \funcname{ret} (same effect as using \funcname{pop} and \funcname{jmp})
          \end{itemize}
        \item<3-> Otherwise, switches to the C stack to be able to execute C code
        \item<4-> Calls \funcname{caml\_stash\_backtrace}, a C function provided by the runtime\ignorespaces
          \onslide<6->{, with
            \begin{itemize}
              \item \funcarg{exn}: exception value
              \item \funcarg{pc}: code address of the raising point
              \item \funcarg{sp}: stack address of the raising function
              \item \funcarg{trapsp}: stack address of the next handler
            \end{itemize}
          }
      \end{itemize}
      \bigskip
      \mintocaml[firstline=9,lastline=13,highlightlines=12]{../src/backtrace/backtrace.ml}
    \end{column}
    \begin{column}{0.5\textwidth}
      \raggedleft
      \onslide*<1,3-4>{
        \mintc[firstline=190,lastline=192]{../ocaml/runtime/amd64.S}
        \mintc[firstline=234,lastline=237]{../ocaml/runtime/amd64.S}
      }
      \onslide*<2>{
        \mintc[firstline=190,lastline=192,highlightlines={191-192}]{../ocaml/runtime/amd64.S}
        \mintc[firstline=234,lastline=237,highlightlines={235-237}]{../ocaml/runtime/amd64.S}
      }
      \onslide*<1>{\listCamlRaiseExn[highlightlines=705]{}}%
      \onslide*<2>{\listCamlRaiseExn[highlightlines={707-708}]{}}%
      \onslide*<3>{\listCamlRaiseExn[highlightlines={712-714}]{}}%
      \onslide*<4>{\listCamlRaiseExn[highlightlines={715-720}]{}}%
      \onslide*<5>{\providecommand\step{1}\input{diagrams/backtrace/backtrace.tex}}%
      \onslide*<6>{\providecommand\step{2}\input{diagrams/backtrace/backtrace.tex}}%
    \end{column}
  \end{columns}
\end{frame}

\newcommand\listStashBacktrace[1][]{
  \listc[firstline=91,lastline=120,#1]{../ocaml/runtime/backtrace\_nat.c}{runtime/backtrace\_nat.c:91}}

\begin{frame}{Backtraces}{Introducing \funcname{caml\_stash\_backtrace}}
  \begin{columns}[c]
    \begin{column}{0.5\textwidth}
      \minipage[c][0.66\textheight][s]{\columnwidth}
      \begin{itemize}
        \item<1-> A C function provided by the runtime (specific to native code)
        \item<2-> It scans the stack, between the stack pointer at the raising point up to the next trap, in order to collect the names of the detected function frames
          \onslide*<2>{
            \begin{itemize}
              \item And more information, like line number, column number...
            \end{itemize}
          }
        \item<3-> It uses the same technique and information as the Garbage Collector (GC) uses to scan the values live on the stack
          \onslide*<4>{
            \begin{itemize}
              \item \typename{frame\_descr} are at the core of stack unwinding
            \end{itemize}
          }
      \end{itemize}
      \vfill
      \mintocaml[firstline=9,lastline=13,highlightlines=12]{../src/backtrace/backtrace.ml}
      \endminipage
    \end{column}
    \begin{column}{0.5\textwidth}
      \onslide*<1>{\listStashBacktrace[highlightlines=91]{}}
      \onslide*<2>{\listStashBacktrace[highlightlines={91,117-118}]{}}
      \onslide*<3->{\listStashBacktrace[highlightlines={94,106,109-110}]{}}
    \end{column}
  \end{columns}
\end{frame}

\newcommand\listFrameDescr[1][]{
  \listocaml[firstline=111,lastline=124,#1]{../ocaml/asmcomp/emitaux.ml}{asmcomp/emitaux.ml:111}}
\newcommand\listDebugInfo[1][]{
  \listocaml[firstline=38,lastline=49,#1]{../ocaml/lambda/debuginfo.mli}{lambda/debuginfo.mli:38}}

\begin{frame}{Backtraces}{\typename{frame\_descr} and \typename{frame\_debuginfo} in a nutshell}
  \begin{columns}[c]
    \begin{column}{0.5\textwidth}
      \begin{itemize}
        \item<1-> For every return address in an OCaml program, there is a associated \typename{frame\_descr}
        \item<2-> A \typename{frame\_descr} specifies
        \begin{itemize}
          \item<2-> The size of the function stack frame
          \item<3-> Where live values are on the stack (so that the GC can mark them as live and prevent from being swept)
        \end{itemize}
        \item<4-> When compiled with debug information, \typename{frame\_descr} will be linked to a \typename{frame\_debuginfo}
        \begin{itemize}
          \item<5-> Contains information about the position in the source code (file name, line and column number)
        \end{itemize}
      \end{itemize}
    \end{column}
    \begin{column}{0.5\textwidth}
        \onslide*<1>{\listFrameDescr[highlightlines={118-122}]{}}
        \onslide*<2>{\listFrameDescr[highlightlines={120}]{}}
        \onslide*<3>{\listFrameDescr[highlightlines={121}]{}}
        \onslide*<4->{\listFrameDescr[highlightlines={122}]{}}
        \medskip
        \onslide*<1-3>{\listDebugInfo{}}
        \onslide*<4>{\listDebugInfo[highlightlines={38,49}]{}}
        \onslide*<5>{\listDebugInfo[highlightlines={39-46}]{}}
    \end{column}
  \end{columns}
\end{frame}

\begin{frame}{Backtraces}{Scanning the stack with \typename{frame\_descr}}
  \begin{columns}[c]
    \begin{column}{0.5\textwidth}
      \begin{itemize}
        \item Given the return address of the code that raises the exception by calling \funcname{caml\_raise\_exn} (\funcarg{pc} argument of \funcname{caml\_stash\_backtrace})
        \item And the position of the stack pointer (\funcarg{sp} argument of \funcname{caml\_stash\_backtrace})
        \item The frame size given by \typename{frame\_descr}.\funcarg{frame\_size}, allows to find the end of the previous frame
        \item And the return address of the caller of this function
        \bigskip
        \onslide<3->{
        \item Note that traps (here the reddish \funcname{ExnA}, \funcname{ExnB} and \funcname{ExnC} blocks) belong to the frame that installed them, and so the frame size changes when traps are removed
        }
      \end{itemize}
    \end{column}
    \begin{column}{0.5\textwidth}
      \raggedleft
      \onslide*<1>{\providecommand\step{2}\input{diagrams/backtrace/backtrace.tex}}%
      \onslide*<2>{\providecommand\step{1}\input{diagrams/backtrace/scanstack.tex}}%
      \onslide*<3>{\providecommand\step{2}\input{diagrams/backtrace/scanstack.tex}}%
      \onslide*<4>{\providecommand\step{3}\input{diagrams/backtrace/scanstack.tex}}%
      \onslide*<5>{\providecommand\step{4}\input{diagrams/backtrace/scanstack.tex}}%
      \onslide*<6>{\providecommand\step{5}\input{diagrams/backtrace/scanstack.tex}}%
    \end{column}
  \end{columns}
\end{frame}

% Duplicate of previous-previous slide
\begin{frame}{Backtraces}{Continuing on \funcname{caml\_stash\_backtrace}}
  \begin{columns}[c]
    \begin{column}{0.5\textwidth}
      \minipage[c][0.75\textheight][s]{\columnwidth}
      \begin{itemize}
          \color{gray}
        \item A C function provided by the runtime (specific to native code)
        \item It scans the stack, between the stack pointer at the raising point up to the next trap, in order to collect the names of the detected function frames
        \item It uses the same technique and information as the Garbage Collector (GC) uses to scan the values live on the stack
      \end{itemize}
      \begin{itemize}
        \item For each function frame detected, store a pointer to the \typename{frame\_descr} in the \localname{domain\_state->backtrace\_buffer} array, effectively constructing the backtrace
      \end{itemize}
      \onslide<2->{
        \begin{itemize}
            \smallskip
          \item Constructed backtrace:
            \funcname{definitely\_raise},
            \onslide<3->{\funcname{probably\_raise}}
        \end{itemize}
      }
      \vfill
      \mintocaml[firstline=9,lastline=13,highlightlines=12]{../src/backtrace/backtrace.ml}
      \endminipage
    \end{column}
    \begin{column}{0.5\textwidth}
      \raggedleft
      \onslide*<1>{\listStashBacktrace[highlightlines={114-115}]{}}
      \onslide*<2>{\providecommand\step{3}\input{diagrams/backtrace/backtrace.tex}}%
      \onslide*<3>{\providecommand\step{4}\input{diagrams/backtrace/backtrace.tex}}%
    \end{column}
  \end{columns}
\end{frame}

\begin{frame}{Backtraces}{Back to \funcname{caml\_raise\_exn}}
  \begin{columns}[c]
    \begin{column}{0.5\textwidth}
      \minipage[c][0.66\textheight][s]{\columnwidth}
      \begin{itemize}\color{gray}
        \item Checks wheteher exceptions backtrace should be recorded, by checking the \funcarg{domain\_state->backtrace\_active} value
        \item Switches to the C stack to be able to execute C code
        \item Calls \funcname{caml\_stash\_backtrace}, to collect the backtrace up to the next trap
      \end{itemize}
      \onslide*<2->{
        \begin{itemize}
          \item<2-> Executes the exception handler in \funcname{probably\_raise} with \funcname{RESTORE\_EXN\_HANDLER\_OCAML}
            \begin{itemize}
              \item Set \regname{RSP} to \localname{domain\_state->exn\_handler} value
              \item Restore \localname{domain\_state->exn\_handler} from trap
              \item Jump to the handler code with \funcname{ret}
            \end{itemize}
        \end{itemize}
      }
      \vfill
      \onslide*<1>{\mintocaml[firstline=9,lastline=13,highlightlines=12]{../src/backtrace/backtrace.ml}}
      \onslide*<2->{\mintocaml[firstline=15,lastline=21,highlightlines={19-21}]{../src/backtrace/backtrace.ml}}
      \endminipage
    \end{column}
    \begin{column}{0.5\textwidth}
      \centering
      \onslide*<1>{
        \mintc[firstline=190,lastline=192]{../ocaml/runtime/amd64.S}
        \mintc[firstline=234,lastline=237]{../ocaml/runtime/amd64.S}
      }
      \onslide*<2>{
        \mintc[firstline=190,lastline=192,highlightlines={191-192}]{../ocaml/runtime/amd64.S}
        \mintc[firstline=234,lastline=237,highlightlines={235-237}]{../ocaml/runtime/amd64.S}
      }
      \onslide*<1>{\listCamlRaiseExn[highlightlines=720]{}}
      \onslide*<2>{\listCamlRaiseExn[highlightlines=722]{}}
      \onslide*<3->{\providecommand\step{5}\input{diagrams/backtrace/backtrace.tex}}
    \end{column}
  \end{columns}
\end{frame}

\begin{frame}{Backtraces}{\funcname{caml\_probably\_raise}'s handler doesn't catch \typename{ExnC}}
  \begin{columns}[c]
    \begin{column}{0.45\textwidth}
      \minipage[c][0.75\textheight][s]{\columnwidth}
      \begin{itemize}
        \item<1-> \funcname{match \dots with}\footnotemark pattern matching can also match exceptions
        \item<1-> The CMM of \funcname{probably\_raise} shows that this \funcname{match \dots with} is implemented with a pattern matching and a \funcname{try \dots with}
        \item<1-> But this one only matches on \typename{ExnA} exception
        \item<2-> Since the pattern matching fails to catch the exception, \funcname{reraise} is called with our current \typename{ExnC} exception
        \item<3-> Disassembly shows that \funcname{reraise} is actually a call to \funcname{caml\_reraise\_exn}, which is also an assembly function provided by the runtime
      \end{itemize}
      \vfill
      \mintocaml[firstline=15,lastline=21,highlightlines={18-21}]{../src/backtrace/backtrace.ml}
      \endminipage
    \end{column}
    \begin{column}{0.55\textwidth}
      \centering
      \onslide*<1>{\mintcmm[highlightlines={10-18}]{../src/backtrace/camlBacktrace\_\_probably\_raise\_351.cmm}}
      \onslide*<2>{\listcmm[highlightlines=18]{../src/backtrace/camlBacktrace\_\_probably\_raise_351.cmm}{CMM of probably\_raise}}
      \onslide*<3>{\mintobjdump[firstline=5,lastline=35,highlightlines=31]{../src/backtrace/camlBacktrace\_\_probably\_raise_351.objdump}}
    \end{column}
  \end{columns}
  \only<1>{\footnotetext[1]{https://blog.janestreet.com/pattern-matching-and-exception-handling-unite/}}
\end{frame}

\newcommand\listCamlReraiseExn[1][]{
  \listasm[firstline=727,lastline=734,#1]{../ocaml/runtime/amd64.S}{runtime/amd64.S:727}}

\begin{frame}{Backtraces}{Introducing \funcname{caml\_reraise\_exn}}
  \begin{columns}[c]
    \begin{column}{0.5\textwidth}
      \minipage[c][0.66\textheight][s]{\columnwidth}
      \begin{itemize}
        \item<1-> The exception have to be reraised to the next handler
        \item<2-> First, checks if the backtrace should be collected with \funcarg{domain\_state->backtrace\_active}
        \only<3>{
        \begin{itemize}
            \item If not, behaves like a \funcname{raise\_notrace} with \funcname{RESTORE\_EXN\_HANDLER\_OCAML}
        \end{itemize}
        }
        \item<4-> Jumps into \funcname{caml\_raise\_exn}
        \item<5-> Calls \funcname{caml\_stash\_backtrace} again, to scan the next part of the stack: from the trap just pop-ed to the next trap
      \end{itemize}
      \vfill
      \mintocaml[firstline=15,lastline=21,highlightlines={18-21}]{../src/backtrace/backtrace.ml}
      \endminipage
    \end{column}
    \begin{column}{0.5\textwidth}
      \raggedleft
      \onslide*<1>{\listCamlReraiseExn{}}
      \onslide*<2>{\listCamlReraiseExn[highlightlines=729]{}}
      \onslide*<3>{
        \mintc[firstline=190,lastline=192,highlightlines={191-192}]{../ocaml/runtime/amd64.S}
        \mintc[firstline=234,lastline=237,highlightlines={235-237}]{../ocaml/runtime/amd64.S}
        \medskip
        \listCamlReraiseExn[highlightlines=731]{}
      }
      \onslide*<4-5>{
        % TODO refactor with \listCamlReraiseExn
        \mintasm[firstline=727,lastline=734,highlightlines=730]{../ocaml/runtime/amd64.S}
        \medskip
      }
      \onslide*<4>{\listCamlRaiseExn[highlightlines=711]{}}
      \onslide*<5>{\listCamlRaiseExn[highlightlines=720]{}}
    \end{column}
  \end{columns}
\end{frame}

\begin{frame}{Backtraces}{Backtrace collection continues}
  \begin{columns}[c]
    \begin{column}{0.55\textwidth}
      \minipage[c][0.75\textheight][s]{\columnwidth}
      \begin{itemize}
        \item<1-> \funcname{caml\_stash\_backtrace} scans the older part of the stack, up to the next trap
        \item<6-> Controls jumps to the nested exception handler in \funcname{maybe\_raise}, which matches only on \typename{ExnB}
        \item<7-> \funcname{caml\_reraise\_exn} is called again, backtrace collection resumes with \funcname{caml\_stash\_backtrace}
      \end{itemize}
      \medskip
      \begin{itemize}
        \item<2-> Constructed backtrace:
        \begin{itemize}
          \item <2-> \funcname{definitely\_raise}
          \item <2-> \funcname{probably\_raise}
          \item <3-> \funcname{probably\_raise} (re-raise)
          \item <4-> \funcname{maybe\_raise}
          \item <5-> \funcname{main}
          \item <8-> \funcname{main} (re-raise)
        \end{itemize}
      \end{itemize}
      \vfill
      \onslide*<1-5>{\mintocaml[firstline=15,lastline=21]{../src/backtrace/backtrace.ml}}
      \onslide*<6>{\mintocaml[firstline=28,lastline=34,highlightlines=31]{../src/backtrace/backtrace.ml}}
      \onslide*<7->{\mintocaml[firstline=28,lastline=34,highlightlines=33]{../src/backtrace/backtrace.ml}}
      \endminipage
    \end{column}
    \begin{column}{0.45\textwidth}
      \raggedleft
      \onslide*<1>{\providecommand\step{6}\input{diagrams/backtrace/backtrace.tex}}%
      \onslide*<2>{\providecommand\step{6}\input{diagrams/backtrace/backtrace.tex}}%
      \onslide*<3>{\providecommand\step{7}\input{diagrams/backtrace/backtrace.tex}}%
      \onslide*<4>{\providecommand\step{8}\input{diagrams/backtrace/backtrace.tex}}%
      \onslide*<5>{\providecommand\step{9}\input{diagrams/backtrace/backtrace.tex}}%
      \onslide*<6>{\providecommand\step{10}\input{diagrams/backtrace/backtrace.tex}}%
      \onslide*<7>{\providecommand\step{11}\input{diagrams/backtrace/backtrace.tex}}%
      \onslide*<8>{\providecommand\step{12}\input{diagrams/backtrace/backtrace.tex}}%
      \onslide*<9>{\providecommand\step{13}\input{diagrams/backtrace/backtrace.tex}}%
    \end{column}
  \end{columns}
\end{frame}

\begin{frame}{Backtraces}{Printing the backtrace}
  \begin{columns}[c]
    \begin{column}{0.45\textwidth}
      \minipage[c][0.66\textheight][s]{\columnwidth}
      \begin{itemize}
        \item<1-> The exception backtrace can now be printed to \funcarg{stdout} with \funcname{Printexc.print_backtrace}
        \item<2-> First the exception backtrace is retrieved into an OCaml array with \funcname{get_raw_backtrace }
        \item<3-> Which is implemented as the \funcname{caml_get_exception_raw_backtrace} C function in the runtime
        \item<4-> And finally printed with \funcname{print_exception_backtrace}
      \end{itemize}
      \vfill
      \mintocaml[firstline=28,lastline=34,highlightlines=33]{../src/backtrace/backtrace.ml}
      \endminipage
    \end{column}
    \begin{column}{0.55\textwidth}
      \onslide*<1,2>{
        \mintocaml[firstline=100,lastline=101]{../ocaml/stdlib/printexc.ml}
      }
      \only<1>{
        \listocaml[firstline=170,lastline=172]{../ocaml/stdlib/printexc.ml}{stdlib/printexc.ml:170}
      }
      \only<2>{
        \listocaml[firstline=170,lastline=172,highlightlines=172]{../ocaml/stdlib/printexc.ml}{stdlib/printexc.ml:170}
      }
      \only<3>{
        \mintc[firstline=154,lastline=156]{../ocaml/runtime/backtrace.c}
        \listc[firstline=171,lastline=192,highlightlines={184-187}]{../ocaml/runtime/backtrace.c}{runtime/backtrace.c:154}
      }
      \only<4>{
        \listocaml[firstline=155,lastline=165]{../ocaml/stdlib/printexc.ml}{stdlib/printexc.ml:55}
      }
    \end{column}
  \end{columns}
\end{frame}


\subsection{The multiple kinds of raise}
\frameSubsection{}{}

\begin{frame}{Backtraces}{When backtraces are not needed}
  \begin{columns}[c]
    \begin{column}{0.45\textwidth}
      \minipage[c][0.66\textheight][s]{\columnwidth}
      \begin{itemize}
        \item<1-> What if the backtrace for an exception is never needed?
        \item<2-> It's possible to raise the exception explicitly with \funcname{raise\_notrace} to make raising as cheap as possible
        \item<3-> An example of use in the \funcname{dynlink} library of the OCaml compiler
      \end{itemize}
      \vfill
      \onslide<3->{
      \listocaml[firstline=205,lastline=209,highlightlines=207]{../ocaml/otherlibs/dynlink/dynlink\_compilerlibs/misc.ml}{otherlibs/dynlink/dynlink\_compilerlibs/misc.ml:205}
      }
      \endminipage
    \end{column}
    \begin{column}{0.55\textwidth}
    \only<4>{
      \mintcmm[highlightlines=3]{../src/notrace/camlNotrace\_\_fun_347.cmm}
      \listcmm{../src/notrace/camlNotrace\_\_all\_somes\_327.cmm}{all\_somes}
    }
    \onslide*<5->{
      \listobjdump[highlightlines={6-9}]{../src/notrace/camlNotrace\_\_fun_347.objdump}{anonymous function from \funcname{all\_somes}}
    }
    \end{column}
  \end{columns}
\end{frame}

\begin{frame}{Backtraces}{Summary table of the multiple kinds of raise}
  \begin{itemize}
    \item When debugging information is enabled and if the backtrace of an exception isn't needed, it's possible to raise with \funcname{raise\_notrace} for performance
    \item When debugging information is \emph{not} enabled, the compiler optimises \funcname{raise} calls to inline assembly (same effect as using \funcname{raise\_notrace})
  \end{itemize}
  \bigskip
  \centering
  \begin{tabular}{| l | c | c | c | c |}
    \hline
    \parbox{10em}{Debug information \\ (\funcarg{-g} flag)}  & \multicolumn{2}{|c|}{Disabled}                                     & \multicolumn{2}{|c|}{Enabled} \\
    \hline
    Raise kind                                               & \funcname{raise} & \funcname{raise\_notrace}                       & \funcname{raise} & \funcname{raise\_notrace} \\
    \hline
    Compilation                                              & \parbox{8em}{inline assembly\\ (optimization)} & inline assembly   & \funcname{caml\_raise\_exn} & inline assembly \\
    \hline
  \end{tabular}
\end{frame}


%
% Takeaway #5
%
\subsection*{Takeaway \#5}
\frameSubsectionTakeaway{}
\againframe<5>{takeaway}
}%
    \end{column}
  \end{columns}
\end{frame}

\newcommand\listStashBacktrace[1][]{
  \listc[firstline=91,lastline=120,#1]{../ocaml/runtime/backtrace\_nat.c}{runtime/backtrace\_nat.c:91}}

\begin{frame}{Backtraces}{Introducing \funcname{caml\_stash\_backtrace}}
  \begin{columns}[c]
    \begin{column}{0.5\textwidth}
      \minipage[c][0.66\textheight][s]{\columnwidth}
      \begin{itemize}
        \item<1-> A C function provided by the runtime (specific to native code)
        \item<2-> It scans the stack, between the stack pointer at the raising point up to the next trap, in order to collect the names of the detected function frames
          \onslide*<2>{
            \begin{itemize}
              \item And more information, like line number, column number...
            \end{itemize}
          }
        \item<3-> It uses the same technique and information as the Garbage Collector (GC) uses to scan the values live on the stack
          \onslide*<4>{
            \begin{itemize}
              \item \typename{frame\_descr} are at the core of stack unwinding
            \end{itemize}
          }
      \end{itemize}
      \vfill
      \mintocaml[firstline=9,lastline=13,highlightlines=12]{../src/backtrace/backtrace.ml}
      \endminipage
    \end{column}
    \begin{column}{0.5\textwidth}
      \onslide*<1>{\listStashBacktrace[highlightlines=91]{}}
      \onslide*<2>{\listStashBacktrace[highlightlines={91,117-118}]{}}
      \onslide*<3->{\listStashBacktrace[highlightlines={94,106,109-110}]{}}
    \end{column}
  \end{columns}
\end{frame}

\newcommand\listFrameDescr[1][]{
  \listocaml[firstline=111,lastline=124,#1]{../ocaml/asmcomp/emitaux.ml}{asmcomp/emitaux.ml:111}}
\newcommand\listDebugInfo[1][]{
  \listocaml[firstline=38,lastline=49,#1]{../ocaml/lambda/debuginfo.mli}{lambda/debuginfo.mli:38}}

\begin{frame}{Backtraces}{\typename{frame\_descr} and \typename{frame\_debuginfo} in a nutshell}
  \begin{columns}[c]
    \begin{column}{0.5\textwidth}
      \begin{itemize}
        \item<1-> For every return address in an OCaml program, there is a associated \typename{frame\_descr}
        \item<2-> A \typename{frame\_descr} specifies
        \begin{itemize}
          \item<2-> The size of the function stack frame
          \item<3-> Where live values are on the stack (so that the GC can mark them as live and prevent from being swept)
        \end{itemize}
        \item<4-> When compiled with debug information, \typename{frame\_descr} will be linked to a \typename{frame\_debuginfo}
        \begin{itemize}
          \item<5-> Contains information about the position in the source code (file name, line and column number)
        \end{itemize}
      \end{itemize}
    \end{column}
    \begin{column}{0.5\textwidth}
        \onslide*<1>{\listFrameDescr[highlightlines={118-122}]{}}
        \onslide*<2>{\listFrameDescr[highlightlines={120}]{}}
        \onslide*<3>{\listFrameDescr[highlightlines={121}]{}}
        \onslide*<4->{\listFrameDescr[highlightlines={122}]{}}
        \medskip
        \onslide*<1-3>{\listDebugInfo{}}
        \onslide*<4>{\listDebugInfo[highlightlines={38,49}]{}}
        \onslide*<5>{\listDebugInfo[highlightlines={39-46}]{}}
    \end{column}
  \end{columns}
\end{frame}

\begin{frame}{Backtraces}{Scanning the stack with \typename{frame\_descr}}
  \begin{columns}[c]
    \begin{column}{0.5\textwidth}
      \begin{itemize}
        \item Given the return address of the code that raises the exception by calling \funcname{caml\_raise\_exn} (\funcarg{pc} argument of \funcname{caml\_stash\_backtrace})
        \item And the position of the stack pointer (\funcarg{sp} argument of \funcname{caml\_stash\_backtrace})
        \item The frame size given by \typename{frame\_descr}.\funcarg{frame\_size}, allows to find the end of the previous frame
        \item And the return address of the caller of this function
        \bigskip
        \onslide<3->{
        \item Note that traps (here the reddish \funcname{ExnA}, \funcname{ExnB} and \funcname{ExnC} blocks) belong to the frame that installed them, and so the frame size changes when traps are removed
        }
      \end{itemize}
    \end{column}
    \begin{column}{0.5\textwidth}
      \raggedleft
      \onslide*<1>{\providecommand\step{2}%
% Backtraces
%
\subsection{One advantage of raising with debugging information}
\frameSubsection{}{}

\begin{frame}{Backtraces}{Debugging information \& source code}
  \begin{columns}[c]
    \begin{column}{0.5\textwidth}
      \begin{itemize}
        \item Raising an exception is fun and games, but how do we know where the exception originated? $\rightarrow$ With the backtrace associated with the exception!
        \item In order to get a backtrace from the point where an exception was raised, it's necessary to compile with debugging information enabled (\funcarg{-g} flag for the compiler)
        \item Same example as in the `Nested handlers' section
      \end{itemize}
    \end{column}
    \begin{column}{0.5\textwidth}
      \listocaml[firstline=9,lastline=34]{../src/backtrace/backtrace.ml}{backtrace/backtrace.ml}
    \end{column}
  \end{columns}
\end{frame}

\begin{frame}{Backtraces}{CMM and disassembly of \funcname{definitely\_raise}}
  \begin{columns}[c]
    \begin{column}{0.5\textwidth}
      \minipage[c][0.66\textheight][s]{\columnwidth}
      \begin{itemize}
        \item<1-> An effect of compiling with debugging information (\funcarg{-g}): the CMM no longer uses \funcname{raise\_notrace} to raise the exception but \funcname{raise} instead
        \item<2-> Disassembly shows that CMM \funcname{raise} is actually a call to \funcname{caml\_raise\_exn}, a function in assembler provided by the runtime
      \end{itemize}
      \vfill
      \mintocaml[firstline=9,lastline=13,highlightlines=12]{../src/backtrace/backtrace.ml}
      \endminipage
    \end{column}
    \begin{column}{0.5\textwidth}
      \onslide*<1>{
        \mintcmm[highlightlines=5]{../src/backtrace/camlBacktrace\_\_definitely\_raise\_347.cmm}
      }
      \onslide*<2>{
        \mintobjdump[firstline=2,highlightlines=14]{../src/backtrace/camlBacktrace\_\_definitely\_raise\_347.objdump}
      }
    \end{column}
  \end{columns}
\end{frame}

\begin{frame}{Backtraces}{Introducing \funcname{caml\_raise\_exn}}
  \begin{columns}[c]
    \begin{column}{0.5\textwidth}
      \minipage[c][0.66\textheight][s]{\columnwidth}
      \onslide*<1>{
        \begin{itemize}
          \item Raising the exception is performed using \funcname{caml\_raise\_exn} which is
            \begin{itemize}
              \item An assembly function provided by the runtime
              \item Architecture specific
            \end{itemize}
          \item Let's see what it does differently from \funcname{raise\_notrace}\dots
          \item We are about to raise \typename{ExnC} with \funcname{caml\_raise\_exn} from \funcname{definitely\_raise}
        \end{itemize}
      }
      \onslide*<2>{
        \mintobjdump[firstline=2,highlightlines=14]{../src/backtrace/camlBacktrace\_\_definitely\_raise\_347.objdump}
      }
      \vfill
      \mintocaml[firstline=9,lastline=13,highlightlines=12]{../src/backtrace/backtrace.ml}
      \endminipage
    \end{column}
    \begin{column}{0.5\textwidth}
      \centering
      \providecommand\step{1}\input{diagrams/backtrace/backtrace.tex}
    \end{column}
  \end{columns}
\end{frame}

\begin{frame}{Backtraces}{But before that, we need more fields from \typename{caml\_domain\_state}}
  \begin{columns}
    \begin{column}{0.5\textwidth}
      \begin{itemize}
        \item Remember \typename{caml\_domain\_state} the per-domain runtime state?
        \item Some other members are of interest to us now\dots
        \item \localname{backtrace\_active} is used to know if the runtime should collect backtraces
        \item \localname{backtrace\_active} can be set by
          \begin{itemize}
            \item The \funcarg{b} flag in the \funcname{OCAMLRUNPARAM} environment variable
            \item \mintinline{ocaml}{Printexc.record_backtrace : bool -> unit} \footnotemark
          \end{itemize}
      \end{itemize}
    \end{column}
    \begin{column}{0.5\textwidth}
      \begin{listing}
        \mintc[highlightlines={9-11}]{src/ocaml/domain\_state\_backtrace.h}
        \caption{runtime/caml/domain\_state.h}
      \end{listing}
    \end{column}
  \end{columns}
  \footnotetext[1]{https://v2.ocaml.org/api/Printexc.html}
\end{frame}

\newcommand\listCamlRaiseExn[1][]{
  \listasm[firstline=702,lastline=725,#1]{../ocaml/runtime/amd64.S}{runtime/amd64.S:702}}

\begin{frame}{Backtraces}{Walk through \funcname{caml\_raise\_exn}}
  \begin{columns}[T]
    \begin{column}{0.5\textwidth}
      \begin{itemize}
        \item<1-> First checks whether exceptions backtrace should be recorded
        \item<1-> By checking the \funcarg{domain\_state->backtrace\_active} value
        \item<2-> If not, acts as \funcname{raise\_notrace} with \funcname{RESTORE\_EXN\_HANDLER\_OCAML}
          \begin{itemize}
            \item Set \regname{RSP} to \localname{domain\_state->exn\_handler} value
            \item Restore \localname{domain\_state->exn\_handler} from trap
            \item Jump with \funcname{ret} (same effect as using \funcname{pop} and \funcname{jmp})
          \end{itemize}
        \item<3-> Otherwise, switches to the C stack to be able to execute C code
        \item<4-> Calls \funcname{caml\_stash\_backtrace}, a C function provided by the runtime\ignorespaces
          \onslide<6->{, with
            \begin{itemize}
              \item \funcarg{exn}: exception value
              \item \funcarg{pc}: code address of the raising point
              \item \funcarg{sp}: stack address of the raising function
              \item \funcarg{trapsp}: stack address of the next handler
            \end{itemize}
          }
      \end{itemize}
      \bigskip
      \mintocaml[firstline=9,lastline=13,highlightlines=12]{../src/backtrace/backtrace.ml}
    \end{column}
    \begin{column}{0.5\textwidth}
      \raggedleft
      \onslide*<1,3-4>{
        \mintc[firstline=190,lastline=192]{../ocaml/runtime/amd64.S}
        \mintc[firstline=234,lastline=237]{../ocaml/runtime/amd64.S}
      }
      \onslide*<2>{
        \mintc[firstline=190,lastline=192,highlightlines={191-192}]{../ocaml/runtime/amd64.S}
        \mintc[firstline=234,lastline=237,highlightlines={235-237}]{../ocaml/runtime/amd64.S}
      }
      \onslide*<1>{\listCamlRaiseExn[highlightlines=705]{}}%
      \onslide*<2>{\listCamlRaiseExn[highlightlines={707-708}]{}}%
      \onslide*<3>{\listCamlRaiseExn[highlightlines={712-714}]{}}%
      \onslide*<4>{\listCamlRaiseExn[highlightlines={715-720}]{}}%
      \onslide*<5>{\providecommand\step{1}\input{diagrams/backtrace/backtrace.tex}}%
      \onslide*<6>{\providecommand\step{2}\input{diagrams/backtrace/backtrace.tex}}%
    \end{column}
  \end{columns}
\end{frame}

\newcommand\listStashBacktrace[1][]{
  \listc[firstline=91,lastline=120,#1]{../ocaml/runtime/backtrace\_nat.c}{runtime/backtrace\_nat.c:91}}

\begin{frame}{Backtraces}{Introducing \funcname{caml\_stash\_backtrace}}
  \begin{columns}[c]
    \begin{column}{0.5\textwidth}
      \minipage[c][0.66\textheight][s]{\columnwidth}
      \begin{itemize}
        \item<1-> A C function provided by the runtime (specific to native code)
        \item<2-> It scans the stack, between the stack pointer at the raising point up to the next trap, in order to collect the names of the detected function frames
          \onslide*<2>{
            \begin{itemize}
              \item And more information, like line number, column number...
            \end{itemize}
          }
        \item<3-> It uses the same technique and information as the Garbage Collector (GC) uses to scan the values live on the stack
          \onslide*<4>{
            \begin{itemize}
              \item \typename{frame\_descr} are at the core of stack unwinding
            \end{itemize}
          }
      \end{itemize}
      \vfill
      \mintocaml[firstline=9,lastline=13,highlightlines=12]{../src/backtrace/backtrace.ml}
      \endminipage
    \end{column}
    \begin{column}{0.5\textwidth}
      \onslide*<1>{\listStashBacktrace[highlightlines=91]{}}
      \onslide*<2>{\listStashBacktrace[highlightlines={91,117-118}]{}}
      \onslide*<3->{\listStashBacktrace[highlightlines={94,106,109-110}]{}}
    \end{column}
  \end{columns}
\end{frame}

\newcommand\listFrameDescr[1][]{
  \listocaml[firstline=111,lastline=124,#1]{../ocaml/asmcomp/emitaux.ml}{asmcomp/emitaux.ml:111}}
\newcommand\listDebugInfo[1][]{
  \listocaml[firstline=38,lastline=49,#1]{../ocaml/lambda/debuginfo.mli}{lambda/debuginfo.mli:38}}

\begin{frame}{Backtraces}{\typename{frame\_descr} and \typename{frame\_debuginfo} in a nutshell}
  \begin{columns}[c]
    \begin{column}{0.5\textwidth}
      \begin{itemize}
        \item<1-> For every return address in an OCaml program, there is a associated \typename{frame\_descr}
        \item<2-> A \typename{frame\_descr} specifies
        \begin{itemize}
          \item<2-> The size of the function stack frame
          \item<3-> Where live values are on the stack (so that the GC can mark them as live and prevent from being swept)
        \end{itemize}
        \item<4-> When compiled with debug information, \typename{frame\_descr} will be linked to a \typename{frame\_debuginfo}
        \begin{itemize}
          \item<5-> Contains information about the position in the source code (file name, line and column number)
        \end{itemize}
      \end{itemize}
    \end{column}
    \begin{column}{0.5\textwidth}
        \onslide*<1>{\listFrameDescr[highlightlines={118-122}]{}}
        \onslide*<2>{\listFrameDescr[highlightlines={120}]{}}
        \onslide*<3>{\listFrameDescr[highlightlines={121}]{}}
        \onslide*<4->{\listFrameDescr[highlightlines={122}]{}}
        \medskip
        \onslide*<1-3>{\listDebugInfo{}}
        \onslide*<4>{\listDebugInfo[highlightlines={38,49}]{}}
        \onslide*<5>{\listDebugInfo[highlightlines={39-46}]{}}
    \end{column}
  \end{columns}
\end{frame}

\begin{frame}{Backtraces}{Scanning the stack with \typename{frame\_descr}}
  \begin{columns}[c]
    \begin{column}{0.5\textwidth}
      \begin{itemize}
        \item Given the return address of the code that raises the exception by calling \funcname{caml\_raise\_exn} (\funcarg{pc} argument of \funcname{caml\_stash\_backtrace})
        \item And the position of the stack pointer (\funcarg{sp} argument of \funcname{caml\_stash\_backtrace})
        \item The frame size given by \typename{frame\_descr}.\funcarg{frame\_size}, allows to find the end of the previous frame
        \item And the return address of the caller of this function
        \bigskip
        \onslide<3->{
        \item Note that traps (here the reddish \funcname{ExnA}, \funcname{ExnB} and \funcname{ExnC} blocks) belong to the frame that installed them, and so the frame size changes when traps are removed
        }
      \end{itemize}
    \end{column}
    \begin{column}{0.5\textwidth}
      \raggedleft
      \onslide*<1>{\providecommand\step{2}\input{diagrams/backtrace/backtrace.tex}}%
      \onslide*<2>{\providecommand\step{1}\input{diagrams/backtrace/scanstack.tex}}%
      \onslide*<3>{\providecommand\step{2}\input{diagrams/backtrace/scanstack.tex}}%
      \onslide*<4>{\providecommand\step{3}\input{diagrams/backtrace/scanstack.tex}}%
      \onslide*<5>{\providecommand\step{4}\input{diagrams/backtrace/scanstack.tex}}%
      \onslide*<6>{\providecommand\step{5}\input{diagrams/backtrace/scanstack.tex}}%
    \end{column}
  \end{columns}
\end{frame}

% Duplicate of previous-previous slide
\begin{frame}{Backtraces}{Continuing on \funcname{caml\_stash\_backtrace}}
  \begin{columns}[c]
    \begin{column}{0.5\textwidth}
      \minipage[c][0.75\textheight][s]{\columnwidth}
      \begin{itemize}
          \color{gray}
        \item A C function provided by the runtime (specific to native code)
        \item It scans the stack, between the stack pointer at the raising point up to the next trap, in order to collect the names of the detected function frames
        \item It uses the same technique and information as the Garbage Collector (GC) uses to scan the values live on the stack
      \end{itemize}
      \begin{itemize}
        \item For each function frame detected, store a pointer to the \typename{frame\_descr} in the \localname{domain\_state->backtrace\_buffer} array, effectively constructing the backtrace
      \end{itemize}
      \onslide<2->{
        \begin{itemize}
            \smallskip
          \item Constructed backtrace:
            \funcname{definitely\_raise},
            \onslide<3->{\funcname{probably\_raise}}
        \end{itemize}
      }
      \vfill
      \mintocaml[firstline=9,lastline=13,highlightlines=12]{../src/backtrace/backtrace.ml}
      \endminipage
    \end{column}
    \begin{column}{0.5\textwidth}
      \raggedleft
      \onslide*<1>{\listStashBacktrace[highlightlines={114-115}]{}}
      \onslide*<2>{\providecommand\step{3}\input{diagrams/backtrace/backtrace.tex}}%
      \onslide*<3>{\providecommand\step{4}\input{diagrams/backtrace/backtrace.tex}}%
    \end{column}
  \end{columns}
\end{frame}

\begin{frame}{Backtraces}{Back to \funcname{caml\_raise\_exn}}
  \begin{columns}[c]
    \begin{column}{0.5\textwidth}
      \minipage[c][0.66\textheight][s]{\columnwidth}
      \begin{itemize}\color{gray}
        \item Checks wheteher exceptions backtrace should be recorded, by checking the \funcarg{domain\_state->backtrace\_active} value
        \item Switches to the C stack to be able to execute C code
        \item Calls \funcname{caml\_stash\_backtrace}, to collect the backtrace up to the next trap
      \end{itemize}
      \onslide*<2->{
        \begin{itemize}
          \item<2-> Executes the exception handler in \funcname{probably\_raise} with \funcname{RESTORE\_EXN\_HANDLER\_OCAML}
            \begin{itemize}
              \item Set \regname{RSP} to \localname{domain\_state->exn\_handler} value
              \item Restore \localname{domain\_state->exn\_handler} from trap
              \item Jump to the handler code with \funcname{ret}
            \end{itemize}
        \end{itemize}
      }
      \vfill
      \onslide*<1>{\mintocaml[firstline=9,lastline=13,highlightlines=12]{../src/backtrace/backtrace.ml}}
      \onslide*<2->{\mintocaml[firstline=15,lastline=21,highlightlines={19-21}]{../src/backtrace/backtrace.ml}}
      \endminipage
    \end{column}
    \begin{column}{0.5\textwidth}
      \centering
      \onslide*<1>{
        \mintc[firstline=190,lastline=192]{../ocaml/runtime/amd64.S}
        \mintc[firstline=234,lastline=237]{../ocaml/runtime/amd64.S}
      }
      \onslide*<2>{
        \mintc[firstline=190,lastline=192,highlightlines={191-192}]{../ocaml/runtime/amd64.S}
        \mintc[firstline=234,lastline=237,highlightlines={235-237}]{../ocaml/runtime/amd64.S}
      }
      \onslide*<1>{\listCamlRaiseExn[highlightlines=720]{}}
      \onslide*<2>{\listCamlRaiseExn[highlightlines=722]{}}
      \onslide*<3->{\providecommand\step{5}\input{diagrams/backtrace/backtrace.tex}}
    \end{column}
  \end{columns}
\end{frame}

\begin{frame}{Backtraces}{\funcname{caml\_probably\_raise}'s handler doesn't catch \typename{ExnC}}
  \begin{columns}[c]
    \begin{column}{0.45\textwidth}
      \minipage[c][0.75\textheight][s]{\columnwidth}
      \begin{itemize}
        \item<1-> \funcname{match \dots with}\footnotemark pattern matching can also match exceptions
        \item<1-> The CMM of \funcname{probably\_raise} shows that this \funcname{match \dots with} is implemented with a pattern matching and a \funcname{try \dots with}
        \item<1-> But this one only matches on \typename{ExnA} exception
        \item<2-> Since the pattern matching fails to catch the exception, \funcname{reraise} is called with our current \typename{ExnC} exception
        \item<3-> Disassembly shows that \funcname{reraise} is actually a call to \funcname{caml\_reraise\_exn}, which is also an assembly function provided by the runtime
      \end{itemize}
      \vfill
      \mintocaml[firstline=15,lastline=21,highlightlines={18-21}]{../src/backtrace/backtrace.ml}
      \endminipage
    \end{column}
    \begin{column}{0.55\textwidth}
      \centering
      \onslide*<1>{\mintcmm[highlightlines={10-18}]{../src/backtrace/camlBacktrace\_\_probably\_raise\_351.cmm}}
      \onslide*<2>{\listcmm[highlightlines=18]{../src/backtrace/camlBacktrace\_\_probably\_raise_351.cmm}{CMM of probably\_raise}}
      \onslide*<3>{\mintobjdump[firstline=5,lastline=35,highlightlines=31]{../src/backtrace/camlBacktrace\_\_probably\_raise_351.objdump}}
    \end{column}
  \end{columns}
  \only<1>{\footnotetext[1]{https://blog.janestreet.com/pattern-matching-and-exception-handling-unite/}}
\end{frame}

\newcommand\listCamlReraiseExn[1][]{
  \listasm[firstline=727,lastline=734,#1]{../ocaml/runtime/amd64.S}{runtime/amd64.S:727}}

\begin{frame}{Backtraces}{Introducing \funcname{caml\_reraise\_exn}}
  \begin{columns}[c]
    \begin{column}{0.5\textwidth}
      \minipage[c][0.66\textheight][s]{\columnwidth}
      \begin{itemize}
        \item<1-> The exception have to be reraised to the next handler
        \item<2-> First, checks if the backtrace should be collected with \funcarg{domain\_state->backtrace\_active}
        \only<3>{
        \begin{itemize}
            \item If not, behaves like a \funcname{raise\_notrace} with \funcname{RESTORE\_EXN\_HANDLER\_OCAML}
        \end{itemize}
        }
        \item<4-> Jumps into \funcname{caml\_raise\_exn}
        \item<5-> Calls \funcname{caml\_stash\_backtrace} again, to scan the next part of the stack: from the trap just pop-ed to the next trap
      \end{itemize}
      \vfill
      \mintocaml[firstline=15,lastline=21,highlightlines={18-21}]{../src/backtrace/backtrace.ml}
      \endminipage
    \end{column}
    \begin{column}{0.5\textwidth}
      \raggedleft
      \onslide*<1>{\listCamlReraiseExn{}}
      \onslide*<2>{\listCamlReraiseExn[highlightlines=729]{}}
      \onslide*<3>{
        \mintc[firstline=190,lastline=192,highlightlines={191-192}]{../ocaml/runtime/amd64.S}
        \mintc[firstline=234,lastline=237,highlightlines={235-237}]{../ocaml/runtime/amd64.S}
        \medskip
        \listCamlReraiseExn[highlightlines=731]{}
      }
      \onslide*<4-5>{
        % TODO refactor with \listCamlReraiseExn
        \mintasm[firstline=727,lastline=734,highlightlines=730]{../ocaml/runtime/amd64.S}
        \medskip
      }
      \onslide*<4>{\listCamlRaiseExn[highlightlines=711]{}}
      \onslide*<5>{\listCamlRaiseExn[highlightlines=720]{}}
    \end{column}
  \end{columns}
\end{frame}

\begin{frame}{Backtraces}{Backtrace collection continues}
  \begin{columns}[c]
    \begin{column}{0.55\textwidth}
      \minipage[c][0.75\textheight][s]{\columnwidth}
      \begin{itemize}
        \item<1-> \funcname{caml\_stash\_backtrace} scans the older part of the stack, up to the next trap
        \item<6-> Controls jumps to the nested exception handler in \funcname{maybe\_raise}, which matches only on \typename{ExnB}
        \item<7-> \funcname{caml\_reraise\_exn} is called again, backtrace collection resumes with \funcname{caml\_stash\_backtrace}
      \end{itemize}
      \medskip
      \begin{itemize}
        \item<2-> Constructed backtrace:
        \begin{itemize}
          \item <2-> \funcname{definitely\_raise}
          \item <2-> \funcname{probably\_raise}
          \item <3-> \funcname{probably\_raise} (re-raise)
          \item <4-> \funcname{maybe\_raise}
          \item <5-> \funcname{main}
          \item <8-> \funcname{main} (re-raise)
        \end{itemize}
      \end{itemize}
      \vfill
      \onslide*<1-5>{\mintocaml[firstline=15,lastline=21]{../src/backtrace/backtrace.ml}}
      \onslide*<6>{\mintocaml[firstline=28,lastline=34,highlightlines=31]{../src/backtrace/backtrace.ml}}
      \onslide*<7->{\mintocaml[firstline=28,lastline=34,highlightlines=33]{../src/backtrace/backtrace.ml}}
      \endminipage
    \end{column}
    \begin{column}{0.45\textwidth}
      \raggedleft
      \onslide*<1>{\providecommand\step{6}\input{diagrams/backtrace/backtrace.tex}}%
      \onslide*<2>{\providecommand\step{6}\input{diagrams/backtrace/backtrace.tex}}%
      \onslide*<3>{\providecommand\step{7}\input{diagrams/backtrace/backtrace.tex}}%
      \onslide*<4>{\providecommand\step{8}\input{diagrams/backtrace/backtrace.tex}}%
      \onslide*<5>{\providecommand\step{9}\input{diagrams/backtrace/backtrace.tex}}%
      \onslide*<6>{\providecommand\step{10}\input{diagrams/backtrace/backtrace.tex}}%
      \onslide*<7>{\providecommand\step{11}\input{diagrams/backtrace/backtrace.tex}}%
      \onslide*<8>{\providecommand\step{12}\input{diagrams/backtrace/backtrace.tex}}%
      \onslide*<9>{\providecommand\step{13}\input{diagrams/backtrace/backtrace.tex}}%
    \end{column}
  \end{columns}
\end{frame}

\begin{frame}{Backtraces}{Printing the backtrace}
  \begin{columns}[c]
    \begin{column}{0.45\textwidth}
      \minipage[c][0.66\textheight][s]{\columnwidth}
      \begin{itemize}
        \item<1-> The exception backtrace can now be printed to \funcarg{stdout} with \funcname{Printexc.print_backtrace}
        \item<2-> First the exception backtrace is retrieved into an OCaml array with \funcname{get_raw_backtrace }
        \item<3-> Which is implemented as the \funcname{caml_get_exception_raw_backtrace} C function in the runtime
        \item<4-> And finally printed with \funcname{print_exception_backtrace}
      \end{itemize}
      \vfill
      \mintocaml[firstline=28,lastline=34,highlightlines=33]{../src/backtrace/backtrace.ml}
      \endminipage
    \end{column}
    \begin{column}{0.55\textwidth}
      \onslide*<1,2>{
        \mintocaml[firstline=100,lastline=101]{../ocaml/stdlib/printexc.ml}
      }
      \only<1>{
        \listocaml[firstline=170,lastline=172]{../ocaml/stdlib/printexc.ml}{stdlib/printexc.ml:170}
      }
      \only<2>{
        \listocaml[firstline=170,lastline=172,highlightlines=172]{../ocaml/stdlib/printexc.ml}{stdlib/printexc.ml:170}
      }
      \only<3>{
        \mintc[firstline=154,lastline=156]{../ocaml/runtime/backtrace.c}
        \listc[firstline=171,lastline=192,highlightlines={184-187}]{../ocaml/runtime/backtrace.c}{runtime/backtrace.c:154}
      }
      \only<4>{
        \listocaml[firstline=155,lastline=165]{../ocaml/stdlib/printexc.ml}{stdlib/printexc.ml:55}
      }
    \end{column}
  \end{columns}
\end{frame}


\subsection{The multiple kinds of raise}
\frameSubsection{}{}

\begin{frame}{Backtraces}{When backtraces are not needed}
  \begin{columns}[c]
    \begin{column}{0.45\textwidth}
      \minipage[c][0.66\textheight][s]{\columnwidth}
      \begin{itemize}
        \item<1-> What if the backtrace for an exception is never needed?
        \item<2-> It's possible to raise the exception explicitly with \funcname{raise\_notrace} to make raising as cheap as possible
        \item<3-> An example of use in the \funcname{dynlink} library of the OCaml compiler
      \end{itemize}
      \vfill
      \onslide<3->{
      \listocaml[firstline=205,lastline=209,highlightlines=207]{../ocaml/otherlibs/dynlink/dynlink\_compilerlibs/misc.ml}{otherlibs/dynlink/dynlink\_compilerlibs/misc.ml:205}
      }
      \endminipage
    \end{column}
    \begin{column}{0.55\textwidth}
    \only<4>{
      \mintcmm[highlightlines=3]{../src/notrace/camlNotrace\_\_fun_347.cmm}
      \listcmm{../src/notrace/camlNotrace\_\_all\_somes\_327.cmm}{all\_somes}
    }
    \onslide*<5->{
      \listobjdump[highlightlines={6-9}]{../src/notrace/camlNotrace\_\_fun_347.objdump}{anonymous function from \funcname{all\_somes}}
    }
    \end{column}
  \end{columns}
\end{frame}

\begin{frame}{Backtraces}{Summary table of the multiple kinds of raise}
  \begin{itemize}
    \item When debugging information is enabled and if the backtrace of an exception isn't needed, it's possible to raise with \funcname{raise\_notrace} for performance
    \item When debugging information is \emph{not} enabled, the compiler optimises \funcname{raise} calls to inline assembly (same effect as using \funcname{raise\_notrace})
  \end{itemize}
  \bigskip
  \centering
  \begin{tabular}{| l | c | c | c | c |}
    \hline
    \parbox{10em}{Debug information \\ (\funcarg{-g} flag)}  & \multicolumn{2}{|c|}{Disabled}                                     & \multicolumn{2}{|c|}{Enabled} \\
    \hline
    Raise kind                                               & \funcname{raise} & \funcname{raise\_notrace}                       & \funcname{raise} & \funcname{raise\_notrace} \\
    \hline
    Compilation                                              & \parbox{8em}{inline assembly\\ (optimization)} & inline assembly   & \funcname{caml\_raise\_exn} & inline assembly \\
    \hline
  \end{tabular}
\end{frame}


%
% Takeaway #5
%
\subsection*{Takeaway \#5}
\frameSubsectionTakeaway{}
\againframe<5>{takeaway}
}%
      \onslide*<2>{\providecommand\step{1}\input{diagrams/backtrace/scanstack.tex}}%
      \onslide*<3>{\providecommand\step{2}\input{diagrams/backtrace/scanstack.tex}}%
      \onslide*<4>{\providecommand\step{3}\input{diagrams/backtrace/scanstack.tex}}%
      \onslide*<5>{\providecommand\step{4}\input{diagrams/backtrace/scanstack.tex}}%
      \onslide*<6>{\providecommand\step{5}\input{diagrams/backtrace/scanstack.tex}}%
    \end{column}
  \end{columns}
\end{frame}

% Duplicate of previous-previous slide
\begin{frame}{Backtraces}{Continuing on \funcname{caml\_stash\_backtrace}}
  \begin{columns}[c]
    \begin{column}{0.5\textwidth}
      \minipage[c][0.75\textheight][s]{\columnwidth}
      \begin{itemize}
          \color{gray}
        \item A C function provided by the runtime (specific to native code)
        \item It scans the stack, between the stack pointer at the raising point up to the next trap, in order to collect the names of the detected function frames
        \item It uses the same technique and information as the Garbage Collector (GC) uses to scan the values live on the stack
      \end{itemize}
      \begin{itemize}
        \item For each function frame detected, store a pointer to the \typename{frame\_descr} in the \localname{domain\_state->backtrace\_buffer} array, effectively constructing the backtrace
      \end{itemize}
      \onslide<2->{
        \begin{itemize}
            \smallskip
          \item Constructed backtrace:
            \funcname{definitely\_raise},
            \onslide<3->{\funcname{probably\_raise}}
        \end{itemize}
      }
      \vfill
      \mintocaml[firstline=9,lastline=13,highlightlines=12]{../src/backtrace/backtrace.ml}
      \endminipage
    \end{column}
    \begin{column}{0.5\textwidth}
      \raggedleft
      \onslide*<1>{\listStashBacktrace[highlightlines={114-115}]{}}
      \onslide*<2>{\providecommand\step{3}%
% Backtraces
%
\subsection{One advantage of raising with debugging information}
\frameSubsection{}{}

\begin{frame}{Backtraces}{Debugging information \& source code}
  \begin{columns}[c]
    \begin{column}{0.5\textwidth}
      \begin{itemize}
        \item Raising an exception is fun and games, but how do we know where the exception originated? $\rightarrow$ With the backtrace associated with the exception!
        \item In order to get a backtrace from the point where an exception was raised, it's necessary to compile with debugging information enabled (\funcarg{-g} flag for the compiler)
        \item Same example as in the `Nested handlers' section
      \end{itemize}
    \end{column}
    \begin{column}{0.5\textwidth}
      \listocaml[firstline=9,lastline=34]{../src/backtrace/backtrace.ml}{backtrace/backtrace.ml}
    \end{column}
  \end{columns}
\end{frame}

\begin{frame}{Backtraces}{CMM and disassembly of \funcname{definitely\_raise}}
  \begin{columns}[c]
    \begin{column}{0.5\textwidth}
      \minipage[c][0.66\textheight][s]{\columnwidth}
      \begin{itemize}
        \item<1-> An effect of compiling with debugging information (\funcarg{-g}): the CMM no longer uses \funcname{raise\_notrace} to raise the exception but \funcname{raise} instead
        \item<2-> Disassembly shows that CMM \funcname{raise} is actually a call to \funcname{caml\_raise\_exn}, a function in assembler provided by the runtime
      \end{itemize}
      \vfill
      \mintocaml[firstline=9,lastline=13,highlightlines=12]{../src/backtrace/backtrace.ml}
      \endminipage
    \end{column}
    \begin{column}{0.5\textwidth}
      \onslide*<1>{
        \mintcmm[highlightlines=5]{../src/backtrace/camlBacktrace\_\_definitely\_raise\_347.cmm}
      }
      \onslide*<2>{
        \mintobjdump[firstline=2,highlightlines=14]{../src/backtrace/camlBacktrace\_\_definitely\_raise\_347.objdump}
      }
    \end{column}
  \end{columns}
\end{frame}

\begin{frame}{Backtraces}{Introducing \funcname{caml\_raise\_exn}}
  \begin{columns}[c]
    \begin{column}{0.5\textwidth}
      \minipage[c][0.66\textheight][s]{\columnwidth}
      \onslide*<1>{
        \begin{itemize}
          \item Raising the exception is performed using \funcname{caml\_raise\_exn} which is
            \begin{itemize}
              \item An assembly function provided by the runtime
              \item Architecture specific
            \end{itemize}
          \item Let's see what it does differently from \funcname{raise\_notrace}\dots
          \item We are about to raise \typename{ExnC} with \funcname{caml\_raise\_exn} from \funcname{definitely\_raise}
        \end{itemize}
      }
      \onslide*<2>{
        \mintobjdump[firstline=2,highlightlines=14]{../src/backtrace/camlBacktrace\_\_definitely\_raise\_347.objdump}
      }
      \vfill
      \mintocaml[firstline=9,lastline=13,highlightlines=12]{../src/backtrace/backtrace.ml}
      \endminipage
    \end{column}
    \begin{column}{0.5\textwidth}
      \centering
      \providecommand\step{1}\input{diagrams/backtrace/backtrace.tex}
    \end{column}
  \end{columns}
\end{frame}

\begin{frame}{Backtraces}{But before that, we need more fields from \typename{caml\_domain\_state}}
  \begin{columns}
    \begin{column}{0.5\textwidth}
      \begin{itemize}
        \item Remember \typename{caml\_domain\_state} the per-domain runtime state?
        \item Some other members are of interest to us now\dots
        \item \localname{backtrace\_active} is used to know if the runtime should collect backtraces
        \item \localname{backtrace\_active} can be set by
          \begin{itemize}
            \item The \funcarg{b} flag in the \funcname{OCAMLRUNPARAM} environment variable
            \item \mintinline{ocaml}{Printexc.record_backtrace : bool -> unit} \footnotemark
          \end{itemize}
      \end{itemize}
    \end{column}
    \begin{column}{0.5\textwidth}
      \begin{listing}
        \mintc[highlightlines={9-11}]{src/ocaml/domain\_state\_backtrace.h}
        \caption{runtime/caml/domain\_state.h}
      \end{listing}
    \end{column}
  \end{columns}
  \footnotetext[1]{https://v2.ocaml.org/api/Printexc.html}
\end{frame}

\newcommand\listCamlRaiseExn[1][]{
  \listasm[firstline=702,lastline=725,#1]{../ocaml/runtime/amd64.S}{runtime/amd64.S:702}}

\begin{frame}{Backtraces}{Walk through \funcname{caml\_raise\_exn}}
  \begin{columns}[T]
    \begin{column}{0.5\textwidth}
      \begin{itemize}
        \item<1-> First checks whether exceptions backtrace should be recorded
        \item<1-> By checking the \funcarg{domain\_state->backtrace\_active} value
        \item<2-> If not, acts as \funcname{raise\_notrace} with \funcname{RESTORE\_EXN\_HANDLER\_OCAML}
          \begin{itemize}
            \item Set \regname{RSP} to \localname{domain\_state->exn\_handler} value
            \item Restore \localname{domain\_state->exn\_handler} from trap
            \item Jump with \funcname{ret} (same effect as using \funcname{pop} and \funcname{jmp})
          \end{itemize}
        \item<3-> Otherwise, switches to the C stack to be able to execute C code
        \item<4-> Calls \funcname{caml\_stash\_backtrace}, a C function provided by the runtime\ignorespaces
          \onslide<6->{, with
            \begin{itemize}
              \item \funcarg{exn}: exception value
              \item \funcarg{pc}: code address of the raising point
              \item \funcarg{sp}: stack address of the raising function
              \item \funcarg{trapsp}: stack address of the next handler
            \end{itemize}
          }
      \end{itemize}
      \bigskip
      \mintocaml[firstline=9,lastline=13,highlightlines=12]{../src/backtrace/backtrace.ml}
    \end{column}
    \begin{column}{0.5\textwidth}
      \raggedleft
      \onslide*<1,3-4>{
        \mintc[firstline=190,lastline=192]{../ocaml/runtime/amd64.S}
        \mintc[firstline=234,lastline=237]{../ocaml/runtime/amd64.S}
      }
      \onslide*<2>{
        \mintc[firstline=190,lastline=192,highlightlines={191-192}]{../ocaml/runtime/amd64.S}
        \mintc[firstline=234,lastline=237,highlightlines={235-237}]{../ocaml/runtime/amd64.S}
      }
      \onslide*<1>{\listCamlRaiseExn[highlightlines=705]{}}%
      \onslide*<2>{\listCamlRaiseExn[highlightlines={707-708}]{}}%
      \onslide*<3>{\listCamlRaiseExn[highlightlines={712-714}]{}}%
      \onslide*<4>{\listCamlRaiseExn[highlightlines={715-720}]{}}%
      \onslide*<5>{\providecommand\step{1}\input{diagrams/backtrace/backtrace.tex}}%
      \onslide*<6>{\providecommand\step{2}\input{diagrams/backtrace/backtrace.tex}}%
    \end{column}
  \end{columns}
\end{frame}

\newcommand\listStashBacktrace[1][]{
  \listc[firstline=91,lastline=120,#1]{../ocaml/runtime/backtrace\_nat.c}{runtime/backtrace\_nat.c:91}}

\begin{frame}{Backtraces}{Introducing \funcname{caml\_stash\_backtrace}}
  \begin{columns}[c]
    \begin{column}{0.5\textwidth}
      \minipage[c][0.66\textheight][s]{\columnwidth}
      \begin{itemize}
        \item<1-> A C function provided by the runtime (specific to native code)
        \item<2-> It scans the stack, between the stack pointer at the raising point up to the next trap, in order to collect the names of the detected function frames
          \onslide*<2>{
            \begin{itemize}
              \item And more information, like line number, column number...
            \end{itemize}
          }
        \item<3-> It uses the same technique and information as the Garbage Collector (GC) uses to scan the values live on the stack
          \onslide*<4>{
            \begin{itemize}
              \item \typename{frame\_descr} are at the core of stack unwinding
            \end{itemize}
          }
      \end{itemize}
      \vfill
      \mintocaml[firstline=9,lastline=13,highlightlines=12]{../src/backtrace/backtrace.ml}
      \endminipage
    \end{column}
    \begin{column}{0.5\textwidth}
      \onslide*<1>{\listStashBacktrace[highlightlines=91]{}}
      \onslide*<2>{\listStashBacktrace[highlightlines={91,117-118}]{}}
      \onslide*<3->{\listStashBacktrace[highlightlines={94,106,109-110}]{}}
    \end{column}
  \end{columns}
\end{frame}

\newcommand\listFrameDescr[1][]{
  \listocaml[firstline=111,lastline=124,#1]{../ocaml/asmcomp/emitaux.ml}{asmcomp/emitaux.ml:111}}
\newcommand\listDebugInfo[1][]{
  \listocaml[firstline=38,lastline=49,#1]{../ocaml/lambda/debuginfo.mli}{lambda/debuginfo.mli:38}}

\begin{frame}{Backtraces}{\typename{frame\_descr} and \typename{frame\_debuginfo} in a nutshell}
  \begin{columns}[c]
    \begin{column}{0.5\textwidth}
      \begin{itemize}
        \item<1-> For every return address in an OCaml program, there is a associated \typename{frame\_descr}
        \item<2-> A \typename{frame\_descr} specifies
        \begin{itemize}
          \item<2-> The size of the function stack frame
          \item<3-> Where live values are on the stack (so that the GC can mark them as live and prevent from being swept)
        \end{itemize}
        \item<4-> When compiled with debug information, \typename{frame\_descr} will be linked to a \typename{frame\_debuginfo}
        \begin{itemize}
          \item<5-> Contains information about the position in the source code (file name, line and column number)
        \end{itemize}
      \end{itemize}
    \end{column}
    \begin{column}{0.5\textwidth}
        \onslide*<1>{\listFrameDescr[highlightlines={118-122}]{}}
        \onslide*<2>{\listFrameDescr[highlightlines={120}]{}}
        \onslide*<3>{\listFrameDescr[highlightlines={121}]{}}
        \onslide*<4->{\listFrameDescr[highlightlines={122}]{}}
        \medskip
        \onslide*<1-3>{\listDebugInfo{}}
        \onslide*<4>{\listDebugInfo[highlightlines={38,49}]{}}
        \onslide*<5>{\listDebugInfo[highlightlines={39-46}]{}}
    \end{column}
  \end{columns}
\end{frame}

\begin{frame}{Backtraces}{Scanning the stack with \typename{frame\_descr}}
  \begin{columns}[c]
    \begin{column}{0.5\textwidth}
      \begin{itemize}
        \item Given the return address of the code that raises the exception by calling \funcname{caml\_raise\_exn} (\funcarg{pc} argument of \funcname{caml\_stash\_backtrace})
        \item And the position of the stack pointer (\funcarg{sp} argument of \funcname{caml\_stash\_backtrace})
        \item The frame size given by \typename{frame\_descr}.\funcarg{frame\_size}, allows to find the end of the previous frame
        \item And the return address of the caller of this function
        \bigskip
        \onslide<3->{
        \item Note that traps (here the reddish \funcname{ExnA}, \funcname{ExnB} and \funcname{ExnC} blocks) belong to the frame that installed them, and so the frame size changes when traps are removed
        }
      \end{itemize}
    \end{column}
    \begin{column}{0.5\textwidth}
      \raggedleft
      \onslide*<1>{\providecommand\step{2}\input{diagrams/backtrace/backtrace.tex}}%
      \onslide*<2>{\providecommand\step{1}\input{diagrams/backtrace/scanstack.tex}}%
      \onslide*<3>{\providecommand\step{2}\input{diagrams/backtrace/scanstack.tex}}%
      \onslide*<4>{\providecommand\step{3}\input{diagrams/backtrace/scanstack.tex}}%
      \onslide*<5>{\providecommand\step{4}\input{diagrams/backtrace/scanstack.tex}}%
      \onslide*<6>{\providecommand\step{5}\input{diagrams/backtrace/scanstack.tex}}%
    \end{column}
  \end{columns}
\end{frame}

% Duplicate of previous-previous slide
\begin{frame}{Backtraces}{Continuing on \funcname{caml\_stash\_backtrace}}
  \begin{columns}[c]
    \begin{column}{0.5\textwidth}
      \minipage[c][0.75\textheight][s]{\columnwidth}
      \begin{itemize}
          \color{gray}
        \item A C function provided by the runtime (specific to native code)
        \item It scans the stack, between the stack pointer at the raising point up to the next trap, in order to collect the names of the detected function frames
        \item It uses the same technique and information as the Garbage Collector (GC) uses to scan the values live on the stack
      \end{itemize}
      \begin{itemize}
        \item For each function frame detected, store a pointer to the \typename{frame\_descr} in the \localname{domain\_state->backtrace\_buffer} array, effectively constructing the backtrace
      \end{itemize}
      \onslide<2->{
        \begin{itemize}
            \smallskip
          \item Constructed backtrace:
            \funcname{definitely\_raise},
            \onslide<3->{\funcname{probably\_raise}}
        \end{itemize}
      }
      \vfill
      \mintocaml[firstline=9,lastline=13,highlightlines=12]{../src/backtrace/backtrace.ml}
      \endminipage
    \end{column}
    \begin{column}{0.5\textwidth}
      \raggedleft
      \onslide*<1>{\listStashBacktrace[highlightlines={114-115}]{}}
      \onslide*<2>{\providecommand\step{3}\input{diagrams/backtrace/backtrace.tex}}%
      \onslide*<3>{\providecommand\step{4}\input{diagrams/backtrace/backtrace.tex}}%
    \end{column}
  \end{columns}
\end{frame}

\begin{frame}{Backtraces}{Back to \funcname{caml\_raise\_exn}}
  \begin{columns}[c]
    \begin{column}{0.5\textwidth}
      \minipage[c][0.66\textheight][s]{\columnwidth}
      \begin{itemize}\color{gray}
        \item Checks wheteher exceptions backtrace should be recorded, by checking the \funcarg{domain\_state->backtrace\_active} value
        \item Switches to the C stack to be able to execute C code
        \item Calls \funcname{caml\_stash\_backtrace}, to collect the backtrace up to the next trap
      \end{itemize}
      \onslide*<2->{
        \begin{itemize}
          \item<2-> Executes the exception handler in \funcname{probably\_raise} with \funcname{RESTORE\_EXN\_HANDLER\_OCAML}
            \begin{itemize}
              \item Set \regname{RSP} to \localname{domain\_state->exn\_handler} value
              \item Restore \localname{domain\_state->exn\_handler} from trap
              \item Jump to the handler code with \funcname{ret}
            \end{itemize}
        \end{itemize}
      }
      \vfill
      \onslide*<1>{\mintocaml[firstline=9,lastline=13,highlightlines=12]{../src/backtrace/backtrace.ml}}
      \onslide*<2->{\mintocaml[firstline=15,lastline=21,highlightlines={19-21}]{../src/backtrace/backtrace.ml}}
      \endminipage
    \end{column}
    \begin{column}{0.5\textwidth}
      \centering
      \onslide*<1>{
        \mintc[firstline=190,lastline=192]{../ocaml/runtime/amd64.S}
        \mintc[firstline=234,lastline=237]{../ocaml/runtime/amd64.S}
      }
      \onslide*<2>{
        \mintc[firstline=190,lastline=192,highlightlines={191-192}]{../ocaml/runtime/amd64.S}
        \mintc[firstline=234,lastline=237,highlightlines={235-237}]{../ocaml/runtime/amd64.S}
      }
      \onslide*<1>{\listCamlRaiseExn[highlightlines=720]{}}
      \onslide*<2>{\listCamlRaiseExn[highlightlines=722]{}}
      \onslide*<3->{\providecommand\step{5}\input{diagrams/backtrace/backtrace.tex}}
    \end{column}
  \end{columns}
\end{frame}

\begin{frame}{Backtraces}{\funcname{caml\_probably\_raise}'s handler doesn't catch \typename{ExnC}}
  \begin{columns}[c]
    \begin{column}{0.45\textwidth}
      \minipage[c][0.75\textheight][s]{\columnwidth}
      \begin{itemize}
        \item<1-> \funcname{match \dots with}\footnotemark pattern matching can also match exceptions
        \item<1-> The CMM of \funcname{probably\_raise} shows that this \funcname{match \dots with} is implemented with a pattern matching and a \funcname{try \dots with}
        \item<1-> But this one only matches on \typename{ExnA} exception
        \item<2-> Since the pattern matching fails to catch the exception, \funcname{reraise} is called with our current \typename{ExnC} exception
        \item<3-> Disassembly shows that \funcname{reraise} is actually a call to \funcname{caml\_reraise\_exn}, which is also an assembly function provided by the runtime
      \end{itemize}
      \vfill
      \mintocaml[firstline=15,lastline=21,highlightlines={18-21}]{../src/backtrace/backtrace.ml}
      \endminipage
    \end{column}
    \begin{column}{0.55\textwidth}
      \centering
      \onslide*<1>{\mintcmm[highlightlines={10-18}]{../src/backtrace/camlBacktrace\_\_probably\_raise\_351.cmm}}
      \onslide*<2>{\listcmm[highlightlines=18]{../src/backtrace/camlBacktrace\_\_probably\_raise_351.cmm}{CMM of probably\_raise}}
      \onslide*<3>{\mintobjdump[firstline=5,lastline=35,highlightlines=31]{../src/backtrace/camlBacktrace\_\_probably\_raise_351.objdump}}
    \end{column}
  \end{columns}
  \only<1>{\footnotetext[1]{https://blog.janestreet.com/pattern-matching-and-exception-handling-unite/}}
\end{frame}

\newcommand\listCamlReraiseExn[1][]{
  \listasm[firstline=727,lastline=734,#1]{../ocaml/runtime/amd64.S}{runtime/amd64.S:727}}

\begin{frame}{Backtraces}{Introducing \funcname{caml\_reraise\_exn}}
  \begin{columns}[c]
    \begin{column}{0.5\textwidth}
      \minipage[c][0.66\textheight][s]{\columnwidth}
      \begin{itemize}
        \item<1-> The exception have to be reraised to the next handler
        \item<2-> First, checks if the backtrace should be collected with \funcarg{domain\_state->backtrace\_active}
        \only<3>{
        \begin{itemize}
            \item If not, behaves like a \funcname{raise\_notrace} with \funcname{RESTORE\_EXN\_HANDLER\_OCAML}
        \end{itemize}
        }
        \item<4-> Jumps into \funcname{caml\_raise\_exn}
        \item<5-> Calls \funcname{caml\_stash\_backtrace} again, to scan the next part of the stack: from the trap just pop-ed to the next trap
      \end{itemize}
      \vfill
      \mintocaml[firstline=15,lastline=21,highlightlines={18-21}]{../src/backtrace/backtrace.ml}
      \endminipage
    \end{column}
    \begin{column}{0.5\textwidth}
      \raggedleft
      \onslide*<1>{\listCamlReraiseExn{}}
      \onslide*<2>{\listCamlReraiseExn[highlightlines=729]{}}
      \onslide*<3>{
        \mintc[firstline=190,lastline=192,highlightlines={191-192}]{../ocaml/runtime/amd64.S}
        \mintc[firstline=234,lastline=237,highlightlines={235-237}]{../ocaml/runtime/amd64.S}
        \medskip
        \listCamlReraiseExn[highlightlines=731]{}
      }
      \onslide*<4-5>{
        % TODO refactor with \listCamlReraiseExn
        \mintasm[firstline=727,lastline=734,highlightlines=730]{../ocaml/runtime/amd64.S}
        \medskip
      }
      \onslide*<4>{\listCamlRaiseExn[highlightlines=711]{}}
      \onslide*<5>{\listCamlRaiseExn[highlightlines=720]{}}
    \end{column}
  \end{columns}
\end{frame}

\begin{frame}{Backtraces}{Backtrace collection continues}
  \begin{columns}[c]
    \begin{column}{0.55\textwidth}
      \minipage[c][0.75\textheight][s]{\columnwidth}
      \begin{itemize}
        \item<1-> \funcname{caml\_stash\_backtrace} scans the older part of the stack, up to the next trap
        \item<6-> Controls jumps to the nested exception handler in \funcname{maybe\_raise}, which matches only on \typename{ExnB}
        \item<7-> \funcname{caml\_reraise\_exn} is called again, backtrace collection resumes with \funcname{caml\_stash\_backtrace}
      \end{itemize}
      \medskip
      \begin{itemize}
        \item<2-> Constructed backtrace:
        \begin{itemize}
          \item <2-> \funcname{definitely\_raise}
          \item <2-> \funcname{probably\_raise}
          \item <3-> \funcname{probably\_raise} (re-raise)
          \item <4-> \funcname{maybe\_raise}
          \item <5-> \funcname{main}
          \item <8-> \funcname{main} (re-raise)
        \end{itemize}
      \end{itemize}
      \vfill
      \onslide*<1-5>{\mintocaml[firstline=15,lastline=21]{../src/backtrace/backtrace.ml}}
      \onslide*<6>{\mintocaml[firstline=28,lastline=34,highlightlines=31]{../src/backtrace/backtrace.ml}}
      \onslide*<7->{\mintocaml[firstline=28,lastline=34,highlightlines=33]{../src/backtrace/backtrace.ml}}
      \endminipage
    \end{column}
    \begin{column}{0.45\textwidth}
      \raggedleft
      \onslide*<1>{\providecommand\step{6}\input{diagrams/backtrace/backtrace.tex}}%
      \onslide*<2>{\providecommand\step{6}\input{diagrams/backtrace/backtrace.tex}}%
      \onslide*<3>{\providecommand\step{7}\input{diagrams/backtrace/backtrace.tex}}%
      \onslide*<4>{\providecommand\step{8}\input{diagrams/backtrace/backtrace.tex}}%
      \onslide*<5>{\providecommand\step{9}\input{diagrams/backtrace/backtrace.tex}}%
      \onslide*<6>{\providecommand\step{10}\input{diagrams/backtrace/backtrace.tex}}%
      \onslide*<7>{\providecommand\step{11}\input{diagrams/backtrace/backtrace.tex}}%
      \onslide*<8>{\providecommand\step{12}\input{diagrams/backtrace/backtrace.tex}}%
      \onslide*<9>{\providecommand\step{13}\input{diagrams/backtrace/backtrace.tex}}%
    \end{column}
  \end{columns}
\end{frame}

\begin{frame}{Backtraces}{Printing the backtrace}
  \begin{columns}[c]
    \begin{column}{0.45\textwidth}
      \minipage[c][0.66\textheight][s]{\columnwidth}
      \begin{itemize}
        \item<1-> The exception backtrace can now be printed to \funcarg{stdout} with \funcname{Printexc.print_backtrace}
        \item<2-> First the exception backtrace is retrieved into an OCaml array with \funcname{get_raw_backtrace }
        \item<3-> Which is implemented as the \funcname{caml_get_exception_raw_backtrace} C function in the runtime
        \item<4-> And finally printed with \funcname{print_exception_backtrace}
      \end{itemize}
      \vfill
      \mintocaml[firstline=28,lastline=34,highlightlines=33]{../src/backtrace/backtrace.ml}
      \endminipage
    \end{column}
    \begin{column}{0.55\textwidth}
      \onslide*<1,2>{
        \mintocaml[firstline=100,lastline=101]{../ocaml/stdlib/printexc.ml}
      }
      \only<1>{
        \listocaml[firstline=170,lastline=172]{../ocaml/stdlib/printexc.ml}{stdlib/printexc.ml:170}
      }
      \only<2>{
        \listocaml[firstline=170,lastline=172,highlightlines=172]{../ocaml/stdlib/printexc.ml}{stdlib/printexc.ml:170}
      }
      \only<3>{
        \mintc[firstline=154,lastline=156]{../ocaml/runtime/backtrace.c}
        \listc[firstline=171,lastline=192,highlightlines={184-187}]{../ocaml/runtime/backtrace.c}{runtime/backtrace.c:154}
      }
      \only<4>{
        \listocaml[firstline=155,lastline=165]{../ocaml/stdlib/printexc.ml}{stdlib/printexc.ml:55}
      }
    \end{column}
  \end{columns}
\end{frame}


\subsection{The multiple kinds of raise}
\frameSubsection{}{}

\begin{frame}{Backtraces}{When backtraces are not needed}
  \begin{columns}[c]
    \begin{column}{0.45\textwidth}
      \minipage[c][0.66\textheight][s]{\columnwidth}
      \begin{itemize}
        \item<1-> What if the backtrace for an exception is never needed?
        \item<2-> It's possible to raise the exception explicitly with \funcname{raise\_notrace} to make raising as cheap as possible
        \item<3-> An example of use in the \funcname{dynlink} library of the OCaml compiler
      \end{itemize}
      \vfill
      \onslide<3->{
      \listocaml[firstline=205,lastline=209,highlightlines=207]{../ocaml/otherlibs/dynlink/dynlink\_compilerlibs/misc.ml}{otherlibs/dynlink/dynlink\_compilerlibs/misc.ml:205}
      }
      \endminipage
    \end{column}
    \begin{column}{0.55\textwidth}
    \only<4>{
      \mintcmm[highlightlines=3]{../src/notrace/camlNotrace\_\_fun_347.cmm}
      \listcmm{../src/notrace/camlNotrace\_\_all\_somes\_327.cmm}{all\_somes}
    }
    \onslide*<5->{
      \listobjdump[highlightlines={6-9}]{../src/notrace/camlNotrace\_\_fun_347.objdump}{anonymous function from \funcname{all\_somes}}
    }
    \end{column}
  \end{columns}
\end{frame}

\begin{frame}{Backtraces}{Summary table of the multiple kinds of raise}
  \begin{itemize}
    \item When debugging information is enabled and if the backtrace of an exception isn't needed, it's possible to raise with \funcname{raise\_notrace} for performance
    \item When debugging information is \emph{not} enabled, the compiler optimises \funcname{raise} calls to inline assembly (same effect as using \funcname{raise\_notrace})
  \end{itemize}
  \bigskip
  \centering
  \begin{tabular}{| l | c | c | c | c |}
    \hline
    \parbox{10em}{Debug information \\ (\funcarg{-g} flag)}  & \multicolumn{2}{|c|}{Disabled}                                     & \multicolumn{2}{|c|}{Enabled} \\
    \hline
    Raise kind                                               & \funcname{raise} & \funcname{raise\_notrace}                       & \funcname{raise} & \funcname{raise\_notrace} \\
    \hline
    Compilation                                              & \parbox{8em}{inline assembly\\ (optimization)} & inline assembly   & \funcname{caml\_raise\_exn} & inline assembly \\
    \hline
  \end{tabular}
\end{frame}


%
% Takeaway #5
%
\subsection*{Takeaway \#5}
\frameSubsectionTakeaway{}
\againframe<5>{takeaway}
}%
      \onslide*<3>{\providecommand\step{4}%
% Backtraces
%
\subsection{One advantage of raising with debugging information}
\frameSubsection{}{}

\begin{frame}{Backtraces}{Debugging information \& source code}
  \begin{columns}[c]
    \begin{column}{0.5\textwidth}
      \begin{itemize}
        \item Raising an exception is fun and games, but how do we know where the exception originated? $\rightarrow$ With the backtrace associated with the exception!
        \item In order to get a backtrace from the point where an exception was raised, it's necessary to compile with debugging information enabled (\funcarg{-g} flag for the compiler)
        \item Same example as in the `Nested handlers' section
      \end{itemize}
    \end{column}
    \begin{column}{0.5\textwidth}
      \listocaml[firstline=9,lastline=34]{../src/backtrace/backtrace.ml}{backtrace/backtrace.ml}
    \end{column}
  \end{columns}
\end{frame}

\begin{frame}{Backtraces}{CMM and disassembly of \funcname{definitely\_raise}}
  \begin{columns}[c]
    \begin{column}{0.5\textwidth}
      \minipage[c][0.66\textheight][s]{\columnwidth}
      \begin{itemize}
        \item<1-> An effect of compiling with debugging information (\funcarg{-g}): the CMM no longer uses \funcname{raise\_notrace} to raise the exception but \funcname{raise} instead
        \item<2-> Disassembly shows that CMM \funcname{raise} is actually a call to \funcname{caml\_raise\_exn}, a function in assembler provided by the runtime
      \end{itemize}
      \vfill
      \mintocaml[firstline=9,lastline=13,highlightlines=12]{../src/backtrace/backtrace.ml}
      \endminipage
    \end{column}
    \begin{column}{0.5\textwidth}
      \onslide*<1>{
        \mintcmm[highlightlines=5]{../src/backtrace/camlBacktrace\_\_definitely\_raise\_347.cmm}
      }
      \onslide*<2>{
        \mintobjdump[firstline=2,highlightlines=14]{../src/backtrace/camlBacktrace\_\_definitely\_raise\_347.objdump}
      }
    \end{column}
  \end{columns}
\end{frame}

\begin{frame}{Backtraces}{Introducing \funcname{caml\_raise\_exn}}
  \begin{columns}[c]
    \begin{column}{0.5\textwidth}
      \minipage[c][0.66\textheight][s]{\columnwidth}
      \onslide*<1>{
        \begin{itemize}
          \item Raising the exception is performed using \funcname{caml\_raise\_exn} which is
            \begin{itemize}
              \item An assembly function provided by the runtime
              \item Architecture specific
            \end{itemize}
          \item Let's see what it does differently from \funcname{raise\_notrace}\dots
          \item We are about to raise \typename{ExnC} with \funcname{caml\_raise\_exn} from \funcname{definitely\_raise}
        \end{itemize}
      }
      \onslide*<2>{
        \mintobjdump[firstline=2,highlightlines=14]{../src/backtrace/camlBacktrace\_\_definitely\_raise\_347.objdump}
      }
      \vfill
      \mintocaml[firstline=9,lastline=13,highlightlines=12]{../src/backtrace/backtrace.ml}
      \endminipage
    \end{column}
    \begin{column}{0.5\textwidth}
      \centering
      \providecommand\step{1}\input{diagrams/backtrace/backtrace.tex}
    \end{column}
  \end{columns}
\end{frame}

\begin{frame}{Backtraces}{But before that, we need more fields from \typename{caml\_domain\_state}}
  \begin{columns}
    \begin{column}{0.5\textwidth}
      \begin{itemize}
        \item Remember \typename{caml\_domain\_state} the per-domain runtime state?
        \item Some other members are of interest to us now\dots
        \item \localname{backtrace\_active} is used to know if the runtime should collect backtraces
        \item \localname{backtrace\_active} can be set by
          \begin{itemize}
            \item The \funcarg{b} flag in the \funcname{OCAMLRUNPARAM} environment variable
            \item \mintinline{ocaml}{Printexc.record_backtrace : bool -> unit} \footnotemark
          \end{itemize}
      \end{itemize}
    \end{column}
    \begin{column}{0.5\textwidth}
      \begin{listing}
        \mintc[highlightlines={9-11}]{src/ocaml/domain\_state\_backtrace.h}
        \caption{runtime/caml/domain\_state.h}
      \end{listing}
    \end{column}
  \end{columns}
  \footnotetext[1]{https://v2.ocaml.org/api/Printexc.html}
\end{frame}

\newcommand\listCamlRaiseExn[1][]{
  \listasm[firstline=702,lastline=725,#1]{../ocaml/runtime/amd64.S}{runtime/amd64.S:702}}

\begin{frame}{Backtraces}{Walk through \funcname{caml\_raise\_exn}}
  \begin{columns}[T]
    \begin{column}{0.5\textwidth}
      \begin{itemize}
        \item<1-> First checks whether exceptions backtrace should be recorded
        \item<1-> By checking the \funcarg{domain\_state->backtrace\_active} value
        \item<2-> If not, acts as \funcname{raise\_notrace} with \funcname{RESTORE\_EXN\_HANDLER\_OCAML}
          \begin{itemize}
            \item Set \regname{RSP} to \localname{domain\_state->exn\_handler} value
            \item Restore \localname{domain\_state->exn\_handler} from trap
            \item Jump with \funcname{ret} (same effect as using \funcname{pop} and \funcname{jmp})
          \end{itemize}
        \item<3-> Otherwise, switches to the C stack to be able to execute C code
        \item<4-> Calls \funcname{caml\_stash\_backtrace}, a C function provided by the runtime\ignorespaces
          \onslide<6->{, with
            \begin{itemize}
              \item \funcarg{exn}: exception value
              \item \funcarg{pc}: code address of the raising point
              \item \funcarg{sp}: stack address of the raising function
              \item \funcarg{trapsp}: stack address of the next handler
            \end{itemize}
          }
      \end{itemize}
      \bigskip
      \mintocaml[firstline=9,lastline=13,highlightlines=12]{../src/backtrace/backtrace.ml}
    \end{column}
    \begin{column}{0.5\textwidth}
      \raggedleft
      \onslide*<1,3-4>{
        \mintc[firstline=190,lastline=192]{../ocaml/runtime/amd64.S}
        \mintc[firstline=234,lastline=237]{../ocaml/runtime/amd64.S}
      }
      \onslide*<2>{
        \mintc[firstline=190,lastline=192,highlightlines={191-192}]{../ocaml/runtime/amd64.S}
        \mintc[firstline=234,lastline=237,highlightlines={235-237}]{../ocaml/runtime/amd64.S}
      }
      \onslide*<1>{\listCamlRaiseExn[highlightlines=705]{}}%
      \onslide*<2>{\listCamlRaiseExn[highlightlines={707-708}]{}}%
      \onslide*<3>{\listCamlRaiseExn[highlightlines={712-714}]{}}%
      \onslide*<4>{\listCamlRaiseExn[highlightlines={715-720}]{}}%
      \onslide*<5>{\providecommand\step{1}\input{diagrams/backtrace/backtrace.tex}}%
      \onslide*<6>{\providecommand\step{2}\input{diagrams/backtrace/backtrace.tex}}%
    \end{column}
  \end{columns}
\end{frame}

\newcommand\listStashBacktrace[1][]{
  \listc[firstline=91,lastline=120,#1]{../ocaml/runtime/backtrace\_nat.c}{runtime/backtrace\_nat.c:91}}

\begin{frame}{Backtraces}{Introducing \funcname{caml\_stash\_backtrace}}
  \begin{columns}[c]
    \begin{column}{0.5\textwidth}
      \minipage[c][0.66\textheight][s]{\columnwidth}
      \begin{itemize}
        \item<1-> A C function provided by the runtime (specific to native code)
        \item<2-> It scans the stack, between the stack pointer at the raising point up to the next trap, in order to collect the names of the detected function frames
          \onslide*<2>{
            \begin{itemize}
              \item And more information, like line number, column number...
            \end{itemize}
          }
        \item<3-> It uses the same technique and information as the Garbage Collector (GC) uses to scan the values live on the stack
          \onslide*<4>{
            \begin{itemize}
              \item \typename{frame\_descr} are at the core of stack unwinding
            \end{itemize}
          }
      \end{itemize}
      \vfill
      \mintocaml[firstline=9,lastline=13,highlightlines=12]{../src/backtrace/backtrace.ml}
      \endminipage
    \end{column}
    \begin{column}{0.5\textwidth}
      \onslide*<1>{\listStashBacktrace[highlightlines=91]{}}
      \onslide*<2>{\listStashBacktrace[highlightlines={91,117-118}]{}}
      \onslide*<3->{\listStashBacktrace[highlightlines={94,106,109-110}]{}}
    \end{column}
  \end{columns}
\end{frame}

\newcommand\listFrameDescr[1][]{
  \listocaml[firstline=111,lastline=124,#1]{../ocaml/asmcomp/emitaux.ml}{asmcomp/emitaux.ml:111}}
\newcommand\listDebugInfo[1][]{
  \listocaml[firstline=38,lastline=49,#1]{../ocaml/lambda/debuginfo.mli}{lambda/debuginfo.mli:38}}

\begin{frame}{Backtraces}{\typename{frame\_descr} and \typename{frame\_debuginfo} in a nutshell}
  \begin{columns}[c]
    \begin{column}{0.5\textwidth}
      \begin{itemize}
        \item<1-> For every return address in an OCaml program, there is a associated \typename{frame\_descr}
        \item<2-> A \typename{frame\_descr} specifies
        \begin{itemize}
          \item<2-> The size of the function stack frame
          \item<3-> Where live values are on the stack (so that the GC can mark them as live and prevent from being swept)
        \end{itemize}
        \item<4-> When compiled with debug information, \typename{frame\_descr} will be linked to a \typename{frame\_debuginfo}
        \begin{itemize}
          \item<5-> Contains information about the position in the source code (file name, line and column number)
        \end{itemize}
      \end{itemize}
    \end{column}
    \begin{column}{0.5\textwidth}
        \onslide*<1>{\listFrameDescr[highlightlines={118-122}]{}}
        \onslide*<2>{\listFrameDescr[highlightlines={120}]{}}
        \onslide*<3>{\listFrameDescr[highlightlines={121}]{}}
        \onslide*<4->{\listFrameDescr[highlightlines={122}]{}}
        \medskip
        \onslide*<1-3>{\listDebugInfo{}}
        \onslide*<4>{\listDebugInfo[highlightlines={38,49}]{}}
        \onslide*<5>{\listDebugInfo[highlightlines={39-46}]{}}
    \end{column}
  \end{columns}
\end{frame}

\begin{frame}{Backtraces}{Scanning the stack with \typename{frame\_descr}}
  \begin{columns}[c]
    \begin{column}{0.5\textwidth}
      \begin{itemize}
        \item Given the return address of the code that raises the exception by calling \funcname{caml\_raise\_exn} (\funcarg{pc} argument of \funcname{caml\_stash\_backtrace})
        \item And the position of the stack pointer (\funcarg{sp} argument of \funcname{caml\_stash\_backtrace})
        \item The frame size given by \typename{frame\_descr}.\funcarg{frame\_size}, allows to find the end of the previous frame
        \item And the return address of the caller of this function
        \bigskip
        \onslide<3->{
        \item Note that traps (here the reddish \funcname{ExnA}, \funcname{ExnB} and \funcname{ExnC} blocks) belong to the frame that installed them, and so the frame size changes when traps are removed
        }
      \end{itemize}
    \end{column}
    \begin{column}{0.5\textwidth}
      \raggedleft
      \onslide*<1>{\providecommand\step{2}\input{diagrams/backtrace/backtrace.tex}}%
      \onslide*<2>{\providecommand\step{1}\input{diagrams/backtrace/scanstack.tex}}%
      \onslide*<3>{\providecommand\step{2}\input{diagrams/backtrace/scanstack.tex}}%
      \onslide*<4>{\providecommand\step{3}\input{diagrams/backtrace/scanstack.tex}}%
      \onslide*<5>{\providecommand\step{4}\input{diagrams/backtrace/scanstack.tex}}%
      \onslide*<6>{\providecommand\step{5}\input{diagrams/backtrace/scanstack.tex}}%
    \end{column}
  \end{columns}
\end{frame}

% Duplicate of previous-previous slide
\begin{frame}{Backtraces}{Continuing on \funcname{caml\_stash\_backtrace}}
  \begin{columns}[c]
    \begin{column}{0.5\textwidth}
      \minipage[c][0.75\textheight][s]{\columnwidth}
      \begin{itemize}
          \color{gray}
        \item A C function provided by the runtime (specific to native code)
        \item It scans the stack, between the stack pointer at the raising point up to the next trap, in order to collect the names of the detected function frames
        \item It uses the same technique and information as the Garbage Collector (GC) uses to scan the values live on the stack
      \end{itemize}
      \begin{itemize}
        \item For each function frame detected, store a pointer to the \typename{frame\_descr} in the \localname{domain\_state->backtrace\_buffer} array, effectively constructing the backtrace
      \end{itemize}
      \onslide<2->{
        \begin{itemize}
            \smallskip
          \item Constructed backtrace:
            \funcname{definitely\_raise},
            \onslide<3->{\funcname{probably\_raise}}
        \end{itemize}
      }
      \vfill
      \mintocaml[firstline=9,lastline=13,highlightlines=12]{../src/backtrace/backtrace.ml}
      \endminipage
    \end{column}
    \begin{column}{0.5\textwidth}
      \raggedleft
      \onslide*<1>{\listStashBacktrace[highlightlines={114-115}]{}}
      \onslide*<2>{\providecommand\step{3}\input{diagrams/backtrace/backtrace.tex}}%
      \onslide*<3>{\providecommand\step{4}\input{diagrams/backtrace/backtrace.tex}}%
    \end{column}
  \end{columns}
\end{frame}

\begin{frame}{Backtraces}{Back to \funcname{caml\_raise\_exn}}
  \begin{columns}[c]
    \begin{column}{0.5\textwidth}
      \minipage[c][0.66\textheight][s]{\columnwidth}
      \begin{itemize}\color{gray}
        \item Checks wheteher exceptions backtrace should be recorded, by checking the \funcarg{domain\_state->backtrace\_active} value
        \item Switches to the C stack to be able to execute C code
        \item Calls \funcname{caml\_stash\_backtrace}, to collect the backtrace up to the next trap
      \end{itemize}
      \onslide*<2->{
        \begin{itemize}
          \item<2-> Executes the exception handler in \funcname{probably\_raise} with \funcname{RESTORE\_EXN\_HANDLER\_OCAML}
            \begin{itemize}
              \item Set \regname{RSP} to \localname{domain\_state->exn\_handler} value
              \item Restore \localname{domain\_state->exn\_handler} from trap
              \item Jump to the handler code with \funcname{ret}
            \end{itemize}
        \end{itemize}
      }
      \vfill
      \onslide*<1>{\mintocaml[firstline=9,lastline=13,highlightlines=12]{../src/backtrace/backtrace.ml}}
      \onslide*<2->{\mintocaml[firstline=15,lastline=21,highlightlines={19-21}]{../src/backtrace/backtrace.ml}}
      \endminipage
    \end{column}
    \begin{column}{0.5\textwidth}
      \centering
      \onslide*<1>{
        \mintc[firstline=190,lastline=192]{../ocaml/runtime/amd64.S}
        \mintc[firstline=234,lastline=237]{../ocaml/runtime/amd64.S}
      }
      \onslide*<2>{
        \mintc[firstline=190,lastline=192,highlightlines={191-192}]{../ocaml/runtime/amd64.S}
        \mintc[firstline=234,lastline=237,highlightlines={235-237}]{../ocaml/runtime/amd64.S}
      }
      \onslide*<1>{\listCamlRaiseExn[highlightlines=720]{}}
      \onslide*<2>{\listCamlRaiseExn[highlightlines=722]{}}
      \onslide*<3->{\providecommand\step{5}\input{diagrams/backtrace/backtrace.tex}}
    \end{column}
  \end{columns}
\end{frame}

\begin{frame}{Backtraces}{\funcname{caml\_probably\_raise}'s handler doesn't catch \typename{ExnC}}
  \begin{columns}[c]
    \begin{column}{0.45\textwidth}
      \minipage[c][0.75\textheight][s]{\columnwidth}
      \begin{itemize}
        \item<1-> \funcname{match \dots with}\footnotemark pattern matching can also match exceptions
        \item<1-> The CMM of \funcname{probably\_raise} shows that this \funcname{match \dots with} is implemented with a pattern matching and a \funcname{try \dots with}
        \item<1-> But this one only matches on \typename{ExnA} exception
        \item<2-> Since the pattern matching fails to catch the exception, \funcname{reraise} is called with our current \typename{ExnC} exception
        \item<3-> Disassembly shows that \funcname{reraise} is actually a call to \funcname{caml\_reraise\_exn}, which is also an assembly function provided by the runtime
      \end{itemize}
      \vfill
      \mintocaml[firstline=15,lastline=21,highlightlines={18-21}]{../src/backtrace/backtrace.ml}
      \endminipage
    \end{column}
    \begin{column}{0.55\textwidth}
      \centering
      \onslide*<1>{\mintcmm[highlightlines={10-18}]{../src/backtrace/camlBacktrace\_\_probably\_raise\_351.cmm}}
      \onslide*<2>{\listcmm[highlightlines=18]{../src/backtrace/camlBacktrace\_\_probably\_raise_351.cmm}{CMM of probably\_raise}}
      \onslide*<3>{\mintobjdump[firstline=5,lastline=35,highlightlines=31]{../src/backtrace/camlBacktrace\_\_probably\_raise_351.objdump}}
    \end{column}
  \end{columns}
  \only<1>{\footnotetext[1]{https://blog.janestreet.com/pattern-matching-and-exception-handling-unite/}}
\end{frame}

\newcommand\listCamlReraiseExn[1][]{
  \listasm[firstline=727,lastline=734,#1]{../ocaml/runtime/amd64.S}{runtime/amd64.S:727}}

\begin{frame}{Backtraces}{Introducing \funcname{caml\_reraise\_exn}}
  \begin{columns}[c]
    \begin{column}{0.5\textwidth}
      \minipage[c][0.66\textheight][s]{\columnwidth}
      \begin{itemize}
        \item<1-> The exception have to be reraised to the next handler
        \item<2-> First, checks if the backtrace should be collected with \funcarg{domain\_state->backtrace\_active}
        \only<3>{
        \begin{itemize}
            \item If not, behaves like a \funcname{raise\_notrace} with \funcname{RESTORE\_EXN\_HANDLER\_OCAML}
        \end{itemize}
        }
        \item<4-> Jumps into \funcname{caml\_raise\_exn}
        \item<5-> Calls \funcname{caml\_stash\_backtrace} again, to scan the next part of the stack: from the trap just pop-ed to the next trap
      \end{itemize}
      \vfill
      \mintocaml[firstline=15,lastline=21,highlightlines={18-21}]{../src/backtrace/backtrace.ml}
      \endminipage
    \end{column}
    \begin{column}{0.5\textwidth}
      \raggedleft
      \onslide*<1>{\listCamlReraiseExn{}}
      \onslide*<2>{\listCamlReraiseExn[highlightlines=729]{}}
      \onslide*<3>{
        \mintc[firstline=190,lastline=192,highlightlines={191-192}]{../ocaml/runtime/amd64.S}
        \mintc[firstline=234,lastline=237,highlightlines={235-237}]{../ocaml/runtime/amd64.S}
        \medskip
        \listCamlReraiseExn[highlightlines=731]{}
      }
      \onslide*<4-5>{
        % TODO refactor with \listCamlReraiseExn
        \mintasm[firstline=727,lastline=734,highlightlines=730]{../ocaml/runtime/amd64.S}
        \medskip
      }
      \onslide*<4>{\listCamlRaiseExn[highlightlines=711]{}}
      \onslide*<5>{\listCamlRaiseExn[highlightlines=720]{}}
    \end{column}
  \end{columns}
\end{frame}

\begin{frame}{Backtraces}{Backtrace collection continues}
  \begin{columns}[c]
    \begin{column}{0.55\textwidth}
      \minipage[c][0.75\textheight][s]{\columnwidth}
      \begin{itemize}
        \item<1-> \funcname{caml\_stash\_backtrace} scans the older part of the stack, up to the next trap
        \item<6-> Controls jumps to the nested exception handler in \funcname{maybe\_raise}, which matches only on \typename{ExnB}
        \item<7-> \funcname{caml\_reraise\_exn} is called again, backtrace collection resumes with \funcname{caml\_stash\_backtrace}
      \end{itemize}
      \medskip
      \begin{itemize}
        \item<2-> Constructed backtrace:
        \begin{itemize}
          \item <2-> \funcname{definitely\_raise}
          \item <2-> \funcname{probably\_raise}
          \item <3-> \funcname{probably\_raise} (re-raise)
          \item <4-> \funcname{maybe\_raise}
          \item <5-> \funcname{main}
          \item <8-> \funcname{main} (re-raise)
        \end{itemize}
      \end{itemize}
      \vfill
      \onslide*<1-5>{\mintocaml[firstline=15,lastline=21]{../src/backtrace/backtrace.ml}}
      \onslide*<6>{\mintocaml[firstline=28,lastline=34,highlightlines=31]{../src/backtrace/backtrace.ml}}
      \onslide*<7->{\mintocaml[firstline=28,lastline=34,highlightlines=33]{../src/backtrace/backtrace.ml}}
      \endminipage
    \end{column}
    \begin{column}{0.45\textwidth}
      \raggedleft
      \onslide*<1>{\providecommand\step{6}\input{diagrams/backtrace/backtrace.tex}}%
      \onslide*<2>{\providecommand\step{6}\input{diagrams/backtrace/backtrace.tex}}%
      \onslide*<3>{\providecommand\step{7}\input{diagrams/backtrace/backtrace.tex}}%
      \onslide*<4>{\providecommand\step{8}\input{diagrams/backtrace/backtrace.tex}}%
      \onslide*<5>{\providecommand\step{9}\input{diagrams/backtrace/backtrace.tex}}%
      \onslide*<6>{\providecommand\step{10}\input{diagrams/backtrace/backtrace.tex}}%
      \onslide*<7>{\providecommand\step{11}\input{diagrams/backtrace/backtrace.tex}}%
      \onslide*<8>{\providecommand\step{12}\input{diagrams/backtrace/backtrace.tex}}%
      \onslide*<9>{\providecommand\step{13}\input{diagrams/backtrace/backtrace.tex}}%
    \end{column}
  \end{columns}
\end{frame}

\begin{frame}{Backtraces}{Printing the backtrace}
  \begin{columns}[c]
    \begin{column}{0.45\textwidth}
      \minipage[c][0.66\textheight][s]{\columnwidth}
      \begin{itemize}
        \item<1-> The exception backtrace can now be printed to \funcarg{stdout} with \funcname{Printexc.print_backtrace}
        \item<2-> First the exception backtrace is retrieved into an OCaml array with \funcname{get_raw_backtrace }
        \item<3-> Which is implemented as the \funcname{caml_get_exception_raw_backtrace} C function in the runtime
        \item<4-> And finally printed with \funcname{print_exception_backtrace}
      \end{itemize}
      \vfill
      \mintocaml[firstline=28,lastline=34,highlightlines=33]{../src/backtrace/backtrace.ml}
      \endminipage
    \end{column}
    \begin{column}{0.55\textwidth}
      \onslide*<1,2>{
        \mintocaml[firstline=100,lastline=101]{../ocaml/stdlib/printexc.ml}
      }
      \only<1>{
        \listocaml[firstline=170,lastline=172]{../ocaml/stdlib/printexc.ml}{stdlib/printexc.ml:170}
      }
      \only<2>{
        \listocaml[firstline=170,lastline=172,highlightlines=172]{../ocaml/stdlib/printexc.ml}{stdlib/printexc.ml:170}
      }
      \only<3>{
        \mintc[firstline=154,lastline=156]{../ocaml/runtime/backtrace.c}
        \listc[firstline=171,lastline=192,highlightlines={184-187}]{../ocaml/runtime/backtrace.c}{runtime/backtrace.c:154}
      }
      \only<4>{
        \listocaml[firstline=155,lastline=165]{../ocaml/stdlib/printexc.ml}{stdlib/printexc.ml:55}
      }
    \end{column}
  \end{columns}
\end{frame}


\subsection{The multiple kinds of raise}
\frameSubsection{}{}

\begin{frame}{Backtraces}{When backtraces are not needed}
  \begin{columns}[c]
    \begin{column}{0.45\textwidth}
      \minipage[c][0.66\textheight][s]{\columnwidth}
      \begin{itemize}
        \item<1-> What if the backtrace for an exception is never needed?
        \item<2-> It's possible to raise the exception explicitly with \funcname{raise\_notrace} to make raising as cheap as possible
        \item<3-> An example of use in the \funcname{dynlink} library of the OCaml compiler
      \end{itemize}
      \vfill
      \onslide<3->{
      \listocaml[firstline=205,lastline=209,highlightlines=207]{../ocaml/otherlibs/dynlink/dynlink\_compilerlibs/misc.ml}{otherlibs/dynlink/dynlink\_compilerlibs/misc.ml:205}
      }
      \endminipage
    \end{column}
    \begin{column}{0.55\textwidth}
    \only<4>{
      \mintcmm[highlightlines=3]{../src/notrace/camlNotrace\_\_fun_347.cmm}
      \listcmm{../src/notrace/camlNotrace\_\_all\_somes\_327.cmm}{all\_somes}
    }
    \onslide*<5->{
      \listobjdump[highlightlines={6-9}]{../src/notrace/camlNotrace\_\_fun_347.objdump}{anonymous function from \funcname{all\_somes}}
    }
    \end{column}
  \end{columns}
\end{frame}

\begin{frame}{Backtraces}{Summary table of the multiple kinds of raise}
  \begin{itemize}
    \item When debugging information is enabled and if the backtrace of an exception isn't needed, it's possible to raise with \funcname{raise\_notrace} for performance
    \item When debugging information is \emph{not} enabled, the compiler optimises \funcname{raise} calls to inline assembly (same effect as using \funcname{raise\_notrace})
  \end{itemize}
  \bigskip
  \centering
  \begin{tabular}{| l | c | c | c | c |}
    \hline
    \parbox{10em}{Debug information \\ (\funcarg{-g} flag)}  & \multicolumn{2}{|c|}{Disabled}                                     & \multicolumn{2}{|c|}{Enabled} \\
    \hline
    Raise kind                                               & \funcname{raise} & \funcname{raise\_notrace}                       & \funcname{raise} & \funcname{raise\_notrace} \\
    \hline
    Compilation                                              & \parbox{8em}{inline assembly\\ (optimization)} & inline assembly   & \funcname{caml\_raise\_exn} & inline assembly \\
    \hline
  \end{tabular}
\end{frame}


%
% Takeaway #5
%
\subsection*{Takeaway \#5}
\frameSubsectionTakeaway{}
\againframe<5>{takeaway}
}%
    \end{column}
  \end{columns}
\end{frame}

\begin{frame}{Backtraces}{Back to \funcname{caml\_raise\_exn}}
  \begin{columns}[c]
    \begin{column}{0.5\textwidth}
      \minipage[c][0.66\textheight][s]{\columnwidth}
      \begin{itemize}\color{gray}
        \item Checks wheteher exceptions backtrace should be recorded, by checking the \funcarg{domain\_state->backtrace\_active} value
        \item Switches to the C stack to be able to execute C code
        \item Calls \funcname{caml\_stash\_backtrace}, to collect the backtrace up to the next trap
      \end{itemize}
      \onslide*<2->{
        \begin{itemize}
          \item<2-> Executes the exception handler in \funcname{probably\_raise} with \funcname{RESTORE\_EXN\_HANDLER\_OCAML}
            \begin{itemize}
              \item Set \regname{RSP} to \localname{domain\_state->exn\_handler} value
              \item Restore \localname{domain\_state->exn\_handler} from trap
              \item Jump to the handler code with \funcname{ret}
            \end{itemize}
        \end{itemize}
      }
      \vfill
      \onslide*<1>{\mintocaml[firstline=9,lastline=13,highlightlines=12]{../src/backtrace/backtrace.ml}}
      \onslide*<2->{\mintocaml[firstline=15,lastline=21,highlightlines={19-21}]{../src/backtrace/backtrace.ml}}
      \endminipage
    \end{column}
    \begin{column}{0.5\textwidth}
      \centering
      \onslide*<1>{
        \mintc[firstline=190,lastline=192]{../ocaml/runtime/amd64.S}
        \mintc[firstline=234,lastline=237]{../ocaml/runtime/amd64.S}
      }
      \onslide*<2>{
        \mintc[firstline=190,lastline=192,highlightlines={191-192}]{../ocaml/runtime/amd64.S}
        \mintc[firstline=234,lastline=237,highlightlines={235-237}]{../ocaml/runtime/amd64.S}
      }
      \onslide*<1>{\listCamlRaiseExn[highlightlines=720]{}}
      \onslide*<2>{\listCamlRaiseExn[highlightlines=722]{}}
      \onslide*<3->{\providecommand\step{5}%
% Backtraces
%
\subsection{One advantage of raising with debugging information}
\frameSubsection{}{}

\begin{frame}{Backtraces}{Debugging information \& source code}
  \begin{columns}[c]
    \begin{column}{0.5\textwidth}
      \begin{itemize}
        \item Raising an exception is fun and games, but how do we know where the exception originated? $\rightarrow$ With the backtrace associated with the exception!
        \item In order to get a backtrace from the point where an exception was raised, it's necessary to compile with debugging information enabled (\funcarg{-g} flag for the compiler)
        \item Same example as in the `Nested handlers' section
      \end{itemize}
    \end{column}
    \begin{column}{0.5\textwidth}
      \listocaml[firstline=9,lastline=34]{../src/backtrace/backtrace.ml}{backtrace/backtrace.ml}
    \end{column}
  \end{columns}
\end{frame}

\begin{frame}{Backtraces}{CMM and disassembly of \funcname{definitely\_raise}}
  \begin{columns}[c]
    \begin{column}{0.5\textwidth}
      \minipage[c][0.66\textheight][s]{\columnwidth}
      \begin{itemize}
        \item<1-> An effect of compiling with debugging information (\funcarg{-g}): the CMM no longer uses \funcname{raise\_notrace} to raise the exception but \funcname{raise} instead
        \item<2-> Disassembly shows that CMM \funcname{raise} is actually a call to \funcname{caml\_raise\_exn}, a function in assembler provided by the runtime
      \end{itemize}
      \vfill
      \mintocaml[firstline=9,lastline=13,highlightlines=12]{../src/backtrace/backtrace.ml}
      \endminipage
    \end{column}
    \begin{column}{0.5\textwidth}
      \onslide*<1>{
        \mintcmm[highlightlines=5]{../src/backtrace/camlBacktrace\_\_definitely\_raise\_347.cmm}
      }
      \onslide*<2>{
        \mintobjdump[firstline=2,highlightlines=14]{../src/backtrace/camlBacktrace\_\_definitely\_raise\_347.objdump}
      }
    \end{column}
  \end{columns}
\end{frame}

\begin{frame}{Backtraces}{Introducing \funcname{caml\_raise\_exn}}
  \begin{columns}[c]
    \begin{column}{0.5\textwidth}
      \minipage[c][0.66\textheight][s]{\columnwidth}
      \onslide*<1>{
        \begin{itemize}
          \item Raising the exception is performed using \funcname{caml\_raise\_exn} which is
            \begin{itemize}
              \item An assembly function provided by the runtime
              \item Architecture specific
            \end{itemize}
          \item Let's see what it does differently from \funcname{raise\_notrace}\dots
          \item We are about to raise \typename{ExnC} with \funcname{caml\_raise\_exn} from \funcname{definitely\_raise}
        \end{itemize}
      }
      \onslide*<2>{
        \mintobjdump[firstline=2,highlightlines=14]{../src/backtrace/camlBacktrace\_\_definitely\_raise\_347.objdump}
      }
      \vfill
      \mintocaml[firstline=9,lastline=13,highlightlines=12]{../src/backtrace/backtrace.ml}
      \endminipage
    \end{column}
    \begin{column}{0.5\textwidth}
      \centering
      \providecommand\step{1}\input{diagrams/backtrace/backtrace.tex}
    \end{column}
  \end{columns}
\end{frame}

\begin{frame}{Backtraces}{But before that, we need more fields from \typename{caml\_domain\_state}}
  \begin{columns}
    \begin{column}{0.5\textwidth}
      \begin{itemize}
        \item Remember \typename{caml\_domain\_state} the per-domain runtime state?
        \item Some other members are of interest to us now\dots
        \item \localname{backtrace\_active} is used to know if the runtime should collect backtraces
        \item \localname{backtrace\_active} can be set by
          \begin{itemize}
            \item The \funcarg{b} flag in the \funcname{OCAMLRUNPARAM} environment variable
            \item \mintinline{ocaml}{Printexc.record_backtrace : bool -> unit} \footnotemark
          \end{itemize}
      \end{itemize}
    \end{column}
    \begin{column}{0.5\textwidth}
      \begin{listing}
        \mintc[highlightlines={9-11}]{src/ocaml/domain\_state\_backtrace.h}
        \caption{runtime/caml/domain\_state.h}
      \end{listing}
    \end{column}
  \end{columns}
  \footnotetext[1]{https://v2.ocaml.org/api/Printexc.html}
\end{frame}

\newcommand\listCamlRaiseExn[1][]{
  \listasm[firstline=702,lastline=725,#1]{../ocaml/runtime/amd64.S}{runtime/amd64.S:702}}

\begin{frame}{Backtraces}{Walk through \funcname{caml\_raise\_exn}}
  \begin{columns}[T]
    \begin{column}{0.5\textwidth}
      \begin{itemize}
        \item<1-> First checks whether exceptions backtrace should be recorded
        \item<1-> By checking the \funcarg{domain\_state->backtrace\_active} value
        \item<2-> If not, acts as \funcname{raise\_notrace} with \funcname{RESTORE\_EXN\_HANDLER\_OCAML}
          \begin{itemize}
            \item Set \regname{RSP} to \localname{domain\_state->exn\_handler} value
            \item Restore \localname{domain\_state->exn\_handler} from trap
            \item Jump with \funcname{ret} (same effect as using \funcname{pop} and \funcname{jmp})
          \end{itemize}
        \item<3-> Otherwise, switches to the C stack to be able to execute C code
        \item<4-> Calls \funcname{caml\_stash\_backtrace}, a C function provided by the runtime\ignorespaces
          \onslide<6->{, with
            \begin{itemize}
              \item \funcarg{exn}: exception value
              \item \funcarg{pc}: code address of the raising point
              \item \funcarg{sp}: stack address of the raising function
              \item \funcarg{trapsp}: stack address of the next handler
            \end{itemize}
          }
      \end{itemize}
      \bigskip
      \mintocaml[firstline=9,lastline=13,highlightlines=12]{../src/backtrace/backtrace.ml}
    \end{column}
    \begin{column}{0.5\textwidth}
      \raggedleft
      \onslide*<1,3-4>{
        \mintc[firstline=190,lastline=192]{../ocaml/runtime/amd64.S}
        \mintc[firstline=234,lastline=237]{../ocaml/runtime/amd64.S}
      }
      \onslide*<2>{
        \mintc[firstline=190,lastline=192,highlightlines={191-192}]{../ocaml/runtime/amd64.S}
        \mintc[firstline=234,lastline=237,highlightlines={235-237}]{../ocaml/runtime/amd64.S}
      }
      \onslide*<1>{\listCamlRaiseExn[highlightlines=705]{}}%
      \onslide*<2>{\listCamlRaiseExn[highlightlines={707-708}]{}}%
      \onslide*<3>{\listCamlRaiseExn[highlightlines={712-714}]{}}%
      \onslide*<4>{\listCamlRaiseExn[highlightlines={715-720}]{}}%
      \onslide*<5>{\providecommand\step{1}\input{diagrams/backtrace/backtrace.tex}}%
      \onslide*<6>{\providecommand\step{2}\input{diagrams/backtrace/backtrace.tex}}%
    \end{column}
  \end{columns}
\end{frame}

\newcommand\listStashBacktrace[1][]{
  \listc[firstline=91,lastline=120,#1]{../ocaml/runtime/backtrace\_nat.c}{runtime/backtrace\_nat.c:91}}

\begin{frame}{Backtraces}{Introducing \funcname{caml\_stash\_backtrace}}
  \begin{columns}[c]
    \begin{column}{0.5\textwidth}
      \minipage[c][0.66\textheight][s]{\columnwidth}
      \begin{itemize}
        \item<1-> A C function provided by the runtime (specific to native code)
        \item<2-> It scans the stack, between the stack pointer at the raising point up to the next trap, in order to collect the names of the detected function frames
          \onslide*<2>{
            \begin{itemize}
              \item And more information, like line number, column number...
            \end{itemize}
          }
        \item<3-> It uses the same technique and information as the Garbage Collector (GC) uses to scan the values live on the stack
          \onslide*<4>{
            \begin{itemize}
              \item \typename{frame\_descr} are at the core of stack unwinding
            \end{itemize}
          }
      \end{itemize}
      \vfill
      \mintocaml[firstline=9,lastline=13,highlightlines=12]{../src/backtrace/backtrace.ml}
      \endminipage
    \end{column}
    \begin{column}{0.5\textwidth}
      \onslide*<1>{\listStashBacktrace[highlightlines=91]{}}
      \onslide*<2>{\listStashBacktrace[highlightlines={91,117-118}]{}}
      \onslide*<3->{\listStashBacktrace[highlightlines={94,106,109-110}]{}}
    \end{column}
  \end{columns}
\end{frame}

\newcommand\listFrameDescr[1][]{
  \listocaml[firstline=111,lastline=124,#1]{../ocaml/asmcomp/emitaux.ml}{asmcomp/emitaux.ml:111}}
\newcommand\listDebugInfo[1][]{
  \listocaml[firstline=38,lastline=49,#1]{../ocaml/lambda/debuginfo.mli}{lambda/debuginfo.mli:38}}

\begin{frame}{Backtraces}{\typename{frame\_descr} and \typename{frame\_debuginfo} in a nutshell}
  \begin{columns}[c]
    \begin{column}{0.5\textwidth}
      \begin{itemize}
        \item<1-> For every return address in an OCaml program, there is a associated \typename{frame\_descr}
        \item<2-> A \typename{frame\_descr} specifies
        \begin{itemize}
          \item<2-> The size of the function stack frame
          \item<3-> Where live values are on the stack (so that the GC can mark them as live and prevent from being swept)
        \end{itemize}
        \item<4-> When compiled with debug information, \typename{frame\_descr} will be linked to a \typename{frame\_debuginfo}
        \begin{itemize}
          \item<5-> Contains information about the position in the source code (file name, line and column number)
        \end{itemize}
      \end{itemize}
    \end{column}
    \begin{column}{0.5\textwidth}
        \onslide*<1>{\listFrameDescr[highlightlines={118-122}]{}}
        \onslide*<2>{\listFrameDescr[highlightlines={120}]{}}
        \onslide*<3>{\listFrameDescr[highlightlines={121}]{}}
        \onslide*<4->{\listFrameDescr[highlightlines={122}]{}}
        \medskip
        \onslide*<1-3>{\listDebugInfo{}}
        \onslide*<4>{\listDebugInfo[highlightlines={38,49}]{}}
        \onslide*<5>{\listDebugInfo[highlightlines={39-46}]{}}
    \end{column}
  \end{columns}
\end{frame}

\begin{frame}{Backtraces}{Scanning the stack with \typename{frame\_descr}}
  \begin{columns}[c]
    \begin{column}{0.5\textwidth}
      \begin{itemize}
        \item Given the return address of the code that raises the exception by calling \funcname{caml\_raise\_exn} (\funcarg{pc} argument of \funcname{caml\_stash\_backtrace})
        \item And the position of the stack pointer (\funcarg{sp} argument of \funcname{caml\_stash\_backtrace})
        \item The frame size given by \typename{frame\_descr}.\funcarg{frame\_size}, allows to find the end of the previous frame
        \item And the return address of the caller of this function
        \bigskip
        \onslide<3->{
        \item Note that traps (here the reddish \funcname{ExnA}, \funcname{ExnB} and \funcname{ExnC} blocks) belong to the frame that installed them, and so the frame size changes when traps are removed
        }
      \end{itemize}
    \end{column}
    \begin{column}{0.5\textwidth}
      \raggedleft
      \onslide*<1>{\providecommand\step{2}\input{diagrams/backtrace/backtrace.tex}}%
      \onslide*<2>{\providecommand\step{1}\input{diagrams/backtrace/scanstack.tex}}%
      \onslide*<3>{\providecommand\step{2}\input{diagrams/backtrace/scanstack.tex}}%
      \onslide*<4>{\providecommand\step{3}\input{diagrams/backtrace/scanstack.tex}}%
      \onslide*<5>{\providecommand\step{4}\input{diagrams/backtrace/scanstack.tex}}%
      \onslide*<6>{\providecommand\step{5}\input{diagrams/backtrace/scanstack.tex}}%
    \end{column}
  \end{columns}
\end{frame}

% Duplicate of previous-previous slide
\begin{frame}{Backtraces}{Continuing on \funcname{caml\_stash\_backtrace}}
  \begin{columns}[c]
    \begin{column}{0.5\textwidth}
      \minipage[c][0.75\textheight][s]{\columnwidth}
      \begin{itemize}
          \color{gray}
        \item A C function provided by the runtime (specific to native code)
        \item It scans the stack, between the stack pointer at the raising point up to the next trap, in order to collect the names of the detected function frames
        \item It uses the same technique and information as the Garbage Collector (GC) uses to scan the values live on the stack
      \end{itemize}
      \begin{itemize}
        \item For each function frame detected, store a pointer to the \typename{frame\_descr} in the \localname{domain\_state->backtrace\_buffer} array, effectively constructing the backtrace
      \end{itemize}
      \onslide<2->{
        \begin{itemize}
            \smallskip
          \item Constructed backtrace:
            \funcname{definitely\_raise},
            \onslide<3->{\funcname{probably\_raise}}
        \end{itemize}
      }
      \vfill
      \mintocaml[firstline=9,lastline=13,highlightlines=12]{../src/backtrace/backtrace.ml}
      \endminipage
    \end{column}
    \begin{column}{0.5\textwidth}
      \raggedleft
      \onslide*<1>{\listStashBacktrace[highlightlines={114-115}]{}}
      \onslide*<2>{\providecommand\step{3}\input{diagrams/backtrace/backtrace.tex}}%
      \onslide*<3>{\providecommand\step{4}\input{diagrams/backtrace/backtrace.tex}}%
    \end{column}
  \end{columns}
\end{frame}

\begin{frame}{Backtraces}{Back to \funcname{caml\_raise\_exn}}
  \begin{columns}[c]
    \begin{column}{0.5\textwidth}
      \minipage[c][0.66\textheight][s]{\columnwidth}
      \begin{itemize}\color{gray}
        \item Checks wheteher exceptions backtrace should be recorded, by checking the \funcarg{domain\_state->backtrace\_active} value
        \item Switches to the C stack to be able to execute C code
        \item Calls \funcname{caml\_stash\_backtrace}, to collect the backtrace up to the next trap
      \end{itemize}
      \onslide*<2->{
        \begin{itemize}
          \item<2-> Executes the exception handler in \funcname{probably\_raise} with \funcname{RESTORE\_EXN\_HANDLER\_OCAML}
            \begin{itemize}
              \item Set \regname{RSP} to \localname{domain\_state->exn\_handler} value
              \item Restore \localname{domain\_state->exn\_handler} from trap
              \item Jump to the handler code with \funcname{ret}
            \end{itemize}
        \end{itemize}
      }
      \vfill
      \onslide*<1>{\mintocaml[firstline=9,lastline=13,highlightlines=12]{../src/backtrace/backtrace.ml}}
      \onslide*<2->{\mintocaml[firstline=15,lastline=21,highlightlines={19-21}]{../src/backtrace/backtrace.ml}}
      \endminipage
    \end{column}
    \begin{column}{0.5\textwidth}
      \centering
      \onslide*<1>{
        \mintc[firstline=190,lastline=192]{../ocaml/runtime/amd64.S}
        \mintc[firstline=234,lastline=237]{../ocaml/runtime/amd64.S}
      }
      \onslide*<2>{
        \mintc[firstline=190,lastline=192,highlightlines={191-192}]{../ocaml/runtime/amd64.S}
        \mintc[firstline=234,lastline=237,highlightlines={235-237}]{../ocaml/runtime/amd64.S}
      }
      \onslide*<1>{\listCamlRaiseExn[highlightlines=720]{}}
      \onslide*<2>{\listCamlRaiseExn[highlightlines=722]{}}
      \onslide*<3->{\providecommand\step{5}\input{diagrams/backtrace/backtrace.tex}}
    \end{column}
  \end{columns}
\end{frame}

\begin{frame}{Backtraces}{\funcname{caml\_probably\_raise}'s handler doesn't catch \typename{ExnC}}
  \begin{columns}[c]
    \begin{column}{0.45\textwidth}
      \minipage[c][0.75\textheight][s]{\columnwidth}
      \begin{itemize}
        \item<1-> \funcname{match \dots with}\footnotemark pattern matching can also match exceptions
        \item<1-> The CMM of \funcname{probably\_raise} shows that this \funcname{match \dots with} is implemented with a pattern matching and a \funcname{try \dots with}
        \item<1-> But this one only matches on \typename{ExnA} exception
        \item<2-> Since the pattern matching fails to catch the exception, \funcname{reraise} is called with our current \typename{ExnC} exception
        \item<3-> Disassembly shows that \funcname{reraise} is actually a call to \funcname{caml\_reraise\_exn}, which is also an assembly function provided by the runtime
      \end{itemize}
      \vfill
      \mintocaml[firstline=15,lastline=21,highlightlines={18-21}]{../src/backtrace/backtrace.ml}
      \endminipage
    \end{column}
    \begin{column}{0.55\textwidth}
      \centering
      \onslide*<1>{\mintcmm[highlightlines={10-18}]{../src/backtrace/camlBacktrace\_\_probably\_raise\_351.cmm}}
      \onslide*<2>{\listcmm[highlightlines=18]{../src/backtrace/camlBacktrace\_\_probably\_raise_351.cmm}{CMM of probably\_raise}}
      \onslide*<3>{\mintobjdump[firstline=5,lastline=35,highlightlines=31]{../src/backtrace/camlBacktrace\_\_probably\_raise_351.objdump}}
    \end{column}
  \end{columns}
  \only<1>{\footnotetext[1]{https://blog.janestreet.com/pattern-matching-and-exception-handling-unite/}}
\end{frame}

\newcommand\listCamlReraiseExn[1][]{
  \listasm[firstline=727,lastline=734,#1]{../ocaml/runtime/amd64.S}{runtime/amd64.S:727}}

\begin{frame}{Backtraces}{Introducing \funcname{caml\_reraise\_exn}}
  \begin{columns}[c]
    \begin{column}{0.5\textwidth}
      \minipage[c][0.66\textheight][s]{\columnwidth}
      \begin{itemize}
        \item<1-> The exception have to be reraised to the next handler
        \item<2-> First, checks if the backtrace should be collected with \funcarg{domain\_state->backtrace\_active}
        \only<3>{
        \begin{itemize}
            \item If not, behaves like a \funcname{raise\_notrace} with \funcname{RESTORE\_EXN\_HANDLER\_OCAML}
        \end{itemize}
        }
        \item<4-> Jumps into \funcname{caml\_raise\_exn}
        \item<5-> Calls \funcname{caml\_stash\_backtrace} again, to scan the next part of the stack: from the trap just pop-ed to the next trap
      \end{itemize}
      \vfill
      \mintocaml[firstline=15,lastline=21,highlightlines={18-21}]{../src/backtrace/backtrace.ml}
      \endminipage
    \end{column}
    \begin{column}{0.5\textwidth}
      \raggedleft
      \onslide*<1>{\listCamlReraiseExn{}}
      \onslide*<2>{\listCamlReraiseExn[highlightlines=729]{}}
      \onslide*<3>{
        \mintc[firstline=190,lastline=192,highlightlines={191-192}]{../ocaml/runtime/amd64.S}
        \mintc[firstline=234,lastline=237,highlightlines={235-237}]{../ocaml/runtime/amd64.S}
        \medskip
        \listCamlReraiseExn[highlightlines=731]{}
      }
      \onslide*<4-5>{
        % TODO refactor with \listCamlReraiseExn
        \mintasm[firstline=727,lastline=734,highlightlines=730]{../ocaml/runtime/amd64.S}
        \medskip
      }
      \onslide*<4>{\listCamlRaiseExn[highlightlines=711]{}}
      \onslide*<5>{\listCamlRaiseExn[highlightlines=720]{}}
    \end{column}
  \end{columns}
\end{frame}

\begin{frame}{Backtraces}{Backtrace collection continues}
  \begin{columns}[c]
    \begin{column}{0.55\textwidth}
      \minipage[c][0.75\textheight][s]{\columnwidth}
      \begin{itemize}
        \item<1-> \funcname{caml\_stash\_backtrace} scans the older part of the stack, up to the next trap
        \item<6-> Controls jumps to the nested exception handler in \funcname{maybe\_raise}, which matches only on \typename{ExnB}
        \item<7-> \funcname{caml\_reraise\_exn} is called again, backtrace collection resumes with \funcname{caml\_stash\_backtrace}
      \end{itemize}
      \medskip
      \begin{itemize}
        \item<2-> Constructed backtrace:
        \begin{itemize}
          \item <2-> \funcname{definitely\_raise}
          \item <2-> \funcname{probably\_raise}
          \item <3-> \funcname{probably\_raise} (re-raise)
          \item <4-> \funcname{maybe\_raise}
          \item <5-> \funcname{main}
          \item <8-> \funcname{main} (re-raise)
        \end{itemize}
      \end{itemize}
      \vfill
      \onslide*<1-5>{\mintocaml[firstline=15,lastline=21]{../src/backtrace/backtrace.ml}}
      \onslide*<6>{\mintocaml[firstline=28,lastline=34,highlightlines=31]{../src/backtrace/backtrace.ml}}
      \onslide*<7->{\mintocaml[firstline=28,lastline=34,highlightlines=33]{../src/backtrace/backtrace.ml}}
      \endminipage
    \end{column}
    \begin{column}{0.45\textwidth}
      \raggedleft
      \onslide*<1>{\providecommand\step{6}\input{diagrams/backtrace/backtrace.tex}}%
      \onslide*<2>{\providecommand\step{6}\input{diagrams/backtrace/backtrace.tex}}%
      \onslide*<3>{\providecommand\step{7}\input{diagrams/backtrace/backtrace.tex}}%
      \onslide*<4>{\providecommand\step{8}\input{diagrams/backtrace/backtrace.tex}}%
      \onslide*<5>{\providecommand\step{9}\input{diagrams/backtrace/backtrace.tex}}%
      \onslide*<6>{\providecommand\step{10}\input{diagrams/backtrace/backtrace.tex}}%
      \onslide*<7>{\providecommand\step{11}\input{diagrams/backtrace/backtrace.tex}}%
      \onslide*<8>{\providecommand\step{12}\input{diagrams/backtrace/backtrace.tex}}%
      \onslide*<9>{\providecommand\step{13}\input{diagrams/backtrace/backtrace.tex}}%
    \end{column}
  \end{columns}
\end{frame}

\begin{frame}{Backtraces}{Printing the backtrace}
  \begin{columns}[c]
    \begin{column}{0.45\textwidth}
      \minipage[c][0.66\textheight][s]{\columnwidth}
      \begin{itemize}
        \item<1-> The exception backtrace can now be printed to \funcarg{stdout} with \funcname{Printexc.print_backtrace}
        \item<2-> First the exception backtrace is retrieved into an OCaml array with \funcname{get_raw_backtrace }
        \item<3-> Which is implemented as the \funcname{caml_get_exception_raw_backtrace} C function in the runtime
        \item<4-> And finally printed with \funcname{print_exception_backtrace}
      \end{itemize}
      \vfill
      \mintocaml[firstline=28,lastline=34,highlightlines=33]{../src/backtrace/backtrace.ml}
      \endminipage
    \end{column}
    \begin{column}{0.55\textwidth}
      \onslide*<1,2>{
        \mintocaml[firstline=100,lastline=101]{../ocaml/stdlib/printexc.ml}
      }
      \only<1>{
        \listocaml[firstline=170,lastline=172]{../ocaml/stdlib/printexc.ml}{stdlib/printexc.ml:170}
      }
      \only<2>{
        \listocaml[firstline=170,lastline=172,highlightlines=172]{../ocaml/stdlib/printexc.ml}{stdlib/printexc.ml:170}
      }
      \only<3>{
        \mintc[firstline=154,lastline=156]{../ocaml/runtime/backtrace.c}
        \listc[firstline=171,lastline=192,highlightlines={184-187}]{../ocaml/runtime/backtrace.c}{runtime/backtrace.c:154}
      }
      \only<4>{
        \listocaml[firstline=155,lastline=165]{../ocaml/stdlib/printexc.ml}{stdlib/printexc.ml:55}
      }
    \end{column}
  \end{columns}
\end{frame}


\subsection{The multiple kinds of raise}
\frameSubsection{}{}

\begin{frame}{Backtraces}{When backtraces are not needed}
  \begin{columns}[c]
    \begin{column}{0.45\textwidth}
      \minipage[c][0.66\textheight][s]{\columnwidth}
      \begin{itemize}
        \item<1-> What if the backtrace for an exception is never needed?
        \item<2-> It's possible to raise the exception explicitly with \funcname{raise\_notrace} to make raising as cheap as possible
        \item<3-> An example of use in the \funcname{dynlink} library of the OCaml compiler
      \end{itemize}
      \vfill
      \onslide<3->{
      \listocaml[firstline=205,lastline=209,highlightlines=207]{../ocaml/otherlibs/dynlink/dynlink\_compilerlibs/misc.ml}{otherlibs/dynlink/dynlink\_compilerlibs/misc.ml:205}
      }
      \endminipage
    \end{column}
    \begin{column}{0.55\textwidth}
    \only<4>{
      \mintcmm[highlightlines=3]{../src/notrace/camlNotrace\_\_fun_347.cmm}
      \listcmm{../src/notrace/camlNotrace\_\_all\_somes\_327.cmm}{all\_somes}
    }
    \onslide*<5->{
      \listobjdump[highlightlines={6-9}]{../src/notrace/camlNotrace\_\_fun_347.objdump}{anonymous function from \funcname{all\_somes}}
    }
    \end{column}
  \end{columns}
\end{frame}

\begin{frame}{Backtraces}{Summary table of the multiple kinds of raise}
  \begin{itemize}
    \item When debugging information is enabled and if the backtrace of an exception isn't needed, it's possible to raise with \funcname{raise\_notrace} for performance
    \item When debugging information is \emph{not} enabled, the compiler optimises \funcname{raise} calls to inline assembly (same effect as using \funcname{raise\_notrace})
  \end{itemize}
  \bigskip
  \centering
  \begin{tabular}{| l | c | c | c | c |}
    \hline
    \parbox{10em}{Debug information \\ (\funcarg{-g} flag)}  & \multicolumn{2}{|c|}{Disabled}                                     & \multicolumn{2}{|c|}{Enabled} \\
    \hline
    Raise kind                                               & \funcname{raise} & \funcname{raise\_notrace}                       & \funcname{raise} & \funcname{raise\_notrace} \\
    \hline
    Compilation                                              & \parbox{8em}{inline assembly\\ (optimization)} & inline assembly   & \funcname{caml\_raise\_exn} & inline assembly \\
    \hline
  \end{tabular}
\end{frame}


%
% Takeaway #5
%
\subsection*{Takeaway \#5}
\frameSubsectionTakeaway{}
\againframe<5>{takeaway}
}
    \end{column}
  \end{columns}
\end{frame}

\begin{frame}{Backtraces}{\funcname{caml\_probably\_raise}'s handler doesn't catch \typename{ExnC}}
  \begin{columns}[c]
    \begin{column}{0.45\textwidth}
      \minipage[c][0.75\textheight][s]{\columnwidth}
      \begin{itemize}
        \item<1-> \funcname{match \dots with}\footnotemark pattern matching can also match exceptions
        \item<1-> The CMM of \funcname{probably\_raise} shows that this \funcname{match \dots with} is implemented with a pattern matching and a \funcname{try \dots with}
        \item<1-> But this one only matches on \typename{ExnA} exception
        \item<2-> Since the pattern matching fails to catch the exception, \funcname{reraise} is called with our current \typename{ExnC} exception
        \item<3-> Disassembly shows that \funcname{reraise} is actually a call to \funcname{caml\_reraise\_exn}, which is also an assembly function provided by the runtime
      \end{itemize}
      \vfill
      \mintocaml[firstline=15,lastline=21,highlightlines={18-21}]{../src/backtrace/backtrace.ml}
      \endminipage
    \end{column}
    \begin{column}{0.55\textwidth}
      \centering
      \onslide*<1>{\mintcmm[highlightlines={10-18}]{../src/backtrace/camlBacktrace\_\_probably\_raise\_351.cmm}}
      \onslide*<2>{\listcmm[highlightlines=18]{../src/backtrace/camlBacktrace\_\_probably\_raise_351.cmm}{CMM of probably\_raise}}
      \onslide*<3>{\mintobjdump[firstline=5,lastline=35,highlightlines=31]{../src/backtrace/camlBacktrace\_\_probably\_raise_351.objdump}}
    \end{column}
  \end{columns}
  \only<1>{\footnotetext[1]{https://blog.janestreet.com/pattern-matching-and-exception-handling-unite/}}
\end{frame}

\newcommand\listCamlReraiseExn[1][]{
  \listasm[firstline=727,lastline=734,#1]{../ocaml/runtime/amd64.S}{runtime/amd64.S:727}}

\begin{frame}{Backtraces}{Introducing \funcname{caml\_reraise\_exn}}
  \begin{columns}[c]
    \begin{column}{0.5\textwidth}
      \minipage[c][0.66\textheight][s]{\columnwidth}
      \begin{itemize}
        \item<1-> The exception have to be reraised to the next handler
        \item<2-> First, checks if the backtrace should be collected with \funcarg{domain\_state->backtrace\_active}
        \only<3>{
        \begin{itemize}
            \item If not, behaves like a \funcname{raise\_notrace} with \funcname{RESTORE\_EXN\_HANDLER\_OCAML}
        \end{itemize}
        }
        \item<4-> Jumps into \funcname{caml\_raise\_exn}
        \item<5-> Calls \funcname{caml\_stash\_backtrace} again, to scan the next part of the stack: from the trap just pop-ed to the next trap
      \end{itemize}
      \vfill
      \mintocaml[firstline=15,lastline=21,highlightlines={18-21}]{../src/backtrace/backtrace.ml}
      \endminipage
    \end{column}
    \begin{column}{0.5\textwidth}
      \raggedleft
      \onslide*<1>{\listCamlReraiseExn{}}
      \onslide*<2>{\listCamlReraiseExn[highlightlines=729]{}}
      \onslide*<3>{
        \mintc[firstline=190,lastline=192,highlightlines={191-192}]{../ocaml/runtime/amd64.S}
        \mintc[firstline=234,lastline=237,highlightlines={235-237}]{../ocaml/runtime/amd64.S}
        \medskip
        \listCamlReraiseExn[highlightlines=731]{}
      }
      \onslide*<4-5>{
        % TODO refactor with \listCamlReraiseExn
        \mintasm[firstline=727,lastline=734,highlightlines=730]{../ocaml/runtime/amd64.S}
        \medskip
      }
      \onslide*<4>{\listCamlRaiseExn[highlightlines=711]{}}
      \onslide*<5>{\listCamlRaiseExn[highlightlines=720]{}}
    \end{column}
  \end{columns}
\end{frame}

\begin{frame}{Backtraces}{Backtrace collection continues}
  \begin{columns}[c]
    \begin{column}{0.55\textwidth}
      \minipage[c][0.75\textheight][s]{\columnwidth}
      \begin{itemize}
        \item<1-> \funcname{caml\_stash\_backtrace} scans the older part of the stack, up to the next trap
        \item<6-> Controls jumps to the nested exception handler in \funcname{maybe\_raise}, which matches only on \typename{ExnB}
        \item<7-> \funcname{caml\_reraise\_exn} is called again, backtrace collection resumes with \funcname{caml\_stash\_backtrace}
      \end{itemize}
      \medskip
      \begin{itemize}
        \item<2-> Constructed backtrace:
        \begin{itemize}
          \item <2-> \funcname{definitely\_raise}
          \item <2-> \funcname{probably\_raise}
          \item <3-> \funcname{probably\_raise} (re-raise)
          \item <4-> \funcname{maybe\_raise}
          \item <5-> \funcname{main}
          \item <8-> \funcname{main} (re-raise)
        \end{itemize}
      \end{itemize}
      \vfill
      \onslide*<1-5>{\mintocaml[firstline=15,lastline=21]{../src/backtrace/backtrace.ml}}
      \onslide*<6>{\mintocaml[firstline=28,lastline=34,highlightlines=31]{../src/backtrace/backtrace.ml}}
      \onslide*<7->{\mintocaml[firstline=28,lastline=34,highlightlines=33]{../src/backtrace/backtrace.ml}}
      \endminipage
    \end{column}
    \begin{column}{0.45\textwidth}
      \raggedleft
      \onslide*<1>{\providecommand\step{6}%
% Backtraces
%
\subsection{One advantage of raising with debugging information}
\frameSubsection{}{}

\begin{frame}{Backtraces}{Debugging information \& source code}
  \begin{columns}[c]
    \begin{column}{0.5\textwidth}
      \begin{itemize}
        \item Raising an exception is fun and games, but how do we know where the exception originated? $\rightarrow$ With the backtrace associated with the exception!
        \item In order to get a backtrace from the point where an exception was raised, it's necessary to compile with debugging information enabled (\funcarg{-g} flag for the compiler)
        \item Same example as in the `Nested handlers' section
      \end{itemize}
    \end{column}
    \begin{column}{0.5\textwidth}
      \listocaml[firstline=9,lastline=34]{../src/backtrace/backtrace.ml}{backtrace/backtrace.ml}
    \end{column}
  \end{columns}
\end{frame}

\begin{frame}{Backtraces}{CMM and disassembly of \funcname{definitely\_raise}}
  \begin{columns}[c]
    \begin{column}{0.5\textwidth}
      \minipage[c][0.66\textheight][s]{\columnwidth}
      \begin{itemize}
        \item<1-> An effect of compiling with debugging information (\funcarg{-g}): the CMM no longer uses \funcname{raise\_notrace} to raise the exception but \funcname{raise} instead
        \item<2-> Disassembly shows that CMM \funcname{raise} is actually a call to \funcname{caml\_raise\_exn}, a function in assembler provided by the runtime
      \end{itemize}
      \vfill
      \mintocaml[firstline=9,lastline=13,highlightlines=12]{../src/backtrace/backtrace.ml}
      \endminipage
    \end{column}
    \begin{column}{0.5\textwidth}
      \onslide*<1>{
        \mintcmm[highlightlines=5]{../src/backtrace/camlBacktrace\_\_definitely\_raise\_347.cmm}
      }
      \onslide*<2>{
        \mintobjdump[firstline=2,highlightlines=14]{../src/backtrace/camlBacktrace\_\_definitely\_raise\_347.objdump}
      }
    \end{column}
  \end{columns}
\end{frame}

\begin{frame}{Backtraces}{Introducing \funcname{caml\_raise\_exn}}
  \begin{columns}[c]
    \begin{column}{0.5\textwidth}
      \minipage[c][0.66\textheight][s]{\columnwidth}
      \onslide*<1>{
        \begin{itemize}
          \item Raising the exception is performed using \funcname{caml\_raise\_exn} which is
            \begin{itemize}
              \item An assembly function provided by the runtime
              \item Architecture specific
            \end{itemize}
          \item Let's see what it does differently from \funcname{raise\_notrace}\dots
          \item We are about to raise \typename{ExnC} with \funcname{caml\_raise\_exn} from \funcname{definitely\_raise}
        \end{itemize}
      }
      \onslide*<2>{
        \mintobjdump[firstline=2,highlightlines=14]{../src/backtrace/camlBacktrace\_\_definitely\_raise\_347.objdump}
      }
      \vfill
      \mintocaml[firstline=9,lastline=13,highlightlines=12]{../src/backtrace/backtrace.ml}
      \endminipage
    \end{column}
    \begin{column}{0.5\textwidth}
      \centering
      \providecommand\step{1}\input{diagrams/backtrace/backtrace.tex}
    \end{column}
  \end{columns}
\end{frame}

\begin{frame}{Backtraces}{But before that, we need more fields from \typename{caml\_domain\_state}}
  \begin{columns}
    \begin{column}{0.5\textwidth}
      \begin{itemize}
        \item Remember \typename{caml\_domain\_state} the per-domain runtime state?
        \item Some other members are of interest to us now\dots
        \item \localname{backtrace\_active} is used to know if the runtime should collect backtraces
        \item \localname{backtrace\_active} can be set by
          \begin{itemize}
            \item The \funcarg{b} flag in the \funcname{OCAMLRUNPARAM} environment variable
            \item \mintinline{ocaml}{Printexc.record_backtrace : bool -> unit} \footnotemark
          \end{itemize}
      \end{itemize}
    \end{column}
    \begin{column}{0.5\textwidth}
      \begin{listing}
        \mintc[highlightlines={9-11}]{src/ocaml/domain\_state\_backtrace.h}
        \caption{runtime/caml/domain\_state.h}
      \end{listing}
    \end{column}
  \end{columns}
  \footnotetext[1]{https://v2.ocaml.org/api/Printexc.html}
\end{frame}

\newcommand\listCamlRaiseExn[1][]{
  \listasm[firstline=702,lastline=725,#1]{../ocaml/runtime/amd64.S}{runtime/amd64.S:702}}

\begin{frame}{Backtraces}{Walk through \funcname{caml\_raise\_exn}}
  \begin{columns}[T]
    \begin{column}{0.5\textwidth}
      \begin{itemize}
        \item<1-> First checks whether exceptions backtrace should be recorded
        \item<1-> By checking the \funcarg{domain\_state->backtrace\_active} value
        \item<2-> If not, acts as \funcname{raise\_notrace} with \funcname{RESTORE\_EXN\_HANDLER\_OCAML}
          \begin{itemize}
            \item Set \regname{RSP} to \localname{domain\_state->exn\_handler} value
            \item Restore \localname{domain\_state->exn\_handler} from trap
            \item Jump with \funcname{ret} (same effect as using \funcname{pop} and \funcname{jmp})
          \end{itemize}
        \item<3-> Otherwise, switches to the C stack to be able to execute C code
        \item<4-> Calls \funcname{caml\_stash\_backtrace}, a C function provided by the runtime\ignorespaces
          \onslide<6->{, with
            \begin{itemize}
              \item \funcarg{exn}: exception value
              \item \funcarg{pc}: code address of the raising point
              \item \funcarg{sp}: stack address of the raising function
              \item \funcarg{trapsp}: stack address of the next handler
            \end{itemize}
          }
      \end{itemize}
      \bigskip
      \mintocaml[firstline=9,lastline=13,highlightlines=12]{../src/backtrace/backtrace.ml}
    \end{column}
    \begin{column}{0.5\textwidth}
      \raggedleft
      \onslide*<1,3-4>{
        \mintc[firstline=190,lastline=192]{../ocaml/runtime/amd64.S}
        \mintc[firstline=234,lastline=237]{../ocaml/runtime/amd64.S}
      }
      \onslide*<2>{
        \mintc[firstline=190,lastline=192,highlightlines={191-192}]{../ocaml/runtime/amd64.S}
        \mintc[firstline=234,lastline=237,highlightlines={235-237}]{../ocaml/runtime/amd64.S}
      }
      \onslide*<1>{\listCamlRaiseExn[highlightlines=705]{}}%
      \onslide*<2>{\listCamlRaiseExn[highlightlines={707-708}]{}}%
      \onslide*<3>{\listCamlRaiseExn[highlightlines={712-714}]{}}%
      \onslide*<4>{\listCamlRaiseExn[highlightlines={715-720}]{}}%
      \onslide*<5>{\providecommand\step{1}\input{diagrams/backtrace/backtrace.tex}}%
      \onslide*<6>{\providecommand\step{2}\input{diagrams/backtrace/backtrace.tex}}%
    \end{column}
  \end{columns}
\end{frame}

\newcommand\listStashBacktrace[1][]{
  \listc[firstline=91,lastline=120,#1]{../ocaml/runtime/backtrace\_nat.c}{runtime/backtrace\_nat.c:91}}

\begin{frame}{Backtraces}{Introducing \funcname{caml\_stash\_backtrace}}
  \begin{columns}[c]
    \begin{column}{0.5\textwidth}
      \minipage[c][0.66\textheight][s]{\columnwidth}
      \begin{itemize}
        \item<1-> A C function provided by the runtime (specific to native code)
        \item<2-> It scans the stack, between the stack pointer at the raising point up to the next trap, in order to collect the names of the detected function frames
          \onslide*<2>{
            \begin{itemize}
              \item And more information, like line number, column number...
            \end{itemize}
          }
        \item<3-> It uses the same technique and information as the Garbage Collector (GC) uses to scan the values live on the stack
          \onslide*<4>{
            \begin{itemize}
              \item \typename{frame\_descr} are at the core of stack unwinding
            \end{itemize}
          }
      \end{itemize}
      \vfill
      \mintocaml[firstline=9,lastline=13,highlightlines=12]{../src/backtrace/backtrace.ml}
      \endminipage
    \end{column}
    \begin{column}{0.5\textwidth}
      \onslide*<1>{\listStashBacktrace[highlightlines=91]{}}
      \onslide*<2>{\listStashBacktrace[highlightlines={91,117-118}]{}}
      \onslide*<3->{\listStashBacktrace[highlightlines={94,106,109-110}]{}}
    \end{column}
  \end{columns}
\end{frame}

\newcommand\listFrameDescr[1][]{
  \listocaml[firstline=111,lastline=124,#1]{../ocaml/asmcomp/emitaux.ml}{asmcomp/emitaux.ml:111}}
\newcommand\listDebugInfo[1][]{
  \listocaml[firstline=38,lastline=49,#1]{../ocaml/lambda/debuginfo.mli}{lambda/debuginfo.mli:38}}

\begin{frame}{Backtraces}{\typename{frame\_descr} and \typename{frame\_debuginfo} in a nutshell}
  \begin{columns}[c]
    \begin{column}{0.5\textwidth}
      \begin{itemize}
        \item<1-> For every return address in an OCaml program, there is a associated \typename{frame\_descr}
        \item<2-> A \typename{frame\_descr} specifies
        \begin{itemize}
          \item<2-> The size of the function stack frame
          \item<3-> Where live values are on the stack (so that the GC can mark them as live and prevent from being swept)
        \end{itemize}
        \item<4-> When compiled with debug information, \typename{frame\_descr} will be linked to a \typename{frame\_debuginfo}
        \begin{itemize}
          \item<5-> Contains information about the position in the source code (file name, line and column number)
        \end{itemize}
      \end{itemize}
    \end{column}
    \begin{column}{0.5\textwidth}
        \onslide*<1>{\listFrameDescr[highlightlines={118-122}]{}}
        \onslide*<2>{\listFrameDescr[highlightlines={120}]{}}
        \onslide*<3>{\listFrameDescr[highlightlines={121}]{}}
        \onslide*<4->{\listFrameDescr[highlightlines={122}]{}}
        \medskip
        \onslide*<1-3>{\listDebugInfo{}}
        \onslide*<4>{\listDebugInfo[highlightlines={38,49}]{}}
        \onslide*<5>{\listDebugInfo[highlightlines={39-46}]{}}
    \end{column}
  \end{columns}
\end{frame}

\begin{frame}{Backtraces}{Scanning the stack with \typename{frame\_descr}}
  \begin{columns}[c]
    \begin{column}{0.5\textwidth}
      \begin{itemize}
        \item Given the return address of the code that raises the exception by calling \funcname{caml\_raise\_exn} (\funcarg{pc} argument of \funcname{caml\_stash\_backtrace})
        \item And the position of the stack pointer (\funcarg{sp} argument of \funcname{caml\_stash\_backtrace})
        \item The frame size given by \typename{frame\_descr}.\funcarg{frame\_size}, allows to find the end of the previous frame
        \item And the return address of the caller of this function
        \bigskip
        \onslide<3->{
        \item Note that traps (here the reddish \funcname{ExnA}, \funcname{ExnB} and \funcname{ExnC} blocks) belong to the frame that installed them, and so the frame size changes when traps are removed
        }
      \end{itemize}
    \end{column}
    \begin{column}{0.5\textwidth}
      \raggedleft
      \onslide*<1>{\providecommand\step{2}\input{diagrams/backtrace/backtrace.tex}}%
      \onslide*<2>{\providecommand\step{1}\input{diagrams/backtrace/scanstack.tex}}%
      \onslide*<3>{\providecommand\step{2}\input{diagrams/backtrace/scanstack.tex}}%
      \onslide*<4>{\providecommand\step{3}\input{diagrams/backtrace/scanstack.tex}}%
      \onslide*<5>{\providecommand\step{4}\input{diagrams/backtrace/scanstack.tex}}%
      \onslide*<6>{\providecommand\step{5}\input{diagrams/backtrace/scanstack.tex}}%
    \end{column}
  \end{columns}
\end{frame}

% Duplicate of previous-previous slide
\begin{frame}{Backtraces}{Continuing on \funcname{caml\_stash\_backtrace}}
  \begin{columns}[c]
    \begin{column}{0.5\textwidth}
      \minipage[c][0.75\textheight][s]{\columnwidth}
      \begin{itemize}
          \color{gray}
        \item A C function provided by the runtime (specific to native code)
        \item It scans the stack, between the stack pointer at the raising point up to the next trap, in order to collect the names of the detected function frames
        \item It uses the same technique and information as the Garbage Collector (GC) uses to scan the values live on the stack
      \end{itemize}
      \begin{itemize}
        \item For each function frame detected, store a pointer to the \typename{frame\_descr} in the \localname{domain\_state->backtrace\_buffer} array, effectively constructing the backtrace
      \end{itemize}
      \onslide<2->{
        \begin{itemize}
            \smallskip
          \item Constructed backtrace:
            \funcname{definitely\_raise},
            \onslide<3->{\funcname{probably\_raise}}
        \end{itemize}
      }
      \vfill
      \mintocaml[firstline=9,lastline=13,highlightlines=12]{../src/backtrace/backtrace.ml}
      \endminipage
    \end{column}
    \begin{column}{0.5\textwidth}
      \raggedleft
      \onslide*<1>{\listStashBacktrace[highlightlines={114-115}]{}}
      \onslide*<2>{\providecommand\step{3}\input{diagrams/backtrace/backtrace.tex}}%
      \onslide*<3>{\providecommand\step{4}\input{diagrams/backtrace/backtrace.tex}}%
    \end{column}
  \end{columns}
\end{frame}

\begin{frame}{Backtraces}{Back to \funcname{caml\_raise\_exn}}
  \begin{columns}[c]
    \begin{column}{0.5\textwidth}
      \minipage[c][0.66\textheight][s]{\columnwidth}
      \begin{itemize}\color{gray}
        \item Checks wheteher exceptions backtrace should be recorded, by checking the \funcarg{domain\_state->backtrace\_active} value
        \item Switches to the C stack to be able to execute C code
        \item Calls \funcname{caml\_stash\_backtrace}, to collect the backtrace up to the next trap
      \end{itemize}
      \onslide*<2->{
        \begin{itemize}
          \item<2-> Executes the exception handler in \funcname{probably\_raise} with \funcname{RESTORE\_EXN\_HANDLER\_OCAML}
            \begin{itemize}
              \item Set \regname{RSP} to \localname{domain\_state->exn\_handler} value
              \item Restore \localname{domain\_state->exn\_handler} from trap
              \item Jump to the handler code with \funcname{ret}
            \end{itemize}
        \end{itemize}
      }
      \vfill
      \onslide*<1>{\mintocaml[firstline=9,lastline=13,highlightlines=12]{../src/backtrace/backtrace.ml}}
      \onslide*<2->{\mintocaml[firstline=15,lastline=21,highlightlines={19-21}]{../src/backtrace/backtrace.ml}}
      \endminipage
    \end{column}
    \begin{column}{0.5\textwidth}
      \centering
      \onslide*<1>{
        \mintc[firstline=190,lastline=192]{../ocaml/runtime/amd64.S}
        \mintc[firstline=234,lastline=237]{../ocaml/runtime/amd64.S}
      }
      \onslide*<2>{
        \mintc[firstline=190,lastline=192,highlightlines={191-192}]{../ocaml/runtime/amd64.S}
        \mintc[firstline=234,lastline=237,highlightlines={235-237}]{../ocaml/runtime/amd64.S}
      }
      \onslide*<1>{\listCamlRaiseExn[highlightlines=720]{}}
      \onslide*<2>{\listCamlRaiseExn[highlightlines=722]{}}
      \onslide*<3->{\providecommand\step{5}\input{diagrams/backtrace/backtrace.tex}}
    \end{column}
  \end{columns}
\end{frame}

\begin{frame}{Backtraces}{\funcname{caml\_probably\_raise}'s handler doesn't catch \typename{ExnC}}
  \begin{columns}[c]
    \begin{column}{0.45\textwidth}
      \minipage[c][0.75\textheight][s]{\columnwidth}
      \begin{itemize}
        \item<1-> \funcname{match \dots with}\footnotemark pattern matching can also match exceptions
        \item<1-> The CMM of \funcname{probably\_raise} shows that this \funcname{match \dots with} is implemented with a pattern matching and a \funcname{try \dots with}
        \item<1-> But this one only matches on \typename{ExnA} exception
        \item<2-> Since the pattern matching fails to catch the exception, \funcname{reraise} is called with our current \typename{ExnC} exception
        \item<3-> Disassembly shows that \funcname{reraise} is actually a call to \funcname{caml\_reraise\_exn}, which is also an assembly function provided by the runtime
      \end{itemize}
      \vfill
      \mintocaml[firstline=15,lastline=21,highlightlines={18-21}]{../src/backtrace/backtrace.ml}
      \endminipage
    \end{column}
    \begin{column}{0.55\textwidth}
      \centering
      \onslide*<1>{\mintcmm[highlightlines={10-18}]{../src/backtrace/camlBacktrace\_\_probably\_raise\_351.cmm}}
      \onslide*<2>{\listcmm[highlightlines=18]{../src/backtrace/camlBacktrace\_\_probably\_raise_351.cmm}{CMM of probably\_raise}}
      \onslide*<3>{\mintobjdump[firstline=5,lastline=35,highlightlines=31]{../src/backtrace/camlBacktrace\_\_probably\_raise_351.objdump}}
    \end{column}
  \end{columns}
  \only<1>{\footnotetext[1]{https://blog.janestreet.com/pattern-matching-and-exception-handling-unite/}}
\end{frame}

\newcommand\listCamlReraiseExn[1][]{
  \listasm[firstline=727,lastline=734,#1]{../ocaml/runtime/amd64.S}{runtime/amd64.S:727}}

\begin{frame}{Backtraces}{Introducing \funcname{caml\_reraise\_exn}}
  \begin{columns}[c]
    \begin{column}{0.5\textwidth}
      \minipage[c][0.66\textheight][s]{\columnwidth}
      \begin{itemize}
        \item<1-> The exception have to be reraised to the next handler
        \item<2-> First, checks if the backtrace should be collected with \funcarg{domain\_state->backtrace\_active}
        \only<3>{
        \begin{itemize}
            \item If not, behaves like a \funcname{raise\_notrace} with \funcname{RESTORE\_EXN\_HANDLER\_OCAML}
        \end{itemize}
        }
        \item<4-> Jumps into \funcname{caml\_raise\_exn}
        \item<5-> Calls \funcname{caml\_stash\_backtrace} again, to scan the next part of the stack: from the trap just pop-ed to the next trap
      \end{itemize}
      \vfill
      \mintocaml[firstline=15,lastline=21,highlightlines={18-21}]{../src/backtrace/backtrace.ml}
      \endminipage
    \end{column}
    \begin{column}{0.5\textwidth}
      \raggedleft
      \onslide*<1>{\listCamlReraiseExn{}}
      \onslide*<2>{\listCamlReraiseExn[highlightlines=729]{}}
      \onslide*<3>{
        \mintc[firstline=190,lastline=192,highlightlines={191-192}]{../ocaml/runtime/amd64.S}
        \mintc[firstline=234,lastline=237,highlightlines={235-237}]{../ocaml/runtime/amd64.S}
        \medskip
        \listCamlReraiseExn[highlightlines=731]{}
      }
      \onslide*<4-5>{
        % TODO refactor with \listCamlReraiseExn
        \mintasm[firstline=727,lastline=734,highlightlines=730]{../ocaml/runtime/amd64.S}
        \medskip
      }
      \onslide*<4>{\listCamlRaiseExn[highlightlines=711]{}}
      \onslide*<5>{\listCamlRaiseExn[highlightlines=720]{}}
    \end{column}
  \end{columns}
\end{frame}

\begin{frame}{Backtraces}{Backtrace collection continues}
  \begin{columns}[c]
    \begin{column}{0.55\textwidth}
      \minipage[c][0.75\textheight][s]{\columnwidth}
      \begin{itemize}
        \item<1-> \funcname{caml\_stash\_backtrace} scans the older part of the stack, up to the next trap
        \item<6-> Controls jumps to the nested exception handler in \funcname{maybe\_raise}, which matches only on \typename{ExnB}
        \item<7-> \funcname{caml\_reraise\_exn} is called again, backtrace collection resumes with \funcname{caml\_stash\_backtrace}
      \end{itemize}
      \medskip
      \begin{itemize}
        \item<2-> Constructed backtrace:
        \begin{itemize}
          \item <2-> \funcname{definitely\_raise}
          \item <2-> \funcname{probably\_raise}
          \item <3-> \funcname{probably\_raise} (re-raise)
          \item <4-> \funcname{maybe\_raise}
          \item <5-> \funcname{main}
          \item <8-> \funcname{main} (re-raise)
        \end{itemize}
      \end{itemize}
      \vfill
      \onslide*<1-5>{\mintocaml[firstline=15,lastline=21]{../src/backtrace/backtrace.ml}}
      \onslide*<6>{\mintocaml[firstline=28,lastline=34,highlightlines=31]{../src/backtrace/backtrace.ml}}
      \onslide*<7->{\mintocaml[firstline=28,lastline=34,highlightlines=33]{../src/backtrace/backtrace.ml}}
      \endminipage
    \end{column}
    \begin{column}{0.45\textwidth}
      \raggedleft
      \onslide*<1>{\providecommand\step{6}\input{diagrams/backtrace/backtrace.tex}}%
      \onslide*<2>{\providecommand\step{6}\input{diagrams/backtrace/backtrace.tex}}%
      \onslide*<3>{\providecommand\step{7}\input{diagrams/backtrace/backtrace.tex}}%
      \onslide*<4>{\providecommand\step{8}\input{diagrams/backtrace/backtrace.tex}}%
      \onslide*<5>{\providecommand\step{9}\input{diagrams/backtrace/backtrace.tex}}%
      \onslide*<6>{\providecommand\step{10}\input{diagrams/backtrace/backtrace.tex}}%
      \onslide*<7>{\providecommand\step{11}\input{diagrams/backtrace/backtrace.tex}}%
      \onslide*<8>{\providecommand\step{12}\input{diagrams/backtrace/backtrace.tex}}%
      \onslide*<9>{\providecommand\step{13}\input{diagrams/backtrace/backtrace.tex}}%
    \end{column}
  \end{columns}
\end{frame}

\begin{frame}{Backtraces}{Printing the backtrace}
  \begin{columns}[c]
    \begin{column}{0.45\textwidth}
      \minipage[c][0.66\textheight][s]{\columnwidth}
      \begin{itemize}
        \item<1-> The exception backtrace can now be printed to \funcarg{stdout} with \funcname{Printexc.print_backtrace}
        \item<2-> First the exception backtrace is retrieved into an OCaml array with \funcname{get_raw_backtrace }
        \item<3-> Which is implemented as the \funcname{caml_get_exception_raw_backtrace} C function in the runtime
        \item<4-> And finally printed with \funcname{print_exception_backtrace}
      \end{itemize}
      \vfill
      \mintocaml[firstline=28,lastline=34,highlightlines=33]{../src/backtrace/backtrace.ml}
      \endminipage
    \end{column}
    \begin{column}{0.55\textwidth}
      \onslide*<1,2>{
        \mintocaml[firstline=100,lastline=101]{../ocaml/stdlib/printexc.ml}
      }
      \only<1>{
        \listocaml[firstline=170,lastline=172]{../ocaml/stdlib/printexc.ml}{stdlib/printexc.ml:170}
      }
      \only<2>{
        \listocaml[firstline=170,lastline=172,highlightlines=172]{../ocaml/stdlib/printexc.ml}{stdlib/printexc.ml:170}
      }
      \only<3>{
        \mintc[firstline=154,lastline=156]{../ocaml/runtime/backtrace.c}
        \listc[firstline=171,lastline=192,highlightlines={184-187}]{../ocaml/runtime/backtrace.c}{runtime/backtrace.c:154}
      }
      \only<4>{
        \listocaml[firstline=155,lastline=165]{../ocaml/stdlib/printexc.ml}{stdlib/printexc.ml:55}
      }
    \end{column}
  \end{columns}
\end{frame}


\subsection{The multiple kinds of raise}
\frameSubsection{}{}

\begin{frame}{Backtraces}{When backtraces are not needed}
  \begin{columns}[c]
    \begin{column}{0.45\textwidth}
      \minipage[c][0.66\textheight][s]{\columnwidth}
      \begin{itemize}
        \item<1-> What if the backtrace for an exception is never needed?
        \item<2-> It's possible to raise the exception explicitly with \funcname{raise\_notrace} to make raising as cheap as possible
        \item<3-> An example of use in the \funcname{dynlink} library of the OCaml compiler
      \end{itemize}
      \vfill
      \onslide<3->{
      \listocaml[firstline=205,lastline=209,highlightlines=207]{../ocaml/otherlibs/dynlink/dynlink\_compilerlibs/misc.ml}{otherlibs/dynlink/dynlink\_compilerlibs/misc.ml:205}
      }
      \endminipage
    \end{column}
    \begin{column}{0.55\textwidth}
    \only<4>{
      \mintcmm[highlightlines=3]{../src/notrace/camlNotrace\_\_fun_347.cmm}
      \listcmm{../src/notrace/camlNotrace\_\_all\_somes\_327.cmm}{all\_somes}
    }
    \onslide*<5->{
      \listobjdump[highlightlines={6-9}]{../src/notrace/camlNotrace\_\_fun_347.objdump}{anonymous function from \funcname{all\_somes}}
    }
    \end{column}
  \end{columns}
\end{frame}

\begin{frame}{Backtraces}{Summary table of the multiple kinds of raise}
  \begin{itemize}
    \item When debugging information is enabled and if the backtrace of an exception isn't needed, it's possible to raise with \funcname{raise\_notrace} for performance
    \item When debugging information is \emph{not} enabled, the compiler optimises \funcname{raise} calls to inline assembly (same effect as using \funcname{raise\_notrace})
  \end{itemize}
  \bigskip
  \centering
  \begin{tabular}{| l | c | c | c | c |}
    \hline
    \parbox{10em}{Debug information \\ (\funcarg{-g} flag)}  & \multicolumn{2}{|c|}{Disabled}                                     & \multicolumn{2}{|c|}{Enabled} \\
    \hline
    Raise kind                                               & \funcname{raise} & \funcname{raise\_notrace}                       & \funcname{raise} & \funcname{raise\_notrace} \\
    \hline
    Compilation                                              & \parbox{8em}{inline assembly\\ (optimization)} & inline assembly   & \funcname{caml\_raise\_exn} & inline assembly \\
    \hline
  \end{tabular}
\end{frame}


%
% Takeaway #5
%
\subsection*{Takeaway \#5}
\frameSubsectionTakeaway{}
\againframe<5>{takeaway}
}%
      \onslide*<2>{\providecommand\step{6}%
% Backtraces
%
\subsection{One advantage of raising with debugging information}
\frameSubsection{}{}

\begin{frame}{Backtraces}{Debugging information \& source code}
  \begin{columns}[c]
    \begin{column}{0.5\textwidth}
      \begin{itemize}
        \item Raising an exception is fun and games, but how do we know where the exception originated? $\rightarrow$ With the backtrace associated with the exception!
        \item In order to get a backtrace from the point where an exception was raised, it's necessary to compile with debugging information enabled (\funcarg{-g} flag for the compiler)
        \item Same example as in the `Nested handlers' section
      \end{itemize}
    \end{column}
    \begin{column}{0.5\textwidth}
      \listocaml[firstline=9,lastline=34]{../src/backtrace/backtrace.ml}{backtrace/backtrace.ml}
    \end{column}
  \end{columns}
\end{frame}

\begin{frame}{Backtraces}{CMM and disassembly of \funcname{definitely\_raise}}
  \begin{columns}[c]
    \begin{column}{0.5\textwidth}
      \minipage[c][0.66\textheight][s]{\columnwidth}
      \begin{itemize}
        \item<1-> An effect of compiling with debugging information (\funcarg{-g}): the CMM no longer uses \funcname{raise\_notrace} to raise the exception but \funcname{raise} instead
        \item<2-> Disassembly shows that CMM \funcname{raise} is actually a call to \funcname{caml\_raise\_exn}, a function in assembler provided by the runtime
      \end{itemize}
      \vfill
      \mintocaml[firstline=9,lastline=13,highlightlines=12]{../src/backtrace/backtrace.ml}
      \endminipage
    \end{column}
    \begin{column}{0.5\textwidth}
      \onslide*<1>{
        \mintcmm[highlightlines=5]{../src/backtrace/camlBacktrace\_\_definitely\_raise\_347.cmm}
      }
      \onslide*<2>{
        \mintobjdump[firstline=2,highlightlines=14]{../src/backtrace/camlBacktrace\_\_definitely\_raise\_347.objdump}
      }
    \end{column}
  \end{columns}
\end{frame}

\begin{frame}{Backtraces}{Introducing \funcname{caml\_raise\_exn}}
  \begin{columns}[c]
    \begin{column}{0.5\textwidth}
      \minipage[c][0.66\textheight][s]{\columnwidth}
      \onslide*<1>{
        \begin{itemize}
          \item Raising the exception is performed using \funcname{caml\_raise\_exn} which is
            \begin{itemize}
              \item An assembly function provided by the runtime
              \item Architecture specific
            \end{itemize}
          \item Let's see what it does differently from \funcname{raise\_notrace}\dots
          \item We are about to raise \typename{ExnC} with \funcname{caml\_raise\_exn} from \funcname{definitely\_raise}
        \end{itemize}
      }
      \onslide*<2>{
        \mintobjdump[firstline=2,highlightlines=14]{../src/backtrace/camlBacktrace\_\_definitely\_raise\_347.objdump}
      }
      \vfill
      \mintocaml[firstline=9,lastline=13,highlightlines=12]{../src/backtrace/backtrace.ml}
      \endminipage
    \end{column}
    \begin{column}{0.5\textwidth}
      \centering
      \providecommand\step{1}\input{diagrams/backtrace/backtrace.tex}
    \end{column}
  \end{columns}
\end{frame}

\begin{frame}{Backtraces}{But before that, we need more fields from \typename{caml\_domain\_state}}
  \begin{columns}
    \begin{column}{0.5\textwidth}
      \begin{itemize}
        \item Remember \typename{caml\_domain\_state} the per-domain runtime state?
        \item Some other members are of interest to us now\dots
        \item \localname{backtrace\_active} is used to know if the runtime should collect backtraces
        \item \localname{backtrace\_active} can be set by
          \begin{itemize}
            \item The \funcarg{b} flag in the \funcname{OCAMLRUNPARAM} environment variable
            \item \mintinline{ocaml}{Printexc.record_backtrace : bool -> unit} \footnotemark
          \end{itemize}
      \end{itemize}
    \end{column}
    \begin{column}{0.5\textwidth}
      \begin{listing}
        \mintc[highlightlines={9-11}]{src/ocaml/domain\_state\_backtrace.h}
        \caption{runtime/caml/domain\_state.h}
      \end{listing}
    \end{column}
  \end{columns}
  \footnotetext[1]{https://v2.ocaml.org/api/Printexc.html}
\end{frame}

\newcommand\listCamlRaiseExn[1][]{
  \listasm[firstline=702,lastline=725,#1]{../ocaml/runtime/amd64.S}{runtime/amd64.S:702}}

\begin{frame}{Backtraces}{Walk through \funcname{caml\_raise\_exn}}
  \begin{columns}[T]
    \begin{column}{0.5\textwidth}
      \begin{itemize}
        \item<1-> First checks whether exceptions backtrace should be recorded
        \item<1-> By checking the \funcarg{domain\_state->backtrace\_active} value
        \item<2-> If not, acts as \funcname{raise\_notrace} with \funcname{RESTORE\_EXN\_HANDLER\_OCAML}
          \begin{itemize}
            \item Set \regname{RSP} to \localname{domain\_state->exn\_handler} value
            \item Restore \localname{domain\_state->exn\_handler} from trap
            \item Jump with \funcname{ret} (same effect as using \funcname{pop} and \funcname{jmp})
          \end{itemize}
        \item<3-> Otherwise, switches to the C stack to be able to execute C code
        \item<4-> Calls \funcname{caml\_stash\_backtrace}, a C function provided by the runtime\ignorespaces
          \onslide<6->{, with
            \begin{itemize}
              \item \funcarg{exn}: exception value
              \item \funcarg{pc}: code address of the raising point
              \item \funcarg{sp}: stack address of the raising function
              \item \funcarg{trapsp}: stack address of the next handler
            \end{itemize}
          }
      \end{itemize}
      \bigskip
      \mintocaml[firstline=9,lastline=13,highlightlines=12]{../src/backtrace/backtrace.ml}
    \end{column}
    \begin{column}{0.5\textwidth}
      \raggedleft
      \onslide*<1,3-4>{
        \mintc[firstline=190,lastline=192]{../ocaml/runtime/amd64.S}
        \mintc[firstline=234,lastline=237]{../ocaml/runtime/amd64.S}
      }
      \onslide*<2>{
        \mintc[firstline=190,lastline=192,highlightlines={191-192}]{../ocaml/runtime/amd64.S}
        \mintc[firstline=234,lastline=237,highlightlines={235-237}]{../ocaml/runtime/amd64.S}
      }
      \onslide*<1>{\listCamlRaiseExn[highlightlines=705]{}}%
      \onslide*<2>{\listCamlRaiseExn[highlightlines={707-708}]{}}%
      \onslide*<3>{\listCamlRaiseExn[highlightlines={712-714}]{}}%
      \onslide*<4>{\listCamlRaiseExn[highlightlines={715-720}]{}}%
      \onslide*<5>{\providecommand\step{1}\input{diagrams/backtrace/backtrace.tex}}%
      \onslide*<6>{\providecommand\step{2}\input{diagrams/backtrace/backtrace.tex}}%
    \end{column}
  \end{columns}
\end{frame}

\newcommand\listStashBacktrace[1][]{
  \listc[firstline=91,lastline=120,#1]{../ocaml/runtime/backtrace\_nat.c}{runtime/backtrace\_nat.c:91}}

\begin{frame}{Backtraces}{Introducing \funcname{caml\_stash\_backtrace}}
  \begin{columns}[c]
    \begin{column}{0.5\textwidth}
      \minipage[c][0.66\textheight][s]{\columnwidth}
      \begin{itemize}
        \item<1-> A C function provided by the runtime (specific to native code)
        \item<2-> It scans the stack, between the stack pointer at the raising point up to the next trap, in order to collect the names of the detected function frames
          \onslide*<2>{
            \begin{itemize}
              \item And more information, like line number, column number...
            \end{itemize}
          }
        \item<3-> It uses the same technique and information as the Garbage Collector (GC) uses to scan the values live on the stack
          \onslide*<4>{
            \begin{itemize}
              \item \typename{frame\_descr} are at the core of stack unwinding
            \end{itemize}
          }
      \end{itemize}
      \vfill
      \mintocaml[firstline=9,lastline=13,highlightlines=12]{../src/backtrace/backtrace.ml}
      \endminipage
    \end{column}
    \begin{column}{0.5\textwidth}
      \onslide*<1>{\listStashBacktrace[highlightlines=91]{}}
      \onslide*<2>{\listStashBacktrace[highlightlines={91,117-118}]{}}
      \onslide*<3->{\listStashBacktrace[highlightlines={94,106,109-110}]{}}
    \end{column}
  \end{columns}
\end{frame}

\newcommand\listFrameDescr[1][]{
  \listocaml[firstline=111,lastline=124,#1]{../ocaml/asmcomp/emitaux.ml}{asmcomp/emitaux.ml:111}}
\newcommand\listDebugInfo[1][]{
  \listocaml[firstline=38,lastline=49,#1]{../ocaml/lambda/debuginfo.mli}{lambda/debuginfo.mli:38}}

\begin{frame}{Backtraces}{\typename{frame\_descr} and \typename{frame\_debuginfo} in a nutshell}
  \begin{columns}[c]
    \begin{column}{0.5\textwidth}
      \begin{itemize}
        \item<1-> For every return address in an OCaml program, there is a associated \typename{frame\_descr}
        \item<2-> A \typename{frame\_descr} specifies
        \begin{itemize}
          \item<2-> The size of the function stack frame
          \item<3-> Where live values are on the stack (so that the GC can mark them as live and prevent from being swept)
        \end{itemize}
        \item<4-> When compiled with debug information, \typename{frame\_descr} will be linked to a \typename{frame\_debuginfo}
        \begin{itemize}
          \item<5-> Contains information about the position in the source code (file name, line and column number)
        \end{itemize}
      \end{itemize}
    \end{column}
    \begin{column}{0.5\textwidth}
        \onslide*<1>{\listFrameDescr[highlightlines={118-122}]{}}
        \onslide*<2>{\listFrameDescr[highlightlines={120}]{}}
        \onslide*<3>{\listFrameDescr[highlightlines={121}]{}}
        \onslide*<4->{\listFrameDescr[highlightlines={122}]{}}
        \medskip
        \onslide*<1-3>{\listDebugInfo{}}
        \onslide*<4>{\listDebugInfo[highlightlines={38,49}]{}}
        \onslide*<5>{\listDebugInfo[highlightlines={39-46}]{}}
    \end{column}
  \end{columns}
\end{frame}

\begin{frame}{Backtraces}{Scanning the stack with \typename{frame\_descr}}
  \begin{columns}[c]
    \begin{column}{0.5\textwidth}
      \begin{itemize}
        \item Given the return address of the code that raises the exception by calling \funcname{caml\_raise\_exn} (\funcarg{pc} argument of \funcname{caml\_stash\_backtrace})
        \item And the position of the stack pointer (\funcarg{sp} argument of \funcname{caml\_stash\_backtrace})
        \item The frame size given by \typename{frame\_descr}.\funcarg{frame\_size}, allows to find the end of the previous frame
        \item And the return address of the caller of this function
        \bigskip
        \onslide<3->{
        \item Note that traps (here the reddish \funcname{ExnA}, \funcname{ExnB} and \funcname{ExnC} blocks) belong to the frame that installed them, and so the frame size changes when traps are removed
        }
      \end{itemize}
    \end{column}
    \begin{column}{0.5\textwidth}
      \raggedleft
      \onslide*<1>{\providecommand\step{2}\input{diagrams/backtrace/backtrace.tex}}%
      \onslide*<2>{\providecommand\step{1}\input{diagrams/backtrace/scanstack.tex}}%
      \onslide*<3>{\providecommand\step{2}\input{diagrams/backtrace/scanstack.tex}}%
      \onslide*<4>{\providecommand\step{3}\input{diagrams/backtrace/scanstack.tex}}%
      \onslide*<5>{\providecommand\step{4}\input{diagrams/backtrace/scanstack.tex}}%
      \onslide*<6>{\providecommand\step{5}\input{diagrams/backtrace/scanstack.tex}}%
    \end{column}
  \end{columns}
\end{frame}

% Duplicate of previous-previous slide
\begin{frame}{Backtraces}{Continuing on \funcname{caml\_stash\_backtrace}}
  \begin{columns}[c]
    \begin{column}{0.5\textwidth}
      \minipage[c][0.75\textheight][s]{\columnwidth}
      \begin{itemize}
          \color{gray}
        \item A C function provided by the runtime (specific to native code)
        \item It scans the stack, between the stack pointer at the raising point up to the next trap, in order to collect the names of the detected function frames
        \item It uses the same technique and information as the Garbage Collector (GC) uses to scan the values live on the stack
      \end{itemize}
      \begin{itemize}
        \item For each function frame detected, store a pointer to the \typename{frame\_descr} in the \localname{domain\_state->backtrace\_buffer} array, effectively constructing the backtrace
      \end{itemize}
      \onslide<2->{
        \begin{itemize}
            \smallskip
          \item Constructed backtrace:
            \funcname{definitely\_raise},
            \onslide<3->{\funcname{probably\_raise}}
        \end{itemize}
      }
      \vfill
      \mintocaml[firstline=9,lastline=13,highlightlines=12]{../src/backtrace/backtrace.ml}
      \endminipage
    \end{column}
    \begin{column}{0.5\textwidth}
      \raggedleft
      \onslide*<1>{\listStashBacktrace[highlightlines={114-115}]{}}
      \onslide*<2>{\providecommand\step{3}\input{diagrams/backtrace/backtrace.tex}}%
      \onslide*<3>{\providecommand\step{4}\input{diagrams/backtrace/backtrace.tex}}%
    \end{column}
  \end{columns}
\end{frame}

\begin{frame}{Backtraces}{Back to \funcname{caml\_raise\_exn}}
  \begin{columns}[c]
    \begin{column}{0.5\textwidth}
      \minipage[c][0.66\textheight][s]{\columnwidth}
      \begin{itemize}\color{gray}
        \item Checks wheteher exceptions backtrace should be recorded, by checking the \funcarg{domain\_state->backtrace\_active} value
        \item Switches to the C stack to be able to execute C code
        \item Calls \funcname{caml\_stash\_backtrace}, to collect the backtrace up to the next trap
      \end{itemize}
      \onslide*<2->{
        \begin{itemize}
          \item<2-> Executes the exception handler in \funcname{probably\_raise} with \funcname{RESTORE\_EXN\_HANDLER\_OCAML}
            \begin{itemize}
              \item Set \regname{RSP} to \localname{domain\_state->exn\_handler} value
              \item Restore \localname{domain\_state->exn\_handler} from trap
              \item Jump to the handler code with \funcname{ret}
            \end{itemize}
        \end{itemize}
      }
      \vfill
      \onslide*<1>{\mintocaml[firstline=9,lastline=13,highlightlines=12]{../src/backtrace/backtrace.ml}}
      \onslide*<2->{\mintocaml[firstline=15,lastline=21,highlightlines={19-21}]{../src/backtrace/backtrace.ml}}
      \endminipage
    \end{column}
    \begin{column}{0.5\textwidth}
      \centering
      \onslide*<1>{
        \mintc[firstline=190,lastline=192]{../ocaml/runtime/amd64.S}
        \mintc[firstline=234,lastline=237]{../ocaml/runtime/amd64.S}
      }
      \onslide*<2>{
        \mintc[firstline=190,lastline=192,highlightlines={191-192}]{../ocaml/runtime/amd64.S}
        \mintc[firstline=234,lastline=237,highlightlines={235-237}]{../ocaml/runtime/amd64.S}
      }
      \onslide*<1>{\listCamlRaiseExn[highlightlines=720]{}}
      \onslide*<2>{\listCamlRaiseExn[highlightlines=722]{}}
      \onslide*<3->{\providecommand\step{5}\input{diagrams/backtrace/backtrace.tex}}
    \end{column}
  \end{columns}
\end{frame}

\begin{frame}{Backtraces}{\funcname{caml\_probably\_raise}'s handler doesn't catch \typename{ExnC}}
  \begin{columns}[c]
    \begin{column}{0.45\textwidth}
      \minipage[c][0.75\textheight][s]{\columnwidth}
      \begin{itemize}
        \item<1-> \funcname{match \dots with}\footnotemark pattern matching can also match exceptions
        \item<1-> The CMM of \funcname{probably\_raise} shows that this \funcname{match \dots with} is implemented with a pattern matching and a \funcname{try \dots with}
        \item<1-> But this one only matches on \typename{ExnA} exception
        \item<2-> Since the pattern matching fails to catch the exception, \funcname{reraise} is called with our current \typename{ExnC} exception
        \item<3-> Disassembly shows that \funcname{reraise} is actually a call to \funcname{caml\_reraise\_exn}, which is also an assembly function provided by the runtime
      \end{itemize}
      \vfill
      \mintocaml[firstline=15,lastline=21,highlightlines={18-21}]{../src/backtrace/backtrace.ml}
      \endminipage
    \end{column}
    \begin{column}{0.55\textwidth}
      \centering
      \onslide*<1>{\mintcmm[highlightlines={10-18}]{../src/backtrace/camlBacktrace\_\_probably\_raise\_351.cmm}}
      \onslide*<2>{\listcmm[highlightlines=18]{../src/backtrace/camlBacktrace\_\_probably\_raise_351.cmm}{CMM of probably\_raise}}
      \onslide*<3>{\mintobjdump[firstline=5,lastline=35,highlightlines=31]{../src/backtrace/camlBacktrace\_\_probably\_raise_351.objdump}}
    \end{column}
  \end{columns}
  \only<1>{\footnotetext[1]{https://blog.janestreet.com/pattern-matching-and-exception-handling-unite/}}
\end{frame}

\newcommand\listCamlReraiseExn[1][]{
  \listasm[firstline=727,lastline=734,#1]{../ocaml/runtime/amd64.S}{runtime/amd64.S:727}}

\begin{frame}{Backtraces}{Introducing \funcname{caml\_reraise\_exn}}
  \begin{columns}[c]
    \begin{column}{0.5\textwidth}
      \minipage[c][0.66\textheight][s]{\columnwidth}
      \begin{itemize}
        \item<1-> The exception have to be reraised to the next handler
        \item<2-> First, checks if the backtrace should be collected with \funcarg{domain\_state->backtrace\_active}
        \only<3>{
        \begin{itemize}
            \item If not, behaves like a \funcname{raise\_notrace} with \funcname{RESTORE\_EXN\_HANDLER\_OCAML}
        \end{itemize}
        }
        \item<4-> Jumps into \funcname{caml\_raise\_exn}
        \item<5-> Calls \funcname{caml\_stash\_backtrace} again, to scan the next part of the stack: from the trap just pop-ed to the next trap
      \end{itemize}
      \vfill
      \mintocaml[firstline=15,lastline=21,highlightlines={18-21}]{../src/backtrace/backtrace.ml}
      \endminipage
    \end{column}
    \begin{column}{0.5\textwidth}
      \raggedleft
      \onslide*<1>{\listCamlReraiseExn{}}
      \onslide*<2>{\listCamlReraiseExn[highlightlines=729]{}}
      \onslide*<3>{
        \mintc[firstline=190,lastline=192,highlightlines={191-192}]{../ocaml/runtime/amd64.S}
        \mintc[firstline=234,lastline=237,highlightlines={235-237}]{../ocaml/runtime/amd64.S}
        \medskip
        \listCamlReraiseExn[highlightlines=731]{}
      }
      \onslide*<4-5>{
        % TODO refactor with \listCamlReraiseExn
        \mintasm[firstline=727,lastline=734,highlightlines=730]{../ocaml/runtime/amd64.S}
        \medskip
      }
      \onslide*<4>{\listCamlRaiseExn[highlightlines=711]{}}
      \onslide*<5>{\listCamlRaiseExn[highlightlines=720]{}}
    \end{column}
  \end{columns}
\end{frame}

\begin{frame}{Backtraces}{Backtrace collection continues}
  \begin{columns}[c]
    \begin{column}{0.55\textwidth}
      \minipage[c][0.75\textheight][s]{\columnwidth}
      \begin{itemize}
        \item<1-> \funcname{caml\_stash\_backtrace} scans the older part of the stack, up to the next trap
        \item<6-> Controls jumps to the nested exception handler in \funcname{maybe\_raise}, which matches only on \typename{ExnB}
        \item<7-> \funcname{caml\_reraise\_exn} is called again, backtrace collection resumes with \funcname{caml\_stash\_backtrace}
      \end{itemize}
      \medskip
      \begin{itemize}
        \item<2-> Constructed backtrace:
        \begin{itemize}
          \item <2-> \funcname{definitely\_raise}
          \item <2-> \funcname{probably\_raise}
          \item <3-> \funcname{probably\_raise} (re-raise)
          \item <4-> \funcname{maybe\_raise}
          \item <5-> \funcname{main}
          \item <8-> \funcname{main} (re-raise)
        \end{itemize}
      \end{itemize}
      \vfill
      \onslide*<1-5>{\mintocaml[firstline=15,lastline=21]{../src/backtrace/backtrace.ml}}
      \onslide*<6>{\mintocaml[firstline=28,lastline=34,highlightlines=31]{../src/backtrace/backtrace.ml}}
      \onslide*<7->{\mintocaml[firstline=28,lastline=34,highlightlines=33]{../src/backtrace/backtrace.ml}}
      \endminipage
    \end{column}
    \begin{column}{0.45\textwidth}
      \raggedleft
      \onslide*<1>{\providecommand\step{6}\input{diagrams/backtrace/backtrace.tex}}%
      \onslide*<2>{\providecommand\step{6}\input{diagrams/backtrace/backtrace.tex}}%
      \onslide*<3>{\providecommand\step{7}\input{diagrams/backtrace/backtrace.tex}}%
      \onslide*<4>{\providecommand\step{8}\input{diagrams/backtrace/backtrace.tex}}%
      \onslide*<5>{\providecommand\step{9}\input{diagrams/backtrace/backtrace.tex}}%
      \onslide*<6>{\providecommand\step{10}\input{diagrams/backtrace/backtrace.tex}}%
      \onslide*<7>{\providecommand\step{11}\input{diagrams/backtrace/backtrace.tex}}%
      \onslide*<8>{\providecommand\step{12}\input{diagrams/backtrace/backtrace.tex}}%
      \onslide*<9>{\providecommand\step{13}\input{diagrams/backtrace/backtrace.tex}}%
    \end{column}
  \end{columns}
\end{frame}

\begin{frame}{Backtraces}{Printing the backtrace}
  \begin{columns}[c]
    \begin{column}{0.45\textwidth}
      \minipage[c][0.66\textheight][s]{\columnwidth}
      \begin{itemize}
        \item<1-> The exception backtrace can now be printed to \funcarg{stdout} with \funcname{Printexc.print_backtrace}
        \item<2-> First the exception backtrace is retrieved into an OCaml array with \funcname{get_raw_backtrace }
        \item<3-> Which is implemented as the \funcname{caml_get_exception_raw_backtrace} C function in the runtime
        \item<4-> And finally printed with \funcname{print_exception_backtrace}
      \end{itemize}
      \vfill
      \mintocaml[firstline=28,lastline=34,highlightlines=33]{../src/backtrace/backtrace.ml}
      \endminipage
    \end{column}
    \begin{column}{0.55\textwidth}
      \onslide*<1,2>{
        \mintocaml[firstline=100,lastline=101]{../ocaml/stdlib/printexc.ml}
      }
      \only<1>{
        \listocaml[firstline=170,lastline=172]{../ocaml/stdlib/printexc.ml}{stdlib/printexc.ml:170}
      }
      \only<2>{
        \listocaml[firstline=170,lastline=172,highlightlines=172]{../ocaml/stdlib/printexc.ml}{stdlib/printexc.ml:170}
      }
      \only<3>{
        \mintc[firstline=154,lastline=156]{../ocaml/runtime/backtrace.c}
        \listc[firstline=171,lastline=192,highlightlines={184-187}]{../ocaml/runtime/backtrace.c}{runtime/backtrace.c:154}
      }
      \only<4>{
        \listocaml[firstline=155,lastline=165]{../ocaml/stdlib/printexc.ml}{stdlib/printexc.ml:55}
      }
    \end{column}
  \end{columns}
\end{frame}


\subsection{The multiple kinds of raise}
\frameSubsection{}{}

\begin{frame}{Backtraces}{When backtraces are not needed}
  \begin{columns}[c]
    \begin{column}{0.45\textwidth}
      \minipage[c][0.66\textheight][s]{\columnwidth}
      \begin{itemize}
        \item<1-> What if the backtrace for an exception is never needed?
        \item<2-> It's possible to raise the exception explicitly with \funcname{raise\_notrace} to make raising as cheap as possible
        \item<3-> An example of use in the \funcname{dynlink} library of the OCaml compiler
      \end{itemize}
      \vfill
      \onslide<3->{
      \listocaml[firstline=205,lastline=209,highlightlines=207]{../ocaml/otherlibs/dynlink/dynlink\_compilerlibs/misc.ml}{otherlibs/dynlink/dynlink\_compilerlibs/misc.ml:205}
      }
      \endminipage
    \end{column}
    \begin{column}{0.55\textwidth}
    \only<4>{
      \mintcmm[highlightlines=3]{../src/notrace/camlNotrace\_\_fun_347.cmm}
      \listcmm{../src/notrace/camlNotrace\_\_all\_somes\_327.cmm}{all\_somes}
    }
    \onslide*<5->{
      \listobjdump[highlightlines={6-9}]{../src/notrace/camlNotrace\_\_fun_347.objdump}{anonymous function from \funcname{all\_somes}}
    }
    \end{column}
  \end{columns}
\end{frame}

\begin{frame}{Backtraces}{Summary table of the multiple kinds of raise}
  \begin{itemize}
    \item When debugging information is enabled and if the backtrace of an exception isn't needed, it's possible to raise with \funcname{raise\_notrace} for performance
    \item When debugging information is \emph{not} enabled, the compiler optimises \funcname{raise} calls to inline assembly (same effect as using \funcname{raise\_notrace})
  \end{itemize}
  \bigskip
  \centering
  \begin{tabular}{| l | c | c | c | c |}
    \hline
    \parbox{10em}{Debug information \\ (\funcarg{-g} flag)}  & \multicolumn{2}{|c|}{Disabled}                                     & \multicolumn{2}{|c|}{Enabled} \\
    \hline
    Raise kind                                               & \funcname{raise} & \funcname{raise\_notrace}                       & \funcname{raise} & \funcname{raise\_notrace} \\
    \hline
    Compilation                                              & \parbox{8em}{inline assembly\\ (optimization)} & inline assembly   & \funcname{caml\_raise\_exn} & inline assembly \\
    \hline
  \end{tabular}
\end{frame}


%
% Takeaway #5
%
\subsection*{Takeaway \#5}
\frameSubsectionTakeaway{}
\againframe<5>{takeaway}
}%
      \onslide*<3>{\providecommand\step{7}%
% Backtraces
%
\subsection{One advantage of raising with debugging information}
\frameSubsection{}{}

\begin{frame}{Backtraces}{Debugging information \& source code}
  \begin{columns}[c]
    \begin{column}{0.5\textwidth}
      \begin{itemize}
        \item Raising an exception is fun and games, but how do we know where the exception originated? $\rightarrow$ With the backtrace associated with the exception!
        \item In order to get a backtrace from the point where an exception was raised, it's necessary to compile with debugging information enabled (\funcarg{-g} flag for the compiler)
        \item Same example as in the `Nested handlers' section
      \end{itemize}
    \end{column}
    \begin{column}{0.5\textwidth}
      \listocaml[firstline=9,lastline=34]{../src/backtrace/backtrace.ml}{backtrace/backtrace.ml}
    \end{column}
  \end{columns}
\end{frame}

\begin{frame}{Backtraces}{CMM and disassembly of \funcname{definitely\_raise}}
  \begin{columns}[c]
    \begin{column}{0.5\textwidth}
      \minipage[c][0.66\textheight][s]{\columnwidth}
      \begin{itemize}
        \item<1-> An effect of compiling with debugging information (\funcarg{-g}): the CMM no longer uses \funcname{raise\_notrace} to raise the exception but \funcname{raise} instead
        \item<2-> Disassembly shows that CMM \funcname{raise} is actually a call to \funcname{caml\_raise\_exn}, a function in assembler provided by the runtime
      \end{itemize}
      \vfill
      \mintocaml[firstline=9,lastline=13,highlightlines=12]{../src/backtrace/backtrace.ml}
      \endminipage
    \end{column}
    \begin{column}{0.5\textwidth}
      \onslide*<1>{
        \mintcmm[highlightlines=5]{../src/backtrace/camlBacktrace\_\_definitely\_raise\_347.cmm}
      }
      \onslide*<2>{
        \mintobjdump[firstline=2,highlightlines=14]{../src/backtrace/camlBacktrace\_\_definitely\_raise\_347.objdump}
      }
    \end{column}
  \end{columns}
\end{frame}

\begin{frame}{Backtraces}{Introducing \funcname{caml\_raise\_exn}}
  \begin{columns}[c]
    \begin{column}{0.5\textwidth}
      \minipage[c][0.66\textheight][s]{\columnwidth}
      \onslide*<1>{
        \begin{itemize}
          \item Raising the exception is performed using \funcname{caml\_raise\_exn} which is
            \begin{itemize}
              \item An assembly function provided by the runtime
              \item Architecture specific
            \end{itemize}
          \item Let's see what it does differently from \funcname{raise\_notrace}\dots
          \item We are about to raise \typename{ExnC} with \funcname{caml\_raise\_exn} from \funcname{definitely\_raise}
        \end{itemize}
      }
      \onslide*<2>{
        \mintobjdump[firstline=2,highlightlines=14]{../src/backtrace/camlBacktrace\_\_definitely\_raise\_347.objdump}
      }
      \vfill
      \mintocaml[firstline=9,lastline=13,highlightlines=12]{../src/backtrace/backtrace.ml}
      \endminipage
    \end{column}
    \begin{column}{0.5\textwidth}
      \centering
      \providecommand\step{1}\input{diagrams/backtrace/backtrace.tex}
    \end{column}
  \end{columns}
\end{frame}

\begin{frame}{Backtraces}{But before that, we need more fields from \typename{caml\_domain\_state}}
  \begin{columns}
    \begin{column}{0.5\textwidth}
      \begin{itemize}
        \item Remember \typename{caml\_domain\_state} the per-domain runtime state?
        \item Some other members are of interest to us now\dots
        \item \localname{backtrace\_active} is used to know if the runtime should collect backtraces
        \item \localname{backtrace\_active} can be set by
          \begin{itemize}
            \item The \funcarg{b} flag in the \funcname{OCAMLRUNPARAM} environment variable
            \item \mintinline{ocaml}{Printexc.record_backtrace : bool -> unit} \footnotemark
          \end{itemize}
      \end{itemize}
    \end{column}
    \begin{column}{0.5\textwidth}
      \begin{listing}
        \mintc[highlightlines={9-11}]{src/ocaml/domain\_state\_backtrace.h}
        \caption{runtime/caml/domain\_state.h}
      \end{listing}
    \end{column}
  \end{columns}
  \footnotetext[1]{https://v2.ocaml.org/api/Printexc.html}
\end{frame}

\newcommand\listCamlRaiseExn[1][]{
  \listasm[firstline=702,lastline=725,#1]{../ocaml/runtime/amd64.S}{runtime/amd64.S:702}}

\begin{frame}{Backtraces}{Walk through \funcname{caml\_raise\_exn}}
  \begin{columns}[T]
    \begin{column}{0.5\textwidth}
      \begin{itemize}
        \item<1-> First checks whether exceptions backtrace should be recorded
        \item<1-> By checking the \funcarg{domain\_state->backtrace\_active} value
        \item<2-> If not, acts as \funcname{raise\_notrace} with \funcname{RESTORE\_EXN\_HANDLER\_OCAML}
          \begin{itemize}
            \item Set \regname{RSP} to \localname{domain\_state->exn\_handler} value
            \item Restore \localname{domain\_state->exn\_handler} from trap
            \item Jump with \funcname{ret} (same effect as using \funcname{pop} and \funcname{jmp})
          \end{itemize}
        \item<3-> Otherwise, switches to the C stack to be able to execute C code
        \item<4-> Calls \funcname{caml\_stash\_backtrace}, a C function provided by the runtime\ignorespaces
          \onslide<6->{, with
            \begin{itemize}
              \item \funcarg{exn}: exception value
              \item \funcarg{pc}: code address of the raising point
              \item \funcarg{sp}: stack address of the raising function
              \item \funcarg{trapsp}: stack address of the next handler
            \end{itemize}
          }
      \end{itemize}
      \bigskip
      \mintocaml[firstline=9,lastline=13,highlightlines=12]{../src/backtrace/backtrace.ml}
    \end{column}
    \begin{column}{0.5\textwidth}
      \raggedleft
      \onslide*<1,3-4>{
        \mintc[firstline=190,lastline=192]{../ocaml/runtime/amd64.S}
        \mintc[firstline=234,lastline=237]{../ocaml/runtime/amd64.S}
      }
      \onslide*<2>{
        \mintc[firstline=190,lastline=192,highlightlines={191-192}]{../ocaml/runtime/amd64.S}
        \mintc[firstline=234,lastline=237,highlightlines={235-237}]{../ocaml/runtime/amd64.S}
      }
      \onslide*<1>{\listCamlRaiseExn[highlightlines=705]{}}%
      \onslide*<2>{\listCamlRaiseExn[highlightlines={707-708}]{}}%
      \onslide*<3>{\listCamlRaiseExn[highlightlines={712-714}]{}}%
      \onslide*<4>{\listCamlRaiseExn[highlightlines={715-720}]{}}%
      \onslide*<5>{\providecommand\step{1}\input{diagrams/backtrace/backtrace.tex}}%
      \onslide*<6>{\providecommand\step{2}\input{diagrams/backtrace/backtrace.tex}}%
    \end{column}
  \end{columns}
\end{frame}

\newcommand\listStashBacktrace[1][]{
  \listc[firstline=91,lastline=120,#1]{../ocaml/runtime/backtrace\_nat.c}{runtime/backtrace\_nat.c:91}}

\begin{frame}{Backtraces}{Introducing \funcname{caml\_stash\_backtrace}}
  \begin{columns}[c]
    \begin{column}{0.5\textwidth}
      \minipage[c][0.66\textheight][s]{\columnwidth}
      \begin{itemize}
        \item<1-> A C function provided by the runtime (specific to native code)
        \item<2-> It scans the stack, between the stack pointer at the raising point up to the next trap, in order to collect the names of the detected function frames
          \onslide*<2>{
            \begin{itemize}
              \item And more information, like line number, column number...
            \end{itemize}
          }
        \item<3-> It uses the same technique and information as the Garbage Collector (GC) uses to scan the values live on the stack
          \onslide*<4>{
            \begin{itemize}
              \item \typename{frame\_descr} are at the core of stack unwinding
            \end{itemize}
          }
      \end{itemize}
      \vfill
      \mintocaml[firstline=9,lastline=13,highlightlines=12]{../src/backtrace/backtrace.ml}
      \endminipage
    \end{column}
    \begin{column}{0.5\textwidth}
      \onslide*<1>{\listStashBacktrace[highlightlines=91]{}}
      \onslide*<2>{\listStashBacktrace[highlightlines={91,117-118}]{}}
      \onslide*<3->{\listStashBacktrace[highlightlines={94,106,109-110}]{}}
    \end{column}
  \end{columns}
\end{frame}

\newcommand\listFrameDescr[1][]{
  \listocaml[firstline=111,lastline=124,#1]{../ocaml/asmcomp/emitaux.ml}{asmcomp/emitaux.ml:111}}
\newcommand\listDebugInfo[1][]{
  \listocaml[firstline=38,lastline=49,#1]{../ocaml/lambda/debuginfo.mli}{lambda/debuginfo.mli:38}}

\begin{frame}{Backtraces}{\typename{frame\_descr} and \typename{frame\_debuginfo} in a nutshell}
  \begin{columns}[c]
    \begin{column}{0.5\textwidth}
      \begin{itemize}
        \item<1-> For every return address in an OCaml program, there is a associated \typename{frame\_descr}
        \item<2-> A \typename{frame\_descr} specifies
        \begin{itemize}
          \item<2-> The size of the function stack frame
          \item<3-> Where live values are on the stack (so that the GC can mark them as live and prevent from being swept)
        \end{itemize}
        \item<4-> When compiled with debug information, \typename{frame\_descr} will be linked to a \typename{frame\_debuginfo}
        \begin{itemize}
          \item<5-> Contains information about the position in the source code (file name, line and column number)
        \end{itemize}
      \end{itemize}
    \end{column}
    \begin{column}{0.5\textwidth}
        \onslide*<1>{\listFrameDescr[highlightlines={118-122}]{}}
        \onslide*<2>{\listFrameDescr[highlightlines={120}]{}}
        \onslide*<3>{\listFrameDescr[highlightlines={121}]{}}
        \onslide*<4->{\listFrameDescr[highlightlines={122}]{}}
        \medskip
        \onslide*<1-3>{\listDebugInfo{}}
        \onslide*<4>{\listDebugInfo[highlightlines={38,49}]{}}
        \onslide*<5>{\listDebugInfo[highlightlines={39-46}]{}}
    \end{column}
  \end{columns}
\end{frame}

\begin{frame}{Backtraces}{Scanning the stack with \typename{frame\_descr}}
  \begin{columns}[c]
    \begin{column}{0.5\textwidth}
      \begin{itemize}
        \item Given the return address of the code that raises the exception by calling \funcname{caml\_raise\_exn} (\funcarg{pc} argument of \funcname{caml\_stash\_backtrace})
        \item And the position of the stack pointer (\funcarg{sp} argument of \funcname{caml\_stash\_backtrace})
        \item The frame size given by \typename{frame\_descr}.\funcarg{frame\_size}, allows to find the end of the previous frame
        \item And the return address of the caller of this function
        \bigskip
        \onslide<3->{
        \item Note that traps (here the reddish \funcname{ExnA}, \funcname{ExnB} and \funcname{ExnC} blocks) belong to the frame that installed them, and so the frame size changes when traps are removed
        }
      \end{itemize}
    \end{column}
    \begin{column}{0.5\textwidth}
      \raggedleft
      \onslide*<1>{\providecommand\step{2}\input{diagrams/backtrace/backtrace.tex}}%
      \onslide*<2>{\providecommand\step{1}\input{diagrams/backtrace/scanstack.tex}}%
      \onslide*<3>{\providecommand\step{2}\input{diagrams/backtrace/scanstack.tex}}%
      \onslide*<4>{\providecommand\step{3}\input{diagrams/backtrace/scanstack.tex}}%
      \onslide*<5>{\providecommand\step{4}\input{diagrams/backtrace/scanstack.tex}}%
      \onslide*<6>{\providecommand\step{5}\input{diagrams/backtrace/scanstack.tex}}%
    \end{column}
  \end{columns}
\end{frame}

% Duplicate of previous-previous slide
\begin{frame}{Backtraces}{Continuing on \funcname{caml\_stash\_backtrace}}
  \begin{columns}[c]
    \begin{column}{0.5\textwidth}
      \minipage[c][0.75\textheight][s]{\columnwidth}
      \begin{itemize}
          \color{gray}
        \item A C function provided by the runtime (specific to native code)
        \item It scans the stack, between the stack pointer at the raising point up to the next trap, in order to collect the names of the detected function frames
        \item It uses the same technique and information as the Garbage Collector (GC) uses to scan the values live on the stack
      \end{itemize}
      \begin{itemize}
        \item For each function frame detected, store a pointer to the \typename{frame\_descr} in the \localname{domain\_state->backtrace\_buffer} array, effectively constructing the backtrace
      \end{itemize}
      \onslide<2->{
        \begin{itemize}
            \smallskip
          \item Constructed backtrace:
            \funcname{definitely\_raise},
            \onslide<3->{\funcname{probably\_raise}}
        \end{itemize}
      }
      \vfill
      \mintocaml[firstline=9,lastline=13,highlightlines=12]{../src/backtrace/backtrace.ml}
      \endminipage
    \end{column}
    \begin{column}{0.5\textwidth}
      \raggedleft
      \onslide*<1>{\listStashBacktrace[highlightlines={114-115}]{}}
      \onslide*<2>{\providecommand\step{3}\input{diagrams/backtrace/backtrace.tex}}%
      \onslide*<3>{\providecommand\step{4}\input{diagrams/backtrace/backtrace.tex}}%
    \end{column}
  \end{columns}
\end{frame}

\begin{frame}{Backtraces}{Back to \funcname{caml\_raise\_exn}}
  \begin{columns}[c]
    \begin{column}{0.5\textwidth}
      \minipage[c][0.66\textheight][s]{\columnwidth}
      \begin{itemize}\color{gray}
        \item Checks wheteher exceptions backtrace should be recorded, by checking the \funcarg{domain\_state->backtrace\_active} value
        \item Switches to the C stack to be able to execute C code
        \item Calls \funcname{caml\_stash\_backtrace}, to collect the backtrace up to the next trap
      \end{itemize}
      \onslide*<2->{
        \begin{itemize}
          \item<2-> Executes the exception handler in \funcname{probably\_raise} with \funcname{RESTORE\_EXN\_HANDLER\_OCAML}
            \begin{itemize}
              \item Set \regname{RSP} to \localname{domain\_state->exn\_handler} value
              \item Restore \localname{domain\_state->exn\_handler} from trap
              \item Jump to the handler code with \funcname{ret}
            \end{itemize}
        \end{itemize}
      }
      \vfill
      \onslide*<1>{\mintocaml[firstline=9,lastline=13,highlightlines=12]{../src/backtrace/backtrace.ml}}
      \onslide*<2->{\mintocaml[firstline=15,lastline=21,highlightlines={19-21}]{../src/backtrace/backtrace.ml}}
      \endminipage
    \end{column}
    \begin{column}{0.5\textwidth}
      \centering
      \onslide*<1>{
        \mintc[firstline=190,lastline=192]{../ocaml/runtime/amd64.S}
        \mintc[firstline=234,lastline=237]{../ocaml/runtime/amd64.S}
      }
      \onslide*<2>{
        \mintc[firstline=190,lastline=192,highlightlines={191-192}]{../ocaml/runtime/amd64.S}
        \mintc[firstline=234,lastline=237,highlightlines={235-237}]{../ocaml/runtime/amd64.S}
      }
      \onslide*<1>{\listCamlRaiseExn[highlightlines=720]{}}
      \onslide*<2>{\listCamlRaiseExn[highlightlines=722]{}}
      \onslide*<3->{\providecommand\step{5}\input{diagrams/backtrace/backtrace.tex}}
    \end{column}
  \end{columns}
\end{frame}

\begin{frame}{Backtraces}{\funcname{caml\_probably\_raise}'s handler doesn't catch \typename{ExnC}}
  \begin{columns}[c]
    \begin{column}{0.45\textwidth}
      \minipage[c][0.75\textheight][s]{\columnwidth}
      \begin{itemize}
        \item<1-> \funcname{match \dots with}\footnotemark pattern matching can also match exceptions
        \item<1-> The CMM of \funcname{probably\_raise} shows that this \funcname{match \dots with} is implemented with a pattern matching and a \funcname{try \dots with}
        \item<1-> But this one only matches on \typename{ExnA} exception
        \item<2-> Since the pattern matching fails to catch the exception, \funcname{reraise} is called with our current \typename{ExnC} exception
        \item<3-> Disassembly shows that \funcname{reraise} is actually a call to \funcname{caml\_reraise\_exn}, which is also an assembly function provided by the runtime
      \end{itemize}
      \vfill
      \mintocaml[firstline=15,lastline=21,highlightlines={18-21}]{../src/backtrace/backtrace.ml}
      \endminipage
    \end{column}
    \begin{column}{0.55\textwidth}
      \centering
      \onslide*<1>{\mintcmm[highlightlines={10-18}]{../src/backtrace/camlBacktrace\_\_probably\_raise\_351.cmm}}
      \onslide*<2>{\listcmm[highlightlines=18]{../src/backtrace/camlBacktrace\_\_probably\_raise_351.cmm}{CMM of probably\_raise}}
      \onslide*<3>{\mintobjdump[firstline=5,lastline=35,highlightlines=31]{../src/backtrace/camlBacktrace\_\_probably\_raise_351.objdump}}
    \end{column}
  \end{columns}
  \only<1>{\footnotetext[1]{https://blog.janestreet.com/pattern-matching-and-exception-handling-unite/}}
\end{frame}

\newcommand\listCamlReraiseExn[1][]{
  \listasm[firstline=727,lastline=734,#1]{../ocaml/runtime/amd64.S}{runtime/amd64.S:727}}

\begin{frame}{Backtraces}{Introducing \funcname{caml\_reraise\_exn}}
  \begin{columns}[c]
    \begin{column}{0.5\textwidth}
      \minipage[c][0.66\textheight][s]{\columnwidth}
      \begin{itemize}
        \item<1-> The exception have to be reraised to the next handler
        \item<2-> First, checks if the backtrace should be collected with \funcarg{domain\_state->backtrace\_active}
        \only<3>{
        \begin{itemize}
            \item If not, behaves like a \funcname{raise\_notrace} with \funcname{RESTORE\_EXN\_HANDLER\_OCAML}
        \end{itemize}
        }
        \item<4-> Jumps into \funcname{caml\_raise\_exn}
        \item<5-> Calls \funcname{caml\_stash\_backtrace} again, to scan the next part of the stack: from the trap just pop-ed to the next trap
      \end{itemize}
      \vfill
      \mintocaml[firstline=15,lastline=21,highlightlines={18-21}]{../src/backtrace/backtrace.ml}
      \endminipage
    \end{column}
    \begin{column}{0.5\textwidth}
      \raggedleft
      \onslide*<1>{\listCamlReraiseExn{}}
      \onslide*<2>{\listCamlReraiseExn[highlightlines=729]{}}
      \onslide*<3>{
        \mintc[firstline=190,lastline=192,highlightlines={191-192}]{../ocaml/runtime/amd64.S}
        \mintc[firstline=234,lastline=237,highlightlines={235-237}]{../ocaml/runtime/amd64.S}
        \medskip
        \listCamlReraiseExn[highlightlines=731]{}
      }
      \onslide*<4-5>{
        % TODO refactor with \listCamlReraiseExn
        \mintasm[firstline=727,lastline=734,highlightlines=730]{../ocaml/runtime/amd64.S}
        \medskip
      }
      \onslide*<4>{\listCamlRaiseExn[highlightlines=711]{}}
      \onslide*<5>{\listCamlRaiseExn[highlightlines=720]{}}
    \end{column}
  \end{columns}
\end{frame}

\begin{frame}{Backtraces}{Backtrace collection continues}
  \begin{columns}[c]
    \begin{column}{0.55\textwidth}
      \minipage[c][0.75\textheight][s]{\columnwidth}
      \begin{itemize}
        \item<1-> \funcname{caml\_stash\_backtrace} scans the older part of the stack, up to the next trap
        \item<6-> Controls jumps to the nested exception handler in \funcname{maybe\_raise}, which matches only on \typename{ExnB}
        \item<7-> \funcname{caml\_reraise\_exn} is called again, backtrace collection resumes with \funcname{caml\_stash\_backtrace}
      \end{itemize}
      \medskip
      \begin{itemize}
        \item<2-> Constructed backtrace:
        \begin{itemize}
          \item <2-> \funcname{definitely\_raise}
          \item <2-> \funcname{probably\_raise}
          \item <3-> \funcname{probably\_raise} (re-raise)
          \item <4-> \funcname{maybe\_raise}
          \item <5-> \funcname{main}
          \item <8-> \funcname{main} (re-raise)
        \end{itemize}
      \end{itemize}
      \vfill
      \onslide*<1-5>{\mintocaml[firstline=15,lastline=21]{../src/backtrace/backtrace.ml}}
      \onslide*<6>{\mintocaml[firstline=28,lastline=34,highlightlines=31]{../src/backtrace/backtrace.ml}}
      \onslide*<7->{\mintocaml[firstline=28,lastline=34,highlightlines=33]{../src/backtrace/backtrace.ml}}
      \endminipage
    \end{column}
    \begin{column}{0.45\textwidth}
      \raggedleft
      \onslide*<1>{\providecommand\step{6}\input{diagrams/backtrace/backtrace.tex}}%
      \onslide*<2>{\providecommand\step{6}\input{diagrams/backtrace/backtrace.tex}}%
      \onslide*<3>{\providecommand\step{7}\input{diagrams/backtrace/backtrace.tex}}%
      \onslide*<4>{\providecommand\step{8}\input{diagrams/backtrace/backtrace.tex}}%
      \onslide*<5>{\providecommand\step{9}\input{diagrams/backtrace/backtrace.tex}}%
      \onslide*<6>{\providecommand\step{10}\input{diagrams/backtrace/backtrace.tex}}%
      \onslide*<7>{\providecommand\step{11}\input{diagrams/backtrace/backtrace.tex}}%
      \onslide*<8>{\providecommand\step{12}\input{diagrams/backtrace/backtrace.tex}}%
      \onslide*<9>{\providecommand\step{13}\input{diagrams/backtrace/backtrace.tex}}%
    \end{column}
  \end{columns}
\end{frame}

\begin{frame}{Backtraces}{Printing the backtrace}
  \begin{columns}[c]
    \begin{column}{0.45\textwidth}
      \minipage[c][0.66\textheight][s]{\columnwidth}
      \begin{itemize}
        \item<1-> The exception backtrace can now be printed to \funcarg{stdout} with \funcname{Printexc.print_backtrace}
        \item<2-> First the exception backtrace is retrieved into an OCaml array with \funcname{get_raw_backtrace }
        \item<3-> Which is implemented as the \funcname{caml_get_exception_raw_backtrace} C function in the runtime
        \item<4-> And finally printed with \funcname{print_exception_backtrace}
      \end{itemize}
      \vfill
      \mintocaml[firstline=28,lastline=34,highlightlines=33]{../src/backtrace/backtrace.ml}
      \endminipage
    \end{column}
    \begin{column}{0.55\textwidth}
      \onslide*<1,2>{
        \mintocaml[firstline=100,lastline=101]{../ocaml/stdlib/printexc.ml}
      }
      \only<1>{
        \listocaml[firstline=170,lastline=172]{../ocaml/stdlib/printexc.ml}{stdlib/printexc.ml:170}
      }
      \only<2>{
        \listocaml[firstline=170,lastline=172,highlightlines=172]{../ocaml/stdlib/printexc.ml}{stdlib/printexc.ml:170}
      }
      \only<3>{
        \mintc[firstline=154,lastline=156]{../ocaml/runtime/backtrace.c}
        \listc[firstline=171,lastline=192,highlightlines={184-187}]{../ocaml/runtime/backtrace.c}{runtime/backtrace.c:154}
      }
      \only<4>{
        \listocaml[firstline=155,lastline=165]{../ocaml/stdlib/printexc.ml}{stdlib/printexc.ml:55}
      }
    \end{column}
  \end{columns}
\end{frame}


\subsection{The multiple kinds of raise}
\frameSubsection{}{}

\begin{frame}{Backtraces}{When backtraces are not needed}
  \begin{columns}[c]
    \begin{column}{0.45\textwidth}
      \minipage[c][0.66\textheight][s]{\columnwidth}
      \begin{itemize}
        \item<1-> What if the backtrace for an exception is never needed?
        \item<2-> It's possible to raise the exception explicitly with \funcname{raise\_notrace} to make raising as cheap as possible
        \item<3-> An example of use in the \funcname{dynlink} library of the OCaml compiler
      \end{itemize}
      \vfill
      \onslide<3->{
      \listocaml[firstline=205,lastline=209,highlightlines=207]{../ocaml/otherlibs/dynlink/dynlink\_compilerlibs/misc.ml}{otherlibs/dynlink/dynlink\_compilerlibs/misc.ml:205}
      }
      \endminipage
    \end{column}
    \begin{column}{0.55\textwidth}
    \only<4>{
      \mintcmm[highlightlines=3]{../src/notrace/camlNotrace\_\_fun_347.cmm}
      \listcmm{../src/notrace/camlNotrace\_\_all\_somes\_327.cmm}{all\_somes}
    }
    \onslide*<5->{
      \listobjdump[highlightlines={6-9}]{../src/notrace/camlNotrace\_\_fun_347.objdump}{anonymous function from \funcname{all\_somes}}
    }
    \end{column}
  \end{columns}
\end{frame}

\begin{frame}{Backtraces}{Summary table of the multiple kinds of raise}
  \begin{itemize}
    \item When debugging information is enabled and if the backtrace of an exception isn't needed, it's possible to raise with \funcname{raise\_notrace} for performance
    \item When debugging information is \emph{not} enabled, the compiler optimises \funcname{raise} calls to inline assembly (same effect as using \funcname{raise\_notrace})
  \end{itemize}
  \bigskip
  \centering
  \begin{tabular}{| l | c | c | c | c |}
    \hline
    \parbox{10em}{Debug information \\ (\funcarg{-g} flag)}  & \multicolumn{2}{|c|}{Disabled}                                     & \multicolumn{2}{|c|}{Enabled} \\
    \hline
    Raise kind                                               & \funcname{raise} & \funcname{raise\_notrace}                       & \funcname{raise} & \funcname{raise\_notrace} \\
    \hline
    Compilation                                              & \parbox{8em}{inline assembly\\ (optimization)} & inline assembly   & \funcname{caml\_raise\_exn} & inline assembly \\
    \hline
  \end{tabular}
\end{frame}


%
% Takeaway #5
%
\subsection*{Takeaway \#5}
\frameSubsectionTakeaway{}
\againframe<5>{takeaway}
}%
      \onslide*<4>{\providecommand\step{8}%
% Backtraces
%
\subsection{One advantage of raising with debugging information}
\frameSubsection{}{}

\begin{frame}{Backtraces}{Debugging information \& source code}
  \begin{columns}[c]
    \begin{column}{0.5\textwidth}
      \begin{itemize}
        \item Raising an exception is fun and games, but how do we know where the exception originated? $\rightarrow$ With the backtrace associated with the exception!
        \item In order to get a backtrace from the point where an exception was raised, it's necessary to compile with debugging information enabled (\funcarg{-g} flag for the compiler)
        \item Same example as in the `Nested handlers' section
      \end{itemize}
    \end{column}
    \begin{column}{0.5\textwidth}
      \listocaml[firstline=9,lastline=34]{../src/backtrace/backtrace.ml}{backtrace/backtrace.ml}
    \end{column}
  \end{columns}
\end{frame}

\begin{frame}{Backtraces}{CMM and disassembly of \funcname{definitely\_raise}}
  \begin{columns}[c]
    \begin{column}{0.5\textwidth}
      \minipage[c][0.66\textheight][s]{\columnwidth}
      \begin{itemize}
        \item<1-> An effect of compiling with debugging information (\funcarg{-g}): the CMM no longer uses \funcname{raise\_notrace} to raise the exception but \funcname{raise} instead
        \item<2-> Disassembly shows that CMM \funcname{raise} is actually a call to \funcname{caml\_raise\_exn}, a function in assembler provided by the runtime
      \end{itemize}
      \vfill
      \mintocaml[firstline=9,lastline=13,highlightlines=12]{../src/backtrace/backtrace.ml}
      \endminipage
    \end{column}
    \begin{column}{0.5\textwidth}
      \onslide*<1>{
        \mintcmm[highlightlines=5]{../src/backtrace/camlBacktrace\_\_definitely\_raise\_347.cmm}
      }
      \onslide*<2>{
        \mintobjdump[firstline=2,highlightlines=14]{../src/backtrace/camlBacktrace\_\_definitely\_raise\_347.objdump}
      }
    \end{column}
  \end{columns}
\end{frame}

\begin{frame}{Backtraces}{Introducing \funcname{caml\_raise\_exn}}
  \begin{columns}[c]
    \begin{column}{0.5\textwidth}
      \minipage[c][0.66\textheight][s]{\columnwidth}
      \onslide*<1>{
        \begin{itemize}
          \item Raising the exception is performed using \funcname{caml\_raise\_exn} which is
            \begin{itemize}
              \item An assembly function provided by the runtime
              \item Architecture specific
            \end{itemize}
          \item Let's see what it does differently from \funcname{raise\_notrace}\dots
          \item We are about to raise \typename{ExnC} with \funcname{caml\_raise\_exn} from \funcname{definitely\_raise}
        \end{itemize}
      }
      \onslide*<2>{
        \mintobjdump[firstline=2,highlightlines=14]{../src/backtrace/camlBacktrace\_\_definitely\_raise\_347.objdump}
      }
      \vfill
      \mintocaml[firstline=9,lastline=13,highlightlines=12]{../src/backtrace/backtrace.ml}
      \endminipage
    \end{column}
    \begin{column}{0.5\textwidth}
      \centering
      \providecommand\step{1}\input{diagrams/backtrace/backtrace.tex}
    \end{column}
  \end{columns}
\end{frame}

\begin{frame}{Backtraces}{But before that, we need more fields from \typename{caml\_domain\_state}}
  \begin{columns}
    \begin{column}{0.5\textwidth}
      \begin{itemize}
        \item Remember \typename{caml\_domain\_state} the per-domain runtime state?
        \item Some other members are of interest to us now\dots
        \item \localname{backtrace\_active} is used to know if the runtime should collect backtraces
        \item \localname{backtrace\_active} can be set by
          \begin{itemize}
            \item The \funcarg{b} flag in the \funcname{OCAMLRUNPARAM} environment variable
            \item \mintinline{ocaml}{Printexc.record_backtrace : bool -> unit} \footnotemark
          \end{itemize}
      \end{itemize}
    \end{column}
    \begin{column}{0.5\textwidth}
      \begin{listing}
        \mintc[highlightlines={9-11}]{src/ocaml/domain\_state\_backtrace.h}
        \caption{runtime/caml/domain\_state.h}
      \end{listing}
    \end{column}
  \end{columns}
  \footnotetext[1]{https://v2.ocaml.org/api/Printexc.html}
\end{frame}

\newcommand\listCamlRaiseExn[1][]{
  \listasm[firstline=702,lastline=725,#1]{../ocaml/runtime/amd64.S}{runtime/amd64.S:702}}

\begin{frame}{Backtraces}{Walk through \funcname{caml\_raise\_exn}}
  \begin{columns}[T]
    \begin{column}{0.5\textwidth}
      \begin{itemize}
        \item<1-> First checks whether exceptions backtrace should be recorded
        \item<1-> By checking the \funcarg{domain\_state->backtrace\_active} value
        \item<2-> If not, acts as \funcname{raise\_notrace} with \funcname{RESTORE\_EXN\_HANDLER\_OCAML}
          \begin{itemize}
            \item Set \regname{RSP} to \localname{domain\_state->exn\_handler} value
            \item Restore \localname{domain\_state->exn\_handler} from trap
            \item Jump with \funcname{ret} (same effect as using \funcname{pop} and \funcname{jmp})
          \end{itemize}
        \item<3-> Otherwise, switches to the C stack to be able to execute C code
        \item<4-> Calls \funcname{caml\_stash\_backtrace}, a C function provided by the runtime\ignorespaces
          \onslide<6->{, with
            \begin{itemize}
              \item \funcarg{exn}: exception value
              \item \funcarg{pc}: code address of the raising point
              \item \funcarg{sp}: stack address of the raising function
              \item \funcarg{trapsp}: stack address of the next handler
            \end{itemize}
          }
      \end{itemize}
      \bigskip
      \mintocaml[firstline=9,lastline=13,highlightlines=12]{../src/backtrace/backtrace.ml}
    \end{column}
    \begin{column}{0.5\textwidth}
      \raggedleft
      \onslide*<1,3-4>{
        \mintc[firstline=190,lastline=192]{../ocaml/runtime/amd64.S}
        \mintc[firstline=234,lastline=237]{../ocaml/runtime/amd64.S}
      }
      \onslide*<2>{
        \mintc[firstline=190,lastline=192,highlightlines={191-192}]{../ocaml/runtime/amd64.S}
        \mintc[firstline=234,lastline=237,highlightlines={235-237}]{../ocaml/runtime/amd64.S}
      }
      \onslide*<1>{\listCamlRaiseExn[highlightlines=705]{}}%
      \onslide*<2>{\listCamlRaiseExn[highlightlines={707-708}]{}}%
      \onslide*<3>{\listCamlRaiseExn[highlightlines={712-714}]{}}%
      \onslide*<4>{\listCamlRaiseExn[highlightlines={715-720}]{}}%
      \onslide*<5>{\providecommand\step{1}\input{diagrams/backtrace/backtrace.tex}}%
      \onslide*<6>{\providecommand\step{2}\input{diagrams/backtrace/backtrace.tex}}%
    \end{column}
  \end{columns}
\end{frame}

\newcommand\listStashBacktrace[1][]{
  \listc[firstline=91,lastline=120,#1]{../ocaml/runtime/backtrace\_nat.c}{runtime/backtrace\_nat.c:91}}

\begin{frame}{Backtraces}{Introducing \funcname{caml\_stash\_backtrace}}
  \begin{columns}[c]
    \begin{column}{0.5\textwidth}
      \minipage[c][0.66\textheight][s]{\columnwidth}
      \begin{itemize}
        \item<1-> A C function provided by the runtime (specific to native code)
        \item<2-> It scans the stack, between the stack pointer at the raising point up to the next trap, in order to collect the names of the detected function frames
          \onslide*<2>{
            \begin{itemize}
              \item And more information, like line number, column number...
            \end{itemize}
          }
        \item<3-> It uses the same technique and information as the Garbage Collector (GC) uses to scan the values live on the stack
          \onslide*<4>{
            \begin{itemize}
              \item \typename{frame\_descr} are at the core of stack unwinding
            \end{itemize}
          }
      \end{itemize}
      \vfill
      \mintocaml[firstline=9,lastline=13,highlightlines=12]{../src/backtrace/backtrace.ml}
      \endminipage
    \end{column}
    \begin{column}{0.5\textwidth}
      \onslide*<1>{\listStashBacktrace[highlightlines=91]{}}
      \onslide*<2>{\listStashBacktrace[highlightlines={91,117-118}]{}}
      \onslide*<3->{\listStashBacktrace[highlightlines={94,106,109-110}]{}}
    \end{column}
  \end{columns}
\end{frame}

\newcommand\listFrameDescr[1][]{
  \listocaml[firstline=111,lastline=124,#1]{../ocaml/asmcomp/emitaux.ml}{asmcomp/emitaux.ml:111}}
\newcommand\listDebugInfo[1][]{
  \listocaml[firstline=38,lastline=49,#1]{../ocaml/lambda/debuginfo.mli}{lambda/debuginfo.mli:38}}

\begin{frame}{Backtraces}{\typename{frame\_descr} and \typename{frame\_debuginfo} in a nutshell}
  \begin{columns}[c]
    \begin{column}{0.5\textwidth}
      \begin{itemize}
        \item<1-> For every return address in an OCaml program, there is a associated \typename{frame\_descr}
        \item<2-> A \typename{frame\_descr} specifies
        \begin{itemize}
          \item<2-> The size of the function stack frame
          \item<3-> Where live values are on the stack (so that the GC can mark them as live and prevent from being swept)
        \end{itemize}
        \item<4-> When compiled with debug information, \typename{frame\_descr} will be linked to a \typename{frame\_debuginfo}
        \begin{itemize}
          \item<5-> Contains information about the position in the source code (file name, line and column number)
        \end{itemize}
      \end{itemize}
    \end{column}
    \begin{column}{0.5\textwidth}
        \onslide*<1>{\listFrameDescr[highlightlines={118-122}]{}}
        \onslide*<2>{\listFrameDescr[highlightlines={120}]{}}
        \onslide*<3>{\listFrameDescr[highlightlines={121}]{}}
        \onslide*<4->{\listFrameDescr[highlightlines={122}]{}}
        \medskip
        \onslide*<1-3>{\listDebugInfo{}}
        \onslide*<4>{\listDebugInfo[highlightlines={38,49}]{}}
        \onslide*<5>{\listDebugInfo[highlightlines={39-46}]{}}
    \end{column}
  \end{columns}
\end{frame}

\begin{frame}{Backtraces}{Scanning the stack with \typename{frame\_descr}}
  \begin{columns}[c]
    \begin{column}{0.5\textwidth}
      \begin{itemize}
        \item Given the return address of the code that raises the exception by calling \funcname{caml\_raise\_exn} (\funcarg{pc} argument of \funcname{caml\_stash\_backtrace})
        \item And the position of the stack pointer (\funcarg{sp} argument of \funcname{caml\_stash\_backtrace})
        \item The frame size given by \typename{frame\_descr}.\funcarg{frame\_size}, allows to find the end of the previous frame
        \item And the return address of the caller of this function
        \bigskip
        \onslide<3->{
        \item Note that traps (here the reddish \funcname{ExnA}, \funcname{ExnB} and \funcname{ExnC} blocks) belong to the frame that installed them, and so the frame size changes when traps are removed
        }
      \end{itemize}
    \end{column}
    \begin{column}{0.5\textwidth}
      \raggedleft
      \onslide*<1>{\providecommand\step{2}\input{diagrams/backtrace/backtrace.tex}}%
      \onslide*<2>{\providecommand\step{1}\input{diagrams/backtrace/scanstack.tex}}%
      \onslide*<3>{\providecommand\step{2}\input{diagrams/backtrace/scanstack.tex}}%
      \onslide*<4>{\providecommand\step{3}\input{diagrams/backtrace/scanstack.tex}}%
      \onslide*<5>{\providecommand\step{4}\input{diagrams/backtrace/scanstack.tex}}%
      \onslide*<6>{\providecommand\step{5}\input{diagrams/backtrace/scanstack.tex}}%
    \end{column}
  \end{columns}
\end{frame}

% Duplicate of previous-previous slide
\begin{frame}{Backtraces}{Continuing on \funcname{caml\_stash\_backtrace}}
  \begin{columns}[c]
    \begin{column}{0.5\textwidth}
      \minipage[c][0.75\textheight][s]{\columnwidth}
      \begin{itemize}
          \color{gray}
        \item A C function provided by the runtime (specific to native code)
        \item It scans the stack, between the stack pointer at the raising point up to the next trap, in order to collect the names of the detected function frames
        \item It uses the same technique and information as the Garbage Collector (GC) uses to scan the values live on the stack
      \end{itemize}
      \begin{itemize}
        \item For each function frame detected, store a pointer to the \typename{frame\_descr} in the \localname{domain\_state->backtrace\_buffer} array, effectively constructing the backtrace
      \end{itemize}
      \onslide<2->{
        \begin{itemize}
            \smallskip
          \item Constructed backtrace:
            \funcname{definitely\_raise},
            \onslide<3->{\funcname{probably\_raise}}
        \end{itemize}
      }
      \vfill
      \mintocaml[firstline=9,lastline=13,highlightlines=12]{../src/backtrace/backtrace.ml}
      \endminipage
    \end{column}
    \begin{column}{0.5\textwidth}
      \raggedleft
      \onslide*<1>{\listStashBacktrace[highlightlines={114-115}]{}}
      \onslide*<2>{\providecommand\step{3}\input{diagrams/backtrace/backtrace.tex}}%
      \onslide*<3>{\providecommand\step{4}\input{diagrams/backtrace/backtrace.tex}}%
    \end{column}
  \end{columns}
\end{frame}

\begin{frame}{Backtraces}{Back to \funcname{caml\_raise\_exn}}
  \begin{columns}[c]
    \begin{column}{0.5\textwidth}
      \minipage[c][0.66\textheight][s]{\columnwidth}
      \begin{itemize}\color{gray}
        \item Checks wheteher exceptions backtrace should be recorded, by checking the \funcarg{domain\_state->backtrace\_active} value
        \item Switches to the C stack to be able to execute C code
        \item Calls \funcname{caml\_stash\_backtrace}, to collect the backtrace up to the next trap
      \end{itemize}
      \onslide*<2->{
        \begin{itemize}
          \item<2-> Executes the exception handler in \funcname{probably\_raise} with \funcname{RESTORE\_EXN\_HANDLER\_OCAML}
            \begin{itemize}
              \item Set \regname{RSP} to \localname{domain\_state->exn\_handler} value
              \item Restore \localname{domain\_state->exn\_handler} from trap
              \item Jump to the handler code with \funcname{ret}
            \end{itemize}
        \end{itemize}
      }
      \vfill
      \onslide*<1>{\mintocaml[firstline=9,lastline=13,highlightlines=12]{../src/backtrace/backtrace.ml}}
      \onslide*<2->{\mintocaml[firstline=15,lastline=21,highlightlines={19-21}]{../src/backtrace/backtrace.ml}}
      \endminipage
    \end{column}
    \begin{column}{0.5\textwidth}
      \centering
      \onslide*<1>{
        \mintc[firstline=190,lastline=192]{../ocaml/runtime/amd64.S}
        \mintc[firstline=234,lastline=237]{../ocaml/runtime/amd64.S}
      }
      \onslide*<2>{
        \mintc[firstline=190,lastline=192,highlightlines={191-192}]{../ocaml/runtime/amd64.S}
        \mintc[firstline=234,lastline=237,highlightlines={235-237}]{../ocaml/runtime/amd64.S}
      }
      \onslide*<1>{\listCamlRaiseExn[highlightlines=720]{}}
      \onslide*<2>{\listCamlRaiseExn[highlightlines=722]{}}
      \onslide*<3->{\providecommand\step{5}\input{diagrams/backtrace/backtrace.tex}}
    \end{column}
  \end{columns}
\end{frame}

\begin{frame}{Backtraces}{\funcname{caml\_probably\_raise}'s handler doesn't catch \typename{ExnC}}
  \begin{columns}[c]
    \begin{column}{0.45\textwidth}
      \minipage[c][0.75\textheight][s]{\columnwidth}
      \begin{itemize}
        \item<1-> \funcname{match \dots with}\footnotemark pattern matching can also match exceptions
        \item<1-> The CMM of \funcname{probably\_raise} shows that this \funcname{match \dots with} is implemented with a pattern matching and a \funcname{try \dots with}
        \item<1-> But this one only matches on \typename{ExnA} exception
        \item<2-> Since the pattern matching fails to catch the exception, \funcname{reraise} is called with our current \typename{ExnC} exception
        \item<3-> Disassembly shows that \funcname{reraise} is actually a call to \funcname{caml\_reraise\_exn}, which is also an assembly function provided by the runtime
      \end{itemize}
      \vfill
      \mintocaml[firstline=15,lastline=21,highlightlines={18-21}]{../src/backtrace/backtrace.ml}
      \endminipage
    \end{column}
    \begin{column}{0.55\textwidth}
      \centering
      \onslide*<1>{\mintcmm[highlightlines={10-18}]{../src/backtrace/camlBacktrace\_\_probably\_raise\_351.cmm}}
      \onslide*<2>{\listcmm[highlightlines=18]{../src/backtrace/camlBacktrace\_\_probably\_raise_351.cmm}{CMM of probably\_raise}}
      \onslide*<3>{\mintobjdump[firstline=5,lastline=35,highlightlines=31]{../src/backtrace/camlBacktrace\_\_probably\_raise_351.objdump}}
    \end{column}
  \end{columns}
  \only<1>{\footnotetext[1]{https://blog.janestreet.com/pattern-matching-and-exception-handling-unite/}}
\end{frame}

\newcommand\listCamlReraiseExn[1][]{
  \listasm[firstline=727,lastline=734,#1]{../ocaml/runtime/amd64.S}{runtime/amd64.S:727}}

\begin{frame}{Backtraces}{Introducing \funcname{caml\_reraise\_exn}}
  \begin{columns}[c]
    \begin{column}{0.5\textwidth}
      \minipage[c][0.66\textheight][s]{\columnwidth}
      \begin{itemize}
        \item<1-> The exception have to be reraised to the next handler
        \item<2-> First, checks if the backtrace should be collected with \funcarg{domain\_state->backtrace\_active}
        \only<3>{
        \begin{itemize}
            \item If not, behaves like a \funcname{raise\_notrace} with \funcname{RESTORE\_EXN\_HANDLER\_OCAML}
        \end{itemize}
        }
        \item<4-> Jumps into \funcname{caml\_raise\_exn}
        \item<5-> Calls \funcname{caml\_stash\_backtrace} again, to scan the next part of the stack: from the trap just pop-ed to the next trap
      \end{itemize}
      \vfill
      \mintocaml[firstline=15,lastline=21,highlightlines={18-21}]{../src/backtrace/backtrace.ml}
      \endminipage
    \end{column}
    \begin{column}{0.5\textwidth}
      \raggedleft
      \onslide*<1>{\listCamlReraiseExn{}}
      \onslide*<2>{\listCamlReraiseExn[highlightlines=729]{}}
      \onslide*<3>{
        \mintc[firstline=190,lastline=192,highlightlines={191-192}]{../ocaml/runtime/amd64.S}
        \mintc[firstline=234,lastline=237,highlightlines={235-237}]{../ocaml/runtime/amd64.S}
        \medskip
        \listCamlReraiseExn[highlightlines=731]{}
      }
      \onslide*<4-5>{
        % TODO refactor with \listCamlReraiseExn
        \mintasm[firstline=727,lastline=734,highlightlines=730]{../ocaml/runtime/amd64.S}
        \medskip
      }
      \onslide*<4>{\listCamlRaiseExn[highlightlines=711]{}}
      \onslide*<5>{\listCamlRaiseExn[highlightlines=720]{}}
    \end{column}
  \end{columns}
\end{frame}

\begin{frame}{Backtraces}{Backtrace collection continues}
  \begin{columns}[c]
    \begin{column}{0.55\textwidth}
      \minipage[c][0.75\textheight][s]{\columnwidth}
      \begin{itemize}
        \item<1-> \funcname{caml\_stash\_backtrace} scans the older part of the stack, up to the next trap
        \item<6-> Controls jumps to the nested exception handler in \funcname{maybe\_raise}, which matches only on \typename{ExnB}
        \item<7-> \funcname{caml\_reraise\_exn} is called again, backtrace collection resumes with \funcname{caml\_stash\_backtrace}
      \end{itemize}
      \medskip
      \begin{itemize}
        \item<2-> Constructed backtrace:
        \begin{itemize}
          \item <2-> \funcname{definitely\_raise}
          \item <2-> \funcname{probably\_raise}
          \item <3-> \funcname{probably\_raise} (re-raise)
          \item <4-> \funcname{maybe\_raise}
          \item <5-> \funcname{main}
          \item <8-> \funcname{main} (re-raise)
        \end{itemize}
      \end{itemize}
      \vfill
      \onslide*<1-5>{\mintocaml[firstline=15,lastline=21]{../src/backtrace/backtrace.ml}}
      \onslide*<6>{\mintocaml[firstline=28,lastline=34,highlightlines=31]{../src/backtrace/backtrace.ml}}
      \onslide*<7->{\mintocaml[firstline=28,lastline=34,highlightlines=33]{../src/backtrace/backtrace.ml}}
      \endminipage
    \end{column}
    \begin{column}{0.45\textwidth}
      \raggedleft
      \onslide*<1>{\providecommand\step{6}\input{diagrams/backtrace/backtrace.tex}}%
      \onslide*<2>{\providecommand\step{6}\input{diagrams/backtrace/backtrace.tex}}%
      \onslide*<3>{\providecommand\step{7}\input{diagrams/backtrace/backtrace.tex}}%
      \onslide*<4>{\providecommand\step{8}\input{diagrams/backtrace/backtrace.tex}}%
      \onslide*<5>{\providecommand\step{9}\input{diagrams/backtrace/backtrace.tex}}%
      \onslide*<6>{\providecommand\step{10}\input{diagrams/backtrace/backtrace.tex}}%
      \onslide*<7>{\providecommand\step{11}\input{diagrams/backtrace/backtrace.tex}}%
      \onslide*<8>{\providecommand\step{12}\input{diagrams/backtrace/backtrace.tex}}%
      \onslide*<9>{\providecommand\step{13}\input{diagrams/backtrace/backtrace.tex}}%
    \end{column}
  \end{columns}
\end{frame}

\begin{frame}{Backtraces}{Printing the backtrace}
  \begin{columns}[c]
    \begin{column}{0.45\textwidth}
      \minipage[c][0.66\textheight][s]{\columnwidth}
      \begin{itemize}
        \item<1-> The exception backtrace can now be printed to \funcarg{stdout} with \funcname{Printexc.print_backtrace}
        \item<2-> First the exception backtrace is retrieved into an OCaml array with \funcname{get_raw_backtrace }
        \item<3-> Which is implemented as the \funcname{caml_get_exception_raw_backtrace} C function in the runtime
        \item<4-> And finally printed with \funcname{print_exception_backtrace}
      \end{itemize}
      \vfill
      \mintocaml[firstline=28,lastline=34,highlightlines=33]{../src/backtrace/backtrace.ml}
      \endminipage
    \end{column}
    \begin{column}{0.55\textwidth}
      \onslide*<1,2>{
        \mintocaml[firstline=100,lastline=101]{../ocaml/stdlib/printexc.ml}
      }
      \only<1>{
        \listocaml[firstline=170,lastline=172]{../ocaml/stdlib/printexc.ml}{stdlib/printexc.ml:170}
      }
      \only<2>{
        \listocaml[firstline=170,lastline=172,highlightlines=172]{../ocaml/stdlib/printexc.ml}{stdlib/printexc.ml:170}
      }
      \only<3>{
        \mintc[firstline=154,lastline=156]{../ocaml/runtime/backtrace.c}
        \listc[firstline=171,lastline=192,highlightlines={184-187}]{../ocaml/runtime/backtrace.c}{runtime/backtrace.c:154}
      }
      \only<4>{
        \listocaml[firstline=155,lastline=165]{../ocaml/stdlib/printexc.ml}{stdlib/printexc.ml:55}
      }
    \end{column}
  \end{columns}
\end{frame}


\subsection{The multiple kinds of raise}
\frameSubsection{}{}

\begin{frame}{Backtraces}{When backtraces are not needed}
  \begin{columns}[c]
    \begin{column}{0.45\textwidth}
      \minipage[c][0.66\textheight][s]{\columnwidth}
      \begin{itemize}
        \item<1-> What if the backtrace for an exception is never needed?
        \item<2-> It's possible to raise the exception explicitly with \funcname{raise\_notrace} to make raising as cheap as possible
        \item<3-> An example of use in the \funcname{dynlink} library of the OCaml compiler
      \end{itemize}
      \vfill
      \onslide<3->{
      \listocaml[firstline=205,lastline=209,highlightlines=207]{../ocaml/otherlibs/dynlink/dynlink\_compilerlibs/misc.ml}{otherlibs/dynlink/dynlink\_compilerlibs/misc.ml:205}
      }
      \endminipage
    \end{column}
    \begin{column}{0.55\textwidth}
    \only<4>{
      \mintcmm[highlightlines=3]{../src/notrace/camlNotrace\_\_fun_347.cmm}
      \listcmm{../src/notrace/camlNotrace\_\_all\_somes\_327.cmm}{all\_somes}
    }
    \onslide*<5->{
      \listobjdump[highlightlines={6-9}]{../src/notrace/camlNotrace\_\_fun_347.objdump}{anonymous function from \funcname{all\_somes}}
    }
    \end{column}
  \end{columns}
\end{frame}

\begin{frame}{Backtraces}{Summary table of the multiple kinds of raise}
  \begin{itemize}
    \item When debugging information is enabled and if the backtrace of an exception isn't needed, it's possible to raise with \funcname{raise\_notrace} for performance
    \item When debugging information is \emph{not} enabled, the compiler optimises \funcname{raise} calls to inline assembly (same effect as using \funcname{raise\_notrace})
  \end{itemize}
  \bigskip
  \centering
  \begin{tabular}{| l | c | c | c | c |}
    \hline
    \parbox{10em}{Debug information \\ (\funcarg{-g} flag)}  & \multicolumn{2}{|c|}{Disabled}                                     & \multicolumn{2}{|c|}{Enabled} \\
    \hline
    Raise kind                                               & \funcname{raise} & \funcname{raise\_notrace}                       & \funcname{raise} & \funcname{raise\_notrace} \\
    \hline
    Compilation                                              & \parbox{8em}{inline assembly\\ (optimization)} & inline assembly   & \funcname{caml\_raise\_exn} & inline assembly \\
    \hline
  \end{tabular}
\end{frame}


%
% Takeaway #5
%
\subsection*{Takeaway \#5}
\frameSubsectionTakeaway{}
\againframe<5>{takeaway}
}%
      \onslide*<5>{\providecommand\step{9}%
% Backtraces
%
\subsection{One advantage of raising with debugging information}
\frameSubsection{}{}

\begin{frame}{Backtraces}{Debugging information \& source code}
  \begin{columns}[c]
    \begin{column}{0.5\textwidth}
      \begin{itemize}
        \item Raising an exception is fun and games, but how do we know where the exception originated? $\rightarrow$ With the backtrace associated with the exception!
        \item In order to get a backtrace from the point where an exception was raised, it's necessary to compile with debugging information enabled (\funcarg{-g} flag for the compiler)
        \item Same example as in the `Nested handlers' section
      \end{itemize}
    \end{column}
    \begin{column}{0.5\textwidth}
      \listocaml[firstline=9,lastline=34]{../src/backtrace/backtrace.ml}{backtrace/backtrace.ml}
    \end{column}
  \end{columns}
\end{frame}

\begin{frame}{Backtraces}{CMM and disassembly of \funcname{definitely\_raise}}
  \begin{columns}[c]
    \begin{column}{0.5\textwidth}
      \minipage[c][0.66\textheight][s]{\columnwidth}
      \begin{itemize}
        \item<1-> An effect of compiling with debugging information (\funcarg{-g}): the CMM no longer uses \funcname{raise\_notrace} to raise the exception but \funcname{raise} instead
        \item<2-> Disassembly shows that CMM \funcname{raise} is actually a call to \funcname{caml\_raise\_exn}, a function in assembler provided by the runtime
      \end{itemize}
      \vfill
      \mintocaml[firstline=9,lastline=13,highlightlines=12]{../src/backtrace/backtrace.ml}
      \endminipage
    \end{column}
    \begin{column}{0.5\textwidth}
      \onslide*<1>{
        \mintcmm[highlightlines=5]{../src/backtrace/camlBacktrace\_\_definitely\_raise\_347.cmm}
      }
      \onslide*<2>{
        \mintobjdump[firstline=2,highlightlines=14]{../src/backtrace/camlBacktrace\_\_definitely\_raise\_347.objdump}
      }
    \end{column}
  \end{columns}
\end{frame}

\begin{frame}{Backtraces}{Introducing \funcname{caml\_raise\_exn}}
  \begin{columns}[c]
    \begin{column}{0.5\textwidth}
      \minipage[c][0.66\textheight][s]{\columnwidth}
      \onslide*<1>{
        \begin{itemize}
          \item Raising the exception is performed using \funcname{caml\_raise\_exn} which is
            \begin{itemize}
              \item An assembly function provided by the runtime
              \item Architecture specific
            \end{itemize}
          \item Let's see what it does differently from \funcname{raise\_notrace}\dots
          \item We are about to raise \typename{ExnC} with \funcname{caml\_raise\_exn} from \funcname{definitely\_raise}
        \end{itemize}
      }
      \onslide*<2>{
        \mintobjdump[firstline=2,highlightlines=14]{../src/backtrace/camlBacktrace\_\_definitely\_raise\_347.objdump}
      }
      \vfill
      \mintocaml[firstline=9,lastline=13,highlightlines=12]{../src/backtrace/backtrace.ml}
      \endminipage
    \end{column}
    \begin{column}{0.5\textwidth}
      \centering
      \providecommand\step{1}\input{diagrams/backtrace/backtrace.tex}
    \end{column}
  \end{columns}
\end{frame}

\begin{frame}{Backtraces}{But before that, we need more fields from \typename{caml\_domain\_state}}
  \begin{columns}
    \begin{column}{0.5\textwidth}
      \begin{itemize}
        \item Remember \typename{caml\_domain\_state} the per-domain runtime state?
        \item Some other members are of interest to us now\dots
        \item \localname{backtrace\_active} is used to know if the runtime should collect backtraces
        \item \localname{backtrace\_active} can be set by
          \begin{itemize}
            \item The \funcarg{b} flag in the \funcname{OCAMLRUNPARAM} environment variable
            \item \mintinline{ocaml}{Printexc.record_backtrace : bool -> unit} \footnotemark
          \end{itemize}
      \end{itemize}
    \end{column}
    \begin{column}{0.5\textwidth}
      \begin{listing}
        \mintc[highlightlines={9-11}]{src/ocaml/domain\_state\_backtrace.h}
        \caption{runtime/caml/domain\_state.h}
      \end{listing}
    \end{column}
  \end{columns}
  \footnotetext[1]{https://v2.ocaml.org/api/Printexc.html}
\end{frame}

\newcommand\listCamlRaiseExn[1][]{
  \listasm[firstline=702,lastline=725,#1]{../ocaml/runtime/amd64.S}{runtime/amd64.S:702}}

\begin{frame}{Backtraces}{Walk through \funcname{caml\_raise\_exn}}
  \begin{columns}[T]
    \begin{column}{0.5\textwidth}
      \begin{itemize}
        \item<1-> First checks whether exceptions backtrace should be recorded
        \item<1-> By checking the \funcarg{domain\_state->backtrace\_active} value
        \item<2-> If not, acts as \funcname{raise\_notrace} with \funcname{RESTORE\_EXN\_HANDLER\_OCAML}
          \begin{itemize}
            \item Set \regname{RSP} to \localname{domain\_state->exn\_handler} value
            \item Restore \localname{domain\_state->exn\_handler} from trap
            \item Jump with \funcname{ret} (same effect as using \funcname{pop} and \funcname{jmp})
          \end{itemize}
        \item<3-> Otherwise, switches to the C stack to be able to execute C code
        \item<4-> Calls \funcname{caml\_stash\_backtrace}, a C function provided by the runtime\ignorespaces
          \onslide<6->{, with
            \begin{itemize}
              \item \funcarg{exn}: exception value
              \item \funcarg{pc}: code address of the raising point
              \item \funcarg{sp}: stack address of the raising function
              \item \funcarg{trapsp}: stack address of the next handler
            \end{itemize}
          }
      \end{itemize}
      \bigskip
      \mintocaml[firstline=9,lastline=13,highlightlines=12]{../src/backtrace/backtrace.ml}
    \end{column}
    \begin{column}{0.5\textwidth}
      \raggedleft
      \onslide*<1,3-4>{
        \mintc[firstline=190,lastline=192]{../ocaml/runtime/amd64.S}
        \mintc[firstline=234,lastline=237]{../ocaml/runtime/amd64.S}
      }
      \onslide*<2>{
        \mintc[firstline=190,lastline=192,highlightlines={191-192}]{../ocaml/runtime/amd64.S}
        \mintc[firstline=234,lastline=237,highlightlines={235-237}]{../ocaml/runtime/amd64.S}
      }
      \onslide*<1>{\listCamlRaiseExn[highlightlines=705]{}}%
      \onslide*<2>{\listCamlRaiseExn[highlightlines={707-708}]{}}%
      \onslide*<3>{\listCamlRaiseExn[highlightlines={712-714}]{}}%
      \onslide*<4>{\listCamlRaiseExn[highlightlines={715-720}]{}}%
      \onslide*<5>{\providecommand\step{1}\input{diagrams/backtrace/backtrace.tex}}%
      \onslide*<6>{\providecommand\step{2}\input{diagrams/backtrace/backtrace.tex}}%
    \end{column}
  \end{columns}
\end{frame}

\newcommand\listStashBacktrace[1][]{
  \listc[firstline=91,lastline=120,#1]{../ocaml/runtime/backtrace\_nat.c}{runtime/backtrace\_nat.c:91}}

\begin{frame}{Backtraces}{Introducing \funcname{caml\_stash\_backtrace}}
  \begin{columns}[c]
    \begin{column}{0.5\textwidth}
      \minipage[c][0.66\textheight][s]{\columnwidth}
      \begin{itemize}
        \item<1-> A C function provided by the runtime (specific to native code)
        \item<2-> It scans the stack, between the stack pointer at the raising point up to the next trap, in order to collect the names of the detected function frames
          \onslide*<2>{
            \begin{itemize}
              \item And more information, like line number, column number...
            \end{itemize}
          }
        \item<3-> It uses the same technique and information as the Garbage Collector (GC) uses to scan the values live on the stack
          \onslide*<4>{
            \begin{itemize}
              \item \typename{frame\_descr} are at the core of stack unwinding
            \end{itemize}
          }
      \end{itemize}
      \vfill
      \mintocaml[firstline=9,lastline=13,highlightlines=12]{../src/backtrace/backtrace.ml}
      \endminipage
    \end{column}
    \begin{column}{0.5\textwidth}
      \onslide*<1>{\listStashBacktrace[highlightlines=91]{}}
      \onslide*<2>{\listStashBacktrace[highlightlines={91,117-118}]{}}
      \onslide*<3->{\listStashBacktrace[highlightlines={94,106,109-110}]{}}
    \end{column}
  \end{columns}
\end{frame}

\newcommand\listFrameDescr[1][]{
  \listocaml[firstline=111,lastline=124,#1]{../ocaml/asmcomp/emitaux.ml}{asmcomp/emitaux.ml:111}}
\newcommand\listDebugInfo[1][]{
  \listocaml[firstline=38,lastline=49,#1]{../ocaml/lambda/debuginfo.mli}{lambda/debuginfo.mli:38}}

\begin{frame}{Backtraces}{\typename{frame\_descr} and \typename{frame\_debuginfo} in a nutshell}
  \begin{columns}[c]
    \begin{column}{0.5\textwidth}
      \begin{itemize}
        \item<1-> For every return address in an OCaml program, there is a associated \typename{frame\_descr}
        \item<2-> A \typename{frame\_descr} specifies
        \begin{itemize}
          \item<2-> The size of the function stack frame
          \item<3-> Where live values are on the stack (so that the GC can mark them as live and prevent from being swept)
        \end{itemize}
        \item<4-> When compiled with debug information, \typename{frame\_descr} will be linked to a \typename{frame\_debuginfo}
        \begin{itemize}
          \item<5-> Contains information about the position in the source code (file name, line and column number)
        \end{itemize}
      \end{itemize}
    \end{column}
    \begin{column}{0.5\textwidth}
        \onslide*<1>{\listFrameDescr[highlightlines={118-122}]{}}
        \onslide*<2>{\listFrameDescr[highlightlines={120}]{}}
        \onslide*<3>{\listFrameDescr[highlightlines={121}]{}}
        \onslide*<4->{\listFrameDescr[highlightlines={122}]{}}
        \medskip
        \onslide*<1-3>{\listDebugInfo{}}
        \onslide*<4>{\listDebugInfo[highlightlines={38,49}]{}}
        \onslide*<5>{\listDebugInfo[highlightlines={39-46}]{}}
    \end{column}
  \end{columns}
\end{frame}

\begin{frame}{Backtraces}{Scanning the stack with \typename{frame\_descr}}
  \begin{columns}[c]
    \begin{column}{0.5\textwidth}
      \begin{itemize}
        \item Given the return address of the code that raises the exception by calling \funcname{caml\_raise\_exn} (\funcarg{pc} argument of \funcname{caml\_stash\_backtrace})
        \item And the position of the stack pointer (\funcarg{sp} argument of \funcname{caml\_stash\_backtrace})
        \item The frame size given by \typename{frame\_descr}.\funcarg{frame\_size}, allows to find the end of the previous frame
        \item And the return address of the caller of this function
        \bigskip
        \onslide<3->{
        \item Note that traps (here the reddish \funcname{ExnA}, \funcname{ExnB} and \funcname{ExnC} blocks) belong to the frame that installed them, and so the frame size changes when traps are removed
        }
      \end{itemize}
    \end{column}
    \begin{column}{0.5\textwidth}
      \raggedleft
      \onslide*<1>{\providecommand\step{2}\input{diagrams/backtrace/backtrace.tex}}%
      \onslide*<2>{\providecommand\step{1}\input{diagrams/backtrace/scanstack.tex}}%
      \onslide*<3>{\providecommand\step{2}\input{diagrams/backtrace/scanstack.tex}}%
      \onslide*<4>{\providecommand\step{3}\input{diagrams/backtrace/scanstack.tex}}%
      \onslide*<5>{\providecommand\step{4}\input{diagrams/backtrace/scanstack.tex}}%
      \onslide*<6>{\providecommand\step{5}\input{diagrams/backtrace/scanstack.tex}}%
    \end{column}
  \end{columns}
\end{frame}

% Duplicate of previous-previous slide
\begin{frame}{Backtraces}{Continuing on \funcname{caml\_stash\_backtrace}}
  \begin{columns}[c]
    \begin{column}{0.5\textwidth}
      \minipage[c][0.75\textheight][s]{\columnwidth}
      \begin{itemize}
          \color{gray}
        \item A C function provided by the runtime (specific to native code)
        \item It scans the stack, between the stack pointer at the raising point up to the next trap, in order to collect the names of the detected function frames
        \item It uses the same technique and information as the Garbage Collector (GC) uses to scan the values live on the stack
      \end{itemize}
      \begin{itemize}
        \item For each function frame detected, store a pointer to the \typename{frame\_descr} in the \localname{domain\_state->backtrace\_buffer} array, effectively constructing the backtrace
      \end{itemize}
      \onslide<2->{
        \begin{itemize}
            \smallskip
          \item Constructed backtrace:
            \funcname{definitely\_raise},
            \onslide<3->{\funcname{probably\_raise}}
        \end{itemize}
      }
      \vfill
      \mintocaml[firstline=9,lastline=13,highlightlines=12]{../src/backtrace/backtrace.ml}
      \endminipage
    \end{column}
    \begin{column}{0.5\textwidth}
      \raggedleft
      \onslide*<1>{\listStashBacktrace[highlightlines={114-115}]{}}
      \onslide*<2>{\providecommand\step{3}\input{diagrams/backtrace/backtrace.tex}}%
      \onslide*<3>{\providecommand\step{4}\input{diagrams/backtrace/backtrace.tex}}%
    \end{column}
  \end{columns}
\end{frame}

\begin{frame}{Backtraces}{Back to \funcname{caml\_raise\_exn}}
  \begin{columns}[c]
    \begin{column}{0.5\textwidth}
      \minipage[c][0.66\textheight][s]{\columnwidth}
      \begin{itemize}\color{gray}
        \item Checks wheteher exceptions backtrace should be recorded, by checking the \funcarg{domain\_state->backtrace\_active} value
        \item Switches to the C stack to be able to execute C code
        \item Calls \funcname{caml\_stash\_backtrace}, to collect the backtrace up to the next trap
      \end{itemize}
      \onslide*<2->{
        \begin{itemize}
          \item<2-> Executes the exception handler in \funcname{probably\_raise} with \funcname{RESTORE\_EXN\_HANDLER\_OCAML}
            \begin{itemize}
              \item Set \regname{RSP} to \localname{domain\_state->exn\_handler} value
              \item Restore \localname{domain\_state->exn\_handler} from trap
              \item Jump to the handler code with \funcname{ret}
            \end{itemize}
        \end{itemize}
      }
      \vfill
      \onslide*<1>{\mintocaml[firstline=9,lastline=13,highlightlines=12]{../src/backtrace/backtrace.ml}}
      \onslide*<2->{\mintocaml[firstline=15,lastline=21,highlightlines={19-21}]{../src/backtrace/backtrace.ml}}
      \endminipage
    \end{column}
    \begin{column}{0.5\textwidth}
      \centering
      \onslide*<1>{
        \mintc[firstline=190,lastline=192]{../ocaml/runtime/amd64.S}
        \mintc[firstline=234,lastline=237]{../ocaml/runtime/amd64.S}
      }
      \onslide*<2>{
        \mintc[firstline=190,lastline=192,highlightlines={191-192}]{../ocaml/runtime/amd64.S}
        \mintc[firstline=234,lastline=237,highlightlines={235-237}]{../ocaml/runtime/amd64.S}
      }
      \onslide*<1>{\listCamlRaiseExn[highlightlines=720]{}}
      \onslide*<2>{\listCamlRaiseExn[highlightlines=722]{}}
      \onslide*<3->{\providecommand\step{5}\input{diagrams/backtrace/backtrace.tex}}
    \end{column}
  \end{columns}
\end{frame}

\begin{frame}{Backtraces}{\funcname{caml\_probably\_raise}'s handler doesn't catch \typename{ExnC}}
  \begin{columns}[c]
    \begin{column}{0.45\textwidth}
      \minipage[c][0.75\textheight][s]{\columnwidth}
      \begin{itemize}
        \item<1-> \funcname{match \dots with}\footnotemark pattern matching can also match exceptions
        \item<1-> The CMM of \funcname{probably\_raise} shows that this \funcname{match \dots with} is implemented with a pattern matching and a \funcname{try \dots with}
        \item<1-> But this one only matches on \typename{ExnA} exception
        \item<2-> Since the pattern matching fails to catch the exception, \funcname{reraise} is called with our current \typename{ExnC} exception
        \item<3-> Disassembly shows that \funcname{reraise} is actually a call to \funcname{caml\_reraise\_exn}, which is also an assembly function provided by the runtime
      \end{itemize}
      \vfill
      \mintocaml[firstline=15,lastline=21,highlightlines={18-21}]{../src/backtrace/backtrace.ml}
      \endminipage
    \end{column}
    \begin{column}{0.55\textwidth}
      \centering
      \onslide*<1>{\mintcmm[highlightlines={10-18}]{../src/backtrace/camlBacktrace\_\_probably\_raise\_351.cmm}}
      \onslide*<2>{\listcmm[highlightlines=18]{../src/backtrace/camlBacktrace\_\_probably\_raise_351.cmm}{CMM of probably\_raise}}
      \onslide*<3>{\mintobjdump[firstline=5,lastline=35,highlightlines=31]{../src/backtrace/camlBacktrace\_\_probably\_raise_351.objdump}}
    \end{column}
  \end{columns}
  \only<1>{\footnotetext[1]{https://blog.janestreet.com/pattern-matching-and-exception-handling-unite/}}
\end{frame}

\newcommand\listCamlReraiseExn[1][]{
  \listasm[firstline=727,lastline=734,#1]{../ocaml/runtime/amd64.S}{runtime/amd64.S:727}}

\begin{frame}{Backtraces}{Introducing \funcname{caml\_reraise\_exn}}
  \begin{columns}[c]
    \begin{column}{0.5\textwidth}
      \minipage[c][0.66\textheight][s]{\columnwidth}
      \begin{itemize}
        \item<1-> The exception have to be reraised to the next handler
        \item<2-> First, checks if the backtrace should be collected with \funcarg{domain\_state->backtrace\_active}
        \only<3>{
        \begin{itemize}
            \item If not, behaves like a \funcname{raise\_notrace} with \funcname{RESTORE\_EXN\_HANDLER\_OCAML}
        \end{itemize}
        }
        \item<4-> Jumps into \funcname{caml\_raise\_exn}
        \item<5-> Calls \funcname{caml\_stash\_backtrace} again, to scan the next part of the stack: from the trap just pop-ed to the next trap
      \end{itemize}
      \vfill
      \mintocaml[firstline=15,lastline=21,highlightlines={18-21}]{../src/backtrace/backtrace.ml}
      \endminipage
    \end{column}
    \begin{column}{0.5\textwidth}
      \raggedleft
      \onslide*<1>{\listCamlReraiseExn{}}
      \onslide*<2>{\listCamlReraiseExn[highlightlines=729]{}}
      \onslide*<3>{
        \mintc[firstline=190,lastline=192,highlightlines={191-192}]{../ocaml/runtime/amd64.S}
        \mintc[firstline=234,lastline=237,highlightlines={235-237}]{../ocaml/runtime/amd64.S}
        \medskip
        \listCamlReraiseExn[highlightlines=731]{}
      }
      \onslide*<4-5>{
        % TODO refactor with \listCamlReraiseExn
        \mintasm[firstline=727,lastline=734,highlightlines=730]{../ocaml/runtime/amd64.S}
        \medskip
      }
      \onslide*<4>{\listCamlRaiseExn[highlightlines=711]{}}
      \onslide*<5>{\listCamlRaiseExn[highlightlines=720]{}}
    \end{column}
  \end{columns}
\end{frame}

\begin{frame}{Backtraces}{Backtrace collection continues}
  \begin{columns}[c]
    \begin{column}{0.55\textwidth}
      \minipage[c][0.75\textheight][s]{\columnwidth}
      \begin{itemize}
        \item<1-> \funcname{caml\_stash\_backtrace} scans the older part of the stack, up to the next trap
        \item<6-> Controls jumps to the nested exception handler in \funcname{maybe\_raise}, which matches only on \typename{ExnB}
        \item<7-> \funcname{caml\_reraise\_exn} is called again, backtrace collection resumes with \funcname{caml\_stash\_backtrace}
      \end{itemize}
      \medskip
      \begin{itemize}
        \item<2-> Constructed backtrace:
        \begin{itemize}
          \item <2-> \funcname{definitely\_raise}
          \item <2-> \funcname{probably\_raise}
          \item <3-> \funcname{probably\_raise} (re-raise)
          \item <4-> \funcname{maybe\_raise}
          \item <5-> \funcname{main}
          \item <8-> \funcname{main} (re-raise)
        \end{itemize}
      \end{itemize}
      \vfill
      \onslide*<1-5>{\mintocaml[firstline=15,lastline=21]{../src/backtrace/backtrace.ml}}
      \onslide*<6>{\mintocaml[firstline=28,lastline=34,highlightlines=31]{../src/backtrace/backtrace.ml}}
      \onslide*<7->{\mintocaml[firstline=28,lastline=34,highlightlines=33]{../src/backtrace/backtrace.ml}}
      \endminipage
    \end{column}
    \begin{column}{0.45\textwidth}
      \raggedleft
      \onslide*<1>{\providecommand\step{6}\input{diagrams/backtrace/backtrace.tex}}%
      \onslide*<2>{\providecommand\step{6}\input{diagrams/backtrace/backtrace.tex}}%
      \onslide*<3>{\providecommand\step{7}\input{diagrams/backtrace/backtrace.tex}}%
      \onslide*<4>{\providecommand\step{8}\input{diagrams/backtrace/backtrace.tex}}%
      \onslide*<5>{\providecommand\step{9}\input{diagrams/backtrace/backtrace.tex}}%
      \onslide*<6>{\providecommand\step{10}\input{diagrams/backtrace/backtrace.tex}}%
      \onslide*<7>{\providecommand\step{11}\input{diagrams/backtrace/backtrace.tex}}%
      \onslide*<8>{\providecommand\step{12}\input{diagrams/backtrace/backtrace.tex}}%
      \onslide*<9>{\providecommand\step{13}\input{diagrams/backtrace/backtrace.tex}}%
    \end{column}
  \end{columns}
\end{frame}

\begin{frame}{Backtraces}{Printing the backtrace}
  \begin{columns}[c]
    \begin{column}{0.45\textwidth}
      \minipage[c][0.66\textheight][s]{\columnwidth}
      \begin{itemize}
        \item<1-> The exception backtrace can now be printed to \funcarg{stdout} with \funcname{Printexc.print_backtrace}
        \item<2-> First the exception backtrace is retrieved into an OCaml array with \funcname{get_raw_backtrace }
        \item<3-> Which is implemented as the \funcname{caml_get_exception_raw_backtrace} C function in the runtime
        \item<4-> And finally printed with \funcname{print_exception_backtrace}
      \end{itemize}
      \vfill
      \mintocaml[firstline=28,lastline=34,highlightlines=33]{../src/backtrace/backtrace.ml}
      \endminipage
    \end{column}
    \begin{column}{0.55\textwidth}
      \onslide*<1,2>{
        \mintocaml[firstline=100,lastline=101]{../ocaml/stdlib/printexc.ml}
      }
      \only<1>{
        \listocaml[firstline=170,lastline=172]{../ocaml/stdlib/printexc.ml}{stdlib/printexc.ml:170}
      }
      \only<2>{
        \listocaml[firstline=170,lastline=172,highlightlines=172]{../ocaml/stdlib/printexc.ml}{stdlib/printexc.ml:170}
      }
      \only<3>{
        \mintc[firstline=154,lastline=156]{../ocaml/runtime/backtrace.c}
        \listc[firstline=171,lastline=192,highlightlines={184-187}]{../ocaml/runtime/backtrace.c}{runtime/backtrace.c:154}
      }
      \only<4>{
        \listocaml[firstline=155,lastline=165]{../ocaml/stdlib/printexc.ml}{stdlib/printexc.ml:55}
      }
    \end{column}
  \end{columns}
\end{frame}


\subsection{The multiple kinds of raise}
\frameSubsection{}{}

\begin{frame}{Backtraces}{When backtraces are not needed}
  \begin{columns}[c]
    \begin{column}{0.45\textwidth}
      \minipage[c][0.66\textheight][s]{\columnwidth}
      \begin{itemize}
        \item<1-> What if the backtrace for an exception is never needed?
        \item<2-> It's possible to raise the exception explicitly with \funcname{raise\_notrace} to make raising as cheap as possible
        \item<3-> An example of use in the \funcname{dynlink} library of the OCaml compiler
      \end{itemize}
      \vfill
      \onslide<3->{
      \listocaml[firstline=205,lastline=209,highlightlines=207]{../ocaml/otherlibs/dynlink/dynlink\_compilerlibs/misc.ml}{otherlibs/dynlink/dynlink\_compilerlibs/misc.ml:205}
      }
      \endminipage
    \end{column}
    \begin{column}{0.55\textwidth}
    \only<4>{
      \mintcmm[highlightlines=3]{../src/notrace/camlNotrace\_\_fun_347.cmm}
      \listcmm{../src/notrace/camlNotrace\_\_all\_somes\_327.cmm}{all\_somes}
    }
    \onslide*<5->{
      \listobjdump[highlightlines={6-9}]{../src/notrace/camlNotrace\_\_fun_347.objdump}{anonymous function from \funcname{all\_somes}}
    }
    \end{column}
  \end{columns}
\end{frame}

\begin{frame}{Backtraces}{Summary table of the multiple kinds of raise}
  \begin{itemize}
    \item When debugging information is enabled and if the backtrace of an exception isn't needed, it's possible to raise with \funcname{raise\_notrace} for performance
    \item When debugging information is \emph{not} enabled, the compiler optimises \funcname{raise} calls to inline assembly (same effect as using \funcname{raise\_notrace})
  \end{itemize}
  \bigskip
  \centering
  \begin{tabular}{| l | c | c | c | c |}
    \hline
    \parbox{10em}{Debug information \\ (\funcarg{-g} flag)}  & \multicolumn{2}{|c|}{Disabled}                                     & \multicolumn{2}{|c|}{Enabled} \\
    \hline
    Raise kind                                               & \funcname{raise} & \funcname{raise\_notrace}                       & \funcname{raise} & \funcname{raise\_notrace} \\
    \hline
    Compilation                                              & \parbox{8em}{inline assembly\\ (optimization)} & inline assembly   & \funcname{caml\_raise\_exn} & inline assembly \\
    \hline
  \end{tabular}
\end{frame}


%
% Takeaway #5
%
\subsection*{Takeaway \#5}
\frameSubsectionTakeaway{}
\againframe<5>{takeaway}
}%
      \onslide*<6>{\providecommand\step{10}%
% Backtraces
%
\subsection{One advantage of raising with debugging information}
\frameSubsection{}{}

\begin{frame}{Backtraces}{Debugging information \& source code}
  \begin{columns}[c]
    \begin{column}{0.5\textwidth}
      \begin{itemize}
        \item Raising an exception is fun and games, but how do we know where the exception originated? $\rightarrow$ With the backtrace associated with the exception!
        \item In order to get a backtrace from the point where an exception was raised, it's necessary to compile with debugging information enabled (\funcarg{-g} flag for the compiler)
        \item Same example as in the `Nested handlers' section
      \end{itemize}
    \end{column}
    \begin{column}{0.5\textwidth}
      \listocaml[firstline=9,lastline=34]{../src/backtrace/backtrace.ml}{backtrace/backtrace.ml}
    \end{column}
  \end{columns}
\end{frame}

\begin{frame}{Backtraces}{CMM and disassembly of \funcname{definitely\_raise}}
  \begin{columns}[c]
    \begin{column}{0.5\textwidth}
      \minipage[c][0.66\textheight][s]{\columnwidth}
      \begin{itemize}
        \item<1-> An effect of compiling with debugging information (\funcarg{-g}): the CMM no longer uses \funcname{raise\_notrace} to raise the exception but \funcname{raise} instead
        \item<2-> Disassembly shows that CMM \funcname{raise} is actually a call to \funcname{caml\_raise\_exn}, a function in assembler provided by the runtime
      \end{itemize}
      \vfill
      \mintocaml[firstline=9,lastline=13,highlightlines=12]{../src/backtrace/backtrace.ml}
      \endminipage
    \end{column}
    \begin{column}{0.5\textwidth}
      \onslide*<1>{
        \mintcmm[highlightlines=5]{../src/backtrace/camlBacktrace\_\_definitely\_raise\_347.cmm}
      }
      \onslide*<2>{
        \mintobjdump[firstline=2,highlightlines=14]{../src/backtrace/camlBacktrace\_\_definitely\_raise\_347.objdump}
      }
    \end{column}
  \end{columns}
\end{frame}

\begin{frame}{Backtraces}{Introducing \funcname{caml\_raise\_exn}}
  \begin{columns}[c]
    \begin{column}{0.5\textwidth}
      \minipage[c][0.66\textheight][s]{\columnwidth}
      \onslide*<1>{
        \begin{itemize}
          \item Raising the exception is performed using \funcname{caml\_raise\_exn} which is
            \begin{itemize}
              \item An assembly function provided by the runtime
              \item Architecture specific
            \end{itemize}
          \item Let's see what it does differently from \funcname{raise\_notrace}\dots
          \item We are about to raise \typename{ExnC} with \funcname{caml\_raise\_exn} from \funcname{definitely\_raise}
        \end{itemize}
      }
      \onslide*<2>{
        \mintobjdump[firstline=2,highlightlines=14]{../src/backtrace/camlBacktrace\_\_definitely\_raise\_347.objdump}
      }
      \vfill
      \mintocaml[firstline=9,lastline=13,highlightlines=12]{../src/backtrace/backtrace.ml}
      \endminipage
    \end{column}
    \begin{column}{0.5\textwidth}
      \centering
      \providecommand\step{1}\input{diagrams/backtrace/backtrace.tex}
    \end{column}
  \end{columns}
\end{frame}

\begin{frame}{Backtraces}{But before that, we need more fields from \typename{caml\_domain\_state}}
  \begin{columns}
    \begin{column}{0.5\textwidth}
      \begin{itemize}
        \item Remember \typename{caml\_domain\_state} the per-domain runtime state?
        \item Some other members are of interest to us now\dots
        \item \localname{backtrace\_active} is used to know if the runtime should collect backtraces
        \item \localname{backtrace\_active} can be set by
          \begin{itemize}
            \item The \funcarg{b} flag in the \funcname{OCAMLRUNPARAM} environment variable
            \item \mintinline{ocaml}{Printexc.record_backtrace : bool -> unit} \footnotemark
          \end{itemize}
      \end{itemize}
    \end{column}
    \begin{column}{0.5\textwidth}
      \begin{listing}
        \mintc[highlightlines={9-11}]{src/ocaml/domain\_state\_backtrace.h}
        \caption{runtime/caml/domain\_state.h}
      \end{listing}
    \end{column}
  \end{columns}
  \footnotetext[1]{https://v2.ocaml.org/api/Printexc.html}
\end{frame}

\newcommand\listCamlRaiseExn[1][]{
  \listasm[firstline=702,lastline=725,#1]{../ocaml/runtime/amd64.S}{runtime/amd64.S:702}}

\begin{frame}{Backtraces}{Walk through \funcname{caml\_raise\_exn}}
  \begin{columns}[T]
    \begin{column}{0.5\textwidth}
      \begin{itemize}
        \item<1-> First checks whether exceptions backtrace should be recorded
        \item<1-> By checking the \funcarg{domain\_state->backtrace\_active} value
        \item<2-> If not, acts as \funcname{raise\_notrace} with \funcname{RESTORE\_EXN\_HANDLER\_OCAML}
          \begin{itemize}
            \item Set \regname{RSP} to \localname{domain\_state->exn\_handler} value
            \item Restore \localname{domain\_state->exn\_handler} from trap
            \item Jump with \funcname{ret} (same effect as using \funcname{pop} and \funcname{jmp})
          \end{itemize}
        \item<3-> Otherwise, switches to the C stack to be able to execute C code
        \item<4-> Calls \funcname{caml\_stash\_backtrace}, a C function provided by the runtime\ignorespaces
          \onslide<6->{, with
            \begin{itemize}
              \item \funcarg{exn}: exception value
              \item \funcarg{pc}: code address of the raising point
              \item \funcarg{sp}: stack address of the raising function
              \item \funcarg{trapsp}: stack address of the next handler
            \end{itemize}
          }
      \end{itemize}
      \bigskip
      \mintocaml[firstline=9,lastline=13,highlightlines=12]{../src/backtrace/backtrace.ml}
    \end{column}
    \begin{column}{0.5\textwidth}
      \raggedleft
      \onslide*<1,3-4>{
        \mintc[firstline=190,lastline=192]{../ocaml/runtime/amd64.S}
        \mintc[firstline=234,lastline=237]{../ocaml/runtime/amd64.S}
      }
      \onslide*<2>{
        \mintc[firstline=190,lastline=192,highlightlines={191-192}]{../ocaml/runtime/amd64.S}
        \mintc[firstline=234,lastline=237,highlightlines={235-237}]{../ocaml/runtime/amd64.S}
      }
      \onslide*<1>{\listCamlRaiseExn[highlightlines=705]{}}%
      \onslide*<2>{\listCamlRaiseExn[highlightlines={707-708}]{}}%
      \onslide*<3>{\listCamlRaiseExn[highlightlines={712-714}]{}}%
      \onslide*<4>{\listCamlRaiseExn[highlightlines={715-720}]{}}%
      \onslide*<5>{\providecommand\step{1}\input{diagrams/backtrace/backtrace.tex}}%
      \onslide*<6>{\providecommand\step{2}\input{diagrams/backtrace/backtrace.tex}}%
    \end{column}
  \end{columns}
\end{frame}

\newcommand\listStashBacktrace[1][]{
  \listc[firstline=91,lastline=120,#1]{../ocaml/runtime/backtrace\_nat.c}{runtime/backtrace\_nat.c:91}}

\begin{frame}{Backtraces}{Introducing \funcname{caml\_stash\_backtrace}}
  \begin{columns}[c]
    \begin{column}{0.5\textwidth}
      \minipage[c][0.66\textheight][s]{\columnwidth}
      \begin{itemize}
        \item<1-> A C function provided by the runtime (specific to native code)
        \item<2-> It scans the stack, between the stack pointer at the raising point up to the next trap, in order to collect the names of the detected function frames
          \onslide*<2>{
            \begin{itemize}
              \item And more information, like line number, column number...
            \end{itemize}
          }
        \item<3-> It uses the same technique and information as the Garbage Collector (GC) uses to scan the values live on the stack
          \onslide*<4>{
            \begin{itemize}
              \item \typename{frame\_descr} are at the core of stack unwinding
            \end{itemize}
          }
      \end{itemize}
      \vfill
      \mintocaml[firstline=9,lastline=13,highlightlines=12]{../src/backtrace/backtrace.ml}
      \endminipage
    \end{column}
    \begin{column}{0.5\textwidth}
      \onslide*<1>{\listStashBacktrace[highlightlines=91]{}}
      \onslide*<2>{\listStashBacktrace[highlightlines={91,117-118}]{}}
      \onslide*<3->{\listStashBacktrace[highlightlines={94,106,109-110}]{}}
    \end{column}
  \end{columns}
\end{frame}

\newcommand\listFrameDescr[1][]{
  \listocaml[firstline=111,lastline=124,#1]{../ocaml/asmcomp/emitaux.ml}{asmcomp/emitaux.ml:111}}
\newcommand\listDebugInfo[1][]{
  \listocaml[firstline=38,lastline=49,#1]{../ocaml/lambda/debuginfo.mli}{lambda/debuginfo.mli:38}}

\begin{frame}{Backtraces}{\typename{frame\_descr} and \typename{frame\_debuginfo} in a nutshell}
  \begin{columns}[c]
    \begin{column}{0.5\textwidth}
      \begin{itemize}
        \item<1-> For every return address in an OCaml program, there is a associated \typename{frame\_descr}
        \item<2-> A \typename{frame\_descr} specifies
        \begin{itemize}
          \item<2-> The size of the function stack frame
          \item<3-> Where live values are on the stack (so that the GC can mark them as live and prevent from being swept)
        \end{itemize}
        \item<4-> When compiled with debug information, \typename{frame\_descr} will be linked to a \typename{frame\_debuginfo}
        \begin{itemize}
          \item<5-> Contains information about the position in the source code (file name, line and column number)
        \end{itemize}
      \end{itemize}
    \end{column}
    \begin{column}{0.5\textwidth}
        \onslide*<1>{\listFrameDescr[highlightlines={118-122}]{}}
        \onslide*<2>{\listFrameDescr[highlightlines={120}]{}}
        \onslide*<3>{\listFrameDescr[highlightlines={121}]{}}
        \onslide*<4->{\listFrameDescr[highlightlines={122}]{}}
        \medskip
        \onslide*<1-3>{\listDebugInfo{}}
        \onslide*<4>{\listDebugInfo[highlightlines={38,49}]{}}
        \onslide*<5>{\listDebugInfo[highlightlines={39-46}]{}}
    \end{column}
  \end{columns}
\end{frame}

\begin{frame}{Backtraces}{Scanning the stack with \typename{frame\_descr}}
  \begin{columns}[c]
    \begin{column}{0.5\textwidth}
      \begin{itemize}
        \item Given the return address of the code that raises the exception by calling \funcname{caml\_raise\_exn} (\funcarg{pc} argument of \funcname{caml\_stash\_backtrace})
        \item And the position of the stack pointer (\funcarg{sp} argument of \funcname{caml\_stash\_backtrace})
        \item The frame size given by \typename{frame\_descr}.\funcarg{frame\_size}, allows to find the end of the previous frame
        \item And the return address of the caller of this function
        \bigskip
        \onslide<3->{
        \item Note that traps (here the reddish \funcname{ExnA}, \funcname{ExnB} and \funcname{ExnC} blocks) belong to the frame that installed them, and so the frame size changes when traps are removed
        }
      \end{itemize}
    \end{column}
    \begin{column}{0.5\textwidth}
      \raggedleft
      \onslide*<1>{\providecommand\step{2}\input{diagrams/backtrace/backtrace.tex}}%
      \onslide*<2>{\providecommand\step{1}\input{diagrams/backtrace/scanstack.tex}}%
      \onslide*<3>{\providecommand\step{2}\input{diagrams/backtrace/scanstack.tex}}%
      \onslide*<4>{\providecommand\step{3}\input{diagrams/backtrace/scanstack.tex}}%
      \onslide*<5>{\providecommand\step{4}\input{diagrams/backtrace/scanstack.tex}}%
      \onslide*<6>{\providecommand\step{5}\input{diagrams/backtrace/scanstack.tex}}%
    \end{column}
  \end{columns}
\end{frame}

% Duplicate of previous-previous slide
\begin{frame}{Backtraces}{Continuing on \funcname{caml\_stash\_backtrace}}
  \begin{columns}[c]
    \begin{column}{0.5\textwidth}
      \minipage[c][0.75\textheight][s]{\columnwidth}
      \begin{itemize}
          \color{gray}
        \item A C function provided by the runtime (specific to native code)
        \item It scans the stack, between the stack pointer at the raising point up to the next trap, in order to collect the names of the detected function frames
        \item It uses the same technique and information as the Garbage Collector (GC) uses to scan the values live on the stack
      \end{itemize}
      \begin{itemize}
        \item For each function frame detected, store a pointer to the \typename{frame\_descr} in the \localname{domain\_state->backtrace\_buffer} array, effectively constructing the backtrace
      \end{itemize}
      \onslide<2->{
        \begin{itemize}
            \smallskip
          \item Constructed backtrace:
            \funcname{definitely\_raise},
            \onslide<3->{\funcname{probably\_raise}}
        \end{itemize}
      }
      \vfill
      \mintocaml[firstline=9,lastline=13,highlightlines=12]{../src/backtrace/backtrace.ml}
      \endminipage
    \end{column}
    \begin{column}{0.5\textwidth}
      \raggedleft
      \onslide*<1>{\listStashBacktrace[highlightlines={114-115}]{}}
      \onslide*<2>{\providecommand\step{3}\input{diagrams/backtrace/backtrace.tex}}%
      \onslide*<3>{\providecommand\step{4}\input{diagrams/backtrace/backtrace.tex}}%
    \end{column}
  \end{columns}
\end{frame}

\begin{frame}{Backtraces}{Back to \funcname{caml\_raise\_exn}}
  \begin{columns}[c]
    \begin{column}{0.5\textwidth}
      \minipage[c][0.66\textheight][s]{\columnwidth}
      \begin{itemize}\color{gray}
        \item Checks wheteher exceptions backtrace should be recorded, by checking the \funcarg{domain\_state->backtrace\_active} value
        \item Switches to the C stack to be able to execute C code
        \item Calls \funcname{caml\_stash\_backtrace}, to collect the backtrace up to the next trap
      \end{itemize}
      \onslide*<2->{
        \begin{itemize}
          \item<2-> Executes the exception handler in \funcname{probably\_raise} with \funcname{RESTORE\_EXN\_HANDLER\_OCAML}
            \begin{itemize}
              \item Set \regname{RSP} to \localname{domain\_state->exn\_handler} value
              \item Restore \localname{domain\_state->exn\_handler} from trap
              \item Jump to the handler code with \funcname{ret}
            \end{itemize}
        \end{itemize}
      }
      \vfill
      \onslide*<1>{\mintocaml[firstline=9,lastline=13,highlightlines=12]{../src/backtrace/backtrace.ml}}
      \onslide*<2->{\mintocaml[firstline=15,lastline=21,highlightlines={19-21}]{../src/backtrace/backtrace.ml}}
      \endminipage
    \end{column}
    \begin{column}{0.5\textwidth}
      \centering
      \onslide*<1>{
        \mintc[firstline=190,lastline=192]{../ocaml/runtime/amd64.S}
        \mintc[firstline=234,lastline=237]{../ocaml/runtime/amd64.S}
      }
      \onslide*<2>{
        \mintc[firstline=190,lastline=192,highlightlines={191-192}]{../ocaml/runtime/amd64.S}
        \mintc[firstline=234,lastline=237,highlightlines={235-237}]{../ocaml/runtime/amd64.S}
      }
      \onslide*<1>{\listCamlRaiseExn[highlightlines=720]{}}
      \onslide*<2>{\listCamlRaiseExn[highlightlines=722]{}}
      \onslide*<3->{\providecommand\step{5}\input{diagrams/backtrace/backtrace.tex}}
    \end{column}
  \end{columns}
\end{frame}

\begin{frame}{Backtraces}{\funcname{caml\_probably\_raise}'s handler doesn't catch \typename{ExnC}}
  \begin{columns}[c]
    \begin{column}{0.45\textwidth}
      \minipage[c][0.75\textheight][s]{\columnwidth}
      \begin{itemize}
        \item<1-> \funcname{match \dots with}\footnotemark pattern matching can also match exceptions
        \item<1-> The CMM of \funcname{probably\_raise} shows that this \funcname{match \dots with} is implemented with a pattern matching and a \funcname{try \dots with}
        \item<1-> But this one only matches on \typename{ExnA} exception
        \item<2-> Since the pattern matching fails to catch the exception, \funcname{reraise} is called with our current \typename{ExnC} exception
        \item<3-> Disassembly shows that \funcname{reraise} is actually a call to \funcname{caml\_reraise\_exn}, which is also an assembly function provided by the runtime
      \end{itemize}
      \vfill
      \mintocaml[firstline=15,lastline=21,highlightlines={18-21}]{../src/backtrace/backtrace.ml}
      \endminipage
    \end{column}
    \begin{column}{0.55\textwidth}
      \centering
      \onslide*<1>{\mintcmm[highlightlines={10-18}]{../src/backtrace/camlBacktrace\_\_probably\_raise\_351.cmm}}
      \onslide*<2>{\listcmm[highlightlines=18]{../src/backtrace/camlBacktrace\_\_probably\_raise_351.cmm}{CMM of probably\_raise}}
      \onslide*<3>{\mintobjdump[firstline=5,lastline=35,highlightlines=31]{../src/backtrace/camlBacktrace\_\_probably\_raise_351.objdump}}
    \end{column}
  \end{columns}
  \only<1>{\footnotetext[1]{https://blog.janestreet.com/pattern-matching-and-exception-handling-unite/}}
\end{frame}

\newcommand\listCamlReraiseExn[1][]{
  \listasm[firstline=727,lastline=734,#1]{../ocaml/runtime/amd64.S}{runtime/amd64.S:727}}

\begin{frame}{Backtraces}{Introducing \funcname{caml\_reraise\_exn}}
  \begin{columns}[c]
    \begin{column}{0.5\textwidth}
      \minipage[c][0.66\textheight][s]{\columnwidth}
      \begin{itemize}
        \item<1-> The exception have to be reraised to the next handler
        \item<2-> First, checks if the backtrace should be collected with \funcarg{domain\_state->backtrace\_active}
        \only<3>{
        \begin{itemize}
            \item If not, behaves like a \funcname{raise\_notrace} with \funcname{RESTORE\_EXN\_HANDLER\_OCAML}
        \end{itemize}
        }
        \item<4-> Jumps into \funcname{caml\_raise\_exn}
        \item<5-> Calls \funcname{caml\_stash\_backtrace} again, to scan the next part of the stack: from the trap just pop-ed to the next trap
      \end{itemize}
      \vfill
      \mintocaml[firstline=15,lastline=21,highlightlines={18-21}]{../src/backtrace/backtrace.ml}
      \endminipage
    \end{column}
    \begin{column}{0.5\textwidth}
      \raggedleft
      \onslide*<1>{\listCamlReraiseExn{}}
      \onslide*<2>{\listCamlReraiseExn[highlightlines=729]{}}
      \onslide*<3>{
        \mintc[firstline=190,lastline=192,highlightlines={191-192}]{../ocaml/runtime/amd64.S}
        \mintc[firstline=234,lastline=237,highlightlines={235-237}]{../ocaml/runtime/amd64.S}
        \medskip
        \listCamlReraiseExn[highlightlines=731]{}
      }
      \onslide*<4-5>{
        % TODO refactor with \listCamlReraiseExn
        \mintasm[firstline=727,lastline=734,highlightlines=730]{../ocaml/runtime/amd64.S}
        \medskip
      }
      \onslide*<4>{\listCamlRaiseExn[highlightlines=711]{}}
      \onslide*<5>{\listCamlRaiseExn[highlightlines=720]{}}
    \end{column}
  \end{columns}
\end{frame}

\begin{frame}{Backtraces}{Backtrace collection continues}
  \begin{columns}[c]
    \begin{column}{0.55\textwidth}
      \minipage[c][0.75\textheight][s]{\columnwidth}
      \begin{itemize}
        \item<1-> \funcname{caml\_stash\_backtrace} scans the older part of the stack, up to the next trap
        \item<6-> Controls jumps to the nested exception handler in \funcname{maybe\_raise}, which matches only on \typename{ExnB}
        \item<7-> \funcname{caml\_reraise\_exn} is called again, backtrace collection resumes with \funcname{caml\_stash\_backtrace}
      \end{itemize}
      \medskip
      \begin{itemize}
        \item<2-> Constructed backtrace:
        \begin{itemize}
          \item <2-> \funcname{definitely\_raise}
          \item <2-> \funcname{probably\_raise}
          \item <3-> \funcname{probably\_raise} (re-raise)
          \item <4-> \funcname{maybe\_raise}
          \item <5-> \funcname{main}
          \item <8-> \funcname{main} (re-raise)
        \end{itemize}
      \end{itemize}
      \vfill
      \onslide*<1-5>{\mintocaml[firstline=15,lastline=21]{../src/backtrace/backtrace.ml}}
      \onslide*<6>{\mintocaml[firstline=28,lastline=34,highlightlines=31]{../src/backtrace/backtrace.ml}}
      \onslide*<7->{\mintocaml[firstline=28,lastline=34,highlightlines=33]{../src/backtrace/backtrace.ml}}
      \endminipage
    \end{column}
    \begin{column}{0.45\textwidth}
      \raggedleft
      \onslide*<1>{\providecommand\step{6}\input{diagrams/backtrace/backtrace.tex}}%
      \onslide*<2>{\providecommand\step{6}\input{diagrams/backtrace/backtrace.tex}}%
      \onslide*<3>{\providecommand\step{7}\input{diagrams/backtrace/backtrace.tex}}%
      \onslide*<4>{\providecommand\step{8}\input{diagrams/backtrace/backtrace.tex}}%
      \onslide*<5>{\providecommand\step{9}\input{diagrams/backtrace/backtrace.tex}}%
      \onslide*<6>{\providecommand\step{10}\input{diagrams/backtrace/backtrace.tex}}%
      \onslide*<7>{\providecommand\step{11}\input{diagrams/backtrace/backtrace.tex}}%
      \onslide*<8>{\providecommand\step{12}\input{diagrams/backtrace/backtrace.tex}}%
      \onslide*<9>{\providecommand\step{13}\input{diagrams/backtrace/backtrace.tex}}%
    \end{column}
  \end{columns}
\end{frame}

\begin{frame}{Backtraces}{Printing the backtrace}
  \begin{columns}[c]
    \begin{column}{0.45\textwidth}
      \minipage[c][0.66\textheight][s]{\columnwidth}
      \begin{itemize}
        \item<1-> The exception backtrace can now be printed to \funcarg{stdout} with \funcname{Printexc.print_backtrace}
        \item<2-> First the exception backtrace is retrieved into an OCaml array with \funcname{get_raw_backtrace }
        \item<3-> Which is implemented as the \funcname{caml_get_exception_raw_backtrace} C function in the runtime
        \item<4-> And finally printed with \funcname{print_exception_backtrace}
      \end{itemize}
      \vfill
      \mintocaml[firstline=28,lastline=34,highlightlines=33]{../src/backtrace/backtrace.ml}
      \endminipage
    \end{column}
    \begin{column}{0.55\textwidth}
      \onslide*<1,2>{
        \mintocaml[firstline=100,lastline=101]{../ocaml/stdlib/printexc.ml}
      }
      \only<1>{
        \listocaml[firstline=170,lastline=172]{../ocaml/stdlib/printexc.ml}{stdlib/printexc.ml:170}
      }
      \only<2>{
        \listocaml[firstline=170,lastline=172,highlightlines=172]{../ocaml/stdlib/printexc.ml}{stdlib/printexc.ml:170}
      }
      \only<3>{
        \mintc[firstline=154,lastline=156]{../ocaml/runtime/backtrace.c}
        \listc[firstline=171,lastline=192,highlightlines={184-187}]{../ocaml/runtime/backtrace.c}{runtime/backtrace.c:154}
      }
      \only<4>{
        \listocaml[firstline=155,lastline=165]{../ocaml/stdlib/printexc.ml}{stdlib/printexc.ml:55}
      }
    \end{column}
  \end{columns}
\end{frame}


\subsection{The multiple kinds of raise}
\frameSubsection{}{}

\begin{frame}{Backtraces}{When backtraces are not needed}
  \begin{columns}[c]
    \begin{column}{0.45\textwidth}
      \minipage[c][0.66\textheight][s]{\columnwidth}
      \begin{itemize}
        \item<1-> What if the backtrace for an exception is never needed?
        \item<2-> It's possible to raise the exception explicitly with \funcname{raise\_notrace} to make raising as cheap as possible
        \item<3-> An example of use in the \funcname{dynlink} library of the OCaml compiler
      \end{itemize}
      \vfill
      \onslide<3->{
      \listocaml[firstline=205,lastline=209,highlightlines=207]{../ocaml/otherlibs/dynlink/dynlink\_compilerlibs/misc.ml}{otherlibs/dynlink/dynlink\_compilerlibs/misc.ml:205}
      }
      \endminipage
    \end{column}
    \begin{column}{0.55\textwidth}
    \only<4>{
      \mintcmm[highlightlines=3]{../src/notrace/camlNotrace\_\_fun_347.cmm}
      \listcmm{../src/notrace/camlNotrace\_\_all\_somes\_327.cmm}{all\_somes}
    }
    \onslide*<5->{
      \listobjdump[highlightlines={6-9}]{../src/notrace/camlNotrace\_\_fun_347.objdump}{anonymous function from \funcname{all\_somes}}
    }
    \end{column}
  \end{columns}
\end{frame}

\begin{frame}{Backtraces}{Summary table of the multiple kinds of raise}
  \begin{itemize}
    \item When debugging information is enabled and if the backtrace of an exception isn't needed, it's possible to raise with \funcname{raise\_notrace} for performance
    \item When debugging information is \emph{not} enabled, the compiler optimises \funcname{raise} calls to inline assembly (same effect as using \funcname{raise\_notrace})
  \end{itemize}
  \bigskip
  \centering
  \begin{tabular}{| l | c | c | c | c |}
    \hline
    \parbox{10em}{Debug information \\ (\funcarg{-g} flag)}  & \multicolumn{2}{|c|}{Disabled}                                     & \multicolumn{2}{|c|}{Enabled} \\
    \hline
    Raise kind                                               & \funcname{raise} & \funcname{raise\_notrace}                       & \funcname{raise} & \funcname{raise\_notrace} \\
    \hline
    Compilation                                              & \parbox{8em}{inline assembly\\ (optimization)} & inline assembly   & \funcname{caml\_raise\_exn} & inline assembly \\
    \hline
  \end{tabular}
\end{frame}


%
% Takeaway #5
%
\subsection*{Takeaway \#5}
\frameSubsectionTakeaway{}
\againframe<5>{takeaway}
}%
      \onslide*<7>{\providecommand\step{11}%
% Backtraces
%
\subsection{One advantage of raising with debugging information}
\frameSubsection{}{}

\begin{frame}{Backtraces}{Debugging information \& source code}
  \begin{columns}[c]
    \begin{column}{0.5\textwidth}
      \begin{itemize}
        \item Raising an exception is fun and games, but how do we know where the exception originated? $\rightarrow$ With the backtrace associated with the exception!
        \item In order to get a backtrace from the point where an exception was raised, it's necessary to compile with debugging information enabled (\funcarg{-g} flag for the compiler)
        \item Same example as in the `Nested handlers' section
      \end{itemize}
    \end{column}
    \begin{column}{0.5\textwidth}
      \listocaml[firstline=9,lastline=34]{../src/backtrace/backtrace.ml}{backtrace/backtrace.ml}
    \end{column}
  \end{columns}
\end{frame}

\begin{frame}{Backtraces}{CMM and disassembly of \funcname{definitely\_raise}}
  \begin{columns}[c]
    \begin{column}{0.5\textwidth}
      \minipage[c][0.66\textheight][s]{\columnwidth}
      \begin{itemize}
        \item<1-> An effect of compiling with debugging information (\funcarg{-g}): the CMM no longer uses \funcname{raise\_notrace} to raise the exception but \funcname{raise} instead
        \item<2-> Disassembly shows that CMM \funcname{raise} is actually a call to \funcname{caml\_raise\_exn}, a function in assembler provided by the runtime
      \end{itemize}
      \vfill
      \mintocaml[firstline=9,lastline=13,highlightlines=12]{../src/backtrace/backtrace.ml}
      \endminipage
    \end{column}
    \begin{column}{0.5\textwidth}
      \onslide*<1>{
        \mintcmm[highlightlines=5]{../src/backtrace/camlBacktrace\_\_definitely\_raise\_347.cmm}
      }
      \onslide*<2>{
        \mintobjdump[firstline=2,highlightlines=14]{../src/backtrace/camlBacktrace\_\_definitely\_raise\_347.objdump}
      }
    \end{column}
  \end{columns}
\end{frame}

\begin{frame}{Backtraces}{Introducing \funcname{caml\_raise\_exn}}
  \begin{columns}[c]
    \begin{column}{0.5\textwidth}
      \minipage[c][0.66\textheight][s]{\columnwidth}
      \onslide*<1>{
        \begin{itemize}
          \item Raising the exception is performed using \funcname{caml\_raise\_exn} which is
            \begin{itemize}
              \item An assembly function provided by the runtime
              \item Architecture specific
            \end{itemize}
          \item Let's see what it does differently from \funcname{raise\_notrace}\dots
          \item We are about to raise \typename{ExnC} with \funcname{caml\_raise\_exn} from \funcname{definitely\_raise}
        \end{itemize}
      }
      \onslide*<2>{
        \mintobjdump[firstline=2,highlightlines=14]{../src/backtrace/camlBacktrace\_\_definitely\_raise\_347.objdump}
      }
      \vfill
      \mintocaml[firstline=9,lastline=13,highlightlines=12]{../src/backtrace/backtrace.ml}
      \endminipage
    \end{column}
    \begin{column}{0.5\textwidth}
      \centering
      \providecommand\step{1}\input{diagrams/backtrace/backtrace.tex}
    \end{column}
  \end{columns}
\end{frame}

\begin{frame}{Backtraces}{But before that, we need more fields from \typename{caml\_domain\_state}}
  \begin{columns}
    \begin{column}{0.5\textwidth}
      \begin{itemize}
        \item Remember \typename{caml\_domain\_state} the per-domain runtime state?
        \item Some other members are of interest to us now\dots
        \item \localname{backtrace\_active} is used to know if the runtime should collect backtraces
        \item \localname{backtrace\_active} can be set by
          \begin{itemize}
            \item The \funcarg{b} flag in the \funcname{OCAMLRUNPARAM} environment variable
            \item \mintinline{ocaml}{Printexc.record_backtrace : bool -> unit} \footnotemark
          \end{itemize}
      \end{itemize}
    \end{column}
    \begin{column}{0.5\textwidth}
      \begin{listing}
        \mintc[highlightlines={9-11}]{src/ocaml/domain\_state\_backtrace.h}
        \caption{runtime/caml/domain\_state.h}
      \end{listing}
    \end{column}
  \end{columns}
  \footnotetext[1]{https://v2.ocaml.org/api/Printexc.html}
\end{frame}

\newcommand\listCamlRaiseExn[1][]{
  \listasm[firstline=702,lastline=725,#1]{../ocaml/runtime/amd64.S}{runtime/amd64.S:702}}

\begin{frame}{Backtraces}{Walk through \funcname{caml\_raise\_exn}}
  \begin{columns}[T]
    \begin{column}{0.5\textwidth}
      \begin{itemize}
        \item<1-> First checks whether exceptions backtrace should be recorded
        \item<1-> By checking the \funcarg{domain\_state->backtrace\_active} value
        \item<2-> If not, acts as \funcname{raise\_notrace} with \funcname{RESTORE\_EXN\_HANDLER\_OCAML}
          \begin{itemize}
            \item Set \regname{RSP} to \localname{domain\_state->exn\_handler} value
            \item Restore \localname{domain\_state->exn\_handler} from trap
            \item Jump with \funcname{ret} (same effect as using \funcname{pop} and \funcname{jmp})
          \end{itemize}
        \item<3-> Otherwise, switches to the C stack to be able to execute C code
        \item<4-> Calls \funcname{caml\_stash\_backtrace}, a C function provided by the runtime\ignorespaces
          \onslide<6->{, with
            \begin{itemize}
              \item \funcarg{exn}: exception value
              \item \funcarg{pc}: code address of the raising point
              \item \funcarg{sp}: stack address of the raising function
              \item \funcarg{trapsp}: stack address of the next handler
            \end{itemize}
          }
      \end{itemize}
      \bigskip
      \mintocaml[firstline=9,lastline=13,highlightlines=12]{../src/backtrace/backtrace.ml}
    \end{column}
    \begin{column}{0.5\textwidth}
      \raggedleft
      \onslide*<1,3-4>{
        \mintc[firstline=190,lastline=192]{../ocaml/runtime/amd64.S}
        \mintc[firstline=234,lastline=237]{../ocaml/runtime/amd64.S}
      }
      \onslide*<2>{
        \mintc[firstline=190,lastline=192,highlightlines={191-192}]{../ocaml/runtime/amd64.S}
        \mintc[firstline=234,lastline=237,highlightlines={235-237}]{../ocaml/runtime/amd64.S}
      }
      \onslide*<1>{\listCamlRaiseExn[highlightlines=705]{}}%
      \onslide*<2>{\listCamlRaiseExn[highlightlines={707-708}]{}}%
      \onslide*<3>{\listCamlRaiseExn[highlightlines={712-714}]{}}%
      \onslide*<4>{\listCamlRaiseExn[highlightlines={715-720}]{}}%
      \onslide*<5>{\providecommand\step{1}\input{diagrams/backtrace/backtrace.tex}}%
      \onslide*<6>{\providecommand\step{2}\input{diagrams/backtrace/backtrace.tex}}%
    \end{column}
  \end{columns}
\end{frame}

\newcommand\listStashBacktrace[1][]{
  \listc[firstline=91,lastline=120,#1]{../ocaml/runtime/backtrace\_nat.c}{runtime/backtrace\_nat.c:91}}

\begin{frame}{Backtraces}{Introducing \funcname{caml\_stash\_backtrace}}
  \begin{columns}[c]
    \begin{column}{0.5\textwidth}
      \minipage[c][0.66\textheight][s]{\columnwidth}
      \begin{itemize}
        \item<1-> A C function provided by the runtime (specific to native code)
        \item<2-> It scans the stack, between the stack pointer at the raising point up to the next trap, in order to collect the names of the detected function frames
          \onslide*<2>{
            \begin{itemize}
              \item And more information, like line number, column number...
            \end{itemize}
          }
        \item<3-> It uses the same technique and information as the Garbage Collector (GC) uses to scan the values live on the stack
          \onslide*<4>{
            \begin{itemize}
              \item \typename{frame\_descr} are at the core of stack unwinding
            \end{itemize}
          }
      \end{itemize}
      \vfill
      \mintocaml[firstline=9,lastline=13,highlightlines=12]{../src/backtrace/backtrace.ml}
      \endminipage
    \end{column}
    \begin{column}{0.5\textwidth}
      \onslide*<1>{\listStashBacktrace[highlightlines=91]{}}
      \onslide*<2>{\listStashBacktrace[highlightlines={91,117-118}]{}}
      \onslide*<3->{\listStashBacktrace[highlightlines={94,106,109-110}]{}}
    \end{column}
  \end{columns}
\end{frame}

\newcommand\listFrameDescr[1][]{
  \listocaml[firstline=111,lastline=124,#1]{../ocaml/asmcomp/emitaux.ml}{asmcomp/emitaux.ml:111}}
\newcommand\listDebugInfo[1][]{
  \listocaml[firstline=38,lastline=49,#1]{../ocaml/lambda/debuginfo.mli}{lambda/debuginfo.mli:38}}

\begin{frame}{Backtraces}{\typename{frame\_descr} and \typename{frame\_debuginfo} in a nutshell}
  \begin{columns}[c]
    \begin{column}{0.5\textwidth}
      \begin{itemize}
        \item<1-> For every return address in an OCaml program, there is a associated \typename{frame\_descr}
        \item<2-> A \typename{frame\_descr} specifies
        \begin{itemize}
          \item<2-> The size of the function stack frame
          \item<3-> Where live values are on the stack (so that the GC can mark them as live and prevent from being swept)
        \end{itemize}
        \item<4-> When compiled with debug information, \typename{frame\_descr} will be linked to a \typename{frame\_debuginfo}
        \begin{itemize}
          \item<5-> Contains information about the position in the source code (file name, line and column number)
        \end{itemize}
      \end{itemize}
    \end{column}
    \begin{column}{0.5\textwidth}
        \onslide*<1>{\listFrameDescr[highlightlines={118-122}]{}}
        \onslide*<2>{\listFrameDescr[highlightlines={120}]{}}
        \onslide*<3>{\listFrameDescr[highlightlines={121}]{}}
        \onslide*<4->{\listFrameDescr[highlightlines={122}]{}}
        \medskip
        \onslide*<1-3>{\listDebugInfo{}}
        \onslide*<4>{\listDebugInfo[highlightlines={38,49}]{}}
        \onslide*<5>{\listDebugInfo[highlightlines={39-46}]{}}
    \end{column}
  \end{columns}
\end{frame}

\begin{frame}{Backtraces}{Scanning the stack with \typename{frame\_descr}}
  \begin{columns}[c]
    \begin{column}{0.5\textwidth}
      \begin{itemize}
        \item Given the return address of the code that raises the exception by calling \funcname{caml\_raise\_exn} (\funcarg{pc} argument of \funcname{caml\_stash\_backtrace})
        \item And the position of the stack pointer (\funcarg{sp} argument of \funcname{caml\_stash\_backtrace})
        \item The frame size given by \typename{frame\_descr}.\funcarg{frame\_size}, allows to find the end of the previous frame
        \item And the return address of the caller of this function
        \bigskip
        \onslide<3->{
        \item Note that traps (here the reddish \funcname{ExnA}, \funcname{ExnB} and \funcname{ExnC} blocks) belong to the frame that installed them, and so the frame size changes when traps are removed
        }
      \end{itemize}
    \end{column}
    \begin{column}{0.5\textwidth}
      \raggedleft
      \onslide*<1>{\providecommand\step{2}\input{diagrams/backtrace/backtrace.tex}}%
      \onslide*<2>{\providecommand\step{1}\input{diagrams/backtrace/scanstack.tex}}%
      \onslide*<3>{\providecommand\step{2}\input{diagrams/backtrace/scanstack.tex}}%
      \onslide*<4>{\providecommand\step{3}\input{diagrams/backtrace/scanstack.tex}}%
      \onslide*<5>{\providecommand\step{4}\input{diagrams/backtrace/scanstack.tex}}%
      \onslide*<6>{\providecommand\step{5}\input{diagrams/backtrace/scanstack.tex}}%
    \end{column}
  \end{columns}
\end{frame}

% Duplicate of previous-previous slide
\begin{frame}{Backtraces}{Continuing on \funcname{caml\_stash\_backtrace}}
  \begin{columns}[c]
    \begin{column}{0.5\textwidth}
      \minipage[c][0.75\textheight][s]{\columnwidth}
      \begin{itemize}
          \color{gray}
        \item A C function provided by the runtime (specific to native code)
        \item It scans the stack, between the stack pointer at the raising point up to the next trap, in order to collect the names of the detected function frames
        \item It uses the same technique and information as the Garbage Collector (GC) uses to scan the values live on the stack
      \end{itemize}
      \begin{itemize}
        \item For each function frame detected, store a pointer to the \typename{frame\_descr} in the \localname{domain\_state->backtrace\_buffer} array, effectively constructing the backtrace
      \end{itemize}
      \onslide<2->{
        \begin{itemize}
            \smallskip
          \item Constructed backtrace:
            \funcname{definitely\_raise},
            \onslide<3->{\funcname{probably\_raise}}
        \end{itemize}
      }
      \vfill
      \mintocaml[firstline=9,lastline=13,highlightlines=12]{../src/backtrace/backtrace.ml}
      \endminipage
    \end{column}
    \begin{column}{0.5\textwidth}
      \raggedleft
      \onslide*<1>{\listStashBacktrace[highlightlines={114-115}]{}}
      \onslide*<2>{\providecommand\step{3}\input{diagrams/backtrace/backtrace.tex}}%
      \onslide*<3>{\providecommand\step{4}\input{diagrams/backtrace/backtrace.tex}}%
    \end{column}
  \end{columns}
\end{frame}

\begin{frame}{Backtraces}{Back to \funcname{caml\_raise\_exn}}
  \begin{columns}[c]
    \begin{column}{0.5\textwidth}
      \minipage[c][0.66\textheight][s]{\columnwidth}
      \begin{itemize}\color{gray}
        \item Checks wheteher exceptions backtrace should be recorded, by checking the \funcarg{domain\_state->backtrace\_active} value
        \item Switches to the C stack to be able to execute C code
        \item Calls \funcname{caml\_stash\_backtrace}, to collect the backtrace up to the next trap
      \end{itemize}
      \onslide*<2->{
        \begin{itemize}
          \item<2-> Executes the exception handler in \funcname{probably\_raise} with \funcname{RESTORE\_EXN\_HANDLER\_OCAML}
            \begin{itemize}
              \item Set \regname{RSP} to \localname{domain\_state->exn\_handler} value
              \item Restore \localname{domain\_state->exn\_handler} from trap
              \item Jump to the handler code with \funcname{ret}
            \end{itemize}
        \end{itemize}
      }
      \vfill
      \onslide*<1>{\mintocaml[firstline=9,lastline=13,highlightlines=12]{../src/backtrace/backtrace.ml}}
      \onslide*<2->{\mintocaml[firstline=15,lastline=21,highlightlines={19-21}]{../src/backtrace/backtrace.ml}}
      \endminipage
    \end{column}
    \begin{column}{0.5\textwidth}
      \centering
      \onslide*<1>{
        \mintc[firstline=190,lastline=192]{../ocaml/runtime/amd64.S}
        \mintc[firstline=234,lastline=237]{../ocaml/runtime/amd64.S}
      }
      \onslide*<2>{
        \mintc[firstline=190,lastline=192,highlightlines={191-192}]{../ocaml/runtime/amd64.S}
        \mintc[firstline=234,lastline=237,highlightlines={235-237}]{../ocaml/runtime/amd64.S}
      }
      \onslide*<1>{\listCamlRaiseExn[highlightlines=720]{}}
      \onslide*<2>{\listCamlRaiseExn[highlightlines=722]{}}
      \onslide*<3->{\providecommand\step{5}\input{diagrams/backtrace/backtrace.tex}}
    \end{column}
  \end{columns}
\end{frame}

\begin{frame}{Backtraces}{\funcname{caml\_probably\_raise}'s handler doesn't catch \typename{ExnC}}
  \begin{columns}[c]
    \begin{column}{0.45\textwidth}
      \minipage[c][0.75\textheight][s]{\columnwidth}
      \begin{itemize}
        \item<1-> \funcname{match \dots with}\footnotemark pattern matching can also match exceptions
        \item<1-> The CMM of \funcname{probably\_raise} shows that this \funcname{match \dots with} is implemented with a pattern matching and a \funcname{try \dots with}
        \item<1-> But this one only matches on \typename{ExnA} exception
        \item<2-> Since the pattern matching fails to catch the exception, \funcname{reraise} is called with our current \typename{ExnC} exception
        \item<3-> Disassembly shows that \funcname{reraise} is actually a call to \funcname{caml\_reraise\_exn}, which is also an assembly function provided by the runtime
      \end{itemize}
      \vfill
      \mintocaml[firstline=15,lastline=21,highlightlines={18-21}]{../src/backtrace/backtrace.ml}
      \endminipage
    \end{column}
    \begin{column}{0.55\textwidth}
      \centering
      \onslide*<1>{\mintcmm[highlightlines={10-18}]{../src/backtrace/camlBacktrace\_\_probably\_raise\_351.cmm}}
      \onslide*<2>{\listcmm[highlightlines=18]{../src/backtrace/camlBacktrace\_\_probably\_raise_351.cmm}{CMM of probably\_raise}}
      \onslide*<3>{\mintobjdump[firstline=5,lastline=35,highlightlines=31]{../src/backtrace/camlBacktrace\_\_probably\_raise_351.objdump}}
    \end{column}
  \end{columns}
  \only<1>{\footnotetext[1]{https://blog.janestreet.com/pattern-matching-and-exception-handling-unite/}}
\end{frame}

\newcommand\listCamlReraiseExn[1][]{
  \listasm[firstline=727,lastline=734,#1]{../ocaml/runtime/amd64.S}{runtime/amd64.S:727}}

\begin{frame}{Backtraces}{Introducing \funcname{caml\_reraise\_exn}}
  \begin{columns}[c]
    \begin{column}{0.5\textwidth}
      \minipage[c][0.66\textheight][s]{\columnwidth}
      \begin{itemize}
        \item<1-> The exception have to be reraised to the next handler
        \item<2-> First, checks if the backtrace should be collected with \funcarg{domain\_state->backtrace\_active}
        \only<3>{
        \begin{itemize}
            \item If not, behaves like a \funcname{raise\_notrace} with \funcname{RESTORE\_EXN\_HANDLER\_OCAML}
        \end{itemize}
        }
        \item<4-> Jumps into \funcname{caml\_raise\_exn}
        \item<5-> Calls \funcname{caml\_stash\_backtrace} again, to scan the next part of the stack: from the trap just pop-ed to the next trap
      \end{itemize}
      \vfill
      \mintocaml[firstline=15,lastline=21,highlightlines={18-21}]{../src/backtrace/backtrace.ml}
      \endminipage
    \end{column}
    \begin{column}{0.5\textwidth}
      \raggedleft
      \onslide*<1>{\listCamlReraiseExn{}}
      \onslide*<2>{\listCamlReraiseExn[highlightlines=729]{}}
      \onslide*<3>{
        \mintc[firstline=190,lastline=192,highlightlines={191-192}]{../ocaml/runtime/amd64.S}
        \mintc[firstline=234,lastline=237,highlightlines={235-237}]{../ocaml/runtime/amd64.S}
        \medskip
        \listCamlReraiseExn[highlightlines=731]{}
      }
      \onslide*<4-5>{
        % TODO refactor with \listCamlReraiseExn
        \mintasm[firstline=727,lastline=734,highlightlines=730]{../ocaml/runtime/amd64.S}
        \medskip
      }
      \onslide*<4>{\listCamlRaiseExn[highlightlines=711]{}}
      \onslide*<5>{\listCamlRaiseExn[highlightlines=720]{}}
    \end{column}
  \end{columns}
\end{frame}

\begin{frame}{Backtraces}{Backtrace collection continues}
  \begin{columns}[c]
    \begin{column}{0.55\textwidth}
      \minipage[c][0.75\textheight][s]{\columnwidth}
      \begin{itemize}
        \item<1-> \funcname{caml\_stash\_backtrace} scans the older part of the stack, up to the next trap
        \item<6-> Controls jumps to the nested exception handler in \funcname{maybe\_raise}, which matches only on \typename{ExnB}
        \item<7-> \funcname{caml\_reraise\_exn} is called again, backtrace collection resumes with \funcname{caml\_stash\_backtrace}
      \end{itemize}
      \medskip
      \begin{itemize}
        \item<2-> Constructed backtrace:
        \begin{itemize}
          \item <2-> \funcname{definitely\_raise}
          \item <2-> \funcname{probably\_raise}
          \item <3-> \funcname{probably\_raise} (re-raise)
          \item <4-> \funcname{maybe\_raise}
          \item <5-> \funcname{main}
          \item <8-> \funcname{main} (re-raise)
        \end{itemize}
      \end{itemize}
      \vfill
      \onslide*<1-5>{\mintocaml[firstline=15,lastline=21]{../src/backtrace/backtrace.ml}}
      \onslide*<6>{\mintocaml[firstline=28,lastline=34,highlightlines=31]{../src/backtrace/backtrace.ml}}
      \onslide*<7->{\mintocaml[firstline=28,lastline=34,highlightlines=33]{../src/backtrace/backtrace.ml}}
      \endminipage
    \end{column}
    \begin{column}{0.45\textwidth}
      \raggedleft
      \onslide*<1>{\providecommand\step{6}\input{diagrams/backtrace/backtrace.tex}}%
      \onslide*<2>{\providecommand\step{6}\input{diagrams/backtrace/backtrace.tex}}%
      \onslide*<3>{\providecommand\step{7}\input{diagrams/backtrace/backtrace.tex}}%
      \onslide*<4>{\providecommand\step{8}\input{diagrams/backtrace/backtrace.tex}}%
      \onslide*<5>{\providecommand\step{9}\input{diagrams/backtrace/backtrace.tex}}%
      \onslide*<6>{\providecommand\step{10}\input{diagrams/backtrace/backtrace.tex}}%
      \onslide*<7>{\providecommand\step{11}\input{diagrams/backtrace/backtrace.tex}}%
      \onslide*<8>{\providecommand\step{12}\input{diagrams/backtrace/backtrace.tex}}%
      \onslide*<9>{\providecommand\step{13}\input{diagrams/backtrace/backtrace.tex}}%
    \end{column}
  \end{columns}
\end{frame}

\begin{frame}{Backtraces}{Printing the backtrace}
  \begin{columns}[c]
    \begin{column}{0.45\textwidth}
      \minipage[c][0.66\textheight][s]{\columnwidth}
      \begin{itemize}
        \item<1-> The exception backtrace can now be printed to \funcarg{stdout} with \funcname{Printexc.print_backtrace}
        \item<2-> First the exception backtrace is retrieved into an OCaml array with \funcname{get_raw_backtrace }
        \item<3-> Which is implemented as the \funcname{caml_get_exception_raw_backtrace} C function in the runtime
        \item<4-> And finally printed with \funcname{print_exception_backtrace}
      \end{itemize}
      \vfill
      \mintocaml[firstline=28,lastline=34,highlightlines=33]{../src/backtrace/backtrace.ml}
      \endminipage
    \end{column}
    \begin{column}{0.55\textwidth}
      \onslide*<1,2>{
        \mintocaml[firstline=100,lastline=101]{../ocaml/stdlib/printexc.ml}
      }
      \only<1>{
        \listocaml[firstline=170,lastline=172]{../ocaml/stdlib/printexc.ml}{stdlib/printexc.ml:170}
      }
      \only<2>{
        \listocaml[firstline=170,lastline=172,highlightlines=172]{../ocaml/stdlib/printexc.ml}{stdlib/printexc.ml:170}
      }
      \only<3>{
        \mintc[firstline=154,lastline=156]{../ocaml/runtime/backtrace.c}
        \listc[firstline=171,lastline=192,highlightlines={184-187}]{../ocaml/runtime/backtrace.c}{runtime/backtrace.c:154}
      }
      \only<4>{
        \listocaml[firstline=155,lastline=165]{../ocaml/stdlib/printexc.ml}{stdlib/printexc.ml:55}
      }
    \end{column}
  \end{columns}
\end{frame}


\subsection{The multiple kinds of raise}
\frameSubsection{}{}

\begin{frame}{Backtraces}{When backtraces are not needed}
  \begin{columns}[c]
    \begin{column}{0.45\textwidth}
      \minipage[c][0.66\textheight][s]{\columnwidth}
      \begin{itemize}
        \item<1-> What if the backtrace for an exception is never needed?
        \item<2-> It's possible to raise the exception explicitly with \funcname{raise\_notrace} to make raising as cheap as possible
        \item<3-> An example of use in the \funcname{dynlink} library of the OCaml compiler
      \end{itemize}
      \vfill
      \onslide<3->{
      \listocaml[firstline=205,lastline=209,highlightlines=207]{../ocaml/otherlibs/dynlink/dynlink\_compilerlibs/misc.ml}{otherlibs/dynlink/dynlink\_compilerlibs/misc.ml:205}
      }
      \endminipage
    \end{column}
    \begin{column}{0.55\textwidth}
    \only<4>{
      \mintcmm[highlightlines=3]{../src/notrace/camlNotrace\_\_fun_347.cmm}
      \listcmm{../src/notrace/camlNotrace\_\_all\_somes\_327.cmm}{all\_somes}
    }
    \onslide*<5->{
      \listobjdump[highlightlines={6-9}]{../src/notrace/camlNotrace\_\_fun_347.objdump}{anonymous function from \funcname{all\_somes}}
    }
    \end{column}
  \end{columns}
\end{frame}

\begin{frame}{Backtraces}{Summary table of the multiple kinds of raise}
  \begin{itemize}
    \item When debugging information is enabled and if the backtrace of an exception isn't needed, it's possible to raise with \funcname{raise\_notrace} for performance
    \item When debugging information is \emph{not} enabled, the compiler optimises \funcname{raise} calls to inline assembly (same effect as using \funcname{raise\_notrace})
  \end{itemize}
  \bigskip
  \centering
  \begin{tabular}{| l | c | c | c | c |}
    \hline
    \parbox{10em}{Debug information \\ (\funcarg{-g} flag)}  & \multicolumn{2}{|c|}{Disabled}                                     & \multicolumn{2}{|c|}{Enabled} \\
    \hline
    Raise kind                                               & \funcname{raise} & \funcname{raise\_notrace}                       & \funcname{raise} & \funcname{raise\_notrace} \\
    \hline
    Compilation                                              & \parbox{8em}{inline assembly\\ (optimization)} & inline assembly   & \funcname{caml\_raise\_exn} & inline assembly \\
    \hline
  \end{tabular}
\end{frame}


%
% Takeaway #5
%
\subsection*{Takeaway \#5}
\frameSubsectionTakeaway{}
\againframe<5>{takeaway}
}%
      \onslide*<8>{\providecommand\step{12}%
% Backtraces
%
\subsection{One advantage of raising with debugging information}
\frameSubsection{}{}

\begin{frame}{Backtraces}{Debugging information \& source code}
  \begin{columns}[c]
    \begin{column}{0.5\textwidth}
      \begin{itemize}
        \item Raising an exception is fun and games, but how do we know where the exception originated? $\rightarrow$ With the backtrace associated with the exception!
        \item In order to get a backtrace from the point where an exception was raised, it's necessary to compile with debugging information enabled (\funcarg{-g} flag for the compiler)
        \item Same example as in the `Nested handlers' section
      \end{itemize}
    \end{column}
    \begin{column}{0.5\textwidth}
      \listocaml[firstline=9,lastline=34]{../src/backtrace/backtrace.ml}{backtrace/backtrace.ml}
    \end{column}
  \end{columns}
\end{frame}

\begin{frame}{Backtraces}{CMM and disassembly of \funcname{definitely\_raise}}
  \begin{columns}[c]
    \begin{column}{0.5\textwidth}
      \minipage[c][0.66\textheight][s]{\columnwidth}
      \begin{itemize}
        \item<1-> An effect of compiling with debugging information (\funcarg{-g}): the CMM no longer uses \funcname{raise\_notrace} to raise the exception but \funcname{raise} instead
        \item<2-> Disassembly shows that CMM \funcname{raise} is actually a call to \funcname{caml\_raise\_exn}, a function in assembler provided by the runtime
      \end{itemize}
      \vfill
      \mintocaml[firstline=9,lastline=13,highlightlines=12]{../src/backtrace/backtrace.ml}
      \endminipage
    \end{column}
    \begin{column}{0.5\textwidth}
      \onslide*<1>{
        \mintcmm[highlightlines=5]{../src/backtrace/camlBacktrace\_\_definitely\_raise\_347.cmm}
      }
      \onslide*<2>{
        \mintobjdump[firstline=2,highlightlines=14]{../src/backtrace/camlBacktrace\_\_definitely\_raise\_347.objdump}
      }
    \end{column}
  \end{columns}
\end{frame}

\begin{frame}{Backtraces}{Introducing \funcname{caml\_raise\_exn}}
  \begin{columns}[c]
    \begin{column}{0.5\textwidth}
      \minipage[c][0.66\textheight][s]{\columnwidth}
      \onslide*<1>{
        \begin{itemize}
          \item Raising the exception is performed using \funcname{caml\_raise\_exn} which is
            \begin{itemize}
              \item An assembly function provided by the runtime
              \item Architecture specific
            \end{itemize}
          \item Let's see what it does differently from \funcname{raise\_notrace}\dots
          \item We are about to raise \typename{ExnC} with \funcname{caml\_raise\_exn} from \funcname{definitely\_raise}
        \end{itemize}
      }
      \onslide*<2>{
        \mintobjdump[firstline=2,highlightlines=14]{../src/backtrace/camlBacktrace\_\_definitely\_raise\_347.objdump}
      }
      \vfill
      \mintocaml[firstline=9,lastline=13,highlightlines=12]{../src/backtrace/backtrace.ml}
      \endminipage
    \end{column}
    \begin{column}{0.5\textwidth}
      \centering
      \providecommand\step{1}\input{diagrams/backtrace/backtrace.tex}
    \end{column}
  \end{columns}
\end{frame}

\begin{frame}{Backtraces}{But before that, we need more fields from \typename{caml\_domain\_state}}
  \begin{columns}
    \begin{column}{0.5\textwidth}
      \begin{itemize}
        \item Remember \typename{caml\_domain\_state} the per-domain runtime state?
        \item Some other members are of interest to us now\dots
        \item \localname{backtrace\_active} is used to know if the runtime should collect backtraces
        \item \localname{backtrace\_active} can be set by
          \begin{itemize}
            \item The \funcarg{b} flag in the \funcname{OCAMLRUNPARAM} environment variable
            \item \mintinline{ocaml}{Printexc.record_backtrace : bool -> unit} \footnotemark
          \end{itemize}
      \end{itemize}
    \end{column}
    \begin{column}{0.5\textwidth}
      \begin{listing}
        \mintc[highlightlines={9-11}]{src/ocaml/domain\_state\_backtrace.h}
        \caption{runtime/caml/domain\_state.h}
      \end{listing}
    \end{column}
  \end{columns}
  \footnotetext[1]{https://v2.ocaml.org/api/Printexc.html}
\end{frame}

\newcommand\listCamlRaiseExn[1][]{
  \listasm[firstline=702,lastline=725,#1]{../ocaml/runtime/amd64.S}{runtime/amd64.S:702}}

\begin{frame}{Backtraces}{Walk through \funcname{caml\_raise\_exn}}
  \begin{columns}[T]
    \begin{column}{0.5\textwidth}
      \begin{itemize}
        \item<1-> First checks whether exceptions backtrace should be recorded
        \item<1-> By checking the \funcarg{domain\_state->backtrace\_active} value
        \item<2-> If not, acts as \funcname{raise\_notrace} with \funcname{RESTORE\_EXN\_HANDLER\_OCAML}
          \begin{itemize}
            \item Set \regname{RSP} to \localname{domain\_state->exn\_handler} value
            \item Restore \localname{domain\_state->exn\_handler} from trap
            \item Jump with \funcname{ret} (same effect as using \funcname{pop} and \funcname{jmp})
          \end{itemize}
        \item<3-> Otherwise, switches to the C stack to be able to execute C code
        \item<4-> Calls \funcname{caml\_stash\_backtrace}, a C function provided by the runtime\ignorespaces
          \onslide<6->{, with
            \begin{itemize}
              \item \funcarg{exn}: exception value
              \item \funcarg{pc}: code address of the raising point
              \item \funcarg{sp}: stack address of the raising function
              \item \funcarg{trapsp}: stack address of the next handler
            \end{itemize}
          }
      \end{itemize}
      \bigskip
      \mintocaml[firstline=9,lastline=13,highlightlines=12]{../src/backtrace/backtrace.ml}
    \end{column}
    \begin{column}{0.5\textwidth}
      \raggedleft
      \onslide*<1,3-4>{
        \mintc[firstline=190,lastline=192]{../ocaml/runtime/amd64.S}
        \mintc[firstline=234,lastline=237]{../ocaml/runtime/amd64.S}
      }
      \onslide*<2>{
        \mintc[firstline=190,lastline=192,highlightlines={191-192}]{../ocaml/runtime/amd64.S}
        \mintc[firstline=234,lastline=237,highlightlines={235-237}]{../ocaml/runtime/amd64.S}
      }
      \onslide*<1>{\listCamlRaiseExn[highlightlines=705]{}}%
      \onslide*<2>{\listCamlRaiseExn[highlightlines={707-708}]{}}%
      \onslide*<3>{\listCamlRaiseExn[highlightlines={712-714}]{}}%
      \onslide*<4>{\listCamlRaiseExn[highlightlines={715-720}]{}}%
      \onslide*<5>{\providecommand\step{1}\input{diagrams/backtrace/backtrace.tex}}%
      \onslide*<6>{\providecommand\step{2}\input{diagrams/backtrace/backtrace.tex}}%
    \end{column}
  \end{columns}
\end{frame}

\newcommand\listStashBacktrace[1][]{
  \listc[firstline=91,lastline=120,#1]{../ocaml/runtime/backtrace\_nat.c}{runtime/backtrace\_nat.c:91}}

\begin{frame}{Backtraces}{Introducing \funcname{caml\_stash\_backtrace}}
  \begin{columns}[c]
    \begin{column}{0.5\textwidth}
      \minipage[c][0.66\textheight][s]{\columnwidth}
      \begin{itemize}
        \item<1-> A C function provided by the runtime (specific to native code)
        \item<2-> It scans the stack, between the stack pointer at the raising point up to the next trap, in order to collect the names of the detected function frames
          \onslide*<2>{
            \begin{itemize}
              \item And more information, like line number, column number...
            \end{itemize}
          }
        \item<3-> It uses the same technique and information as the Garbage Collector (GC) uses to scan the values live on the stack
          \onslide*<4>{
            \begin{itemize}
              \item \typename{frame\_descr} are at the core of stack unwinding
            \end{itemize}
          }
      \end{itemize}
      \vfill
      \mintocaml[firstline=9,lastline=13,highlightlines=12]{../src/backtrace/backtrace.ml}
      \endminipage
    \end{column}
    \begin{column}{0.5\textwidth}
      \onslide*<1>{\listStashBacktrace[highlightlines=91]{}}
      \onslide*<2>{\listStashBacktrace[highlightlines={91,117-118}]{}}
      \onslide*<3->{\listStashBacktrace[highlightlines={94,106,109-110}]{}}
    \end{column}
  \end{columns}
\end{frame}

\newcommand\listFrameDescr[1][]{
  \listocaml[firstline=111,lastline=124,#1]{../ocaml/asmcomp/emitaux.ml}{asmcomp/emitaux.ml:111}}
\newcommand\listDebugInfo[1][]{
  \listocaml[firstline=38,lastline=49,#1]{../ocaml/lambda/debuginfo.mli}{lambda/debuginfo.mli:38}}

\begin{frame}{Backtraces}{\typename{frame\_descr} and \typename{frame\_debuginfo} in a nutshell}
  \begin{columns}[c]
    \begin{column}{0.5\textwidth}
      \begin{itemize}
        \item<1-> For every return address in an OCaml program, there is a associated \typename{frame\_descr}
        \item<2-> A \typename{frame\_descr} specifies
        \begin{itemize}
          \item<2-> The size of the function stack frame
          \item<3-> Where live values are on the stack (so that the GC can mark them as live and prevent from being swept)
        \end{itemize}
        \item<4-> When compiled with debug information, \typename{frame\_descr} will be linked to a \typename{frame\_debuginfo}
        \begin{itemize}
          \item<5-> Contains information about the position in the source code (file name, line and column number)
        \end{itemize}
      \end{itemize}
    \end{column}
    \begin{column}{0.5\textwidth}
        \onslide*<1>{\listFrameDescr[highlightlines={118-122}]{}}
        \onslide*<2>{\listFrameDescr[highlightlines={120}]{}}
        \onslide*<3>{\listFrameDescr[highlightlines={121}]{}}
        \onslide*<4->{\listFrameDescr[highlightlines={122}]{}}
        \medskip
        \onslide*<1-3>{\listDebugInfo{}}
        \onslide*<4>{\listDebugInfo[highlightlines={38,49}]{}}
        \onslide*<5>{\listDebugInfo[highlightlines={39-46}]{}}
    \end{column}
  \end{columns}
\end{frame}

\begin{frame}{Backtraces}{Scanning the stack with \typename{frame\_descr}}
  \begin{columns}[c]
    \begin{column}{0.5\textwidth}
      \begin{itemize}
        \item Given the return address of the code that raises the exception by calling \funcname{caml\_raise\_exn} (\funcarg{pc} argument of \funcname{caml\_stash\_backtrace})
        \item And the position of the stack pointer (\funcarg{sp} argument of \funcname{caml\_stash\_backtrace})
        \item The frame size given by \typename{frame\_descr}.\funcarg{frame\_size}, allows to find the end of the previous frame
        \item And the return address of the caller of this function
        \bigskip
        \onslide<3->{
        \item Note that traps (here the reddish \funcname{ExnA}, \funcname{ExnB} and \funcname{ExnC} blocks) belong to the frame that installed them, and so the frame size changes when traps are removed
        }
      \end{itemize}
    \end{column}
    \begin{column}{0.5\textwidth}
      \raggedleft
      \onslide*<1>{\providecommand\step{2}\input{diagrams/backtrace/backtrace.tex}}%
      \onslide*<2>{\providecommand\step{1}\input{diagrams/backtrace/scanstack.tex}}%
      \onslide*<3>{\providecommand\step{2}\input{diagrams/backtrace/scanstack.tex}}%
      \onslide*<4>{\providecommand\step{3}\input{diagrams/backtrace/scanstack.tex}}%
      \onslide*<5>{\providecommand\step{4}\input{diagrams/backtrace/scanstack.tex}}%
      \onslide*<6>{\providecommand\step{5}\input{diagrams/backtrace/scanstack.tex}}%
    \end{column}
  \end{columns}
\end{frame}

% Duplicate of previous-previous slide
\begin{frame}{Backtraces}{Continuing on \funcname{caml\_stash\_backtrace}}
  \begin{columns}[c]
    \begin{column}{0.5\textwidth}
      \minipage[c][0.75\textheight][s]{\columnwidth}
      \begin{itemize}
          \color{gray}
        \item A C function provided by the runtime (specific to native code)
        \item It scans the stack, between the stack pointer at the raising point up to the next trap, in order to collect the names of the detected function frames
        \item It uses the same technique and information as the Garbage Collector (GC) uses to scan the values live on the stack
      \end{itemize}
      \begin{itemize}
        \item For each function frame detected, store a pointer to the \typename{frame\_descr} in the \localname{domain\_state->backtrace\_buffer} array, effectively constructing the backtrace
      \end{itemize}
      \onslide<2->{
        \begin{itemize}
            \smallskip
          \item Constructed backtrace:
            \funcname{definitely\_raise},
            \onslide<3->{\funcname{probably\_raise}}
        \end{itemize}
      }
      \vfill
      \mintocaml[firstline=9,lastline=13,highlightlines=12]{../src/backtrace/backtrace.ml}
      \endminipage
    \end{column}
    \begin{column}{0.5\textwidth}
      \raggedleft
      \onslide*<1>{\listStashBacktrace[highlightlines={114-115}]{}}
      \onslide*<2>{\providecommand\step{3}\input{diagrams/backtrace/backtrace.tex}}%
      \onslide*<3>{\providecommand\step{4}\input{diagrams/backtrace/backtrace.tex}}%
    \end{column}
  \end{columns}
\end{frame}

\begin{frame}{Backtraces}{Back to \funcname{caml\_raise\_exn}}
  \begin{columns}[c]
    \begin{column}{0.5\textwidth}
      \minipage[c][0.66\textheight][s]{\columnwidth}
      \begin{itemize}\color{gray}
        \item Checks wheteher exceptions backtrace should be recorded, by checking the \funcarg{domain\_state->backtrace\_active} value
        \item Switches to the C stack to be able to execute C code
        \item Calls \funcname{caml\_stash\_backtrace}, to collect the backtrace up to the next trap
      \end{itemize}
      \onslide*<2->{
        \begin{itemize}
          \item<2-> Executes the exception handler in \funcname{probably\_raise} with \funcname{RESTORE\_EXN\_HANDLER\_OCAML}
            \begin{itemize}
              \item Set \regname{RSP} to \localname{domain\_state->exn\_handler} value
              \item Restore \localname{domain\_state->exn\_handler} from trap
              \item Jump to the handler code with \funcname{ret}
            \end{itemize}
        \end{itemize}
      }
      \vfill
      \onslide*<1>{\mintocaml[firstline=9,lastline=13,highlightlines=12]{../src/backtrace/backtrace.ml}}
      \onslide*<2->{\mintocaml[firstline=15,lastline=21,highlightlines={19-21}]{../src/backtrace/backtrace.ml}}
      \endminipage
    \end{column}
    \begin{column}{0.5\textwidth}
      \centering
      \onslide*<1>{
        \mintc[firstline=190,lastline=192]{../ocaml/runtime/amd64.S}
        \mintc[firstline=234,lastline=237]{../ocaml/runtime/amd64.S}
      }
      \onslide*<2>{
        \mintc[firstline=190,lastline=192,highlightlines={191-192}]{../ocaml/runtime/amd64.S}
        \mintc[firstline=234,lastline=237,highlightlines={235-237}]{../ocaml/runtime/amd64.S}
      }
      \onslide*<1>{\listCamlRaiseExn[highlightlines=720]{}}
      \onslide*<2>{\listCamlRaiseExn[highlightlines=722]{}}
      \onslide*<3->{\providecommand\step{5}\input{diagrams/backtrace/backtrace.tex}}
    \end{column}
  \end{columns}
\end{frame}

\begin{frame}{Backtraces}{\funcname{caml\_probably\_raise}'s handler doesn't catch \typename{ExnC}}
  \begin{columns}[c]
    \begin{column}{0.45\textwidth}
      \minipage[c][0.75\textheight][s]{\columnwidth}
      \begin{itemize}
        \item<1-> \funcname{match \dots with}\footnotemark pattern matching can also match exceptions
        \item<1-> The CMM of \funcname{probably\_raise} shows that this \funcname{match \dots with} is implemented with a pattern matching and a \funcname{try \dots with}
        \item<1-> But this one only matches on \typename{ExnA} exception
        \item<2-> Since the pattern matching fails to catch the exception, \funcname{reraise} is called with our current \typename{ExnC} exception
        \item<3-> Disassembly shows that \funcname{reraise} is actually a call to \funcname{caml\_reraise\_exn}, which is also an assembly function provided by the runtime
      \end{itemize}
      \vfill
      \mintocaml[firstline=15,lastline=21,highlightlines={18-21}]{../src/backtrace/backtrace.ml}
      \endminipage
    \end{column}
    \begin{column}{0.55\textwidth}
      \centering
      \onslide*<1>{\mintcmm[highlightlines={10-18}]{../src/backtrace/camlBacktrace\_\_probably\_raise\_351.cmm}}
      \onslide*<2>{\listcmm[highlightlines=18]{../src/backtrace/camlBacktrace\_\_probably\_raise_351.cmm}{CMM of probably\_raise}}
      \onslide*<3>{\mintobjdump[firstline=5,lastline=35,highlightlines=31]{../src/backtrace/camlBacktrace\_\_probably\_raise_351.objdump}}
    \end{column}
  \end{columns}
  \only<1>{\footnotetext[1]{https://blog.janestreet.com/pattern-matching-and-exception-handling-unite/}}
\end{frame}

\newcommand\listCamlReraiseExn[1][]{
  \listasm[firstline=727,lastline=734,#1]{../ocaml/runtime/amd64.S}{runtime/amd64.S:727}}

\begin{frame}{Backtraces}{Introducing \funcname{caml\_reraise\_exn}}
  \begin{columns}[c]
    \begin{column}{0.5\textwidth}
      \minipage[c][0.66\textheight][s]{\columnwidth}
      \begin{itemize}
        \item<1-> The exception have to be reraised to the next handler
        \item<2-> First, checks if the backtrace should be collected with \funcarg{domain\_state->backtrace\_active}
        \only<3>{
        \begin{itemize}
            \item If not, behaves like a \funcname{raise\_notrace} with \funcname{RESTORE\_EXN\_HANDLER\_OCAML}
        \end{itemize}
        }
        \item<4-> Jumps into \funcname{caml\_raise\_exn}
        \item<5-> Calls \funcname{caml\_stash\_backtrace} again, to scan the next part of the stack: from the trap just pop-ed to the next trap
      \end{itemize}
      \vfill
      \mintocaml[firstline=15,lastline=21,highlightlines={18-21}]{../src/backtrace/backtrace.ml}
      \endminipage
    \end{column}
    \begin{column}{0.5\textwidth}
      \raggedleft
      \onslide*<1>{\listCamlReraiseExn{}}
      \onslide*<2>{\listCamlReraiseExn[highlightlines=729]{}}
      \onslide*<3>{
        \mintc[firstline=190,lastline=192,highlightlines={191-192}]{../ocaml/runtime/amd64.S}
        \mintc[firstline=234,lastline=237,highlightlines={235-237}]{../ocaml/runtime/amd64.S}
        \medskip
        \listCamlReraiseExn[highlightlines=731]{}
      }
      \onslide*<4-5>{
        % TODO refactor with \listCamlReraiseExn
        \mintasm[firstline=727,lastline=734,highlightlines=730]{../ocaml/runtime/amd64.S}
        \medskip
      }
      \onslide*<4>{\listCamlRaiseExn[highlightlines=711]{}}
      \onslide*<5>{\listCamlRaiseExn[highlightlines=720]{}}
    \end{column}
  \end{columns}
\end{frame}

\begin{frame}{Backtraces}{Backtrace collection continues}
  \begin{columns}[c]
    \begin{column}{0.55\textwidth}
      \minipage[c][0.75\textheight][s]{\columnwidth}
      \begin{itemize}
        \item<1-> \funcname{caml\_stash\_backtrace} scans the older part of the stack, up to the next trap
        \item<6-> Controls jumps to the nested exception handler in \funcname{maybe\_raise}, which matches only on \typename{ExnB}
        \item<7-> \funcname{caml\_reraise\_exn} is called again, backtrace collection resumes with \funcname{caml\_stash\_backtrace}
      \end{itemize}
      \medskip
      \begin{itemize}
        \item<2-> Constructed backtrace:
        \begin{itemize}
          \item <2-> \funcname{definitely\_raise}
          \item <2-> \funcname{probably\_raise}
          \item <3-> \funcname{probably\_raise} (re-raise)
          \item <4-> \funcname{maybe\_raise}
          \item <5-> \funcname{main}
          \item <8-> \funcname{main} (re-raise)
        \end{itemize}
      \end{itemize}
      \vfill
      \onslide*<1-5>{\mintocaml[firstline=15,lastline=21]{../src/backtrace/backtrace.ml}}
      \onslide*<6>{\mintocaml[firstline=28,lastline=34,highlightlines=31]{../src/backtrace/backtrace.ml}}
      \onslide*<7->{\mintocaml[firstline=28,lastline=34,highlightlines=33]{../src/backtrace/backtrace.ml}}
      \endminipage
    \end{column}
    \begin{column}{0.45\textwidth}
      \raggedleft
      \onslide*<1>{\providecommand\step{6}\input{diagrams/backtrace/backtrace.tex}}%
      \onslide*<2>{\providecommand\step{6}\input{diagrams/backtrace/backtrace.tex}}%
      \onslide*<3>{\providecommand\step{7}\input{diagrams/backtrace/backtrace.tex}}%
      \onslide*<4>{\providecommand\step{8}\input{diagrams/backtrace/backtrace.tex}}%
      \onslide*<5>{\providecommand\step{9}\input{diagrams/backtrace/backtrace.tex}}%
      \onslide*<6>{\providecommand\step{10}\input{diagrams/backtrace/backtrace.tex}}%
      \onslide*<7>{\providecommand\step{11}\input{diagrams/backtrace/backtrace.tex}}%
      \onslide*<8>{\providecommand\step{12}\input{diagrams/backtrace/backtrace.tex}}%
      \onslide*<9>{\providecommand\step{13}\input{diagrams/backtrace/backtrace.tex}}%
    \end{column}
  \end{columns}
\end{frame}

\begin{frame}{Backtraces}{Printing the backtrace}
  \begin{columns}[c]
    \begin{column}{0.45\textwidth}
      \minipage[c][0.66\textheight][s]{\columnwidth}
      \begin{itemize}
        \item<1-> The exception backtrace can now be printed to \funcarg{stdout} with \funcname{Printexc.print_backtrace}
        \item<2-> First the exception backtrace is retrieved into an OCaml array with \funcname{get_raw_backtrace }
        \item<3-> Which is implemented as the \funcname{caml_get_exception_raw_backtrace} C function in the runtime
        \item<4-> And finally printed with \funcname{print_exception_backtrace}
      \end{itemize}
      \vfill
      \mintocaml[firstline=28,lastline=34,highlightlines=33]{../src/backtrace/backtrace.ml}
      \endminipage
    \end{column}
    \begin{column}{0.55\textwidth}
      \onslide*<1,2>{
        \mintocaml[firstline=100,lastline=101]{../ocaml/stdlib/printexc.ml}
      }
      \only<1>{
        \listocaml[firstline=170,lastline=172]{../ocaml/stdlib/printexc.ml}{stdlib/printexc.ml:170}
      }
      \only<2>{
        \listocaml[firstline=170,lastline=172,highlightlines=172]{../ocaml/stdlib/printexc.ml}{stdlib/printexc.ml:170}
      }
      \only<3>{
        \mintc[firstline=154,lastline=156]{../ocaml/runtime/backtrace.c}
        \listc[firstline=171,lastline=192,highlightlines={184-187}]{../ocaml/runtime/backtrace.c}{runtime/backtrace.c:154}
      }
      \only<4>{
        \listocaml[firstline=155,lastline=165]{../ocaml/stdlib/printexc.ml}{stdlib/printexc.ml:55}
      }
    \end{column}
  \end{columns}
\end{frame}


\subsection{The multiple kinds of raise}
\frameSubsection{}{}

\begin{frame}{Backtraces}{When backtraces are not needed}
  \begin{columns}[c]
    \begin{column}{0.45\textwidth}
      \minipage[c][0.66\textheight][s]{\columnwidth}
      \begin{itemize}
        \item<1-> What if the backtrace for an exception is never needed?
        \item<2-> It's possible to raise the exception explicitly with \funcname{raise\_notrace} to make raising as cheap as possible
        \item<3-> An example of use in the \funcname{dynlink} library of the OCaml compiler
      \end{itemize}
      \vfill
      \onslide<3->{
      \listocaml[firstline=205,lastline=209,highlightlines=207]{../ocaml/otherlibs/dynlink/dynlink\_compilerlibs/misc.ml}{otherlibs/dynlink/dynlink\_compilerlibs/misc.ml:205}
      }
      \endminipage
    \end{column}
    \begin{column}{0.55\textwidth}
    \only<4>{
      \mintcmm[highlightlines=3]{../src/notrace/camlNotrace\_\_fun_347.cmm}
      \listcmm{../src/notrace/camlNotrace\_\_all\_somes\_327.cmm}{all\_somes}
    }
    \onslide*<5->{
      \listobjdump[highlightlines={6-9}]{../src/notrace/camlNotrace\_\_fun_347.objdump}{anonymous function from \funcname{all\_somes}}
    }
    \end{column}
  \end{columns}
\end{frame}

\begin{frame}{Backtraces}{Summary table of the multiple kinds of raise}
  \begin{itemize}
    \item When debugging information is enabled and if the backtrace of an exception isn't needed, it's possible to raise with \funcname{raise\_notrace} for performance
    \item When debugging information is \emph{not} enabled, the compiler optimises \funcname{raise} calls to inline assembly (same effect as using \funcname{raise\_notrace})
  \end{itemize}
  \bigskip
  \centering
  \begin{tabular}{| l | c | c | c | c |}
    \hline
    \parbox{10em}{Debug information \\ (\funcarg{-g} flag)}  & \multicolumn{2}{|c|}{Disabled}                                     & \multicolumn{2}{|c|}{Enabled} \\
    \hline
    Raise kind                                               & \funcname{raise} & \funcname{raise\_notrace}                       & \funcname{raise} & \funcname{raise\_notrace} \\
    \hline
    Compilation                                              & \parbox{8em}{inline assembly\\ (optimization)} & inline assembly   & \funcname{caml\_raise\_exn} & inline assembly \\
    \hline
  \end{tabular}
\end{frame}


%
% Takeaway #5
%
\subsection*{Takeaway \#5}
\frameSubsectionTakeaway{}
\againframe<5>{takeaway}
}%
      \onslide*<9>{\providecommand\step{13}%
% Backtraces
%
\subsection{One advantage of raising with debugging information}
\frameSubsection{}{}

\begin{frame}{Backtraces}{Debugging information \& source code}
  \begin{columns}[c]
    \begin{column}{0.5\textwidth}
      \begin{itemize}
        \item Raising an exception is fun and games, but how do we know where the exception originated? $\rightarrow$ With the backtrace associated with the exception!
        \item In order to get a backtrace from the point where an exception was raised, it's necessary to compile with debugging information enabled (\funcarg{-g} flag for the compiler)
        \item Same example as in the `Nested handlers' section
      \end{itemize}
    \end{column}
    \begin{column}{0.5\textwidth}
      \listocaml[firstline=9,lastline=34]{../src/backtrace/backtrace.ml}{backtrace/backtrace.ml}
    \end{column}
  \end{columns}
\end{frame}

\begin{frame}{Backtraces}{CMM and disassembly of \funcname{definitely\_raise}}
  \begin{columns}[c]
    \begin{column}{0.5\textwidth}
      \minipage[c][0.66\textheight][s]{\columnwidth}
      \begin{itemize}
        \item<1-> An effect of compiling with debugging information (\funcarg{-g}): the CMM no longer uses \funcname{raise\_notrace} to raise the exception but \funcname{raise} instead
        \item<2-> Disassembly shows that CMM \funcname{raise} is actually a call to \funcname{caml\_raise\_exn}, a function in assembler provided by the runtime
      \end{itemize}
      \vfill
      \mintocaml[firstline=9,lastline=13,highlightlines=12]{../src/backtrace/backtrace.ml}
      \endminipage
    \end{column}
    \begin{column}{0.5\textwidth}
      \onslide*<1>{
        \mintcmm[highlightlines=5]{../src/backtrace/camlBacktrace\_\_definitely\_raise\_347.cmm}
      }
      \onslide*<2>{
        \mintobjdump[firstline=2,highlightlines=14]{../src/backtrace/camlBacktrace\_\_definitely\_raise\_347.objdump}
      }
    \end{column}
  \end{columns}
\end{frame}

\begin{frame}{Backtraces}{Introducing \funcname{caml\_raise\_exn}}
  \begin{columns}[c]
    \begin{column}{0.5\textwidth}
      \minipage[c][0.66\textheight][s]{\columnwidth}
      \onslide*<1>{
        \begin{itemize}
          \item Raising the exception is performed using \funcname{caml\_raise\_exn} which is
            \begin{itemize}
              \item An assembly function provided by the runtime
              \item Architecture specific
            \end{itemize}
          \item Let's see what it does differently from \funcname{raise\_notrace}\dots
          \item We are about to raise \typename{ExnC} with \funcname{caml\_raise\_exn} from \funcname{definitely\_raise}
        \end{itemize}
      }
      \onslide*<2>{
        \mintobjdump[firstline=2,highlightlines=14]{../src/backtrace/camlBacktrace\_\_definitely\_raise\_347.objdump}
      }
      \vfill
      \mintocaml[firstline=9,lastline=13,highlightlines=12]{../src/backtrace/backtrace.ml}
      \endminipage
    \end{column}
    \begin{column}{0.5\textwidth}
      \centering
      \providecommand\step{1}\input{diagrams/backtrace/backtrace.tex}
    \end{column}
  \end{columns}
\end{frame}

\begin{frame}{Backtraces}{But before that, we need more fields from \typename{caml\_domain\_state}}
  \begin{columns}
    \begin{column}{0.5\textwidth}
      \begin{itemize}
        \item Remember \typename{caml\_domain\_state} the per-domain runtime state?
        \item Some other members are of interest to us now\dots
        \item \localname{backtrace\_active} is used to know if the runtime should collect backtraces
        \item \localname{backtrace\_active} can be set by
          \begin{itemize}
            \item The \funcarg{b} flag in the \funcname{OCAMLRUNPARAM} environment variable
            \item \mintinline{ocaml}{Printexc.record_backtrace : bool -> unit} \footnotemark
          \end{itemize}
      \end{itemize}
    \end{column}
    \begin{column}{0.5\textwidth}
      \begin{listing}
        \mintc[highlightlines={9-11}]{src/ocaml/domain\_state\_backtrace.h}
        \caption{runtime/caml/domain\_state.h}
      \end{listing}
    \end{column}
  \end{columns}
  \footnotetext[1]{https://v2.ocaml.org/api/Printexc.html}
\end{frame}

\newcommand\listCamlRaiseExn[1][]{
  \listasm[firstline=702,lastline=725,#1]{../ocaml/runtime/amd64.S}{runtime/amd64.S:702}}

\begin{frame}{Backtraces}{Walk through \funcname{caml\_raise\_exn}}
  \begin{columns}[T]
    \begin{column}{0.5\textwidth}
      \begin{itemize}
        \item<1-> First checks whether exceptions backtrace should be recorded
        \item<1-> By checking the \funcarg{domain\_state->backtrace\_active} value
        \item<2-> If not, acts as \funcname{raise\_notrace} with \funcname{RESTORE\_EXN\_HANDLER\_OCAML}
          \begin{itemize}
            \item Set \regname{RSP} to \localname{domain\_state->exn\_handler} value
            \item Restore \localname{domain\_state->exn\_handler} from trap
            \item Jump with \funcname{ret} (same effect as using \funcname{pop} and \funcname{jmp})
          \end{itemize}
        \item<3-> Otherwise, switches to the C stack to be able to execute C code
        \item<4-> Calls \funcname{caml\_stash\_backtrace}, a C function provided by the runtime\ignorespaces
          \onslide<6->{, with
            \begin{itemize}
              \item \funcarg{exn}: exception value
              \item \funcarg{pc}: code address of the raising point
              \item \funcarg{sp}: stack address of the raising function
              \item \funcarg{trapsp}: stack address of the next handler
            \end{itemize}
          }
      \end{itemize}
      \bigskip
      \mintocaml[firstline=9,lastline=13,highlightlines=12]{../src/backtrace/backtrace.ml}
    \end{column}
    \begin{column}{0.5\textwidth}
      \raggedleft
      \onslide*<1,3-4>{
        \mintc[firstline=190,lastline=192]{../ocaml/runtime/amd64.S}
        \mintc[firstline=234,lastline=237]{../ocaml/runtime/amd64.S}
      }
      \onslide*<2>{
        \mintc[firstline=190,lastline=192,highlightlines={191-192}]{../ocaml/runtime/amd64.S}
        \mintc[firstline=234,lastline=237,highlightlines={235-237}]{../ocaml/runtime/amd64.S}
      }
      \onslide*<1>{\listCamlRaiseExn[highlightlines=705]{}}%
      \onslide*<2>{\listCamlRaiseExn[highlightlines={707-708}]{}}%
      \onslide*<3>{\listCamlRaiseExn[highlightlines={712-714}]{}}%
      \onslide*<4>{\listCamlRaiseExn[highlightlines={715-720}]{}}%
      \onslide*<5>{\providecommand\step{1}\input{diagrams/backtrace/backtrace.tex}}%
      \onslide*<6>{\providecommand\step{2}\input{diagrams/backtrace/backtrace.tex}}%
    \end{column}
  \end{columns}
\end{frame}

\newcommand\listStashBacktrace[1][]{
  \listc[firstline=91,lastline=120,#1]{../ocaml/runtime/backtrace\_nat.c}{runtime/backtrace\_nat.c:91}}

\begin{frame}{Backtraces}{Introducing \funcname{caml\_stash\_backtrace}}
  \begin{columns}[c]
    \begin{column}{0.5\textwidth}
      \minipage[c][0.66\textheight][s]{\columnwidth}
      \begin{itemize}
        \item<1-> A C function provided by the runtime (specific to native code)
        \item<2-> It scans the stack, between the stack pointer at the raising point up to the next trap, in order to collect the names of the detected function frames
          \onslide*<2>{
            \begin{itemize}
              \item And more information, like line number, column number...
            \end{itemize}
          }
        \item<3-> It uses the same technique and information as the Garbage Collector (GC) uses to scan the values live on the stack
          \onslide*<4>{
            \begin{itemize}
              \item \typename{frame\_descr} are at the core of stack unwinding
            \end{itemize}
          }
      \end{itemize}
      \vfill
      \mintocaml[firstline=9,lastline=13,highlightlines=12]{../src/backtrace/backtrace.ml}
      \endminipage
    \end{column}
    \begin{column}{0.5\textwidth}
      \onslide*<1>{\listStashBacktrace[highlightlines=91]{}}
      \onslide*<2>{\listStashBacktrace[highlightlines={91,117-118}]{}}
      \onslide*<3->{\listStashBacktrace[highlightlines={94,106,109-110}]{}}
    \end{column}
  \end{columns}
\end{frame}

\newcommand\listFrameDescr[1][]{
  \listocaml[firstline=111,lastline=124,#1]{../ocaml/asmcomp/emitaux.ml}{asmcomp/emitaux.ml:111}}
\newcommand\listDebugInfo[1][]{
  \listocaml[firstline=38,lastline=49,#1]{../ocaml/lambda/debuginfo.mli}{lambda/debuginfo.mli:38}}

\begin{frame}{Backtraces}{\typename{frame\_descr} and \typename{frame\_debuginfo} in a nutshell}
  \begin{columns}[c]
    \begin{column}{0.5\textwidth}
      \begin{itemize}
        \item<1-> For every return address in an OCaml program, there is a associated \typename{frame\_descr}
        \item<2-> A \typename{frame\_descr} specifies
        \begin{itemize}
          \item<2-> The size of the function stack frame
          \item<3-> Where live values are on the stack (so that the GC can mark them as live and prevent from being swept)
        \end{itemize}
        \item<4-> When compiled with debug information, \typename{frame\_descr} will be linked to a \typename{frame\_debuginfo}
        \begin{itemize}
          \item<5-> Contains information about the position in the source code (file name, line and column number)
        \end{itemize}
      \end{itemize}
    \end{column}
    \begin{column}{0.5\textwidth}
        \onslide*<1>{\listFrameDescr[highlightlines={118-122}]{}}
        \onslide*<2>{\listFrameDescr[highlightlines={120}]{}}
        \onslide*<3>{\listFrameDescr[highlightlines={121}]{}}
        \onslide*<4->{\listFrameDescr[highlightlines={122}]{}}
        \medskip
        \onslide*<1-3>{\listDebugInfo{}}
        \onslide*<4>{\listDebugInfo[highlightlines={38,49}]{}}
        \onslide*<5>{\listDebugInfo[highlightlines={39-46}]{}}
    \end{column}
  \end{columns}
\end{frame}

\begin{frame}{Backtraces}{Scanning the stack with \typename{frame\_descr}}
  \begin{columns}[c]
    \begin{column}{0.5\textwidth}
      \begin{itemize}
        \item Given the return address of the code that raises the exception by calling \funcname{caml\_raise\_exn} (\funcarg{pc} argument of \funcname{caml\_stash\_backtrace})
        \item And the position of the stack pointer (\funcarg{sp} argument of \funcname{caml\_stash\_backtrace})
        \item The frame size given by \typename{frame\_descr}.\funcarg{frame\_size}, allows to find the end of the previous frame
        \item And the return address of the caller of this function
        \bigskip
        \onslide<3->{
        \item Note that traps (here the reddish \funcname{ExnA}, \funcname{ExnB} and \funcname{ExnC} blocks) belong to the frame that installed them, and so the frame size changes when traps are removed
        }
      \end{itemize}
    \end{column}
    \begin{column}{0.5\textwidth}
      \raggedleft
      \onslide*<1>{\providecommand\step{2}\input{diagrams/backtrace/backtrace.tex}}%
      \onslide*<2>{\providecommand\step{1}\input{diagrams/backtrace/scanstack.tex}}%
      \onslide*<3>{\providecommand\step{2}\input{diagrams/backtrace/scanstack.tex}}%
      \onslide*<4>{\providecommand\step{3}\input{diagrams/backtrace/scanstack.tex}}%
      \onslide*<5>{\providecommand\step{4}\input{diagrams/backtrace/scanstack.tex}}%
      \onslide*<6>{\providecommand\step{5}\input{diagrams/backtrace/scanstack.tex}}%
    \end{column}
  \end{columns}
\end{frame}

% Duplicate of previous-previous slide
\begin{frame}{Backtraces}{Continuing on \funcname{caml\_stash\_backtrace}}
  \begin{columns}[c]
    \begin{column}{0.5\textwidth}
      \minipage[c][0.75\textheight][s]{\columnwidth}
      \begin{itemize}
          \color{gray}
        \item A C function provided by the runtime (specific to native code)
        \item It scans the stack, between the stack pointer at the raising point up to the next trap, in order to collect the names of the detected function frames
        \item It uses the same technique and information as the Garbage Collector (GC) uses to scan the values live on the stack
      \end{itemize}
      \begin{itemize}
        \item For each function frame detected, store a pointer to the \typename{frame\_descr} in the \localname{domain\_state->backtrace\_buffer} array, effectively constructing the backtrace
      \end{itemize}
      \onslide<2->{
        \begin{itemize}
            \smallskip
          \item Constructed backtrace:
            \funcname{definitely\_raise},
            \onslide<3->{\funcname{probably\_raise}}
        \end{itemize}
      }
      \vfill
      \mintocaml[firstline=9,lastline=13,highlightlines=12]{../src/backtrace/backtrace.ml}
      \endminipage
    \end{column}
    \begin{column}{0.5\textwidth}
      \raggedleft
      \onslide*<1>{\listStashBacktrace[highlightlines={114-115}]{}}
      \onslide*<2>{\providecommand\step{3}\input{diagrams/backtrace/backtrace.tex}}%
      \onslide*<3>{\providecommand\step{4}\input{diagrams/backtrace/backtrace.tex}}%
    \end{column}
  \end{columns}
\end{frame}

\begin{frame}{Backtraces}{Back to \funcname{caml\_raise\_exn}}
  \begin{columns}[c]
    \begin{column}{0.5\textwidth}
      \minipage[c][0.66\textheight][s]{\columnwidth}
      \begin{itemize}\color{gray}
        \item Checks wheteher exceptions backtrace should be recorded, by checking the \funcarg{domain\_state->backtrace\_active} value
        \item Switches to the C stack to be able to execute C code
        \item Calls \funcname{caml\_stash\_backtrace}, to collect the backtrace up to the next trap
      \end{itemize}
      \onslide*<2->{
        \begin{itemize}
          \item<2-> Executes the exception handler in \funcname{probably\_raise} with \funcname{RESTORE\_EXN\_HANDLER\_OCAML}
            \begin{itemize}
              \item Set \regname{RSP} to \localname{domain\_state->exn\_handler} value
              \item Restore \localname{domain\_state->exn\_handler} from trap
              \item Jump to the handler code with \funcname{ret}
            \end{itemize}
        \end{itemize}
      }
      \vfill
      \onslide*<1>{\mintocaml[firstline=9,lastline=13,highlightlines=12]{../src/backtrace/backtrace.ml}}
      \onslide*<2->{\mintocaml[firstline=15,lastline=21,highlightlines={19-21}]{../src/backtrace/backtrace.ml}}
      \endminipage
    \end{column}
    \begin{column}{0.5\textwidth}
      \centering
      \onslide*<1>{
        \mintc[firstline=190,lastline=192]{../ocaml/runtime/amd64.S}
        \mintc[firstline=234,lastline=237]{../ocaml/runtime/amd64.S}
      }
      \onslide*<2>{
        \mintc[firstline=190,lastline=192,highlightlines={191-192}]{../ocaml/runtime/amd64.S}
        \mintc[firstline=234,lastline=237,highlightlines={235-237}]{../ocaml/runtime/amd64.S}
      }
      \onslide*<1>{\listCamlRaiseExn[highlightlines=720]{}}
      \onslide*<2>{\listCamlRaiseExn[highlightlines=722]{}}
      \onslide*<3->{\providecommand\step{5}\input{diagrams/backtrace/backtrace.tex}}
    \end{column}
  \end{columns}
\end{frame}

\begin{frame}{Backtraces}{\funcname{caml\_probably\_raise}'s handler doesn't catch \typename{ExnC}}
  \begin{columns}[c]
    \begin{column}{0.45\textwidth}
      \minipage[c][0.75\textheight][s]{\columnwidth}
      \begin{itemize}
        \item<1-> \funcname{match \dots with}\footnotemark pattern matching can also match exceptions
        \item<1-> The CMM of \funcname{probably\_raise} shows that this \funcname{match \dots with} is implemented with a pattern matching and a \funcname{try \dots with}
        \item<1-> But this one only matches on \typename{ExnA} exception
        \item<2-> Since the pattern matching fails to catch the exception, \funcname{reraise} is called with our current \typename{ExnC} exception
        \item<3-> Disassembly shows that \funcname{reraise} is actually a call to \funcname{caml\_reraise\_exn}, which is also an assembly function provided by the runtime
      \end{itemize}
      \vfill
      \mintocaml[firstline=15,lastline=21,highlightlines={18-21}]{../src/backtrace/backtrace.ml}
      \endminipage
    \end{column}
    \begin{column}{0.55\textwidth}
      \centering
      \onslide*<1>{\mintcmm[highlightlines={10-18}]{../src/backtrace/camlBacktrace\_\_probably\_raise\_351.cmm}}
      \onslide*<2>{\listcmm[highlightlines=18]{../src/backtrace/camlBacktrace\_\_probably\_raise_351.cmm}{CMM of probably\_raise}}
      \onslide*<3>{\mintobjdump[firstline=5,lastline=35,highlightlines=31]{../src/backtrace/camlBacktrace\_\_probably\_raise_351.objdump}}
    \end{column}
  \end{columns}
  \only<1>{\footnotetext[1]{https://blog.janestreet.com/pattern-matching-and-exception-handling-unite/}}
\end{frame}

\newcommand\listCamlReraiseExn[1][]{
  \listasm[firstline=727,lastline=734,#1]{../ocaml/runtime/amd64.S}{runtime/amd64.S:727}}

\begin{frame}{Backtraces}{Introducing \funcname{caml\_reraise\_exn}}
  \begin{columns}[c]
    \begin{column}{0.5\textwidth}
      \minipage[c][0.66\textheight][s]{\columnwidth}
      \begin{itemize}
        \item<1-> The exception have to be reraised to the next handler
        \item<2-> First, checks if the backtrace should be collected with \funcarg{domain\_state->backtrace\_active}
        \only<3>{
        \begin{itemize}
            \item If not, behaves like a \funcname{raise\_notrace} with \funcname{RESTORE\_EXN\_HANDLER\_OCAML}
        \end{itemize}
        }
        \item<4-> Jumps into \funcname{caml\_raise\_exn}
        \item<5-> Calls \funcname{caml\_stash\_backtrace} again, to scan the next part of the stack: from the trap just pop-ed to the next trap
      \end{itemize}
      \vfill
      \mintocaml[firstline=15,lastline=21,highlightlines={18-21}]{../src/backtrace/backtrace.ml}
      \endminipage
    \end{column}
    \begin{column}{0.5\textwidth}
      \raggedleft
      \onslide*<1>{\listCamlReraiseExn{}}
      \onslide*<2>{\listCamlReraiseExn[highlightlines=729]{}}
      \onslide*<3>{
        \mintc[firstline=190,lastline=192,highlightlines={191-192}]{../ocaml/runtime/amd64.S}
        \mintc[firstline=234,lastline=237,highlightlines={235-237}]{../ocaml/runtime/amd64.S}
        \medskip
        \listCamlReraiseExn[highlightlines=731]{}
      }
      \onslide*<4-5>{
        % TODO refactor with \listCamlReraiseExn
        \mintasm[firstline=727,lastline=734,highlightlines=730]{../ocaml/runtime/amd64.S}
        \medskip
      }
      \onslide*<4>{\listCamlRaiseExn[highlightlines=711]{}}
      \onslide*<5>{\listCamlRaiseExn[highlightlines=720]{}}
    \end{column}
  \end{columns}
\end{frame}

\begin{frame}{Backtraces}{Backtrace collection continues}
  \begin{columns}[c]
    \begin{column}{0.55\textwidth}
      \minipage[c][0.75\textheight][s]{\columnwidth}
      \begin{itemize}
        \item<1-> \funcname{caml\_stash\_backtrace} scans the older part of the stack, up to the next trap
        \item<6-> Controls jumps to the nested exception handler in \funcname{maybe\_raise}, which matches only on \typename{ExnB}
        \item<7-> \funcname{caml\_reraise\_exn} is called again, backtrace collection resumes with \funcname{caml\_stash\_backtrace}
      \end{itemize}
      \medskip
      \begin{itemize}
        \item<2-> Constructed backtrace:
        \begin{itemize}
          \item <2-> \funcname{definitely\_raise}
          \item <2-> \funcname{probably\_raise}
          \item <3-> \funcname{probably\_raise} (re-raise)
          \item <4-> \funcname{maybe\_raise}
          \item <5-> \funcname{main}
          \item <8-> \funcname{main} (re-raise)
        \end{itemize}
      \end{itemize}
      \vfill
      \onslide*<1-5>{\mintocaml[firstline=15,lastline=21]{../src/backtrace/backtrace.ml}}
      \onslide*<6>{\mintocaml[firstline=28,lastline=34,highlightlines=31]{../src/backtrace/backtrace.ml}}
      \onslide*<7->{\mintocaml[firstline=28,lastline=34,highlightlines=33]{../src/backtrace/backtrace.ml}}
      \endminipage
    \end{column}
    \begin{column}{0.45\textwidth}
      \raggedleft
      \onslide*<1>{\providecommand\step{6}\input{diagrams/backtrace/backtrace.tex}}%
      \onslide*<2>{\providecommand\step{6}\input{diagrams/backtrace/backtrace.tex}}%
      \onslide*<3>{\providecommand\step{7}\input{diagrams/backtrace/backtrace.tex}}%
      \onslide*<4>{\providecommand\step{8}\input{diagrams/backtrace/backtrace.tex}}%
      \onslide*<5>{\providecommand\step{9}\input{diagrams/backtrace/backtrace.tex}}%
      \onslide*<6>{\providecommand\step{10}\input{diagrams/backtrace/backtrace.tex}}%
      \onslide*<7>{\providecommand\step{11}\input{diagrams/backtrace/backtrace.tex}}%
      \onslide*<8>{\providecommand\step{12}\input{diagrams/backtrace/backtrace.tex}}%
      \onslide*<9>{\providecommand\step{13}\input{diagrams/backtrace/backtrace.tex}}%
    \end{column}
  \end{columns}
\end{frame}

\begin{frame}{Backtraces}{Printing the backtrace}
  \begin{columns}[c]
    \begin{column}{0.45\textwidth}
      \minipage[c][0.66\textheight][s]{\columnwidth}
      \begin{itemize}
        \item<1-> The exception backtrace can now be printed to \funcarg{stdout} with \funcname{Printexc.print_backtrace}
        \item<2-> First the exception backtrace is retrieved into an OCaml array with \funcname{get_raw_backtrace }
        \item<3-> Which is implemented as the \funcname{caml_get_exception_raw_backtrace} C function in the runtime
        \item<4-> And finally printed with \funcname{print_exception_backtrace}
      \end{itemize}
      \vfill
      \mintocaml[firstline=28,lastline=34,highlightlines=33]{../src/backtrace/backtrace.ml}
      \endminipage
    \end{column}
    \begin{column}{0.55\textwidth}
      \onslide*<1,2>{
        \mintocaml[firstline=100,lastline=101]{../ocaml/stdlib/printexc.ml}
      }
      \only<1>{
        \listocaml[firstline=170,lastline=172]{../ocaml/stdlib/printexc.ml}{stdlib/printexc.ml:170}
      }
      \only<2>{
        \listocaml[firstline=170,lastline=172,highlightlines=172]{../ocaml/stdlib/printexc.ml}{stdlib/printexc.ml:170}
      }
      \only<3>{
        \mintc[firstline=154,lastline=156]{../ocaml/runtime/backtrace.c}
        \listc[firstline=171,lastline=192,highlightlines={184-187}]{../ocaml/runtime/backtrace.c}{runtime/backtrace.c:154}
      }
      \only<4>{
        \listocaml[firstline=155,lastline=165]{../ocaml/stdlib/printexc.ml}{stdlib/printexc.ml:55}
      }
    \end{column}
  \end{columns}
\end{frame}


\subsection{The multiple kinds of raise}
\frameSubsection{}{}

\begin{frame}{Backtraces}{When backtraces are not needed}
  \begin{columns}[c]
    \begin{column}{0.45\textwidth}
      \minipage[c][0.66\textheight][s]{\columnwidth}
      \begin{itemize}
        \item<1-> What if the backtrace for an exception is never needed?
        \item<2-> It's possible to raise the exception explicitly with \funcname{raise\_notrace} to make raising as cheap as possible
        \item<3-> An example of use in the \funcname{dynlink} library of the OCaml compiler
      \end{itemize}
      \vfill
      \onslide<3->{
      \listocaml[firstline=205,lastline=209,highlightlines=207]{../ocaml/otherlibs/dynlink/dynlink\_compilerlibs/misc.ml}{otherlibs/dynlink/dynlink\_compilerlibs/misc.ml:205}
      }
      \endminipage
    \end{column}
    \begin{column}{0.55\textwidth}
    \only<4>{
      \mintcmm[highlightlines=3]{../src/notrace/camlNotrace\_\_fun_347.cmm}
      \listcmm{../src/notrace/camlNotrace\_\_all\_somes\_327.cmm}{all\_somes}
    }
    \onslide*<5->{
      \listobjdump[highlightlines={6-9}]{../src/notrace/camlNotrace\_\_fun_347.objdump}{anonymous function from \funcname{all\_somes}}
    }
    \end{column}
  \end{columns}
\end{frame}

\begin{frame}{Backtraces}{Summary table of the multiple kinds of raise}
  \begin{itemize}
    \item When debugging information is enabled and if the backtrace of an exception isn't needed, it's possible to raise with \funcname{raise\_notrace} for performance
    \item When debugging information is \emph{not} enabled, the compiler optimises \funcname{raise} calls to inline assembly (same effect as using \funcname{raise\_notrace})
  \end{itemize}
  \bigskip
  \centering
  \begin{tabular}{| l | c | c | c | c |}
    \hline
    \parbox{10em}{Debug information \\ (\funcarg{-g} flag)}  & \multicolumn{2}{|c|}{Disabled}                                     & \multicolumn{2}{|c|}{Enabled} \\
    \hline
    Raise kind                                               & \funcname{raise} & \funcname{raise\_notrace}                       & \funcname{raise} & \funcname{raise\_notrace} \\
    \hline
    Compilation                                              & \parbox{8em}{inline assembly\\ (optimization)} & inline assembly   & \funcname{caml\_raise\_exn} & inline assembly \\
    \hline
  \end{tabular}
\end{frame}


%
% Takeaway #5
%
\subsection*{Takeaway \#5}
\frameSubsectionTakeaway{}
\againframe<5>{takeaway}
}%
    \end{column}
  \end{columns}
\end{frame}

\begin{frame}{Backtraces}{Printing the backtrace}
  \begin{columns}[c]
    \begin{column}{0.45\textwidth}
      \minipage[c][0.66\textheight][s]{\columnwidth}
      \begin{itemize}
        \item<1-> The exception backtrace can now be printed to \funcarg{stdout} with \funcname{Printexc.print_backtrace}
        \item<2-> First the exception backtrace is retrieved into an OCaml array with \funcname{get_raw_backtrace }
        \item<3-> Which is implemented as the \funcname{caml_get_exception_raw_backtrace} C function in the runtime
        \item<4-> And finally printed with \funcname{print_exception_backtrace}
      \end{itemize}
      \vfill
      \mintocaml[firstline=28,lastline=34,highlightlines=33]{../src/backtrace/backtrace.ml}
      \endminipage
    \end{column}
    \begin{column}{0.55\textwidth}
      \onslide*<1,2>{
        \mintocaml[firstline=100,lastline=101]{../ocaml/stdlib/printexc.ml}
      }
      \only<1>{
        \listocaml[firstline=170,lastline=172]{../ocaml/stdlib/printexc.ml}{stdlib/printexc.ml:170}
      }
      \only<2>{
        \listocaml[firstline=170,lastline=172,highlightlines=172]{../ocaml/stdlib/printexc.ml}{stdlib/printexc.ml:170}
      }
      \only<3>{
        \mintc[firstline=154,lastline=156]{../ocaml/runtime/backtrace.c}
        \listc[firstline=171,lastline=192,highlightlines={184-187}]{../ocaml/runtime/backtrace.c}{runtime/backtrace.c:154}
      }
      \only<4>{
        \listocaml[firstline=155,lastline=165]{../ocaml/stdlib/printexc.ml}{stdlib/printexc.ml:55}
      }
    \end{column}
  \end{columns}
\end{frame}


\subsection{The multiple kinds of raise}
\frameSubsection{}{}

\begin{frame}{Backtraces}{When backtraces are not needed}
  \begin{columns}[c]
    \begin{column}{0.45\textwidth}
      \minipage[c][0.66\textheight][s]{\columnwidth}
      \begin{itemize}
        \item<1-> What if the backtrace for an exception is never needed?
        \item<2-> It's possible to raise the exception explicitly with \funcname{raise\_notrace} to make raising as cheap as possible
        \item<3-> An example of use in the \funcname{dynlink} library of the OCaml compiler
      \end{itemize}
      \vfill
      \onslide<3->{
      \listocaml[firstline=205,lastline=209,highlightlines=207]{../ocaml/otherlibs/dynlink/dynlink\_compilerlibs/misc.ml}{otherlibs/dynlink/dynlink\_compilerlibs/misc.ml:205}
      }
      \endminipage
    \end{column}
    \begin{column}{0.55\textwidth}
    \only<4>{
      \mintcmm[highlightlines=3]{../src/notrace/camlNotrace\_\_fun_347.cmm}
      \listcmm{../src/notrace/camlNotrace\_\_all\_somes\_327.cmm}{all\_somes}
    }
    \onslide*<5->{
      \listobjdump[highlightlines={6-9}]{../src/notrace/camlNotrace\_\_fun_347.objdump}{anonymous function from \funcname{all\_somes}}
    }
    \end{column}
  \end{columns}
\end{frame}

\begin{frame}{Backtraces}{Summary table of the multiple kinds of raise}
  \begin{itemize}
    \item When debugging information is enabled and if the backtrace of an exception isn't needed, it's possible to raise with \funcname{raise\_notrace} for performance
    \item When debugging information is \emph{not} enabled, the compiler optimises \funcname{raise} calls to inline assembly (same effect as using \funcname{raise\_notrace})
  \end{itemize}
  \bigskip
  \centering
  \begin{tabular}{| l | c | c | c | c |}
    \hline
    \parbox{10em}{Debug information \\ (\funcarg{-g} flag)}  & \multicolumn{2}{|c|}{Disabled}                                     & \multicolumn{2}{|c|}{Enabled} \\
    \hline
    Raise kind                                               & \funcname{raise} & \funcname{raise\_notrace}                       & \funcname{raise} & \funcname{raise\_notrace} \\
    \hline
    Compilation                                              & \parbox{8em}{inline assembly\\ (optimization)} & inline assembly   & \funcname{caml\_raise\_exn} & inline assembly \\
    \hline
  \end{tabular}
\end{frame}


%
% Takeaway #5
%
\subsection*{Takeaway \#5}
\frameSubsectionTakeaway{}
\againframe<5>{takeaway}
}%
      \onslide*<6>{\providecommand\step{10}%
% Backtraces
%
\subsection{One advantage of raising with debugging information}
\frameSubsection{}{}

\begin{frame}{Backtraces}{Debugging information \& source code}
  \begin{columns}[c]
    \begin{column}{0.5\textwidth}
      \begin{itemize}
        \item Raising an exception is fun and games, but how do we know where the exception originated? $\rightarrow$ With the backtrace associated with the exception!
        \item In order to get a backtrace from the point where an exception was raised, it's necessary to compile with debugging information enabled (\funcarg{-g} flag for the compiler)
        \item Same example as in the `Nested handlers' section
      \end{itemize}
    \end{column}
    \begin{column}{0.5\textwidth}
      \listocaml[firstline=9,lastline=34]{../src/backtrace/backtrace.ml}{backtrace/backtrace.ml}
    \end{column}
  \end{columns}
\end{frame}

\begin{frame}{Backtraces}{CMM and disassembly of \funcname{definitely\_raise}}
  \begin{columns}[c]
    \begin{column}{0.5\textwidth}
      \minipage[c][0.66\textheight][s]{\columnwidth}
      \begin{itemize}
        \item<1-> An effect of compiling with debugging information (\funcarg{-g}): the CMM no longer uses \funcname{raise\_notrace} to raise the exception but \funcname{raise} instead
        \item<2-> Disassembly shows that CMM \funcname{raise} is actually a call to \funcname{caml\_raise\_exn}, a function in assembler provided by the runtime
      \end{itemize}
      \vfill
      \mintocaml[firstline=9,lastline=13,highlightlines=12]{../src/backtrace/backtrace.ml}
      \endminipage
    \end{column}
    \begin{column}{0.5\textwidth}
      \onslide*<1>{
        \mintcmm[highlightlines=5]{../src/backtrace/camlBacktrace\_\_definitely\_raise\_347.cmm}
      }
      \onslide*<2>{
        \mintobjdump[firstline=2,highlightlines=14]{../src/backtrace/camlBacktrace\_\_definitely\_raise\_347.objdump}
      }
    \end{column}
  \end{columns}
\end{frame}

\begin{frame}{Backtraces}{Introducing \funcname{caml\_raise\_exn}}
  \begin{columns}[c]
    \begin{column}{0.5\textwidth}
      \minipage[c][0.66\textheight][s]{\columnwidth}
      \onslide*<1>{
        \begin{itemize}
          \item Raising the exception is performed using \funcname{caml\_raise\_exn} which is
            \begin{itemize}
              \item An assembly function provided by the runtime
              \item Architecture specific
            \end{itemize}
          \item Let's see what it does differently from \funcname{raise\_notrace}\dots
          \item We are about to raise \typename{ExnC} with \funcname{caml\_raise\_exn} from \funcname{definitely\_raise}
        \end{itemize}
      }
      \onslide*<2>{
        \mintobjdump[firstline=2,highlightlines=14]{../src/backtrace/camlBacktrace\_\_definitely\_raise\_347.objdump}
      }
      \vfill
      \mintocaml[firstline=9,lastline=13,highlightlines=12]{../src/backtrace/backtrace.ml}
      \endminipage
    \end{column}
    \begin{column}{0.5\textwidth}
      \centering
      \providecommand\step{1}%
% Backtraces
%
\subsection{One advantage of raising with debugging information}
\frameSubsection{}{}

\begin{frame}{Backtraces}{Debugging information \& source code}
  \begin{columns}[c]
    \begin{column}{0.5\textwidth}
      \begin{itemize}
        \item Raising an exception is fun and games, but how do we know where the exception originated? $\rightarrow$ With the backtrace associated with the exception!
        \item In order to get a backtrace from the point where an exception was raised, it's necessary to compile with debugging information enabled (\funcarg{-g} flag for the compiler)
        \item Same example as in the `Nested handlers' section
      \end{itemize}
    \end{column}
    \begin{column}{0.5\textwidth}
      \listocaml[firstline=9,lastline=34]{../src/backtrace/backtrace.ml}{backtrace/backtrace.ml}
    \end{column}
  \end{columns}
\end{frame}

\begin{frame}{Backtraces}{CMM and disassembly of \funcname{definitely\_raise}}
  \begin{columns}[c]
    \begin{column}{0.5\textwidth}
      \minipage[c][0.66\textheight][s]{\columnwidth}
      \begin{itemize}
        \item<1-> An effect of compiling with debugging information (\funcarg{-g}): the CMM no longer uses \funcname{raise\_notrace} to raise the exception but \funcname{raise} instead
        \item<2-> Disassembly shows that CMM \funcname{raise} is actually a call to \funcname{caml\_raise\_exn}, a function in assembler provided by the runtime
      \end{itemize}
      \vfill
      \mintocaml[firstline=9,lastline=13,highlightlines=12]{../src/backtrace/backtrace.ml}
      \endminipage
    \end{column}
    \begin{column}{0.5\textwidth}
      \onslide*<1>{
        \mintcmm[highlightlines=5]{../src/backtrace/camlBacktrace\_\_definitely\_raise\_347.cmm}
      }
      \onslide*<2>{
        \mintobjdump[firstline=2,highlightlines=14]{../src/backtrace/camlBacktrace\_\_definitely\_raise\_347.objdump}
      }
    \end{column}
  \end{columns}
\end{frame}

\begin{frame}{Backtraces}{Introducing \funcname{caml\_raise\_exn}}
  \begin{columns}[c]
    \begin{column}{0.5\textwidth}
      \minipage[c][0.66\textheight][s]{\columnwidth}
      \onslide*<1>{
        \begin{itemize}
          \item Raising the exception is performed using \funcname{caml\_raise\_exn} which is
            \begin{itemize}
              \item An assembly function provided by the runtime
              \item Architecture specific
            \end{itemize}
          \item Let's see what it does differently from \funcname{raise\_notrace}\dots
          \item We are about to raise \typename{ExnC} with \funcname{caml\_raise\_exn} from \funcname{definitely\_raise}
        \end{itemize}
      }
      \onslide*<2>{
        \mintobjdump[firstline=2,highlightlines=14]{../src/backtrace/camlBacktrace\_\_definitely\_raise\_347.objdump}
      }
      \vfill
      \mintocaml[firstline=9,lastline=13,highlightlines=12]{../src/backtrace/backtrace.ml}
      \endminipage
    \end{column}
    \begin{column}{0.5\textwidth}
      \centering
      \providecommand\step{1}\input{diagrams/backtrace/backtrace.tex}
    \end{column}
  \end{columns}
\end{frame}

\begin{frame}{Backtraces}{But before that, we need more fields from \typename{caml\_domain\_state}}
  \begin{columns}
    \begin{column}{0.5\textwidth}
      \begin{itemize}
        \item Remember \typename{caml\_domain\_state} the per-domain runtime state?
        \item Some other members are of interest to us now\dots
        \item \localname{backtrace\_active} is used to know if the runtime should collect backtraces
        \item \localname{backtrace\_active} can be set by
          \begin{itemize}
            \item The \funcarg{b} flag in the \funcname{OCAMLRUNPARAM} environment variable
            \item \mintinline{ocaml}{Printexc.record_backtrace : bool -> unit} \footnotemark
          \end{itemize}
      \end{itemize}
    \end{column}
    \begin{column}{0.5\textwidth}
      \begin{listing}
        \mintc[highlightlines={9-11}]{src/ocaml/domain\_state\_backtrace.h}
        \caption{runtime/caml/domain\_state.h}
      \end{listing}
    \end{column}
  \end{columns}
  \footnotetext[1]{https://v2.ocaml.org/api/Printexc.html}
\end{frame}

\newcommand\listCamlRaiseExn[1][]{
  \listasm[firstline=702,lastline=725,#1]{../ocaml/runtime/amd64.S}{runtime/amd64.S:702}}

\begin{frame}{Backtraces}{Walk through \funcname{caml\_raise\_exn}}
  \begin{columns}[T]
    \begin{column}{0.5\textwidth}
      \begin{itemize}
        \item<1-> First checks whether exceptions backtrace should be recorded
        \item<1-> By checking the \funcarg{domain\_state->backtrace\_active} value
        \item<2-> If not, acts as \funcname{raise\_notrace} with \funcname{RESTORE\_EXN\_HANDLER\_OCAML}
          \begin{itemize}
            \item Set \regname{RSP} to \localname{domain\_state->exn\_handler} value
            \item Restore \localname{domain\_state->exn\_handler} from trap
            \item Jump with \funcname{ret} (same effect as using \funcname{pop} and \funcname{jmp})
          \end{itemize}
        \item<3-> Otherwise, switches to the C stack to be able to execute C code
        \item<4-> Calls \funcname{caml\_stash\_backtrace}, a C function provided by the runtime\ignorespaces
          \onslide<6->{, with
            \begin{itemize}
              \item \funcarg{exn}: exception value
              \item \funcarg{pc}: code address of the raising point
              \item \funcarg{sp}: stack address of the raising function
              \item \funcarg{trapsp}: stack address of the next handler
            \end{itemize}
          }
      \end{itemize}
      \bigskip
      \mintocaml[firstline=9,lastline=13,highlightlines=12]{../src/backtrace/backtrace.ml}
    \end{column}
    \begin{column}{0.5\textwidth}
      \raggedleft
      \onslide*<1,3-4>{
        \mintc[firstline=190,lastline=192]{../ocaml/runtime/amd64.S}
        \mintc[firstline=234,lastline=237]{../ocaml/runtime/amd64.S}
      }
      \onslide*<2>{
        \mintc[firstline=190,lastline=192,highlightlines={191-192}]{../ocaml/runtime/amd64.S}
        \mintc[firstline=234,lastline=237,highlightlines={235-237}]{../ocaml/runtime/amd64.S}
      }
      \onslide*<1>{\listCamlRaiseExn[highlightlines=705]{}}%
      \onslide*<2>{\listCamlRaiseExn[highlightlines={707-708}]{}}%
      \onslide*<3>{\listCamlRaiseExn[highlightlines={712-714}]{}}%
      \onslide*<4>{\listCamlRaiseExn[highlightlines={715-720}]{}}%
      \onslide*<5>{\providecommand\step{1}\input{diagrams/backtrace/backtrace.tex}}%
      \onslide*<6>{\providecommand\step{2}\input{diagrams/backtrace/backtrace.tex}}%
    \end{column}
  \end{columns}
\end{frame}

\newcommand\listStashBacktrace[1][]{
  \listc[firstline=91,lastline=120,#1]{../ocaml/runtime/backtrace\_nat.c}{runtime/backtrace\_nat.c:91}}

\begin{frame}{Backtraces}{Introducing \funcname{caml\_stash\_backtrace}}
  \begin{columns}[c]
    \begin{column}{0.5\textwidth}
      \minipage[c][0.66\textheight][s]{\columnwidth}
      \begin{itemize}
        \item<1-> A C function provided by the runtime (specific to native code)
        \item<2-> It scans the stack, between the stack pointer at the raising point up to the next trap, in order to collect the names of the detected function frames
          \onslide*<2>{
            \begin{itemize}
              \item And more information, like line number, column number...
            \end{itemize}
          }
        \item<3-> It uses the same technique and information as the Garbage Collector (GC) uses to scan the values live on the stack
          \onslide*<4>{
            \begin{itemize}
              \item \typename{frame\_descr} are at the core of stack unwinding
            \end{itemize}
          }
      \end{itemize}
      \vfill
      \mintocaml[firstline=9,lastline=13,highlightlines=12]{../src/backtrace/backtrace.ml}
      \endminipage
    \end{column}
    \begin{column}{0.5\textwidth}
      \onslide*<1>{\listStashBacktrace[highlightlines=91]{}}
      \onslide*<2>{\listStashBacktrace[highlightlines={91,117-118}]{}}
      \onslide*<3->{\listStashBacktrace[highlightlines={94,106,109-110}]{}}
    \end{column}
  \end{columns}
\end{frame}

\newcommand\listFrameDescr[1][]{
  \listocaml[firstline=111,lastline=124,#1]{../ocaml/asmcomp/emitaux.ml}{asmcomp/emitaux.ml:111}}
\newcommand\listDebugInfo[1][]{
  \listocaml[firstline=38,lastline=49,#1]{../ocaml/lambda/debuginfo.mli}{lambda/debuginfo.mli:38}}

\begin{frame}{Backtraces}{\typename{frame\_descr} and \typename{frame\_debuginfo} in a nutshell}
  \begin{columns}[c]
    \begin{column}{0.5\textwidth}
      \begin{itemize}
        \item<1-> For every return address in an OCaml program, there is a associated \typename{frame\_descr}
        \item<2-> A \typename{frame\_descr} specifies
        \begin{itemize}
          \item<2-> The size of the function stack frame
          \item<3-> Where live values are on the stack (so that the GC can mark them as live and prevent from being swept)
        \end{itemize}
        \item<4-> When compiled with debug information, \typename{frame\_descr} will be linked to a \typename{frame\_debuginfo}
        \begin{itemize}
          \item<5-> Contains information about the position in the source code (file name, line and column number)
        \end{itemize}
      \end{itemize}
    \end{column}
    \begin{column}{0.5\textwidth}
        \onslide*<1>{\listFrameDescr[highlightlines={118-122}]{}}
        \onslide*<2>{\listFrameDescr[highlightlines={120}]{}}
        \onslide*<3>{\listFrameDescr[highlightlines={121}]{}}
        \onslide*<4->{\listFrameDescr[highlightlines={122}]{}}
        \medskip
        \onslide*<1-3>{\listDebugInfo{}}
        \onslide*<4>{\listDebugInfo[highlightlines={38,49}]{}}
        \onslide*<5>{\listDebugInfo[highlightlines={39-46}]{}}
    \end{column}
  \end{columns}
\end{frame}

\begin{frame}{Backtraces}{Scanning the stack with \typename{frame\_descr}}
  \begin{columns}[c]
    \begin{column}{0.5\textwidth}
      \begin{itemize}
        \item Given the return address of the code that raises the exception by calling \funcname{caml\_raise\_exn} (\funcarg{pc} argument of \funcname{caml\_stash\_backtrace})
        \item And the position of the stack pointer (\funcarg{sp} argument of \funcname{caml\_stash\_backtrace})
        \item The frame size given by \typename{frame\_descr}.\funcarg{frame\_size}, allows to find the end of the previous frame
        \item And the return address of the caller of this function
        \bigskip
        \onslide<3->{
        \item Note that traps (here the reddish \funcname{ExnA}, \funcname{ExnB} and \funcname{ExnC} blocks) belong to the frame that installed them, and so the frame size changes when traps are removed
        }
      \end{itemize}
    \end{column}
    \begin{column}{0.5\textwidth}
      \raggedleft
      \onslide*<1>{\providecommand\step{2}\input{diagrams/backtrace/backtrace.tex}}%
      \onslide*<2>{\providecommand\step{1}\input{diagrams/backtrace/scanstack.tex}}%
      \onslide*<3>{\providecommand\step{2}\input{diagrams/backtrace/scanstack.tex}}%
      \onslide*<4>{\providecommand\step{3}\input{diagrams/backtrace/scanstack.tex}}%
      \onslide*<5>{\providecommand\step{4}\input{diagrams/backtrace/scanstack.tex}}%
      \onslide*<6>{\providecommand\step{5}\input{diagrams/backtrace/scanstack.tex}}%
    \end{column}
  \end{columns}
\end{frame}

% Duplicate of previous-previous slide
\begin{frame}{Backtraces}{Continuing on \funcname{caml\_stash\_backtrace}}
  \begin{columns}[c]
    \begin{column}{0.5\textwidth}
      \minipage[c][0.75\textheight][s]{\columnwidth}
      \begin{itemize}
          \color{gray}
        \item A C function provided by the runtime (specific to native code)
        \item It scans the stack, between the stack pointer at the raising point up to the next trap, in order to collect the names of the detected function frames
        \item It uses the same technique and information as the Garbage Collector (GC) uses to scan the values live on the stack
      \end{itemize}
      \begin{itemize}
        \item For each function frame detected, store a pointer to the \typename{frame\_descr} in the \localname{domain\_state->backtrace\_buffer} array, effectively constructing the backtrace
      \end{itemize}
      \onslide<2->{
        \begin{itemize}
            \smallskip
          \item Constructed backtrace:
            \funcname{definitely\_raise},
            \onslide<3->{\funcname{probably\_raise}}
        \end{itemize}
      }
      \vfill
      \mintocaml[firstline=9,lastline=13,highlightlines=12]{../src/backtrace/backtrace.ml}
      \endminipage
    \end{column}
    \begin{column}{0.5\textwidth}
      \raggedleft
      \onslide*<1>{\listStashBacktrace[highlightlines={114-115}]{}}
      \onslide*<2>{\providecommand\step{3}\input{diagrams/backtrace/backtrace.tex}}%
      \onslide*<3>{\providecommand\step{4}\input{diagrams/backtrace/backtrace.tex}}%
    \end{column}
  \end{columns}
\end{frame}

\begin{frame}{Backtraces}{Back to \funcname{caml\_raise\_exn}}
  \begin{columns}[c]
    \begin{column}{0.5\textwidth}
      \minipage[c][0.66\textheight][s]{\columnwidth}
      \begin{itemize}\color{gray}
        \item Checks wheteher exceptions backtrace should be recorded, by checking the \funcarg{domain\_state->backtrace\_active} value
        \item Switches to the C stack to be able to execute C code
        \item Calls \funcname{caml\_stash\_backtrace}, to collect the backtrace up to the next trap
      \end{itemize}
      \onslide*<2->{
        \begin{itemize}
          \item<2-> Executes the exception handler in \funcname{probably\_raise} with \funcname{RESTORE\_EXN\_HANDLER\_OCAML}
            \begin{itemize}
              \item Set \regname{RSP} to \localname{domain\_state->exn\_handler} value
              \item Restore \localname{domain\_state->exn\_handler} from trap
              \item Jump to the handler code with \funcname{ret}
            \end{itemize}
        \end{itemize}
      }
      \vfill
      \onslide*<1>{\mintocaml[firstline=9,lastline=13,highlightlines=12]{../src/backtrace/backtrace.ml}}
      \onslide*<2->{\mintocaml[firstline=15,lastline=21,highlightlines={19-21}]{../src/backtrace/backtrace.ml}}
      \endminipage
    \end{column}
    \begin{column}{0.5\textwidth}
      \centering
      \onslide*<1>{
        \mintc[firstline=190,lastline=192]{../ocaml/runtime/amd64.S}
        \mintc[firstline=234,lastline=237]{../ocaml/runtime/amd64.S}
      }
      \onslide*<2>{
        \mintc[firstline=190,lastline=192,highlightlines={191-192}]{../ocaml/runtime/amd64.S}
        \mintc[firstline=234,lastline=237,highlightlines={235-237}]{../ocaml/runtime/amd64.S}
      }
      \onslide*<1>{\listCamlRaiseExn[highlightlines=720]{}}
      \onslide*<2>{\listCamlRaiseExn[highlightlines=722]{}}
      \onslide*<3->{\providecommand\step{5}\input{diagrams/backtrace/backtrace.tex}}
    \end{column}
  \end{columns}
\end{frame}

\begin{frame}{Backtraces}{\funcname{caml\_probably\_raise}'s handler doesn't catch \typename{ExnC}}
  \begin{columns}[c]
    \begin{column}{0.45\textwidth}
      \minipage[c][0.75\textheight][s]{\columnwidth}
      \begin{itemize}
        \item<1-> \funcname{match \dots with}\footnotemark pattern matching can also match exceptions
        \item<1-> The CMM of \funcname{probably\_raise} shows that this \funcname{match \dots with} is implemented with a pattern matching and a \funcname{try \dots with}
        \item<1-> But this one only matches on \typename{ExnA} exception
        \item<2-> Since the pattern matching fails to catch the exception, \funcname{reraise} is called with our current \typename{ExnC} exception
        \item<3-> Disassembly shows that \funcname{reraise} is actually a call to \funcname{caml\_reraise\_exn}, which is also an assembly function provided by the runtime
      \end{itemize}
      \vfill
      \mintocaml[firstline=15,lastline=21,highlightlines={18-21}]{../src/backtrace/backtrace.ml}
      \endminipage
    \end{column}
    \begin{column}{0.55\textwidth}
      \centering
      \onslide*<1>{\mintcmm[highlightlines={10-18}]{../src/backtrace/camlBacktrace\_\_probably\_raise\_351.cmm}}
      \onslide*<2>{\listcmm[highlightlines=18]{../src/backtrace/camlBacktrace\_\_probably\_raise_351.cmm}{CMM of probably\_raise}}
      \onslide*<3>{\mintobjdump[firstline=5,lastline=35,highlightlines=31]{../src/backtrace/camlBacktrace\_\_probably\_raise_351.objdump}}
    \end{column}
  \end{columns}
  \only<1>{\footnotetext[1]{https://blog.janestreet.com/pattern-matching-and-exception-handling-unite/}}
\end{frame}

\newcommand\listCamlReraiseExn[1][]{
  \listasm[firstline=727,lastline=734,#1]{../ocaml/runtime/amd64.S}{runtime/amd64.S:727}}

\begin{frame}{Backtraces}{Introducing \funcname{caml\_reraise\_exn}}
  \begin{columns}[c]
    \begin{column}{0.5\textwidth}
      \minipage[c][0.66\textheight][s]{\columnwidth}
      \begin{itemize}
        \item<1-> The exception have to be reraised to the next handler
        \item<2-> First, checks if the backtrace should be collected with \funcarg{domain\_state->backtrace\_active}
        \only<3>{
        \begin{itemize}
            \item If not, behaves like a \funcname{raise\_notrace} with \funcname{RESTORE\_EXN\_HANDLER\_OCAML}
        \end{itemize}
        }
        \item<4-> Jumps into \funcname{caml\_raise\_exn}
        \item<5-> Calls \funcname{caml\_stash\_backtrace} again, to scan the next part of the stack: from the trap just pop-ed to the next trap
      \end{itemize}
      \vfill
      \mintocaml[firstline=15,lastline=21,highlightlines={18-21}]{../src/backtrace/backtrace.ml}
      \endminipage
    \end{column}
    \begin{column}{0.5\textwidth}
      \raggedleft
      \onslide*<1>{\listCamlReraiseExn{}}
      \onslide*<2>{\listCamlReraiseExn[highlightlines=729]{}}
      \onslide*<3>{
        \mintc[firstline=190,lastline=192,highlightlines={191-192}]{../ocaml/runtime/amd64.S}
        \mintc[firstline=234,lastline=237,highlightlines={235-237}]{../ocaml/runtime/amd64.S}
        \medskip
        \listCamlReraiseExn[highlightlines=731]{}
      }
      \onslide*<4-5>{
        % TODO refactor with \listCamlReraiseExn
        \mintasm[firstline=727,lastline=734,highlightlines=730]{../ocaml/runtime/amd64.S}
        \medskip
      }
      \onslide*<4>{\listCamlRaiseExn[highlightlines=711]{}}
      \onslide*<5>{\listCamlRaiseExn[highlightlines=720]{}}
    \end{column}
  \end{columns}
\end{frame}

\begin{frame}{Backtraces}{Backtrace collection continues}
  \begin{columns}[c]
    \begin{column}{0.55\textwidth}
      \minipage[c][0.75\textheight][s]{\columnwidth}
      \begin{itemize}
        \item<1-> \funcname{caml\_stash\_backtrace} scans the older part of the stack, up to the next trap
        \item<6-> Controls jumps to the nested exception handler in \funcname{maybe\_raise}, which matches only on \typename{ExnB}
        \item<7-> \funcname{caml\_reraise\_exn} is called again, backtrace collection resumes with \funcname{caml\_stash\_backtrace}
      \end{itemize}
      \medskip
      \begin{itemize}
        \item<2-> Constructed backtrace:
        \begin{itemize}
          \item <2-> \funcname{definitely\_raise}
          \item <2-> \funcname{probably\_raise}
          \item <3-> \funcname{probably\_raise} (re-raise)
          \item <4-> \funcname{maybe\_raise}
          \item <5-> \funcname{main}
          \item <8-> \funcname{main} (re-raise)
        \end{itemize}
      \end{itemize}
      \vfill
      \onslide*<1-5>{\mintocaml[firstline=15,lastline=21]{../src/backtrace/backtrace.ml}}
      \onslide*<6>{\mintocaml[firstline=28,lastline=34,highlightlines=31]{../src/backtrace/backtrace.ml}}
      \onslide*<7->{\mintocaml[firstline=28,lastline=34,highlightlines=33]{../src/backtrace/backtrace.ml}}
      \endminipage
    \end{column}
    \begin{column}{0.45\textwidth}
      \raggedleft
      \onslide*<1>{\providecommand\step{6}\input{diagrams/backtrace/backtrace.tex}}%
      \onslide*<2>{\providecommand\step{6}\input{diagrams/backtrace/backtrace.tex}}%
      \onslide*<3>{\providecommand\step{7}\input{diagrams/backtrace/backtrace.tex}}%
      \onslide*<4>{\providecommand\step{8}\input{diagrams/backtrace/backtrace.tex}}%
      \onslide*<5>{\providecommand\step{9}\input{diagrams/backtrace/backtrace.tex}}%
      \onslide*<6>{\providecommand\step{10}\input{diagrams/backtrace/backtrace.tex}}%
      \onslide*<7>{\providecommand\step{11}\input{diagrams/backtrace/backtrace.tex}}%
      \onslide*<8>{\providecommand\step{12}\input{diagrams/backtrace/backtrace.tex}}%
      \onslide*<9>{\providecommand\step{13}\input{diagrams/backtrace/backtrace.tex}}%
    \end{column}
  \end{columns}
\end{frame}

\begin{frame}{Backtraces}{Printing the backtrace}
  \begin{columns}[c]
    \begin{column}{0.45\textwidth}
      \minipage[c][0.66\textheight][s]{\columnwidth}
      \begin{itemize}
        \item<1-> The exception backtrace can now be printed to \funcarg{stdout} with \funcname{Printexc.print_backtrace}
        \item<2-> First the exception backtrace is retrieved into an OCaml array with \funcname{get_raw_backtrace }
        \item<3-> Which is implemented as the \funcname{caml_get_exception_raw_backtrace} C function in the runtime
        \item<4-> And finally printed with \funcname{print_exception_backtrace}
      \end{itemize}
      \vfill
      \mintocaml[firstline=28,lastline=34,highlightlines=33]{../src/backtrace/backtrace.ml}
      \endminipage
    \end{column}
    \begin{column}{0.55\textwidth}
      \onslide*<1,2>{
        \mintocaml[firstline=100,lastline=101]{../ocaml/stdlib/printexc.ml}
      }
      \only<1>{
        \listocaml[firstline=170,lastline=172]{../ocaml/stdlib/printexc.ml}{stdlib/printexc.ml:170}
      }
      \only<2>{
        \listocaml[firstline=170,lastline=172,highlightlines=172]{../ocaml/stdlib/printexc.ml}{stdlib/printexc.ml:170}
      }
      \only<3>{
        \mintc[firstline=154,lastline=156]{../ocaml/runtime/backtrace.c}
        \listc[firstline=171,lastline=192,highlightlines={184-187}]{../ocaml/runtime/backtrace.c}{runtime/backtrace.c:154}
      }
      \only<4>{
        \listocaml[firstline=155,lastline=165]{../ocaml/stdlib/printexc.ml}{stdlib/printexc.ml:55}
      }
    \end{column}
  \end{columns}
\end{frame}


\subsection{The multiple kinds of raise}
\frameSubsection{}{}

\begin{frame}{Backtraces}{When backtraces are not needed}
  \begin{columns}[c]
    \begin{column}{0.45\textwidth}
      \minipage[c][0.66\textheight][s]{\columnwidth}
      \begin{itemize}
        \item<1-> What if the backtrace for an exception is never needed?
        \item<2-> It's possible to raise the exception explicitly with \funcname{raise\_notrace} to make raising as cheap as possible
        \item<3-> An example of use in the \funcname{dynlink} library of the OCaml compiler
      \end{itemize}
      \vfill
      \onslide<3->{
      \listocaml[firstline=205,lastline=209,highlightlines=207]{../ocaml/otherlibs/dynlink/dynlink\_compilerlibs/misc.ml}{otherlibs/dynlink/dynlink\_compilerlibs/misc.ml:205}
      }
      \endminipage
    \end{column}
    \begin{column}{0.55\textwidth}
    \only<4>{
      \mintcmm[highlightlines=3]{../src/notrace/camlNotrace\_\_fun_347.cmm}
      \listcmm{../src/notrace/camlNotrace\_\_all\_somes\_327.cmm}{all\_somes}
    }
    \onslide*<5->{
      \listobjdump[highlightlines={6-9}]{../src/notrace/camlNotrace\_\_fun_347.objdump}{anonymous function from \funcname{all\_somes}}
    }
    \end{column}
  \end{columns}
\end{frame}

\begin{frame}{Backtraces}{Summary table of the multiple kinds of raise}
  \begin{itemize}
    \item When debugging information is enabled and if the backtrace of an exception isn't needed, it's possible to raise with \funcname{raise\_notrace} for performance
    \item When debugging information is \emph{not} enabled, the compiler optimises \funcname{raise} calls to inline assembly (same effect as using \funcname{raise\_notrace})
  \end{itemize}
  \bigskip
  \centering
  \begin{tabular}{| l | c | c | c | c |}
    \hline
    \parbox{10em}{Debug information \\ (\funcarg{-g} flag)}  & \multicolumn{2}{|c|}{Disabled}                                     & \multicolumn{2}{|c|}{Enabled} \\
    \hline
    Raise kind                                               & \funcname{raise} & \funcname{raise\_notrace}                       & \funcname{raise} & \funcname{raise\_notrace} \\
    \hline
    Compilation                                              & \parbox{8em}{inline assembly\\ (optimization)} & inline assembly   & \funcname{caml\_raise\_exn} & inline assembly \\
    \hline
  \end{tabular}
\end{frame}


%
% Takeaway #5
%
\subsection*{Takeaway \#5}
\frameSubsectionTakeaway{}
\againframe<5>{takeaway}

    \end{column}
  \end{columns}
\end{frame}

\begin{frame}{Backtraces}{But before that, we need more fields from \typename{caml\_domain\_state}}
  \begin{columns}
    \begin{column}{0.5\textwidth}
      \begin{itemize}
        \item Remember \typename{caml\_domain\_state} the per-domain runtime state?
        \item Some other members are of interest to us now\dots
        \item \localname{backtrace\_active} is used to know if the runtime should collect backtraces
        \item \localname{backtrace\_active} can be set by
          \begin{itemize}
            \item The \funcarg{b} flag in the \funcname{OCAMLRUNPARAM} environment variable
            \item \mintinline{ocaml}{Printexc.record_backtrace : bool -> unit} \footnotemark
          \end{itemize}
      \end{itemize}
    \end{column}
    \begin{column}{0.5\textwidth}
      \begin{listing}
        \mintc[highlightlines={9-11}]{src/ocaml/domain\_state\_backtrace.h}
        \caption{runtime/caml/domain\_state.h}
      \end{listing}
    \end{column}
  \end{columns}
  \footnotetext[1]{https://v2.ocaml.org/api/Printexc.html}
\end{frame}

\newcommand\listCamlRaiseExn[1][]{
  \listasm[firstline=702,lastline=725,#1]{../ocaml/runtime/amd64.S}{runtime/amd64.S:702}}

\begin{frame}{Backtraces}{Walk through \funcname{caml\_raise\_exn}}
  \begin{columns}[T]
    \begin{column}{0.5\textwidth}
      \begin{itemize}
        \item<1-> First checks whether exceptions backtrace should be recorded
        \item<1-> By checking the \funcarg{domain\_state->backtrace\_active} value
        \item<2-> If not, acts as \funcname{raise\_notrace} with \funcname{RESTORE\_EXN\_HANDLER\_OCAML}
          \begin{itemize}
            \item Set \regname{RSP} to \localname{domain\_state->exn\_handler} value
            \item Restore \localname{domain\_state->exn\_handler} from trap
            \item Jump with \funcname{ret} (same effect as using \funcname{pop} and \funcname{jmp})
          \end{itemize}
        \item<3-> Otherwise, switches to the C stack to be able to execute C code
        \item<4-> Calls \funcname{caml\_stash\_backtrace}, a C function provided by the runtime\ignorespaces
          \onslide<6->{, with
            \begin{itemize}
              \item \funcarg{exn}: exception value
              \item \funcarg{pc}: code address of the raising point
              \item \funcarg{sp}: stack address of the raising function
              \item \funcarg{trapsp}: stack address of the next handler
            \end{itemize}
          }
      \end{itemize}
      \bigskip
      \mintocaml[firstline=9,lastline=13,highlightlines=12]{../src/backtrace/backtrace.ml}
    \end{column}
    \begin{column}{0.5\textwidth}
      \raggedleft
      \onslide*<1,3-4>{
        \mintc[firstline=190,lastline=192]{../ocaml/runtime/amd64.S}
        \mintc[firstline=234,lastline=237]{../ocaml/runtime/amd64.S}
      }
      \onslide*<2>{
        \mintc[firstline=190,lastline=192,highlightlines={191-192}]{../ocaml/runtime/amd64.S}
        \mintc[firstline=234,lastline=237,highlightlines={235-237}]{../ocaml/runtime/amd64.S}
      }
      \onslide*<1>{\listCamlRaiseExn[highlightlines=705]{}}%
      \onslide*<2>{\listCamlRaiseExn[highlightlines={707-708}]{}}%
      \onslide*<3>{\listCamlRaiseExn[highlightlines={712-714}]{}}%
      \onslide*<4>{\listCamlRaiseExn[highlightlines={715-720}]{}}%
      \onslide*<5>{\providecommand\step{1}%
% Backtraces
%
\subsection{One advantage of raising with debugging information}
\frameSubsection{}{}

\begin{frame}{Backtraces}{Debugging information \& source code}
  \begin{columns}[c]
    \begin{column}{0.5\textwidth}
      \begin{itemize}
        \item Raising an exception is fun and games, but how do we know where the exception originated? $\rightarrow$ With the backtrace associated with the exception!
        \item In order to get a backtrace from the point where an exception was raised, it's necessary to compile with debugging information enabled (\funcarg{-g} flag for the compiler)
        \item Same example as in the `Nested handlers' section
      \end{itemize}
    \end{column}
    \begin{column}{0.5\textwidth}
      \listocaml[firstline=9,lastline=34]{../src/backtrace/backtrace.ml}{backtrace/backtrace.ml}
    \end{column}
  \end{columns}
\end{frame}

\begin{frame}{Backtraces}{CMM and disassembly of \funcname{definitely\_raise}}
  \begin{columns}[c]
    \begin{column}{0.5\textwidth}
      \minipage[c][0.66\textheight][s]{\columnwidth}
      \begin{itemize}
        \item<1-> An effect of compiling with debugging information (\funcarg{-g}): the CMM no longer uses \funcname{raise\_notrace} to raise the exception but \funcname{raise} instead
        \item<2-> Disassembly shows that CMM \funcname{raise} is actually a call to \funcname{caml\_raise\_exn}, a function in assembler provided by the runtime
      \end{itemize}
      \vfill
      \mintocaml[firstline=9,lastline=13,highlightlines=12]{../src/backtrace/backtrace.ml}
      \endminipage
    \end{column}
    \begin{column}{0.5\textwidth}
      \onslide*<1>{
        \mintcmm[highlightlines=5]{../src/backtrace/camlBacktrace\_\_definitely\_raise\_347.cmm}
      }
      \onslide*<2>{
        \mintobjdump[firstline=2,highlightlines=14]{../src/backtrace/camlBacktrace\_\_definitely\_raise\_347.objdump}
      }
    \end{column}
  \end{columns}
\end{frame}

\begin{frame}{Backtraces}{Introducing \funcname{caml\_raise\_exn}}
  \begin{columns}[c]
    \begin{column}{0.5\textwidth}
      \minipage[c][0.66\textheight][s]{\columnwidth}
      \onslide*<1>{
        \begin{itemize}
          \item Raising the exception is performed using \funcname{caml\_raise\_exn} which is
            \begin{itemize}
              \item An assembly function provided by the runtime
              \item Architecture specific
            \end{itemize}
          \item Let's see what it does differently from \funcname{raise\_notrace}\dots
          \item We are about to raise \typename{ExnC} with \funcname{caml\_raise\_exn} from \funcname{definitely\_raise}
        \end{itemize}
      }
      \onslide*<2>{
        \mintobjdump[firstline=2,highlightlines=14]{../src/backtrace/camlBacktrace\_\_definitely\_raise\_347.objdump}
      }
      \vfill
      \mintocaml[firstline=9,lastline=13,highlightlines=12]{../src/backtrace/backtrace.ml}
      \endminipage
    \end{column}
    \begin{column}{0.5\textwidth}
      \centering
      \providecommand\step{1}\input{diagrams/backtrace/backtrace.tex}
    \end{column}
  \end{columns}
\end{frame}

\begin{frame}{Backtraces}{But before that, we need more fields from \typename{caml\_domain\_state}}
  \begin{columns}
    \begin{column}{0.5\textwidth}
      \begin{itemize}
        \item Remember \typename{caml\_domain\_state} the per-domain runtime state?
        \item Some other members are of interest to us now\dots
        \item \localname{backtrace\_active} is used to know if the runtime should collect backtraces
        \item \localname{backtrace\_active} can be set by
          \begin{itemize}
            \item The \funcarg{b} flag in the \funcname{OCAMLRUNPARAM} environment variable
            \item \mintinline{ocaml}{Printexc.record_backtrace : bool -> unit} \footnotemark
          \end{itemize}
      \end{itemize}
    \end{column}
    \begin{column}{0.5\textwidth}
      \begin{listing}
        \mintc[highlightlines={9-11}]{src/ocaml/domain\_state\_backtrace.h}
        \caption{runtime/caml/domain\_state.h}
      \end{listing}
    \end{column}
  \end{columns}
  \footnotetext[1]{https://v2.ocaml.org/api/Printexc.html}
\end{frame}

\newcommand\listCamlRaiseExn[1][]{
  \listasm[firstline=702,lastline=725,#1]{../ocaml/runtime/amd64.S}{runtime/amd64.S:702}}

\begin{frame}{Backtraces}{Walk through \funcname{caml\_raise\_exn}}
  \begin{columns}[T]
    \begin{column}{0.5\textwidth}
      \begin{itemize}
        \item<1-> First checks whether exceptions backtrace should be recorded
        \item<1-> By checking the \funcarg{domain\_state->backtrace\_active} value
        \item<2-> If not, acts as \funcname{raise\_notrace} with \funcname{RESTORE\_EXN\_HANDLER\_OCAML}
          \begin{itemize}
            \item Set \regname{RSP} to \localname{domain\_state->exn\_handler} value
            \item Restore \localname{domain\_state->exn\_handler} from trap
            \item Jump with \funcname{ret} (same effect as using \funcname{pop} and \funcname{jmp})
          \end{itemize}
        \item<3-> Otherwise, switches to the C stack to be able to execute C code
        \item<4-> Calls \funcname{caml\_stash\_backtrace}, a C function provided by the runtime\ignorespaces
          \onslide<6->{, with
            \begin{itemize}
              \item \funcarg{exn}: exception value
              \item \funcarg{pc}: code address of the raising point
              \item \funcarg{sp}: stack address of the raising function
              \item \funcarg{trapsp}: stack address of the next handler
            \end{itemize}
          }
      \end{itemize}
      \bigskip
      \mintocaml[firstline=9,lastline=13,highlightlines=12]{../src/backtrace/backtrace.ml}
    \end{column}
    \begin{column}{0.5\textwidth}
      \raggedleft
      \onslide*<1,3-4>{
        \mintc[firstline=190,lastline=192]{../ocaml/runtime/amd64.S}
        \mintc[firstline=234,lastline=237]{../ocaml/runtime/amd64.S}
      }
      \onslide*<2>{
        \mintc[firstline=190,lastline=192,highlightlines={191-192}]{../ocaml/runtime/amd64.S}
        \mintc[firstline=234,lastline=237,highlightlines={235-237}]{../ocaml/runtime/amd64.S}
      }
      \onslide*<1>{\listCamlRaiseExn[highlightlines=705]{}}%
      \onslide*<2>{\listCamlRaiseExn[highlightlines={707-708}]{}}%
      \onslide*<3>{\listCamlRaiseExn[highlightlines={712-714}]{}}%
      \onslide*<4>{\listCamlRaiseExn[highlightlines={715-720}]{}}%
      \onslide*<5>{\providecommand\step{1}\input{diagrams/backtrace/backtrace.tex}}%
      \onslide*<6>{\providecommand\step{2}\input{diagrams/backtrace/backtrace.tex}}%
    \end{column}
  \end{columns}
\end{frame}

\newcommand\listStashBacktrace[1][]{
  \listc[firstline=91,lastline=120,#1]{../ocaml/runtime/backtrace\_nat.c}{runtime/backtrace\_nat.c:91}}

\begin{frame}{Backtraces}{Introducing \funcname{caml\_stash\_backtrace}}
  \begin{columns}[c]
    \begin{column}{0.5\textwidth}
      \minipage[c][0.66\textheight][s]{\columnwidth}
      \begin{itemize}
        \item<1-> A C function provided by the runtime (specific to native code)
        \item<2-> It scans the stack, between the stack pointer at the raising point up to the next trap, in order to collect the names of the detected function frames
          \onslide*<2>{
            \begin{itemize}
              \item And more information, like line number, column number...
            \end{itemize}
          }
        \item<3-> It uses the same technique and information as the Garbage Collector (GC) uses to scan the values live on the stack
          \onslide*<4>{
            \begin{itemize}
              \item \typename{frame\_descr} are at the core of stack unwinding
            \end{itemize}
          }
      \end{itemize}
      \vfill
      \mintocaml[firstline=9,lastline=13,highlightlines=12]{../src/backtrace/backtrace.ml}
      \endminipage
    \end{column}
    \begin{column}{0.5\textwidth}
      \onslide*<1>{\listStashBacktrace[highlightlines=91]{}}
      \onslide*<2>{\listStashBacktrace[highlightlines={91,117-118}]{}}
      \onslide*<3->{\listStashBacktrace[highlightlines={94,106,109-110}]{}}
    \end{column}
  \end{columns}
\end{frame}

\newcommand\listFrameDescr[1][]{
  \listocaml[firstline=111,lastline=124,#1]{../ocaml/asmcomp/emitaux.ml}{asmcomp/emitaux.ml:111}}
\newcommand\listDebugInfo[1][]{
  \listocaml[firstline=38,lastline=49,#1]{../ocaml/lambda/debuginfo.mli}{lambda/debuginfo.mli:38}}

\begin{frame}{Backtraces}{\typename{frame\_descr} and \typename{frame\_debuginfo} in a nutshell}
  \begin{columns}[c]
    \begin{column}{0.5\textwidth}
      \begin{itemize}
        \item<1-> For every return address in an OCaml program, there is a associated \typename{frame\_descr}
        \item<2-> A \typename{frame\_descr} specifies
        \begin{itemize}
          \item<2-> The size of the function stack frame
          \item<3-> Where live values are on the stack (so that the GC can mark them as live and prevent from being swept)
        \end{itemize}
        \item<4-> When compiled with debug information, \typename{frame\_descr} will be linked to a \typename{frame\_debuginfo}
        \begin{itemize}
          \item<5-> Contains information about the position in the source code (file name, line and column number)
        \end{itemize}
      \end{itemize}
    \end{column}
    \begin{column}{0.5\textwidth}
        \onslide*<1>{\listFrameDescr[highlightlines={118-122}]{}}
        \onslide*<2>{\listFrameDescr[highlightlines={120}]{}}
        \onslide*<3>{\listFrameDescr[highlightlines={121}]{}}
        \onslide*<4->{\listFrameDescr[highlightlines={122}]{}}
        \medskip
        \onslide*<1-3>{\listDebugInfo{}}
        \onslide*<4>{\listDebugInfo[highlightlines={38,49}]{}}
        \onslide*<5>{\listDebugInfo[highlightlines={39-46}]{}}
    \end{column}
  \end{columns}
\end{frame}

\begin{frame}{Backtraces}{Scanning the stack with \typename{frame\_descr}}
  \begin{columns}[c]
    \begin{column}{0.5\textwidth}
      \begin{itemize}
        \item Given the return address of the code that raises the exception by calling \funcname{caml\_raise\_exn} (\funcarg{pc} argument of \funcname{caml\_stash\_backtrace})
        \item And the position of the stack pointer (\funcarg{sp} argument of \funcname{caml\_stash\_backtrace})
        \item The frame size given by \typename{frame\_descr}.\funcarg{frame\_size}, allows to find the end of the previous frame
        \item And the return address of the caller of this function
        \bigskip
        \onslide<3->{
        \item Note that traps (here the reddish \funcname{ExnA}, \funcname{ExnB} and \funcname{ExnC} blocks) belong to the frame that installed them, and so the frame size changes when traps are removed
        }
      \end{itemize}
    \end{column}
    \begin{column}{0.5\textwidth}
      \raggedleft
      \onslide*<1>{\providecommand\step{2}\input{diagrams/backtrace/backtrace.tex}}%
      \onslide*<2>{\providecommand\step{1}\input{diagrams/backtrace/scanstack.tex}}%
      \onslide*<3>{\providecommand\step{2}\input{diagrams/backtrace/scanstack.tex}}%
      \onslide*<4>{\providecommand\step{3}\input{diagrams/backtrace/scanstack.tex}}%
      \onslide*<5>{\providecommand\step{4}\input{diagrams/backtrace/scanstack.tex}}%
      \onslide*<6>{\providecommand\step{5}\input{diagrams/backtrace/scanstack.tex}}%
    \end{column}
  \end{columns}
\end{frame}

% Duplicate of previous-previous slide
\begin{frame}{Backtraces}{Continuing on \funcname{caml\_stash\_backtrace}}
  \begin{columns}[c]
    \begin{column}{0.5\textwidth}
      \minipage[c][0.75\textheight][s]{\columnwidth}
      \begin{itemize}
          \color{gray}
        \item A C function provided by the runtime (specific to native code)
        \item It scans the stack, between the stack pointer at the raising point up to the next trap, in order to collect the names of the detected function frames
        \item It uses the same technique and information as the Garbage Collector (GC) uses to scan the values live on the stack
      \end{itemize}
      \begin{itemize}
        \item For each function frame detected, store a pointer to the \typename{frame\_descr} in the \localname{domain\_state->backtrace\_buffer} array, effectively constructing the backtrace
      \end{itemize}
      \onslide<2->{
        \begin{itemize}
            \smallskip
          \item Constructed backtrace:
            \funcname{definitely\_raise},
            \onslide<3->{\funcname{probably\_raise}}
        \end{itemize}
      }
      \vfill
      \mintocaml[firstline=9,lastline=13,highlightlines=12]{../src/backtrace/backtrace.ml}
      \endminipage
    \end{column}
    \begin{column}{0.5\textwidth}
      \raggedleft
      \onslide*<1>{\listStashBacktrace[highlightlines={114-115}]{}}
      \onslide*<2>{\providecommand\step{3}\input{diagrams/backtrace/backtrace.tex}}%
      \onslide*<3>{\providecommand\step{4}\input{diagrams/backtrace/backtrace.tex}}%
    \end{column}
  \end{columns}
\end{frame}

\begin{frame}{Backtraces}{Back to \funcname{caml\_raise\_exn}}
  \begin{columns}[c]
    \begin{column}{0.5\textwidth}
      \minipage[c][0.66\textheight][s]{\columnwidth}
      \begin{itemize}\color{gray}
        \item Checks wheteher exceptions backtrace should be recorded, by checking the \funcarg{domain\_state->backtrace\_active} value
        \item Switches to the C stack to be able to execute C code
        \item Calls \funcname{caml\_stash\_backtrace}, to collect the backtrace up to the next trap
      \end{itemize}
      \onslide*<2->{
        \begin{itemize}
          \item<2-> Executes the exception handler in \funcname{probably\_raise} with \funcname{RESTORE\_EXN\_HANDLER\_OCAML}
            \begin{itemize}
              \item Set \regname{RSP} to \localname{domain\_state->exn\_handler} value
              \item Restore \localname{domain\_state->exn\_handler} from trap
              \item Jump to the handler code with \funcname{ret}
            \end{itemize}
        \end{itemize}
      }
      \vfill
      \onslide*<1>{\mintocaml[firstline=9,lastline=13,highlightlines=12]{../src/backtrace/backtrace.ml}}
      \onslide*<2->{\mintocaml[firstline=15,lastline=21,highlightlines={19-21}]{../src/backtrace/backtrace.ml}}
      \endminipage
    \end{column}
    \begin{column}{0.5\textwidth}
      \centering
      \onslide*<1>{
        \mintc[firstline=190,lastline=192]{../ocaml/runtime/amd64.S}
        \mintc[firstline=234,lastline=237]{../ocaml/runtime/amd64.S}
      }
      \onslide*<2>{
        \mintc[firstline=190,lastline=192,highlightlines={191-192}]{../ocaml/runtime/amd64.S}
        \mintc[firstline=234,lastline=237,highlightlines={235-237}]{../ocaml/runtime/amd64.S}
      }
      \onslide*<1>{\listCamlRaiseExn[highlightlines=720]{}}
      \onslide*<2>{\listCamlRaiseExn[highlightlines=722]{}}
      \onslide*<3->{\providecommand\step{5}\input{diagrams/backtrace/backtrace.tex}}
    \end{column}
  \end{columns}
\end{frame}

\begin{frame}{Backtraces}{\funcname{caml\_probably\_raise}'s handler doesn't catch \typename{ExnC}}
  \begin{columns}[c]
    \begin{column}{0.45\textwidth}
      \minipage[c][0.75\textheight][s]{\columnwidth}
      \begin{itemize}
        \item<1-> \funcname{match \dots with}\footnotemark pattern matching can also match exceptions
        \item<1-> The CMM of \funcname{probably\_raise} shows that this \funcname{match \dots with} is implemented with a pattern matching and a \funcname{try \dots with}
        \item<1-> But this one only matches on \typename{ExnA} exception
        \item<2-> Since the pattern matching fails to catch the exception, \funcname{reraise} is called with our current \typename{ExnC} exception
        \item<3-> Disassembly shows that \funcname{reraise} is actually a call to \funcname{caml\_reraise\_exn}, which is also an assembly function provided by the runtime
      \end{itemize}
      \vfill
      \mintocaml[firstline=15,lastline=21,highlightlines={18-21}]{../src/backtrace/backtrace.ml}
      \endminipage
    \end{column}
    \begin{column}{0.55\textwidth}
      \centering
      \onslide*<1>{\mintcmm[highlightlines={10-18}]{../src/backtrace/camlBacktrace\_\_probably\_raise\_351.cmm}}
      \onslide*<2>{\listcmm[highlightlines=18]{../src/backtrace/camlBacktrace\_\_probably\_raise_351.cmm}{CMM of probably\_raise}}
      \onslide*<3>{\mintobjdump[firstline=5,lastline=35,highlightlines=31]{../src/backtrace/camlBacktrace\_\_probably\_raise_351.objdump}}
    \end{column}
  \end{columns}
  \only<1>{\footnotetext[1]{https://blog.janestreet.com/pattern-matching-and-exception-handling-unite/}}
\end{frame}

\newcommand\listCamlReraiseExn[1][]{
  \listasm[firstline=727,lastline=734,#1]{../ocaml/runtime/amd64.S}{runtime/amd64.S:727}}

\begin{frame}{Backtraces}{Introducing \funcname{caml\_reraise\_exn}}
  \begin{columns}[c]
    \begin{column}{0.5\textwidth}
      \minipage[c][0.66\textheight][s]{\columnwidth}
      \begin{itemize}
        \item<1-> The exception have to be reraised to the next handler
        \item<2-> First, checks if the backtrace should be collected with \funcarg{domain\_state->backtrace\_active}
        \only<3>{
        \begin{itemize}
            \item If not, behaves like a \funcname{raise\_notrace} with \funcname{RESTORE\_EXN\_HANDLER\_OCAML}
        \end{itemize}
        }
        \item<4-> Jumps into \funcname{caml\_raise\_exn}
        \item<5-> Calls \funcname{caml\_stash\_backtrace} again, to scan the next part of the stack: from the trap just pop-ed to the next trap
      \end{itemize}
      \vfill
      \mintocaml[firstline=15,lastline=21,highlightlines={18-21}]{../src/backtrace/backtrace.ml}
      \endminipage
    \end{column}
    \begin{column}{0.5\textwidth}
      \raggedleft
      \onslide*<1>{\listCamlReraiseExn{}}
      \onslide*<2>{\listCamlReraiseExn[highlightlines=729]{}}
      \onslide*<3>{
        \mintc[firstline=190,lastline=192,highlightlines={191-192}]{../ocaml/runtime/amd64.S}
        \mintc[firstline=234,lastline=237,highlightlines={235-237}]{../ocaml/runtime/amd64.S}
        \medskip
        \listCamlReraiseExn[highlightlines=731]{}
      }
      \onslide*<4-5>{
        % TODO refactor with \listCamlReraiseExn
        \mintasm[firstline=727,lastline=734,highlightlines=730]{../ocaml/runtime/amd64.S}
        \medskip
      }
      \onslide*<4>{\listCamlRaiseExn[highlightlines=711]{}}
      \onslide*<5>{\listCamlRaiseExn[highlightlines=720]{}}
    \end{column}
  \end{columns}
\end{frame}

\begin{frame}{Backtraces}{Backtrace collection continues}
  \begin{columns}[c]
    \begin{column}{0.55\textwidth}
      \minipage[c][0.75\textheight][s]{\columnwidth}
      \begin{itemize}
        \item<1-> \funcname{caml\_stash\_backtrace} scans the older part of the stack, up to the next trap
        \item<6-> Controls jumps to the nested exception handler in \funcname{maybe\_raise}, which matches only on \typename{ExnB}
        \item<7-> \funcname{caml\_reraise\_exn} is called again, backtrace collection resumes with \funcname{caml\_stash\_backtrace}
      \end{itemize}
      \medskip
      \begin{itemize}
        \item<2-> Constructed backtrace:
        \begin{itemize}
          \item <2-> \funcname{definitely\_raise}
          \item <2-> \funcname{probably\_raise}
          \item <3-> \funcname{probably\_raise} (re-raise)
          \item <4-> \funcname{maybe\_raise}
          \item <5-> \funcname{main}
          \item <8-> \funcname{main} (re-raise)
        \end{itemize}
      \end{itemize}
      \vfill
      \onslide*<1-5>{\mintocaml[firstline=15,lastline=21]{../src/backtrace/backtrace.ml}}
      \onslide*<6>{\mintocaml[firstline=28,lastline=34,highlightlines=31]{../src/backtrace/backtrace.ml}}
      \onslide*<7->{\mintocaml[firstline=28,lastline=34,highlightlines=33]{../src/backtrace/backtrace.ml}}
      \endminipage
    \end{column}
    \begin{column}{0.45\textwidth}
      \raggedleft
      \onslide*<1>{\providecommand\step{6}\input{diagrams/backtrace/backtrace.tex}}%
      \onslide*<2>{\providecommand\step{6}\input{diagrams/backtrace/backtrace.tex}}%
      \onslide*<3>{\providecommand\step{7}\input{diagrams/backtrace/backtrace.tex}}%
      \onslide*<4>{\providecommand\step{8}\input{diagrams/backtrace/backtrace.tex}}%
      \onslide*<5>{\providecommand\step{9}\input{diagrams/backtrace/backtrace.tex}}%
      \onslide*<6>{\providecommand\step{10}\input{diagrams/backtrace/backtrace.tex}}%
      \onslide*<7>{\providecommand\step{11}\input{diagrams/backtrace/backtrace.tex}}%
      \onslide*<8>{\providecommand\step{12}\input{diagrams/backtrace/backtrace.tex}}%
      \onslide*<9>{\providecommand\step{13}\input{diagrams/backtrace/backtrace.tex}}%
    \end{column}
  \end{columns}
\end{frame}

\begin{frame}{Backtraces}{Printing the backtrace}
  \begin{columns}[c]
    \begin{column}{0.45\textwidth}
      \minipage[c][0.66\textheight][s]{\columnwidth}
      \begin{itemize}
        \item<1-> The exception backtrace can now be printed to \funcarg{stdout} with \funcname{Printexc.print_backtrace}
        \item<2-> First the exception backtrace is retrieved into an OCaml array with \funcname{get_raw_backtrace }
        \item<3-> Which is implemented as the \funcname{caml_get_exception_raw_backtrace} C function in the runtime
        \item<4-> And finally printed with \funcname{print_exception_backtrace}
      \end{itemize}
      \vfill
      \mintocaml[firstline=28,lastline=34,highlightlines=33]{../src/backtrace/backtrace.ml}
      \endminipage
    \end{column}
    \begin{column}{0.55\textwidth}
      \onslide*<1,2>{
        \mintocaml[firstline=100,lastline=101]{../ocaml/stdlib/printexc.ml}
      }
      \only<1>{
        \listocaml[firstline=170,lastline=172]{../ocaml/stdlib/printexc.ml}{stdlib/printexc.ml:170}
      }
      \only<2>{
        \listocaml[firstline=170,lastline=172,highlightlines=172]{../ocaml/stdlib/printexc.ml}{stdlib/printexc.ml:170}
      }
      \only<3>{
        \mintc[firstline=154,lastline=156]{../ocaml/runtime/backtrace.c}
        \listc[firstline=171,lastline=192,highlightlines={184-187}]{../ocaml/runtime/backtrace.c}{runtime/backtrace.c:154}
      }
      \only<4>{
        \listocaml[firstline=155,lastline=165]{../ocaml/stdlib/printexc.ml}{stdlib/printexc.ml:55}
      }
    \end{column}
  \end{columns}
\end{frame}


\subsection{The multiple kinds of raise}
\frameSubsection{}{}

\begin{frame}{Backtraces}{When backtraces are not needed}
  \begin{columns}[c]
    \begin{column}{0.45\textwidth}
      \minipage[c][0.66\textheight][s]{\columnwidth}
      \begin{itemize}
        \item<1-> What if the backtrace for an exception is never needed?
        \item<2-> It's possible to raise the exception explicitly with \funcname{raise\_notrace} to make raising as cheap as possible
        \item<3-> An example of use in the \funcname{dynlink} library of the OCaml compiler
      \end{itemize}
      \vfill
      \onslide<3->{
      \listocaml[firstline=205,lastline=209,highlightlines=207]{../ocaml/otherlibs/dynlink/dynlink\_compilerlibs/misc.ml}{otherlibs/dynlink/dynlink\_compilerlibs/misc.ml:205}
      }
      \endminipage
    \end{column}
    \begin{column}{0.55\textwidth}
    \only<4>{
      \mintcmm[highlightlines=3]{../src/notrace/camlNotrace\_\_fun_347.cmm}
      \listcmm{../src/notrace/camlNotrace\_\_all\_somes\_327.cmm}{all\_somes}
    }
    \onslide*<5->{
      \listobjdump[highlightlines={6-9}]{../src/notrace/camlNotrace\_\_fun_347.objdump}{anonymous function from \funcname{all\_somes}}
    }
    \end{column}
  \end{columns}
\end{frame}

\begin{frame}{Backtraces}{Summary table of the multiple kinds of raise}
  \begin{itemize}
    \item When debugging information is enabled and if the backtrace of an exception isn't needed, it's possible to raise with \funcname{raise\_notrace} for performance
    \item When debugging information is \emph{not} enabled, the compiler optimises \funcname{raise} calls to inline assembly (same effect as using \funcname{raise\_notrace})
  \end{itemize}
  \bigskip
  \centering
  \begin{tabular}{| l | c | c | c | c |}
    \hline
    \parbox{10em}{Debug information \\ (\funcarg{-g} flag)}  & \multicolumn{2}{|c|}{Disabled}                                     & \multicolumn{2}{|c|}{Enabled} \\
    \hline
    Raise kind                                               & \funcname{raise} & \funcname{raise\_notrace}                       & \funcname{raise} & \funcname{raise\_notrace} \\
    \hline
    Compilation                                              & \parbox{8em}{inline assembly\\ (optimization)} & inline assembly   & \funcname{caml\_raise\_exn} & inline assembly \\
    \hline
  \end{tabular}
\end{frame}


%
% Takeaway #5
%
\subsection*{Takeaway \#5}
\frameSubsectionTakeaway{}
\againframe<5>{takeaway}
}%
      \onslide*<6>{\providecommand\step{2}%
% Backtraces
%
\subsection{One advantage of raising with debugging information}
\frameSubsection{}{}

\begin{frame}{Backtraces}{Debugging information \& source code}
  \begin{columns}[c]
    \begin{column}{0.5\textwidth}
      \begin{itemize}
        \item Raising an exception is fun and games, but how do we know where the exception originated? $\rightarrow$ With the backtrace associated with the exception!
        \item In order to get a backtrace from the point where an exception was raised, it's necessary to compile with debugging information enabled (\funcarg{-g} flag for the compiler)
        \item Same example as in the `Nested handlers' section
      \end{itemize}
    \end{column}
    \begin{column}{0.5\textwidth}
      \listocaml[firstline=9,lastline=34]{../src/backtrace/backtrace.ml}{backtrace/backtrace.ml}
    \end{column}
  \end{columns}
\end{frame}

\begin{frame}{Backtraces}{CMM and disassembly of \funcname{definitely\_raise}}
  \begin{columns}[c]
    \begin{column}{0.5\textwidth}
      \minipage[c][0.66\textheight][s]{\columnwidth}
      \begin{itemize}
        \item<1-> An effect of compiling with debugging information (\funcarg{-g}): the CMM no longer uses \funcname{raise\_notrace} to raise the exception but \funcname{raise} instead
        \item<2-> Disassembly shows that CMM \funcname{raise} is actually a call to \funcname{caml\_raise\_exn}, a function in assembler provided by the runtime
      \end{itemize}
      \vfill
      \mintocaml[firstline=9,lastline=13,highlightlines=12]{../src/backtrace/backtrace.ml}
      \endminipage
    \end{column}
    \begin{column}{0.5\textwidth}
      \onslide*<1>{
        \mintcmm[highlightlines=5]{../src/backtrace/camlBacktrace\_\_definitely\_raise\_347.cmm}
      }
      \onslide*<2>{
        \mintobjdump[firstline=2,highlightlines=14]{../src/backtrace/camlBacktrace\_\_definitely\_raise\_347.objdump}
      }
    \end{column}
  \end{columns}
\end{frame}

\begin{frame}{Backtraces}{Introducing \funcname{caml\_raise\_exn}}
  \begin{columns}[c]
    \begin{column}{0.5\textwidth}
      \minipage[c][0.66\textheight][s]{\columnwidth}
      \onslide*<1>{
        \begin{itemize}
          \item Raising the exception is performed using \funcname{caml\_raise\_exn} which is
            \begin{itemize}
              \item An assembly function provided by the runtime
              \item Architecture specific
            \end{itemize}
          \item Let's see what it does differently from \funcname{raise\_notrace}\dots
          \item We are about to raise \typename{ExnC} with \funcname{caml\_raise\_exn} from \funcname{definitely\_raise}
        \end{itemize}
      }
      \onslide*<2>{
        \mintobjdump[firstline=2,highlightlines=14]{../src/backtrace/camlBacktrace\_\_definitely\_raise\_347.objdump}
      }
      \vfill
      \mintocaml[firstline=9,lastline=13,highlightlines=12]{../src/backtrace/backtrace.ml}
      \endminipage
    \end{column}
    \begin{column}{0.5\textwidth}
      \centering
      \providecommand\step{1}\input{diagrams/backtrace/backtrace.tex}
    \end{column}
  \end{columns}
\end{frame}

\begin{frame}{Backtraces}{But before that, we need more fields from \typename{caml\_domain\_state}}
  \begin{columns}
    \begin{column}{0.5\textwidth}
      \begin{itemize}
        \item Remember \typename{caml\_domain\_state} the per-domain runtime state?
        \item Some other members are of interest to us now\dots
        \item \localname{backtrace\_active} is used to know if the runtime should collect backtraces
        \item \localname{backtrace\_active} can be set by
          \begin{itemize}
            \item The \funcarg{b} flag in the \funcname{OCAMLRUNPARAM} environment variable
            \item \mintinline{ocaml}{Printexc.record_backtrace : bool -> unit} \footnotemark
          \end{itemize}
      \end{itemize}
    \end{column}
    \begin{column}{0.5\textwidth}
      \begin{listing}
        \mintc[highlightlines={9-11}]{src/ocaml/domain\_state\_backtrace.h}
        \caption{runtime/caml/domain\_state.h}
      \end{listing}
    \end{column}
  \end{columns}
  \footnotetext[1]{https://v2.ocaml.org/api/Printexc.html}
\end{frame}

\newcommand\listCamlRaiseExn[1][]{
  \listasm[firstline=702,lastline=725,#1]{../ocaml/runtime/amd64.S}{runtime/amd64.S:702}}

\begin{frame}{Backtraces}{Walk through \funcname{caml\_raise\_exn}}
  \begin{columns}[T]
    \begin{column}{0.5\textwidth}
      \begin{itemize}
        \item<1-> First checks whether exceptions backtrace should be recorded
        \item<1-> By checking the \funcarg{domain\_state->backtrace\_active} value
        \item<2-> If not, acts as \funcname{raise\_notrace} with \funcname{RESTORE\_EXN\_HANDLER\_OCAML}
          \begin{itemize}
            \item Set \regname{RSP} to \localname{domain\_state->exn\_handler} value
            \item Restore \localname{domain\_state->exn\_handler} from trap
            \item Jump with \funcname{ret} (same effect as using \funcname{pop} and \funcname{jmp})
          \end{itemize}
        \item<3-> Otherwise, switches to the C stack to be able to execute C code
        \item<4-> Calls \funcname{caml\_stash\_backtrace}, a C function provided by the runtime\ignorespaces
          \onslide<6->{, with
            \begin{itemize}
              \item \funcarg{exn}: exception value
              \item \funcarg{pc}: code address of the raising point
              \item \funcarg{sp}: stack address of the raising function
              \item \funcarg{trapsp}: stack address of the next handler
            \end{itemize}
          }
      \end{itemize}
      \bigskip
      \mintocaml[firstline=9,lastline=13,highlightlines=12]{../src/backtrace/backtrace.ml}
    \end{column}
    \begin{column}{0.5\textwidth}
      \raggedleft
      \onslide*<1,3-4>{
        \mintc[firstline=190,lastline=192]{../ocaml/runtime/amd64.S}
        \mintc[firstline=234,lastline=237]{../ocaml/runtime/amd64.S}
      }
      \onslide*<2>{
        \mintc[firstline=190,lastline=192,highlightlines={191-192}]{../ocaml/runtime/amd64.S}
        \mintc[firstline=234,lastline=237,highlightlines={235-237}]{../ocaml/runtime/amd64.S}
      }
      \onslide*<1>{\listCamlRaiseExn[highlightlines=705]{}}%
      \onslide*<2>{\listCamlRaiseExn[highlightlines={707-708}]{}}%
      \onslide*<3>{\listCamlRaiseExn[highlightlines={712-714}]{}}%
      \onslide*<4>{\listCamlRaiseExn[highlightlines={715-720}]{}}%
      \onslide*<5>{\providecommand\step{1}\input{diagrams/backtrace/backtrace.tex}}%
      \onslide*<6>{\providecommand\step{2}\input{diagrams/backtrace/backtrace.tex}}%
    \end{column}
  \end{columns}
\end{frame}

\newcommand\listStashBacktrace[1][]{
  \listc[firstline=91,lastline=120,#1]{../ocaml/runtime/backtrace\_nat.c}{runtime/backtrace\_nat.c:91}}

\begin{frame}{Backtraces}{Introducing \funcname{caml\_stash\_backtrace}}
  \begin{columns}[c]
    \begin{column}{0.5\textwidth}
      \minipage[c][0.66\textheight][s]{\columnwidth}
      \begin{itemize}
        \item<1-> A C function provided by the runtime (specific to native code)
        \item<2-> It scans the stack, between the stack pointer at the raising point up to the next trap, in order to collect the names of the detected function frames
          \onslide*<2>{
            \begin{itemize}
              \item And more information, like line number, column number...
            \end{itemize}
          }
        \item<3-> It uses the same technique and information as the Garbage Collector (GC) uses to scan the values live on the stack
          \onslide*<4>{
            \begin{itemize}
              \item \typename{frame\_descr} are at the core of stack unwinding
            \end{itemize}
          }
      \end{itemize}
      \vfill
      \mintocaml[firstline=9,lastline=13,highlightlines=12]{../src/backtrace/backtrace.ml}
      \endminipage
    \end{column}
    \begin{column}{0.5\textwidth}
      \onslide*<1>{\listStashBacktrace[highlightlines=91]{}}
      \onslide*<2>{\listStashBacktrace[highlightlines={91,117-118}]{}}
      \onslide*<3->{\listStashBacktrace[highlightlines={94,106,109-110}]{}}
    \end{column}
  \end{columns}
\end{frame}

\newcommand\listFrameDescr[1][]{
  \listocaml[firstline=111,lastline=124,#1]{../ocaml/asmcomp/emitaux.ml}{asmcomp/emitaux.ml:111}}
\newcommand\listDebugInfo[1][]{
  \listocaml[firstline=38,lastline=49,#1]{../ocaml/lambda/debuginfo.mli}{lambda/debuginfo.mli:38}}

\begin{frame}{Backtraces}{\typename{frame\_descr} and \typename{frame\_debuginfo} in a nutshell}
  \begin{columns}[c]
    \begin{column}{0.5\textwidth}
      \begin{itemize}
        \item<1-> For every return address in an OCaml program, there is a associated \typename{frame\_descr}
        \item<2-> A \typename{frame\_descr} specifies
        \begin{itemize}
          \item<2-> The size of the function stack frame
          \item<3-> Where live values are on the stack (so that the GC can mark them as live and prevent from being swept)
        \end{itemize}
        \item<4-> When compiled with debug information, \typename{frame\_descr} will be linked to a \typename{frame\_debuginfo}
        \begin{itemize}
          \item<5-> Contains information about the position in the source code (file name, line and column number)
        \end{itemize}
      \end{itemize}
    \end{column}
    \begin{column}{0.5\textwidth}
        \onslide*<1>{\listFrameDescr[highlightlines={118-122}]{}}
        \onslide*<2>{\listFrameDescr[highlightlines={120}]{}}
        \onslide*<3>{\listFrameDescr[highlightlines={121}]{}}
        \onslide*<4->{\listFrameDescr[highlightlines={122}]{}}
        \medskip
        \onslide*<1-3>{\listDebugInfo{}}
        \onslide*<4>{\listDebugInfo[highlightlines={38,49}]{}}
        \onslide*<5>{\listDebugInfo[highlightlines={39-46}]{}}
    \end{column}
  \end{columns}
\end{frame}

\begin{frame}{Backtraces}{Scanning the stack with \typename{frame\_descr}}
  \begin{columns}[c]
    \begin{column}{0.5\textwidth}
      \begin{itemize}
        \item Given the return address of the code that raises the exception by calling \funcname{caml\_raise\_exn} (\funcarg{pc} argument of \funcname{caml\_stash\_backtrace})
        \item And the position of the stack pointer (\funcarg{sp} argument of \funcname{caml\_stash\_backtrace})
        \item The frame size given by \typename{frame\_descr}.\funcarg{frame\_size}, allows to find the end of the previous frame
        \item And the return address of the caller of this function
        \bigskip
        \onslide<3->{
        \item Note that traps (here the reddish \funcname{ExnA}, \funcname{ExnB} and \funcname{ExnC} blocks) belong to the frame that installed them, and so the frame size changes when traps are removed
        }
      \end{itemize}
    \end{column}
    \begin{column}{0.5\textwidth}
      \raggedleft
      \onslide*<1>{\providecommand\step{2}\input{diagrams/backtrace/backtrace.tex}}%
      \onslide*<2>{\providecommand\step{1}\input{diagrams/backtrace/scanstack.tex}}%
      \onslide*<3>{\providecommand\step{2}\input{diagrams/backtrace/scanstack.tex}}%
      \onslide*<4>{\providecommand\step{3}\input{diagrams/backtrace/scanstack.tex}}%
      \onslide*<5>{\providecommand\step{4}\input{diagrams/backtrace/scanstack.tex}}%
      \onslide*<6>{\providecommand\step{5}\input{diagrams/backtrace/scanstack.tex}}%
    \end{column}
  \end{columns}
\end{frame}

% Duplicate of previous-previous slide
\begin{frame}{Backtraces}{Continuing on \funcname{caml\_stash\_backtrace}}
  \begin{columns}[c]
    \begin{column}{0.5\textwidth}
      \minipage[c][0.75\textheight][s]{\columnwidth}
      \begin{itemize}
          \color{gray}
        \item A C function provided by the runtime (specific to native code)
        \item It scans the stack, between the stack pointer at the raising point up to the next trap, in order to collect the names of the detected function frames
        \item It uses the same technique and information as the Garbage Collector (GC) uses to scan the values live on the stack
      \end{itemize}
      \begin{itemize}
        \item For each function frame detected, store a pointer to the \typename{frame\_descr} in the \localname{domain\_state->backtrace\_buffer} array, effectively constructing the backtrace
      \end{itemize}
      \onslide<2->{
        \begin{itemize}
            \smallskip
          \item Constructed backtrace:
            \funcname{definitely\_raise},
            \onslide<3->{\funcname{probably\_raise}}
        \end{itemize}
      }
      \vfill
      \mintocaml[firstline=9,lastline=13,highlightlines=12]{../src/backtrace/backtrace.ml}
      \endminipage
    \end{column}
    \begin{column}{0.5\textwidth}
      \raggedleft
      \onslide*<1>{\listStashBacktrace[highlightlines={114-115}]{}}
      \onslide*<2>{\providecommand\step{3}\input{diagrams/backtrace/backtrace.tex}}%
      \onslide*<3>{\providecommand\step{4}\input{diagrams/backtrace/backtrace.tex}}%
    \end{column}
  \end{columns}
\end{frame}

\begin{frame}{Backtraces}{Back to \funcname{caml\_raise\_exn}}
  \begin{columns}[c]
    \begin{column}{0.5\textwidth}
      \minipage[c][0.66\textheight][s]{\columnwidth}
      \begin{itemize}\color{gray}
        \item Checks wheteher exceptions backtrace should be recorded, by checking the \funcarg{domain\_state->backtrace\_active} value
        \item Switches to the C stack to be able to execute C code
        \item Calls \funcname{caml\_stash\_backtrace}, to collect the backtrace up to the next trap
      \end{itemize}
      \onslide*<2->{
        \begin{itemize}
          \item<2-> Executes the exception handler in \funcname{probably\_raise} with \funcname{RESTORE\_EXN\_HANDLER\_OCAML}
            \begin{itemize}
              \item Set \regname{RSP} to \localname{domain\_state->exn\_handler} value
              \item Restore \localname{domain\_state->exn\_handler} from trap
              \item Jump to the handler code with \funcname{ret}
            \end{itemize}
        \end{itemize}
      }
      \vfill
      \onslide*<1>{\mintocaml[firstline=9,lastline=13,highlightlines=12]{../src/backtrace/backtrace.ml}}
      \onslide*<2->{\mintocaml[firstline=15,lastline=21,highlightlines={19-21}]{../src/backtrace/backtrace.ml}}
      \endminipage
    \end{column}
    \begin{column}{0.5\textwidth}
      \centering
      \onslide*<1>{
        \mintc[firstline=190,lastline=192]{../ocaml/runtime/amd64.S}
        \mintc[firstline=234,lastline=237]{../ocaml/runtime/amd64.S}
      }
      \onslide*<2>{
        \mintc[firstline=190,lastline=192,highlightlines={191-192}]{../ocaml/runtime/amd64.S}
        \mintc[firstline=234,lastline=237,highlightlines={235-237}]{../ocaml/runtime/amd64.S}
      }
      \onslide*<1>{\listCamlRaiseExn[highlightlines=720]{}}
      \onslide*<2>{\listCamlRaiseExn[highlightlines=722]{}}
      \onslide*<3->{\providecommand\step{5}\input{diagrams/backtrace/backtrace.tex}}
    \end{column}
  \end{columns}
\end{frame}

\begin{frame}{Backtraces}{\funcname{caml\_probably\_raise}'s handler doesn't catch \typename{ExnC}}
  \begin{columns}[c]
    \begin{column}{0.45\textwidth}
      \minipage[c][0.75\textheight][s]{\columnwidth}
      \begin{itemize}
        \item<1-> \funcname{match \dots with}\footnotemark pattern matching can also match exceptions
        \item<1-> The CMM of \funcname{probably\_raise} shows that this \funcname{match \dots with} is implemented with a pattern matching and a \funcname{try \dots with}
        \item<1-> But this one only matches on \typename{ExnA} exception
        \item<2-> Since the pattern matching fails to catch the exception, \funcname{reraise} is called with our current \typename{ExnC} exception
        \item<3-> Disassembly shows that \funcname{reraise} is actually a call to \funcname{caml\_reraise\_exn}, which is also an assembly function provided by the runtime
      \end{itemize}
      \vfill
      \mintocaml[firstline=15,lastline=21,highlightlines={18-21}]{../src/backtrace/backtrace.ml}
      \endminipage
    \end{column}
    \begin{column}{0.55\textwidth}
      \centering
      \onslide*<1>{\mintcmm[highlightlines={10-18}]{../src/backtrace/camlBacktrace\_\_probably\_raise\_351.cmm}}
      \onslide*<2>{\listcmm[highlightlines=18]{../src/backtrace/camlBacktrace\_\_probably\_raise_351.cmm}{CMM of probably\_raise}}
      \onslide*<3>{\mintobjdump[firstline=5,lastline=35,highlightlines=31]{../src/backtrace/camlBacktrace\_\_probably\_raise_351.objdump}}
    \end{column}
  \end{columns}
  \only<1>{\footnotetext[1]{https://blog.janestreet.com/pattern-matching-and-exception-handling-unite/}}
\end{frame}

\newcommand\listCamlReraiseExn[1][]{
  \listasm[firstline=727,lastline=734,#1]{../ocaml/runtime/amd64.S}{runtime/amd64.S:727}}

\begin{frame}{Backtraces}{Introducing \funcname{caml\_reraise\_exn}}
  \begin{columns}[c]
    \begin{column}{0.5\textwidth}
      \minipage[c][0.66\textheight][s]{\columnwidth}
      \begin{itemize}
        \item<1-> The exception have to be reraised to the next handler
        \item<2-> First, checks if the backtrace should be collected with \funcarg{domain\_state->backtrace\_active}
        \only<3>{
        \begin{itemize}
            \item If not, behaves like a \funcname{raise\_notrace} with \funcname{RESTORE\_EXN\_HANDLER\_OCAML}
        \end{itemize}
        }
        \item<4-> Jumps into \funcname{caml\_raise\_exn}
        \item<5-> Calls \funcname{caml\_stash\_backtrace} again, to scan the next part of the stack: from the trap just pop-ed to the next trap
      \end{itemize}
      \vfill
      \mintocaml[firstline=15,lastline=21,highlightlines={18-21}]{../src/backtrace/backtrace.ml}
      \endminipage
    \end{column}
    \begin{column}{0.5\textwidth}
      \raggedleft
      \onslide*<1>{\listCamlReraiseExn{}}
      \onslide*<2>{\listCamlReraiseExn[highlightlines=729]{}}
      \onslide*<3>{
        \mintc[firstline=190,lastline=192,highlightlines={191-192}]{../ocaml/runtime/amd64.S}
        \mintc[firstline=234,lastline=237,highlightlines={235-237}]{../ocaml/runtime/amd64.S}
        \medskip
        \listCamlReraiseExn[highlightlines=731]{}
      }
      \onslide*<4-5>{
        % TODO refactor with \listCamlReraiseExn
        \mintasm[firstline=727,lastline=734,highlightlines=730]{../ocaml/runtime/amd64.S}
        \medskip
      }
      \onslide*<4>{\listCamlRaiseExn[highlightlines=711]{}}
      \onslide*<5>{\listCamlRaiseExn[highlightlines=720]{}}
    \end{column}
  \end{columns}
\end{frame}

\begin{frame}{Backtraces}{Backtrace collection continues}
  \begin{columns}[c]
    \begin{column}{0.55\textwidth}
      \minipage[c][0.75\textheight][s]{\columnwidth}
      \begin{itemize}
        \item<1-> \funcname{caml\_stash\_backtrace} scans the older part of the stack, up to the next trap
        \item<6-> Controls jumps to the nested exception handler in \funcname{maybe\_raise}, which matches only on \typename{ExnB}
        \item<7-> \funcname{caml\_reraise\_exn} is called again, backtrace collection resumes with \funcname{caml\_stash\_backtrace}
      \end{itemize}
      \medskip
      \begin{itemize}
        \item<2-> Constructed backtrace:
        \begin{itemize}
          \item <2-> \funcname{definitely\_raise}
          \item <2-> \funcname{probably\_raise}
          \item <3-> \funcname{probably\_raise} (re-raise)
          \item <4-> \funcname{maybe\_raise}
          \item <5-> \funcname{main}
          \item <8-> \funcname{main} (re-raise)
        \end{itemize}
      \end{itemize}
      \vfill
      \onslide*<1-5>{\mintocaml[firstline=15,lastline=21]{../src/backtrace/backtrace.ml}}
      \onslide*<6>{\mintocaml[firstline=28,lastline=34,highlightlines=31]{../src/backtrace/backtrace.ml}}
      \onslide*<7->{\mintocaml[firstline=28,lastline=34,highlightlines=33]{../src/backtrace/backtrace.ml}}
      \endminipage
    \end{column}
    \begin{column}{0.45\textwidth}
      \raggedleft
      \onslide*<1>{\providecommand\step{6}\input{diagrams/backtrace/backtrace.tex}}%
      \onslide*<2>{\providecommand\step{6}\input{diagrams/backtrace/backtrace.tex}}%
      \onslide*<3>{\providecommand\step{7}\input{diagrams/backtrace/backtrace.tex}}%
      \onslide*<4>{\providecommand\step{8}\input{diagrams/backtrace/backtrace.tex}}%
      \onslide*<5>{\providecommand\step{9}\input{diagrams/backtrace/backtrace.tex}}%
      \onslide*<6>{\providecommand\step{10}\input{diagrams/backtrace/backtrace.tex}}%
      \onslide*<7>{\providecommand\step{11}\input{diagrams/backtrace/backtrace.tex}}%
      \onslide*<8>{\providecommand\step{12}\input{diagrams/backtrace/backtrace.tex}}%
      \onslide*<9>{\providecommand\step{13}\input{diagrams/backtrace/backtrace.tex}}%
    \end{column}
  \end{columns}
\end{frame}

\begin{frame}{Backtraces}{Printing the backtrace}
  \begin{columns}[c]
    \begin{column}{0.45\textwidth}
      \minipage[c][0.66\textheight][s]{\columnwidth}
      \begin{itemize}
        \item<1-> The exception backtrace can now be printed to \funcarg{stdout} with \funcname{Printexc.print_backtrace}
        \item<2-> First the exception backtrace is retrieved into an OCaml array with \funcname{get_raw_backtrace }
        \item<3-> Which is implemented as the \funcname{caml_get_exception_raw_backtrace} C function in the runtime
        \item<4-> And finally printed with \funcname{print_exception_backtrace}
      \end{itemize}
      \vfill
      \mintocaml[firstline=28,lastline=34,highlightlines=33]{../src/backtrace/backtrace.ml}
      \endminipage
    \end{column}
    \begin{column}{0.55\textwidth}
      \onslide*<1,2>{
        \mintocaml[firstline=100,lastline=101]{../ocaml/stdlib/printexc.ml}
      }
      \only<1>{
        \listocaml[firstline=170,lastline=172]{../ocaml/stdlib/printexc.ml}{stdlib/printexc.ml:170}
      }
      \only<2>{
        \listocaml[firstline=170,lastline=172,highlightlines=172]{../ocaml/stdlib/printexc.ml}{stdlib/printexc.ml:170}
      }
      \only<3>{
        \mintc[firstline=154,lastline=156]{../ocaml/runtime/backtrace.c}
        \listc[firstline=171,lastline=192,highlightlines={184-187}]{../ocaml/runtime/backtrace.c}{runtime/backtrace.c:154}
      }
      \only<4>{
        \listocaml[firstline=155,lastline=165]{../ocaml/stdlib/printexc.ml}{stdlib/printexc.ml:55}
      }
    \end{column}
  \end{columns}
\end{frame}


\subsection{The multiple kinds of raise}
\frameSubsection{}{}

\begin{frame}{Backtraces}{When backtraces are not needed}
  \begin{columns}[c]
    \begin{column}{0.45\textwidth}
      \minipage[c][0.66\textheight][s]{\columnwidth}
      \begin{itemize}
        \item<1-> What if the backtrace for an exception is never needed?
        \item<2-> It's possible to raise the exception explicitly with \funcname{raise\_notrace} to make raising as cheap as possible
        \item<3-> An example of use in the \funcname{dynlink} library of the OCaml compiler
      \end{itemize}
      \vfill
      \onslide<3->{
      \listocaml[firstline=205,lastline=209,highlightlines=207]{../ocaml/otherlibs/dynlink/dynlink\_compilerlibs/misc.ml}{otherlibs/dynlink/dynlink\_compilerlibs/misc.ml:205}
      }
      \endminipage
    \end{column}
    \begin{column}{0.55\textwidth}
    \only<4>{
      \mintcmm[highlightlines=3]{../src/notrace/camlNotrace\_\_fun_347.cmm}
      \listcmm{../src/notrace/camlNotrace\_\_all\_somes\_327.cmm}{all\_somes}
    }
    \onslide*<5->{
      \listobjdump[highlightlines={6-9}]{../src/notrace/camlNotrace\_\_fun_347.objdump}{anonymous function from \funcname{all\_somes}}
    }
    \end{column}
  \end{columns}
\end{frame}

\begin{frame}{Backtraces}{Summary table of the multiple kinds of raise}
  \begin{itemize}
    \item When debugging information is enabled and if the backtrace of an exception isn't needed, it's possible to raise with \funcname{raise\_notrace} for performance
    \item When debugging information is \emph{not} enabled, the compiler optimises \funcname{raise} calls to inline assembly (same effect as using \funcname{raise\_notrace})
  \end{itemize}
  \bigskip
  \centering
  \begin{tabular}{| l | c | c | c | c |}
    \hline
    \parbox{10em}{Debug information \\ (\funcarg{-g} flag)}  & \multicolumn{2}{|c|}{Disabled}                                     & \multicolumn{2}{|c|}{Enabled} \\
    \hline
    Raise kind                                               & \funcname{raise} & \funcname{raise\_notrace}                       & \funcname{raise} & \funcname{raise\_notrace} \\
    \hline
    Compilation                                              & \parbox{8em}{inline assembly\\ (optimization)} & inline assembly   & \funcname{caml\_raise\_exn} & inline assembly \\
    \hline
  \end{tabular}
\end{frame}


%
% Takeaway #5
%
\subsection*{Takeaway \#5}
\frameSubsectionTakeaway{}
\againframe<5>{takeaway}
}%
    \end{column}
  \end{columns}
\end{frame}

\newcommand\listStashBacktrace[1][]{
  \listc[firstline=91,lastline=120,#1]{../ocaml/runtime/backtrace\_nat.c}{runtime/backtrace\_nat.c:91}}

\begin{frame}{Backtraces}{Introducing \funcname{caml\_stash\_backtrace}}
  \begin{columns}[c]
    \begin{column}{0.5\textwidth}
      \minipage[c][0.66\textheight][s]{\columnwidth}
      \begin{itemize}
        \item<1-> A C function provided by the runtime (specific to native code)
        \item<2-> It scans the stack, between the stack pointer at the raising point up to the next trap, in order to collect the names of the detected function frames
          \onslide*<2>{
            \begin{itemize}
              \item And more information, like line number, column number...
            \end{itemize}
          }
        \item<3-> It uses the same technique and information as the Garbage Collector (GC) uses to scan the values live on the stack
          \onslide*<4>{
            \begin{itemize}
              \item \typename{frame\_descr} are at the core of stack unwinding
            \end{itemize}
          }
      \end{itemize}
      \vfill
      \mintocaml[firstline=9,lastline=13,highlightlines=12]{../src/backtrace/backtrace.ml}
      \endminipage
    \end{column}
    \begin{column}{0.5\textwidth}
      \onslide*<1>{\listStashBacktrace[highlightlines=91]{}}
      \onslide*<2>{\listStashBacktrace[highlightlines={91,117-118}]{}}
      \onslide*<3->{\listStashBacktrace[highlightlines={94,106,109-110}]{}}
    \end{column}
  \end{columns}
\end{frame}

\newcommand\listFrameDescr[1][]{
  \listocaml[firstline=111,lastline=124,#1]{../ocaml/asmcomp/emitaux.ml}{asmcomp/emitaux.ml:111}}
\newcommand\listDebugInfo[1][]{
  \listocaml[firstline=38,lastline=49,#1]{../ocaml/lambda/debuginfo.mli}{lambda/debuginfo.mli:38}}

\begin{frame}{Backtraces}{\typename{frame\_descr} and \typename{frame\_debuginfo} in a nutshell}
  \begin{columns}[c]
    \begin{column}{0.5\textwidth}
      \begin{itemize}
        \item<1-> For every return address in an OCaml program, there is a associated \typename{frame\_descr}
        \item<2-> A \typename{frame\_descr} specifies
        \begin{itemize}
          \item<2-> The size of the function stack frame
          \item<3-> Where live values are on the stack (so that the GC can mark them as live and prevent from being swept)
        \end{itemize}
        \item<4-> When compiled with debug information, \typename{frame\_descr} will be linked to a \typename{frame\_debuginfo}
        \begin{itemize}
          \item<5-> Contains information about the position in the source code (file name, line and column number)
        \end{itemize}
      \end{itemize}
    \end{column}
    \begin{column}{0.5\textwidth}
        \onslide*<1>{\listFrameDescr[highlightlines={118-122}]{}}
        \onslide*<2>{\listFrameDescr[highlightlines={120}]{}}
        \onslide*<3>{\listFrameDescr[highlightlines={121}]{}}
        \onslide*<4->{\listFrameDescr[highlightlines={122}]{}}
        \medskip
        \onslide*<1-3>{\listDebugInfo{}}
        \onslide*<4>{\listDebugInfo[highlightlines={38,49}]{}}
        \onslide*<5>{\listDebugInfo[highlightlines={39-46}]{}}
    \end{column}
  \end{columns}
\end{frame}

\begin{frame}{Backtraces}{Scanning the stack with \typename{frame\_descr}}
  \begin{columns}[c]
    \begin{column}{0.5\textwidth}
      \begin{itemize}
        \item Given the return address of the code that raises the exception by calling \funcname{caml\_raise\_exn} (\funcarg{pc} argument of \funcname{caml\_stash\_backtrace})
        \item And the position of the stack pointer (\funcarg{sp} argument of \funcname{caml\_stash\_backtrace})
        \item The frame size given by \typename{frame\_descr}.\funcarg{frame\_size}, allows to find the end of the previous frame
        \item And the return address of the caller of this function
        \bigskip
        \onslide<3->{
        \item Note that traps (here the reddish \funcname{ExnA}, \funcname{ExnB} and \funcname{ExnC} blocks) belong to the frame that installed them, and so the frame size changes when traps are removed
        }
      \end{itemize}
    \end{column}
    \begin{column}{0.5\textwidth}
      \raggedleft
      \onslide*<1>{\providecommand\step{2}%
% Backtraces
%
\subsection{One advantage of raising with debugging information}
\frameSubsection{}{}

\begin{frame}{Backtraces}{Debugging information \& source code}
  \begin{columns}[c]
    \begin{column}{0.5\textwidth}
      \begin{itemize}
        \item Raising an exception is fun and games, but how do we know where the exception originated? $\rightarrow$ With the backtrace associated with the exception!
        \item In order to get a backtrace from the point where an exception was raised, it's necessary to compile with debugging information enabled (\funcarg{-g} flag for the compiler)
        \item Same example as in the `Nested handlers' section
      \end{itemize}
    \end{column}
    \begin{column}{0.5\textwidth}
      \listocaml[firstline=9,lastline=34]{../src/backtrace/backtrace.ml}{backtrace/backtrace.ml}
    \end{column}
  \end{columns}
\end{frame}

\begin{frame}{Backtraces}{CMM and disassembly of \funcname{definitely\_raise}}
  \begin{columns}[c]
    \begin{column}{0.5\textwidth}
      \minipage[c][0.66\textheight][s]{\columnwidth}
      \begin{itemize}
        \item<1-> An effect of compiling with debugging information (\funcarg{-g}): the CMM no longer uses \funcname{raise\_notrace} to raise the exception but \funcname{raise} instead
        \item<2-> Disassembly shows that CMM \funcname{raise} is actually a call to \funcname{caml\_raise\_exn}, a function in assembler provided by the runtime
      \end{itemize}
      \vfill
      \mintocaml[firstline=9,lastline=13,highlightlines=12]{../src/backtrace/backtrace.ml}
      \endminipage
    \end{column}
    \begin{column}{0.5\textwidth}
      \onslide*<1>{
        \mintcmm[highlightlines=5]{../src/backtrace/camlBacktrace\_\_definitely\_raise\_347.cmm}
      }
      \onslide*<2>{
        \mintobjdump[firstline=2,highlightlines=14]{../src/backtrace/camlBacktrace\_\_definitely\_raise\_347.objdump}
      }
    \end{column}
  \end{columns}
\end{frame}

\begin{frame}{Backtraces}{Introducing \funcname{caml\_raise\_exn}}
  \begin{columns}[c]
    \begin{column}{0.5\textwidth}
      \minipage[c][0.66\textheight][s]{\columnwidth}
      \onslide*<1>{
        \begin{itemize}
          \item Raising the exception is performed using \funcname{caml\_raise\_exn} which is
            \begin{itemize}
              \item An assembly function provided by the runtime
              \item Architecture specific
            \end{itemize}
          \item Let's see what it does differently from \funcname{raise\_notrace}\dots
          \item We are about to raise \typename{ExnC} with \funcname{caml\_raise\_exn} from \funcname{definitely\_raise}
        \end{itemize}
      }
      \onslide*<2>{
        \mintobjdump[firstline=2,highlightlines=14]{../src/backtrace/camlBacktrace\_\_definitely\_raise\_347.objdump}
      }
      \vfill
      \mintocaml[firstline=9,lastline=13,highlightlines=12]{../src/backtrace/backtrace.ml}
      \endminipage
    \end{column}
    \begin{column}{0.5\textwidth}
      \centering
      \providecommand\step{1}\input{diagrams/backtrace/backtrace.tex}
    \end{column}
  \end{columns}
\end{frame}

\begin{frame}{Backtraces}{But before that, we need more fields from \typename{caml\_domain\_state}}
  \begin{columns}
    \begin{column}{0.5\textwidth}
      \begin{itemize}
        \item Remember \typename{caml\_domain\_state} the per-domain runtime state?
        \item Some other members are of interest to us now\dots
        \item \localname{backtrace\_active} is used to know if the runtime should collect backtraces
        \item \localname{backtrace\_active} can be set by
          \begin{itemize}
            \item The \funcarg{b} flag in the \funcname{OCAMLRUNPARAM} environment variable
            \item \mintinline{ocaml}{Printexc.record_backtrace : bool -> unit} \footnotemark
          \end{itemize}
      \end{itemize}
    \end{column}
    \begin{column}{0.5\textwidth}
      \begin{listing}
        \mintc[highlightlines={9-11}]{src/ocaml/domain\_state\_backtrace.h}
        \caption{runtime/caml/domain\_state.h}
      \end{listing}
    \end{column}
  \end{columns}
  \footnotetext[1]{https://v2.ocaml.org/api/Printexc.html}
\end{frame}

\newcommand\listCamlRaiseExn[1][]{
  \listasm[firstline=702,lastline=725,#1]{../ocaml/runtime/amd64.S}{runtime/amd64.S:702}}

\begin{frame}{Backtraces}{Walk through \funcname{caml\_raise\_exn}}
  \begin{columns}[T]
    \begin{column}{0.5\textwidth}
      \begin{itemize}
        \item<1-> First checks whether exceptions backtrace should be recorded
        \item<1-> By checking the \funcarg{domain\_state->backtrace\_active} value
        \item<2-> If not, acts as \funcname{raise\_notrace} with \funcname{RESTORE\_EXN\_HANDLER\_OCAML}
          \begin{itemize}
            \item Set \regname{RSP} to \localname{domain\_state->exn\_handler} value
            \item Restore \localname{domain\_state->exn\_handler} from trap
            \item Jump with \funcname{ret} (same effect as using \funcname{pop} and \funcname{jmp})
          \end{itemize}
        \item<3-> Otherwise, switches to the C stack to be able to execute C code
        \item<4-> Calls \funcname{caml\_stash\_backtrace}, a C function provided by the runtime\ignorespaces
          \onslide<6->{, with
            \begin{itemize}
              \item \funcarg{exn}: exception value
              \item \funcarg{pc}: code address of the raising point
              \item \funcarg{sp}: stack address of the raising function
              \item \funcarg{trapsp}: stack address of the next handler
            \end{itemize}
          }
      \end{itemize}
      \bigskip
      \mintocaml[firstline=9,lastline=13,highlightlines=12]{../src/backtrace/backtrace.ml}
    \end{column}
    \begin{column}{0.5\textwidth}
      \raggedleft
      \onslide*<1,3-4>{
        \mintc[firstline=190,lastline=192]{../ocaml/runtime/amd64.S}
        \mintc[firstline=234,lastline=237]{../ocaml/runtime/amd64.S}
      }
      \onslide*<2>{
        \mintc[firstline=190,lastline=192,highlightlines={191-192}]{../ocaml/runtime/amd64.S}
        \mintc[firstline=234,lastline=237,highlightlines={235-237}]{../ocaml/runtime/amd64.S}
      }
      \onslide*<1>{\listCamlRaiseExn[highlightlines=705]{}}%
      \onslide*<2>{\listCamlRaiseExn[highlightlines={707-708}]{}}%
      \onslide*<3>{\listCamlRaiseExn[highlightlines={712-714}]{}}%
      \onslide*<4>{\listCamlRaiseExn[highlightlines={715-720}]{}}%
      \onslide*<5>{\providecommand\step{1}\input{diagrams/backtrace/backtrace.tex}}%
      \onslide*<6>{\providecommand\step{2}\input{diagrams/backtrace/backtrace.tex}}%
    \end{column}
  \end{columns}
\end{frame}

\newcommand\listStashBacktrace[1][]{
  \listc[firstline=91,lastline=120,#1]{../ocaml/runtime/backtrace\_nat.c}{runtime/backtrace\_nat.c:91}}

\begin{frame}{Backtraces}{Introducing \funcname{caml\_stash\_backtrace}}
  \begin{columns}[c]
    \begin{column}{0.5\textwidth}
      \minipage[c][0.66\textheight][s]{\columnwidth}
      \begin{itemize}
        \item<1-> A C function provided by the runtime (specific to native code)
        \item<2-> It scans the stack, between the stack pointer at the raising point up to the next trap, in order to collect the names of the detected function frames
          \onslide*<2>{
            \begin{itemize}
              \item And more information, like line number, column number...
            \end{itemize}
          }
        \item<3-> It uses the same technique and information as the Garbage Collector (GC) uses to scan the values live on the stack
          \onslide*<4>{
            \begin{itemize}
              \item \typename{frame\_descr} are at the core of stack unwinding
            \end{itemize}
          }
      \end{itemize}
      \vfill
      \mintocaml[firstline=9,lastline=13,highlightlines=12]{../src/backtrace/backtrace.ml}
      \endminipage
    \end{column}
    \begin{column}{0.5\textwidth}
      \onslide*<1>{\listStashBacktrace[highlightlines=91]{}}
      \onslide*<2>{\listStashBacktrace[highlightlines={91,117-118}]{}}
      \onslide*<3->{\listStashBacktrace[highlightlines={94,106,109-110}]{}}
    \end{column}
  \end{columns}
\end{frame}

\newcommand\listFrameDescr[1][]{
  \listocaml[firstline=111,lastline=124,#1]{../ocaml/asmcomp/emitaux.ml}{asmcomp/emitaux.ml:111}}
\newcommand\listDebugInfo[1][]{
  \listocaml[firstline=38,lastline=49,#1]{../ocaml/lambda/debuginfo.mli}{lambda/debuginfo.mli:38}}

\begin{frame}{Backtraces}{\typename{frame\_descr} and \typename{frame\_debuginfo} in a nutshell}
  \begin{columns}[c]
    \begin{column}{0.5\textwidth}
      \begin{itemize}
        \item<1-> For every return address in an OCaml program, there is a associated \typename{frame\_descr}
        \item<2-> A \typename{frame\_descr} specifies
        \begin{itemize}
          \item<2-> The size of the function stack frame
          \item<3-> Where live values are on the stack (so that the GC can mark them as live and prevent from being swept)
        \end{itemize}
        \item<4-> When compiled with debug information, \typename{frame\_descr} will be linked to a \typename{frame\_debuginfo}
        \begin{itemize}
          \item<5-> Contains information about the position in the source code (file name, line and column number)
        \end{itemize}
      \end{itemize}
    \end{column}
    \begin{column}{0.5\textwidth}
        \onslide*<1>{\listFrameDescr[highlightlines={118-122}]{}}
        \onslide*<2>{\listFrameDescr[highlightlines={120}]{}}
        \onslide*<3>{\listFrameDescr[highlightlines={121}]{}}
        \onslide*<4->{\listFrameDescr[highlightlines={122}]{}}
        \medskip
        \onslide*<1-3>{\listDebugInfo{}}
        \onslide*<4>{\listDebugInfo[highlightlines={38,49}]{}}
        \onslide*<5>{\listDebugInfo[highlightlines={39-46}]{}}
    \end{column}
  \end{columns}
\end{frame}

\begin{frame}{Backtraces}{Scanning the stack with \typename{frame\_descr}}
  \begin{columns}[c]
    \begin{column}{0.5\textwidth}
      \begin{itemize}
        \item Given the return address of the code that raises the exception by calling \funcname{caml\_raise\_exn} (\funcarg{pc} argument of \funcname{caml\_stash\_backtrace})
        \item And the position of the stack pointer (\funcarg{sp} argument of \funcname{caml\_stash\_backtrace})
        \item The frame size given by \typename{frame\_descr}.\funcarg{frame\_size}, allows to find the end of the previous frame
        \item And the return address of the caller of this function
        \bigskip
        \onslide<3->{
        \item Note that traps (here the reddish \funcname{ExnA}, \funcname{ExnB} and \funcname{ExnC} blocks) belong to the frame that installed them, and so the frame size changes when traps are removed
        }
      \end{itemize}
    \end{column}
    \begin{column}{0.5\textwidth}
      \raggedleft
      \onslide*<1>{\providecommand\step{2}\input{diagrams/backtrace/backtrace.tex}}%
      \onslide*<2>{\providecommand\step{1}\input{diagrams/backtrace/scanstack.tex}}%
      \onslide*<3>{\providecommand\step{2}\input{diagrams/backtrace/scanstack.tex}}%
      \onslide*<4>{\providecommand\step{3}\input{diagrams/backtrace/scanstack.tex}}%
      \onslide*<5>{\providecommand\step{4}\input{diagrams/backtrace/scanstack.tex}}%
      \onslide*<6>{\providecommand\step{5}\input{diagrams/backtrace/scanstack.tex}}%
    \end{column}
  \end{columns}
\end{frame}

% Duplicate of previous-previous slide
\begin{frame}{Backtraces}{Continuing on \funcname{caml\_stash\_backtrace}}
  \begin{columns}[c]
    \begin{column}{0.5\textwidth}
      \minipage[c][0.75\textheight][s]{\columnwidth}
      \begin{itemize}
          \color{gray}
        \item A C function provided by the runtime (specific to native code)
        \item It scans the stack, between the stack pointer at the raising point up to the next trap, in order to collect the names of the detected function frames
        \item It uses the same technique and information as the Garbage Collector (GC) uses to scan the values live on the stack
      \end{itemize}
      \begin{itemize}
        \item For each function frame detected, store a pointer to the \typename{frame\_descr} in the \localname{domain\_state->backtrace\_buffer} array, effectively constructing the backtrace
      \end{itemize}
      \onslide<2->{
        \begin{itemize}
            \smallskip
          \item Constructed backtrace:
            \funcname{definitely\_raise},
            \onslide<3->{\funcname{probably\_raise}}
        \end{itemize}
      }
      \vfill
      \mintocaml[firstline=9,lastline=13,highlightlines=12]{../src/backtrace/backtrace.ml}
      \endminipage
    \end{column}
    \begin{column}{0.5\textwidth}
      \raggedleft
      \onslide*<1>{\listStashBacktrace[highlightlines={114-115}]{}}
      \onslide*<2>{\providecommand\step{3}\input{diagrams/backtrace/backtrace.tex}}%
      \onslide*<3>{\providecommand\step{4}\input{diagrams/backtrace/backtrace.tex}}%
    \end{column}
  \end{columns}
\end{frame}

\begin{frame}{Backtraces}{Back to \funcname{caml\_raise\_exn}}
  \begin{columns}[c]
    \begin{column}{0.5\textwidth}
      \minipage[c][0.66\textheight][s]{\columnwidth}
      \begin{itemize}\color{gray}
        \item Checks wheteher exceptions backtrace should be recorded, by checking the \funcarg{domain\_state->backtrace\_active} value
        \item Switches to the C stack to be able to execute C code
        \item Calls \funcname{caml\_stash\_backtrace}, to collect the backtrace up to the next trap
      \end{itemize}
      \onslide*<2->{
        \begin{itemize}
          \item<2-> Executes the exception handler in \funcname{probably\_raise} with \funcname{RESTORE\_EXN\_HANDLER\_OCAML}
            \begin{itemize}
              \item Set \regname{RSP} to \localname{domain\_state->exn\_handler} value
              \item Restore \localname{domain\_state->exn\_handler} from trap
              \item Jump to the handler code with \funcname{ret}
            \end{itemize}
        \end{itemize}
      }
      \vfill
      \onslide*<1>{\mintocaml[firstline=9,lastline=13,highlightlines=12]{../src/backtrace/backtrace.ml}}
      \onslide*<2->{\mintocaml[firstline=15,lastline=21,highlightlines={19-21}]{../src/backtrace/backtrace.ml}}
      \endminipage
    \end{column}
    \begin{column}{0.5\textwidth}
      \centering
      \onslide*<1>{
        \mintc[firstline=190,lastline=192]{../ocaml/runtime/amd64.S}
        \mintc[firstline=234,lastline=237]{../ocaml/runtime/amd64.S}
      }
      \onslide*<2>{
        \mintc[firstline=190,lastline=192,highlightlines={191-192}]{../ocaml/runtime/amd64.S}
        \mintc[firstline=234,lastline=237,highlightlines={235-237}]{../ocaml/runtime/amd64.S}
      }
      \onslide*<1>{\listCamlRaiseExn[highlightlines=720]{}}
      \onslide*<2>{\listCamlRaiseExn[highlightlines=722]{}}
      \onslide*<3->{\providecommand\step{5}\input{diagrams/backtrace/backtrace.tex}}
    \end{column}
  \end{columns}
\end{frame}

\begin{frame}{Backtraces}{\funcname{caml\_probably\_raise}'s handler doesn't catch \typename{ExnC}}
  \begin{columns}[c]
    \begin{column}{0.45\textwidth}
      \minipage[c][0.75\textheight][s]{\columnwidth}
      \begin{itemize}
        \item<1-> \funcname{match \dots with}\footnotemark pattern matching can also match exceptions
        \item<1-> The CMM of \funcname{probably\_raise} shows that this \funcname{match \dots with} is implemented with a pattern matching and a \funcname{try \dots with}
        \item<1-> But this one only matches on \typename{ExnA} exception
        \item<2-> Since the pattern matching fails to catch the exception, \funcname{reraise} is called with our current \typename{ExnC} exception
        \item<3-> Disassembly shows that \funcname{reraise} is actually a call to \funcname{caml\_reraise\_exn}, which is also an assembly function provided by the runtime
      \end{itemize}
      \vfill
      \mintocaml[firstline=15,lastline=21,highlightlines={18-21}]{../src/backtrace/backtrace.ml}
      \endminipage
    \end{column}
    \begin{column}{0.55\textwidth}
      \centering
      \onslide*<1>{\mintcmm[highlightlines={10-18}]{../src/backtrace/camlBacktrace\_\_probably\_raise\_351.cmm}}
      \onslide*<2>{\listcmm[highlightlines=18]{../src/backtrace/camlBacktrace\_\_probably\_raise_351.cmm}{CMM of probably\_raise}}
      \onslide*<3>{\mintobjdump[firstline=5,lastline=35,highlightlines=31]{../src/backtrace/camlBacktrace\_\_probably\_raise_351.objdump}}
    \end{column}
  \end{columns}
  \only<1>{\footnotetext[1]{https://blog.janestreet.com/pattern-matching-and-exception-handling-unite/}}
\end{frame}

\newcommand\listCamlReraiseExn[1][]{
  \listasm[firstline=727,lastline=734,#1]{../ocaml/runtime/amd64.S}{runtime/amd64.S:727}}

\begin{frame}{Backtraces}{Introducing \funcname{caml\_reraise\_exn}}
  \begin{columns}[c]
    \begin{column}{0.5\textwidth}
      \minipage[c][0.66\textheight][s]{\columnwidth}
      \begin{itemize}
        \item<1-> The exception have to be reraised to the next handler
        \item<2-> First, checks if the backtrace should be collected with \funcarg{domain\_state->backtrace\_active}
        \only<3>{
        \begin{itemize}
            \item If not, behaves like a \funcname{raise\_notrace} with \funcname{RESTORE\_EXN\_HANDLER\_OCAML}
        \end{itemize}
        }
        \item<4-> Jumps into \funcname{caml\_raise\_exn}
        \item<5-> Calls \funcname{caml\_stash\_backtrace} again, to scan the next part of the stack: from the trap just pop-ed to the next trap
      \end{itemize}
      \vfill
      \mintocaml[firstline=15,lastline=21,highlightlines={18-21}]{../src/backtrace/backtrace.ml}
      \endminipage
    \end{column}
    \begin{column}{0.5\textwidth}
      \raggedleft
      \onslide*<1>{\listCamlReraiseExn{}}
      \onslide*<2>{\listCamlReraiseExn[highlightlines=729]{}}
      \onslide*<3>{
        \mintc[firstline=190,lastline=192,highlightlines={191-192}]{../ocaml/runtime/amd64.S}
        \mintc[firstline=234,lastline=237,highlightlines={235-237}]{../ocaml/runtime/amd64.S}
        \medskip
        \listCamlReraiseExn[highlightlines=731]{}
      }
      \onslide*<4-5>{
        % TODO refactor with \listCamlReraiseExn
        \mintasm[firstline=727,lastline=734,highlightlines=730]{../ocaml/runtime/amd64.S}
        \medskip
      }
      \onslide*<4>{\listCamlRaiseExn[highlightlines=711]{}}
      \onslide*<5>{\listCamlRaiseExn[highlightlines=720]{}}
    \end{column}
  \end{columns}
\end{frame}

\begin{frame}{Backtraces}{Backtrace collection continues}
  \begin{columns}[c]
    \begin{column}{0.55\textwidth}
      \minipage[c][0.75\textheight][s]{\columnwidth}
      \begin{itemize}
        \item<1-> \funcname{caml\_stash\_backtrace} scans the older part of the stack, up to the next trap
        \item<6-> Controls jumps to the nested exception handler in \funcname{maybe\_raise}, which matches only on \typename{ExnB}
        \item<7-> \funcname{caml\_reraise\_exn} is called again, backtrace collection resumes with \funcname{caml\_stash\_backtrace}
      \end{itemize}
      \medskip
      \begin{itemize}
        \item<2-> Constructed backtrace:
        \begin{itemize}
          \item <2-> \funcname{definitely\_raise}
          \item <2-> \funcname{probably\_raise}
          \item <3-> \funcname{probably\_raise} (re-raise)
          \item <4-> \funcname{maybe\_raise}
          \item <5-> \funcname{main}
          \item <8-> \funcname{main} (re-raise)
        \end{itemize}
      \end{itemize}
      \vfill
      \onslide*<1-5>{\mintocaml[firstline=15,lastline=21]{../src/backtrace/backtrace.ml}}
      \onslide*<6>{\mintocaml[firstline=28,lastline=34,highlightlines=31]{../src/backtrace/backtrace.ml}}
      \onslide*<7->{\mintocaml[firstline=28,lastline=34,highlightlines=33]{../src/backtrace/backtrace.ml}}
      \endminipage
    \end{column}
    \begin{column}{0.45\textwidth}
      \raggedleft
      \onslide*<1>{\providecommand\step{6}\input{diagrams/backtrace/backtrace.tex}}%
      \onslide*<2>{\providecommand\step{6}\input{diagrams/backtrace/backtrace.tex}}%
      \onslide*<3>{\providecommand\step{7}\input{diagrams/backtrace/backtrace.tex}}%
      \onslide*<4>{\providecommand\step{8}\input{diagrams/backtrace/backtrace.tex}}%
      \onslide*<5>{\providecommand\step{9}\input{diagrams/backtrace/backtrace.tex}}%
      \onslide*<6>{\providecommand\step{10}\input{diagrams/backtrace/backtrace.tex}}%
      \onslide*<7>{\providecommand\step{11}\input{diagrams/backtrace/backtrace.tex}}%
      \onslide*<8>{\providecommand\step{12}\input{diagrams/backtrace/backtrace.tex}}%
      \onslide*<9>{\providecommand\step{13}\input{diagrams/backtrace/backtrace.tex}}%
    \end{column}
  \end{columns}
\end{frame}

\begin{frame}{Backtraces}{Printing the backtrace}
  \begin{columns}[c]
    \begin{column}{0.45\textwidth}
      \minipage[c][0.66\textheight][s]{\columnwidth}
      \begin{itemize}
        \item<1-> The exception backtrace can now be printed to \funcarg{stdout} with \funcname{Printexc.print_backtrace}
        \item<2-> First the exception backtrace is retrieved into an OCaml array with \funcname{get_raw_backtrace }
        \item<3-> Which is implemented as the \funcname{caml_get_exception_raw_backtrace} C function in the runtime
        \item<4-> And finally printed with \funcname{print_exception_backtrace}
      \end{itemize}
      \vfill
      \mintocaml[firstline=28,lastline=34,highlightlines=33]{../src/backtrace/backtrace.ml}
      \endminipage
    \end{column}
    \begin{column}{0.55\textwidth}
      \onslide*<1,2>{
        \mintocaml[firstline=100,lastline=101]{../ocaml/stdlib/printexc.ml}
      }
      \only<1>{
        \listocaml[firstline=170,lastline=172]{../ocaml/stdlib/printexc.ml}{stdlib/printexc.ml:170}
      }
      \only<2>{
        \listocaml[firstline=170,lastline=172,highlightlines=172]{../ocaml/stdlib/printexc.ml}{stdlib/printexc.ml:170}
      }
      \only<3>{
        \mintc[firstline=154,lastline=156]{../ocaml/runtime/backtrace.c}
        \listc[firstline=171,lastline=192,highlightlines={184-187}]{../ocaml/runtime/backtrace.c}{runtime/backtrace.c:154}
      }
      \only<4>{
        \listocaml[firstline=155,lastline=165]{../ocaml/stdlib/printexc.ml}{stdlib/printexc.ml:55}
      }
    \end{column}
  \end{columns}
\end{frame}


\subsection{The multiple kinds of raise}
\frameSubsection{}{}

\begin{frame}{Backtraces}{When backtraces are not needed}
  \begin{columns}[c]
    \begin{column}{0.45\textwidth}
      \minipage[c][0.66\textheight][s]{\columnwidth}
      \begin{itemize}
        \item<1-> What if the backtrace for an exception is never needed?
        \item<2-> It's possible to raise the exception explicitly with \funcname{raise\_notrace} to make raising as cheap as possible
        \item<3-> An example of use in the \funcname{dynlink} library of the OCaml compiler
      \end{itemize}
      \vfill
      \onslide<3->{
      \listocaml[firstline=205,lastline=209,highlightlines=207]{../ocaml/otherlibs/dynlink/dynlink\_compilerlibs/misc.ml}{otherlibs/dynlink/dynlink\_compilerlibs/misc.ml:205}
      }
      \endminipage
    \end{column}
    \begin{column}{0.55\textwidth}
    \only<4>{
      \mintcmm[highlightlines=3]{../src/notrace/camlNotrace\_\_fun_347.cmm}
      \listcmm{../src/notrace/camlNotrace\_\_all\_somes\_327.cmm}{all\_somes}
    }
    \onslide*<5->{
      \listobjdump[highlightlines={6-9}]{../src/notrace/camlNotrace\_\_fun_347.objdump}{anonymous function from \funcname{all\_somes}}
    }
    \end{column}
  \end{columns}
\end{frame}

\begin{frame}{Backtraces}{Summary table of the multiple kinds of raise}
  \begin{itemize}
    \item When debugging information is enabled and if the backtrace of an exception isn't needed, it's possible to raise with \funcname{raise\_notrace} for performance
    \item When debugging information is \emph{not} enabled, the compiler optimises \funcname{raise} calls to inline assembly (same effect as using \funcname{raise\_notrace})
  \end{itemize}
  \bigskip
  \centering
  \begin{tabular}{| l | c | c | c | c |}
    \hline
    \parbox{10em}{Debug information \\ (\funcarg{-g} flag)}  & \multicolumn{2}{|c|}{Disabled}                                     & \multicolumn{2}{|c|}{Enabled} \\
    \hline
    Raise kind                                               & \funcname{raise} & \funcname{raise\_notrace}                       & \funcname{raise} & \funcname{raise\_notrace} \\
    \hline
    Compilation                                              & \parbox{8em}{inline assembly\\ (optimization)} & inline assembly   & \funcname{caml\_raise\_exn} & inline assembly \\
    \hline
  \end{tabular}
\end{frame}


%
% Takeaway #5
%
\subsection*{Takeaway \#5}
\frameSubsectionTakeaway{}
\againframe<5>{takeaway}
}%
      \onslide*<2>{\providecommand\step{1}\input{diagrams/backtrace/scanstack.tex}}%
      \onslide*<3>{\providecommand\step{2}\input{diagrams/backtrace/scanstack.tex}}%
      \onslide*<4>{\providecommand\step{3}\input{diagrams/backtrace/scanstack.tex}}%
      \onslide*<5>{\providecommand\step{4}\input{diagrams/backtrace/scanstack.tex}}%
      \onslide*<6>{\providecommand\step{5}\input{diagrams/backtrace/scanstack.tex}}%
    \end{column}
  \end{columns}
\end{frame}

% Duplicate of previous-previous slide
\begin{frame}{Backtraces}{Continuing on \funcname{caml\_stash\_backtrace}}
  \begin{columns}[c]
    \begin{column}{0.5\textwidth}
      \minipage[c][0.75\textheight][s]{\columnwidth}
      \begin{itemize}
          \color{gray}
        \item A C function provided by the runtime (specific to native code)
        \item It scans the stack, between the stack pointer at the raising point up to the next trap, in order to collect the names of the detected function frames
        \item It uses the same technique and information as the Garbage Collector (GC) uses to scan the values live on the stack
      \end{itemize}
      \begin{itemize}
        \item For each function frame detected, store a pointer to the \typename{frame\_descr} in the \localname{domain\_state->backtrace\_buffer} array, effectively constructing the backtrace
      \end{itemize}
      \onslide<2->{
        \begin{itemize}
            \smallskip
          \item Constructed backtrace:
            \funcname{definitely\_raise},
            \onslide<3->{\funcname{probably\_raise}}
        \end{itemize}
      }
      \vfill
      \mintocaml[firstline=9,lastline=13,highlightlines=12]{../src/backtrace/backtrace.ml}
      \endminipage
    \end{column}
    \begin{column}{0.5\textwidth}
      \raggedleft
      \onslide*<1>{\listStashBacktrace[highlightlines={114-115}]{}}
      \onslide*<2>{\providecommand\step{3}%
% Backtraces
%
\subsection{One advantage of raising with debugging information}
\frameSubsection{}{}

\begin{frame}{Backtraces}{Debugging information \& source code}
  \begin{columns}[c]
    \begin{column}{0.5\textwidth}
      \begin{itemize}
        \item Raising an exception is fun and games, but how do we know where the exception originated? $\rightarrow$ With the backtrace associated with the exception!
        \item In order to get a backtrace from the point where an exception was raised, it's necessary to compile with debugging information enabled (\funcarg{-g} flag for the compiler)
        \item Same example as in the `Nested handlers' section
      \end{itemize}
    \end{column}
    \begin{column}{0.5\textwidth}
      \listocaml[firstline=9,lastline=34]{../src/backtrace/backtrace.ml}{backtrace/backtrace.ml}
    \end{column}
  \end{columns}
\end{frame}

\begin{frame}{Backtraces}{CMM and disassembly of \funcname{definitely\_raise}}
  \begin{columns}[c]
    \begin{column}{0.5\textwidth}
      \minipage[c][0.66\textheight][s]{\columnwidth}
      \begin{itemize}
        \item<1-> An effect of compiling with debugging information (\funcarg{-g}): the CMM no longer uses \funcname{raise\_notrace} to raise the exception but \funcname{raise} instead
        \item<2-> Disassembly shows that CMM \funcname{raise} is actually a call to \funcname{caml\_raise\_exn}, a function in assembler provided by the runtime
      \end{itemize}
      \vfill
      \mintocaml[firstline=9,lastline=13,highlightlines=12]{../src/backtrace/backtrace.ml}
      \endminipage
    \end{column}
    \begin{column}{0.5\textwidth}
      \onslide*<1>{
        \mintcmm[highlightlines=5]{../src/backtrace/camlBacktrace\_\_definitely\_raise\_347.cmm}
      }
      \onslide*<2>{
        \mintobjdump[firstline=2,highlightlines=14]{../src/backtrace/camlBacktrace\_\_definitely\_raise\_347.objdump}
      }
    \end{column}
  \end{columns}
\end{frame}

\begin{frame}{Backtraces}{Introducing \funcname{caml\_raise\_exn}}
  \begin{columns}[c]
    \begin{column}{0.5\textwidth}
      \minipage[c][0.66\textheight][s]{\columnwidth}
      \onslide*<1>{
        \begin{itemize}
          \item Raising the exception is performed using \funcname{caml\_raise\_exn} which is
            \begin{itemize}
              \item An assembly function provided by the runtime
              \item Architecture specific
            \end{itemize}
          \item Let's see what it does differently from \funcname{raise\_notrace}\dots
          \item We are about to raise \typename{ExnC} with \funcname{caml\_raise\_exn} from \funcname{definitely\_raise}
        \end{itemize}
      }
      \onslide*<2>{
        \mintobjdump[firstline=2,highlightlines=14]{../src/backtrace/camlBacktrace\_\_definitely\_raise\_347.objdump}
      }
      \vfill
      \mintocaml[firstline=9,lastline=13,highlightlines=12]{../src/backtrace/backtrace.ml}
      \endminipage
    \end{column}
    \begin{column}{0.5\textwidth}
      \centering
      \providecommand\step{1}\input{diagrams/backtrace/backtrace.tex}
    \end{column}
  \end{columns}
\end{frame}

\begin{frame}{Backtraces}{But before that, we need more fields from \typename{caml\_domain\_state}}
  \begin{columns}
    \begin{column}{0.5\textwidth}
      \begin{itemize}
        \item Remember \typename{caml\_domain\_state} the per-domain runtime state?
        \item Some other members are of interest to us now\dots
        \item \localname{backtrace\_active} is used to know if the runtime should collect backtraces
        \item \localname{backtrace\_active} can be set by
          \begin{itemize}
            \item The \funcarg{b} flag in the \funcname{OCAMLRUNPARAM} environment variable
            \item \mintinline{ocaml}{Printexc.record_backtrace : bool -> unit} \footnotemark
          \end{itemize}
      \end{itemize}
    \end{column}
    \begin{column}{0.5\textwidth}
      \begin{listing}
        \mintc[highlightlines={9-11}]{src/ocaml/domain\_state\_backtrace.h}
        \caption{runtime/caml/domain\_state.h}
      \end{listing}
    \end{column}
  \end{columns}
  \footnotetext[1]{https://v2.ocaml.org/api/Printexc.html}
\end{frame}

\newcommand\listCamlRaiseExn[1][]{
  \listasm[firstline=702,lastline=725,#1]{../ocaml/runtime/amd64.S}{runtime/amd64.S:702}}

\begin{frame}{Backtraces}{Walk through \funcname{caml\_raise\_exn}}
  \begin{columns}[T]
    \begin{column}{0.5\textwidth}
      \begin{itemize}
        \item<1-> First checks whether exceptions backtrace should be recorded
        \item<1-> By checking the \funcarg{domain\_state->backtrace\_active} value
        \item<2-> If not, acts as \funcname{raise\_notrace} with \funcname{RESTORE\_EXN\_HANDLER\_OCAML}
          \begin{itemize}
            \item Set \regname{RSP} to \localname{domain\_state->exn\_handler} value
            \item Restore \localname{domain\_state->exn\_handler} from trap
            \item Jump with \funcname{ret} (same effect as using \funcname{pop} and \funcname{jmp})
          \end{itemize}
        \item<3-> Otherwise, switches to the C stack to be able to execute C code
        \item<4-> Calls \funcname{caml\_stash\_backtrace}, a C function provided by the runtime\ignorespaces
          \onslide<6->{, with
            \begin{itemize}
              \item \funcarg{exn}: exception value
              \item \funcarg{pc}: code address of the raising point
              \item \funcarg{sp}: stack address of the raising function
              \item \funcarg{trapsp}: stack address of the next handler
            \end{itemize}
          }
      \end{itemize}
      \bigskip
      \mintocaml[firstline=9,lastline=13,highlightlines=12]{../src/backtrace/backtrace.ml}
    \end{column}
    \begin{column}{0.5\textwidth}
      \raggedleft
      \onslide*<1,3-4>{
        \mintc[firstline=190,lastline=192]{../ocaml/runtime/amd64.S}
        \mintc[firstline=234,lastline=237]{../ocaml/runtime/amd64.S}
      }
      \onslide*<2>{
        \mintc[firstline=190,lastline=192,highlightlines={191-192}]{../ocaml/runtime/amd64.S}
        \mintc[firstline=234,lastline=237,highlightlines={235-237}]{../ocaml/runtime/amd64.S}
      }
      \onslide*<1>{\listCamlRaiseExn[highlightlines=705]{}}%
      \onslide*<2>{\listCamlRaiseExn[highlightlines={707-708}]{}}%
      \onslide*<3>{\listCamlRaiseExn[highlightlines={712-714}]{}}%
      \onslide*<4>{\listCamlRaiseExn[highlightlines={715-720}]{}}%
      \onslide*<5>{\providecommand\step{1}\input{diagrams/backtrace/backtrace.tex}}%
      \onslide*<6>{\providecommand\step{2}\input{diagrams/backtrace/backtrace.tex}}%
    \end{column}
  \end{columns}
\end{frame}

\newcommand\listStashBacktrace[1][]{
  \listc[firstline=91,lastline=120,#1]{../ocaml/runtime/backtrace\_nat.c}{runtime/backtrace\_nat.c:91}}

\begin{frame}{Backtraces}{Introducing \funcname{caml\_stash\_backtrace}}
  \begin{columns}[c]
    \begin{column}{0.5\textwidth}
      \minipage[c][0.66\textheight][s]{\columnwidth}
      \begin{itemize}
        \item<1-> A C function provided by the runtime (specific to native code)
        \item<2-> It scans the stack, between the stack pointer at the raising point up to the next trap, in order to collect the names of the detected function frames
          \onslide*<2>{
            \begin{itemize}
              \item And more information, like line number, column number...
            \end{itemize}
          }
        \item<3-> It uses the same technique and information as the Garbage Collector (GC) uses to scan the values live on the stack
          \onslide*<4>{
            \begin{itemize}
              \item \typename{frame\_descr} are at the core of stack unwinding
            \end{itemize}
          }
      \end{itemize}
      \vfill
      \mintocaml[firstline=9,lastline=13,highlightlines=12]{../src/backtrace/backtrace.ml}
      \endminipage
    \end{column}
    \begin{column}{0.5\textwidth}
      \onslide*<1>{\listStashBacktrace[highlightlines=91]{}}
      \onslide*<2>{\listStashBacktrace[highlightlines={91,117-118}]{}}
      \onslide*<3->{\listStashBacktrace[highlightlines={94,106,109-110}]{}}
    \end{column}
  \end{columns}
\end{frame}

\newcommand\listFrameDescr[1][]{
  \listocaml[firstline=111,lastline=124,#1]{../ocaml/asmcomp/emitaux.ml}{asmcomp/emitaux.ml:111}}
\newcommand\listDebugInfo[1][]{
  \listocaml[firstline=38,lastline=49,#1]{../ocaml/lambda/debuginfo.mli}{lambda/debuginfo.mli:38}}

\begin{frame}{Backtraces}{\typename{frame\_descr} and \typename{frame\_debuginfo} in a nutshell}
  \begin{columns}[c]
    \begin{column}{0.5\textwidth}
      \begin{itemize}
        \item<1-> For every return address in an OCaml program, there is a associated \typename{frame\_descr}
        \item<2-> A \typename{frame\_descr} specifies
        \begin{itemize}
          \item<2-> The size of the function stack frame
          \item<3-> Where live values are on the stack (so that the GC can mark them as live and prevent from being swept)
        \end{itemize}
        \item<4-> When compiled with debug information, \typename{frame\_descr} will be linked to a \typename{frame\_debuginfo}
        \begin{itemize}
          \item<5-> Contains information about the position in the source code (file name, line and column number)
        \end{itemize}
      \end{itemize}
    \end{column}
    \begin{column}{0.5\textwidth}
        \onslide*<1>{\listFrameDescr[highlightlines={118-122}]{}}
        \onslide*<2>{\listFrameDescr[highlightlines={120}]{}}
        \onslide*<3>{\listFrameDescr[highlightlines={121}]{}}
        \onslide*<4->{\listFrameDescr[highlightlines={122}]{}}
        \medskip
        \onslide*<1-3>{\listDebugInfo{}}
        \onslide*<4>{\listDebugInfo[highlightlines={38,49}]{}}
        \onslide*<5>{\listDebugInfo[highlightlines={39-46}]{}}
    \end{column}
  \end{columns}
\end{frame}

\begin{frame}{Backtraces}{Scanning the stack with \typename{frame\_descr}}
  \begin{columns}[c]
    \begin{column}{0.5\textwidth}
      \begin{itemize}
        \item Given the return address of the code that raises the exception by calling \funcname{caml\_raise\_exn} (\funcarg{pc} argument of \funcname{caml\_stash\_backtrace})
        \item And the position of the stack pointer (\funcarg{sp} argument of \funcname{caml\_stash\_backtrace})
        \item The frame size given by \typename{frame\_descr}.\funcarg{frame\_size}, allows to find the end of the previous frame
        \item And the return address of the caller of this function
        \bigskip
        \onslide<3->{
        \item Note that traps (here the reddish \funcname{ExnA}, \funcname{ExnB} and \funcname{ExnC} blocks) belong to the frame that installed them, and so the frame size changes when traps are removed
        }
      \end{itemize}
    \end{column}
    \begin{column}{0.5\textwidth}
      \raggedleft
      \onslide*<1>{\providecommand\step{2}\input{diagrams/backtrace/backtrace.tex}}%
      \onslide*<2>{\providecommand\step{1}\input{diagrams/backtrace/scanstack.tex}}%
      \onslide*<3>{\providecommand\step{2}\input{diagrams/backtrace/scanstack.tex}}%
      \onslide*<4>{\providecommand\step{3}\input{diagrams/backtrace/scanstack.tex}}%
      \onslide*<5>{\providecommand\step{4}\input{diagrams/backtrace/scanstack.tex}}%
      \onslide*<6>{\providecommand\step{5}\input{diagrams/backtrace/scanstack.tex}}%
    \end{column}
  \end{columns}
\end{frame}

% Duplicate of previous-previous slide
\begin{frame}{Backtraces}{Continuing on \funcname{caml\_stash\_backtrace}}
  \begin{columns}[c]
    \begin{column}{0.5\textwidth}
      \minipage[c][0.75\textheight][s]{\columnwidth}
      \begin{itemize}
          \color{gray}
        \item A C function provided by the runtime (specific to native code)
        \item It scans the stack, between the stack pointer at the raising point up to the next trap, in order to collect the names of the detected function frames
        \item It uses the same technique and information as the Garbage Collector (GC) uses to scan the values live on the stack
      \end{itemize}
      \begin{itemize}
        \item For each function frame detected, store a pointer to the \typename{frame\_descr} in the \localname{domain\_state->backtrace\_buffer} array, effectively constructing the backtrace
      \end{itemize}
      \onslide<2->{
        \begin{itemize}
            \smallskip
          \item Constructed backtrace:
            \funcname{definitely\_raise},
            \onslide<3->{\funcname{probably\_raise}}
        \end{itemize}
      }
      \vfill
      \mintocaml[firstline=9,lastline=13,highlightlines=12]{../src/backtrace/backtrace.ml}
      \endminipage
    \end{column}
    \begin{column}{0.5\textwidth}
      \raggedleft
      \onslide*<1>{\listStashBacktrace[highlightlines={114-115}]{}}
      \onslide*<2>{\providecommand\step{3}\input{diagrams/backtrace/backtrace.tex}}%
      \onslide*<3>{\providecommand\step{4}\input{diagrams/backtrace/backtrace.tex}}%
    \end{column}
  \end{columns}
\end{frame}

\begin{frame}{Backtraces}{Back to \funcname{caml\_raise\_exn}}
  \begin{columns}[c]
    \begin{column}{0.5\textwidth}
      \minipage[c][0.66\textheight][s]{\columnwidth}
      \begin{itemize}\color{gray}
        \item Checks wheteher exceptions backtrace should be recorded, by checking the \funcarg{domain\_state->backtrace\_active} value
        \item Switches to the C stack to be able to execute C code
        \item Calls \funcname{caml\_stash\_backtrace}, to collect the backtrace up to the next trap
      \end{itemize}
      \onslide*<2->{
        \begin{itemize}
          \item<2-> Executes the exception handler in \funcname{probably\_raise} with \funcname{RESTORE\_EXN\_HANDLER\_OCAML}
            \begin{itemize}
              \item Set \regname{RSP} to \localname{domain\_state->exn\_handler} value
              \item Restore \localname{domain\_state->exn\_handler} from trap
              \item Jump to the handler code with \funcname{ret}
            \end{itemize}
        \end{itemize}
      }
      \vfill
      \onslide*<1>{\mintocaml[firstline=9,lastline=13,highlightlines=12]{../src/backtrace/backtrace.ml}}
      \onslide*<2->{\mintocaml[firstline=15,lastline=21,highlightlines={19-21}]{../src/backtrace/backtrace.ml}}
      \endminipage
    \end{column}
    \begin{column}{0.5\textwidth}
      \centering
      \onslide*<1>{
        \mintc[firstline=190,lastline=192]{../ocaml/runtime/amd64.S}
        \mintc[firstline=234,lastline=237]{../ocaml/runtime/amd64.S}
      }
      \onslide*<2>{
        \mintc[firstline=190,lastline=192,highlightlines={191-192}]{../ocaml/runtime/amd64.S}
        \mintc[firstline=234,lastline=237,highlightlines={235-237}]{../ocaml/runtime/amd64.S}
      }
      \onslide*<1>{\listCamlRaiseExn[highlightlines=720]{}}
      \onslide*<2>{\listCamlRaiseExn[highlightlines=722]{}}
      \onslide*<3->{\providecommand\step{5}\input{diagrams/backtrace/backtrace.tex}}
    \end{column}
  \end{columns}
\end{frame}

\begin{frame}{Backtraces}{\funcname{caml\_probably\_raise}'s handler doesn't catch \typename{ExnC}}
  \begin{columns}[c]
    \begin{column}{0.45\textwidth}
      \minipage[c][0.75\textheight][s]{\columnwidth}
      \begin{itemize}
        \item<1-> \funcname{match \dots with}\footnotemark pattern matching can also match exceptions
        \item<1-> The CMM of \funcname{probably\_raise} shows that this \funcname{match \dots with} is implemented with a pattern matching and a \funcname{try \dots with}
        \item<1-> But this one only matches on \typename{ExnA} exception
        \item<2-> Since the pattern matching fails to catch the exception, \funcname{reraise} is called with our current \typename{ExnC} exception
        \item<3-> Disassembly shows that \funcname{reraise} is actually a call to \funcname{caml\_reraise\_exn}, which is also an assembly function provided by the runtime
      \end{itemize}
      \vfill
      \mintocaml[firstline=15,lastline=21,highlightlines={18-21}]{../src/backtrace/backtrace.ml}
      \endminipage
    \end{column}
    \begin{column}{0.55\textwidth}
      \centering
      \onslide*<1>{\mintcmm[highlightlines={10-18}]{../src/backtrace/camlBacktrace\_\_probably\_raise\_351.cmm}}
      \onslide*<2>{\listcmm[highlightlines=18]{../src/backtrace/camlBacktrace\_\_probably\_raise_351.cmm}{CMM of probably\_raise}}
      \onslide*<3>{\mintobjdump[firstline=5,lastline=35,highlightlines=31]{../src/backtrace/camlBacktrace\_\_probably\_raise_351.objdump}}
    \end{column}
  \end{columns}
  \only<1>{\footnotetext[1]{https://blog.janestreet.com/pattern-matching-and-exception-handling-unite/}}
\end{frame}

\newcommand\listCamlReraiseExn[1][]{
  \listasm[firstline=727,lastline=734,#1]{../ocaml/runtime/amd64.S}{runtime/amd64.S:727}}

\begin{frame}{Backtraces}{Introducing \funcname{caml\_reraise\_exn}}
  \begin{columns}[c]
    \begin{column}{0.5\textwidth}
      \minipage[c][0.66\textheight][s]{\columnwidth}
      \begin{itemize}
        \item<1-> The exception have to be reraised to the next handler
        \item<2-> First, checks if the backtrace should be collected with \funcarg{domain\_state->backtrace\_active}
        \only<3>{
        \begin{itemize}
            \item If not, behaves like a \funcname{raise\_notrace} with \funcname{RESTORE\_EXN\_HANDLER\_OCAML}
        \end{itemize}
        }
        \item<4-> Jumps into \funcname{caml\_raise\_exn}
        \item<5-> Calls \funcname{caml\_stash\_backtrace} again, to scan the next part of the stack: from the trap just pop-ed to the next trap
      \end{itemize}
      \vfill
      \mintocaml[firstline=15,lastline=21,highlightlines={18-21}]{../src/backtrace/backtrace.ml}
      \endminipage
    \end{column}
    \begin{column}{0.5\textwidth}
      \raggedleft
      \onslide*<1>{\listCamlReraiseExn{}}
      \onslide*<2>{\listCamlReraiseExn[highlightlines=729]{}}
      \onslide*<3>{
        \mintc[firstline=190,lastline=192,highlightlines={191-192}]{../ocaml/runtime/amd64.S}
        \mintc[firstline=234,lastline=237,highlightlines={235-237}]{../ocaml/runtime/amd64.S}
        \medskip
        \listCamlReraiseExn[highlightlines=731]{}
      }
      \onslide*<4-5>{
        % TODO refactor with \listCamlReraiseExn
        \mintasm[firstline=727,lastline=734,highlightlines=730]{../ocaml/runtime/amd64.S}
        \medskip
      }
      \onslide*<4>{\listCamlRaiseExn[highlightlines=711]{}}
      \onslide*<5>{\listCamlRaiseExn[highlightlines=720]{}}
    \end{column}
  \end{columns}
\end{frame}

\begin{frame}{Backtraces}{Backtrace collection continues}
  \begin{columns}[c]
    \begin{column}{0.55\textwidth}
      \minipage[c][0.75\textheight][s]{\columnwidth}
      \begin{itemize}
        \item<1-> \funcname{caml\_stash\_backtrace} scans the older part of the stack, up to the next trap
        \item<6-> Controls jumps to the nested exception handler in \funcname{maybe\_raise}, which matches only on \typename{ExnB}
        \item<7-> \funcname{caml\_reraise\_exn} is called again, backtrace collection resumes with \funcname{caml\_stash\_backtrace}
      \end{itemize}
      \medskip
      \begin{itemize}
        \item<2-> Constructed backtrace:
        \begin{itemize}
          \item <2-> \funcname{definitely\_raise}
          \item <2-> \funcname{probably\_raise}
          \item <3-> \funcname{probably\_raise} (re-raise)
          \item <4-> \funcname{maybe\_raise}
          \item <5-> \funcname{main}
          \item <8-> \funcname{main} (re-raise)
        \end{itemize}
      \end{itemize}
      \vfill
      \onslide*<1-5>{\mintocaml[firstline=15,lastline=21]{../src/backtrace/backtrace.ml}}
      \onslide*<6>{\mintocaml[firstline=28,lastline=34,highlightlines=31]{../src/backtrace/backtrace.ml}}
      \onslide*<7->{\mintocaml[firstline=28,lastline=34,highlightlines=33]{../src/backtrace/backtrace.ml}}
      \endminipage
    \end{column}
    \begin{column}{0.45\textwidth}
      \raggedleft
      \onslide*<1>{\providecommand\step{6}\input{diagrams/backtrace/backtrace.tex}}%
      \onslide*<2>{\providecommand\step{6}\input{diagrams/backtrace/backtrace.tex}}%
      \onslide*<3>{\providecommand\step{7}\input{diagrams/backtrace/backtrace.tex}}%
      \onslide*<4>{\providecommand\step{8}\input{diagrams/backtrace/backtrace.tex}}%
      \onslide*<5>{\providecommand\step{9}\input{diagrams/backtrace/backtrace.tex}}%
      \onslide*<6>{\providecommand\step{10}\input{diagrams/backtrace/backtrace.tex}}%
      \onslide*<7>{\providecommand\step{11}\input{diagrams/backtrace/backtrace.tex}}%
      \onslide*<8>{\providecommand\step{12}\input{diagrams/backtrace/backtrace.tex}}%
      \onslide*<9>{\providecommand\step{13}\input{diagrams/backtrace/backtrace.tex}}%
    \end{column}
  \end{columns}
\end{frame}

\begin{frame}{Backtraces}{Printing the backtrace}
  \begin{columns}[c]
    \begin{column}{0.45\textwidth}
      \minipage[c][0.66\textheight][s]{\columnwidth}
      \begin{itemize}
        \item<1-> The exception backtrace can now be printed to \funcarg{stdout} with \funcname{Printexc.print_backtrace}
        \item<2-> First the exception backtrace is retrieved into an OCaml array with \funcname{get_raw_backtrace }
        \item<3-> Which is implemented as the \funcname{caml_get_exception_raw_backtrace} C function in the runtime
        \item<4-> And finally printed with \funcname{print_exception_backtrace}
      \end{itemize}
      \vfill
      \mintocaml[firstline=28,lastline=34,highlightlines=33]{../src/backtrace/backtrace.ml}
      \endminipage
    \end{column}
    \begin{column}{0.55\textwidth}
      \onslide*<1,2>{
        \mintocaml[firstline=100,lastline=101]{../ocaml/stdlib/printexc.ml}
      }
      \only<1>{
        \listocaml[firstline=170,lastline=172]{../ocaml/stdlib/printexc.ml}{stdlib/printexc.ml:170}
      }
      \only<2>{
        \listocaml[firstline=170,lastline=172,highlightlines=172]{../ocaml/stdlib/printexc.ml}{stdlib/printexc.ml:170}
      }
      \only<3>{
        \mintc[firstline=154,lastline=156]{../ocaml/runtime/backtrace.c}
        \listc[firstline=171,lastline=192,highlightlines={184-187}]{../ocaml/runtime/backtrace.c}{runtime/backtrace.c:154}
      }
      \only<4>{
        \listocaml[firstline=155,lastline=165]{../ocaml/stdlib/printexc.ml}{stdlib/printexc.ml:55}
      }
    \end{column}
  \end{columns}
\end{frame}


\subsection{The multiple kinds of raise}
\frameSubsection{}{}

\begin{frame}{Backtraces}{When backtraces are not needed}
  \begin{columns}[c]
    \begin{column}{0.45\textwidth}
      \minipage[c][0.66\textheight][s]{\columnwidth}
      \begin{itemize}
        \item<1-> What if the backtrace for an exception is never needed?
        \item<2-> It's possible to raise the exception explicitly with \funcname{raise\_notrace} to make raising as cheap as possible
        \item<3-> An example of use in the \funcname{dynlink} library of the OCaml compiler
      \end{itemize}
      \vfill
      \onslide<3->{
      \listocaml[firstline=205,lastline=209,highlightlines=207]{../ocaml/otherlibs/dynlink/dynlink\_compilerlibs/misc.ml}{otherlibs/dynlink/dynlink\_compilerlibs/misc.ml:205}
      }
      \endminipage
    \end{column}
    \begin{column}{0.55\textwidth}
    \only<4>{
      \mintcmm[highlightlines=3]{../src/notrace/camlNotrace\_\_fun_347.cmm}
      \listcmm{../src/notrace/camlNotrace\_\_all\_somes\_327.cmm}{all\_somes}
    }
    \onslide*<5->{
      \listobjdump[highlightlines={6-9}]{../src/notrace/camlNotrace\_\_fun_347.objdump}{anonymous function from \funcname{all\_somes}}
    }
    \end{column}
  \end{columns}
\end{frame}

\begin{frame}{Backtraces}{Summary table of the multiple kinds of raise}
  \begin{itemize}
    \item When debugging information is enabled and if the backtrace of an exception isn't needed, it's possible to raise with \funcname{raise\_notrace} for performance
    \item When debugging information is \emph{not} enabled, the compiler optimises \funcname{raise} calls to inline assembly (same effect as using \funcname{raise\_notrace})
  \end{itemize}
  \bigskip
  \centering
  \begin{tabular}{| l | c | c | c | c |}
    \hline
    \parbox{10em}{Debug information \\ (\funcarg{-g} flag)}  & \multicolumn{2}{|c|}{Disabled}                                     & \multicolumn{2}{|c|}{Enabled} \\
    \hline
    Raise kind                                               & \funcname{raise} & \funcname{raise\_notrace}                       & \funcname{raise} & \funcname{raise\_notrace} \\
    \hline
    Compilation                                              & \parbox{8em}{inline assembly\\ (optimization)} & inline assembly   & \funcname{caml\_raise\_exn} & inline assembly \\
    \hline
  \end{tabular}
\end{frame}


%
% Takeaway #5
%
\subsection*{Takeaway \#5}
\frameSubsectionTakeaway{}
\againframe<5>{takeaway}
}%
      \onslide*<3>{\providecommand\step{4}%
% Backtraces
%
\subsection{One advantage of raising with debugging information}
\frameSubsection{}{}

\begin{frame}{Backtraces}{Debugging information \& source code}
  \begin{columns}[c]
    \begin{column}{0.5\textwidth}
      \begin{itemize}
        \item Raising an exception is fun and games, but how do we know where the exception originated? $\rightarrow$ With the backtrace associated with the exception!
        \item In order to get a backtrace from the point where an exception was raised, it's necessary to compile with debugging information enabled (\funcarg{-g} flag for the compiler)
        \item Same example as in the `Nested handlers' section
      \end{itemize}
    \end{column}
    \begin{column}{0.5\textwidth}
      \listocaml[firstline=9,lastline=34]{../src/backtrace/backtrace.ml}{backtrace/backtrace.ml}
    \end{column}
  \end{columns}
\end{frame}

\begin{frame}{Backtraces}{CMM and disassembly of \funcname{definitely\_raise}}
  \begin{columns}[c]
    \begin{column}{0.5\textwidth}
      \minipage[c][0.66\textheight][s]{\columnwidth}
      \begin{itemize}
        \item<1-> An effect of compiling with debugging information (\funcarg{-g}): the CMM no longer uses \funcname{raise\_notrace} to raise the exception but \funcname{raise} instead
        \item<2-> Disassembly shows that CMM \funcname{raise} is actually a call to \funcname{caml\_raise\_exn}, a function in assembler provided by the runtime
      \end{itemize}
      \vfill
      \mintocaml[firstline=9,lastline=13,highlightlines=12]{../src/backtrace/backtrace.ml}
      \endminipage
    \end{column}
    \begin{column}{0.5\textwidth}
      \onslide*<1>{
        \mintcmm[highlightlines=5]{../src/backtrace/camlBacktrace\_\_definitely\_raise\_347.cmm}
      }
      \onslide*<2>{
        \mintobjdump[firstline=2,highlightlines=14]{../src/backtrace/camlBacktrace\_\_definitely\_raise\_347.objdump}
      }
    \end{column}
  \end{columns}
\end{frame}

\begin{frame}{Backtraces}{Introducing \funcname{caml\_raise\_exn}}
  \begin{columns}[c]
    \begin{column}{0.5\textwidth}
      \minipage[c][0.66\textheight][s]{\columnwidth}
      \onslide*<1>{
        \begin{itemize}
          \item Raising the exception is performed using \funcname{caml\_raise\_exn} which is
            \begin{itemize}
              \item An assembly function provided by the runtime
              \item Architecture specific
            \end{itemize}
          \item Let's see what it does differently from \funcname{raise\_notrace}\dots
          \item We are about to raise \typename{ExnC} with \funcname{caml\_raise\_exn} from \funcname{definitely\_raise}
        \end{itemize}
      }
      \onslide*<2>{
        \mintobjdump[firstline=2,highlightlines=14]{../src/backtrace/camlBacktrace\_\_definitely\_raise\_347.objdump}
      }
      \vfill
      \mintocaml[firstline=9,lastline=13,highlightlines=12]{../src/backtrace/backtrace.ml}
      \endminipage
    \end{column}
    \begin{column}{0.5\textwidth}
      \centering
      \providecommand\step{1}\input{diagrams/backtrace/backtrace.tex}
    \end{column}
  \end{columns}
\end{frame}

\begin{frame}{Backtraces}{But before that, we need more fields from \typename{caml\_domain\_state}}
  \begin{columns}
    \begin{column}{0.5\textwidth}
      \begin{itemize}
        \item Remember \typename{caml\_domain\_state} the per-domain runtime state?
        \item Some other members are of interest to us now\dots
        \item \localname{backtrace\_active} is used to know if the runtime should collect backtraces
        \item \localname{backtrace\_active} can be set by
          \begin{itemize}
            \item The \funcarg{b} flag in the \funcname{OCAMLRUNPARAM} environment variable
            \item \mintinline{ocaml}{Printexc.record_backtrace : bool -> unit} \footnotemark
          \end{itemize}
      \end{itemize}
    \end{column}
    \begin{column}{0.5\textwidth}
      \begin{listing}
        \mintc[highlightlines={9-11}]{src/ocaml/domain\_state\_backtrace.h}
        \caption{runtime/caml/domain\_state.h}
      \end{listing}
    \end{column}
  \end{columns}
  \footnotetext[1]{https://v2.ocaml.org/api/Printexc.html}
\end{frame}

\newcommand\listCamlRaiseExn[1][]{
  \listasm[firstline=702,lastline=725,#1]{../ocaml/runtime/amd64.S}{runtime/amd64.S:702}}

\begin{frame}{Backtraces}{Walk through \funcname{caml\_raise\_exn}}
  \begin{columns}[T]
    \begin{column}{0.5\textwidth}
      \begin{itemize}
        \item<1-> First checks whether exceptions backtrace should be recorded
        \item<1-> By checking the \funcarg{domain\_state->backtrace\_active} value
        \item<2-> If not, acts as \funcname{raise\_notrace} with \funcname{RESTORE\_EXN\_HANDLER\_OCAML}
          \begin{itemize}
            \item Set \regname{RSP} to \localname{domain\_state->exn\_handler} value
            \item Restore \localname{domain\_state->exn\_handler} from trap
            \item Jump with \funcname{ret} (same effect as using \funcname{pop} and \funcname{jmp})
          \end{itemize}
        \item<3-> Otherwise, switches to the C stack to be able to execute C code
        \item<4-> Calls \funcname{caml\_stash\_backtrace}, a C function provided by the runtime\ignorespaces
          \onslide<6->{, with
            \begin{itemize}
              \item \funcarg{exn}: exception value
              \item \funcarg{pc}: code address of the raising point
              \item \funcarg{sp}: stack address of the raising function
              \item \funcarg{trapsp}: stack address of the next handler
            \end{itemize}
          }
      \end{itemize}
      \bigskip
      \mintocaml[firstline=9,lastline=13,highlightlines=12]{../src/backtrace/backtrace.ml}
    \end{column}
    \begin{column}{0.5\textwidth}
      \raggedleft
      \onslide*<1,3-4>{
        \mintc[firstline=190,lastline=192]{../ocaml/runtime/amd64.S}
        \mintc[firstline=234,lastline=237]{../ocaml/runtime/amd64.S}
      }
      \onslide*<2>{
        \mintc[firstline=190,lastline=192,highlightlines={191-192}]{../ocaml/runtime/amd64.S}
        \mintc[firstline=234,lastline=237,highlightlines={235-237}]{../ocaml/runtime/amd64.S}
      }
      \onslide*<1>{\listCamlRaiseExn[highlightlines=705]{}}%
      \onslide*<2>{\listCamlRaiseExn[highlightlines={707-708}]{}}%
      \onslide*<3>{\listCamlRaiseExn[highlightlines={712-714}]{}}%
      \onslide*<4>{\listCamlRaiseExn[highlightlines={715-720}]{}}%
      \onslide*<5>{\providecommand\step{1}\input{diagrams/backtrace/backtrace.tex}}%
      \onslide*<6>{\providecommand\step{2}\input{diagrams/backtrace/backtrace.tex}}%
    \end{column}
  \end{columns}
\end{frame}

\newcommand\listStashBacktrace[1][]{
  \listc[firstline=91,lastline=120,#1]{../ocaml/runtime/backtrace\_nat.c}{runtime/backtrace\_nat.c:91}}

\begin{frame}{Backtraces}{Introducing \funcname{caml\_stash\_backtrace}}
  \begin{columns}[c]
    \begin{column}{0.5\textwidth}
      \minipage[c][0.66\textheight][s]{\columnwidth}
      \begin{itemize}
        \item<1-> A C function provided by the runtime (specific to native code)
        \item<2-> It scans the stack, between the stack pointer at the raising point up to the next trap, in order to collect the names of the detected function frames
          \onslide*<2>{
            \begin{itemize}
              \item And more information, like line number, column number...
            \end{itemize}
          }
        \item<3-> It uses the same technique and information as the Garbage Collector (GC) uses to scan the values live on the stack
          \onslide*<4>{
            \begin{itemize}
              \item \typename{frame\_descr} are at the core of stack unwinding
            \end{itemize}
          }
      \end{itemize}
      \vfill
      \mintocaml[firstline=9,lastline=13,highlightlines=12]{../src/backtrace/backtrace.ml}
      \endminipage
    \end{column}
    \begin{column}{0.5\textwidth}
      \onslide*<1>{\listStashBacktrace[highlightlines=91]{}}
      \onslide*<2>{\listStashBacktrace[highlightlines={91,117-118}]{}}
      \onslide*<3->{\listStashBacktrace[highlightlines={94,106,109-110}]{}}
    \end{column}
  \end{columns}
\end{frame}

\newcommand\listFrameDescr[1][]{
  \listocaml[firstline=111,lastline=124,#1]{../ocaml/asmcomp/emitaux.ml}{asmcomp/emitaux.ml:111}}
\newcommand\listDebugInfo[1][]{
  \listocaml[firstline=38,lastline=49,#1]{../ocaml/lambda/debuginfo.mli}{lambda/debuginfo.mli:38}}

\begin{frame}{Backtraces}{\typename{frame\_descr} and \typename{frame\_debuginfo} in a nutshell}
  \begin{columns}[c]
    \begin{column}{0.5\textwidth}
      \begin{itemize}
        \item<1-> For every return address in an OCaml program, there is a associated \typename{frame\_descr}
        \item<2-> A \typename{frame\_descr} specifies
        \begin{itemize}
          \item<2-> The size of the function stack frame
          \item<3-> Where live values are on the stack (so that the GC can mark them as live and prevent from being swept)
        \end{itemize}
        \item<4-> When compiled with debug information, \typename{frame\_descr} will be linked to a \typename{frame\_debuginfo}
        \begin{itemize}
          \item<5-> Contains information about the position in the source code (file name, line and column number)
        \end{itemize}
      \end{itemize}
    \end{column}
    \begin{column}{0.5\textwidth}
        \onslide*<1>{\listFrameDescr[highlightlines={118-122}]{}}
        \onslide*<2>{\listFrameDescr[highlightlines={120}]{}}
        \onslide*<3>{\listFrameDescr[highlightlines={121}]{}}
        \onslide*<4->{\listFrameDescr[highlightlines={122}]{}}
        \medskip
        \onslide*<1-3>{\listDebugInfo{}}
        \onslide*<4>{\listDebugInfo[highlightlines={38,49}]{}}
        \onslide*<5>{\listDebugInfo[highlightlines={39-46}]{}}
    \end{column}
  \end{columns}
\end{frame}

\begin{frame}{Backtraces}{Scanning the stack with \typename{frame\_descr}}
  \begin{columns}[c]
    \begin{column}{0.5\textwidth}
      \begin{itemize}
        \item Given the return address of the code that raises the exception by calling \funcname{caml\_raise\_exn} (\funcarg{pc} argument of \funcname{caml\_stash\_backtrace})
        \item And the position of the stack pointer (\funcarg{sp} argument of \funcname{caml\_stash\_backtrace})
        \item The frame size given by \typename{frame\_descr}.\funcarg{frame\_size}, allows to find the end of the previous frame
        \item And the return address of the caller of this function
        \bigskip
        \onslide<3->{
        \item Note that traps (here the reddish \funcname{ExnA}, \funcname{ExnB} and \funcname{ExnC} blocks) belong to the frame that installed them, and so the frame size changes when traps are removed
        }
      \end{itemize}
    \end{column}
    \begin{column}{0.5\textwidth}
      \raggedleft
      \onslide*<1>{\providecommand\step{2}\input{diagrams/backtrace/backtrace.tex}}%
      \onslide*<2>{\providecommand\step{1}\input{diagrams/backtrace/scanstack.tex}}%
      \onslide*<3>{\providecommand\step{2}\input{diagrams/backtrace/scanstack.tex}}%
      \onslide*<4>{\providecommand\step{3}\input{diagrams/backtrace/scanstack.tex}}%
      \onslide*<5>{\providecommand\step{4}\input{diagrams/backtrace/scanstack.tex}}%
      \onslide*<6>{\providecommand\step{5}\input{diagrams/backtrace/scanstack.tex}}%
    \end{column}
  \end{columns}
\end{frame}

% Duplicate of previous-previous slide
\begin{frame}{Backtraces}{Continuing on \funcname{caml\_stash\_backtrace}}
  \begin{columns}[c]
    \begin{column}{0.5\textwidth}
      \minipage[c][0.75\textheight][s]{\columnwidth}
      \begin{itemize}
          \color{gray}
        \item A C function provided by the runtime (specific to native code)
        \item It scans the stack, between the stack pointer at the raising point up to the next trap, in order to collect the names of the detected function frames
        \item It uses the same technique and information as the Garbage Collector (GC) uses to scan the values live on the stack
      \end{itemize}
      \begin{itemize}
        \item For each function frame detected, store a pointer to the \typename{frame\_descr} in the \localname{domain\_state->backtrace\_buffer} array, effectively constructing the backtrace
      \end{itemize}
      \onslide<2->{
        \begin{itemize}
            \smallskip
          \item Constructed backtrace:
            \funcname{definitely\_raise},
            \onslide<3->{\funcname{probably\_raise}}
        \end{itemize}
      }
      \vfill
      \mintocaml[firstline=9,lastline=13,highlightlines=12]{../src/backtrace/backtrace.ml}
      \endminipage
    \end{column}
    \begin{column}{0.5\textwidth}
      \raggedleft
      \onslide*<1>{\listStashBacktrace[highlightlines={114-115}]{}}
      \onslide*<2>{\providecommand\step{3}\input{diagrams/backtrace/backtrace.tex}}%
      \onslide*<3>{\providecommand\step{4}\input{diagrams/backtrace/backtrace.tex}}%
    \end{column}
  \end{columns}
\end{frame}

\begin{frame}{Backtraces}{Back to \funcname{caml\_raise\_exn}}
  \begin{columns}[c]
    \begin{column}{0.5\textwidth}
      \minipage[c][0.66\textheight][s]{\columnwidth}
      \begin{itemize}\color{gray}
        \item Checks wheteher exceptions backtrace should be recorded, by checking the \funcarg{domain\_state->backtrace\_active} value
        \item Switches to the C stack to be able to execute C code
        \item Calls \funcname{caml\_stash\_backtrace}, to collect the backtrace up to the next trap
      \end{itemize}
      \onslide*<2->{
        \begin{itemize}
          \item<2-> Executes the exception handler in \funcname{probably\_raise} with \funcname{RESTORE\_EXN\_HANDLER\_OCAML}
            \begin{itemize}
              \item Set \regname{RSP} to \localname{domain\_state->exn\_handler} value
              \item Restore \localname{domain\_state->exn\_handler} from trap
              \item Jump to the handler code with \funcname{ret}
            \end{itemize}
        \end{itemize}
      }
      \vfill
      \onslide*<1>{\mintocaml[firstline=9,lastline=13,highlightlines=12]{../src/backtrace/backtrace.ml}}
      \onslide*<2->{\mintocaml[firstline=15,lastline=21,highlightlines={19-21}]{../src/backtrace/backtrace.ml}}
      \endminipage
    \end{column}
    \begin{column}{0.5\textwidth}
      \centering
      \onslide*<1>{
        \mintc[firstline=190,lastline=192]{../ocaml/runtime/amd64.S}
        \mintc[firstline=234,lastline=237]{../ocaml/runtime/amd64.S}
      }
      \onslide*<2>{
        \mintc[firstline=190,lastline=192,highlightlines={191-192}]{../ocaml/runtime/amd64.S}
        \mintc[firstline=234,lastline=237,highlightlines={235-237}]{../ocaml/runtime/amd64.S}
      }
      \onslide*<1>{\listCamlRaiseExn[highlightlines=720]{}}
      \onslide*<2>{\listCamlRaiseExn[highlightlines=722]{}}
      \onslide*<3->{\providecommand\step{5}\input{diagrams/backtrace/backtrace.tex}}
    \end{column}
  \end{columns}
\end{frame}

\begin{frame}{Backtraces}{\funcname{caml\_probably\_raise}'s handler doesn't catch \typename{ExnC}}
  \begin{columns}[c]
    \begin{column}{0.45\textwidth}
      \minipage[c][0.75\textheight][s]{\columnwidth}
      \begin{itemize}
        \item<1-> \funcname{match \dots with}\footnotemark pattern matching can also match exceptions
        \item<1-> The CMM of \funcname{probably\_raise} shows that this \funcname{match \dots with} is implemented with a pattern matching and a \funcname{try \dots with}
        \item<1-> But this one only matches on \typename{ExnA} exception
        \item<2-> Since the pattern matching fails to catch the exception, \funcname{reraise} is called with our current \typename{ExnC} exception
        \item<3-> Disassembly shows that \funcname{reraise} is actually a call to \funcname{caml\_reraise\_exn}, which is also an assembly function provided by the runtime
      \end{itemize}
      \vfill
      \mintocaml[firstline=15,lastline=21,highlightlines={18-21}]{../src/backtrace/backtrace.ml}
      \endminipage
    \end{column}
    \begin{column}{0.55\textwidth}
      \centering
      \onslide*<1>{\mintcmm[highlightlines={10-18}]{../src/backtrace/camlBacktrace\_\_probably\_raise\_351.cmm}}
      \onslide*<2>{\listcmm[highlightlines=18]{../src/backtrace/camlBacktrace\_\_probably\_raise_351.cmm}{CMM of probably\_raise}}
      \onslide*<3>{\mintobjdump[firstline=5,lastline=35,highlightlines=31]{../src/backtrace/camlBacktrace\_\_probably\_raise_351.objdump}}
    \end{column}
  \end{columns}
  \only<1>{\footnotetext[1]{https://blog.janestreet.com/pattern-matching-and-exception-handling-unite/}}
\end{frame}

\newcommand\listCamlReraiseExn[1][]{
  \listasm[firstline=727,lastline=734,#1]{../ocaml/runtime/amd64.S}{runtime/amd64.S:727}}

\begin{frame}{Backtraces}{Introducing \funcname{caml\_reraise\_exn}}
  \begin{columns}[c]
    \begin{column}{0.5\textwidth}
      \minipage[c][0.66\textheight][s]{\columnwidth}
      \begin{itemize}
        \item<1-> The exception have to be reraised to the next handler
        \item<2-> First, checks if the backtrace should be collected with \funcarg{domain\_state->backtrace\_active}
        \only<3>{
        \begin{itemize}
            \item If not, behaves like a \funcname{raise\_notrace} with \funcname{RESTORE\_EXN\_HANDLER\_OCAML}
        \end{itemize}
        }
        \item<4-> Jumps into \funcname{caml\_raise\_exn}
        \item<5-> Calls \funcname{caml\_stash\_backtrace} again, to scan the next part of the stack: from the trap just pop-ed to the next trap
      \end{itemize}
      \vfill
      \mintocaml[firstline=15,lastline=21,highlightlines={18-21}]{../src/backtrace/backtrace.ml}
      \endminipage
    \end{column}
    \begin{column}{0.5\textwidth}
      \raggedleft
      \onslide*<1>{\listCamlReraiseExn{}}
      \onslide*<2>{\listCamlReraiseExn[highlightlines=729]{}}
      \onslide*<3>{
        \mintc[firstline=190,lastline=192,highlightlines={191-192}]{../ocaml/runtime/amd64.S}
        \mintc[firstline=234,lastline=237,highlightlines={235-237}]{../ocaml/runtime/amd64.S}
        \medskip
        \listCamlReraiseExn[highlightlines=731]{}
      }
      \onslide*<4-5>{
        % TODO refactor with \listCamlReraiseExn
        \mintasm[firstline=727,lastline=734,highlightlines=730]{../ocaml/runtime/amd64.S}
        \medskip
      }
      \onslide*<4>{\listCamlRaiseExn[highlightlines=711]{}}
      \onslide*<5>{\listCamlRaiseExn[highlightlines=720]{}}
    \end{column}
  \end{columns}
\end{frame}

\begin{frame}{Backtraces}{Backtrace collection continues}
  \begin{columns}[c]
    \begin{column}{0.55\textwidth}
      \minipage[c][0.75\textheight][s]{\columnwidth}
      \begin{itemize}
        \item<1-> \funcname{caml\_stash\_backtrace} scans the older part of the stack, up to the next trap
        \item<6-> Controls jumps to the nested exception handler in \funcname{maybe\_raise}, which matches only on \typename{ExnB}
        \item<7-> \funcname{caml\_reraise\_exn} is called again, backtrace collection resumes with \funcname{caml\_stash\_backtrace}
      \end{itemize}
      \medskip
      \begin{itemize}
        \item<2-> Constructed backtrace:
        \begin{itemize}
          \item <2-> \funcname{definitely\_raise}
          \item <2-> \funcname{probably\_raise}
          \item <3-> \funcname{probably\_raise} (re-raise)
          \item <4-> \funcname{maybe\_raise}
          \item <5-> \funcname{main}
          \item <8-> \funcname{main} (re-raise)
        \end{itemize}
      \end{itemize}
      \vfill
      \onslide*<1-5>{\mintocaml[firstline=15,lastline=21]{../src/backtrace/backtrace.ml}}
      \onslide*<6>{\mintocaml[firstline=28,lastline=34,highlightlines=31]{../src/backtrace/backtrace.ml}}
      \onslide*<7->{\mintocaml[firstline=28,lastline=34,highlightlines=33]{../src/backtrace/backtrace.ml}}
      \endminipage
    \end{column}
    \begin{column}{0.45\textwidth}
      \raggedleft
      \onslide*<1>{\providecommand\step{6}\input{diagrams/backtrace/backtrace.tex}}%
      \onslide*<2>{\providecommand\step{6}\input{diagrams/backtrace/backtrace.tex}}%
      \onslide*<3>{\providecommand\step{7}\input{diagrams/backtrace/backtrace.tex}}%
      \onslide*<4>{\providecommand\step{8}\input{diagrams/backtrace/backtrace.tex}}%
      \onslide*<5>{\providecommand\step{9}\input{diagrams/backtrace/backtrace.tex}}%
      \onslide*<6>{\providecommand\step{10}\input{diagrams/backtrace/backtrace.tex}}%
      \onslide*<7>{\providecommand\step{11}\input{diagrams/backtrace/backtrace.tex}}%
      \onslide*<8>{\providecommand\step{12}\input{diagrams/backtrace/backtrace.tex}}%
      \onslide*<9>{\providecommand\step{13}\input{diagrams/backtrace/backtrace.tex}}%
    \end{column}
  \end{columns}
\end{frame}

\begin{frame}{Backtraces}{Printing the backtrace}
  \begin{columns}[c]
    \begin{column}{0.45\textwidth}
      \minipage[c][0.66\textheight][s]{\columnwidth}
      \begin{itemize}
        \item<1-> The exception backtrace can now be printed to \funcarg{stdout} with \funcname{Printexc.print_backtrace}
        \item<2-> First the exception backtrace is retrieved into an OCaml array with \funcname{get_raw_backtrace }
        \item<3-> Which is implemented as the \funcname{caml_get_exception_raw_backtrace} C function in the runtime
        \item<4-> And finally printed with \funcname{print_exception_backtrace}
      \end{itemize}
      \vfill
      \mintocaml[firstline=28,lastline=34,highlightlines=33]{../src/backtrace/backtrace.ml}
      \endminipage
    \end{column}
    \begin{column}{0.55\textwidth}
      \onslide*<1,2>{
        \mintocaml[firstline=100,lastline=101]{../ocaml/stdlib/printexc.ml}
      }
      \only<1>{
        \listocaml[firstline=170,lastline=172]{../ocaml/stdlib/printexc.ml}{stdlib/printexc.ml:170}
      }
      \only<2>{
        \listocaml[firstline=170,lastline=172,highlightlines=172]{../ocaml/stdlib/printexc.ml}{stdlib/printexc.ml:170}
      }
      \only<3>{
        \mintc[firstline=154,lastline=156]{../ocaml/runtime/backtrace.c}
        \listc[firstline=171,lastline=192,highlightlines={184-187}]{../ocaml/runtime/backtrace.c}{runtime/backtrace.c:154}
      }
      \only<4>{
        \listocaml[firstline=155,lastline=165]{../ocaml/stdlib/printexc.ml}{stdlib/printexc.ml:55}
      }
    \end{column}
  \end{columns}
\end{frame}


\subsection{The multiple kinds of raise}
\frameSubsection{}{}

\begin{frame}{Backtraces}{When backtraces are not needed}
  \begin{columns}[c]
    \begin{column}{0.45\textwidth}
      \minipage[c][0.66\textheight][s]{\columnwidth}
      \begin{itemize}
        \item<1-> What if the backtrace for an exception is never needed?
        \item<2-> It's possible to raise the exception explicitly with \funcname{raise\_notrace} to make raising as cheap as possible
        \item<3-> An example of use in the \funcname{dynlink} library of the OCaml compiler
      \end{itemize}
      \vfill
      \onslide<3->{
      \listocaml[firstline=205,lastline=209,highlightlines=207]{../ocaml/otherlibs/dynlink/dynlink\_compilerlibs/misc.ml}{otherlibs/dynlink/dynlink\_compilerlibs/misc.ml:205}
      }
      \endminipage
    \end{column}
    \begin{column}{0.55\textwidth}
    \only<4>{
      \mintcmm[highlightlines=3]{../src/notrace/camlNotrace\_\_fun_347.cmm}
      \listcmm{../src/notrace/camlNotrace\_\_all\_somes\_327.cmm}{all\_somes}
    }
    \onslide*<5->{
      \listobjdump[highlightlines={6-9}]{../src/notrace/camlNotrace\_\_fun_347.objdump}{anonymous function from \funcname{all\_somes}}
    }
    \end{column}
  \end{columns}
\end{frame}

\begin{frame}{Backtraces}{Summary table of the multiple kinds of raise}
  \begin{itemize}
    \item When debugging information is enabled and if the backtrace of an exception isn't needed, it's possible to raise with \funcname{raise\_notrace} for performance
    \item When debugging information is \emph{not} enabled, the compiler optimises \funcname{raise} calls to inline assembly (same effect as using \funcname{raise\_notrace})
  \end{itemize}
  \bigskip
  \centering
  \begin{tabular}{| l | c | c | c | c |}
    \hline
    \parbox{10em}{Debug information \\ (\funcarg{-g} flag)}  & \multicolumn{2}{|c|}{Disabled}                                     & \multicolumn{2}{|c|}{Enabled} \\
    \hline
    Raise kind                                               & \funcname{raise} & \funcname{raise\_notrace}                       & \funcname{raise} & \funcname{raise\_notrace} \\
    \hline
    Compilation                                              & \parbox{8em}{inline assembly\\ (optimization)} & inline assembly   & \funcname{caml\_raise\_exn} & inline assembly \\
    \hline
  \end{tabular}
\end{frame}


%
% Takeaway #5
%
\subsection*{Takeaway \#5}
\frameSubsectionTakeaway{}
\againframe<5>{takeaway}
}%
    \end{column}
  \end{columns}
\end{frame}

\begin{frame}{Backtraces}{Back to \funcname{caml\_raise\_exn}}
  \begin{columns}[c]
    \begin{column}{0.5\textwidth}
      \minipage[c][0.66\textheight][s]{\columnwidth}
      \begin{itemize}\color{gray}
        \item Checks wheteher exceptions backtrace should be recorded, by checking the \funcarg{domain\_state->backtrace\_active} value
        \item Switches to the C stack to be able to execute C code
        \item Calls \funcname{caml\_stash\_backtrace}, to collect the backtrace up to the next trap
      \end{itemize}
      \onslide*<2->{
        \begin{itemize}
          \item<2-> Executes the exception handler in \funcname{probably\_raise} with \funcname{RESTORE\_EXN\_HANDLER\_OCAML}
            \begin{itemize}
              \item Set \regname{RSP} to \localname{domain\_state->exn\_handler} value
              \item Restore \localname{domain\_state->exn\_handler} from trap
              \item Jump to the handler code with \funcname{ret}
            \end{itemize}
        \end{itemize}
      }
      \vfill
      \onslide*<1>{\mintocaml[firstline=9,lastline=13,highlightlines=12]{../src/backtrace/backtrace.ml}}
      \onslide*<2->{\mintocaml[firstline=15,lastline=21,highlightlines={19-21}]{../src/backtrace/backtrace.ml}}
      \endminipage
    \end{column}
    \begin{column}{0.5\textwidth}
      \centering
      \onslide*<1>{
        \mintc[firstline=190,lastline=192]{../ocaml/runtime/amd64.S}
        \mintc[firstline=234,lastline=237]{../ocaml/runtime/amd64.S}
      }
      \onslide*<2>{
        \mintc[firstline=190,lastline=192,highlightlines={191-192}]{../ocaml/runtime/amd64.S}
        \mintc[firstline=234,lastline=237,highlightlines={235-237}]{../ocaml/runtime/amd64.S}
      }
      \onslide*<1>{\listCamlRaiseExn[highlightlines=720]{}}
      \onslide*<2>{\listCamlRaiseExn[highlightlines=722]{}}
      \onslide*<3->{\providecommand\step{5}%
% Backtraces
%
\subsection{One advantage of raising with debugging information}
\frameSubsection{}{}

\begin{frame}{Backtraces}{Debugging information \& source code}
  \begin{columns}[c]
    \begin{column}{0.5\textwidth}
      \begin{itemize}
        \item Raising an exception is fun and games, but how do we know where the exception originated? $\rightarrow$ With the backtrace associated with the exception!
        \item In order to get a backtrace from the point where an exception was raised, it's necessary to compile with debugging information enabled (\funcarg{-g} flag for the compiler)
        \item Same example as in the `Nested handlers' section
      \end{itemize}
    \end{column}
    \begin{column}{0.5\textwidth}
      \listocaml[firstline=9,lastline=34]{../src/backtrace/backtrace.ml}{backtrace/backtrace.ml}
    \end{column}
  \end{columns}
\end{frame}

\begin{frame}{Backtraces}{CMM and disassembly of \funcname{definitely\_raise}}
  \begin{columns}[c]
    \begin{column}{0.5\textwidth}
      \minipage[c][0.66\textheight][s]{\columnwidth}
      \begin{itemize}
        \item<1-> An effect of compiling with debugging information (\funcarg{-g}): the CMM no longer uses \funcname{raise\_notrace} to raise the exception but \funcname{raise} instead
        \item<2-> Disassembly shows that CMM \funcname{raise} is actually a call to \funcname{caml\_raise\_exn}, a function in assembler provided by the runtime
      \end{itemize}
      \vfill
      \mintocaml[firstline=9,lastline=13,highlightlines=12]{../src/backtrace/backtrace.ml}
      \endminipage
    \end{column}
    \begin{column}{0.5\textwidth}
      \onslide*<1>{
        \mintcmm[highlightlines=5]{../src/backtrace/camlBacktrace\_\_definitely\_raise\_347.cmm}
      }
      \onslide*<2>{
        \mintobjdump[firstline=2,highlightlines=14]{../src/backtrace/camlBacktrace\_\_definitely\_raise\_347.objdump}
      }
    \end{column}
  \end{columns}
\end{frame}

\begin{frame}{Backtraces}{Introducing \funcname{caml\_raise\_exn}}
  \begin{columns}[c]
    \begin{column}{0.5\textwidth}
      \minipage[c][0.66\textheight][s]{\columnwidth}
      \onslide*<1>{
        \begin{itemize}
          \item Raising the exception is performed using \funcname{caml\_raise\_exn} which is
            \begin{itemize}
              \item An assembly function provided by the runtime
              \item Architecture specific
            \end{itemize}
          \item Let's see what it does differently from \funcname{raise\_notrace}\dots
          \item We are about to raise \typename{ExnC} with \funcname{caml\_raise\_exn} from \funcname{definitely\_raise}
        \end{itemize}
      }
      \onslide*<2>{
        \mintobjdump[firstline=2,highlightlines=14]{../src/backtrace/camlBacktrace\_\_definitely\_raise\_347.objdump}
      }
      \vfill
      \mintocaml[firstline=9,lastline=13,highlightlines=12]{../src/backtrace/backtrace.ml}
      \endminipage
    \end{column}
    \begin{column}{0.5\textwidth}
      \centering
      \providecommand\step{1}\input{diagrams/backtrace/backtrace.tex}
    \end{column}
  \end{columns}
\end{frame}

\begin{frame}{Backtraces}{But before that, we need more fields from \typename{caml\_domain\_state}}
  \begin{columns}
    \begin{column}{0.5\textwidth}
      \begin{itemize}
        \item Remember \typename{caml\_domain\_state} the per-domain runtime state?
        \item Some other members are of interest to us now\dots
        \item \localname{backtrace\_active} is used to know if the runtime should collect backtraces
        \item \localname{backtrace\_active} can be set by
          \begin{itemize}
            \item The \funcarg{b} flag in the \funcname{OCAMLRUNPARAM} environment variable
            \item \mintinline{ocaml}{Printexc.record_backtrace : bool -> unit} \footnotemark
          \end{itemize}
      \end{itemize}
    \end{column}
    \begin{column}{0.5\textwidth}
      \begin{listing}
        \mintc[highlightlines={9-11}]{src/ocaml/domain\_state\_backtrace.h}
        \caption{runtime/caml/domain\_state.h}
      \end{listing}
    \end{column}
  \end{columns}
  \footnotetext[1]{https://v2.ocaml.org/api/Printexc.html}
\end{frame}

\newcommand\listCamlRaiseExn[1][]{
  \listasm[firstline=702,lastline=725,#1]{../ocaml/runtime/amd64.S}{runtime/amd64.S:702}}

\begin{frame}{Backtraces}{Walk through \funcname{caml\_raise\_exn}}
  \begin{columns}[T]
    \begin{column}{0.5\textwidth}
      \begin{itemize}
        \item<1-> First checks whether exceptions backtrace should be recorded
        \item<1-> By checking the \funcarg{domain\_state->backtrace\_active} value
        \item<2-> If not, acts as \funcname{raise\_notrace} with \funcname{RESTORE\_EXN\_HANDLER\_OCAML}
          \begin{itemize}
            \item Set \regname{RSP} to \localname{domain\_state->exn\_handler} value
            \item Restore \localname{domain\_state->exn\_handler} from trap
            \item Jump with \funcname{ret} (same effect as using \funcname{pop} and \funcname{jmp})
          \end{itemize}
        \item<3-> Otherwise, switches to the C stack to be able to execute C code
        \item<4-> Calls \funcname{caml\_stash\_backtrace}, a C function provided by the runtime\ignorespaces
          \onslide<6->{, with
            \begin{itemize}
              \item \funcarg{exn}: exception value
              \item \funcarg{pc}: code address of the raising point
              \item \funcarg{sp}: stack address of the raising function
              \item \funcarg{trapsp}: stack address of the next handler
            \end{itemize}
          }
      \end{itemize}
      \bigskip
      \mintocaml[firstline=9,lastline=13,highlightlines=12]{../src/backtrace/backtrace.ml}
    \end{column}
    \begin{column}{0.5\textwidth}
      \raggedleft
      \onslide*<1,3-4>{
        \mintc[firstline=190,lastline=192]{../ocaml/runtime/amd64.S}
        \mintc[firstline=234,lastline=237]{../ocaml/runtime/amd64.S}
      }
      \onslide*<2>{
        \mintc[firstline=190,lastline=192,highlightlines={191-192}]{../ocaml/runtime/amd64.S}
        \mintc[firstline=234,lastline=237,highlightlines={235-237}]{../ocaml/runtime/amd64.S}
      }
      \onslide*<1>{\listCamlRaiseExn[highlightlines=705]{}}%
      \onslide*<2>{\listCamlRaiseExn[highlightlines={707-708}]{}}%
      \onslide*<3>{\listCamlRaiseExn[highlightlines={712-714}]{}}%
      \onslide*<4>{\listCamlRaiseExn[highlightlines={715-720}]{}}%
      \onslide*<5>{\providecommand\step{1}\input{diagrams/backtrace/backtrace.tex}}%
      \onslide*<6>{\providecommand\step{2}\input{diagrams/backtrace/backtrace.tex}}%
    \end{column}
  \end{columns}
\end{frame}

\newcommand\listStashBacktrace[1][]{
  \listc[firstline=91,lastline=120,#1]{../ocaml/runtime/backtrace\_nat.c}{runtime/backtrace\_nat.c:91}}

\begin{frame}{Backtraces}{Introducing \funcname{caml\_stash\_backtrace}}
  \begin{columns}[c]
    \begin{column}{0.5\textwidth}
      \minipage[c][0.66\textheight][s]{\columnwidth}
      \begin{itemize}
        \item<1-> A C function provided by the runtime (specific to native code)
        \item<2-> It scans the stack, between the stack pointer at the raising point up to the next trap, in order to collect the names of the detected function frames
          \onslide*<2>{
            \begin{itemize}
              \item And more information, like line number, column number...
            \end{itemize}
          }
        \item<3-> It uses the same technique and information as the Garbage Collector (GC) uses to scan the values live on the stack
          \onslide*<4>{
            \begin{itemize}
              \item \typename{frame\_descr} are at the core of stack unwinding
            \end{itemize}
          }
      \end{itemize}
      \vfill
      \mintocaml[firstline=9,lastline=13,highlightlines=12]{../src/backtrace/backtrace.ml}
      \endminipage
    \end{column}
    \begin{column}{0.5\textwidth}
      \onslide*<1>{\listStashBacktrace[highlightlines=91]{}}
      \onslide*<2>{\listStashBacktrace[highlightlines={91,117-118}]{}}
      \onslide*<3->{\listStashBacktrace[highlightlines={94,106,109-110}]{}}
    \end{column}
  \end{columns}
\end{frame}

\newcommand\listFrameDescr[1][]{
  \listocaml[firstline=111,lastline=124,#1]{../ocaml/asmcomp/emitaux.ml}{asmcomp/emitaux.ml:111}}
\newcommand\listDebugInfo[1][]{
  \listocaml[firstline=38,lastline=49,#1]{../ocaml/lambda/debuginfo.mli}{lambda/debuginfo.mli:38}}

\begin{frame}{Backtraces}{\typename{frame\_descr} and \typename{frame\_debuginfo} in a nutshell}
  \begin{columns}[c]
    \begin{column}{0.5\textwidth}
      \begin{itemize}
        \item<1-> For every return address in an OCaml program, there is a associated \typename{frame\_descr}
        \item<2-> A \typename{frame\_descr} specifies
        \begin{itemize}
          \item<2-> The size of the function stack frame
          \item<3-> Where live values are on the stack (so that the GC can mark them as live and prevent from being swept)
        \end{itemize}
        \item<4-> When compiled with debug information, \typename{frame\_descr} will be linked to a \typename{frame\_debuginfo}
        \begin{itemize}
          \item<5-> Contains information about the position in the source code (file name, line and column number)
        \end{itemize}
      \end{itemize}
    \end{column}
    \begin{column}{0.5\textwidth}
        \onslide*<1>{\listFrameDescr[highlightlines={118-122}]{}}
        \onslide*<2>{\listFrameDescr[highlightlines={120}]{}}
        \onslide*<3>{\listFrameDescr[highlightlines={121}]{}}
        \onslide*<4->{\listFrameDescr[highlightlines={122}]{}}
        \medskip
        \onslide*<1-3>{\listDebugInfo{}}
        \onslide*<4>{\listDebugInfo[highlightlines={38,49}]{}}
        \onslide*<5>{\listDebugInfo[highlightlines={39-46}]{}}
    \end{column}
  \end{columns}
\end{frame}

\begin{frame}{Backtraces}{Scanning the stack with \typename{frame\_descr}}
  \begin{columns}[c]
    \begin{column}{0.5\textwidth}
      \begin{itemize}
        \item Given the return address of the code that raises the exception by calling \funcname{caml\_raise\_exn} (\funcarg{pc} argument of \funcname{caml\_stash\_backtrace})
        \item And the position of the stack pointer (\funcarg{sp} argument of \funcname{caml\_stash\_backtrace})
        \item The frame size given by \typename{frame\_descr}.\funcarg{frame\_size}, allows to find the end of the previous frame
        \item And the return address of the caller of this function
        \bigskip
        \onslide<3->{
        \item Note that traps (here the reddish \funcname{ExnA}, \funcname{ExnB} and \funcname{ExnC} blocks) belong to the frame that installed them, and so the frame size changes when traps are removed
        }
      \end{itemize}
    \end{column}
    \begin{column}{0.5\textwidth}
      \raggedleft
      \onslide*<1>{\providecommand\step{2}\input{diagrams/backtrace/backtrace.tex}}%
      \onslide*<2>{\providecommand\step{1}\input{diagrams/backtrace/scanstack.tex}}%
      \onslide*<3>{\providecommand\step{2}\input{diagrams/backtrace/scanstack.tex}}%
      \onslide*<4>{\providecommand\step{3}\input{diagrams/backtrace/scanstack.tex}}%
      \onslide*<5>{\providecommand\step{4}\input{diagrams/backtrace/scanstack.tex}}%
      \onslide*<6>{\providecommand\step{5}\input{diagrams/backtrace/scanstack.tex}}%
    \end{column}
  \end{columns}
\end{frame}

% Duplicate of previous-previous slide
\begin{frame}{Backtraces}{Continuing on \funcname{caml\_stash\_backtrace}}
  \begin{columns}[c]
    \begin{column}{0.5\textwidth}
      \minipage[c][0.75\textheight][s]{\columnwidth}
      \begin{itemize}
          \color{gray}
        \item A C function provided by the runtime (specific to native code)
        \item It scans the stack, between the stack pointer at the raising point up to the next trap, in order to collect the names of the detected function frames
        \item It uses the same technique and information as the Garbage Collector (GC) uses to scan the values live on the stack
      \end{itemize}
      \begin{itemize}
        \item For each function frame detected, store a pointer to the \typename{frame\_descr} in the \localname{domain\_state->backtrace\_buffer} array, effectively constructing the backtrace
      \end{itemize}
      \onslide<2->{
        \begin{itemize}
            \smallskip
          \item Constructed backtrace:
            \funcname{definitely\_raise},
            \onslide<3->{\funcname{probably\_raise}}
        \end{itemize}
      }
      \vfill
      \mintocaml[firstline=9,lastline=13,highlightlines=12]{../src/backtrace/backtrace.ml}
      \endminipage
    \end{column}
    \begin{column}{0.5\textwidth}
      \raggedleft
      \onslide*<1>{\listStashBacktrace[highlightlines={114-115}]{}}
      \onslide*<2>{\providecommand\step{3}\input{diagrams/backtrace/backtrace.tex}}%
      \onslide*<3>{\providecommand\step{4}\input{diagrams/backtrace/backtrace.tex}}%
    \end{column}
  \end{columns}
\end{frame}

\begin{frame}{Backtraces}{Back to \funcname{caml\_raise\_exn}}
  \begin{columns}[c]
    \begin{column}{0.5\textwidth}
      \minipage[c][0.66\textheight][s]{\columnwidth}
      \begin{itemize}\color{gray}
        \item Checks wheteher exceptions backtrace should be recorded, by checking the \funcarg{domain\_state->backtrace\_active} value
        \item Switches to the C stack to be able to execute C code
        \item Calls \funcname{caml\_stash\_backtrace}, to collect the backtrace up to the next trap
      \end{itemize}
      \onslide*<2->{
        \begin{itemize}
          \item<2-> Executes the exception handler in \funcname{probably\_raise} with \funcname{RESTORE\_EXN\_HANDLER\_OCAML}
            \begin{itemize}
              \item Set \regname{RSP} to \localname{domain\_state->exn\_handler} value
              \item Restore \localname{domain\_state->exn\_handler} from trap
              \item Jump to the handler code with \funcname{ret}
            \end{itemize}
        \end{itemize}
      }
      \vfill
      \onslide*<1>{\mintocaml[firstline=9,lastline=13,highlightlines=12]{../src/backtrace/backtrace.ml}}
      \onslide*<2->{\mintocaml[firstline=15,lastline=21,highlightlines={19-21}]{../src/backtrace/backtrace.ml}}
      \endminipage
    \end{column}
    \begin{column}{0.5\textwidth}
      \centering
      \onslide*<1>{
        \mintc[firstline=190,lastline=192]{../ocaml/runtime/amd64.S}
        \mintc[firstline=234,lastline=237]{../ocaml/runtime/amd64.S}
      }
      \onslide*<2>{
        \mintc[firstline=190,lastline=192,highlightlines={191-192}]{../ocaml/runtime/amd64.S}
        \mintc[firstline=234,lastline=237,highlightlines={235-237}]{../ocaml/runtime/amd64.S}
      }
      \onslide*<1>{\listCamlRaiseExn[highlightlines=720]{}}
      \onslide*<2>{\listCamlRaiseExn[highlightlines=722]{}}
      \onslide*<3->{\providecommand\step{5}\input{diagrams/backtrace/backtrace.tex}}
    \end{column}
  \end{columns}
\end{frame}

\begin{frame}{Backtraces}{\funcname{caml\_probably\_raise}'s handler doesn't catch \typename{ExnC}}
  \begin{columns}[c]
    \begin{column}{0.45\textwidth}
      \minipage[c][0.75\textheight][s]{\columnwidth}
      \begin{itemize}
        \item<1-> \funcname{match \dots with}\footnotemark pattern matching can also match exceptions
        \item<1-> The CMM of \funcname{probably\_raise} shows that this \funcname{match \dots with} is implemented with a pattern matching and a \funcname{try \dots with}
        \item<1-> But this one only matches on \typename{ExnA} exception
        \item<2-> Since the pattern matching fails to catch the exception, \funcname{reraise} is called with our current \typename{ExnC} exception
        \item<3-> Disassembly shows that \funcname{reraise} is actually a call to \funcname{caml\_reraise\_exn}, which is also an assembly function provided by the runtime
      \end{itemize}
      \vfill
      \mintocaml[firstline=15,lastline=21,highlightlines={18-21}]{../src/backtrace/backtrace.ml}
      \endminipage
    \end{column}
    \begin{column}{0.55\textwidth}
      \centering
      \onslide*<1>{\mintcmm[highlightlines={10-18}]{../src/backtrace/camlBacktrace\_\_probably\_raise\_351.cmm}}
      \onslide*<2>{\listcmm[highlightlines=18]{../src/backtrace/camlBacktrace\_\_probably\_raise_351.cmm}{CMM of probably\_raise}}
      \onslide*<3>{\mintobjdump[firstline=5,lastline=35,highlightlines=31]{../src/backtrace/camlBacktrace\_\_probably\_raise_351.objdump}}
    \end{column}
  \end{columns}
  \only<1>{\footnotetext[1]{https://blog.janestreet.com/pattern-matching-and-exception-handling-unite/}}
\end{frame}

\newcommand\listCamlReraiseExn[1][]{
  \listasm[firstline=727,lastline=734,#1]{../ocaml/runtime/amd64.S}{runtime/amd64.S:727}}

\begin{frame}{Backtraces}{Introducing \funcname{caml\_reraise\_exn}}
  \begin{columns}[c]
    \begin{column}{0.5\textwidth}
      \minipage[c][0.66\textheight][s]{\columnwidth}
      \begin{itemize}
        \item<1-> The exception have to be reraised to the next handler
        \item<2-> First, checks if the backtrace should be collected with \funcarg{domain\_state->backtrace\_active}
        \only<3>{
        \begin{itemize}
            \item If not, behaves like a \funcname{raise\_notrace} with \funcname{RESTORE\_EXN\_HANDLER\_OCAML}
        \end{itemize}
        }
        \item<4-> Jumps into \funcname{caml\_raise\_exn}
        \item<5-> Calls \funcname{caml\_stash\_backtrace} again, to scan the next part of the stack: from the trap just pop-ed to the next trap
      \end{itemize}
      \vfill
      \mintocaml[firstline=15,lastline=21,highlightlines={18-21}]{../src/backtrace/backtrace.ml}
      \endminipage
    \end{column}
    \begin{column}{0.5\textwidth}
      \raggedleft
      \onslide*<1>{\listCamlReraiseExn{}}
      \onslide*<2>{\listCamlReraiseExn[highlightlines=729]{}}
      \onslide*<3>{
        \mintc[firstline=190,lastline=192,highlightlines={191-192}]{../ocaml/runtime/amd64.S}
        \mintc[firstline=234,lastline=237,highlightlines={235-237}]{../ocaml/runtime/amd64.S}
        \medskip
        \listCamlReraiseExn[highlightlines=731]{}
      }
      \onslide*<4-5>{
        % TODO refactor with \listCamlReraiseExn
        \mintasm[firstline=727,lastline=734,highlightlines=730]{../ocaml/runtime/amd64.S}
        \medskip
      }
      \onslide*<4>{\listCamlRaiseExn[highlightlines=711]{}}
      \onslide*<5>{\listCamlRaiseExn[highlightlines=720]{}}
    \end{column}
  \end{columns}
\end{frame}

\begin{frame}{Backtraces}{Backtrace collection continues}
  \begin{columns}[c]
    \begin{column}{0.55\textwidth}
      \minipage[c][0.75\textheight][s]{\columnwidth}
      \begin{itemize}
        \item<1-> \funcname{caml\_stash\_backtrace} scans the older part of the stack, up to the next trap
        \item<6-> Controls jumps to the nested exception handler in \funcname{maybe\_raise}, which matches only on \typename{ExnB}
        \item<7-> \funcname{caml\_reraise\_exn} is called again, backtrace collection resumes with \funcname{caml\_stash\_backtrace}
      \end{itemize}
      \medskip
      \begin{itemize}
        \item<2-> Constructed backtrace:
        \begin{itemize}
          \item <2-> \funcname{definitely\_raise}
          \item <2-> \funcname{probably\_raise}
          \item <3-> \funcname{probably\_raise} (re-raise)
          \item <4-> \funcname{maybe\_raise}
          \item <5-> \funcname{main}
          \item <8-> \funcname{main} (re-raise)
        \end{itemize}
      \end{itemize}
      \vfill
      \onslide*<1-5>{\mintocaml[firstline=15,lastline=21]{../src/backtrace/backtrace.ml}}
      \onslide*<6>{\mintocaml[firstline=28,lastline=34,highlightlines=31]{../src/backtrace/backtrace.ml}}
      \onslide*<7->{\mintocaml[firstline=28,lastline=34,highlightlines=33]{../src/backtrace/backtrace.ml}}
      \endminipage
    \end{column}
    \begin{column}{0.45\textwidth}
      \raggedleft
      \onslide*<1>{\providecommand\step{6}\input{diagrams/backtrace/backtrace.tex}}%
      \onslide*<2>{\providecommand\step{6}\input{diagrams/backtrace/backtrace.tex}}%
      \onslide*<3>{\providecommand\step{7}\input{diagrams/backtrace/backtrace.tex}}%
      \onslide*<4>{\providecommand\step{8}\input{diagrams/backtrace/backtrace.tex}}%
      \onslide*<5>{\providecommand\step{9}\input{diagrams/backtrace/backtrace.tex}}%
      \onslide*<6>{\providecommand\step{10}\input{diagrams/backtrace/backtrace.tex}}%
      \onslide*<7>{\providecommand\step{11}\input{diagrams/backtrace/backtrace.tex}}%
      \onslide*<8>{\providecommand\step{12}\input{diagrams/backtrace/backtrace.tex}}%
      \onslide*<9>{\providecommand\step{13}\input{diagrams/backtrace/backtrace.tex}}%
    \end{column}
  \end{columns}
\end{frame}

\begin{frame}{Backtraces}{Printing the backtrace}
  \begin{columns}[c]
    \begin{column}{0.45\textwidth}
      \minipage[c][0.66\textheight][s]{\columnwidth}
      \begin{itemize}
        \item<1-> The exception backtrace can now be printed to \funcarg{stdout} with \funcname{Printexc.print_backtrace}
        \item<2-> First the exception backtrace is retrieved into an OCaml array with \funcname{get_raw_backtrace }
        \item<3-> Which is implemented as the \funcname{caml_get_exception_raw_backtrace} C function in the runtime
        \item<4-> And finally printed with \funcname{print_exception_backtrace}
      \end{itemize}
      \vfill
      \mintocaml[firstline=28,lastline=34,highlightlines=33]{../src/backtrace/backtrace.ml}
      \endminipage
    \end{column}
    \begin{column}{0.55\textwidth}
      \onslide*<1,2>{
        \mintocaml[firstline=100,lastline=101]{../ocaml/stdlib/printexc.ml}
      }
      \only<1>{
        \listocaml[firstline=170,lastline=172]{../ocaml/stdlib/printexc.ml}{stdlib/printexc.ml:170}
      }
      \only<2>{
        \listocaml[firstline=170,lastline=172,highlightlines=172]{../ocaml/stdlib/printexc.ml}{stdlib/printexc.ml:170}
      }
      \only<3>{
        \mintc[firstline=154,lastline=156]{../ocaml/runtime/backtrace.c}
        \listc[firstline=171,lastline=192,highlightlines={184-187}]{../ocaml/runtime/backtrace.c}{runtime/backtrace.c:154}
      }
      \only<4>{
        \listocaml[firstline=155,lastline=165]{../ocaml/stdlib/printexc.ml}{stdlib/printexc.ml:55}
      }
    \end{column}
  \end{columns}
\end{frame}


\subsection{The multiple kinds of raise}
\frameSubsection{}{}

\begin{frame}{Backtraces}{When backtraces are not needed}
  \begin{columns}[c]
    \begin{column}{0.45\textwidth}
      \minipage[c][0.66\textheight][s]{\columnwidth}
      \begin{itemize}
        \item<1-> What if the backtrace for an exception is never needed?
        \item<2-> It's possible to raise the exception explicitly with \funcname{raise\_notrace} to make raising as cheap as possible
        \item<3-> An example of use in the \funcname{dynlink} library of the OCaml compiler
      \end{itemize}
      \vfill
      \onslide<3->{
      \listocaml[firstline=205,lastline=209,highlightlines=207]{../ocaml/otherlibs/dynlink/dynlink\_compilerlibs/misc.ml}{otherlibs/dynlink/dynlink\_compilerlibs/misc.ml:205}
      }
      \endminipage
    \end{column}
    \begin{column}{0.55\textwidth}
    \only<4>{
      \mintcmm[highlightlines=3]{../src/notrace/camlNotrace\_\_fun_347.cmm}
      \listcmm{../src/notrace/camlNotrace\_\_all\_somes\_327.cmm}{all\_somes}
    }
    \onslide*<5->{
      \listobjdump[highlightlines={6-9}]{../src/notrace/camlNotrace\_\_fun_347.objdump}{anonymous function from \funcname{all\_somes}}
    }
    \end{column}
  \end{columns}
\end{frame}

\begin{frame}{Backtraces}{Summary table of the multiple kinds of raise}
  \begin{itemize}
    \item When debugging information is enabled and if the backtrace of an exception isn't needed, it's possible to raise with \funcname{raise\_notrace} for performance
    \item When debugging information is \emph{not} enabled, the compiler optimises \funcname{raise} calls to inline assembly (same effect as using \funcname{raise\_notrace})
  \end{itemize}
  \bigskip
  \centering
  \begin{tabular}{| l | c | c | c | c |}
    \hline
    \parbox{10em}{Debug information \\ (\funcarg{-g} flag)}  & \multicolumn{2}{|c|}{Disabled}                                     & \multicolumn{2}{|c|}{Enabled} \\
    \hline
    Raise kind                                               & \funcname{raise} & \funcname{raise\_notrace}                       & \funcname{raise} & \funcname{raise\_notrace} \\
    \hline
    Compilation                                              & \parbox{8em}{inline assembly\\ (optimization)} & inline assembly   & \funcname{caml\_raise\_exn} & inline assembly \\
    \hline
  \end{tabular}
\end{frame}


%
% Takeaway #5
%
\subsection*{Takeaway \#5}
\frameSubsectionTakeaway{}
\againframe<5>{takeaway}
}
    \end{column}
  \end{columns}
\end{frame}

\begin{frame}{Backtraces}{\funcname{caml\_probably\_raise}'s handler doesn't catch \typename{ExnC}}
  \begin{columns}[c]
    \begin{column}{0.45\textwidth}
      \minipage[c][0.75\textheight][s]{\columnwidth}
      \begin{itemize}
        \item<1-> \funcname{match \dots with}\footnotemark pattern matching can also match exceptions
        \item<1-> The CMM of \funcname{probably\_raise} shows that this \funcname{match \dots with} is implemented with a pattern matching and a \funcname{try \dots with}
        \item<1-> But this one only matches on \typename{ExnA} exception
        \item<2-> Since the pattern matching fails to catch the exception, \funcname{reraise} is called with our current \typename{ExnC} exception
        \item<3-> Disassembly shows that \funcname{reraise} is actually a call to \funcname{caml\_reraise\_exn}, which is also an assembly function provided by the runtime
      \end{itemize}
      \vfill
      \mintocaml[firstline=15,lastline=21,highlightlines={18-21}]{../src/backtrace/backtrace.ml}
      \endminipage
    \end{column}
    \begin{column}{0.55\textwidth}
      \centering
      \onslide*<1>{\mintcmm[highlightlines={10-18}]{../src/backtrace/camlBacktrace\_\_probably\_raise\_351.cmm}}
      \onslide*<2>{\listcmm[highlightlines=18]{../src/backtrace/camlBacktrace\_\_probably\_raise_351.cmm}{CMM of probably\_raise}}
      \onslide*<3>{\mintobjdump[firstline=5,lastline=35,highlightlines=31]{../src/backtrace/camlBacktrace\_\_probably\_raise_351.objdump}}
    \end{column}
  \end{columns}
  \only<1>{\footnotetext[1]{https://blog.janestreet.com/pattern-matching-and-exception-handling-unite/}}
\end{frame}

\newcommand\listCamlReraiseExn[1][]{
  \listasm[firstline=727,lastline=734,#1]{../ocaml/runtime/amd64.S}{runtime/amd64.S:727}}

\begin{frame}{Backtraces}{Introducing \funcname{caml\_reraise\_exn}}
  \begin{columns}[c]
    \begin{column}{0.5\textwidth}
      \minipage[c][0.66\textheight][s]{\columnwidth}
      \begin{itemize}
        \item<1-> The exception have to be reraised to the next handler
        \item<2-> First, checks if the backtrace should be collected with \funcarg{domain\_state->backtrace\_active}
        \only<3>{
        \begin{itemize}
            \item If not, behaves like a \funcname{raise\_notrace} with \funcname{RESTORE\_EXN\_HANDLER\_OCAML}
        \end{itemize}
        }
        \item<4-> Jumps into \funcname{caml\_raise\_exn}
        \item<5-> Calls \funcname{caml\_stash\_backtrace} again, to scan the next part of the stack: from the trap just pop-ed to the next trap
      \end{itemize}
      \vfill
      \mintocaml[firstline=15,lastline=21,highlightlines={18-21}]{../src/backtrace/backtrace.ml}
      \endminipage
    \end{column}
    \begin{column}{0.5\textwidth}
      \raggedleft
      \onslide*<1>{\listCamlReraiseExn{}}
      \onslide*<2>{\listCamlReraiseExn[highlightlines=729]{}}
      \onslide*<3>{
        \mintc[firstline=190,lastline=192,highlightlines={191-192}]{../ocaml/runtime/amd64.S}
        \mintc[firstline=234,lastline=237,highlightlines={235-237}]{../ocaml/runtime/amd64.S}
        \medskip
        \listCamlReraiseExn[highlightlines=731]{}
      }
      \onslide*<4-5>{
        % TODO refactor with \listCamlReraiseExn
        \mintasm[firstline=727,lastline=734,highlightlines=730]{../ocaml/runtime/amd64.S}
        \medskip
      }
      \onslide*<4>{\listCamlRaiseExn[highlightlines=711]{}}
      \onslide*<5>{\listCamlRaiseExn[highlightlines=720]{}}
    \end{column}
  \end{columns}
\end{frame}

\begin{frame}{Backtraces}{Backtrace collection continues}
  \begin{columns}[c]
    \begin{column}{0.55\textwidth}
      \minipage[c][0.75\textheight][s]{\columnwidth}
      \begin{itemize}
        \item<1-> \funcname{caml\_stash\_backtrace} scans the older part of the stack, up to the next trap
        \item<6-> Controls jumps to the nested exception handler in \funcname{maybe\_raise}, which matches only on \typename{ExnB}
        \item<7-> \funcname{caml\_reraise\_exn} is called again, backtrace collection resumes with \funcname{caml\_stash\_backtrace}
      \end{itemize}
      \medskip
      \begin{itemize}
        \item<2-> Constructed backtrace:
        \begin{itemize}
          \item <2-> \funcname{definitely\_raise}
          \item <2-> \funcname{probably\_raise}
          \item <3-> \funcname{probably\_raise} (re-raise)
          \item <4-> \funcname{maybe\_raise}
          \item <5-> \funcname{main}
          \item <8-> \funcname{main} (re-raise)
        \end{itemize}
      \end{itemize}
      \vfill
      \onslide*<1-5>{\mintocaml[firstline=15,lastline=21]{../src/backtrace/backtrace.ml}}
      \onslide*<6>{\mintocaml[firstline=28,lastline=34,highlightlines=31]{../src/backtrace/backtrace.ml}}
      \onslide*<7->{\mintocaml[firstline=28,lastline=34,highlightlines=33]{../src/backtrace/backtrace.ml}}
      \endminipage
    \end{column}
    \begin{column}{0.45\textwidth}
      \raggedleft
      \onslide*<1>{\providecommand\step{6}%
% Backtraces
%
\subsection{One advantage of raising with debugging information}
\frameSubsection{}{}

\begin{frame}{Backtraces}{Debugging information \& source code}
  \begin{columns}[c]
    \begin{column}{0.5\textwidth}
      \begin{itemize}
        \item Raising an exception is fun and games, but how do we know where the exception originated? $\rightarrow$ With the backtrace associated with the exception!
        \item In order to get a backtrace from the point where an exception was raised, it's necessary to compile with debugging information enabled (\funcarg{-g} flag for the compiler)
        \item Same example as in the `Nested handlers' section
      \end{itemize}
    \end{column}
    \begin{column}{0.5\textwidth}
      \listocaml[firstline=9,lastline=34]{../src/backtrace/backtrace.ml}{backtrace/backtrace.ml}
    \end{column}
  \end{columns}
\end{frame}

\begin{frame}{Backtraces}{CMM and disassembly of \funcname{definitely\_raise}}
  \begin{columns}[c]
    \begin{column}{0.5\textwidth}
      \minipage[c][0.66\textheight][s]{\columnwidth}
      \begin{itemize}
        \item<1-> An effect of compiling with debugging information (\funcarg{-g}): the CMM no longer uses \funcname{raise\_notrace} to raise the exception but \funcname{raise} instead
        \item<2-> Disassembly shows that CMM \funcname{raise} is actually a call to \funcname{caml\_raise\_exn}, a function in assembler provided by the runtime
      \end{itemize}
      \vfill
      \mintocaml[firstline=9,lastline=13,highlightlines=12]{../src/backtrace/backtrace.ml}
      \endminipage
    \end{column}
    \begin{column}{0.5\textwidth}
      \onslide*<1>{
        \mintcmm[highlightlines=5]{../src/backtrace/camlBacktrace\_\_definitely\_raise\_347.cmm}
      }
      \onslide*<2>{
        \mintobjdump[firstline=2,highlightlines=14]{../src/backtrace/camlBacktrace\_\_definitely\_raise\_347.objdump}
      }
    \end{column}
  \end{columns}
\end{frame}

\begin{frame}{Backtraces}{Introducing \funcname{caml\_raise\_exn}}
  \begin{columns}[c]
    \begin{column}{0.5\textwidth}
      \minipage[c][0.66\textheight][s]{\columnwidth}
      \onslide*<1>{
        \begin{itemize}
          \item Raising the exception is performed using \funcname{caml\_raise\_exn} which is
            \begin{itemize}
              \item An assembly function provided by the runtime
              \item Architecture specific
            \end{itemize}
          \item Let's see what it does differently from \funcname{raise\_notrace}\dots
          \item We are about to raise \typename{ExnC} with \funcname{caml\_raise\_exn} from \funcname{definitely\_raise}
        \end{itemize}
      }
      \onslide*<2>{
        \mintobjdump[firstline=2,highlightlines=14]{../src/backtrace/camlBacktrace\_\_definitely\_raise\_347.objdump}
      }
      \vfill
      \mintocaml[firstline=9,lastline=13,highlightlines=12]{../src/backtrace/backtrace.ml}
      \endminipage
    \end{column}
    \begin{column}{0.5\textwidth}
      \centering
      \providecommand\step{1}\input{diagrams/backtrace/backtrace.tex}
    \end{column}
  \end{columns}
\end{frame}

\begin{frame}{Backtraces}{But before that, we need more fields from \typename{caml\_domain\_state}}
  \begin{columns}
    \begin{column}{0.5\textwidth}
      \begin{itemize}
        \item Remember \typename{caml\_domain\_state} the per-domain runtime state?
        \item Some other members are of interest to us now\dots
        \item \localname{backtrace\_active} is used to know if the runtime should collect backtraces
        \item \localname{backtrace\_active} can be set by
          \begin{itemize}
            \item The \funcarg{b} flag in the \funcname{OCAMLRUNPARAM} environment variable
            \item \mintinline{ocaml}{Printexc.record_backtrace : bool -> unit} \footnotemark
          \end{itemize}
      \end{itemize}
    \end{column}
    \begin{column}{0.5\textwidth}
      \begin{listing}
        \mintc[highlightlines={9-11}]{src/ocaml/domain\_state\_backtrace.h}
        \caption{runtime/caml/domain\_state.h}
      \end{listing}
    \end{column}
  \end{columns}
  \footnotetext[1]{https://v2.ocaml.org/api/Printexc.html}
\end{frame}

\newcommand\listCamlRaiseExn[1][]{
  \listasm[firstline=702,lastline=725,#1]{../ocaml/runtime/amd64.S}{runtime/amd64.S:702}}

\begin{frame}{Backtraces}{Walk through \funcname{caml\_raise\_exn}}
  \begin{columns}[T]
    \begin{column}{0.5\textwidth}
      \begin{itemize}
        \item<1-> First checks whether exceptions backtrace should be recorded
        \item<1-> By checking the \funcarg{domain\_state->backtrace\_active} value
        \item<2-> If not, acts as \funcname{raise\_notrace} with \funcname{RESTORE\_EXN\_HANDLER\_OCAML}
          \begin{itemize}
            \item Set \regname{RSP} to \localname{domain\_state->exn\_handler} value
            \item Restore \localname{domain\_state->exn\_handler} from trap
            \item Jump with \funcname{ret} (same effect as using \funcname{pop} and \funcname{jmp})
          \end{itemize}
        \item<3-> Otherwise, switches to the C stack to be able to execute C code
        \item<4-> Calls \funcname{caml\_stash\_backtrace}, a C function provided by the runtime\ignorespaces
          \onslide<6->{, with
            \begin{itemize}
              \item \funcarg{exn}: exception value
              \item \funcarg{pc}: code address of the raising point
              \item \funcarg{sp}: stack address of the raising function
              \item \funcarg{trapsp}: stack address of the next handler
            \end{itemize}
          }
      \end{itemize}
      \bigskip
      \mintocaml[firstline=9,lastline=13,highlightlines=12]{../src/backtrace/backtrace.ml}
    \end{column}
    \begin{column}{0.5\textwidth}
      \raggedleft
      \onslide*<1,3-4>{
        \mintc[firstline=190,lastline=192]{../ocaml/runtime/amd64.S}
        \mintc[firstline=234,lastline=237]{../ocaml/runtime/amd64.S}
      }
      \onslide*<2>{
        \mintc[firstline=190,lastline=192,highlightlines={191-192}]{../ocaml/runtime/amd64.S}
        \mintc[firstline=234,lastline=237,highlightlines={235-237}]{../ocaml/runtime/amd64.S}
      }
      \onslide*<1>{\listCamlRaiseExn[highlightlines=705]{}}%
      \onslide*<2>{\listCamlRaiseExn[highlightlines={707-708}]{}}%
      \onslide*<3>{\listCamlRaiseExn[highlightlines={712-714}]{}}%
      \onslide*<4>{\listCamlRaiseExn[highlightlines={715-720}]{}}%
      \onslide*<5>{\providecommand\step{1}\input{diagrams/backtrace/backtrace.tex}}%
      \onslide*<6>{\providecommand\step{2}\input{diagrams/backtrace/backtrace.tex}}%
    \end{column}
  \end{columns}
\end{frame}

\newcommand\listStashBacktrace[1][]{
  \listc[firstline=91,lastline=120,#1]{../ocaml/runtime/backtrace\_nat.c}{runtime/backtrace\_nat.c:91}}

\begin{frame}{Backtraces}{Introducing \funcname{caml\_stash\_backtrace}}
  \begin{columns}[c]
    \begin{column}{0.5\textwidth}
      \minipage[c][0.66\textheight][s]{\columnwidth}
      \begin{itemize}
        \item<1-> A C function provided by the runtime (specific to native code)
        \item<2-> It scans the stack, between the stack pointer at the raising point up to the next trap, in order to collect the names of the detected function frames
          \onslide*<2>{
            \begin{itemize}
              \item And more information, like line number, column number...
            \end{itemize}
          }
        \item<3-> It uses the same technique and information as the Garbage Collector (GC) uses to scan the values live on the stack
          \onslide*<4>{
            \begin{itemize}
              \item \typename{frame\_descr} are at the core of stack unwinding
            \end{itemize}
          }
      \end{itemize}
      \vfill
      \mintocaml[firstline=9,lastline=13,highlightlines=12]{../src/backtrace/backtrace.ml}
      \endminipage
    \end{column}
    \begin{column}{0.5\textwidth}
      \onslide*<1>{\listStashBacktrace[highlightlines=91]{}}
      \onslide*<2>{\listStashBacktrace[highlightlines={91,117-118}]{}}
      \onslide*<3->{\listStashBacktrace[highlightlines={94,106,109-110}]{}}
    \end{column}
  \end{columns}
\end{frame}

\newcommand\listFrameDescr[1][]{
  \listocaml[firstline=111,lastline=124,#1]{../ocaml/asmcomp/emitaux.ml}{asmcomp/emitaux.ml:111}}
\newcommand\listDebugInfo[1][]{
  \listocaml[firstline=38,lastline=49,#1]{../ocaml/lambda/debuginfo.mli}{lambda/debuginfo.mli:38}}

\begin{frame}{Backtraces}{\typename{frame\_descr} and \typename{frame\_debuginfo} in a nutshell}
  \begin{columns}[c]
    \begin{column}{0.5\textwidth}
      \begin{itemize}
        \item<1-> For every return address in an OCaml program, there is a associated \typename{frame\_descr}
        \item<2-> A \typename{frame\_descr} specifies
        \begin{itemize}
          \item<2-> The size of the function stack frame
          \item<3-> Where live values are on the stack (so that the GC can mark them as live and prevent from being swept)
        \end{itemize}
        \item<4-> When compiled with debug information, \typename{frame\_descr} will be linked to a \typename{frame\_debuginfo}
        \begin{itemize}
          \item<5-> Contains information about the position in the source code (file name, line and column number)
        \end{itemize}
      \end{itemize}
    \end{column}
    \begin{column}{0.5\textwidth}
        \onslide*<1>{\listFrameDescr[highlightlines={118-122}]{}}
        \onslide*<2>{\listFrameDescr[highlightlines={120}]{}}
        \onslide*<3>{\listFrameDescr[highlightlines={121}]{}}
        \onslide*<4->{\listFrameDescr[highlightlines={122}]{}}
        \medskip
        \onslide*<1-3>{\listDebugInfo{}}
        \onslide*<4>{\listDebugInfo[highlightlines={38,49}]{}}
        \onslide*<5>{\listDebugInfo[highlightlines={39-46}]{}}
    \end{column}
  \end{columns}
\end{frame}

\begin{frame}{Backtraces}{Scanning the stack with \typename{frame\_descr}}
  \begin{columns}[c]
    \begin{column}{0.5\textwidth}
      \begin{itemize}
        \item Given the return address of the code that raises the exception by calling \funcname{caml\_raise\_exn} (\funcarg{pc} argument of \funcname{caml\_stash\_backtrace})
        \item And the position of the stack pointer (\funcarg{sp} argument of \funcname{caml\_stash\_backtrace})
        \item The frame size given by \typename{frame\_descr}.\funcarg{frame\_size}, allows to find the end of the previous frame
        \item And the return address of the caller of this function
        \bigskip
        \onslide<3->{
        \item Note that traps (here the reddish \funcname{ExnA}, \funcname{ExnB} and \funcname{ExnC} blocks) belong to the frame that installed them, and so the frame size changes when traps are removed
        }
      \end{itemize}
    \end{column}
    \begin{column}{0.5\textwidth}
      \raggedleft
      \onslide*<1>{\providecommand\step{2}\input{diagrams/backtrace/backtrace.tex}}%
      \onslide*<2>{\providecommand\step{1}\input{diagrams/backtrace/scanstack.tex}}%
      \onslide*<3>{\providecommand\step{2}\input{diagrams/backtrace/scanstack.tex}}%
      \onslide*<4>{\providecommand\step{3}\input{diagrams/backtrace/scanstack.tex}}%
      \onslide*<5>{\providecommand\step{4}\input{diagrams/backtrace/scanstack.tex}}%
      \onslide*<6>{\providecommand\step{5}\input{diagrams/backtrace/scanstack.tex}}%
    \end{column}
  \end{columns}
\end{frame}

% Duplicate of previous-previous slide
\begin{frame}{Backtraces}{Continuing on \funcname{caml\_stash\_backtrace}}
  \begin{columns}[c]
    \begin{column}{0.5\textwidth}
      \minipage[c][0.75\textheight][s]{\columnwidth}
      \begin{itemize}
          \color{gray}
        \item A C function provided by the runtime (specific to native code)
        \item It scans the stack, between the stack pointer at the raising point up to the next trap, in order to collect the names of the detected function frames
        \item It uses the same technique and information as the Garbage Collector (GC) uses to scan the values live on the stack
      \end{itemize}
      \begin{itemize}
        \item For each function frame detected, store a pointer to the \typename{frame\_descr} in the \localname{domain\_state->backtrace\_buffer} array, effectively constructing the backtrace
      \end{itemize}
      \onslide<2->{
        \begin{itemize}
            \smallskip
          \item Constructed backtrace:
            \funcname{definitely\_raise},
            \onslide<3->{\funcname{probably\_raise}}
        \end{itemize}
      }
      \vfill
      \mintocaml[firstline=9,lastline=13,highlightlines=12]{../src/backtrace/backtrace.ml}
      \endminipage
    \end{column}
    \begin{column}{0.5\textwidth}
      \raggedleft
      \onslide*<1>{\listStashBacktrace[highlightlines={114-115}]{}}
      \onslide*<2>{\providecommand\step{3}\input{diagrams/backtrace/backtrace.tex}}%
      \onslide*<3>{\providecommand\step{4}\input{diagrams/backtrace/backtrace.tex}}%
    \end{column}
  \end{columns}
\end{frame}

\begin{frame}{Backtraces}{Back to \funcname{caml\_raise\_exn}}
  \begin{columns}[c]
    \begin{column}{0.5\textwidth}
      \minipage[c][0.66\textheight][s]{\columnwidth}
      \begin{itemize}\color{gray}
        \item Checks wheteher exceptions backtrace should be recorded, by checking the \funcarg{domain\_state->backtrace\_active} value
        \item Switches to the C stack to be able to execute C code
        \item Calls \funcname{caml\_stash\_backtrace}, to collect the backtrace up to the next trap
      \end{itemize}
      \onslide*<2->{
        \begin{itemize}
          \item<2-> Executes the exception handler in \funcname{probably\_raise} with \funcname{RESTORE\_EXN\_HANDLER\_OCAML}
            \begin{itemize}
              \item Set \regname{RSP} to \localname{domain\_state->exn\_handler} value
              \item Restore \localname{domain\_state->exn\_handler} from trap
              \item Jump to the handler code with \funcname{ret}
            \end{itemize}
        \end{itemize}
      }
      \vfill
      \onslide*<1>{\mintocaml[firstline=9,lastline=13,highlightlines=12]{../src/backtrace/backtrace.ml}}
      \onslide*<2->{\mintocaml[firstline=15,lastline=21,highlightlines={19-21}]{../src/backtrace/backtrace.ml}}
      \endminipage
    \end{column}
    \begin{column}{0.5\textwidth}
      \centering
      \onslide*<1>{
        \mintc[firstline=190,lastline=192]{../ocaml/runtime/amd64.S}
        \mintc[firstline=234,lastline=237]{../ocaml/runtime/amd64.S}
      }
      \onslide*<2>{
        \mintc[firstline=190,lastline=192,highlightlines={191-192}]{../ocaml/runtime/amd64.S}
        \mintc[firstline=234,lastline=237,highlightlines={235-237}]{../ocaml/runtime/amd64.S}
      }
      \onslide*<1>{\listCamlRaiseExn[highlightlines=720]{}}
      \onslide*<2>{\listCamlRaiseExn[highlightlines=722]{}}
      \onslide*<3->{\providecommand\step{5}\input{diagrams/backtrace/backtrace.tex}}
    \end{column}
  \end{columns}
\end{frame}

\begin{frame}{Backtraces}{\funcname{caml\_probably\_raise}'s handler doesn't catch \typename{ExnC}}
  \begin{columns}[c]
    \begin{column}{0.45\textwidth}
      \minipage[c][0.75\textheight][s]{\columnwidth}
      \begin{itemize}
        \item<1-> \funcname{match \dots with}\footnotemark pattern matching can also match exceptions
        \item<1-> The CMM of \funcname{probably\_raise} shows that this \funcname{match \dots with} is implemented with a pattern matching and a \funcname{try \dots with}
        \item<1-> But this one only matches on \typename{ExnA} exception
        \item<2-> Since the pattern matching fails to catch the exception, \funcname{reraise} is called with our current \typename{ExnC} exception
        \item<3-> Disassembly shows that \funcname{reraise} is actually a call to \funcname{caml\_reraise\_exn}, which is also an assembly function provided by the runtime
      \end{itemize}
      \vfill
      \mintocaml[firstline=15,lastline=21,highlightlines={18-21}]{../src/backtrace/backtrace.ml}
      \endminipage
    \end{column}
    \begin{column}{0.55\textwidth}
      \centering
      \onslide*<1>{\mintcmm[highlightlines={10-18}]{../src/backtrace/camlBacktrace\_\_probably\_raise\_351.cmm}}
      \onslide*<2>{\listcmm[highlightlines=18]{../src/backtrace/camlBacktrace\_\_probably\_raise_351.cmm}{CMM of probably\_raise}}
      \onslide*<3>{\mintobjdump[firstline=5,lastline=35,highlightlines=31]{../src/backtrace/camlBacktrace\_\_probably\_raise_351.objdump}}
    \end{column}
  \end{columns}
  \only<1>{\footnotetext[1]{https://blog.janestreet.com/pattern-matching-and-exception-handling-unite/}}
\end{frame}

\newcommand\listCamlReraiseExn[1][]{
  \listasm[firstline=727,lastline=734,#1]{../ocaml/runtime/amd64.S}{runtime/amd64.S:727}}

\begin{frame}{Backtraces}{Introducing \funcname{caml\_reraise\_exn}}
  \begin{columns}[c]
    \begin{column}{0.5\textwidth}
      \minipage[c][0.66\textheight][s]{\columnwidth}
      \begin{itemize}
        \item<1-> The exception have to be reraised to the next handler
        \item<2-> First, checks if the backtrace should be collected with \funcarg{domain\_state->backtrace\_active}
        \only<3>{
        \begin{itemize}
            \item If not, behaves like a \funcname{raise\_notrace} with \funcname{RESTORE\_EXN\_HANDLER\_OCAML}
        \end{itemize}
        }
        \item<4-> Jumps into \funcname{caml\_raise\_exn}
        \item<5-> Calls \funcname{caml\_stash\_backtrace} again, to scan the next part of the stack: from the trap just pop-ed to the next trap
      \end{itemize}
      \vfill
      \mintocaml[firstline=15,lastline=21,highlightlines={18-21}]{../src/backtrace/backtrace.ml}
      \endminipage
    \end{column}
    \begin{column}{0.5\textwidth}
      \raggedleft
      \onslide*<1>{\listCamlReraiseExn{}}
      \onslide*<2>{\listCamlReraiseExn[highlightlines=729]{}}
      \onslide*<3>{
        \mintc[firstline=190,lastline=192,highlightlines={191-192}]{../ocaml/runtime/amd64.S}
        \mintc[firstline=234,lastline=237,highlightlines={235-237}]{../ocaml/runtime/amd64.S}
        \medskip
        \listCamlReraiseExn[highlightlines=731]{}
      }
      \onslide*<4-5>{
        % TODO refactor with \listCamlReraiseExn
        \mintasm[firstline=727,lastline=734,highlightlines=730]{../ocaml/runtime/amd64.S}
        \medskip
      }
      \onslide*<4>{\listCamlRaiseExn[highlightlines=711]{}}
      \onslide*<5>{\listCamlRaiseExn[highlightlines=720]{}}
    \end{column}
  \end{columns}
\end{frame}

\begin{frame}{Backtraces}{Backtrace collection continues}
  \begin{columns}[c]
    \begin{column}{0.55\textwidth}
      \minipage[c][0.75\textheight][s]{\columnwidth}
      \begin{itemize}
        \item<1-> \funcname{caml\_stash\_backtrace} scans the older part of the stack, up to the next trap
        \item<6-> Controls jumps to the nested exception handler in \funcname{maybe\_raise}, which matches only on \typename{ExnB}
        \item<7-> \funcname{caml\_reraise\_exn} is called again, backtrace collection resumes with \funcname{caml\_stash\_backtrace}
      \end{itemize}
      \medskip
      \begin{itemize}
        \item<2-> Constructed backtrace:
        \begin{itemize}
          \item <2-> \funcname{definitely\_raise}
          \item <2-> \funcname{probably\_raise}
          \item <3-> \funcname{probably\_raise} (re-raise)
          \item <4-> \funcname{maybe\_raise}
          \item <5-> \funcname{main}
          \item <8-> \funcname{main} (re-raise)
        \end{itemize}
      \end{itemize}
      \vfill
      \onslide*<1-5>{\mintocaml[firstline=15,lastline=21]{../src/backtrace/backtrace.ml}}
      \onslide*<6>{\mintocaml[firstline=28,lastline=34,highlightlines=31]{../src/backtrace/backtrace.ml}}
      \onslide*<7->{\mintocaml[firstline=28,lastline=34,highlightlines=33]{../src/backtrace/backtrace.ml}}
      \endminipage
    \end{column}
    \begin{column}{0.45\textwidth}
      \raggedleft
      \onslide*<1>{\providecommand\step{6}\input{diagrams/backtrace/backtrace.tex}}%
      \onslide*<2>{\providecommand\step{6}\input{diagrams/backtrace/backtrace.tex}}%
      \onslide*<3>{\providecommand\step{7}\input{diagrams/backtrace/backtrace.tex}}%
      \onslide*<4>{\providecommand\step{8}\input{diagrams/backtrace/backtrace.tex}}%
      \onslide*<5>{\providecommand\step{9}\input{diagrams/backtrace/backtrace.tex}}%
      \onslide*<6>{\providecommand\step{10}\input{diagrams/backtrace/backtrace.tex}}%
      \onslide*<7>{\providecommand\step{11}\input{diagrams/backtrace/backtrace.tex}}%
      \onslide*<8>{\providecommand\step{12}\input{diagrams/backtrace/backtrace.tex}}%
      \onslide*<9>{\providecommand\step{13}\input{diagrams/backtrace/backtrace.tex}}%
    \end{column}
  \end{columns}
\end{frame}

\begin{frame}{Backtraces}{Printing the backtrace}
  \begin{columns}[c]
    \begin{column}{0.45\textwidth}
      \minipage[c][0.66\textheight][s]{\columnwidth}
      \begin{itemize}
        \item<1-> The exception backtrace can now be printed to \funcarg{stdout} with \funcname{Printexc.print_backtrace}
        \item<2-> First the exception backtrace is retrieved into an OCaml array with \funcname{get_raw_backtrace }
        \item<3-> Which is implemented as the \funcname{caml_get_exception_raw_backtrace} C function in the runtime
        \item<4-> And finally printed with \funcname{print_exception_backtrace}
      \end{itemize}
      \vfill
      \mintocaml[firstline=28,lastline=34,highlightlines=33]{../src/backtrace/backtrace.ml}
      \endminipage
    \end{column}
    \begin{column}{0.55\textwidth}
      \onslide*<1,2>{
        \mintocaml[firstline=100,lastline=101]{../ocaml/stdlib/printexc.ml}
      }
      \only<1>{
        \listocaml[firstline=170,lastline=172]{../ocaml/stdlib/printexc.ml}{stdlib/printexc.ml:170}
      }
      \only<2>{
        \listocaml[firstline=170,lastline=172,highlightlines=172]{../ocaml/stdlib/printexc.ml}{stdlib/printexc.ml:170}
      }
      \only<3>{
        \mintc[firstline=154,lastline=156]{../ocaml/runtime/backtrace.c}
        \listc[firstline=171,lastline=192,highlightlines={184-187}]{../ocaml/runtime/backtrace.c}{runtime/backtrace.c:154}
      }
      \only<4>{
        \listocaml[firstline=155,lastline=165]{../ocaml/stdlib/printexc.ml}{stdlib/printexc.ml:55}
      }
    \end{column}
  \end{columns}
\end{frame}


\subsection{The multiple kinds of raise}
\frameSubsection{}{}

\begin{frame}{Backtraces}{When backtraces are not needed}
  \begin{columns}[c]
    \begin{column}{0.45\textwidth}
      \minipage[c][0.66\textheight][s]{\columnwidth}
      \begin{itemize}
        \item<1-> What if the backtrace for an exception is never needed?
        \item<2-> It's possible to raise the exception explicitly with \funcname{raise\_notrace} to make raising as cheap as possible
        \item<3-> An example of use in the \funcname{dynlink} library of the OCaml compiler
      \end{itemize}
      \vfill
      \onslide<3->{
      \listocaml[firstline=205,lastline=209,highlightlines=207]{../ocaml/otherlibs/dynlink/dynlink\_compilerlibs/misc.ml}{otherlibs/dynlink/dynlink\_compilerlibs/misc.ml:205}
      }
      \endminipage
    \end{column}
    \begin{column}{0.55\textwidth}
    \only<4>{
      \mintcmm[highlightlines=3]{../src/notrace/camlNotrace\_\_fun_347.cmm}
      \listcmm{../src/notrace/camlNotrace\_\_all\_somes\_327.cmm}{all\_somes}
    }
    \onslide*<5->{
      \listobjdump[highlightlines={6-9}]{../src/notrace/camlNotrace\_\_fun_347.objdump}{anonymous function from \funcname{all\_somes}}
    }
    \end{column}
  \end{columns}
\end{frame}

\begin{frame}{Backtraces}{Summary table of the multiple kinds of raise}
  \begin{itemize}
    \item When debugging information is enabled and if the backtrace of an exception isn't needed, it's possible to raise with \funcname{raise\_notrace} for performance
    \item When debugging information is \emph{not} enabled, the compiler optimises \funcname{raise} calls to inline assembly (same effect as using \funcname{raise\_notrace})
  \end{itemize}
  \bigskip
  \centering
  \begin{tabular}{| l | c | c | c | c |}
    \hline
    \parbox{10em}{Debug information \\ (\funcarg{-g} flag)}  & \multicolumn{2}{|c|}{Disabled}                                     & \multicolumn{2}{|c|}{Enabled} \\
    \hline
    Raise kind                                               & \funcname{raise} & \funcname{raise\_notrace}                       & \funcname{raise} & \funcname{raise\_notrace} \\
    \hline
    Compilation                                              & \parbox{8em}{inline assembly\\ (optimization)} & inline assembly   & \funcname{caml\_raise\_exn} & inline assembly \\
    \hline
  \end{tabular}
\end{frame}


%
% Takeaway #5
%
\subsection*{Takeaway \#5}
\frameSubsectionTakeaway{}
\againframe<5>{takeaway}
}%
      \onslide*<2>{\providecommand\step{6}%
% Backtraces
%
\subsection{One advantage of raising with debugging information}
\frameSubsection{}{}

\begin{frame}{Backtraces}{Debugging information \& source code}
  \begin{columns}[c]
    \begin{column}{0.5\textwidth}
      \begin{itemize}
        \item Raising an exception is fun and games, but how do we know where the exception originated? $\rightarrow$ With the backtrace associated with the exception!
        \item In order to get a backtrace from the point where an exception was raised, it's necessary to compile with debugging information enabled (\funcarg{-g} flag for the compiler)
        \item Same example as in the `Nested handlers' section
      \end{itemize}
    \end{column}
    \begin{column}{0.5\textwidth}
      \listocaml[firstline=9,lastline=34]{../src/backtrace/backtrace.ml}{backtrace/backtrace.ml}
    \end{column}
  \end{columns}
\end{frame}

\begin{frame}{Backtraces}{CMM and disassembly of \funcname{definitely\_raise}}
  \begin{columns}[c]
    \begin{column}{0.5\textwidth}
      \minipage[c][0.66\textheight][s]{\columnwidth}
      \begin{itemize}
        \item<1-> An effect of compiling with debugging information (\funcarg{-g}): the CMM no longer uses \funcname{raise\_notrace} to raise the exception but \funcname{raise} instead
        \item<2-> Disassembly shows that CMM \funcname{raise} is actually a call to \funcname{caml\_raise\_exn}, a function in assembler provided by the runtime
      \end{itemize}
      \vfill
      \mintocaml[firstline=9,lastline=13,highlightlines=12]{../src/backtrace/backtrace.ml}
      \endminipage
    \end{column}
    \begin{column}{0.5\textwidth}
      \onslide*<1>{
        \mintcmm[highlightlines=5]{../src/backtrace/camlBacktrace\_\_definitely\_raise\_347.cmm}
      }
      \onslide*<2>{
        \mintobjdump[firstline=2,highlightlines=14]{../src/backtrace/camlBacktrace\_\_definitely\_raise\_347.objdump}
      }
    \end{column}
  \end{columns}
\end{frame}

\begin{frame}{Backtraces}{Introducing \funcname{caml\_raise\_exn}}
  \begin{columns}[c]
    \begin{column}{0.5\textwidth}
      \minipage[c][0.66\textheight][s]{\columnwidth}
      \onslide*<1>{
        \begin{itemize}
          \item Raising the exception is performed using \funcname{caml\_raise\_exn} which is
            \begin{itemize}
              \item An assembly function provided by the runtime
              \item Architecture specific
            \end{itemize}
          \item Let's see what it does differently from \funcname{raise\_notrace}\dots
          \item We are about to raise \typename{ExnC} with \funcname{caml\_raise\_exn} from \funcname{definitely\_raise}
        \end{itemize}
      }
      \onslide*<2>{
        \mintobjdump[firstline=2,highlightlines=14]{../src/backtrace/camlBacktrace\_\_definitely\_raise\_347.objdump}
      }
      \vfill
      \mintocaml[firstline=9,lastline=13,highlightlines=12]{../src/backtrace/backtrace.ml}
      \endminipage
    \end{column}
    \begin{column}{0.5\textwidth}
      \centering
      \providecommand\step{1}\input{diagrams/backtrace/backtrace.tex}
    \end{column}
  \end{columns}
\end{frame}

\begin{frame}{Backtraces}{But before that, we need more fields from \typename{caml\_domain\_state}}
  \begin{columns}
    \begin{column}{0.5\textwidth}
      \begin{itemize}
        \item Remember \typename{caml\_domain\_state} the per-domain runtime state?
        \item Some other members are of interest to us now\dots
        \item \localname{backtrace\_active} is used to know if the runtime should collect backtraces
        \item \localname{backtrace\_active} can be set by
          \begin{itemize}
            \item The \funcarg{b} flag in the \funcname{OCAMLRUNPARAM} environment variable
            \item \mintinline{ocaml}{Printexc.record_backtrace : bool -> unit} \footnotemark
          \end{itemize}
      \end{itemize}
    \end{column}
    \begin{column}{0.5\textwidth}
      \begin{listing}
        \mintc[highlightlines={9-11}]{src/ocaml/domain\_state\_backtrace.h}
        \caption{runtime/caml/domain\_state.h}
      \end{listing}
    \end{column}
  \end{columns}
  \footnotetext[1]{https://v2.ocaml.org/api/Printexc.html}
\end{frame}

\newcommand\listCamlRaiseExn[1][]{
  \listasm[firstline=702,lastline=725,#1]{../ocaml/runtime/amd64.S}{runtime/amd64.S:702}}

\begin{frame}{Backtraces}{Walk through \funcname{caml\_raise\_exn}}
  \begin{columns}[T]
    \begin{column}{0.5\textwidth}
      \begin{itemize}
        \item<1-> First checks whether exceptions backtrace should be recorded
        \item<1-> By checking the \funcarg{domain\_state->backtrace\_active} value
        \item<2-> If not, acts as \funcname{raise\_notrace} with \funcname{RESTORE\_EXN\_HANDLER\_OCAML}
          \begin{itemize}
            \item Set \regname{RSP} to \localname{domain\_state->exn\_handler} value
            \item Restore \localname{domain\_state->exn\_handler} from trap
            \item Jump with \funcname{ret} (same effect as using \funcname{pop} and \funcname{jmp})
          \end{itemize}
        \item<3-> Otherwise, switches to the C stack to be able to execute C code
        \item<4-> Calls \funcname{caml\_stash\_backtrace}, a C function provided by the runtime\ignorespaces
          \onslide<6->{, with
            \begin{itemize}
              \item \funcarg{exn}: exception value
              \item \funcarg{pc}: code address of the raising point
              \item \funcarg{sp}: stack address of the raising function
              \item \funcarg{trapsp}: stack address of the next handler
            \end{itemize}
          }
      \end{itemize}
      \bigskip
      \mintocaml[firstline=9,lastline=13,highlightlines=12]{../src/backtrace/backtrace.ml}
    \end{column}
    \begin{column}{0.5\textwidth}
      \raggedleft
      \onslide*<1,3-4>{
        \mintc[firstline=190,lastline=192]{../ocaml/runtime/amd64.S}
        \mintc[firstline=234,lastline=237]{../ocaml/runtime/amd64.S}
      }
      \onslide*<2>{
        \mintc[firstline=190,lastline=192,highlightlines={191-192}]{../ocaml/runtime/amd64.S}
        \mintc[firstline=234,lastline=237,highlightlines={235-237}]{../ocaml/runtime/amd64.S}
      }
      \onslide*<1>{\listCamlRaiseExn[highlightlines=705]{}}%
      \onslide*<2>{\listCamlRaiseExn[highlightlines={707-708}]{}}%
      \onslide*<3>{\listCamlRaiseExn[highlightlines={712-714}]{}}%
      \onslide*<4>{\listCamlRaiseExn[highlightlines={715-720}]{}}%
      \onslide*<5>{\providecommand\step{1}\input{diagrams/backtrace/backtrace.tex}}%
      \onslide*<6>{\providecommand\step{2}\input{diagrams/backtrace/backtrace.tex}}%
    \end{column}
  \end{columns}
\end{frame}

\newcommand\listStashBacktrace[1][]{
  \listc[firstline=91,lastline=120,#1]{../ocaml/runtime/backtrace\_nat.c}{runtime/backtrace\_nat.c:91}}

\begin{frame}{Backtraces}{Introducing \funcname{caml\_stash\_backtrace}}
  \begin{columns}[c]
    \begin{column}{0.5\textwidth}
      \minipage[c][0.66\textheight][s]{\columnwidth}
      \begin{itemize}
        \item<1-> A C function provided by the runtime (specific to native code)
        \item<2-> It scans the stack, between the stack pointer at the raising point up to the next trap, in order to collect the names of the detected function frames
          \onslide*<2>{
            \begin{itemize}
              \item And more information, like line number, column number...
            \end{itemize}
          }
        \item<3-> It uses the same technique and information as the Garbage Collector (GC) uses to scan the values live on the stack
          \onslide*<4>{
            \begin{itemize}
              \item \typename{frame\_descr} are at the core of stack unwinding
            \end{itemize}
          }
      \end{itemize}
      \vfill
      \mintocaml[firstline=9,lastline=13,highlightlines=12]{../src/backtrace/backtrace.ml}
      \endminipage
    \end{column}
    \begin{column}{0.5\textwidth}
      \onslide*<1>{\listStashBacktrace[highlightlines=91]{}}
      \onslide*<2>{\listStashBacktrace[highlightlines={91,117-118}]{}}
      \onslide*<3->{\listStashBacktrace[highlightlines={94,106,109-110}]{}}
    \end{column}
  \end{columns}
\end{frame}

\newcommand\listFrameDescr[1][]{
  \listocaml[firstline=111,lastline=124,#1]{../ocaml/asmcomp/emitaux.ml}{asmcomp/emitaux.ml:111}}
\newcommand\listDebugInfo[1][]{
  \listocaml[firstline=38,lastline=49,#1]{../ocaml/lambda/debuginfo.mli}{lambda/debuginfo.mli:38}}

\begin{frame}{Backtraces}{\typename{frame\_descr} and \typename{frame\_debuginfo} in a nutshell}
  \begin{columns}[c]
    \begin{column}{0.5\textwidth}
      \begin{itemize}
        \item<1-> For every return address in an OCaml program, there is a associated \typename{frame\_descr}
        \item<2-> A \typename{frame\_descr} specifies
        \begin{itemize}
          \item<2-> The size of the function stack frame
          \item<3-> Where live values are on the stack (so that the GC can mark them as live and prevent from being swept)
        \end{itemize}
        \item<4-> When compiled with debug information, \typename{frame\_descr} will be linked to a \typename{frame\_debuginfo}
        \begin{itemize}
          \item<5-> Contains information about the position in the source code (file name, line and column number)
        \end{itemize}
      \end{itemize}
    \end{column}
    \begin{column}{0.5\textwidth}
        \onslide*<1>{\listFrameDescr[highlightlines={118-122}]{}}
        \onslide*<2>{\listFrameDescr[highlightlines={120}]{}}
        \onslide*<3>{\listFrameDescr[highlightlines={121}]{}}
        \onslide*<4->{\listFrameDescr[highlightlines={122}]{}}
        \medskip
        \onslide*<1-3>{\listDebugInfo{}}
        \onslide*<4>{\listDebugInfo[highlightlines={38,49}]{}}
        \onslide*<5>{\listDebugInfo[highlightlines={39-46}]{}}
    \end{column}
  \end{columns}
\end{frame}

\begin{frame}{Backtraces}{Scanning the stack with \typename{frame\_descr}}
  \begin{columns}[c]
    \begin{column}{0.5\textwidth}
      \begin{itemize}
        \item Given the return address of the code that raises the exception by calling \funcname{caml\_raise\_exn} (\funcarg{pc} argument of \funcname{caml\_stash\_backtrace})
        \item And the position of the stack pointer (\funcarg{sp} argument of \funcname{caml\_stash\_backtrace})
        \item The frame size given by \typename{frame\_descr}.\funcarg{frame\_size}, allows to find the end of the previous frame
        \item And the return address of the caller of this function
        \bigskip
        \onslide<3->{
        \item Note that traps (here the reddish \funcname{ExnA}, \funcname{ExnB} and \funcname{ExnC} blocks) belong to the frame that installed them, and so the frame size changes when traps are removed
        }
      \end{itemize}
    \end{column}
    \begin{column}{0.5\textwidth}
      \raggedleft
      \onslide*<1>{\providecommand\step{2}\input{diagrams/backtrace/backtrace.tex}}%
      \onslide*<2>{\providecommand\step{1}\input{diagrams/backtrace/scanstack.tex}}%
      \onslide*<3>{\providecommand\step{2}\input{diagrams/backtrace/scanstack.tex}}%
      \onslide*<4>{\providecommand\step{3}\input{diagrams/backtrace/scanstack.tex}}%
      \onslide*<5>{\providecommand\step{4}\input{diagrams/backtrace/scanstack.tex}}%
      \onslide*<6>{\providecommand\step{5}\input{diagrams/backtrace/scanstack.tex}}%
    \end{column}
  \end{columns}
\end{frame}

% Duplicate of previous-previous slide
\begin{frame}{Backtraces}{Continuing on \funcname{caml\_stash\_backtrace}}
  \begin{columns}[c]
    \begin{column}{0.5\textwidth}
      \minipage[c][0.75\textheight][s]{\columnwidth}
      \begin{itemize}
          \color{gray}
        \item A C function provided by the runtime (specific to native code)
        \item It scans the stack, between the stack pointer at the raising point up to the next trap, in order to collect the names of the detected function frames
        \item It uses the same technique and information as the Garbage Collector (GC) uses to scan the values live on the stack
      \end{itemize}
      \begin{itemize}
        \item For each function frame detected, store a pointer to the \typename{frame\_descr} in the \localname{domain\_state->backtrace\_buffer} array, effectively constructing the backtrace
      \end{itemize}
      \onslide<2->{
        \begin{itemize}
            \smallskip
          \item Constructed backtrace:
            \funcname{definitely\_raise},
            \onslide<3->{\funcname{probably\_raise}}
        \end{itemize}
      }
      \vfill
      \mintocaml[firstline=9,lastline=13,highlightlines=12]{../src/backtrace/backtrace.ml}
      \endminipage
    \end{column}
    \begin{column}{0.5\textwidth}
      \raggedleft
      \onslide*<1>{\listStashBacktrace[highlightlines={114-115}]{}}
      \onslide*<2>{\providecommand\step{3}\input{diagrams/backtrace/backtrace.tex}}%
      \onslide*<3>{\providecommand\step{4}\input{diagrams/backtrace/backtrace.tex}}%
    \end{column}
  \end{columns}
\end{frame}

\begin{frame}{Backtraces}{Back to \funcname{caml\_raise\_exn}}
  \begin{columns}[c]
    \begin{column}{0.5\textwidth}
      \minipage[c][0.66\textheight][s]{\columnwidth}
      \begin{itemize}\color{gray}
        \item Checks wheteher exceptions backtrace should be recorded, by checking the \funcarg{domain\_state->backtrace\_active} value
        \item Switches to the C stack to be able to execute C code
        \item Calls \funcname{caml\_stash\_backtrace}, to collect the backtrace up to the next trap
      \end{itemize}
      \onslide*<2->{
        \begin{itemize}
          \item<2-> Executes the exception handler in \funcname{probably\_raise} with \funcname{RESTORE\_EXN\_HANDLER\_OCAML}
            \begin{itemize}
              \item Set \regname{RSP} to \localname{domain\_state->exn\_handler} value
              \item Restore \localname{domain\_state->exn\_handler} from trap
              \item Jump to the handler code with \funcname{ret}
            \end{itemize}
        \end{itemize}
      }
      \vfill
      \onslide*<1>{\mintocaml[firstline=9,lastline=13,highlightlines=12]{../src/backtrace/backtrace.ml}}
      \onslide*<2->{\mintocaml[firstline=15,lastline=21,highlightlines={19-21}]{../src/backtrace/backtrace.ml}}
      \endminipage
    \end{column}
    \begin{column}{0.5\textwidth}
      \centering
      \onslide*<1>{
        \mintc[firstline=190,lastline=192]{../ocaml/runtime/amd64.S}
        \mintc[firstline=234,lastline=237]{../ocaml/runtime/amd64.S}
      }
      \onslide*<2>{
        \mintc[firstline=190,lastline=192,highlightlines={191-192}]{../ocaml/runtime/amd64.S}
        \mintc[firstline=234,lastline=237,highlightlines={235-237}]{../ocaml/runtime/amd64.S}
      }
      \onslide*<1>{\listCamlRaiseExn[highlightlines=720]{}}
      \onslide*<2>{\listCamlRaiseExn[highlightlines=722]{}}
      \onslide*<3->{\providecommand\step{5}\input{diagrams/backtrace/backtrace.tex}}
    \end{column}
  \end{columns}
\end{frame}

\begin{frame}{Backtraces}{\funcname{caml\_probably\_raise}'s handler doesn't catch \typename{ExnC}}
  \begin{columns}[c]
    \begin{column}{0.45\textwidth}
      \minipage[c][0.75\textheight][s]{\columnwidth}
      \begin{itemize}
        \item<1-> \funcname{match \dots with}\footnotemark pattern matching can also match exceptions
        \item<1-> The CMM of \funcname{probably\_raise} shows that this \funcname{match \dots with} is implemented with a pattern matching and a \funcname{try \dots with}
        \item<1-> But this one only matches on \typename{ExnA} exception
        \item<2-> Since the pattern matching fails to catch the exception, \funcname{reraise} is called with our current \typename{ExnC} exception
        \item<3-> Disassembly shows that \funcname{reraise} is actually a call to \funcname{caml\_reraise\_exn}, which is also an assembly function provided by the runtime
      \end{itemize}
      \vfill
      \mintocaml[firstline=15,lastline=21,highlightlines={18-21}]{../src/backtrace/backtrace.ml}
      \endminipage
    \end{column}
    \begin{column}{0.55\textwidth}
      \centering
      \onslide*<1>{\mintcmm[highlightlines={10-18}]{../src/backtrace/camlBacktrace\_\_probably\_raise\_351.cmm}}
      \onslide*<2>{\listcmm[highlightlines=18]{../src/backtrace/camlBacktrace\_\_probably\_raise_351.cmm}{CMM of probably\_raise}}
      \onslide*<3>{\mintobjdump[firstline=5,lastline=35,highlightlines=31]{../src/backtrace/camlBacktrace\_\_probably\_raise_351.objdump}}
    \end{column}
  \end{columns}
  \only<1>{\footnotetext[1]{https://blog.janestreet.com/pattern-matching-and-exception-handling-unite/}}
\end{frame}

\newcommand\listCamlReraiseExn[1][]{
  \listasm[firstline=727,lastline=734,#1]{../ocaml/runtime/amd64.S}{runtime/amd64.S:727}}

\begin{frame}{Backtraces}{Introducing \funcname{caml\_reraise\_exn}}
  \begin{columns}[c]
    \begin{column}{0.5\textwidth}
      \minipage[c][0.66\textheight][s]{\columnwidth}
      \begin{itemize}
        \item<1-> The exception have to be reraised to the next handler
        \item<2-> First, checks if the backtrace should be collected with \funcarg{domain\_state->backtrace\_active}
        \only<3>{
        \begin{itemize}
            \item If not, behaves like a \funcname{raise\_notrace} with \funcname{RESTORE\_EXN\_HANDLER\_OCAML}
        \end{itemize}
        }
        \item<4-> Jumps into \funcname{caml\_raise\_exn}
        \item<5-> Calls \funcname{caml\_stash\_backtrace} again, to scan the next part of the stack: from the trap just pop-ed to the next trap
      \end{itemize}
      \vfill
      \mintocaml[firstline=15,lastline=21,highlightlines={18-21}]{../src/backtrace/backtrace.ml}
      \endminipage
    \end{column}
    \begin{column}{0.5\textwidth}
      \raggedleft
      \onslide*<1>{\listCamlReraiseExn{}}
      \onslide*<2>{\listCamlReraiseExn[highlightlines=729]{}}
      \onslide*<3>{
        \mintc[firstline=190,lastline=192,highlightlines={191-192}]{../ocaml/runtime/amd64.S}
        \mintc[firstline=234,lastline=237,highlightlines={235-237}]{../ocaml/runtime/amd64.S}
        \medskip
        \listCamlReraiseExn[highlightlines=731]{}
      }
      \onslide*<4-5>{
        % TODO refactor with \listCamlReraiseExn
        \mintasm[firstline=727,lastline=734,highlightlines=730]{../ocaml/runtime/amd64.S}
        \medskip
      }
      \onslide*<4>{\listCamlRaiseExn[highlightlines=711]{}}
      \onslide*<5>{\listCamlRaiseExn[highlightlines=720]{}}
    \end{column}
  \end{columns}
\end{frame}

\begin{frame}{Backtraces}{Backtrace collection continues}
  \begin{columns}[c]
    \begin{column}{0.55\textwidth}
      \minipage[c][0.75\textheight][s]{\columnwidth}
      \begin{itemize}
        \item<1-> \funcname{caml\_stash\_backtrace} scans the older part of the stack, up to the next trap
        \item<6-> Controls jumps to the nested exception handler in \funcname{maybe\_raise}, which matches only on \typename{ExnB}
        \item<7-> \funcname{caml\_reraise\_exn} is called again, backtrace collection resumes with \funcname{caml\_stash\_backtrace}
      \end{itemize}
      \medskip
      \begin{itemize}
        \item<2-> Constructed backtrace:
        \begin{itemize}
          \item <2-> \funcname{definitely\_raise}
          \item <2-> \funcname{probably\_raise}
          \item <3-> \funcname{probably\_raise} (re-raise)
          \item <4-> \funcname{maybe\_raise}
          \item <5-> \funcname{main}
          \item <8-> \funcname{main} (re-raise)
        \end{itemize}
      \end{itemize}
      \vfill
      \onslide*<1-5>{\mintocaml[firstline=15,lastline=21]{../src/backtrace/backtrace.ml}}
      \onslide*<6>{\mintocaml[firstline=28,lastline=34,highlightlines=31]{../src/backtrace/backtrace.ml}}
      \onslide*<7->{\mintocaml[firstline=28,lastline=34,highlightlines=33]{../src/backtrace/backtrace.ml}}
      \endminipage
    \end{column}
    \begin{column}{0.45\textwidth}
      \raggedleft
      \onslide*<1>{\providecommand\step{6}\input{diagrams/backtrace/backtrace.tex}}%
      \onslide*<2>{\providecommand\step{6}\input{diagrams/backtrace/backtrace.tex}}%
      \onslide*<3>{\providecommand\step{7}\input{diagrams/backtrace/backtrace.tex}}%
      \onslide*<4>{\providecommand\step{8}\input{diagrams/backtrace/backtrace.tex}}%
      \onslide*<5>{\providecommand\step{9}\input{diagrams/backtrace/backtrace.tex}}%
      \onslide*<6>{\providecommand\step{10}\input{diagrams/backtrace/backtrace.tex}}%
      \onslide*<7>{\providecommand\step{11}\input{diagrams/backtrace/backtrace.tex}}%
      \onslide*<8>{\providecommand\step{12}\input{diagrams/backtrace/backtrace.tex}}%
      \onslide*<9>{\providecommand\step{13}\input{diagrams/backtrace/backtrace.tex}}%
    \end{column}
  \end{columns}
\end{frame}

\begin{frame}{Backtraces}{Printing the backtrace}
  \begin{columns}[c]
    \begin{column}{0.45\textwidth}
      \minipage[c][0.66\textheight][s]{\columnwidth}
      \begin{itemize}
        \item<1-> The exception backtrace can now be printed to \funcarg{stdout} with \funcname{Printexc.print_backtrace}
        \item<2-> First the exception backtrace is retrieved into an OCaml array with \funcname{get_raw_backtrace }
        \item<3-> Which is implemented as the \funcname{caml_get_exception_raw_backtrace} C function in the runtime
        \item<4-> And finally printed with \funcname{print_exception_backtrace}
      \end{itemize}
      \vfill
      \mintocaml[firstline=28,lastline=34,highlightlines=33]{../src/backtrace/backtrace.ml}
      \endminipage
    \end{column}
    \begin{column}{0.55\textwidth}
      \onslide*<1,2>{
        \mintocaml[firstline=100,lastline=101]{../ocaml/stdlib/printexc.ml}
      }
      \only<1>{
        \listocaml[firstline=170,lastline=172]{../ocaml/stdlib/printexc.ml}{stdlib/printexc.ml:170}
      }
      \only<2>{
        \listocaml[firstline=170,lastline=172,highlightlines=172]{../ocaml/stdlib/printexc.ml}{stdlib/printexc.ml:170}
      }
      \only<3>{
        \mintc[firstline=154,lastline=156]{../ocaml/runtime/backtrace.c}
        \listc[firstline=171,lastline=192,highlightlines={184-187}]{../ocaml/runtime/backtrace.c}{runtime/backtrace.c:154}
      }
      \only<4>{
        \listocaml[firstline=155,lastline=165]{../ocaml/stdlib/printexc.ml}{stdlib/printexc.ml:55}
      }
    \end{column}
  \end{columns}
\end{frame}


\subsection{The multiple kinds of raise}
\frameSubsection{}{}

\begin{frame}{Backtraces}{When backtraces are not needed}
  \begin{columns}[c]
    \begin{column}{0.45\textwidth}
      \minipage[c][0.66\textheight][s]{\columnwidth}
      \begin{itemize}
        \item<1-> What if the backtrace for an exception is never needed?
        \item<2-> It's possible to raise the exception explicitly with \funcname{raise\_notrace} to make raising as cheap as possible
        \item<3-> An example of use in the \funcname{dynlink} library of the OCaml compiler
      \end{itemize}
      \vfill
      \onslide<3->{
      \listocaml[firstline=205,lastline=209,highlightlines=207]{../ocaml/otherlibs/dynlink/dynlink\_compilerlibs/misc.ml}{otherlibs/dynlink/dynlink\_compilerlibs/misc.ml:205}
      }
      \endminipage
    \end{column}
    \begin{column}{0.55\textwidth}
    \only<4>{
      \mintcmm[highlightlines=3]{../src/notrace/camlNotrace\_\_fun_347.cmm}
      \listcmm{../src/notrace/camlNotrace\_\_all\_somes\_327.cmm}{all\_somes}
    }
    \onslide*<5->{
      \listobjdump[highlightlines={6-9}]{../src/notrace/camlNotrace\_\_fun_347.objdump}{anonymous function from \funcname{all\_somes}}
    }
    \end{column}
  \end{columns}
\end{frame}

\begin{frame}{Backtraces}{Summary table of the multiple kinds of raise}
  \begin{itemize}
    \item When debugging information is enabled and if the backtrace of an exception isn't needed, it's possible to raise with \funcname{raise\_notrace} for performance
    \item When debugging information is \emph{not} enabled, the compiler optimises \funcname{raise} calls to inline assembly (same effect as using \funcname{raise\_notrace})
  \end{itemize}
  \bigskip
  \centering
  \begin{tabular}{| l | c | c | c | c |}
    \hline
    \parbox{10em}{Debug information \\ (\funcarg{-g} flag)}  & \multicolumn{2}{|c|}{Disabled}                                     & \multicolumn{2}{|c|}{Enabled} \\
    \hline
    Raise kind                                               & \funcname{raise} & \funcname{raise\_notrace}                       & \funcname{raise} & \funcname{raise\_notrace} \\
    \hline
    Compilation                                              & \parbox{8em}{inline assembly\\ (optimization)} & inline assembly   & \funcname{caml\_raise\_exn} & inline assembly \\
    \hline
  \end{tabular}
\end{frame}


%
% Takeaway #5
%
\subsection*{Takeaway \#5}
\frameSubsectionTakeaway{}
\againframe<5>{takeaway}
}%
      \onslide*<3>{\providecommand\step{7}%
% Backtraces
%
\subsection{One advantage of raising with debugging information}
\frameSubsection{}{}

\begin{frame}{Backtraces}{Debugging information \& source code}
  \begin{columns}[c]
    \begin{column}{0.5\textwidth}
      \begin{itemize}
        \item Raising an exception is fun and games, but how do we know where the exception originated? $\rightarrow$ With the backtrace associated with the exception!
        \item In order to get a backtrace from the point where an exception was raised, it's necessary to compile with debugging information enabled (\funcarg{-g} flag for the compiler)
        \item Same example as in the `Nested handlers' section
      \end{itemize}
    \end{column}
    \begin{column}{0.5\textwidth}
      \listocaml[firstline=9,lastline=34]{../src/backtrace/backtrace.ml}{backtrace/backtrace.ml}
    \end{column}
  \end{columns}
\end{frame}

\begin{frame}{Backtraces}{CMM and disassembly of \funcname{definitely\_raise}}
  \begin{columns}[c]
    \begin{column}{0.5\textwidth}
      \minipage[c][0.66\textheight][s]{\columnwidth}
      \begin{itemize}
        \item<1-> An effect of compiling with debugging information (\funcarg{-g}): the CMM no longer uses \funcname{raise\_notrace} to raise the exception but \funcname{raise} instead
        \item<2-> Disassembly shows that CMM \funcname{raise} is actually a call to \funcname{caml\_raise\_exn}, a function in assembler provided by the runtime
      \end{itemize}
      \vfill
      \mintocaml[firstline=9,lastline=13,highlightlines=12]{../src/backtrace/backtrace.ml}
      \endminipage
    \end{column}
    \begin{column}{0.5\textwidth}
      \onslide*<1>{
        \mintcmm[highlightlines=5]{../src/backtrace/camlBacktrace\_\_definitely\_raise\_347.cmm}
      }
      \onslide*<2>{
        \mintobjdump[firstline=2,highlightlines=14]{../src/backtrace/camlBacktrace\_\_definitely\_raise\_347.objdump}
      }
    \end{column}
  \end{columns}
\end{frame}

\begin{frame}{Backtraces}{Introducing \funcname{caml\_raise\_exn}}
  \begin{columns}[c]
    \begin{column}{0.5\textwidth}
      \minipage[c][0.66\textheight][s]{\columnwidth}
      \onslide*<1>{
        \begin{itemize}
          \item Raising the exception is performed using \funcname{caml\_raise\_exn} which is
            \begin{itemize}
              \item An assembly function provided by the runtime
              \item Architecture specific
            \end{itemize}
          \item Let's see what it does differently from \funcname{raise\_notrace}\dots
          \item We are about to raise \typename{ExnC} with \funcname{caml\_raise\_exn} from \funcname{definitely\_raise}
        \end{itemize}
      }
      \onslide*<2>{
        \mintobjdump[firstline=2,highlightlines=14]{../src/backtrace/camlBacktrace\_\_definitely\_raise\_347.objdump}
      }
      \vfill
      \mintocaml[firstline=9,lastline=13,highlightlines=12]{../src/backtrace/backtrace.ml}
      \endminipage
    \end{column}
    \begin{column}{0.5\textwidth}
      \centering
      \providecommand\step{1}\input{diagrams/backtrace/backtrace.tex}
    \end{column}
  \end{columns}
\end{frame}

\begin{frame}{Backtraces}{But before that, we need more fields from \typename{caml\_domain\_state}}
  \begin{columns}
    \begin{column}{0.5\textwidth}
      \begin{itemize}
        \item Remember \typename{caml\_domain\_state} the per-domain runtime state?
        \item Some other members are of interest to us now\dots
        \item \localname{backtrace\_active} is used to know if the runtime should collect backtraces
        \item \localname{backtrace\_active} can be set by
          \begin{itemize}
            \item The \funcarg{b} flag in the \funcname{OCAMLRUNPARAM} environment variable
            \item \mintinline{ocaml}{Printexc.record_backtrace : bool -> unit} \footnotemark
          \end{itemize}
      \end{itemize}
    \end{column}
    \begin{column}{0.5\textwidth}
      \begin{listing}
        \mintc[highlightlines={9-11}]{src/ocaml/domain\_state\_backtrace.h}
        \caption{runtime/caml/domain\_state.h}
      \end{listing}
    \end{column}
  \end{columns}
  \footnotetext[1]{https://v2.ocaml.org/api/Printexc.html}
\end{frame}

\newcommand\listCamlRaiseExn[1][]{
  \listasm[firstline=702,lastline=725,#1]{../ocaml/runtime/amd64.S}{runtime/amd64.S:702}}

\begin{frame}{Backtraces}{Walk through \funcname{caml\_raise\_exn}}
  \begin{columns}[T]
    \begin{column}{0.5\textwidth}
      \begin{itemize}
        \item<1-> First checks whether exceptions backtrace should be recorded
        \item<1-> By checking the \funcarg{domain\_state->backtrace\_active} value
        \item<2-> If not, acts as \funcname{raise\_notrace} with \funcname{RESTORE\_EXN\_HANDLER\_OCAML}
          \begin{itemize}
            \item Set \regname{RSP} to \localname{domain\_state->exn\_handler} value
            \item Restore \localname{domain\_state->exn\_handler} from trap
            \item Jump with \funcname{ret} (same effect as using \funcname{pop} and \funcname{jmp})
          \end{itemize}
        \item<3-> Otherwise, switches to the C stack to be able to execute C code
        \item<4-> Calls \funcname{caml\_stash\_backtrace}, a C function provided by the runtime\ignorespaces
          \onslide<6->{, with
            \begin{itemize}
              \item \funcarg{exn}: exception value
              \item \funcarg{pc}: code address of the raising point
              \item \funcarg{sp}: stack address of the raising function
              \item \funcarg{trapsp}: stack address of the next handler
            \end{itemize}
          }
      \end{itemize}
      \bigskip
      \mintocaml[firstline=9,lastline=13,highlightlines=12]{../src/backtrace/backtrace.ml}
    \end{column}
    \begin{column}{0.5\textwidth}
      \raggedleft
      \onslide*<1,3-4>{
        \mintc[firstline=190,lastline=192]{../ocaml/runtime/amd64.S}
        \mintc[firstline=234,lastline=237]{../ocaml/runtime/amd64.S}
      }
      \onslide*<2>{
        \mintc[firstline=190,lastline=192,highlightlines={191-192}]{../ocaml/runtime/amd64.S}
        \mintc[firstline=234,lastline=237,highlightlines={235-237}]{../ocaml/runtime/amd64.S}
      }
      \onslide*<1>{\listCamlRaiseExn[highlightlines=705]{}}%
      \onslide*<2>{\listCamlRaiseExn[highlightlines={707-708}]{}}%
      \onslide*<3>{\listCamlRaiseExn[highlightlines={712-714}]{}}%
      \onslide*<4>{\listCamlRaiseExn[highlightlines={715-720}]{}}%
      \onslide*<5>{\providecommand\step{1}\input{diagrams/backtrace/backtrace.tex}}%
      \onslide*<6>{\providecommand\step{2}\input{diagrams/backtrace/backtrace.tex}}%
    \end{column}
  \end{columns}
\end{frame}

\newcommand\listStashBacktrace[1][]{
  \listc[firstline=91,lastline=120,#1]{../ocaml/runtime/backtrace\_nat.c}{runtime/backtrace\_nat.c:91}}

\begin{frame}{Backtraces}{Introducing \funcname{caml\_stash\_backtrace}}
  \begin{columns}[c]
    \begin{column}{0.5\textwidth}
      \minipage[c][0.66\textheight][s]{\columnwidth}
      \begin{itemize}
        \item<1-> A C function provided by the runtime (specific to native code)
        \item<2-> It scans the stack, between the stack pointer at the raising point up to the next trap, in order to collect the names of the detected function frames
          \onslide*<2>{
            \begin{itemize}
              \item And more information, like line number, column number...
            \end{itemize}
          }
        \item<3-> It uses the same technique and information as the Garbage Collector (GC) uses to scan the values live on the stack
          \onslide*<4>{
            \begin{itemize}
              \item \typename{frame\_descr} are at the core of stack unwinding
            \end{itemize}
          }
      \end{itemize}
      \vfill
      \mintocaml[firstline=9,lastline=13,highlightlines=12]{../src/backtrace/backtrace.ml}
      \endminipage
    \end{column}
    \begin{column}{0.5\textwidth}
      \onslide*<1>{\listStashBacktrace[highlightlines=91]{}}
      \onslide*<2>{\listStashBacktrace[highlightlines={91,117-118}]{}}
      \onslide*<3->{\listStashBacktrace[highlightlines={94,106,109-110}]{}}
    \end{column}
  \end{columns}
\end{frame}

\newcommand\listFrameDescr[1][]{
  \listocaml[firstline=111,lastline=124,#1]{../ocaml/asmcomp/emitaux.ml}{asmcomp/emitaux.ml:111}}
\newcommand\listDebugInfo[1][]{
  \listocaml[firstline=38,lastline=49,#1]{../ocaml/lambda/debuginfo.mli}{lambda/debuginfo.mli:38}}

\begin{frame}{Backtraces}{\typename{frame\_descr} and \typename{frame\_debuginfo} in a nutshell}
  \begin{columns}[c]
    \begin{column}{0.5\textwidth}
      \begin{itemize}
        \item<1-> For every return address in an OCaml program, there is a associated \typename{frame\_descr}
        \item<2-> A \typename{frame\_descr} specifies
        \begin{itemize}
          \item<2-> The size of the function stack frame
          \item<3-> Where live values are on the stack (so that the GC can mark them as live and prevent from being swept)
        \end{itemize}
        \item<4-> When compiled with debug information, \typename{frame\_descr} will be linked to a \typename{frame\_debuginfo}
        \begin{itemize}
          \item<5-> Contains information about the position in the source code (file name, line and column number)
        \end{itemize}
      \end{itemize}
    \end{column}
    \begin{column}{0.5\textwidth}
        \onslide*<1>{\listFrameDescr[highlightlines={118-122}]{}}
        \onslide*<2>{\listFrameDescr[highlightlines={120}]{}}
        \onslide*<3>{\listFrameDescr[highlightlines={121}]{}}
        \onslide*<4->{\listFrameDescr[highlightlines={122}]{}}
        \medskip
        \onslide*<1-3>{\listDebugInfo{}}
        \onslide*<4>{\listDebugInfo[highlightlines={38,49}]{}}
        \onslide*<5>{\listDebugInfo[highlightlines={39-46}]{}}
    \end{column}
  \end{columns}
\end{frame}

\begin{frame}{Backtraces}{Scanning the stack with \typename{frame\_descr}}
  \begin{columns}[c]
    \begin{column}{0.5\textwidth}
      \begin{itemize}
        \item Given the return address of the code that raises the exception by calling \funcname{caml\_raise\_exn} (\funcarg{pc} argument of \funcname{caml\_stash\_backtrace})
        \item And the position of the stack pointer (\funcarg{sp} argument of \funcname{caml\_stash\_backtrace})
        \item The frame size given by \typename{frame\_descr}.\funcarg{frame\_size}, allows to find the end of the previous frame
        \item And the return address of the caller of this function
        \bigskip
        \onslide<3->{
        \item Note that traps (here the reddish \funcname{ExnA}, \funcname{ExnB} and \funcname{ExnC} blocks) belong to the frame that installed them, and so the frame size changes when traps are removed
        }
      \end{itemize}
    \end{column}
    \begin{column}{0.5\textwidth}
      \raggedleft
      \onslide*<1>{\providecommand\step{2}\input{diagrams/backtrace/backtrace.tex}}%
      \onslide*<2>{\providecommand\step{1}\input{diagrams/backtrace/scanstack.tex}}%
      \onslide*<3>{\providecommand\step{2}\input{diagrams/backtrace/scanstack.tex}}%
      \onslide*<4>{\providecommand\step{3}\input{diagrams/backtrace/scanstack.tex}}%
      \onslide*<5>{\providecommand\step{4}\input{diagrams/backtrace/scanstack.tex}}%
      \onslide*<6>{\providecommand\step{5}\input{diagrams/backtrace/scanstack.tex}}%
    \end{column}
  \end{columns}
\end{frame}

% Duplicate of previous-previous slide
\begin{frame}{Backtraces}{Continuing on \funcname{caml\_stash\_backtrace}}
  \begin{columns}[c]
    \begin{column}{0.5\textwidth}
      \minipage[c][0.75\textheight][s]{\columnwidth}
      \begin{itemize}
          \color{gray}
        \item A C function provided by the runtime (specific to native code)
        \item It scans the stack, between the stack pointer at the raising point up to the next trap, in order to collect the names of the detected function frames
        \item It uses the same technique and information as the Garbage Collector (GC) uses to scan the values live on the stack
      \end{itemize}
      \begin{itemize}
        \item For each function frame detected, store a pointer to the \typename{frame\_descr} in the \localname{domain\_state->backtrace\_buffer} array, effectively constructing the backtrace
      \end{itemize}
      \onslide<2->{
        \begin{itemize}
            \smallskip
          \item Constructed backtrace:
            \funcname{definitely\_raise},
            \onslide<3->{\funcname{probably\_raise}}
        \end{itemize}
      }
      \vfill
      \mintocaml[firstline=9,lastline=13,highlightlines=12]{../src/backtrace/backtrace.ml}
      \endminipage
    \end{column}
    \begin{column}{0.5\textwidth}
      \raggedleft
      \onslide*<1>{\listStashBacktrace[highlightlines={114-115}]{}}
      \onslide*<2>{\providecommand\step{3}\input{diagrams/backtrace/backtrace.tex}}%
      \onslide*<3>{\providecommand\step{4}\input{diagrams/backtrace/backtrace.tex}}%
    \end{column}
  \end{columns}
\end{frame}

\begin{frame}{Backtraces}{Back to \funcname{caml\_raise\_exn}}
  \begin{columns}[c]
    \begin{column}{0.5\textwidth}
      \minipage[c][0.66\textheight][s]{\columnwidth}
      \begin{itemize}\color{gray}
        \item Checks wheteher exceptions backtrace should be recorded, by checking the \funcarg{domain\_state->backtrace\_active} value
        \item Switches to the C stack to be able to execute C code
        \item Calls \funcname{caml\_stash\_backtrace}, to collect the backtrace up to the next trap
      \end{itemize}
      \onslide*<2->{
        \begin{itemize}
          \item<2-> Executes the exception handler in \funcname{probably\_raise} with \funcname{RESTORE\_EXN\_HANDLER\_OCAML}
            \begin{itemize}
              \item Set \regname{RSP} to \localname{domain\_state->exn\_handler} value
              \item Restore \localname{domain\_state->exn\_handler} from trap
              \item Jump to the handler code with \funcname{ret}
            \end{itemize}
        \end{itemize}
      }
      \vfill
      \onslide*<1>{\mintocaml[firstline=9,lastline=13,highlightlines=12]{../src/backtrace/backtrace.ml}}
      \onslide*<2->{\mintocaml[firstline=15,lastline=21,highlightlines={19-21}]{../src/backtrace/backtrace.ml}}
      \endminipage
    \end{column}
    \begin{column}{0.5\textwidth}
      \centering
      \onslide*<1>{
        \mintc[firstline=190,lastline=192]{../ocaml/runtime/amd64.S}
        \mintc[firstline=234,lastline=237]{../ocaml/runtime/amd64.S}
      }
      \onslide*<2>{
        \mintc[firstline=190,lastline=192,highlightlines={191-192}]{../ocaml/runtime/amd64.S}
        \mintc[firstline=234,lastline=237,highlightlines={235-237}]{../ocaml/runtime/amd64.S}
      }
      \onslide*<1>{\listCamlRaiseExn[highlightlines=720]{}}
      \onslide*<2>{\listCamlRaiseExn[highlightlines=722]{}}
      \onslide*<3->{\providecommand\step{5}\input{diagrams/backtrace/backtrace.tex}}
    \end{column}
  \end{columns}
\end{frame}

\begin{frame}{Backtraces}{\funcname{caml\_probably\_raise}'s handler doesn't catch \typename{ExnC}}
  \begin{columns}[c]
    \begin{column}{0.45\textwidth}
      \minipage[c][0.75\textheight][s]{\columnwidth}
      \begin{itemize}
        \item<1-> \funcname{match \dots with}\footnotemark pattern matching can also match exceptions
        \item<1-> The CMM of \funcname{probably\_raise} shows that this \funcname{match \dots with} is implemented with a pattern matching and a \funcname{try \dots with}
        \item<1-> But this one only matches on \typename{ExnA} exception
        \item<2-> Since the pattern matching fails to catch the exception, \funcname{reraise} is called with our current \typename{ExnC} exception
        \item<3-> Disassembly shows that \funcname{reraise} is actually a call to \funcname{caml\_reraise\_exn}, which is also an assembly function provided by the runtime
      \end{itemize}
      \vfill
      \mintocaml[firstline=15,lastline=21,highlightlines={18-21}]{../src/backtrace/backtrace.ml}
      \endminipage
    \end{column}
    \begin{column}{0.55\textwidth}
      \centering
      \onslide*<1>{\mintcmm[highlightlines={10-18}]{../src/backtrace/camlBacktrace\_\_probably\_raise\_351.cmm}}
      \onslide*<2>{\listcmm[highlightlines=18]{../src/backtrace/camlBacktrace\_\_probably\_raise_351.cmm}{CMM of probably\_raise}}
      \onslide*<3>{\mintobjdump[firstline=5,lastline=35,highlightlines=31]{../src/backtrace/camlBacktrace\_\_probably\_raise_351.objdump}}
    \end{column}
  \end{columns}
  \only<1>{\footnotetext[1]{https://blog.janestreet.com/pattern-matching-and-exception-handling-unite/}}
\end{frame}

\newcommand\listCamlReraiseExn[1][]{
  \listasm[firstline=727,lastline=734,#1]{../ocaml/runtime/amd64.S}{runtime/amd64.S:727}}

\begin{frame}{Backtraces}{Introducing \funcname{caml\_reraise\_exn}}
  \begin{columns}[c]
    \begin{column}{0.5\textwidth}
      \minipage[c][0.66\textheight][s]{\columnwidth}
      \begin{itemize}
        \item<1-> The exception have to be reraised to the next handler
        \item<2-> First, checks if the backtrace should be collected with \funcarg{domain\_state->backtrace\_active}
        \only<3>{
        \begin{itemize}
            \item If not, behaves like a \funcname{raise\_notrace} with \funcname{RESTORE\_EXN\_HANDLER\_OCAML}
        \end{itemize}
        }
        \item<4-> Jumps into \funcname{caml\_raise\_exn}
        \item<5-> Calls \funcname{caml\_stash\_backtrace} again, to scan the next part of the stack: from the trap just pop-ed to the next trap
      \end{itemize}
      \vfill
      \mintocaml[firstline=15,lastline=21,highlightlines={18-21}]{../src/backtrace/backtrace.ml}
      \endminipage
    \end{column}
    \begin{column}{0.5\textwidth}
      \raggedleft
      \onslide*<1>{\listCamlReraiseExn{}}
      \onslide*<2>{\listCamlReraiseExn[highlightlines=729]{}}
      \onslide*<3>{
        \mintc[firstline=190,lastline=192,highlightlines={191-192}]{../ocaml/runtime/amd64.S}
        \mintc[firstline=234,lastline=237,highlightlines={235-237}]{../ocaml/runtime/amd64.S}
        \medskip
        \listCamlReraiseExn[highlightlines=731]{}
      }
      \onslide*<4-5>{
        % TODO refactor with \listCamlReraiseExn
        \mintasm[firstline=727,lastline=734,highlightlines=730]{../ocaml/runtime/amd64.S}
        \medskip
      }
      \onslide*<4>{\listCamlRaiseExn[highlightlines=711]{}}
      \onslide*<5>{\listCamlRaiseExn[highlightlines=720]{}}
    \end{column}
  \end{columns}
\end{frame}

\begin{frame}{Backtraces}{Backtrace collection continues}
  \begin{columns}[c]
    \begin{column}{0.55\textwidth}
      \minipage[c][0.75\textheight][s]{\columnwidth}
      \begin{itemize}
        \item<1-> \funcname{caml\_stash\_backtrace} scans the older part of the stack, up to the next trap
        \item<6-> Controls jumps to the nested exception handler in \funcname{maybe\_raise}, which matches only on \typename{ExnB}
        \item<7-> \funcname{caml\_reraise\_exn} is called again, backtrace collection resumes with \funcname{caml\_stash\_backtrace}
      \end{itemize}
      \medskip
      \begin{itemize}
        \item<2-> Constructed backtrace:
        \begin{itemize}
          \item <2-> \funcname{definitely\_raise}
          \item <2-> \funcname{probably\_raise}
          \item <3-> \funcname{probably\_raise} (re-raise)
          \item <4-> \funcname{maybe\_raise}
          \item <5-> \funcname{main}
          \item <8-> \funcname{main} (re-raise)
        \end{itemize}
      \end{itemize}
      \vfill
      \onslide*<1-5>{\mintocaml[firstline=15,lastline=21]{../src/backtrace/backtrace.ml}}
      \onslide*<6>{\mintocaml[firstline=28,lastline=34,highlightlines=31]{../src/backtrace/backtrace.ml}}
      \onslide*<7->{\mintocaml[firstline=28,lastline=34,highlightlines=33]{../src/backtrace/backtrace.ml}}
      \endminipage
    \end{column}
    \begin{column}{0.45\textwidth}
      \raggedleft
      \onslide*<1>{\providecommand\step{6}\input{diagrams/backtrace/backtrace.tex}}%
      \onslide*<2>{\providecommand\step{6}\input{diagrams/backtrace/backtrace.tex}}%
      \onslide*<3>{\providecommand\step{7}\input{diagrams/backtrace/backtrace.tex}}%
      \onslide*<4>{\providecommand\step{8}\input{diagrams/backtrace/backtrace.tex}}%
      \onslide*<5>{\providecommand\step{9}\input{diagrams/backtrace/backtrace.tex}}%
      \onslide*<6>{\providecommand\step{10}\input{diagrams/backtrace/backtrace.tex}}%
      \onslide*<7>{\providecommand\step{11}\input{diagrams/backtrace/backtrace.tex}}%
      \onslide*<8>{\providecommand\step{12}\input{diagrams/backtrace/backtrace.tex}}%
      \onslide*<9>{\providecommand\step{13}\input{diagrams/backtrace/backtrace.tex}}%
    \end{column}
  \end{columns}
\end{frame}

\begin{frame}{Backtraces}{Printing the backtrace}
  \begin{columns}[c]
    \begin{column}{0.45\textwidth}
      \minipage[c][0.66\textheight][s]{\columnwidth}
      \begin{itemize}
        \item<1-> The exception backtrace can now be printed to \funcarg{stdout} with \funcname{Printexc.print_backtrace}
        \item<2-> First the exception backtrace is retrieved into an OCaml array with \funcname{get_raw_backtrace }
        \item<3-> Which is implemented as the \funcname{caml_get_exception_raw_backtrace} C function in the runtime
        \item<4-> And finally printed with \funcname{print_exception_backtrace}
      \end{itemize}
      \vfill
      \mintocaml[firstline=28,lastline=34,highlightlines=33]{../src/backtrace/backtrace.ml}
      \endminipage
    \end{column}
    \begin{column}{0.55\textwidth}
      \onslide*<1,2>{
        \mintocaml[firstline=100,lastline=101]{../ocaml/stdlib/printexc.ml}
      }
      \only<1>{
        \listocaml[firstline=170,lastline=172]{../ocaml/stdlib/printexc.ml}{stdlib/printexc.ml:170}
      }
      \only<2>{
        \listocaml[firstline=170,lastline=172,highlightlines=172]{../ocaml/stdlib/printexc.ml}{stdlib/printexc.ml:170}
      }
      \only<3>{
        \mintc[firstline=154,lastline=156]{../ocaml/runtime/backtrace.c}
        \listc[firstline=171,lastline=192,highlightlines={184-187}]{../ocaml/runtime/backtrace.c}{runtime/backtrace.c:154}
      }
      \only<4>{
        \listocaml[firstline=155,lastline=165]{../ocaml/stdlib/printexc.ml}{stdlib/printexc.ml:55}
      }
    \end{column}
  \end{columns}
\end{frame}


\subsection{The multiple kinds of raise}
\frameSubsection{}{}

\begin{frame}{Backtraces}{When backtraces are not needed}
  \begin{columns}[c]
    \begin{column}{0.45\textwidth}
      \minipage[c][0.66\textheight][s]{\columnwidth}
      \begin{itemize}
        \item<1-> What if the backtrace for an exception is never needed?
        \item<2-> It's possible to raise the exception explicitly with \funcname{raise\_notrace} to make raising as cheap as possible
        \item<3-> An example of use in the \funcname{dynlink} library of the OCaml compiler
      \end{itemize}
      \vfill
      \onslide<3->{
      \listocaml[firstline=205,lastline=209,highlightlines=207]{../ocaml/otherlibs/dynlink/dynlink\_compilerlibs/misc.ml}{otherlibs/dynlink/dynlink\_compilerlibs/misc.ml:205}
      }
      \endminipage
    \end{column}
    \begin{column}{0.55\textwidth}
    \only<4>{
      \mintcmm[highlightlines=3]{../src/notrace/camlNotrace\_\_fun_347.cmm}
      \listcmm{../src/notrace/camlNotrace\_\_all\_somes\_327.cmm}{all\_somes}
    }
    \onslide*<5->{
      \listobjdump[highlightlines={6-9}]{../src/notrace/camlNotrace\_\_fun_347.objdump}{anonymous function from \funcname{all\_somes}}
    }
    \end{column}
  \end{columns}
\end{frame}

\begin{frame}{Backtraces}{Summary table of the multiple kinds of raise}
  \begin{itemize}
    \item When debugging information is enabled and if the backtrace of an exception isn't needed, it's possible to raise with \funcname{raise\_notrace} for performance
    \item When debugging information is \emph{not} enabled, the compiler optimises \funcname{raise} calls to inline assembly (same effect as using \funcname{raise\_notrace})
  \end{itemize}
  \bigskip
  \centering
  \begin{tabular}{| l | c | c | c | c |}
    \hline
    \parbox{10em}{Debug information \\ (\funcarg{-g} flag)}  & \multicolumn{2}{|c|}{Disabled}                                     & \multicolumn{2}{|c|}{Enabled} \\
    \hline
    Raise kind                                               & \funcname{raise} & \funcname{raise\_notrace}                       & \funcname{raise} & \funcname{raise\_notrace} \\
    \hline
    Compilation                                              & \parbox{8em}{inline assembly\\ (optimization)} & inline assembly   & \funcname{caml\_raise\_exn} & inline assembly \\
    \hline
  \end{tabular}
\end{frame}


%
% Takeaway #5
%
\subsection*{Takeaway \#5}
\frameSubsectionTakeaway{}
\againframe<5>{takeaway}
}%
      \onslide*<4>{\providecommand\step{8}%
% Backtraces
%
\subsection{One advantage of raising with debugging information}
\frameSubsection{}{}

\begin{frame}{Backtraces}{Debugging information \& source code}
  \begin{columns}[c]
    \begin{column}{0.5\textwidth}
      \begin{itemize}
        \item Raising an exception is fun and games, but how do we know where the exception originated? $\rightarrow$ With the backtrace associated with the exception!
        \item In order to get a backtrace from the point where an exception was raised, it's necessary to compile with debugging information enabled (\funcarg{-g} flag for the compiler)
        \item Same example as in the `Nested handlers' section
      \end{itemize}
    \end{column}
    \begin{column}{0.5\textwidth}
      \listocaml[firstline=9,lastline=34]{../src/backtrace/backtrace.ml}{backtrace/backtrace.ml}
    \end{column}
  \end{columns}
\end{frame}

\begin{frame}{Backtraces}{CMM and disassembly of \funcname{definitely\_raise}}
  \begin{columns}[c]
    \begin{column}{0.5\textwidth}
      \minipage[c][0.66\textheight][s]{\columnwidth}
      \begin{itemize}
        \item<1-> An effect of compiling with debugging information (\funcarg{-g}): the CMM no longer uses \funcname{raise\_notrace} to raise the exception but \funcname{raise} instead
        \item<2-> Disassembly shows that CMM \funcname{raise} is actually a call to \funcname{caml\_raise\_exn}, a function in assembler provided by the runtime
      \end{itemize}
      \vfill
      \mintocaml[firstline=9,lastline=13,highlightlines=12]{../src/backtrace/backtrace.ml}
      \endminipage
    \end{column}
    \begin{column}{0.5\textwidth}
      \onslide*<1>{
        \mintcmm[highlightlines=5]{../src/backtrace/camlBacktrace\_\_definitely\_raise\_347.cmm}
      }
      \onslide*<2>{
        \mintobjdump[firstline=2,highlightlines=14]{../src/backtrace/camlBacktrace\_\_definitely\_raise\_347.objdump}
      }
    \end{column}
  \end{columns}
\end{frame}

\begin{frame}{Backtraces}{Introducing \funcname{caml\_raise\_exn}}
  \begin{columns}[c]
    \begin{column}{0.5\textwidth}
      \minipage[c][0.66\textheight][s]{\columnwidth}
      \onslide*<1>{
        \begin{itemize}
          \item Raising the exception is performed using \funcname{caml\_raise\_exn} which is
            \begin{itemize}
              \item An assembly function provided by the runtime
              \item Architecture specific
            \end{itemize}
          \item Let's see what it does differently from \funcname{raise\_notrace}\dots
          \item We are about to raise \typename{ExnC} with \funcname{caml\_raise\_exn} from \funcname{definitely\_raise}
        \end{itemize}
      }
      \onslide*<2>{
        \mintobjdump[firstline=2,highlightlines=14]{../src/backtrace/camlBacktrace\_\_definitely\_raise\_347.objdump}
      }
      \vfill
      \mintocaml[firstline=9,lastline=13,highlightlines=12]{../src/backtrace/backtrace.ml}
      \endminipage
    \end{column}
    \begin{column}{0.5\textwidth}
      \centering
      \providecommand\step{1}\input{diagrams/backtrace/backtrace.tex}
    \end{column}
  \end{columns}
\end{frame}

\begin{frame}{Backtraces}{But before that, we need more fields from \typename{caml\_domain\_state}}
  \begin{columns}
    \begin{column}{0.5\textwidth}
      \begin{itemize}
        \item Remember \typename{caml\_domain\_state} the per-domain runtime state?
        \item Some other members are of interest to us now\dots
        \item \localname{backtrace\_active} is used to know if the runtime should collect backtraces
        \item \localname{backtrace\_active} can be set by
          \begin{itemize}
            \item The \funcarg{b} flag in the \funcname{OCAMLRUNPARAM} environment variable
            \item \mintinline{ocaml}{Printexc.record_backtrace : bool -> unit} \footnotemark
          \end{itemize}
      \end{itemize}
    \end{column}
    \begin{column}{0.5\textwidth}
      \begin{listing}
        \mintc[highlightlines={9-11}]{src/ocaml/domain\_state\_backtrace.h}
        \caption{runtime/caml/domain\_state.h}
      \end{listing}
    \end{column}
  \end{columns}
  \footnotetext[1]{https://v2.ocaml.org/api/Printexc.html}
\end{frame}

\newcommand\listCamlRaiseExn[1][]{
  \listasm[firstline=702,lastline=725,#1]{../ocaml/runtime/amd64.S}{runtime/amd64.S:702}}

\begin{frame}{Backtraces}{Walk through \funcname{caml\_raise\_exn}}
  \begin{columns}[T]
    \begin{column}{0.5\textwidth}
      \begin{itemize}
        \item<1-> First checks whether exceptions backtrace should be recorded
        \item<1-> By checking the \funcarg{domain\_state->backtrace\_active} value
        \item<2-> If not, acts as \funcname{raise\_notrace} with \funcname{RESTORE\_EXN\_HANDLER\_OCAML}
          \begin{itemize}
            \item Set \regname{RSP} to \localname{domain\_state->exn\_handler} value
            \item Restore \localname{domain\_state->exn\_handler} from trap
            \item Jump with \funcname{ret} (same effect as using \funcname{pop} and \funcname{jmp})
          \end{itemize}
        \item<3-> Otherwise, switches to the C stack to be able to execute C code
        \item<4-> Calls \funcname{caml\_stash\_backtrace}, a C function provided by the runtime\ignorespaces
          \onslide<6->{, with
            \begin{itemize}
              \item \funcarg{exn}: exception value
              \item \funcarg{pc}: code address of the raising point
              \item \funcarg{sp}: stack address of the raising function
              \item \funcarg{trapsp}: stack address of the next handler
            \end{itemize}
          }
      \end{itemize}
      \bigskip
      \mintocaml[firstline=9,lastline=13,highlightlines=12]{../src/backtrace/backtrace.ml}
    \end{column}
    \begin{column}{0.5\textwidth}
      \raggedleft
      \onslide*<1,3-4>{
        \mintc[firstline=190,lastline=192]{../ocaml/runtime/amd64.S}
        \mintc[firstline=234,lastline=237]{../ocaml/runtime/amd64.S}
      }
      \onslide*<2>{
        \mintc[firstline=190,lastline=192,highlightlines={191-192}]{../ocaml/runtime/amd64.S}
        \mintc[firstline=234,lastline=237,highlightlines={235-237}]{../ocaml/runtime/amd64.S}
      }
      \onslide*<1>{\listCamlRaiseExn[highlightlines=705]{}}%
      \onslide*<2>{\listCamlRaiseExn[highlightlines={707-708}]{}}%
      \onslide*<3>{\listCamlRaiseExn[highlightlines={712-714}]{}}%
      \onslide*<4>{\listCamlRaiseExn[highlightlines={715-720}]{}}%
      \onslide*<5>{\providecommand\step{1}\input{diagrams/backtrace/backtrace.tex}}%
      \onslide*<6>{\providecommand\step{2}\input{diagrams/backtrace/backtrace.tex}}%
    \end{column}
  \end{columns}
\end{frame}

\newcommand\listStashBacktrace[1][]{
  \listc[firstline=91,lastline=120,#1]{../ocaml/runtime/backtrace\_nat.c}{runtime/backtrace\_nat.c:91}}

\begin{frame}{Backtraces}{Introducing \funcname{caml\_stash\_backtrace}}
  \begin{columns}[c]
    \begin{column}{0.5\textwidth}
      \minipage[c][0.66\textheight][s]{\columnwidth}
      \begin{itemize}
        \item<1-> A C function provided by the runtime (specific to native code)
        \item<2-> It scans the stack, between the stack pointer at the raising point up to the next trap, in order to collect the names of the detected function frames
          \onslide*<2>{
            \begin{itemize}
              \item And more information, like line number, column number...
            \end{itemize}
          }
        \item<3-> It uses the same technique and information as the Garbage Collector (GC) uses to scan the values live on the stack
          \onslide*<4>{
            \begin{itemize}
              \item \typename{frame\_descr} are at the core of stack unwinding
            \end{itemize}
          }
      \end{itemize}
      \vfill
      \mintocaml[firstline=9,lastline=13,highlightlines=12]{../src/backtrace/backtrace.ml}
      \endminipage
    \end{column}
    \begin{column}{0.5\textwidth}
      \onslide*<1>{\listStashBacktrace[highlightlines=91]{}}
      \onslide*<2>{\listStashBacktrace[highlightlines={91,117-118}]{}}
      \onslide*<3->{\listStashBacktrace[highlightlines={94,106,109-110}]{}}
    \end{column}
  \end{columns}
\end{frame}

\newcommand\listFrameDescr[1][]{
  \listocaml[firstline=111,lastline=124,#1]{../ocaml/asmcomp/emitaux.ml}{asmcomp/emitaux.ml:111}}
\newcommand\listDebugInfo[1][]{
  \listocaml[firstline=38,lastline=49,#1]{../ocaml/lambda/debuginfo.mli}{lambda/debuginfo.mli:38}}

\begin{frame}{Backtraces}{\typename{frame\_descr} and \typename{frame\_debuginfo} in a nutshell}
  \begin{columns}[c]
    \begin{column}{0.5\textwidth}
      \begin{itemize}
        \item<1-> For every return address in an OCaml program, there is a associated \typename{frame\_descr}
        \item<2-> A \typename{frame\_descr} specifies
        \begin{itemize}
          \item<2-> The size of the function stack frame
          \item<3-> Where live values are on the stack (so that the GC can mark them as live and prevent from being swept)
        \end{itemize}
        \item<4-> When compiled with debug information, \typename{frame\_descr} will be linked to a \typename{frame\_debuginfo}
        \begin{itemize}
          \item<5-> Contains information about the position in the source code (file name, line and column number)
        \end{itemize}
      \end{itemize}
    \end{column}
    \begin{column}{0.5\textwidth}
        \onslide*<1>{\listFrameDescr[highlightlines={118-122}]{}}
        \onslide*<2>{\listFrameDescr[highlightlines={120}]{}}
        \onslide*<3>{\listFrameDescr[highlightlines={121}]{}}
        \onslide*<4->{\listFrameDescr[highlightlines={122}]{}}
        \medskip
        \onslide*<1-3>{\listDebugInfo{}}
        \onslide*<4>{\listDebugInfo[highlightlines={38,49}]{}}
        \onslide*<5>{\listDebugInfo[highlightlines={39-46}]{}}
    \end{column}
  \end{columns}
\end{frame}

\begin{frame}{Backtraces}{Scanning the stack with \typename{frame\_descr}}
  \begin{columns}[c]
    \begin{column}{0.5\textwidth}
      \begin{itemize}
        \item Given the return address of the code that raises the exception by calling \funcname{caml\_raise\_exn} (\funcarg{pc} argument of \funcname{caml\_stash\_backtrace})
        \item And the position of the stack pointer (\funcarg{sp} argument of \funcname{caml\_stash\_backtrace})
        \item The frame size given by \typename{frame\_descr}.\funcarg{frame\_size}, allows to find the end of the previous frame
        \item And the return address of the caller of this function
        \bigskip
        \onslide<3->{
        \item Note that traps (here the reddish \funcname{ExnA}, \funcname{ExnB} and \funcname{ExnC} blocks) belong to the frame that installed them, and so the frame size changes when traps are removed
        }
      \end{itemize}
    \end{column}
    \begin{column}{0.5\textwidth}
      \raggedleft
      \onslide*<1>{\providecommand\step{2}\input{diagrams/backtrace/backtrace.tex}}%
      \onslide*<2>{\providecommand\step{1}\input{diagrams/backtrace/scanstack.tex}}%
      \onslide*<3>{\providecommand\step{2}\input{diagrams/backtrace/scanstack.tex}}%
      \onslide*<4>{\providecommand\step{3}\input{diagrams/backtrace/scanstack.tex}}%
      \onslide*<5>{\providecommand\step{4}\input{diagrams/backtrace/scanstack.tex}}%
      \onslide*<6>{\providecommand\step{5}\input{diagrams/backtrace/scanstack.tex}}%
    \end{column}
  \end{columns}
\end{frame}

% Duplicate of previous-previous slide
\begin{frame}{Backtraces}{Continuing on \funcname{caml\_stash\_backtrace}}
  \begin{columns}[c]
    \begin{column}{0.5\textwidth}
      \minipage[c][0.75\textheight][s]{\columnwidth}
      \begin{itemize}
          \color{gray}
        \item A C function provided by the runtime (specific to native code)
        \item It scans the stack, between the stack pointer at the raising point up to the next trap, in order to collect the names of the detected function frames
        \item It uses the same technique and information as the Garbage Collector (GC) uses to scan the values live on the stack
      \end{itemize}
      \begin{itemize}
        \item For each function frame detected, store a pointer to the \typename{frame\_descr} in the \localname{domain\_state->backtrace\_buffer} array, effectively constructing the backtrace
      \end{itemize}
      \onslide<2->{
        \begin{itemize}
            \smallskip
          \item Constructed backtrace:
            \funcname{definitely\_raise},
            \onslide<3->{\funcname{probably\_raise}}
        \end{itemize}
      }
      \vfill
      \mintocaml[firstline=9,lastline=13,highlightlines=12]{../src/backtrace/backtrace.ml}
      \endminipage
    \end{column}
    \begin{column}{0.5\textwidth}
      \raggedleft
      \onslide*<1>{\listStashBacktrace[highlightlines={114-115}]{}}
      \onslide*<2>{\providecommand\step{3}\input{diagrams/backtrace/backtrace.tex}}%
      \onslide*<3>{\providecommand\step{4}\input{diagrams/backtrace/backtrace.tex}}%
    \end{column}
  \end{columns}
\end{frame}

\begin{frame}{Backtraces}{Back to \funcname{caml\_raise\_exn}}
  \begin{columns}[c]
    \begin{column}{0.5\textwidth}
      \minipage[c][0.66\textheight][s]{\columnwidth}
      \begin{itemize}\color{gray}
        \item Checks wheteher exceptions backtrace should be recorded, by checking the \funcarg{domain\_state->backtrace\_active} value
        \item Switches to the C stack to be able to execute C code
        \item Calls \funcname{caml\_stash\_backtrace}, to collect the backtrace up to the next trap
      \end{itemize}
      \onslide*<2->{
        \begin{itemize}
          \item<2-> Executes the exception handler in \funcname{probably\_raise} with \funcname{RESTORE\_EXN\_HANDLER\_OCAML}
            \begin{itemize}
              \item Set \regname{RSP} to \localname{domain\_state->exn\_handler} value
              \item Restore \localname{domain\_state->exn\_handler} from trap
              \item Jump to the handler code with \funcname{ret}
            \end{itemize}
        \end{itemize}
      }
      \vfill
      \onslide*<1>{\mintocaml[firstline=9,lastline=13,highlightlines=12]{../src/backtrace/backtrace.ml}}
      \onslide*<2->{\mintocaml[firstline=15,lastline=21,highlightlines={19-21}]{../src/backtrace/backtrace.ml}}
      \endminipage
    \end{column}
    \begin{column}{0.5\textwidth}
      \centering
      \onslide*<1>{
        \mintc[firstline=190,lastline=192]{../ocaml/runtime/amd64.S}
        \mintc[firstline=234,lastline=237]{../ocaml/runtime/amd64.S}
      }
      \onslide*<2>{
        \mintc[firstline=190,lastline=192,highlightlines={191-192}]{../ocaml/runtime/amd64.S}
        \mintc[firstline=234,lastline=237,highlightlines={235-237}]{../ocaml/runtime/amd64.S}
      }
      \onslide*<1>{\listCamlRaiseExn[highlightlines=720]{}}
      \onslide*<2>{\listCamlRaiseExn[highlightlines=722]{}}
      \onslide*<3->{\providecommand\step{5}\input{diagrams/backtrace/backtrace.tex}}
    \end{column}
  \end{columns}
\end{frame}

\begin{frame}{Backtraces}{\funcname{caml\_probably\_raise}'s handler doesn't catch \typename{ExnC}}
  \begin{columns}[c]
    \begin{column}{0.45\textwidth}
      \minipage[c][0.75\textheight][s]{\columnwidth}
      \begin{itemize}
        \item<1-> \funcname{match \dots with}\footnotemark pattern matching can also match exceptions
        \item<1-> The CMM of \funcname{probably\_raise} shows that this \funcname{match \dots with} is implemented with a pattern matching and a \funcname{try \dots with}
        \item<1-> But this one only matches on \typename{ExnA} exception
        \item<2-> Since the pattern matching fails to catch the exception, \funcname{reraise} is called with our current \typename{ExnC} exception
        \item<3-> Disassembly shows that \funcname{reraise} is actually a call to \funcname{caml\_reraise\_exn}, which is also an assembly function provided by the runtime
      \end{itemize}
      \vfill
      \mintocaml[firstline=15,lastline=21,highlightlines={18-21}]{../src/backtrace/backtrace.ml}
      \endminipage
    \end{column}
    \begin{column}{0.55\textwidth}
      \centering
      \onslide*<1>{\mintcmm[highlightlines={10-18}]{../src/backtrace/camlBacktrace\_\_probably\_raise\_351.cmm}}
      \onslide*<2>{\listcmm[highlightlines=18]{../src/backtrace/camlBacktrace\_\_probably\_raise_351.cmm}{CMM of probably\_raise}}
      \onslide*<3>{\mintobjdump[firstline=5,lastline=35,highlightlines=31]{../src/backtrace/camlBacktrace\_\_probably\_raise_351.objdump}}
    \end{column}
  \end{columns}
  \only<1>{\footnotetext[1]{https://blog.janestreet.com/pattern-matching-and-exception-handling-unite/}}
\end{frame}

\newcommand\listCamlReraiseExn[1][]{
  \listasm[firstline=727,lastline=734,#1]{../ocaml/runtime/amd64.S}{runtime/amd64.S:727}}

\begin{frame}{Backtraces}{Introducing \funcname{caml\_reraise\_exn}}
  \begin{columns}[c]
    \begin{column}{0.5\textwidth}
      \minipage[c][0.66\textheight][s]{\columnwidth}
      \begin{itemize}
        \item<1-> The exception have to be reraised to the next handler
        \item<2-> First, checks if the backtrace should be collected with \funcarg{domain\_state->backtrace\_active}
        \only<3>{
        \begin{itemize}
            \item If not, behaves like a \funcname{raise\_notrace} with \funcname{RESTORE\_EXN\_HANDLER\_OCAML}
        \end{itemize}
        }
        \item<4-> Jumps into \funcname{caml\_raise\_exn}
        \item<5-> Calls \funcname{caml\_stash\_backtrace} again, to scan the next part of the stack: from the trap just pop-ed to the next trap
      \end{itemize}
      \vfill
      \mintocaml[firstline=15,lastline=21,highlightlines={18-21}]{../src/backtrace/backtrace.ml}
      \endminipage
    \end{column}
    \begin{column}{0.5\textwidth}
      \raggedleft
      \onslide*<1>{\listCamlReraiseExn{}}
      \onslide*<2>{\listCamlReraiseExn[highlightlines=729]{}}
      \onslide*<3>{
        \mintc[firstline=190,lastline=192,highlightlines={191-192}]{../ocaml/runtime/amd64.S}
        \mintc[firstline=234,lastline=237,highlightlines={235-237}]{../ocaml/runtime/amd64.S}
        \medskip
        \listCamlReraiseExn[highlightlines=731]{}
      }
      \onslide*<4-5>{
        % TODO refactor with \listCamlReraiseExn
        \mintasm[firstline=727,lastline=734,highlightlines=730]{../ocaml/runtime/amd64.S}
        \medskip
      }
      \onslide*<4>{\listCamlRaiseExn[highlightlines=711]{}}
      \onslide*<5>{\listCamlRaiseExn[highlightlines=720]{}}
    \end{column}
  \end{columns}
\end{frame}

\begin{frame}{Backtraces}{Backtrace collection continues}
  \begin{columns}[c]
    \begin{column}{0.55\textwidth}
      \minipage[c][0.75\textheight][s]{\columnwidth}
      \begin{itemize}
        \item<1-> \funcname{caml\_stash\_backtrace} scans the older part of the stack, up to the next trap
        \item<6-> Controls jumps to the nested exception handler in \funcname{maybe\_raise}, which matches only on \typename{ExnB}
        \item<7-> \funcname{caml\_reraise\_exn} is called again, backtrace collection resumes with \funcname{caml\_stash\_backtrace}
      \end{itemize}
      \medskip
      \begin{itemize}
        \item<2-> Constructed backtrace:
        \begin{itemize}
          \item <2-> \funcname{definitely\_raise}
          \item <2-> \funcname{probably\_raise}
          \item <3-> \funcname{probably\_raise} (re-raise)
          \item <4-> \funcname{maybe\_raise}
          \item <5-> \funcname{main}
          \item <8-> \funcname{main} (re-raise)
        \end{itemize}
      \end{itemize}
      \vfill
      \onslide*<1-5>{\mintocaml[firstline=15,lastline=21]{../src/backtrace/backtrace.ml}}
      \onslide*<6>{\mintocaml[firstline=28,lastline=34,highlightlines=31]{../src/backtrace/backtrace.ml}}
      \onslide*<7->{\mintocaml[firstline=28,lastline=34,highlightlines=33]{../src/backtrace/backtrace.ml}}
      \endminipage
    \end{column}
    \begin{column}{0.45\textwidth}
      \raggedleft
      \onslide*<1>{\providecommand\step{6}\input{diagrams/backtrace/backtrace.tex}}%
      \onslide*<2>{\providecommand\step{6}\input{diagrams/backtrace/backtrace.tex}}%
      \onslide*<3>{\providecommand\step{7}\input{diagrams/backtrace/backtrace.tex}}%
      \onslide*<4>{\providecommand\step{8}\input{diagrams/backtrace/backtrace.tex}}%
      \onslide*<5>{\providecommand\step{9}\input{diagrams/backtrace/backtrace.tex}}%
      \onslide*<6>{\providecommand\step{10}\input{diagrams/backtrace/backtrace.tex}}%
      \onslide*<7>{\providecommand\step{11}\input{diagrams/backtrace/backtrace.tex}}%
      \onslide*<8>{\providecommand\step{12}\input{diagrams/backtrace/backtrace.tex}}%
      \onslide*<9>{\providecommand\step{13}\input{diagrams/backtrace/backtrace.tex}}%
    \end{column}
  \end{columns}
\end{frame}

\begin{frame}{Backtraces}{Printing the backtrace}
  \begin{columns}[c]
    \begin{column}{0.45\textwidth}
      \minipage[c][0.66\textheight][s]{\columnwidth}
      \begin{itemize}
        \item<1-> The exception backtrace can now be printed to \funcarg{stdout} with \funcname{Printexc.print_backtrace}
        \item<2-> First the exception backtrace is retrieved into an OCaml array with \funcname{get_raw_backtrace }
        \item<3-> Which is implemented as the \funcname{caml_get_exception_raw_backtrace} C function in the runtime
        \item<4-> And finally printed with \funcname{print_exception_backtrace}
      \end{itemize}
      \vfill
      \mintocaml[firstline=28,lastline=34,highlightlines=33]{../src/backtrace/backtrace.ml}
      \endminipage
    \end{column}
    \begin{column}{0.55\textwidth}
      \onslide*<1,2>{
        \mintocaml[firstline=100,lastline=101]{../ocaml/stdlib/printexc.ml}
      }
      \only<1>{
        \listocaml[firstline=170,lastline=172]{../ocaml/stdlib/printexc.ml}{stdlib/printexc.ml:170}
      }
      \only<2>{
        \listocaml[firstline=170,lastline=172,highlightlines=172]{../ocaml/stdlib/printexc.ml}{stdlib/printexc.ml:170}
      }
      \only<3>{
        \mintc[firstline=154,lastline=156]{../ocaml/runtime/backtrace.c}
        \listc[firstline=171,lastline=192,highlightlines={184-187}]{../ocaml/runtime/backtrace.c}{runtime/backtrace.c:154}
      }
      \only<4>{
        \listocaml[firstline=155,lastline=165]{../ocaml/stdlib/printexc.ml}{stdlib/printexc.ml:55}
      }
    \end{column}
  \end{columns}
\end{frame}


\subsection{The multiple kinds of raise}
\frameSubsection{}{}

\begin{frame}{Backtraces}{When backtraces are not needed}
  \begin{columns}[c]
    \begin{column}{0.45\textwidth}
      \minipage[c][0.66\textheight][s]{\columnwidth}
      \begin{itemize}
        \item<1-> What if the backtrace for an exception is never needed?
        \item<2-> It's possible to raise the exception explicitly with \funcname{raise\_notrace} to make raising as cheap as possible
        \item<3-> An example of use in the \funcname{dynlink} library of the OCaml compiler
      \end{itemize}
      \vfill
      \onslide<3->{
      \listocaml[firstline=205,lastline=209,highlightlines=207]{../ocaml/otherlibs/dynlink/dynlink\_compilerlibs/misc.ml}{otherlibs/dynlink/dynlink\_compilerlibs/misc.ml:205}
      }
      \endminipage
    \end{column}
    \begin{column}{0.55\textwidth}
    \only<4>{
      \mintcmm[highlightlines=3]{../src/notrace/camlNotrace\_\_fun_347.cmm}
      \listcmm{../src/notrace/camlNotrace\_\_all\_somes\_327.cmm}{all\_somes}
    }
    \onslide*<5->{
      \listobjdump[highlightlines={6-9}]{../src/notrace/camlNotrace\_\_fun_347.objdump}{anonymous function from \funcname{all\_somes}}
    }
    \end{column}
  \end{columns}
\end{frame}

\begin{frame}{Backtraces}{Summary table of the multiple kinds of raise}
  \begin{itemize}
    \item When debugging information is enabled and if the backtrace of an exception isn't needed, it's possible to raise with \funcname{raise\_notrace} for performance
    \item When debugging information is \emph{not} enabled, the compiler optimises \funcname{raise} calls to inline assembly (same effect as using \funcname{raise\_notrace})
  \end{itemize}
  \bigskip
  \centering
  \begin{tabular}{| l | c | c | c | c |}
    \hline
    \parbox{10em}{Debug information \\ (\funcarg{-g} flag)}  & \multicolumn{2}{|c|}{Disabled}                                     & \multicolumn{2}{|c|}{Enabled} \\
    \hline
    Raise kind                                               & \funcname{raise} & \funcname{raise\_notrace}                       & \funcname{raise} & \funcname{raise\_notrace} \\
    \hline
    Compilation                                              & \parbox{8em}{inline assembly\\ (optimization)} & inline assembly   & \funcname{caml\_raise\_exn} & inline assembly \\
    \hline
  \end{tabular}
\end{frame}


%
% Takeaway #5
%
\subsection*{Takeaway \#5}
\frameSubsectionTakeaway{}
\againframe<5>{takeaway}
}%
      \onslide*<5>{\providecommand\step{9}%
% Backtraces
%
\subsection{One advantage of raising with debugging information}
\frameSubsection{}{}

\begin{frame}{Backtraces}{Debugging information \& source code}
  \begin{columns}[c]
    \begin{column}{0.5\textwidth}
      \begin{itemize}
        \item Raising an exception is fun and games, but how do we know where the exception originated? $\rightarrow$ With the backtrace associated with the exception!
        \item In order to get a backtrace from the point where an exception was raised, it's necessary to compile with debugging information enabled (\funcarg{-g} flag for the compiler)
        \item Same example as in the `Nested handlers' section
      \end{itemize}
    \end{column}
    \begin{column}{0.5\textwidth}
      \listocaml[firstline=9,lastline=34]{../src/backtrace/backtrace.ml}{backtrace/backtrace.ml}
    \end{column}
  \end{columns}
\end{frame}

\begin{frame}{Backtraces}{CMM and disassembly of \funcname{definitely\_raise}}
  \begin{columns}[c]
    \begin{column}{0.5\textwidth}
      \minipage[c][0.66\textheight][s]{\columnwidth}
      \begin{itemize}
        \item<1-> An effect of compiling with debugging information (\funcarg{-g}): the CMM no longer uses \funcname{raise\_notrace} to raise the exception but \funcname{raise} instead
        \item<2-> Disassembly shows that CMM \funcname{raise} is actually a call to \funcname{caml\_raise\_exn}, a function in assembler provided by the runtime
      \end{itemize}
      \vfill
      \mintocaml[firstline=9,lastline=13,highlightlines=12]{../src/backtrace/backtrace.ml}
      \endminipage
    \end{column}
    \begin{column}{0.5\textwidth}
      \onslide*<1>{
        \mintcmm[highlightlines=5]{../src/backtrace/camlBacktrace\_\_definitely\_raise\_347.cmm}
      }
      \onslide*<2>{
        \mintobjdump[firstline=2,highlightlines=14]{../src/backtrace/camlBacktrace\_\_definitely\_raise\_347.objdump}
      }
    \end{column}
  \end{columns}
\end{frame}

\begin{frame}{Backtraces}{Introducing \funcname{caml\_raise\_exn}}
  \begin{columns}[c]
    \begin{column}{0.5\textwidth}
      \minipage[c][0.66\textheight][s]{\columnwidth}
      \onslide*<1>{
        \begin{itemize}
          \item Raising the exception is performed using \funcname{caml\_raise\_exn} which is
            \begin{itemize}
              \item An assembly function provided by the runtime
              \item Architecture specific
            \end{itemize}
          \item Let's see what it does differently from \funcname{raise\_notrace}\dots
          \item We are about to raise \typename{ExnC} with \funcname{caml\_raise\_exn} from \funcname{definitely\_raise}
        \end{itemize}
      }
      \onslide*<2>{
        \mintobjdump[firstline=2,highlightlines=14]{../src/backtrace/camlBacktrace\_\_definitely\_raise\_347.objdump}
      }
      \vfill
      \mintocaml[firstline=9,lastline=13,highlightlines=12]{../src/backtrace/backtrace.ml}
      \endminipage
    \end{column}
    \begin{column}{0.5\textwidth}
      \centering
      \providecommand\step{1}\input{diagrams/backtrace/backtrace.tex}
    \end{column}
  \end{columns}
\end{frame}

\begin{frame}{Backtraces}{But before that, we need more fields from \typename{caml\_domain\_state}}
  \begin{columns}
    \begin{column}{0.5\textwidth}
      \begin{itemize}
        \item Remember \typename{caml\_domain\_state} the per-domain runtime state?
        \item Some other members are of interest to us now\dots
        \item \localname{backtrace\_active} is used to know if the runtime should collect backtraces
        \item \localname{backtrace\_active} can be set by
          \begin{itemize}
            \item The \funcarg{b} flag in the \funcname{OCAMLRUNPARAM} environment variable
            \item \mintinline{ocaml}{Printexc.record_backtrace : bool -> unit} \footnotemark
          \end{itemize}
      \end{itemize}
    \end{column}
    \begin{column}{0.5\textwidth}
      \begin{listing}
        \mintc[highlightlines={9-11}]{src/ocaml/domain\_state\_backtrace.h}
        \caption{runtime/caml/domain\_state.h}
      \end{listing}
    \end{column}
  \end{columns}
  \footnotetext[1]{https://v2.ocaml.org/api/Printexc.html}
\end{frame}

\newcommand\listCamlRaiseExn[1][]{
  \listasm[firstline=702,lastline=725,#1]{../ocaml/runtime/amd64.S}{runtime/amd64.S:702}}

\begin{frame}{Backtraces}{Walk through \funcname{caml\_raise\_exn}}
  \begin{columns}[T]
    \begin{column}{0.5\textwidth}
      \begin{itemize}
        \item<1-> First checks whether exceptions backtrace should be recorded
        \item<1-> By checking the \funcarg{domain\_state->backtrace\_active} value
        \item<2-> If not, acts as \funcname{raise\_notrace} with \funcname{RESTORE\_EXN\_HANDLER\_OCAML}
          \begin{itemize}
            \item Set \regname{RSP} to \localname{domain\_state->exn\_handler} value
            \item Restore \localname{domain\_state->exn\_handler} from trap
            \item Jump with \funcname{ret} (same effect as using \funcname{pop} and \funcname{jmp})
          \end{itemize}
        \item<3-> Otherwise, switches to the C stack to be able to execute C code
        \item<4-> Calls \funcname{caml\_stash\_backtrace}, a C function provided by the runtime\ignorespaces
          \onslide<6->{, with
            \begin{itemize}
              \item \funcarg{exn}: exception value
              \item \funcarg{pc}: code address of the raising point
              \item \funcarg{sp}: stack address of the raising function
              \item \funcarg{trapsp}: stack address of the next handler
            \end{itemize}
          }
      \end{itemize}
      \bigskip
      \mintocaml[firstline=9,lastline=13,highlightlines=12]{../src/backtrace/backtrace.ml}
    \end{column}
    \begin{column}{0.5\textwidth}
      \raggedleft
      \onslide*<1,3-4>{
        \mintc[firstline=190,lastline=192]{../ocaml/runtime/amd64.S}
        \mintc[firstline=234,lastline=237]{../ocaml/runtime/amd64.S}
      }
      \onslide*<2>{
        \mintc[firstline=190,lastline=192,highlightlines={191-192}]{../ocaml/runtime/amd64.S}
        \mintc[firstline=234,lastline=237,highlightlines={235-237}]{../ocaml/runtime/amd64.S}
      }
      \onslide*<1>{\listCamlRaiseExn[highlightlines=705]{}}%
      \onslide*<2>{\listCamlRaiseExn[highlightlines={707-708}]{}}%
      \onslide*<3>{\listCamlRaiseExn[highlightlines={712-714}]{}}%
      \onslide*<4>{\listCamlRaiseExn[highlightlines={715-720}]{}}%
      \onslide*<5>{\providecommand\step{1}\input{diagrams/backtrace/backtrace.tex}}%
      \onslide*<6>{\providecommand\step{2}\input{diagrams/backtrace/backtrace.tex}}%
    \end{column}
  \end{columns}
\end{frame}

\newcommand\listStashBacktrace[1][]{
  \listc[firstline=91,lastline=120,#1]{../ocaml/runtime/backtrace\_nat.c}{runtime/backtrace\_nat.c:91}}

\begin{frame}{Backtraces}{Introducing \funcname{caml\_stash\_backtrace}}
  \begin{columns}[c]
    \begin{column}{0.5\textwidth}
      \minipage[c][0.66\textheight][s]{\columnwidth}
      \begin{itemize}
        \item<1-> A C function provided by the runtime (specific to native code)
        \item<2-> It scans the stack, between the stack pointer at the raising point up to the next trap, in order to collect the names of the detected function frames
          \onslide*<2>{
            \begin{itemize}
              \item And more information, like line number, column number...
            \end{itemize}
          }
        \item<3-> It uses the same technique and information as the Garbage Collector (GC) uses to scan the values live on the stack
          \onslide*<4>{
            \begin{itemize}
              \item \typename{frame\_descr} are at the core of stack unwinding
            \end{itemize}
          }
      \end{itemize}
      \vfill
      \mintocaml[firstline=9,lastline=13,highlightlines=12]{../src/backtrace/backtrace.ml}
      \endminipage
    \end{column}
    \begin{column}{0.5\textwidth}
      \onslide*<1>{\listStashBacktrace[highlightlines=91]{}}
      \onslide*<2>{\listStashBacktrace[highlightlines={91,117-118}]{}}
      \onslide*<3->{\listStashBacktrace[highlightlines={94,106,109-110}]{}}
    \end{column}
  \end{columns}
\end{frame}

\newcommand\listFrameDescr[1][]{
  \listocaml[firstline=111,lastline=124,#1]{../ocaml/asmcomp/emitaux.ml}{asmcomp/emitaux.ml:111}}
\newcommand\listDebugInfo[1][]{
  \listocaml[firstline=38,lastline=49,#1]{../ocaml/lambda/debuginfo.mli}{lambda/debuginfo.mli:38}}

\begin{frame}{Backtraces}{\typename{frame\_descr} and \typename{frame\_debuginfo} in a nutshell}
  \begin{columns}[c]
    \begin{column}{0.5\textwidth}
      \begin{itemize}
        \item<1-> For every return address in an OCaml program, there is a associated \typename{frame\_descr}
        \item<2-> A \typename{frame\_descr} specifies
        \begin{itemize}
          \item<2-> The size of the function stack frame
          \item<3-> Where live values are on the stack (so that the GC can mark them as live and prevent from being swept)
        \end{itemize}
        \item<4-> When compiled with debug information, \typename{frame\_descr} will be linked to a \typename{frame\_debuginfo}
        \begin{itemize}
          \item<5-> Contains information about the position in the source code (file name, line and column number)
        \end{itemize}
      \end{itemize}
    \end{column}
    \begin{column}{0.5\textwidth}
        \onslide*<1>{\listFrameDescr[highlightlines={118-122}]{}}
        \onslide*<2>{\listFrameDescr[highlightlines={120}]{}}
        \onslide*<3>{\listFrameDescr[highlightlines={121}]{}}
        \onslide*<4->{\listFrameDescr[highlightlines={122}]{}}
        \medskip
        \onslide*<1-3>{\listDebugInfo{}}
        \onslide*<4>{\listDebugInfo[highlightlines={38,49}]{}}
        \onslide*<5>{\listDebugInfo[highlightlines={39-46}]{}}
    \end{column}
  \end{columns}
\end{frame}

\begin{frame}{Backtraces}{Scanning the stack with \typename{frame\_descr}}
  \begin{columns}[c]
    \begin{column}{0.5\textwidth}
      \begin{itemize}
        \item Given the return address of the code that raises the exception by calling \funcname{caml\_raise\_exn} (\funcarg{pc} argument of \funcname{caml\_stash\_backtrace})
        \item And the position of the stack pointer (\funcarg{sp} argument of \funcname{caml\_stash\_backtrace})
        \item The frame size given by \typename{frame\_descr}.\funcarg{frame\_size}, allows to find the end of the previous frame
        \item And the return address of the caller of this function
        \bigskip
        \onslide<3->{
        \item Note that traps (here the reddish \funcname{ExnA}, \funcname{ExnB} and \funcname{ExnC} blocks) belong to the frame that installed them, and so the frame size changes when traps are removed
        }
      \end{itemize}
    \end{column}
    \begin{column}{0.5\textwidth}
      \raggedleft
      \onslide*<1>{\providecommand\step{2}\input{diagrams/backtrace/backtrace.tex}}%
      \onslide*<2>{\providecommand\step{1}\input{diagrams/backtrace/scanstack.tex}}%
      \onslide*<3>{\providecommand\step{2}\input{diagrams/backtrace/scanstack.tex}}%
      \onslide*<4>{\providecommand\step{3}\input{diagrams/backtrace/scanstack.tex}}%
      \onslide*<5>{\providecommand\step{4}\input{diagrams/backtrace/scanstack.tex}}%
      \onslide*<6>{\providecommand\step{5}\input{diagrams/backtrace/scanstack.tex}}%
    \end{column}
  \end{columns}
\end{frame}

% Duplicate of previous-previous slide
\begin{frame}{Backtraces}{Continuing on \funcname{caml\_stash\_backtrace}}
  \begin{columns}[c]
    \begin{column}{0.5\textwidth}
      \minipage[c][0.75\textheight][s]{\columnwidth}
      \begin{itemize}
          \color{gray}
        \item A C function provided by the runtime (specific to native code)
        \item It scans the stack, between the stack pointer at the raising point up to the next trap, in order to collect the names of the detected function frames
        \item It uses the same technique and information as the Garbage Collector (GC) uses to scan the values live on the stack
      \end{itemize}
      \begin{itemize}
        \item For each function frame detected, store a pointer to the \typename{frame\_descr} in the \localname{domain\_state->backtrace\_buffer} array, effectively constructing the backtrace
      \end{itemize}
      \onslide<2->{
        \begin{itemize}
            \smallskip
          \item Constructed backtrace:
            \funcname{definitely\_raise},
            \onslide<3->{\funcname{probably\_raise}}
        \end{itemize}
      }
      \vfill
      \mintocaml[firstline=9,lastline=13,highlightlines=12]{../src/backtrace/backtrace.ml}
      \endminipage
    \end{column}
    \begin{column}{0.5\textwidth}
      \raggedleft
      \onslide*<1>{\listStashBacktrace[highlightlines={114-115}]{}}
      \onslide*<2>{\providecommand\step{3}\input{diagrams/backtrace/backtrace.tex}}%
      \onslide*<3>{\providecommand\step{4}\input{diagrams/backtrace/backtrace.tex}}%
    \end{column}
  \end{columns}
\end{frame}

\begin{frame}{Backtraces}{Back to \funcname{caml\_raise\_exn}}
  \begin{columns}[c]
    \begin{column}{0.5\textwidth}
      \minipage[c][0.66\textheight][s]{\columnwidth}
      \begin{itemize}\color{gray}
        \item Checks wheteher exceptions backtrace should be recorded, by checking the \funcarg{domain\_state->backtrace\_active} value
        \item Switches to the C stack to be able to execute C code
        \item Calls \funcname{caml\_stash\_backtrace}, to collect the backtrace up to the next trap
      \end{itemize}
      \onslide*<2->{
        \begin{itemize}
          \item<2-> Executes the exception handler in \funcname{probably\_raise} with \funcname{RESTORE\_EXN\_HANDLER\_OCAML}
            \begin{itemize}
              \item Set \regname{RSP} to \localname{domain\_state->exn\_handler} value
              \item Restore \localname{domain\_state->exn\_handler} from trap
              \item Jump to the handler code with \funcname{ret}
            \end{itemize}
        \end{itemize}
      }
      \vfill
      \onslide*<1>{\mintocaml[firstline=9,lastline=13,highlightlines=12]{../src/backtrace/backtrace.ml}}
      \onslide*<2->{\mintocaml[firstline=15,lastline=21,highlightlines={19-21}]{../src/backtrace/backtrace.ml}}
      \endminipage
    \end{column}
    \begin{column}{0.5\textwidth}
      \centering
      \onslide*<1>{
        \mintc[firstline=190,lastline=192]{../ocaml/runtime/amd64.S}
        \mintc[firstline=234,lastline=237]{../ocaml/runtime/amd64.S}
      }
      \onslide*<2>{
        \mintc[firstline=190,lastline=192,highlightlines={191-192}]{../ocaml/runtime/amd64.S}
        \mintc[firstline=234,lastline=237,highlightlines={235-237}]{../ocaml/runtime/amd64.S}
      }
      \onslide*<1>{\listCamlRaiseExn[highlightlines=720]{}}
      \onslide*<2>{\listCamlRaiseExn[highlightlines=722]{}}
      \onslide*<3->{\providecommand\step{5}\input{diagrams/backtrace/backtrace.tex}}
    \end{column}
  \end{columns}
\end{frame}

\begin{frame}{Backtraces}{\funcname{caml\_probably\_raise}'s handler doesn't catch \typename{ExnC}}
  \begin{columns}[c]
    \begin{column}{0.45\textwidth}
      \minipage[c][0.75\textheight][s]{\columnwidth}
      \begin{itemize}
        \item<1-> \funcname{match \dots with}\footnotemark pattern matching can also match exceptions
        \item<1-> The CMM of \funcname{probably\_raise} shows that this \funcname{match \dots with} is implemented with a pattern matching and a \funcname{try \dots with}
        \item<1-> But this one only matches on \typename{ExnA} exception
        \item<2-> Since the pattern matching fails to catch the exception, \funcname{reraise} is called with our current \typename{ExnC} exception
        \item<3-> Disassembly shows that \funcname{reraise} is actually a call to \funcname{caml\_reraise\_exn}, which is also an assembly function provided by the runtime
      \end{itemize}
      \vfill
      \mintocaml[firstline=15,lastline=21,highlightlines={18-21}]{../src/backtrace/backtrace.ml}
      \endminipage
    \end{column}
    \begin{column}{0.55\textwidth}
      \centering
      \onslide*<1>{\mintcmm[highlightlines={10-18}]{../src/backtrace/camlBacktrace\_\_probably\_raise\_351.cmm}}
      \onslide*<2>{\listcmm[highlightlines=18]{../src/backtrace/camlBacktrace\_\_probably\_raise_351.cmm}{CMM of probably\_raise}}
      \onslide*<3>{\mintobjdump[firstline=5,lastline=35,highlightlines=31]{../src/backtrace/camlBacktrace\_\_probably\_raise_351.objdump}}
    \end{column}
  \end{columns}
  \only<1>{\footnotetext[1]{https://blog.janestreet.com/pattern-matching-and-exception-handling-unite/}}
\end{frame}

\newcommand\listCamlReraiseExn[1][]{
  \listasm[firstline=727,lastline=734,#1]{../ocaml/runtime/amd64.S}{runtime/amd64.S:727}}

\begin{frame}{Backtraces}{Introducing \funcname{caml\_reraise\_exn}}
  \begin{columns}[c]
    \begin{column}{0.5\textwidth}
      \minipage[c][0.66\textheight][s]{\columnwidth}
      \begin{itemize}
        \item<1-> The exception have to be reraised to the next handler
        \item<2-> First, checks if the backtrace should be collected with \funcarg{domain\_state->backtrace\_active}
        \only<3>{
        \begin{itemize}
            \item If not, behaves like a \funcname{raise\_notrace} with \funcname{RESTORE\_EXN\_HANDLER\_OCAML}
        \end{itemize}
        }
        \item<4-> Jumps into \funcname{caml\_raise\_exn}
        \item<5-> Calls \funcname{caml\_stash\_backtrace} again, to scan the next part of the stack: from the trap just pop-ed to the next trap
      \end{itemize}
      \vfill
      \mintocaml[firstline=15,lastline=21,highlightlines={18-21}]{../src/backtrace/backtrace.ml}
      \endminipage
    \end{column}
    \begin{column}{0.5\textwidth}
      \raggedleft
      \onslide*<1>{\listCamlReraiseExn{}}
      \onslide*<2>{\listCamlReraiseExn[highlightlines=729]{}}
      \onslide*<3>{
        \mintc[firstline=190,lastline=192,highlightlines={191-192}]{../ocaml/runtime/amd64.S}
        \mintc[firstline=234,lastline=237,highlightlines={235-237}]{../ocaml/runtime/amd64.S}
        \medskip
        \listCamlReraiseExn[highlightlines=731]{}
      }
      \onslide*<4-5>{
        % TODO refactor with \listCamlReraiseExn
        \mintasm[firstline=727,lastline=734,highlightlines=730]{../ocaml/runtime/amd64.S}
        \medskip
      }
      \onslide*<4>{\listCamlRaiseExn[highlightlines=711]{}}
      \onslide*<5>{\listCamlRaiseExn[highlightlines=720]{}}
    \end{column}
  \end{columns}
\end{frame}

\begin{frame}{Backtraces}{Backtrace collection continues}
  \begin{columns}[c]
    \begin{column}{0.55\textwidth}
      \minipage[c][0.75\textheight][s]{\columnwidth}
      \begin{itemize}
        \item<1-> \funcname{caml\_stash\_backtrace} scans the older part of the stack, up to the next trap
        \item<6-> Controls jumps to the nested exception handler in \funcname{maybe\_raise}, which matches only on \typename{ExnB}
        \item<7-> \funcname{caml\_reraise\_exn} is called again, backtrace collection resumes with \funcname{caml\_stash\_backtrace}
      \end{itemize}
      \medskip
      \begin{itemize}
        \item<2-> Constructed backtrace:
        \begin{itemize}
          \item <2-> \funcname{definitely\_raise}
          \item <2-> \funcname{probably\_raise}
          \item <3-> \funcname{probably\_raise} (re-raise)
          \item <4-> \funcname{maybe\_raise}
          \item <5-> \funcname{main}
          \item <8-> \funcname{main} (re-raise)
        \end{itemize}
      \end{itemize}
      \vfill
      \onslide*<1-5>{\mintocaml[firstline=15,lastline=21]{../src/backtrace/backtrace.ml}}
      \onslide*<6>{\mintocaml[firstline=28,lastline=34,highlightlines=31]{../src/backtrace/backtrace.ml}}
      \onslide*<7->{\mintocaml[firstline=28,lastline=34,highlightlines=33]{../src/backtrace/backtrace.ml}}
      \endminipage
    \end{column}
    \begin{column}{0.45\textwidth}
      \raggedleft
      \onslide*<1>{\providecommand\step{6}\input{diagrams/backtrace/backtrace.tex}}%
      \onslide*<2>{\providecommand\step{6}\input{diagrams/backtrace/backtrace.tex}}%
      \onslide*<3>{\providecommand\step{7}\input{diagrams/backtrace/backtrace.tex}}%
      \onslide*<4>{\providecommand\step{8}\input{diagrams/backtrace/backtrace.tex}}%
      \onslide*<5>{\providecommand\step{9}\input{diagrams/backtrace/backtrace.tex}}%
      \onslide*<6>{\providecommand\step{10}\input{diagrams/backtrace/backtrace.tex}}%
      \onslide*<7>{\providecommand\step{11}\input{diagrams/backtrace/backtrace.tex}}%
      \onslide*<8>{\providecommand\step{12}\input{diagrams/backtrace/backtrace.tex}}%
      \onslide*<9>{\providecommand\step{13}\input{diagrams/backtrace/backtrace.tex}}%
    \end{column}
  \end{columns}
\end{frame}

\begin{frame}{Backtraces}{Printing the backtrace}
  \begin{columns}[c]
    \begin{column}{0.45\textwidth}
      \minipage[c][0.66\textheight][s]{\columnwidth}
      \begin{itemize}
        \item<1-> The exception backtrace can now be printed to \funcarg{stdout} with \funcname{Printexc.print_backtrace}
        \item<2-> First the exception backtrace is retrieved into an OCaml array with \funcname{get_raw_backtrace }
        \item<3-> Which is implemented as the \funcname{caml_get_exception_raw_backtrace} C function in the runtime
        \item<4-> And finally printed with \funcname{print_exception_backtrace}
      \end{itemize}
      \vfill
      \mintocaml[firstline=28,lastline=34,highlightlines=33]{../src/backtrace/backtrace.ml}
      \endminipage
    \end{column}
    \begin{column}{0.55\textwidth}
      \onslide*<1,2>{
        \mintocaml[firstline=100,lastline=101]{../ocaml/stdlib/printexc.ml}
      }
      \only<1>{
        \listocaml[firstline=170,lastline=172]{../ocaml/stdlib/printexc.ml}{stdlib/printexc.ml:170}
      }
      \only<2>{
        \listocaml[firstline=170,lastline=172,highlightlines=172]{../ocaml/stdlib/printexc.ml}{stdlib/printexc.ml:170}
      }
      \only<3>{
        \mintc[firstline=154,lastline=156]{../ocaml/runtime/backtrace.c}
        \listc[firstline=171,lastline=192,highlightlines={184-187}]{../ocaml/runtime/backtrace.c}{runtime/backtrace.c:154}
      }
      \only<4>{
        \listocaml[firstline=155,lastline=165]{../ocaml/stdlib/printexc.ml}{stdlib/printexc.ml:55}
      }
    \end{column}
  \end{columns}
\end{frame}


\subsection{The multiple kinds of raise}
\frameSubsection{}{}

\begin{frame}{Backtraces}{When backtraces are not needed}
  \begin{columns}[c]
    \begin{column}{0.45\textwidth}
      \minipage[c][0.66\textheight][s]{\columnwidth}
      \begin{itemize}
        \item<1-> What if the backtrace for an exception is never needed?
        \item<2-> It's possible to raise the exception explicitly with \funcname{raise\_notrace} to make raising as cheap as possible
        \item<3-> An example of use in the \funcname{dynlink} library of the OCaml compiler
      \end{itemize}
      \vfill
      \onslide<3->{
      \listocaml[firstline=205,lastline=209,highlightlines=207]{../ocaml/otherlibs/dynlink/dynlink\_compilerlibs/misc.ml}{otherlibs/dynlink/dynlink\_compilerlibs/misc.ml:205}
      }
      \endminipage
    \end{column}
    \begin{column}{0.55\textwidth}
    \only<4>{
      \mintcmm[highlightlines=3]{../src/notrace/camlNotrace\_\_fun_347.cmm}
      \listcmm{../src/notrace/camlNotrace\_\_all\_somes\_327.cmm}{all\_somes}
    }
    \onslide*<5->{
      \listobjdump[highlightlines={6-9}]{../src/notrace/camlNotrace\_\_fun_347.objdump}{anonymous function from \funcname{all\_somes}}
    }
    \end{column}
  \end{columns}
\end{frame}

\begin{frame}{Backtraces}{Summary table of the multiple kinds of raise}
  \begin{itemize}
    \item When debugging information is enabled and if the backtrace of an exception isn't needed, it's possible to raise with \funcname{raise\_notrace} for performance
    \item When debugging information is \emph{not} enabled, the compiler optimises \funcname{raise} calls to inline assembly (same effect as using \funcname{raise\_notrace})
  \end{itemize}
  \bigskip
  \centering
  \begin{tabular}{| l | c | c | c | c |}
    \hline
    \parbox{10em}{Debug information \\ (\funcarg{-g} flag)}  & \multicolumn{2}{|c|}{Disabled}                                     & \multicolumn{2}{|c|}{Enabled} \\
    \hline
    Raise kind                                               & \funcname{raise} & \funcname{raise\_notrace}                       & \funcname{raise} & \funcname{raise\_notrace} \\
    \hline
    Compilation                                              & \parbox{8em}{inline assembly\\ (optimization)} & inline assembly   & \funcname{caml\_raise\_exn} & inline assembly \\
    \hline
  \end{tabular}
\end{frame}


%
% Takeaway #5
%
\subsection*{Takeaway \#5}
\frameSubsectionTakeaway{}
\againframe<5>{takeaway}
}%
      \onslide*<6>{\providecommand\step{10}%
% Backtraces
%
\subsection{One advantage of raising with debugging information}
\frameSubsection{}{}

\begin{frame}{Backtraces}{Debugging information \& source code}
  \begin{columns}[c]
    \begin{column}{0.5\textwidth}
      \begin{itemize}
        \item Raising an exception is fun and games, but how do we know where the exception originated? $\rightarrow$ With the backtrace associated with the exception!
        \item In order to get a backtrace from the point where an exception was raised, it's necessary to compile with debugging information enabled (\funcarg{-g} flag for the compiler)
        \item Same example as in the `Nested handlers' section
      \end{itemize}
    \end{column}
    \begin{column}{0.5\textwidth}
      \listocaml[firstline=9,lastline=34]{../src/backtrace/backtrace.ml}{backtrace/backtrace.ml}
    \end{column}
  \end{columns}
\end{frame}

\begin{frame}{Backtraces}{CMM and disassembly of \funcname{definitely\_raise}}
  \begin{columns}[c]
    \begin{column}{0.5\textwidth}
      \minipage[c][0.66\textheight][s]{\columnwidth}
      \begin{itemize}
        \item<1-> An effect of compiling with debugging information (\funcarg{-g}): the CMM no longer uses \funcname{raise\_notrace} to raise the exception but \funcname{raise} instead
        \item<2-> Disassembly shows that CMM \funcname{raise} is actually a call to \funcname{caml\_raise\_exn}, a function in assembler provided by the runtime
      \end{itemize}
      \vfill
      \mintocaml[firstline=9,lastline=13,highlightlines=12]{../src/backtrace/backtrace.ml}
      \endminipage
    \end{column}
    \begin{column}{0.5\textwidth}
      \onslide*<1>{
        \mintcmm[highlightlines=5]{../src/backtrace/camlBacktrace\_\_definitely\_raise\_347.cmm}
      }
      \onslide*<2>{
        \mintobjdump[firstline=2,highlightlines=14]{../src/backtrace/camlBacktrace\_\_definitely\_raise\_347.objdump}
      }
    \end{column}
  \end{columns}
\end{frame}

\begin{frame}{Backtraces}{Introducing \funcname{caml\_raise\_exn}}
  \begin{columns}[c]
    \begin{column}{0.5\textwidth}
      \minipage[c][0.66\textheight][s]{\columnwidth}
      \onslide*<1>{
        \begin{itemize}
          \item Raising the exception is performed using \funcname{caml\_raise\_exn} which is
            \begin{itemize}
              \item An assembly function provided by the runtime
              \item Architecture specific
            \end{itemize}
          \item Let's see what it does differently from \funcname{raise\_notrace}\dots
          \item We are about to raise \typename{ExnC} with \funcname{caml\_raise\_exn} from \funcname{definitely\_raise}
        \end{itemize}
      }
      \onslide*<2>{
        \mintobjdump[firstline=2,highlightlines=14]{../src/backtrace/camlBacktrace\_\_definitely\_raise\_347.objdump}
      }
      \vfill
      \mintocaml[firstline=9,lastline=13,highlightlines=12]{../src/backtrace/backtrace.ml}
      \endminipage
    \end{column}
    \begin{column}{0.5\textwidth}
      \centering
      \providecommand\step{1}\input{diagrams/backtrace/backtrace.tex}
    \end{column}
  \end{columns}
\end{frame}

\begin{frame}{Backtraces}{But before that, we need more fields from \typename{caml\_domain\_state}}
  \begin{columns}
    \begin{column}{0.5\textwidth}
      \begin{itemize}
        \item Remember \typename{caml\_domain\_state} the per-domain runtime state?
        \item Some other members are of interest to us now\dots
        \item \localname{backtrace\_active} is used to know if the runtime should collect backtraces
        \item \localname{backtrace\_active} can be set by
          \begin{itemize}
            \item The \funcarg{b} flag in the \funcname{OCAMLRUNPARAM} environment variable
            \item \mintinline{ocaml}{Printexc.record_backtrace : bool -> unit} \footnotemark
          \end{itemize}
      \end{itemize}
    \end{column}
    \begin{column}{0.5\textwidth}
      \begin{listing}
        \mintc[highlightlines={9-11}]{src/ocaml/domain\_state\_backtrace.h}
        \caption{runtime/caml/domain\_state.h}
      \end{listing}
    \end{column}
  \end{columns}
  \footnotetext[1]{https://v2.ocaml.org/api/Printexc.html}
\end{frame}

\newcommand\listCamlRaiseExn[1][]{
  \listasm[firstline=702,lastline=725,#1]{../ocaml/runtime/amd64.S}{runtime/amd64.S:702}}

\begin{frame}{Backtraces}{Walk through \funcname{caml\_raise\_exn}}
  \begin{columns}[T]
    \begin{column}{0.5\textwidth}
      \begin{itemize}
        \item<1-> First checks whether exceptions backtrace should be recorded
        \item<1-> By checking the \funcarg{domain\_state->backtrace\_active} value
        \item<2-> If not, acts as \funcname{raise\_notrace} with \funcname{RESTORE\_EXN\_HANDLER\_OCAML}
          \begin{itemize}
            \item Set \regname{RSP} to \localname{domain\_state->exn\_handler} value
            \item Restore \localname{domain\_state->exn\_handler} from trap
            \item Jump with \funcname{ret} (same effect as using \funcname{pop} and \funcname{jmp})
          \end{itemize}
        \item<3-> Otherwise, switches to the C stack to be able to execute C code
        \item<4-> Calls \funcname{caml\_stash\_backtrace}, a C function provided by the runtime\ignorespaces
          \onslide<6->{, with
            \begin{itemize}
              \item \funcarg{exn}: exception value
              \item \funcarg{pc}: code address of the raising point
              \item \funcarg{sp}: stack address of the raising function
              \item \funcarg{trapsp}: stack address of the next handler
            \end{itemize}
          }
      \end{itemize}
      \bigskip
      \mintocaml[firstline=9,lastline=13,highlightlines=12]{../src/backtrace/backtrace.ml}
    \end{column}
    \begin{column}{0.5\textwidth}
      \raggedleft
      \onslide*<1,3-4>{
        \mintc[firstline=190,lastline=192]{../ocaml/runtime/amd64.S}
        \mintc[firstline=234,lastline=237]{../ocaml/runtime/amd64.S}
      }
      \onslide*<2>{
        \mintc[firstline=190,lastline=192,highlightlines={191-192}]{../ocaml/runtime/amd64.S}
        \mintc[firstline=234,lastline=237,highlightlines={235-237}]{../ocaml/runtime/amd64.S}
      }
      \onslide*<1>{\listCamlRaiseExn[highlightlines=705]{}}%
      \onslide*<2>{\listCamlRaiseExn[highlightlines={707-708}]{}}%
      \onslide*<3>{\listCamlRaiseExn[highlightlines={712-714}]{}}%
      \onslide*<4>{\listCamlRaiseExn[highlightlines={715-720}]{}}%
      \onslide*<5>{\providecommand\step{1}\input{diagrams/backtrace/backtrace.tex}}%
      \onslide*<6>{\providecommand\step{2}\input{diagrams/backtrace/backtrace.tex}}%
    \end{column}
  \end{columns}
\end{frame}

\newcommand\listStashBacktrace[1][]{
  \listc[firstline=91,lastline=120,#1]{../ocaml/runtime/backtrace\_nat.c}{runtime/backtrace\_nat.c:91}}

\begin{frame}{Backtraces}{Introducing \funcname{caml\_stash\_backtrace}}
  \begin{columns}[c]
    \begin{column}{0.5\textwidth}
      \minipage[c][0.66\textheight][s]{\columnwidth}
      \begin{itemize}
        \item<1-> A C function provided by the runtime (specific to native code)
        \item<2-> It scans the stack, between the stack pointer at the raising point up to the next trap, in order to collect the names of the detected function frames
          \onslide*<2>{
            \begin{itemize}
              \item And more information, like line number, column number...
            \end{itemize}
          }
        \item<3-> It uses the same technique and information as the Garbage Collector (GC) uses to scan the values live on the stack
          \onslide*<4>{
            \begin{itemize}
              \item \typename{frame\_descr} are at the core of stack unwinding
            \end{itemize}
          }
      \end{itemize}
      \vfill
      \mintocaml[firstline=9,lastline=13,highlightlines=12]{../src/backtrace/backtrace.ml}
      \endminipage
    \end{column}
    \begin{column}{0.5\textwidth}
      \onslide*<1>{\listStashBacktrace[highlightlines=91]{}}
      \onslide*<2>{\listStashBacktrace[highlightlines={91,117-118}]{}}
      \onslide*<3->{\listStashBacktrace[highlightlines={94,106,109-110}]{}}
    \end{column}
  \end{columns}
\end{frame}

\newcommand\listFrameDescr[1][]{
  \listocaml[firstline=111,lastline=124,#1]{../ocaml/asmcomp/emitaux.ml}{asmcomp/emitaux.ml:111}}
\newcommand\listDebugInfo[1][]{
  \listocaml[firstline=38,lastline=49,#1]{../ocaml/lambda/debuginfo.mli}{lambda/debuginfo.mli:38}}

\begin{frame}{Backtraces}{\typename{frame\_descr} and \typename{frame\_debuginfo} in a nutshell}
  \begin{columns}[c]
    \begin{column}{0.5\textwidth}
      \begin{itemize}
        \item<1-> For every return address in an OCaml program, there is a associated \typename{frame\_descr}
        \item<2-> A \typename{frame\_descr} specifies
        \begin{itemize}
          \item<2-> The size of the function stack frame
          \item<3-> Where live values are on the stack (so that the GC can mark them as live and prevent from being swept)
        \end{itemize}
        \item<4-> When compiled with debug information, \typename{frame\_descr} will be linked to a \typename{frame\_debuginfo}
        \begin{itemize}
          \item<5-> Contains information about the position in the source code (file name, line and column number)
        \end{itemize}
      \end{itemize}
    \end{column}
    \begin{column}{0.5\textwidth}
        \onslide*<1>{\listFrameDescr[highlightlines={118-122}]{}}
        \onslide*<2>{\listFrameDescr[highlightlines={120}]{}}
        \onslide*<3>{\listFrameDescr[highlightlines={121}]{}}
        \onslide*<4->{\listFrameDescr[highlightlines={122}]{}}
        \medskip
        \onslide*<1-3>{\listDebugInfo{}}
        \onslide*<4>{\listDebugInfo[highlightlines={38,49}]{}}
        \onslide*<5>{\listDebugInfo[highlightlines={39-46}]{}}
    \end{column}
  \end{columns}
\end{frame}

\begin{frame}{Backtraces}{Scanning the stack with \typename{frame\_descr}}
  \begin{columns}[c]
    \begin{column}{0.5\textwidth}
      \begin{itemize}
        \item Given the return address of the code that raises the exception by calling \funcname{caml\_raise\_exn} (\funcarg{pc} argument of \funcname{caml\_stash\_backtrace})
        \item And the position of the stack pointer (\funcarg{sp} argument of \funcname{caml\_stash\_backtrace})
        \item The frame size given by \typename{frame\_descr}.\funcarg{frame\_size}, allows to find the end of the previous frame
        \item And the return address of the caller of this function
        \bigskip
        \onslide<3->{
        \item Note that traps (here the reddish \funcname{ExnA}, \funcname{ExnB} and \funcname{ExnC} blocks) belong to the frame that installed them, and so the frame size changes when traps are removed
        }
      \end{itemize}
    \end{column}
    \begin{column}{0.5\textwidth}
      \raggedleft
      \onslide*<1>{\providecommand\step{2}\input{diagrams/backtrace/backtrace.tex}}%
      \onslide*<2>{\providecommand\step{1}\input{diagrams/backtrace/scanstack.tex}}%
      \onslide*<3>{\providecommand\step{2}\input{diagrams/backtrace/scanstack.tex}}%
      \onslide*<4>{\providecommand\step{3}\input{diagrams/backtrace/scanstack.tex}}%
      \onslide*<5>{\providecommand\step{4}\input{diagrams/backtrace/scanstack.tex}}%
      \onslide*<6>{\providecommand\step{5}\input{diagrams/backtrace/scanstack.tex}}%
    \end{column}
  \end{columns}
\end{frame}

% Duplicate of previous-previous slide
\begin{frame}{Backtraces}{Continuing on \funcname{caml\_stash\_backtrace}}
  \begin{columns}[c]
    \begin{column}{0.5\textwidth}
      \minipage[c][0.75\textheight][s]{\columnwidth}
      \begin{itemize}
          \color{gray}
        \item A C function provided by the runtime (specific to native code)
        \item It scans the stack, between the stack pointer at the raising point up to the next trap, in order to collect the names of the detected function frames
        \item It uses the same technique and information as the Garbage Collector (GC) uses to scan the values live on the stack
      \end{itemize}
      \begin{itemize}
        \item For each function frame detected, store a pointer to the \typename{frame\_descr} in the \localname{domain\_state->backtrace\_buffer} array, effectively constructing the backtrace
      \end{itemize}
      \onslide<2->{
        \begin{itemize}
            \smallskip
          \item Constructed backtrace:
            \funcname{definitely\_raise},
            \onslide<3->{\funcname{probably\_raise}}
        \end{itemize}
      }
      \vfill
      \mintocaml[firstline=9,lastline=13,highlightlines=12]{../src/backtrace/backtrace.ml}
      \endminipage
    \end{column}
    \begin{column}{0.5\textwidth}
      \raggedleft
      \onslide*<1>{\listStashBacktrace[highlightlines={114-115}]{}}
      \onslide*<2>{\providecommand\step{3}\input{diagrams/backtrace/backtrace.tex}}%
      \onslide*<3>{\providecommand\step{4}\input{diagrams/backtrace/backtrace.tex}}%
    \end{column}
  \end{columns}
\end{frame}

\begin{frame}{Backtraces}{Back to \funcname{caml\_raise\_exn}}
  \begin{columns}[c]
    \begin{column}{0.5\textwidth}
      \minipage[c][0.66\textheight][s]{\columnwidth}
      \begin{itemize}\color{gray}
        \item Checks wheteher exceptions backtrace should be recorded, by checking the \funcarg{domain\_state->backtrace\_active} value
        \item Switches to the C stack to be able to execute C code
        \item Calls \funcname{caml\_stash\_backtrace}, to collect the backtrace up to the next trap
      \end{itemize}
      \onslide*<2->{
        \begin{itemize}
          \item<2-> Executes the exception handler in \funcname{probably\_raise} with \funcname{RESTORE\_EXN\_HANDLER\_OCAML}
            \begin{itemize}
              \item Set \regname{RSP} to \localname{domain\_state->exn\_handler} value
              \item Restore \localname{domain\_state->exn\_handler} from trap
              \item Jump to the handler code with \funcname{ret}
            \end{itemize}
        \end{itemize}
      }
      \vfill
      \onslide*<1>{\mintocaml[firstline=9,lastline=13,highlightlines=12]{../src/backtrace/backtrace.ml}}
      \onslide*<2->{\mintocaml[firstline=15,lastline=21,highlightlines={19-21}]{../src/backtrace/backtrace.ml}}
      \endminipage
    \end{column}
    \begin{column}{0.5\textwidth}
      \centering
      \onslide*<1>{
        \mintc[firstline=190,lastline=192]{../ocaml/runtime/amd64.S}
        \mintc[firstline=234,lastline=237]{../ocaml/runtime/amd64.S}
      }
      \onslide*<2>{
        \mintc[firstline=190,lastline=192,highlightlines={191-192}]{../ocaml/runtime/amd64.S}
        \mintc[firstline=234,lastline=237,highlightlines={235-237}]{../ocaml/runtime/amd64.S}
      }
      \onslide*<1>{\listCamlRaiseExn[highlightlines=720]{}}
      \onslide*<2>{\listCamlRaiseExn[highlightlines=722]{}}
      \onslide*<3->{\providecommand\step{5}\input{diagrams/backtrace/backtrace.tex}}
    \end{column}
  \end{columns}
\end{frame}

\begin{frame}{Backtraces}{\funcname{caml\_probably\_raise}'s handler doesn't catch \typename{ExnC}}
  \begin{columns}[c]
    \begin{column}{0.45\textwidth}
      \minipage[c][0.75\textheight][s]{\columnwidth}
      \begin{itemize}
        \item<1-> \funcname{match \dots with}\footnotemark pattern matching can also match exceptions
        \item<1-> The CMM of \funcname{probably\_raise} shows that this \funcname{match \dots with} is implemented with a pattern matching and a \funcname{try \dots with}
        \item<1-> But this one only matches on \typename{ExnA} exception
        \item<2-> Since the pattern matching fails to catch the exception, \funcname{reraise} is called with our current \typename{ExnC} exception
        \item<3-> Disassembly shows that \funcname{reraise} is actually a call to \funcname{caml\_reraise\_exn}, which is also an assembly function provided by the runtime
      \end{itemize}
      \vfill
      \mintocaml[firstline=15,lastline=21,highlightlines={18-21}]{../src/backtrace/backtrace.ml}
      \endminipage
    \end{column}
    \begin{column}{0.55\textwidth}
      \centering
      \onslide*<1>{\mintcmm[highlightlines={10-18}]{../src/backtrace/camlBacktrace\_\_probably\_raise\_351.cmm}}
      \onslide*<2>{\listcmm[highlightlines=18]{../src/backtrace/camlBacktrace\_\_probably\_raise_351.cmm}{CMM of probably\_raise}}
      \onslide*<3>{\mintobjdump[firstline=5,lastline=35,highlightlines=31]{../src/backtrace/camlBacktrace\_\_probably\_raise_351.objdump}}
    \end{column}
  \end{columns}
  \only<1>{\footnotetext[1]{https://blog.janestreet.com/pattern-matching-and-exception-handling-unite/}}
\end{frame}

\newcommand\listCamlReraiseExn[1][]{
  \listasm[firstline=727,lastline=734,#1]{../ocaml/runtime/amd64.S}{runtime/amd64.S:727}}

\begin{frame}{Backtraces}{Introducing \funcname{caml\_reraise\_exn}}
  \begin{columns}[c]
    \begin{column}{0.5\textwidth}
      \minipage[c][0.66\textheight][s]{\columnwidth}
      \begin{itemize}
        \item<1-> The exception have to be reraised to the next handler
        \item<2-> First, checks if the backtrace should be collected with \funcarg{domain\_state->backtrace\_active}
        \only<3>{
        \begin{itemize}
            \item If not, behaves like a \funcname{raise\_notrace} with \funcname{RESTORE\_EXN\_HANDLER\_OCAML}
        \end{itemize}
        }
        \item<4-> Jumps into \funcname{caml\_raise\_exn}
        \item<5-> Calls \funcname{caml\_stash\_backtrace} again, to scan the next part of the stack: from the trap just pop-ed to the next trap
      \end{itemize}
      \vfill
      \mintocaml[firstline=15,lastline=21,highlightlines={18-21}]{../src/backtrace/backtrace.ml}
      \endminipage
    \end{column}
    \begin{column}{0.5\textwidth}
      \raggedleft
      \onslide*<1>{\listCamlReraiseExn{}}
      \onslide*<2>{\listCamlReraiseExn[highlightlines=729]{}}
      \onslide*<3>{
        \mintc[firstline=190,lastline=192,highlightlines={191-192}]{../ocaml/runtime/amd64.S}
        \mintc[firstline=234,lastline=237,highlightlines={235-237}]{../ocaml/runtime/amd64.S}
        \medskip
        \listCamlReraiseExn[highlightlines=731]{}
      }
      \onslide*<4-5>{
        % TODO refactor with \listCamlReraiseExn
        \mintasm[firstline=727,lastline=734,highlightlines=730]{../ocaml/runtime/amd64.S}
        \medskip
      }
      \onslide*<4>{\listCamlRaiseExn[highlightlines=711]{}}
      \onslide*<5>{\listCamlRaiseExn[highlightlines=720]{}}
    \end{column}
  \end{columns}
\end{frame}

\begin{frame}{Backtraces}{Backtrace collection continues}
  \begin{columns}[c]
    \begin{column}{0.55\textwidth}
      \minipage[c][0.75\textheight][s]{\columnwidth}
      \begin{itemize}
        \item<1-> \funcname{caml\_stash\_backtrace} scans the older part of the stack, up to the next trap
        \item<6-> Controls jumps to the nested exception handler in \funcname{maybe\_raise}, which matches only on \typename{ExnB}
        \item<7-> \funcname{caml\_reraise\_exn} is called again, backtrace collection resumes with \funcname{caml\_stash\_backtrace}
      \end{itemize}
      \medskip
      \begin{itemize}
        \item<2-> Constructed backtrace:
        \begin{itemize}
          \item <2-> \funcname{definitely\_raise}
          \item <2-> \funcname{probably\_raise}
          \item <3-> \funcname{probably\_raise} (re-raise)
          \item <4-> \funcname{maybe\_raise}
          \item <5-> \funcname{main}
          \item <8-> \funcname{main} (re-raise)
        \end{itemize}
      \end{itemize}
      \vfill
      \onslide*<1-5>{\mintocaml[firstline=15,lastline=21]{../src/backtrace/backtrace.ml}}
      \onslide*<6>{\mintocaml[firstline=28,lastline=34,highlightlines=31]{../src/backtrace/backtrace.ml}}
      \onslide*<7->{\mintocaml[firstline=28,lastline=34,highlightlines=33]{../src/backtrace/backtrace.ml}}
      \endminipage
    \end{column}
    \begin{column}{0.45\textwidth}
      \raggedleft
      \onslide*<1>{\providecommand\step{6}\input{diagrams/backtrace/backtrace.tex}}%
      \onslide*<2>{\providecommand\step{6}\input{diagrams/backtrace/backtrace.tex}}%
      \onslide*<3>{\providecommand\step{7}\input{diagrams/backtrace/backtrace.tex}}%
      \onslide*<4>{\providecommand\step{8}\input{diagrams/backtrace/backtrace.tex}}%
      \onslide*<5>{\providecommand\step{9}\input{diagrams/backtrace/backtrace.tex}}%
      \onslide*<6>{\providecommand\step{10}\input{diagrams/backtrace/backtrace.tex}}%
      \onslide*<7>{\providecommand\step{11}\input{diagrams/backtrace/backtrace.tex}}%
      \onslide*<8>{\providecommand\step{12}\input{diagrams/backtrace/backtrace.tex}}%
      \onslide*<9>{\providecommand\step{13}\input{diagrams/backtrace/backtrace.tex}}%
    \end{column}
  \end{columns}
\end{frame}

\begin{frame}{Backtraces}{Printing the backtrace}
  \begin{columns}[c]
    \begin{column}{0.45\textwidth}
      \minipage[c][0.66\textheight][s]{\columnwidth}
      \begin{itemize}
        \item<1-> The exception backtrace can now be printed to \funcarg{stdout} with \funcname{Printexc.print_backtrace}
        \item<2-> First the exception backtrace is retrieved into an OCaml array with \funcname{get_raw_backtrace }
        \item<3-> Which is implemented as the \funcname{caml_get_exception_raw_backtrace} C function in the runtime
        \item<4-> And finally printed with \funcname{print_exception_backtrace}
      \end{itemize}
      \vfill
      \mintocaml[firstline=28,lastline=34,highlightlines=33]{../src/backtrace/backtrace.ml}
      \endminipage
    \end{column}
    \begin{column}{0.55\textwidth}
      \onslide*<1,2>{
        \mintocaml[firstline=100,lastline=101]{../ocaml/stdlib/printexc.ml}
      }
      \only<1>{
        \listocaml[firstline=170,lastline=172]{../ocaml/stdlib/printexc.ml}{stdlib/printexc.ml:170}
      }
      \only<2>{
        \listocaml[firstline=170,lastline=172,highlightlines=172]{../ocaml/stdlib/printexc.ml}{stdlib/printexc.ml:170}
      }
      \only<3>{
        \mintc[firstline=154,lastline=156]{../ocaml/runtime/backtrace.c}
        \listc[firstline=171,lastline=192,highlightlines={184-187}]{../ocaml/runtime/backtrace.c}{runtime/backtrace.c:154}
      }
      \only<4>{
        \listocaml[firstline=155,lastline=165]{../ocaml/stdlib/printexc.ml}{stdlib/printexc.ml:55}
      }
    \end{column}
  \end{columns}
\end{frame}


\subsection{The multiple kinds of raise}
\frameSubsection{}{}

\begin{frame}{Backtraces}{When backtraces are not needed}
  \begin{columns}[c]
    \begin{column}{0.45\textwidth}
      \minipage[c][0.66\textheight][s]{\columnwidth}
      \begin{itemize}
        \item<1-> What if the backtrace for an exception is never needed?
        \item<2-> It's possible to raise the exception explicitly with \funcname{raise\_notrace} to make raising as cheap as possible
        \item<3-> An example of use in the \funcname{dynlink} library of the OCaml compiler
      \end{itemize}
      \vfill
      \onslide<3->{
      \listocaml[firstline=205,lastline=209,highlightlines=207]{../ocaml/otherlibs/dynlink/dynlink\_compilerlibs/misc.ml}{otherlibs/dynlink/dynlink\_compilerlibs/misc.ml:205}
      }
      \endminipage
    \end{column}
    \begin{column}{0.55\textwidth}
    \only<4>{
      \mintcmm[highlightlines=3]{../src/notrace/camlNotrace\_\_fun_347.cmm}
      \listcmm{../src/notrace/camlNotrace\_\_all\_somes\_327.cmm}{all\_somes}
    }
    \onslide*<5->{
      \listobjdump[highlightlines={6-9}]{../src/notrace/camlNotrace\_\_fun_347.objdump}{anonymous function from \funcname{all\_somes}}
    }
    \end{column}
  \end{columns}
\end{frame}

\begin{frame}{Backtraces}{Summary table of the multiple kinds of raise}
  \begin{itemize}
    \item When debugging information is enabled and if the backtrace of an exception isn't needed, it's possible to raise with \funcname{raise\_notrace} for performance
    \item When debugging information is \emph{not} enabled, the compiler optimises \funcname{raise} calls to inline assembly (same effect as using \funcname{raise\_notrace})
  \end{itemize}
  \bigskip
  \centering
  \begin{tabular}{| l | c | c | c | c |}
    \hline
    \parbox{10em}{Debug information \\ (\funcarg{-g} flag)}  & \multicolumn{2}{|c|}{Disabled}                                     & \multicolumn{2}{|c|}{Enabled} \\
    \hline
    Raise kind                                               & \funcname{raise} & \funcname{raise\_notrace}                       & \funcname{raise} & \funcname{raise\_notrace} \\
    \hline
    Compilation                                              & \parbox{8em}{inline assembly\\ (optimization)} & inline assembly   & \funcname{caml\_raise\_exn} & inline assembly \\
    \hline
  \end{tabular}
\end{frame}


%
% Takeaway #5
%
\subsection*{Takeaway \#5}
\frameSubsectionTakeaway{}
\againframe<5>{takeaway}
}%
      \onslide*<7>{\providecommand\step{11}%
% Backtraces
%
\subsection{One advantage of raising with debugging information}
\frameSubsection{}{}

\begin{frame}{Backtraces}{Debugging information \& source code}
  \begin{columns}[c]
    \begin{column}{0.5\textwidth}
      \begin{itemize}
        \item Raising an exception is fun and games, but how do we know where the exception originated? $\rightarrow$ With the backtrace associated with the exception!
        \item In order to get a backtrace from the point where an exception was raised, it's necessary to compile with debugging information enabled (\funcarg{-g} flag for the compiler)
        \item Same example as in the `Nested handlers' section
      \end{itemize}
    \end{column}
    \begin{column}{0.5\textwidth}
      \listocaml[firstline=9,lastline=34]{../src/backtrace/backtrace.ml}{backtrace/backtrace.ml}
    \end{column}
  \end{columns}
\end{frame}

\begin{frame}{Backtraces}{CMM and disassembly of \funcname{definitely\_raise}}
  \begin{columns}[c]
    \begin{column}{0.5\textwidth}
      \minipage[c][0.66\textheight][s]{\columnwidth}
      \begin{itemize}
        \item<1-> An effect of compiling with debugging information (\funcarg{-g}): the CMM no longer uses \funcname{raise\_notrace} to raise the exception but \funcname{raise} instead
        \item<2-> Disassembly shows that CMM \funcname{raise} is actually a call to \funcname{caml\_raise\_exn}, a function in assembler provided by the runtime
      \end{itemize}
      \vfill
      \mintocaml[firstline=9,lastline=13,highlightlines=12]{../src/backtrace/backtrace.ml}
      \endminipage
    \end{column}
    \begin{column}{0.5\textwidth}
      \onslide*<1>{
        \mintcmm[highlightlines=5]{../src/backtrace/camlBacktrace\_\_definitely\_raise\_347.cmm}
      }
      \onslide*<2>{
        \mintobjdump[firstline=2,highlightlines=14]{../src/backtrace/camlBacktrace\_\_definitely\_raise\_347.objdump}
      }
    \end{column}
  \end{columns}
\end{frame}

\begin{frame}{Backtraces}{Introducing \funcname{caml\_raise\_exn}}
  \begin{columns}[c]
    \begin{column}{0.5\textwidth}
      \minipage[c][0.66\textheight][s]{\columnwidth}
      \onslide*<1>{
        \begin{itemize}
          \item Raising the exception is performed using \funcname{caml\_raise\_exn} which is
            \begin{itemize}
              \item An assembly function provided by the runtime
              \item Architecture specific
            \end{itemize}
          \item Let's see what it does differently from \funcname{raise\_notrace}\dots
          \item We are about to raise \typename{ExnC} with \funcname{caml\_raise\_exn} from \funcname{definitely\_raise}
        \end{itemize}
      }
      \onslide*<2>{
        \mintobjdump[firstline=2,highlightlines=14]{../src/backtrace/camlBacktrace\_\_definitely\_raise\_347.objdump}
      }
      \vfill
      \mintocaml[firstline=9,lastline=13,highlightlines=12]{../src/backtrace/backtrace.ml}
      \endminipage
    \end{column}
    \begin{column}{0.5\textwidth}
      \centering
      \providecommand\step{1}\input{diagrams/backtrace/backtrace.tex}
    \end{column}
  \end{columns}
\end{frame}

\begin{frame}{Backtraces}{But before that, we need more fields from \typename{caml\_domain\_state}}
  \begin{columns}
    \begin{column}{0.5\textwidth}
      \begin{itemize}
        \item Remember \typename{caml\_domain\_state} the per-domain runtime state?
        \item Some other members are of interest to us now\dots
        \item \localname{backtrace\_active} is used to know if the runtime should collect backtraces
        \item \localname{backtrace\_active} can be set by
          \begin{itemize}
            \item The \funcarg{b} flag in the \funcname{OCAMLRUNPARAM} environment variable
            \item \mintinline{ocaml}{Printexc.record_backtrace : bool -> unit} \footnotemark
          \end{itemize}
      \end{itemize}
    \end{column}
    \begin{column}{0.5\textwidth}
      \begin{listing}
        \mintc[highlightlines={9-11}]{src/ocaml/domain\_state\_backtrace.h}
        \caption{runtime/caml/domain\_state.h}
      \end{listing}
    \end{column}
  \end{columns}
  \footnotetext[1]{https://v2.ocaml.org/api/Printexc.html}
\end{frame}

\newcommand\listCamlRaiseExn[1][]{
  \listasm[firstline=702,lastline=725,#1]{../ocaml/runtime/amd64.S}{runtime/amd64.S:702}}

\begin{frame}{Backtraces}{Walk through \funcname{caml\_raise\_exn}}
  \begin{columns}[T]
    \begin{column}{0.5\textwidth}
      \begin{itemize}
        \item<1-> First checks whether exceptions backtrace should be recorded
        \item<1-> By checking the \funcarg{domain\_state->backtrace\_active} value
        \item<2-> If not, acts as \funcname{raise\_notrace} with \funcname{RESTORE\_EXN\_HANDLER\_OCAML}
          \begin{itemize}
            \item Set \regname{RSP} to \localname{domain\_state->exn\_handler} value
            \item Restore \localname{domain\_state->exn\_handler} from trap
            \item Jump with \funcname{ret} (same effect as using \funcname{pop} and \funcname{jmp})
          \end{itemize}
        \item<3-> Otherwise, switches to the C stack to be able to execute C code
        \item<4-> Calls \funcname{caml\_stash\_backtrace}, a C function provided by the runtime\ignorespaces
          \onslide<6->{, with
            \begin{itemize}
              \item \funcarg{exn}: exception value
              \item \funcarg{pc}: code address of the raising point
              \item \funcarg{sp}: stack address of the raising function
              \item \funcarg{trapsp}: stack address of the next handler
            \end{itemize}
          }
      \end{itemize}
      \bigskip
      \mintocaml[firstline=9,lastline=13,highlightlines=12]{../src/backtrace/backtrace.ml}
    \end{column}
    \begin{column}{0.5\textwidth}
      \raggedleft
      \onslide*<1,3-4>{
        \mintc[firstline=190,lastline=192]{../ocaml/runtime/amd64.S}
        \mintc[firstline=234,lastline=237]{../ocaml/runtime/amd64.S}
      }
      \onslide*<2>{
        \mintc[firstline=190,lastline=192,highlightlines={191-192}]{../ocaml/runtime/amd64.S}
        \mintc[firstline=234,lastline=237,highlightlines={235-237}]{../ocaml/runtime/amd64.S}
      }
      \onslide*<1>{\listCamlRaiseExn[highlightlines=705]{}}%
      \onslide*<2>{\listCamlRaiseExn[highlightlines={707-708}]{}}%
      \onslide*<3>{\listCamlRaiseExn[highlightlines={712-714}]{}}%
      \onslide*<4>{\listCamlRaiseExn[highlightlines={715-720}]{}}%
      \onslide*<5>{\providecommand\step{1}\input{diagrams/backtrace/backtrace.tex}}%
      \onslide*<6>{\providecommand\step{2}\input{diagrams/backtrace/backtrace.tex}}%
    \end{column}
  \end{columns}
\end{frame}

\newcommand\listStashBacktrace[1][]{
  \listc[firstline=91,lastline=120,#1]{../ocaml/runtime/backtrace\_nat.c}{runtime/backtrace\_nat.c:91}}

\begin{frame}{Backtraces}{Introducing \funcname{caml\_stash\_backtrace}}
  \begin{columns}[c]
    \begin{column}{0.5\textwidth}
      \minipage[c][0.66\textheight][s]{\columnwidth}
      \begin{itemize}
        \item<1-> A C function provided by the runtime (specific to native code)
        \item<2-> It scans the stack, between the stack pointer at the raising point up to the next trap, in order to collect the names of the detected function frames
          \onslide*<2>{
            \begin{itemize}
              \item And more information, like line number, column number...
            \end{itemize}
          }
        \item<3-> It uses the same technique and information as the Garbage Collector (GC) uses to scan the values live on the stack
          \onslide*<4>{
            \begin{itemize}
              \item \typename{frame\_descr} are at the core of stack unwinding
            \end{itemize}
          }
      \end{itemize}
      \vfill
      \mintocaml[firstline=9,lastline=13,highlightlines=12]{../src/backtrace/backtrace.ml}
      \endminipage
    \end{column}
    \begin{column}{0.5\textwidth}
      \onslide*<1>{\listStashBacktrace[highlightlines=91]{}}
      \onslide*<2>{\listStashBacktrace[highlightlines={91,117-118}]{}}
      \onslide*<3->{\listStashBacktrace[highlightlines={94,106,109-110}]{}}
    \end{column}
  \end{columns}
\end{frame}

\newcommand\listFrameDescr[1][]{
  \listocaml[firstline=111,lastline=124,#1]{../ocaml/asmcomp/emitaux.ml}{asmcomp/emitaux.ml:111}}
\newcommand\listDebugInfo[1][]{
  \listocaml[firstline=38,lastline=49,#1]{../ocaml/lambda/debuginfo.mli}{lambda/debuginfo.mli:38}}

\begin{frame}{Backtraces}{\typename{frame\_descr} and \typename{frame\_debuginfo} in a nutshell}
  \begin{columns}[c]
    \begin{column}{0.5\textwidth}
      \begin{itemize}
        \item<1-> For every return address in an OCaml program, there is a associated \typename{frame\_descr}
        \item<2-> A \typename{frame\_descr} specifies
        \begin{itemize}
          \item<2-> The size of the function stack frame
          \item<3-> Where live values are on the stack (so that the GC can mark them as live and prevent from being swept)
        \end{itemize}
        \item<4-> When compiled with debug information, \typename{frame\_descr} will be linked to a \typename{frame\_debuginfo}
        \begin{itemize}
          \item<5-> Contains information about the position in the source code (file name, line and column number)
        \end{itemize}
      \end{itemize}
    \end{column}
    \begin{column}{0.5\textwidth}
        \onslide*<1>{\listFrameDescr[highlightlines={118-122}]{}}
        \onslide*<2>{\listFrameDescr[highlightlines={120}]{}}
        \onslide*<3>{\listFrameDescr[highlightlines={121}]{}}
        \onslide*<4->{\listFrameDescr[highlightlines={122}]{}}
        \medskip
        \onslide*<1-3>{\listDebugInfo{}}
        \onslide*<4>{\listDebugInfo[highlightlines={38,49}]{}}
        \onslide*<5>{\listDebugInfo[highlightlines={39-46}]{}}
    \end{column}
  \end{columns}
\end{frame}

\begin{frame}{Backtraces}{Scanning the stack with \typename{frame\_descr}}
  \begin{columns}[c]
    \begin{column}{0.5\textwidth}
      \begin{itemize}
        \item Given the return address of the code that raises the exception by calling \funcname{caml\_raise\_exn} (\funcarg{pc} argument of \funcname{caml\_stash\_backtrace})
        \item And the position of the stack pointer (\funcarg{sp} argument of \funcname{caml\_stash\_backtrace})
        \item The frame size given by \typename{frame\_descr}.\funcarg{frame\_size}, allows to find the end of the previous frame
        \item And the return address of the caller of this function
        \bigskip
        \onslide<3->{
        \item Note that traps (here the reddish \funcname{ExnA}, \funcname{ExnB} and \funcname{ExnC} blocks) belong to the frame that installed them, and so the frame size changes when traps are removed
        }
      \end{itemize}
    \end{column}
    \begin{column}{0.5\textwidth}
      \raggedleft
      \onslide*<1>{\providecommand\step{2}\input{diagrams/backtrace/backtrace.tex}}%
      \onslide*<2>{\providecommand\step{1}\input{diagrams/backtrace/scanstack.tex}}%
      \onslide*<3>{\providecommand\step{2}\input{diagrams/backtrace/scanstack.tex}}%
      \onslide*<4>{\providecommand\step{3}\input{diagrams/backtrace/scanstack.tex}}%
      \onslide*<5>{\providecommand\step{4}\input{diagrams/backtrace/scanstack.tex}}%
      \onslide*<6>{\providecommand\step{5}\input{diagrams/backtrace/scanstack.tex}}%
    \end{column}
  \end{columns}
\end{frame}

% Duplicate of previous-previous slide
\begin{frame}{Backtraces}{Continuing on \funcname{caml\_stash\_backtrace}}
  \begin{columns}[c]
    \begin{column}{0.5\textwidth}
      \minipage[c][0.75\textheight][s]{\columnwidth}
      \begin{itemize}
          \color{gray}
        \item A C function provided by the runtime (specific to native code)
        \item It scans the stack, between the stack pointer at the raising point up to the next trap, in order to collect the names of the detected function frames
        \item It uses the same technique and information as the Garbage Collector (GC) uses to scan the values live on the stack
      \end{itemize}
      \begin{itemize}
        \item For each function frame detected, store a pointer to the \typename{frame\_descr} in the \localname{domain\_state->backtrace\_buffer} array, effectively constructing the backtrace
      \end{itemize}
      \onslide<2->{
        \begin{itemize}
            \smallskip
          \item Constructed backtrace:
            \funcname{definitely\_raise},
            \onslide<3->{\funcname{probably\_raise}}
        \end{itemize}
      }
      \vfill
      \mintocaml[firstline=9,lastline=13,highlightlines=12]{../src/backtrace/backtrace.ml}
      \endminipage
    \end{column}
    \begin{column}{0.5\textwidth}
      \raggedleft
      \onslide*<1>{\listStashBacktrace[highlightlines={114-115}]{}}
      \onslide*<2>{\providecommand\step{3}\input{diagrams/backtrace/backtrace.tex}}%
      \onslide*<3>{\providecommand\step{4}\input{diagrams/backtrace/backtrace.tex}}%
    \end{column}
  \end{columns}
\end{frame}

\begin{frame}{Backtraces}{Back to \funcname{caml\_raise\_exn}}
  \begin{columns}[c]
    \begin{column}{0.5\textwidth}
      \minipage[c][0.66\textheight][s]{\columnwidth}
      \begin{itemize}\color{gray}
        \item Checks wheteher exceptions backtrace should be recorded, by checking the \funcarg{domain\_state->backtrace\_active} value
        \item Switches to the C stack to be able to execute C code
        \item Calls \funcname{caml\_stash\_backtrace}, to collect the backtrace up to the next trap
      \end{itemize}
      \onslide*<2->{
        \begin{itemize}
          \item<2-> Executes the exception handler in \funcname{probably\_raise} with \funcname{RESTORE\_EXN\_HANDLER\_OCAML}
            \begin{itemize}
              \item Set \regname{RSP} to \localname{domain\_state->exn\_handler} value
              \item Restore \localname{domain\_state->exn\_handler} from trap
              \item Jump to the handler code with \funcname{ret}
            \end{itemize}
        \end{itemize}
      }
      \vfill
      \onslide*<1>{\mintocaml[firstline=9,lastline=13,highlightlines=12]{../src/backtrace/backtrace.ml}}
      \onslide*<2->{\mintocaml[firstline=15,lastline=21,highlightlines={19-21}]{../src/backtrace/backtrace.ml}}
      \endminipage
    \end{column}
    \begin{column}{0.5\textwidth}
      \centering
      \onslide*<1>{
        \mintc[firstline=190,lastline=192]{../ocaml/runtime/amd64.S}
        \mintc[firstline=234,lastline=237]{../ocaml/runtime/amd64.S}
      }
      \onslide*<2>{
        \mintc[firstline=190,lastline=192,highlightlines={191-192}]{../ocaml/runtime/amd64.S}
        \mintc[firstline=234,lastline=237,highlightlines={235-237}]{../ocaml/runtime/amd64.S}
      }
      \onslide*<1>{\listCamlRaiseExn[highlightlines=720]{}}
      \onslide*<2>{\listCamlRaiseExn[highlightlines=722]{}}
      \onslide*<3->{\providecommand\step{5}\input{diagrams/backtrace/backtrace.tex}}
    \end{column}
  \end{columns}
\end{frame}

\begin{frame}{Backtraces}{\funcname{caml\_probably\_raise}'s handler doesn't catch \typename{ExnC}}
  \begin{columns}[c]
    \begin{column}{0.45\textwidth}
      \minipage[c][0.75\textheight][s]{\columnwidth}
      \begin{itemize}
        \item<1-> \funcname{match \dots with}\footnotemark pattern matching can also match exceptions
        \item<1-> The CMM of \funcname{probably\_raise} shows that this \funcname{match \dots with} is implemented with a pattern matching and a \funcname{try \dots with}
        \item<1-> But this one only matches on \typename{ExnA} exception
        \item<2-> Since the pattern matching fails to catch the exception, \funcname{reraise} is called with our current \typename{ExnC} exception
        \item<3-> Disassembly shows that \funcname{reraise} is actually a call to \funcname{caml\_reraise\_exn}, which is also an assembly function provided by the runtime
      \end{itemize}
      \vfill
      \mintocaml[firstline=15,lastline=21,highlightlines={18-21}]{../src/backtrace/backtrace.ml}
      \endminipage
    \end{column}
    \begin{column}{0.55\textwidth}
      \centering
      \onslide*<1>{\mintcmm[highlightlines={10-18}]{../src/backtrace/camlBacktrace\_\_probably\_raise\_351.cmm}}
      \onslide*<2>{\listcmm[highlightlines=18]{../src/backtrace/camlBacktrace\_\_probably\_raise_351.cmm}{CMM of probably\_raise}}
      \onslide*<3>{\mintobjdump[firstline=5,lastline=35,highlightlines=31]{../src/backtrace/camlBacktrace\_\_probably\_raise_351.objdump}}
    \end{column}
  \end{columns}
  \only<1>{\footnotetext[1]{https://blog.janestreet.com/pattern-matching-and-exception-handling-unite/}}
\end{frame}

\newcommand\listCamlReraiseExn[1][]{
  \listasm[firstline=727,lastline=734,#1]{../ocaml/runtime/amd64.S}{runtime/amd64.S:727}}

\begin{frame}{Backtraces}{Introducing \funcname{caml\_reraise\_exn}}
  \begin{columns}[c]
    \begin{column}{0.5\textwidth}
      \minipage[c][0.66\textheight][s]{\columnwidth}
      \begin{itemize}
        \item<1-> The exception have to be reraised to the next handler
        \item<2-> First, checks if the backtrace should be collected with \funcarg{domain\_state->backtrace\_active}
        \only<3>{
        \begin{itemize}
            \item If not, behaves like a \funcname{raise\_notrace} with \funcname{RESTORE\_EXN\_HANDLER\_OCAML}
        \end{itemize}
        }
        \item<4-> Jumps into \funcname{caml\_raise\_exn}
        \item<5-> Calls \funcname{caml\_stash\_backtrace} again, to scan the next part of the stack: from the trap just pop-ed to the next trap
      \end{itemize}
      \vfill
      \mintocaml[firstline=15,lastline=21,highlightlines={18-21}]{../src/backtrace/backtrace.ml}
      \endminipage
    \end{column}
    \begin{column}{0.5\textwidth}
      \raggedleft
      \onslide*<1>{\listCamlReraiseExn{}}
      \onslide*<2>{\listCamlReraiseExn[highlightlines=729]{}}
      \onslide*<3>{
        \mintc[firstline=190,lastline=192,highlightlines={191-192}]{../ocaml/runtime/amd64.S}
        \mintc[firstline=234,lastline=237,highlightlines={235-237}]{../ocaml/runtime/amd64.S}
        \medskip
        \listCamlReraiseExn[highlightlines=731]{}
      }
      \onslide*<4-5>{
        % TODO refactor with \listCamlReraiseExn
        \mintasm[firstline=727,lastline=734,highlightlines=730]{../ocaml/runtime/amd64.S}
        \medskip
      }
      \onslide*<4>{\listCamlRaiseExn[highlightlines=711]{}}
      \onslide*<5>{\listCamlRaiseExn[highlightlines=720]{}}
    \end{column}
  \end{columns}
\end{frame}

\begin{frame}{Backtraces}{Backtrace collection continues}
  \begin{columns}[c]
    \begin{column}{0.55\textwidth}
      \minipage[c][0.75\textheight][s]{\columnwidth}
      \begin{itemize}
        \item<1-> \funcname{caml\_stash\_backtrace} scans the older part of the stack, up to the next trap
        \item<6-> Controls jumps to the nested exception handler in \funcname{maybe\_raise}, which matches only on \typename{ExnB}
        \item<7-> \funcname{caml\_reraise\_exn} is called again, backtrace collection resumes with \funcname{caml\_stash\_backtrace}
      \end{itemize}
      \medskip
      \begin{itemize}
        \item<2-> Constructed backtrace:
        \begin{itemize}
          \item <2-> \funcname{definitely\_raise}
          \item <2-> \funcname{probably\_raise}
          \item <3-> \funcname{probably\_raise} (re-raise)
          \item <4-> \funcname{maybe\_raise}
          \item <5-> \funcname{main}
          \item <8-> \funcname{main} (re-raise)
        \end{itemize}
      \end{itemize}
      \vfill
      \onslide*<1-5>{\mintocaml[firstline=15,lastline=21]{../src/backtrace/backtrace.ml}}
      \onslide*<6>{\mintocaml[firstline=28,lastline=34,highlightlines=31]{../src/backtrace/backtrace.ml}}
      \onslide*<7->{\mintocaml[firstline=28,lastline=34,highlightlines=33]{../src/backtrace/backtrace.ml}}
      \endminipage
    \end{column}
    \begin{column}{0.45\textwidth}
      \raggedleft
      \onslide*<1>{\providecommand\step{6}\input{diagrams/backtrace/backtrace.tex}}%
      \onslide*<2>{\providecommand\step{6}\input{diagrams/backtrace/backtrace.tex}}%
      \onslide*<3>{\providecommand\step{7}\input{diagrams/backtrace/backtrace.tex}}%
      \onslide*<4>{\providecommand\step{8}\input{diagrams/backtrace/backtrace.tex}}%
      \onslide*<5>{\providecommand\step{9}\input{diagrams/backtrace/backtrace.tex}}%
      \onslide*<6>{\providecommand\step{10}\input{diagrams/backtrace/backtrace.tex}}%
      \onslide*<7>{\providecommand\step{11}\input{diagrams/backtrace/backtrace.tex}}%
      \onslide*<8>{\providecommand\step{12}\input{diagrams/backtrace/backtrace.tex}}%
      \onslide*<9>{\providecommand\step{13}\input{diagrams/backtrace/backtrace.tex}}%
    \end{column}
  \end{columns}
\end{frame}

\begin{frame}{Backtraces}{Printing the backtrace}
  \begin{columns}[c]
    \begin{column}{0.45\textwidth}
      \minipage[c][0.66\textheight][s]{\columnwidth}
      \begin{itemize}
        \item<1-> The exception backtrace can now be printed to \funcarg{stdout} with \funcname{Printexc.print_backtrace}
        \item<2-> First the exception backtrace is retrieved into an OCaml array with \funcname{get_raw_backtrace }
        \item<3-> Which is implemented as the \funcname{caml_get_exception_raw_backtrace} C function in the runtime
        \item<4-> And finally printed with \funcname{print_exception_backtrace}
      \end{itemize}
      \vfill
      \mintocaml[firstline=28,lastline=34,highlightlines=33]{../src/backtrace/backtrace.ml}
      \endminipage
    \end{column}
    \begin{column}{0.55\textwidth}
      \onslide*<1,2>{
        \mintocaml[firstline=100,lastline=101]{../ocaml/stdlib/printexc.ml}
      }
      \only<1>{
        \listocaml[firstline=170,lastline=172]{../ocaml/stdlib/printexc.ml}{stdlib/printexc.ml:170}
      }
      \only<2>{
        \listocaml[firstline=170,lastline=172,highlightlines=172]{../ocaml/stdlib/printexc.ml}{stdlib/printexc.ml:170}
      }
      \only<3>{
        \mintc[firstline=154,lastline=156]{../ocaml/runtime/backtrace.c}
        \listc[firstline=171,lastline=192,highlightlines={184-187}]{../ocaml/runtime/backtrace.c}{runtime/backtrace.c:154}
      }
      \only<4>{
        \listocaml[firstline=155,lastline=165]{../ocaml/stdlib/printexc.ml}{stdlib/printexc.ml:55}
      }
    \end{column}
  \end{columns}
\end{frame}


\subsection{The multiple kinds of raise}
\frameSubsection{}{}

\begin{frame}{Backtraces}{When backtraces are not needed}
  \begin{columns}[c]
    \begin{column}{0.45\textwidth}
      \minipage[c][0.66\textheight][s]{\columnwidth}
      \begin{itemize}
        \item<1-> What if the backtrace for an exception is never needed?
        \item<2-> It's possible to raise the exception explicitly with \funcname{raise\_notrace} to make raising as cheap as possible
        \item<3-> An example of use in the \funcname{dynlink} library of the OCaml compiler
      \end{itemize}
      \vfill
      \onslide<3->{
      \listocaml[firstline=205,lastline=209,highlightlines=207]{../ocaml/otherlibs/dynlink/dynlink\_compilerlibs/misc.ml}{otherlibs/dynlink/dynlink\_compilerlibs/misc.ml:205}
      }
      \endminipage
    \end{column}
    \begin{column}{0.55\textwidth}
    \only<4>{
      \mintcmm[highlightlines=3]{../src/notrace/camlNotrace\_\_fun_347.cmm}
      \listcmm{../src/notrace/camlNotrace\_\_all\_somes\_327.cmm}{all\_somes}
    }
    \onslide*<5->{
      \listobjdump[highlightlines={6-9}]{../src/notrace/camlNotrace\_\_fun_347.objdump}{anonymous function from \funcname{all\_somes}}
    }
    \end{column}
  \end{columns}
\end{frame}

\begin{frame}{Backtraces}{Summary table of the multiple kinds of raise}
  \begin{itemize}
    \item When debugging information is enabled and if the backtrace of an exception isn't needed, it's possible to raise with \funcname{raise\_notrace} for performance
    \item When debugging information is \emph{not} enabled, the compiler optimises \funcname{raise} calls to inline assembly (same effect as using \funcname{raise\_notrace})
  \end{itemize}
  \bigskip
  \centering
  \begin{tabular}{| l | c | c | c | c |}
    \hline
    \parbox{10em}{Debug information \\ (\funcarg{-g} flag)}  & \multicolumn{2}{|c|}{Disabled}                                     & \multicolumn{2}{|c|}{Enabled} \\
    \hline
    Raise kind                                               & \funcname{raise} & \funcname{raise\_notrace}                       & \funcname{raise} & \funcname{raise\_notrace} \\
    \hline
    Compilation                                              & \parbox{8em}{inline assembly\\ (optimization)} & inline assembly   & \funcname{caml\_raise\_exn} & inline assembly \\
    \hline
  \end{tabular}
\end{frame}


%
% Takeaway #5
%
\subsection*{Takeaway \#5}
\frameSubsectionTakeaway{}
\againframe<5>{takeaway}
}%
      \onslide*<8>{\providecommand\step{12}%
% Backtraces
%
\subsection{One advantage of raising with debugging information}
\frameSubsection{}{}

\begin{frame}{Backtraces}{Debugging information \& source code}
  \begin{columns}[c]
    \begin{column}{0.5\textwidth}
      \begin{itemize}
        \item Raising an exception is fun and games, but how do we know where the exception originated? $\rightarrow$ With the backtrace associated with the exception!
        \item In order to get a backtrace from the point where an exception was raised, it's necessary to compile with debugging information enabled (\funcarg{-g} flag for the compiler)
        \item Same example as in the `Nested handlers' section
      \end{itemize}
    \end{column}
    \begin{column}{0.5\textwidth}
      \listocaml[firstline=9,lastline=34]{../src/backtrace/backtrace.ml}{backtrace/backtrace.ml}
    \end{column}
  \end{columns}
\end{frame}

\begin{frame}{Backtraces}{CMM and disassembly of \funcname{definitely\_raise}}
  \begin{columns}[c]
    \begin{column}{0.5\textwidth}
      \minipage[c][0.66\textheight][s]{\columnwidth}
      \begin{itemize}
        \item<1-> An effect of compiling with debugging information (\funcarg{-g}): the CMM no longer uses \funcname{raise\_notrace} to raise the exception but \funcname{raise} instead
        \item<2-> Disassembly shows that CMM \funcname{raise} is actually a call to \funcname{caml\_raise\_exn}, a function in assembler provided by the runtime
      \end{itemize}
      \vfill
      \mintocaml[firstline=9,lastline=13,highlightlines=12]{../src/backtrace/backtrace.ml}
      \endminipage
    \end{column}
    \begin{column}{0.5\textwidth}
      \onslide*<1>{
        \mintcmm[highlightlines=5]{../src/backtrace/camlBacktrace\_\_definitely\_raise\_347.cmm}
      }
      \onslide*<2>{
        \mintobjdump[firstline=2,highlightlines=14]{../src/backtrace/camlBacktrace\_\_definitely\_raise\_347.objdump}
      }
    \end{column}
  \end{columns}
\end{frame}

\begin{frame}{Backtraces}{Introducing \funcname{caml\_raise\_exn}}
  \begin{columns}[c]
    \begin{column}{0.5\textwidth}
      \minipage[c][0.66\textheight][s]{\columnwidth}
      \onslide*<1>{
        \begin{itemize}
          \item Raising the exception is performed using \funcname{caml\_raise\_exn} which is
            \begin{itemize}
              \item An assembly function provided by the runtime
              \item Architecture specific
            \end{itemize}
          \item Let's see what it does differently from \funcname{raise\_notrace}\dots
          \item We are about to raise \typename{ExnC} with \funcname{caml\_raise\_exn} from \funcname{definitely\_raise}
        \end{itemize}
      }
      \onslide*<2>{
        \mintobjdump[firstline=2,highlightlines=14]{../src/backtrace/camlBacktrace\_\_definitely\_raise\_347.objdump}
      }
      \vfill
      \mintocaml[firstline=9,lastline=13,highlightlines=12]{../src/backtrace/backtrace.ml}
      \endminipage
    \end{column}
    \begin{column}{0.5\textwidth}
      \centering
      \providecommand\step{1}\input{diagrams/backtrace/backtrace.tex}
    \end{column}
  \end{columns}
\end{frame}

\begin{frame}{Backtraces}{But before that, we need more fields from \typename{caml\_domain\_state}}
  \begin{columns}
    \begin{column}{0.5\textwidth}
      \begin{itemize}
        \item Remember \typename{caml\_domain\_state} the per-domain runtime state?
        \item Some other members are of interest to us now\dots
        \item \localname{backtrace\_active} is used to know if the runtime should collect backtraces
        \item \localname{backtrace\_active} can be set by
          \begin{itemize}
            \item The \funcarg{b} flag in the \funcname{OCAMLRUNPARAM} environment variable
            \item \mintinline{ocaml}{Printexc.record_backtrace : bool -> unit} \footnotemark
          \end{itemize}
      \end{itemize}
    \end{column}
    \begin{column}{0.5\textwidth}
      \begin{listing}
        \mintc[highlightlines={9-11}]{src/ocaml/domain\_state\_backtrace.h}
        \caption{runtime/caml/domain\_state.h}
      \end{listing}
    \end{column}
  \end{columns}
  \footnotetext[1]{https://v2.ocaml.org/api/Printexc.html}
\end{frame}

\newcommand\listCamlRaiseExn[1][]{
  \listasm[firstline=702,lastline=725,#1]{../ocaml/runtime/amd64.S}{runtime/amd64.S:702}}

\begin{frame}{Backtraces}{Walk through \funcname{caml\_raise\_exn}}
  \begin{columns}[T]
    \begin{column}{0.5\textwidth}
      \begin{itemize}
        \item<1-> First checks whether exceptions backtrace should be recorded
        \item<1-> By checking the \funcarg{domain\_state->backtrace\_active} value
        \item<2-> If not, acts as \funcname{raise\_notrace} with \funcname{RESTORE\_EXN\_HANDLER\_OCAML}
          \begin{itemize}
            \item Set \regname{RSP} to \localname{domain\_state->exn\_handler} value
            \item Restore \localname{domain\_state->exn\_handler} from trap
            \item Jump with \funcname{ret} (same effect as using \funcname{pop} and \funcname{jmp})
          \end{itemize}
        \item<3-> Otherwise, switches to the C stack to be able to execute C code
        \item<4-> Calls \funcname{caml\_stash\_backtrace}, a C function provided by the runtime\ignorespaces
          \onslide<6->{, with
            \begin{itemize}
              \item \funcarg{exn}: exception value
              \item \funcarg{pc}: code address of the raising point
              \item \funcarg{sp}: stack address of the raising function
              \item \funcarg{trapsp}: stack address of the next handler
            \end{itemize}
          }
      \end{itemize}
      \bigskip
      \mintocaml[firstline=9,lastline=13,highlightlines=12]{../src/backtrace/backtrace.ml}
    \end{column}
    \begin{column}{0.5\textwidth}
      \raggedleft
      \onslide*<1,3-4>{
        \mintc[firstline=190,lastline=192]{../ocaml/runtime/amd64.S}
        \mintc[firstline=234,lastline=237]{../ocaml/runtime/amd64.S}
      }
      \onslide*<2>{
        \mintc[firstline=190,lastline=192,highlightlines={191-192}]{../ocaml/runtime/amd64.S}
        \mintc[firstline=234,lastline=237,highlightlines={235-237}]{../ocaml/runtime/amd64.S}
      }
      \onslide*<1>{\listCamlRaiseExn[highlightlines=705]{}}%
      \onslide*<2>{\listCamlRaiseExn[highlightlines={707-708}]{}}%
      \onslide*<3>{\listCamlRaiseExn[highlightlines={712-714}]{}}%
      \onslide*<4>{\listCamlRaiseExn[highlightlines={715-720}]{}}%
      \onslide*<5>{\providecommand\step{1}\input{diagrams/backtrace/backtrace.tex}}%
      \onslide*<6>{\providecommand\step{2}\input{diagrams/backtrace/backtrace.tex}}%
    \end{column}
  \end{columns}
\end{frame}

\newcommand\listStashBacktrace[1][]{
  \listc[firstline=91,lastline=120,#1]{../ocaml/runtime/backtrace\_nat.c}{runtime/backtrace\_nat.c:91}}

\begin{frame}{Backtraces}{Introducing \funcname{caml\_stash\_backtrace}}
  \begin{columns}[c]
    \begin{column}{0.5\textwidth}
      \minipage[c][0.66\textheight][s]{\columnwidth}
      \begin{itemize}
        \item<1-> A C function provided by the runtime (specific to native code)
        \item<2-> It scans the stack, between the stack pointer at the raising point up to the next trap, in order to collect the names of the detected function frames
          \onslide*<2>{
            \begin{itemize}
              \item And more information, like line number, column number...
            \end{itemize}
          }
        \item<3-> It uses the same technique and information as the Garbage Collector (GC) uses to scan the values live on the stack
          \onslide*<4>{
            \begin{itemize}
              \item \typename{frame\_descr} are at the core of stack unwinding
            \end{itemize}
          }
      \end{itemize}
      \vfill
      \mintocaml[firstline=9,lastline=13,highlightlines=12]{../src/backtrace/backtrace.ml}
      \endminipage
    \end{column}
    \begin{column}{0.5\textwidth}
      \onslide*<1>{\listStashBacktrace[highlightlines=91]{}}
      \onslide*<2>{\listStashBacktrace[highlightlines={91,117-118}]{}}
      \onslide*<3->{\listStashBacktrace[highlightlines={94,106,109-110}]{}}
    \end{column}
  \end{columns}
\end{frame}

\newcommand\listFrameDescr[1][]{
  \listocaml[firstline=111,lastline=124,#1]{../ocaml/asmcomp/emitaux.ml}{asmcomp/emitaux.ml:111}}
\newcommand\listDebugInfo[1][]{
  \listocaml[firstline=38,lastline=49,#1]{../ocaml/lambda/debuginfo.mli}{lambda/debuginfo.mli:38}}

\begin{frame}{Backtraces}{\typename{frame\_descr} and \typename{frame\_debuginfo} in a nutshell}
  \begin{columns}[c]
    \begin{column}{0.5\textwidth}
      \begin{itemize}
        \item<1-> For every return address in an OCaml program, there is a associated \typename{frame\_descr}
        \item<2-> A \typename{frame\_descr} specifies
        \begin{itemize}
          \item<2-> The size of the function stack frame
          \item<3-> Where live values are on the stack (so that the GC can mark them as live and prevent from being swept)
        \end{itemize}
        \item<4-> When compiled with debug information, \typename{frame\_descr} will be linked to a \typename{frame\_debuginfo}
        \begin{itemize}
          \item<5-> Contains information about the position in the source code (file name, line and column number)
        \end{itemize}
      \end{itemize}
    \end{column}
    \begin{column}{0.5\textwidth}
        \onslide*<1>{\listFrameDescr[highlightlines={118-122}]{}}
        \onslide*<2>{\listFrameDescr[highlightlines={120}]{}}
        \onslide*<3>{\listFrameDescr[highlightlines={121}]{}}
        \onslide*<4->{\listFrameDescr[highlightlines={122}]{}}
        \medskip
        \onslide*<1-3>{\listDebugInfo{}}
        \onslide*<4>{\listDebugInfo[highlightlines={38,49}]{}}
        \onslide*<5>{\listDebugInfo[highlightlines={39-46}]{}}
    \end{column}
  \end{columns}
\end{frame}

\begin{frame}{Backtraces}{Scanning the stack with \typename{frame\_descr}}
  \begin{columns}[c]
    \begin{column}{0.5\textwidth}
      \begin{itemize}
        \item Given the return address of the code that raises the exception by calling \funcname{caml\_raise\_exn} (\funcarg{pc} argument of \funcname{caml\_stash\_backtrace})
        \item And the position of the stack pointer (\funcarg{sp} argument of \funcname{caml\_stash\_backtrace})
        \item The frame size given by \typename{frame\_descr}.\funcarg{frame\_size}, allows to find the end of the previous frame
        \item And the return address of the caller of this function
        \bigskip
        \onslide<3->{
        \item Note that traps (here the reddish \funcname{ExnA}, \funcname{ExnB} and \funcname{ExnC} blocks) belong to the frame that installed them, and so the frame size changes when traps are removed
        }
      \end{itemize}
    \end{column}
    \begin{column}{0.5\textwidth}
      \raggedleft
      \onslide*<1>{\providecommand\step{2}\input{diagrams/backtrace/backtrace.tex}}%
      \onslide*<2>{\providecommand\step{1}\input{diagrams/backtrace/scanstack.tex}}%
      \onslide*<3>{\providecommand\step{2}\input{diagrams/backtrace/scanstack.tex}}%
      \onslide*<4>{\providecommand\step{3}\input{diagrams/backtrace/scanstack.tex}}%
      \onslide*<5>{\providecommand\step{4}\input{diagrams/backtrace/scanstack.tex}}%
      \onslide*<6>{\providecommand\step{5}\input{diagrams/backtrace/scanstack.tex}}%
    \end{column}
  \end{columns}
\end{frame}

% Duplicate of previous-previous slide
\begin{frame}{Backtraces}{Continuing on \funcname{caml\_stash\_backtrace}}
  \begin{columns}[c]
    \begin{column}{0.5\textwidth}
      \minipage[c][0.75\textheight][s]{\columnwidth}
      \begin{itemize}
          \color{gray}
        \item A C function provided by the runtime (specific to native code)
        \item It scans the stack, between the stack pointer at the raising point up to the next trap, in order to collect the names of the detected function frames
        \item It uses the same technique and information as the Garbage Collector (GC) uses to scan the values live on the stack
      \end{itemize}
      \begin{itemize}
        \item For each function frame detected, store a pointer to the \typename{frame\_descr} in the \localname{domain\_state->backtrace\_buffer} array, effectively constructing the backtrace
      \end{itemize}
      \onslide<2->{
        \begin{itemize}
            \smallskip
          \item Constructed backtrace:
            \funcname{definitely\_raise},
            \onslide<3->{\funcname{probably\_raise}}
        \end{itemize}
      }
      \vfill
      \mintocaml[firstline=9,lastline=13,highlightlines=12]{../src/backtrace/backtrace.ml}
      \endminipage
    \end{column}
    \begin{column}{0.5\textwidth}
      \raggedleft
      \onslide*<1>{\listStashBacktrace[highlightlines={114-115}]{}}
      \onslide*<2>{\providecommand\step{3}\input{diagrams/backtrace/backtrace.tex}}%
      \onslide*<3>{\providecommand\step{4}\input{diagrams/backtrace/backtrace.tex}}%
    \end{column}
  \end{columns}
\end{frame}

\begin{frame}{Backtraces}{Back to \funcname{caml\_raise\_exn}}
  \begin{columns}[c]
    \begin{column}{0.5\textwidth}
      \minipage[c][0.66\textheight][s]{\columnwidth}
      \begin{itemize}\color{gray}
        \item Checks wheteher exceptions backtrace should be recorded, by checking the \funcarg{domain\_state->backtrace\_active} value
        \item Switches to the C stack to be able to execute C code
        \item Calls \funcname{caml\_stash\_backtrace}, to collect the backtrace up to the next trap
      \end{itemize}
      \onslide*<2->{
        \begin{itemize}
          \item<2-> Executes the exception handler in \funcname{probably\_raise} with \funcname{RESTORE\_EXN\_HANDLER\_OCAML}
            \begin{itemize}
              \item Set \regname{RSP} to \localname{domain\_state->exn\_handler} value
              \item Restore \localname{domain\_state->exn\_handler} from trap
              \item Jump to the handler code with \funcname{ret}
            \end{itemize}
        \end{itemize}
      }
      \vfill
      \onslide*<1>{\mintocaml[firstline=9,lastline=13,highlightlines=12]{../src/backtrace/backtrace.ml}}
      \onslide*<2->{\mintocaml[firstline=15,lastline=21,highlightlines={19-21}]{../src/backtrace/backtrace.ml}}
      \endminipage
    \end{column}
    \begin{column}{0.5\textwidth}
      \centering
      \onslide*<1>{
        \mintc[firstline=190,lastline=192]{../ocaml/runtime/amd64.S}
        \mintc[firstline=234,lastline=237]{../ocaml/runtime/amd64.S}
      }
      \onslide*<2>{
        \mintc[firstline=190,lastline=192,highlightlines={191-192}]{../ocaml/runtime/amd64.S}
        \mintc[firstline=234,lastline=237,highlightlines={235-237}]{../ocaml/runtime/amd64.S}
      }
      \onslide*<1>{\listCamlRaiseExn[highlightlines=720]{}}
      \onslide*<2>{\listCamlRaiseExn[highlightlines=722]{}}
      \onslide*<3->{\providecommand\step{5}\input{diagrams/backtrace/backtrace.tex}}
    \end{column}
  \end{columns}
\end{frame}

\begin{frame}{Backtraces}{\funcname{caml\_probably\_raise}'s handler doesn't catch \typename{ExnC}}
  \begin{columns}[c]
    \begin{column}{0.45\textwidth}
      \minipage[c][0.75\textheight][s]{\columnwidth}
      \begin{itemize}
        \item<1-> \funcname{match \dots with}\footnotemark pattern matching can also match exceptions
        \item<1-> The CMM of \funcname{probably\_raise} shows that this \funcname{match \dots with} is implemented with a pattern matching and a \funcname{try \dots with}
        \item<1-> But this one only matches on \typename{ExnA} exception
        \item<2-> Since the pattern matching fails to catch the exception, \funcname{reraise} is called with our current \typename{ExnC} exception
        \item<3-> Disassembly shows that \funcname{reraise} is actually a call to \funcname{caml\_reraise\_exn}, which is also an assembly function provided by the runtime
      \end{itemize}
      \vfill
      \mintocaml[firstline=15,lastline=21,highlightlines={18-21}]{../src/backtrace/backtrace.ml}
      \endminipage
    \end{column}
    \begin{column}{0.55\textwidth}
      \centering
      \onslide*<1>{\mintcmm[highlightlines={10-18}]{../src/backtrace/camlBacktrace\_\_probably\_raise\_351.cmm}}
      \onslide*<2>{\listcmm[highlightlines=18]{../src/backtrace/camlBacktrace\_\_probably\_raise_351.cmm}{CMM of probably\_raise}}
      \onslide*<3>{\mintobjdump[firstline=5,lastline=35,highlightlines=31]{../src/backtrace/camlBacktrace\_\_probably\_raise_351.objdump}}
    \end{column}
  \end{columns}
  \only<1>{\footnotetext[1]{https://blog.janestreet.com/pattern-matching-and-exception-handling-unite/}}
\end{frame}

\newcommand\listCamlReraiseExn[1][]{
  \listasm[firstline=727,lastline=734,#1]{../ocaml/runtime/amd64.S}{runtime/amd64.S:727}}

\begin{frame}{Backtraces}{Introducing \funcname{caml\_reraise\_exn}}
  \begin{columns}[c]
    \begin{column}{0.5\textwidth}
      \minipage[c][0.66\textheight][s]{\columnwidth}
      \begin{itemize}
        \item<1-> The exception have to be reraised to the next handler
        \item<2-> First, checks if the backtrace should be collected with \funcarg{domain\_state->backtrace\_active}
        \only<3>{
        \begin{itemize}
            \item If not, behaves like a \funcname{raise\_notrace} with \funcname{RESTORE\_EXN\_HANDLER\_OCAML}
        \end{itemize}
        }
        \item<4-> Jumps into \funcname{caml\_raise\_exn}
        \item<5-> Calls \funcname{caml\_stash\_backtrace} again, to scan the next part of the stack: from the trap just pop-ed to the next trap
      \end{itemize}
      \vfill
      \mintocaml[firstline=15,lastline=21,highlightlines={18-21}]{../src/backtrace/backtrace.ml}
      \endminipage
    \end{column}
    \begin{column}{0.5\textwidth}
      \raggedleft
      \onslide*<1>{\listCamlReraiseExn{}}
      \onslide*<2>{\listCamlReraiseExn[highlightlines=729]{}}
      \onslide*<3>{
        \mintc[firstline=190,lastline=192,highlightlines={191-192}]{../ocaml/runtime/amd64.S}
        \mintc[firstline=234,lastline=237,highlightlines={235-237}]{../ocaml/runtime/amd64.S}
        \medskip
        \listCamlReraiseExn[highlightlines=731]{}
      }
      \onslide*<4-5>{
        % TODO refactor with \listCamlReraiseExn
        \mintasm[firstline=727,lastline=734,highlightlines=730]{../ocaml/runtime/amd64.S}
        \medskip
      }
      \onslide*<4>{\listCamlRaiseExn[highlightlines=711]{}}
      \onslide*<5>{\listCamlRaiseExn[highlightlines=720]{}}
    \end{column}
  \end{columns}
\end{frame}

\begin{frame}{Backtraces}{Backtrace collection continues}
  \begin{columns}[c]
    \begin{column}{0.55\textwidth}
      \minipage[c][0.75\textheight][s]{\columnwidth}
      \begin{itemize}
        \item<1-> \funcname{caml\_stash\_backtrace} scans the older part of the stack, up to the next trap
        \item<6-> Controls jumps to the nested exception handler in \funcname{maybe\_raise}, which matches only on \typename{ExnB}
        \item<7-> \funcname{caml\_reraise\_exn} is called again, backtrace collection resumes with \funcname{caml\_stash\_backtrace}
      \end{itemize}
      \medskip
      \begin{itemize}
        \item<2-> Constructed backtrace:
        \begin{itemize}
          \item <2-> \funcname{definitely\_raise}
          \item <2-> \funcname{probably\_raise}
          \item <3-> \funcname{probably\_raise} (re-raise)
          \item <4-> \funcname{maybe\_raise}
          \item <5-> \funcname{main}
          \item <8-> \funcname{main} (re-raise)
        \end{itemize}
      \end{itemize}
      \vfill
      \onslide*<1-5>{\mintocaml[firstline=15,lastline=21]{../src/backtrace/backtrace.ml}}
      \onslide*<6>{\mintocaml[firstline=28,lastline=34,highlightlines=31]{../src/backtrace/backtrace.ml}}
      \onslide*<7->{\mintocaml[firstline=28,lastline=34,highlightlines=33]{../src/backtrace/backtrace.ml}}
      \endminipage
    \end{column}
    \begin{column}{0.45\textwidth}
      \raggedleft
      \onslide*<1>{\providecommand\step{6}\input{diagrams/backtrace/backtrace.tex}}%
      \onslide*<2>{\providecommand\step{6}\input{diagrams/backtrace/backtrace.tex}}%
      \onslide*<3>{\providecommand\step{7}\input{diagrams/backtrace/backtrace.tex}}%
      \onslide*<4>{\providecommand\step{8}\input{diagrams/backtrace/backtrace.tex}}%
      \onslide*<5>{\providecommand\step{9}\input{diagrams/backtrace/backtrace.tex}}%
      \onslide*<6>{\providecommand\step{10}\input{diagrams/backtrace/backtrace.tex}}%
      \onslide*<7>{\providecommand\step{11}\input{diagrams/backtrace/backtrace.tex}}%
      \onslide*<8>{\providecommand\step{12}\input{diagrams/backtrace/backtrace.tex}}%
      \onslide*<9>{\providecommand\step{13}\input{diagrams/backtrace/backtrace.tex}}%
    \end{column}
  \end{columns}
\end{frame}

\begin{frame}{Backtraces}{Printing the backtrace}
  \begin{columns}[c]
    \begin{column}{0.45\textwidth}
      \minipage[c][0.66\textheight][s]{\columnwidth}
      \begin{itemize}
        \item<1-> The exception backtrace can now be printed to \funcarg{stdout} with \funcname{Printexc.print_backtrace}
        \item<2-> First the exception backtrace is retrieved into an OCaml array with \funcname{get_raw_backtrace }
        \item<3-> Which is implemented as the \funcname{caml_get_exception_raw_backtrace} C function in the runtime
        \item<4-> And finally printed with \funcname{print_exception_backtrace}
      \end{itemize}
      \vfill
      \mintocaml[firstline=28,lastline=34,highlightlines=33]{../src/backtrace/backtrace.ml}
      \endminipage
    \end{column}
    \begin{column}{0.55\textwidth}
      \onslide*<1,2>{
        \mintocaml[firstline=100,lastline=101]{../ocaml/stdlib/printexc.ml}
      }
      \only<1>{
        \listocaml[firstline=170,lastline=172]{../ocaml/stdlib/printexc.ml}{stdlib/printexc.ml:170}
      }
      \only<2>{
        \listocaml[firstline=170,lastline=172,highlightlines=172]{../ocaml/stdlib/printexc.ml}{stdlib/printexc.ml:170}
      }
      \only<3>{
        \mintc[firstline=154,lastline=156]{../ocaml/runtime/backtrace.c}
        \listc[firstline=171,lastline=192,highlightlines={184-187}]{../ocaml/runtime/backtrace.c}{runtime/backtrace.c:154}
      }
      \only<4>{
        \listocaml[firstline=155,lastline=165]{../ocaml/stdlib/printexc.ml}{stdlib/printexc.ml:55}
      }
    \end{column}
  \end{columns}
\end{frame}


\subsection{The multiple kinds of raise}
\frameSubsection{}{}

\begin{frame}{Backtraces}{When backtraces are not needed}
  \begin{columns}[c]
    \begin{column}{0.45\textwidth}
      \minipage[c][0.66\textheight][s]{\columnwidth}
      \begin{itemize}
        \item<1-> What if the backtrace for an exception is never needed?
        \item<2-> It's possible to raise the exception explicitly with \funcname{raise\_notrace} to make raising as cheap as possible
        \item<3-> An example of use in the \funcname{dynlink} library of the OCaml compiler
      \end{itemize}
      \vfill
      \onslide<3->{
      \listocaml[firstline=205,lastline=209,highlightlines=207]{../ocaml/otherlibs/dynlink/dynlink\_compilerlibs/misc.ml}{otherlibs/dynlink/dynlink\_compilerlibs/misc.ml:205}
      }
      \endminipage
    \end{column}
    \begin{column}{0.55\textwidth}
    \only<4>{
      \mintcmm[highlightlines=3]{../src/notrace/camlNotrace\_\_fun_347.cmm}
      \listcmm{../src/notrace/camlNotrace\_\_all\_somes\_327.cmm}{all\_somes}
    }
    \onslide*<5->{
      \listobjdump[highlightlines={6-9}]{../src/notrace/camlNotrace\_\_fun_347.objdump}{anonymous function from \funcname{all\_somes}}
    }
    \end{column}
  \end{columns}
\end{frame}

\begin{frame}{Backtraces}{Summary table of the multiple kinds of raise}
  \begin{itemize}
    \item When debugging information is enabled and if the backtrace of an exception isn't needed, it's possible to raise with \funcname{raise\_notrace} for performance
    \item When debugging information is \emph{not} enabled, the compiler optimises \funcname{raise} calls to inline assembly (same effect as using \funcname{raise\_notrace})
  \end{itemize}
  \bigskip
  \centering
  \begin{tabular}{| l | c | c | c | c |}
    \hline
    \parbox{10em}{Debug information \\ (\funcarg{-g} flag)}  & \multicolumn{2}{|c|}{Disabled}                                     & \multicolumn{2}{|c|}{Enabled} \\
    \hline
    Raise kind                                               & \funcname{raise} & \funcname{raise\_notrace}                       & \funcname{raise} & \funcname{raise\_notrace} \\
    \hline
    Compilation                                              & \parbox{8em}{inline assembly\\ (optimization)} & inline assembly   & \funcname{caml\_raise\_exn} & inline assembly \\
    \hline
  \end{tabular}
\end{frame}


%
% Takeaway #5
%
\subsection*{Takeaway \#5}
\frameSubsectionTakeaway{}
\againframe<5>{takeaway}
}%
      \onslide*<9>{\providecommand\step{13}%
% Backtraces
%
\subsection{One advantage of raising with debugging information}
\frameSubsection{}{}

\begin{frame}{Backtraces}{Debugging information \& source code}
  \begin{columns}[c]
    \begin{column}{0.5\textwidth}
      \begin{itemize}
        \item Raising an exception is fun and games, but how do we know where the exception originated? $\rightarrow$ With the backtrace associated with the exception!
        \item In order to get a backtrace from the point where an exception was raised, it's necessary to compile with debugging information enabled (\funcarg{-g} flag for the compiler)
        \item Same example as in the `Nested handlers' section
      \end{itemize}
    \end{column}
    \begin{column}{0.5\textwidth}
      \listocaml[firstline=9,lastline=34]{../src/backtrace/backtrace.ml}{backtrace/backtrace.ml}
    \end{column}
  \end{columns}
\end{frame}

\begin{frame}{Backtraces}{CMM and disassembly of \funcname{definitely\_raise}}
  \begin{columns}[c]
    \begin{column}{0.5\textwidth}
      \minipage[c][0.66\textheight][s]{\columnwidth}
      \begin{itemize}
        \item<1-> An effect of compiling with debugging information (\funcarg{-g}): the CMM no longer uses \funcname{raise\_notrace} to raise the exception but \funcname{raise} instead
        \item<2-> Disassembly shows that CMM \funcname{raise} is actually a call to \funcname{caml\_raise\_exn}, a function in assembler provided by the runtime
      \end{itemize}
      \vfill
      \mintocaml[firstline=9,lastline=13,highlightlines=12]{../src/backtrace/backtrace.ml}
      \endminipage
    \end{column}
    \begin{column}{0.5\textwidth}
      \onslide*<1>{
        \mintcmm[highlightlines=5]{../src/backtrace/camlBacktrace\_\_definitely\_raise\_347.cmm}
      }
      \onslide*<2>{
        \mintobjdump[firstline=2,highlightlines=14]{../src/backtrace/camlBacktrace\_\_definitely\_raise\_347.objdump}
      }
    \end{column}
  \end{columns}
\end{frame}

\begin{frame}{Backtraces}{Introducing \funcname{caml\_raise\_exn}}
  \begin{columns}[c]
    \begin{column}{0.5\textwidth}
      \minipage[c][0.66\textheight][s]{\columnwidth}
      \onslide*<1>{
        \begin{itemize}
          \item Raising the exception is performed using \funcname{caml\_raise\_exn} which is
            \begin{itemize}
              \item An assembly function provided by the runtime
              \item Architecture specific
            \end{itemize}
          \item Let's see what it does differently from \funcname{raise\_notrace}\dots
          \item We are about to raise \typename{ExnC} with \funcname{caml\_raise\_exn} from \funcname{definitely\_raise}
        \end{itemize}
      }
      \onslide*<2>{
        \mintobjdump[firstline=2,highlightlines=14]{../src/backtrace/camlBacktrace\_\_definitely\_raise\_347.objdump}
      }
      \vfill
      \mintocaml[firstline=9,lastline=13,highlightlines=12]{../src/backtrace/backtrace.ml}
      \endminipage
    \end{column}
    \begin{column}{0.5\textwidth}
      \centering
      \providecommand\step{1}\input{diagrams/backtrace/backtrace.tex}
    \end{column}
  \end{columns}
\end{frame}

\begin{frame}{Backtraces}{But before that, we need more fields from \typename{caml\_domain\_state}}
  \begin{columns}
    \begin{column}{0.5\textwidth}
      \begin{itemize}
        \item Remember \typename{caml\_domain\_state} the per-domain runtime state?
        \item Some other members are of interest to us now\dots
        \item \localname{backtrace\_active} is used to know if the runtime should collect backtraces
        \item \localname{backtrace\_active} can be set by
          \begin{itemize}
            \item The \funcarg{b} flag in the \funcname{OCAMLRUNPARAM} environment variable
            \item \mintinline{ocaml}{Printexc.record_backtrace : bool -> unit} \footnotemark
          \end{itemize}
      \end{itemize}
    \end{column}
    \begin{column}{0.5\textwidth}
      \begin{listing}
        \mintc[highlightlines={9-11}]{src/ocaml/domain\_state\_backtrace.h}
        \caption{runtime/caml/domain\_state.h}
      \end{listing}
    \end{column}
  \end{columns}
  \footnotetext[1]{https://v2.ocaml.org/api/Printexc.html}
\end{frame}

\newcommand\listCamlRaiseExn[1][]{
  \listasm[firstline=702,lastline=725,#1]{../ocaml/runtime/amd64.S}{runtime/amd64.S:702}}

\begin{frame}{Backtraces}{Walk through \funcname{caml\_raise\_exn}}
  \begin{columns}[T]
    \begin{column}{0.5\textwidth}
      \begin{itemize}
        \item<1-> First checks whether exceptions backtrace should be recorded
        \item<1-> By checking the \funcarg{domain\_state->backtrace\_active} value
        \item<2-> If not, acts as \funcname{raise\_notrace} with \funcname{RESTORE\_EXN\_HANDLER\_OCAML}
          \begin{itemize}
            \item Set \regname{RSP} to \localname{domain\_state->exn\_handler} value
            \item Restore \localname{domain\_state->exn\_handler} from trap
            \item Jump with \funcname{ret} (same effect as using \funcname{pop} and \funcname{jmp})
          \end{itemize}
        \item<3-> Otherwise, switches to the C stack to be able to execute C code
        \item<4-> Calls \funcname{caml\_stash\_backtrace}, a C function provided by the runtime\ignorespaces
          \onslide<6->{, with
            \begin{itemize}
              \item \funcarg{exn}: exception value
              \item \funcarg{pc}: code address of the raising point
              \item \funcarg{sp}: stack address of the raising function
              \item \funcarg{trapsp}: stack address of the next handler
            \end{itemize}
          }
      \end{itemize}
      \bigskip
      \mintocaml[firstline=9,lastline=13,highlightlines=12]{../src/backtrace/backtrace.ml}
    \end{column}
    \begin{column}{0.5\textwidth}
      \raggedleft
      \onslide*<1,3-4>{
        \mintc[firstline=190,lastline=192]{../ocaml/runtime/amd64.S}
        \mintc[firstline=234,lastline=237]{../ocaml/runtime/amd64.S}
      }
      \onslide*<2>{
        \mintc[firstline=190,lastline=192,highlightlines={191-192}]{../ocaml/runtime/amd64.S}
        \mintc[firstline=234,lastline=237,highlightlines={235-237}]{../ocaml/runtime/amd64.S}
      }
      \onslide*<1>{\listCamlRaiseExn[highlightlines=705]{}}%
      \onslide*<2>{\listCamlRaiseExn[highlightlines={707-708}]{}}%
      \onslide*<3>{\listCamlRaiseExn[highlightlines={712-714}]{}}%
      \onslide*<4>{\listCamlRaiseExn[highlightlines={715-720}]{}}%
      \onslide*<5>{\providecommand\step{1}\input{diagrams/backtrace/backtrace.tex}}%
      \onslide*<6>{\providecommand\step{2}\input{diagrams/backtrace/backtrace.tex}}%
    \end{column}
  \end{columns}
\end{frame}

\newcommand\listStashBacktrace[1][]{
  \listc[firstline=91,lastline=120,#1]{../ocaml/runtime/backtrace\_nat.c}{runtime/backtrace\_nat.c:91}}

\begin{frame}{Backtraces}{Introducing \funcname{caml\_stash\_backtrace}}
  \begin{columns}[c]
    \begin{column}{0.5\textwidth}
      \minipage[c][0.66\textheight][s]{\columnwidth}
      \begin{itemize}
        \item<1-> A C function provided by the runtime (specific to native code)
        \item<2-> It scans the stack, between the stack pointer at the raising point up to the next trap, in order to collect the names of the detected function frames
          \onslide*<2>{
            \begin{itemize}
              \item And more information, like line number, column number...
            \end{itemize}
          }
        \item<3-> It uses the same technique and information as the Garbage Collector (GC) uses to scan the values live on the stack
          \onslide*<4>{
            \begin{itemize}
              \item \typename{frame\_descr} are at the core of stack unwinding
            \end{itemize}
          }
      \end{itemize}
      \vfill
      \mintocaml[firstline=9,lastline=13,highlightlines=12]{../src/backtrace/backtrace.ml}
      \endminipage
    \end{column}
    \begin{column}{0.5\textwidth}
      \onslide*<1>{\listStashBacktrace[highlightlines=91]{}}
      \onslide*<2>{\listStashBacktrace[highlightlines={91,117-118}]{}}
      \onslide*<3->{\listStashBacktrace[highlightlines={94,106,109-110}]{}}
    \end{column}
  \end{columns}
\end{frame}

\newcommand\listFrameDescr[1][]{
  \listocaml[firstline=111,lastline=124,#1]{../ocaml/asmcomp/emitaux.ml}{asmcomp/emitaux.ml:111}}
\newcommand\listDebugInfo[1][]{
  \listocaml[firstline=38,lastline=49,#1]{../ocaml/lambda/debuginfo.mli}{lambda/debuginfo.mli:38}}

\begin{frame}{Backtraces}{\typename{frame\_descr} and \typename{frame\_debuginfo} in a nutshell}
  \begin{columns}[c]
    \begin{column}{0.5\textwidth}
      \begin{itemize}
        \item<1-> For every return address in an OCaml program, there is a associated \typename{frame\_descr}
        \item<2-> A \typename{frame\_descr} specifies
        \begin{itemize}
          \item<2-> The size of the function stack frame
          \item<3-> Where live values are on the stack (so that the GC can mark them as live and prevent from being swept)
        \end{itemize}
        \item<4-> When compiled with debug information, \typename{frame\_descr} will be linked to a \typename{frame\_debuginfo}
        \begin{itemize}
          \item<5-> Contains information about the position in the source code (file name, line and column number)
        \end{itemize}
      \end{itemize}
    \end{column}
    \begin{column}{0.5\textwidth}
        \onslide*<1>{\listFrameDescr[highlightlines={118-122}]{}}
        \onslide*<2>{\listFrameDescr[highlightlines={120}]{}}
        \onslide*<3>{\listFrameDescr[highlightlines={121}]{}}
        \onslide*<4->{\listFrameDescr[highlightlines={122}]{}}
        \medskip
        \onslide*<1-3>{\listDebugInfo{}}
        \onslide*<4>{\listDebugInfo[highlightlines={38,49}]{}}
        \onslide*<5>{\listDebugInfo[highlightlines={39-46}]{}}
    \end{column}
  \end{columns}
\end{frame}

\begin{frame}{Backtraces}{Scanning the stack with \typename{frame\_descr}}
  \begin{columns}[c]
    \begin{column}{0.5\textwidth}
      \begin{itemize}
        \item Given the return address of the code that raises the exception by calling \funcname{caml\_raise\_exn} (\funcarg{pc} argument of \funcname{caml\_stash\_backtrace})
        \item And the position of the stack pointer (\funcarg{sp} argument of \funcname{caml\_stash\_backtrace})
        \item The frame size given by \typename{frame\_descr}.\funcarg{frame\_size}, allows to find the end of the previous frame
        \item And the return address of the caller of this function
        \bigskip
        \onslide<3->{
        \item Note that traps (here the reddish \funcname{ExnA}, \funcname{ExnB} and \funcname{ExnC} blocks) belong to the frame that installed them, and so the frame size changes when traps are removed
        }
      \end{itemize}
    \end{column}
    \begin{column}{0.5\textwidth}
      \raggedleft
      \onslide*<1>{\providecommand\step{2}\input{diagrams/backtrace/backtrace.tex}}%
      \onslide*<2>{\providecommand\step{1}\input{diagrams/backtrace/scanstack.tex}}%
      \onslide*<3>{\providecommand\step{2}\input{diagrams/backtrace/scanstack.tex}}%
      \onslide*<4>{\providecommand\step{3}\input{diagrams/backtrace/scanstack.tex}}%
      \onslide*<5>{\providecommand\step{4}\input{diagrams/backtrace/scanstack.tex}}%
      \onslide*<6>{\providecommand\step{5}\input{diagrams/backtrace/scanstack.tex}}%
    \end{column}
  \end{columns}
\end{frame}

% Duplicate of previous-previous slide
\begin{frame}{Backtraces}{Continuing on \funcname{caml\_stash\_backtrace}}
  \begin{columns}[c]
    \begin{column}{0.5\textwidth}
      \minipage[c][0.75\textheight][s]{\columnwidth}
      \begin{itemize}
          \color{gray}
        \item A C function provided by the runtime (specific to native code)
        \item It scans the stack, between the stack pointer at the raising point up to the next trap, in order to collect the names of the detected function frames
        \item It uses the same technique and information as the Garbage Collector (GC) uses to scan the values live on the stack
      \end{itemize}
      \begin{itemize}
        \item For each function frame detected, store a pointer to the \typename{frame\_descr} in the \localname{domain\_state->backtrace\_buffer} array, effectively constructing the backtrace
      \end{itemize}
      \onslide<2->{
        \begin{itemize}
            \smallskip
          \item Constructed backtrace:
            \funcname{definitely\_raise},
            \onslide<3->{\funcname{probably\_raise}}
        \end{itemize}
      }
      \vfill
      \mintocaml[firstline=9,lastline=13,highlightlines=12]{../src/backtrace/backtrace.ml}
      \endminipage
    \end{column}
    \begin{column}{0.5\textwidth}
      \raggedleft
      \onslide*<1>{\listStashBacktrace[highlightlines={114-115}]{}}
      \onslide*<2>{\providecommand\step{3}\input{diagrams/backtrace/backtrace.tex}}%
      \onslide*<3>{\providecommand\step{4}\input{diagrams/backtrace/backtrace.tex}}%
    \end{column}
  \end{columns}
\end{frame}

\begin{frame}{Backtraces}{Back to \funcname{caml\_raise\_exn}}
  \begin{columns}[c]
    \begin{column}{0.5\textwidth}
      \minipage[c][0.66\textheight][s]{\columnwidth}
      \begin{itemize}\color{gray}
        \item Checks wheteher exceptions backtrace should be recorded, by checking the \funcarg{domain\_state->backtrace\_active} value
        \item Switches to the C stack to be able to execute C code
        \item Calls \funcname{caml\_stash\_backtrace}, to collect the backtrace up to the next trap
      \end{itemize}
      \onslide*<2->{
        \begin{itemize}
          \item<2-> Executes the exception handler in \funcname{probably\_raise} with \funcname{RESTORE\_EXN\_HANDLER\_OCAML}
            \begin{itemize}
              \item Set \regname{RSP} to \localname{domain\_state->exn\_handler} value
              \item Restore \localname{domain\_state->exn\_handler} from trap
              \item Jump to the handler code with \funcname{ret}
            \end{itemize}
        \end{itemize}
      }
      \vfill
      \onslide*<1>{\mintocaml[firstline=9,lastline=13,highlightlines=12]{../src/backtrace/backtrace.ml}}
      \onslide*<2->{\mintocaml[firstline=15,lastline=21,highlightlines={19-21}]{../src/backtrace/backtrace.ml}}
      \endminipage
    \end{column}
    \begin{column}{0.5\textwidth}
      \centering
      \onslide*<1>{
        \mintc[firstline=190,lastline=192]{../ocaml/runtime/amd64.S}
        \mintc[firstline=234,lastline=237]{../ocaml/runtime/amd64.S}
      }
      \onslide*<2>{
        \mintc[firstline=190,lastline=192,highlightlines={191-192}]{../ocaml/runtime/amd64.S}
        \mintc[firstline=234,lastline=237,highlightlines={235-237}]{../ocaml/runtime/amd64.S}
      }
      \onslide*<1>{\listCamlRaiseExn[highlightlines=720]{}}
      \onslide*<2>{\listCamlRaiseExn[highlightlines=722]{}}
      \onslide*<3->{\providecommand\step{5}\input{diagrams/backtrace/backtrace.tex}}
    \end{column}
  \end{columns}
\end{frame}

\begin{frame}{Backtraces}{\funcname{caml\_probably\_raise}'s handler doesn't catch \typename{ExnC}}
  \begin{columns}[c]
    \begin{column}{0.45\textwidth}
      \minipage[c][0.75\textheight][s]{\columnwidth}
      \begin{itemize}
        \item<1-> \funcname{match \dots with}\footnotemark pattern matching can also match exceptions
        \item<1-> The CMM of \funcname{probably\_raise} shows that this \funcname{match \dots with} is implemented with a pattern matching and a \funcname{try \dots with}
        \item<1-> But this one only matches on \typename{ExnA} exception
        \item<2-> Since the pattern matching fails to catch the exception, \funcname{reraise} is called with our current \typename{ExnC} exception
        \item<3-> Disassembly shows that \funcname{reraise} is actually a call to \funcname{caml\_reraise\_exn}, which is also an assembly function provided by the runtime
      \end{itemize}
      \vfill
      \mintocaml[firstline=15,lastline=21,highlightlines={18-21}]{../src/backtrace/backtrace.ml}
      \endminipage
    \end{column}
    \begin{column}{0.55\textwidth}
      \centering
      \onslide*<1>{\mintcmm[highlightlines={10-18}]{../src/backtrace/camlBacktrace\_\_probably\_raise\_351.cmm}}
      \onslide*<2>{\listcmm[highlightlines=18]{../src/backtrace/camlBacktrace\_\_probably\_raise_351.cmm}{CMM of probably\_raise}}
      \onslide*<3>{\mintobjdump[firstline=5,lastline=35,highlightlines=31]{../src/backtrace/camlBacktrace\_\_probably\_raise_351.objdump}}
    \end{column}
  \end{columns}
  \only<1>{\footnotetext[1]{https://blog.janestreet.com/pattern-matching-and-exception-handling-unite/}}
\end{frame}

\newcommand\listCamlReraiseExn[1][]{
  \listasm[firstline=727,lastline=734,#1]{../ocaml/runtime/amd64.S}{runtime/amd64.S:727}}

\begin{frame}{Backtraces}{Introducing \funcname{caml\_reraise\_exn}}
  \begin{columns}[c]
    \begin{column}{0.5\textwidth}
      \minipage[c][0.66\textheight][s]{\columnwidth}
      \begin{itemize}
        \item<1-> The exception have to be reraised to the next handler
        \item<2-> First, checks if the backtrace should be collected with \funcarg{domain\_state->backtrace\_active}
        \only<3>{
        \begin{itemize}
            \item If not, behaves like a \funcname{raise\_notrace} with \funcname{RESTORE\_EXN\_HANDLER\_OCAML}
        \end{itemize}
        }
        \item<4-> Jumps into \funcname{caml\_raise\_exn}
        \item<5-> Calls \funcname{caml\_stash\_backtrace} again, to scan the next part of the stack: from the trap just pop-ed to the next trap
      \end{itemize}
      \vfill
      \mintocaml[firstline=15,lastline=21,highlightlines={18-21}]{../src/backtrace/backtrace.ml}
      \endminipage
    \end{column}
    \begin{column}{0.5\textwidth}
      \raggedleft
      \onslide*<1>{\listCamlReraiseExn{}}
      \onslide*<2>{\listCamlReraiseExn[highlightlines=729]{}}
      \onslide*<3>{
        \mintc[firstline=190,lastline=192,highlightlines={191-192}]{../ocaml/runtime/amd64.S}
        \mintc[firstline=234,lastline=237,highlightlines={235-237}]{../ocaml/runtime/amd64.S}
        \medskip
        \listCamlReraiseExn[highlightlines=731]{}
      }
      \onslide*<4-5>{
        % TODO refactor with \listCamlReraiseExn
        \mintasm[firstline=727,lastline=734,highlightlines=730]{../ocaml/runtime/amd64.S}
        \medskip
      }
      \onslide*<4>{\listCamlRaiseExn[highlightlines=711]{}}
      \onslide*<5>{\listCamlRaiseExn[highlightlines=720]{}}
    \end{column}
  \end{columns}
\end{frame}

\begin{frame}{Backtraces}{Backtrace collection continues}
  \begin{columns}[c]
    \begin{column}{0.55\textwidth}
      \minipage[c][0.75\textheight][s]{\columnwidth}
      \begin{itemize}
        \item<1-> \funcname{caml\_stash\_backtrace} scans the older part of the stack, up to the next trap
        \item<6-> Controls jumps to the nested exception handler in \funcname{maybe\_raise}, which matches only on \typename{ExnB}
        \item<7-> \funcname{caml\_reraise\_exn} is called again, backtrace collection resumes with \funcname{caml\_stash\_backtrace}
      \end{itemize}
      \medskip
      \begin{itemize}
        \item<2-> Constructed backtrace:
        \begin{itemize}
          \item <2-> \funcname{definitely\_raise}
          \item <2-> \funcname{probably\_raise}
          \item <3-> \funcname{probably\_raise} (re-raise)
          \item <4-> \funcname{maybe\_raise}
          \item <5-> \funcname{main}
          \item <8-> \funcname{main} (re-raise)
        \end{itemize}
      \end{itemize}
      \vfill
      \onslide*<1-5>{\mintocaml[firstline=15,lastline=21]{../src/backtrace/backtrace.ml}}
      \onslide*<6>{\mintocaml[firstline=28,lastline=34,highlightlines=31]{../src/backtrace/backtrace.ml}}
      \onslide*<7->{\mintocaml[firstline=28,lastline=34,highlightlines=33]{../src/backtrace/backtrace.ml}}
      \endminipage
    \end{column}
    \begin{column}{0.45\textwidth}
      \raggedleft
      \onslide*<1>{\providecommand\step{6}\input{diagrams/backtrace/backtrace.tex}}%
      \onslide*<2>{\providecommand\step{6}\input{diagrams/backtrace/backtrace.tex}}%
      \onslide*<3>{\providecommand\step{7}\input{diagrams/backtrace/backtrace.tex}}%
      \onslide*<4>{\providecommand\step{8}\input{diagrams/backtrace/backtrace.tex}}%
      \onslide*<5>{\providecommand\step{9}\input{diagrams/backtrace/backtrace.tex}}%
      \onslide*<6>{\providecommand\step{10}\input{diagrams/backtrace/backtrace.tex}}%
      \onslide*<7>{\providecommand\step{11}\input{diagrams/backtrace/backtrace.tex}}%
      \onslide*<8>{\providecommand\step{12}\input{diagrams/backtrace/backtrace.tex}}%
      \onslide*<9>{\providecommand\step{13}\input{diagrams/backtrace/backtrace.tex}}%
    \end{column}
  \end{columns}
\end{frame}

\begin{frame}{Backtraces}{Printing the backtrace}
  \begin{columns}[c]
    \begin{column}{0.45\textwidth}
      \minipage[c][0.66\textheight][s]{\columnwidth}
      \begin{itemize}
        \item<1-> The exception backtrace can now be printed to \funcarg{stdout} with \funcname{Printexc.print_backtrace}
        \item<2-> First the exception backtrace is retrieved into an OCaml array with \funcname{get_raw_backtrace }
        \item<3-> Which is implemented as the \funcname{caml_get_exception_raw_backtrace} C function in the runtime
        \item<4-> And finally printed with \funcname{print_exception_backtrace}
      \end{itemize}
      \vfill
      \mintocaml[firstline=28,lastline=34,highlightlines=33]{../src/backtrace/backtrace.ml}
      \endminipage
    \end{column}
    \begin{column}{0.55\textwidth}
      \onslide*<1,2>{
        \mintocaml[firstline=100,lastline=101]{../ocaml/stdlib/printexc.ml}
      }
      \only<1>{
        \listocaml[firstline=170,lastline=172]{../ocaml/stdlib/printexc.ml}{stdlib/printexc.ml:170}
      }
      \only<2>{
        \listocaml[firstline=170,lastline=172,highlightlines=172]{../ocaml/stdlib/printexc.ml}{stdlib/printexc.ml:170}
      }
      \only<3>{
        \mintc[firstline=154,lastline=156]{../ocaml/runtime/backtrace.c}
        \listc[firstline=171,lastline=192,highlightlines={184-187}]{../ocaml/runtime/backtrace.c}{runtime/backtrace.c:154}
      }
      \only<4>{
        \listocaml[firstline=155,lastline=165]{../ocaml/stdlib/printexc.ml}{stdlib/printexc.ml:55}
      }
    \end{column}
  \end{columns}
\end{frame}


\subsection{The multiple kinds of raise}
\frameSubsection{}{}

\begin{frame}{Backtraces}{When backtraces are not needed}
  \begin{columns}[c]
    \begin{column}{0.45\textwidth}
      \minipage[c][0.66\textheight][s]{\columnwidth}
      \begin{itemize}
        \item<1-> What if the backtrace for an exception is never needed?
        \item<2-> It's possible to raise the exception explicitly with \funcname{raise\_notrace} to make raising as cheap as possible
        \item<3-> An example of use in the \funcname{dynlink} library of the OCaml compiler
      \end{itemize}
      \vfill
      \onslide<3->{
      \listocaml[firstline=205,lastline=209,highlightlines=207]{../ocaml/otherlibs/dynlink/dynlink\_compilerlibs/misc.ml}{otherlibs/dynlink/dynlink\_compilerlibs/misc.ml:205}
      }
      \endminipage
    \end{column}
    \begin{column}{0.55\textwidth}
    \only<4>{
      \mintcmm[highlightlines=3]{../src/notrace/camlNotrace\_\_fun_347.cmm}
      \listcmm{../src/notrace/camlNotrace\_\_all\_somes\_327.cmm}{all\_somes}
    }
    \onslide*<5->{
      \listobjdump[highlightlines={6-9}]{../src/notrace/camlNotrace\_\_fun_347.objdump}{anonymous function from \funcname{all\_somes}}
    }
    \end{column}
  \end{columns}
\end{frame}

\begin{frame}{Backtraces}{Summary table of the multiple kinds of raise}
  \begin{itemize}
    \item When debugging information is enabled and if the backtrace of an exception isn't needed, it's possible to raise with \funcname{raise\_notrace} for performance
    \item When debugging information is \emph{not} enabled, the compiler optimises \funcname{raise} calls to inline assembly (same effect as using \funcname{raise\_notrace})
  \end{itemize}
  \bigskip
  \centering
  \begin{tabular}{| l | c | c | c | c |}
    \hline
    \parbox{10em}{Debug information \\ (\funcarg{-g} flag)}  & \multicolumn{2}{|c|}{Disabled}                                     & \multicolumn{2}{|c|}{Enabled} \\
    \hline
    Raise kind                                               & \funcname{raise} & \funcname{raise\_notrace}                       & \funcname{raise} & \funcname{raise\_notrace} \\
    \hline
    Compilation                                              & \parbox{8em}{inline assembly\\ (optimization)} & inline assembly   & \funcname{caml\_raise\_exn} & inline assembly \\
    \hline
  \end{tabular}
\end{frame}


%
% Takeaway #5
%
\subsection*{Takeaway \#5}
\frameSubsectionTakeaway{}
\againframe<5>{takeaway}
}%
    \end{column}
  \end{columns}
\end{frame}

\begin{frame}{Backtraces}{Printing the backtrace}
  \begin{columns}[c]
    \begin{column}{0.45\textwidth}
      \minipage[c][0.66\textheight][s]{\columnwidth}
      \begin{itemize}
        \item<1-> The exception backtrace can now be printed to \funcarg{stdout} with \funcname{Printexc.print_backtrace}
        \item<2-> First the exception backtrace is retrieved into an OCaml array with \funcname{get_raw_backtrace }
        \item<3-> Which is implemented as the \funcname{caml_get_exception_raw_backtrace} C function in the runtime
        \item<4-> And finally printed with \funcname{print_exception_backtrace}
      \end{itemize}
      \vfill
      \mintocaml[firstline=28,lastline=34,highlightlines=33]{../src/backtrace/backtrace.ml}
      \endminipage
    \end{column}
    \begin{column}{0.55\textwidth}
      \onslide*<1,2>{
        \mintocaml[firstline=100,lastline=101]{../ocaml/stdlib/printexc.ml}
      }
      \only<1>{
        \listocaml[firstline=170,lastline=172]{../ocaml/stdlib/printexc.ml}{stdlib/printexc.ml:170}
      }
      \only<2>{
        \listocaml[firstline=170,lastline=172,highlightlines=172]{../ocaml/stdlib/printexc.ml}{stdlib/printexc.ml:170}
      }
      \only<3>{
        \mintc[firstline=154,lastline=156]{../ocaml/runtime/backtrace.c}
        \listc[firstline=171,lastline=192,highlightlines={184-187}]{../ocaml/runtime/backtrace.c}{runtime/backtrace.c:154}
      }
      \only<4>{
        \listocaml[firstline=155,lastline=165]{../ocaml/stdlib/printexc.ml}{stdlib/printexc.ml:55}
      }
    \end{column}
  \end{columns}
\end{frame}


\subsection{The multiple kinds of raise}
\frameSubsection{}{}

\begin{frame}{Backtraces}{When backtraces are not needed}
  \begin{columns}[c]
    \begin{column}{0.45\textwidth}
      \minipage[c][0.66\textheight][s]{\columnwidth}
      \begin{itemize}
        \item<1-> What if the backtrace for an exception is never needed?
        \item<2-> It's possible to raise the exception explicitly with \funcname{raise\_notrace} to make raising as cheap as possible
        \item<3-> An example of use in the \funcname{dynlink} library of the OCaml compiler
      \end{itemize}
      \vfill
      \onslide<3->{
      \listocaml[firstline=205,lastline=209,highlightlines=207]{../ocaml/otherlibs/dynlink/dynlink\_compilerlibs/misc.ml}{otherlibs/dynlink/dynlink\_compilerlibs/misc.ml:205}
      }
      \endminipage
    \end{column}
    \begin{column}{0.55\textwidth}
    \only<4>{
      \mintcmm[highlightlines=3]{../src/notrace/camlNotrace\_\_fun_347.cmm}
      \listcmm{../src/notrace/camlNotrace\_\_all\_somes\_327.cmm}{all\_somes}
    }
    \onslide*<5->{
      \listobjdump[highlightlines={6-9}]{../src/notrace/camlNotrace\_\_fun_347.objdump}{anonymous function from \funcname{all\_somes}}
    }
    \end{column}
  \end{columns}
\end{frame}

\begin{frame}{Backtraces}{Summary table of the multiple kinds of raise}
  \begin{itemize}
    \item When debugging information is enabled and if the backtrace of an exception isn't needed, it's possible to raise with \funcname{raise\_notrace} for performance
    \item When debugging information is \emph{not} enabled, the compiler optimises \funcname{raise} calls to inline assembly (same effect as using \funcname{raise\_notrace})
  \end{itemize}
  \bigskip
  \centering
  \begin{tabular}{| l | c | c | c | c |}
    \hline
    \parbox{10em}{Debug information \\ (\funcarg{-g} flag)}  & \multicolumn{2}{|c|}{Disabled}                                     & \multicolumn{2}{|c|}{Enabled} \\
    \hline
    Raise kind                                               & \funcname{raise} & \funcname{raise\_notrace}                       & \funcname{raise} & \funcname{raise\_notrace} \\
    \hline
    Compilation                                              & \parbox{8em}{inline assembly\\ (optimization)} & inline assembly   & \funcname{caml\_raise\_exn} & inline assembly \\
    \hline
  \end{tabular}
\end{frame}


%
% Takeaway #5
%
\subsection*{Takeaway \#5}
\frameSubsectionTakeaway{}
\againframe<5>{takeaway}
}%
      \onslide*<7>{\providecommand\step{11}%
% Backtraces
%
\subsection{One advantage of raising with debugging information}
\frameSubsection{}{}

\begin{frame}{Backtraces}{Debugging information \& source code}
  \begin{columns}[c]
    \begin{column}{0.5\textwidth}
      \begin{itemize}
        \item Raising an exception is fun and games, but how do we know where the exception originated? $\rightarrow$ With the backtrace associated with the exception!
        \item In order to get a backtrace from the point where an exception was raised, it's necessary to compile with debugging information enabled (\funcarg{-g} flag for the compiler)
        \item Same example as in the `Nested handlers' section
      \end{itemize}
    \end{column}
    \begin{column}{0.5\textwidth}
      \listocaml[firstline=9,lastline=34]{../src/backtrace/backtrace.ml}{backtrace/backtrace.ml}
    \end{column}
  \end{columns}
\end{frame}

\begin{frame}{Backtraces}{CMM and disassembly of \funcname{definitely\_raise}}
  \begin{columns}[c]
    \begin{column}{0.5\textwidth}
      \minipage[c][0.66\textheight][s]{\columnwidth}
      \begin{itemize}
        \item<1-> An effect of compiling with debugging information (\funcarg{-g}): the CMM no longer uses \funcname{raise\_notrace} to raise the exception but \funcname{raise} instead
        \item<2-> Disassembly shows that CMM \funcname{raise} is actually a call to \funcname{caml\_raise\_exn}, a function in assembler provided by the runtime
      \end{itemize}
      \vfill
      \mintocaml[firstline=9,lastline=13,highlightlines=12]{../src/backtrace/backtrace.ml}
      \endminipage
    \end{column}
    \begin{column}{0.5\textwidth}
      \onslide*<1>{
        \mintcmm[highlightlines=5]{../src/backtrace/camlBacktrace\_\_definitely\_raise\_347.cmm}
      }
      \onslide*<2>{
        \mintobjdump[firstline=2,highlightlines=14]{../src/backtrace/camlBacktrace\_\_definitely\_raise\_347.objdump}
      }
    \end{column}
  \end{columns}
\end{frame}

\begin{frame}{Backtraces}{Introducing \funcname{caml\_raise\_exn}}
  \begin{columns}[c]
    \begin{column}{0.5\textwidth}
      \minipage[c][0.66\textheight][s]{\columnwidth}
      \onslide*<1>{
        \begin{itemize}
          \item Raising the exception is performed using \funcname{caml\_raise\_exn} which is
            \begin{itemize}
              \item An assembly function provided by the runtime
              \item Architecture specific
            \end{itemize}
          \item Let's see what it does differently from \funcname{raise\_notrace}\dots
          \item We are about to raise \typename{ExnC} with \funcname{caml\_raise\_exn} from \funcname{definitely\_raise}
        \end{itemize}
      }
      \onslide*<2>{
        \mintobjdump[firstline=2,highlightlines=14]{../src/backtrace/camlBacktrace\_\_definitely\_raise\_347.objdump}
      }
      \vfill
      \mintocaml[firstline=9,lastline=13,highlightlines=12]{../src/backtrace/backtrace.ml}
      \endminipage
    \end{column}
    \begin{column}{0.5\textwidth}
      \centering
      \providecommand\step{1}%
% Backtraces
%
\subsection{One advantage of raising with debugging information}
\frameSubsection{}{}

\begin{frame}{Backtraces}{Debugging information \& source code}
  \begin{columns}[c]
    \begin{column}{0.5\textwidth}
      \begin{itemize}
        \item Raising an exception is fun and games, but how do we know where the exception originated? $\rightarrow$ With the backtrace associated with the exception!
        \item In order to get a backtrace from the point where an exception was raised, it's necessary to compile with debugging information enabled (\funcarg{-g} flag for the compiler)
        \item Same example as in the `Nested handlers' section
      \end{itemize}
    \end{column}
    \begin{column}{0.5\textwidth}
      \listocaml[firstline=9,lastline=34]{../src/backtrace/backtrace.ml}{backtrace/backtrace.ml}
    \end{column}
  \end{columns}
\end{frame}

\begin{frame}{Backtraces}{CMM and disassembly of \funcname{definitely\_raise}}
  \begin{columns}[c]
    \begin{column}{0.5\textwidth}
      \minipage[c][0.66\textheight][s]{\columnwidth}
      \begin{itemize}
        \item<1-> An effect of compiling with debugging information (\funcarg{-g}): the CMM no longer uses \funcname{raise\_notrace} to raise the exception but \funcname{raise} instead
        \item<2-> Disassembly shows that CMM \funcname{raise} is actually a call to \funcname{caml\_raise\_exn}, a function in assembler provided by the runtime
      \end{itemize}
      \vfill
      \mintocaml[firstline=9,lastline=13,highlightlines=12]{../src/backtrace/backtrace.ml}
      \endminipage
    \end{column}
    \begin{column}{0.5\textwidth}
      \onslide*<1>{
        \mintcmm[highlightlines=5]{../src/backtrace/camlBacktrace\_\_definitely\_raise\_347.cmm}
      }
      \onslide*<2>{
        \mintobjdump[firstline=2,highlightlines=14]{../src/backtrace/camlBacktrace\_\_definitely\_raise\_347.objdump}
      }
    \end{column}
  \end{columns}
\end{frame}

\begin{frame}{Backtraces}{Introducing \funcname{caml\_raise\_exn}}
  \begin{columns}[c]
    \begin{column}{0.5\textwidth}
      \minipage[c][0.66\textheight][s]{\columnwidth}
      \onslide*<1>{
        \begin{itemize}
          \item Raising the exception is performed using \funcname{caml\_raise\_exn} which is
            \begin{itemize}
              \item An assembly function provided by the runtime
              \item Architecture specific
            \end{itemize}
          \item Let's see what it does differently from \funcname{raise\_notrace}\dots
          \item We are about to raise \typename{ExnC} with \funcname{caml\_raise\_exn} from \funcname{definitely\_raise}
        \end{itemize}
      }
      \onslide*<2>{
        \mintobjdump[firstline=2,highlightlines=14]{../src/backtrace/camlBacktrace\_\_definitely\_raise\_347.objdump}
      }
      \vfill
      \mintocaml[firstline=9,lastline=13,highlightlines=12]{../src/backtrace/backtrace.ml}
      \endminipage
    \end{column}
    \begin{column}{0.5\textwidth}
      \centering
      \providecommand\step{1}\input{diagrams/backtrace/backtrace.tex}
    \end{column}
  \end{columns}
\end{frame}

\begin{frame}{Backtraces}{But before that, we need more fields from \typename{caml\_domain\_state}}
  \begin{columns}
    \begin{column}{0.5\textwidth}
      \begin{itemize}
        \item Remember \typename{caml\_domain\_state} the per-domain runtime state?
        \item Some other members are of interest to us now\dots
        \item \localname{backtrace\_active} is used to know if the runtime should collect backtraces
        \item \localname{backtrace\_active} can be set by
          \begin{itemize}
            \item The \funcarg{b} flag in the \funcname{OCAMLRUNPARAM} environment variable
            \item \mintinline{ocaml}{Printexc.record_backtrace : bool -> unit} \footnotemark
          \end{itemize}
      \end{itemize}
    \end{column}
    \begin{column}{0.5\textwidth}
      \begin{listing}
        \mintc[highlightlines={9-11}]{src/ocaml/domain\_state\_backtrace.h}
        \caption{runtime/caml/domain\_state.h}
      \end{listing}
    \end{column}
  \end{columns}
  \footnotetext[1]{https://v2.ocaml.org/api/Printexc.html}
\end{frame}

\newcommand\listCamlRaiseExn[1][]{
  \listasm[firstline=702,lastline=725,#1]{../ocaml/runtime/amd64.S}{runtime/amd64.S:702}}

\begin{frame}{Backtraces}{Walk through \funcname{caml\_raise\_exn}}
  \begin{columns}[T]
    \begin{column}{0.5\textwidth}
      \begin{itemize}
        \item<1-> First checks whether exceptions backtrace should be recorded
        \item<1-> By checking the \funcarg{domain\_state->backtrace\_active} value
        \item<2-> If not, acts as \funcname{raise\_notrace} with \funcname{RESTORE\_EXN\_HANDLER\_OCAML}
          \begin{itemize}
            \item Set \regname{RSP} to \localname{domain\_state->exn\_handler} value
            \item Restore \localname{domain\_state->exn\_handler} from trap
            \item Jump with \funcname{ret} (same effect as using \funcname{pop} and \funcname{jmp})
          \end{itemize}
        \item<3-> Otherwise, switches to the C stack to be able to execute C code
        \item<4-> Calls \funcname{caml\_stash\_backtrace}, a C function provided by the runtime\ignorespaces
          \onslide<6->{, with
            \begin{itemize}
              \item \funcarg{exn}: exception value
              \item \funcarg{pc}: code address of the raising point
              \item \funcarg{sp}: stack address of the raising function
              \item \funcarg{trapsp}: stack address of the next handler
            \end{itemize}
          }
      \end{itemize}
      \bigskip
      \mintocaml[firstline=9,lastline=13,highlightlines=12]{../src/backtrace/backtrace.ml}
    \end{column}
    \begin{column}{0.5\textwidth}
      \raggedleft
      \onslide*<1,3-4>{
        \mintc[firstline=190,lastline=192]{../ocaml/runtime/amd64.S}
        \mintc[firstline=234,lastline=237]{../ocaml/runtime/amd64.S}
      }
      \onslide*<2>{
        \mintc[firstline=190,lastline=192,highlightlines={191-192}]{../ocaml/runtime/amd64.S}
        \mintc[firstline=234,lastline=237,highlightlines={235-237}]{../ocaml/runtime/amd64.S}
      }
      \onslide*<1>{\listCamlRaiseExn[highlightlines=705]{}}%
      \onslide*<2>{\listCamlRaiseExn[highlightlines={707-708}]{}}%
      \onslide*<3>{\listCamlRaiseExn[highlightlines={712-714}]{}}%
      \onslide*<4>{\listCamlRaiseExn[highlightlines={715-720}]{}}%
      \onslide*<5>{\providecommand\step{1}\input{diagrams/backtrace/backtrace.tex}}%
      \onslide*<6>{\providecommand\step{2}\input{diagrams/backtrace/backtrace.tex}}%
    \end{column}
  \end{columns}
\end{frame}

\newcommand\listStashBacktrace[1][]{
  \listc[firstline=91,lastline=120,#1]{../ocaml/runtime/backtrace\_nat.c}{runtime/backtrace\_nat.c:91}}

\begin{frame}{Backtraces}{Introducing \funcname{caml\_stash\_backtrace}}
  \begin{columns}[c]
    \begin{column}{0.5\textwidth}
      \minipage[c][0.66\textheight][s]{\columnwidth}
      \begin{itemize}
        \item<1-> A C function provided by the runtime (specific to native code)
        \item<2-> It scans the stack, between the stack pointer at the raising point up to the next trap, in order to collect the names of the detected function frames
          \onslide*<2>{
            \begin{itemize}
              \item And more information, like line number, column number...
            \end{itemize}
          }
        \item<3-> It uses the same technique and information as the Garbage Collector (GC) uses to scan the values live on the stack
          \onslide*<4>{
            \begin{itemize}
              \item \typename{frame\_descr} are at the core of stack unwinding
            \end{itemize}
          }
      \end{itemize}
      \vfill
      \mintocaml[firstline=9,lastline=13,highlightlines=12]{../src/backtrace/backtrace.ml}
      \endminipage
    \end{column}
    \begin{column}{0.5\textwidth}
      \onslide*<1>{\listStashBacktrace[highlightlines=91]{}}
      \onslide*<2>{\listStashBacktrace[highlightlines={91,117-118}]{}}
      \onslide*<3->{\listStashBacktrace[highlightlines={94,106,109-110}]{}}
    \end{column}
  \end{columns}
\end{frame}

\newcommand\listFrameDescr[1][]{
  \listocaml[firstline=111,lastline=124,#1]{../ocaml/asmcomp/emitaux.ml}{asmcomp/emitaux.ml:111}}
\newcommand\listDebugInfo[1][]{
  \listocaml[firstline=38,lastline=49,#1]{../ocaml/lambda/debuginfo.mli}{lambda/debuginfo.mli:38}}

\begin{frame}{Backtraces}{\typename{frame\_descr} and \typename{frame\_debuginfo} in a nutshell}
  \begin{columns}[c]
    \begin{column}{0.5\textwidth}
      \begin{itemize}
        \item<1-> For every return address in an OCaml program, there is a associated \typename{frame\_descr}
        \item<2-> A \typename{frame\_descr} specifies
        \begin{itemize}
          \item<2-> The size of the function stack frame
          \item<3-> Where live values are on the stack (so that the GC can mark them as live and prevent from being swept)
        \end{itemize}
        \item<4-> When compiled with debug information, \typename{frame\_descr} will be linked to a \typename{frame\_debuginfo}
        \begin{itemize}
          \item<5-> Contains information about the position in the source code (file name, line and column number)
        \end{itemize}
      \end{itemize}
    \end{column}
    \begin{column}{0.5\textwidth}
        \onslide*<1>{\listFrameDescr[highlightlines={118-122}]{}}
        \onslide*<2>{\listFrameDescr[highlightlines={120}]{}}
        \onslide*<3>{\listFrameDescr[highlightlines={121}]{}}
        \onslide*<4->{\listFrameDescr[highlightlines={122}]{}}
        \medskip
        \onslide*<1-3>{\listDebugInfo{}}
        \onslide*<4>{\listDebugInfo[highlightlines={38,49}]{}}
        \onslide*<5>{\listDebugInfo[highlightlines={39-46}]{}}
    \end{column}
  \end{columns}
\end{frame}

\begin{frame}{Backtraces}{Scanning the stack with \typename{frame\_descr}}
  \begin{columns}[c]
    \begin{column}{0.5\textwidth}
      \begin{itemize}
        \item Given the return address of the code that raises the exception by calling \funcname{caml\_raise\_exn} (\funcarg{pc} argument of \funcname{caml\_stash\_backtrace})
        \item And the position of the stack pointer (\funcarg{sp} argument of \funcname{caml\_stash\_backtrace})
        \item The frame size given by \typename{frame\_descr}.\funcarg{frame\_size}, allows to find the end of the previous frame
        \item And the return address of the caller of this function
        \bigskip
        \onslide<3->{
        \item Note that traps (here the reddish \funcname{ExnA}, \funcname{ExnB} and \funcname{ExnC} blocks) belong to the frame that installed them, and so the frame size changes when traps are removed
        }
      \end{itemize}
    \end{column}
    \begin{column}{0.5\textwidth}
      \raggedleft
      \onslide*<1>{\providecommand\step{2}\input{diagrams/backtrace/backtrace.tex}}%
      \onslide*<2>{\providecommand\step{1}\input{diagrams/backtrace/scanstack.tex}}%
      \onslide*<3>{\providecommand\step{2}\input{diagrams/backtrace/scanstack.tex}}%
      \onslide*<4>{\providecommand\step{3}\input{diagrams/backtrace/scanstack.tex}}%
      \onslide*<5>{\providecommand\step{4}\input{diagrams/backtrace/scanstack.tex}}%
      \onslide*<6>{\providecommand\step{5}\input{diagrams/backtrace/scanstack.tex}}%
    \end{column}
  \end{columns}
\end{frame}

% Duplicate of previous-previous slide
\begin{frame}{Backtraces}{Continuing on \funcname{caml\_stash\_backtrace}}
  \begin{columns}[c]
    \begin{column}{0.5\textwidth}
      \minipage[c][0.75\textheight][s]{\columnwidth}
      \begin{itemize}
          \color{gray}
        \item A C function provided by the runtime (specific to native code)
        \item It scans the stack, between the stack pointer at the raising point up to the next trap, in order to collect the names of the detected function frames
        \item It uses the same technique and information as the Garbage Collector (GC) uses to scan the values live on the stack
      \end{itemize}
      \begin{itemize}
        \item For each function frame detected, store a pointer to the \typename{frame\_descr} in the \localname{domain\_state->backtrace\_buffer} array, effectively constructing the backtrace
      \end{itemize}
      \onslide<2->{
        \begin{itemize}
            \smallskip
          \item Constructed backtrace:
            \funcname{definitely\_raise},
            \onslide<3->{\funcname{probably\_raise}}
        \end{itemize}
      }
      \vfill
      \mintocaml[firstline=9,lastline=13,highlightlines=12]{../src/backtrace/backtrace.ml}
      \endminipage
    \end{column}
    \begin{column}{0.5\textwidth}
      \raggedleft
      \onslide*<1>{\listStashBacktrace[highlightlines={114-115}]{}}
      \onslide*<2>{\providecommand\step{3}\input{diagrams/backtrace/backtrace.tex}}%
      \onslide*<3>{\providecommand\step{4}\input{diagrams/backtrace/backtrace.tex}}%
    \end{column}
  \end{columns}
\end{frame}

\begin{frame}{Backtraces}{Back to \funcname{caml\_raise\_exn}}
  \begin{columns}[c]
    \begin{column}{0.5\textwidth}
      \minipage[c][0.66\textheight][s]{\columnwidth}
      \begin{itemize}\color{gray}
        \item Checks wheteher exceptions backtrace should be recorded, by checking the \funcarg{domain\_state->backtrace\_active} value
        \item Switches to the C stack to be able to execute C code
        \item Calls \funcname{caml\_stash\_backtrace}, to collect the backtrace up to the next trap
      \end{itemize}
      \onslide*<2->{
        \begin{itemize}
          \item<2-> Executes the exception handler in \funcname{probably\_raise} with \funcname{RESTORE\_EXN\_HANDLER\_OCAML}
            \begin{itemize}
              \item Set \regname{RSP} to \localname{domain\_state->exn\_handler} value
              \item Restore \localname{domain\_state->exn\_handler} from trap
              \item Jump to the handler code with \funcname{ret}
            \end{itemize}
        \end{itemize}
      }
      \vfill
      \onslide*<1>{\mintocaml[firstline=9,lastline=13,highlightlines=12]{../src/backtrace/backtrace.ml}}
      \onslide*<2->{\mintocaml[firstline=15,lastline=21,highlightlines={19-21}]{../src/backtrace/backtrace.ml}}
      \endminipage
    \end{column}
    \begin{column}{0.5\textwidth}
      \centering
      \onslide*<1>{
        \mintc[firstline=190,lastline=192]{../ocaml/runtime/amd64.S}
        \mintc[firstline=234,lastline=237]{../ocaml/runtime/amd64.S}
      }
      \onslide*<2>{
        \mintc[firstline=190,lastline=192,highlightlines={191-192}]{../ocaml/runtime/amd64.S}
        \mintc[firstline=234,lastline=237,highlightlines={235-237}]{../ocaml/runtime/amd64.S}
      }
      \onslide*<1>{\listCamlRaiseExn[highlightlines=720]{}}
      \onslide*<2>{\listCamlRaiseExn[highlightlines=722]{}}
      \onslide*<3->{\providecommand\step{5}\input{diagrams/backtrace/backtrace.tex}}
    \end{column}
  \end{columns}
\end{frame}

\begin{frame}{Backtraces}{\funcname{caml\_probably\_raise}'s handler doesn't catch \typename{ExnC}}
  \begin{columns}[c]
    \begin{column}{0.45\textwidth}
      \minipage[c][0.75\textheight][s]{\columnwidth}
      \begin{itemize}
        \item<1-> \funcname{match \dots with}\footnotemark pattern matching can also match exceptions
        \item<1-> The CMM of \funcname{probably\_raise} shows that this \funcname{match \dots with} is implemented with a pattern matching and a \funcname{try \dots with}
        \item<1-> But this one only matches on \typename{ExnA} exception
        \item<2-> Since the pattern matching fails to catch the exception, \funcname{reraise} is called with our current \typename{ExnC} exception
        \item<3-> Disassembly shows that \funcname{reraise} is actually a call to \funcname{caml\_reraise\_exn}, which is also an assembly function provided by the runtime
      \end{itemize}
      \vfill
      \mintocaml[firstline=15,lastline=21,highlightlines={18-21}]{../src/backtrace/backtrace.ml}
      \endminipage
    \end{column}
    \begin{column}{0.55\textwidth}
      \centering
      \onslide*<1>{\mintcmm[highlightlines={10-18}]{../src/backtrace/camlBacktrace\_\_probably\_raise\_351.cmm}}
      \onslide*<2>{\listcmm[highlightlines=18]{../src/backtrace/camlBacktrace\_\_probably\_raise_351.cmm}{CMM of probably\_raise}}
      \onslide*<3>{\mintobjdump[firstline=5,lastline=35,highlightlines=31]{../src/backtrace/camlBacktrace\_\_probably\_raise_351.objdump}}
    \end{column}
  \end{columns}
  \only<1>{\footnotetext[1]{https://blog.janestreet.com/pattern-matching-and-exception-handling-unite/}}
\end{frame}

\newcommand\listCamlReraiseExn[1][]{
  \listasm[firstline=727,lastline=734,#1]{../ocaml/runtime/amd64.S}{runtime/amd64.S:727}}

\begin{frame}{Backtraces}{Introducing \funcname{caml\_reraise\_exn}}
  \begin{columns}[c]
    \begin{column}{0.5\textwidth}
      \minipage[c][0.66\textheight][s]{\columnwidth}
      \begin{itemize}
        \item<1-> The exception have to be reraised to the next handler
        \item<2-> First, checks if the backtrace should be collected with \funcarg{domain\_state->backtrace\_active}
        \only<3>{
        \begin{itemize}
            \item If not, behaves like a \funcname{raise\_notrace} with \funcname{RESTORE\_EXN\_HANDLER\_OCAML}
        \end{itemize}
        }
        \item<4-> Jumps into \funcname{caml\_raise\_exn}
        \item<5-> Calls \funcname{caml\_stash\_backtrace} again, to scan the next part of the stack: from the trap just pop-ed to the next trap
      \end{itemize}
      \vfill
      \mintocaml[firstline=15,lastline=21,highlightlines={18-21}]{../src/backtrace/backtrace.ml}
      \endminipage
    \end{column}
    \begin{column}{0.5\textwidth}
      \raggedleft
      \onslide*<1>{\listCamlReraiseExn{}}
      \onslide*<2>{\listCamlReraiseExn[highlightlines=729]{}}
      \onslide*<3>{
        \mintc[firstline=190,lastline=192,highlightlines={191-192}]{../ocaml/runtime/amd64.S}
        \mintc[firstline=234,lastline=237,highlightlines={235-237}]{../ocaml/runtime/amd64.S}
        \medskip
        \listCamlReraiseExn[highlightlines=731]{}
      }
      \onslide*<4-5>{
        % TODO refactor with \listCamlReraiseExn
        \mintasm[firstline=727,lastline=734,highlightlines=730]{../ocaml/runtime/amd64.S}
        \medskip
      }
      \onslide*<4>{\listCamlRaiseExn[highlightlines=711]{}}
      \onslide*<5>{\listCamlRaiseExn[highlightlines=720]{}}
    \end{column}
  \end{columns}
\end{frame}

\begin{frame}{Backtraces}{Backtrace collection continues}
  \begin{columns}[c]
    \begin{column}{0.55\textwidth}
      \minipage[c][0.75\textheight][s]{\columnwidth}
      \begin{itemize}
        \item<1-> \funcname{caml\_stash\_backtrace} scans the older part of the stack, up to the next trap
        \item<6-> Controls jumps to the nested exception handler in \funcname{maybe\_raise}, which matches only on \typename{ExnB}
        \item<7-> \funcname{caml\_reraise\_exn} is called again, backtrace collection resumes with \funcname{caml\_stash\_backtrace}
      \end{itemize}
      \medskip
      \begin{itemize}
        \item<2-> Constructed backtrace:
        \begin{itemize}
          \item <2-> \funcname{definitely\_raise}
          \item <2-> \funcname{probably\_raise}
          \item <3-> \funcname{probably\_raise} (re-raise)
          \item <4-> \funcname{maybe\_raise}
          \item <5-> \funcname{main}
          \item <8-> \funcname{main} (re-raise)
        \end{itemize}
      \end{itemize}
      \vfill
      \onslide*<1-5>{\mintocaml[firstline=15,lastline=21]{../src/backtrace/backtrace.ml}}
      \onslide*<6>{\mintocaml[firstline=28,lastline=34,highlightlines=31]{../src/backtrace/backtrace.ml}}
      \onslide*<7->{\mintocaml[firstline=28,lastline=34,highlightlines=33]{../src/backtrace/backtrace.ml}}
      \endminipage
    \end{column}
    \begin{column}{0.45\textwidth}
      \raggedleft
      \onslide*<1>{\providecommand\step{6}\input{diagrams/backtrace/backtrace.tex}}%
      \onslide*<2>{\providecommand\step{6}\input{diagrams/backtrace/backtrace.tex}}%
      \onslide*<3>{\providecommand\step{7}\input{diagrams/backtrace/backtrace.tex}}%
      \onslide*<4>{\providecommand\step{8}\input{diagrams/backtrace/backtrace.tex}}%
      \onslide*<5>{\providecommand\step{9}\input{diagrams/backtrace/backtrace.tex}}%
      \onslide*<6>{\providecommand\step{10}\input{diagrams/backtrace/backtrace.tex}}%
      \onslide*<7>{\providecommand\step{11}\input{diagrams/backtrace/backtrace.tex}}%
      \onslide*<8>{\providecommand\step{12}\input{diagrams/backtrace/backtrace.tex}}%
      \onslide*<9>{\providecommand\step{13}\input{diagrams/backtrace/backtrace.tex}}%
    \end{column}
  \end{columns}
\end{frame}

\begin{frame}{Backtraces}{Printing the backtrace}
  \begin{columns}[c]
    \begin{column}{0.45\textwidth}
      \minipage[c][0.66\textheight][s]{\columnwidth}
      \begin{itemize}
        \item<1-> The exception backtrace can now be printed to \funcarg{stdout} with \funcname{Printexc.print_backtrace}
        \item<2-> First the exception backtrace is retrieved into an OCaml array with \funcname{get_raw_backtrace }
        \item<3-> Which is implemented as the \funcname{caml_get_exception_raw_backtrace} C function in the runtime
        \item<4-> And finally printed with \funcname{print_exception_backtrace}
      \end{itemize}
      \vfill
      \mintocaml[firstline=28,lastline=34,highlightlines=33]{../src/backtrace/backtrace.ml}
      \endminipage
    \end{column}
    \begin{column}{0.55\textwidth}
      \onslide*<1,2>{
        \mintocaml[firstline=100,lastline=101]{../ocaml/stdlib/printexc.ml}
      }
      \only<1>{
        \listocaml[firstline=170,lastline=172]{../ocaml/stdlib/printexc.ml}{stdlib/printexc.ml:170}
      }
      \only<2>{
        \listocaml[firstline=170,lastline=172,highlightlines=172]{../ocaml/stdlib/printexc.ml}{stdlib/printexc.ml:170}
      }
      \only<3>{
        \mintc[firstline=154,lastline=156]{../ocaml/runtime/backtrace.c}
        \listc[firstline=171,lastline=192,highlightlines={184-187}]{../ocaml/runtime/backtrace.c}{runtime/backtrace.c:154}
      }
      \only<4>{
        \listocaml[firstline=155,lastline=165]{../ocaml/stdlib/printexc.ml}{stdlib/printexc.ml:55}
      }
    \end{column}
  \end{columns}
\end{frame}


\subsection{The multiple kinds of raise}
\frameSubsection{}{}

\begin{frame}{Backtraces}{When backtraces are not needed}
  \begin{columns}[c]
    \begin{column}{0.45\textwidth}
      \minipage[c][0.66\textheight][s]{\columnwidth}
      \begin{itemize}
        \item<1-> What if the backtrace for an exception is never needed?
        \item<2-> It's possible to raise the exception explicitly with \funcname{raise\_notrace} to make raising as cheap as possible
        \item<3-> An example of use in the \funcname{dynlink} library of the OCaml compiler
      \end{itemize}
      \vfill
      \onslide<3->{
      \listocaml[firstline=205,lastline=209,highlightlines=207]{../ocaml/otherlibs/dynlink/dynlink\_compilerlibs/misc.ml}{otherlibs/dynlink/dynlink\_compilerlibs/misc.ml:205}
      }
      \endminipage
    \end{column}
    \begin{column}{0.55\textwidth}
    \only<4>{
      \mintcmm[highlightlines=3]{../src/notrace/camlNotrace\_\_fun_347.cmm}
      \listcmm{../src/notrace/camlNotrace\_\_all\_somes\_327.cmm}{all\_somes}
    }
    \onslide*<5->{
      \listobjdump[highlightlines={6-9}]{../src/notrace/camlNotrace\_\_fun_347.objdump}{anonymous function from \funcname{all\_somes}}
    }
    \end{column}
  \end{columns}
\end{frame}

\begin{frame}{Backtraces}{Summary table of the multiple kinds of raise}
  \begin{itemize}
    \item When debugging information is enabled and if the backtrace of an exception isn't needed, it's possible to raise with \funcname{raise\_notrace} for performance
    \item When debugging information is \emph{not} enabled, the compiler optimises \funcname{raise} calls to inline assembly (same effect as using \funcname{raise\_notrace})
  \end{itemize}
  \bigskip
  \centering
  \begin{tabular}{| l | c | c | c | c |}
    \hline
    \parbox{10em}{Debug information \\ (\funcarg{-g} flag)}  & \multicolumn{2}{|c|}{Disabled}                                     & \multicolumn{2}{|c|}{Enabled} \\
    \hline
    Raise kind                                               & \funcname{raise} & \funcname{raise\_notrace}                       & \funcname{raise} & \funcname{raise\_notrace} \\
    \hline
    Compilation                                              & \parbox{8em}{inline assembly\\ (optimization)} & inline assembly   & \funcname{caml\_raise\_exn} & inline assembly \\
    \hline
  \end{tabular}
\end{frame}


%
% Takeaway #5
%
\subsection*{Takeaway \#5}
\frameSubsectionTakeaway{}
\againframe<5>{takeaway}

    \end{column}
  \end{columns}
\end{frame}

\begin{frame}{Backtraces}{But before that, we need more fields from \typename{caml\_domain\_state}}
  \begin{columns}
    \begin{column}{0.5\textwidth}
      \begin{itemize}
        \item Remember \typename{caml\_domain\_state} the per-domain runtime state?
        \item Some other members are of interest to us now\dots
        \item \localname{backtrace\_active} is used to know if the runtime should collect backtraces
        \item \localname{backtrace\_active} can be set by
          \begin{itemize}
            \item The \funcarg{b} flag in the \funcname{OCAMLRUNPARAM} environment variable
            \item \mintinline{ocaml}{Printexc.record_backtrace : bool -> unit} \footnotemark
          \end{itemize}
      \end{itemize}
    \end{column}
    \begin{column}{0.5\textwidth}
      \begin{listing}
        \mintc[highlightlines={9-11}]{src/ocaml/domain\_state\_backtrace.h}
        \caption{runtime/caml/domain\_state.h}
      \end{listing}
    \end{column}
  \end{columns}
  \footnotetext[1]{https://v2.ocaml.org/api/Printexc.html}
\end{frame}

\newcommand\listCamlRaiseExn[1][]{
  \listasm[firstline=702,lastline=725,#1]{../ocaml/runtime/amd64.S}{runtime/amd64.S:702}}

\begin{frame}{Backtraces}{Walk through \funcname{caml\_raise\_exn}}
  \begin{columns}[T]
    \begin{column}{0.5\textwidth}
      \begin{itemize}
        \item<1-> First checks whether exceptions backtrace should be recorded
        \item<1-> By checking the \funcarg{domain\_state->backtrace\_active} value
        \item<2-> If not, acts as \funcname{raise\_notrace} with \funcname{RESTORE\_EXN\_HANDLER\_OCAML}
          \begin{itemize}
            \item Set \regname{RSP} to \localname{domain\_state->exn\_handler} value
            \item Restore \localname{domain\_state->exn\_handler} from trap
            \item Jump with \funcname{ret} (same effect as using \funcname{pop} and \funcname{jmp})
          \end{itemize}
        \item<3-> Otherwise, switches to the C stack to be able to execute C code
        \item<4-> Calls \funcname{caml\_stash\_backtrace}, a C function provided by the runtime\ignorespaces
          \onslide<6->{, with
            \begin{itemize}
              \item \funcarg{exn}: exception value
              \item \funcarg{pc}: code address of the raising point
              \item \funcarg{sp}: stack address of the raising function
              \item \funcarg{trapsp}: stack address of the next handler
            \end{itemize}
          }
      \end{itemize}
      \bigskip
      \mintocaml[firstline=9,lastline=13,highlightlines=12]{../src/backtrace/backtrace.ml}
    \end{column}
    \begin{column}{0.5\textwidth}
      \raggedleft
      \onslide*<1,3-4>{
        \mintc[firstline=190,lastline=192]{../ocaml/runtime/amd64.S}
        \mintc[firstline=234,lastline=237]{../ocaml/runtime/amd64.S}
      }
      \onslide*<2>{
        \mintc[firstline=190,lastline=192,highlightlines={191-192}]{../ocaml/runtime/amd64.S}
        \mintc[firstline=234,lastline=237,highlightlines={235-237}]{../ocaml/runtime/amd64.S}
      }
      \onslide*<1>{\listCamlRaiseExn[highlightlines=705]{}}%
      \onslide*<2>{\listCamlRaiseExn[highlightlines={707-708}]{}}%
      \onslide*<3>{\listCamlRaiseExn[highlightlines={712-714}]{}}%
      \onslide*<4>{\listCamlRaiseExn[highlightlines={715-720}]{}}%
      \onslide*<5>{\providecommand\step{1}%
% Backtraces
%
\subsection{One advantage of raising with debugging information}
\frameSubsection{}{}

\begin{frame}{Backtraces}{Debugging information \& source code}
  \begin{columns}[c]
    \begin{column}{0.5\textwidth}
      \begin{itemize}
        \item Raising an exception is fun and games, but how do we know where the exception originated? $\rightarrow$ With the backtrace associated with the exception!
        \item In order to get a backtrace from the point where an exception was raised, it's necessary to compile with debugging information enabled (\funcarg{-g} flag for the compiler)
        \item Same example as in the `Nested handlers' section
      \end{itemize}
    \end{column}
    \begin{column}{0.5\textwidth}
      \listocaml[firstline=9,lastline=34]{../src/backtrace/backtrace.ml}{backtrace/backtrace.ml}
    \end{column}
  \end{columns}
\end{frame}

\begin{frame}{Backtraces}{CMM and disassembly of \funcname{definitely\_raise}}
  \begin{columns}[c]
    \begin{column}{0.5\textwidth}
      \minipage[c][0.66\textheight][s]{\columnwidth}
      \begin{itemize}
        \item<1-> An effect of compiling with debugging information (\funcarg{-g}): the CMM no longer uses \funcname{raise\_notrace} to raise the exception but \funcname{raise} instead
        \item<2-> Disassembly shows that CMM \funcname{raise} is actually a call to \funcname{caml\_raise\_exn}, a function in assembler provided by the runtime
      \end{itemize}
      \vfill
      \mintocaml[firstline=9,lastline=13,highlightlines=12]{../src/backtrace/backtrace.ml}
      \endminipage
    \end{column}
    \begin{column}{0.5\textwidth}
      \onslide*<1>{
        \mintcmm[highlightlines=5]{../src/backtrace/camlBacktrace\_\_definitely\_raise\_347.cmm}
      }
      \onslide*<2>{
        \mintobjdump[firstline=2,highlightlines=14]{../src/backtrace/camlBacktrace\_\_definitely\_raise\_347.objdump}
      }
    \end{column}
  \end{columns}
\end{frame}

\begin{frame}{Backtraces}{Introducing \funcname{caml\_raise\_exn}}
  \begin{columns}[c]
    \begin{column}{0.5\textwidth}
      \minipage[c][0.66\textheight][s]{\columnwidth}
      \onslide*<1>{
        \begin{itemize}
          \item Raising the exception is performed using \funcname{caml\_raise\_exn} which is
            \begin{itemize}
              \item An assembly function provided by the runtime
              \item Architecture specific
            \end{itemize}
          \item Let's see what it does differently from \funcname{raise\_notrace}\dots
          \item We are about to raise \typename{ExnC} with \funcname{caml\_raise\_exn} from \funcname{definitely\_raise}
        \end{itemize}
      }
      \onslide*<2>{
        \mintobjdump[firstline=2,highlightlines=14]{../src/backtrace/camlBacktrace\_\_definitely\_raise\_347.objdump}
      }
      \vfill
      \mintocaml[firstline=9,lastline=13,highlightlines=12]{../src/backtrace/backtrace.ml}
      \endminipage
    \end{column}
    \begin{column}{0.5\textwidth}
      \centering
      \providecommand\step{1}\input{diagrams/backtrace/backtrace.tex}
    \end{column}
  \end{columns}
\end{frame}

\begin{frame}{Backtraces}{But before that, we need more fields from \typename{caml\_domain\_state}}
  \begin{columns}
    \begin{column}{0.5\textwidth}
      \begin{itemize}
        \item Remember \typename{caml\_domain\_state} the per-domain runtime state?
        \item Some other members are of interest to us now\dots
        \item \localname{backtrace\_active} is used to know if the runtime should collect backtraces
        \item \localname{backtrace\_active} can be set by
          \begin{itemize}
            \item The \funcarg{b} flag in the \funcname{OCAMLRUNPARAM} environment variable
            \item \mintinline{ocaml}{Printexc.record_backtrace : bool -> unit} \footnotemark
          \end{itemize}
      \end{itemize}
    \end{column}
    \begin{column}{0.5\textwidth}
      \begin{listing}
        \mintc[highlightlines={9-11}]{src/ocaml/domain\_state\_backtrace.h}
        \caption{runtime/caml/domain\_state.h}
      \end{listing}
    \end{column}
  \end{columns}
  \footnotetext[1]{https://v2.ocaml.org/api/Printexc.html}
\end{frame}

\newcommand\listCamlRaiseExn[1][]{
  \listasm[firstline=702,lastline=725,#1]{../ocaml/runtime/amd64.S}{runtime/amd64.S:702}}

\begin{frame}{Backtraces}{Walk through \funcname{caml\_raise\_exn}}
  \begin{columns}[T]
    \begin{column}{0.5\textwidth}
      \begin{itemize}
        \item<1-> First checks whether exceptions backtrace should be recorded
        \item<1-> By checking the \funcarg{domain\_state->backtrace\_active} value
        \item<2-> If not, acts as \funcname{raise\_notrace} with \funcname{RESTORE\_EXN\_HANDLER\_OCAML}
          \begin{itemize}
            \item Set \regname{RSP} to \localname{domain\_state->exn\_handler} value
            \item Restore \localname{domain\_state->exn\_handler} from trap
            \item Jump with \funcname{ret} (same effect as using \funcname{pop} and \funcname{jmp})
          \end{itemize}
        \item<3-> Otherwise, switches to the C stack to be able to execute C code
        \item<4-> Calls \funcname{caml\_stash\_backtrace}, a C function provided by the runtime\ignorespaces
          \onslide<6->{, with
            \begin{itemize}
              \item \funcarg{exn}: exception value
              \item \funcarg{pc}: code address of the raising point
              \item \funcarg{sp}: stack address of the raising function
              \item \funcarg{trapsp}: stack address of the next handler
            \end{itemize}
          }
      \end{itemize}
      \bigskip
      \mintocaml[firstline=9,lastline=13,highlightlines=12]{../src/backtrace/backtrace.ml}
    \end{column}
    \begin{column}{0.5\textwidth}
      \raggedleft
      \onslide*<1,3-4>{
        \mintc[firstline=190,lastline=192]{../ocaml/runtime/amd64.S}
        \mintc[firstline=234,lastline=237]{../ocaml/runtime/amd64.S}
      }
      \onslide*<2>{
        \mintc[firstline=190,lastline=192,highlightlines={191-192}]{../ocaml/runtime/amd64.S}
        \mintc[firstline=234,lastline=237,highlightlines={235-237}]{../ocaml/runtime/amd64.S}
      }
      \onslide*<1>{\listCamlRaiseExn[highlightlines=705]{}}%
      \onslide*<2>{\listCamlRaiseExn[highlightlines={707-708}]{}}%
      \onslide*<3>{\listCamlRaiseExn[highlightlines={712-714}]{}}%
      \onslide*<4>{\listCamlRaiseExn[highlightlines={715-720}]{}}%
      \onslide*<5>{\providecommand\step{1}\input{diagrams/backtrace/backtrace.tex}}%
      \onslide*<6>{\providecommand\step{2}\input{diagrams/backtrace/backtrace.tex}}%
    \end{column}
  \end{columns}
\end{frame}

\newcommand\listStashBacktrace[1][]{
  \listc[firstline=91,lastline=120,#1]{../ocaml/runtime/backtrace\_nat.c}{runtime/backtrace\_nat.c:91}}

\begin{frame}{Backtraces}{Introducing \funcname{caml\_stash\_backtrace}}
  \begin{columns}[c]
    \begin{column}{0.5\textwidth}
      \minipage[c][0.66\textheight][s]{\columnwidth}
      \begin{itemize}
        \item<1-> A C function provided by the runtime (specific to native code)
        \item<2-> It scans the stack, between the stack pointer at the raising point up to the next trap, in order to collect the names of the detected function frames
          \onslide*<2>{
            \begin{itemize}
              \item And more information, like line number, column number...
            \end{itemize}
          }
        \item<3-> It uses the same technique and information as the Garbage Collector (GC) uses to scan the values live on the stack
          \onslide*<4>{
            \begin{itemize}
              \item \typename{frame\_descr} are at the core of stack unwinding
            \end{itemize}
          }
      \end{itemize}
      \vfill
      \mintocaml[firstline=9,lastline=13,highlightlines=12]{../src/backtrace/backtrace.ml}
      \endminipage
    \end{column}
    \begin{column}{0.5\textwidth}
      \onslide*<1>{\listStashBacktrace[highlightlines=91]{}}
      \onslide*<2>{\listStashBacktrace[highlightlines={91,117-118}]{}}
      \onslide*<3->{\listStashBacktrace[highlightlines={94,106,109-110}]{}}
    \end{column}
  \end{columns}
\end{frame}

\newcommand\listFrameDescr[1][]{
  \listocaml[firstline=111,lastline=124,#1]{../ocaml/asmcomp/emitaux.ml}{asmcomp/emitaux.ml:111}}
\newcommand\listDebugInfo[1][]{
  \listocaml[firstline=38,lastline=49,#1]{../ocaml/lambda/debuginfo.mli}{lambda/debuginfo.mli:38}}

\begin{frame}{Backtraces}{\typename{frame\_descr} and \typename{frame\_debuginfo} in a nutshell}
  \begin{columns}[c]
    \begin{column}{0.5\textwidth}
      \begin{itemize}
        \item<1-> For every return address in an OCaml program, there is a associated \typename{frame\_descr}
        \item<2-> A \typename{frame\_descr} specifies
        \begin{itemize}
          \item<2-> The size of the function stack frame
          \item<3-> Where live values are on the stack (so that the GC can mark them as live and prevent from being swept)
        \end{itemize}
        \item<4-> When compiled with debug information, \typename{frame\_descr} will be linked to a \typename{frame\_debuginfo}
        \begin{itemize}
          \item<5-> Contains information about the position in the source code (file name, line and column number)
        \end{itemize}
      \end{itemize}
    \end{column}
    \begin{column}{0.5\textwidth}
        \onslide*<1>{\listFrameDescr[highlightlines={118-122}]{}}
        \onslide*<2>{\listFrameDescr[highlightlines={120}]{}}
        \onslide*<3>{\listFrameDescr[highlightlines={121}]{}}
        \onslide*<4->{\listFrameDescr[highlightlines={122}]{}}
        \medskip
        \onslide*<1-3>{\listDebugInfo{}}
        \onslide*<4>{\listDebugInfo[highlightlines={38,49}]{}}
        \onslide*<5>{\listDebugInfo[highlightlines={39-46}]{}}
    \end{column}
  \end{columns}
\end{frame}

\begin{frame}{Backtraces}{Scanning the stack with \typename{frame\_descr}}
  \begin{columns}[c]
    \begin{column}{0.5\textwidth}
      \begin{itemize}
        \item Given the return address of the code that raises the exception by calling \funcname{caml\_raise\_exn} (\funcarg{pc} argument of \funcname{caml\_stash\_backtrace})
        \item And the position of the stack pointer (\funcarg{sp} argument of \funcname{caml\_stash\_backtrace})
        \item The frame size given by \typename{frame\_descr}.\funcarg{frame\_size}, allows to find the end of the previous frame
        \item And the return address of the caller of this function
        \bigskip
        \onslide<3->{
        \item Note that traps (here the reddish \funcname{ExnA}, \funcname{ExnB} and \funcname{ExnC} blocks) belong to the frame that installed them, and so the frame size changes when traps are removed
        }
      \end{itemize}
    \end{column}
    \begin{column}{0.5\textwidth}
      \raggedleft
      \onslide*<1>{\providecommand\step{2}\input{diagrams/backtrace/backtrace.tex}}%
      \onslide*<2>{\providecommand\step{1}\input{diagrams/backtrace/scanstack.tex}}%
      \onslide*<3>{\providecommand\step{2}\input{diagrams/backtrace/scanstack.tex}}%
      \onslide*<4>{\providecommand\step{3}\input{diagrams/backtrace/scanstack.tex}}%
      \onslide*<5>{\providecommand\step{4}\input{diagrams/backtrace/scanstack.tex}}%
      \onslide*<6>{\providecommand\step{5}\input{diagrams/backtrace/scanstack.tex}}%
    \end{column}
  \end{columns}
\end{frame}

% Duplicate of previous-previous slide
\begin{frame}{Backtraces}{Continuing on \funcname{caml\_stash\_backtrace}}
  \begin{columns}[c]
    \begin{column}{0.5\textwidth}
      \minipage[c][0.75\textheight][s]{\columnwidth}
      \begin{itemize}
          \color{gray}
        \item A C function provided by the runtime (specific to native code)
        \item It scans the stack, between the stack pointer at the raising point up to the next trap, in order to collect the names of the detected function frames
        \item It uses the same technique and information as the Garbage Collector (GC) uses to scan the values live on the stack
      \end{itemize}
      \begin{itemize}
        \item For each function frame detected, store a pointer to the \typename{frame\_descr} in the \localname{domain\_state->backtrace\_buffer} array, effectively constructing the backtrace
      \end{itemize}
      \onslide<2->{
        \begin{itemize}
            \smallskip
          \item Constructed backtrace:
            \funcname{definitely\_raise},
            \onslide<3->{\funcname{probably\_raise}}
        \end{itemize}
      }
      \vfill
      \mintocaml[firstline=9,lastline=13,highlightlines=12]{../src/backtrace/backtrace.ml}
      \endminipage
    \end{column}
    \begin{column}{0.5\textwidth}
      \raggedleft
      \onslide*<1>{\listStashBacktrace[highlightlines={114-115}]{}}
      \onslide*<2>{\providecommand\step{3}\input{diagrams/backtrace/backtrace.tex}}%
      \onslide*<3>{\providecommand\step{4}\input{diagrams/backtrace/backtrace.tex}}%
    \end{column}
  \end{columns}
\end{frame}

\begin{frame}{Backtraces}{Back to \funcname{caml\_raise\_exn}}
  \begin{columns}[c]
    \begin{column}{0.5\textwidth}
      \minipage[c][0.66\textheight][s]{\columnwidth}
      \begin{itemize}\color{gray}
        \item Checks wheteher exceptions backtrace should be recorded, by checking the \funcarg{domain\_state->backtrace\_active} value
        \item Switches to the C stack to be able to execute C code
        \item Calls \funcname{caml\_stash\_backtrace}, to collect the backtrace up to the next trap
      \end{itemize}
      \onslide*<2->{
        \begin{itemize}
          \item<2-> Executes the exception handler in \funcname{probably\_raise} with \funcname{RESTORE\_EXN\_HANDLER\_OCAML}
            \begin{itemize}
              \item Set \regname{RSP} to \localname{domain\_state->exn\_handler} value
              \item Restore \localname{domain\_state->exn\_handler} from trap
              \item Jump to the handler code with \funcname{ret}
            \end{itemize}
        \end{itemize}
      }
      \vfill
      \onslide*<1>{\mintocaml[firstline=9,lastline=13,highlightlines=12]{../src/backtrace/backtrace.ml}}
      \onslide*<2->{\mintocaml[firstline=15,lastline=21,highlightlines={19-21}]{../src/backtrace/backtrace.ml}}
      \endminipage
    \end{column}
    \begin{column}{0.5\textwidth}
      \centering
      \onslide*<1>{
        \mintc[firstline=190,lastline=192]{../ocaml/runtime/amd64.S}
        \mintc[firstline=234,lastline=237]{../ocaml/runtime/amd64.S}
      }
      \onslide*<2>{
        \mintc[firstline=190,lastline=192,highlightlines={191-192}]{../ocaml/runtime/amd64.S}
        \mintc[firstline=234,lastline=237,highlightlines={235-237}]{../ocaml/runtime/amd64.S}
      }
      \onslide*<1>{\listCamlRaiseExn[highlightlines=720]{}}
      \onslide*<2>{\listCamlRaiseExn[highlightlines=722]{}}
      \onslide*<3->{\providecommand\step{5}\input{diagrams/backtrace/backtrace.tex}}
    \end{column}
  \end{columns}
\end{frame}

\begin{frame}{Backtraces}{\funcname{caml\_probably\_raise}'s handler doesn't catch \typename{ExnC}}
  \begin{columns}[c]
    \begin{column}{0.45\textwidth}
      \minipage[c][0.75\textheight][s]{\columnwidth}
      \begin{itemize}
        \item<1-> \funcname{match \dots with}\footnotemark pattern matching can also match exceptions
        \item<1-> The CMM of \funcname{probably\_raise} shows that this \funcname{match \dots with} is implemented with a pattern matching and a \funcname{try \dots with}
        \item<1-> But this one only matches on \typename{ExnA} exception
        \item<2-> Since the pattern matching fails to catch the exception, \funcname{reraise} is called with our current \typename{ExnC} exception
        \item<3-> Disassembly shows that \funcname{reraise} is actually a call to \funcname{caml\_reraise\_exn}, which is also an assembly function provided by the runtime
      \end{itemize}
      \vfill
      \mintocaml[firstline=15,lastline=21,highlightlines={18-21}]{../src/backtrace/backtrace.ml}
      \endminipage
    \end{column}
    \begin{column}{0.55\textwidth}
      \centering
      \onslide*<1>{\mintcmm[highlightlines={10-18}]{../src/backtrace/camlBacktrace\_\_probably\_raise\_351.cmm}}
      \onslide*<2>{\listcmm[highlightlines=18]{../src/backtrace/camlBacktrace\_\_probably\_raise_351.cmm}{CMM of probably\_raise}}
      \onslide*<3>{\mintobjdump[firstline=5,lastline=35,highlightlines=31]{../src/backtrace/camlBacktrace\_\_probably\_raise_351.objdump}}
    \end{column}
  \end{columns}
  \only<1>{\footnotetext[1]{https://blog.janestreet.com/pattern-matching-and-exception-handling-unite/}}
\end{frame}

\newcommand\listCamlReraiseExn[1][]{
  \listasm[firstline=727,lastline=734,#1]{../ocaml/runtime/amd64.S}{runtime/amd64.S:727}}

\begin{frame}{Backtraces}{Introducing \funcname{caml\_reraise\_exn}}
  \begin{columns}[c]
    \begin{column}{0.5\textwidth}
      \minipage[c][0.66\textheight][s]{\columnwidth}
      \begin{itemize}
        \item<1-> The exception have to be reraised to the next handler
        \item<2-> First, checks if the backtrace should be collected with \funcarg{domain\_state->backtrace\_active}
        \only<3>{
        \begin{itemize}
            \item If not, behaves like a \funcname{raise\_notrace} with \funcname{RESTORE\_EXN\_HANDLER\_OCAML}
        \end{itemize}
        }
        \item<4-> Jumps into \funcname{caml\_raise\_exn}
        \item<5-> Calls \funcname{caml\_stash\_backtrace} again, to scan the next part of the stack: from the trap just pop-ed to the next trap
      \end{itemize}
      \vfill
      \mintocaml[firstline=15,lastline=21,highlightlines={18-21}]{../src/backtrace/backtrace.ml}
      \endminipage
    \end{column}
    \begin{column}{0.5\textwidth}
      \raggedleft
      \onslide*<1>{\listCamlReraiseExn{}}
      \onslide*<2>{\listCamlReraiseExn[highlightlines=729]{}}
      \onslide*<3>{
        \mintc[firstline=190,lastline=192,highlightlines={191-192}]{../ocaml/runtime/amd64.S}
        \mintc[firstline=234,lastline=237,highlightlines={235-237}]{../ocaml/runtime/amd64.S}
        \medskip
        \listCamlReraiseExn[highlightlines=731]{}
      }
      \onslide*<4-5>{
        % TODO refactor with \listCamlReraiseExn
        \mintasm[firstline=727,lastline=734,highlightlines=730]{../ocaml/runtime/amd64.S}
        \medskip
      }
      \onslide*<4>{\listCamlRaiseExn[highlightlines=711]{}}
      \onslide*<5>{\listCamlRaiseExn[highlightlines=720]{}}
    \end{column}
  \end{columns}
\end{frame}

\begin{frame}{Backtraces}{Backtrace collection continues}
  \begin{columns}[c]
    \begin{column}{0.55\textwidth}
      \minipage[c][0.75\textheight][s]{\columnwidth}
      \begin{itemize}
        \item<1-> \funcname{caml\_stash\_backtrace} scans the older part of the stack, up to the next trap
        \item<6-> Controls jumps to the nested exception handler in \funcname{maybe\_raise}, which matches only on \typename{ExnB}
        \item<7-> \funcname{caml\_reraise\_exn} is called again, backtrace collection resumes with \funcname{caml\_stash\_backtrace}
      \end{itemize}
      \medskip
      \begin{itemize}
        \item<2-> Constructed backtrace:
        \begin{itemize}
          \item <2-> \funcname{definitely\_raise}
          \item <2-> \funcname{probably\_raise}
          \item <3-> \funcname{probably\_raise} (re-raise)
          \item <4-> \funcname{maybe\_raise}
          \item <5-> \funcname{main}
          \item <8-> \funcname{main} (re-raise)
        \end{itemize}
      \end{itemize}
      \vfill
      \onslide*<1-5>{\mintocaml[firstline=15,lastline=21]{../src/backtrace/backtrace.ml}}
      \onslide*<6>{\mintocaml[firstline=28,lastline=34,highlightlines=31]{../src/backtrace/backtrace.ml}}
      \onslide*<7->{\mintocaml[firstline=28,lastline=34,highlightlines=33]{../src/backtrace/backtrace.ml}}
      \endminipage
    \end{column}
    \begin{column}{0.45\textwidth}
      \raggedleft
      \onslide*<1>{\providecommand\step{6}\input{diagrams/backtrace/backtrace.tex}}%
      \onslide*<2>{\providecommand\step{6}\input{diagrams/backtrace/backtrace.tex}}%
      \onslide*<3>{\providecommand\step{7}\input{diagrams/backtrace/backtrace.tex}}%
      \onslide*<4>{\providecommand\step{8}\input{diagrams/backtrace/backtrace.tex}}%
      \onslide*<5>{\providecommand\step{9}\input{diagrams/backtrace/backtrace.tex}}%
      \onslide*<6>{\providecommand\step{10}\input{diagrams/backtrace/backtrace.tex}}%
      \onslide*<7>{\providecommand\step{11}\input{diagrams/backtrace/backtrace.tex}}%
      \onslide*<8>{\providecommand\step{12}\input{diagrams/backtrace/backtrace.tex}}%
      \onslide*<9>{\providecommand\step{13}\input{diagrams/backtrace/backtrace.tex}}%
    \end{column}
  \end{columns}
\end{frame}

\begin{frame}{Backtraces}{Printing the backtrace}
  \begin{columns}[c]
    \begin{column}{0.45\textwidth}
      \minipage[c][0.66\textheight][s]{\columnwidth}
      \begin{itemize}
        \item<1-> The exception backtrace can now be printed to \funcarg{stdout} with \funcname{Printexc.print_backtrace}
        \item<2-> First the exception backtrace is retrieved into an OCaml array with \funcname{get_raw_backtrace }
        \item<3-> Which is implemented as the \funcname{caml_get_exception_raw_backtrace} C function in the runtime
        \item<4-> And finally printed with \funcname{print_exception_backtrace}
      \end{itemize}
      \vfill
      \mintocaml[firstline=28,lastline=34,highlightlines=33]{../src/backtrace/backtrace.ml}
      \endminipage
    \end{column}
    \begin{column}{0.55\textwidth}
      \onslide*<1,2>{
        \mintocaml[firstline=100,lastline=101]{../ocaml/stdlib/printexc.ml}
      }
      \only<1>{
        \listocaml[firstline=170,lastline=172]{../ocaml/stdlib/printexc.ml}{stdlib/printexc.ml:170}
      }
      \only<2>{
        \listocaml[firstline=170,lastline=172,highlightlines=172]{../ocaml/stdlib/printexc.ml}{stdlib/printexc.ml:170}
      }
      \only<3>{
        \mintc[firstline=154,lastline=156]{../ocaml/runtime/backtrace.c}
        \listc[firstline=171,lastline=192,highlightlines={184-187}]{../ocaml/runtime/backtrace.c}{runtime/backtrace.c:154}
      }
      \only<4>{
        \listocaml[firstline=155,lastline=165]{../ocaml/stdlib/printexc.ml}{stdlib/printexc.ml:55}
      }
    \end{column}
  \end{columns}
\end{frame}


\subsection{The multiple kinds of raise}
\frameSubsection{}{}

\begin{frame}{Backtraces}{When backtraces are not needed}
  \begin{columns}[c]
    \begin{column}{0.45\textwidth}
      \minipage[c][0.66\textheight][s]{\columnwidth}
      \begin{itemize}
        \item<1-> What if the backtrace for an exception is never needed?
        \item<2-> It's possible to raise the exception explicitly with \funcname{raise\_notrace} to make raising as cheap as possible
        \item<3-> An example of use in the \funcname{dynlink} library of the OCaml compiler
      \end{itemize}
      \vfill
      \onslide<3->{
      \listocaml[firstline=205,lastline=209,highlightlines=207]{../ocaml/otherlibs/dynlink/dynlink\_compilerlibs/misc.ml}{otherlibs/dynlink/dynlink\_compilerlibs/misc.ml:205}
      }
      \endminipage
    \end{column}
    \begin{column}{0.55\textwidth}
    \only<4>{
      \mintcmm[highlightlines=3]{../src/notrace/camlNotrace\_\_fun_347.cmm}
      \listcmm{../src/notrace/camlNotrace\_\_all\_somes\_327.cmm}{all\_somes}
    }
    \onslide*<5->{
      \listobjdump[highlightlines={6-9}]{../src/notrace/camlNotrace\_\_fun_347.objdump}{anonymous function from \funcname{all\_somes}}
    }
    \end{column}
  \end{columns}
\end{frame}

\begin{frame}{Backtraces}{Summary table of the multiple kinds of raise}
  \begin{itemize}
    \item When debugging information is enabled and if the backtrace of an exception isn't needed, it's possible to raise with \funcname{raise\_notrace} for performance
    \item When debugging information is \emph{not} enabled, the compiler optimises \funcname{raise} calls to inline assembly (same effect as using \funcname{raise\_notrace})
  \end{itemize}
  \bigskip
  \centering
  \begin{tabular}{| l | c | c | c | c |}
    \hline
    \parbox{10em}{Debug information \\ (\funcarg{-g} flag)}  & \multicolumn{2}{|c|}{Disabled}                                     & \multicolumn{2}{|c|}{Enabled} \\
    \hline
    Raise kind                                               & \funcname{raise} & \funcname{raise\_notrace}                       & \funcname{raise} & \funcname{raise\_notrace} \\
    \hline
    Compilation                                              & \parbox{8em}{inline assembly\\ (optimization)} & inline assembly   & \funcname{caml\_raise\_exn} & inline assembly \\
    \hline
  \end{tabular}
\end{frame}


%
% Takeaway #5
%
\subsection*{Takeaway \#5}
\frameSubsectionTakeaway{}
\againframe<5>{takeaway}
}%
      \onslide*<6>{\providecommand\step{2}%
% Backtraces
%
\subsection{One advantage of raising with debugging information}
\frameSubsection{}{}

\begin{frame}{Backtraces}{Debugging information \& source code}
  \begin{columns}[c]
    \begin{column}{0.5\textwidth}
      \begin{itemize}
        \item Raising an exception is fun and games, but how do we know where the exception originated? $\rightarrow$ With the backtrace associated with the exception!
        \item In order to get a backtrace from the point where an exception was raised, it's necessary to compile with debugging information enabled (\funcarg{-g} flag for the compiler)
        \item Same example as in the `Nested handlers' section
      \end{itemize}
    \end{column}
    \begin{column}{0.5\textwidth}
      \listocaml[firstline=9,lastline=34]{../src/backtrace/backtrace.ml}{backtrace/backtrace.ml}
    \end{column}
  \end{columns}
\end{frame}

\begin{frame}{Backtraces}{CMM and disassembly of \funcname{definitely\_raise}}
  \begin{columns}[c]
    \begin{column}{0.5\textwidth}
      \minipage[c][0.66\textheight][s]{\columnwidth}
      \begin{itemize}
        \item<1-> An effect of compiling with debugging information (\funcarg{-g}): the CMM no longer uses \funcname{raise\_notrace} to raise the exception but \funcname{raise} instead
        \item<2-> Disassembly shows that CMM \funcname{raise} is actually a call to \funcname{caml\_raise\_exn}, a function in assembler provided by the runtime
      \end{itemize}
      \vfill
      \mintocaml[firstline=9,lastline=13,highlightlines=12]{../src/backtrace/backtrace.ml}
      \endminipage
    \end{column}
    \begin{column}{0.5\textwidth}
      \onslide*<1>{
        \mintcmm[highlightlines=5]{../src/backtrace/camlBacktrace\_\_definitely\_raise\_347.cmm}
      }
      \onslide*<2>{
        \mintobjdump[firstline=2,highlightlines=14]{../src/backtrace/camlBacktrace\_\_definitely\_raise\_347.objdump}
      }
    \end{column}
  \end{columns}
\end{frame}

\begin{frame}{Backtraces}{Introducing \funcname{caml\_raise\_exn}}
  \begin{columns}[c]
    \begin{column}{0.5\textwidth}
      \minipage[c][0.66\textheight][s]{\columnwidth}
      \onslide*<1>{
        \begin{itemize}
          \item Raising the exception is performed using \funcname{caml\_raise\_exn} which is
            \begin{itemize}
              \item An assembly function provided by the runtime
              \item Architecture specific
            \end{itemize}
          \item Let's see what it does differently from \funcname{raise\_notrace}\dots
          \item We are about to raise \typename{ExnC} with \funcname{caml\_raise\_exn} from \funcname{definitely\_raise}
        \end{itemize}
      }
      \onslide*<2>{
        \mintobjdump[firstline=2,highlightlines=14]{../src/backtrace/camlBacktrace\_\_definitely\_raise\_347.objdump}
      }
      \vfill
      \mintocaml[firstline=9,lastline=13,highlightlines=12]{../src/backtrace/backtrace.ml}
      \endminipage
    \end{column}
    \begin{column}{0.5\textwidth}
      \centering
      \providecommand\step{1}\input{diagrams/backtrace/backtrace.tex}
    \end{column}
  \end{columns}
\end{frame}

\begin{frame}{Backtraces}{But before that, we need more fields from \typename{caml\_domain\_state}}
  \begin{columns}
    \begin{column}{0.5\textwidth}
      \begin{itemize}
        \item Remember \typename{caml\_domain\_state} the per-domain runtime state?
        \item Some other members are of interest to us now\dots
        \item \localname{backtrace\_active} is used to know if the runtime should collect backtraces
        \item \localname{backtrace\_active} can be set by
          \begin{itemize}
            \item The \funcarg{b} flag in the \funcname{OCAMLRUNPARAM} environment variable
            \item \mintinline{ocaml}{Printexc.record_backtrace : bool -> unit} \footnotemark
          \end{itemize}
      \end{itemize}
    \end{column}
    \begin{column}{0.5\textwidth}
      \begin{listing}
        \mintc[highlightlines={9-11}]{src/ocaml/domain\_state\_backtrace.h}
        \caption{runtime/caml/domain\_state.h}
      \end{listing}
    \end{column}
  \end{columns}
  \footnotetext[1]{https://v2.ocaml.org/api/Printexc.html}
\end{frame}

\newcommand\listCamlRaiseExn[1][]{
  \listasm[firstline=702,lastline=725,#1]{../ocaml/runtime/amd64.S}{runtime/amd64.S:702}}

\begin{frame}{Backtraces}{Walk through \funcname{caml\_raise\_exn}}
  \begin{columns}[T]
    \begin{column}{0.5\textwidth}
      \begin{itemize}
        \item<1-> First checks whether exceptions backtrace should be recorded
        \item<1-> By checking the \funcarg{domain\_state->backtrace\_active} value
        \item<2-> If not, acts as \funcname{raise\_notrace} with \funcname{RESTORE\_EXN\_HANDLER\_OCAML}
          \begin{itemize}
            \item Set \regname{RSP} to \localname{domain\_state->exn\_handler} value
            \item Restore \localname{domain\_state->exn\_handler} from trap
            \item Jump with \funcname{ret} (same effect as using \funcname{pop} and \funcname{jmp})
          \end{itemize}
        \item<3-> Otherwise, switches to the C stack to be able to execute C code
        \item<4-> Calls \funcname{caml\_stash\_backtrace}, a C function provided by the runtime\ignorespaces
          \onslide<6->{, with
            \begin{itemize}
              \item \funcarg{exn}: exception value
              \item \funcarg{pc}: code address of the raising point
              \item \funcarg{sp}: stack address of the raising function
              \item \funcarg{trapsp}: stack address of the next handler
            \end{itemize}
          }
      \end{itemize}
      \bigskip
      \mintocaml[firstline=9,lastline=13,highlightlines=12]{../src/backtrace/backtrace.ml}
    \end{column}
    \begin{column}{0.5\textwidth}
      \raggedleft
      \onslide*<1,3-4>{
        \mintc[firstline=190,lastline=192]{../ocaml/runtime/amd64.S}
        \mintc[firstline=234,lastline=237]{../ocaml/runtime/amd64.S}
      }
      \onslide*<2>{
        \mintc[firstline=190,lastline=192,highlightlines={191-192}]{../ocaml/runtime/amd64.S}
        \mintc[firstline=234,lastline=237,highlightlines={235-237}]{../ocaml/runtime/amd64.S}
      }
      \onslide*<1>{\listCamlRaiseExn[highlightlines=705]{}}%
      \onslide*<2>{\listCamlRaiseExn[highlightlines={707-708}]{}}%
      \onslide*<3>{\listCamlRaiseExn[highlightlines={712-714}]{}}%
      \onslide*<4>{\listCamlRaiseExn[highlightlines={715-720}]{}}%
      \onslide*<5>{\providecommand\step{1}\input{diagrams/backtrace/backtrace.tex}}%
      \onslide*<6>{\providecommand\step{2}\input{diagrams/backtrace/backtrace.tex}}%
    \end{column}
  \end{columns}
\end{frame}

\newcommand\listStashBacktrace[1][]{
  \listc[firstline=91,lastline=120,#1]{../ocaml/runtime/backtrace\_nat.c}{runtime/backtrace\_nat.c:91}}

\begin{frame}{Backtraces}{Introducing \funcname{caml\_stash\_backtrace}}
  \begin{columns}[c]
    \begin{column}{0.5\textwidth}
      \minipage[c][0.66\textheight][s]{\columnwidth}
      \begin{itemize}
        \item<1-> A C function provided by the runtime (specific to native code)
        \item<2-> It scans the stack, between the stack pointer at the raising point up to the next trap, in order to collect the names of the detected function frames
          \onslide*<2>{
            \begin{itemize}
              \item And more information, like line number, column number...
            \end{itemize}
          }
        \item<3-> It uses the same technique and information as the Garbage Collector (GC) uses to scan the values live on the stack
          \onslide*<4>{
            \begin{itemize}
              \item \typename{frame\_descr} are at the core of stack unwinding
            \end{itemize}
          }
      \end{itemize}
      \vfill
      \mintocaml[firstline=9,lastline=13,highlightlines=12]{../src/backtrace/backtrace.ml}
      \endminipage
    \end{column}
    \begin{column}{0.5\textwidth}
      \onslide*<1>{\listStashBacktrace[highlightlines=91]{}}
      \onslide*<2>{\listStashBacktrace[highlightlines={91,117-118}]{}}
      \onslide*<3->{\listStashBacktrace[highlightlines={94,106,109-110}]{}}
    \end{column}
  \end{columns}
\end{frame}

\newcommand\listFrameDescr[1][]{
  \listocaml[firstline=111,lastline=124,#1]{../ocaml/asmcomp/emitaux.ml}{asmcomp/emitaux.ml:111}}
\newcommand\listDebugInfo[1][]{
  \listocaml[firstline=38,lastline=49,#1]{../ocaml/lambda/debuginfo.mli}{lambda/debuginfo.mli:38}}

\begin{frame}{Backtraces}{\typename{frame\_descr} and \typename{frame\_debuginfo} in a nutshell}
  \begin{columns}[c]
    \begin{column}{0.5\textwidth}
      \begin{itemize}
        \item<1-> For every return address in an OCaml program, there is a associated \typename{frame\_descr}
        \item<2-> A \typename{frame\_descr} specifies
        \begin{itemize}
          \item<2-> The size of the function stack frame
          \item<3-> Where live values are on the stack (so that the GC can mark them as live and prevent from being swept)
        \end{itemize}
        \item<4-> When compiled with debug information, \typename{frame\_descr} will be linked to a \typename{frame\_debuginfo}
        \begin{itemize}
          \item<5-> Contains information about the position in the source code (file name, line and column number)
        \end{itemize}
      \end{itemize}
    \end{column}
    \begin{column}{0.5\textwidth}
        \onslide*<1>{\listFrameDescr[highlightlines={118-122}]{}}
        \onslide*<2>{\listFrameDescr[highlightlines={120}]{}}
        \onslide*<3>{\listFrameDescr[highlightlines={121}]{}}
        \onslide*<4->{\listFrameDescr[highlightlines={122}]{}}
        \medskip
        \onslide*<1-3>{\listDebugInfo{}}
        \onslide*<4>{\listDebugInfo[highlightlines={38,49}]{}}
        \onslide*<5>{\listDebugInfo[highlightlines={39-46}]{}}
    \end{column}
  \end{columns}
\end{frame}

\begin{frame}{Backtraces}{Scanning the stack with \typename{frame\_descr}}
  \begin{columns}[c]
    \begin{column}{0.5\textwidth}
      \begin{itemize}
        \item Given the return address of the code that raises the exception by calling \funcname{caml\_raise\_exn} (\funcarg{pc} argument of \funcname{caml\_stash\_backtrace})
        \item And the position of the stack pointer (\funcarg{sp} argument of \funcname{caml\_stash\_backtrace})
        \item The frame size given by \typename{frame\_descr}.\funcarg{frame\_size}, allows to find the end of the previous frame
        \item And the return address of the caller of this function
        \bigskip
        \onslide<3->{
        \item Note that traps (here the reddish \funcname{ExnA}, \funcname{ExnB} and \funcname{ExnC} blocks) belong to the frame that installed them, and so the frame size changes when traps are removed
        }
      \end{itemize}
    \end{column}
    \begin{column}{0.5\textwidth}
      \raggedleft
      \onslide*<1>{\providecommand\step{2}\input{diagrams/backtrace/backtrace.tex}}%
      \onslide*<2>{\providecommand\step{1}\input{diagrams/backtrace/scanstack.tex}}%
      \onslide*<3>{\providecommand\step{2}\input{diagrams/backtrace/scanstack.tex}}%
      \onslide*<4>{\providecommand\step{3}\input{diagrams/backtrace/scanstack.tex}}%
      \onslide*<5>{\providecommand\step{4}\input{diagrams/backtrace/scanstack.tex}}%
      \onslide*<6>{\providecommand\step{5}\input{diagrams/backtrace/scanstack.tex}}%
    \end{column}
  \end{columns}
\end{frame}

% Duplicate of previous-previous slide
\begin{frame}{Backtraces}{Continuing on \funcname{caml\_stash\_backtrace}}
  \begin{columns}[c]
    \begin{column}{0.5\textwidth}
      \minipage[c][0.75\textheight][s]{\columnwidth}
      \begin{itemize}
          \color{gray}
        \item A C function provided by the runtime (specific to native code)
        \item It scans the stack, between the stack pointer at the raising point up to the next trap, in order to collect the names of the detected function frames
        \item It uses the same technique and information as the Garbage Collector (GC) uses to scan the values live on the stack
      \end{itemize}
      \begin{itemize}
        \item For each function frame detected, store a pointer to the \typename{frame\_descr} in the \localname{domain\_state->backtrace\_buffer} array, effectively constructing the backtrace
      \end{itemize}
      \onslide<2->{
        \begin{itemize}
            \smallskip
          \item Constructed backtrace:
            \funcname{definitely\_raise},
            \onslide<3->{\funcname{probably\_raise}}
        \end{itemize}
      }
      \vfill
      \mintocaml[firstline=9,lastline=13,highlightlines=12]{../src/backtrace/backtrace.ml}
      \endminipage
    \end{column}
    \begin{column}{0.5\textwidth}
      \raggedleft
      \onslide*<1>{\listStashBacktrace[highlightlines={114-115}]{}}
      \onslide*<2>{\providecommand\step{3}\input{diagrams/backtrace/backtrace.tex}}%
      \onslide*<3>{\providecommand\step{4}\input{diagrams/backtrace/backtrace.tex}}%
    \end{column}
  \end{columns}
\end{frame}

\begin{frame}{Backtraces}{Back to \funcname{caml\_raise\_exn}}
  \begin{columns}[c]
    \begin{column}{0.5\textwidth}
      \minipage[c][0.66\textheight][s]{\columnwidth}
      \begin{itemize}\color{gray}
        \item Checks wheteher exceptions backtrace should be recorded, by checking the \funcarg{domain\_state->backtrace\_active} value
        \item Switches to the C stack to be able to execute C code
        \item Calls \funcname{caml\_stash\_backtrace}, to collect the backtrace up to the next trap
      \end{itemize}
      \onslide*<2->{
        \begin{itemize}
          \item<2-> Executes the exception handler in \funcname{probably\_raise} with \funcname{RESTORE\_EXN\_HANDLER\_OCAML}
            \begin{itemize}
              \item Set \regname{RSP} to \localname{domain\_state->exn\_handler} value
              \item Restore \localname{domain\_state->exn\_handler} from trap
              \item Jump to the handler code with \funcname{ret}
            \end{itemize}
        \end{itemize}
      }
      \vfill
      \onslide*<1>{\mintocaml[firstline=9,lastline=13,highlightlines=12]{../src/backtrace/backtrace.ml}}
      \onslide*<2->{\mintocaml[firstline=15,lastline=21,highlightlines={19-21}]{../src/backtrace/backtrace.ml}}
      \endminipage
    \end{column}
    \begin{column}{0.5\textwidth}
      \centering
      \onslide*<1>{
        \mintc[firstline=190,lastline=192]{../ocaml/runtime/amd64.S}
        \mintc[firstline=234,lastline=237]{../ocaml/runtime/amd64.S}
      }
      \onslide*<2>{
        \mintc[firstline=190,lastline=192,highlightlines={191-192}]{../ocaml/runtime/amd64.S}
        \mintc[firstline=234,lastline=237,highlightlines={235-237}]{../ocaml/runtime/amd64.S}
      }
      \onslide*<1>{\listCamlRaiseExn[highlightlines=720]{}}
      \onslide*<2>{\listCamlRaiseExn[highlightlines=722]{}}
      \onslide*<3->{\providecommand\step{5}\input{diagrams/backtrace/backtrace.tex}}
    \end{column}
  \end{columns}
\end{frame}

\begin{frame}{Backtraces}{\funcname{caml\_probably\_raise}'s handler doesn't catch \typename{ExnC}}
  \begin{columns}[c]
    \begin{column}{0.45\textwidth}
      \minipage[c][0.75\textheight][s]{\columnwidth}
      \begin{itemize}
        \item<1-> \funcname{match \dots with}\footnotemark pattern matching can also match exceptions
        \item<1-> The CMM of \funcname{probably\_raise} shows that this \funcname{match \dots with} is implemented with a pattern matching and a \funcname{try \dots with}
        \item<1-> But this one only matches on \typename{ExnA} exception
        \item<2-> Since the pattern matching fails to catch the exception, \funcname{reraise} is called with our current \typename{ExnC} exception
        \item<3-> Disassembly shows that \funcname{reraise} is actually a call to \funcname{caml\_reraise\_exn}, which is also an assembly function provided by the runtime
      \end{itemize}
      \vfill
      \mintocaml[firstline=15,lastline=21,highlightlines={18-21}]{../src/backtrace/backtrace.ml}
      \endminipage
    \end{column}
    \begin{column}{0.55\textwidth}
      \centering
      \onslide*<1>{\mintcmm[highlightlines={10-18}]{../src/backtrace/camlBacktrace\_\_probably\_raise\_351.cmm}}
      \onslide*<2>{\listcmm[highlightlines=18]{../src/backtrace/camlBacktrace\_\_probably\_raise_351.cmm}{CMM of probably\_raise}}
      \onslide*<3>{\mintobjdump[firstline=5,lastline=35,highlightlines=31]{../src/backtrace/camlBacktrace\_\_probably\_raise_351.objdump}}
    \end{column}
  \end{columns}
  \only<1>{\footnotetext[1]{https://blog.janestreet.com/pattern-matching-and-exception-handling-unite/}}
\end{frame}

\newcommand\listCamlReraiseExn[1][]{
  \listasm[firstline=727,lastline=734,#1]{../ocaml/runtime/amd64.S}{runtime/amd64.S:727}}

\begin{frame}{Backtraces}{Introducing \funcname{caml\_reraise\_exn}}
  \begin{columns}[c]
    \begin{column}{0.5\textwidth}
      \minipage[c][0.66\textheight][s]{\columnwidth}
      \begin{itemize}
        \item<1-> The exception have to be reraised to the next handler
        \item<2-> First, checks if the backtrace should be collected with \funcarg{domain\_state->backtrace\_active}
        \only<3>{
        \begin{itemize}
            \item If not, behaves like a \funcname{raise\_notrace} with \funcname{RESTORE\_EXN\_HANDLER\_OCAML}
        \end{itemize}
        }
        \item<4-> Jumps into \funcname{caml\_raise\_exn}
        \item<5-> Calls \funcname{caml\_stash\_backtrace} again, to scan the next part of the stack: from the trap just pop-ed to the next trap
      \end{itemize}
      \vfill
      \mintocaml[firstline=15,lastline=21,highlightlines={18-21}]{../src/backtrace/backtrace.ml}
      \endminipage
    \end{column}
    \begin{column}{0.5\textwidth}
      \raggedleft
      \onslide*<1>{\listCamlReraiseExn{}}
      \onslide*<2>{\listCamlReraiseExn[highlightlines=729]{}}
      \onslide*<3>{
        \mintc[firstline=190,lastline=192,highlightlines={191-192}]{../ocaml/runtime/amd64.S}
        \mintc[firstline=234,lastline=237,highlightlines={235-237}]{../ocaml/runtime/amd64.S}
        \medskip
        \listCamlReraiseExn[highlightlines=731]{}
      }
      \onslide*<4-5>{
        % TODO refactor with \listCamlReraiseExn
        \mintasm[firstline=727,lastline=734,highlightlines=730]{../ocaml/runtime/amd64.S}
        \medskip
      }
      \onslide*<4>{\listCamlRaiseExn[highlightlines=711]{}}
      \onslide*<5>{\listCamlRaiseExn[highlightlines=720]{}}
    \end{column}
  \end{columns}
\end{frame}

\begin{frame}{Backtraces}{Backtrace collection continues}
  \begin{columns}[c]
    \begin{column}{0.55\textwidth}
      \minipage[c][0.75\textheight][s]{\columnwidth}
      \begin{itemize}
        \item<1-> \funcname{caml\_stash\_backtrace} scans the older part of the stack, up to the next trap
        \item<6-> Controls jumps to the nested exception handler in \funcname{maybe\_raise}, which matches only on \typename{ExnB}
        \item<7-> \funcname{caml\_reraise\_exn} is called again, backtrace collection resumes with \funcname{caml\_stash\_backtrace}
      \end{itemize}
      \medskip
      \begin{itemize}
        \item<2-> Constructed backtrace:
        \begin{itemize}
          \item <2-> \funcname{definitely\_raise}
          \item <2-> \funcname{probably\_raise}
          \item <3-> \funcname{probably\_raise} (re-raise)
          \item <4-> \funcname{maybe\_raise}
          \item <5-> \funcname{main}
          \item <8-> \funcname{main} (re-raise)
        \end{itemize}
      \end{itemize}
      \vfill
      \onslide*<1-5>{\mintocaml[firstline=15,lastline=21]{../src/backtrace/backtrace.ml}}
      \onslide*<6>{\mintocaml[firstline=28,lastline=34,highlightlines=31]{../src/backtrace/backtrace.ml}}
      \onslide*<7->{\mintocaml[firstline=28,lastline=34,highlightlines=33]{../src/backtrace/backtrace.ml}}
      \endminipage
    \end{column}
    \begin{column}{0.45\textwidth}
      \raggedleft
      \onslide*<1>{\providecommand\step{6}\input{diagrams/backtrace/backtrace.tex}}%
      \onslide*<2>{\providecommand\step{6}\input{diagrams/backtrace/backtrace.tex}}%
      \onslide*<3>{\providecommand\step{7}\input{diagrams/backtrace/backtrace.tex}}%
      \onslide*<4>{\providecommand\step{8}\input{diagrams/backtrace/backtrace.tex}}%
      \onslide*<5>{\providecommand\step{9}\input{diagrams/backtrace/backtrace.tex}}%
      \onslide*<6>{\providecommand\step{10}\input{diagrams/backtrace/backtrace.tex}}%
      \onslide*<7>{\providecommand\step{11}\input{diagrams/backtrace/backtrace.tex}}%
      \onslide*<8>{\providecommand\step{12}\input{diagrams/backtrace/backtrace.tex}}%
      \onslide*<9>{\providecommand\step{13}\input{diagrams/backtrace/backtrace.tex}}%
    \end{column}
  \end{columns}
\end{frame}

\begin{frame}{Backtraces}{Printing the backtrace}
  \begin{columns}[c]
    \begin{column}{0.45\textwidth}
      \minipage[c][0.66\textheight][s]{\columnwidth}
      \begin{itemize}
        \item<1-> The exception backtrace can now be printed to \funcarg{stdout} with \funcname{Printexc.print_backtrace}
        \item<2-> First the exception backtrace is retrieved into an OCaml array with \funcname{get_raw_backtrace }
        \item<3-> Which is implemented as the \funcname{caml_get_exception_raw_backtrace} C function in the runtime
        \item<4-> And finally printed with \funcname{print_exception_backtrace}
      \end{itemize}
      \vfill
      \mintocaml[firstline=28,lastline=34,highlightlines=33]{../src/backtrace/backtrace.ml}
      \endminipage
    \end{column}
    \begin{column}{0.55\textwidth}
      \onslide*<1,2>{
        \mintocaml[firstline=100,lastline=101]{../ocaml/stdlib/printexc.ml}
      }
      \only<1>{
        \listocaml[firstline=170,lastline=172]{../ocaml/stdlib/printexc.ml}{stdlib/printexc.ml:170}
      }
      \only<2>{
        \listocaml[firstline=170,lastline=172,highlightlines=172]{../ocaml/stdlib/printexc.ml}{stdlib/printexc.ml:170}
      }
      \only<3>{
        \mintc[firstline=154,lastline=156]{../ocaml/runtime/backtrace.c}
        \listc[firstline=171,lastline=192,highlightlines={184-187}]{../ocaml/runtime/backtrace.c}{runtime/backtrace.c:154}
      }
      \only<4>{
        \listocaml[firstline=155,lastline=165]{../ocaml/stdlib/printexc.ml}{stdlib/printexc.ml:55}
      }
    \end{column}
  \end{columns}
\end{frame}


\subsection{The multiple kinds of raise}
\frameSubsection{}{}

\begin{frame}{Backtraces}{When backtraces are not needed}
  \begin{columns}[c]
    \begin{column}{0.45\textwidth}
      \minipage[c][0.66\textheight][s]{\columnwidth}
      \begin{itemize}
        \item<1-> What if the backtrace for an exception is never needed?
        \item<2-> It's possible to raise the exception explicitly with \funcname{raise\_notrace} to make raising as cheap as possible
        \item<3-> An example of use in the \funcname{dynlink} library of the OCaml compiler
      \end{itemize}
      \vfill
      \onslide<3->{
      \listocaml[firstline=205,lastline=209,highlightlines=207]{../ocaml/otherlibs/dynlink/dynlink\_compilerlibs/misc.ml}{otherlibs/dynlink/dynlink\_compilerlibs/misc.ml:205}
      }
      \endminipage
    \end{column}
    \begin{column}{0.55\textwidth}
    \only<4>{
      \mintcmm[highlightlines=3]{../src/notrace/camlNotrace\_\_fun_347.cmm}
      \listcmm{../src/notrace/camlNotrace\_\_all\_somes\_327.cmm}{all\_somes}
    }
    \onslide*<5->{
      \listobjdump[highlightlines={6-9}]{../src/notrace/camlNotrace\_\_fun_347.objdump}{anonymous function from \funcname{all\_somes}}
    }
    \end{column}
  \end{columns}
\end{frame}

\begin{frame}{Backtraces}{Summary table of the multiple kinds of raise}
  \begin{itemize}
    \item When debugging information is enabled and if the backtrace of an exception isn't needed, it's possible to raise with \funcname{raise\_notrace} for performance
    \item When debugging information is \emph{not} enabled, the compiler optimises \funcname{raise} calls to inline assembly (same effect as using \funcname{raise\_notrace})
  \end{itemize}
  \bigskip
  \centering
  \begin{tabular}{| l | c | c | c | c |}
    \hline
    \parbox{10em}{Debug information \\ (\funcarg{-g} flag)}  & \multicolumn{2}{|c|}{Disabled}                                     & \multicolumn{2}{|c|}{Enabled} \\
    \hline
    Raise kind                                               & \funcname{raise} & \funcname{raise\_notrace}                       & \funcname{raise} & \funcname{raise\_notrace} \\
    \hline
    Compilation                                              & \parbox{8em}{inline assembly\\ (optimization)} & inline assembly   & \funcname{caml\_raise\_exn} & inline assembly \\
    \hline
  \end{tabular}
\end{frame}


%
% Takeaway #5
%
\subsection*{Takeaway \#5}
\frameSubsectionTakeaway{}
\againframe<5>{takeaway}
}%
    \end{column}
  \end{columns}
\end{frame}

\newcommand\listStashBacktrace[1][]{
  \listc[firstline=91,lastline=120,#1]{../ocaml/runtime/backtrace\_nat.c}{runtime/backtrace\_nat.c:91}}

\begin{frame}{Backtraces}{Introducing \funcname{caml\_stash\_backtrace}}
  \begin{columns}[c]
    \begin{column}{0.5\textwidth}
      \minipage[c][0.66\textheight][s]{\columnwidth}
      \begin{itemize}
        \item<1-> A C function provided by the runtime (specific to native code)
        \item<2-> It scans the stack, between the stack pointer at the raising point up to the next trap, in order to collect the names of the detected function frames
          \onslide*<2>{
            \begin{itemize}
              \item And more information, like line number, column number...
            \end{itemize}
          }
        \item<3-> It uses the same technique and information as the Garbage Collector (GC) uses to scan the values live on the stack
          \onslide*<4>{
            \begin{itemize}
              \item \typename{frame\_descr} are at the core of stack unwinding
            \end{itemize}
          }
      \end{itemize}
      \vfill
      \mintocaml[firstline=9,lastline=13,highlightlines=12]{../src/backtrace/backtrace.ml}
      \endminipage
    \end{column}
    \begin{column}{0.5\textwidth}
      \onslide*<1>{\listStashBacktrace[highlightlines=91]{}}
      \onslide*<2>{\listStashBacktrace[highlightlines={91,117-118}]{}}
      \onslide*<3->{\listStashBacktrace[highlightlines={94,106,109-110}]{}}
    \end{column}
  \end{columns}
\end{frame}

\newcommand\listFrameDescr[1][]{
  \listocaml[firstline=111,lastline=124,#1]{../ocaml/asmcomp/emitaux.ml}{asmcomp/emitaux.ml:111}}
\newcommand\listDebugInfo[1][]{
  \listocaml[firstline=38,lastline=49,#1]{../ocaml/lambda/debuginfo.mli}{lambda/debuginfo.mli:38}}

\begin{frame}{Backtraces}{\typename{frame\_descr} and \typename{frame\_debuginfo} in a nutshell}
  \begin{columns}[c]
    \begin{column}{0.5\textwidth}
      \begin{itemize}
        \item<1-> For every return address in an OCaml program, there is a associated \typename{frame\_descr}
        \item<2-> A \typename{frame\_descr} specifies
        \begin{itemize}
          \item<2-> The size of the function stack frame
          \item<3-> Where live values are on the stack (so that the GC can mark them as live and prevent from being swept)
        \end{itemize}
        \item<4-> When compiled with debug information, \typename{frame\_descr} will be linked to a \typename{frame\_debuginfo}
        \begin{itemize}
          \item<5-> Contains information about the position in the source code (file name, line and column number)
        \end{itemize}
      \end{itemize}
    \end{column}
    \begin{column}{0.5\textwidth}
        \onslide*<1>{\listFrameDescr[highlightlines={118-122}]{}}
        \onslide*<2>{\listFrameDescr[highlightlines={120}]{}}
        \onslide*<3>{\listFrameDescr[highlightlines={121}]{}}
        \onslide*<4->{\listFrameDescr[highlightlines={122}]{}}
        \medskip
        \onslide*<1-3>{\listDebugInfo{}}
        \onslide*<4>{\listDebugInfo[highlightlines={38,49}]{}}
        \onslide*<5>{\listDebugInfo[highlightlines={39-46}]{}}
    \end{column}
  \end{columns}
\end{frame}

\begin{frame}{Backtraces}{Scanning the stack with \typename{frame\_descr}}
  \begin{columns}[c]
    \begin{column}{0.5\textwidth}
      \begin{itemize}
        \item Given the return address of the code that raises the exception by calling \funcname{caml\_raise\_exn} (\funcarg{pc} argument of \funcname{caml\_stash\_backtrace})
        \item And the position of the stack pointer (\funcarg{sp} argument of \funcname{caml\_stash\_backtrace})
        \item The frame size given by \typename{frame\_descr}.\funcarg{frame\_size}, allows to find the end of the previous frame
        \item And the return address of the caller of this function
        \bigskip
        \onslide<3->{
        \item Note that traps (here the reddish \funcname{ExnA}, \funcname{ExnB} and \funcname{ExnC} blocks) belong to the frame that installed them, and so the frame size changes when traps are removed
        }
      \end{itemize}
    \end{column}
    \begin{column}{0.5\textwidth}
      \raggedleft
      \onslide*<1>{\providecommand\step{2}%
% Backtraces
%
\subsection{One advantage of raising with debugging information}
\frameSubsection{}{}

\begin{frame}{Backtraces}{Debugging information \& source code}
  \begin{columns}[c]
    \begin{column}{0.5\textwidth}
      \begin{itemize}
        \item Raising an exception is fun and games, but how do we know where the exception originated? $\rightarrow$ With the backtrace associated with the exception!
        \item In order to get a backtrace from the point where an exception was raised, it's necessary to compile with debugging information enabled (\funcarg{-g} flag for the compiler)
        \item Same example as in the `Nested handlers' section
      \end{itemize}
    \end{column}
    \begin{column}{0.5\textwidth}
      \listocaml[firstline=9,lastline=34]{../src/backtrace/backtrace.ml}{backtrace/backtrace.ml}
    \end{column}
  \end{columns}
\end{frame}

\begin{frame}{Backtraces}{CMM and disassembly of \funcname{definitely\_raise}}
  \begin{columns}[c]
    \begin{column}{0.5\textwidth}
      \minipage[c][0.66\textheight][s]{\columnwidth}
      \begin{itemize}
        \item<1-> An effect of compiling with debugging information (\funcarg{-g}): the CMM no longer uses \funcname{raise\_notrace} to raise the exception but \funcname{raise} instead
        \item<2-> Disassembly shows that CMM \funcname{raise} is actually a call to \funcname{caml\_raise\_exn}, a function in assembler provided by the runtime
      \end{itemize}
      \vfill
      \mintocaml[firstline=9,lastline=13,highlightlines=12]{../src/backtrace/backtrace.ml}
      \endminipage
    \end{column}
    \begin{column}{0.5\textwidth}
      \onslide*<1>{
        \mintcmm[highlightlines=5]{../src/backtrace/camlBacktrace\_\_definitely\_raise\_347.cmm}
      }
      \onslide*<2>{
        \mintobjdump[firstline=2,highlightlines=14]{../src/backtrace/camlBacktrace\_\_definitely\_raise\_347.objdump}
      }
    \end{column}
  \end{columns}
\end{frame}

\begin{frame}{Backtraces}{Introducing \funcname{caml\_raise\_exn}}
  \begin{columns}[c]
    \begin{column}{0.5\textwidth}
      \minipage[c][0.66\textheight][s]{\columnwidth}
      \onslide*<1>{
        \begin{itemize}
          \item Raising the exception is performed using \funcname{caml\_raise\_exn} which is
            \begin{itemize}
              \item An assembly function provided by the runtime
              \item Architecture specific
            \end{itemize}
          \item Let's see what it does differently from \funcname{raise\_notrace}\dots
          \item We are about to raise \typename{ExnC} with \funcname{caml\_raise\_exn} from \funcname{definitely\_raise}
        \end{itemize}
      }
      \onslide*<2>{
        \mintobjdump[firstline=2,highlightlines=14]{../src/backtrace/camlBacktrace\_\_definitely\_raise\_347.objdump}
      }
      \vfill
      \mintocaml[firstline=9,lastline=13,highlightlines=12]{../src/backtrace/backtrace.ml}
      \endminipage
    \end{column}
    \begin{column}{0.5\textwidth}
      \centering
      \providecommand\step{1}\input{diagrams/backtrace/backtrace.tex}
    \end{column}
  \end{columns}
\end{frame}

\begin{frame}{Backtraces}{But before that, we need more fields from \typename{caml\_domain\_state}}
  \begin{columns}
    \begin{column}{0.5\textwidth}
      \begin{itemize}
        \item Remember \typename{caml\_domain\_state} the per-domain runtime state?
        \item Some other members are of interest to us now\dots
        \item \localname{backtrace\_active} is used to know if the runtime should collect backtraces
        \item \localname{backtrace\_active} can be set by
          \begin{itemize}
            \item The \funcarg{b} flag in the \funcname{OCAMLRUNPARAM} environment variable
            \item \mintinline{ocaml}{Printexc.record_backtrace : bool -> unit} \footnotemark
          \end{itemize}
      \end{itemize}
    \end{column}
    \begin{column}{0.5\textwidth}
      \begin{listing}
        \mintc[highlightlines={9-11}]{src/ocaml/domain\_state\_backtrace.h}
        \caption{runtime/caml/domain\_state.h}
      \end{listing}
    \end{column}
  \end{columns}
  \footnotetext[1]{https://v2.ocaml.org/api/Printexc.html}
\end{frame}

\newcommand\listCamlRaiseExn[1][]{
  \listasm[firstline=702,lastline=725,#1]{../ocaml/runtime/amd64.S}{runtime/amd64.S:702}}

\begin{frame}{Backtraces}{Walk through \funcname{caml\_raise\_exn}}
  \begin{columns}[T]
    \begin{column}{0.5\textwidth}
      \begin{itemize}
        \item<1-> First checks whether exceptions backtrace should be recorded
        \item<1-> By checking the \funcarg{domain\_state->backtrace\_active} value
        \item<2-> If not, acts as \funcname{raise\_notrace} with \funcname{RESTORE\_EXN\_HANDLER\_OCAML}
          \begin{itemize}
            \item Set \regname{RSP} to \localname{domain\_state->exn\_handler} value
            \item Restore \localname{domain\_state->exn\_handler} from trap
            \item Jump with \funcname{ret} (same effect as using \funcname{pop} and \funcname{jmp})
          \end{itemize}
        \item<3-> Otherwise, switches to the C stack to be able to execute C code
        \item<4-> Calls \funcname{caml\_stash\_backtrace}, a C function provided by the runtime\ignorespaces
          \onslide<6->{, with
            \begin{itemize}
              \item \funcarg{exn}: exception value
              \item \funcarg{pc}: code address of the raising point
              \item \funcarg{sp}: stack address of the raising function
              \item \funcarg{trapsp}: stack address of the next handler
            \end{itemize}
          }
      \end{itemize}
      \bigskip
      \mintocaml[firstline=9,lastline=13,highlightlines=12]{../src/backtrace/backtrace.ml}
    \end{column}
    \begin{column}{0.5\textwidth}
      \raggedleft
      \onslide*<1,3-4>{
        \mintc[firstline=190,lastline=192]{../ocaml/runtime/amd64.S}
        \mintc[firstline=234,lastline=237]{../ocaml/runtime/amd64.S}
      }
      \onslide*<2>{
        \mintc[firstline=190,lastline=192,highlightlines={191-192}]{../ocaml/runtime/amd64.S}
        \mintc[firstline=234,lastline=237,highlightlines={235-237}]{../ocaml/runtime/amd64.S}
      }
      \onslide*<1>{\listCamlRaiseExn[highlightlines=705]{}}%
      \onslide*<2>{\listCamlRaiseExn[highlightlines={707-708}]{}}%
      \onslide*<3>{\listCamlRaiseExn[highlightlines={712-714}]{}}%
      \onslide*<4>{\listCamlRaiseExn[highlightlines={715-720}]{}}%
      \onslide*<5>{\providecommand\step{1}\input{diagrams/backtrace/backtrace.tex}}%
      \onslide*<6>{\providecommand\step{2}\input{diagrams/backtrace/backtrace.tex}}%
    \end{column}
  \end{columns}
\end{frame}

\newcommand\listStashBacktrace[1][]{
  \listc[firstline=91,lastline=120,#1]{../ocaml/runtime/backtrace\_nat.c}{runtime/backtrace\_nat.c:91}}

\begin{frame}{Backtraces}{Introducing \funcname{caml\_stash\_backtrace}}
  \begin{columns}[c]
    \begin{column}{0.5\textwidth}
      \minipage[c][0.66\textheight][s]{\columnwidth}
      \begin{itemize}
        \item<1-> A C function provided by the runtime (specific to native code)
        \item<2-> It scans the stack, between the stack pointer at the raising point up to the next trap, in order to collect the names of the detected function frames
          \onslide*<2>{
            \begin{itemize}
              \item And more information, like line number, column number...
            \end{itemize}
          }
        \item<3-> It uses the same technique and information as the Garbage Collector (GC) uses to scan the values live on the stack
          \onslide*<4>{
            \begin{itemize}
              \item \typename{frame\_descr} are at the core of stack unwinding
            \end{itemize}
          }
      \end{itemize}
      \vfill
      \mintocaml[firstline=9,lastline=13,highlightlines=12]{../src/backtrace/backtrace.ml}
      \endminipage
    \end{column}
    \begin{column}{0.5\textwidth}
      \onslide*<1>{\listStashBacktrace[highlightlines=91]{}}
      \onslide*<2>{\listStashBacktrace[highlightlines={91,117-118}]{}}
      \onslide*<3->{\listStashBacktrace[highlightlines={94,106,109-110}]{}}
    \end{column}
  \end{columns}
\end{frame}

\newcommand\listFrameDescr[1][]{
  \listocaml[firstline=111,lastline=124,#1]{../ocaml/asmcomp/emitaux.ml}{asmcomp/emitaux.ml:111}}
\newcommand\listDebugInfo[1][]{
  \listocaml[firstline=38,lastline=49,#1]{../ocaml/lambda/debuginfo.mli}{lambda/debuginfo.mli:38}}

\begin{frame}{Backtraces}{\typename{frame\_descr} and \typename{frame\_debuginfo} in a nutshell}
  \begin{columns}[c]
    \begin{column}{0.5\textwidth}
      \begin{itemize}
        \item<1-> For every return address in an OCaml program, there is a associated \typename{frame\_descr}
        \item<2-> A \typename{frame\_descr} specifies
        \begin{itemize}
          \item<2-> The size of the function stack frame
          \item<3-> Where live values are on the stack (so that the GC can mark them as live and prevent from being swept)
        \end{itemize}
        \item<4-> When compiled with debug information, \typename{frame\_descr} will be linked to a \typename{frame\_debuginfo}
        \begin{itemize}
          \item<5-> Contains information about the position in the source code (file name, line and column number)
        \end{itemize}
      \end{itemize}
    \end{column}
    \begin{column}{0.5\textwidth}
        \onslide*<1>{\listFrameDescr[highlightlines={118-122}]{}}
        \onslide*<2>{\listFrameDescr[highlightlines={120}]{}}
        \onslide*<3>{\listFrameDescr[highlightlines={121}]{}}
        \onslide*<4->{\listFrameDescr[highlightlines={122}]{}}
        \medskip
        \onslide*<1-3>{\listDebugInfo{}}
        \onslide*<4>{\listDebugInfo[highlightlines={38,49}]{}}
        \onslide*<5>{\listDebugInfo[highlightlines={39-46}]{}}
    \end{column}
  \end{columns}
\end{frame}

\begin{frame}{Backtraces}{Scanning the stack with \typename{frame\_descr}}
  \begin{columns}[c]
    \begin{column}{0.5\textwidth}
      \begin{itemize}
        \item Given the return address of the code that raises the exception by calling \funcname{caml\_raise\_exn} (\funcarg{pc} argument of \funcname{caml\_stash\_backtrace})
        \item And the position of the stack pointer (\funcarg{sp} argument of \funcname{caml\_stash\_backtrace})
        \item The frame size given by \typename{frame\_descr}.\funcarg{frame\_size}, allows to find the end of the previous frame
        \item And the return address of the caller of this function
        \bigskip
        \onslide<3->{
        \item Note that traps (here the reddish \funcname{ExnA}, \funcname{ExnB} and \funcname{ExnC} blocks) belong to the frame that installed them, and so the frame size changes when traps are removed
        }
      \end{itemize}
    \end{column}
    \begin{column}{0.5\textwidth}
      \raggedleft
      \onslide*<1>{\providecommand\step{2}\input{diagrams/backtrace/backtrace.tex}}%
      \onslide*<2>{\providecommand\step{1}\input{diagrams/backtrace/scanstack.tex}}%
      \onslide*<3>{\providecommand\step{2}\input{diagrams/backtrace/scanstack.tex}}%
      \onslide*<4>{\providecommand\step{3}\input{diagrams/backtrace/scanstack.tex}}%
      \onslide*<5>{\providecommand\step{4}\input{diagrams/backtrace/scanstack.tex}}%
      \onslide*<6>{\providecommand\step{5}\input{diagrams/backtrace/scanstack.tex}}%
    \end{column}
  \end{columns}
\end{frame}

% Duplicate of previous-previous slide
\begin{frame}{Backtraces}{Continuing on \funcname{caml\_stash\_backtrace}}
  \begin{columns}[c]
    \begin{column}{0.5\textwidth}
      \minipage[c][0.75\textheight][s]{\columnwidth}
      \begin{itemize}
          \color{gray}
        \item A C function provided by the runtime (specific to native code)
        \item It scans the stack, between the stack pointer at the raising point up to the next trap, in order to collect the names of the detected function frames
        \item It uses the same technique and information as the Garbage Collector (GC) uses to scan the values live on the stack
      \end{itemize}
      \begin{itemize}
        \item For each function frame detected, store a pointer to the \typename{frame\_descr} in the \localname{domain\_state->backtrace\_buffer} array, effectively constructing the backtrace
      \end{itemize}
      \onslide<2->{
        \begin{itemize}
            \smallskip
          \item Constructed backtrace:
            \funcname{definitely\_raise},
            \onslide<3->{\funcname{probably\_raise}}
        \end{itemize}
      }
      \vfill
      \mintocaml[firstline=9,lastline=13,highlightlines=12]{../src/backtrace/backtrace.ml}
      \endminipage
    \end{column}
    \begin{column}{0.5\textwidth}
      \raggedleft
      \onslide*<1>{\listStashBacktrace[highlightlines={114-115}]{}}
      \onslide*<2>{\providecommand\step{3}\input{diagrams/backtrace/backtrace.tex}}%
      \onslide*<3>{\providecommand\step{4}\input{diagrams/backtrace/backtrace.tex}}%
    \end{column}
  \end{columns}
\end{frame}

\begin{frame}{Backtraces}{Back to \funcname{caml\_raise\_exn}}
  \begin{columns}[c]
    \begin{column}{0.5\textwidth}
      \minipage[c][0.66\textheight][s]{\columnwidth}
      \begin{itemize}\color{gray}
        \item Checks wheteher exceptions backtrace should be recorded, by checking the \funcarg{domain\_state->backtrace\_active} value
        \item Switches to the C stack to be able to execute C code
        \item Calls \funcname{caml\_stash\_backtrace}, to collect the backtrace up to the next trap
      \end{itemize}
      \onslide*<2->{
        \begin{itemize}
          \item<2-> Executes the exception handler in \funcname{probably\_raise} with \funcname{RESTORE\_EXN\_HANDLER\_OCAML}
            \begin{itemize}
              \item Set \regname{RSP} to \localname{domain\_state->exn\_handler} value
              \item Restore \localname{domain\_state->exn\_handler} from trap
              \item Jump to the handler code with \funcname{ret}
            \end{itemize}
        \end{itemize}
      }
      \vfill
      \onslide*<1>{\mintocaml[firstline=9,lastline=13,highlightlines=12]{../src/backtrace/backtrace.ml}}
      \onslide*<2->{\mintocaml[firstline=15,lastline=21,highlightlines={19-21}]{../src/backtrace/backtrace.ml}}
      \endminipage
    \end{column}
    \begin{column}{0.5\textwidth}
      \centering
      \onslide*<1>{
        \mintc[firstline=190,lastline=192]{../ocaml/runtime/amd64.S}
        \mintc[firstline=234,lastline=237]{../ocaml/runtime/amd64.S}
      }
      \onslide*<2>{
        \mintc[firstline=190,lastline=192,highlightlines={191-192}]{../ocaml/runtime/amd64.S}
        \mintc[firstline=234,lastline=237,highlightlines={235-237}]{../ocaml/runtime/amd64.S}
      }
      \onslide*<1>{\listCamlRaiseExn[highlightlines=720]{}}
      \onslide*<2>{\listCamlRaiseExn[highlightlines=722]{}}
      \onslide*<3->{\providecommand\step{5}\input{diagrams/backtrace/backtrace.tex}}
    \end{column}
  \end{columns}
\end{frame}

\begin{frame}{Backtraces}{\funcname{caml\_probably\_raise}'s handler doesn't catch \typename{ExnC}}
  \begin{columns}[c]
    \begin{column}{0.45\textwidth}
      \minipage[c][0.75\textheight][s]{\columnwidth}
      \begin{itemize}
        \item<1-> \funcname{match \dots with}\footnotemark pattern matching can also match exceptions
        \item<1-> The CMM of \funcname{probably\_raise} shows that this \funcname{match \dots with} is implemented with a pattern matching and a \funcname{try \dots with}
        \item<1-> But this one only matches on \typename{ExnA} exception
        \item<2-> Since the pattern matching fails to catch the exception, \funcname{reraise} is called with our current \typename{ExnC} exception
        \item<3-> Disassembly shows that \funcname{reraise} is actually a call to \funcname{caml\_reraise\_exn}, which is also an assembly function provided by the runtime
      \end{itemize}
      \vfill
      \mintocaml[firstline=15,lastline=21,highlightlines={18-21}]{../src/backtrace/backtrace.ml}
      \endminipage
    \end{column}
    \begin{column}{0.55\textwidth}
      \centering
      \onslide*<1>{\mintcmm[highlightlines={10-18}]{../src/backtrace/camlBacktrace\_\_probably\_raise\_351.cmm}}
      \onslide*<2>{\listcmm[highlightlines=18]{../src/backtrace/camlBacktrace\_\_probably\_raise_351.cmm}{CMM of probably\_raise}}
      \onslide*<3>{\mintobjdump[firstline=5,lastline=35,highlightlines=31]{../src/backtrace/camlBacktrace\_\_probably\_raise_351.objdump}}
    \end{column}
  \end{columns}
  \only<1>{\footnotetext[1]{https://blog.janestreet.com/pattern-matching-and-exception-handling-unite/}}
\end{frame}

\newcommand\listCamlReraiseExn[1][]{
  \listasm[firstline=727,lastline=734,#1]{../ocaml/runtime/amd64.S}{runtime/amd64.S:727}}

\begin{frame}{Backtraces}{Introducing \funcname{caml\_reraise\_exn}}
  \begin{columns}[c]
    \begin{column}{0.5\textwidth}
      \minipage[c][0.66\textheight][s]{\columnwidth}
      \begin{itemize}
        \item<1-> The exception have to be reraised to the next handler
        \item<2-> First, checks if the backtrace should be collected with \funcarg{domain\_state->backtrace\_active}
        \only<3>{
        \begin{itemize}
            \item If not, behaves like a \funcname{raise\_notrace} with \funcname{RESTORE\_EXN\_HANDLER\_OCAML}
        \end{itemize}
        }
        \item<4-> Jumps into \funcname{caml\_raise\_exn}
        \item<5-> Calls \funcname{caml\_stash\_backtrace} again, to scan the next part of the stack: from the trap just pop-ed to the next trap
      \end{itemize}
      \vfill
      \mintocaml[firstline=15,lastline=21,highlightlines={18-21}]{../src/backtrace/backtrace.ml}
      \endminipage
    \end{column}
    \begin{column}{0.5\textwidth}
      \raggedleft
      \onslide*<1>{\listCamlReraiseExn{}}
      \onslide*<2>{\listCamlReraiseExn[highlightlines=729]{}}
      \onslide*<3>{
        \mintc[firstline=190,lastline=192,highlightlines={191-192}]{../ocaml/runtime/amd64.S}
        \mintc[firstline=234,lastline=237,highlightlines={235-237}]{../ocaml/runtime/amd64.S}
        \medskip
        \listCamlReraiseExn[highlightlines=731]{}
      }
      \onslide*<4-5>{
        % TODO refactor with \listCamlReraiseExn
        \mintasm[firstline=727,lastline=734,highlightlines=730]{../ocaml/runtime/amd64.S}
        \medskip
      }
      \onslide*<4>{\listCamlRaiseExn[highlightlines=711]{}}
      \onslide*<5>{\listCamlRaiseExn[highlightlines=720]{}}
    \end{column}
  \end{columns}
\end{frame}

\begin{frame}{Backtraces}{Backtrace collection continues}
  \begin{columns}[c]
    \begin{column}{0.55\textwidth}
      \minipage[c][0.75\textheight][s]{\columnwidth}
      \begin{itemize}
        \item<1-> \funcname{caml\_stash\_backtrace} scans the older part of the stack, up to the next trap
        \item<6-> Controls jumps to the nested exception handler in \funcname{maybe\_raise}, which matches only on \typename{ExnB}
        \item<7-> \funcname{caml\_reraise\_exn} is called again, backtrace collection resumes with \funcname{caml\_stash\_backtrace}
      \end{itemize}
      \medskip
      \begin{itemize}
        \item<2-> Constructed backtrace:
        \begin{itemize}
          \item <2-> \funcname{definitely\_raise}
          \item <2-> \funcname{probably\_raise}
          \item <3-> \funcname{probably\_raise} (re-raise)
          \item <4-> \funcname{maybe\_raise}
          \item <5-> \funcname{main}
          \item <8-> \funcname{main} (re-raise)
        \end{itemize}
      \end{itemize}
      \vfill
      \onslide*<1-5>{\mintocaml[firstline=15,lastline=21]{../src/backtrace/backtrace.ml}}
      \onslide*<6>{\mintocaml[firstline=28,lastline=34,highlightlines=31]{../src/backtrace/backtrace.ml}}
      \onslide*<7->{\mintocaml[firstline=28,lastline=34,highlightlines=33]{../src/backtrace/backtrace.ml}}
      \endminipage
    \end{column}
    \begin{column}{0.45\textwidth}
      \raggedleft
      \onslide*<1>{\providecommand\step{6}\input{diagrams/backtrace/backtrace.tex}}%
      \onslide*<2>{\providecommand\step{6}\input{diagrams/backtrace/backtrace.tex}}%
      \onslide*<3>{\providecommand\step{7}\input{diagrams/backtrace/backtrace.tex}}%
      \onslide*<4>{\providecommand\step{8}\input{diagrams/backtrace/backtrace.tex}}%
      \onslide*<5>{\providecommand\step{9}\input{diagrams/backtrace/backtrace.tex}}%
      \onslide*<6>{\providecommand\step{10}\input{diagrams/backtrace/backtrace.tex}}%
      \onslide*<7>{\providecommand\step{11}\input{diagrams/backtrace/backtrace.tex}}%
      \onslide*<8>{\providecommand\step{12}\input{diagrams/backtrace/backtrace.tex}}%
      \onslide*<9>{\providecommand\step{13}\input{diagrams/backtrace/backtrace.tex}}%
    \end{column}
  \end{columns}
\end{frame}

\begin{frame}{Backtraces}{Printing the backtrace}
  \begin{columns}[c]
    \begin{column}{0.45\textwidth}
      \minipage[c][0.66\textheight][s]{\columnwidth}
      \begin{itemize}
        \item<1-> The exception backtrace can now be printed to \funcarg{stdout} with \funcname{Printexc.print_backtrace}
        \item<2-> First the exception backtrace is retrieved into an OCaml array with \funcname{get_raw_backtrace }
        \item<3-> Which is implemented as the \funcname{caml_get_exception_raw_backtrace} C function in the runtime
        \item<4-> And finally printed with \funcname{print_exception_backtrace}
      \end{itemize}
      \vfill
      \mintocaml[firstline=28,lastline=34,highlightlines=33]{../src/backtrace/backtrace.ml}
      \endminipage
    \end{column}
    \begin{column}{0.55\textwidth}
      \onslide*<1,2>{
        \mintocaml[firstline=100,lastline=101]{../ocaml/stdlib/printexc.ml}
      }
      \only<1>{
        \listocaml[firstline=170,lastline=172]{../ocaml/stdlib/printexc.ml}{stdlib/printexc.ml:170}
      }
      \only<2>{
        \listocaml[firstline=170,lastline=172,highlightlines=172]{../ocaml/stdlib/printexc.ml}{stdlib/printexc.ml:170}
      }
      \only<3>{
        \mintc[firstline=154,lastline=156]{../ocaml/runtime/backtrace.c}
        \listc[firstline=171,lastline=192,highlightlines={184-187}]{../ocaml/runtime/backtrace.c}{runtime/backtrace.c:154}
      }
      \only<4>{
        \listocaml[firstline=155,lastline=165]{../ocaml/stdlib/printexc.ml}{stdlib/printexc.ml:55}
      }
    \end{column}
  \end{columns}
\end{frame}


\subsection{The multiple kinds of raise}
\frameSubsection{}{}

\begin{frame}{Backtraces}{When backtraces are not needed}
  \begin{columns}[c]
    \begin{column}{0.45\textwidth}
      \minipage[c][0.66\textheight][s]{\columnwidth}
      \begin{itemize}
        \item<1-> What if the backtrace for an exception is never needed?
        \item<2-> It's possible to raise the exception explicitly with \funcname{raise\_notrace} to make raising as cheap as possible
        \item<3-> An example of use in the \funcname{dynlink} library of the OCaml compiler
      \end{itemize}
      \vfill
      \onslide<3->{
      \listocaml[firstline=205,lastline=209,highlightlines=207]{../ocaml/otherlibs/dynlink/dynlink\_compilerlibs/misc.ml}{otherlibs/dynlink/dynlink\_compilerlibs/misc.ml:205}
      }
      \endminipage
    \end{column}
    \begin{column}{0.55\textwidth}
    \only<4>{
      \mintcmm[highlightlines=3]{../src/notrace/camlNotrace\_\_fun_347.cmm}
      \listcmm{../src/notrace/camlNotrace\_\_all\_somes\_327.cmm}{all\_somes}
    }
    \onslide*<5->{
      \listobjdump[highlightlines={6-9}]{../src/notrace/camlNotrace\_\_fun_347.objdump}{anonymous function from \funcname{all\_somes}}
    }
    \end{column}
  \end{columns}
\end{frame}

\begin{frame}{Backtraces}{Summary table of the multiple kinds of raise}
  \begin{itemize}
    \item When debugging information is enabled and if the backtrace of an exception isn't needed, it's possible to raise with \funcname{raise\_notrace} for performance
    \item When debugging information is \emph{not} enabled, the compiler optimises \funcname{raise} calls to inline assembly (same effect as using \funcname{raise\_notrace})
  \end{itemize}
  \bigskip
  \centering
  \begin{tabular}{| l | c | c | c | c |}
    \hline
    \parbox{10em}{Debug information \\ (\funcarg{-g} flag)}  & \multicolumn{2}{|c|}{Disabled}                                     & \multicolumn{2}{|c|}{Enabled} \\
    \hline
    Raise kind                                               & \funcname{raise} & \funcname{raise\_notrace}                       & \funcname{raise} & \funcname{raise\_notrace} \\
    \hline
    Compilation                                              & \parbox{8em}{inline assembly\\ (optimization)} & inline assembly   & \funcname{caml\_raise\_exn} & inline assembly \\
    \hline
  \end{tabular}
\end{frame}


%
% Takeaway #5
%
\subsection*{Takeaway \#5}
\frameSubsectionTakeaway{}
\againframe<5>{takeaway}
}%
      \onslide*<2>{\providecommand\step{1}\input{diagrams/backtrace/scanstack.tex}}%
      \onslide*<3>{\providecommand\step{2}\input{diagrams/backtrace/scanstack.tex}}%
      \onslide*<4>{\providecommand\step{3}\input{diagrams/backtrace/scanstack.tex}}%
      \onslide*<5>{\providecommand\step{4}\input{diagrams/backtrace/scanstack.tex}}%
      \onslide*<6>{\providecommand\step{5}\input{diagrams/backtrace/scanstack.tex}}%
    \end{column}
  \end{columns}
\end{frame}

% Duplicate of previous-previous slide
\begin{frame}{Backtraces}{Continuing on \funcname{caml\_stash\_backtrace}}
  \begin{columns}[c]
    \begin{column}{0.5\textwidth}
      \minipage[c][0.75\textheight][s]{\columnwidth}
      \begin{itemize}
          \color{gray}
        \item A C function provided by the runtime (specific to native code)
        \item It scans the stack, between the stack pointer at the raising point up to the next trap, in order to collect the names of the detected function frames
        \item It uses the same technique and information as the Garbage Collector (GC) uses to scan the values live on the stack
      \end{itemize}
      \begin{itemize}
        \item For each function frame detected, store a pointer to the \typename{frame\_descr} in the \localname{domain\_state->backtrace\_buffer} array, effectively constructing the backtrace
      \end{itemize}
      \onslide<2->{
        \begin{itemize}
            \smallskip
          \item Constructed backtrace:
            \funcname{definitely\_raise},
            \onslide<3->{\funcname{probably\_raise}}
        \end{itemize}
      }
      \vfill
      \mintocaml[firstline=9,lastline=13,highlightlines=12]{../src/backtrace/backtrace.ml}
      \endminipage
    \end{column}
    \begin{column}{0.5\textwidth}
      \raggedleft
      \onslide*<1>{\listStashBacktrace[highlightlines={114-115}]{}}
      \onslide*<2>{\providecommand\step{3}%
% Backtraces
%
\subsection{One advantage of raising with debugging information}
\frameSubsection{}{}

\begin{frame}{Backtraces}{Debugging information \& source code}
  \begin{columns}[c]
    \begin{column}{0.5\textwidth}
      \begin{itemize}
        \item Raising an exception is fun and games, but how do we know where the exception originated? $\rightarrow$ With the backtrace associated with the exception!
        \item In order to get a backtrace from the point where an exception was raised, it's necessary to compile with debugging information enabled (\funcarg{-g} flag for the compiler)
        \item Same example as in the `Nested handlers' section
      \end{itemize}
    \end{column}
    \begin{column}{0.5\textwidth}
      \listocaml[firstline=9,lastline=34]{../src/backtrace/backtrace.ml}{backtrace/backtrace.ml}
    \end{column}
  \end{columns}
\end{frame}

\begin{frame}{Backtraces}{CMM and disassembly of \funcname{definitely\_raise}}
  \begin{columns}[c]
    \begin{column}{0.5\textwidth}
      \minipage[c][0.66\textheight][s]{\columnwidth}
      \begin{itemize}
        \item<1-> An effect of compiling with debugging information (\funcarg{-g}): the CMM no longer uses \funcname{raise\_notrace} to raise the exception but \funcname{raise} instead
        \item<2-> Disassembly shows that CMM \funcname{raise} is actually a call to \funcname{caml\_raise\_exn}, a function in assembler provided by the runtime
      \end{itemize}
      \vfill
      \mintocaml[firstline=9,lastline=13,highlightlines=12]{../src/backtrace/backtrace.ml}
      \endminipage
    \end{column}
    \begin{column}{0.5\textwidth}
      \onslide*<1>{
        \mintcmm[highlightlines=5]{../src/backtrace/camlBacktrace\_\_definitely\_raise\_347.cmm}
      }
      \onslide*<2>{
        \mintobjdump[firstline=2,highlightlines=14]{../src/backtrace/camlBacktrace\_\_definitely\_raise\_347.objdump}
      }
    \end{column}
  \end{columns}
\end{frame}

\begin{frame}{Backtraces}{Introducing \funcname{caml\_raise\_exn}}
  \begin{columns}[c]
    \begin{column}{0.5\textwidth}
      \minipage[c][0.66\textheight][s]{\columnwidth}
      \onslide*<1>{
        \begin{itemize}
          \item Raising the exception is performed using \funcname{caml\_raise\_exn} which is
            \begin{itemize}
              \item An assembly function provided by the runtime
              \item Architecture specific
            \end{itemize}
          \item Let's see what it does differently from \funcname{raise\_notrace}\dots
          \item We are about to raise \typename{ExnC} with \funcname{caml\_raise\_exn} from \funcname{definitely\_raise}
        \end{itemize}
      }
      \onslide*<2>{
        \mintobjdump[firstline=2,highlightlines=14]{../src/backtrace/camlBacktrace\_\_definitely\_raise\_347.objdump}
      }
      \vfill
      \mintocaml[firstline=9,lastline=13,highlightlines=12]{../src/backtrace/backtrace.ml}
      \endminipage
    \end{column}
    \begin{column}{0.5\textwidth}
      \centering
      \providecommand\step{1}\input{diagrams/backtrace/backtrace.tex}
    \end{column}
  \end{columns}
\end{frame}

\begin{frame}{Backtraces}{But before that, we need more fields from \typename{caml\_domain\_state}}
  \begin{columns}
    \begin{column}{0.5\textwidth}
      \begin{itemize}
        \item Remember \typename{caml\_domain\_state} the per-domain runtime state?
        \item Some other members are of interest to us now\dots
        \item \localname{backtrace\_active} is used to know if the runtime should collect backtraces
        \item \localname{backtrace\_active} can be set by
          \begin{itemize}
            \item The \funcarg{b} flag in the \funcname{OCAMLRUNPARAM} environment variable
            \item \mintinline{ocaml}{Printexc.record_backtrace : bool -> unit} \footnotemark
          \end{itemize}
      \end{itemize}
    \end{column}
    \begin{column}{0.5\textwidth}
      \begin{listing}
        \mintc[highlightlines={9-11}]{src/ocaml/domain\_state\_backtrace.h}
        \caption{runtime/caml/domain\_state.h}
      \end{listing}
    \end{column}
  \end{columns}
  \footnotetext[1]{https://v2.ocaml.org/api/Printexc.html}
\end{frame}

\newcommand\listCamlRaiseExn[1][]{
  \listasm[firstline=702,lastline=725,#1]{../ocaml/runtime/amd64.S}{runtime/amd64.S:702}}

\begin{frame}{Backtraces}{Walk through \funcname{caml\_raise\_exn}}
  \begin{columns}[T]
    \begin{column}{0.5\textwidth}
      \begin{itemize}
        \item<1-> First checks whether exceptions backtrace should be recorded
        \item<1-> By checking the \funcarg{domain\_state->backtrace\_active} value
        \item<2-> If not, acts as \funcname{raise\_notrace} with \funcname{RESTORE\_EXN\_HANDLER\_OCAML}
          \begin{itemize}
            \item Set \regname{RSP} to \localname{domain\_state->exn\_handler} value
            \item Restore \localname{domain\_state->exn\_handler} from trap
            \item Jump with \funcname{ret} (same effect as using \funcname{pop} and \funcname{jmp})
          \end{itemize}
        \item<3-> Otherwise, switches to the C stack to be able to execute C code
        \item<4-> Calls \funcname{caml\_stash\_backtrace}, a C function provided by the runtime\ignorespaces
          \onslide<6->{, with
            \begin{itemize}
              \item \funcarg{exn}: exception value
              \item \funcarg{pc}: code address of the raising point
              \item \funcarg{sp}: stack address of the raising function
              \item \funcarg{trapsp}: stack address of the next handler
            \end{itemize}
          }
      \end{itemize}
      \bigskip
      \mintocaml[firstline=9,lastline=13,highlightlines=12]{../src/backtrace/backtrace.ml}
    \end{column}
    \begin{column}{0.5\textwidth}
      \raggedleft
      \onslide*<1,3-4>{
        \mintc[firstline=190,lastline=192]{../ocaml/runtime/amd64.S}
        \mintc[firstline=234,lastline=237]{../ocaml/runtime/amd64.S}
      }
      \onslide*<2>{
        \mintc[firstline=190,lastline=192,highlightlines={191-192}]{../ocaml/runtime/amd64.S}
        \mintc[firstline=234,lastline=237,highlightlines={235-237}]{../ocaml/runtime/amd64.S}
      }
      \onslide*<1>{\listCamlRaiseExn[highlightlines=705]{}}%
      \onslide*<2>{\listCamlRaiseExn[highlightlines={707-708}]{}}%
      \onslide*<3>{\listCamlRaiseExn[highlightlines={712-714}]{}}%
      \onslide*<4>{\listCamlRaiseExn[highlightlines={715-720}]{}}%
      \onslide*<5>{\providecommand\step{1}\input{diagrams/backtrace/backtrace.tex}}%
      \onslide*<6>{\providecommand\step{2}\input{diagrams/backtrace/backtrace.tex}}%
    \end{column}
  \end{columns}
\end{frame}

\newcommand\listStashBacktrace[1][]{
  \listc[firstline=91,lastline=120,#1]{../ocaml/runtime/backtrace\_nat.c}{runtime/backtrace\_nat.c:91}}

\begin{frame}{Backtraces}{Introducing \funcname{caml\_stash\_backtrace}}
  \begin{columns}[c]
    \begin{column}{0.5\textwidth}
      \minipage[c][0.66\textheight][s]{\columnwidth}
      \begin{itemize}
        \item<1-> A C function provided by the runtime (specific to native code)
        \item<2-> It scans the stack, between the stack pointer at the raising point up to the next trap, in order to collect the names of the detected function frames
          \onslide*<2>{
            \begin{itemize}
              \item And more information, like line number, column number...
            \end{itemize}
          }
        \item<3-> It uses the same technique and information as the Garbage Collector (GC) uses to scan the values live on the stack
          \onslide*<4>{
            \begin{itemize}
              \item \typename{frame\_descr} are at the core of stack unwinding
            \end{itemize}
          }
      \end{itemize}
      \vfill
      \mintocaml[firstline=9,lastline=13,highlightlines=12]{../src/backtrace/backtrace.ml}
      \endminipage
    \end{column}
    \begin{column}{0.5\textwidth}
      \onslide*<1>{\listStashBacktrace[highlightlines=91]{}}
      \onslide*<2>{\listStashBacktrace[highlightlines={91,117-118}]{}}
      \onslide*<3->{\listStashBacktrace[highlightlines={94,106,109-110}]{}}
    \end{column}
  \end{columns}
\end{frame}

\newcommand\listFrameDescr[1][]{
  \listocaml[firstline=111,lastline=124,#1]{../ocaml/asmcomp/emitaux.ml}{asmcomp/emitaux.ml:111}}
\newcommand\listDebugInfo[1][]{
  \listocaml[firstline=38,lastline=49,#1]{../ocaml/lambda/debuginfo.mli}{lambda/debuginfo.mli:38}}

\begin{frame}{Backtraces}{\typename{frame\_descr} and \typename{frame\_debuginfo} in a nutshell}
  \begin{columns}[c]
    \begin{column}{0.5\textwidth}
      \begin{itemize}
        \item<1-> For every return address in an OCaml program, there is a associated \typename{frame\_descr}
        \item<2-> A \typename{frame\_descr} specifies
        \begin{itemize}
          \item<2-> The size of the function stack frame
          \item<3-> Where live values are on the stack (so that the GC can mark them as live and prevent from being swept)
        \end{itemize}
        \item<4-> When compiled with debug information, \typename{frame\_descr} will be linked to a \typename{frame\_debuginfo}
        \begin{itemize}
          \item<5-> Contains information about the position in the source code (file name, line and column number)
        \end{itemize}
      \end{itemize}
    \end{column}
    \begin{column}{0.5\textwidth}
        \onslide*<1>{\listFrameDescr[highlightlines={118-122}]{}}
        \onslide*<2>{\listFrameDescr[highlightlines={120}]{}}
        \onslide*<3>{\listFrameDescr[highlightlines={121}]{}}
        \onslide*<4->{\listFrameDescr[highlightlines={122}]{}}
        \medskip
        \onslide*<1-3>{\listDebugInfo{}}
        \onslide*<4>{\listDebugInfo[highlightlines={38,49}]{}}
        \onslide*<5>{\listDebugInfo[highlightlines={39-46}]{}}
    \end{column}
  \end{columns}
\end{frame}

\begin{frame}{Backtraces}{Scanning the stack with \typename{frame\_descr}}
  \begin{columns}[c]
    \begin{column}{0.5\textwidth}
      \begin{itemize}
        \item Given the return address of the code that raises the exception by calling \funcname{caml\_raise\_exn} (\funcarg{pc} argument of \funcname{caml\_stash\_backtrace})
        \item And the position of the stack pointer (\funcarg{sp} argument of \funcname{caml\_stash\_backtrace})
        \item The frame size given by \typename{frame\_descr}.\funcarg{frame\_size}, allows to find the end of the previous frame
        \item And the return address of the caller of this function
        \bigskip
        \onslide<3->{
        \item Note that traps (here the reddish \funcname{ExnA}, \funcname{ExnB} and \funcname{ExnC} blocks) belong to the frame that installed them, and so the frame size changes when traps are removed
        }
      \end{itemize}
    \end{column}
    \begin{column}{0.5\textwidth}
      \raggedleft
      \onslide*<1>{\providecommand\step{2}\input{diagrams/backtrace/backtrace.tex}}%
      \onslide*<2>{\providecommand\step{1}\input{diagrams/backtrace/scanstack.tex}}%
      \onslide*<3>{\providecommand\step{2}\input{diagrams/backtrace/scanstack.tex}}%
      \onslide*<4>{\providecommand\step{3}\input{diagrams/backtrace/scanstack.tex}}%
      \onslide*<5>{\providecommand\step{4}\input{diagrams/backtrace/scanstack.tex}}%
      \onslide*<6>{\providecommand\step{5}\input{diagrams/backtrace/scanstack.tex}}%
    \end{column}
  \end{columns}
\end{frame}

% Duplicate of previous-previous slide
\begin{frame}{Backtraces}{Continuing on \funcname{caml\_stash\_backtrace}}
  \begin{columns}[c]
    \begin{column}{0.5\textwidth}
      \minipage[c][0.75\textheight][s]{\columnwidth}
      \begin{itemize}
          \color{gray}
        \item A C function provided by the runtime (specific to native code)
        \item It scans the stack, between the stack pointer at the raising point up to the next trap, in order to collect the names of the detected function frames
        \item It uses the same technique and information as the Garbage Collector (GC) uses to scan the values live on the stack
      \end{itemize}
      \begin{itemize}
        \item For each function frame detected, store a pointer to the \typename{frame\_descr} in the \localname{domain\_state->backtrace\_buffer} array, effectively constructing the backtrace
      \end{itemize}
      \onslide<2->{
        \begin{itemize}
            \smallskip
          \item Constructed backtrace:
            \funcname{definitely\_raise},
            \onslide<3->{\funcname{probably\_raise}}
        \end{itemize}
      }
      \vfill
      \mintocaml[firstline=9,lastline=13,highlightlines=12]{../src/backtrace/backtrace.ml}
      \endminipage
    \end{column}
    \begin{column}{0.5\textwidth}
      \raggedleft
      \onslide*<1>{\listStashBacktrace[highlightlines={114-115}]{}}
      \onslide*<2>{\providecommand\step{3}\input{diagrams/backtrace/backtrace.tex}}%
      \onslide*<3>{\providecommand\step{4}\input{diagrams/backtrace/backtrace.tex}}%
    \end{column}
  \end{columns}
\end{frame}

\begin{frame}{Backtraces}{Back to \funcname{caml\_raise\_exn}}
  \begin{columns}[c]
    \begin{column}{0.5\textwidth}
      \minipage[c][0.66\textheight][s]{\columnwidth}
      \begin{itemize}\color{gray}
        \item Checks wheteher exceptions backtrace should be recorded, by checking the \funcarg{domain\_state->backtrace\_active} value
        \item Switches to the C stack to be able to execute C code
        \item Calls \funcname{caml\_stash\_backtrace}, to collect the backtrace up to the next trap
      \end{itemize}
      \onslide*<2->{
        \begin{itemize}
          \item<2-> Executes the exception handler in \funcname{probably\_raise} with \funcname{RESTORE\_EXN\_HANDLER\_OCAML}
            \begin{itemize}
              \item Set \regname{RSP} to \localname{domain\_state->exn\_handler} value
              \item Restore \localname{domain\_state->exn\_handler} from trap
              \item Jump to the handler code with \funcname{ret}
            \end{itemize}
        \end{itemize}
      }
      \vfill
      \onslide*<1>{\mintocaml[firstline=9,lastline=13,highlightlines=12]{../src/backtrace/backtrace.ml}}
      \onslide*<2->{\mintocaml[firstline=15,lastline=21,highlightlines={19-21}]{../src/backtrace/backtrace.ml}}
      \endminipage
    \end{column}
    \begin{column}{0.5\textwidth}
      \centering
      \onslide*<1>{
        \mintc[firstline=190,lastline=192]{../ocaml/runtime/amd64.S}
        \mintc[firstline=234,lastline=237]{../ocaml/runtime/amd64.S}
      }
      \onslide*<2>{
        \mintc[firstline=190,lastline=192,highlightlines={191-192}]{../ocaml/runtime/amd64.S}
        \mintc[firstline=234,lastline=237,highlightlines={235-237}]{../ocaml/runtime/amd64.S}
      }
      \onslide*<1>{\listCamlRaiseExn[highlightlines=720]{}}
      \onslide*<2>{\listCamlRaiseExn[highlightlines=722]{}}
      \onslide*<3->{\providecommand\step{5}\input{diagrams/backtrace/backtrace.tex}}
    \end{column}
  \end{columns}
\end{frame}

\begin{frame}{Backtraces}{\funcname{caml\_probably\_raise}'s handler doesn't catch \typename{ExnC}}
  \begin{columns}[c]
    \begin{column}{0.45\textwidth}
      \minipage[c][0.75\textheight][s]{\columnwidth}
      \begin{itemize}
        \item<1-> \funcname{match \dots with}\footnotemark pattern matching can also match exceptions
        \item<1-> The CMM of \funcname{probably\_raise} shows that this \funcname{match \dots with} is implemented with a pattern matching and a \funcname{try \dots with}
        \item<1-> But this one only matches on \typename{ExnA} exception
        \item<2-> Since the pattern matching fails to catch the exception, \funcname{reraise} is called with our current \typename{ExnC} exception
        \item<3-> Disassembly shows that \funcname{reraise} is actually a call to \funcname{caml\_reraise\_exn}, which is also an assembly function provided by the runtime
      \end{itemize}
      \vfill
      \mintocaml[firstline=15,lastline=21,highlightlines={18-21}]{../src/backtrace/backtrace.ml}
      \endminipage
    \end{column}
    \begin{column}{0.55\textwidth}
      \centering
      \onslide*<1>{\mintcmm[highlightlines={10-18}]{../src/backtrace/camlBacktrace\_\_probably\_raise\_351.cmm}}
      \onslide*<2>{\listcmm[highlightlines=18]{../src/backtrace/camlBacktrace\_\_probably\_raise_351.cmm}{CMM of probably\_raise}}
      \onslide*<3>{\mintobjdump[firstline=5,lastline=35,highlightlines=31]{../src/backtrace/camlBacktrace\_\_probably\_raise_351.objdump}}
    \end{column}
  \end{columns}
  \only<1>{\footnotetext[1]{https://blog.janestreet.com/pattern-matching-and-exception-handling-unite/}}
\end{frame}

\newcommand\listCamlReraiseExn[1][]{
  \listasm[firstline=727,lastline=734,#1]{../ocaml/runtime/amd64.S}{runtime/amd64.S:727}}

\begin{frame}{Backtraces}{Introducing \funcname{caml\_reraise\_exn}}
  \begin{columns}[c]
    \begin{column}{0.5\textwidth}
      \minipage[c][0.66\textheight][s]{\columnwidth}
      \begin{itemize}
        \item<1-> The exception have to be reraised to the next handler
        \item<2-> First, checks if the backtrace should be collected with \funcarg{domain\_state->backtrace\_active}
        \only<3>{
        \begin{itemize}
            \item If not, behaves like a \funcname{raise\_notrace} with \funcname{RESTORE\_EXN\_HANDLER\_OCAML}
        \end{itemize}
        }
        \item<4-> Jumps into \funcname{caml\_raise\_exn}
        \item<5-> Calls \funcname{caml\_stash\_backtrace} again, to scan the next part of the stack: from the trap just pop-ed to the next trap
      \end{itemize}
      \vfill
      \mintocaml[firstline=15,lastline=21,highlightlines={18-21}]{../src/backtrace/backtrace.ml}
      \endminipage
    \end{column}
    \begin{column}{0.5\textwidth}
      \raggedleft
      \onslide*<1>{\listCamlReraiseExn{}}
      \onslide*<2>{\listCamlReraiseExn[highlightlines=729]{}}
      \onslide*<3>{
        \mintc[firstline=190,lastline=192,highlightlines={191-192}]{../ocaml/runtime/amd64.S}
        \mintc[firstline=234,lastline=237,highlightlines={235-237}]{../ocaml/runtime/amd64.S}
        \medskip
        \listCamlReraiseExn[highlightlines=731]{}
      }
      \onslide*<4-5>{
        % TODO refactor with \listCamlReraiseExn
        \mintasm[firstline=727,lastline=734,highlightlines=730]{../ocaml/runtime/amd64.S}
        \medskip
      }
      \onslide*<4>{\listCamlRaiseExn[highlightlines=711]{}}
      \onslide*<5>{\listCamlRaiseExn[highlightlines=720]{}}
    \end{column}
  \end{columns}
\end{frame}

\begin{frame}{Backtraces}{Backtrace collection continues}
  \begin{columns}[c]
    \begin{column}{0.55\textwidth}
      \minipage[c][0.75\textheight][s]{\columnwidth}
      \begin{itemize}
        \item<1-> \funcname{caml\_stash\_backtrace} scans the older part of the stack, up to the next trap
        \item<6-> Controls jumps to the nested exception handler in \funcname{maybe\_raise}, which matches only on \typename{ExnB}
        \item<7-> \funcname{caml\_reraise\_exn} is called again, backtrace collection resumes with \funcname{caml\_stash\_backtrace}
      \end{itemize}
      \medskip
      \begin{itemize}
        \item<2-> Constructed backtrace:
        \begin{itemize}
          \item <2-> \funcname{definitely\_raise}
          \item <2-> \funcname{probably\_raise}
          \item <3-> \funcname{probably\_raise} (re-raise)
          \item <4-> \funcname{maybe\_raise}
          \item <5-> \funcname{main}
          \item <8-> \funcname{main} (re-raise)
        \end{itemize}
      \end{itemize}
      \vfill
      \onslide*<1-5>{\mintocaml[firstline=15,lastline=21]{../src/backtrace/backtrace.ml}}
      \onslide*<6>{\mintocaml[firstline=28,lastline=34,highlightlines=31]{../src/backtrace/backtrace.ml}}
      \onslide*<7->{\mintocaml[firstline=28,lastline=34,highlightlines=33]{../src/backtrace/backtrace.ml}}
      \endminipage
    \end{column}
    \begin{column}{0.45\textwidth}
      \raggedleft
      \onslide*<1>{\providecommand\step{6}\input{diagrams/backtrace/backtrace.tex}}%
      \onslide*<2>{\providecommand\step{6}\input{diagrams/backtrace/backtrace.tex}}%
      \onslide*<3>{\providecommand\step{7}\input{diagrams/backtrace/backtrace.tex}}%
      \onslide*<4>{\providecommand\step{8}\input{diagrams/backtrace/backtrace.tex}}%
      \onslide*<5>{\providecommand\step{9}\input{diagrams/backtrace/backtrace.tex}}%
      \onslide*<6>{\providecommand\step{10}\input{diagrams/backtrace/backtrace.tex}}%
      \onslide*<7>{\providecommand\step{11}\input{diagrams/backtrace/backtrace.tex}}%
      \onslide*<8>{\providecommand\step{12}\input{diagrams/backtrace/backtrace.tex}}%
      \onslide*<9>{\providecommand\step{13}\input{diagrams/backtrace/backtrace.tex}}%
    \end{column}
  \end{columns}
\end{frame}

\begin{frame}{Backtraces}{Printing the backtrace}
  \begin{columns}[c]
    \begin{column}{0.45\textwidth}
      \minipage[c][0.66\textheight][s]{\columnwidth}
      \begin{itemize}
        \item<1-> The exception backtrace can now be printed to \funcarg{stdout} with \funcname{Printexc.print_backtrace}
        \item<2-> First the exception backtrace is retrieved into an OCaml array with \funcname{get_raw_backtrace }
        \item<3-> Which is implemented as the \funcname{caml_get_exception_raw_backtrace} C function in the runtime
        \item<4-> And finally printed with \funcname{print_exception_backtrace}
      \end{itemize}
      \vfill
      \mintocaml[firstline=28,lastline=34,highlightlines=33]{../src/backtrace/backtrace.ml}
      \endminipage
    \end{column}
    \begin{column}{0.55\textwidth}
      \onslide*<1,2>{
        \mintocaml[firstline=100,lastline=101]{../ocaml/stdlib/printexc.ml}
      }
      \only<1>{
        \listocaml[firstline=170,lastline=172]{../ocaml/stdlib/printexc.ml}{stdlib/printexc.ml:170}
      }
      \only<2>{
        \listocaml[firstline=170,lastline=172,highlightlines=172]{../ocaml/stdlib/printexc.ml}{stdlib/printexc.ml:170}
      }
      \only<3>{
        \mintc[firstline=154,lastline=156]{../ocaml/runtime/backtrace.c}
        \listc[firstline=171,lastline=192,highlightlines={184-187}]{../ocaml/runtime/backtrace.c}{runtime/backtrace.c:154}
      }
      \only<4>{
        \listocaml[firstline=155,lastline=165]{../ocaml/stdlib/printexc.ml}{stdlib/printexc.ml:55}
      }
    \end{column}
  \end{columns}
\end{frame}


\subsection{The multiple kinds of raise}
\frameSubsection{}{}

\begin{frame}{Backtraces}{When backtraces are not needed}
  \begin{columns}[c]
    \begin{column}{0.45\textwidth}
      \minipage[c][0.66\textheight][s]{\columnwidth}
      \begin{itemize}
        \item<1-> What if the backtrace for an exception is never needed?
        \item<2-> It's possible to raise the exception explicitly with \funcname{raise\_notrace} to make raising as cheap as possible
        \item<3-> An example of use in the \funcname{dynlink} library of the OCaml compiler
      \end{itemize}
      \vfill
      \onslide<3->{
      \listocaml[firstline=205,lastline=209,highlightlines=207]{../ocaml/otherlibs/dynlink/dynlink\_compilerlibs/misc.ml}{otherlibs/dynlink/dynlink\_compilerlibs/misc.ml:205}
      }
      \endminipage
    \end{column}
    \begin{column}{0.55\textwidth}
    \only<4>{
      \mintcmm[highlightlines=3]{../src/notrace/camlNotrace\_\_fun_347.cmm}
      \listcmm{../src/notrace/camlNotrace\_\_all\_somes\_327.cmm}{all\_somes}
    }
    \onslide*<5->{
      \listobjdump[highlightlines={6-9}]{../src/notrace/camlNotrace\_\_fun_347.objdump}{anonymous function from \funcname{all\_somes}}
    }
    \end{column}
  \end{columns}
\end{frame}

\begin{frame}{Backtraces}{Summary table of the multiple kinds of raise}
  \begin{itemize}
    \item When debugging information is enabled and if the backtrace of an exception isn't needed, it's possible to raise with \funcname{raise\_notrace} for performance
    \item When debugging information is \emph{not} enabled, the compiler optimises \funcname{raise} calls to inline assembly (same effect as using \funcname{raise\_notrace})
  \end{itemize}
  \bigskip
  \centering
  \begin{tabular}{| l | c | c | c | c |}
    \hline
    \parbox{10em}{Debug information \\ (\funcarg{-g} flag)}  & \multicolumn{2}{|c|}{Disabled}                                     & \multicolumn{2}{|c|}{Enabled} \\
    \hline
    Raise kind                                               & \funcname{raise} & \funcname{raise\_notrace}                       & \funcname{raise} & \funcname{raise\_notrace} \\
    \hline
    Compilation                                              & \parbox{8em}{inline assembly\\ (optimization)} & inline assembly   & \funcname{caml\_raise\_exn} & inline assembly \\
    \hline
  \end{tabular}
\end{frame}


%
% Takeaway #5
%
\subsection*{Takeaway \#5}
\frameSubsectionTakeaway{}
\againframe<5>{takeaway}
}%
      \onslide*<3>{\providecommand\step{4}%
% Backtraces
%
\subsection{One advantage of raising with debugging information}
\frameSubsection{}{}

\begin{frame}{Backtraces}{Debugging information \& source code}
  \begin{columns}[c]
    \begin{column}{0.5\textwidth}
      \begin{itemize}
        \item Raising an exception is fun and games, but how do we know where the exception originated? $\rightarrow$ With the backtrace associated with the exception!
        \item In order to get a backtrace from the point where an exception was raised, it's necessary to compile with debugging information enabled (\funcarg{-g} flag for the compiler)
        \item Same example as in the `Nested handlers' section
      \end{itemize}
    \end{column}
    \begin{column}{0.5\textwidth}
      \listocaml[firstline=9,lastline=34]{../src/backtrace/backtrace.ml}{backtrace/backtrace.ml}
    \end{column}
  \end{columns}
\end{frame}

\begin{frame}{Backtraces}{CMM and disassembly of \funcname{definitely\_raise}}
  \begin{columns}[c]
    \begin{column}{0.5\textwidth}
      \minipage[c][0.66\textheight][s]{\columnwidth}
      \begin{itemize}
        \item<1-> An effect of compiling with debugging information (\funcarg{-g}): the CMM no longer uses \funcname{raise\_notrace} to raise the exception but \funcname{raise} instead
        \item<2-> Disassembly shows that CMM \funcname{raise} is actually a call to \funcname{caml\_raise\_exn}, a function in assembler provided by the runtime
      \end{itemize}
      \vfill
      \mintocaml[firstline=9,lastline=13,highlightlines=12]{../src/backtrace/backtrace.ml}
      \endminipage
    \end{column}
    \begin{column}{0.5\textwidth}
      \onslide*<1>{
        \mintcmm[highlightlines=5]{../src/backtrace/camlBacktrace\_\_definitely\_raise\_347.cmm}
      }
      \onslide*<2>{
        \mintobjdump[firstline=2,highlightlines=14]{../src/backtrace/camlBacktrace\_\_definitely\_raise\_347.objdump}
      }
    \end{column}
  \end{columns}
\end{frame}

\begin{frame}{Backtraces}{Introducing \funcname{caml\_raise\_exn}}
  \begin{columns}[c]
    \begin{column}{0.5\textwidth}
      \minipage[c][0.66\textheight][s]{\columnwidth}
      \onslide*<1>{
        \begin{itemize}
          \item Raising the exception is performed using \funcname{caml\_raise\_exn} which is
            \begin{itemize}
              \item An assembly function provided by the runtime
              \item Architecture specific
            \end{itemize}
          \item Let's see what it does differently from \funcname{raise\_notrace}\dots
          \item We are about to raise \typename{ExnC} with \funcname{caml\_raise\_exn} from \funcname{definitely\_raise}
        \end{itemize}
      }
      \onslide*<2>{
        \mintobjdump[firstline=2,highlightlines=14]{../src/backtrace/camlBacktrace\_\_definitely\_raise\_347.objdump}
      }
      \vfill
      \mintocaml[firstline=9,lastline=13,highlightlines=12]{../src/backtrace/backtrace.ml}
      \endminipage
    \end{column}
    \begin{column}{0.5\textwidth}
      \centering
      \providecommand\step{1}\input{diagrams/backtrace/backtrace.tex}
    \end{column}
  \end{columns}
\end{frame}

\begin{frame}{Backtraces}{But before that, we need more fields from \typename{caml\_domain\_state}}
  \begin{columns}
    \begin{column}{0.5\textwidth}
      \begin{itemize}
        \item Remember \typename{caml\_domain\_state} the per-domain runtime state?
        \item Some other members are of interest to us now\dots
        \item \localname{backtrace\_active} is used to know if the runtime should collect backtraces
        \item \localname{backtrace\_active} can be set by
          \begin{itemize}
            \item The \funcarg{b} flag in the \funcname{OCAMLRUNPARAM} environment variable
            \item \mintinline{ocaml}{Printexc.record_backtrace : bool -> unit} \footnotemark
          \end{itemize}
      \end{itemize}
    \end{column}
    \begin{column}{0.5\textwidth}
      \begin{listing}
        \mintc[highlightlines={9-11}]{src/ocaml/domain\_state\_backtrace.h}
        \caption{runtime/caml/domain\_state.h}
      \end{listing}
    \end{column}
  \end{columns}
  \footnotetext[1]{https://v2.ocaml.org/api/Printexc.html}
\end{frame}

\newcommand\listCamlRaiseExn[1][]{
  \listasm[firstline=702,lastline=725,#1]{../ocaml/runtime/amd64.S}{runtime/amd64.S:702}}

\begin{frame}{Backtraces}{Walk through \funcname{caml\_raise\_exn}}
  \begin{columns}[T]
    \begin{column}{0.5\textwidth}
      \begin{itemize}
        \item<1-> First checks whether exceptions backtrace should be recorded
        \item<1-> By checking the \funcarg{domain\_state->backtrace\_active} value
        \item<2-> If not, acts as \funcname{raise\_notrace} with \funcname{RESTORE\_EXN\_HANDLER\_OCAML}
          \begin{itemize}
            \item Set \regname{RSP} to \localname{domain\_state->exn\_handler} value
            \item Restore \localname{domain\_state->exn\_handler} from trap
            \item Jump with \funcname{ret} (same effect as using \funcname{pop} and \funcname{jmp})
          \end{itemize}
        \item<3-> Otherwise, switches to the C stack to be able to execute C code
        \item<4-> Calls \funcname{caml\_stash\_backtrace}, a C function provided by the runtime\ignorespaces
          \onslide<6->{, with
            \begin{itemize}
              \item \funcarg{exn}: exception value
              \item \funcarg{pc}: code address of the raising point
              \item \funcarg{sp}: stack address of the raising function
              \item \funcarg{trapsp}: stack address of the next handler
            \end{itemize}
          }
      \end{itemize}
      \bigskip
      \mintocaml[firstline=9,lastline=13,highlightlines=12]{../src/backtrace/backtrace.ml}
    \end{column}
    \begin{column}{0.5\textwidth}
      \raggedleft
      \onslide*<1,3-4>{
        \mintc[firstline=190,lastline=192]{../ocaml/runtime/amd64.S}
        \mintc[firstline=234,lastline=237]{../ocaml/runtime/amd64.S}
      }
      \onslide*<2>{
        \mintc[firstline=190,lastline=192,highlightlines={191-192}]{../ocaml/runtime/amd64.S}
        \mintc[firstline=234,lastline=237,highlightlines={235-237}]{../ocaml/runtime/amd64.S}
      }
      \onslide*<1>{\listCamlRaiseExn[highlightlines=705]{}}%
      \onslide*<2>{\listCamlRaiseExn[highlightlines={707-708}]{}}%
      \onslide*<3>{\listCamlRaiseExn[highlightlines={712-714}]{}}%
      \onslide*<4>{\listCamlRaiseExn[highlightlines={715-720}]{}}%
      \onslide*<5>{\providecommand\step{1}\input{diagrams/backtrace/backtrace.tex}}%
      \onslide*<6>{\providecommand\step{2}\input{diagrams/backtrace/backtrace.tex}}%
    \end{column}
  \end{columns}
\end{frame}

\newcommand\listStashBacktrace[1][]{
  \listc[firstline=91,lastline=120,#1]{../ocaml/runtime/backtrace\_nat.c}{runtime/backtrace\_nat.c:91}}

\begin{frame}{Backtraces}{Introducing \funcname{caml\_stash\_backtrace}}
  \begin{columns}[c]
    \begin{column}{0.5\textwidth}
      \minipage[c][0.66\textheight][s]{\columnwidth}
      \begin{itemize}
        \item<1-> A C function provided by the runtime (specific to native code)
        \item<2-> It scans the stack, between the stack pointer at the raising point up to the next trap, in order to collect the names of the detected function frames
          \onslide*<2>{
            \begin{itemize}
              \item And more information, like line number, column number...
            \end{itemize}
          }
        \item<3-> It uses the same technique and information as the Garbage Collector (GC) uses to scan the values live on the stack
          \onslide*<4>{
            \begin{itemize}
              \item \typename{frame\_descr} are at the core of stack unwinding
            \end{itemize}
          }
      \end{itemize}
      \vfill
      \mintocaml[firstline=9,lastline=13,highlightlines=12]{../src/backtrace/backtrace.ml}
      \endminipage
    \end{column}
    \begin{column}{0.5\textwidth}
      \onslide*<1>{\listStashBacktrace[highlightlines=91]{}}
      \onslide*<2>{\listStashBacktrace[highlightlines={91,117-118}]{}}
      \onslide*<3->{\listStashBacktrace[highlightlines={94,106,109-110}]{}}
    \end{column}
  \end{columns}
\end{frame}

\newcommand\listFrameDescr[1][]{
  \listocaml[firstline=111,lastline=124,#1]{../ocaml/asmcomp/emitaux.ml}{asmcomp/emitaux.ml:111}}
\newcommand\listDebugInfo[1][]{
  \listocaml[firstline=38,lastline=49,#1]{../ocaml/lambda/debuginfo.mli}{lambda/debuginfo.mli:38}}

\begin{frame}{Backtraces}{\typename{frame\_descr} and \typename{frame\_debuginfo} in a nutshell}
  \begin{columns}[c]
    \begin{column}{0.5\textwidth}
      \begin{itemize}
        \item<1-> For every return address in an OCaml program, there is a associated \typename{frame\_descr}
        \item<2-> A \typename{frame\_descr} specifies
        \begin{itemize}
          \item<2-> The size of the function stack frame
          \item<3-> Where live values are on the stack (so that the GC can mark them as live and prevent from being swept)
        \end{itemize}
        \item<4-> When compiled with debug information, \typename{frame\_descr} will be linked to a \typename{frame\_debuginfo}
        \begin{itemize}
          \item<5-> Contains information about the position in the source code (file name, line and column number)
        \end{itemize}
      \end{itemize}
    \end{column}
    \begin{column}{0.5\textwidth}
        \onslide*<1>{\listFrameDescr[highlightlines={118-122}]{}}
        \onslide*<2>{\listFrameDescr[highlightlines={120}]{}}
        \onslide*<3>{\listFrameDescr[highlightlines={121}]{}}
        \onslide*<4->{\listFrameDescr[highlightlines={122}]{}}
        \medskip
        \onslide*<1-3>{\listDebugInfo{}}
        \onslide*<4>{\listDebugInfo[highlightlines={38,49}]{}}
        \onslide*<5>{\listDebugInfo[highlightlines={39-46}]{}}
    \end{column}
  \end{columns}
\end{frame}

\begin{frame}{Backtraces}{Scanning the stack with \typename{frame\_descr}}
  \begin{columns}[c]
    \begin{column}{0.5\textwidth}
      \begin{itemize}
        \item Given the return address of the code that raises the exception by calling \funcname{caml\_raise\_exn} (\funcarg{pc} argument of \funcname{caml\_stash\_backtrace})
        \item And the position of the stack pointer (\funcarg{sp} argument of \funcname{caml\_stash\_backtrace})
        \item The frame size given by \typename{frame\_descr}.\funcarg{frame\_size}, allows to find the end of the previous frame
        \item And the return address of the caller of this function
        \bigskip
        \onslide<3->{
        \item Note that traps (here the reddish \funcname{ExnA}, \funcname{ExnB} and \funcname{ExnC} blocks) belong to the frame that installed them, and so the frame size changes when traps are removed
        }
      \end{itemize}
    \end{column}
    \begin{column}{0.5\textwidth}
      \raggedleft
      \onslide*<1>{\providecommand\step{2}\input{diagrams/backtrace/backtrace.tex}}%
      \onslide*<2>{\providecommand\step{1}\input{diagrams/backtrace/scanstack.tex}}%
      \onslide*<3>{\providecommand\step{2}\input{diagrams/backtrace/scanstack.tex}}%
      \onslide*<4>{\providecommand\step{3}\input{diagrams/backtrace/scanstack.tex}}%
      \onslide*<5>{\providecommand\step{4}\input{diagrams/backtrace/scanstack.tex}}%
      \onslide*<6>{\providecommand\step{5}\input{diagrams/backtrace/scanstack.tex}}%
    \end{column}
  \end{columns}
\end{frame}

% Duplicate of previous-previous slide
\begin{frame}{Backtraces}{Continuing on \funcname{caml\_stash\_backtrace}}
  \begin{columns}[c]
    \begin{column}{0.5\textwidth}
      \minipage[c][0.75\textheight][s]{\columnwidth}
      \begin{itemize}
          \color{gray}
        \item A C function provided by the runtime (specific to native code)
        \item It scans the stack, between the stack pointer at the raising point up to the next trap, in order to collect the names of the detected function frames
        \item It uses the same technique and information as the Garbage Collector (GC) uses to scan the values live on the stack
      \end{itemize}
      \begin{itemize}
        \item For each function frame detected, store a pointer to the \typename{frame\_descr} in the \localname{domain\_state->backtrace\_buffer} array, effectively constructing the backtrace
      \end{itemize}
      \onslide<2->{
        \begin{itemize}
            \smallskip
          \item Constructed backtrace:
            \funcname{definitely\_raise},
            \onslide<3->{\funcname{probably\_raise}}
        \end{itemize}
      }
      \vfill
      \mintocaml[firstline=9,lastline=13,highlightlines=12]{../src/backtrace/backtrace.ml}
      \endminipage
    \end{column}
    \begin{column}{0.5\textwidth}
      \raggedleft
      \onslide*<1>{\listStashBacktrace[highlightlines={114-115}]{}}
      \onslide*<2>{\providecommand\step{3}\input{diagrams/backtrace/backtrace.tex}}%
      \onslide*<3>{\providecommand\step{4}\input{diagrams/backtrace/backtrace.tex}}%
    \end{column}
  \end{columns}
\end{frame}

\begin{frame}{Backtraces}{Back to \funcname{caml\_raise\_exn}}
  \begin{columns}[c]
    \begin{column}{0.5\textwidth}
      \minipage[c][0.66\textheight][s]{\columnwidth}
      \begin{itemize}\color{gray}
        \item Checks wheteher exceptions backtrace should be recorded, by checking the \funcarg{domain\_state->backtrace\_active} value
        \item Switches to the C stack to be able to execute C code
        \item Calls \funcname{caml\_stash\_backtrace}, to collect the backtrace up to the next trap
      \end{itemize}
      \onslide*<2->{
        \begin{itemize}
          \item<2-> Executes the exception handler in \funcname{probably\_raise} with \funcname{RESTORE\_EXN\_HANDLER\_OCAML}
            \begin{itemize}
              \item Set \regname{RSP} to \localname{domain\_state->exn\_handler} value
              \item Restore \localname{domain\_state->exn\_handler} from trap
              \item Jump to the handler code with \funcname{ret}
            \end{itemize}
        \end{itemize}
      }
      \vfill
      \onslide*<1>{\mintocaml[firstline=9,lastline=13,highlightlines=12]{../src/backtrace/backtrace.ml}}
      \onslide*<2->{\mintocaml[firstline=15,lastline=21,highlightlines={19-21}]{../src/backtrace/backtrace.ml}}
      \endminipage
    \end{column}
    \begin{column}{0.5\textwidth}
      \centering
      \onslide*<1>{
        \mintc[firstline=190,lastline=192]{../ocaml/runtime/amd64.S}
        \mintc[firstline=234,lastline=237]{../ocaml/runtime/amd64.S}
      }
      \onslide*<2>{
        \mintc[firstline=190,lastline=192,highlightlines={191-192}]{../ocaml/runtime/amd64.S}
        \mintc[firstline=234,lastline=237,highlightlines={235-237}]{../ocaml/runtime/amd64.S}
      }
      \onslide*<1>{\listCamlRaiseExn[highlightlines=720]{}}
      \onslide*<2>{\listCamlRaiseExn[highlightlines=722]{}}
      \onslide*<3->{\providecommand\step{5}\input{diagrams/backtrace/backtrace.tex}}
    \end{column}
  \end{columns}
\end{frame}

\begin{frame}{Backtraces}{\funcname{caml\_probably\_raise}'s handler doesn't catch \typename{ExnC}}
  \begin{columns}[c]
    \begin{column}{0.45\textwidth}
      \minipage[c][0.75\textheight][s]{\columnwidth}
      \begin{itemize}
        \item<1-> \funcname{match \dots with}\footnotemark pattern matching can also match exceptions
        \item<1-> The CMM of \funcname{probably\_raise} shows that this \funcname{match \dots with} is implemented with a pattern matching and a \funcname{try \dots with}
        \item<1-> But this one only matches on \typename{ExnA} exception
        \item<2-> Since the pattern matching fails to catch the exception, \funcname{reraise} is called with our current \typename{ExnC} exception
        \item<3-> Disassembly shows that \funcname{reraise} is actually a call to \funcname{caml\_reraise\_exn}, which is also an assembly function provided by the runtime
      \end{itemize}
      \vfill
      \mintocaml[firstline=15,lastline=21,highlightlines={18-21}]{../src/backtrace/backtrace.ml}
      \endminipage
    \end{column}
    \begin{column}{0.55\textwidth}
      \centering
      \onslide*<1>{\mintcmm[highlightlines={10-18}]{../src/backtrace/camlBacktrace\_\_probably\_raise\_351.cmm}}
      \onslide*<2>{\listcmm[highlightlines=18]{../src/backtrace/camlBacktrace\_\_probably\_raise_351.cmm}{CMM of probably\_raise}}
      \onslide*<3>{\mintobjdump[firstline=5,lastline=35,highlightlines=31]{../src/backtrace/camlBacktrace\_\_probably\_raise_351.objdump}}
    \end{column}
  \end{columns}
  \only<1>{\footnotetext[1]{https://blog.janestreet.com/pattern-matching-and-exception-handling-unite/}}
\end{frame}

\newcommand\listCamlReraiseExn[1][]{
  \listasm[firstline=727,lastline=734,#1]{../ocaml/runtime/amd64.S}{runtime/amd64.S:727}}

\begin{frame}{Backtraces}{Introducing \funcname{caml\_reraise\_exn}}
  \begin{columns}[c]
    \begin{column}{0.5\textwidth}
      \minipage[c][0.66\textheight][s]{\columnwidth}
      \begin{itemize}
        \item<1-> The exception have to be reraised to the next handler
        \item<2-> First, checks if the backtrace should be collected with \funcarg{domain\_state->backtrace\_active}
        \only<3>{
        \begin{itemize}
            \item If not, behaves like a \funcname{raise\_notrace} with \funcname{RESTORE\_EXN\_HANDLER\_OCAML}
        \end{itemize}
        }
        \item<4-> Jumps into \funcname{caml\_raise\_exn}
        \item<5-> Calls \funcname{caml\_stash\_backtrace} again, to scan the next part of the stack: from the trap just pop-ed to the next trap
      \end{itemize}
      \vfill
      \mintocaml[firstline=15,lastline=21,highlightlines={18-21}]{../src/backtrace/backtrace.ml}
      \endminipage
    \end{column}
    \begin{column}{0.5\textwidth}
      \raggedleft
      \onslide*<1>{\listCamlReraiseExn{}}
      \onslide*<2>{\listCamlReraiseExn[highlightlines=729]{}}
      \onslide*<3>{
        \mintc[firstline=190,lastline=192,highlightlines={191-192}]{../ocaml/runtime/amd64.S}
        \mintc[firstline=234,lastline=237,highlightlines={235-237}]{../ocaml/runtime/amd64.S}
        \medskip
        \listCamlReraiseExn[highlightlines=731]{}
      }
      \onslide*<4-5>{
        % TODO refactor with \listCamlReraiseExn
        \mintasm[firstline=727,lastline=734,highlightlines=730]{../ocaml/runtime/amd64.S}
        \medskip
      }
      \onslide*<4>{\listCamlRaiseExn[highlightlines=711]{}}
      \onslide*<5>{\listCamlRaiseExn[highlightlines=720]{}}
    \end{column}
  \end{columns}
\end{frame}

\begin{frame}{Backtraces}{Backtrace collection continues}
  \begin{columns}[c]
    \begin{column}{0.55\textwidth}
      \minipage[c][0.75\textheight][s]{\columnwidth}
      \begin{itemize}
        \item<1-> \funcname{caml\_stash\_backtrace} scans the older part of the stack, up to the next trap
        \item<6-> Controls jumps to the nested exception handler in \funcname{maybe\_raise}, which matches only on \typename{ExnB}
        \item<7-> \funcname{caml\_reraise\_exn} is called again, backtrace collection resumes with \funcname{caml\_stash\_backtrace}
      \end{itemize}
      \medskip
      \begin{itemize}
        \item<2-> Constructed backtrace:
        \begin{itemize}
          \item <2-> \funcname{definitely\_raise}
          \item <2-> \funcname{probably\_raise}
          \item <3-> \funcname{probably\_raise} (re-raise)
          \item <4-> \funcname{maybe\_raise}
          \item <5-> \funcname{main}
          \item <8-> \funcname{main} (re-raise)
        \end{itemize}
      \end{itemize}
      \vfill
      \onslide*<1-5>{\mintocaml[firstline=15,lastline=21]{../src/backtrace/backtrace.ml}}
      \onslide*<6>{\mintocaml[firstline=28,lastline=34,highlightlines=31]{../src/backtrace/backtrace.ml}}
      \onslide*<7->{\mintocaml[firstline=28,lastline=34,highlightlines=33]{../src/backtrace/backtrace.ml}}
      \endminipage
    \end{column}
    \begin{column}{0.45\textwidth}
      \raggedleft
      \onslide*<1>{\providecommand\step{6}\input{diagrams/backtrace/backtrace.tex}}%
      \onslide*<2>{\providecommand\step{6}\input{diagrams/backtrace/backtrace.tex}}%
      \onslide*<3>{\providecommand\step{7}\input{diagrams/backtrace/backtrace.tex}}%
      \onslide*<4>{\providecommand\step{8}\input{diagrams/backtrace/backtrace.tex}}%
      \onslide*<5>{\providecommand\step{9}\input{diagrams/backtrace/backtrace.tex}}%
      \onslide*<6>{\providecommand\step{10}\input{diagrams/backtrace/backtrace.tex}}%
      \onslide*<7>{\providecommand\step{11}\input{diagrams/backtrace/backtrace.tex}}%
      \onslide*<8>{\providecommand\step{12}\input{diagrams/backtrace/backtrace.tex}}%
      \onslide*<9>{\providecommand\step{13}\input{diagrams/backtrace/backtrace.tex}}%
    \end{column}
  \end{columns}
\end{frame}

\begin{frame}{Backtraces}{Printing the backtrace}
  \begin{columns}[c]
    \begin{column}{0.45\textwidth}
      \minipage[c][0.66\textheight][s]{\columnwidth}
      \begin{itemize}
        \item<1-> The exception backtrace can now be printed to \funcarg{stdout} with \funcname{Printexc.print_backtrace}
        \item<2-> First the exception backtrace is retrieved into an OCaml array with \funcname{get_raw_backtrace }
        \item<3-> Which is implemented as the \funcname{caml_get_exception_raw_backtrace} C function in the runtime
        \item<4-> And finally printed with \funcname{print_exception_backtrace}
      \end{itemize}
      \vfill
      \mintocaml[firstline=28,lastline=34,highlightlines=33]{../src/backtrace/backtrace.ml}
      \endminipage
    \end{column}
    \begin{column}{0.55\textwidth}
      \onslide*<1,2>{
        \mintocaml[firstline=100,lastline=101]{../ocaml/stdlib/printexc.ml}
      }
      \only<1>{
        \listocaml[firstline=170,lastline=172]{../ocaml/stdlib/printexc.ml}{stdlib/printexc.ml:170}
      }
      \only<2>{
        \listocaml[firstline=170,lastline=172,highlightlines=172]{../ocaml/stdlib/printexc.ml}{stdlib/printexc.ml:170}
      }
      \only<3>{
        \mintc[firstline=154,lastline=156]{../ocaml/runtime/backtrace.c}
        \listc[firstline=171,lastline=192,highlightlines={184-187}]{../ocaml/runtime/backtrace.c}{runtime/backtrace.c:154}
      }
      \only<4>{
        \listocaml[firstline=155,lastline=165]{../ocaml/stdlib/printexc.ml}{stdlib/printexc.ml:55}
      }
    \end{column}
  \end{columns}
\end{frame}


\subsection{The multiple kinds of raise}
\frameSubsection{}{}

\begin{frame}{Backtraces}{When backtraces are not needed}
  \begin{columns}[c]
    \begin{column}{0.45\textwidth}
      \minipage[c][0.66\textheight][s]{\columnwidth}
      \begin{itemize}
        \item<1-> What if the backtrace for an exception is never needed?
        \item<2-> It's possible to raise the exception explicitly with \funcname{raise\_notrace} to make raising as cheap as possible
        \item<3-> An example of use in the \funcname{dynlink} library of the OCaml compiler
      \end{itemize}
      \vfill
      \onslide<3->{
      \listocaml[firstline=205,lastline=209,highlightlines=207]{../ocaml/otherlibs/dynlink/dynlink\_compilerlibs/misc.ml}{otherlibs/dynlink/dynlink\_compilerlibs/misc.ml:205}
      }
      \endminipage
    \end{column}
    \begin{column}{0.55\textwidth}
    \only<4>{
      \mintcmm[highlightlines=3]{../src/notrace/camlNotrace\_\_fun_347.cmm}
      \listcmm{../src/notrace/camlNotrace\_\_all\_somes\_327.cmm}{all\_somes}
    }
    \onslide*<5->{
      \listobjdump[highlightlines={6-9}]{../src/notrace/camlNotrace\_\_fun_347.objdump}{anonymous function from \funcname{all\_somes}}
    }
    \end{column}
  \end{columns}
\end{frame}

\begin{frame}{Backtraces}{Summary table of the multiple kinds of raise}
  \begin{itemize}
    \item When debugging information is enabled and if the backtrace of an exception isn't needed, it's possible to raise with \funcname{raise\_notrace} for performance
    \item When debugging information is \emph{not} enabled, the compiler optimises \funcname{raise} calls to inline assembly (same effect as using \funcname{raise\_notrace})
  \end{itemize}
  \bigskip
  \centering
  \begin{tabular}{| l | c | c | c | c |}
    \hline
    \parbox{10em}{Debug information \\ (\funcarg{-g} flag)}  & \multicolumn{2}{|c|}{Disabled}                                     & \multicolumn{2}{|c|}{Enabled} \\
    \hline
    Raise kind                                               & \funcname{raise} & \funcname{raise\_notrace}                       & \funcname{raise} & \funcname{raise\_notrace} \\
    \hline
    Compilation                                              & \parbox{8em}{inline assembly\\ (optimization)} & inline assembly   & \funcname{caml\_raise\_exn} & inline assembly \\
    \hline
  \end{tabular}
\end{frame}


%
% Takeaway #5
%
\subsection*{Takeaway \#5}
\frameSubsectionTakeaway{}
\againframe<5>{takeaway}
}%
    \end{column}
  \end{columns}
\end{frame}

\begin{frame}{Backtraces}{Back to \funcname{caml\_raise\_exn}}
  \begin{columns}[c]
    \begin{column}{0.5\textwidth}
      \minipage[c][0.66\textheight][s]{\columnwidth}
      \begin{itemize}\color{gray}
        \item Checks wheteher exceptions backtrace should be recorded, by checking the \funcarg{domain\_state->backtrace\_active} value
        \item Switches to the C stack to be able to execute C code
        \item Calls \funcname{caml\_stash\_backtrace}, to collect the backtrace up to the next trap
      \end{itemize}
      \onslide*<2->{
        \begin{itemize}
          \item<2-> Executes the exception handler in \funcname{probably\_raise} with \funcname{RESTORE\_EXN\_HANDLER\_OCAML}
            \begin{itemize}
              \item Set \regname{RSP} to \localname{domain\_state->exn\_handler} value
              \item Restore \localname{domain\_state->exn\_handler} from trap
              \item Jump to the handler code with \funcname{ret}
            \end{itemize}
        \end{itemize}
      }
      \vfill
      \onslide*<1>{\mintocaml[firstline=9,lastline=13,highlightlines=12]{../src/backtrace/backtrace.ml}}
      \onslide*<2->{\mintocaml[firstline=15,lastline=21,highlightlines={19-21}]{../src/backtrace/backtrace.ml}}
      \endminipage
    \end{column}
    \begin{column}{0.5\textwidth}
      \centering
      \onslide*<1>{
        \mintc[firstline=190,lastline=192]{../ocaml/runtime/amd64.S}
        \mintc[firstline=234,lastline=237]{../ocaml/runtime/amd64.S}
      }
      \onslide*<2>{
        \mintc[firstline=190,lastline=192,highlightlines={191-192}]{../ocaml/runtime/amd64.S}
        \mintc[firstline=234,lastline=237,highlightlines={235-237}]{../ocaml/runtime/amd64.S}
      }
      \onslide*<1>{\listCamlRaiseExn[highlightlines=720]{}}
      \onslide*<2>{\listCamlRaiseExn[highlightlines=722]{}}
      \onslide*<3->{\providecommand\step{5}%
% Backtraces
%
\subsection{One advantage of raising with debugging information}
\frameSubsection{}{}

\begin{frame}{Backtraces}{Debugging information \& source code}
  \begin{columns}[c]
    \begin{column}{0.5\textwidth}
      \begin{itemize}
        \item Raising an exception is fun and games, but how do we know where the exception originated? $\rightarrow$ With the backtrace associated with the exception!
        \item In order to get a backtrace from the point where an exception was raised, it's necessary to compile with debugging information enabled (\funcarg{-g} flag for the compiler)
        \item Same example as in the `Nested handlers' section
      \end{itemize}
    \end{column}
    \begin{column}{0.5\textwidth}
      \listocaml[firstline=9,lastline=34]{../src/backtrace/backtrace.ml}{backtrace/backtrace.ml}
    \end{column}
  \end{columns}
\end{frame}

\begin{frame}{Backtraces}{CMM and disassembly of \funcname{definitely\_raise}}
  \begin{columns}[c]
    \begin{column}{0.5\textwidth}
      \minipage[c][0.66\textheight][s]{\columnwidth}
      \begin{itemize}
        \item<1-> An effect of compiling with debugging information (\funcarg{-g}): the CMM no longer uses \funcname{raise\_notrace} to raise the exception but \funcname{raise} instead
        \item<2-> Disassembly shows that CMM \funcname{raise} is actually a call to \funcname{caml\_raise\_exn}, a function in assembler provided by the runtime
      \end{itemize}
      \vfill
      \mintocaml[firstline=9,lastline=13,highlightlines=12]{../src/backtrace/backtrace.ml}
      \endminipage
    \end{column}
    \begin{column}{0.5\textwidth}
      \onslide*<1>{
        \mintcmm[highlightlines=5]{../src/backtrace/camlBacktrace\_\_definitely\_raise\_347.cmm}
      }
      \onslide*<2>{
        \mintobjdump[firstline=2,highlightlines=14]{../src/backtrace/camlBacktrace\_\_definitely\_raise\_347.objdump}
      }
    \end{column}
  \end{columns}
\end{frame}

\begin{frame}{Backtraces}{Introducing \funcname{caml\_raise\_exn}}
  \begin{columns}[c]
    \begin{column}{0.5\textwidth}
      \minipage[c][0.66\textheight][s]{\columnwidth}
      \onslide*<1>{
        \begin{itemize}
          \item Raising the exception is performed using \funcname{caml\_raise\_exn} which is
            \begin{itemize}
              \item An assembly function provided by the runtime
              \item Architecture specific
            \end{itemize}
          \item Let's see what it does differently from \funcname{raise\_notrace}\dots
          \item We are about to raise \typename{ExnC} with \funcname{caml\_raise\_exn} from \funcname{definitely\_raise}
        \end{itemize}
      }
      \onslide*<2>{
        \mintobjdump[firstline=2,highlightlines=14]{../src/backtrace/camlBacktrace\_\_definitely\_raise\_347.objdump}
      }
      \vfill
      \mintocaml[firstline=9,lastline=13,highlightlines=12]{../src/backtrace/backtrace.ml}
      \endminipage
    \end{column}
    \begin{column}{0.5\textwidth}
      \centering
      \providecommand\step{1}\input{diagrams/backtrace/backtrace.tex}
    \end{column}
  \end{columns}
\end{frame}

\begin{frame}{Backtraces}{But before that, we need more fields from \typename{caml\_domain\_state}}
  \begin{columns}
    \begin{column}{0.5\textwidth}
      \begin{itemize}
        \item Remember \typename{caml\_domain\_state} the per-domain runtime state?
        \item Some other members are of interest to us now\dots
        \item \localname{backtrace\_active} is used to know if the runtime should collect backtraces
        \item \localname{backtrace\_active} can be set by
          \begin{itemize}
            \item The \funcarg{b} flag in the \funcname{OCAMLRUNPARAM} environment variable
            \item \mintinline{ocaml}{Printexc.record_backtrace : bool -> unit} \footnotemark
          \end{itemize}
      \end{itemize}
    \end{column}
    \begin{column}{0.5\textwidth}
      \begin{listing}
        \mintc[highlightlines={9-11}]{src/ocaml/domain\_state\_backtrace.h}
        \caption{runtime/caml/domain\_state.h}
      \end{listing}
    \end{column}
  \end{columns}
  \footnotetext[1]{https://v2.ocaml.org/api/Printexc.html}
\end{frame}

\newcommand\listCamlRaiseExn[1][]{
  \listasm[firstline=702,lastline=725,#1]{../ocaml/runtime/amd64.S}{runtime/amd64.S:702}}

\begin{frame}{Backtraces}{Walk through \funcname{caml\_raise\_exn}}
  \begin{columns}[T]
    \begin{column}{0.5\textwidth}
      \begin{itemize}
        \item<1-> First checks whether exceptions backtrace should be recorded
        \item<1-> By checking the \funcarg{domain\_state->backtrace\_active} value
        \item<2-> If not, acts as \funcname{raise\_notrace} with \funcname{RESTORE\_EXN\_HANDLER\_OCAML}
          \begin{itemize}
            \item Set \regname{RSP} to \localname{domain\_state->exn\_handler} value
            \item Restore \localname{domain\_state->exn\_handler} from trap
            \item Jump with \funcname{ret} (same effect as using \funcname{pop} and \funcname{jmp})
          \end{itemize}
        \item<3-> Otherwise, switches to the C stack to be able to execute C code
        \item<4-> Calls \funcname{caml\_stash\_backtrace}, a C function provided by the runtime\ignorespaces
          \onslide<6->{, with
            \begin{itemize}
              \item \funcarg{exn}: exception value
              \item \funcarg{pc}: code address of the raising point
              \item \funcarg{sp}: stack address of the raising function
              \item \funcarg{trapsp}: stack address of the next handler
            \end{itemize}
          }
      \end{itemize}
      \bigskip
      \mintocaml[firstline=9,lastline=13,highlightlines=12]{../src/backtrace/backtrace.ml}
    \end{column}
    \begin{column}{0.5\textwidth}
      \raggedleft
      \onslide*<1,3-4>{
        \mintc[firstline=190,lastline=192]{../ocaml/runtime/amd64.S}
        \mintc[firstline=234,lastline=237]{../ocaml/runtime/amd64.S}
      }
      \onslide*<2>{
        \mintc[firstline=190,lastline=192,highlightlines={191-192}]{../ocaml/runtime/amd64.S}
        \mintc[firstline=234,lastline=237,highlightlines={235-237}]{../ocaml/runtime/amd64.S}
      }
      \onslide*<1>{\listCamlRaiseExn[highlightlines=705]{}}%
      \onslide*<2>{\listCamlRaiseExn[highlightlines={707-708}]{}}%
      \onslide*<3>{\listCamlRaiseExn[highlightlines={712-714}]{}}%
      \onslide*<4>{\listCamlRaiseExn[highlightlines={715-720}]{}}%
      \onslide*<5>{\providecommand\step{1}\input{diagrams/backtrace/backtrace.tex}}%
      \onslide*<6>{\providecommand\step{2}\input{diagrams/backtrace/backtrace.tex}}%
    \end{column}
  \end{columns}
\end{frame}

\newcommand\listStashBacktrace[1][]{
  \listc[firstline=91,lastline=120,#1]{../ocaml/runtime/backtrace\_nat.c}{runtime/backtrace\_nat.c:91}}

\begin{frame}{Backtraces}{Introducing \funcname{caml\_stash\_backtrace}}
  \begin{columns}[c]
    \begin{column}{0.5\textwidth}
      \minipage[c][0.66\textheight][s]{\columnwidth}
      \begin{itemize}
        \item<1-> A C function provided by the runtime (specific to native code)
        \item<2-> It scans the stack, between the stack pointer at the raising point up to the next trap, in order to collect the names of the detected function frames
          \onslide*<2>{
            \begin{itemize}
              \item And more information, like line number, column number...
            \end{itemize}
          }
        \item<3-> It uses the same technique and information as the Garbage Collector (GC) uses to scan the values live on the stack
          \onslide*<4>{
            \begin{itemize}
              \item \typename{frame\_descr} are at the core of stack unwinding
            \end{itemize}
          }
      \end{itemize}
      \vfill
      \mintocaml[firstline=9,lastline=13,highlightlines=12]{../src/backtrace/backtrace.ml}
      \endminipage
    \end{column}
    \begin{column}{0.5\textwidth}
      \onslide*<1>{\listStashBacktrace[highlightlines=91]{}}
      \onslide*<2>{\listStashBacktrace[highlightlines={91,117-118}]{}}
      \onslide*<3->{\listStashBacktrace[highlightlines={94,106,109-110}]{}}
    \end{column}
  \end{columns}
\end{frame}

\newcommand\listFrameDescr[1][]{
  \listocaml[firstline=111,lastline=124,#1]{../ocaml/asmcomp/emitaux.ml}{asmcomp/emitaux.ml:111}}
\newcommand\listDebugInfo[1][]{
  \listocaml[firstline=38,lastline=49,#1]{../ocaml/lambda/debuginfo.mli}{lambda/debuginfo.mli:38}}

\begin{frame}{Backtraces}{\typename{frame\_descr} and \typename{frame\_debuginfo} in a nutshell}
  \begin{columns}[c]
    \begin{column}{0.5\textwidth}
      \begin{itemize}
        \item<1-> For every return address in an OCaml program, there is a associated \typename{frame\_descr}
        \item<2-> A \typename{frame\_descr} specifies
        \begin{itemize}
          \item<2-> The size of the function stack frame
          \item<3-> Where live values are on the stack (so that the GC can mark them as live and prevent from being swept)
        \end{itemize}
        \item<4-> When compiled with debug information, \typename{frame\_descr} will be linked to a \typename{frame\_debuginfo}
        \begin{itemize}
          \item<5-> Contains information about the position in the source code (file name, line and column number)
        \end{itemize}
      \end{itemize}
    \end{column}
    \begin{column}{0.5\textwidth}
        \onslide*<1>{\listFrameDescr[highlightlines={118-122}]{}}
        \onslide*<2>{\listFrameDescr[highlightlines={120}]{}}
        \onslide*<3>{\listFrameDescr[highlightlines={121}]{}}
        \onslide*<4->{\listFrameDescr[highlightlines={122}]{}}
        \medskip
        \onslide*<1-3>{\listDebugInfo{}}
        \onslide*<4>{\listDebugInfo[highlightlines={38,49}]{}}
        \onslide*<5>{\listDebugInfo[highlightlines={39-46}]{}}
    \end{column}
  \end{columns}
\end{frame}

\begin{frame}{Backtraces}{Scanning the stack with \typename{frame\_descr}}
  \begin{columns}[c]
    \begin{column}{0.5\textwidth}
      \begin{itemize}
        \item Given the return address of the code that raises the exception by calling \funcname{caml\_raise\_exn} (\funcarg{pc} argument of \funcname{caml\_stash\_backtrace})
        \item And the position of the stack pointer (\funcarg{sp} argument of \funcname{caml\_stash\_backtrace})
        \item The frame size given by \typename{frame\_descr}.\funcarg{frame\_size}, allows to find the end of the previous frame
        \item And the return address of the caller of this function
        \bigskip
        \onslide<3->{
        \item Note that traps (here the reddish \funcname{ExnA}, \funcname{ExnB} and \funcname{ExnC} blocks) belong to the frame that installed them, and so the frame size changes when traps are removed
        }
      \end{itemize}
    \end{column}
    \begin{column}{0.5\textwidth}
      \raggedleft
      \onslide*<1>{\providecommand\step{2}\input{diagrams/backtrace/backtrace.tex}}%
      \onslide*<2>{\providecommand\step{1}\input{diagrams/backtrace/scanstack.tex}}%
      \onslide*<3>{\providecommand\step{2}\input{diagrams/backtrace/scanstack.tex}}%
      \onslide*<4>{\providecommand\step{3}\input{diagrams/backtrace/scanstack.tex}}%
      \onslide*<5>{\providecommand\step{4}\input{diagrams/backtrace/scanstack.tex}}%
      \onslide*<6>{\providecommand\step{5}\input{diagrams/backtrace/scanstack.tex}}%
    \end{column}
  \end{columns}
\end{frame}

% Duplicate of previous-previous slide
\begin{frame}{Backtraces}{Continuing on \funcname{caml\_stash\_backtrace}}
  \begin{columns}[c]
    \begin{column}{0.5\textwidth}
      \minipage[c][0.75\textheight][s]{\columnwidth}
      \begin{itemize}
          \color{gray}
        \item A C function provided by the runtime (specific to native code)
        \item It scans the stack, between the stack pointer at the raising point up to the next trap, in order to collect the names of the detected function frames
        \item It uses the same technique and information as the Garbage Collector (GC) uses to scan the values live on the stack
      \end{itemize}
      \begin{itemize}
        \item For each function frame detected, store a pointer to the \typename{frame\_descr} in the \localname{domain\_state->backtrace\_buffer} array, effectively constructing the backtrace
      \end{itemize}
      \onslide<2->{
        \begin{itemize}
            \smallskip
          \item Constructed backtrace:
            \funcname{definitely\_raise},
            \onslide<3->{\funcname{probably\_raise}}
        \end{itemize}
      }
      \vfill
      \mintocaml[firstline=9,lastline=13,highlightlines=12]{../src/backtrace/backtrace.ml}
      \endminipage
    \end{column}
    \begin{column}{0.5\textwidth}
      \raggedleft
      \onslide*<1>{\listStashBacktrace[highlightlines={114-115}]{}}
      \onslide*<2>{\providecommand\step{3}\input{diagrams/backtrace/backtrace.tex}}%
      \onslide*<3>{\providecommand\step{4}\input{diagrams/backtrace/backtrace.tex}}%
    \end{column}
  \end{columns}
\end{frame}

\begin{frame}{Backtraces}{Back to \funcname{caml\_raise\_exn}}
  \begin{columns}[c]
    \begin{column}{0.5\textwidth}
      \minipage[c][0.66\textheight][s]{\columnwidth}
      \begin{itemize}\color{gray}
        \item Checks wheteher exceptions backtrace should be recorded, by checking the \funcarg{domain\_state->backtrace\_active} value
        \item Switches to the C stack to be able to execute C code
        \item Calls \funcname{caml\_stash\_backtrace}, to collect the backtrace up to the next trap
      \end{itemize}
      \onslide*<2->{
        \begin{itemize}
          \item<2-> Executes the exception handler in \funcname{probably\_raise} with \funcname{RESTORE\_EXN\_HANDLER\_OCAML}
            \begin{itemize}
              \item Set \regname{RSP} to \localname{domain\_state->exn\_handler} value
              \item Restore \localname{domain\_state->exn\_handler} from trap
              \item Jump to the handler code with \funcname{ret}
            \end{itemize}
        \end{itemize}
      }
      \vfill
      \onslide*<1>{\mintocaml[firstline=9,lastline=13,highlightlines=12]{../src/backtrace/backtrace.ml}}
      \onslide*<2->{\mintocaml[firstline=15,lastline=21,highlightlines={19-21}]{../src/backtrace/backtrace.ml}}
      \endminipage
    \end{column}
    \begin{column}{0.5\textwidth}
      \centering
      \onslide*<1>{
        \mintc[firstline=190,lastline=192]{../ocaml/runtime/amd64.S}
        \mintc[firstline=234,lastline=237]{../ocaml/runtime/amd64.S}
      }
      \onslide*<2>{
        \mintc[firstline=190,lastline=192,highlightlines={191-192}]{../ocaml/runtime/amd64.S}
        \mintc[firstline=234,lastline=237,highlightlines={235-237}]{../ocaml/runtime/amd64.S}
      }
      \onslide*<1>{\listCamlRaiseExn[highlightlines=720]{}}
      \onslide*<2>{\listCamlRaiseExn[highlightlines=722]{}}
      \onslide*<3->{\providecommand\step{5}\input{diagrams/backtrace/backtrace.tex}}
    \end{column}
  \end{columns}
\end{frame}

\begin{frame}{Backtraces}{\funcname{caml\_probably\_raise}'s handler doesn't catch \typename{ExnC}}
  \begin{columns}[c]
    \begin{column}{0.45\textwidth}
      \minipage[c][0.75\textheight][s]{\columnwidth}
      \begin{itemize}
        \item<1-> \funcname{match \dots with}\footnotemark pattern matching can also match exceptions
        \item<1-> The CMM of \funcname{probably\_raise} shows that this \funcname{match \dots with} is implemented with a pattern matching and a \funcname{try \dots with}
        \item<1-> But this one only matches on \typename{ExnA} exception
        \item<2-> Since the pattern matching fails to catch the exception, \funcname{reraise} is called with our current \typename{ExnC} exception
        \item<3-> Disassembly shows that \funcname{reraise} is actually a call to \funcname{caml\_reraise\_exn}, which is also an assembly function provided by the runtime
      \end{itemize}
      \vfill
      \mintocaml[firstline=15,lastline=21,highlightlines={18-21}]{../src/backtrace/backtrace.ml}
      \endminipage
    \end{column}
    \begin{column}{0.55\textwidth}
      \centering
      \onslide*<1>{\mintcmm[highlightlines={10-18}]{../src/backtrace/camlBacktrace\_\_probably\_raise\_351.cmm}}
      \onslide*<2>{\listcmm[highlightlines=18]{../src/backtrace/camlBacktrace\_\_probably\_raise_351.cmm}{CMM of probably\_raise}}
      \onslide*<3>{\mintobjdump[firstline=5,lastline=35,highlightlines=31]{../src/backtrace/camlBacktrace\_\_probably\_raise_351.objdump}}
    \end{column}
  \end{columns}
  \only<1>{\footnotetext[1]{https://blog.janestreet.com/pattern-matching-and-exception-handling-unite/}}
\end{frame}

\newcommand\listCamlReraiseExn[1][]{
  \listasm[firstline=727,lastline=734,#1]{../ocaml/runtime/amd64.S}{runtime/amd64.S:727}}

\begin{frame}{Backtraces}{Introducing \funcname{caml\_reraise\_exn}}
  \begin{columns}[c]
    \begin{column}{0.5\textwidth}
      \minipage[c][0.66\textheight][s]{\columnwidth}
      \begin{itemize}
        \item<1-> The exception have to be reraised to the next handler
        \item<2-> First, checks if the backtrace should be collected with \funcarg{domain\_state->backtrace\_active}
        \only<3>{
        \begin{itemize}
            \item If not, behaves like a \funcname{raise\_notrace} with \funcname{RESTORE\_EXN\_HANDLER\_OCAML}
        \end{itemize}
        }
        \item<4-> Jumps into \funcname{caml\_raise\_exn}
        \item<5-> Calls \funcname{caml\_stash\_backtrace} again, to scan the next part of the stack: from the trap just pop-ed to the next trap
      \end{itemize}
      \vfill
      \mintocaml[firstline=15,lastline=21,highlightlines={18-21}]{../src/backtrace/backtrace.ml}
      \endminipage
    \end{column}
    \begin{column}{0.5\textwidth}
      \raggedleft
      \onslide*<1>{\listCamlReraiseExn{}}
      \onslide*<2>{\listCamlReraiseExn[highlightlines=729]{}}
      \onslide*<3>{
        \mintc[firstline=190,lastline=192,highlightlines={191-192}]{../ocaml/runtime/amd64.S}
        \mintc[firstline=234,lastline=237,highlightlines={235-237}]{../ocaml/runtime/amd64.S}
        \medskip
        \listCamlReraiseExn[highlightlines=731]{}
      }
      \onslide*<4-5>{
        % TODO refactor with \listCamlReraiseExn
        \mintasm[firstline=727,lastline=734,highlightlines=730]{../ocaml/runtime/amd64.S}
        \medskip
      }
      \onslide*<4>{\listCamlRaiseExn[highlightlines=711]{}}
      \onslide*<5>{\listCamlRaiseExn[highlightlines=720]{}}
    \end{column}
  \end{columns}
\end{frame}

\begin{frame}{Backtraces}{Backtrace collection continues}
  \begin{columns}[c]
    \begin{column}{0.55\textwidth}
      \minipage[c][0.75\textheight][s]{\columnwidth}
      \begin{itemize}
        \item<1-> \funcname{caml\_stash\_backtrace} scans the older part of the stack, up to the next trap
        \item<6-> Controls jumps to the nested exception handler in \funcname{maybe\_raise}, which matches only on \typename{ExnB}
        \item<7-> \funcname{caml\_reraise\_exn} is called again, backtrace collection resumes with \funcname{caml\_stash\_backtrace}
      \end{itemize}
      \medskip
      \begin{itemize}
        \item<2-> Constructed backtrace:
        \begin{itemize}
          \item <2-> \funcname{definitely\_raise}
          \item <2-> \funcname{probably\_raise}
          \item <3-> \funcname{probably\_raise} (re-raise)
          \item <4-> \funcname{maybe\_raise}
          \item <5-> \funcname{main}
          \item <8-> \funcname{main} (re-raise)
        \end{itemize}
      \end{itemize}
      \vfill
      \onslide*<1-5>{\mintocaml[firstline=15,lastline=21]{../src/backtrace/backtrace.ml}}
      \onslide*<6>{\mintocaml[firstline=28,lastline=34,highlightlines=31]{../src/backtrace/backtrace.ml}}
      \onslide*<7->{\mintocaml[firstline=28,lastline=34,highlightlines=33]{../src/backtrace/backtrace.ml}}
      \endminipage
    \end{column}
    \begin{column}{0.45\textwidth}
      \raggedleft
      \onslide*<1>{\providecommand\step{6}\input{diagrams/backtrace/backtrace.tex}}%
      \onslide*<2>{\providecommand\step{6}\input{diagrams/backtrace/backtrace.tex}}%
      \onslide*<3>{\providecommand\step{7}\input{diagrams/backtrace/backtrace.tex}}%
      \onslide*<4>{\providecommand\step{8}\input{diagrams/backtrace/backtrace.tex}}%
      \onslide*<5>{\providecommand\step{9}\input{diagrams/backtrace/backtrace.tex}}%
      \onslide*<6>{\providecommand\step{10}\input{diagrams/backtrace/backtrace.tex}}%
      \onslide*<7>{\providecommand\step{11}\input{diagrams/backtrace/backtrace.tex}}%
      \onslide*<8>{\providecommand\step{12}\input{diagrams/backtrace/backtrace.tex}}%
      \onslide*<9>{\providecommand\step{13}\input{diagrams/backtrace/backtrace.tex}}%
    \end{column}
  \end{columns}
\end{frame}

\begin{frame}{Backtraces}{Printing the backtrace}
  \begin{columns}[c]
    \begin{column}{0.45\textwidth}
      \minipage[c][0.66\textheight][s]{\columnwidth}
      \begin{itemize}
        \item<1-> The exception backtrace can now be printed to \funcarg{stdout} with \funcname{Printexc.print_backtrace}
        \item<2-> First the exception backtrace is retrieved into an OCaml array with \funcname{get_raw_backtrace }
        \item<3-> Which is implemented as the \funcname{caml_get_exception_raw_backtrace} C function in the runtime
        \item<4-> And finally printed with \funcname{print_exception_backtrace}
      \end{itemize}
      \vfill
      \mintocaml[firstline=28,lastline=34,highlightlines=33]{../src/backtrace/backtrace.ml}
      \endminipage
    \end{column}
    \begin{column}{0.55\textwidth}
      \onslide*<1,2>{
        \mintocaml[firstline=100,lastline=101]{../ocaml/stdlib/printexc.ml}
      }
      \only<1>{
        \listocaml[firstline=170,lastline=172]{../ocaml/stdlib/printexc.ml}{stdlib/printexc.ml:170}
      }
      \only<2>{
        \listocaml[firstline=170,lastline=172,highlightlines=172]{../ocaml/stdlib/printexc.ml}{stdlib/printexc.ml:170}
      }
      \only<3>{
        \mintc[firstline=154,lastline=156]{../ocaml/runtime/backtrace.c}
        \listc[firstline=171,lastline=192,highlightlines={184-187}]{../ocaml/runtime/backtrace.c}{runtime/backtrace.c:154}
      }
      \only<4>{
        \listocaml[firstline=155,lastline=165]{../ocaml/stdlib/printexc.ml}{stdlib/printexc.ml:55}
      }
    \end{column}
  \end{columns}
\end{frame}


\subsection{The multiple kinds of raise}
\frameSubsection{}{}

\begin{frame}{Backtraces}{When backtraces are not needed}
  \begin{columns}[c]
    \begin{column}{0.45\textwidth}
      \minipage[c][0.66\textheight][s]{\columnwidth}
      \begin{itemize}
        \item<1-> What if the backtrace for an exception is never needed?
        \item<2-> It's possible to raise the exception explicitly with \funcname{raise\_notrace} to make raising as cheap as possible
        \item<3-> An example of use in the \funcname{dynlink} library of the OCaml compiler
      \end{itemize}
      \vfill
      \onslide<3->{
      \listocaml[firstline=205,lastline=209,highlightlines=207]{../ocaml/otherlibs/dynlink/dynlink\_compilerlibs/misc.ml}{otherlibs/dynlink/dynlink\_compilerlibs/misc.ml:205}
      }
      \endminipage
    \end{column}
    \begin{column}{0.55\textwidth}
    \only<4>{
      \mintcmm[highlightlines=3]{../src/notrace/camlNotrace\_\_fun_347.cmm}
      \listcmm{../src/notrace/camlNotrace\_\_all\_somes\_327.cmm}{all\_somes}
    }
    \onslide*<5->{
      \listobjdump[highlightlines={6-9}]{../src/notrace/camlNotrace\_\_fun_347.objdump}{anonymous function from \funcname{all\_somes}}
    }
    \end{column}
  \end{columns}
\end{frame}

\begin{frame}{Backtraces}{Summary table of the multiple kinds of raise}
  \begin{itemize}
    \item When debugging information is enabled and if the backtrace of an exception isn't needed, it's possible to raise with \funcname{raise\_notrace} for performance
    \item When debugging information is \emph{not} enabled, the compiler optimises \funcname{raise} calls to inline assembly (same effect as using \funcname{raise\_notrace})
  \end{itemize}
  \bigskip
  \centering
  \begin{tabular}{| l | c | c | c | c |}
    \hline
    \parbox{10em}{Debug information \\ (\funcarg{-g} flag)}  & \multicolumn{2}{|c|}{Disabled}                                     & \multicolumn{2}{|c|}{Enabled} \\
    \hline
    Raise kind                                               & \funcname{raise} & \funcname{raise\_notrace}                       & \funcname{raise} & \funcname{raise\_notrace} \\
    \hline
    Compilation                                              & \parbox{8em}{inline assembly\\ (optimization)} & inline assembly   & \funcname{caml\_raise\_exn} & inline assembly \\
    \hline
  \end{tabular}
\end{frame}


%
% Takeaway #5
%
\subsection*{Takeaway \#5}
\frameSubsectionTakeaway{}
\againframe<5>{takeaway}
}
    \end{column}
  \end{columns}
\end{frame}

\begin{frame}{Backtraces}{\funcname{caml\_probably\_raise}'s handler doesn't catch \typename{ExnC}}
  \begin{columns}[c]
    \begin{column}{0.45\textwidth}
      \minipage[c][0.75\textheight][s]{\columnwidth}
      \begin{itemize}
        \item<1-> \funcname{match \dots with}\footnotemark pattern matching can also match exceptions
        \item<1-> The CMM of \funcname{probably\_raise} shows that this \funcname{match \dots with} is implemented with a pattern matching and a \funcname{try \dots with}
        \item<1-> But this one only matches on \typename{ExnA} exception
        \item<2-> Since the pattern matching fails to catch the exception, \funcname{reraise} is called with our current \typename{ExnC} exception
        \item<3-> Disassembly shows that \funcname{reraise} is actually a call to \funcname{caml\_reraise\_exn}, which is also an assembly function provided by the runtime
      \end{itemize}
      \vfill
      \mintocaml[firstline=15,lastline=21,highlightlines={18-21}]{../src/backtrace/backtrace.ml}
      \endminipage
    \end{column}
    \begin{column}{0.55\textwidth}
      \centering
      \onslide*<1>{\mintcmm[highlightlines={10-18}]{../src/backtrace/camlBacktrace\_\_probably\_raise\_351.cmm}}
      \onslide*<2>{\listcmm[highlightlines=18]{../src/backtrace/camlBacktrace\_\_probably\_raise_351.cmm}{CMM of probably\_raise}}
      \onslide*<3>{\mintobjdump[firstline=5,lastline=35,highlightlines=31]{../src/backtrace/camlBacktrace\_\_probably\_raise_351.objdump}}
    \end{column}
  \end{columns}
  \only<1>{\footnotetext[1]{https://blog.janestreet.com/pattern-matching-and-exception-handling-unite/}}
\end{frame}

\newcommand\listCamlReraiseExn[1][]{
  \listasm[firstline=727,lastline=734,#1]{../ocaml/runtime/amd64.S}{runtime/amd64.S:727}}

\begin{frame}{Backtraces}{Introducing \funcname{caml\_reraise\_exn}}
  \begin{columns}[c]
    \begin{column}{0.5\textwidth}
      \minipage[c][0.66\textheight][s]{\columnwidth}
      \begin{itemize}
        \item<1-> The exception have to be reraised to the next handler
        \item<2-> First, checks if the backtrace should be collected with \funcarg{domain\_state->backtrace\_active}
        \only<3>{
        \begin{itemize}
            \item If not, behaves like a \funcname{raise\_notrace} with \funcname{RESTORE\_EXN\_HANDLER\_OCAML}
        \end{itemize}
        }
        \item<4-> Jumps into \funcname{caml\_raise\_exn}
        \item<5-> Calls \funcname{caml\_stash\_backtrace} again, to scan the next part of the stack: from the trap just pop-ed to the next trap
      \end{itemize}
      \vfill
      \mintocaml[firstline=15,lastline=21,highlightlines={18-21}]{../src/backtrace/backtrace.ml}
      \endminipage
    \end{column}
    \begin{column}{0.5\textwidth}
      \raggedleft
      \onslide*<1>{\listCamlReraiseExn{}}
      \onslide*<2>{\listCamlReraiseExn[highlightlines=729]{}}
      \onslide*<3>{
        \mintc[firstline=190,lastline=192,highlightlines={191-192}]{../ocaml/runtime/amd64.S}
        \mintc[firstline=234,lastline=237,highlightlines={235-237}]{../ocaml/runtime/amd64.S}
        \medskip
        \listCamlReraiseExn[highlightlines=731]{}
      }
      \onslide*<4-5>{
        % TODO refactor with \listCamlReraiseExn
        \mintasm[firstline=727,lastline=734,highlightlines=730]{../ocaml/runtime/amd64.S}
        \medskip
      }
      \onslide*<4>{\listCamlRaiseExn[highlightlines=711]{}}
      \onslide*<5>{\listCamlRaiseExn[highlightlines=720]{}}
    \end{column}
  \end{columns}
\end{frame}

\begin{frame}{Backtraces}{Backtrace collection continues}
  \begin{columns}[c]
    \begin{column}{0.55\textwidth}
      \minipage[c][0.75\textheight][s]{\columnwidth}
      \begin{itemize}
        \item<1-> \funcname{caml\_stash\_backtrace} scans the older part of the stack, up to the next trap
        \item<6-> Controls jumps to the nested exception handler in \funcname{maybe\_raise}, which matches only on \typename{ExnB}
        \item<7-> \funcname{caml\_reraise\_exn} is called again, backtrace collection resumes with \funcname{caml\_stash\_backtrace}
      \end{itemize}
      \medskip
      \begin{itemize}
        \item<2-> Constructed backtrace:
        \begin{itemize}
          \item <2-> \funcname{definitely\_raise}
          \item <2-> \funcname{probably\_raise}
          \item <3-> \funcname{probably\_raise} (re-raise)
          \item <4-> \funcname{maybe\_raise}
          \item <5-> \funcname{main}
          \item <8-> \funcname{main} (re-raise)
        \end{itemize}
      \end{itemize}
      \vfill
      \onslide*<1-5>{\mintocaml[firstline=15,lastline=21]{../src/backtrace/backtrace.ml}}
      \onslide*<6>{\mintocaml[firstline=28,lastline=34,highlightlines=31]{../src/backtrace/backtrace.ml}}
      \onslide*<7->{\mintocaml[firstline=28,lastline=34,highlightlines=33]{../src/backtrace/backtrace.ml}}
      \endminipage
    \end{column}
    \begin{column}{0.45\textwidth}
      \raggedleft
      \onslide*<1>{\providecommand\step{6}%
% Backtraces
%
\subsection{One advantage of raising with debugging information}
\frameSubsection{}{}

\begin{frame}{Backtraces}{Debugging information \& source code}
  \begin{columns}[c]
    \begin{column}{0.5\textwidth}
      \begin{itemize}
        \item Raising an exception is fun and games, but how do we know where the exception originated? $\rightarrow$ With the backtrace associated with the exception!
        \item In order to get a backtrace from the point where an exception was raised, it's necessary to compile with debugging information enabled (\funcarg{-g} flag for the compiler)
        \item Same example as in the `Nested handlers' section
      \end{itemize}
    \end{column}
    \begin{column}{0.5\textwidth}
      \listocaml[firstline=9,lastline=34]{../src/backtrace/backtrace.ml}{backtrace/backtrace.ml}
    \end{column}
  \end{columns}
\end{frame}

\begin{frame}{Backtraces}{CMM and disassembly of \funcname{definitely\_raise}}
  \begin{columns}[c]
    \begin{column}{0.5\textwidth}
      \minipage[c][0.66\textheight][s]{\columnwidth}
      \begin{itemize}
        \item<1-> An effect of compiling with debugging information (\funcarg{-g}): the CMM no longer uses \funcname{raise\_notrace} to raise the exception but \funcname{raise} instead
        \item<2-> Disassembly shows that CMM \funcname{raise} is actually a call to \funcname{caml\_raise\_exn}, a function in assembler provided by the runtime
      \end{itemize}
      \vfill
      \mintocaml[firstline=9,lastline=13,highlightlines=12]{../src/backtrace/backtrace.ml}
      \endminipage
    \end{column}
    \begin{column}{0.5\textwidth}
      \onslide*<1>{
        \mintcmm[highlightlines=5]{../src/backtrace/camlBacktrace\_\_definitely\_raise\_347.cmm}
      }
      \onslide*<2>{
        \mintobjdump[firstline=2,highlightlines=14]{../src/backtrace/camlBacktrace\_\_definitely\_raise\_347.objdump}
      }
    \end{column}
  \end{columns}
\end{frame}

\begin{frame}{Backtraces}{Introducing \funcname{caml\_raise\_exn}}
  \begin{columns}[c]
    \begin{column}{0.5\textwidth}
      \minipage[c][0.66\textheight][s]{\columnwidth}
      \onslide*<1>{
        \begin{itemize}
          \item Raising the exception is performed using \funcname{caml\_raise\_exn} which is
            \begin{itemize}
              \item An assembly function provided by the runtime
              \item Architecture specific
            \end{itemize}
          \item Let's see what it does differently from \funcname{raise\_notrace}\dots
          \item We are about to raise \typename{ExnC} with \funcname{caml\_raise\_exn} from \funcname{definitely\_raise}
        \end{itemize}
      }
      \onslide*<2>{
        \mintobjdump[firstline=2,highlightlines=14]{../src/backtrace/camlBacktrace\_\_definitely\_raise\_347.objdump}
      }
      \vfill
      \mintocaml[firstline=9,lastline=13,highlightlines=12]{../src/backtrace/backtrace.ml}
      \endminipage
    \end{column}
    \begin{column}{0.5\textwidth}
      \centering
      \providecommand\step{1}\input{diagrams/backtrace/backtrace.tex}
    \end{column}
  \end{columns}
\end{frame}

\begin{frame}{Backtraces}{But before that, we need more fields from \typename{caml\_domain\_state}}
  \begin{columns}
    \begin{column}{0.5\textwidth}
      \begin{itemize}
        \item Remember \typename{caml\_domain\_state} the per-domain runtime state?
        \item Some other members are of interest to us now\dots
        \item \localname{backtrace\_active} is used to know if the runtime should collect backtraces
        \item \localname{backtrace\_active} can be set by
          \begin{itemize}
            \item The \funcarg{b} flag in the \funcname{OCAMLRUNPARAM} environment variable
            \item \mintinline{ocaml}{Printexc.record_backtrace : bool -> unit} \footnotemark
          \end{itemize}
      \end{itemize}
    \end{column}
    \begin{column}{0.5\textwidth}
      \begin{listing}
        \mintc[highlightlines={9-11}]{src/ocaml/domain\_state\_backtrace.h}
        \caption{runtime/caml/domain\_state.h}
      \end{listing}
    \end{column}
  \end{columns}
  \footnotetext[1]{https://v2.ocaml.org/api/Printexc.html}
\end{frame}

\newcommand\listCamlRaiseExn[1][]{
  \listasm[firstline=702,lastline=725,#1]{../ocaml/runtime/amd64.S}{runtime/amd64.S:702}}

\begin{frame}{Backtraces}{Walk through \funcname{caml\_raise\_exn}}
  \begin{columns}[T]
    \begin{column}{0.5\textwidth}
      \begin{itemize}
        \item<1-> First checks whether exceptions backtrace should be recorded
        \item<1-> By checking the \funcarg{domain\_state->backtrace\_active} value
        \item<2-> If not, acts as \funcname{raise\_notrace} with \funcname{RESTORE\_EXN\_HANDLER\_OCAML}
          \begin{itemize}
            \item Set \regname{RSP} to \localname{domain\_state->exn\_handler} value
            \item Restore \localname{domain\_state->exn\_handler} from trap
            \item Jump with \funcname{ret} (same effect as using \funcname{pop} and \funcname{jmp})
          \end{itemize}
        \item<3-> Otherwise, switches to the C stack to be able to execute C code
        \item<4-> Calls \funcname{caml\_stash\_backtrace}, a C function provided by the runtime\ignorespaces
          \onslide<6->{, with
            \begin{itemize}
              \item \funcarg{exn}: exception value
              \item \funcarg{pc}: code address of the raising point
              \item \funcarg{sp}: stack address of the raising function
              \item \funcarg{trapsp}: stack address of the next handler
            \end{itemize}
          }
      \end{itemize}
      \bigskip
      \mintocaml[firstline=9,lastline=13,highlightlines=12]{../src/backtrace/backtrace.ml}
    \end{column}
    \begin{column}{0.5\textwidth}
      \raggedleft
      \onslide*<1,3-4>{
        \mintc[firstline=190,lastline=192]{../ocaml/runtime/amd64.S}
        \mintc[firstline=234,lastline=237]{../ocaml/runtime/amd64.S}
      }
      \onslide*<2>{
        \mintc[firstline=190,lastline=192,highlightlines={191-192}]{../ocaml/runtime/amd64.S}
        \mintc[firstline=234,lastline=237,highlightlines={235-237}]{../ocaml/runtime/amd64.S}
      }
      \onslide*<1>{\listCamlRaiseExn[highlightlines=705]{}}%
      \onslide*<2>{\listCamlRaiseExn[highlightlines={707-708}]{}}%
      \onslide*<3>{\listCamlRaiseExn[highlightlines={712-714}]{}}%
      \onslide*<4>{\listCamlRaiseExn[highlightlines={715-720}]{}}%
      \onslide*<5>{\providecommand\step{1}\input{diagrams/backtrace/backtrace.tex}}%
      \onslide*<6>{\providecommand\step{2}\input{diagrams/backtrace/backtrace.tex}}%
    \end{column}
  \end{columns}
\end{frame}

\newcommand\listStashBacktrace[1][]{
  \listc[firstline=91,lastline=120,#1]{../ocaml/runtime/backtrace\_nat.c}{runtime/backtrace\_nat.c:91}}

\begin{frame}{Backtraces}{Introducing \funcname{caml\_stash\_backtrace}}
  \begin{columns}[c]
    \begin{column}{0.5\textwidth}
      \minipage[c][0.66\textheight][s]{\columnwidth}
      \begin{itemize}
        \item<1-> A C function provided by the runtime (specific to native code)
        \item<2-> It scans the stack, between the stack pointer at the raising point up to the next trap, in order to collect the names of the detected function frames
          \onslide*<2>{
            \begin{itemize}
              \item And more information, like line number, column number...
            \end{itemize}
          }
        \item<3-> It uses the same technique and information as the Garbage Collector (GC) uses to scan the values live on the stack
          \onslide*<4>{
            \begin{itemize}
              \item \typename{frame\_descr} are at the core of stack unwinding
            \end{itemize}
          }
      \end{itemize}
      \vfill
      \mintocaml[firstline=9,lastline=13,highlightlines=12]{../src/backtrace/backtrace.ml}
      \endminipage
    \end{column}
    \begin{column}{0.5\textwidth}
      \onslide*<1>{\listStashBacktrace[highlightlines=91]{}}
      \onslide*<2>{\listStashBacktrace[highlightlines={91,117-118}]{}}
      \onslide*<3->{\listStashBacktrace[highlightlines={94,106,109-110}]{}}
    \end{column}
  \end{columns}
\end{frame}

\newcommand\listFrameDescr[1][]{
  \listocaml[firstline=111,lastline=124,#1]{../ocaml/asmcomp/emitaux.ml}{asmcomp/emitaux.ml:111}}
\newcommand\listDebugInfo[1][]{
  \listocaml[firstline=38,lastline=49,#1]{../ocaml/lambda/debuginfo.mli}{lambda/debuginfo.mli:38}}

\begin{frame}{Backtraces}{\typename{frame\_descr} and \typename{frame\_debuginfo} in a nutshell}
  \begin{columns}[c]
    \begin{column}{0.5\textwidth}
      \begin{itemize}
        \item<1-> For every return address in an OCaml program, there is a associated \typename{frame\_descr}
        \item<2-> A \typename{frame\_descr} specifies
        \begin{itemize}
          \item<2-> The size of the function stack frame
          \item<3-> Where live values are on the stack (so that the GC can mark them as live and prevent from being swept)
        \end{itemize}
        \item<4-> When compiled with debug information, \typename{frame\_descr} will be linked to a \typename{frame\_debuginfo}
        \begin{itemize}
          \item<5-> Contains information about the position in the source code (file name, line and column number)
        \end{itemize}
      \end{itemize}
    \end{column}
    \begin{column}{0.5\textwidth}
        \onslide*<1>{\listFrameDescr[highlightlines={118-122}]{}}
        \onslide*<2>{\listFrameDescr[highlightlines={120}]{}}
        \onslide*<3>{\listFrameDescr[highlightlines={121}]{}}
        \onslide*<4->{\listFrameDescr[highlightlines={122}]{}}
        \medskip
        \onslide*<1-3>{\listDebugInfo{}}
        \onslide*<4>{\listDebugInfo[highlightlines={38,49}]{}}
        \onslide*<5>{\listDebugInfo[highlightlines={39-46}]{}}
    \end{column}
  \end{columns}
\end{frame}

\begin{frame}{Backtraces}{Scanning the stack with \typename{frame\_descr}}
  \begin{columns}[c]
    \begin{column}{0.5\textwidth}
      \begin{itemize}
        \item Given the return address of the code that raises the exception by calling \funcname{caml\_raise\_exn} (\funcarg{pc} argument of \funcname{caml\_stash\_backtrace})
        \item And the position of the stack pointer (\funcarg{sp} argument of \funcname{caml\_stash\_backtrace})
        \item The frame size given by \typename{frame\_descr}.\funcarg{frame\_size}, allows to find the end of the previous frame
        \item And the return address of the caller of this function
        \bigskip
        \onslide<3->{
        \item Note that traps (here the reddish \funcname{ExnA}, \funcname{ExnB} and \funcname{ExnC} blocks) belong to the frame that installed them, and so the frame size changes when traps are removed
        }
      \end{itemize}
    \end{column}
    \begin{column}{0.5\textwidth}
      \raggedleft
      \onslide*<1>{\providecommand\step{2}\input{diagrams/backtrace/backtrace.tex}}%
      \onslide*<2>{\providecommand\step{1}\input{diagrams/backtrace/scanstack.tex}}%
      \onslide*<3>{\providecommand\step{2}\input{diagrams/backtrace/scanstack.tex}}%
      \onslide*<4>{\providecommand\step{3}\input{diagrams/backtrace/scanstack.tex}}%
      \onslide*<5>{\providecommand\step{4}\input{diagrams/backtrace/scanstack.tex}}%
      \onslide*<6>{\providecommand\step{5}\input{diagrams/backtrace/scanstack.tex}}%
    \end{column}
  \end{columns}
\end{frame}

% Duplicate of previous-previous slide
\begin{frame}{Backtraces}{Continuing on \funcname{caml\_stash\_backtrace}}
  \begin{columns}[c]
    \begin{column}{0.5\textwidth}
      \minipage[c][0.75\textheight][s]{\columnwidth}
      \begin{itemize}
          \color{gray}
        \item A C function provided by the runtime (specific to native code)
        \item It scans the stack, between the stack pointer at the raising point up to the next trap, in order to collect the names of the detected function frames
        \item It uses the same technique and information as the Garbage Collector (GC) uses to scan the values live on the stack
      \end{itemize}
      \begin{itemize}
        \item For each function frame detected, store a pointer to the \typename{frame\_descr} in the \localname{domain\_state->backtrace\_buffer} array, effectively constructing the backtrace
      \end{itemize}
      \onslide<2->{
        \begin{itemize}
            \smallskip
          \item Constructed backtrace:
            \funcname{definitely\_raise},
            \onslide<3->{\funcname{probably\_raise}}
        \end{itemize}
      }
      \vfill
      \mintocaml[firstline=9,lastline=13,highlightlines=12]{../src/backtrace/backtrace.ml}
      \endminipage
    \end{column}
    \begin{column}{0.5\textwidth}
      \raggedleft
      \onslide*<1>{\listStashBacktrace[highlightlines={114-115}]{}}
      \onslide*<2>{\providecommand\step{3}\input{diagrams/backtrace/backtrace.tex}}%
      \onslide*<3>{\providecommand\step{4}\input{diagrams/backtrace/backtrace.tex}}%
    \end{column}
  \end{columns}
\end{frame}

\begin{frame}{Backtraces}{Back to \funcname{caml\_raise\_exn}}
  \begin{columns}[c]
    \begin{column}{0.5\textwidth}
      \minipage[c][0.66\textheight][s]{\columnwidth}
      \begin{itemize}\color{gray}
        \item Checks wheteher exceptions backtrace should be recorded, by checking the \funcarg{domain\_state->backtrace\_active} value
        \item Switches to the C stack to be able to execute C code
        \item Calls \funcname{caml\_stash\_backtrace}, to collect the backtrace up to the next trap
      \end{itemize}
      \onslide*<2->{
        \begin{itemize}
          \item<2-> Executes the exception handler in \funcname{probably\_raise} with \funcname{RESTORE\_EXN\_HANDLER\_OCAML}
            \begin{itemize}
              \item Set \regname{RSP} to \localname{domain\_state->exn\_handler} value
              \item Restore \localname{domain\_state->exn\_handler} from trap
              \item Jump to the handler code with \funcname{ret}
            \end{itemize}
        \end{itemize}
      }
      \vfill
      \onslide*<1>{\mintocaml[firstline=9,lastline=13,highlightlines=12]{../src/backtrace/backtrace.ml}}
      \onslide*<2->{\mintocaml[firstline=15,lastline=21,highlightlines={19-21}]{../src/backtrace/backtrace.ml}}
      \endminipage
    \end{column}
    \begin{column}{0.5\textwidth}
      \centering
      \onslide*<1>{
        \mintc[firstline=190,lastline=192]{../ocaml/runtime/amd64.S}
        \mintc[firstline=234,lastline=237]{../ocaml/runtime/amd64.S}
      }
      \onslide*<2>{
        \mintc[firstline=190,lastline=192,highlightlines={191-192}]{../ocaml/runtime/amd64.S}
        \mintc[firstline=234,lastline=237,highlightlines={235-237}]{../ocaml/runtime/amd64.S}
      }
      \onslide*<1>{\listCamlRaiseExn[highlightlines=720]{}}
      \onslide*<2>{\listCamlRaiseExn[highlightlines=722]{}}
      \onslide*<3->{\providecommand\step{5}\input{diagrams/backtrace/backtrace.tex}}
    \end{column}
  \end{columns}
\end{frame}

\begin{frame}{Backtraces}{\funcname{caml\_probably\_raise}'s handler doesn't catch \typename{ExnC}}
  \begin{columns}[c]
    \begin{column}{0.45\textwidth}
      \minipage[c][0.75\textheight][s]{\columnwidth}
      \begin{itemize}
        \item<1-> \funcname{match \dots with}\footnotemark pattern matching can also match exceptions
        \item<1-> The CMM of \funcname{probably\_raise} shows that this \funcname{match \dots with} is implemented with a pattern matching and a \funcname{try \dots with}
        \item<1-> But this one only matches on \typename{ExnA} exception
        \item<2-> Since the pattern matching fails to catch the exception, \funcname{reraise} is called with our current \typename{ExnC} exception
        \item<3-> Disassembly shows that \funcname{reraise} is actually a call to \funcname{caml\_reraise\_exn}, which is also an assembly function provided by the runtime
      \end{itemize}
      \vfill
      \mintocaml[firstline=15,lastline=21,highlightlines={18-21}]{../src/backtrace/backtrace.ml}
      \endminipage
    \end{column}
    \begin{column}{0.55\textwidth}
      \centering
      \onslide*<1>{\mintcmm[highlightlines={10-18}]{../src/backtrace/camlBacktrace\_\_probably\_raise\_351.cmm}}
      \onslide*<2>{\listcmm[highlightlines=18]{../src/backtrace/camlBacktrace\_\_probably\_raise_351.cmm}{CMM of probably\_raise}}
      \onslide*<3>{\mintobjdump[firstline=5,lastline=35,highlightlines=31]{../src/backtrace/camlBacktrace\_\_probably\_raise_351.objdump}}
    \end{column}
  \end{columns}
  \only<1>{\footnotetext[1]{https://blog.janestreet.com/pattern-matching-and-exception-handling-unite/}}
\end{frame}

\newcommand\listCamlReraiseExn[1][]{
  \listasm[firstline=727,lastline=734,#1]{../ocaml/runtime/amd64.S}{runtime/amd64.S:727}}

\begin{frame}{Backtraces}{Introducing \funcname{caml\_reraise\_exn}}
  \begin{columns}[c]
    \begin{column}{0.5\textwidth}
      \minipage[c][0.66\textheight][s]{\columnwidth}
      \begin{itemize}
        \item<1-> The exception have to be reraised to the next handler
        \item<2-> First, checks if the backtrace should be collected with \funcarg{domain\_state->backtrace\_active}
        \only<3>{
        \begin{itemize}
            \item If not, behaves like a \funcname{raise\_notrace} with \funcname{RESTORE\_EXN\_HANDLER\_OCAML}
        \end{itemize}
        }
        \item<4-> Jumps into \funcname{caml\_raise\_exn}
        \item<5-> Calls \funcname{caml\_stash\_backtrace} again, to scan the next part of the stack: from the trap just pop-ed to the next trap
      \end{itemize}
      \vfill
      \mintocaml[firstline=15,lastline=21,highlightlines={18-21}]{../src/backtrace/backtrace.ml}
      \endminipage
    \end{column}
    \begin{column}{0.5\textwidth}
      \raggedleft
      \onslide*<1>{\listCamlReraiseExn{}}
      \onslide*<2>{\listCamlReraiseExn[highlightlines=729]{}}
      \onslide*<3>{
        \mintc[firstline=190,lastline=192,highlightlines={191-192}]{../ocaml/runtime/amd64.S}
        \mintc[firstline=234,lastline=237,highlightlines={235-237}]{../ocaml/runtime/amd64.S}
        \medskip
        \listCamlReraiseExn[highlightlines=731]{}
      }
      \onslide*<4-5>{
        % TODO refactor with \listCamlReraiseExn
        \mintasm[firstline=727,lastline=734,highlightlines=730]{../ocaml/runtime/amd64.S}
        \medskip
      }
      \onslide*<4>{\listCamlRaiseExn[highlightlines=711]{}}
      \onslide*<5>{\listCamlRaiseExn[highlightlines=720]{}}
    \end{column}
  \end{columns}
\end{frame}

\begin{frame}{Backtraces}{Backtrace collection continues}
  \begin{columns}[c]
    \begin{column}{0.55\textwidth}
      \minipage[c][0.75\textheight][s]{\columnwidth}
      \begin{itemize}
        \item<1-> \funcname{caml\_stash\_backtrace} scans the older part of the stack, up to the next trap
        \item<6-> Controls jumps to the nested exception handler in \funcname{maybe\_raise}, which matches only on \typename{ExnB}
        \item<7-> \funcname{caml\_reraise\_exn} is called again, backtrace collection resumes with \funcname{caml\_stash\_backtrace}
      \end{itemize}
      \medskip
      \begin{itemize}
        \item<2-> Constructed backtrace:
        \begin{itemize}
          \item <2-> \funcname{definitely\_raise}
          \item <2-> \funcname{probably\_raise}
          \item <3-> \funcname{probably\_raise} (re-raise)
          \item <4-> \funcname{maybe\_raise}
          \item <5-> \funcname{main}
          \item <8-> \funcname{main} (re-raise)
        \end{itemize}
      \end{itemize}
      \vfill
      \onslide*<1-5>{\mintocaml[firstline=15,lastline=21]{../src/backtrace/backtrace.ml}}
      \onslide*<6>{\mintocaml[firstline=28,lastline=34,highlightlines=31]{../src/backtrace/backtrace.ml}}
      \onslide*<7->{\mintocaml[firstline=28,lastline=34,highlightlines=33]{../src/backtrace/backtrace.ml}}
      \endminipage
    \end{column}
    \begin{column}{0.45\textwidth}
      \raggedleft
      \onslide*<1>{\providecommand\step{6}\input{diagrams/backtrace/backtrace.tex}}%
      \onslide*<2>{\providecommand\step{6}\input{diagrams/backtrace/backtrace.tex}}%
      \onslide*<3>{\providecommand\step{7}\input{diagrams/backtrace/backtrace.tex}}%
      \onslide*<4>{\providecommand\step{8}\input{diagrams/backtrace/backtrace.tex}}%
      \onslide*<5>{\providecommand\step{9}\input{diagrams/backtrace/backtrace.tex}}%
      \onslide*<6>{\providecommand\step{10}\input{diagrams/backtrace/backtrace.tex}}%
      \onslide*<7>{\providecommand\step{11}\input{diagrams/backtrace/backtrace.tex}}%
      \onslide*<8>{\providecommand\step{12}\input{diagrams/backtrace/backtrace.tex}}%
      \onslide*<9>{\providecommand\step{13}\input{diagrams/backtrace/backtrace.tex}}%
    \end{column}
  \end{columns}
\end{frame}

\begin{frame}{Backtraces}{Printing the backtrace}
  \begin{columns}[c]
    \begin{column}{0.45\textwidth}
      \minipage[c][0.66\textheight][s]{\columnwidth}
      \begin{itemize}
        \item<1-> The exception backtrace can now be printed to \funcarg{stdout} with \funcname{Printexc.print_backtrace}
        \item<2-> First the exception backtrace is retrieved into an OCaml array with \funcname{get_raw_backtrace }
        \item<3-> Which is implemented as the \funcname{caml_get_exception_raw_backtrace} C function in the runtime
        \item<4-> And finally printed with \funcname{print_exception_backtrace}
      \end{itemize}
      \vfill
      \mintocaml[firstline=28,lastline=34,highlightlines=33]{../src/backtrace/backtrace.ml}
      \endminipage
    \end{column}
    \begin{column}{0.55\textwidth}
      \onslide*<1,2>{
        \mintocaml[firstline=100,lastline=101]{../ocaml/stdlib/printexc.ml}
      }
      \only<1>{
        \listocaml[firstline=170,lastline=172]{../ocaml/stdlib/printexc.ml}{stdlib/printexc.ml:170}
      }
      \only<2>{
        \listocaml[firstline=170,lastline=172,highlightlines=172]{../ocaml/stdlib/printexc.ml}{stdlib/printexc.ml:170}
      }
      \only<3>{
        \mintc[firstline=154,lastline=156]{../ocaml/runtime/backtrace.c}
        \listc[firstline=171,lastline=192,highlightlines={184-187}]{../ocaml/runtime/backtrace.c}{runtime/backtrace.c:154}
      }
      \only<4>{
        \listocaml[firstline=155,lastline=165]{../ocaml/stdlib/printexc.ml}{stdlib/printexc.ml:55}
      }
    \end{column}
  \end{columns}
\end{frame}


\subsection{The multiple kinds of raise}
\frameSubsection{}{}

\begin{frame}{Backtraces}{When backtraces are not needed}
  \begin{columns}[c]
    \begin{column}{0.45\textwidth}
      \minipage[c][0.66\textheight][s]{\columnwidth}
      \begin{itemize}
        \item<1-> What if the backtrace for an exception is never needed?
        \item<2-> It's possible to raise the exception explicitly with \funcname{raise\_notrace} to make raising as cheap as possible
        \item<3-> An example of use in the \funcname{dynlink} library of the OCaml compiler
      \end{itemize}
      \vfill
      \onslide<3->{
      \listocaml[firstline=205,lastline=209,highlightlines=207]{../ocaml/otherlibs/dynlink/dynlink\_compilerlibs/misc.ml}{otherlibs/dynlink/dynlink\_compilerlibs/misc.ml:205}
      }
      \endminipage
    \end{column}
    \begin{column}{0.55\textwidth}
    \only<4>{
      \mintcmm[highlightlines=3]{../src/notrace/camlNotrace\_\_fun_347.cmm}
      \listcmm{../src/notrace/camlNotrace\_\_all\_somes\_327.cmm}{all\_somes}
    }
    \onslide*<5->{
      \listobjdump[highlightlines={6-9}]{../src/notrace/camlNotrace\_\_fun_347.objdump}{anonymous function from \funcname{all\_somes}}
    }
    \end{column}
  \end{columns}
\end{frame}

\begin{frame}{Backtraces}{Summary table of the multiple kinds of raise}
  \begin{itemize}
    \item When debugging information is enabled and if the backtrace of an exception isn't needed, it's possible to raise with \funcname{raise\_notrace} for performance
    \item When debugging information is \emph{not} enabled, the compiler optimises \funcname{raise} calls to inline assembly (same effect as using \funcname{raise\_notrace})
  \end{itemize}
  \bigskip
  \centering
  \begin{tabular}{| l | c | c | c | c |}
    \hline
    \parbox{10em}{Debug information \\ (\funcarg{-g} flag)}  & \multicolumn{2}{|c|}{Disabled}                                     & \multicolumn{2}{|c|}{Enabled} \\
    \hline
    Raise kind                                               & \funcname{raise} & \funcname{raise\_notrace}                       & \funcname{raise} & \funcname{raise\_notrace} \\
    \hline
    Compilation                                              & \parbox{8em}{inline assembly\\ (optimization)} & inline assembly   & \funcname{caml\_raise\_exn} & inline assembly \\
    \hline
  \end{tabular}
\end{frame}


%
% Takeaway #5
%
\subsection*{Takeaway \#5}
\frameSubsectionTakeaway{}
\againframe<5>{takeaway}
}%
      \onslide*<2>{\providecommand\step{6}%
% Backtraces
%
\subsection{One advantage of raising with debugging information}
\frameSubsection{}{}

\begin{frame}{Backtraces}{Debugging information \& source code}
  \begin{columns}[c]
    \begin{column}{0.5\textwidth}
      \begin{itemize}
        \item Raising an exception is fun and games, but how do we know where the exception originated? $\rightarrow$ With the backtrace associated with the exception!
        \item In order to get a backtrace from the point where an exception was raised, it's necessary to compile with debugging information enabled (\funcarg{-g} flag for the compiler)
        \item Same example as in the `Nested handlers' section
      \end{itemize}
    \end{column}
    \begin{column}{0.5\textwidth}
      \listocaml[firstline=9,lastline=34]{../src/backtrace/backtrace.ml}{backtrace/backtrace.ml}
    \end{column}
  \end{columns}
\end{frame}

\begin{frame}{Backtraces}{CMM and disassembly of \funcname{definitely\_raise}}
  \begin{columns}[c]
    \begin{column}{0.5\textwidth}
      \minipage[c][0.66\textheight][s]{\columnwidth}
      \begin{itemize}
        \item<1-> An effect of compiling with debugging information (\funcarg{-g}): the CMM no longer uses \funcname{raise\_notrace} to raise the exception but \funcname{raise} instead
        \item<2-> Disassembly shows that CMM \funcname{raise} is actually a call to \funcname{caml\_raise\_exn}, a function in assembler provided by the runtime
      \end{itemize}
      \vfill
      \mintocaml[firstline=9,lastline=13,highlightlines=12]{../src/backtrace/backtrace.ml}
      \endminipage
    \end{column}
    \begin{column}{0.5\textwidth}
      \onslide*<1>{
        \mintcmm[highlightlines=5]{../src/backtrace/camlBacktrace\_\_definitely\_raise\_347.cmm}
      }
      \onslide*<2>{
        \mintobjdump[firstline=2,highlightlines=14]{../src/backtrace/camlBacktrace\_\_definitely\_raise\_347.objdump}
      }
    \end{column}
  \end{columns}
\end{frame}

\begin{frame}{Backtraces}{Introducing \funcname{caml\_raise\_exn}}
  \begin{columns}[c]
    \begin{column}{0.5\textwidth}
      \minipage[c][0.66\textheight][s]{\columnwidth}
      \onslide*<1>{
        \begin{itemize}
          \item Raising the exception is performed using \funcname{caml\_raise\_exn} which is
            \begin{itemize}
              \item An assembly function provided by the runtime
              \item Architecture specific
            \end{itemize}
          \item Let's see what it does differently from \funcname{raise\_notrace}\dots
          \item We are about to raise \typename{ExnC} with \funcname{caml\_raise\_exn} from \funcname{definitely\_raise}
        \end{itemize}
      }
      \onslide*<2>{
        \mintobjdump[firstline=2,highlightlines=14]{../src/backtrace/camlBacktrace\_\_definitely\_raise\_347.objdump}
      }
      \vfill
      \mintocaml[firstline=9,lastline=13,highlightlines=12]{../src/backtrace/backtrace.ml}
      \endminipage
    \end{column}
    \begin{column}{0.5\textwidth}
      \centering
      \providecommand\step{1}\input{diagrams/backtrace/backtrace.tex}
    \end{column}
  \end{columns}
\end{frame}

\begin{frame}{Backtraces}{But before that, we need more fields from \typename{caml\_domain\_state}}
  \begin{columns}
    \begin{column}{0.5\textwidth}
      \begin{itemize}
        \item Remember \typename{caml\_domain\_state} the per-domain runtime state?
        \item Some other members are of interest to us now\dots
        \item \localname{backtrace\_active} is used to know if the runtime should collect backtraces
        \item \localname{backtrace\_active} can be set by
          \begin{itemize}
            \item The \funcarg{b} flag in the \funcname{OCAMLRUNPARAM} environment variable
            \item \mintinline{ocaml}{Printexc.record_backtrace : bool -> unit} \footnotemark
          \end{itemize}
      \end{itemize}
    \end{column}
    \begin{column}{0.5\textwidth}
      \begin{listing}
        \mintc[highlightlines={9-11}]{src/ocaml/domain\_state\_backtrace.h}
        \caption{runtime/caml/domain\_state.h}
      \end{listing}
    \end{column}
  \end{columns}
  \footnotetext[1]{https://v2.ocaml.org/api/Printexc.html}
\end{frame}

\newcommand\listCamlRaiseExn[1][]{
  \listasm[firstline=702,lastline=725,#1]{../ocaml/runtime/amd64.S}{runtime/amd64.S:702}}

\begin{frame}{Backtraces}{Walk through \funcname{caml\_raise\_exn}}
  \begin{columns}[T]
    \begin{column}{0.5\textwidth}
      \begin{itemize}
        \item<1-> First checks whether exceptions backtrace should be recorded
        \item<1-> By checking the \funcarg{domain\_state->backtrace\_active} value
        \item<2-> If not, acts as \funcname{raise\_notrace} with \funcname{RESTORE\_EXN\_HANDLER\_OCAML}
          \begin{itemize}
            \item Set \regname{RSP} to \localname{domain\_state->exn\_handler} value
            \item Restore \localname{domain\_state->exn\_handler} from trap
            \item Jump with \funcname{ret} (same effect as using \funcname{pop} and \funcname{jmp})
          \end{itemize}
        \item<3-> Otherwise, switches to the C stack to be able to execute C code
        \item<4-> Calls \funcname{caml\_stash\_backtrace}, a C function provided by the runtime\ignorespaces
          \onslide<6->{, with
            \begin{itemize}
              \item \funcarg{exn}: exception value
              \item \funcarg{pc}: code address of the raising point
              \item \funcarg{sp}: stack address of the raising function
              \item \funcarg{trapsp}: stack address of the next handler
            \end{itemize}
          }
      \end{itemize}
      \bigskip
      \mintocaml[firstline=9,lastline=13,highlightlines=12]{../src/backtrace/backtrace.ml}
    \end{column}
    \begin{column}{0.5\textwidth}
      \raggedleft
      \onslide*<1,3-4>{
        \mintc[firstline=190,lastline=192]{../ocaml/runtime/amd64.S}
        \mintc[firstline=234,lastline=237]{../ocaml/runtime/amd64.S}
      }
      \onslide*<2>{
        \mintc[firstline=190,lastline=192,highlightlines={191-192}]{../ocaml/runtime/amd64.S}
        \mintc[firstline=234,lastline=237,highlightlines={235-237}]{../ocaml/runtime/amd64.S}
      }
      \onslide*<1>{\listCamlRaiseExn[highlightlines=705]{}}%
      \onslide*<2>{\listCamlRaiseExn[highlightlines={707-708}]{}}%
      \onslide*<3>{\listCamlRaiseExn[highlightlines={712-714}]{}}%
      \onslide*<4>{\listCamlRaiseExn[highlightlines={715-720}]{}}%
      \onslide*<5>{\providecommand\step{1}\input{diagrams/backtrace/backtrace.tex}}%
      \onslide*<6>{\providecommand\step{2}\input{diagrams/backtrace/backtrace.tex}}%
    \end{column}
  \end{columns}
\end{frame}

\newcommand\listStashBacktrace[1][]{
  \listc[firstline=91,lastline=120,#1]{../ocaml/runtime/backtrace\_nat.c}{runtime/backtrace\_nat.c:91}}

\begin{frame}{Backtraces}{Introducing \funcname{caml\_stash\_backtrace}}
  \begin{columns}[c]
    \begin{column}{0.5\textwidth}
      \minipage[c][0.66\textheight][s]{\columnwidth}
      \begin{itemize}
        \item<1-> A C function provided by the runtime (specific to native code)
        \item<2-> It scans the stack, between the stack pointer at the raising point up to the next trap, in order to collect the names of the detected function frames
          \onslide*<2>{
            \begin{itemize}
              \item And more information, like line number, column number...
            \end{itemize}
          }
        \item<3-> It uses the same technique and information as the Garbage Collector (GC) uses to scan the values live on the stack
          \onslide*<4>{
            \begin{itemize}
              \item \typename{frame\_descr} are at the core of stack unwinding
            \end{itemize}
          }
      \end{itemize}
      \vfill
      \mintocaml[firstline=9,lastline=13,highlightlines=12]{../src/backtrace/backtrace.ml}
      \endminipage
    \end{column}
    \begin{column}{0.5\textwidth}
      \onslide*<1>{\listStashBacktrace[highlightlines=91]{}}
      \onslide*<2>{\listStashBacktrace[highlightlines={91,117-118}]{}}
      \onslide*<3->{\listStashBacktrace[highlightlines={94,106,109-110}]{}}
    \end{column}
  \end{columns}
\end{frame}

\newcommand\listFrameDescr[1][]{
  \listocaml[firstline=111,lastline=124,#1]{../ocaml/asmcomp/emitaux.ml}{asmcomp/emitaux.ml:111}}
\newcommand\listDebugInfo[1][]{
  \listocaml[firstline=38,lastline=49,#1]{../ocaml/lambda/debuginfo.mli}{lambda/debuginfo.mli:38}}

\begin{frame}{Backtraces}{\typename{frame\_descr} and \typename{frame\_debuginfo} in a nutshell}
  \begin{columns}[c]
    \begin{column}{0.5\textwidth}
      \begin{itemize}
        \item<1-> For every return address in an OCaml program, there is a associated \typename{frame\_descr}
        \item<2-> A \typename{frame\_descr} specifies
        \begin{itemize}
          \item<2-> The size of the function stack frame
          \item<3-> Where live values are on the stack (so that the GC can mark them as live and prevent from being swept)
        \end{itemize}
        \item<4-> When compiled with debug information, \typename{frame\_descr} will be linked to a \typename{frame\_debuginfo}
        \begin{itemize}
          \item<5-> Contains information about the position in the source code (file name, line and column number)
        \end{itemize}
      \end{itemize}
    \end{column}
    \begin{column}{0.5\textwidth}
        \onslide*<1>{\listFrameDescr[highlightlines={118-122}]{}}
        \onslide*<2>{\listFrameDescr[highlightlines={120}]{}}
        \onslide*<3>{\listFrameDescr[highlightlines={121}]{}}
        \onslide*<4->{\listFrameDescr[highlightlines={122}]{}}
        \medskip
        \onslide*<1-3>{\listDebugInfo{}}
        \onslide*<4>{\listDebugInfo[highlightlines={38,49}]{}}
        \onslide*<5>{\listDebugInfo[highlightlines={39-46}]{}}
    \end{column}
  \end{columns}
\end{frame}

\begin{frame}{Backtraces}{Scanning the stack with \typename{frame\_descr}}
  \begin{columns}[c]
    \begin{column}{0.5\textwidth}
      \begin{itemize}
        \item Given the return address of the code that raises the exception by calling \funcname{caml\_raise\_exn} (\funcarg{pc} argument of \funcname{caml\_stash\_backtrace})
        \item And the position of the stack pointer (\funcarg{sp} argument of \funcname{caml\_stash\_backtrace})
        \item The frame size given by \typename{frame\_descr}.\funcarg{frame\_size}, allows to find the end of the previous frame
        \item And the return address of the caller of this function
        \bigskip
        \onslide<3->{
        \item Note that traps (here the reddish \funcname{ExnA}, \funcname{ExnB} and \funcname{ExnC} blocks) belong to the frame that installed them, and so the frame size changes when traps are removed
        }
      \end{itemize}
    \end{column}
    \begin{column}{0.5\textwidth}
      \raggedleft
      \onslide*<1>{\providecommand\step{2}\input{diagrams/backtrace/backtrace.tex}}%
      \onslide*<2>{\providecommand\step{1}\input{diagrams/backtrace/scanstack.tex}}%
      \onslide*<3>{\providecommand\step{2}\input{diagrams/backtrace/scanstack.tex}}%
      \onslide*<4>{\providecommand\step{3}\input{diagrams/backtrace/scanstack.tex}}%
      \onslide*<5>{\providecommand\step{4}\input{diagrams/backtrace/scanstack.tex}}%
      \onslide*<6>{\providecommand\step{5}\input{diagrams/backtrace/scanstack.tex}}%
    \end{column}
  \end{columns}
\end{frame}

% Duplicate of previous-previous slide
\begin{frame}{Backtraces}{Continuing on \funcname{caml\_stash\_backtrace}}
  \begin{columns}[c]
    \begin{column}{0.5\textwidth}
      \minipage[c][0.75\textheight][s]{\columnwidth}
      \begin{itemize}
          \color{gray}
        \item A C function provided by the runtime (specific to native code)
        \item It scans the stack, between the stack pointer at the raising point up to the next trap, in order to collect the names of the detected function frames
        \item It uses the same technique and information as the Garbage Collector (GC) uses to scan the values live on the stack
      \end{itemize}
      \begin{itemize}
        \item For each function frame detected, store a pointer to the \typename{frame\_descr} in the \localname{domain\_state->backtrace\_buffer} array, effectively constructing the backtrace
      \end{itemize}
      \onslide<2->{
        \begin{itemize}
            \smallskip
          \item Constructed backtrace:
            \funcname{definitely\_raise},
            \onslide<3->{\funcname{probably\_raise}}
        \end{itemize}
      }
      \vfill
      \mintocaml[firstline=9,lastline=13,highlightlines=12]{../src/backtrace/backtrace.ml}
      \endminipage
    \end{column}
    \begin{column}{0.5\textwidth}
      \raggedleft
      \onslide*<1>{\listStashBacktrace[highlightlines={114-115}]{}}
      \onslide*<2>{\providecommand\step{3}\input{diagrams/backtrace/backtrace.tex}}%
      \onslide*<3>{\providecommand\step{4}\input{diagrams/backtrace/backtrace.tex}}%
    \end{column}
  \end{columns}
\end{frame}

\begin{frame}{Backtraces}{Back to \funcname{caml\_raise\_exn}}
  \begin{columns}[c]
    \begin{column}{0.5\textwidth}
      \minipage[c][0.66\textheight][s]{\columnwidth}
      \begin{itemize}\color{gray}
        \item Checks wheteher exceptions backtrace should be recorded, by checking the \funcarg{domain\_state->backtrace\_active} value
        \item Switches to the C stack to be able to execute C code
        \item Calls \funcname{caml\_stash\_backtrace}, to collect the backtrace up to the next trap
      \end{itemize}
      \onslide*<2->{
        \begin{itemize}
          \item<2-> Executes the exception handler in \funcname{probably\_raise} with \funcname{RESTORE\_EXN\_HANDLER\_OCAML}
            \begin{itemize}
              \item Set \regname{RSP} to \localname{domain\_state->exn\_handler} value
              \item Restore \localname{domain\_state->exn\_handler} from trap
              \item Jump to the handler code with \funcname{ret}
            \end{itemize}
        \end{itemize}
      }
      \vfill
      \onslide*<1>{\mintocaml[firstline=9,lastline=13,highlightlines=12]{../src/backtrace/backtrace.ml}}
      \onslide*<2->{\mintocaml[firstline=15,lastline=21,highlightlines={19-21}]{../src/backtrace/backtrace.ml}}
      \endminipage
    \end{column}
    \begin{column}{0.5\textwidth}
      \centering
      \onslide*<1>{
        \mintc[firstline=190,lastline=192]{../ocaml/runtime/amd64.S}
        \mintc[firstline=234,lastline=237]{../ocaml/runtime/amd64.S}
      }
      \onslide*<2>{
        \mintc[firstline=190,lastline=192,highlightlines={191-192}]{../ocaml/runtime/amd64.S}
        \mintc[firstline=234,lastline=237,highlightlines={235-237}]{../ocaml/runtime/amd64.S}
      }
      \onslide*<1>{\listCamlRaiseExn[highlightlines=720]{}}
      \onslide*<2>{\listCamlRaiseExn[highlightlines=722]{}}
      \onslide*<3->{\providecommand\step{5}\input{diagrams/backtrace/backtrace.tex}}
    \end{column}
  \end{columns}
\end{frame}

\begin{frame}{Backtraces}{\funcname{caml\_probably\_raise}'s handler doesn't catch \typename{ExnC}}
  \begin{columns}[c]
    \begin{column}{0.45\textwidth}
      \minipage[c][0.75\textheight][s]{\columnwidth}
      \begin{itemize}
        \item<1-> \funcname{match \dots with}\footnotemark pattern matching can also match exceptions
        \item<1-> The CMM of \funcname{probably\_raise} shows that this \funcname{match \dots with} is implemented with a pattern matching and a \funcname{try \dots with}
        \item<1-> But this one only matches on \typename{ExnA} exception
        \item<2-> Since the pattern matching fails to catch the exception, \funcname{reraise} is called with our current \typename{ExnC} exception
        \item<3-> Disassembly shows that \funcname{reraise} is actually a call to \funcname{caml\_reraise\_exn}, which is also an assembly function provided by the runtime
      \end{itemize}
      \vfill
      \mintocaml[firstline=15,lastline=21,highlightlines={18-21}]{../src/backtrace/backtrace.ml}
      \endminipage
    \end{column}
    \begin{column}{0.55\textwidth}
      \centering
      \onslide*<1>{\mintcmm[highlightlines={10-18}]{../src/backtrace/camlBacktrace\_\_probably\_raise\_351.cmm}}
      \onslide*<2>{\listcmm[highlightlines=18]{../src/backtrace/camlBacktrace\_\_probably\_raise_351.cmm}{CMM of probably\_raise}}
      \onslide*<3>{\mintobjdump[firstline=5,lastline=35,highlightlines=31]{../src/backtrace/camlBacktrace\_\_probably\_raise_351.objdump}}
    \end{column}
  \end{columns}
  \only<1>{\footnotetext[1]{https://blog.janestreet.com/pattern-matching-and-exception-handling-unite/}}
\end{frame}

\newcommand\listCamlReraiseExn[1][]{
  \listasm[firstline=727,lastline=734,#1]{../ocaml/runtime/amd64.S}{runtime/amd64.S:727}}

\begin{frame}{Backtraces}{Introducing \funcname{caml\_reraise\_exn}}
  \begin{columns}[c]
    \begin{column}{0.5\textwidth}
      \minipage[c][0.66\textheight][s]{\columnwidth}
      \begin{itemize}
        \item<1-> The exception have to be reraised to the next handler
        \item<2-> First, checks if the backtrace should be collected with \funcarg{domain\_state->backtrace\_active}
        \only<3>{
        \begin{itemize}
            \item If not, behaves like a \funcname{raise\_notrace} with \funcname{RESTORE\_EXN\_HANDLER\_OCAML}
        \end{itemize}
        }
        \item<4-> Jumps into \funcname{caml\_raise\_exn}
        \item<5-> Calls \funcname{caml\_stash\_backtrace} again, to scan the next part of the stack: from the trap just pop-ed to the next trap
      \end{itemize}
      \vfill
      \mintocaml[firstline=15,lastline=21,highlightlines={18-21}]{../src/backtrace/backtrace.ml}
      \endminipage
    \end{column}
    \begin{column}{0.5\textwidth}
      \raggedleft
      \onslide*<1>{\listCamlReraiseExn{}}
      \onslide*<2>{\listCamlReraiseExn[highlightlines=729]{}}
      \onslide*<3>{
        \mintc[firstline=190,lastline=192,highlightlines={191-192}]{../ocaml/runtime/amd64.S}
        \mintc[firstline=234,lastline=237,highlightlines={235-237}]{../ocaml/runtime/amd64.S}
        \medskip
        \listCamlReraiseExn[highlightlines=731]{}
      }
      \onslide*<4-5>{
        % TODO refactor with \listCamlReraiseExn
        \mintasm[firstline=727,lastline=734,highlightlines=730]{../ocaml/runtime/amd64.S}
        \medskip
      }
      \onslide*<4>{\listCamlRaiseExn[highlightlines=711]{}}
      \onslide*<5>{\listCamlRaiseExn[highlightlines=720]{}}
    \end{column}
  \end{columns}
\end{frame}

\begin{frame}{Backtraces}{Backtrace collection continues}
  \begin{columns}[c]
    \begin{column}{0.55\textwidth}
      \minipage[c][0.75\textheight][s]{\columnwidth}
      \begin{itemize}
        \item<1-> \funcname{caml\_stash\_backtrace} scans the older part of the stack, up to the next trap
        \item<6-> Controls jumps to the nested exception handler in \funcname{maybe\_raise}, which matches only on \typename{ExnB}
        \item<7-> \funcname{caml\_reraise\_exn} is called again, backtrace collection resumes with \funcname{caml\_stash\_backtrace}
      \end{itemize}
      \medskip
      \begin{itemize}
        \item<2-> Constructed backtrace:
        \begin{itemize}
          \item <2-> \funcname{definitely\_raise}
          \item <2-> \funcname{probably\_raise}
          \item <3-> \funcname{probably\_raise} (re-raise)
          \item <4-> \funcname{maybe\_raise}
          \item <5-> \funcname{main}
          \item <8-> \funcname{main} (re-raise)
        \end{itemize}
      \end{itemize}
      \vfill
      \onslide*<1-5>{\mintocaml[firstline=15,lastline=21]{../src/backtrace/backtrace.ml}}
      \onslide*<6>{\mintocaml[firstline=28,lastline=34,highlightlines=31]{../src/backtrace/backtrace.ml}}
      \onslide*<7->{\mintocaml[firstline=28,lastline=34,highlightlines=33]{../src/backtrace/backtrace.ml}}
      \endminipage
    \end{column}
    \begin{column}{0.45\textwidth}
      \raggedleft
      \onslide*<1>{\providecommand\step{6}\input{diagrams/backtrace/backtrace.tex}}%
      \onslide*<2>{\providecommand\step{6}\input{diagrams/backtrace/backtrace.tex}}%
      \onslide*<3>{\providecommand\step{7}\input{diagrams/backtrace/backtrace.tex}}%
      \onslide*<4>{\providecommand\step{8}\input{diagrams/backtrace/backtrace.tex}}%
      \onslide*<5>{\providecommand\step{9}\input{diagrams/backtrace/backtrace.tex}}%
      \onslide*<6>{\providecommand\step{10}\input{diagrams/backtrace/backtrace.tex}}%
      \onslide*<7>{\providecommand\step{11}\input{diagrams/backtrace/backtrace.tex}}%
      \onslide*<8>{\providecommand\step{12}\input{diagrams/backtrace/backtrace.tex}}%
      \onslide*<9>{\providecommand\step{13}\input{diagrams/backtrace/backtrace.tex}}%
    \end{column}
  \end{columns}
\end{frame}

\begin{frame}{Backtraces}{Printing the backtrace}
  \begin{columns}[c]
    \begin{column}{0.45\textwidth}
      \minipage[c][0.66\textheight][s]{\columnwidth}
      \begin{itemize}
        \item<1-> The exception backtrace can now be printed to \funcarg{stdout} with \funcname{Printexc.print_backtrace}
        \item<2-> First the exception backtrace is retrieved into an OCaml array with \funcname{get_raw_backtrace }
        \item<3-> Which is implemented as the \funcname{caml_get_exception_raw_backtrace} C function in the runtime
        \item<4-> And finally printed with \funcname{print_exception_backtrace}
      \end{itemize}
      \vfill
      \mintocaml[firstline=28,lastline=34,highlightlines=33]{../src/backtrace/backtrace.ml}
      \endminipage
    \end{column}
    \begin{column}{0.55\textwidth}
      \onslide*<1,2>{
        \mintocaml[firstline=100,lastline=101]{../ocaml/stdlib/printexc.ml}
      }
      \only<1>{
        \listocaml[firstline=170,lastline=172]{../ocaml/stdlib/printexc.ml}{stdlib/printexc.ml:170}
      }
      \only<2>{
        \listocaml[firstline=170,lastline=172,highlightlines=172]{../ocaml/stdlib/printexc.ml}{stdlib/printexc.ml:170}
      }
      \only<3>{
        \mintc[firstline=154,lastline=156]{../ocaml/runtime/backtrace.c}
        \listc[firstline=171,lastline=192,highlightlines={184-187}]{../ocaml/runtime/backtrace.c}{runtime/backtrace.c:154}
      }
      \only<4>{
        \listocaml[firstline=155,lastline=165]{../ocaml/stdlib/printexc.ml}{stdlib/printexc.ml:55}
      }
    \end{column}
  \end{columns}
\end{frame}


\subsection{The multiple kinds of raise}
\frameSubsection{}{}

\begin{frame}{Backtraces}{When backtraces are not needed}
  \begin{columns}[c]
    \begin{column}{0.45\textwidth}
      \minipage[c][0.66\textheight][s]{\columnwidth}
      \begin{itemize}
        \item<1-> What if the backtrace for an exception is never needed?
        \item<2-> It's possible to raise the exception explicitly with \funcname{raise\_notrace} to make raising as cheap as possible
        \item<3-> An example of use in the \funcname{dynlink} library of the OCaml compiler
      \end{itemize}
      \vfill
      \onslide<3->{
      \listocaml[firstline=205,lastline=209,highlightlines=207]{../ocaml/otherlibs/dynlink/dynlink\_compilerlibs/misc.ml}{otherlibs/dynlink/dynlink\_compilerlibs/misc.ml:205}
      }
      \endminipage
    \end{column}
    \begin{column}{0.55\textwidth}
    \only<4>{
      \mintcmm[highlightlines=3]{../src/notrace/camlNotrace\_\_fun_347.cmm}
      \listcmm{../src/notrace/camlNotrace\_\_all\_somes\_327.cmm}{all\_somes}
    }
    \onslide*<5->{
      \listobjdump[highlightlines={6-9}]{../src/notrace/camlNotrace\_\_fun_347.objdump}{anonymous function from \funcname{all\_somes}}
    }
    \end{column}
  \end{columns}
\end{frame}

\begin{frame}{Backtraces}{Summary table of the multiple kinds of raise}
  \begin{itemize}
    \item When debugging information is enabled and if the backtrace of an exception isn't needed, it's possible to raise with \funcname{raise\_notrace} for performance
    \item When debugging information is \emph{not} enabled, the compiler optimises \funcname{raise} calls to inline assembly (same effect as using \funcname{raise\_notrace})
  \end{itemize}
  \bigskip
  \centering
  \begin{tabular}{| l | c | c | c | c |}
    \hline
    \parbox{10em}{Debug information \\ (\funcarg{-g} flag)}  & \multicolumn{2}{|c|}{Disabled}                                     & \multicolumn{2}{|c|}{Enabled} \\
    \hline
    Raise kind                                               & \funcname{raise} & \funcname{raise\_notrace}                       & \funcname{raise} & \funcname{raise\_notrace} \\
    \hline
    Compilation                                              & \parbox{8em}{inline assembly\\ (optimization)} & inline assembly   & \funcname{caml\_raise\_exn} & inline assembly \\
    \hline
  \end{tabular}
\end{frame}


%
% Takeaway #5
%
\subsection*{Takeaway \#5}
\frameSubsectionTakeaway{}
\againframe<5>{takeaway}
}%
      \onslide*<3>{\providecommand\step{7}%
% Backtraces
%
\subsection{One advantage of raising with debugging information}
\frameSubsection{}{}

\begin{frame}{Backtraces}{Debugging information \& source code}
  \begin{columns}[c]
    \begin{column}{0.5\textwidth}
      \begin{itemize}
        \item Raising an exception is fun and games, but how do we know where the exception originated? $\rightarrow$ With the backtrace associated with the exception!
        \item In order to get a backtrace from the point where an exception was raised, it's necessary to compile with debugging information enabled (\funcarg{-g} flag for the compiler)
        \item Same example as in the `Nested handlers' section
      \end{itemize}
    \end{column}
    \begin{column}{0.5\textwidth}
      \listocaml[firstline=9,lastline=34]{../src/backtrace/backtrace.ml}{backtrace/backtrace.ml}
    \end{column}
  \end{columns}
\end{frame}

\begin{frame}{Backtraces}{CMM and disassembly of \funcname{definitely\_raise}}
  \begin{columns}[c]
    \begin{column}{0.5\textwidth}
      \minipage[c][0.66\textheight][s]{\columnwidth}
      \begin{itemize}
        \item<1-> An effect of compiling with debugging information (\funcarg{-g}): the CMM no longer uses \funcname{raise\_notrace} to raise the exception but \funcname{raise} instead
        \item<2-> Disassembly shows that CMM \funcname{raise} is actually a call to \funcname{caml\_raise\_exn}, a function in assembler provided by the runtime
      \end{itemize}
      \vfill
      \mintocaml[firstline=9,lastline=13,highlightlines=12]{../src/backtrace/backtrace.ml}
      \endminipage
    \end{column}
    \begin{column}{0.5\textwidth}
      \onslide*<1>{
        \mintcmm[highlightlines=5]{../src/backtrace/camlBacktrace\_\_definitely\_raise\_347.cmm}
      }
      \onslide*<2>{
        \mintobjdump[firstline=2,highlightlines=14]{../src/backtrace/camlBacktrace\_\_definitely\_raise\_347.objdump}
      }
    \end{column}
  \end{columns}
\end{frame}

\begin{frame}{Backtraces}{Introducing \funcname{caml\_raise\_exn}}
  \begin{columns}[c]
    \begin{column}{0.5\textwidth}
      \minipage[c][0.66\textheight][s]{\columnwidth}
      \onslide*<1>{
        \begin{itemize}
          \item Raising the exception is performed using \funcname{caml\_raise\_exn} which is
            \begin{itemize}
              \item An assembly function provided by the runtime
              \item Architecture specific
            \end{itemize}
          \item Let's see what it does differently from \funcname{raise\_notrace}\dots
          \item We are about to raise \typename{ExnC} with \funcname{caml\_raise\_exn} from \funcname{definitely\_raise}
        \end{itemize}
      }
      \onslide*<2>{
        \mintobjdump[firstline=2,highlightlines=14]{../src/backtrace/camlBacktrace\_\_definitely\_raise\_347.objdump}
      }
      \vfill
      \mintocaml[firstline=9,lastline=13,highlightlines=12]{../src/backtrace/backtrace.ml}
      \endminipage
    \end{column}
    \begin{column}{0.5\textwidth}
      \centering
      \providecommand\step{1}\input{diagrams/backtrace/backtrace.tex}
    \end{column}
  \end{columns}
\end{frame}

\begin{frame}{Backtraces}{But before that, we need more fields from \typename{caml\_domain\_state}}
  \begin{columns}
    \begin{column}{0.5\textwidth}
      \begin{itemize}
        \item Remember \typename{caml\_domain\_state} the per-domain runtime state?
        \item Some other members are of interest to us now\dots
        \item \localname{backtrace\_active} is used to know if the runtime should collect backtraces
        \item \localname{backtrace\_active} can be set by
          \begin{itemize}
            \item The \funcarg{b} flag in the \funcname{OCAMLRUNPARAM} environment variable
            \item \mintinline{ocaml}{Printexc.record_backtrace : bool -> unit} \footnotemark
          \end{itemize}
      \end{itemize}
    \end{column}
    \begin{column}{0.5\textwidth}
      \begin{listing}
        \mintc[highlightlines={9-11}]{src/ocaml/domain\_state\_backtrace.h}
        \caption{runtime/caml/domain\_state.h}
      \end{listing}
    \end{column}
  \end{columns}
  \footnotetext[1]{https://v2.ocaml.org/api/Printexc.html}
\end{frame}

\newcommand\listCamlRaiseExn[1][]{
  \listasm[firstline=702,lastline=725,#1]{../ocaml/runtime/amd64.S}{runtime/amd64.S:702}}

\begin{frame}{Backtraces}{Walk through \funcname{caml\_raise\_exn}}
  \begin{columns}[T]
    \begin{column}{0.5\textwidth}
      \begin{itemize}
        \item<1-> First checks whether exceptions backtrace should be recorded
        \item<1-> By checking the \funcarg{domain\_state->backtrace\_active} value
        \item<2-> If not, acts as \funcname{raise\_notrace} with \funcname{RESTORE\_EXN\_HANDLER\_OCAML}
          \begin{itemize}
            \item Set \regname{RSP} to \localname{domain\_state->exn\_handler} value
            \item Restore \localname{domain\_state->exn\_handler} from trap
            \item Jump with \funcname{ret} (same effect as using \funcname{pop} and \funcname{jmp})
          \end{itemize}
        \item<3-> Otherwise, switches to the C stack to be able to execute C code
        \item<4-> Calls \funcname{caml\_stash\_backtrace}, a C function provided by the runtime\ignorespaces
          \onslide<6->{, with
            \begin{itemize}
              \item \funcarg{exn}: exception value
              \item \funcarg{pc}: code address of the raising point
              \item \funcarg{sp}: stack address of the raising function
              \item \funcarg{trapsp}: stack address of the next handler
            \end{itemize}
          }
      \end{itemize}
      \bigskip
      \mintocaml[firstline=9,lastline=13,highlightlines=12]{../src/backtrace/backtrace.ml}
    \end{column}
    \begin{column}{0.5\textwidth}
      \raggedleft
      \onslide*<1,3-4>{
        \mintc[firstline=190,lastline=192]{../ocaml/runtime/amd64.S}
        \mintc[firstline=234,lastline=237]{../ocaml/runtime/amd64.S}
      }
      \onslide*<2>{
        \mintc[firstline=190,lastline=192,highlightlines={191-192}]{../ocaml/runtime/amd64.S}
        \mintc[firstline=234,lastline=237,highlightlines={235-237}]{../ocaml/runtime/amd64.S}
      }
      \onslide*<1>{\listCamlRaiseExn[highlightlines=705]{}}%
      \onslide*<2>{\listCamlRaiseExn[highlightlines={707-708}]{}}%
      \onslide*<3>{\listCamlRaiseExn[highlightlines={712-714}]{}}%
      \onslide*<4>{\listCamlRaiseExn[highlightlines={715-720}]{}}%
      \onslide*<5>{\providecommand\step{1}\input{diagrams/backtrace/backtrace.tex}}%
      \onslide*<6>{\providecommand\step{2}\input{diagrams/backtrace/backtrace.tex}}%
    \end{column}
  \end{columns}
\end{frame}

\newcommand\listStashBacktrace[1][]{
  \listc[firstline=91,lastline=120,#1]{../ocaml/runtime/backtrace\_nat.c}{runtime/backtrace\_nat.c:91}}

\begin{frame}{Backtraces}{Introducing \funcname{caml\_stash\_backtrace}}
  \begin{columns}[c]
    \begin{column}{0.5\textwidth}
      \minipage[c][0.66\textheight][s]{\columnwidth}
      \begin{itemize}
        \item<1-> A C function provided by the runtime (specific to native code)
        \item<2-> It scans the stack, between the stack pointer at the raising point up to the next trap, in order to collect the names of the detected function frames
          \onslide*<2>{
            \begin{itemize}
              \item And more information, like line number, column number...
            \end{itemize}
          }
        \item<3-> It uses the same technique and information as the Garbage Collector (GC) uses to scan the values live on the stack
          \onslide*<4>{
            \begin{itemize}
              \item \typename{frame\_descr} are at the core of stack unwinding
            \end{itemize}
          }
      \end{itemize}
      \vfill
      \mintocaml[firstline=9,lastline=13,highlightlines=12]{../src/backtrace/backtrace.ml}
      \endminipage
    \end{column}
    \begin{column}{0.5\textwidth}
      \onslide*<1>{\listStashBacktrace[highlightlines=91]{}}
      \onslide*<2>{\listStashBacktrace[highlightlines={91,117-118}]{}}
      \onslide*<3->{\listStashBacktrace[highlightlines={94,106,109-110}]{}}
    \end{column}
  \end{columns}
\end{frame}

\newcommand\listFrameDescr[1][]{
  \listocaml[firstline=111,lastline=124,#1]{../ocaml/asmcomp/emitaux.ml}{asmcomp/emitaux.ml:111}}
\newcommand\listDebugInfo[1][]{
  \listocaml[firstline=38,lastline=49,#1]{../ocaml/lambda/debuginfo.mli}{lambda/debuginfo.mli:38}}

\begin{frame}{Backtraces}{\typename{frame\_descr} and \typename{frame\_debuginfo} in a nutshell}
  \begin{columns}[c]
    \begin{column}{0.5\textwidth}
      \begin{itemize}
        \item<1-> For every return address in an OCaml program, there is a associated \typename{frame\_descr}
        \item<2-> A \typename{frame\_descr} specifies
        \begin{itemize}
          \item<2-> The size of the function stack frame
          \item<3-> Where live values are on the stack (so that the GC can mark them as live and prevent from being swept)
        \end{itemize}
        \item<4-> When compiled with debug information, \typename{frame\_descr} will be linked to a \typename{frame\_debuginfo}
        \begin{itemize}
          \item<5-> Contains information about the position in the source code (file name, line and column number)
        \end{itemize}
      \end{itemize}
    \end{column}
    \begin{column}{0.5\textwidth}
        \onslide*<1>{\listFrameDescr[highlightlines={118-122}]{}}
        \onslide*<2>{\listFrameDescr[highlightlines={120}]{}}
        \onslide*<3>{\listFrameDescr[highlightlines={121}]{}}
        \onslide*<4->{\listFrameDescr[highlightlines={122}]{}}
        \medskip
        \onslide*<1-3>{\listDebugInfo{}}
        \onslide*<4>{\listDebugInfo[highlightlines={38,49}]{}}
        \onslide*<5>{\listDebugInfo[highlightlines={39-46}]{}}
    \end{column}
  \end{columns}
\end{frame}

\begin{frame}{Backtraces}{Scanning the stack with \typename{frame\_descr}}
  \begin{columns}[c]
    \begin{column}{0.5\textwidth}
      \begin{itemize}
        \item Given the return address of the code that raises the exception by calling \funcname{caml\_raise\_exn} (\funcarg{pc} argument of \funcname{caml\_stash\_backtrace})
        \item And the position of the stack pointer (\funcarg{sp} argument of \funcname{caml\_stash\_backtrace})
        \item The frame size given by \typename{frame\_descr}.\funcarg{frame\_size}, allows to find the end of the previous frame
        \item And the return address of the caller of this function
        \bigskip
        \onslide<3->{
        \item Note that traps (here the reddish \funcname{ExnA}, \funcname{ExnB} and \funcname{ExnC} blocks) belong to the frame that installed them, and so the frame size changes when traps are removed
        }
      \end{itemize}
    \end{column}
    \begin{column}{0.5\textwidth}
      \raggedleft
      \onslide*<1>{\providecommand\step{2}\input{diagrams/backtrace/backtrace.tex}}%
      \onslide*<2>{\providecommand\step{1}\input{diagrams/backtrace/scanstack.tex}}%
      \onslide*<3>{\providecommand\step{2}\input{diagrams/backtrace/scanstack.tex}}%
      \onslide*<4>{\providecommand\step{3}\input{diagrams/backtrace/scanstack.tex}}%
      \onslide*<5>{\providecommand\step{4}\input{diagrams/backtrace/scanstack.tex}}%
      \onslide*<6>{\providecommand\step{5}\input{diagrams/backtrace/scanstack.tex}}%
    \end{column}
  \end{columns}
\end{frame}

% Duplicate of previous-previous slide
\begin{frame}{Backtraces}{Continuing on \funcname{caml\_stash\_backtrace}}
  \begin{columns}[c]
    \begin{column}{0.5\textwidth}
      \minipage[c][0.75\textheight][s]{\columnwidth}
      \begin{itemize}
          \color{gray}
        \item A C function provided by the runtime (specific to native code)
        \item It scans the stack, between the stack pointer at the raising point up to the next trap, in order to collect the names of the detected function frames
        \item It uses the same technique and information as the Garbage Collector (GC) uses to scan the values live on the stack
      \end{itemize}
      \begin{itemize}
        \item For each function frame detected, store a pointer to the \typename{frame\_descr} in the \localname{domain\_state->backtrace\_buffer} array, effectively constructing the backtrace
      \end{itemize}
      \onslide<2->{
        \begin{itemize}
            \smallskip
          \item Constructed backtrace:
            \funcname{definitely\_raise},
            \onslide<3->{\funcname{probably\_raise}}
        \end{itemize}
      }
      \vfill
      \mintocaml[firstline=9,lastline=13,highlightlines=12]{../src/backtrace/backtrace.ml}
      \endminipage
    \end{column}
    \begin{column}{0.5\textwidth}
      \raggedleft
      \onslide*<1>{\listStashBacktrace[highlightlines={114-115}]{}}
      \onslide*<2>{\providecommand\step{3}\input{diagrams/backtrace/backtrace.tex}}%
      \onslide*<3>{\providecommand\step{4}\input{diagrams/backtrace/backtrace.tex}}%
    \end{column}
  \end{columns}
\end{frame}

\begin{frame}{Backtraces}{Back to \funcname{caml\_raise\_exn}}
  \begin{columns}[c]
    \begin{column}{0.5\textwidth}
      \minipage[c][0.66\textheight][s]{\columnwidth}
      \begin{itemize}\color{gray}
        \item Checks wheteher exceptions backtrace should be recorded, by checking the \funcarg{domain\_state->backtrace\_active} value
        \item Switches to the C stack to be able to execute C code
        \item Calls \funcname{caml\_stash\_backtrace}, to collect the backtrace up to the next trap
      \end{itemize}
      \onslide*<2->{
        \begin{itemize}
          \item<2-> Executes the exception handler in \funcname{probably\_raise} with \funcname{RESTORE\_EXN\_HANDLER\_OCAML}
            \begin{itemize}
              \item Set \regname{RSP} to \localname{domain\_state->exn\_handler} value
              \item Restore \localname{domain\_state->exn\_handler} from trap
              \item Jump to the handler code with \funcname{ret}
            \end{itemize}
        \end{itemize}
      }
      \vfill
      \onslide*<1>{\mintocaml[firstline=9,lastline=13,highlightlines=12]{../src/backtrace/backtrace.ml}}
      \onslide*<2->{\mintocaml[firstline=15,lastline=21,highlightlines={19-21}]{../src/backtrace/backtrace.ml}}
      \endminipage
    \end{column}
    \begin{column}{0.5\textwidth}
      \centering
      \onslide*<1>{
        \mintc[firstline=190,lastline=192]{../ocaml/runtime/amd64.S}
        \mintc[firstline=234,lastline=237]{../ocaml/runtime/amd64.S}
      }
      \onslide*<2>{
        \mintc[firstline=190,lastline=192,highlightlines={191-192}]{../ocaml/runtime/amd64.S}
        \mintc[firstline=234,lastline=237,highlightlines={235-237}]{../ocaml/runtime/amd64.S}
      }
      \onslide*<1>{\listCamlRaiseExn[highlightlines=720]{}}
      \onslide*<2>{\listCamlRaiseExn[highlightlines=722]{}}
      \onslide*<3->{\providecommand\step{5}\input{diagrams/backtrace/backtrace.tex}}
    \end{column}
  \end{columns}
\end{frame}

\begin{frame}{Backtraces}{\funcname{caml\_probably\_raise}'s handler doesn't catch \typename{ExnC}}
  \begin{columns}[c]
    \begin{column}{0.45\textwidth}
      \minipage[c][0.75\textheight][s]{\columnwidth}
      \begin{itemize}
        \item<1-> \funcname{match \dots with}\footnotemark pattern matching can also match exceptions
        \item<1-> The CMM of \funcname{probably\_raise} shows that this \funcname{match \dots with} is implemented with a pattern matching and a \funcname{try \dots with}
        \item<1-> But this one only matches on \typename{ExnA} exception
        \item<2-> Since the pattern matching fails to catch the exception, \funcname{reraise} is called with our current \typename{ExnC} exception
        \item<3-> Disassembly shows that \funcname{reraise} is actually a call to \funcname{caml\_reraise\_exn}, which is also an assembly function provided by the runtime
      \end{itemize}
      \vfill
      \mintocaml[firstline=15,lastline=21,highlightlines={18-21}]{../src/backtrace/backtrace.ml}
      \endminipage
    \end{column}
    \begin{column}{0.55\textwidth}
      \centering
      \onslide*<1>{\mintcmm[highlightlines={10-18}]{../src/backtrace/camlBacktrace\_\_probably\_raise\_351.cmm}}
      \onslide*<2>{\listcmm[highlightlines=18]{../src/backtrace/camlBacktrace\_\_probably\_raise_351.cmm}{CMM of probably\_raise}}
      \onslide*<3>{\mintobjdump[firstline=5,lastline=35,highlightlines=31]{../src/backtrace/camlBacktrace\_\_probably\_raise_351.objdump}}
    \end{column}
  \end{columns}
  \only<1>{\footnotetext[1]{https://blog.janestreet.com/pattern-matching-and-exception-handling-unite/}}
\end{frame}

\newcommand\listCamlReraiseExn[1][]{
  \listasm[firstline=727,lastline=734,#1]{../ocaml/runtime/amd64.S}{runtime/amd64.S:727}}

\begin{frame}{Backtraces}{Introducing \funcname{caml\_reraise\_exn}}
  \begin{columns}[c]
    \begin{column}{0.5\textwidth}
      \minipage[c][0.66\textheight][s]{\columnwidth}
      \begin{itemize}
        \item<1-> The exception have to be reraised to the next handler
        \item<2-> First, checks if the backtrace should be collected with \funcarg{domain\_state->backtrace\_active}
        \only<3>{
        \begin{itemize}
            \item If not, behaves like a \funcname{raise\_notrace} with \funcname{RESTORE\_EXN\_HANDLER\_OCAML}
        \end{itemize}
        }
        \item<4-> Jumps into \funcname{caml\_raise\_exn}
        \item<5-> Calls \funcname{caml\_stash\_backtrace} again, to scan the next part of the stack: from the trap just pop-ed to the next trap
      \end{itemize}
      \vfill
      \mintocaml[firstline=15,lastline=21,highlightlines={18-21}]{../src/backtrace/backtrace.ml}
      \endminipage
    \end{column}
    \begin{column}{0.5\textwidth}
      \raggedleft
      \onslide*<1>{\listCamlReraiseExn{}}
      \onslide*<2>{\listCamlReraiseExn[highlightlines=729]{}}
      \onslide*<3>{
        \mintc[firstline=190,lastline=192,highlightlines={191-192}]{../ocaml/runtime/amd64.S}
        \mintc[firstline=234,lastline=237,highlightlines={235-237}]{../ocaml/runtime/amd64.S}
        \medskip
        \listCamlReraiseExn[highlightlines=731]{}
      }
      \onslide*<4-5>{
        % TODO refactor with \listCamlReraiseExn
        \mintasm[firstline=727,lastline=734,highlightlines=730]{../ocaml/runtime/amd64.S}
        \medskip
      }
      \onslide*<4>{\listCamlRaiseExn[highlightlines=711]{}}
      \onslide*<5>{\listCamlRaiseExn[highlightlines=720]{}}
    \end{column}
  \end{columns}
\end{frame}

\begin{frame}{Backtraces}{Backtrace collection continues}
  \begin{columns}[c]
    \begin{column}{0.55\textwidth}
      \minipage[c][0.75\textheight][s]{\columnwidth}
      \begin{itemize}
        \item<1-> \funcname{caml\_stash\_backtrace} scans the older part of the stack, up to the next trap
        \item<6-> Controls jumps to the nested exception handler in \funcname{maybe\_raise}, which matches only on \typename{ExnB}
        \item<7-> \funcname{caml\_reraise\_exn} is called again, backtrace collection resumes with \funcname{caml\_stash\_backtrace}
      \end{itemize}
      \medskip
      \begin{itemize}
        \item<2-> Constructed backtrace:
        \begin{itemize}
          \item <2-> \funcname{definitely\_raise}
          \item <2-> \funcname{probably\_raise}
          \item <3-> \funcname{probably\_raise} (re-raise)
          \item <4-> \funcname{maybe\_raise}
          \item <5-> \funcname{main}
          \item <8-> \funcname{main} (re-raise)
        \end{itemize}
      \end{itemize}
      \vfill
      \onslide*<1-5>{\mintocaml[firstline=15,lastline=21]{../src/backtrace/backtrace.ml}}
      \onslide*<6>{\mintocaml[firstline=28,lastline=34,highlightlines=31]{../src/backtrace/backtrace.ml}}
      \onslide*<7->{\mintocaml[firstline=28,lastline=34,highlightlines=33]{../src/backtrace/backtrace.ml}}
      \endminipage
    \end{column}
    \begin{column}{0.45\textwidth}
      \raggedleft
      \onslide*<1>{\providecommand\step{6}\input{diagrams/backtrace/backtrace.tex}}%
      \onslide*<2>{\providecommand\step{6}\input{diagrams/backtrace/backtrace.tex}}%
      \onslide*<3>{\providecommand\step{7}\input{diagrams/backtrace/backtrace.tex}}%
      \onslide*<4>{\providecommand\step{8}\input{diagrams/backtrace/backtrace.tex}}%
      \onslide*<5>{\providecommand\step{9}\input{diagrams/backtrace/backtrace.tex}}%
      \onslide*<6>{\providecommand\step{10}\input{diagrams/backtrace/backtrace.tex}}%
      \onslide*<7>{\providecommand\step{11}\input{diagrams/backtrace/backtrace.tex}}%
      \onslide*<8>{\providecommand\step{12}\input{diagrams/backtrace/backtrace.tex}}%
      \onslide*<9>{\providecommand\step{13}\input{diagrams/backtrace/backtrace.tex}}%
    \end{column}
  \end{columns}
\end{frame}

\begin{frame}{Backtraces}{Printing the backtrace}
  \begin{columns}[c]
    \begin{column}{0.45\textwidth}
      \minipage[c][0.66\textheight][s]{\columnwidth}
      \begin{itemize}
        \item<1-> The exception backtrace can now be printed to \funcarg{stdout} with \funcname{Printexc.print_backtrace}
        \item<2-> First the exception backtrace is retrieved into an OCaml array with \funcname{get_raw_backtrace }
        \item<3-> Which is implemented as the \funcname{caml_get_exception_raw_backtrace} C function in the runtime
        \item<4-> And finally printed with \funcname{print_exception_backtrace}
      \end{itemize}
      \vfill
      \mintocaml[firstline=28,lastline=34,highlightlines=33]{../src/backtrace/backtrace.ml}
      \endminipage
    \end{column}
    \begin{column}{0.55\textwidth}
      \onslide*<1,2>{
        \mintocaml[firstline=100,lastline=101]{../ocaml/stdlib/printexc.ml}
      }
      \only<1>{
        \listocaml[firstline=170,lastline=172]{../ocaml/stdlib/printexc.ml}{stdlib/printexc.ml:170}
      }
      \only<2>{
        \listocaml[firstline=170,lastline=172,highlightlines=172]{../ocaml/stdlib/printexc.ml}{stdlib/printexc.ml:170}
      }
      \only<3>{
        \mintc[firstline=154,lastline=156]{../ocaml/runtime/backtrace.c}
        \listc[firstline=171,lastline=192,highlightlines={184-187}]{../ocaml/runtime/backtrace.c}{runtime/backtrace.c:154}
      }
      \only<4>{
        \listocaml[firstline=155,lastline=165]{../ocaml/stdlib/printexc.ml}{stdlib/printexc.ml:55}
      }
    \end{column}
  \end{columns}
\end{frame}


\subsection{The multiple kinds of raise}
\frameSubsection{}{}

\begin{frame}{Backtraces}{When backtraces are not needed}
  \begin{columns}[c]
    \begin{column}{0.45\textwidth}
      \minipage[c][0.66\textheight][s]{\columnwidth}
      \begin{itemize}
        \item<1-> What if the backtrace for an exception is never needed?
        \item<2-> It's possible to raise the exception explicitly with \funcname{raise\_notrace} to make raising as cheap as possible
        \item<3-> An example of use in the \funcname{dynlink} library of the OCaml compiler
      \end{itemize}
      \vfill
      \onslide<3->{
      \listocaml[firstline=205,lastline=209,highlightlines=207]{../ocaml/otherlibs/dynlink/dynlink\_compilerlibs/misc.ml}{otherlibs/dynlink/dynlink\_compilerlibs/misc.ml:205}
      }
      \endminipage
    \end{column}
    \begin{column}{0.55\textwidth}
    \only<4>{
      \mintcmm[highlightlines=3]{../src/notrace/camlNotrace\_\_fun_347.cmm}
      \listcmm{../src/notrace/camlNotrace\_\_all\_somes\_327.cmm}{all\_somes}
    }
    \onslide*<5->{
      \listobjdump[highlightlines={6-9}]{../src/notrace/camlNotrace\_\_fun_347.objdump}{anonymous function from \funcname{all\_somes}}
    }
    \end{column}
  \end{columns}
\end{frame}

\begin{frame}{Backtraces}{Summary table of the multiple kinds of raise}
  \begin{itemize}
    \item When debugging information is enabled and if the backtrace of an exception isn't needed, it's possible to raise with \funcname{raise\_notrace} for performance
    \item When debugging information is \emph{not} enabled, the compiler optimises \funcname{raise} calls to inline assembly (same effect as using \funcname{raise\_notrace})
  \end{itemize}
  \bigskip
  \centering
  \begin{tabular}{| l | c | c | c | c |}
    \hline
    \parbox{10em}{Debug information \\ (\funcarg{-g} flag)}  & \multicolumn{2}{|c|}{Disabled}                                     & \multicolumn{2}{|c|}{Enabled} \\
    \hline
    Raise kind                                               & \funcname{raise} & \funcname{raise\_notrace}                       & \funcname{raise} & \funcname{raise\_notrace} \\
    \hline
    Compilation                                              & \parbox{8em}{inline assembly\\ (optimization)} & inline assembly   & \funcname{caml\_raise\_exn} & inline assembly \\
    \hline
  \end{tabular}
\end{frame}


%
% Takeaway #5
%
\subsection*{Takeaway \#5}
\frameSubsectionTakeaway{}
\againframe<5>{takeaway}
}%
      \onslide*<4>{\providecommand\step{8}%
% Backtraces
%
\subsection{One advantage of raising with debugging information}
\frameSubsection{}{}

\begin{frame}{Backtraces}{Debugging information \& source code}
  \begin{columns}[c]
    \begin{column}{0.5\textwidth}
      \begin{itemize}
        \item Raising an exception is fun and games, but how do we know where the exception originated? $\rightarrow$ With the backtrace associated with the exception!
        \item In order to get a backtrace from the point where an exception was raised, it's necessary to compile with debugging information enabled (\funcarg{-g} flag for the compiler)
        \item Same example as in the `Nested handlers' section
      \end{itemize}
    \end{column}
    \begin{column}{0.5\textwidth}
      \listocaml[firstline=9,lastline=34]{../src/backtrace/backtrace.ml}{backtrace/backtrace.ml}
    \end{column}
  \end{columns}
\end{frame}

\begin{frame}{Backtraces}{CMM and disassembly of \funcname{definitely\_raise}}
  \begin{columns}[c]
    \begin{column}{0.5\textwidth}
      \minipage[c][0.66\textheight][s]{\columnwidth}
      \begin{itemize}
        \item<1-> An effect of compiling with debugging information (\funcarg{-g}): the CMM no longer uses \funcname{raise\_notrace} to raise the exception but \funcname{raise} instead
        \item<2-> Disassembly shows that CMM \funcname{raise} is actually a call to \funcname{caml\_raise\_exn}, a function in assembler provided by the runtime
      \end{itemize}
      \vfill
      \mintocaml[firstline=9,lastline=13,highlightlines=12]{../src/backtrace/backtrace.ml}
      \endminipage
    \end{column}
    \begin{column}{0.5\textwidth}
      \onslide*<1>{
        \mintcmm[highlightlines=5]{../src/backtrace/camlBacktrace\_\_definitely\_raise\_347.cmm}
      }
      \onslide*<2>{
        \mintobjdump[firstline=2,highlightlines=14]{../src/backtrace/camlBacktrace\_\_definitely\_raise\_347.objdump}
      }
    \end{column}
  \end{columns}
\end{frame}

\begin{frame}{Backtraces}{Introducing \funcname{caml\_raise\_exn}}
  \begin{columns}[c]
    \begin{column}{0.5\textwidth}
      \minipage[c][0.66\textheight][s]{\columnwidth}
      \onslide*<1>{
        \begin{itemize}
          \item Raising the exception is performed using \funcname{caml\_raise\_exn} which is
            \begin{itemize}
              \item An assembly function provided by the runtime
              \item Architecture specific
            \end{itemize}
          \item Let's see what it does differently from \funcname{raise\_notrace}\dots
          \item We are about to raise \typename{ExnC} with \funcname{caml\_raise\_exn} from \funcname{definitely\_raise}
        \end{itemize}
      }
      \onslide*<2>{
        \mintobjdump[firstline=2,highlightlines=14]{../src/backtrace/camlBacktrace\_\_definitely\_raise\_347.objdump}
      }
      \vfill
      \mintocaml[firstline=9,lastline=13,highlightlines=12]{../src/backtrace/backtrace.ml}
      \endminipage
    \end{column}
    \begin{column}{0.5\textwidth}
      \centering
      \providecommand\step{1}\input{diagrams/backtrace/backtrace.tex}
    \end{column}
  \end{columns}
\end{frame}

\begin{frame}{Backtraces}{But before that, we need more fields from \typename{caml\_domain\_state}}
  \begin{columns}
    \begin{column}{0.5\textwidth}
      \begin{itemize}
        \item Remember \typename{caml\_domain\_state} the per-domain runtime state?
        \item Some other members are of interest to us now\dots
        \item \localname{backtrace\_active} is used to know if the runtime should collect backtraces
        \item \localname{backtrace\_active} can be set by
          \begin{itemize}
            \item The \funcarg{b} flag in the \funcname{OCAMLRUNPARAM} environment variable
            \item \mintinline{ocaml}{Printexc.record_backtrace : bool -> unit} \footnotemark
          \end{itemize}
      \end{itemize}
    \end{column}
    \begin{column}{0.5\textwidth}
      \begin{listing}
        \mintc[highlightlines={9-11}]{src/ocaml/domain\_state\_backtrace.h}
        \caption{runtime/caml/domain\_state.h}
      \end{listing}
    \end{column}
  \end{columns}
  \footnotetext[1]{https://v2.ocaml.org/api/Printexc.html}
\end{frame}

\newcommand\listCamlRaiseExn[1][]{
  \listasm[firstline=702,lastline=725,#1]{../ocaml/runtime/amd64.S}{runtime/amd64.S:702}}

\begin{frame}{Backtraces}{Walk through \funcname{caml\_raise\_exn}}
  \begin{columns}[T]
    \begin{column}{0.5\textwidth}
      \begin{itemize}
        \item<1-> First checks whether exceptions backtrace should be recorded
        \item<1-> By checking the \funcarg{domain\_state->backtrace\_active} value
        \item<2-> If not, acts as \funcname{raise\_notrace} with \funcname{RESTORE\_EXN\_HANDLER\_OCAML}
          \begin{itemize}
            \item Set \regname{RSP} to \localname{domain\_state->exn\_handler} value
            \item Restore \localname{domain\_state->exn\_handler} from trap
            \item Jump with \funcname{ret} (same effect as using \funcname{pop} and \funcname{jmp})
          \end{itemize}
        \item<3-> Otherwise, switches to the C stack to be able to execute C code
        \item<4-> Calls \funcname{caml\_stash\_backtrace}, a C function provided by the runtime\ignorespaces
          \onslide<6->{, with
            \begin{itemize}
              \item \funcarg{exn}: exception value
              \item \funcarg{pc}: code address of the raising point
              \item \funcarg{sp}: stack address of the raising function
              \item \funcarg{trapsp}: stack address of the next handler
            \end{itemize}
          }
      \end{itemize}
      \bigskip
      \mintocaml[firstline=9,lastline=13,highlightlines=12]{../src/backtrace/backtrace.ml}
    \end{column}
    \begin{column}{0.5\textwidth}
      \raggedleft
      \onslide*<1,3-4>{
        \mintc[firstline=190,lastline=192]{../ocaml/runtime/amd64.S}
        \mintc[firstline=234,lastline=237]{../ocaml/runtime/amd64.S}
      }
      \onslide*<2>{
        \mintc[firstline=190,lastline=192,highlightlines={191-192}]{../ocaml/runtime/amd64.S}
        \mintc[firstline=234,lastline=237,highlightlines={235-237}]{../ocaml/runtime/amd64.S}
      }
      \onslide*<1>{\listCamlRaiseExn[highlightlines=705]{}}%
      \onslide*<2>{\listCamlRaiseExn[highlightlines={707-708}]{}}%
      \onslide*<3>{\listCamlRaiseExn[highlightlines={712-714}]{}}%
      \onslide*<4>{\listCamlRaiseExn[highlightlines={715-720}]{}}%
      \onslide*<5>{\providecommand\step{1}\input{diagrams/backtrace/backtrace.tex}}%
      \onslide*<6>{\providecommand\step{2}\input{diagrams/backtrace/backtrace.tex}}%
    \end{column}
  \end{columns}
\end{frame}

\newcommand\listStashBacktrace[1][]{
  \listc[firstline=91,lastline=120,#1]{../ocaml/runtime/backtrace\_nat.c}{runtime/backtrace\_nat.c:91}}

\begin{frame}{Backtraces}{Introducing \funcname{caml\_stash\_backtrace}}
  \begin{columns}[c]
    \begin{column}{0.5\textwidth}
      \minipage[c][0.66\textheight][s]{\columnwidth}
      \begin{itemize}
        \item<1-> A C function provided by the runtime (specific to native code)
        \item<2-> It scans the stack, between the stack pointer at the raising point up to the next trap, in order to collect the names of the detected function frames
          \onslide*<2>{
            \begin{itemize}
              \item And more information, like line number, column number...
            \end{itemize}
          }
        \item<3-> It uses the same technique and information as the Garbage Collector (GC) uses to scan the values live on the stack
          \onslide*<4>{
            \begin{itemize}
              \item \typename{frame\_descr} are at the core of stack unwinding
            \end{itemize}
          }
      \end{itemize}
      \vfill
      \mintocaml[firstline=9,lastline=13,highlightlines=12]{../src/backtrace/backtrace.ml}
      \endminipage
    \end{column}
    \begin{column}{0.5\textwidth}
      \onslide*<1>{\listStashBacktrace[highlightlines=91]{}}
      \onslide*<2>{\listStashBacktrace[highlightlines={91,117-118}]{}}
      \onslide*<3->{\listStashBacktrace[highlightlines={94,106,109-110}]{}}
    \end{column}
  \end{columns}
\end{frame}

\newcommand\listFrameDescr[1][]{
  \listocaml[firstline=111,lastline=124,#1]{../ocaml/asmcomp/emitaux.ml}{asmcomp/emitaux.ml:111}}
\newcommand\listDebugInfo[1][]{
  \listocaml[firstline=38,lastline=49,#1]{../ocaml/lambda/debuginfo.mli}{lambda/debuginfo.mli:38}}

\begin{frame}{Backtraces}{\typename{frame\_descr} and \typename{frame\_debuginfo} in a nutshell}
  \begin{columns}[c]
    \begin{column}{0.5\textwidth}
      \begin{itemize}
        \item<1-> For every return address in an OCaml program, there is a associated \typename{frame\_descr}
        \item<2-> A \typename{frame\_descr} specifies
        \begin{itemize}
          \item<2-> The size of the function stack frame
          \item<3-> Where live values are on the stack (so that the GC can mark them as live and prevent from being swept)
        \end{itemize}
        \item<4-> When compiled with debug information, \typename{frame\_descr} will be linked to a \typename{frame\_debuginfo}
        \begin{itemize}
          \item<5-> Contains information about the position in the source code (file name, line and column number)
        \end{itemize}
      \end{itemize}
    \end{column}
    \begin{column}{0.5\textwidth}
        \onslide*<1>{\listFrameDescr[highlightlines={118-122}]{}}
        \onslide*<2>{\listFrameDescr[highlightlines={120}]{}}
        \onslide*<3>{\listFrameDescr[highlightlines={121}]{}}
        \onslide*<4->{\listFrameDescr[highlightlines={122}]{}}
        \medskip
        \onslide*<1-3>{\listDebugInfo{}}
        \onslide*<4>{\listDebugInfo[highlightlines={38,49}]{}}
        \onslide*<5>{\listDebugInfo[highlightlines={39-46}]{}}
    \end{column}
  \end{columns}
\end{frame}

\begin{frame}{Backtraces}{Scanning the stack with \typename{frame\_descr}}
  \begin{columns}[c]
    \begin{column}{0.5\textwidth}
      \begin{itemize}
        \item Given the return address of the code that raises the exception by calling \funcname{caml\_raise\_exn} (\funcarg{pc} argument of \funcname{caml\_stash\_backtrace})
        \item And the position of the stack pointer (\funcarg{sp} argument of \funcname{caml\_stash\_backtrace})
        \item The frame size given by \typename{frame\_descr}.\funcarg{frame\_size}, allows to find the end of the previous frame
        \item And the return address of the caller of this function
        \bigskip
        \onslide<3->{
        \item Note that traps (here the reddish \funcname{ExnA}, \funcname{ExnB} and \funcname{ExnC} blocks) belong to the frame that installed them, and so the frame size changes when traps are removed
        }
      \end{itemize}
    \end{column}
    \begin{column}{0.5\textwidth}
      \raggedleft
      \onslide*<1>{\providecommand\step{2}\input{diagrams/backtrace/backtrace.tex}}%
      \onslide*<2>{\providecommand\step{1}\input{diagrams/backtrace/scanstack.tex}}%
      \onslide*<3>{\providecommand\step{2}\input{diagrams/backtrace/scanstack.tex}}%
      \onslide*<4>{\providecommand\step{3}\input{diagrams/backtrace/scanstack.tex}}%
      \onslide*<5>{\providecommand\step{4}\input{diagrams/backtrace/scanstack.tex}}%
      \onslide*<6>{\providecommand\step{5}\input{diagrams/backtrace/scanstack.tex}}%
    \end{column}
  \end{columns}
\end{frame}

% Duplicate of previous-previous slide
\begin{frame}{Backtraces}{Continuing on \funcname{caml\_stash\_backtrace}}
  \begin{columns}[c]
    \begin{column}{0.5\textwidth}
      \minipage[c][0.75\textheight][s]{\columnwidth}
      \begin{itemize}
          \color{gray}
        \item A C function provided by the runtime (specific to native code)
        \item It scans the stack, between the stack pointer at the raising point up to the next trap, in order to collect the names of the detected function frames
        \item It uses the same technique and information as the Garbage Collector (GC) uses to scan the values live on the stack
      \end{itemize}
      \begin{itemize}
        \item For each function frame detected, store a pointer to the \typename{frame\_descr} in the \localname{domain\_state->backtrace\_buffer} array, effectively constructing the backtrace
      \end{itemize}
      \onslide<2->{
        \begin{itemize}
            \smallskip
          \item Constructed backtrace:
            \funcname{definitely\_raise},
            \onslide<3->{\funcname{probably\_raise}}
        \end{itemize}
      }
      \vfill
      \mintocaml[firstline=9,lastline=13,highlightlines=12]{../src/backtrace/backtrace.ml}
      \endminipage
    \end{column}
    \begin{column}{0.5\textwidth}
      \raggedleft
      \onslide*<1>{\listStashBacktrace[highlightlines={114-115}]{}}
      \onslide*<2>{\providecommand\step{3}\input{diagrams/backtrace/backtrace.tex}}%
      \onslide*<3>{\providecommand\step{4}\input{diagrams/backtrace/backtrace.tex}}%
    \end{column}
  \end{columns}
\end{frame}

\begin{frame}{Backtraces}{Back to \funcname{caml\_raise\_exn}}
  \begin{columns}[c]
    \begin{column}{0.5\textwidth}
      \minipage[c][0.66\textheight][s]{\columnwidth}
      \begin{itemize}\color{gray}
        \item Checks wheteher exceptions backtrace should be recorded, by checking the \funcarg{domain\_state->backtrace\_active} value
        \item Switches to the C stack to be able to execute C code
        \item Calls \funcname{caml\_stash\_backtrace}, to collect the backtrace up to the next trap
      \end{itemize}
      \onslide*<2->{
        \begin{itemize}
          \item<2-> Executes the exception handler in \funcname{probably\_raise} with \funcname{RESTORE\_EXN\_HANDLER\_OCAML}
            \begin{itemize}
              \item Set \regname{RSP} to \localname{domain\_state->exn\_handler} value
              \item Restore \localname{domain\_state->exn\_handler} from trap
              \item Jump to the handler code with \funcname{ret}
            \end{itemize}
        \end{itemize}
      }
      \vfill
      \onslide*<1>{\mintocaml[firstline=9,lastline=13,highlightlines=12]{../src/backtrace/backtrace.ml}}
      \onslide*<2->{\mintocaml[firstline=15,lastline=21,highlightlines={19-21}]{../src/backtrace/backtrace.ml}}
      \endminipage
    \end{column}
    \begin{column}{0.5\textwidth}
      \centering
      \onslide*<1>{
        \mintc[firstline=190,lastline=192]{../ocaml/runtime/amd64.S}
        \mintc[firstline=234,lastline=237]{../ocaml/runtime/amd64.S}
      }
      \onslide*<2>{
        \mintc[firstline=190,lastline=192,highlightlines={191-192}]{../ocaml/runtime/amd64.S}
        \mintc[firstline=234,lastline=237,highlightlines={235-237}]{../ocaml/runtime/amd64.S}
      }
      \onslide*<1>{\listCamlRaiseExn[highlightlines=720]{}}
      \onslide*<2>{\listCamlRaiseExn[highlightlines=722]{}}
      \onslide*<3->{\providecommand\step{5}\input{diagrams/backtrace/backtrace.tex}}
    \end{column}
  \end{columns}
\end{frame}

\begin{frame}{Backtraces}{\funcname{caml\_probably\_raise}'s handler doesn't catch \typename{ExnC}}
  \begin{columns}[c]
    \begin{column}{0.45\textwidth}
      \minipage[c][0.75\textheight][s]{\columnwidth}
      \begin{itemize}
        \item<1-> \funcname{match \dots with}\footnotemark pattern matching can also match exceptions
        \item<1-> The CMM of \funcname{probably\_raise} shows that this \funcname{match \dots with} is implemented with a pattern matching and a \funcname{try \dots with}
        \item<1-> But this one only matches on \typename{ExnA} exception
        \item<2-> Since the pattern matching fails to catch the exception, \funcname{reraise} is called with our current \typename{ExnC} exception
        \item<3-> Disassembly shows that \funcname{reraise} is actually a call to \funcname{caml\_reraise\_exn}, which is also an assembly function provided by the runtime
      \end{itemize}
      \vfill
      \mintocaml[firstline=15,lastline=21,highlightlines={18-21}]{../src/backtrace/backtrace.ml}
      \endminipage
    \end{column}
    \begin{column}{0.55\textwidth}
      \centering
      \onslide*<1>{\mintcmm[highlightlines={10-18}]{../src/backtrace/camlBacktrace\_\_probably\_raise\_351.cmm}}
      \onslide*<2>{\listcmm[highlightlines=18]{../src/backtrace/camlBacktrace\_\_probably\_raise_351.cmm}{CMM of probably\_raise}}
      \onslide*<3>{\mintobjdump[firstline=5,lastline=35,highlightlines=31]{../src/backtrace/camlBacktrace\_\_probably\_raise_351.objdump}}
    \end{column}
  \end{columns}
  \only<1>{\footnotetext[1]{https://blog.janestreet.com/pattern-matching-and-exception-handling-unite/}}
\end{frame}

\newcommand\listCamlReraiseExn[1][]{
  \listasm[firstline=727,lastline=734,#1]{../ocaml/runtime/amd64.S}{runtime/amd64.S:727}}

\begin{frame}{Backtraces}{Introducing \funcname{caml\_reraise\_exn}}
  \begin{columns}[c]
    \begin{column}{0.5\textwidth}
      \minipage[c][0.66\textheight][s]{\columnwidth}
      \begin{itemize}
        \item<1-> The exception have to be reraised to the next handler
        \item<2-> First, checks if the backtrace should be collected with \funcarg{domain\_state->backtrace\_active}
        \only<3>{
        \begin{itemize}
            \item If not, behaves like a \funcname{raise\_notrace} with \funcname{RESTORE\_EXN\_HANDLER\_OCAML}
        \end{itemize}
        }
        \item<4-> Jumps into \funcname{caml\_raise\_exn}
        \item<5-> Calls \funcname{caml\_stash\_backtrace} again, to scan the next part of the stack: from the trap just pop-ed to the next trap
      \end{itemize}
      \vfill
      \mintocaml[firstline=15,lastline=21,highlightlines={18-21}]{../src/backtrace/backtrace.ml}
      \endminipage
    \end{column}
    \begin{column}{0.5\textwidth}
      \raggedleft
      \onslide*<1>{\listCamlReraiseExn{}}
      \onslide*<2>{\listCamlReraiseExn[highlightlines=729]{}}
      \onslide*<3>{
        \mintc[firstline=190,lastline=192,highlightlines={191-192}]{../ocaml/runtime/amd64.S}
        \mintc[firstline=234,lastline=237,highlightlines={235-237}]{../ocaml/runtime/amd64.S}
        \medskip
        \listCamlReraiseExn[highlightlines=731]{}
      }
      \onslide*<4-5>{
        % TODO refactor with \listCamlReraiseExn
        \mintasm[firstline=727,lastline=734,highlightlines=730]{../ocaml/runtime/amd64.S}
        \medskip
      }
      \onslide*<4>{\listCamlRaiseExn[highlightlines=711]{}}
      \onslide*<5>{\listCamlRaiseExn[highlightlines=720]{}}
    \end{column}
  \end{columns}
\end{frame}

\begin{frame}{Backtraces}{Backtrace collection continues}
  \begin{columns}[c]
    \begin{column}{0.55\textwidth}
      \minipage[c][0.75\textheight][s]{\columnwidth}
      \begin{itemize}
        \item<1-> \funcname{caml\_stash\_backtrace} scans the older part of the stack, up to the next trap
        \item<6-> Controls jumps to the nested exception handler in \funcname{maybe\_raise}, which matches only on \typename{ExnB}
        \item<7-> \funcname{caml\_reraise\_exn} is called again, backtrace collection resumes with \funcname{caml\_stash\_backtrace}
      \end{itemize}
      \medskip
      \begin{itemize}
        \item<2-> Constructed backtrace:
        \begin{itemize}
          \item <2-> \funcname{definitely\_raise}
          \item <2-> \funcname{probably\_raise}
          \item <3-> \funcname{probably\_raise} (re-raise)
          \item <4-> \funcname{maybe\_raise}
          \item <5-> \funcname{main}
          \item <8-> \funcname{main} (re-raise)
        \end{itemize}
      \end{itemize}
      \vfill
      \onslide*<1-5>{\mintocaml[firstline=15,lastline=21]{../src/backtrace/backtrace.ml}}
      \onslide*<6>{\mintocaml[firstline=28,lastline=34,highlightlines=31]{../src/backtrace/backtrace.ml}}
      \onslide*<7->{\mintocaml[firstline=28,lastline=34,highlightlines=33]{../src/backtrace/backtrace.ml}}
      \endminipage
    \end{column}
    \begin{column}{0.45\textwidth}
      \raggedleft
      \onslide*<1>{\providecommand\step{6}\input{diagrams/backtrace/backtrace.tex}}%
      \onslide*<2>{\providecommand\step{6}\input{diagrams/backtrace/backtrace.tex}}%
      \onslide*<3>{\providecommand\step{7}\input{diagrams/backtrace/backtrace.tex}}%
      \onslide*<4>{\providecommand\step{8}\input{diagrams/backtrace/backtrace.tex}}%
      \onslide*<5>{\providecommand\step{9}\input{diagrams/backtrace/backtrace.tex}}%
      \onslide*<6>{\providecommand\step{10}\input{diagrams/backtrace/backtrace.tex}}%
      \onslide*<7>{\providecommand\step{11}\input{diagrams/backtrace/backtrace.tex}}%
      \onslide*<8>{\providecommand\step{12}\input{diagrams/backtrace/backtrace.tex}}%
      \onslide*<9>{\providecommand\step{13}\input{diagrams/backtrace/backtrace.tex}}%
    \end{column}
  \end{columns}
\end{frame}

\begin{frame}{Backtraces}{Printing the backtrace}
  \begin{columns}[c]
    \begin{column}{0.45\textwidth}
      \minipage[c][0.66\textheight][s]{\columnwidth}
      \begin{itemize}
        \item<1-> The exception backtrace can now be printed to \funcarg{stdout} with \funcname{Printexc.print_backtrace}
        \item<2-> First the exception backtrace is retrieved into an OCaml array with \funcname{get_raw_backtrace }
        \item<3-> Which is implemented as the \funcname{caml_get_exception_raw_backtrace} C function in the runtime
        \item<4-> And finally printed with \funcname{print_exception_backtrace}
      \end{itemize}
      \vfill
      \mintocaml[firstline=28,lastline=34,highlightlines=33]{../src/backtrace/backtrace.ml}
      \endminipage
    \end{column}
    \begin{column}{0.55\textwidth}
      \onslide*<1,2>{
        \mintocaml[firstline=100,lastline=101]{../ocaml/stdlib/printexc.ml}
      }
      \only<1>{
        \listocaml[firstline=170,lastline=172]{../ocaml/stdlib/printexc.ml}{stdlib/printexc.ml:170}
      }
      \only<2>{
        \listocaml[firstline=170,lastline=172,highlightlines=172]{../ocaml/stdlib/printexc.ml}{stdlib/printexc.ml:170}
      }
      \only<3>{
        \mintc[firstline=154,lastline=156]{../ocaml/runtime/backtrace.c}
        \listc[firstline=171,lastline=192,highlightlines={184-187}]{../ocaml/runtime/backtrace.c}{runtime/backtrace.c:154}
      }
      \only<4>{
        \listocaml[firstline=155,lastline=165]{../ocaml/stdlib/printexc.ml}{stdlib/printexc.ml:55}
      }
    \end{column}
  \end{columns}
\end{frame}


\subsection{The multiple kinds of raise}
\frameSubsection{}{}

\begin{frame}{Backtraces}{When backtraces are not needed}
  \begin{columns}[c]
    \begin{column}{0.45\textwidth}
      \minipage[c][0.66\textheight][s]{\columnwidth}
      \begin{itemize}
        \item<1-> What if the backtrace for an exception is never needed?
        \item<2-> It's possible to raise the exception explicitly with \funcname{raise\_notrace} to make raising as cheap as possible
        \item<3-> An example of use in the \funcname{dynlink} library of the OCaml compiler
      \end{itemize}
      \vfill
      \onslide<3->{
      \listocaml[firstline=205,lastline=209,highlightlines=207]{../ocaml/otherlibs/dynlink/dynlink\_compilerlibs/misc.ml}{otherlibs/dynlink/dynlink\_compilerlibs/misc.ml:205}
      }
      \endminipage
    \end{column}
    \begin{column}{0.55\textwidth}
    \only<4>{
      \mintcmm[highlightlines=3]{../src/notrace/camlNotrace\_\_fun_347.cmm}
      \listcmm{../src/notrace/camlNotrace\_\_all\_somes\_327.cmm}{all\_somes}
    }
    \onslide*<5->{
      \listobjdump[highlightlines={6-9}]{../src/notrace/camlNotrace\_\_fun_347.objdump}{anonymous function from \funcname{all\_somes}}
    }
    \end{column}
  \end{columns}
\end{frame}

\begin{frame}{Backtraces}{Summary table of the multiple kinds of raise}
  \begin{itemize}
    \item When debugging information is enabled and if the backtrace of an exception isn't needed, it's possible to raise with \funcname{raise\_notrace} for performance
    \item When debugging information is \emph{not} enabled, the compiler optimises \funcname{raise} calls to inline assembly (same effect as using \funcname{raise\_notrace})
  \end{itemize}
  \bigskip
  \centering
  \begin{tabular}{| l | c | c | c | c |}
    \hline
    \parbox{10em}{Debug information \\ (\funcarg{-g} flag)}  & \multicolumn{2}{|c|}{Disabled}                                     & \multicolumn{2}{|c|}{Enabled} \\
    \hline
    Raise kind                                               & \funcname{raise} & \funcname{raise\_notrace}                       & \funcname{raise} & \funcname{raise\_notrace} \\
    \hline
    Compilation                                              & \parbox{8em}{inline assembly\\ (optimization)} & inline assembly   & \funcname{caml\_raise\_exn} & inline assembly \\
    \hline
  \end{tabular}
\end{frame}


%
% Takeaway #5
%
\subsection*{Takeaway \#5}
\frameSubsectionTakeaway{}
\againframe<5>{takeaway}
}%
      \onslide*<5>{\providecommand\step{9}%
% Backtraces
%
\subsection{One advantage of raising with debugging information}
\frameSubsection{}{}

\begin{frame}{Backtraces}{Debugging information \& source code}
  \begin{columns}[c]
    \begin{column}{0.5\textwidth}
      \begin{itemize}
        \item Raising an exception is fun and games, but how do we know where the exception originated? $\rightarrow$ With the backtrace associated with the exception!
        \item In order to get a backtrace from the point where an exception was raised, it's necessary to compile with debugging information enabled (\funcarg{-g} flag for the compiler)
        \item Same example as in the `Nested handlers' section
      \end{itemize}
    \end{column}
    \begin{column}{0.5\textwidth}
      \listocaml[firstline=9,lastline=34]{../src/backtrace/backtrace.ml}{backtrace/backtrace.ml}
    \end{column}
  \end{columns}
\end{frame}

\begin{frame}{Backtraces}{CMM and disassembly of \funcname{definitely\_raise}}
  \begin{columns}[c]
    \begin{column}{0.5\textwidth}
      \minipage[c][0.66\textheight][s]{\columnwidth}
      \begin{itemize}
        \item<1-> An effect of compiling with debugging information (\funcarg{-g}): the CMM no longer uses \funcname{raise\_notrace} to raise the exception but \funcname{raise} instead
        \item<2-> Disassembly shows that CMM \funcname{raise} is actually a call to \funcname{caml\_raise\_exn}, a function in assembler provided by the runtime
      \end{itemize}
      \vfill
      \mintocaml[firstline=9,lastline=13,highlightlines=12]{../src/backtrace/backtrace.ml}
      \endminipage
    \end{column}
    \begin{column}{0.5\textwidth}
      \onslide*<1>{
        \mintcmm[highlightlines=5]{../src/backtrace/camlBacktrace\_\_definitely\_raise\_347.cmm}
      }
      \onslide*<2>{
        \mintobjdump[firstline=2,highlightlines=14]{../src/backtrace/camlBacktrace\_\_definitely\_raise\_347.objdump}
      }
    \end{column}
  \end{columns}
\end{frame}

\begin{frame}{Backtraces}{Introducing \funcname{caml\_raise\_exn}}
  \begin{columns}[c]
    \begin{column}{0.5\textwidth}
      \minipage[c][0.66\textheight][s]{\columnwidth}
      \onslide*<1>{
        \begin{itemize}
          \item Raising the exception is performed using \funcname{caml\_raise\_exn} which is
            \begin{itemize}
              \item An assembly function provided by the runtime
              \item Architecture specific
            \end{itemize}
          \item Let's see what it does differently from \funcname{raise\_notrace}\dots
          \item We are about to raise \typename{ExnC} with \funcname{caml\_raise\_exn} from \funcname{definitely\_raise}
        \end{itemize}
      }
      \onslide*<2>{
        \mintobjdump[firstline=2,highlightlines=14]{../src/backtrace/camlBacktrace\_\_definitely\_raise\_347.objdump}
      }
      \vfill
      \mintocaml[firstline=9,lastline=13,highlightlines=12]{../src/backtrace/backtrace.ml}
      \endminipage
    \end{column}
    \begin{column}{0.5\textwidth}
      \centering
      \providecommand\step{1}\input{diagrams/backtrace/backtrace.tex}
    \end{column}
  \end{columns}
\end{frame}

\begin{frame}{Backtraces}{But before that, we need more fields from \typename{caml\_domain\_state}}
  \begin{columns}
    \begin{column}{0.5\textwidth}
      \begin{itemize}
        \item Remember \typename{caml\_domain\_state} the per-domain runtime state?
        \item Some other members are of interest to us now\dots
        \item \localname{backtrace\_active} is used to know if the runtime should collect backtraces
        \item \localname{backtrace\_active} can be set by
          \begin{itemize}
            \item The \funcarg{b} flag in the \funcname{OCAMLRUNPARAM} environment variable
            \item \mintinline{ocaml}{Printexc.record_backtrace : bool -> unit} \footnotemark
          \end{itemize}
      \end{itemize}
    \end{column}
    \begin{column}{0.5\textwidth}
      \begin{listing}
        \mintc[highlightlines={9-11}]{src/ocaml/domain\_state\_backtrace.h}
        \caption{runtime/caml/domain\_state.h}
      \end{listing}
    \end{column}
  \end{columns}
  \footnotetext[1]{https://v2.ocaml.org/api/Printexc.html}
\end{frame}

\newcommand\listCamlRaiseExn[1][]{
  \listasm[firstline=702,lastline=725,#1]{../ocaml/runtime/amd64.S}{runtime/amd64.S:702}}

\begin{frame}{Backtraces}{Walk through \funcname{caml\_raise\_exn}}
  \begin{columns}[T]
    \begin{column}{0.5\textwidth}
      \begin{itemize}
        \item<1-> First checks whether exceptions backtrace should be recorded
        \item<1-> By checking the \funcarg{domain\_state->backtrace\_active} value
        \item<2-> If not, acts as \funcname{raise\_notrace} with \funcname{RESTORE\_EXN\_HANDLER\_OCAML}
          \begin{itemize}
            \item Set \regname{RSP} to \localname{domain\_state->exn\_handler} value
            \item Restore \localname{domain\_state->exn\_handler} from trap
            \item Jump with \funcname{ret} (same effect as using \funcname{pop} and \funcname{jmp})
          \end{itemize}
        \item<3-> Otherwise, switches to the C stack to be able to execute C code
        \item<4-> Calls \funcname{caml\_stash\_backtrace}, a C function provided by the runtime\ignorespaces
          \onslide<6->{, with
            \begin{itemize}
              \item \funcarg{exn}: exception value
              \item \funcarg{pc}: code address of the raising point
              \item \funcarg{sp}: stack address of the raising function
              \item \funcarg{trapsp}: stack address of the next handler
            \end{itemize}
          }
      \end{itemize}
      \bigskip
      \mintocaml[firstline=9,lastline=13,highlightlines=12]{../src/backtrace/backtrace.ml}
    \end{column}
    \begin{column}{0.5\textwidth}
      \raggedleft
      \onslide*<1,3-4>{
        \mintc[firstline=190,lastline=192]{../ocaml/runtime/amd64.S}
        \mintc[firstline=234,lastline=237]{../ocaml/runtime/amd64.S}
      }
      \onslide*<2>{
        \mintc[firstline=190,lastline=192,highlightlines={191-192}]{../ocaml/runtime/amd64.S}
        \mintc[firstline=234,lastline=237,highlightlines={235-237}]{../ocaml/runtime/amd64.S}
      }
      \onslide*<1>{\listCamlRaiseExn[highlightlines=705]{}}%
      \onslide*<2>{\listCamlRaiseExn[highlightlines={707-708}]{}}%
      \onslide*<3>{\listCamlRaiseExn[highlightlines={712-714}]{}}%
      \onslide*<4>{\listCamlRaiseExn[highlightlines={715-720}]{}}%
      \onslide*<5>{\providecommand\step{1}\input{diagrams/backtrace/backtrace.tex}}%
      \onslide*<6>{\providecommand\step{2}\input{diagrams/backtrace/backtrace.tex}}%
    \end{column}
  \end{columns}
\end{frame}

\newcommand\listStashBacktrace[1][]{
  \listc[firstline=91,lastline=120,#1]{../ocaml/runtime/backtrace\_nat.c}{runtime/backtrace\_nat.c:91}}

\begin{frame}{Backtraces}{Introducing \funcname{caml\_stash\_backtrace}}
  \begin{columns}[c]
    \begin{column}{0.5\textwidth}
      \minipage[c][0.66\textheight][s]{\columnwidth}
      \begin{itemize}
        \item<1-> A C function provided by the runtime (specific to native code)
        \item<2-> It scans the stack, between the stack pointer at the raising point up to the next trap, in order to collect the names of the detected function frames
          \onslide*<2>{
            \begin{itemize}
              \item And more information, like line number, column number...
            \end{itemize}
          }
        \item<3-> It uses the same technique and information as the Garbage Collector (GC) uses to scan the values live on the stack
          \onslide*<4>{
            \begin{itemize}
              \item \typename{frame\_descr} are at the core of stack unwinding
            \end{itemize}
          }
      \end{itemize}
      \vfill
      \mintocaml[firstline=9,lastline=13,highlightlines=12]{../src/backtrace/backtrace.ml}
      \endminipage
    \end{column}
    \begin{column}{0.5\textwidth}
      \onslide*<1>{\listStashBacktrace[highlightlines=91]{}}
      \onslide*<2>{\listStashBacktrace[highlightlines={91,117-118}]{}}
      \onslide*<3->{\listStashBacktrace[highlightlines={94,106,109-110}]{}}
    \end{column}
  \end{columns}
\end{frame}

\newcommand\listFrameDescr[1][]{
  \listocaml[firstline=111,lastline=124,#1]{../ocaml/asmcomp/emitaux.ml}{asmcomp/emitaux.ml:111}}
\newcommand\listDebugInfo[1][]{
  \listocaml[firstline=38,lastline=49,#1]{../ocaml/lambda/debuginfo.mli}{lambda/debuginfo.mli:38}}

\begin{frame}{Backtraces}{\typename{frame\_descr} and \typename{frame\_debuginfo} in a nutshell}
  \begin{columns}[c]
    \begin{column}{0.5\textwidth}
      \begin{itemize}
        \item<1-> For every return address in an OCaml program, there is a associated \typename{frame\_descr}
        \item<2-> A \typename{frame\_descr} specifies
        \begin{itemize}
          \item<2-> The size of the function stack frame
          \item<3-> Where live values are on the stack (so that the GC can mark them as live and prevent from being swept)
        \end{itemize}
        \item<4-> When compiled with debug information, \typename{frame\_descr} will be linked to a \typename{frame\_debuginfo}
        \begin{itemize}
          \item<5-> Contains information about the position in the source code (file name, line and column number)
        \end{itemize}
      \end{itemize}
    \end{column}
    \begin{column}{0.5\textwidth}
        \onslide*<1>{\listFrameDescr[highlightlines={118-122}]{}}
        \onslide*<2>{\listFrameDescr[highlightlines={120}]{}}
        \onslide*<3>{\listFrameDescr[highlightlines={121}]{}}
        \onslide*<4->{\listFrameDescr[highlightlines={122}]{}}
        \medskip
        \onslide*<1-3>{\listDebugInfo{}}
        \onslide*<4>{\listDebugInfo[highlightlines={38,49}]{}}
        \onslide*<5>{\listDebugInfo[highlightlines={39-46}]{}}
    \end{column}
  \end{columns}
\end{frame}

\begin{frame}{Backtraces}{Scanning the stack with \typename{frame\_descr}}
  \begin{columns}[c]
    \begin{column}{0.5\textwidth}
      \begin{itemize}
        \item Given the return address of the code that raises the exception by calling \funcname{caml\_raise\_exn} (\funcarg{pc} argument of \funcname{caml\_stash\_backtrace})
        \item And the position of the stack pointer (\funcarg{sp} argument of \funcname{caml\_stash\_backtrace})
        \item The frame size given by \typename{frame\_descr}.\funcarg{frame\_size}, allows to find the end of the previous frame
        \item And the return address of the caller of this function
        \bigskip
        \onslide<3->{
        \item Note that traps (here the reddish \funcname{ExnA}, \funcname{ExnB} and \funcname{ExnC} blocks) belong to the frame that installed them, and so the frame size changes when traps are removed
        }
      \end{itemize}
    \end{column}
    \begin{column}{0.5\textwidth}
      \raggedleft
      \onslide*<1>{\providecommand\step{2}\input{diagrams/backtrace/backtrace.tex}}%
      \onslide*<2>{\providecommand\step{1}\input{diagrams/backtrace/scanstack.tex}}%
      \onslide*<3>{\providecommand\step{2}\input{diagrams/backtrace/scanstack.tex}}%
      \onslide*<4>{\providecommand\step{3}\input{diagrams/backtrace/scanstack.tex}}%
      \onslide*<5>{\providecommand\step{4}\input{diagrams/backtrace/scanstack.tex}}%
      \onslide*<6>{\providecommand\step{5}\input{diagrams/backtrace/scanstack.tex}}%
    \end{column}
  \end{columns}
\end{frame}

% Duplicate of previous-previous slide
\begin{frame}{Backtraces}{Continuing on \funcname{caml\_stash\_backtrace}}
  \begin{columns}[c]
    \begin{column}{0.5\textwidth}
      \minipage[c][0.75\textheight][s]{\columnwidth}
      \begin{itemize}
          \color{gray}
        \item A C function provided by the runtime (specific to native code)
        \item It scans the stack, between the stack pointer at the raising point up to the next trap, in order to collect the names of the detected function frames
        \item It uses the same technique and information as the Garbage Collector (GC) uses to scan the values live on the stack
      \end{itemize}
      \begin{itemize}
        \item For each function frame detected, store a pointer to the \typename{frame\_descr} in the \localname{domain\_state->backtrace\_buffer} array, effectively constructing the backtrace
      \end{itemize}
      \onslide<2->{
        \begin{itemize}
            \smallskip
          \item Constructed backtrace:
            \funcname{definitely\_raise},
            \onslide<3->{\funcname{probably\_raise}}
        \end{itemize}
      }
      \vfill
      \mintocaml[firstline=9,lastline=13,highlightlines=12]{../src/backtrace/backtrace.ml}
      \endminipage
    \end{column}
    \begin{column}{0.5\textwidth}
      \raggedleft
      \onslide*<1>{\listStashBacktrace[highlightlines={114-115}]{}}
      \onslide*<2>{\providecommand\step{3}\input{diagrams/backtrace/backtrace.tex}}%
      \onslide*<3>{\providecommand\step{4}\input{diagrams/backtrace/backtrace.tex}}%
    \end{column}
  \end{columns}
\end{frame}

\begin{frame}{Backtraces}{Back to \funcname{caml\_raise\_exn}}
  \begin{columns}[c]
    \begin{column}{0.5\textwidth}
      \minipage[c][0.66\textheight][s]{\columnwidth}
      \begin{itemize}\color{gray}
        \item Checks wheteher exceptions backtrace should be recorded, by checking the \funcarg{domain\_state->backtrace\_active} value
        \item Switches to the C stack to be able to execute C code
        \item Calls \funcname{caml\_stash\_backtrace}, to collect the backtrace up to the next trap
      \end{itemize}
      \onslide*<2->{
        \begin{itemize}
          \item<2-> Executes the exception handler in \funcname{probably\_raise} with \funcname{RESTORE\_EXN\_HANDLER\_OCAML}
            \begin{itemize}
              \item Set \regname{RSP} to \localname{domain\_state->exn\_handler} value
              \item Restore \localname{domain\_state->exn\_handler} from trap
              \item Jump to the handler code with \funcname{ret}
            \end{itemize}
        \end{itemize}
      }
      \vfill
      \onslide*<1>{\mintocaml[firstline=9,lastline=13,highlightlines=12]{../src/backtrace/backtrace.ml}}
      \onslide*<2->{\mintocaml[firstline=15,lastline=21,highlightlines={19-21}]{../src/backtrace/backtrace.ml}}
      \endminipage
    \end{column}
    \begin{column}{0.5\textwidth}
      \centering
      \onslide*<1>{
        \mintc[firstline=190,lastline=192]{../ocaml/runtime/amd64.S}
        \mintc[firstline=234,lastline=237]{../ocaml/runtime/amd64.S}
      }
      \onslide*<2>{
        \mintc[firstline=190,lastline=192,highlightlines={191-192}]{../ocaml/runtime/amd64.S}
        \mintc[firstline=234,lastline=237,highlightlines={235-237}]{../ocaml/runtime/amd64.S}
      }
      \onslide*<1>{\listCamlRaiseExn[highlightlines=720]{}}
      \onslide*<2>{\listCamlRaiseExn[highlightlines=722]{}}
      \onslide*<3->{\providecommand\step{5}\input{diagrams/backtrace/backtrace.tex}}
    \end{column}
  \end{columns}
\end{frame}

\begin{frame}{Backtraces}{\funcname{caml\_probably\_raise}'s handler doesn't catch \typename{ExnC}}
  \begin{columns}[c]
    \begin{column}{0.45\textwidth}
      \minipage[c][0.75\textheight][s]{\columnwidth}
      \begin{itemize}
        \item<1-> \funcname{match \dots with}\footnotemark pattern matching can also match exceptions
        \item<1-> The CMM of \funcname{probably\_raise} shows that this \funcname{match \dots with} is implemented with a pattern matching and a \funcname{try \dots with}
        \item<1-> But this one only matches on \typename{ExnA} exception
        \item<2-> Since the pattern matching fails to catch the exception, \funcname{reraise} is called with our current \typename{ExnC} exception
        \item<3-> Disassembly shows that \funcname{reraise} is actually a call to \funcname{caml\_reraise\_exn}, which is also an assembly function provided by the runtime
      \end{itemize}
      \vfill
      \mintocaml[firstline=15,lastline=21,highlightlines={18-21}]{../src/backtrace/backtrace.ml}
      \endminipage
    \end{column}
    \begin{column}{0.55\textwidth}
      \centering
      \onslide*<1>{\mintcmm[highlightlines={10-18}]{../src/backtrace/camlBacktrace\_\_probably\_raise\_351.cmm}}
      \onslide*<2>{\listcmm[highlightlines=18]{../src/backtrace/camlBacktrace\_\_probably\_raise_351.cmm}{CMM of probably\_raise}}
      \onslide*<3>{\mintobjdump[firstline=5,lastline=35,highlightlines=31]{../src/backtrace/camlBacktrace\_\_probably\_raise_351.objdump}}
    \end{column}
  \end{columns}
  \only<1>{\footnotetext[1]{https://blog.janestreet.com/pattern-matching-and-exception-handling-unite/}}
\end{frame}

\newcommand\listCamlReraiseExn[1][]{
  \listasm[firstline=727,lastline=734,#1]{../ocaml/runtime/amd64.S}{runtime/amd64.S:727}}

\begin{frame}{Backtraces}{Introducing \funcname{caml\_reraise\_exn}}
  \begin{columns}[c]
    \begin{column}{0.5\textwidth}
      \minipage[c][0.66\textheight][s]{\columnwidth}
      \begin{itemize}
        \item<1-> The exception have to be reraised to the next handler
        \item<2-> First, checks if the backtrace should be collected with \funcarg{domain\_state->backtrace\_active}
        \only<3>{
        \begin{itemize}
            \item If not, behaves like a \funcname{raise\_notrace} with \funcname{RESTORE\_EXN\_HANDLER\_OCAML}
        \end{itemize}
        }
        \item<4-> Jumps into \funcname{caml\_raise\_exn}
        \item<5-> Calls \funcname{caml\_stash\_backtrace} again, to scan the next part of the stack: from the trap just pop-ed to the next trap
      \end{itemize}
      \vfill
      \mintocaml[firstline=15,lastline=21,highlightlines={18-21}]{../src/backtrace/backtrace.ml}
      \endminipage
    \end{column}
    \begin{column}{0.5\textwidth}
      \raggedleft
      \onslide*<1>{\listCamlReraiseExn{}}
      \onslide*<2>{\listCamlReraiseExn[highlightlines=729]{}}
      \onslide*<3>{
        \mintc[firstline=190,lastline=192,highlightlines={191-192}]{../ocaml/runtime/amd64.S}
        \mintc[firstline=234,lastline=237,highlightlines={235-237}]{../ocaml/runtime/amd64.S}
        \medskip
        \listCamlReraiseExn[highlightlines=731]{}
      }
      \onslide*<4-5>{
        % TODO refactor with \listCamlReraiseExn
        \mintasm[firstline=727,lastline=734,highlightlines=730]{../ocaml/runtime/amd64.S}
        \medskip
      }
      \onslide*<4>{\listCamlRaiseExn[highlightlines=711]{}}
      \onslide*<5>{\listCamlRaiseExn[highlightlines=720]{}}
    \end{column}
  \end{columns}
\end{frame}

\begin{frame}{Backtraces}{Backtrace collection continues}
  \begin{columns}[c]
    \begin{column}{0.55\textwidth}
      \minipage[c][0.75\textheight][s]{\columnwidth}
      \begin{itemize}
        \item<1-> \funcname{caml\_stash\_backtrace} scans the older part of the stack, up to the next trap
        \item<6-> Controls jumps to the nested exception handler in \funcname{maybe\_raise}, which matches only on \typename{ExnB}
        \item<7-> \funcname{caml\_reraise\_exn} is called again, backtrace collection resumes with \funcname{caml\_stash\_backtrace}
      \end{itemize}
      \medskip
      \begin{itemize}
        \item<2-> Constructed backtrace:
        \begin{itemize}
          \item <2-> \funcname{definitely\_raise}
          \item <2-> \funcname{probably\_raise}
          \item <3-> \funcname{probably\_raise} (re-raise)
          \item <4-> \funcname{maybe\_raise}
          \item <5-> \funcname{main}
          \item <8-> \funcname{main} (re-raise)
        \end{itemize}
      \end{itemize}
      \vfill
      \onslide*<1-5>{\mintocaml[firstline=15,lastline=21]{../src/backtrace/backtrace.ml}}
      \onslide*<6>{\mintocaml[firstline=28,lastline=34,highlightlines=31]{../src/backtrace/backtrace.ml}}
      \onslide*<7->{\mintocaml[firstline=28,lastline=34,highlightlines=33]{../src/backtrace/backtrace.ml}}
      \endminipage
    \end{column}
    \begin{column}{0.45\textwidth}
      \raggedleft
      \onslide*<1>{\providecommand\step{6}\input{diagrams/backtrace/backtrace.tex}}%
      \onslide*<2>{\providecommand\step{6}\input{diagrams/backtrace/backtrace.tex}}%
      \onslide*<3>{\providecommand\step{7}\input{diagrams/backtrace/backtrace.tex}}%
      \onslide*<4>{\providecommand\step{8}\input{diagrams/backtrace/backtrace.tex}}%
      \onslide*<5>{\providecommand\step{9}\input{diagrams/backtrace/backtrace.tex}}%
      \onslide*<6>{\providecommand\step{10}\input{diagrams/backtrace/backtrace.tex}}%
      \onslide*<7>{\providecommand\step{11}\input{diagrams/backtrace/backtrace.tex}}%
      \onslide*<8>{\providecommand\step{12}\input{diagrams/backtrace/backtrace.tex}}%
      \onslide*<9>{\providecommand\step{13}\input{diagrams/backtrace/backtrace.tex}}%
    \end{column}
  \end{columns}
\end{frame}

\begin{frame}{Backtraces}{Printing the backtrace}
  \begin{columns}[c]
    \begin{column}{0.45\textwidth}
      \minipage[c][0.66\textheight][s]{\columnwidth}
      \begin{itemize}
        \item<1-> The exception backtrace can now be printed to \funcarg{stdout} with \funcname{Printexc.print_backtrace}
        \item<2-> First the exception backtrace is retrieved into an OCaml array with \funcname{get_raw_backtrace }
        \item<3-> Which is implemented as the \funcname{caml_get_exception_raw_backtrace} C function in the runtime
        \item<4-> And finally printed with \funcname{print_exception_backtrace}
      \end{itemize}
      \vfill
      \mintocaml[firstline=28,lastline=34,highlightlines=33]{../src/backtrace/backtrace.ml}
      \endminipage
    \end{column}
    \begin{column}{0.55\textwidth}
      \onslide*<1,2>{
        \mintocaml[firstline=100,lastline=101]{../ocaml/stdlib/printexc.ml}
      }
      \only<1>{
        \listocaml[firstline=170,lastline=172]{../ocaml/stdlib/printexc.ml}{stdlib/printexc.ml:170}
      }
      \only<2>{
        \listocaml[firstline=170,lastline=172,highlightlines=172]{../ocaml/stdlib/printexc.ml}{stdlib/printexc.ml:170}
      }
      \only<3>{
        \mintc[firstline=154,lastline=156]{../ocaml/runtime/backtrace.c}
        \listc[firstline=171,lastline=192,highlightlines={184-187}]{../ocaml/runtime/backtrace.c}{runtime/backtrace.c:154}
      }
      \only<4>{
        \listocaml[firstline=155,lastline=165]{../ocaml/stdlib/printexc.ml}{stdlib/printexc.ml:55}
      }
    \end{column}
  \end{columns}
\end{frame}


\subsection{The multiple kinds of raise}
\frameSubsection{}{}

\begin{frame}{Backtraces}{When backtraces are not needed}
  \begin{columns}[c]
    \begin{column}{0.45\textwidth}
      \minipage[c][0.66\textheight][s]{\columnwidth}
      \begin{itemize}
        \item<1-> What if the backtrace for an exception is never needed?
        \item<2-> It's possible to raise the exception explicitly with \funcname{raise\_notrace} to make raising as cheap as possible
        \item<3-> An example of use in the \funcname{dynlink} library of the OCaml compiler
      \end{itemize}
      \vfill
      \onslide<3->{
      \listocaml[firstline=205,lastline=209,highlightlines=207]{../ocaml/otherlibs/dynlink/dynlink\_compilerlibs/misc.ml}{otherlibs/dynlink/dynlink\_compilerlibs/misc.ml:205}
      }
      \endminipage
    \end{column}
    \begin{column}{0.55\textwidth}
    \only<4>{
      \mintcmm[highlightlines=3]{../src/notrace/camlNotrace\_\_fun_347.cmm}
      \listcmm{../src/notrace/camlNotrace\_\_all\_somes\_327.cmm}{all\_somes}
    }
    \onslide*<5->{
      \listobjdump[highlightlines={6-9}]{../src/notrace/camlNotrace\_\_fun_347.objdump}{anonymous function from \funcname{all\_somes}}
    }
    \end{column}
  \end{columns}
\end{frame}

\begin{frame}{Backtraces}{Summary table of the multiple kinds of raise}
  \begin{itemize}
    \item When debugging information is enabled and if the backtrace of an exception isn't needed, it's possible to raise with \funcname{raise\_notrace} for performance
    \item When debugging information is \emph{not} enabled, the compiler optimises \funcname{raise} calls to inline assembly (same effect as using \funcname{raise\_notrace})
  \end{itemize}
  \bigskip
  \centering
  \begin{tabular}{| l | c | c | c | c |}
    \hline
    \parbox{10em}{Debug information \\ (\funcarg{-g} flag)}  & \multicolumn{2}{|c|}{Disabled}                                     & \multicolumn{2}{|c|}{Enabled} \\
    \hline
    Raise kind                                               & \funcname{raise} & \funcname{raise\_notrace}                       & \funcname{raise} & \funcname{raise\_notrace} \\
    \hline
    Compilation                                              & \parbox{8em}{inline assembly\\ (optimization)} & inline assembly   & \funcname{caml\_raise\_exn} & inline assembly \\
    \hline
  \end{tabular}
\end{frame}


%
% Takeaway #5
%
\subsection*{Takeaway \#5}
\frameSubsectionTakeaway{}
\againframe<5>{takeaway}
}%
      \onslide*<6>{\providecommand\step{10}%
% Backtraces
%
\subsection{One advantage of raising with debugging information}
\frameSubsection{}{}

\begin{frame}{Backtraces}{Debugging information \& source code}
  \begin{columns}[c]
    \begin{column}{0.5\textwidth}
      \begin{itemize}
        \item Raising an exception is fun and games, but how do we know where the exception originated? $\rightarrow$ With the backtrace associated with the exception!
        \item In order to get a backtrace from the point where an exception was raised, it's necessary to compile with debugging information enabled (\funcarg{-g} flag for the compiler)
        \item Same example as in the `Nested handlers' section
      \end{itemize}
    \end{column}
    \begin{column}{0.5\textwidth}
      \listocaml[firstline=9,lastline=34]{../src/backtrace/backtrace.ml}{backtrace/backtrace.ml}
    \end{column}
  \end{columns}
\end{frame}

\begin{frame}{Backtraces}{CMM and disassembly of \funcname{definitely\_raise}}
  \begin{columns}[c]
    \begin{column}{0.5\textwidth}
      \minipage[c][0.66\textheight][s]{\columnwidth}
      \begin{itemize}
        \item<1-> An effect of compiling with debugging information (\funcarg{-g}): the CMM no longer uses \funcname{raise\_notrace} to raise the exception but \funcname{raise} instead
        \item<2-> Disassembly shows that CMM \funcname{raise} is actually a call to \funcname{caml\_raise\_exn}, a function in assembler provided by the runtime
      \end{itemize}
      \vfill
      \mintocaml[firstline=9,lastline=13,highlightlines=12]{../src/backtrace/backtrace.ml}
      \endminipage
    \end{column}
    \begin{column}{0.5\textwidth}
      \onslide*<1>{
        \mintcmm[highlightlines=5]{../src/backtrace/camlBacktrace\_\_definitely\_raise\_347.cmm}
      }
      \onslide*<2>{
        \mintobjdump[firstline=2,highlightlines=14]{../src/backtrace/camlBacktrace\_\_definitely\_raise\_347.objdump}
      }
    \end{column}
  \end{columns}
\end{frame}

\begin{frame}{Backtraces}{Introducing \funcname{caml\_raise\_exn}}
  \begin{columns}[c]
    \begin{column}{0.5\textwidth}
      \minipage[c][0.66\textheight][s]{\columnwidth}
      \onslide*<1>{
        \begin{itemize}
          \item Raising the exception is performed using \funcname{caml\_raise\_exn} which is
            \begin{itemize}
              \item An assembly function provided by the runtime
              \item Architecture specific
            \end{itemize}
          \item Let's see what it does differently from \funcname{raise\_notrace}\dots
          \item We are about to raise \typename{ExnC} with \funcname{caml\_raise\_exn} from \funcname{definitely\_raise}
        \end{itemize}
      }
      \onslide*<2>{
        \mintobjdump[firstline=2,highlightlines=14]{../src/backtrace/camlBacktrace\_\_definitely\_raise\_347.objdump}
      }
      \vfill
      \mintocaml[firstline=9,lastline=13,highlightlines=12]{../src/backtrace/backtrace.ml}
      \endminipage
    \end{column}
    \begin{column}{0.5\textwidth}
      \centering
      \providecommand\step{1}\input{diagrams/backtrace/backtrace.tex}
    \end{column}
  \end{columns}
\end{frame}

\begin{frame}{Backtraces}{But before that, we need more fields from \typename{caml\_domain\_state}}
  \begin{columns}
    \begin{column}{0.5\textwidth}
      \begin{itemize}
        \item Remember \typename{caml\_domain\_state} the per-domain runtime state?
        \item Some other members are of interest to us now\dots
        \item \localname{backtrace\_active} is used to know if the runtime should collect backtraces
        \item \localname{backtrace\_active} can be set by
          \begin{itemize}
            \item The \funcarg{b} flag in the \funcname{OCAMLRUNPARAM} environment variable
            \item \mintinline{ocaml}{Printexc.record_backtrace : bool -> unit} \footnotemark
          \end{itemize}
      \end{itemize}
    \end{column}
    \begin{column}{0.5\textwidth}
      \begin{listing}
        \mintc[highlightlines={9-11}]{src/ocaml/domain\_state\_backtrace.h}
        \caption{runtime/caml/domain\_state.h}
      \end{listing}
    \end{column}
  \end{columns}
  \footnotetext[1]{https://v2.ocaml.org/api/Printexc.html}
\end{frame}

\newcommand\listCamlRaiseExn[1][]{
  \listasm[firstline=702,lastline=725,#1]{../ocaml/runtime/amd64.S}{runtime/amd64.S:702}}

\begin{frame}{Backtraces}{Walk through \funcname{caml\_raise\_exn}}
  \begin{columns}[T]
    \begin{column}{0.5\textwidth}
      \begin{itemize}
        \item<1-> First checks whether exceptions backtrace should be recorded
        \item<1-> By checking the \funcarg{domain\_state->backtrace\_active} value
        \item<2-> If not, acts as \funcname{raise\_notrace} with \funcname{RESTORE\_EXN\_HANDLER\_OCAML}
          \begin{itemize}
            \item Set \regname{RSP} to \localname{domain\_state->exn\_handler} value
            \item Restore \localname{domain\_state->exn\_handler} from trap
            \item Jump with \funcname{ret} (same effect as using \funcname{pop} and \funcname{jmp})
          \end{itemize}
        \item<3-> Otherwise, switches to the C stack to be able to execute C code
        \item<4-> Calls \funcname{caml\_stash\_backtrace}, a C function provided by the runtime\ignorespaces
          \onslide<6->{, with
            \begin{itemize}
              \item \funcarg{exn}: exception value
              \item \funcarg{pc}: code address of the raising point
              \item \funcarg{sp}: stack address of the raising function
              \item \funcarg{trapsp}: stack address of the next handler
            \end{itemize}
          }
      \end{itemize}
      \bigskip
      \mintocaml[firstline=9,lastline=13,highlightlines=12]{../src/backtrace/backtrace.ml}
    \end{column}
    \begin{column}{0.5\textwidth}
      \raggedleft
      \onslide*<1,3-4>{
        \mintc[firstline=190,lastline=192]{../ocaml/runtime/amd64.S}
        \mintc[firstline=234,lastline=237]{../ocaml/runtime/amd64.S}
      }
      \onslide*<2>{
        \mintc[firstline=190,lastline=192,highlightlines={191-192}]{../ocaml/runtime/amd64.S}
        \mintc[firstline=234,lastline=237,highlightlines={235-237}]{../ocaml/runtime/amd64.S}
      }
      \onslide*<1>{\listCamlRaiseExn[highlightlines=705]{}}%
      \onslide*<2>{\listCamlRaiseExn[highlightlines={707-708}]{}}%
      \onslide*<3>{\listCamlRaiseExn[highlightlines={712-714}]{}}%
      \onslide*<4>{\listCamlRaiseExn[highlightlines={715-720}]{}}%
      \onslide*<5>{\providecommand\step{1}\input{diagrams/backtrace/backtrace.tex}}%
      \onslide*<6>{\providecommand\step{2}\input{diagrams/backtrace/backtrace.tex}}%
    \end{column}
  \end{columns}
\end{frame}

\newcommand\listStashBacktrace[1][]{
  \listc[firstline=91,lastline=120,#1]{../ocaml/runtime/backtrace\_nat.c}{runtime/backtrace\_nat.c:91}}

\begin{frame}{Backtraces}{Introducing \funcname{caml\_stash\_backtrace}}
  \begin{columns}[c]
    \begin{column}{0.5\textwidth}
      \minipage[c][0.66\textheight][s]{\columnwidth}
      \begin{itemize}
        \item<1-> A C function provided by the runtime (specific to native code)
        \item<2-> It scans the stack, between the stack pointer at the raising point up to the next trap, in order to collect the names of the detected function frames
          \onslide*<2>{
            \begin{itemize}
              \item And more information, like line number, column number...
            \end{itemize}
          }
        \item<3-> It uses the same technique and information as the Garbage Collector (GC) uses to scan the values live on the stack
          \onslide*<4>{
            \begin{itemize}
              \item \typename{frame\_descr} are at the core of stack unwinding
            \end{itemize}
          }
      \end{itemize}
      \vfill
      \mintocaml[firstline=9,lastline=13,highlightlines=12]{../src/backtrace/backtrace.ml}
      \endminipage
    \end{column}
    \begin{column}{0.5\textwidth}
      \onslide*<1>{\listStashBacktrace[highlightlines=91]{}}
      \onslide*<2>{\listStashBacktrace[highlightlines={91,117-118}]{}}
      \onslide*<3->{\listStashBacktrace[highlightlines={94,106,109-110}]{}}
    \end{column}
  \end{columns}
\end{frame}

\newcommand\listFrameDescr[1][]{
  \listocaml[firstline=111,lastline=124,#1]{../ocaml/asmcomp/emitaux.ml}{asmcomp/emitaux.ml:111}}
\newcommand\listDebugInfo[1][]{
  \listocaml[firstline=38,lastline=49,#1]{../ocaml/lambda/debuginfo.mli}{lambda/debuginfo.mli:38}}

\begin{frame}{Backtraces}{\typename{frame\_descr} and \typename{frame\_debuginfo} in a nutshell}
  \begin{columns}[c]
    \begin{column}{0.5\textwidth}
      \begin{itemize}
        \item<1-> For every return address in an OCaml program, there is a associated \typename{frame\_descr}
        \item<2-> A \typename{frame\_descr} specifies
        \begin{itemize}
          \item<2-> The size of the function stack frame
          \item<3-> Where live values are on the stack (so that the GC can mark them as live and prevent from being swept)
        \end{itemize}
        \item<4-> When compiled with debug information, \typename{frame\_descr} will be linked to a \typename{frame\_debuginfo}
        \begin{itemize}
          \item<5-> Contains information about the position in the source code (file name, line and column number)
        \end{itemize}
      \end{itemize}
    \end{column}
    \begin{column}{0.5\textwidth}
        \onslide*<1>{\listFrameDescr[highlightlines={118-122}]{}}
        \onslide*<2>{\listFrameDescr[highlightlines={120}]{}}
        \onslide*<3>{\listFrameDescr[highlightlines={121}]{}}
        \onslide*<4->{\listFrameDescr[highlightlines={122}]{}}
        \medskip
        \onslide*<1-3>{\listDebugInfo{}}
        \onslide*<4>{\listDebugInfo[highlightlines={38,49}]{}}
        \onslide*<5>{\listDebugInfo[highlightlines={39-46}]{}}
    \end{column}
  \end{columns}
\end{frame}

\begin{frame}{Backtraces}{Scanning the stack with \typename{frame\_descr}}
  \begin{columns}[c]
    \begin{column}{0.5\textwidth}
      \begin{itemize}
        \item Given the return address of the code that raises the exception by calling \funcname{caml\_raise\_exn} (\funcarg{pc} argument of \funcname{caml\_stash\_backtrace})
        \item And the position of the stack pointer (\funcarg{sp} argument of \funcname{caml\_stash\_backtrace})
        \item The frame size given by \typename{frame\_descr}.\funcarg{frame\_size}, allows to find the end of the previous frame
        \item And the return address of the caller of this function
        \bigskip
        \onslide<3->{
        \item Note that traps (here the reddish \funcname{ExnA}, \funcname{ExnB} and \funcname{ExnC} blocks) belong to the frame that installed them, and so the frame size changes when traps are removed
        }
      \end{itemize}
    \end{column}
    \begin{column}{0.5\textwidth}
      \raggedleft
      \onslide*<1>{\providecommand\step{2}\input{diagrams/backtrace/backtrace.tex}}%
      \onslide*<2>{\providecommand\step{1}\input{diagrams/backtrace/scanstack.tex}}%
      \onslide*<3>{\providecommand\step{2}\input{diagrams/backtrace/scanstack.tex}}%
      \onslide*<4>{\providecommand\step{3}\input{diagrams/backtrace/scanstack.tex}}%
      \onslide*<5>{\providecommand\step{4}\input{diagrams/backtrace/scanstack.tex}}%
      \onslide*<6>{\providecommand\step{5}\input{diagrams/backtrace/scanstack.tex}}%
    \end{column}
  \end{columns}
\end{frame}

% Duplicate of previous-previous slide
\begin{frame}{Backtraces}{Continuing on \funcname{caml\_stash\_backtrace}}
  \begin{columns}[c]
    \begin{column}{0.5\textwidth}
      \minipage[c][0.75\textheight][s]{\columnwidth}
      \begin{itemize}
          \color{gray}
        \item A C function provided by the runtime (specific to native code)
        \item It scans the stack, between the stack pointer at the raising point up to the next trap, in order to collect the names of the detected function frames
        \item It uses the same technique and information as the Garbage Collector (GC) uses to scan the values live on the stack
      \end{itemize}
      \begin{itemize}
        \item For each function frame detected, store a pointer to the \typename{frame\_descr} in the \localname{domain\_state->backtrace\_buffer} array, effectively constructing the backtrace
      \end{itemize}
      \onslide<2->{
        \begin{itemize}
            \smallskip
          \item Constructed backtrace:
            \funcname{definitely\_raise},
            \onslide<3->{\funcname{probably\_raise}}
        \end{itemize}
      }
      \vfill
      \mintocaml[firstline=9,lastline=13,highlightlines=12]{../src/backtrace/backtrace.ml}
      \endminipage
    \end{column}
    \begin{column}{0.5\textwidth}
      \raggedleft
      \onslide*<1>{\listStashBacktrace[highlightlines={114-115}]{}}
      \onslide*<2>{\providecommand\step{3}\input{diagrams/backtrace/backtrace.tex}}%
      \onslide*<3>{\providecommand\step{4}\input{diagrams/backtrace/backtrace.tex}}%
    \end{column}
  \end{columns}
\end{frame}

\begin{frame}{Backtraces}{Back to \funcname{caml\_raise\_exn}}
  \begin{columns}[c]
    \begin{column}{0.5\textwidth}
      \minipage[c][0.66\textheight][s]{\columnwidth}
      \begin{itemize}\color{gray}
        \item Checks wheteher exceptions backtrace should be recorded, by checking the \funcarg{domain\_state->backtrace\_active} value
        \item Switches to the C stack to be able to execute C code
        \item Calls \funcname{caml\_stash\_backtrace}, to collect the backtrace up to the next trap
      \end{itemize}
      \onslide*<2->{
        \begin{itemize}
          \item<2-> Executes the exception handler in \funcname{probably\_raise} with \funcname{RESTORE\_EXN\_HANDLER\_OCAML}
            \begin{itemize}
              \item Set \regname{RSP} to \localname{domain\_state->exn\_handler} value
              \item Restore \localname{domain\_state->exn\_handler} from trap
              \item Jump to the handler code with \funcname{ret}
            \end{itemize}
        \end{itemize}
      }
      \vfill
      \onslide*<1>{\mintocaml[firstline=9,lastline=13,highlightlines=12]{../src/backtrace/backtrace.ml}}
      \onslide*<2->{\mintocaml[firstline=15,lastline=21,highlightlines={19-21}]{../src/backtrace/backtrace.ml}}
      \endminipage
    \end{column}
    \begin{column}{0.5\textwidth}
      \centering
      \onslide*<1>{
        \mintc[firstline=190,lastline=192]{../ocaml/runtime/amd64.S}
        \mintc[firstline=234,lastline=237]{../ocaml/runtime/amd64.S}
      }
      \onslide*<2>{
        \mintc[firstline=190,lastline=192,highlightlines={191-192}]{../ocaml/runtime/amd64.S}
        \mintc[firstline=234,lastline=237,highlightlines={235-237}]{../ocaml/runtime/amd64.S}
      }
      \onslide*<1>{\listCamlRaiseExn[highlightlines=720]{}}
      \onslide*<2>{\listCamlRaiseExn[highlightlines=722]{}}
      \onslide*<3->{\providecommand\step{5}\input{diagrams/backtrace/backtrace.tex}}
    \end{column}
  \end{columns}
\end{frame}

\begin{frame}{Backtraces}{\funcname{caml\_probably\_raise}'s handler doesn't catch \typename{ExnC}}
  \begin{columns}[c]
    \begin{column}{0.45\textwidth}
      \minipage[c][0.75\textheight][s]{\columnwidth}
      \begin{itemize}
        \item<1-> \funcname{match \dots with}\footnotemark pattern matching can also match exceptions
        \item<1-> The CMM of \funcname{probably\_raise} shows that this \funcname{match \dots with} is implemented with a pattern matching and a \funcname{try \dots with}
        \item<1-> But this one only matches on \typename{ExnA} exception
        \item<2-> Since the pattern matching fails to catch the exception, \funcname{reraise} is called with our current \typename{ExnC} exception
        \item<3-> Disassembly shows that \funcname{reraise} is actually a call to \funcname{caml\_reraise\_exn}, which is also an assembly function provided by the runtime
      \end{itemize}
      \vfill
      \mintocaml[firstline=15,lastline=21,highlightlines={18-21}]{../src/backtrace/backtrace.ml}
      \endminipage
    \end{column}
    \begin{column}{0.55\textwidth}
      \centering
      \onslide*<1>{\mintcmm[highlightlines={10-18}]{../src/backtrace/camlBacktrace\_\_probably\_raise\_351.cmm}}
      \onslide*<2>{\listcmm[highlightlines=18]{../src/backtrace/camlBacktrace\_\_probably\_raise_351.cmm}{CMM of probably\_raise}}
      \onslide*<3>{\mintobjdump[firstline=5,lastline=35,highlightlines=31]{../src/backtrace/camlBacktrace\_\_probably\_raise_351.objdump}}
    \end{column}
  \end{columns}
  \only<1>{\footnotetext[1]{https://blog.janestreet.com/pattern-matching-and-exception-handling-unite/}}
\end{frame}

\newcommand\listCamlReraiseExn[1][]{
  \listasm[firstline=727,lastline=734,#1]{../ocaml/runtime/amd64.S}{runtime/amd64.S:727}}

\begin{frame}{Backtraces}{Introducing \funcname{caml\_reraise\_exn}}
  \begin{columns}[c]
    \begin{column}{0.5\textwidth}
      \minipage[c][0.66\textheight][s]{\columnwidth}
      \begin{itemize}
        \item<1-> The exception have to be reraised to the next handler
        \item<2-> First, checks if the backtrace should be collected with \funcarg{domain\_state->backtrace\_active}
        \only<3>{
        \begin{itemize}
            \item If not, behaves like a \funcname{raise\_notrace} with \funcname{RESTORE\_EXN\_HANDLER\_OCAML}
        \end{itemize}
        }
        \item<4-> Jumps into \funcname{caml\_raise\_exn}
        \item<5-> Calls \funcname{caml\_stash\_backtrace} again, to scan the next part of the stack: from the trap just pop-ed to the next trap
      \end{itemize}
      \vfill
      \mintocaml[firstline=15,lastline=21,highlightlines={18-21}]{../src/backtrace/backtrace.ml}
      \endminipage
    \end{column}
    \begin{column}{0.5\textwidth}
      \raggedleft
      \onslide*<1>{\listCamlReraiseExn{}}
      \onslide*<2>{\listCamlReraiseExn[highlightlines=729]{}}
      \onslide*<3>{
        \mintc[firstline=190,lastline=192,highlightlines={191-192}]{../ocaml/runtime/amd64.S}
        \mintc[firstline=234,lastline=237,highlightlines={235-237}]{../ocaml/runtime/amd64.S}
        \medskip
        \listCamlReraiseExn[highlightlines=731]{}
      }
      \onslide*<4-5>{
        % TODO refactor with \listCamlReraiseExn
        \mintasm[firstline=727,lastline=734,highlightlines=730]{../ocaml/runtime/amd64.S}
        \medskip
      }
      \onslide*<4>{\listCamlRaiseExn[highlightlines=711]{}}
      \onslide*<5>{\listCamlRaiseExn[highlightlines=720]{}}
    \end{column}
  \end{columns}
\end{frame}

\begin{frame}{Backtraces}{Backtrace collection continues}
  \begin{columns}[c]
    \begin{column}{0.55\textwidth}
      \minipage[c][0.75\textheight][s]{\columnwidth}
      \begin{itemize}
        \item<1-> \funcname{caml\_stash\_backtrace} scans the older part of the stack, up to the next trap
        \item<6-> Controls jumps to the nested exception handler in \funcname{maybe\_raise}, which matches only on \typename{ExnB}
        \item<7-> \funcname{caml\_reraise\_exn} is called again, backtrace collection resumes with \funcname{caml\_stash\_backtrace}
      \end{itemize}
      \medskip
      \begin{itemize}
        \item<2-> Constructed backtrace:
        \begin{itemize}
          \item <2-> \funcname{definitely\_raise}
          \item <2-> \funcname{probably\_raise}
          \item <3-> \funcname{probably\_raise} (re-raise)
          \item <4-> \funcname{maybe\_raise}
          \item <5-> \funcname{main}
          \item <8-> \funcname{main} (re-raise)
        \end{itemize}
      \end{itemize}
      \vfill
      \onslide*<1-5>{\mintocaml[firstline=15,lastline=21]{../src/backtrace/backtrace.ml}}
      \onslide*<6>{\mintocaml[firstline=28,lastline=34,highlightlines=31]{../src/backtrace/backtrace.ml}}
      \onslide*<7->{\mintocaml[firstline=28,lastline=34,highlightlines=33]{../src/backtrace/backtrace.ml}}
      \endminipage
    \end{column}
    \begin{column}{0.45\textwidth}
      \raggedleft
      \onslide*<1>{\providecommand\step{6}\input{diagrams/backtrace/backtrace.tex}}%
      \onslide*<2>{\providecommand\step{6}\input{diagrams/backtrace/backtrace.tex}}%
      \onslide*<3>{\providecommand\step{7}\input{diagrams/backtrace/backtrace.tex}}%
      \onslide*<4>{\providecommand\step{8}\input{diagrams/backtrace/backtrace.tex}}%
      \onslide*<5>{\providecommand\step{9}\input{diagrams/backtrace/backtrace.tex}}%
      \onslide*<6>{\providecommand\step{10}\input{diagrams/backtrace/backtrace.tex}}%
      \onslide*<7>{\providecommand\step{11}\input{diagrams/backtrace/backtrace.tex}}%
      \onslide*<8>{\providecommand\step{12}\input{diagrams/backtrace/backtrace.tex}}%
      \onslide*<9>{\providecommand\step{13}\input{diagrams/backtrace/backtrace.tex}}%
    \end{column}
  \end{columns}
\end{frame}

\begin{frame}{Backtraces}{Printing the backtrace}
  \begin{columns}[c]
    \begin{column}{0.45\textwidth}
      \minipage[c][0.66\textheight][s]{\columnwidth}
      \begin{itemize}
        \item<1-> The exception backtrace can now be printed to \funcarg{stdout} with \funcname{Printexc.print_backtrace}
        \item<2-> First the exception backtrace is retrieved into an OCaml array with \funcname{get_raw_backtrace }
        \item<3-> Which is implemented as the \funcname{caml_get_exception_raw_backtrace} C function in the runtime
        \item<4-> And finally printed with \funcname{print_exception_backtrace}
      \end{itemize}
      \vfill
      \mintocaml[firstline=28,lastline=34,highlightlines=33]{../src/backtrace/backtrace.ml}
      \endminipage
    \end{column}
    \begin{column}{0.55\textwidth}
      \onslide*<1,2>{
        \mintocaml[firstline=100,lastline=101]{../ocaml/stdlib/printexc.ml}
      }
      \only<1>{
        \listocaml[firstline=170,lastline=172]{../ocaml/stdlib/printexc.ml}{stdlib/printexc.ml:170}
      }
      \only<2>{
        \listocaml[firstline=170,lastline=172,highlightlines=172]{../ocaml/stdlib/printexc.ml}{stdlib/printexc.ml:170}
      }
      \only<3>{
        \mintc[firstline=154,lastline=156]{../ocaml/runtime/backtrace.c}
        \listc[firstline=171,lastline=192,highlightlines={184-187}]{../ocaml/runtime/backtrace.c}{runtime/backtrace.c:154}
      }
      \only<4>{
        \listocaml[firstline=155,lastline=165]{../ocaml/stdlib/printexc.ml}{stdlib/printexc.ml:55}
      }
    \end{column}
  \end{columns}
\end{frame}


\subsection{The multiple kinds of raise}
\frameSubsection{}{}

\begin{frame}{Backtraces}{When backtraces are not needed}
  \begin{columns}[c]
    \begin{column}{0.45\textwidth}
      \minipage[c][0.66\textheight][s]{\columnwidth}
      \begin{itemize}
        \item<1-> What if the backtrace for an exception is never needed?
        \item<2-> It's possible to raise the exception explicitly with \funcname{raise\_notrace} to make raising as cheap as possible
        \item<3-> An example of use in the \funcname{dynlink} library of the OCaml compiler
      \end{itemize}
      \vfill
      \onslide<3->{
      \listocaml[firstline=205,lastline=209,highlightlines=207]{../ocaml/otherlibs/dynlink/dynlink\_compilerlibs/misc.ml}{otherlibs/dynlink/dynlink\_compilerlibs/misc.ml:205}
      }
      \endminipage
    \end{column}
    \begin{column}{0.55\textwidth}
    \only<4>{
      \mintcmm[highlightlines=3]{../src/notrace/camlNotrace\_\_fun_347.cmm}
      \listcmm{../src/notrace/camlNotrace\_\_all\_somes\_327.cmm}{all\_somes}
    }
    \onslide*<5->{
      \listobjdump[highlightlines={6-9}]{../src/notrace/camlNotrace\_\_fun_347.objdump}{anonymous function from \funcname{all\_somes}}
    }
    \end{column}
  \end{columns}
\end{frame}

\begin{frame}{Backtraces}{Summary table of the multiple kinds of raise}
  \begin{itemize}
    \item When debugging information is enabled and if the backtrace of an exception isn't needed, it's possible to raise with \funcname{raise\_notrace} for performance
    \item When debugging information is \emph{not} enabled, the compiler optimises \funcname{raise} calls to inline assembly (same effect as using \funcname{raise\_notrace})
  \end{itemize}
  \bigskip
  \centering
  \begin{tabular}{| l | c | c | c | c |}
    \hline
    \parbox{10em}{Debug information \\ (\funcarg{-g} flag)}  & \multicolumn{2}{|c|}{Disabled}                                     & \multicolumn{2}{|c|}{Enabled} \\
    \hline
    Raise kind                                               & \funcname{raise} & \funcname{raise\_notrace}                       & \funcname{raise} & \funcname{raise\_notrace} \\
    \hline
    Compilation                                              & \parbox{8em}{inline assembly\\ (optimization)} & inline assembly   & \funcname{caml\_raise\_exn} & inline assembly \\
    \hline
  \end{tabular}
\end{frame}


%
% Takeaway #5
%
\subsection*{Takeaway \#5}
\frameSubsectionTakeaway{}
\againframe<5>{takeaway}
}%
      \onslide*<7>{\providecommand\step{11}%
% Backtraces
%
\subsection{One advantage of raising with debugging information}
\frameSubsection{}{}

\begin{frame}{Backtraces}{Debugging information \& source code}
  \begin{columns}[c]
    \begin{column}{0.5\textwidth}
      \begin{itemize}
        \item Raising an exception is fun and games, but how do we know where the exception originated? $\rightarrow$ With the backtrace associated with the exception!
        \item In order to get a backtrace from the point where an exception was raised, it's necessary to compile with debugging information enabled (\funcarg{-g} flag for the compiler)
        \item Same example as in the `Nested handlers' section
      \end{itemize}
    \end{column}
    \begin{column}{0.5\textwidth}
      \listocaml[firstline=9,lastline=34]{../src/backtrace/backtrace.ml}{backtrace/backtrace.ml}
    \end{column}
  \end{columns}
\end{frame}

\begin{frame}{Backtraces}{CMM and disassembly of \funcname{definitely\_raise}}
  \begin{columns}[c]
    \begin{column}{0.5\textwidth}
      \minipage[c][0.66\textheight][s]{\columnwidth}
      \begin{itemize}
        \item<1-> An effect of compiling with debugging information (\funcarg{-g}): the CMM no longer uses \funcname{raise\_notrace} to raise the exception but \funcname{raise} instead
        \item<2-> Disassembly shows that CMM \funcname{raise} is actually a call to \funcname{caml\_raise\_exn}, a function in assembler provided by the runtime
      \end{itemize}
      \vfill
      \mintocaml[firstline=9,lastline=13,highlightlines=12]{../src/backtrace/backtrace.ml}
      \endminipage
    \end{column}
    \begin{column}{0.5\textwidth}
      \onslide*<1>{
        \mintcmm[highlightlines=5]{../src/backtrace/camlBacktrace\_\_definitely\_raise\_347.cmm}
      }
      \onslide*<2>{
        \mintobjdump[firstline=2,highlightlines=14]{../src/backtrace/camlBacktrace\_\_definitely\_raise\_347.objdump}
      }
    \end{column}
  \end{columns}
\end{frame}

\begin{frame}{Backtraces}{Introducing \funcname{caml\_raise\_exn}}
  \begin{columns}[c]
    \begin{column}{0.5\textwidth}
      \minipage[c][0.66\textheight][s]{\columnwidth}
      \onslide*<1>{
        \begin{itemize}
          \item Raising the exception is performed using \funcname{caml\_raise\_exn} which is
            \begin{itemize}
              \item An assembly function provided by the runtime
              \item Architecture specific
            \end{itemize}
          \item Let's see what it does differently from \funcname{raise\_notrace}\dots
          \item We are about to raise \typename{ExnC} with \funcname{caml\_raise\_exn} from \funcname{definitely\_raise}
        \end{itemize}
      }
      \onslide*<2>{
        \mintobjdump[firstline=2,highlightlines=14]{../src/backtrace/camlBacktrace\_\_definitely\_raise\_347.objdump}
      }
      \vfill
      \mintocaml[firstline=9,lastline=13,highlightlines=12]{../src/backtrace/backtrace.ml}
      \endminipage
    \end{column}
    \begin{column}{0.5\textwidth}
      \centering
      \providecommand\step{1}\input{diagrams/backtrace/backtrace.tex}
    \end{column}
  \end{columns}
\end{frame}

\begin{frame}{Backtraces}{But before that, we need more fields from \typename{caml\_domain\_state}}
  \begin{columns}
    \begin{column}{0.5\textwidth}
      \begin{itemize}
        \item Remember \typename{caml\_domain\_state} the per-domain runtime state?
        \item Some other members are of interest to us now\dots
        \item \localname{backtrace\_active} is used to know if the runtime should collect backtraces
        \item \localname{backtrace\_active} can be set by
          \begin{itemize}
            \item The \funcarg{b} flag in the \funcname{OCAMLRUNPARAM} environment variable
            \item \mintinline{ocaml}{Printexc.record_backtrace : bool -> unit} \footnotemark
          \end{itemize}
      \end{itemize}
    \end{column}
    \begin{column}{0.5\textwidth}
      \begin{listing}
        \mintc[highlightlines={9-11}]{src/ocaml/domain\_state\_backtrace.h}
        \caption{runtime/caml/domain\_state.h}
      \end{listing}
    \end{column}
  \end{columns}
  \footnotetext[1]{https://v2.ocaml.org/api/Printexc.html}
\end{frame}

\newcommand\listCamlRaiseExn[1][]{
  \listasm[firstline=702,lastline=725,#1]{../ocaml/runtime/amd64.S}{runtime/amd64.S:702}}

\begin{frame}{Backtraces}{Walk through \funcname{caml\_raise\_exn}}
  \begin{columns}[T]
    \begin{column}{0.5\textwidth}
      \begin{itemize}
        \item<1-> First checks whether exceptions backtrace should be recorded
        \item<1-> By checking the \funcarg{domain\_state->backtrace\_active} value
        \item<2-> If not, acts as \funcname{raise\_notrace} with \funcname{RESTORE\_EXN\_HANDLER\_OCAML}
          \begin{itemize}
            \item Set \regname{RSP} to \localname{domain\_state->exn\_handler} value
            \item Restore \localname{domain\_state->exn\_handler} from trap
            \item Jump with \funcname{ret} (same effect as using \funcname{pop} and \funcname{jmp})
          \end{itemize}
        \item<3-> Otherwise, switches to the C stack to be able to execute C code
        \item<4-> Calls \funcname{caml\_stash\_backtrace}, a C function provided by the runtime\ignorespaces
          \onslide<6->{, with
            \begin{itemize}
              \item \funcarg{exn}: exception value
              \item \funcarg{pc}: code address of the raising point
              \item \funcarg{sp}: stack address of the raising function
              \item \funcarg{trapsp}: stack address of the next handler
            \end{itemize}
          }
      \end{itemize}
      \bigskip
      \mintocaml[firstline=9,lastline=13,highlightlines=12]{../src/backtrace/backtrace.ml}
    \end{column}
    \begin{column}{0.5\textwidth}
      \raggedleft
      \onslide*<1,3-4>{
        \mintc[firstline=190,lastline=192]{../ocaml/runtime/amd64.S}
        \mintc[firstline=234,lastline=237]{../ocaml/runtime/amd64.S}
      }
      \onslide*<2>{
        \mintc[firstline=190,lastline=192,highlightlines={191-192}]{../ocaml/runtime/amd64.S}
        \mintc[firstline=234,lastline=237,highlightlines={235-237}]{../ocaml/runtime/amd64.S}
      }
      \onslide*<1>{\listCamlRaiseExn[highlightlines=705]{}}%
      \onslide*<2>{\listCamlRaiseExn[highlightlines={707-708}]{}}%
      \onslide*<3>{\listCamlRaiseExn[highlightlines={712-714}]{}}%
      \onslide*<4>{\listCamlRaiseExn[highlightlines={715-720}]{}}%
      \onslide*<5>{\providecommand\step{1}\input{diagrams/backtrace/backtrace.tex}}%
      \onslide*<6>{\providecommand\step{2}\input{diagrams/backtrace/backtrace.tex}}%
    \end{column}
  \end{columns}
\end{frame}

\newcommand\listStashBacktrace[1][]{
  \listc[firstline=91,lastline=120,#1]{../ocaml/runtime/backtrace\_nat.c}{runtime/backtrace\_nat.c:91}}

\begin{frame}{Backtraces}{Introducing \funcname{caml\_stash\_backtrace}}
  \begin{columns}[c]
    \begin{column}{0.5\textwidth}
      \minipage[c][0.66\textheight][s]{\columnwidth}
      \begin{itemize}
        \item<1-> A C function provided by the runtime (specific to native code)
        \item<2-> It scans the stack, between the stack pointer at the raising point up to the next trap, in order to collect the names of the detected function frames
          \onslide*<2>{
            \begin{itemize}
              \item And more information, like line number, column number...
            \end{itemize}
          }
        \item<3-> It uses the same technique and information as the Garbage Collector (GC) uses to scan the values live on the stack
          \onslide*<4>{
            \begin{itemize}
              \item \typename{frame\_descr} are at the core of stack unwinding
            \end{itemize}
          }
      \end{itemize}
      \vfill
      \mintocaml[firstline=9,lastline=13,highlightlines=12]{../src/backtrace/backtrace.ml}
      \endminipage
    \end{column}
    \begin{column}{0.5\textwidth}
      \onslide*<1>{\listStashBacktrace[highlightlines=91]{}}
      \onslide*<2>{\listStashBacktrace[highlightlines={91,117-118}]{}}
      \onslide*<3->{\listStashBacktrace[highlightlines={94,106,109-110}]{}}
    \end{column}
  \end{columns}
\end{frame}

\newcommand\listFrameDescr[1][]{
  \listocaml[firstline=111,lastline=124,#1]{../ocaml/asmcomp/emitaux.ml}{asmcomp/emitaux.ml:111}}
\newcommand\listDebugInfo[1][]{
  \listocaml[firstline=38,lastline=49,#1]{../ocaml/lambda/debuginfo.mli}{lambda/debuginfo.mli:38}}

\begin{frame}{Backtraces}{\typename{frame\_descr} and \typename{frame\_debuginfo} in a nutshell}
  \begin{columns}[c]
    \begin{column}{0.5\textwidth}
      \begin{itemize}
        \item<1-> For every return address in an OCaml program, there is a associated \typename{frame\_descr}
        \item<2-> A \typename{frame\_descr} specifies
        \begin{itemize}
          \item<2-> The size of the function stack frame
          \item<3-> Where live values are on the stack (so that the GC can mark them as live and prevent from being swept)
        \end{itemize}
        \item<4-> When compiled with debug information, \typename{frame\_descr} will be linked to a \typename{frame\_debuginfo}
        \begin{itemize}
          \item<5-> Contains information about the position in the source code (file name, line and column number)
        \end{itemize}
      \end{itemize}
    \end{column}
    \begin{column}{0.5\textwidth}
        \onslide*<1>{\listFrameDescr[highlightlines={118-122}]{}}
        \onslide*<2>{\listFrameDescr[highlightlines={120}]{}}
        \onslide*<3>{\listFrameDescr[highlightlines={121}]{}}
        \onslide*<4->{\listFrameDescr[highlightlines={122}]{}}
        \medskip
        \onslide*<1-3>{\listDebugInfo{}}
        \onslide*<4>{\listDebugInfo[highlightlines={38,49}]{}}
        \onslide*<5>{\listDebugInfo[highlightlines={39-46}]{}}
    \end{column}
  \end{columns}
\end{frame}

\begin{frame}{Backtraces}{Scanning the stack with \typename{frame\_descr}}
  \begin{columns}[c]
    \begin{column}{0.5\textwidth}
      \begin{itemize}
        \item Given the return address of the code that raises the exception by calling \funcname{caml\_raise\_exn} (\funcarg{pc} argument of \funcname{caml\_stash\_backtrace})
        \item And the position of the stack pointer (\funcarg{sp} argument of \funcname{caml\_stash\_backtrace})
        \item The frame size given by \typename{frame\_descr}.\funcarg{frame\_size}, allows to find the end of the previous frame
        \item And the return address of the caller of this function
        \bigskip
        \onslide<3->{
        \item Note that traps (here the reddish \funcname{ExnA}, \funcname{ExnB} and \funcname{ExnC} blocks) belong to the frame that installed them, and so the frame size changes when traps are removed
        }
      \end{itemize}
    \end{column}
    \begin{column}{0.5\textwidth}
      \raggedleft
      \onslide*<1>{\providecommand\step{2}\input{diagrams/backtrace/backtrace.tex}}%
      \onslide*<2>{\providecommand\step{1}\input{diagrams/backtrace/scanstack.tex}}%
      \onslide*<3>{\providecommand\step{2}\input{diagrams/backtrace/scanstack.tex}}%
      \onslide*<4>{\providecommand\step{3}\input{diagrams/backtrace/scanstack.tex}}%
      \onslide*<5>{\providecommand\step{4}\input{diagrams/backtrace/scanstack.tex}}%
      \onslide*<6>{\providecommand\step{5}\input{diagrams/backtrace/scanstack.tex}}%
    \end{column}
  \end{columns}
\end{frame}

% Duplicate of previous-previous slide
\begin{frame}{Backtraces}{Continuing on \funcname{caml\_stash\_backtrace}}
  \begin{columns}[c]
    \begin{column}{0.5\textwidth}
      \minipage[c][0.75\textheight][s]{\columnwidth}
      \begin{itemize}
          \color{gray}
        \item A C function provided by the runtime (specific to native code)
        \item It scans the stack, between the stack pointer at the raising point up to the next trap, in order to collect the names of the detected function frames
        \item It uses the same technique and information as the Garbage Collector (GC) uses to scan the values live on the stack
      \end{itemize}
      \begin{itemize}
        \item For each function frame detected, store a pointer to the \typename{frame\_descr} in the \localname{domain\_state->backtrace\_buffer} array, effectively constructing the backtrace
      \end{itemize}
      \onslide<2->{
        \begin{itemize}
            \smallskip
          \item Constructed backtrace:
            \funcname{definitely\_raise},
            \onslide<3->{\funcname{probably\_raise}}
        \end{itemize}
      }
      \vfill
      \mintocaml[firstline=9,lastline=13,highlightlines=12]{../src/backtrace/backtrace.ml}
      \endminipage
    \end{column}
    \begin{column}{0.5\textwidth}
      \raggedleft
      \onslide*<1>{\listStashBacktrace[highlightlines={114-115}]{}}
      \onslide*<2>{\providecommand\step{3}\input{diagrams/backtrace/backtrace.tex}}%
      \onslide*<3>{\providecommand\step{4}\input{diagrams/backtrace/backtrace.tex}}%
    \end{column}
  \end{columns}
\end{frame}

\begin{frame}{Backtraces}{Back to \funcname{caml\_raise\_exn}}
  \begin{columns}[c]
    \begin{column}{0.5\textwidth}
      \minipage[c][0.66\textheight][s]{\columnwidth}
      \begin{itemize}\color{gray}
        \item Checks wheteher exceptions backtrace should be recorded, by checking the \funcarg{domain\_state->backtrace\_active} value
        \item Switches to the C stack to be able to execute C code
        \item Calls \funcname{caml\_stash\_backtrace}, to collect the backtrace up to the next trap
      \end{itemize}
      \onslide*<2->{
        \begin{itemize}
          \item<2-> Executes the exception handler in \funcname{probably\_raise} with \funcname{RESTORE\_EXN\_HANDLER\_OCAML}
            \begin{itemize}
              \item Set \regname{RSP} to \localname{domain\_state->exn\_handler} value
              \item Restore \localname{domain\_state->exn\_handler} from trap
              \item Jump to the handler code with \funcname{ret}
            \end{itemize}
        \end{itemize}
      }
      \vfill
      \onslide*<1>{\mintocaml[firstline=9,lastline=13,highlightlines=12]{../src/backtrace/backtrace.ml}}
      \onslide*<2->{\mintocaml[firstline=15,lastline=21,highlightlines={19-21}]{../src/backtrace/backtrace.ml}}
      \endminipage
    \end{column}
    \begin{column}{0.5\textwidth}
      \centering
      \onslide*<1>{
        \mintc[firstline=190,lastline=192]{../ocaml/runtime/amd64.S}
        \mintc[firstline=234,lastline=237]{../ocaml/runtime/amd64.S}
      }
      \onslide*<2>{
        \mintc[firstline=190,lastline=192,highlightlines={191-192}]{../ocaml/runtime/amd64.S}
        \mintc[firstline=234,lastline=237,highlightlines={235-237}]{../ocaml/runtime/amd64.S}
      }
      \onslide*<1>{\listCamlRaiseExn[highlightlines=720]{}}
      \onslide*<2>{\listCamlRaiseExn[highlightlines=722]{}}
      \onslide*<3->{\providecommand\step{5}\input{diagrams/backtrace/backtrace.tex}}
    \end{column}
  \end{columns}
\end{frame}

\begin{frame}{Backtraces}{\funcname{caml\_probably\_raise}'s handler doesn't catch \typename{ExnC}}
  \begin{columns}[c]
    \begin{column}{0.45\textwidth}
      \minipage[c][0.75\textheight][s]{\columnwidth}
      \begin{itemize}
        \item<1-> \funcname{match \dots with}\footnotemark pattern matching can also match exceptions
        \item<1-> The CMM of \funcname{probably\_raise} shows that this \funcname{match \dots with} is implemented with a pattern matching and a \funcname{try \dots with}
        \item<1-> But this one only matches on \typename{ExnA} exception
        \item<2-> Since the pattern matching fails to catch the exception, \funcname{reraise} is called with our current \typename{ExnC} exception
        \item<3-> Disassembly shows that \funcname{reraise} is actually a call to \funcname{caml\_reraise\_exn}, which is also an assembly function provided by the runtime
      \end{itemize}
      \vfill
      \mintocaml[firstline=15,lastline=21,highlightlines={18-21}]{../src/backtrace/backtrace.ml}
      \endminipage
    \end{column}
    \begin{column}{0.55\textwidth}
      \centering
      \onslide*<1>{\mintcmm[highlightlines={10-18}]{../src/backtrace/camlBacktrace\_\_probably\_raise\_351.cmm}}
      \onslide*<2>{\listcmm[highlightlines=18]{../src/backtrace/camlBacktrace\_\_probably\_raise_351.cmm}{CMM of probably\_raise}}
      \onslide*<3>{\mintobjdump[firstline=5,lastline=35,highlightlines=31]{../src/backtrace/camlBacktrace\_\_probably\_raise_351.objdump}}
    \end{column}
  \end{columns}
  \only<1>{\footnotetext[1]{https://blog.janestreet.com/pattern-matching-and-exception-handling-unite/}}
\end{frame}

\newcommand\listCamlReraiseExn[1][]{
  \listasm[firstline=727,lastline=734,#1]{../ocaml/runtime/amd64.S}{runtime/amd64.S:727}}

\begin{frame}{Backtraces}{Introducing \funcname{caml\_reraise\_exn}}
  \begin{columns}[c]
    \begin{column}{0.5\textwidth}
      \minipage[c][0.66\textheight][s]{\columnwidth}
      \begin{itemize}
        \item<1-> The exception have to be reraised to the next handler
        \item<2-> First, checks if the backtrace should be collected with \funcarg{domain\_state->backtrace\_active}
        \only<3>{
        \begin{itemize}
            \item If not, behaves like a \funcname{raise\_notrace} with \funcname{RESTORE\_EXN\_HANDLER\_OCAML}
        \end{itemize}
        }
        \item<4-> Jumps into \funcname{caml\_raise\_exn}
        \item<5-> Calls \funcname{caml\_stash\_backtrace} again, to scan the next part of the stack: from the trap just pop-ed to the next trap
      \end{itemize}
      \vfill
      \mintocaml[firstline=15,lastline=21,highlightlines={18-21}]{../src/backtrace/backtrace.ml}
      \endminipage
    \end{column}
    \begin{column}{0.5\textwidth}
      \raggedleft
      \onslide*<1>{\listCamlReraiseExn{}}
      \onslide*<2>{\listCamlReraiseExn[highlightlines=729]{}}
      \onslide*<3>{
        \mintc[firstline=190,lastline=192,highlightlines={191-192}]{../ocaml/runtime/amd64.S}
        \mintc[firstline=234,lastline=237,highlightlines={235-237}]{../ocaml/runtime/amd64.S}
        \medskip
        \listCamlReraiseExn[highlightlines=731]{}
      }
      \onslide*<4-5>{
        % TODO refactor with \listCamlReraiseExn
        \mintasm[firstline=727,lastline=734,highlightlines=730]{../ocaml/runtime/amd64.S}
        \medskip
      }
      \onslide*<4>{\listCamlRaiseExn[highlightlines=711]{}}
      \onslide*<5>{\listCamlRaiseExn[highlightlines=720]{}}
    \end{column}
  \end{columns}
\end{frame}

\begin{frame}{Backtraces}{Backtrace collection continues}
  \begin{columns}[c]
    \begin{column}{0.55\textwidth}
      \minipage[c][0.75\textheight][s]{\columnwidth}
      \begin{itemize}
        \item<1-> \funcname{caml\_stash\_backtrace} scans the older part of the stack, up to the next trap
        \item<6-> Controls jumps to the nested exception handler in \funcname{maybe\_raise}, which matches only on \typename{ExnB}
        \item<7-> \funcname{caml\_reraise\_exn} is called again, backtrace collection resumes with \funcname{caml\_stash\_backtrace}
      \end{itemize}
      \medskip
      \begin{itemize}
        \item<2-> Constructed backtrace:
        \begin{itemize}
          \item <2-> \funcname{definitely\_raise}
          \item <2-> \funcname{probably\_raise}
          \item <3-> \funcname{probably\_raise} (re-raise)
          \item <4-> \funcname{maybe\_raise}
          \item <5-> \funcname{main}
          \item <8-> \funcname{main} (re-raise)
        \end{itemize}
      \end{itemize}
      \vfill
      \onslide*<1-5>{\mintocaml[firstline=15,lastline=21]{../src/backtrace/backtrace.ml}}
      \onslide*<6>{\mintocaml[firstline=28,lastline=34,highlightlines=31]{../src/backtrace/backtrace.ml}}
      \onslide*<7->{\mintocaml[firstline=28,lastline=34,highlightlines=33]{../src/backtrace/backtrace.ml}}
      \endminipage
    \end{column}
    \begin{column}{0.45\textwidth}
      \raggedleft
      \onslide*<1>{\providecommand\step{6}\input{diagrams/backtrace/backtrace.tex}}%
      \onslide*<2>{\providecommand\step{6}\input{diagrams/backtrace/backtrace.tex}}%
      \onslide*<3>{\providecommand\step{7}\input{diagrams/backtrace/backtrace.tex}}%
      \onslide*<4>{\providecommand\step{8}\input{diagrams/backtrace/backtrace.tex}}%
      \onslide*<5>{\providecommand\step{9}\input{diagrams/backtrace/backtrace.tex}}%
      \onslide*<6>{\providecommand\step{10}\input{diagrams/backtrace/backtrace.tex}}%
      \onslide*<7>{\providecommand\step{11}\input{diagrams/backtrace/backtrace.tex}}%
      \onslide*<8>{\providecommand\step{12}\input{diagrams/backtrace/backtrace.tex}}%
      \onslide*<9>{\providecommand\step{13}\input{diagrams/backtrace/backtrace.tex}}%
    \end{column}
  \end{columns}
\end{frame}

\begin{frame}{Backtraces}{Printing the backtrace}
  \begin{columns}[c]
    \begin{column}{0.45\textwidth}
      \minipage[c][0.66\textheight][s]{\columnwidth}
      \begin{itemize}
        \item<1-> The exception backtrace can now be printed to \funcarg{stdout} with \funcname{Printexc.print_backtrace}
        \item<2-> First the exception backtrace is retrieved into an OCaml array with \funcname{get_raw_backtrace }
        \item<3-> Which is implemented as the \funcname{caml_get_exception_raw_backtrace} C function in the runtime
        \item<4-> And finally printed with \funcname{print_exception_backtrace}
      \end{itemize}
      \vfill
      \mintocaml[firstline=28,lastline=34,highlightlines=33]{../src/backtrace/backtrace.ml}
      \endminipage
    \end{column}
    \begin{column}{0.55\textwidth}
      \onslide*<1,2>{
        \mintocaml[firstline=100,lastline=101]{../ocaml/stdlib/printexc.ml}
      }
      \only<1>{
        \listocaml[firstline=170,lastline=172]{../ocaml/stdlib/printexc.ml}{stdlib/printexc.ml:170}
      }
      \only<2>{
        \listocaml[firstline=170,lastline=172,highlightlines=172]{../ocaml/stdlib/printexc.ml}{stdlib/printexc.ml:170}
      }
      \only<3>{
        \mintc[firstline=154,lastline=156]{../ocaml/runtime/backtrace.c}
        \listc[firstline=171,lastline=192,highlightlines={184-187}]{../ocaml/runtime/backtrace.c}{runtime/backtrace.c:154}
      }
      \only<4>{
        \listocaml[firstline=155,lastline=165]{../ocaml/stdlib/printexc.ml}{stdlib/printexc.ml:55}
      }
    \end{column}
  \end{columns}
\end{frame}


\subsection{The multiple kinds of raise}
\frameSubsection{}{}

\begin{frame}{Backtraces}{When backtraces are not needed}
  \begin{columns}[c]
    \begin{column}{0.45\textwidth}
      \minipage[c][0.66\textheight][s]{\columnwidth}
      \begin{itemize}
        \item<1-> What if the backtrace for an exception is never needed?
        \item<2-> It's possible to raise the exception explicitly with \funcname{raise\_notrace} to make raising as cheap as possible
        \item<3-> An example of use in the \funcname{dynlink} library of the OCaml compiler
      \end{itemize}
      \vfill
      \onslide<3->{
      \listocaml[firstline=205,lastline=209,highlightlines=207]{../ocaml/otherlibs/dynlink/dynlink\_compilerlibs/misc.ml}{otherlibs/dynlink/dynlink\_compilerlibs/misc.ml:205}
      }
      \endminipage
    \end{column}
    \begin{column}{0.55\textwidth}
    \only<4>{
      \mintcmm[highlightlines=3]{../src/notrace/camlNotrace\_\_fun_347.cmm}
      \listcmm{../src/notrace/camlNotrace\_\_all\_somes\_327.cmm}{all\_somes}
    }
    \onslide*<5->{
      \listobjdump[highlightlines={6-9}]{../src/notrace/camlNotrace\_\_fun_347.objdump}{anonymous function from \funcname{all\_somes}}
    }
    \end{column}
  \end{columns}
\end{frame}

\begin{frame}{Backtraces}{Summary table of the multiple kinds of raise}
  \begin{itemize}
    \item When debugging information is enabled and if the backtrace of an exception isn't needed, it's possible to raise with \funcname{raise\_notrace} for performance
    \item When debugging information is \emph{not} enabled, the compiler optimises \funcname{raise} calls to inline assembly (same effect as using \funcname{raise\_notrace})
  \end{itemize}
  \bigskip
  \centering
  \begin{tabular}{| l | c | c | c | c |}
    \hline
    \parbox{10em}{Debug information \\ (\funcarg{-g} flag)}  & \multicolumn{2}{|c|}{Disabled}                                     & \multicolumn{2}{|c|}{Enabled} \\
    \hline
    Raise kind                                               & \funcname{raise} & \funcname{raise\_notrace}                       & \funcname{raise} & \funcname{raise\_notrace} \\
    \hline
    Compilation                                              & \parbox{8em}{inline assembly\\ (optimization)} & inline assembly   & \funcname{caml\_raise\_exn} & inline assembly \\
    \hline
  \end{tabular}
\end{frame}


%
% Takeaway #5
%
\subsection*{Takeaway \#5}
\frameSubsectionTakeaway{}
\againframe<5>{takeaway}
}%
      \onslide*<8>{\providecommand\step{12}%
% Backtraces
%
\subsection{One advantage of raising with debugging information}
\frameSubsection{}{}

\begin{frame}{Backtraces}{Debugging information \& source code}
  \begin{columns}[c]
    \begin{column}{0.5\textwidth}
      \begin{itemize}
        \item Raising an exception is fun and games, but how do we know where the exception originated? $\rightarrow$ With the backtrace associated with the exception!
        \item In order to get a backtrace from the point where an exception was raised, it's necessary to compile with debugging information enabled (\funcarg{-g} flag for the compiler)
        \item Same example as in the `Nested handlers' section
      \end{itemize}
    \end{column}
    \begin{column}{0.5\textwidth}
      \listocaml[firstline=9,lastline=34]{../src/backtrace/backtrace.ml}{backtrace/backtrace.ml}
    \end{column}
  \end{columns}
\end{frame}

\begin{frame}{Backtraces}{CMM and disassembly of \funcname{definitely\_raise}}
  \begin{columns}[c]
    \begin{column}{0.5\textwidth}
      \minipage[c][0.66\textheight][s]{\columnwidth}
      \begin{itemize}
        \item<1-> An effect of compiling with debugging information (\funcarg{-g}): the CMM no longer uses \funcname{raise\_notrace} to raise the exception but \funcname{raise} instead
        \item<2-> Disassembly shows that CMM \funcname{raise} is actually a call to \funcname{caml\_raise\_exn}, a function in assembler provided by the runtime
      \end{itemize}
      \vfill
      \mintocaml[firstline=9,lastline=13,highlightlines=12]{../src/backtrace/backtrace.ml}
      \endminipage
    \end{column}
    \begin{column}{0.5\textwidth}
      \onslide*<1>{
        \mintcmm[highlightlines=5]{../src/backtrace/camlBacktrace\_\_definitely\_raise\_347.cmm}
      }
      \onslide*<2>{
        \mintobjdump[firstline=2,highlightlines=14]{../src/backtrace/camlBacktrace\_\_definitely\_raise\_347.objdump}
      }
    \end{column}
  \end{columns}
\end{frame}

\begin{frame}{Backtraces}{Introducing \funcname{caml\_raise\_exn}}
  \begin{columns}[c]
    \begin{column}{0.5\textwidth}
      \minipage[c][0.66\textheight][s]{\columnwidth}
      \onslide*<1>{
        \begin{itemize}
          \item Raising the exception is performed using \funcname{caml\_raise\_exn} which is
            \begin{itemize}
              \item An assembly function provided by the runtime
              \item Architecture specific
            \end{itemize}
          \item Let's see what it does differently from \funcname{raise\_notrace}\dots
          \item We are about to raise \typename{ExnC} with \funcname{caml\_raise\_exn} from \funcname{definitely\_raise}
        \end{itemize}
      }
      \onslide*<2>{
        \mintobjdump[firstline=2,highlightlines=14]{../src/backtrace/camlBacktrace\_\_definitely\_raise\_347.objdump}
      }
      \vfill
      \mintocaml[firstline=9,lastline=13,highlightlines=12]{../src/backtrace/backtrace.ml}
      \endminipage
    \end{column}
    \begin{column}{0.5\textwidth}
      \centering
      \providecommand\step{1}\input{diagrams/backtrace/backtrace.tex}
    \end{column}
  \end{columns}
\end{frame}

\begin{frame}{Backtraces}{But before that, we need more fields from \typename{caml\_domain\_state}}
  \begin{columns}
    \begin{column}{0.5\textwidth}
      \begin{itemize}
        \item Remember \typename{caml\_domain\_state} the per-domain runtime state?
        \item Some other members are of interest to us now\dots
        \item \localname{backtrace\_active} is used to know if the runtime should collect backtraces
        \item \localname{backtrace\_active} can be set by
          \begin{itemize}
            \item The \funcarg{b} flag in the \funcname{OCAMLRUNPARAM} environment variable
            \item \mintinline{ocaml}{Printexc.record_backtrace : bool -> unit} \footnotemark
          \end{itemize}
      \end{itemize}
    \end{column}
    \begin{column}{0.5\textwidth}
      \begin{listing}
        \mintc[highlightlines={9-11}]{src/ocaml/domain\_state\_backtrace.h}
        \caption{runtime/caml/domain\_state.h}
      \end{listing}
    \end{column}
  \end{columns}
  \footnotetext[1]{https://v2.ocaml.org/api/Printexc.html}
\end{frame}

\newcommand\listCamlRaiseExn[1][]{
  \listasm[firstline=702,lastline=725,#1]{../ocaml/runtime/amd64.S}{runtime/amd64.S:702}}

\begin{frame}{Backtraces}{Walk through \funcname{caml\_raise\_exn}}
  \begin{columns}[T]
    \begin{column}{0.5\textwidth}
      \begin{itemize}
        \item<1-> First checks whether exceptions backtrace should be recorded
        \item<1-> By checking the \funcarg{domain\_state->backtrace\_active} value
        \item<2-> If not, acts as \funcname{raise\_notrace} with \funcname{RESTORE\_EXN\_HANDLER\_OCAML}
          \begin{itemize}
            \item Set \regname{RSP} to \localname{domain\_state->exn\_handler} value
            \item Restore \localname{domain\_state->exn\_handler} from trap
            \item Jump with \funcname{ret} (same effect as using \funcname{pop} and \funcname{jmp})
          \end{itemize}
        \item<3-> Otherwise, switches to the C stack to be able to execute C code
        \item<4-> Calls \funcname{caml\_stash\_backtrace}, a C function provided by the runtime\ignorespaces
          \onslide<6->{, with
            \begin{itemize}
              \item \funcarg{exn}: exception value
              \item \funcarg{pc}: code address of the raising point
              \item \funcarg{sp}: stack address of the raising function
              \item \funcarg{trapsp}: stack address of the next handler
            \end{itemize}
          }
      \end{itemize}
      \bigskip
      \mintocaml[firstline=9,lastline=13,highlightlines=12]{../src/backtrace/backtrace.ml}
    \end{column}
    \begin{column}{0.5\textwidth}
      \raggedleft
      \onslide*<1,3-4>{
        \mintc[firstline=190,lastline=192]{../ocaml/runtime/amd64.S}
        \mintc[firstline=234,lastline=237]{../ocaml/runtime/amd64.S}
      }
      \onslide*<2>{
        \mintc[firstline=190,lastline=192,highlightlines={191-192}]{../ocaml/runtime/amd64.S}
        \mintc[firstline=234,lastline=237,highlightlines={235-237}]{../ocaml/runtime/amd64.S}
      }
      \onslide*<1>{\listCamlRaiseExn[highlightlines=705]{}}%
      \onslide*<2>{\listCamlRaiseExn[highlightlines={707-708}]{}}%
      \onslide*<3>{\listCamlRaiseExn[highlightlines={712-714}]{}}%
      \onslide*<4>{\listCamlRaiseExn[highlightlines={715-720}]{}}%
      \onslide*<5>{\providecommand\step{1}\input{diagrams/backtrace/backtrace.tex}}%
      \onslide*<6>{\providecommand\step{2}\input{diagrams/backtrace/backtrace.tex}}%
    \end{column}
  \end{columns}
\end{frame}

\newcommand\listStashBacktrace[1][]{
  \listc[firstline=91,lastline=120,#1]{../ocaml/runtime/backtrace\_nat.c}{runtime/backtrace\_nat.c:91}}

\begin{frame}{Backtraces}{Introducing \funcname{caml\_stash\_backtrace}}
  \begin{columns}[c]
    \begin{column}{0.5\textwidth}
      \minipage[c][0.66\textheight][s]{\columnwidth}
      \begin{itemize}
        \item<1-> A C function provided by the runtime (specific to native code)
        \item<2-> It scans the stack, between the stack pointer at the raising point up to the next trap, in order to collect the names of the detected function frames
          \onslide*<2>{
            \begin{itemize}
              \item And more information, like line number, column number...
            \end{itemize}
          }
        \item<3-> It uses the same technique and information as the Garbage Collector (GC) uses to scan the values live on the stack
          \onslide*<4>{
            \begin{itemize}
              \item \typename{frame\_descr} are at the core of stack unwinding
            \end{itemize}
          }
      \end{itemize}
      \vfill
      \mintocaml[firstline=9,lastline=13,highlightlines=12]{../src/backtrace/backtrace.ml}
      \endminipage
    \end{column}
    \begin{column}{0.5\textwidth}
      \onslide*<1>{\listStashBacktrace[highlightlines=91]{}}
      \onslide*<2>{\listStashBacktrace[highlightlines={91,117-118}]{}}
      \onslide*<3->{\listStashBacktrace[highlightlines={94,106,109-110}]{}}
    \end{column}
  \end{columns}
\end{frame}

\newcommand\listFrameDescr[1][]{
  \listocaml[firstline=111,lastline=124,#1]{../ocaml/asmcomp/emitaux.ml}{asmcomp/emitaux.ml:111}}
\newcommand\listDebugInfo[1][]{
  \listocaml[firstline=38,lastline=49,#1]{../ocaml/lambda/debuginfo.mli}{lambda/debuginfo.mli:38}}

\begin{frame}{Backtraces}{\typename{frame\_descr} and \typename{frame\_debuginfo} in a nutshell}
  \begin{columns}[c]
    \begin{column}{0.5\textwidth}
      \begin{itemize}
        \item<1-> For every return address in an OCaml program, there is a associated \typename{frame\_descr}
        \item<2-> A \typename{frame\_descr} specifies
        \begin{itemize}
          \item<2-> The size of the function stack frame
          \item<3-> Where live values are on the stack (so that the GC can mark them as live and prevent from being swept)
        \end{itemize}
        \item<4-> When compiled with debug information, \typename{frame\_descr} will be linked to a \typename{frame\_debuginfo}
        \begin{itemize}
          \item<5-> Contains information about the position in the source code (file name, line and column number)
        \end{itemize}
      \end{itemize}
    \end{column}
    \begin{column}{0.5\textwidth}
        \onslide*<1>{\listFrameDescr[highlightlines={118-122}]{}}
        \onslide*<2>{\listFrameDescr[highlightlines={120}]{}}
        \onslide*<3>{\listFrameDescr[highlightlines={121}]{}}
        \onslide*<4->{\listFrameDescr[highlightlines={122}]{}}
        \medskip
        \onslide*<1-3>{\listDebugInfo{}}
        \onslide*<4>{\listDebugInfo[highlightlines={38,49}]{}}
        \onslide*<5>{\listDebugInfo[highlightlines={39-46}]{}}
    \end{column}
  \end{columns}
\end{frame}

\begin{frame}{Backtraces}{Scanning the stack with \typename{frame\_descr}}
  \begin{columns}[c]
    \begin{column}{0.5\textwidth}
      \begin{itemize}
        \item Given the return address of the code that raises the exception by calling \funcname{caml\_raise\_exn} (\funcarg{pc} argument of \funcname{caml\_stash\_backtrace})
        \item And the position of the stack pointer (\funcarg{sp} argument of \funcname{caml\_stash\_backtrace})
        \item The frame size given by \typename{frame\_descr}.\funcarg{frame\_size}, allows to find the end of the previous frame
        \item And the return address of the caller of this function
        \bigskip
        \onslide<3->{
        \item Note that traps (here the reddish \funcname{ExnA}, \funcname{ExnB} and \funcname{ExnC} blocks) belong to the frame that installed them, and so the frame size changes when traps are removed
        }
      \end{itemize}
    \end{column}
    \begin{column}{0.5\textwidth}
      \raggedleft
      \onslide*<1>{\providecommand\step{2}\input{diagrams/backtrace/backtrace.tex}}%
      \onslide*<2>{\providecommand\step{1}\input{diagrams/backtrace/scanstack.tex}}%
      \onslide*<3>{\providecommand\step{2}\input{diagrams/backtrace/scanstack.tex}}%
      \onslide*<4>{\providecommand\step{3}\input{diagrams/backtrace/scanstack.tex}}%
      \onslide*<5>{\providecommand\step{4}\input{diagrams/backtrace/scanstack.tex}}%
      \onslide*<6>{\providecommand\step{5}\input{diagrams/backtrace/scanstack.tex}}%
    \end{column}
  \end{columns}
\end{frame}

% Duplicate of previous-previous slide
\begin{frame}{Backtraces}{Continuing on \funcname{caml\_stash\_backtrace}}
  \begin{columns}[c]
    \begin{column}{0.5\textwidth}
      \minipage[c][0.75\textheight][s]{\columnwidth}
      \begin{itemize}
          \color{gray}
        \item A C function provided by the runtime (specific to native code)
        \item It scans the stack, between the stack pointer at the raising point up to the next trap, in order to collect the names of the detected function frames
        \item It uses the same technique and information as the Garbage Collector (GC) uses to scan the values live on the stack
      \end{itemize}
      \begin{itemize}
        \item For each function frame detected, store a pointer to the \typename{frame\_descr} in the \localname{domain\_state->backtrace\_buffer} array, effectively constructing the backtrace
      \end{itemize}
      \onslide<2->{
        \begin{itemize}
            \smallskip
          \item Constructed backtrace:
            \funcname{definitely\_raise},
            \onslide<3->{\funcname{probably\_raise}}
        \end{itemize}
      }
      \vfill
      \mintocaml[firstline=9,lastline=13,highlightlines=12]{../src/backtrace/backtrace.ml}
      \endminipage
    \end{column}
    \begin{column}{0.5\textwidth}
      \raggedleft
      \onslide*<1>{\listStashBacktrace[highlightlines={114-115}]{}}
      \onslide*<2>{\providecommand\step{3}\input{diagrams/backtrace/backtrace.tex}}%
      \onslide*<3>{\providecommand\step{4}\input{diagrams/backtrace/backtrace.tex}}%
    \end{column}
  \end{columns}
\end{frame}

\begin{frame}{Backtraces}{Back to \funcname{caml\_raise\_exn}}
  \begin{columns}[c]
    \begin{column}{0.5\textwidth}
      \minipage[c][0.66\textheight][s]{\columnwidth}
      \begin{itemize}\color{gray}
        \item Checks wheteher exceptions backtrace should be recorded, by checking the \funcarg{domain\_state->backtrace\_active} value
        \item Switches to the C stack to be able to execute C code
        \item Calls \funcname{caml\_stash\_backtrace}, to collect the backtrace up to the next trap
      \end{itemize}
      \onslide*<2->{
        \begin{itemize}
          \item<2-> Executes the exception handler in \funcname{probably\_raise} with \funcname{RESTORE\_EXN\_HANDLER\_OCAML}
            \begin{itemize}
              \item Set \regname{RSP} to \localname{domain\_state->exn\_handler} value
              \item Restore \localname{domain\_state->exn\_handler} from trap
              \item Jump to the handler code with \funcname{ret}
            \end{itemize}
        \end{itemize}
      }
      \vfill
      \onslide*<1>{\mintocaml[firstline=9,lastline=13,highlightlines=12]{../src/backtrace/backtrace.ml}}
      \onslide*<2->{\mintocaml[firstline=15,lastline=21,highlightlines={19-21}]{../src/backtrace/backtrace.ml}}
      \endminipage
    \end{column}
    \begin{column}{0.5\textwidth}
      \centering
      \onslide*<1>{
        \mintc[firstline=190,lastline=192]{../ocaml/runtime/amd64.S}
        \mintc[firstline=234,lastline=237]{../ocaml/runtime/amd64.S}
      }
      \onslide*<2>{
        \mintc[firstline=190,lastline=192,highlightlines={191-192}]{../ocaml/runtime/amd64.S}
        \mintc[firstline=234,lastline=237,highlightlines={235-237}]{../ocaml/runtime/amd64.S}
      }
      \onslide*<1>{\listCamlRaiseExn[highlightlines=720]{}}
      \onslide*<2>{\listCamlRaiseExn[highlightlines=722]{}}
      \onslide*<3->{\providecommand\step{5}\input{diagrams/backtrace/backtrace.tex}}
    \end{column}
  \end{columns}
\end{frame}

\begin{frame}{Backtraces}{\funcname{caml\_probably\_raise}'s handler doesn't catch \typename{ExnC}}
  \begin{columns}[c]
    \begin{column}{0.45\textwidth}
      \minipage[c][0.75\textheight][s]{\columnwidth}
      \begin{itemize}
        \item<1-> \funcname{match \dots with}\footnotemark pattern matching can also match exceptions
        \item<1-> The CMM of \funcname{probably\_raise} shows that this \funcname{match \dots with} is implemented with a pattern matching and a \funcname{try \dots with}
        \item<1-> But this one only matches on \typename{ExnA} exception
        \item<2-> Since the pattern matching fails to catch the exception, \funcname{reraise} is called with our current \typename{ExnC} exception
        \item<3-> Disassembly shows that \funcname{reraise} is actually a call to \funcname{caml\_reraise\_exn}, which is also an assembly function provided by the runtime
      \end{itemize}
      \vfill
      \mintocaml[firstline=15,lastline=21,highlightlines={18-21}]{../src/backtrace/backtrace.ml}
      \endminipage
    \end{column}
    \begin{column}{0.55\textwidth}
      \centering
      \onslide*<1>{\mintcmm[highlightlines={10-18}]{../src/backtrace/camlBacktrace\_\_probably\_raise\_351.cmm}}
      \onslide*<2>{\listcmm[highlightlines=18]{../src/backtrace/camlBacktrace\_\_probably\_raise_351.cmm}{CMM of probably\_raise}}
      \onslide*<3>{\mintobjdump[firstline=5,lastline=35,highlightlines=31]{../src/backtrace/camlBacktrace\_\_probably\_raise_351.objdump}}
    \end{column}
  \end{columns}
  \only<1>{\footnotetext[1]{https://blog.janestreet.com/pattern-matching-and-exception-handling-unite/}}
\end{frame}

\newcommand\listCamlReraiseExn[1][]{
  \listasm[firstline=727,lastline=734,#1]{../ocaml/runtime/amd64.S}{runtime/amd64.S:727}}

\begin{frame}{Backtraces}{Introducing \funcname{caml\_reraise\_exn}}
  \begin{columns}[c]
    \begin{column}{0.5\textwidth}
      \minipage[c][0.66\textheight][s]{\columnwidth}
      \begin{itemize}
        \item<1-> The exception have to be reraised to the next handler
        \item<2-> First, checks if the backtrace should be collected with \funcarg{domain\_state->backtrace\_active}
        \only<3>{
        \begin{itemize}
            \item If not, behaves like a \funcname{raise\_notrace} with \funcname{RESTORE\_EXN\_HANDLER\_OCAML}
        \end{itemize}
        }
        \item<4-> Jumps into \funcname{caml\_raise\_exn}
        \item<5-> Calls \funcname{caml\_stash\_backtrace} again, to scan the next part of the stack: from the trap just pop-ed to the next trap
      \end{itemize}
      \vfill
      \mintocaml[firstline=15,lastline=21,highlightlines={18-21}]{../src/backtrace/backtrace.ml}
      \endminipage
    \end{column}
    \begin{column}{0.5\textwidth}
      \raggedleft
      \onslide*<1>{\listCamlReraiseExn{}}
      \onslide*<2>{\listCamlReraiseExn[highlightlines=729]{}}
      \onslide*<3>{
        \mintc[firstline=190,lastline=192,highlightlines={191-192}]{../ocaml/runtime/amd64.S}
        \mintc[firstline=234,lastline=237,highlightlines={235-237}]{../ocaml/runtime/amd64.S}
        \medskip
        \listCamlReraiseExn[highlightlines=731]{}
      }
      \onslide*<4-5>{
        % TODO refactor with \listCamlReraiseExn
        \mintasm[firstline=727,lastline=734,highlightlines=730]{../ocaml/runtime/amd64.S}
        \medskip
      }
      \onslide*<4>{\listCamlRaiseExn[highlightlines=711]{}}
      \onslide*<5>{\listCamlRaiseExn[highlightlines=720]{}}
    \end{column}
  \end{columns}
\end{frame}

\begin{frame}{Backtraces}{Backtrace collection continues}
  \begin{columns}[c]
    \begin{column}{0.55\textwidth}
      \minipage[c][0.75\textheight][s]{\columnwidth}
      \begin{itemize}
        \item<1-> \funcname{caml\_stash\_backtrace} scans the older part of the stack, up to the next trap
        \item<6-> Controls jumps to the nested exception handler in \funcname{maybe\_raise}, which matches only on \typename{ExnB}
        \item<7-> \funcname{caml\_reraise\_exn} is called again, backtrace collection resumes with \funcname{caml\_stash\_backtrace}
      \end{itemize}
      \medskip
      \begin{itemize}
        \item<2-> Constructed backtrace:
        \begin{itemize}
          \item <2-> \funcname{definitely\_raise}
          \item <2-> \funcname{probably\_raise}
          \item <3-> \funcname{probably\_raise} (re-raise)
          \item <4-> \funcname{maybe\_raise}
          \item <5-> \funcname{main}
          \item <8-> \funcname{main} (re-raise)
        \end{itemize}
      \end{itemize}
      \vfill
      \onslide*<1-5>{\mintocaml[firstline=15,lastline=21]{../src/backtrace/backtrace.ml}}
      \onslide*<6>{\mintocaml[firstline=28,lastline=34,highlightlines=31]{../src/backtrace/backtrace.ml}}
      \onslide*<7->{\mintocaml[firstline=28,lastline=34,highlightlines=33]{../src/backtrace/backtrace.ml}}
      \endminipage
    \end{column}
    \begin{column}{0.45\textwidth}
      \raggedleft
      \onslide*<1>{\providecommand\step{6}\input{diagrams/backtrace/backtrace.tex}}%
      \onslide*<2>{\providecommand\step{6}\input{diagrams/backtrace/backtrace.tex}}%
      \onslide*<3>{\providecommand\step{7}\input{diagrams/backtrace/backtrace.tex}}%
      \onslide*<4>{\providecommand\step{8}\input{diagrams/backtrace/backtrace.tex}}%
      \onslide*<5>{\providecommand\step{9}\input{diagrams/backtrace/backtrace.tex}}%
      \onslide*<6>{\providecommand\step{10}\input{diagrams/backtrace/backtrace.tex}}%
      \onslide*<7>{\providecommand\step{11}\input{diagrams/backtrace/backtrace.tex}}%
      \onslide*<8>{\providecommand\step{12}\input{diagrams/backtrace/backtrace.tex}}%
      \onslide*<9>{\providecommand\step{13}\input{diagrams/backtrace/backtrace.tex}}%
    \end{column}
  \end{columns}
\end{frame}

\begin{frame}{Backtraces}{Printing the backtrace}
  \begin{columns}[c]
    \begin{column}{0.45\textwidth}
      \minipage[c][0.66\textheight][s]{\columnwidth}
      \begin{itemize}
        \item<1-> The exception backtrace can now be printed to \funcarg{stdout} with \funcname{Printexc.print_backtrace}
        \item<2-> First the exception backtrace is retrieved into an OCaml array with \funcname{get_raw_backtrace }
        \item<3-> Which is implemented as the \funcname{caml_get_exception_raw_backtrace} C function in the runtime
        \item<4-> And finally printed with \funcname{print_exception_backtrace}
      \end{itemize}
      \vfill
      \mintocaml[firstline=28,lastline=34,highlightlines=33]{../src/backtrace/backtrace.ml}
      \endminipage
    \end{column}
    \begin{column}{0.55\textwidth}
      \onslide*<1,2>{
        \mintocaml[firstline=100,lastline=101]{../ocaml/stdlib/printexc.ml}
      }
      \only<1>{
        \listocaml[firstline=170,lastline=172]{../ocaml/stdlib/printexc.ml}{stdlib/printexc.ml:170}
      }
      \only<2>{
        \listocaml[firstline=170,lastline=172,highlightlines=172]{../ocaml/stdlib/printexc.ml}{stdlib/printexc.ml:170}
      }
      \only<3>{
        \mintc[firstline=154,lastline=156]{../ocaml/runtime/backtrace.c}
        \listc[firstline=171,lastline=192,highlightlines={184-187}]{../ocaml/runtime/backtrace.c}{runtime/backtrace.c:154}
      }
      \only<4>{
        \listocaml[firstline=155,lastline=165]{../ocaml/stdlib/printexc.ml}{stdlib/printexc.ml:55}
      }
    \end{column}
  \end{columns}
\end{frame}


\subsection{The multiple kinds of raise}
\frameSubsection{}{}

\begin{frame}{Backtraces}{When backtraces are not needed}
  \begin{columns}[c]
    \begin{column}{0.45\textwidth}
      \minipage[c][0.66\textheight][s]{\columnwidth}
      \begin{itemize}
        \item<1-> What if the backtrace for an exception is never needed?
        \item<2-> It's possible to raise the exception explicitly with \funcname{raise\_notrace} to make raising as cheap as possible
        \item<3-> An example of use in the \funcname{dynlink} library of the OCaml compiler
      \end{itemize}
      \vfill
      \onslide<3->{
      \listocaml[firstline=205,lastline=209,highlightlines=207]{../ocaml/otherlibs/dynlink/dynlink\_compilerlibs/misc.ml}{otherlibs/dynlink/dynlink\_compilerlibs/misc.ml:205}
      }
      \endminipage
    \end{column}
    \begin{column}{0.55\textwidth}
    \only<4>{
      \mintcmm[highlightlines=3]{../src/notrace/camlNotrace\_\_fun_347.cmm}
      \listcmm{../src/notrace/camlNotrace\_\_all\_somes\_327.cmm}{all\_somes}
    }
    \onslide*<5->{
      \listobjdump[highlightlines={6-9}]{../src/notrace/camlNotrace\_\_fun_347.objdump}{anonymous function from \funcname{all\_somes}}
    }
    \end{column}
  \end{columns}
\end{frame}

\begin{frame}{Backtraces}{Summary table of the multiple kinds of raise}
  \begin{itemize}
    \item When debugging information is enabled and if the backtrace of an exception isn't needed, it's possible to raise with \funcname{raise\_notrace} for performance
    \item When debugging information is \emph{not} enabled, the compiler optimises \funcname{raise} calls to inline assembly (same effect as using \funcname{raise\_notrace})
  \end{itemize}
  \bigskip
  \centering
  \begin{tabular}{| l | c | c | c | c |}
    \hline
    \parbox{10em}{Debug information \\ (\funcarg{-g} flag)}  & \multicolumn{2}{|c|}{Disabled}                                     & \multicolumn{2}{|c|}{Enabled} \\
    \hline
    Raise kind                                               & \funcname{raise} & \funcname{raise\_notrace}                       & \funcname{raise} & \funcname{raise\_notrace} \\
    \hline
    Compilation                                              & \parbox{8em}{inline assembly\\ (optimization)} & inline assembly   & \funcname{caml\_raise\_exn} & inline assembly \\
    \hline
  \end{tabular}
\end{frame}


%
% Takeaway #5
%
\subsection*{Takeaway \#5}
\frameSubsectionTakeaway{}
\againframe<5>{takeaway}
}%
      \onslide*<9>{\providecommand\step{13}%
% Backtraces
%
\subsection{One advantage of raising with debugging information}
\frameSubsection{}{}

\begin{frame}{Backtraces}{Debugging information \& source code}
  \begin{columns}[c]
    \begin{column}{0.5\textwidth}
      \begin{itemize}
        \item Raising an exception is fun and games, but how do we know where the exception originated? $\rightarrow$ With the backtrace associated with the exception!
        \item In order to get a backtrace from the point where an exception was raised, it's necessary to compile with debugging information enabled (\funcarg{-g} flag for the compiler)
        \item Same example as in the `Nested handlers' section
      \end{itemize}
    \end{column}
    \begin{column}{0.5\textwidth}
      \listocaml[firstline=9,lastline=34]{../src/backtrace/backtrace.ml}{backtrace/backtrace.ml}
    \end{column}
  \end{columns}
\end{frame}

\begin{frame}{Backtraces}{CMM and disassembly of \funcname{definitely\_raise}}
  \begin{columns}[c]
    \begin{column}{0.5\textwidth}
      \minipage[c][0.66\textheight][s]{\columnwidth}
      \begin{itemize}
        \item<1-> An effect of compiling with debugging information (\funcarg{-g}): the CMM no longer uses \funcname{raise\_notrace} to raise the exception but \funcname{raise} instead
        \item<2-> Disassembly shows that CMM \funcname{raise} is actually a call to \funcname{caml\_raise\_exn}, a function in assembler provided by the runtime
      \end{itemize}
      \vfill
      \mintocaml[firstline=9,lastline=13,highlightlines=12]{../src/backtrace/backtrace.ml}
      \endminipage
    \end{column}
    \begin{column}{0.5\textwidth}
      \onslide*<1>{
        \mintcmm[highlightlines=5]{../src/backtrace/camlBacktrace\_\_definitely\_raise\_347.cmm}
      }
      \onslide*<2>{
        \mintobjdump[firstline=2,highlightlines=14]{../src/backtrace/camlBacktrace\_\_definitely\_raise\_347.objdump}
      }
    \end{column}
  \end{columns}
\end{frame}

\begin{frame}{Backtraces}{Introducing \funcname{caml\_raise\_exn}}
  \begin{columns}[c]
    \begin{column}{0.5\textwidth}
      \minipage[c][0.66\textheight][s]{\columnwidth}
      \onslide*<1>{
        \begin{itemize}
          \item Raising the exception is performed using \funcname{caml\_raise\_exn} which is
            \begin{itemize}
              \item An assembly function provided by the runtime
              \item Architecture specific
            \end{itemize}
          \item Let's see what it does differently from \funcname{raise\_notrace}\dots
          \item We are about to raise \typename{ExnC} with \funcname{caml\_raise\_exn} from \funcname{definitely\_raise}
        \end{itemize}
      }
      \onslide*<2>{
        \mintobjdump[firstline=2,highlightlines=14]{../src/backtrace/camlBacktrace\_\_definitely\_raise\_347.objdump}
      }
      \vfill
      \mintocaml[firstline=9,lastline=13,highlightlines=12]{../src/backtrace/backtrace.ml}
      \endminipage
    \end{column}
    \begin{column}{0.5\textwidth}
      \centering
      \providecommand\step{1}\input{diagrams/backtrace/backtrace.tex}
    \end{column}
  \end{columns}
\end{frame}

\begin{frame}{Backtraces}{But before that, we need more fields from \typename{caml\_domain\_state}}
  \begin{columns}
    \begin{column}{0.5\textwidth}
      \begin{itemize}
        \item Remember \typename{caml\_domain\_state} the per-domain runtime state?
        \item Some other members are of interest to us now\dots
        \item \localname{backtrace\_active} is used to know if the runtime should collect backtraces
        \item \localname{backtrace\_active} can be set by
          \begin{itemize}
            \item The \funcarg{b} flag in the \funcname{OCAMLRUNPARAM} environment variable
            \item \mintinline{ocaml}{Printexc.record_backtrace : bool -> unit} \footnotemark
          \end{itemize}
      \end{itemize}
    \end{column}
    \begin{column}{0.5\textwidth}
      \begin{listing}
        \mintc[highlightlines={9-11}]{src/ocaml/domain\_state\_backtrace.h}
        \caption{runtime/caml/domain\_state.h}
      \end{listing}
    \end{column}
  \end{columns}
  \footnotetext[1]{https://v2.ocaml.org/api/Printexc.html}
\end{frame}

\newcommand\listCamlRaiseExn[1][]{
  \listasm[firstline=702,lastline=725,#1]{../ocaml/runtime/amd64.S}{runtime/amd64.S:702}}

\begin{frame}{Backtraces}{Walk through \funcname{caml\_raise\_exn}}
  \begin{columns}[T]
    \begin{column}{0.5\textwidth}
      \begin{itemize}
        \item<1-> First checks whether exceptions backtrace should be recorded
        \item<1-> By checking the \funcarg{domain\_state->backtrace\_active} value
        \item<2-> If not, acts as \funcname{raise\_notrace} with \funcname{RESTORE\_EXN\_HANDLER\_OCAML}
          \begin{itemize}
            \item Set \regname{RSP} to \localname{domain\_state->exn\_handler} value
            \item Restore \localname{domain\_state->exn\_handler} from trap
            \item Jump with \funcname{ret} (same effect as using \funcname{pop} and \funcname{jmp})
          \end{itemize}
        \item<3-> Otherwise, switches to the C stack to be able to execute C code
        \item<4-> Calls \funcname{caml\_stash\_backtrace}, a C function provided by the runtime\ignorespaces
          \onslide<6->{, with
            \begin{itemize}
              \item \funcarg{exn}: exception value
              \item \funcarg{pc}: code address of the raising point
              \item \funcarg{sp}: stack address of the raising function
              \item \funcarg{trapsp}: stack address of the next handler
            \end{itemize}
          }
      \end{itemize}
      \bigskip
      \mintocaml[firstline=9,lastline=13,highlightlines=12]{../src/backtrace/backtrace.ml}
    \end{column}
    \begin{column}{0.5\textwidth}
      \raggedleft
      \onslide*<1,3-4>{
        \mintc[firstline=190,lastline=192]{../ocaml/runtime/amd64.S}
        \mintc[firstline=234,lastline=237]{../ocaml/runtime/amd64.S}
      }
      \onslide*<2>{
        \mintc[firstline=190,lastline=192,highlightlines={191-192}]{../ocaml/runtime/amd64.S}
        \mintc[firstline=234,lastline=237,highlightlines={235-237}]{../ocaml/runtime/amd64.S}
      }
      \onslide*<1>{\listCamlRaiseExn[highlightlines=705]{}}%
      \onslide*<2>{\listCamlRaiseExn[highlightlines={707-708}]{}}%
      \onslide*<3>{\listCamlRaiseExn[highlightlines={712-714}]{}}%
      \onslide*<4>{\listCamlRaiseExn[highlightlines={715-720}]{}}%
      \onslide*<5>{\providecommand\step{1}\input{diagrams/backtrace/backtrace.tex}}%
      \onslide*<6>{\providecommand\step{2}\input{diagrams/backtrace/backtrace.tex}}%
    \end{column}
  \end{columns}
\end{frame}

\newcommand\listStashBacktrace[1][]{
  \listc[firstline=91,lastline=120,#1]{../ocaml/runtime/backtrace\_nat.c}{runtime/backtrace\_nat.c:91}}

\begin{frame}{Backtraces}{Introducing \funcname{caml\_stash\_backtrace}}
  \begin{columns}[c]
    \begin{column}{0.5\textwidth}
      \minipage[c][0.66\textheight][s]{\columnwidth}
      \begin{itemize}
        \item<1-> A C function provided by the runtime (specific to native code)
        \item<2-> It scans the stack, between the stack pointer at the raising point up to the next trap, in order to collect the names of the detected function frames
          \onslide*<2>{
            \begin{itemize}
              \item And more information, like line number, column number...
            \end{itemize}
          }
        \item<3-> It uses the same technique and information as the Garbage Collector (GC) uses to scan the values live on the stack
          \onslide*<4>{
            \begin{itemize}
              \item \typename{frame\_descr} are at the core of stack unwinding
            \end{itemize}
          }
      \end{itemize}
      \vfill
      \mintocaml[firstline=9,lastline=13,highlightlines=12]{../src/backtrace/backtrace.ml}
      \endminipage
    \end{column}
    \begin{column}{0.5\textwidth}
      \onslide*<1>{\listStashBacktrace[highlightlines=91]{}}
      \onslide*<2>{\listStashBacktrace[highlightlines={91,117-118}]{}}
      \onslide*<3->{\listStashBacktrace[highlightlines={94,106,109-110}]{}}
    \end{column}
  \end{columns}
\end{frame}

\newcommand\listFrameDescr[1][]{
  \listocaml[firstline=111,lastline=124,#1]{../ocaml/asmcomp/emitaux.ml}{asmcomp/emitaux.ml:111}}
\newcommand\listDebugInfo[1][]{
  \listocaml[firstline=38,lastline=49,#1]{../ocaml/lambda/debuginfo.mli}{lambda/debuginfo.mli:38}}

\begin{frame}{Backtraces}{\typename{frame\_descr} and \typename{frame\_debuginfo} in a nutshell}
  \begin{columns}[c]
    \begin{column}{0.5\textwidth}
      \begin{itemize}
        \item<1-> For every return address in an OCaml program, there is a associated \typename{frame\_descr}
        \item<2-> A \typename{frame\_descr} specifies
        \begin{itemize}
          \item<2-> The size of the function stack frame
          \item<3-> Where live values are on the stack (so that the GC can mark them as live and prevent from being swept)
        \end{itemize}
        \item<4-> When compiled with debug information, \typename{frame\_descr} will be linked to a \typename{frame\_debuginfo}
        \begin{itemize}
          \item<5-> Contains information about the position in the source code (file name, line and column number)
        \end{itemize}
      \end{itemize}
    \end{column}
    \begin{column}{0.5\textwidth}
        \onslide*<1>{\listFrameDescr[highlightlines={118-122}]{}}
        \onslide*<2>{\listFrameDescr[highlightlines={120}]{}}
        \onslide*<3>{\listFrameDescr[highlightlines={121}]{}}
        \onslide*<4->{\listFrameDescr[highlightlines={122}]{}}
        \medskip
        \onslide*<1-3>{\listDebugInfo{}}
        \onslide*<4>{\listDebugInfo[highlightlines={38,49}]{}}
        \onslide*<5>{\listDebugInfo[highlightlines={39-46}]{}}
    \end{column}
  \end{columns}
\end{frame}

\begin{frame}{Backtraces}{Scanning the stack with \typename{frame\_descr}}
  \begin{columns}[c]
    \begin{column}{0.5\textwidth}
      \begin{itemize}
        \item Given the return address of the code that raises the exception by calling \funcname{caml\_raise\_exn} (\funcarg{pc} argument of \funcname{caml\_stash\_backtrace})
        \item And the position of the stack pointer (\funcarg{sp} argument of \funcname{caml\_stash\_backtrace})
        \item The frame size given by \typename{frame\_descr}.\funcarg{frame\_size}, allows to find the end of the previous frame
        \item And the return address of the caller of this function
        \bigskip
        \onslide<3->{
        \item Note that traps (here the reddish \funcname{ExnA}, \funcname{ExnB} and \funcname{ExnC} blocks) belong to the frame that installed them, and so the frame size changes when traps are removed
        }
      \end{itemize}
    \end{column}
    \begin{column}{0.5\textwidth}
      \raggedleft
      \onslide*<1>{\providecommand\step{2}\input{diagrams/backtrace/backtrace.tex}}%
      \onslide*<2>{\providecommand\step{1}\input{diagrams/backtrace/scanstack.tex}}%
      \onslide*<3>{\providecommand\step{2}\input{diagrams/backtrace/scanstack.tex}}%
      \onslide*<4>{\providecommand\step{3}\input{diagrams/backtrace/scanstack.tex}}%
      \onslide*<5>{\providecommand\step{4}\input{diagrams/backtrace/scanstack.tex}}%
      \onslide*<6>{\providecommand\step{5}\input{diagrams/backtrace/scanstack.tex}}%
    \end{column}
  \end{columns}
\end{frame}

% Duplicate of previous-previous slide
\begin{frame}{Backtraces}{Continuing on \funcname{caml\_stash\_backtrace}}
  \begin{columns}[c]
    \begin{column}{0.5\textwidth}
      \minipage[c][0.75\textheight][s]{\columnwidth}
      \begin{itemize}
          \color{gray}
        \item A C function provided by the runtime (specific to native code)
        \item It scans the stack, between the stack pointer at the raising point up to the next trap, in order to collect the names of the detected function frames
        \item It uses the same technique and information as the Garbage Collector (GC) uses to scan the values live on the stack
      \end{itemize}
      \begin{itemize}
        \item For each function frame detected, store a pointer to the \typename{frame\_descr} in the \localname{domain\_state->backtrace\_buffer} array, effectively constructing the backtrace
      \end{itemize}
      \onslide<2->{
        \begin{itemize}
            \smallskip
          \item Constructed backtrace:
            \funcname{definitely\_raise},
            \onslide<3->{\funcname{probably\_raise}}
        \end{itemize}
      }
      \vfill
      \mintocaml[firstline=9,lastline=13,highlightlines=12]{../src/backtrace/backtrace.ml}
      \endminipage
    \end{column}
    \begin{column}{0.5\textwidth}
      \raggedleft
      \onslide*<1>{\listStashBacktrace[highlightlines={114-115}]{}}
      \onslide*<2>{\providecommand\step{3}\input{diagrams/backtrace/backtrace.tex}}%
      \onslide*<3>{\providecommand\step{4}\input{diagrams/backtrace/backtrace.tex}}%
    \end{column}
  \end{columns}
\end{frame}

\begin{frame}{Backtraces}{Back to \funcname{caml\_raise\_exn}}
  \begin{columns}[c]
    \begin{column}{0.5\textwidth}
      \minipage[c][0.66\textheight][s]{\columnwidth}
      \begin{itemize}\color{gray}
        \item Checks wheteher exceptions backtrace should be recorded, by checking the \funcarg{domain\_state->backtrace\_active} value
        \item Switches to the C stack to be able to execute C code
        \item Calls \funcname{caml\_stash\_backtrace}, to collect the backtrace up to the next trap
      \end{itemize}
      \onslide*<2->{
        \begin{itemize}
          \item<2-> Executes the exception handler in \funcname{probably\_raise} with \funcname{RESTORE\_EXN\_HANDLER\_OCAML}
            \begin{itemize}
              \item Set \regname{RSP} to \localname{domain\_state->exn\_handler} value
              \item Restore \localname{domain\_state->exn\_handler} from trap
              \item Jump to the handler code with \funcname{ret}
            \end{itemize}
        \end{itemize}
      }
      \vfill
      \onslide*<1>{\mintocaml[firstline=9,lastline=13,highlightlines=12]{../src/backtrace/backtrace.ml}}
      \onslide*<2->{\mintocaml[firstline=15,lastline=21,highlightlines={19-21}]{../src/backtrace/backtrace.ml}}
      \endminipage
    \end{column}
    \begin{column}{0.5\textwidth}
      \centering
      \onslide*<1>{
        \mintc[firstline=190,lastline=192]{../ocaml/runtime/amd64.S}
        \mintc[firstline=234,lastline=237]{../ocaml/runtime/amd64.S}
      }
      \onslide*<2>{
        \mintc[firstline=190,lastline=192,highlightlines={191-192}]{../ocaml/runtime/amd64.S}
        \mintc[firstline=234,lastline=237,highlightlines={235-237}]{../ocaml/runtime/amd64.S}
      }
      \onslide*<1>{\listCamlRaiseExn[highlightlines=720]{}}
      \onslide*<2>{\listCamlRaiseExn[highlightlines=722]{}}
      \onslide*<3->{\providecommand\step{5}\input{diagrams/backtrace/backtrace.tex}}
    \end{column}
  \end{columns}
\end{frame}

\begin{frame}{Backtraces}{\funcname{caml\_probably\_raise}'s handler doesn't catch \typename{ExnC}}
  \begin{columns}[c]
    \begin{column}{0.45\textwidth}
      \minipage[c][0.75\textheight][s]{\columnwidth}
      \begin{itemize}
        \item<1-> \funcname{match \dots with}\footnotemark pattern matching can also match exceptions
        \item<1-> The CMM of \funcname{probably\_raise} shows that this \funcname{match \dots with} is implemented with a pattern matching and a \funcname{try \dots with}
        \item<1-> But this one only matches on \typename{ExnA} exception
        \item<2-> Since the pattern matching fails to catch the exception, \funcname{reraise} is called with our current \typename{ExnC} exception
        \item<3-> Disassembly shows that \funcname{reraise} is actually a call to \funcname{caml\_reraise\_exn}, which is also an assembly function provided by the runtime
      \end{itemize}
      \vfill
      \mintocaml[firstline=15,lastline=21,highlightlines={18-21}]{../src/backtrace/backtrace.ml}
      \endminipage
    \end{column}
    \begin{column}{0.55\textwidth}
      \centering
      \onslide*<1>{\mintcmm[highlightlines={10-18}]{../src/backtrace/camlBacktrace\_\_probably\_raise\_351.cmm}}
      \onslide*<2>{\listcmm[highlightlines=18]{../src/backtrace/camlBacktrace\_\_probably\_raise_351.cmm}{CMM of probably\_raise}}
      \onslide*<3>{\mintobjdump[firstline=5,lastline=35,highlightlines=31]{../src/backtrace/camlBacktrace\_\_probably\_raise_351.objdump}}
    \end{column}
  \end{columns}
  \only<1>{\footnotetext[1]{https://blog.janestreet.com/pattern-matching-and-exception-handling-unite/}}
\end{frame}

\newcommand\listCamlReraiseExn[1][]{
  \listasm[firstline=727,lastline=734,#1]{../ocaml/runtime/amd64.S}{runtime/amd64.S:727}}

\begin{frame}{Backtraces}{Introducing \funcname{caml\_reraise\_exn}}
  \begin{columns}[c]
    \begin{column}{0.5\textwidth}
      \minipage[c][0.66\textheight][s]{\columnwidth}
      \begin{itemize}
        \item<1-> The exception have to be reraised to the next handler
        \item<2-> First, checks if the backtrace should be collected with \funcarg{domain\_state->backtrace\_active}
        \only<3>{
        \begin{itemize}
            \item If not, behaves like a \funcname{raise\_notrace} with \funcname{RESTORE\_EXN\_HANDLER\_OCAML}
        \end{itemize}
        }
        \item<4-> Jumps into \funcname{caml\_raise\_exn}
        \item<5-> Calls \funcname{caml\_stash\_backtrace} again, to scan the next part of the stack: from the trap just pop-ed to the next trap
      \end{itemize}
      \vfill
      \mintocaml[firstline=15,lastline=21,highlightlines={18-21}]{../src/backtrace/backtrace.ml}
      \endminipage
    \end{column}
    \begin{column}{0.5\textwidth}
      \raggedleft
      \onslide*<1>{\listCamlReraiseExn{}}
      \onslide*<2>{\listCamlReraiseExn[highlightlines=729]{}}
      \onslide*<3>{
        \mintc[firstline=190,lastline=192,highlightlines={191-192}]{../ocaml/runtime/amd64.S}
        \mintc[firstline=234,lastline=237,highlightlines={235-237}]{../ocaml/runtime/amd64.S}
        \medskip
        \listCamlReraiseExn[highlightlines=731]{}
      }
      \onslide*<4-5>{
        % TODO refactor with \listCamlReraiseExn
        \mintasm[firstline=727,lastline=734,highlightlines=730]{../ocaml/runtime/amd64.S}
        \medskip
      }
      \onslide*<4>{\listCamlRaiseExn[highlightlines=711]{}}
      \onslide*<5>{\listCamlRaiseExn[highlightlines=720]{}}
    \end{column}
  \end{columns}
\end{frame}

\begin{frame}{Backtraces}{Backtrace collection continues}
  \begin{columns}[c]
    \begin{column}{0.55\textwidth}
      \minipage[c][0.75\textheight][s]{\columnwidth}
      \begin{itemize}
        \item<1-> \funcname{caml\_stash\_backtrace} scans the older part of the stack, up to the next trap
        \item<6-> Controls jumps to the nested exception handler in \funcname{maybe\_raise}, which matches only on \typename{ExnB}
        \item<7-> \funcname{caml\_reraise\_exn} is called again, backtrace collection resumes with \funcname{caml\_stash\_backtrace}
      \end{itemize}
      \medskip
      \begin{itemize}
        \item<2-> Constructed backtrace:
        \begin{itemize}
          \item <2-> \funcname{definitely\_raise}
          \item <2-> \funcname{probably\_raise}
          \item <3-> \funcname{probably\_raise} (re-raise)
          \item <4-> \funcname{maybe\_raise}
          \item <5-> \funcname{main}
          \item <8-> \funcname{main} (re-raise)
        \end{itemize}
      \end{itemize}
      \vfill
      \onslide*<1-5>{\mintocaml[firstline=15,lastline=21]{../src/backtrace/backtrace.ml}}
      \onslide*<6>{\mintocaml[firstline=28,lastline=34,highlightlines=31]{../src/backtrace/backtrace.ml}}
      \onslide*<7->{\mintocaml[firstline=28,lastline=34,highlightlines=33]{../src/backtrace/backtrace.ml}}
      \endminipage
    \end{column}
    \begin{column}{0.45\textwidth}
      \raggedleft
      \onslide*<1>{\providecommand\step{6}\input{diagrams/backtrace/backtrace.tex}}%
      \onslide*<2>{\providecommand\step{6}\input{diagrams/backtrace/backtrace.tex}}%
      \onslide*<3>{\providecommand\step{7}\input{diagrams/backtrace/backtrace.tex}}%
      \onslide*<4>{\providecommand\step{8}\input{diagrams/backtrace/backtrace.tex}}%
      \onslide*<5>{\providecommand\step{9}\input{diagrams/backtrace/backtrace.tex}}%
      \onslide*<6>{\providecommand\step{10}\input{diagrams/backtrace/backtrace.tex}}%
      \onslide*<7>{\providecommand\step{11}\input{diagrams/backtrace/backtrace.tex}}%
      \onslide*<8>{\providecommand\step{12}\input{diagrams/backtrace/backtrace.tex}}%
      \onslide*<9>{\providecommand\step{13}\input{diagrams/backtrace/backtrace.tex}}%
    \end{column}
  \end{columns}
\end{frame}

\begin{frame}{Backtraces}{Printing the backtrace}
  \begin{columns}[c]
    \begin{column}{0.45\textwidth}
      \minipage[c][0.66\textheight][s]{\columnwidth}
      \begin{itemize}
        \item<1-> The exception backtrace can now be printed to \funcarg{stdout} with \funcname{Printexc.print_backtrace}
        \item<2-> First the exception backtrace is retrieved into an OCaml array with \funcname{get_raw_backtrace }
        \item<3-> Which is implemented as the \funcname{caml_get_exception_raw_backtrace} C function in the runtime
        \item<4-> And finally printed with \funcname{print_exception_backtrace}
      \end{itemize}
      \vfill
      \mintocaml[firstline=28,lastline=34,highlightlines=33]{../src/backtrace/backtrace.ml}
      \endminipage
    \end{column}
    \begin{column}{0.55\textwidth}
      \onslide*<1,2>{
        \mintocaml[firstline=100,lastline=101]{../ocaml/stdlib/printexc.ml}
      }
      \only<1>{
        \listocaml[firstline=170,lastline=172]{../ocaml/stdlib/printexc.ml}{stdlib/printexc.ml:170}
      }
      \only<2>{
        \listocaml[firstline=170,lastline=172,highlightlines=172]{../ocaml/stdlib/printexc.ml}{stdlib/printexc.ml:170}
      }
      \only<3>{
        \mintc[firstline=154,lastline=156]{../ocaml/runtime/backtrace.c}
        \listc[firstline=171,lastline=192,highlightlines={184-187}]{../ocaml/runtime/backtrace.c}{runtime/backtrace.c:154}
      }
      \only<4>{
        \listocaml[firstline=155,lastline=165]{../ocaml/stdlib/printexc.ml}{stdlib/printexc.ml:55}
      }
    \end{column}
  \end{columns}
\end{frame}


\subsection{The multiple kinds of raise}
\frameSubsection{}{}

\begin{frame}{Backtraces}{When backtraces are not needed}
  \begin{columns}[c]
    \begin{column}{0.45\textwidth}
      \minipage[c][0.66\textheight][s]{\columnwidth}
      \begin{itemize}
        \item<1-> What if the backtrace for an exception is never needed?
        \item<2-> It's possible to raise the exception explicitly with \funcname{raise\_notrace} to make raising as cheap as possible
        \item<3-> An example of use in the \funcname{dynlink} library of the OCaml compiler
      \end{itemize}
      \vfill
      \onslide<3->{
      \listocaml[firstline=205,lastline=209,highlightlines=207]{../ocaml/otherlibs/dynlink/dynlink\_compilerlibs/misc.ml}{otherlibs/dynlink/dynlink\_compilerlibs/misc.ml:205}
      }
      \endminipage
    \end{column}
    \begin{column}{0.55\textwidth}
    \only<4>{
      \mintcmm[highlightlines=3]{../src/notrace/camlNotrace\_\_fun_347.cmm}
      \listcmm{../src/notrace/camlNotrace\_\_all\_somes\_327.cmm}{all\_somes}
    }
    \onslide*<5->{
      \listobjdump[highlightlines={6-9}]{../src/notrace/camlNotrace\_\_fun_347.objdump}{anonymous function from \funcname{all\_somes}}
    }
    \end{column}
  \end{columns}
\end{frame}

\begin{frame}{Backtraces}{Summary table of the multiple kinds of raise}
  \begin{itemize}
    \item When debugging information is enabled and if the backtrace of an exception isn't needed, it's possible to raise with \funcname{raise\_notrace} for performance
    \item When debugging information is \emph{not} enabled, the compiler optimises \funcname{raise} calls to inline assembly (same effect as using \funcname{raise\_notrace})
  \end{itemize}
  \bigskip
  \centering
  \begin{tabular}{| l | c | c | c | c |}
    \hline
    \parbox{10em}{Debug information \\ (\funcarg{-g} flag)}  & \multicolumn{2}{|c|}{Disabled}                                     & \multicolumn{2}{|c|}{Enabled} \\
    \hline
    Raise kind                                               & \funcname{raise} & \funcname{raise\_notrace}                       & \funcname{raise} & \funcname{raise\_notrace} \\
    \hline
    Compilation                                              & \parbox{8em}{inline assembly\\ (optimization)} & inline assembly   & \funcname{caml\_raise\_exn} & inline assembly \\
    \hline
  \end{tabular}
\end{frame}


%
% Takeaway #5
%
\subsection*{Takeaway \#5}
\frameSubsectionTakeaway{}
\againframe<5>{takeaway}
}%
    \end{column}
  \end{columns}
\end{frame}

\begin{frame}{Backtraces}{Printing the backtrace}
  \begin{columns}[c]
    \begin{column}{0.45\textwidth}
      \minipage[c][0.66\textheight][s]{\columnwidth}
      \begin{itemize}
        \item<1-> The exception backtrace can now be printed to \funcarg{stdout} with \funcname{Printexc.print_backtrace}
        \item<2-> First the exception backtrace is retrieved into an OCaml array with \funcname{get_raw_backtrace }
        \item<3-> Which is implemented as the \funcname{caml_get_exception_raw_backtrace} C function in the runtime
        \item<4-> And finally printed with \funcname{print_exception_backtrace}
      \end{itemize}
      \vfill
      \mintocaml[firstline=28,lastline=34,highlightlines=33]{../src/backtrace/backtrace.ml}
      \endminipage
    \end{column}
    \begin{column}{0.55\textwidth}
      \onslide*<1,2>{
        \mintocaml[firstline=100,lastline=101]{../ocaml/stdlib/printexc.ml}
      }
      \only<1>{
        \listocaml[firstline=170,lastline=172]{../ocaml/stdlib/printexc.ml}{stdlib/printexc.ml:170}
      }
      \only<2>{
        \listocaml[firstline=170,lastline=172,highlightlines=172]{../ocaml/stdlib/printexc.ml}{stdlib/printexc.ml:170}
      }
      \only<3>{
        \mintc[firstline=154,lastline=156]{../ocaml/runtime/backtrace.c}
        \listc[firstline=171,lastline=192,highlightlines={184-187}]{../ocaml/runtime/backtrace.c}{runtime/backtrace.c:154}
      }
      \only<4>{
        \listocaml[firstline=155,lastline=165]{../ocaml/stdlib/printexc.ml}{stdlib/printexc.ml:55}
      }
    \end{column}
  \end{columns}
\end{frame}


\subsection{The multiple kinds of raise}
\frameSubsection{}{}

\begin{frame}{Backtraces}{When backtraces are not needed}
  \begin{columns}[c]
    \begin{column}{0.45\textwidth}
      \minipage[c][0.66\textheight][s]{\columnwidth}
      \begin{itemize}
        \item<1-> What if the backtrace for an exception is never needed?
        \item<2-> It's possible to raise the exception explicitly with \funcname{raise\_notrace} to make raising as cheap as possible
        \item<3-> An example of use in the \funcname{dynlink} library of the OCaml compiler
      \end{itemize}
      \vfill
      \onslide<3->{
      \listocaml[firstline=205,lastline=209,highlightlines=207]{../ocaml/otherlibs/dynlink/dynlink\_compilerlibs/misc.ml}{otherlibs/dynlink/dynlink\_compilerlibs/misc.ml:205}
      }
      \endminipage
    \end{column}
    \begin{column}{0.55\textwidth}
    \only<4>{
      \mintcmm[highlightlines=3]{../src/notrace/camlNotrace\_\_fun_347.cmm}
      \listcmm{../src/notrace/camlNotrace\_\_all\_somes\_327.cmm}{all\_somes}
    }
    \onslide*<5->{
      \listobjdump[highlightlines={6-9}]{../src/notrace/camlNotrace\_\_fun_347.objdump}{anonymous function from \funcname{all\_somes}}
    }
    \end{column}
  \end{columns}
\end{frame}

\begin{frame}{Backtraces}{Summary table of the multiple kinds of raise}
  \begin{itemize}
    \item When debugging information is enabled and if the backtrace of an exception isn't needed, it's possible to raise with \funcname{raise\_notrace} for performance
    \item When debugging information is \emph{not} enabled, the compiler optimises \funcname{raise} calls to inline assembly (same effect as using \funcname{raise\_notrace})
  \end{itemize}
  \bigskip
  \centering
  \begin{tabular}{| l | c | c | c | c |}
    \hline
    \parbox{10em}{Debug information \\ (\funcarg{-g} flag)}  & \multicolumn{2}{|c|}{Disabled}                                     & \multicolumn{2}{|c|}{Enabled} \\
    \hline
    Raise kind                                               & \funcname{raise} & \funcname{raise\_notrace}                       & \funcname{raise} & \funcname{raise\_notrace} \\
    \hline
    Compilation                                              & \parbox{8em}{inline assembly\\ (optimization)} & inline assembly   & \funcname{caml\_raise\_exn} & inline assembly \\
    \hline
  \end{tabular}
\end{frame}


%
% Takeaway #5
%
\subsection*{Takeaway \#5}
\frameSubsectionTakeaway{}
\againframe<5>{takeaway}
}%
      \onslide*<8>{\providecommand\step{12}%
% Backtraces
%
\subsection{One advantage of raising with debugging information}
\frameSubsection{}{}

\begin{frame}{Backtraces}{Debugging information \& source code}
  \begin{columns}[c]
    \begin{column}{0.5\textwidth}
      \begin{itemize}
        \item Raising an exception is fun and games, but how do we know where the exception originated? $\rightarrow$ With the backtrace associated with the exception!
        \item In order to get a backtrace from the point where an exception was raised, it's necessary to compile with debugging information enabled (\funcarg{-g} flag for the compiler)
        \item Same example as in the `Nested handlers' section
      \end{itemize}
    \end{column}
    \begin{column}{0.5\textwidth}
      \listocaml[firstline=9,lastline=34]{../src/backtrace/backtrace.ml}{backtrace/backtrace.ml}
    \end{column}
  \end{columns}
\end{frame}

\begin{frame}{Backtraces}{CMM and disassembly of \funcname{definitely\_raise}}
  \begin{columns}[c]
    \begin{column}{0.5\textwidth}
      \minipage[c][0.66\textheight][s]{\columnwidth}
      \begin{itemize}
        \item<1-> An effect of compiling with debugging information (\funcarg{-g}): the CMM no longer uses \funcname{raise\_notrace} to raise the exception but \funcname{raise} instead
        \item<2-> Disassembly shows that CMM \funcname{raise} is actually a call to \funcname{caml\_raise\_exn}, a function in assembler provided by the runtime
      \end{itemize}
      \vfill
      \mintocaml[firstline=9,lastline=13,highlightlines=12]{../src/backtrace/backtrace.ml}
      \endminipage
    \end{column}
    \begin{column}{0.5\textwidth}
      \onslide*<1>{
        \mintcmm[highlightlines=5]{../src/backtrace/camlBacktrace\_\_definitely\_raise\_347.cmm}
      }
      \onslide*<2>{
        \mintobjdump[firstline=2,highlightlines=14]{../src/backtrace/camlBacktrace\_\_definitely\_raise\_347.objdump}
      }
    \end{column}
  \end{columns}
\end{frame}

\begin{frame}{Backtraces}{Introducing \funcname{caml\_raise\_exn}}
  \begin{columns}[c]
    \begin{column}{0.5\textwidth}
      \minipage[c][0.66\textheight][s]{\columnwidth}
      \onslide*<1>{
        \begin{itemize}
          \item Raising the exception is performed using \funcname{caml\_raise\_exn} which is
            \begin{itemize}
              \item An assembly function provided by the runtime
              \item Architecture specific
            \end{itemize}
          \item Let's see what it does differently from \funcname{raise\_notrace}\dots
          \item We are about to raise \typename{ExnC} with \funcname{caml\_raise\_exn} from \funcname{definitely\_raise}
        \end{itemize}
      }
      \onslide*<2>{
        \mintobjdump[firstline=2,highlightlines=14]{../src/backtrace/camlBacktrace\_\_definitely\_raise\_347.objdump}
      }
      \vfill
      \mintocaml[firstline=9,lastline=13,highlightlines=12]{../src/backtrace/backtrace.ml}
      \endminipage
    \end{column}
    \begin{column}{0.5\textwidth}
      \centering
      \providecommand\step{1}%
% Backtraces
%
\subsection{One advantage of raising with debugging information}
\frameSubsection{}{}

\begin{frame}{Backtraces}{Debugging information \& source code}
  \begin{columns}[c]
    \begin{column}{0.5\textwidth}
      \begin{itemize}
        \item Raising an exception is fun and games, but how do we know where the exception originated? $\rightarrow$ With the backtrace associated with the exception!
        \item In order to get a backtrace from the point where an exception was raised, it's necessary to compile with debugging information enabled (\funcarg{-g} flag for the compiler)
        \item Same example as in the `Nested handlers' section
      \end{itemize}
    \end{column}
    \begin{column}{0.5\textwidth}
      \listocaml[firstline=9,lastline=34]{../src/backtrace/backtrace.ml}{backtrace/backtrace.ml}
    \end{column}
  \end{columns}
\end{frame}

\begin{frame}{Backtraces}{CMM and disassembly of \funcname{definitely\_raise}}
  \begin{columns}[c]
    \begin{column}{0.5\textwidth}
      \minipage[c][0.66\textheight][s]{\columnwidth}
      \begin{itemize}
        \item<1-> An effect of compiling with debugging information (\funcarg{-g}): the CMM no longer uses \funcname{raise\_notrace} to raise the exception but \funcname{raise} instead
        \item<2-> Disassembly shows that CMM \funcname{raise} is actually a call to \funcname{caml\_raise\_exn}, a function in assembler provided by the runtime
      \end{itemize}
      \vfill
      \mintocaml[firstline=9,lastline=13,highlightlines=12]{../src/backtrace/backtrace.ml}
      \endminipage
    \end{column}
    \begin{column}{0.5\textwidth}
      \onslide*<1>{
        \mintcmm[highlightlines=5]{../src/backtrace/camlBacktrace\_\_definitely\_raise\_347.cmm}
      }
      \onslide*<2>{
        \mintobjdump[firstline=2,highlightlines=14]{../src/backtrace/camlBacktrace\_\_definitely\_raise\_347.objdump}
      }
    \end{column}
  \end{columns}
\end{frame}

\begin{frame}{Backtraces}{Introducing \funcname{caml\_raise\_exn}}
  \begin{columns}[c]
    \begin{column}{0.5\textwidth}
      \minipage[c][0.66\textheight][s]{\columnwidth}
      \onslide*<1>{
        \begin{itemize}
          \item Raising the exception is performed using \funcname{caml\_raise\_exn} which is
            \begin{itemize}
              \item An assembly function provided by the runtime
              \item Architecture specific
            \end{itemize}
          \item Let's see what it does differently from \funcname{raise\_notrace}\dots
          \item We are about to raise \typename{ExnC} with \funcname{caml\_raise\_exn} from \funcname{definitely\_raise}
        \end{itemize}
      }
      \onslide*<2>{
        \mintobjdump[firstline=2,highlightlines=14]{../src/backtrace/camlBacktrace\_\_definitely\_raise\_347.objdump}
      }
      \vfill
      \mintocaml[firstline=9,lastline=13,highlightlines=12]{../src/backtrace/backtrace.ml}
      \endminipage
    \end{column}
    \begin{column}{0.5\textwidth}
      \centering
      \providecommand\step{1}\input{diagrams/backtrace/backtrace.tex}
    \end{column}
  \end{columns}
\end{frame}

\begin{frame}{Backtraces}{But before that, we need more fields from \typename{caml\_domain\_state}}
  \begin{columns}
    \begin{column}{0.5\textwidth}
      \begin{itemize}
        \item Remember \typename{caml\_domain\_state} the per-domain runtime state?
        \item Some other members are of interest to us now\dots
        \item \localname{backtrace\_active} is used to know if the runtime should collect backtraces
        \item \localname{backtrace\_active} can be set by
          \begin{itemize}
            \item The \funcarg{b} flag in the \funcname{OCAMLRUNPARAM} environment variable
            \item \mintinline{ocaml}{Printexc.record_backtrace : bool -> unit} \footnotemark
          \end{itemize}
      \end{itemize}
    \end{column}
    \begin{column}{0.5\textwidth}
      \begin{listing}
        \mintc[highlightlines={9-11}]{src/ocaml/domain\_state\_backtrace.h}
        \caption{runtime/caml/domain\_state.h}
      \end{listing}
    \end{column}
  \end{columns}
  \footnotetext[1]{https://v2.ocaml.org/api/Printexc.html}
\end{frame}

\newcommand\listCamlRaiseExn[1][]{
  \listasm[firstline=702,lastline=725,#1]{../ocaml/runtime/amd64.S}{runtime/amd64.S:702}}

\begin{frame}{Backtraces}{Walk through \funcname{caml\_raise\_exn}}
  \begin{columns}[T]
    \begin{column}{0.5\textwidth}
      \begin{itemize}
        \item<1-> First checks whether exceptions backtrace should be recorded
        \item<1-> By checking the \funcarg{domain\_state->backtrace\_active} value
        \item<2-> If not, acts as \funcname{raise\_notrace} with \funcname{RESTORE\_EXN\_HANDLER\_OCAML}
          \begin{itemize}
            \item Set \regname{RSP} to \localname{domain\_state->exn\_handler} value
            \item Restore \localname{domain\_state->exn\_handler} from trap
            \item Jump with \funcname{ret} (same effect as using \funcname{pop} and \funcname{jmp})
          \end{itemize}
        \item<3-> Otherwise, switches to the C stack to be able to execute C code
        \item<4-> Calls \funcname{caml\_stash\_backtrace}, a C function provided by the runtime\ignorespaces
          \onslide<6->{, with
            \begin{itemize}
              \item \funcarg{exn}: exception value
              \item \funcarg{pc}: code address of the raising point
              \item \funcarg{sp}: stack address of the raising function
              \item \funcarg{trapsp}: stack address of the next handler
            \end{itemize}
          }
      \end{itemize}
      \bigskip
      \mintocaml[firstline=9,lastline=13,highlightlines=12]{../src/backtrace/backtrace.ml}
    \end{column}
    \begin{column}{0.5\textwidth}
      \raggedleft
      \onslide*<1,3-4>{
        \mintc[firstline=190,lastline=192]{../ocaml/runtime/amd64.S}
        \mintc[firstline=234,lastline=237]{../ocaml/runtime/amd64.S}
      }
      \onslide*<2>{
        \mintc[firstline=190,lastline=192,highlightlines={191-192}]{../ocaml/runtime/amd64.S}
        \mintc[firstline=234,lastline=237,highlightlines={235-237}]{../ocaml/runtime/amd64.S}
      }
      \onslide*<1>{\listCamlRaiseExn[highlightlines=705]{}}%
      \onslide*<2>{\listCamlRaiseExn[highlightlines={707-708}]{}}%
      \onslide*<3>{\listCamlRaiseExn[highlightlines={712-714}]{}}%
      \onslide*<4>{\listCamlRaiseExn[highlightlines={715-720}]{}}%
      \onslide*<5>{\providecommand\step{1}\input{diagrams/backtrace/backtrace.tex}}%
      \onslide*<6>{\providecommand\step{2}\input{diagrams/backtrace/backtrace.tex}}%
    \end{column}
  \end{columns}
\end{frame}

\newcommand\listStashBacktrace[1][]{
  \listc[firstline=91,lastline=120,#1]{../ocaml/runtime/backtrace\_nat.c}{runtime/backtrace\_nat.c:91}}

\begin{frame}{Backtraces}{Introducing \funcname{caml\_stash\_backtrace}}
  \begin{columns}[c]
    \begin{column}{0.5\textwidth}
      \minipage[c][0.66\textheight][s]{\columnwidth}
      \begin{itemize}
        \item<1-> A C function provided by the runtime (specific to native code)
        \item<2-> It scans the stack, between the stack pointer at the raising point up to the next trap, in order to collect the names of the detected function frames
          \onslide*<2>{
            \begin{itemize}
              \item And more information, like line number, column number...
            \end{itemize}
          }
        \item<3-> It uses the same technique and information as the Garbage Collector (GC) uses to scan the values live on the stack
          \onslide*<4>{
            \begin{itemize}
              \item \typename{frame\_descr} are at the core of stack unwinding
            \end{itemize}
          }
      \end{itemize}
      \vfill
      \mintocaml[firstline=9,lastline=13,highlightlines=12]{../src/backtrace/backtrace.ml}
      \endminipage
    \end{column}
    \begin{column}{0.5\textwidth}
      \onslide*<1>{\listStashBacktrace[highlightlines=91]{}}
      \onslide*<2>{\listStashBacktrace[highlightlines={91,117-118}]{}}
      \onslide*<3->{\listStashBacktrace[highlightlines={94,106,109-110}]{}}
    \end{column}
  \end{columns}
\end{frame}

\newcommand\listFrameDescr[1][]{
  \listocaml[firstline=111,lastline=124,#1]{../ocaml/asmcomp/emitaux.ml}{asmcomp/emitaux.ml:111}}
\newcommand\listDebugInfo[1][]{
  \listocaml[firstline=38,lastline=49,#1]{../ocaml/lambda/debuginfo.mli}{lambda/debuginfo.mli:38}}

\begin{frame}{Backtraces}{\typename{frame\_descr} and \typename{frame\_debuginfo} in a nutshell}
  \begin{columns}[c]
    \begin{column}{0.5\textwidth}
      \begin{itemize}
        \item<1-> For every return address in an OCaml program, there is a associated \typename{frame\_descr}
        \item<2-> A \typename{frame\_descr} specifies
        \begin{itemize}
          \item<2-> The size of the function stack frame
          \item<3-> Where live values are on the stack (so that the GC can mark them as live and prevent from being swept)
        \end{itemize}
        \item<4-> When compiled with debug information, \typename{frame\_descr} will be linked to a \typename{frame\_debuginfo}
        \begin{itemize}
          \item<5-> Contains information about the position in the source code (file name, line and column number)
        \end{itemize}
      \end{itemize}
    \end{column}
    \begin{column}{0.5\textwidth}
        \onslide*<1>{\listFrameDescr[highlightlines={118-122}]{}}
        \onslide*<2>{\listFrameDescr[highlightlines={120}]{}}
        \onslide*<3>{\listFrameDescr[highlightlines={121}]{}}
        \onslide*<4->{\listFrameDescr[highlightlines={122}]{}}
        \medskip
        \onslide*<1-3>{\listDebugInfo{}}
        \onslide*<4>{\listDebugInfo[highlightlines={38,49}]{}}
        \onslide*<5>{\listDebugInfo[highlightlines={39-46}]{}}
    \end{column}
  \end{columns}
\end{frame}

\begin{frame}{Backtraces}{Scanning the stack with \typename{frame\_descr}}
  \begin{columns}[c]
    \begin{column}{0.5\textwidth}
      \begin{itemize}
        \item Given the return address of the code that raises the exception by calling \funcname{caml\_raise\_exn} (\funcarg{pc} argument of \funcname{caml\_stash\_backtrace})
        \item And the position of the stack pointer (\funcarg{sp} argument of \funcname{caml\_stash\_backtrace})
        \item The frame size given by \typename{frame\_descr}.\funcarg{frame\_size}, allows to find the end of the previous frame
        \item And the return address of the caller of this function
        \bigskip
        \onslide<3->{
        \item Note that traps (here the reddish \funcname{ExnA}, \funcname{ExnB} and \funcname{ExnC} blocks) belong to the frame that installed them, and so the frame size changes when traps are removed
        }
      \end{itemize}
    \end{column}
    \begin{column}{0.5\textwidth}
      \raggedleft
      \onslide*<1>{\providecommand\step{2}\input{diagrams/backtrace/backtrace.tex}}%
      \onslide*<2>{\providecommand\step{1}\input{diagrams/backtrace/scanstack.tex}}%
      \onslide*<3>{\providecommand\step{2}\input{diagrams/backtrace/scanstack.tex}}%
      \onslide*<4>{\providecommand\step{3}\input{diagrams/backtrace/scanstack.tex}}%
      \onslide*<5>{\providecommand\step{4}\input{diagrams/backtrace/scanstack.tex}}%
      \onslide*<6>{\providecommand\step{5}\input{diagrams/backtrace/scanstack.tex}}%
    \end{column}
  \end{columns}
\end{frame}

% Duplicate of previous-previous slide
\begin{frame}{Backtraces}{Continuing on \funcname{caml\_stash\_backtrace}}
  \begin{columns}[c]
    \begin{column}{0.5\textwidth}
      \minipage[c][0.75\textheight][s]{\columnwidth}
      \begin{itemize}
          \color{gray}
        \item A C function provided by the runtime (specific to native code)
        \item It scans the stack, between the stack pointer at the raising point up to the next trap, in order to collect the names of the detected function frames
        \item It uses the same technique and information as the Garbage Collector (GC) uses to scan the values live on the stack
      \end{itemize}
      \begin{itemize}
        \item For each function frame detected, store a pointer to the \typename{frame\_descr} in the \localname{domain\_state->backtrace\_buffer} array, effectively constructing the backtrace
      \end{itemize}
      \onslide<2->{
        \begin{itemize}
            \smallskip
          \item Constructed backtrace:
            \funcname{definitely\_raise},
            \onslide<3->{\funcname{probably\_raise}}
        \end{itemize}
      }
      \vfill
      \mintocaml[firstline=9,lastline=13,highlightlines=12]{../src/backtrace/backtrace.ml}
      \endminipage
    \end{column}
    \begin{column}{0.5\textwidth}
      \raggedleft
      \onslide*<1>{\listStashBacktrace[highlightlines={114-115}]{}}
      \onslide*<2>{\providecommand\step{3}\input{diagrams/backtrace/backtrace.tex}}%
      \onslide*<3>{\providecommand\step{4}\input{diagrams/backtrace/backtrace.tex}}%
    \end{column}
  \end{columns}
\end{frame}

\begin{frame}{Backtraces}{Back to \funcname{caml\_raise\_exn}}
  \begin{columns}[c]
    \begin{column}{0.5\textwidth}
      \minipage[c][0.66\textheight][s]{\columnwidth}
      \begin{itemize}\color{gray}
        \item Checks wheteher exceptions backtrace should be recorded, by checking the \funcarg{domain\_state->backtrace\_active} value
        \item Switches to the C stack to be able to execute C code
        \item Calls \funcname{caml\_stash\_backtrace}, to collect the backtrace up to the next trap
      \end{itemize}
      \onslide*<2->{
        \begin{itemize}
          \item<2-> Executes the exception handler in \funcname{probably\_raise} with \funcname{RESTORE\_EXN\_HANDLER\_OCAML}
            \begin{itemize}
              \item Set \regname{RSP} to \localname{domain\_state->exn\_handler} value
              \item Restore \localname{domain\_state->exn\_handler} from trap
              \item Jump to the handler code with \funcname{ret}
            \end{itemize}
        \end{itemize}
      }
      \vfill
      \onslide*<1>{\mintocaml[firstline=9,lastline=13,highlightlines=12]{../src/backtrace/backtrace.ml}}
      \onslide*<2->{\mintocaml[firstline=15,lastline=21,highlightlines={19-21}]{../src/backtrace/backtrace.ml}}
      \endminipage
    \end{column}
    \begin{column}{0.5\textwidth}
      \centering
      \onslide*<1>{
        \mintc[firstline=190,lastline=192]{../ocaml/runtime/amd64.S}
        \mintc[firstline=234,lastline=237]{../ocaml/runtime/amd64.S}
      }
      \onslide*<2>{
        \mintc[firstline=190,lastline=192,highlightlines={191-192}]{../ocaml/runtime/amd64.S}
        \mintc[firstline=234,lastline=237,highlightlines={235-237}]{../ocaml/runtime/amd64.S}
      }
      \onslide*<1>{\listCamlRaiseExn[highlightlines=720]{}}
      \onslide*<2>{\listCamlRaiseExn[highlightlines=722]{}}
      \onslide*<3->{\providecommand\step{5}\input{diagrams/backtrace/backtrace.tex}}
    \end{column}
  \end{columns}
\end{frame}

\begin{frame}{Backtraces}{\funcname{caml\_probably\_raise}'s handler doesn't catch \typename{ExnC}}
  \begin{columns}[c]
    \begin{column}{0.45\textwidth}
      \minipage[c][0.75\textheight][s]{\columnwidth}
      \begin{itemize}
        \item<1-> \funcname{match \dots with}\footnotemark pattern matching can also match exceptions
        \item<1-> The CMM of \funcname{probably\_raise} shows that this \funcname{match \dots with} is implemented with a pattern matching and a \funcname{try \dots with}
        \item<1-> But this one only matches on \typename{ExnA} exception
        \item<2-> Since the pattern matching fails to catch the exception, \funcname{reraise} is called with our current \typename{ExnC} exception
        \item<3-> Disassembly shows that \funcname{reraise} is actually a call to \funcname{caml\_reraise\_exn}, which is also an assembly function provided by the runtime
      \end{itemize}
      \vfill
      \mintocaml[firstline=15,lastline=21,highlightlines={18-21}]{../src/backtrace/backtrace.ml}
      \endminipage
    \end{column}
    \begin{column}{0.55\textwidth}
      \centering
      \onslide*<1>{\mintcmm[highlightlines={10-18}]{../src/backtrace/camlBacktrace\_\_probably\_raise\_351.cmm}}
      \onslide*<2>{\listcmm[highlightlines=18]{../src/backtrace/camlBacktrace\_\_probably\_raise_351.cmm}{CMM of probably\_raise}}
      \onslide*<3>{\mintobjdump[firstline=5,lastline=35,highlightlines=31]{../src/backtrace/camlBacktrace\_\_probably\_raise_351.objdump}}
    \end{column}
  \end{columns}
  \only<1>{\footnotetext[1]{https://blog.janestreet.com/pattern-matching-and-exception-handling-unite/}}
\end{frame}

\newcommand\listCamlReraiseExn[1][]{
  \listasm[firstline=727,lastline=734,#1]{../ocaml/runtime/amd64.S}{runtime/amd64.S:727}}

\begin{frame}{Backtraces}{Introducing \funcname{caml\_reraise\_exn}}
  \begin{columns}[c]
    \begin{column}{0.5\textwidth}
      \minipage[c][0.66\textheight][s]{\columnwidth}
      \begin{itemize}
        \item<1-> The exception have to be reraised to the next handler
        \item<2-> First, checks if the backtrace should be collected with \funcarg{domain\_state->backtrace\_active}
        \only<3>{
        \begin{itemize}
            \item If not, behaves like a \funcname{raise\_notrace} with \funcname{RESTORE\_EXN\_HANDLER\_OCAML}
        \end{itemize}
        }
        \item<4-> Jumps into \funcname{caml\_raise\_exn}
        \item<5-> Calls \funcname{caml\_stash\_backtrace} again, to scan the next part of the stack: from the trap just pop-ed to the next trap
      \end{itemize}
      \vfill
      \mintocaml[firstline=15,lastline=21,highlightlines={18-21}]{../src/backtrace/backtrace.ml}
      \endminipage
    \end{column}
    \begin{column}{0.5\textwidth}
      \raggedleft
      \onslide*<1>{\listCamlReraiseExn{}}
      \onslide*<2>{\listCamlReraiseExn[highlightlines=729]{}}
      \onslide*<3>{
        \mintc[firstline=190,lastline=192,highlightlines={191-192}]{../ocaml/runtime/amd64.S}
        \mintc[firstline=234,lastline=237,highlightlines={235-237}]{../ocaml/runtime/amd64.S}
        \medskip
        \listCamlReraiseExn[highlightlines=731]{}
      }
      \onslide*<4-5>{
        % TODO refactor with \listCamlReraiseExn
        \mintasm[firstline=727,lastline=734,highlightlines=730]{../ocaml/runtime/amd64.S}
        \medskip
      }
      \onslide*<4>{\listCamlRaiseExn[highlightlines=711]{}}
      \onslide*<5>{\listCamlRaiseExn[highlightlines=720]{}}
    \end{column}
  \end{columns}
\end{frame}

\begin{frame}{Backtraces}{Backtrace collection continues}
  \begin{columns}[c]
    \begin{column}{0.55\textwidth}
      \minipage[c][0.75\textheight][s]{\columnwidth}
      \begin{itemize}
        \item<1-> \funcname{caml\_stash\_backtrace} scans the older part of the stack, up to the next trap
        \item<6-> Controls jumps to the nested exception handler in \funcname{maybe\_raise}, which matches only on \typename{ExnB}
        \item<7-> \funcname{caml\_reraise\_exn} is called again, backtrace collection resumes with \funcname{caml\_stash\_backtrace}
      \end{itemize}
      \medskip
      \begin{itemize}
        \item<2-> Constructed backtrace:
        \begin{itemize}
          \item <2-> \funcname{definitely\_raise}
          \item <2-> \funcname{probably\_raise}
          \item <3-> \funcname{probably\_raise} (re-raise)
          \item <4-> \funcname{maybe\_raise}
          \item <5-> \funcname{main}
          \item <8-> \funcname{main} (re-raise)
        \end{itemize}
      \end{itemize}
      \vfill
      \onslide*<1-5>{\mintocaml[firstline=15,lastline=21]{../src/backtrace/backtrace.ml}}
      \onslide*<6>{\mintocaml[firstline=28,lastline=34,highlightlines=31]{../src/backtrace/backtrace.ml}}
      \onslide*<7->{\mintocaml[firstline=28,lastline=34,highlightlines=33]{../src/backtrace/backtrace.ml}}
      \endminipage
    \end{column}
    \begin{column}{0.45\textwidth}
      \raggedleft
      \onslide*<1>{\providecommand\step{6}\input{diagrams/backtrace/backtrace.tex}}%
      \onslide*<2>{\providecommand\step{6}\input{diagrams/backtrace/backtrace.tex}}%
      \onslide*<3>{\providecommand\step{7}\input{diagrams/backtrace/backtrace.tex}}%
      \onslide*<4>{\providecommand\step{8}\input{diagrams/backtrace/backtrace.tex}}%
      \onslide*<5>{\providecommand\step{9}\input{diagrams/backtrace/backtrace.tex}}%
      \onslide*<6>{\providecommand\step{10}\input{diagrams/backtrace/backtrace.tex}}%
      \onslide*<7>{\providecommand\step{11}\input{diagrams/backtrace/backtrace.tex}}%
      \onslide*<8>{\providecommand\step{12}\input{diagrams/backtrace/backtrace.tex}}%
      \onslide*<9>{\providecommand\step{13}\input{diagrams/backtrace/backtrace.tex}}%
    \end{column}
  \end{columns}
\end{frame}

\begin{frame}{Backtraces}{Printing the backtrace}
  \begin{columns}[c]
    \begin{column}{0.45\textwidth}
      \minipage[c][0.66\textheight][s]{\columnwidth}
      \begin{itemize}
        \item<1-> The exception backtrace can now be printed to \funcarg{stdout} with \funcname{Printexc.print_backtrace}
        \item<2-> First the exception backtrace is retrieved into an OCaml array with \funcname{get_raw_backtrace }
        \item<3-> Which is implemented as the \funcname{caml_get_exception_raw_backtrace} C function in the runtime
        \item<4-> And finally printed with \funcname{print_exception_backtrace}
      \end{itemize}
      \vfill
      \mintocaml[firstline=28,lastline=34,highlightlines=33]{../src/backtrace/backtrace.ml}
      \endminipage
    \end{column}
    \begin{column}{0.55\textwidth}
      \onslide*<1,2>{
        \mintocaml[firstline=100,lastline=101]{../ocaml/stdlib/printexc.ml}
      }
      \only<1>{
        \listocaml[firstline=170,lastline=172]{../ocaml/stdlib/printexc.ml}{stdlib/printexc.ml:170}
      }
      \only<2>{
        \listocaml[firstline=170,lastline=172,highlightlines=172]{../ocaml/stdlib/printexc.ml}{stdlib/printexc.ml:170}
      }
      \only<3>{
        \mintc[firstline=154,lastline=156]{../ocaml/runtime/backtrace.c}
        \listc[firstline=171,lastline=192,highlightlines={184-187}]{../ocaml/runtime/backtrace.c}{runtime/backtrace.c:154}
      }
      \only<4>{
        \listocaml[firstline=155,lastline=165]{../ocaml/stdlib/printexc.ml}{stdlib/printexc.ml:55}
      }
    \end{column}
  \end{columns}
\end{frame}


\subsection{The multiple kinds of raise}
\frameSubsection{}{}

\begin{frame}{Backtraces}{When backtraces are not needed}
  \begin{columns}[c]
    \begin{column}{0.45\textwidth}
      \minipage[c][0.66\textheight][s]{\columnwidth}
      \begin{itemize}
        \item<1-> What if the backtrace for an exception is never needed?
        \item<2-> It's possible to raise the exception explicitly with \funcname{raise\_notrace} to make raising as cheap as possible
        \item<3-> An example of use in the \funcname{dynlink} library of the OCaml compiler
      \end{itemize}
      \vfill
      \onslide<3->{
      \listocaml[firstline=205,lastline=209,highlightlines=207]{../ocaml/otherlibs/dynlink/dynlink\_compilerlibs/misc.ml}{otherlibs/dynlink/dynlink\_compilerlibs/misc.ml:205}
      }
      \endminipage
    \end{column}
    \begin{column}{0.55\textwidth}
    \only<4>{
      \mintcmm[highlightlines=3]{../src/notrace/camlNotrace\_\_fun_347.cmm}
      \listcmm{../src/notrace/camlNotrace\_\_all\_somes\_327.cmm}{all\_somes}
    }
    \onslide*<5->{
      \listobjdump[highlightlines={6-9}]{../src/notrace/camlNotrace\_\_fun_347.objdump}{anonymous function from \funcname{all\_somes}}
    }
    \end{column}
  \end{columns}
\end{frame}

\begin{frame}{Backtraces}{Summary table of the multiple kinds of raise}
  \begin{itemize}
    \item When debugging information is enabled and if the backtrace of an exception isn't needed, it's possible to raise with \funcname{raise\_notrace} for performance
    \item When debugging information is \emph{not} enabled, the compiler optimises \funcname{raise} calls to inline assembly (same effect as using \funcname{raise\_notrace})
  \end{itemize}
  \bigskip
  \centering
  \begin{tabular}{| l | c | c | c | c |}
    \hline
    \parbox{10em}{Debug information \\ (\funcarg{-g} flag)}  & \multicolumn{2}{|c|}{Disabled}                                     & \multicolumn{2}{|c|}{Enabled} \\
    \hline
    Raise kind                                               & \funcname{raise} & \funcname{raise\_notrace}                       & \funcname{raise} & \funcname{raise\_notrace} \\
    \hline
    Compilation                                              & \parbox{8em}{inline assembly\\ (optimization)} & inline assembly   & \funcname{caml\_raise\_exn} & inline assembly \\
    \hline
  \end{tabular}
\end{frame}


%
% Takeaway #5
%
\subsection*{Takeaway \#5}
\frameSubsectionTakeaway{}
\againframe<5>{takeaway}

    \end{column}
  \end{columns}
\end{frame}

\begin{frame}{Backtraces}{But before that, we need more fields from \typename{caml\_domain\_state}}
  \begin{columns}
    \begin{column}{0.5\textwidth}
      \begin{itemize}
        \item Remember \typename{caml\_domain\_state} the per-domain runtime state?
        \item Some other members are of interest to us now\dots
        \item \localname{backtrace\_active} is used to know if the runtime should collect backtraces
        \item \localname{backtrace\_active} can be set by
          \begin{itemize}
            \item The \funcarg{b} flag in the \funcname{OCAMLRUNPARAM} environment variable
            \item \mintinline{ocaml}{Printexc.record_backtrace : bool -> unit} \footnotemark
          \end{itemize}
      \end{itemize}
    \end{column}
    \begin{column}{0.5\textwidth}
      \begin{listing}
        \mintc[highlightlines={9-11}]{src/ocaml/domain\_state\_backtrace.h}
        \caption{runtime/caml/domain\_state.h}
      \end{listing}
    \end{column}
  \end{columns}
  \footnotetext[1]{https://v2.ocaml.org/api/Printexc.html}
\end{frame}

\newcommand\listCamlRaiseExn[1][]{
  \listasm[firstline=702,lastline=725,#1]{../ocaml/runtime/amd64.S}{runtime/amd64.S:702}}

\begin{frame}{Backtraces}{Walk through \funcname{caml\_raise\_exn}}
  \begin{columns}[T]
    \begin{column}{0.5\textwidth}
      \begin{itemize}
        \item<1-> First checks whether exceptions backtrace should be recorded
        \item<1-> By checking the \funcarg{domain\_state->backtrace\_active} value
        \item<2-> If not, acts as \funcname{raise\_notrace} with \funcname{RESTORE\_EXN\_HANDLER\_OCAML}
          \begin{itemize}
            \item Set \regname{RSP} to \localname{domain\_state->exn\_handler} value
            \item Restore \localname{domain\_state->exn\_handler} from trap
            \item Jump with \funcname{ret} (same effect as using \funcname{pop} and \funcname{jmp})
          \end{itemize}
        \item<3-> Otherwise, switches to the C stack to be able to execute C code
        \item<4-> Calls \funcname{caml\_stash\_backtrace}, a C function provided by the runtime\ignorespaces
          \onslide<6->{, with
            \begin{itemize}
              \item \funcarg{exn}: exception value
              \item \funcarg{pc}: code address of the raising point
              \item \funcarg{sp}: stack address of the raising function
              \item \funcarg{trapsp}: stack address of the next handler
            \end{itemize}
          }
      \end{itemize}
      \bigskip
      \mintocaml[firstline=9,lastline=13,highlightlines=12]{../src/backtrace/backtrace.ml}
    \end{column}
    \begin{column}{0.5\textwidth}
      \raggedleft
      \onslide*<1,3-4>{
        \mintc[firstline=190,lastline=192]{../ocaml/runtime/amd64.S}
        \mintc[firstline=234,lastline=237]{../ocaml/runtime/amd64.S}
      }
      \onslide*<2>{
        \mintc[firstline=190,lastline=192,highlightlines={191-192}]{../ocaml/runtime/amd64.S}
        \mintc[firstline=234,lastline=237,highlightlines={235-237}]{../ocaml/runtime/amd64.S}
      }
      \onslide*<1>{\listCamlRaiseExn[highlightlines=705]{}}%
      \onslide*<2>{\listCamlRaiseExn[highlightlines={707-708}]{}}%
      \onslide*<3>{\listCamlRaiseExn[highlightlines={712-714}]{}}%
      \onslide*<4>{\listCamlRaiseExn[highlightlines={715-720}]{}}%
      \onslide*<5>{\providecommand\step{1}%
% Backtraces
%
\subsection{One advantage of raising with debugging information}
\frameSubsection{}{}

\begin{frame}{Backtraces}{Debugging information \& source code}
  \begin{columns}[c]
    \begin{column}{0.5\textwidth}
      \begin{itemize}
        \item Raising an exception is fun and games, but how do we know where the exception originated? $\rightarrow$ With the backtrace associated with the exception!
        \item In order to get a backtrace from the point where an exception was raised, it's necessary to compile with debugging information enabled (\funcarg{-g} flag for the compiler)
        \item Same example as in the `Nested handlers' section
      \end{itemize}
    \end{column}
    \begin{column}{0.5\textwidth}
      \listocaml[firstline=9,lastline=34]{../src/backtrace/backtrace.ml}{backtrace/backtrace.ml}
    \end{column}
  \end{columns}
\end{frame}

\begin{frame}{Backtraces}{CMM and disassembly of \funcname{definitely\_raise}}
  \begin{columns}[c]
    \begin{column}{0.5\textwidth}
      \minipage[c][0.66\textheight][s]{\columnwidth}
      \begin{itemize}
        \item<1-> An effect of compiling with debugging information (\funcarg{-g}): the CMM no longer uses \funcname{raise\_notrace} to raise the exception but \funcname{raise} instead
        \item<2-> Disassembly shows that CMM \funcname{raise} is actually a call to \funcname{caml\_raise\_exn}, a function in assembler provided by the runtime
      \end{itemize}
      \vfill
      \mintocaml[firstline=9,lastline=13,highlightlines=12]{../src/backtrace/backtrace.ml}
      \endminipage
    \end{column}
    \begin{column}{0.5\textwidth}
      \onslide*<1>{
        \mintcmm[highlightlines=5]{../src/backtrace/camlBacktrace\_\_definitely\_raise\_347.cmm}
      }
      \onslide*<2>{
        \mintobjdump[firstline=2,highlightlines=14]{../src/backtrace/camlBacktrace\_\_definitely\_raise\_347.objdump}
      }
    \end{column}
  \end{columns}
\end{frame}

\begin{frame}{Backtraces}{Introducing \funcname{caml\_raise\_exn}}
  \begin{columns}[c]
    \begin{column}{0.5\textwidth}
      \minipage[c][0.66\textheight][s]{\columnwidth}
      \onslide*<1>{
        \begin{itemize}
          \item Raising the exception is performed using \funcname{caml\_raise\_exn} which is
            \begin{itemize}
              \item An assembly function provided by the runtime
              \item Architecture specific
            \end{itemize}
          \item Let's see what it does differently from \funcname{raise\_notrace}\dots
          \item We are about to raise \typename{ExnC} with \funcname{caml\_raise\_exn} from \funcname{definitely\_raise}
        \end{itemize}
      }
      \onslide*<2>{
        \mintobjdump[firstline=2,highlightlines=14]{../src/backtrace/camlBacktrace\_\_definitely\_raise\_347.objdump}
      }
      \vfill
      \mintocaml[firstline=9,lastline=13,highlightlines=12]{../src/backtrace/backtrace.ml}
      \endminipage
    \end{column}
    \begin{column}{0.5\textwidth}
      \centering
      \providecommand\step{1}\input{diagrams/backtrace/backtrace.tex}
    \end{column}
  \end{columns}
\end{frame}

\begin{frame}{Backtraces}{But before that, we need more fields from \typename{caml\_domain\_state}}
  \begin{columns}
    \begin{column}{0.5\textwidth}
      \begin{itemize}
        \item Remember \typename{caml\_domain\_state} the per-domain runtime state?
        \item Some other members are of interest to us now\dots
        \item \localname{backtrace\_active} is used to know if the runtime should collect backtraces
        \item \localname{backtrace\_active} can be set by
          \begin{itemize}
            \item The \funcarg{b} flag in the \funcname{OCAMLRUNPARAM} environment variable
            \item \mintinline{ocaml}{Printexc.record_backtrace : bool -> unit} \footnotemark
          \end{itemize}
      \end{itemize}
    \end{column}
    \begin{column}{0.5\textwidth}
      \begin{listing}
        \mintc[highlightlines={9-11}]{src/ocaml/domain\_state\_backtrace.h}
        \caption{runtime/caml/domain\_state.h}
      \end{listing}
    \end{column}
  \end{columns}
  \footnotetext[1]{https://v2.ocaml.org/api/Printexc.html}
\end{frame}

\newcommand\listCamlRaiseExn[1][]{
  \listasm[firstline=702,lastline=725,#1]{../ocaml/runtime/amd64.S}{runtime/amd64.S:702}}

\begin{frame}{Backtraces}{Walk through \funcname{caml\_raise\_exn}}
  \begin{columns}[T]
    \begin{column}{0.5\textwidth}
      \begin{itemize}
        \item<1-> First checks whether exceptions backtrace should be recorded
        \item<1-> By checking the \funcarg{domain\_state->backtrace\_active} value
        \item<2-> If not, acts as \funcname{raise\_notrace} with \funcname{RESTORE\_EXN\_HANDLER\_OCAML}
          \begin{itemize}
            \item Set \regname{RSP} to \localname{domain\_state->exn\_handler} value
            \item Restore \localname{domain\_state->exn\_handler} from trap
            \item Jump with \funcname{ret} (same effect as using \funcname{pop} and \funcname{jmp})
          \end{itemize}
        \item<3-> Otherwise, switches to the C stack to be able to execute C code
        \item<4-> Calls \funcname{caml\_stash\_backtrace}, a C function provided by the runtime\ignorespaces
          \onslide<6->{, with
            \begin{itemize}
              \item \funcarg{exn}: exception value
              \item \funcarg{pc}: code address of the raising point
              \item \funcarg{sp}: stack address of the raising function
              \item \funcarg{trapsp}: stack address of the next handler
            \end{itemize}
          }
      \end{itemize}
      \bigskip
      \mintocaml[firstline=9,lastline=13,highlightlines=12]{../src/backtrace/backtrace.ml}
    \end{column}
    \begin{column}{0.5\textwidth}
      \raggedleft
      \onslide*<1,3-4>{
        \mintc[firstline=190,lastline=192]{../ocaml/runtime/amd64.S}
        \mintc[firstline=234,lastline=237]{../ocaml/runtime/amd64.S}
      }
      \onslide*<2>{
        \mintc[firstline=190,lastline=192,highlightlines={191-192}]{../ocaml/runtime/amd64.S}
        \mintc[firstline=234,lastline=237,highlightlines={235-237}]{../ocaml/runtime/amd64.S}
      }
      \onslide*<1>{\listCamlRaiseExn[highlightlines=705]{}}%
      \onslide*<2>{\listCamlRaiseExn[highlightlines={707-708}]{}}%
      \onslide*<3>{\listCamlRaiseExn[highlightlines={712-714}]{}}%
      \onslide*<4>{\listCamlRaiseExn[highlightlines={715-720}]{}}%
      \onslide*<5>{\providecommand\step{1}\input{diagrams/backtrace/backtrace.tex}}%
      \onslide*<6>{\providecommand\step{2}\input{diagrams/backtrace/backtrace.tex}}%
    \end{column}
  \end{columns}
\end{frame}

\newcommand\listStashBacktrace[1][]{
  \listc[firstline=91,lastline=120,#1]{../ocaml/runtime/backtrace\_nat.c}{runtime/backtrace\_nat.c:91}}

\begin{frame}{Backtraces}{Introducing \funcname{caml\_stash\_backtrace}}
  \begin{columns}[c]
    \begin{column}{0.5\textwidth}
      \minipage[c][0.66\textheight][s]{\columnwidth}
      \begin{itemize}
        \item<1-> A C function provided by the runtime (specific to native code)
        \item<2-> It scans the stack, between the stack pointer at the raising point up to the next trap, in order to collect the names of the detected function frames
          \onslide*<2>{
            \begin{itemize}
              \item And more information, like line number, column number...
            \end{itemize}
          }
        \item<3-> It uses the same technique and information as the Garbage Collector (GC) uses to scan the values live on the stack
          \onslide*<4>{
            \begin{itemize}
              \item \typename{frame\_descr} are at the core of stack unwinding
            \end{itemize}
          }
      \end{itemize}
      \vfill
      \mintocaml[firstline=9,lastline=13,highlightlines=12]{../src/backtrace/backtrace.ml}
      \endminipage
    \end{column}
    \begin{column}{0.5\textwidth}
      \onslide*<1>{\listStashBacktrace[highlightlines=91]{}}
      \onslide*<2>{\listStashBacktrace[highlightlines={91,117-118}]{}}
      \onslide*<3->{\listStashBacktrace[highlightlines={94,106,109-110}]{}}
    \end{column}
  \end{columns}
\end{frame}

\newcommand\listFrameDescr[1][]{
  \listocaml[firstline=111,lastline=124,#1]{../ocaml/asmcomp/emitaux.ml}{asmcomp/emitaux.ml:111}}
\newcommand\listDebugInfo[1][]{
  \listocaml[firstline=38,lastline=49,#1]{../ocaml/lambda/debuginfo.mli}{lambda/debuginfo.mli:38}}

\begin{frame}{Backtraces}{\typename{frame\_descr} and \typename{frame\_debuginfo} in a nutshell}
  \begin{columns}[c]
    \begin{column}{0.5\textwidth}
      \begin{itemize}
        \item<1-> For every return address in an OCaml program, there is a associated \typename{frame\_descr}
        \item<2-> A \typename{frame\_descr} specifies
        \begin{itemize}
          \item<2-> The size of the function stack frame
          \item<3-> Where live values are on the stack (so that the GC can mark them as live and prevent from being swept)
        \end{itemize}
        \item<4-> When compiled with debug information, \typename{frame\_descr} will be linked to a \typename{frame\_debuginfo}
        \begin{itemize}
          \item<5-> Contains information about the position in the source code (file name, line and column number)
        \end{itemize}
      \end{itemize}
    \end{column}
    \begin{column}{0.5\textwidth}
        \onslide*<1>{\listFrameDescr[highlightlines={118-122}]{}}
        \onslide*<2>{\listFrameDescr[highlightlines={120}]{}}
        \onslide*<3>{\listFrameDescr[highlightlines={121}]{}}
        \onslide*<4->{\listFrameDescr[highlightlines={122}]{}}
        \medskip
        \onslide*<1-3>{\listDebugInfo{}}
        \onslide*<4>{\listDebugInfo[highlightlines={38,49}]{}}
        \onslide*<5>{\listDebugInfo[highlightlines={39-46}]{}}
    \end{column}
  \end{columns}
\end{frame}

\begin{frame}{Backtraces}{Scanning the stack with \typename{frame\_descr}}
  \begin{columns}[c]
    \begin{column}{0.5\textwidth}
      \begin{itemize}
        \item Given the return address of the code that raises the exception by calling \funcname{caml\_raise\_exn} (\funcarg{pc} argument of \funcname{caml\_stash\_backtrace})
        \item And the position of the stack pointer (\funcarg{sp} argument of \funcname{caml\_stash\_backtrace})
        \item The frame size given by \typename{frame\_descr}.\funcarg{frame\_size}, allows to find the end of the previous frame
        \item And the return address of the caller of this function
        \bigskip
        \onslide<3->{
        \item Note that traps (here the reddish \funcname{ExnA}, \funcname{ExnB} and \funcname{ExnC} blocks) belong to the frame that installed them, and so the frame size changes when traps are removed
        }
      \end{itemize}
    \end{column}
    \begin{column}{0.5\textwidth}
      \raggedleft
      \onslide*<1>{\providecommand\step{2}\input{diagrams/backtrace/backtrace.tex}}%
      \onslide*<2>{\providecommand\step{1}\input{diagrams/backtrace/scanstack.tex}}%
      \onslide*<3>{\providecommand\step{2}\input{diagrams/backtrace/scanstack.tex}}%
      \onslide*<4>{\providecommand\step{3}\input{diagrams/backtrace/scanstack.tex}}%
      \onslide*<5>{\providecommand\step{4}\input{diagrams/backtrace/scanstack.tex}}%
      \onslide*<6>{\providecommand\step{5}\input{diagrams/backtrace/scanstack.tex}}%
    \end{column}
  \end{columns}
\end{frame}

% Duplicate of previous-previous slide
\begin{frame}{Backtraces}{Continuing on \funcname{caml\_stash\_backtrace}}
  \begin{columns}[c]
    \begin{column}{0.5\textwidth}
      \minipage[c][0.75\textheight][s]{\columnwidth}
      \begin{itemize}
          \color{gray}
        \item A C function provided by the runtime (specific to native code)
        \item It scans the stack, between the stack pointer at the raising point up to the next trap, in order to collect the names of the detected function frames
        \item It uses the same technique and information as the Garbage Collector (GC) uses to scan the values live on the stack
      \end{itemize}
      \begin{itemize}
        \item For each function frame detected, store a pointer to the \typename{frame\_descr} in the \localname{domain\_state->backtrace\_buffer} array, effectively constructing the backtrace
      \end{itemize}
      \onslide<2->{
        \begin{itemize}
            \smallskip
          \item Constructed backtrace:
            \funcname{definitely\_raise},
            \onslide<3->{\funcname{probably\_raise}}
        \end{itemize}
      }
      \vfill
      \mintocaml[firstline=9,lastline=13,highlightlines=12]{../src/backtrace/backtrace.ml}
      \endminipage
    \end{column}
    \begin{column}{0.5\textwidth}
      \raggedleft
      \onslide*<1>{\listStashBacktrace[highlightlines={114-115}]{}}
      \onslide*<2>{\providecommand\step{3}\input{diagrams/backtrace/backtrace.tex}}%
      \onslide*<3>{\providecommand\step{4}\input{diagrams/backtrace/backtrace.tex}}%
    \end{column}
  \end{columns}
\end{frame}

\begin{frame}{Backtraces}{Back to \funcname{caml\_raise\_exn}}
  \begin{columns}[c]
    \begin{column}{0.5\textwidth}
      \minipage[c][0.66\textheight][s]{\columnwidth}
      \begin{itemize}\color{gray}
        \item Checks wheteher exceptions backtrace should be recorded, by checking the \funcarg{domain\_state->backtrace\_active} value
        \item Switches to the C stack to be able to execute C code
        \item Calls \funcname{caml\_stash\_backtrace}, to collect the backtrace up to the next trap
      \end{itemize}
      \onslide*<2->{
        \begin{itemize}
          \item<2-> Executes the exception handler in \funcname{probably\_raise} with \funcname{RESTORE\_EXN\_HANDLER\_OCAML}
            \begin{itemize}
              \item Set \regname{RSP} to \localname{domain\_state->exn\_handler} value
              \item Restore \localname{domain\_state->exn\_handler} from trap
              \item Jump to the handler code with \funcname{ret}
            \end{itemize}
        \end{itemize}
      }
      \vfill
      \onslide*<1>{\mintocaml[firstline=9,lastline=13,highlightlines=12]{../src/backtrace/backtrace.ml}}
      \onslide*<2->{\mintocaml[firstline=15,lastline=21,highlightlines={19-21}]{../src/backtrace/backtrace.ml}}
      \endminipage
    \end{column}
    \begin{column}{0.5\textwidth}
      \centering
      \onslide*<1>{
        \mintc[firstline=190,lastline=192]{../ocaml/runtime/amd64.S}
        \mintc[firstline=234,lastline=237]{../ocaml/runtime/amd64.S}
      }
      \onslide*<2>{
        \mintc[firstline=190,lastline=192,highlightlines={191-192}]{../ocaml/runtime/amd64.S}
        \mintc[firstline=234,lastline=237,highlightlines={235-237}]{../ocaml/runtime/amd64.S}
      }
      \onslide*<1>{\listCamlRaiseExn[highlightlines=720]{}}
      \onslide*<2>{\listCamlRaiseExn[highlightlines=722]{}}
      \onslide*<3->{\providecommand\step{5}\input{diagrams/backtrace/backtrace.tex}}
    \end{column}
  \end{columns}
\end{frame}

\begin{frame}{Backtraces}{\funcname{caml\_probably\_raise}'s handler doesn't catch \typename{ExnC}}
  \begin{columns}[c]
    \begin{column}{0.45\textwidth}
      \minipage[c][0.75\textheight][s]{\columnwidth}
      \begin{itemize}
        \item<1-> \funcname{match \dots with}\footnotemark pattern matching can also match exceptions
        \item<1-> The CMM of \funcname{probably\_raise} shows that this \funcname{match \dots with} is implemented with a pattern matching and a \funcname{try \dots with}
        \item<1-> But this one only matches on \typename{ExnA} exception
        \item<2-> Since the pattern matching fails to catch the exception, \funcname{reraise} is called with our current \typename{ExnC} exception
        \item<3-> Disassembly shows that \funcname{reraise} is actually a call to \funcname{caml\_reraise\_exn}, which is also an assembly function provided by the runtime
      \end{itemize}
      \vfill
      \mintocaml[firstline=15,lastline=21,highlightlines={18-21}]{../src/backtrace/backtrace.ml}
      \endminipage
    \end{column}
    \begin{column}{0.55\textwidth}
      \centering
      \onslide*<1>{\mintcmm[highlightlines={10-18}]{../src/backtrace/camlBacktrace\_\_probably\_raise\_351.cmm}}
      \onslide*<2>{\listcmm[highlightlines=18]{../src/backtrace/camlBacktrace\_\_probably\_raise_351.cmm}{CMM of probably\_raise}}
      \onslide*<3>{\mintobjdump[firstline=5,lastline=35,highlightlines=31]{../src/backtrace/camlBacktrace\_\_probably\_raise_351.objdump}}
    \end{column}
  \end{columns}
  \only<1>{\footnotetext[1]{https://blog.janestreet.com/pattern-matching-and-exception-handling-unite/}}
\end{frame}

\newcommand\listCamlReraiseExn[1][]{
  \listasm[firstline=727,lastline=734,#1]{../ocaml/runtime/amd64.S}{runtime/amd64.S:727}}

\begin{frame}{Backtraces}{Introducing \funcname{caml\_reraise\_exn}}
  \begin{columns}[c]
    \begin{column}{0.5\textwidth}
      \minipage[c][0.66\textheight][s]{\columnwidth}
      \begin{itemize}
        \item<1-> The exception have to be reraised to the next handler
        \item<2-> First, checks if the backtrace should be collected with \funcarg{domain\_state->backtrace\_active}
        \only<3>{
        \begin{itemize}
            \item If not, behaves like a \funcname{raise\_notrace} with \funcname{RESTORE\_EXN\_HANDLER\_OCAML}
        \end{itemize}
        }
        \item<4-> Jumps into \funcname{caml\_raise\_exn}
        \item<5-> Calls \funcname{caml\_stash\_backtrace} again, to scan the next part of the stack: from the trap just pop-ed to the next trap
      \end{itemize}
      \vfill
      \mintocaml[firstline=15,lastline=21,highlightlines={18-21}]{../src/backtrace/backtrace.ml}
      \endminipage
    \end{column}
    \begin{column}{0.5\textwidth}
      \raggedleft
      \onslide*<1>{\listCamlReraiseExn{}}
      \onslide*<2>{\listCamlReraiseExn[highlightlines=729]{}}
      \onslide*<3>{
        \mintc[firstline=190,lastline=192,highlightlines={191-192}]{../ocaml/runtime/amd64.S}
        \mintc[firstline=234,lastline=237,highlightlines={235-237}]{../ocaml/runtime/amd64.S}
        \medskip
        \listCamlReraiseExn[highlightlines=731]{}
      }
      \onslide*<4-5>{
        % TODO refactor with \listCamlReraiseExn
        \mintasm[firstline=727,lastline=734,highlightlines=730]{../ocaml/runtime/amd64.S}
        \medskip
      }
      \onslide*<4>{\listCamlRaiseExn[highlightlines=711]{}}
      \onslide*<5>{\listCamlRaiseExn[highlightlines=720]{}}
    \end{column}
  \end{columns}
\end{frame}

\begin{frame}{Backtraces}{Backtrace collection continues}
  \begin{columns}[c]
    \begin{column}{0.55\textwidth}
      \minipage[c][0.75\textheight][s]{\columnwidth}
      \begin{itemize}
        \item<1-> \funcname{caml\_stash\_backtrace} scans the older part of the stack, up to the next trap
        \item<6-> Controls jumps to the nested exception handler in \funcname{maybe\_raise}, which matches only on \typename{ExnB}
        \item<7-> \funcname{caml\_reraise\_exn} is called again, backtrace collection resumes with \funcname{caml\_stash\_backtrace}
      \end{itemize}
      \medskip
      \begin{itemize}
        \item<2-> Constructed backtrace:
        \begin{itemize}
          \item <2-> \funcname{definitely\_raise}
          \item <2-> \funcname{probably\_raise}
          \item <3-> \funcname{probably\_raise} (re-raise)
          \item <4-> \funcname{maybe\_raise}
          \item <5-> \funcname{main}
          \item <8-> \funcname{main} (re-raise)
        \end{itemize}
      \end{itemize}
      \vfill
      \onslide*<1-5>{\mintocaml[firstline=15,lastline=21]{../src/backtrace/backtrace.ml}}
      \onslide*<6>{\mintocaml[firstline=28,lastline=34,highlightlines=31]{../src/backtrace/backtrace.ml}}
      \onslide*<7->{\mintocaml[firstline=28,lastline=34,highlightlines=33]{../src/backtrace/backtrace.ml}}
      \endminipage
    \end{column}
    \begin{column}{0.45\textwidth}
      \raggedleft
      \onslide*<1>{\providecommand\step{6}\input{diagrams/backtrace/backtrace.tex}}%
      \onslide*<2>{\providecommand\step{6}\input{diagrams/backtrace/backtrace.tex}}%
      \onslide*<3>{\providecommand\step{7}\input{diagrams/backtrace/backtrace.tex}}%
      \onslide*<4>{\providecommand\step{8}\input{diagrams/backtrace/backtrace.tex}}%
      \onslide*<5>{\providecommand\step{9}\input{diagrams/backtrace/backtrace.tex}}%
      \onslide*<6>{\providecommand\step{10}\input{diagrams/backtrace/backtrace.tex}}%
      \onslide*<7>{\providecommand\step{11}\input{diagrams/backtrace/backtrace.tex}}%
      \onslide*<8>{\providecommand\step{12}\input{diagrams/backtrace/backtrace.tex}}%
      \onslide*<9>{\providecommand\step{13}\input{diagrams/backtrace/backtrace.tex}}%
    \end{column}
  \end{columns}
\end{frame}

\begin{frame}{Backtraces}{Printing the backtrace}
  \begin{columns}[c]
    \begin{column}{0.45\textwidth}
      \minipage[c][0.66\textheight][s]{\columnwidth}
      \begin{itemize}
        \item<1-> The exception backtrace can now be printed to \funcarg{stdout} with \funcname{Printexc.print_backtrace}
        \item<2-> First the exception backtrace is retrieved into an OCaml array with \funcname{get_raw_backtrace }
        \item<3-> Which is implemented as the \funcname{caml_get_exception_raw_backtrace} C function in the runtime
        \item<4-> And finally printed with \funcname{print_exception_backtrace}
      \end{itemize}
      \vfill
      \mintocaml[firstline=28,lastline=34,highlightlines=33]{../src/backtrace/backtrace.ml}
      \endminipage
    \end{column}
    \begin{column}{0.55\textwidth}
      \onslide*<1,2>{
        \mintocaml[firstline=100,lastline=101]{../ocaml/stdlib/printexc.ml}
      }
      \only<1>{
        \listocaml[firstline=170,lastline=172]{../ocaml/stdlib/printexc.ml}{stdlib/printexc.ml:170}
      }
      \only<2>{
        \listocaml[firstline=170,lastline=172,highlightlines=172]{../ocaml/stdlib/printexc.ml}{stdlib/printexc.ml:170}
      }
      \only<3>{
        \mintc[firstline=154,lastline=156]{../ocaml/runtime/backtrace.c}
        \listc[firstline=171,lastline=192,highlightlines={184-187}]{../ocaml/runtime/backtrace.c}{runtime/backtrace.c:154}
      }
      \only<4>{
        \listocaml[firstline=155,lastline=165]{../ocaml/stdlib/printexc.ml}{stdlib/printexc.ml:55}
      }
    \end{column}
  \end{columns}
\end{frame}


\subsection{The multiple kinds of raise}
\frameSubsection{}{}

\begin{frame}{Backtraces}{When backtraces are not needed}
  \begin{columns}[c]
    \begin{column}{0.45\textwidth}
      \minipage[c][0.66\textheight][s]{\columnwidth}
      \begin{itemize}
        \item<1-> What if the backtrace for an exception is never needed?
        \item<2-> It's possible to raise the exception explicitly with \funcname{raise\_notrace} to make raising as cheap as possible
        \item<3-> An example of use in the \funcname{dynlink} library of the OCaml compiler
      \end{itemize}
      \vfill
      \onslide<3->{
      \listocaml[firstline=205,lastline=209,highlightlines=207]{../ocaml/otherlibs/dynlink/dynlink\_compilerlibs/misc.ml}{otherlibs/dynlink/dynlink\_compilerlibs/misc.ml:205}
      }
      \endminipage
    \end{column}
    \begin{column}{0.55\textwidth}
    \only<4>{
      \mintcmm[highlightlines=3]{../src/notrace/camlNotrace\_\_fun_347.cmm}
      \listcmm{../src/notrace/camlNotrace\_\_all\_somes\_327.cmm}{all\_somes}
    }
    \onslide*<5->{
      \listobjdump[highlightlines={6-9}]{../src/notrace/camlNotrace\_\_fun_347.objdump}{anonymous function from \funcname{all\_somes}}
    }
    \end{column}
  \end{columns}
\end{frame}

\begin{frame}{Backtraces}{Summary table of the multiple kinds of raise}
  \begin{itemize}
    \item When debugging information is enabled and if the backtrace of an exception isn't needed, it's possible to raise with \funcname{raise\_notrace} for performance
    \item When debugging information is \emph{not} enabled, the compiler optimises \funcname{raise} calls to inline assembly (same effect as using \funcname{raise\_notrace})
  \end{itemize}
  \bigskip
  \centering
  \begin{tabular}{| l | c | c | c | c |}
    \hline
    \parbox{10em}{Debug information \\ (\funcarg{-g} flag)}  & \multicolumn{2}{|c|}{Disabled}                                     & \multicolumn{2}{|c|}{Enabled} \\
    \hline
    Raise kind                                               & \funcname{raise} & \funcname{raise\_notrace}                       & \funcname{raise} & \funcname{raise\_notrace} \\
    \hline
    Compilation                                              & \parbox{8em}{inline assembly\\ (optimization)} & inline assembly   & \funcname{caml\_raise\_exn} & inline assembly \\
    \hline
  \end{tabular}
\end{frame}


%
% Takeaway #5
%
\subsection*{Takeaway \#5}
\frameSubsectionTakeaway{}
\againframe<5>{takeaway}
}%
      \onslide*<6>{\providecommand\step{2}%
% Backtraces
%
\subsection{One advantage of raising with debugging information}
\frameSubsection{}{}

\begin{frame}{Backtraces}{Debugging information \& source code}
  \begin{columns}[c]
    \begin{column}{0.5\textwidth}
      \begin{itemize}
        \item Raising an exception is fun and games, but how do we know where the exception originated? $\rightarrow$ With the backtrace associated with the exception!
        \item In order to get a backtrace from the point where an exception was raised, it's necessary to compile with debugging information enabled (\funcarg{-g} flag for the compiler)
        \item Same example as in the `Nested handlers' section
      \end{itemize}
    \end{column}
    \begin{column}{0.5\textwidth}
      \listocaml[firstline=9,lastline=34]{../src/backtrace/backtrace.ml}{backtrace/backtrace.ml}
    \end{column}
  \end{columns}
\end{frame}

\begin{frame}{Backtraces}{CMM and disassembly of \funcname{definitely\_raise}}
  \begin{columns}[c]
    \begin{column}{0.5\textwidth}
      \minipage[c][0.66\textheight][s]{\columnwidth}
      \begin{itemize}
        \item<1-> An effect of compiling with debugging information (\funcarg{-g}): the CMM no longer uses \funcname{raise\_notrace} to raise the exception but \funcname{raise} instead
        \item<2-> Disassembly shows that CMM \funcname{raise} is actually a call to \funcname{caml\_raise\_exn}, a function in assembler provided by the runtime
      \end{itemize}
      \vfill
      \mintocaml[firstline=9,lastline=13,highlightlines=12]{../src/backtrace/backtrace.ml}
      \endminipage
    \end{column}
    \begin{column}{0.5\textwidth}
      \onslide*<1>{
        \mintcmm[highlightlines=5]{../src/backtrace/camlBacktrace\_\_definitely\_raise\_347.cmm}
      }
      \onslide*<2>{
        \mintobjdump[firstline=2,highlightlines=14]{../src/backtrace/camlBacktrace\_\_definitely\_raise\_347.objdump}
      }
    \end{column}
  \end{columns}
\end{frame}

\begin{frame}{Backtraces}{Introducing \funcname{caml\_raise\_exn}}
  \begin{columns}[c]
    \begin{column}{0.5\textwidth}
      \minipage[c][0.66\textheight][s]{\columnwidth}
      \onslide*<1>{
        \begin{itemize}
          \item Raising the exception is performed using \funcname{caml\_raise\_exn} which is
            \begin{itemize}
              \item An assembly function provided by the runtime
              \item Architecture specific
            \end{itemize}
          \item Let's see what it does differently from \funcname{raise\_notrace}\dots
          \item We are about to raise \typename{ExnC} with \funcname{caml\_raise\_exn} from \funcname{definitely\_raise}
        \end{itemize}
      }
      \onslide*<2>{
        \mintobjdump[firstline=2,highlightlines=14]{../src/backtrace/camlBacktrace\_\_definitely\_raise\_347.objdump}
      }
      \vfill
      \mintocaml[firstline=9,lastline=13,highlightlines=12]{../src/backtrace/backtrace.ml}
      \endminipage
    \end{column}
    \begin{column}{0.5\textwidth}
      \centering
      \providecommand\step{1}\input{diagrams/backtrace/backtrace.tex}
    \end{column}
  \end{columns}
\end{frame}

\begin{frame}{Backtraces}{But before that, we need more fields from \typename{caml\_domain\_state}}
  \begin{columns}
    \begin{column}{0.5\textwidth}
      \begin{itemize}
        \item Remember \typename{caml\_domain\_state} the per-domain runtime state?
        \item Some other members are of interest to us now\dots
        \item \localname{backtrace\_active} is used to know if the runtime should collect backtraces
        \item \localname{backtrace\_active} can be set by
          \begin{itemize}
            \item The \funcarg{b} flag in the \funcname{OCAMLRUNPARAM} environment variable
            \item \mintinline{ocaml}{Printexc.record_backtrace : bool -> unit} \footnotemark
          \end{itemize}
      \end{itemize}
    \end{column}
    \begin{column}{0.5\textwidth}
      \begin{listing}
        \mintc[highlightlines={9-11}]{src/ocaml/domain\_state\_backtrace.h}
        \caption{runtime/caml/domain\_state.h}
      \end{listing}
    \end{column}
  \end{columns}
  \footnotetext[1]{https://v2.ocaml.org/api/Printexc.html}
\end{frame}

\newcommand\listCamlRaiseExn[1][]{
  \listasm[firstline=702,lastline=725,#1]{../ocaml/runtime/amd64.S}{runtime/amd64.S:702}}

\begin{frame}{Backtraces}{Walk through \funcname{caml\_raise\_exn}}
  \begin{columns}[T]
    \begin{column}{0.5\textwidth}
      \begin{itemize}
        \item<1-> First checks whether exceptions backtrace should be recorded
        \item<1-> By checking the \funcarg{domain\_state->backtrace\_active} value
        \item<2-> If not, acts as \funcname{raise\_notrace} with \funcname{RESTORE\_EXN\_HANDLER\_OCAML}
          \begin{itemize}
            \item Set \regname{RSP} to \localname{domain\_state->exn\_handler} value
            \item Restore \localname{domain\_state->exn\_handler} from trap
            \item Jump with \funcname{ret} (same effect as using \funcname{pop} and \funcname{jmp})
          \end{itemize}
        \item<3-> Otherwise, switches to the C stack to be able to execute C code
        \item<4-> Calls \funcname{caml\_stash\_backtrace}, a C function provided by the runtime\ignorespaces
          \onslide<6->{, with
            \begin{itemize}
              \item \funcarg{exn}: exception value
              \item \funcarg{pc}: code address of the raising point
              \item \funcarg{sp}: stack address of the raising function
              \item \funcarg{trapsp}: stack address of the next handler
            \end{itemize}
          }
      \end{itemize}
      \bigskip
      \mintocaml[firstline=9,lastline=13,highlightlines=12]{../src/backtrace/backtrace.ml}
    \end{column}
    \begin{column}{0.5\textwidth}
      \raggedleft
      \onslide*<1,3-4>{
        \mintc[firstline=190,lastline=192]{../ocaml/runtime/amd64.S}
        \mintc[firstline=234,lastline=237]{../ocaml/runtime/amd64.S}
      }
      \onslide*<2>{
        \mintc[firstline=190,lastline=192,highlightlines={191-192}]{../ocaml/runtime/amd64.S}
        \mintc[firstline=234,lastline=237,highlightlines={235-237}]{../ocaml/runtime/amd64.S}
      }
      \onslide*<1>{\listCamlRaiseExn[highlightlines=705]{}}%
      \onslide*<2>{\listCamlRaiseExn[highlightlines={707-708}]{}}%
      \onslide*<3>{\listCamlRaiseExn[highlightlines={712-714}]{}}%
      \onslide*<4>{\listCamlRaiseExn[highlightlines={715-720}]{}}%
      \onslide*<5>{\providecommand\step{1}\input{diagrams/backtrace/backtrace.tex}}%
      \onslide*<6>{\providecommand\step{2}\input{diagrams/backtrace/backtrace.tex}}%
    \end{column}
  \end{columns}
\end{frame}

\newcommand\listStashBacktrace[1][]{
  \listc[firstline=91,lastline=120,#1]{../ocaml/runtime/backtrace\_nat.c}{runtime/backtrace\_nat.c:91}}

\begin{frame}{Backtraces}{Introducing \funcname{caml\_stash\_backtrace}}
  \begin{columns}[c]
    \begin{column}{0.5\textwidth}
      \minipage[c][0.66\textheight][s]{\columnwidth}
      \begin{itemize}
        \item<1-> A C function provided by the runtime (specific to native code)
        \item<2-> It scans the stack, between the stack pointer at the raising point up to the next trap, in order to collect the names of the detected function frames
          \onslide*<2>{
            \begin{itemize}
              \item And more information, like line number, column number...
            \end{itemize}
          }
        \item<3-> It uses the same technique and information as the Garbage Collector (GC) uses to scan the values live on the stack
          \onslide*<4>{
            \begin{itemize}
              \item \typename{frame\_descr} are at the core of stack unwinding
            \end{itemize}
          }
      \end{itemize}
      \vfill
      \mintocaml[firstline=9,lastline=13,highlightlines=12]{../src/backtrace/backtrace.ml}
      \endminipage
    \end{column}
    \begin{column}{0.5\textwidth}
      \onslide*<1>{\listStashBacktrace[highlightlines=91]{}}
      \onslide*<2>{\listStashBacktrace[highlightlines={91,117-118}]{}}
      \onslide*<3->{\listStashBacktrace[highlightlines={94,106,109-110}]{}}
    \end{column}
  \end{columns}
\end{frame}

\newcommand\listFrameDescr[1][]{
  \listocaml[firstline=111,lastline=124,#1]{../ocaml/asmcomp/emitaux.ml}{asmcomp/emitaux.ml:111}}
\newcommand\listDebugInfo[1][]{
  \listocaml[firstline=38,lastline=49,#1]{../ocaml/lambda/debuginfo.mli}{lambda/debuginfo.mli:38}}

\begin{frame}{Backtraces}{\typename{frame\_descr} and \typename{frame\_debuginfo} in a nutshell}
  \begin{columns}[c]
    \begin{column}{0.5\textwidth}
      \begin{itemize}
        \item<1-> For every return address in an OCaml program, there is a associated \typename{frame\_descr}
        \item<2-> A \typename{frame\_descr} specifies
        \begin{itemize}
          \item<2-> The size of the function stack frame
          \item<3-> Where live values are on the stack (so that the GC can mark them as live and prevent from being swept)
        \end{itemize}
        \item<4-> When compiled with debug information, \typename{frame\_descr} will be linked to a \typename{frame\_debuginfo}
        \begin{itemize}
          \item<5-> Contains information about the position in the source code (file name, line and column number)
        \end{itemize}
      \end{itemize}
    \end{column}
    \begin{column}{0.5\textwidth}
        \onslide*<1>{\listFrameDescr[highlightlines={118-122}]{}}
        \onslide*<2>{\listFrameDescr[highlightlines={120}]{}}
        \onslide*<3>{\listFrameDescr[highlightlines={121}]{}}
        \onslide*<4->{\listFrameDescr[highlightlines={122}]{}}
        \medskip
        \onslide*<1-3>{\listDebugInfo{}}
        \onslide*<4>{\listDebugInfo[highlightlines={38,49}]{}}
        \onslide*<5>{\listDebugInfo[highlightlines={39-46}]{}}
    \end{column}
  \end{columns}
\end{frame}

\begin{frame}{Backtraces}{Scanning the stack with \typename{frame\_descr}}
  \begin{columns}[c]
    \begin{column}{0.5\textwidth}
      \begin{itemize}
        \item Given the return address of the code that raises the exception by calling \funcname{caml\_raise\_exn} (\funcarg{pc} argument of \funcname{caml\_stash\_backtrace})
        \item And the position of the stack pointer (\funcarg{sp} argument of \funcname{caml\_stash\_backtrace})
        \item The frame size given by \typename{frame\_descr}.\funcarg{frame\_size}, allows to find the end of the previous frame
        \item And the return address of the caller of this function
        \bigskip
        \onslide<3->{
        \item Note that traps (here the reddish \funcname{ExnA}, \funcname{ExnB} and \funcname{ExnC} blocks) belong to the frame that installed them, and so the frame size changes when traps are removed
        }
      \end{itemize}
    \end{column}
    \begin{column}{0.5\textwidth}
      \raggedleft
      \onslide*<1>{\providecommand\step{2}\input{diagrams/backtrace/backtrace.tex}}%
      \onslide*<2>{\providecommand\step{1}\input{diagrams/backtrace/scanstack.tex}}%
      \onslide*<3>{\providecommand\step{2}\input{diagrams/backtrace/scanstack.tex}}%
      \onslide*<4>{\providecommand\step{3}\input{diagrams/backtrace/scanstack.tex}}%
      \onslide*<5>{\providecommand\step{4}\input{diagrams/backtrace/scanstack.tex}}%
      \onslide*<6>{\providecommand\step{5}\input{diagrams/backtrace/scanstack.tex}}%
    \end{column}
  \end{columns}
\end{frame}

% Duplicate of previous-previous slide
\begin{frame}{Backtraces}{Continuing on \funcname{caml\_stash\_backtrace}}
  \begin{columns}[c]
    \begin{column}{0.5\textwidth}
      \minipage[c][0.75\textheight][s]{\columnwidth}
      \begin{itemize}
          \color{gray}
        \item A C function provided by the runtime (specific to native code)
        \item It scans the stack, between the stack pointer at the raising point up to the next trap, in order to collect the names of the detected function frames
        \item It uses the same technique and information as the Garbage Collector (GC) uses to scan the values live on the stack
      \end{itemize}
      \begin{itemize}
        \item For each function frame detected, store a pointer to the \typename{frame\_descr} in the \localname{domain\_state->backtrace\_buffer} array, effectively constructing the backtrace
      \end{itemize}
      \onslide<2->{
        \begin{itemize}
            \smallskip
          \item Constructed backtrace:
            \funcname{definitely\_raise},
            \onslide<3->{\funcname{probably\_raise}}
        \end{itemize}
      }
      \vfill
      \mintocaml[firstline=9,lastline=13,highlightlines=12]{../src/backtrace/backtrace.ml}
      \endminipage
    \end{column}
    \begin{column}{0.5\textwidth}
      \raggedleft
      \onslide*<1>{\listStashBacktrace[highlightlines={114-115}]{}}
      \onslide*<2>{\providecommand\step{3}\input{diagrams/backtrace/backtrace.tex}}%
      \onslide*<3>{\providecommand\step{4}\input{diagrams/backtrace/backtrace.tex}}%
    \end{column}
  \end{columns}
\end{frame}

\begin{frame}{Backtraces}{Back to \funcname{caml\_raise\_exn}}
  \begin{columns}[c]
    \begin{column}{0.5\textwidth}
      \minipage[c][0.66\textheight][s]{\columnwidth}
      \begin{itemize}\color{gray}
        \item Checks wheteher exceptions backtrace should be recorded, by checking the \funcarg{domain\_state->backtrace\_active} value
        \item Switches to the C stack to be able to execute C code
        \item Calls \funcname{caml\_stash\_backtrace}, to collect the backtrace up to the next trap
      \end{itemize}
      \onslide*<2->{
        \begin{itemize}
          \item<2-> Executes the exception handler in \funcname{probably\_raise} with \funcname{RESTORE\_EXN\_HANDLER\_OCAML}
            \begin{itemize}
              \item Set \regname{RSP} to \localname{domain\_state->exn\_handler} value
              \item Restore \localname{domain\_state->exn\_handler} from trap
              \item Jump to the handler code with \funcname{ret}
            \end{itemize}
        \end{itemize}
      }
      \vfill
      \onslide*<1>{\mintocaml[firstline=9,lastline=13,highlightlines=12]{../src/backtrace/backtrace.ml}}
      \onslide*<2->{\mintocaml[firstline=15,lastline=21,highlightlines={19-21}]{../src/backtrace/backtrace.ml}}
      \endminipage
    \end{column}
    \begin{column}{0.5\textwidth}
      \centering
      \onslide*<1>{
        \mintc[firstline=190,lastline=192]{../ocaml/runtime/amd64.S}
        \mintc[firstline=234,lastline=237]{../ocaml/runtime/amd64.S}
      }
      \onslide*<2>{
        \mintc[firstline=190,lastline=192,highlightlines={191-192}]{../ocaml/runtime/amd64.S}
        \mintc[firstline=234,lastline=237,highlightlines={235-237}]{../ocaml/runtime/amd64.S}
      }
      \onslide*<1>{\listCamlRaiseExn[highlightlines=720]{}}
      \onslide*<2>{\listCamlRaiseExn[highlightlines=722]{}}
      \onslide*<3->{\providecommand\step{5}\input{diagrams/backtrace/backtrace.tex}}
    \end{column}
  \end{columns}
\end{frame}

\begin{frame}{Backtraces}{\funcname{caml\_probably\_raise}'s handler doesn't catch \typename{ExnC}}
  \begin{columns}[c]
    \begin{column}{0.45\textwidth}
      \minipage[c][0.75\textheight][s]{\columnwidth}
      \begin{itemize}
        \item<1-> \funcname{match \dots with}\footnotemark pattern matching can also match exceptions
        \item<1-> The CMM of \funcname{probably\_raise} shows that this \funcname{match \dots with} is implemented with a pattern matching and a \funcname{try \dots with}
        \item<1-> But this one only matches on \typename{ExnA} exception
        \item<2-> Since the pattern matching fails to catch the exception, \funcname{reraise} is called with our current \typename{ExnC} exception
        \item<3-> Disassembly shows that \funcname{reraise} is actually a call to \funcname{caml\_reraise\_exn}, which is also an assembly function provided by the runtime
      \end{itemize}
      \vfill
      \mintocaml[firstline=15,lastline=21,highlightlines={18-21}]{../src/backtrace/backtrace.ml}
      \endminipage
    \end{column}
    \begin{column}{0.55\textwidth}
      \centering
      \onslide*<1>{\mintcmm[highlightlines={10-18}]{../src/backtrace/camlBacktrace\_\_probably\_raise\_351.cmm}}
      \onslide*<2>{\listcmm[highlightlines=18]{../src/backtrace/camlBacktrace\_\_probably\_raise_351.cmm}{CMM of probably\_raise}}
      \onslide*<3>{\mintobjdump[firstline=5,lastline=35,highlightlines=31]{../src/backtrace/camlBacktrace\_\_probably\_raise_351.objdump}}
    \end{column}
  \end{columns}
  \only<1>{\footnotetext[1]{https://blog.janestreet.com/pattern-matching-and-exception-handling-unite/}}
\end{frame}

\newcommand\listCamlReraiseExn[1][]{
  \listasm[firstline=727,lastline=734,#1]{../ocaml/runtime/amd64.S}{runtime/amd64.S:727}}

\begin{frame}{Backtraces}{Introducing \funcname{caml\_reraise\_exn}}
  \begin{columns}[c]
    \begin{column}{0.5\textwidth}
      \minipage[c][0.66\textheight][s]{\columnwidth}
      \begin{itemize}
        \item<1-> The exception have to be reraised to the next handler
        \item<2-> First, checks if the backtrace should be collected with \funcarg{domain\_state->backtrace\_active}
        \only<3>{
        \begin{itemize}
            \item If not, behaves like a \funcname{raise\_notrace} with \funcname{RESTORE\_EXN\_HANDLER\_OCAML}
        \end{itemize}
        }
        \item<4-> Jumps into \funcname{caml\_raise\_exn}
        \item<5-> Calls \funcname{caml\_stash\_backtrace} again, to scan the next part of the stack: from the trap just pop-ed to the next trap
      \end{itemize}
      \vfill
      \mintocaml[firstline=15,lastline=21,highlightlines={18-21}]{../src/backtrace/backtrace.ml}
      \endminipage
    \end{column}
    \begin{column}{0.5\textwidth}
      \raggedleft
      \onslide*<1>{\listCamlReraiseExn{}}
      \onslide*<2>{\listCamlReraiseExn[highlightlines=729]{}}
      \onslide*<3>{
        \mintc[firstline=190,lastline=192,highlightlines={191-192}]{../ocaml/runtime/amd64.S}
        \mintc[firstline=234,lastline=237,highlightlines={235-237}]{../ocaml/runtime/amd64.S}
        \medskip
        \listCamlReraiseExn[highlightlines=731]{}
      }
      \onslide*<4-5>{
        % TODO refactor with \listCamlReraiseExn
        \mintasm[firstline=727,lastline=734,highlightlines=730]{../ocaml/runtime/amd64.S}
        \medskip
      }
      \onslide*<4>{\listCamlRaiseExn[highlightlines=711]{}}
      \onslide*<5>{\listCamlRaiseExn[highlightlines=720]{}}
    \end{column}
  \end{columns}
\end{frame}

\begin{frame}{Backtraces}{Backtrace collection continues}
  \begin{columns}[c]
    \begin{column}{0.55\textwidth}
      \minipage[c][0.75\textheight][s]{\columnwidth}
      \begin{itemize}
        \item<1-> \funcname{caml\_stash\_backtrace} scans the older part of the stack, up to the next trap
        \item<6-> Controls jumps to the nested exception handler in \funcname{maybe\_raise}, which matches only on \typename{ExnB}
        \item<7-> \funcname{caml\_reraise\_exn} is called again, backtrace collection resumes with \funcname{caml\_stash\_backtrace}
      \end{itemize}
      \medskip
      \begin{itemize}
        \item<2-> Constructed backtrace:
        \begin{itemize}
          \item <2-> \funcname{definitely\_raise}
          \item <2-> \funcname{probably\_raise}
          \item <3-> \funcname{probably\_raise} (re-raise)
          \item <4-> \funcname{maybe\_raise}
          \item <5-> \funcname{main}
          \item <8-> \funcname{main} (re-raise)
        \end{itemize}
      \end{itemize}
      \vfill
      \onslide*<1-5>{\mintocaml[firstline=15,lastline=21]{../src/backtrace/backtrace.ml}}
      \onslide*<6>{\mintocaml[firstline=28,lastline=34,highlightlines=31]{../src/backtrace/backtrace.ml}}
      \onslide*<7->{\mintocaml[firstline=28,lastline=34,highlightlines=33]{../src/backtrace/backtrace.ml}}
      \endminipage
    \end{column}
    \begin{column}{0.45\textwidth}
      \raggedleft
      \onslide*<1>{\providecommand\step{6}\input{diagrams/backtrace/backtrace.tex}}%
      \onslide*<2>{\providecommand\step{6}\input{diagrams/backtrace/backtrace.tex}}%
      \onslide*<3>{\providecommand\step{7}\input{diagrams/backtrace/backtrace.tex}}%
      \onslide*<4>{\providecommand\step{8}\input{diagrams/backtrace/backtrace.tex}}%
      \onslide*<5>{\providecommand\step{9}\input{diagrams/backtrace/backtrace.tex}}%
      \onslide*<6>{\providecommand\step{10}\input{diagrams/backtrace/backtrace.tex}}%
      \onslide*<7>{\providecommand\step{11}\input{diagrams/backtrace/backtrace.tex}}%
      \onslide*<8>{\providecommand\step{12}\input{diagrams/backtrace/backtrace.tex}}%
      \onslide*<9>{\providecommand\step{13}\input{diagrams/backtrace/backtrace.tex}}%
    \end{column}
  \end{columns}
\end{frame}

\begin{frame}{Backtraces}{Printing the backtrace}
  \begin{columns}[c]
    \begin{column}{0.45\textwidth}
      \minipage[c][0.66\textheight][s]{\columnwidth}
      \begin{itemize}
        \item<1-> The exception backtrace can now be printed to \funcarg{stdout} with \funcname{Printexc.print_backtrace}
        \item<2-> First the exception backtrace is retrieved into an OCaml array with \funcname{get_raw_backtrace }
        \item<3-> Which is implemented as the \funcname{caml_get_exception_raw_backtrace} C function in the runtime
        \item<4-> And finally printed with \funcname{print_exception_backtrace}
      \end{itemize}
      \vfill
      \mintocaml[firstline=28,lastline=34,highlightlines=33]{../src/backtrace/backtrace.ml}
      \endminipage
    \end{column}
    \begin{column}{0.55\textwidth}
      \onslide*<1,2>{
        \mintocaml[firstline=100,lastline=101]{../ocaml/stdlib/printexc.ml}
      }
      \only<1>{
        \listocaml[firstline=170,lastline=172]{../ocaml/stdlib/printexc.ml}{stdlib/printexc.ml:170}
      }
      \only<2>{
        \listocaml[firstline=170,lastline=172,highlightlines=172]{../ocaml/stdlib/printexc.ml}{stdlib/printexc.ml:170}
      }
      \only<3>{
        \mintc[firstline=154,lastline=156]{../ocaml/runtime/backtrace.c}
        \listc[firstline=171,lastline=192,highlightlines={184-187}]{../ocaml/runtime/backtrace.c}{runtime/backtrace.c:154}
      }
      \only<4>{
        \listocaml[firstline=155,lastline=165]{../ocaml/stdlib/printexc.ml}{stdlib/printexc.ml:55}
      }
    \end{column}
  \end{columns}
\end{frame}


\subsection{The multiple kinds of raise}
\frameSubsection{}{}

\begin{frame}{Backtraces}{When backtraces are not needed}
  \begin{columns}[c]
    \begin{column}{0.45\textwidth}
      \minipage[c][0.66\textheight][s]{\columnwidth}
      \begin{itemize}
        \item<1-> What if the backtrace for an exception is never needed?
        \item<2-> It's possible to raise the exception explicitly with \funcname{raise\_notrace} to make raising as cheap as possible
        \item<3-> An example of use in the \funcname{dynlink} library of the OCaml compiler
      \end{itemize}
      \vfill
      \onslide<3->{
      \listocaml[firstline=205,lastline=209,highlightlines=207]{../ocaml/otherlibs/dynlink/dynlink\_compilerlibs/misc.ml}{otherlibs/dynlink/dynlink\_compilerlibs/misc.ml:205}
      }
      \endminipage
    \end{column}
    \begin{column}{0.55\textwidth}
    \only<4>{
      \mintcmm[highlightlines=3]{../src/notrace/camlNotrace\_\_fun_347.cmm}
      \listcmm{../src/notrace/camlNotrace\_\_all\_somes\_327.cmm}{all\_somes}
    }
    \onslide*<5->{
      \listobjdump[highlightlines={6-9}]{../src/notrace/camlNotrace\_\_fun_347.objdump}{anonymous function from \funcname{all\_somes}}
    }
    \end{column}
  \end{columns}
\end{frame}

\begin{frame}{Backtraces}{Summary table of the multiple kinds of raise}
  \begin{itemize}
    \item When debugging information is enabled and if the backtrace of an exception isn't needed, it's possible to raise with \funcname{raise\_notrace} for performance
    \item When debugging information is \emph{not} enabled, the compiler optimises \funcname{raise} calls to inline assembly (same effect as using \funcname{raise\_notrace})
  \end{itemize}
  \bigskip
  \centering
  \begin{tabular}{| l | c | c | c | c |}
    \hline
    \parbox{10em}{Debug information \\ (\funcarg{-g} flag)}  & \multicolumn{2}{|c|}{Disabled}                                     & \multicolumn{2}{|c|}{Enabled} \\
    \hline
    Raise kind                                               & \funcname{raise} & \funcname{raise\_notrace}                       & \funcname{raise} & \funcname{raise\_notrace} \\
    \hline
    Compilation                                              & \parbox{8em}{inline assembly\\ (optimization)} & inline assembly   & \funcname{caml\_raise\_exn} & inline assembly \\
    \hline
  \end{tabular}
\end{frame}


%
% Takeaway #5
%
\subsection*{Takeaway \#5}
\frameSubsectionTakeaway{}
\againframe<5>{takeaway}
}%
    \end{column}
  \end{columns}
\end{frame}

\newcommand\listStashBacktrace[1][]{
  \listc[firstline=91,lastline=120,#1]{../ocaml/runtime/backtrace\_nat.c}{runtime/backtrace\_nat.c:91}}

\begin{frame}{Backtraces}{Introducing \funcname{caml\_stash\_backtrace}}
  \begin{columns}[c]
    \begin{column}{0.5\textwidth}
      \minipage[c][0.66\textheight][s]{\columnwidth}
      \begin{itemize}
        \item<1-> A C function provided by the runtime (specific to native code)
        \item<2-> It scans the stack, between the stack pointer at the raising point up to the next trap, in order to collect the names of the detected function frames
          \onslide*<2>{
            \begin{itemize}
              \item And more information, like line number, column number...
            \end{itemize}
          }
        \item<3-> It uses the same technique and information as the Garbage Collector (GC) uses to scan the values live on the stack
          \onslide*<4>{
            \begin{itemize}
              \item \typename{frame\_descr} are at the core of stack unwinding
            \end{itemize}
          }
      \end{itemize}
      \vfill
      \mintocaml[firstline=9,lastline=13,highlightlines=12]{../src/backtrace/backtrace.ml}
      \endminipage
    \end{column}
    \begin{column}{0.5\textwidth}
      \onslide*<1>{\listStashBacktrace[highlightlines=91]{}}
      \onslide*<2>{\listStashBacktrace[highlightlines={91,117-118}]{}}
      \onslide*<3->{\listStashBacktrace[highlightlines={94,106,109-110}]{}}
    \end{column}
  \end{columns}
\end{frame}

\newcommand\listFrameDescr[1][]{
  \listocaml[firstline=111,lastline=124,#1]{../ocaml/asmcomp/emitaux.ml}{asmcomp/emitaux.ml:111}}
\newcommand\listDebugInfo[1][]{
  \listocaml[firstline=38,lastline=49,#1]{../ocaml/lambda/debuginfo.mli}{lambda/debuginfo.mli:38}}

\begin{frame}{Backtraces}{\typename{frame\_descr} and \typename{frame\_debuginfo} in a nutshell}
  \begin{columns}[c]
    \begin{column}{0.5\textwidth}
      \begin{itemize}
        \item<1-> For every return address in an OCaml program, there is a associated \typename{frame\_descr}
        \item<2-> A \typename{frame\_descr} specifies
        \begin{itemize}
          \item<2-> The size of the function stack frame
          \item<3-> Where live values are on the stack (so that the GC can mark them as live and prevent from being swept)
        \end{itemize}
        \item<4-> When compiled with debug information, \typename{frame\_descr} will be linked to a \typename{frame\_debuginfo}
        \begin{itemize}
          \item<5-> Contains information about the position in the source code (file name, line and column number)
        \end{itemize}
      \end{itemize}
    \end{column}
    \begin{column}{0.5\textwidth}
        \onslide*<1>{\listFrameDescr[highlightlines={118-122}]{}}
        \onslide*<2>{\listFrameDescr[highlightlines={120}]{}}
        \onslide*<3>{\listFrameDescr[highlightlines={121}]{}}
        \onslide*<4->{\listFrameDescr[highlightlines={122}]{}}
        \medskip
        \onslide*<1-3>{\listDebugInfo{}}
        \onslide*<4>{\listDebugInfo[highlightlines={38,49}]{}}
        \onslide*<5>{\listDebugInfo[highlightlines={39-46}]{}}
    \end{column}
  \end{columns}
\end{frame}

\begin{frame}{Backtraces}{Scanning the stack with \typename{frame\_descr}}
  \begin{columns}[c]
    \begin{column}{0.5\textwidth}
      \begin{itemize}
        \item Given the return address of the code that raises the exception by calling \funcname{caml\_raise\_exn} (\funcarg{pc} argument of \funcname{caml\_stash\_backtrace})
        \item And the position of the stack pointer (\funcarg{sp} argument of \funcname{caml\_stash\_backtrace})
        \item The frame size given by \typename{frame\_descr}.\funcarg{frame\_size}, allows to find the end of the previous frame
        \item And the return address of the caller of this function
        \bigskip
        \onslide<3->{
        \item Note that traps (here the reddish \funcname{ExnA}, \funcname{ExnB} and \funcname{ExnC} blocks) belong to the frame that installed them, and so the frame size changes when traps are removed
        }
      \end{itemize}
    \end{column}
    \begin{column}{0.5\textwidth}
      \raggedleft
      \onslide*<1>{\providecommand\step{2}%
% Backtraces
%
\subsection{One advantage of raising with debugging information}
\frameSubsection{}{}

\begin{frame}{Backtraces}{Debugging information \& source code}
  \begin{columns}[c]
    \begin{column}{0.5\textwidth}
      \begin{itemize}
        \item Raising an exception is fun and games, but how do we know where the exception originated? $\rightarrow$ With the backtrace associated with the exception!
        \item In order to get a backtrace from the point where an exception was raised, it's necessary to compile with debugging information enabled (\funcarg{-g} flag for the compiler)
        \item Same example as in the `Nested handlers' section
      \end{itemize}
    \end{column}
    \begin{column}{0.5\textwidth}
      \listocaml[firstline=9,lastline=34]{../src/backtrace/backtrace.ml}{backtrace/backtrace.ml}
    \end{column}
  \end{columns}
\end{frame}

\begin{frame}{Backtraces}{CMM and disassembly of \funcname{definitely\_raise}}
  \begin{columns}[c]
    \begin{column}{0.5\textwidth}
      \minipage[c][0.66\textheight][s]{\columnwidth}
      \begin{itemize}
        \item<1-> An effect of compiling with debugging information (\funcarg{-g}): the CMM no longer uses \funcname{raise\_notrace} to raise the exception but \funcname{raise} instead
        \item<2-> Disassembly shows that CMM \funcname{raise} is actually a call to \funcname{caml\_raise\_exn}, a function in assembler provided by the runtime
      \end{itemize}
      \vfill
      \mintocaml[firstline=9,lastline=13,highlightlines=12]{../src/backtrace/backtrace.ml}
      \endminipage
    \end{column}
    \begin{column}{0.5\textwidth}
      \onslide*<1>{
        \mintcmm[highlightlines=5]{../src/backtrace/camlBacktrace\_\_definitely\_raise\_347.cmm}
      }
      \onslide*<2>{
        \mintobjdump[firstline=2,highlightlines=14]{../src/backtrace/camlBacktrace\_\_definitely\_raise\_347.objdump}
      }
    \end{column}
  \end{columns}
\end{frame}

\begin{frame}{Backtraces}{Introducing \funcname{caml\_raise\_exn}}
  \begin{columns}[c]
    \begin{column}{0.5\textwidth}
      \minipage[c][0.66\textheight][s]{\columnwidth}
      \onslide*<1>{
        \begin{itemize}
          \item Raising the exception is performed using \funcname{caml\_raise\_exn} which is
            \begin{itemize}
              \item An assembly function provided by the runtime
              \item Architecture specific
            \end{itemize}
          \item Let's see what it does differently from \funcname{raise\_notrace}\dots
          \item We are about to raise \typename{ExnC} with \funcname{caml\_raise\_exn} from \funcname{definitely\_raise}
        \end{itemize}
      }
      \onslide*<2>{
        \mintobjdump[firstline=2,highlightlines=14]{../src/backtrace/camlBacktrace\_\_definitely\_raise\_347.objdump}
      }
      \vfill
      \mintocaml[firstline=9,lastline=13,highlightlines=12]{../src/backtrace/backtrace.ml}
      \endminipage
    \end{column}
    \begin{column}{0.5\textwidth}
      \centering
      \providecommand\step{1}\input{diagrams/backtrace/backtrace.tex}
    \end{column}
  \end{columns}
\end{frame}

\begin{frame}{Backtraces}{But before that, we need more fields from \typename{caml\_domain\_state}}
  \begin{columns}
    \begin{column}{0.5\textwidth}
      \begin{itemize}
        \item Remember \typename{caml\_domain\_state} the per-domain runtime state?
        \item Some other members are of interest to us now\dots
        \item \localname{backtrace\_active} is used to know if the runtime should collect backtraces
        \item \localname{backtrace\_active} can be set by
          \begin{itemize}
            \item The \funcarg{b} flag in the \funcname{OCAMLRUNPARAM} environment variable
            \item \mintinline{ocaml}{Printexc.record_backtrace : bool -> unit} \footnotemark
          \end{itemize}
      \end{itemize}
    \end{column}
    \begin{column}{0.5\textwidth}
      \begin{listing}
        \mintc[highlightlines={9-11}]{src/ocaml/domain\_state\_backtrace.h}
        \caption{runtime/caml/domain\_state.h}
      \end{listing}
    \end{column}
  \end{columns}
  \footnotetext[1]{https://v2.ocaml.org/api/Printexc.html}
\end{frame}

\newcommand\listCamlRaiseExn[1][]{
  \listasm[firstline=702,lastline=725,#1]{../ocaml/runtime/amd64.S}{runtime/amd64.S:702}}

\begin{frame}{Backtraces}{Walk through \funcname{caml\_raise\_exn}}
  \begin{columns}[T]
    \begin{column}{0.5\textwidth}
      \begin{itemize}
        \item<1-> First checks whether exceptions backtrace should be recorded
        \item<1-> By checking the \funcarg{domain\_state->backtrace\_active} value
        \item<2-> If not, acts as \funcname{raise\_notrace} with \funcname{RESTORE\_EXN\_HANDLER\_OCAML}
          \begin{itemize}
            \item Set \regname{RSP} to \localname{domain\_state->exn\_handler} value
            \item Restore \localname{domain\_state->exn\_handler} from trap
            \item Jump with \funcname{ret} (same effect as using \funcname{pop} and \funcname{jmp})
          \end{itemize}
        \item<3-> Otherwise, switches to the C stack to be able to execute C code
        \item<4-> Calls \funcname{caml\_stash\_backtrace}, a C function provided by the runtime\ignorespaces
          \onslide<6->{, with
            \begin{itemize}
              \item \funcarg{exn}: exception value
              \item \funcarg{pc}: code address of the raising point
              \item \funcarg{sp}: stack address of the raising function
              \item \funcarg{trapsp}: stack address of the next handler
            \end{itemize}
          }
      \end{itemize}
      \bigskip
      \mintocaml[firstline=9,lastline=13,highlightlines=12]{../src/backtrace/backtrace.ml}
    \end{column}
    \begin{column}{0.5\textwidth}
      \raggedleft
      \onslide*<1,3-4>{
        \mintc[firstline=190,lastline=192]{../ocaml/runtime/amd64.S}
        \mintc[firstline=234,lastline=237]{../ocaml/runtime/amd64.S}
      }
      \onslide*<2>{
        \mintc[firstline=190,lastline=192,highlightlines={191-192}]{../ocaml/runtime/amd64.S}
        \mintc[firstline=234,lastline=237,highlightlines={235-237}]{../ocaml/runtime/amd64.S}
      }
      \onslide*<1>{\listCamlRaiseExn[highlightlines=705]{}}%
      \onslide*<2>{\listCamlRaiseExn[highlightlines={707-708}]{}}%
      \onslide*<3>{\listCamlRaiseExn[highlightlines={712-714}]{}}%
      \onslide*<4>{\listCamlRaiseExn[highlightlines={715-720}]{}}%
      \onslide*<5>{\providecommand\step{1}\input{diagrams/backtrace/backtrace.tex}}%
      \onslide*<6>{\providecommand\step{2}\input{diagrams/backtrace/backtrace.tex}}%
    \end{column}
  \end{columns}
\end{frame}

\newcommand\listStashBacktrace[1][]{
  \listc[firstline=91,lastline=120,#1]{../ocaml/runtime/backtrace\_nat.c}{runtime/backtrace\_nat.c:91}}

\begin{frame}{Backtraces}{Introducing \funcname{caml\_stash\_backtrace}}
  \begin{columns}[c]
    \begin{column}{0.5\textwidth}
      \minipage[c][0.66\textheight][s]{\columnwidth}
      \begin{itemize}
        \item<1-> A C function provided by the runtime (specific to native code)
        \item<2-> It scans the stack, between the stack pointer at the raising point up to the next trap, in order to collect the names of the detected function frames
          \onslide*<2>{
            \begin{itemize}
              \item And more information, like line number, column number...
            \end{itemize}
          }
        \item<3-> It uses the same technique and information as the Garbage Collector (GC) uses to scan the values live on the stack
          \onslide*<4>{
            \begin{itemize}
              \item \typename{frame\_descr} are at the core of stack unwinding
            \end{itemize}
          }
      \end{itemize}
      \vfill
      \mintocaml[firstline=9,lastline=13,highlightlines=12]{../src/backtrace/backtrace.ml}
      \endminipage
    \end{column}
    \begin{column}{0.5\textwidth}
      \onslide*<1>{\listStashBacktrace[highlightlines=91]{}}
      \onslide*<2>{\listStashBacktrace[highlightlines={91,117-118}]{}}
      \onslide*<3->{\listStashBacktrace[highlightlines={94,106,109-110}]{}}
    \end{column}
  \end{columns}
\end{frame}

\newcommand\listFrameDescr[1][]{
  \listocaml[firstline=111,lastline=124,#1]{../ocaml/asmcomp/emitaux.ml}{asmcomp/emitaux.ml:111}}
\newcommand\listDebugInfo[1][]{
  \listocaml[firstline=38,lastline=49,#1]{../ocaml/lambda/debuginfo.mli}{lambda/debuginfo.mli:38}}

\begin{frame}{Backtraces}{\typename{frame\_descr} and \typename{frame\_debuginfo} in a nutshell}
  \begin{columns}[c]
    \begin{column}{0.5\textwidth}
      \begin{itemize}
        \item<1-> For every return address in an OCaml program, there is a associated \typename{frame\_descr}
        \item<2-> A \typename{frame\_descr} specifies
        \begin{itemize}
          \item<2-> The size of the function stack frame
          \item<3-> Where live values are on the stack (so that the GC can mark them as live and prevent from being swept)
        \end{itemize}
        \item<4-> When compiled with debug information, \typename{frame\_descr} will be linked to a \typename{frame\_debuginfo}
        \begin{itemize}
          \item<5-> Contains information about the position in the source code (file name, line and column number)
        \end{itemize}
      \end{itemize}
    \end{column}
    \begin{column}{0.5\textwidth}
        \onslide*<1>{\listFrameDescr[highlightlines={118-122}]{}}
        \onslide*<2>{\listFrameDescr[highlightlines={120}]{}}
        \onslide*<3>{\listFrameDescr[highlightlines={121}]{}}
        \onslide*<4->{\listFrameDescr[highlightlines={122}]{}}
        \medskip
        \onslide*<1-3>{\listDebugInfo{}}
        \onslide*<4>{\listDebugInfo[highlightlines={38,49}]{}}
        \onslide*<5>{\listDebugInfo[highlightlines={39-46}]{}}
    \end{column}
  \end{columns}
\end{frame}

\begin{frame}{Backtraces}{Scanning the stack with \typename{frame\_descr}}
  \begin{columns}[c]
    \begin{column}{0.5\textwidth}
      \begin{itemize}
        \item Given the return address of the code that raises the exception by calling \funcname{caml\_raise\_exn} (\funcarg{pc} argument of \funcname{caml\_stash\_backtrace})
        \item And the position of the stack pointer (\funcarg{sp} argument of \funcname{caml\_stash\_backtrace})
        \item The frame size given by \typename{frame\_descr}.\funcarg{frame\_size}, allows to find the end of the previous frame
        \item And the return address of the caller of this function
        \bigskip
        \onslide<3->{
        \item Note that traps (here the reddish \funcname{ExnA}, \funcname{ExnB} and \funcname{ExnC} blocks) belong to the frame that installed them, and so the frame size changes when traps are removed
        }
      \end{itemize}
    \end{column}
    \begin{column}{0.5\textwidth}
      \raggedleft
      \onslide*<1>{\providecommand\step{2}\input{diagrams/backtrace/backtrace.tex}}%
      \onslide*<2>{\providecommand\step{1}\input{diagrams/backtrace/scanstack.tex}}%
      \onslide*<3>{\providecommand\step{2}\input{diagrams/backtrace/scanstack.tex}}%
      \onslide*<4>{\providecommand\step{3}\input{diagrams/backtrace/scanstack.tex}}%
      \onslide*<5>{\providecommand\step{4}\input{diagrams/backtrace/scanstack.tex}}%
      \onslide*<6>{\providecommand\step{5}\input{diagrams/backtrace/scanstack.tex}}%
    \end{column}
  \end{columns}
\end{frame}

% Duplicate of previous-previous slide
\begin{frame}{Backtraces}{Continuing on \funcname{caml\_stash\_backtrace}}
  \begin{columns}[c]
    \begin{column}{0.5\textwidth}
      \minipage[c][0.75\textheight][s]{\columnwidth}
      \begin{itemize}
          \color{gray}
        \item A C function provided by the runtime (specific to native code)
        \item It scans the stack, between the stack pointer at the raising point up to the next trap, in order to collect the names of the detected function frames
        \item It uses the same technique and information as the Garbage Collector (GC) uses to scan the values live on the stack
      \end{itemize}
      \begin{itemize}
        \item For each function frame detected, store a pointer to the \typename{frame\_descr} in the \localname{domain\_state->backtrace\_buffer} array, effectively constructing the backtrace
      \end{itemize}
      \onslide<2->{
        \begin{itemize}
            \smallskip
          \item Constructed backtrace:
            \funcname{definitely\_raise},
            \onslide<3->{\funcname{probably\_raise}}
        \end{itemize}
      }
      \vfill
      \mintocaml[firstline=9,lastline=13,highlightlines=12]{../src/backtrace/backtrace.ml}
      \endminipage
    \end{column}
    \begin{column}{0.5\textwidth}
      \raggedleft
      \onslide*<1>{\listStashBacktrace[highlightlines={114-115}]{}}
      \onslide*<2>{\providecommand\step{3}\input{diagrams/backtrace/backtrace.tex}}%
      \onslide*<3>{\providecommand\step{4}\input{diagrams/backtrace/backtrace.tex}}%
    \end{column}
  \end{columns}
\end{frame}

\begin{frame}{Backtraces}{Back to \funcname{caml\_raise\_exn}}
  \begin{columns}[c]
    \begin{column}{0.5\textwidth}
      \minipage[c][0.66\textheight][s]{\columnwidth}
      \begin{itemize}\color{gray}
        \item Checks wheteher exceptions backtrace should be recorded, by checking the \funcarg{domain\_state->backtrace\_active} value
        \item Switches to the C stack to be able to execute C code
        \item Calls \funcname{caml\_stash\_backtrace}, to collect the backtrace up to the next trap
      \end{itemize}
      \onslide*<2->{
        \begin{itemize}
          \item<2-> Executes the exception handler in \funcname{probably\_raise} with \funcname{RESTORE\_EXN\_HANDLER\_OCAML}
            \begin{itemize}
              \item Set \regname{RSP} to \localname{domain\_state->exn\_handler} value
              \item Restore \localname{domain\_state->exn\_handler} from trap
              \item Jump to the handler code with \funcname{ret}
            \end{itemize}
        \end{itemize}
      }
      \vfill
      \onslide*<1>{\mintocaml[firstline=9,lastline=13,highlightlines=12]{../src/backtrace/backtrace.ml}}
      \onslide*<2->{\mintocaml[firstline=15,lastline=21,highlightlines={19-21}]{../src/backtrace/backtrace.ml}}
      \endminipage
    \end{column}
    \begin{column}{0.5\textwidth}
      \centering
      \onslide*<1>{
        \mintc[firstline=190,lastline=192]{../ocaml/runtime/amd64.S}
        \mintc[firstline=234,lastline=237]{../ocaml/runtime/amd64.S}
      }
      \onslide*<2>{
        \mintc[firstline=190,lastline=192,highlightlines={191-192}]{../ocaml/runtime/amd64.S}
        \mintc[firstline=234,lastline=237,highlightlines={235-237}]{../ocaml/runtime/amd64.S}
      }
      \onslide*<1>{\listCamlRaiseExn[highlightlines=720]{}}
      \onslide*<2>{\listCamlRaiseExn[highlightlines=722]{}}
      \onslide*<3->{\providecommand\step{5}\input{diagrams/backtrace/backtrace.tex}}
    \end{column}
  \end{columns}
\end{frame}

\begin{frame}{Backtraces}{\funcname{caml\_probably\_raise}'s handler doesn't catch \typename{ExnC}}
  \begin{columns}[c]
    \begin{column}{0.45\textwidth}
      \minipage[c][0.75\textheight][s]{\columnwidth}
      \begin{itemize}
        \item<1-> \funcname{match \dots with}\footnotemark pattern matching can also match exceptions
        \item<1-> The CMM of \funcname{probably\_raise} shows that this \funcname{match \dots with} is implemented with a pattern matching and a \funcname{try \dots with}
        \item<1-> But this one only matches on \typename{ExnA} exception
        \item<2-> Since the pattern matching fails to catch the exception, \funcname{reraise} is called with our current \typename{ExnC} exception
        \item<3-> Disassembly shows that \funcname{reraise} is actually a call to \funcname{caml\_reraise\_exn}, which is also an assembly function provided by the runtime
      \end{itemize}
      \vfill
      \mintocaml[firstline=15,lastline=21,highlightlines={18-21}]{../src/backtrace/backtrace.ml}
      \endminipage
    \end{column}
    \begin{column}{0.55\textwidth}
      \centering
      \onslide*<1>{\mintcmm[highlightlines={10-18}]{../src/backtrace/camlBacktrace\_\_probably\_raise\_351.cmm}}
      \onslide*<2>{\listcmm[highlightlines=18]{../src/backtrace/camlBacktrace\_\_probably\_raise_351.cmm}{CMM of probably\_raise}}
      \onslide*<3>{\mintobjdump[firstline=5,lastline=35,highlightlines=31]{../src/backtrace/camlBacktrace\_\_probably\_raise_351.objdump}}
    \end{column}
  \end{columns}
  \only<1>{\footnotetext[1]{https://blog.janestreet.com/pattern-matching-and-exception-handling-unite/}}
\end{frame}

\newcommand\listCamlReraiseExn[1][]{
  \listasm[firstline=727,lastline=734,#1]{../ocaml/runtime/amd64.S}{runtime/amd64.S:727}}

\begin{frame}{Backtraces}{Introducing \funcname{caml\_reraise\_exn}}
  \begin{columns}[c]
    \begin{column}{0.5\textwidth}
      \minipage[c][0.66\textheight][s]{\columnwidth}
      \begin{itemize}
        \item<1-> The exception have to be reraised to the next handler
        \item<2-> First, checks if the backtrace should be collected with \funcarg{domain\_state->backtrace\_active}
        \only<3>{
        \begin{itemize}
            \item If not, behaves like a \funcname{raise\_notrace} with \funcname{RESTORE\_EXN\_HANDLER\_OCAML}
        \end{itemize}
        }
        \item<4-> Jumps into \funcname{caml\_raise\_exn}
        \item<5-> Calls \funcname{caml\_stash\_backtrace} again, to scan the next part of the stack: from the trap just pop-ed to the next trap
      \end{itemize}
      \vfill
      \mintocaml[firstline=15,lastline=21,highlightlines={18-21}]{../src/backtrace/backtrace.ml}
      \endminipage
    \end{column}
    \begin{column}{0.5\textwidth}
      \raggedleft
      \onslide*<1>{\listCamlReraiseExn{}}
      \onslide*<2>{\listCamlReraiseExn[highlightlines=729]{}}
      \onslide*<3>{
        \mintc[firstline=190,lastline=192,highlightlines={191-192}]{../ocaml/runtime/amd64.S}
        \mintc[firstline=234,lastline=237,highlightlines={235-237}]{../ocaml/runtime/amd64.S}
        \medskip
        \listCamlReraiseExn[highlightlines=731]{}
      }
      \onslide*<4-5>{
        % TODO refactor with \listCamlReraiseExn
        \mintasm[firstline=727,lastline=734,highlightlines=730]{../ocaml/runtime/amd64.S}
        \medskip
      }
      \onslide*<4>{\listCamlRaiseExn[highlightlines=711]{}}
      \onslide*<5>{\listCamlRaiseExn[highlightlines=720]{}}
    \end{column}
  \end{columns}
\end{frame}

\begin{frame}{Backtraces}{Backtrace collection continues}
  \begin{columns}[c]
    \begin{column}{0.55\textwidth}
      \minipage[c][0.75\textheight][s]{\columnwidth}
      \begin{itemize}
        \item<1-> \funcname{caml\_stash\_backtrace} scans the older part of the stack, up to the next trap
        \item<6-> Controls jumps to the nested exception handler in \funcname{maybe\_raise}, which matches only on \typename{ExnB}
        \item<7-> \funcname{caml\_reraise\_exn} is called again, backtrace collection resumes with \funcname{caml\_stash\_backtrace}
      \end{itemize}
      \medskip
      \begin{itemize}
        \item<2-> Constructed backtrace:
        \begin{itemize}
          \item <2-> \funcname{definitely\_raise}
          \item <2-> \funcname{probably\_raise}
          \item <3-> \funcname{probably\_raise} (re-raise)
          \item <4-> \funcname{maybe\_raise}
          \item <5-> \funcname{main}
          \item <8-> \funcname{main} (re-raise)
        \end{itemize}
      \end{itemize}
      \vfill
      \onslide*<1-5>{\mintocaml[firstline=15,lastline=21]{../src/backtrace/backtrace.ml}}
      \onslide*<6>{\mintocaml[firstline=28,lastline=34,highlightlines=31]{../src/backtrace/backtrace.ml}}
      \onslide*<7->{\mintocaml[firstline=28,lastline=34,highlightlines=33]{../src/backtrace/backtrace.ml}}
      \endminipage
    \end{column}
    \begin{column}{0.45\textwidth}
      \raggedleft
      \onslide*<1>{\providecommand\step{6}\input{diagrams/backtrace/backtrace.tex}}%
      \onslide*<2>{\providecommand\step{6}\input{diagrams/backtrace/backtrace.tex}}%
      \onslide*<3>{\providecommand\step{7}\input{diagrams/backtrace/backtrace.tex}}%
      \onslide*<4>{\providecommand\step{8}\input{diagrams/backtrace/backtrace.tex}}%
      \onslide*<5>{\providecommand\step{9}\input{diagrams/backtrace/backtrace.tex}}%
      \onslide*<6>{\providecommand\step{10}\input{diagrams/backtrace/backtrace.tex}}%
      \onslide*<7>{\providecommand\step{11}\input{diagrams/backtrace/backtrace.tex}}%
      \onslide*<8>{\providecommand\step{12}\input{diagrams/backtrace/backtrace.tex}}%
      \onslide*<9>{\providecommand\step{13}\input{diagrams/backtrace/backtrace.tex}}%
    \end{column}
  \end{columns}
\end{frame}

\begin{frame}{Backtraces}{Printing the backtrace}
  \begin{columns}[c]
    \begin{column}{0.45\textwidth}
      \minipage[c][0.66\textheight][s]{\columnwidth}
      \begin{itemize}
        \item<1-> The exception backtrace can now be printed to \funcarg{stdout} with \funcname{Printexc.print_backtrace}
        \item<2-> First the exception backtrace is retrieved into an OCaml array with \funcname{get_raw_backtrace }
        \item<3-> Which is implemented as the \funcname{caml_get_exception_raw_backtrace} C function in the runtime
        \item<4-> And finally printed with \funcname{print_exception_backtrace}
      \end{itemize}
      \vfill
      \mintocaml[firstline=28,lastline=34,highlightlines=33]{../src/backtrace/backtrace.ml}
      \endminipage
    \end{column}
    \begin{column}{0.55\textwidth}
      \onslide*<1,2>{
        \mintocaml[firstline=100,lastline=101]{../ocaml/stdlib/printexc.ml}
      }
      \only<1>{
        \listocaml[firstline=170,lastline=172]{../ocaml/stdlib/printexc.ml}{stdlib/printexc.ml:170}
      }
      \only<2>{
        \listocaml[firstline=170,lastline=172,highlightlines=172]{../ocaml/stdlib/printexc.ml}{stdlib/printexc.ml:170}
      }
      \only<3>{
        \mintc[firstline=154,lastline=156]{../ocaml/runtime/backtrace.c}
        \listc[firstline=171,lastline=192,highlightlines={184-187}]{../ocaml/runtime/backtrace.c}{runtime/backtrace.c:154}
      }
      \only<4>{
        \listocaml[firstline=155,lastline=165]{../ocaml/stdlib/printexc.ml}{stdlib/printexc.ml:55}
      }
    \end{column}
  \end{columns}
\end{frame}


\subsection{The multiple kinds of raise}
\frameSubsection{}{}

\begin{frame}{Backtraces}{When backtraces are not needed}
  \begin{columns}[c]
    \begin{column}{0.45\textwidth}
      \minipage[c][0.66\textheight][s]{\columnwidth}
      \begin{itemize}
        \item<1-> What if the backtrace for an exception is never needed?
        \item<2-> It's possible to raise the exception explicitly with \funcname{raise\_notrace} to make raising as cheap as possible
        \item<3-> An example of use in the \funcname{dynlink} library of the OCaml compiler
      \end{itemize}
      \vfill
      \onslide<3->{
      \listocaml[firstline=205,lastline=209,highlightlines=207]{../ocaml/otherlibs/dynlink/dynlink\_compilerlibs/misc.ml}{otherlibs/dynlink/dynlink\_compilerlibs/misc.ml:205}
      }
      \endminipage
    \end{column}
    \begin{column}{0.55\textwidth}
    \only<4>{
      \mintcmm[highlightlines=3]{../src/notrace/camlNotrace\_\_fun_347.cmm}
      \listcmm{../src/notrace/camlNotrace\_\_all\_somes\_327.cmm}{all\_somes}
    }
    \onslide*<5->{
      \listobjdump[highlightlines={6-9}]{../src/notrace/camlNotrace\_\_fun_347.objdump}{anonymous function from \funcname{all\_somes}}
    }
    \end{column}
  \end{columns}
\end{frame}

\begin{frame}{Backtraces}{Summary table of the multiple kinds of raise}
  \begin{itemize}
    \item When debugging information is enabled and if the backtrace of an exception isn't needed, it's possible to raise with \funcname{raise\_notrace} for performance
    \item When debugging information is \emph{not} enabled, the compiler optimises \funcname{raise} calls to inline assembly (same effect as using \funcname{raise\_notrace})
  \end{itemize}
  \bigskip
  \centering
  \begin{tabular}{| l | c | c | c | c |}
    \hline
    \parbox{10em}{Debug information \\ (\funcarg{-g} flag)}  & \multicolumn{2}{|c|}{Disabled}                                     & \multicolumn{2}{|c|}{Enabled} \\
    \hline
    Raise kind                                               & \funcname{raise} & \funcname{raise\_notrace}                       & \funcname{raise} & \funcname{raise\_notrace} \\
    \hline
    Compilation                                              & \parbox{8em}{inline assembly\\ (optimization)} & inline assembly   & \funcname{caml\_raise\_exn} & inline assembly \\
    \hline
  \end{tabular}
\end{frame}


%
% Takeaway #5
%
\subsection*{Takeaway \#5}
\frameSubsectionTakeaway{}
\againframe<5>{takeaway}
}%
      \onslide*<2>{\providecommand\step{1}\input{diagrams/backtrace/scanstack.tex}}%
      \onslide*<3>{\providecommand\step{2}\input{diagrams/backtrace/scanstack.tex}}%
      \onslide*<4>{\providecommand\step{3}\input{diagrams/backtrace/scanstack.tex}}%
      \onslide*<5>{\providecommand\step{4}\input{diagrams/backtrace/scanstack.tex}}%
      \onslide*<6>{\providecommand\step{5}\input{diagrams/backtrace/scanstack.tex}}%
    \end{column}
  \end{columns}
\end{frame}

% Duplicate of previous-previous slide
\begin{frame}{Backtraces}{Continuing on \funcname{caml\_stash\_backtrace}}
  \begin{columns}[c]
    \begin{column}{0.5\textwidth}
      \minipage[c][0.75\textheight][s]{\columnwidth}
      \begin{itemize}
          \color{gray}
        \item A C function provided by the runtime (specific to native code)
        \item It scans the stack, between the stack pointer at the raising point up to the next trap, in order to collect the names of the detected function frames
        \item It uses the same technique and information as the Garbage Collector (GC) uses to scan the values live on the stack
      \end{itemize}
      \begin{itemize}
        \item For each function frame detected, store a pointer to the \typename{frame\_descr} in the \localname{domain\_state->backtrace\_buffer} array, effectively constructing the backtrace
      \end{itemize}
      \onslide<2->{
        \begin{itemize}
            \smallskip
          \item Constructed backtrace:
            \funcname{definitely\_raise},
            \onslide<3->{\funcname{probably\_raise}}
        \end{itemize}
      }
      \vfill
      \mintocaml[firstline=9,lastline=13,highlightlines=12]{../src/backtrace/backtrace.ml}
      \endminipage
    \end{column}
    \begin{column}{0.5\textwidth}
      \raggedleft
      \onslide*<1>{\listStashBacktrace[highlightlines={114-115}]{}}
      \onslide*<2>{\providecommand\step{3}%
% Backtraces
%
\subsection{One advantage of raising with debugging information}
\frameSubsection{}{}

\begin{frame}{Backtraces}{Debugging information \& source code}
  \begin{columns}[c]
    \begin{column}{0.5\textwidth}
      \begin{itemize}
        \item Raising an exception is fun and games, but how do we know where the exception originated? $\rightarrow$ With the backtrace associated with the exception!
        \item In order to get a backtrace from the point where an exception was raised, it's necessary to compile with debugging information enabled (\funcarg{-g} flag for the compiler)
        \item Same example as in the `Nested handlers' section
      \end{itemize}
    \end{column}
    \begin{column}{0.5\textwidth}
      \listocaml[firstline=9,lastline=34]{../src/backtrace/backtrace.ml}{backtrace/backtrace.ml}
    \end{column}
  \end{columns}
\end{frame}

\begin{frame}{Backtraces}{CMM and disassembly of \funcname{definitely\_raise}}
  \begin{columns}[c]
    \begin{column}{0.5\textwidth}
      \minipage[c][0.66\textheight][s]{\columnwidth}
      \begin{itemize}
        \item<1-> An effect of compiling with debugging information (\funcarg{-g}): the CMM no longer uses \funcname{raise\_notrace} to raise the exception but \funcname{raise} instead
        \item<2-> Disassembly shows that CMM \funcname{raise} is actually a call to \funcname{caml\_raise\_exn}, a function in assembler provided by the runtime
      \end{itemize}
      \vfill
      \mintocaml[firstline=9,lastline=13,highlightlines=12]{../src/backtrace/backtrace.ml}
      \endminipage
    \end{column}
    \begin{column}{0.5\textwidth}
      \onslide*<1>{
        \mintcmm[highlightlines=5]{../src/backtrace/camlBacktrace\_\_definitely\_raise\_347.cmm}
      }
      \onslide*<2>{
        \mintobjdump[firstline=2,highlightlines=14]{../src/backtrace/camlBacktrace\_\_definitely\_raise\_347.objdump}
      }
    \end{column}
  \end{columns}
\end{frame}

\begin{frame}{Backtraces}{Introducing \funcname{caml\_raise\_exn}}
  \begin{columns}[c]
    \begin{column}{0.5\textwidth}
      \minipage[c][0.66\textheight][s]{\columnwidth}
      \onslide*<1>{
        \begin{itemize}
          \item Raising the exception is performed using \funcname{caml\_raise\_exn} which is
            \begin{itemize}
              \item An assembly function provided by the runtime
              \item Architecture specific
            \end{itemize}
          \item Let's see what it does differently from \funcname{raise\_notrace}\dots
          \item We are about to raise \typename{ExnC} with \funcname{caml\_raise\_exn} from \funcname{definitely\_raise}
        \end{itemize}
      }
      \onslide*<2>{
        \mintobjdump[firstline=2,highlightlines=14]{../src/backtrace/camlBacktrace\_\_definitely\_raise\_347.objdump}
      }
      \vfill
      \mintocaml[firstline=9,lastline=13,highlightlines=12]{../src/backtrace/backtrace.ml}
      \endminipage
    \end{column}
    \begin{column}{0.5\textwidth}
      \centering
      \providecommand\step{1}\input{diagrams/backtrace/backtrace.tex}
    \end{column}
  \end{columns}
\end{frame}

\begin{frame}{Backtraces}{But before that, we need more fields from \typename{caml\_domain\_state}}
  \begin{columns}
    \begin{column}{0.5\textwidth}
      \begin{itemize}
        \item Remember \typename{caml\_domain\_state} the per-domain runtime state?
        \item Some other members are of interest to us now\dots
        \item \localname{backtrace\_active} is used to know if the runtime should collect backtraces
        \item \localname{backtrace\_active} can be set by
          \begin{itemize}
            \item The \funcarg{b} flag in the \funcname{OCAMLRUNPARAM} environment variable
            \item \mintinline{ocaml}{Printexc.record_backtrace : bool -> unit} \footnotemark
          \end{itemize}
      \end{itemize}
    \end{column}
    \begin{column}{0.5\textwidth}
      \begin{listing}
        \mintc[highlightlines={9-11}]{src/ocaml/domain\_state\_backtrace.h}
        \caption{runtime/caml/domain\_state.h}
      \end{listing}
    \end{column}
  \end{columns}
  \footnotetext[1]{https://v2.ocaml.org/api/Printexc.html}
\end{frame}

\newcommand\listCamlRaiseExn[1][]{
  \listasm[firstline=702,lastline=725,#1]{../ocaml/runtime/amd64.S}{runtime/amd64.S:702}}

\begin{frame}{Backtraces}{Walk through \funcname{caml\_raise\_exn}}
  \begin{columns}[T]
    \begin{column}{0.5\textwidth}
      \begin{itemize}
        \item<1-> First checks whether exceptions backtrace should be recorded
        \item<1-> By checking the \funcarg{domain\_state->backtrace\_active} value
        \item<2-> If not, acts as \funcname{raise\_notrace} with \funcname{RESTORE\_EXN\_HANDLER\_OCAML}
          \begin{itemize}
            \item Set \regname{RSP} to \localname{domain\_state->exn\_handler} value
            \item Restore \localname{domain\_state->exn\_handler} from trap
            \item Jump with \funcname{ret} (same effect as using \funcname{pop} and \funcname{jmp})
          \end{itemize}
        \item<3-> Otherwise, switches to the C stack to be able to execute C code
        \item<4-> Calls \funcname{caml\_stash\_backtrace}, a C function provided by the runtime\ignorespaces
          \onslide<6->{, with
            \begin{itemize}
              \item \funcarg{exn}: exception value
              \item \funcarg{pc}: code address of the raising point
              \item \funcarg{sp}: stack address of the raising function
              \item \funcarg{trapsp}: stack address of the next handler
            \end{itemize}
          }
      \end{itemize}
      \bigskip
      \mintocaml[firstline=9,lastline=13,highlightlines=12]{../src/backtrace/backtrace.ml}
    \end{column}
    \begin{column}{0.5\textwidth}
      \raggedleft
      \onslide*<1,3-4>{
        \mintc[firstline=190,lastline=192]{../ocaml/runtime/amd64.S}
        \mintc[firstline=234,lastline=237]{../ocaml/runtime/amd64.S}
      }
      \onslide*<2>{
        \mintc[firstline=190,lastline=192,highlightlines={191-192}]{../ocaml/runtime/amd64.S}
        \mintc[firstline=234,lastline=237,highlightlines={235-237}]{../ocaml/runtime/amd64.S}
      }
      \onslide*<1>{\listCamlRaiseExn[highlightlines=705]{}}%
      \onslide*<2>{\listCamlRaiseExn[highlightlines={707-708}]{}}%
      \onslide*<3>{\listCamlRaiseExn[highlightlines={712-714}]{}}%
      \onslide*<4>{\listCamlRaiseExn[highlightlines={715-720}]{}}%
      \onslide*<5>{\providecommand\step{1}\input{diagrams/backtrace/backtrace.tex}}%
      \onslide*<6>{\providecommand\step{2}\input{diagrams/backtrace/backtrace.tex}}%
    \end{column}
  \end{columns}
\end{frame}

\newcommand\listStashBacktrace[1][]{
  \listc[firstline=91,lastline=120,#1]{../ocaml/runtime/backtrace\_nat.c}{runtime/backtrace\_nat.c:91}}

\begin{frame}{Backtraces}{Introducing \funcname{caml\_stash\_backtrace}}
  \begin{columns}[c]
    \begin{column}{0.5\textwidth}
      \minipage[c][0.66\textheight][s]{\columnwidth}
      \begin{itemize}
        \item<1-> A C function provided by the runtime (specific to native code)
        \item<2-> It scans the stack, between the stack pointer at the raising point up to the next trap, in order to collect the names of the detected function frames
          \onslide*<2>{
            \begin{itemize}
              \item And more information, like line number, column number...
            \end{itemize}
          }
        \item<3-> It uses the same technique and information as the Garbage Collector (GC) uses to scan the values live on the stack
          \onslide*<4>{
            \begin{itemize}
              \item \typename{frame\_descr} are at the core of stack unwinding
            \end{itemize}
          }
      \end{itemize}
      \vfill
      \mintocaml[firstline=9,lastline=13,highlightlines=12]{../src/backtrace/backtrace.ml}
      \endminipage
    \end{column}
    \begin{column}{0.5\textwidth}
      \onslide*<1>{\listStashBacktrace[highlightlines=91]{}}
      \onslide*<2>{\listStashBacktrace[highlightlines={91,117-118}]{}}
      \onslide*<3->{\listStashBacktrace[highlightlines={94,106,109-110}]{}}
    \end{column}
  \end{columns}
\end{frame}

\newcommand\listFrameDescr[1][]{
  \listocaml[firstline=111,lastline=124,#1]{../ocaml/asmcomp/emitaux.ml}{asmcomp/emitaux.ml:111}}
\newcommand\listDebugInfo[1][]{
  \listocaml[firstline=38,lastline=49,#1]{../ocaml/lambda/debuginfo.mli}{lambda/debuginfo.mli:38}}

\begin{frame}{Backtraces}{\typename{frame\_descr} and \typename{frame\_debuginfo} in a nutshell}
  \begin{columns}[c]
    \begin{column}{0.5\textwidth}
      \begin{itemize}
        \item<1-> For every return address in an OCaml program, there is a associated \typename{frame\_descr}
        \item<2-> A \typename{frame\_descr} specifies
        \begin{itemize}
          \item<2-> The size of the function stack frame
          \item<3-> Where live values are on the stack (so that the GC can mark them as live and prevent from being swept)
        \end{itemize}
        \item<4-> When compiled with debug information, \typename{frame\_descr} will be linked to a \typename{frame\_debuginfo}
        \begin{itemize}
          \item<5-> Contains information about the position in the source code (file name, line and column number)
        \end{itemize}
      \end{itemize}
    \end{column}
    \begin{column}{0.5\textwidth}
        \onslide*<1>{\listFrameDescr[highlightlines={118-122}]{}}
        \onslide*<2>{\listFrameDescr[highlightlines={120}]{}}
        \onslide*<3>{\listFrameDescr[highlightlines={121}]{}}
        \onslide*<4->{\listFrameDescr[highlightlines={122}]{}}
        \medskip
        \onslide*<1-3>{\listDebugInfo{}}
        \onslide*<4>{\listDebugInfo[highlightlines={38,49}]{}}
        \onslide*<5>{\listDebugInfo[highlightlines={39-46}]{}}
    \end{column}
  \end{columns}
\end{frame}

\begin{frame}{Backtraces}{Scanning the stack with \typename{frame\_descr}}
  \begin{columns}[c]
    \begin{column}{0.5\textwidth}
      \begin{itemize}
        \item Given the return address of the code that raises the exception by calling \funcname{caml\_raise\_exn} (\funcarg{pc} argument of \funcname{caml\_stash\_backtrace})
        \item And the position of the stack pointer (\funcarg{sp} argument of \funcname{caml\_stash\_backtrace})
        \item The frame size given by \typename{frame\_descr}.\funcarg{frame\_size}, allows to find the end of the previous frame
        \item And the return address of the caller of this function
        \bigskip
        \onslide<3->{
        \item Note that traps (here the reddish \funcname{ExnA}, \funcname{ExnB} and \funcname{ExnC} blocks) belong to the frame that installed them, and so the frame size changes when traps are removed
        }
      \end{itemize}
    \end{column}
    \begin{column}{0.5\textwidth}
      \raggedleft
      \onslide*<1>{\providecommand\step{2}\input{diagrams/backtrace/backtrace.tex}}%
      \onslide*<2>{\providecommand\step{1}\input{diagrams/backtrace/scanstack.tex}}%
      \onslide*<3>{\providecommand\step{2}\input{diagrams/backtrace/scanstack.tex}}%
      \onslide*<4>{\providecommand\step{3}\input{diagrams/backtrace/scanstack.tex}}%
      \onslide*<5>{\providecommand\step{4}\input{diagrams/backtrace/scanstack.tex}}%
      \onslide*<6>{\providecommand\step{5}\input{diagrams/backtrace/scanstack.tex}}%
    \end{column}
  \end{columns}
\end{frame}

% Duplicate of previous-previous slide
\begin{frame}{Backtraces}{Continuing on \funcname{caml\_stash\_backtrace}}
  \begin{columns}[c]
    \begin{column}{0.5\textwidth}
      \minipage[c][0.75\textheight][s]{\columnwidth}
      \begin{itemize}
          \color{gray}
        \item A C function provided by the runtime (specific to native code)
        \item It scans the stack, between the stack pointer at the raising point up to the next trap, in order to collect the names of the detected function frames
        \item It uses the same technique and information as the Garbage Collector (GC) uses to scan the values live on the stack
      \end{itemize}
      \begin{itemize}
        \item For each function frame detected, store a pointer to the \typename{frame\_descr} in the \localname{domain\_state->backtrace\_buffer} array, effectively constructing the backtrace
      \end{itemize}
      \onslide<2->{
        \begin{itemize}
            \smallskip
          \item Constructed backtrace:
            \funcname{definitely\_raise},
            \onslide<3->{\funcname{probably\_raise}}
        \end{itemize}
      }
      \vfill
      \mintocaml[firstline=9,lastline=13,highlightlines=12]{../src/backtrace/backtrace.ml}
      \endminipage
    \end{column}
    \begin{column}{0.5\textwidth}
      \raggedleft
      \onslide*<1>{\listStashBacktrace[highlightlines={114-115}]{}}
      \onslide*<2>{\providecommand\step{3}\input{diagrams/backtrace/backtrace.tex}}%
      \onslide*<3>{\providecommand\step{4}\input{diagrams/backtrace/backtrace.tex}}%
    \end{column}
  \end{columns}
\end{frame}

\begin{frame}{Backtraces}{Back to \funcname{caml\_raise\_exn}}
  \begin{columns}[c]
    \begin{column}{0.5\textwidth}
      \minipage[c][0.66\textheight][s]{\columnwidth}
      \begin{itemize}\color{gray}
        \item Checks wheteher exceptions backtrace should be recorded, by checking the \funcarg{domain\_state->backtrace\_active} value
        \item Switches to the C stack to be able to execute C code
        \item Calls \funcname{caml\_stash\_backtrace}, to collect the backtrace up to the next trap
      \end{itemize}
      \onslide*<2->{
        \begin{itemize}
          \item<2-> Executes the exception handler in \funcname{probably\_raise} with \funcname{RESTORE\_EXN\_HANDLER\_OCAML}
            \begin{itemize}
              \item Set \regname{RSP} to \localname{domain\_state->exn\_handler} value
              \item Restore \localname{domain\_state->exn\_handler} from trap
              \item Jump to the handler code with \funcname{ret}
            \end{itemize}
        \end{itemize}
      }
      \vfill
      \onslide*<1>{\mintocaml[firstline=9,lastline=13,highlightlines=12]{../src/backtrace/backtrace.ml}}
      \onslide*<2->{\mintocaml[firstline=15,lastline=21,highlightlines={19-21}]{../src/backtrace/backtrace.ml}}
      \endminipage
    \end{column}
    \begin{column}{0.5\textwidth}
      \centering
      \onslide*<1>{
        \mintc[firstline=190,lastline=192]{../ocaml/runtime/amd64.S}
        \mintc[firstline=234,lastline=237]{../ocaml/runtime/amd64.S}
      }
      \onslide*<2>{
        \mintc[firstline=190,lastline=192,highlightlines={191-192}]{../ocaml/runtime/amd64.S}
        \mintc[firstline=234,lastline=237,highlightlines={235-237}]{../ocaml/runtime/amd64.S}
      }
      \onslide*<1>{\listCamlRaiseExn[highlightlines=720]{}}
      \onslide*<2>{\listCamlRaiseExn[highlightlines=722]{}}
      \onslide*<3->{\providecommand\step{5}\input{diagrams/backtrace/backtrace.tex}}
    \end{column}
  \end{columns}
\end{frame}

\begin{frame}{Backtraces}{\funcname{caml\_probably\_raise}'s handler doesn't catch \typename{ExnC}}
  \begin{columns}[c]
    \begin{column}{0.45\textwidth}
      \minipage[c][0.75\textheight][s]{\columnwidth}
      \begin{itemize}
        \item<1-> \funcname{match \dots with}\footnotemark pattern matching can also match exceptions
        \item<1-> The CMM of \funcname{probably\_raise} shows that this \funcname{match \dots with} is implemented with a pattern matching and a \funcname{try \dots with}
        \item<1-> But this one only matches on \typename{ExnA} exception
        \item<2-> Since the pattern matching fails to catch the exception, \funcname{reraise} is called with our current \typename{ExnC} exception
        \item<3-> Disassembly shows that \funcname{reraise} is actually a call to \funcname{caml\_reraise\_exn}, which is also an assembly function provided by the runtime
      \end{itemize}
      \vfill
      \mintocaml[firstline=15,lastline=21,highlightlines={18-21}]{../src/backtrace/backtrace.ml}
      \endminipage
    \end{column}
    \begin{column}{0.55\textwidth}
      \centering
      \onslide*<1>{\mintcmm[highlightlines={10-18}]{../src/backtrace/camlBacktrace\_\_probably\_raise\_351.cmm}}
      \onslide*<2>{\listcmm[highlightlines=18]{../src/backtrace/camlBacktrace\_\_probably\_raise_351.cmm}{CMM of probably\_raise}}
      \onslide*<3>{\mintobjdump[firstline=5,lastline=35,highlightlines=31]{../src/backtrace/camlBacktrace\_\_probably\_raise_351.objdump}}
    \end{column}
  \end{columns}
  \only<1>{\footnotetext[1]{https://blog.janestreet.com/pattern-matching-and-exception-handling-unite/}}
\end{frame}

\newcommand\listCamlReraiseExn[1][]{
  \listasm[firstline=727,lastline=734,#1]{../ocaml/runtime/amd64.S}{runtime/amd64.S:727}}

\begin{frame}{Backtraces}{Introducing \funcname{caml\_reraise\_exn}}
  \begin{columns}[c]
    \begin{column}{0.5\textwidth}
      \minipage[c][0.66\textheight][s]{\columnwidth}
      \begin{itemize}
        \item<1-> The exception have to be reraised to the next handler
        \item<2-> First, checks if the backtrace should be collected with \funcarg{domain\_state->backtrace\_active}
        \only<3>{
        \begin{itemize}
            \item If not, behaves like a \funcname{raise\_notrace} with \funcname{RESTORE\_EXN\_HANDLER\_OCAML}
        \end{itemize}
        }
        \item<4-> Jumps into \funcname{caml\_raise\_exn}
        \item<5-> Calls \funcname{caml\_stash\_backtrace} again, to scan the next part of the stack: from the trap just pop-ed to the next trap
      \end{itemize}
      \vfill
      \mintocaml[firstline=15,lastline=21,highlightlines={18-21}]{../src/backtrace/backtrace.ml}
      \endminipage
    \end{column}
    \begin{column}{0.5\textwidth}
      \raggedleft
      \onslide*<1>{\listCamlReraiseExn{}}
      \onslide*<2>{\listCamlReraiseExn[highlightlines=729]{}}
      \onslide*<3>{
        \mintc[firstline=190,lastline=192,highlightlines={191-192}]{../ocaml/runtime/amd64.S}
        \mintc[firstline=234,lastline=237,highlightlines={235-237}]{../ocaml/runtime/amd64.S}
        \medskip
        \listCamlReraiseExn[highlightlines=731]{}
      }
      \onslide*<4-5>{
        % TODO refactor with \listCamlReraiseExn
        \mintasm[firstline=727,lastline=734,highlightlines=730]{../ocaml/runtime/amd64.S}
        \medskip
      }
      \onslide*<4>{\listCamlRaiseExn[highlightlines=711]{}}
      \onslide*<5>{\listCamlRaiseExn[highlightlines=720]{}}
    \end{column}
  \end{columns}
\end{frame}

\begin{frame}{Backtraces}{Backtrace collection continues}
  \begin{columns}[c]
    \begin{column}{0.55\textwidth}
      \minipage[c][0.75\textheight][s]{\columnwidth}
      \begin{itemize}
        \item<1-> \funcname{caml\_stash\_backtrace} scans the older part of the stack, up to the next trap
        \item<6-> Controls jumps to the nested exception handler in \funcname{maybe\_raise}, which matches only on \typename{ExnB}
        \item<7-> \funcname{caml\_reraise\_exn} is called again, backtrace collection resumes with \funcname{caml\_stash\_backtrace}
      \end{itemize}
      \medskip
      \begin{itemize}
        \item<2-> Constructed backtrace:
        \begin{itemize}
          \item <2-> \funcname{definitely\_raise}
          \item <2-> \funcname{probably\_raise}
          \item <3-> \funcname{probably\_raise} (re-raise)
          \item <4-> \funcname{maybe\_raise}
          \item <5-> \funcname{main}
          \item <8-> \funcname{main} (re-raise)
        \end{itemize}
      \end{itemize}
      \vfill
      \onslide*<1-5>{\mintocaml[firstline=15,lastline=21]{../src/backtrace/backtrace.ml}}
      \onslide*<6>{\mintocaml[firstline=28,lastline=34,highlightlines=31]{../src/backtrace/backtrace.ml}}
      \onslide*<7->{\mintocaml[firstline=28,lastline=34,highlightlines=33]{../src/backtrace/backtrace.ml}}
      \endminipage
    \end{column}
    \begin{column}{0.45\textwidth}
      \raggedleft
      \onslide*<1>{\providecommand\step{6}\input{diagrams/backtrace/backtrace.tex}}%
      \onslide*<2>{\providecommand\step{6}\input{diagrams/backtrace/backtrace.tex}}%
      \onslide*<3>{\providecommand\step{7}\input{diagrams/backtrace/backtrace.tex}}%
      \onslide*<4>{\providecommand\step{8}\input{diagrams/backtrace/backtrace.tex}}%
      \onslide*<5>{\providecommand\step{9}\input{diagrams/backtrace/backtrace.tex}}%
      \onslide*<6>{\providecommand\step{10}\input{diagrams/backtrace/backtrace.tex}}%
      \onslide*<7>{\providecommand\step{11}\input{diagrams/backtrace/backtrace.tex}}%
      \onslide*<8>{\providecommand\step{12}\input{diagrams/backtrace/backtrace.tex}}%
      \onslide*<9>{\providecommand\step{13}\input{diagrams/backtrace/backtrace.tex}}%
    \end{column}
  \end{columns}
\end{frame}

\begin{frame}{Backtraces}{Printing the backtrace}
  \begin{columns}[c]
    \begin{column}{0.45\textwidth}
      \minipage[c][0.66\textheight][s]{\columnwidth}
      \begin{itemize}
        \item<1-> The exception backtrace can now be printed to \funcarg{stdout} with \funcname{Printexc.print_backtrace}
        \item<2-> First the exception backtrace is retrieved into an OCaml array with \funcname{get_raw_backtrace }
        \item<3-> Which is implemented as the \funcname{caml_get_exception_raw_backtrace} C function in the runtime
        \item<4-> And finally printed with \funcname{print_exception_backtrace}
      \end{itemize}
      \vfill
      \mintocaml[firstline=28,lastline=34,highlightlines=33]{../src/backtrace/backtrace.ml}
      \endminipage
    \end{column}
    \begin{column}{0.55\textwidth}
      \onslide*<1,2>{
        \mintocaml[firstline=100,lastline=101]{../ocaml/stdlib/printexc.ml}
      }
      \only<1>{
        \listocaml[firstline=170,lastline=172]{../ocaml/stdlib/printexc.ml}{stdlib/printexc.ml:170}
      }
      \only<2>{
        \listocaml[firstline=170,lastline=172,highlightlines=172]{../ocaml/stdlib/printexc.ml}{stdlib/printexc.ml:170}
      }
      \only<3>{
        \mintc[firstline=154,lastline=156]{../ocaml/runtime/backtrace.c}
        \listc[firstline=171,lastline=192,highlightlines={184-187}]{../ocaml/runtime/backtrace.c}{runtime/backtrace.c:154}
      }
      \only<4>{
        \listocaml[firstline=155,lastline=165]{../ocaml/stdlib/printexc.ml}{stdlib/printexc.ml:55}
      }
    \end{column}
  \end{columns}
\end{frame}


\subsection{The multiple kinds of raise}
\frameSubsection{}{}

\begin{frame}{Backtraces}{When backtraces are not needed}
  \begin{columns}[c]
    \begin{column}{0.45\textwidth}
      \minipage[c][0.66\textheight][s]{\columnwidth}
      \begin{itemize}
        \item<1-> What if the backtrace for an exception is never needed?
        \item<2-> It's possible to raise the exception explicitly with \funcname{raise\_notrace} to make raising as cheap as possible
        \item<3-> An example of use in the \funcname{dynlink} library of the OCaml compiler
      \end{itemize}
      \vfill
      \onslide<3->{
      \listocaml[firstline=205,lastline=209,highlightlines=207]{../ocaml/otherlibs/dynlink/dynlink\_compilerlibs/misc.ml}{otherlibs/dynlink/dynlink\_compilerlibs/misc.ml:205}
      }
      \endminipage
    \end{column}
    \begin{column}{0.55\textwidth}
    \only<4>{
      \mintcmm[highlightlines=3]{../src/notrace/camlNotrace\_\_fun_347.cmm}
      \listcmm{../src/notrace/camlNotrace\_\_all\_somes\_327.cmm}{all\_somes}
    }
    \onslide*<5->{
      \listobjdump[highlightlines={6-9}]{../src/notrace/camlNotrace\_\_fun_347.objdump}{anonymous function from \funcname{all\_somes}}
    }
    \end{column}
  \end{columns}
\end{frame}

\begin{frame}{Backtraces}{Summary table of the multiple kinds of raise}
  \begin{itemize}
    \item When debugging information is enabled and if the backtrace of an exception isn't needed, it's possible to raise with \funcname{raise\_notrace} for performance
    \item When debugging information is \emph{not} enabled, the compiler optimises \funcname{raise} calls to inline assembly (same effect as using \funcname{raise\_notrace})
  \end{itemize}
  \bigskip
  \centering
  \begin{tabular}{| l | c | c | c | c |}
    \hline
    \parbox{10em}{Debug information \\ (\funcarg{-g} flag)}  & \multicolumn{2}{|c|}{Disabled}                                     & \multicolumn{2}{|c|}{Enabled} \\
    \hline
    Raise kind                                               & \funcname{raise} & \funcname{raise\_notrace}                       & \funcname{raise} & \funcname{raise\_notrace} \\
    \hline
    Compilation                                              & \parbox{8em}{inline assembly\\ (optimization)} & inline assembly   & \funcname{caml\_raise\_exn} & inline assembly \\
    \hline
  \end{tabular}
\end{frame}


%
% Takeaway #5
%
\subsection*{Takeaway \#5}
\frameSubsectionTakeaway{}
\againframe<5>{takeaway}
}%
      \onslide*<3>{\providecommand\step{4}%
% Backtraces
%
\subsection{One advantage of raising with debugging information}
\frameSubsection{}{}

\begin{frame}{Backtraces}{Debugging information \& source code}
  \begin{columns}[c]
    \begin{column}{0.5\textwidth}
      \begin{itemize}
        \item Raising an exception is fun and games, but how do we know where the exception originated? $\rightarrow$ With the backtrace associated with the exception!
        \item In order to get a backtrace from the point where an exception was raised, it's necessary to compile with debugging information enabled (\funcarg{-g} flag for the compiler)
        \item Same example as in the `Nested handlers' section
      \end{itemize}
    \end{column}
    \begin{column}{0.5\textwidth}
      \listocaml[firstline=9,lastline=34]{../src/backtrace/backtrace.ml}{backtrace/backtrace.ml}
    \end{column}
  \end{columns}
\end{frame}

\begin{frame}{Backtraces}{CMM and disassembly of \funcname{definitely\_raise}}
  \begin{columns}[c]
    \begin{column}{0.5\textwidth}
      \minipage[c][0.66\textheight][s]{\columnwidth}
      \begin{itemize}
        \item<1-> An effect of compiling with debugging information (\funcarg{-g}): the CMM no longer uses \funcname{raise\_notrace} to raise the exception but \funcname{raise} instead
        \item<2-> Disassembly shows that CMM \funcname{raise} is actually a call to \funcname{caml\_raise\_exn}, a function in assembler provided by the runtime
      \end{itemize}
      \vfill
      \mintocaml[firstline=9,lastline=13,highlightlines=12]{../src/backtrace/backtrace.ml}
      \endminipage
    \end{column}
    \begin{column}{0.5\textwidth}
      \onslide*<1>{
        \mintcmm[highlightlines=5]{../src/backtrace/camlBacktrace\_\_definitely\_raise\_347.cmm}
      }
      \onslide*<2>{
        \mintobjdump[firstline=2,highlightlines=14]{../src/backtrace/camlBacktrace\_\_definitely\_raise\_347.objdump}
      }
    \end{column}
  \end{columns}
\end{frame}

\begin{frame}{Backtraces}{Introducing \funcname{caml\_raise\_exn}}
  \begin{columns}[c]
    \begin{column}{0.5\textwidth}
      \minipage[c][0.66\textheight][s]{\columnwidth}
      \onslide*<1>{
        \begin{itemize}
          \item Raising the exception is performed using \funcname{caml\_raise\_exn} which is
            \begin{itemize}
              \item An assembly function provided by the runtime
              \item Architecture specific
            \end{itemize}
          \item Let's see what it does differently from \funcname{raise\_notrace}\dots
          \item We are about to raise \typename{ExnC} with \funcname{caml\_raise\_exn} from \funcname{definitely\_raise}
        \end{itemize}
      }
      \onslide*<2>{
        \mintobjdump[firstline=2,highlightlines=14]{../src/backtrace/camlBacktrace\_\_definitely\_raise\_347.objdump}
      }
      \vfill
      \mintocaml[firstline=9,lastline=13,highlightlines=12]{../src/backtrace/backtrace.ml}
      \endminipage
    \end{column}
    \begin{column}{0.5\textwidth}
      \centering
      \providecommand\step{1}\input{diagrams/backtrace/backtrace.tex}
    \end{column}
  \end{columns}
\end{frame}

\begin{frame}{Backtraces}{But before that, we need more fields from \typename{caml\_domain\_state}}
  \begin{columns}
    \begin{column}{0.5\textwidth}
      \begin{itemize}
        \item Remember \typename{caml\_domain\_state} the per-domain runtime state?
        \item Some other members are of interest to us now\dots
        \item \localname{backtrace\_active} is used to know if the runtime should collect backtraces
        \item \localname{backtrace\_active} can be set by
          \begin{itemize}
            \item The \funcarg{b} flag in the \funcname{OCAMLRUNPARAM} environment variable
            \item \mintinline{ocaml}{Printexc.record_backtrace : bool -> unit} \footnotemark
          \end{itemize}
      \end{itemize}
    \end{column}
    \begin{column}{0.5\textwidth}
      \begin{listing}
        \mintc[highlightlines={9-11}]{src/ocaml/domain\_state\_backtrace.h}
        \caption{runtime/caml/domain\_state.h}
      \end{listing}
    \end{column}
  \end{columns}
  \footnotetext[1]{https://v2.ocaml.org/api/Printexc.html}
\end{frame}

\newcommand\listCamlRaiseExn[1][]{
  \listasm[firstline=702,lastline=725,#1]{../ocaml/runtime/amd64.S}{runtime/amd64.S:702}}

\begin{frame}{Backtraces}{Walk through \funcname{caml\_raise\_exn}}
  \begin{columns}[T]
    \begin{column}{0.5\textwidth}
      \begin{itemize}
        \item<1-> First checks whether exceptions backtrace should be recorded
        \item<1-> By checking the \funcarg{domain\_state->backtrace\_active} value
        \item<2-> If not, acts as \funcname{raise\_notrace} with \funcname{RESTORE\_EXN\_HANDLER\_OCAML}
          \begin{itemize}
            \item Set \regname{RSP} to \localname{domain\_state->exn\_handler} value
            \item Restore \localname{domain\_state->exn\_handler} from trap
            \item Jump with \funcname{ret} (same effect as using \funcname{pop} and \funcname{jmp})
          \end{itemize}
        \item<3-> Otherwise, switches to the C stack to be able to execute C code
        \item<4-> Calls \funcname{caml\_stash\_backtrace}, a C function provided by the runtime\ignorespaces
          \onslide<6->{, with
            \begin{itemize}
              \item \funcarg{exn}: exception value
              \item \funcarg{pc}: code address of the raising point
              \item \funcarg{sp}: stack address of the raising function
              \item \funcarg{trapsp}: stack address of the next handler
            \end{itemize}
          }
      \end{itemize}
      \bigskip
      \mintocaml[firstline=9,lastline=13,highlightlines=12]{../src/backtrace/backtrace.ml}
    \end{column}
    \begin{column}{0.5\textwidth}
      \raggedleft
      \onslide*<1,3-4>{
        \mintc[firstline=190,lastline=192]{../ocaml/runtime/amd64.S}
        \mintc[firstline=234,lastline=237]{../ocaml/runtime/amd64.S}
      }
      \onslide*<2>{
        \mintc[firstline=190,lastline=192,highlightlines={191-192}]{../ocaml/runtime/amd64.S}
        \mintc[firstline=234,lastline=237,highlightlines={235-237}]{../ocaml/runtime/amd64.S}
      }
      \onslide*<1>{\listCamlRaiseExn[highlightlines=705]{}}%
      \onslide*<2>{\listCamlRaiseExn[highlightlines={707-708}]{}}%
      \onslide*<3>{\listCamlRaiseExn[highlightlines={712-714}]{}}%
      \onslide*<4>{\listCamlRaiseExn[highlightlines={715-720}]{}}%
      \onslide*<5>{\providecommand\step{1}\input{diagrams/backtrace/backtrace.tex}}%
      \onslide*<6>{\providecommand\step{2}\input{diagrams/backtrace/backtrace.tex}}%
    \end{column}
  \end{columns}
\end{frame}

\newcommand\listStashBacktrace[1][]{
  \listc[firstline=91,lastline=120,#1]{../ocaml/runtime/backtrace\_nat.c}{runtime/backtrace\_nat.c:91}}

\begin{frame}{Backtraces}{Introducing \funcname{caml\_stash\_backtrace}}
  \begin{columns}[c]
    \begin{column}{0.5\textwidth}
      \minipage[c][0.66\textheight][s]{\columnwidth}
      \begin{itemize}
        \item<1-> A C function provided by the runtime (specific to native code)
        \item<2-> It scans the stack, between the stack pointer at the raising point up to the next trap, in order to collect the names of the detected function frames
          \onslide*<2>{
            \begin{itemize}
              \item And more information, like line number, column number...
            \end{itemize}
          }
        \item<3-> It uses the same technique and information as the Garbage Collector (GC) uses to scan the values live on the stack
          \onslide*<4>{
            \begin{itemize}
              \item \typename{frame\_descr} are at the core of stack unwinding
            \end{itemize}
          }
      \end{itemize}
      \vfill
      \mintocaml[firstline=9,lastline=13,highlightlines=12]{../src/backtrace/backtrace.ml}
      \endminipage
    \end{column}
    \begin{column}{0.5\textwidth}
      \onslide*<1>{\listStashBacktrace[highlightlines=91]{}}
      \onslide*<2>{\listStashBacktrace[highlightlines={91,117-118}]{}}
      \onslide*<3->{\listStashBacktrace[highlightlines={94,106,109-110}]{}}
    \end{column}
  \end{columns}
\end{frame}

\newcommand\listFrameDescr[1][]{
  \listocaml[firstline=111,lastline=124,#1]{../ocaml/asmcomp/emitaux.ml}{asmcomp/emitaux.ml:111}}
\newcommand\listDebugInfo[1][]{
  \listocaml[firstline=38,lastline=49,#1]{../ocaml/lambda/debuginfo.mli}{lambda/debuginfo.mli:38}}

\begin{frame}{Backtraces}{\typename{frame\_descr} and \typename{frame\_debuginfo} in a nutshell}
  \begin{columns}[c]
    \begin{column}{0.5\textwidth}
      \begin{itemize}
        \item<1-> For every return address in an OCaml program, there is a associated \typename{frame\_descr}
        \item<2-> A \typename{frame\_descr} specifies
        \begin{itemize}
          \item<2-> The size of the function stack frame
          \item<3-> Where live values are on the stack (so that the GC can mark them as live and prevent from being swept)
        \end{itemize}
        \item<4-> When compiled with debug information, \typename{frame\_descr} will be linked to a \typename{frame\_debuginfo}
        \begin{itemize}
          \item<5-> Contains information about the position in the source code (file name, line and column number)
        \end{itemize}
      \end{itemize}
    \end{column}
    \begin{column}{0.5\textwidth}
        \onslide*<1>{\listFrameDescr[highlightlines={118-122}]{}}
        \onslide*<2>{\listFrameDescr[highlightlines={120}]{}}
        \onslide*<3>{\listFrameDescr[highlightlines={121}]{}}
        \onslide*<4->{\listFrameDescr[highlightlines={122}]{}}
        \medskip
        \onslide*<1-3>{\listDebugInfo{}}
        \onslide*<4>{\listDebugInfo[highlightlines={38,49}]{}}
        \onslide*<5>{\listDebugInfo[highlightlines={39-46}]{}}
    \end{column}
  \end{columns}
\end{frame}

\begin{frame}{Backtraces}{Scanning the stack with \typename{frame\_descr}}
  \begin{columns}[c]
    \begin{column}{0.5\textwidth}
      \begin{itemize}
        \item Given the return address of the code that raises the exception by calling \funcname{caml\_raise\_exn} (\funcarg{pc} argument of \funcname{caml\_stash\_backtrace})
        \item And the position of the stack pointer (\funcarg{sp} argument of \funcname{caml\_stash\_backtrace})
        \item The frame size given by \typename{frame\_descr}.\funcarg{frame\_size}, allows to find the end of the previous frame
        \item And the return address of the caller of this function
        \bigskip
        \onslide<3->{
        \item Note that traps (here the reddish \funcname{ExnA}, \funcname{ExnB} and \funcname{ExnC} blocks) belong to the frame that installed them, and so the frame size changes when traps are removed
        }
      \end{itemize}
    \end{column}
    \begin{column}{0.5\textwidth}
      \raggedleft
      \onslide*<1>{\providecommand\step{2}\input{diagrams/backtrace/backtrace.tex}}%
      \onslide*<2>{\providecommand\step{1}\input{diagrams/backtrace/scanstack.tex}}%
      \onslide*<3>{\providecommand\step{2}\input{diagrams/backtrace/scanstack.tex}}%
      \onslide*<4>{\providecommand\step{3}\input{diagrams/backtrace/scanstack.tex}}%
      \onslide*<5>{\providecommand\step{4}\input{diagrams/backtrace/scanstack.tex}}%
      \onslide*<6>{\providecommand\step{5}\input{diagrams/backtrace/scanstack.tex}}%
    \end{column}
  \end{columns}
\end{frame}

% Duplicate of previous-previous slide
\begin{frame}{Backtraces}{Continuing on \funcname{caml\_stash\_backtrace}}
  \begin{columns}[c]
    \begin{column}{0.5\textwidth}
      \minipage[c][0.75\textheight][s]{\columnwidth}
      \begin{itemize}
          \color{gray}
        \item A C function provided by the runtime (specific to native code)
        \item It scans the stack, between the stack pointer at the raising point up to the next trap, in order to collect the names of the detected function frames
        \item It uses the same technique and information as the Garbage Collector (GC) uses to scan the values live on the stack
      \end{itemize}
      \begin{itemize}
        \item For each function frame detected, store a pointer to the \typename{frame\_descr} in the \localname{domain\_state->backtrace\_buffer} array, effectively constructing the backtrace
      \end{itemize}
      \onslide<2->{
        \begin{itemize}
            \smallskip
          \item Constructed backtrace:
            \funcname{definitely\_raise},
            \onslide<3->{\funcname{probably\_raise}}
        \end{itemize}
      }
      \vfill
      \mintocaml[firstline=9,lastline=13,highlightlines=12]{../src/backtrace/backtrace.ml}
      \endminipage
    \end{column}
    \begin{column}{0.5\textwidth}
      \raggedleft
      \onslide*<1>{\listStashBacktrace[highlightlines={114-115}]{}}
      \onslide*<2>{\providecommand\step{3}\input{diagrams/backtrace/backtrace.tex}}%
      \onslide*<3>{\providecommand\step{4}\input{diagrams/backtrace/backtrace.tex}}%
    \end{column}
  \end{columns}
\end{frame}

\begin{frame}{Backtraces}{Back to \funcname{caml\_raise\_exn}}
  \begin{columns}[c]
    \begin{column}{0.5\textwidth}
      \minipage[c][0.66\textheight][s]{\columnwidth}
      \begin{itemize}\color{gray}
        \item Checks wheteher exceptions backtrace should be recorded, by checking the \funcarg{domain\_state->backtrace\_active} value
        \item Switches to the C stack to be able to execute C code
        \item Calls \funcname{caml\_stash\_backtrace}, to collect the backtrace up to the next trap
      \end{itemize}
      \onslide*<2->{
        \begin{itemize}
          \item<2-> Executes the exception handler in \funcname{probably\_raise} with \funcname{RESTORE\_EXN\_HANDLER\_OCAML}
            \begin{itemize}
              \item Set \regname{RSP} to \localname{domain\_state->exn\_handler} value
              \item Restore \localname{domain\_state->exn\_handler} from trap
              \item Jump to the handler code with \funcname{ret}
            \end{itemize}
        \end{itemize}
      }
      \vfill
      \onslide*<1>{\mintocaml[firstline=9,lastline=13,highlightlines=12]{../src/backtrace/backtrace.ml}}
      \onslide*<2->{\mintocaml[firstline=15,lastline=21,highlightlines={19-21}]{../src/backtrace/backtrace.ml}}
      \endminipage
    \end{column}
    \begin{column}{0.5\textwidth}
      \centering
      \onslide*<1>{
        \mintc[firstline=190,lastline=192]{../ocaml/runtime/amd64.S}
        \mintc[firstline=234,lastline=237]{../ocaml/runtime/amd64.S}
      }
      \onslide*<2>{
        \mintc[firstline=190,lastline=192,highlightlines={191-192}]{../ocaml/runtime/amd64.S}
        \mintc[firstline=234,lastline=237,highlightlines={235-237}]{../ocaml/runtime/amd64.S}
      }
      \onslide*<1>{\listCamlRaiseExn[highlightlines=720]{}}
      \onslide*<2>{\listCamlRaiseExn[highlightlines=722]{}}
      \onslide*<3->{\providecommand\step{5}\input{diagrams/backtrace/backtrace.tex}}
    \end{column}
  \end{columns}
\end{frame}

\begin{frame}{Backtraces}{\funcname{caml\_probably\_raise}'s handler doesn't catch \typename{ExnC}}
  \begin{columns}[c]
    \begin{column}{0.45\textwidth}
      \minipage[c][0.75\textheight][s]{\columnwidth}
      \begin{itemize}
        \item<1-> \funcname{match \dots with}\footnotemark pattern matching can also match exceptions
        \item<1-> The CMM of \funcname{probably\_raise} shows that this \funcname{match \dots with} is implemented with a pattern matching and a \funcname{try \dots with}
        \item<1-> But this one only matches on \typename{ExnA} exception
        \item<2-> Since the pattern matching fails to catch the exception, \funcname{reraise} is called with our current \typename{ExnC} exception
        \item<3-> Disassembly shows that \funcname{reraise} is actually a call to \funcname{caml\_reraise\_exn}, which is also an assembly function provided by the runtime
      \end{itemize}
      \vfill
      \mintocaml[firstline=15,lastline=21,highlightlines={18-21}]{../src/backtrace/backtrace.ml}
      \endminipage
    \end{column}
    \begin{column}{0.55\textwidth}
      \centering
      \onslide*<1>{\mintcmm[highlightlines={10-18}]{../src/backtrace/camlBacktrace\_\_probably\_raise\_351.cmm}}
      \onslide*<2>{\listcmm[highlightlines=18]{../src/backtrace/camlBacktrace\_\_probably\_raise_351.cmm}{CMM of probably\_raise}}
      \onslide*<3>{\mintobjdump[firstline=5,lastline=35,highlightlines=31]{../src/backtrace/camlBacktrace\_\_probably\_raise_351.objdump}}
    \end{column}
  \end{columns}
  \only<1>{\footnotetext[1]{https://blog.janestreet.com/pattern-matching-and-exception-handling-unite/}}
\end{frame}

\newcommand\listCamlReraiseExn[1][]{
  \listasm[firstline=727,lastline=734,#1]{../ocaml/runtime/amd64.S}{runtime/amd64.S:727}}

\begin{frame}{Backtraces}{Introducing \funcname{caml\_reraise\_exn}}
  \begin{columns}[c]
    \begin{column}{0.5\textwidth}
      \minipage[c][0.66\textheight][s]{\columnwidth}
      \begin{itemize}
        \item<1-> The exception have to be reraised to the next handler
        \item<2-> First, checks if the backtrace should be collected with \funcarg{domain\_state->backtrace\_active}
        \only<3>{
        \begin{itemize}
            \item If not, behaves like a \funcname{raise\_notrace} with \funcname{RESTORE\_EXN\_HANDLER\_OCAML}
        \end{itemize}
        }
        \item<4-> Jumps into \funcname{caml\_raise\_exn}
        \item<5-> Calls \funcname{caml\_stash\_backtrace} again, to scan the next part of the stack: from the trap just pop-ed to the next trap
      \end{itemize}
      \vfill
      \mintocaml[firstline=15,lastline=21,highlightlines={18-21}]{../src/backtrace/backtrace.ml}
      \endminipage
    \end{column}
    \begin{column}{0.5\textwidth}
      \raggedleft
      \onslide*<1>{\listCamlReraiseExn{}}
      \onslide*<2>{\listCamlReraiseExn[highlightlines=729]{}}
      \onslide*<3>{
        \mintc[firstline=190,lastline=192,highlightlines={191-192}]{../ocaml/runtime/amd64.S}
        \mintc[firstline=234,lastline=237,highlightlines={235-237}]{../ocaml/runtime/amd64.S}
        \medskip
        \listCamlReraiseExn[highlightlines=731]{}
      }
      \onslide*<4-5>{
        % TODO refactor with \listCamlReraiseExn
        \mintasm[firstline=727,lastline=734,highlightlines=730]{../ocaml/runtime/amd64.S}
        \medskip
      }
      \onslide*<4>{\listCamlRaiseExn[highlightlines=711]{}}
      \onslide*<5>{\listCamlRaiseExn[highlightlines=720]{}}
    \end{column}
  \end{columns}
\end{frame}

\begin{frame}{Backtraces}{Backtrace collection continues}
  \begin{columns}[c]
    \begin{column}{0.55\textwidth}
      \minipage[c][0.75\textheight][s]{\columnwidth}
      \begin{itemize}
        \item<1-> \funcname{caml\_stash\_backtrace} scans the older part of the stack, up to the next trap
        \item<6-> Controls jumps to the nested exception handler in \funcname{maybe\_raise}, which matches only on \typename{ExnB}
        \item<7-> \funcname{caml\_reraise\_exn} is called again, backtrace collection resumes with \funcname{caml\_stash\_backtrace}
      \end{itemize}
      \medskip
      \begin{itemize}
        \item<2-> Constructed backtrace:
        \begin{itemize}
          \item <2-> \funcname{definitely\_raise}
          \item <2-> \funcname{probably\_raise}
          \item <3-> \funcname{probably\_raise} (re-raise)
          \item <4-> \funcname{maybe\_raise}
          \item <5-> \funcname{main}
          \item <8-> \funcname{main} (re-raise)
        \end{itemize}
      \end{itemize}
      \vfill
      \onslide*<1-5>{\mintocaml[firstline=15,lastline=21]{../src/backtrace/backtrace.ml}}
      \onslide*<6>{\mintocaml[firstline=28,lastline=34,highlightlines=31]{../src/backtrace/backtrace.ml}}
      \onslide*<7->{\mintocaml[firstline=28,lastline=34,highlightlines=33]{../src/backtrace/backtrace.ml}}
      \endminipage
    \end{column}
    \begin{column}{0.45\textwidth}
      \raggedleft
      \onslide*<1>{\providecommand\step{6}\input{diagrams/backtrace/backtrace.tex}}%
      \onslide*<2>{\providecommand\step{6}\input{diagrams/backtrace/backtrace.tex}}%
      \onslide*<3>{\providecommand\step{7}\input{diagrams/backtrace/backtrace.tex}}%
      \onslide*<4>{\providecommand\step{8}\input{diagrams/backtrace/backtrace.tex}}%
      \onslide*<5>{\providecommand\step{9}\input{diagrams/backtrace/backtrace.tex}}%
      \onslide*<6>{\providecommand\step{10}\input{diagrams/backtrace/backtrace.tex}}%
      \onslide*<7>{\providecommand\step{11}\input{diagrams/backtrace/backtrace.tex}}%
      \onslide*<8>{\providecommand\step{12}\input{diagrams/backtrace/backtrace.tex}}%
      \onslide*<9>{\providecommand\step{13}\input{diagrams/backtrace/backtrace.tex}}%
    \end{column}
  \end{columns}
\end{frame}

\begin{frame}{Backtraces}{Printing the backtrace}
  \begin{columns}[c]
    \begin{column}{0.45\textwidth}
      \minipage[c][0.66\textheight][s]{\columnwidth}
      \begin{itemize}
        \item<1-> The exception backtrace can now be printed to \funcarg{stdout} with \funcname{Printexc.print_backtrace}
        \item<2-> First the exception backtrace is retrieved into an OCaml array with \funcname{get_raw_backtrace }
        \item<3-> Which is implemented as the \funcname{caml_get_exception_raw_backtrace} C function in the runtime
        \item<4-> And finally printed with \funcname{print_exception_backtrace}
      \end{itemize}
      \vfill
      \mintocaml[firstline=28,lastline=34,highlightlines=33]{../src/backtrace/backtrace.ml}
      \endminipage
    \end{column}
    \begin{column}{0.55\textwidth}
      \onslide*<1,2>{
        \mintocaml[firstline=100,lastline=101]{../ocaml/stdlib/printexc.ml}
      }
      \only<1>{
        \listocaml[firstline=170,lastline=172]{../ocaml/stdlib/printexc.ml}{stdlib/printexc.ml:170}
      }
      \only<2>{
        \listocaml[firstline=170,lastline=172,highlightlines=172]{../ocaml/stdlib/printexc.ml}{stdlib/printexc.ml:170}
      }
      \only<3>{
        \mintc[firstline=154,lastline=156]{../ocaml/runtime/backtrace.c}
        \listc[firstline=171,lastline=192,highlightlines={184-187}]{../ocaml/runtime/backtrace.c}{runtime/backtrace.c:154}
      }
      \only<4>{
        \listocaml[firstline=155,lastline=165]{../ocaml/stdlib/printexc.ml}{stdlib/printexc.ml:55}
      }
    \end{column}
  \end{columns}
\end{frame}


\subsection{The multiple kinds of raise}
\frameSubsection{}{}

\begin{frame}{Backtraces}{When backtraces are not needed}
  \begin{columns}[c]
    \begin{column}{0.45\textwidth}
      \minipage[c][0.66\textheight][s]{\columnwidth}
      \begin{itemize}
        \item<1-> What if the backtrace for an exception is never needed?
        \item<2-> It's possible to raise the exception explicitly with \funcname{raise\_notrace} to make raising as cheap as possible
        \item<3-> An example of use in the \funcname{dynlink} library of the OCaml compiler
      \end{itemize}
      \vfill
      \onslide<3->{
      \listocaml[firstline=205,lastline=209,highlightlines=207]{../ocaml/otherlibs/dynlink/dynlink\_compilerlibs/misc.ml}{otherlibs/dynlink/dynlink\_compilerlibs/misc.ml:205}
      }
      \endminipage
    \end{column}
    \begin{column}{0.55\textwidth}
    \only<4>{
      \mintcmm[highlightlines=3]{../src/notrace/camlNotrace\_\_fun_347.cmm}
      \listcmm{../src/notrace/camlNotrace\_\_all\_somes\_327.cmm}{all\_somes}
    }
    \onslide*<5->{
      \listobjdump[highlightlines={6-9}]{../src/notrace/camlNotrace\_\_fun_347.objdump}{anonymous function from \funcname{all\_somes}}
    }
    \end{column}
  \end{columns}
\end{frame}

\begin{frame}{Backtraces}{Summary table of the multiple kinds of raise}
  \begin{itemize}
    \item When debugging information is enabled and if the backtrace of an exception isn't needed, it's possible to raise with \funcname{raise\_notrace} for performance
    \item When debugging information is \emph{not} enabled, the compiler optimises \funcname{raise} calls to inline assembly (same effect as using \funcname{raise\_notrace})
  \end{itemize}
  \bigskip
  \centering
  \begin{tabular}{| l | c | c | c | c |}
    \hline
    \parbox{10em}{Debug information \\ (\funcarg{-g} flag)}  & \multicolumn{2}{|c|}{Disabled}                                     & \multicolumn{2}{|c|}{Enabled} \\
    \hline
    Raise kind                                               & \funcname{raise} & \funcname{raise\_notrace}                       & \funcname{raise} & \funcname{raise\_notrace} \\
    \hline
    Compilation                                              & \parbox{8em}{inline assembly\\ (optimization)} & inline assembly   & \funcname{caml\_raise\_exn} & inline assembly \\
    \hline
  \end{tabular}
\end{frame}


%
% Takeaway #5
%
\subsection*{Takeaway \#5}
\frameSubsectionTakeaway{}
\againframe<5>{takeaway}
}%
    \end{column}
  \end{columns}
\end{frame}

\begin{frame}{Backtraces}{Back to \funcname{caml\_raise\_exn}}
  \begin{columns}[c]
    \begin{column}{0.5\textwidth}
      \minipage[c][0.66\textheight][s]{\columnwidth}
      \begin{itemize}\color{gray}
        \item Checks wheteher exceptions backtrace should be recorded, by checking the \funcarg{domain\_state->backtrace\_active} value
        \item Switches to the C stack to be able to execute C code
        \item Calls \funcname{caml\_stash\_backtrace}, to collect the backtrace up to the next trap
      \end{itemize}
      \onslide*<2->{
        \begin{itemize}
          \item<2-> Executes the exception handler in \funcname{probably\_raise} with \funcname{RESTORE\_EXN\_HANDLER\_OCAML}
            \begin{itemize}
              \item Set \regname{RSP} to \localname{domain\_state->exn\_handler} value
              \item Restore \localname{domain\_state->exn\_handler} from trap
              \item Jump to the handler code with \funcname{ret}
            \end{itemize}
        \end{itemize}
      }
      \vfill
      \onslide*<1>{\mintocaml[firstline=9,lastline=13,highlightlines=12]{../src/backtrace/backtrace.ml}}
      \onslide*<2->{\mintocaml[firstline=15,lastline=21,highlightlines={19-21}]{../src/backtrace/backtrace.ml}}
      \endminipage
    \end{column}
    \begin{column}{0.5\textwidth}
      \centering
      \onslide*<1>{
        \mintc[firstline=190,lastline=192]{../ocaml/runtime/amd64.S}
        \mintc[firstline=234,lastline=237]{../ocaml/runtime/amd64.S}
      }
      \onslide*<2>{
        \mintc[firstline=190,lastline=192,highlightlines={191-192}]{../ocaml/runtime/amd64.S}
        \mintc[firstline=234,lastline=237,highlightlines={235-237}]{../ocaml/runtime/amd64.S}
      }
      \onslide*<1>{\listCamlRaiseExn[highlightlines=720]{}}
      \onslide*<2>{\listCamlRaiseExn[highlightlines=722]{}}
      \onslide*<3->{\providecommand\step{5}%
% Backtraces
%
\subsection{One advantage of raising with debugging information}
\frameSubsection{}{}

\begin{frame}{Backtraces}{Debugging information \& source code}
  \begin{columns}[c]
    \begin{column}{0.5\textwidth}
      \begin{itemize}
        \item Raising an exception is fun and games, but how do we know where the exception originated? $\rightarrow$ With the backtrace associated with the exception!
        \item In order to get a backtrace from the point where an exception was raised, it's necessary to compile with debugging information enabled (\funcarg{-g} flag for the compiler)
        \item Same example as in the `Nested handlers' section
      \end{itemize}
    \end{column}
    \begin{column}{0.5\textwidth}
      \listocaml[firstline=9,lastline=34]{../src/backtrace/backtrace.ml}{backtrace/backtrace.ml}
    \end{column}
  \end{columns}
\end{frame}

\begin{frame}{Backtraces}{CMM and disassembly of \funcname{definitely\_raise}}
  \begin{columns}[c]
    \begin{column}{0.5\textwidth}
      \minipage[c][0.66\textheight][s]{\columnwidth}
      \begin{itemize}
        \item<1-> An effect of compiling with debugging information (\funcarg{-g}): the CMM no longer uses \funcname{raise\_notrace} to raise the exception but \funcname{raise} instead
        \item<2-> Disassembly shows that CMM \funcname{raise} is actually a call to \funcname{caml\_raise\_exn}, a function in assembler provided by the runtime
      \end{itemize}
      \vfill
      \mintocaml[firstline=9,lastline=13,highlightlines=12]{../src/backtrace/backtrace.ml}
      \endminipage
    \end{column}
    \begin{column}{0.5\textwidth}
      \onslide*<1>{
        \mintcmm[highlightlines=5]{../src/backtrace/camlBacktrace\_\_definitely\_raise\_347.cmm}
      }
      \onslide*<2>{
        \mintobjdump[firstline=2,highlightlines=14]{../src/backtrace/camlBacktrace\_\_definitely\_raise\_347.objdump}
      }
    \end{column}
  \end{columns}
\end{frame}

\begin{frame}{Backtraces}{Introducing \funcname{caml\_raise\_exn}}
  \begin{columns}[c]
    \begin{column}{0.5\textwidth}
      \minipage[c][0.66\textheight][s]{\columnwidth}
      \onslide*<1>{
        \begin{itemize}
          \item Raising the exception is performed using \funcname{caml\_raise\_exn} which is
            \begin{itemize}
              \item An assembly function provided by the runtime
              \item Architecture specific
            \end{itemize}
          \item Let's see what it does differently from \funcname{raise\_notrace}\dots
          \item We are about to raise \typename{ExnC} with \funcname{caml\_raise\_exn} from \funcname{definitely\_raise}
        \end{itemize}
      }
      \onslide*<2>{
        \mintobjdump[firstline=2,highlightlines=14]{../src/backtrace/camlBacktrace\_\_definitely\_raise\_347.objdump}
      }
      \vfill
      \mintocaml[firstline=9,lastline=13,highlightlines=12]{../src/backtrace/backtrace.ml}
      \endminipage
    \end{column}
    \begin{column}{0.5\textwidth}
      \centering
      \providecommand\step{1}\input{diagrams/backtrace/backtrace.tex}
    \end{column}
  \end{columns}
\end{frame}

\begin{frame}{Backtraces}{But before that, we need more fields from \typename{caml\_domain\_state}}
  \begin{columns}
    \begin{column}{0.5\textwidth}
      \begin{itemize}
        \item Remember \typename{caml\_domain\_state} the per-domain runtime state?
        \item Some other members are of interest to us now\dots
        \item \localname{backtrace\_active} is used to know if the runtime should collect backtraces
        \item \localname{backtrace\_active} can be set by
          \begin{itemize}
            \item The \funcarg{b} flag in the \funcname{OCAMLRUNPARAM} environment variable
            \item \mintinline{ocaml}{Printexc.record_backtrace : bool -> unit} \footnotemark
          \end{itemize}
      \end{itemize}
    \end{column}
    \begin{column}{0.5\textwidth}
      \begin{listing}
        \mintc[highlightlines={9-11}]{src/ocaml/domain\_state\_backtrace.h}
        \caption{runtime/caml/domain\_state.h}
      \end{listing}
    \end{column}
  \end{columns}
  \footnotetext[1]{https://v2.ocaml.org/api/Printexc.html}
\end{frame}

\newcommand\listCamlRaiseExn[1][]{
  \listasm[firstline=702,lastline=725,#1]{../ocaml/runtime/amd64.S}{runtime/amd64.S:702}}

\begin{frame}{Backtraces}{Walk through \funcname{caml\_raise\_exn}}
  \begin{columns}[T]
    \begin{column}{0.5\textwidth}
      \begin{itemize}
        \item<1-> First checks whether exceptions backtrace should be recorded
        \item<1-> By checking the \funcarg{domain\_state->backtrace\_active} value
        \item<2-> If not, acts as \funcname{raise\_notrace} with \funcname{RESTORE\_EXN\_HANDLER\_OCAML}
          \begin{itemize}
            \item Set \regname{RSP} to \localname{domain\_state->exn\_handler} value
            \item Restore \localname{domain\_state->exn\_handler} from trap
            \item Jump with \funcname{ret} (same effect as using \funcname{pop} and \funcname{jmp})
          \end{itemize}
        \item<3-> Otherwise, switches to the C stack to be able to execute C code
        \item<4-> Calls \funcname{caml\_stash\_backtrace}, a C function provided by the runtime\ignorespaces
          \onslide<6->{, with
            \begin{itemize}
              \item \funcarg{exn}: exception value
              \item \funcarg{pc}: code address of the raising point
              \item \funcarg{sp}: stack address of the raising function
              \item \funcarg{trapsp}: stack address of the next handler
            \end{itemize}
          }
      \end{itemize}
      \bigskip
      \mintocaml[firstline=9,lastline=13,highlightlines=12]{../src/backtrace/backtrace.ml}
    \end{column}
    \begin{column}{0.5\textwidth}
      \raggedleft
      \onslide*<1,3-4>{
        \mintc[firstline=190,lastline=192]{../ocaml/runtime/amd64.S}
        \mintc[firstline=234,lastline=237]{../ocaml/runtime/amd64.S}
      }
      \onslide*<2>{
        \mintc[firstline=190,lastline=192,highlightlines={191-192}]{../ocaml/runtime/amd64.S}
        \mintc[firstline=234,lastline=237,highlightlines={235-237}]{../ocaml/runtime/amd64.S}
      }
      \onslide*<1>{\listCamlRaiseExn[highlightlines=705]{}}%
      \onslide*<2>{\listCamlRaiseExn[highlightlines={707-708}]{}}%
      \onslide*<3>{\listCamlRaiseExn[highlightlines={712-714}]{}}%
      \onslide*<4>{\listCamlRaiseExn[highlightlines={715-720}]{}}%
      \onslide*<5>{\providecommand\step{1}\input{diagrams/backtrace/backtrace.tex}}%
      \onslide*<6>{\providecommand\step{2}\input{diagrams/backtrace/backtrace.tex}}%
    \end{column}
  \end{columns}
\end{frame}

\newcommand\listStashBacktrace[1][]{
  \listc[firstline=91,lastline=120,#1]{../ocaml/runtime/backtrace\_nat.c}{runtime/backtrace\_nat.c:91}}

\begin{frame}{Backtraces}{Introducing \funcname{caml\_stash\_backtrace}}
  \begin{columns}[c]
    \begin{column}{0.5\textwidth}
      \minipage[c][0.66\textheight][s]{\columnwidth}
      \begin{itemize}
        \item<1-> A C function provided by the runtime (specific to native code)
        \item<2-> It scans the stack, between the stack pointer at the raising point up to the next trap, in order to collect the names of the detected function frames
          \onslide*<2>{
            \begin{itemize}
              \item And more information, like line number, column number...
            \end{itemize}
          }
        \item<3-> It uses the same technique and information as the Garbage Collector (GC) uses to scan the values live on the stack
          \onslide*<4>{
            \begin{itemize}
              \item \typename{frame\_descr} are at the core of stack unwinding
            \end{itemize}
          }
      \end{itemize}
      \vfill
      \mintocaml[firstline=9,lastline=13,highlightlines=12]{../src/backtrace/backtrace.ml}
      \endminipage
    \end{column}
    \begin{column}{0.5\textwidth}
      \onslide*<1>{\listStashBacktrace[highlightlines=91]{}}
      \onslide*<2>{\listStashBacktrace[highlightlines={91,117-118}]{}}
      \onslide*<3->{\listStashBacktrace[highlightlines={94,106,109-110}]{}}
    \end{column}
  \end{columns}
\end{frame}

\newcommand\listFrameDescr[1][]{
  \listocaml[firstline=111,lastline=124,#1]{../ocaml/asmcomp/emitaux.ml}{asmcomp/emitaux.ml:111}}
\newcommand\listDebugInfo[1][]{
  \listocaml[firstline=38,lastline=49,#1]{../ocaml/lambda/debuginfo.mli}{lambda/debuginfo.mli:38}}

\begin{frame}{Backtraces}{\typename{frame\_descr} and \typename{frame\_debuginfo} in a nutshell}
  \begin{columns}[c]
    \begin{column}{0.5\textwidth}
      \begin{itemize}
        \item<1-> For every return address in an OCaml program, there is a associated \typename{frame\_descr}
        \item<2-> A \typename{frame\_descr} specifies
        \begin{itemize}
          \item<2-> The size of the function stack frame
          \item<3-> Where live values are on the stack (so that the GC can mark them as live and prevent from being swept)
        \end{itemize}
        \item<4-> When compiled with debug information, \typename{frame\_descr} will be linked to a \typename{frame\_debuginfo}
        \begin{itemize}
          \item<5-> Contains information about the position in the source code (file name, line and column number)
        \end{itemize}
      \end{itemize}
    \end{column}
    \begin{column}{0.5\textwidth}
        \onslide*<1>{\listFrameDescr[highlightlines={118-122}]{}}
        \onslide*<2>{\listFrameDescr[highlightlines={120}]{}}
        \onslide*<3>{\listFrameDescr[highlightlines={121}]{}}
        \onslide*<4->{\listFrameDescr[highlightlines={122}]{}}
        \medskip
        \onslide*<1-3>{\listDebugInfo{}}
        \onslide*<4>{\listDebugInfo[highlightlines={38,49}]{}}
        \onslide*<5>{\listDebugInfo[highlightlines={39-46}]{}}
    \end{column}
  \end{columns}
\end{frame}

\begin{frame}{Backtraces}{Scanning the stack with \typename{frame\_descr}}
  \begin{columns}[c]
    \begin{column}{0.5\textwidth}
      \begin{itemize}
        \item Given the return address of the code that raises the exception by calling \funcname{caml\_raise\_exn} (\funcarg{pc} argument of \funcname{caml\_stash\_backtrace})
        \item And the position of the stack pointer (\funcarg{sp} argument of \funcname{caml\_stash\_backtrace})
        \item The frame size given by \typename{frame\_descr}.\funcarg{frame\_size}, allows to find the end of the previous frame
        \item And the return address of the caller of this function
        \bigskip
        \onslide<3->{
        \item Note that traps (here the reddish \funcname{ExnA}, \funcname{ExnB} and \funcname{ExnC} blocks) belong to the frame that installed them, and so the frame size changes when traps are removed
        }
      \end{itemize}
    \end{column}
    \begin{column}{0.5\textwidth}
      \raggedleft
      \onslide*<1>{\providecommand\step{2}\input{diagrams/backtrace/backtrace.tex}}%
      \onslide*<2>{\providecommand\step{1}\input{diagrams/backtrace/scanstack.tex}}%
      \onslide*<3>{\providecommand\step{2}\input{diagrams/backtrace/scanstack.tex}}%
      \onslide*<4>{\providecommand\step{3}\input{diagrams/backtrace/scanstack.tex}}%
      \onslide*<5>{\providecommand\step{4}\input{diagrams/backtrace/scanstack.tex}}%
      \onslide*<6>{\providecommand\step{5}\input{diagrams/backtrace/scanstack.tex}}%
    \end{column}
  \end{columns}
\end{frame}

% Duplicate of previous-previous slide
\begin{frame}{Backtraces}{Continuing on \funcname{caml\_stash\_backtrace}}
  \begin{columns}[c]
    \begin{column}{0.5\textwidth}
      \minipage[c][0.75\textheight][s]{\columnwidth}
      \begin{itemize}
          \color{gray}
        \item A C function provided by the runtime (specific to native code)
        \item It scans the stack, between the stack pointer at the raising point up to the next trap, in order to collect the names of the detected function frames
        \item It uses the same technique and information as the Garbage Collector (GC) uses to scan the values live on the stack
      \end{itemize}
      \begin{itemize}
        \item For each function frame detected, store a pointer to the \typename{frame\_descr} in the \localname{domain\_state->backtrace\_buffer} array, effectively constructing the backtrace
      \end{itemize}
      \onslide<2->{
        \begin{itemize}
            \smallskip
          \item Constructed backtrace:
            \funcname{definitely\_raise},
            \onslide<3->{\funcname{probably\_raise}}
        \end{itemize}
      }
      \vfill
      \mintocaml[firstline=9,lastline=13,highlightlines=12]{../src/backtrace/backtrace.ml}
      \endminipage
    \end{column}
    \begin{column}{0.5\textwidth}
      \raggedleft
      \onslide*<1>{\listStashBacktrace[highlightlines={114-115}]{}}
      \onslide*<2>{\providecommand\step{3}\input{diagrams/backtrace/backtrace.tex}}%
      \onslide*<3>{\providecommand\step{4}\input{diagrams/backtrace/backtrace.tex}}%
    \end{column}
  \end{columns}
\end{frame}

\begin{frame}{Backtraces}{Back to \funcname{caml\_raise\_exn}}
  \begin{columns}[c]
    \begin{column}{0.5\textwidth}
      \minipage[c][0.66\textheight][s]{\columnwidth}
      \begin{itemize}\color{gray}
        \item Checks wheteher exceptions backtrace should be recorded, by checking the \funcarg{domain\_state->backtrace\_active} value
        \item Switches to the C stack to be able to execute C code
        \item Calls \funcname{caml\_stash\_backtrace}, to collect the backtrace up to the next trap
      \end{itemize}
      \onslide*<2->{
        \begin{itemize}
          \item<2-> Executes the exception handler in \funcname{probably\_raise} with \funcname{RESTORE\_EXN\_HANDLER\_OCAML}
            \begin{itemize}
              \item Set \regname{RSP} to \localname{domain\_state->exn\_handler} value
              \item Restore \localname{domain\_state->exn\_handler} from trap
              \item Jump to the handler code with \funcname{ret}
            \end{itemize}
        \end{itemize}
      }
      \vfill
      \onslide*<1>{\mintocaml[firstline=9,lastline=13,highlightlines=12]{../src/backtrace/backtrace.ml}}
      \onslide*<2->{\mintocaml[firstline=15,lastline=21,highlightlines={19-21}]{../src/backtrace/backtrace.ml}}
      \endminipage
    \end{column}
    \begin{column}{0.5\textwidth}
      \centering
      \onslide*<1>{
        \mintc[firstline=190,lastline=192]{../ocaml/runtime/amd64.S}
        \mintc[firstline=234,lastline=237]{../ocaml/runtime/amd64.S}
      }
      \onslide*<2>{
        \mintc[firstline=190,lastline=192,highlightlines={191-192}]{../ocaml/runtime/amd64.S}
        \mintc[firstline=234,lastline=237,highlightlines={235-237}]{../ocaml/runtime/amd64.S}
      }
      \onslide*<1>{\listCamlRaiseExn[highlightlines=720]{}}
      \onslide*<2>{\listCamlRaiseExn[highlightlines=722]{}}
      \onslide*<3->{\providecommand\step{5}\input{diagrams/backtrace/backtrace.tex}}
    \end{column}
  \end{columns}
\end{frame}

\begin{frame}{Backtraces}{\funcname{caml\_probably\_raise}'s handler doesn't catch \typename{ExnC}}
  \begin{columns}[c]
    \begin{column}{0.45\textwidth}
      \minipage[c][0.75\textheight][s]{\columnwidth}
      \begin{itemize}
        \item<1-> \funcname{match \dots with}\footnotemark pattern matching can also match exceptions
        \item<1-> The CMM of \funcname{probably\_raise} shows that this \funcname{match \dots with} is implemented with a pattern matching and a \funcname{try \dots with}
        \item<1-> But this one only matches on \typename{ExnA} exception
        \item<2-> Since the pattern matching fails to catch the exception, \funcname{reraise} is called with our current \typename{ExnC} exception
        \item<3-> Disassembly shows that \funcname{reraise} is actually a call to \funcname{caml\_reraise\_exn}, which is also an assembly function provided by the runtime
      \end{itemize}
      \vfill
      \mintocaml[firstline=15,lastline=21,highlightlines={18-21}]{../src/backtrace/backtrace.ml}
      \endminipage
    \end{column}
    \begin{column}{0.55\textwidth}
      \centering
      \onslide*<1>{\mintcmm[highlightlines={10-18}]{../src/backtrace/camlBacktrace\_\_probably\_raise\_351.cmm}}
      \onslide*<2>{\listcmm[highlightlines=18]{../src/backtrace/camlBacktrace\_\_probably\_raise_351.cmm}{CMM of probably\_raise}}
      \onslide*<3>{\mintobjdump[firstline=5,lastline=35,highlightlines=31]{../src/backtrace/camlBacktrace\_\_probably\_raise_351.objdump}}
    \end{column}
  \end{columns}
  \only<1>{\footnotetext[1]{https://blog.janestreet.com/pattern-matching-and-exception-handling-unite/}}
\end{frame}

\newcommand\listCamlReraiseExn[1][]{
  \listasm[firstline=727,lastline=734,#1]{../ocaml/runtime/amd64.S}{runtime/amd64.S:727}}

\begin{frame}{Backtraces}{Introducing \funcname{caml\_reraise\_exn}}
  \begin{columns}[c]
    \begin{column}{0.5\textwidth}
      \minipage[c][0.66\textheight][s]{\columnwidth}
      \begin{itemize}
        \item<1-> The exception have to be reraised to the next handler
        \item<2-> First, checks if the backtrace should be collected with \funcarg{domain\_state->backtrace\_active}
        \only<3>{
        \begin{itemize}
            \item If not, behaves like a \funcname{raise\_notrace} with \funcname{RESTORE\_EXN\_HANDLER\_OCAML}
        \end{itemize}
        }
        \item<4-> Jumps into \funcname{caml\_raise\_exn}
        \item<5-> Calls \funcname{caml\_stash\_backtrace} again, to scan the next part of the stack: from the trap just pop-ed to the next trap
      \end{itemize}
      \vfill
      \mintocaml[firstline=15,lastline=21,highlightlines={18-21}]{../src/backtrace/backtrace.ml}
      \endminipage
    \end{column}
    \begin{column}{0.5\textwidth}
      \raggedleft
      \onslide*<1>{\listCamlReraiseExn{}}
      \onslide*<2>{\listCamlReraiseExn[highlightlines=729]{}}
      \onslide*<3>{
        \mintc[firstline=190,lastline=192,highlightlines={191-192}]{../ocaml/runtime/amd64.S}
        \mintc[firstline=234,lastline=237,highlightlines={235-237}]{../ocaml/runtime/amd64.S}
        \medskip
        \listCamlReraiseExn[highlightlines=731]{}
      }
      \onslide*<4-5>{
        % TODO refactor with \listCamlReraiseExn
        \mintasm[firstline=727,lastline=734,highlightlines=730]{../ocaml/runtime/amd64.S}
        \medskip
      }
      \onslide*<4>{\listCamlRaiseExn[highlightlines=711]{}}
      \onslide*<5>{\listCamlRaiseExn[highlightlines=720]{}}
    \end{column}
  \end{columns}
\end{frame}

\begin{frame}{Backtraces}{Backtrace collection continues}
  \begin{columns}[c]
    \begin{column}{0.55\textwidth}
      \minipage[c][0.75\textheight][s]{\columnwidth}
      \begin{itemize}
        \item<1-> \funcname{caml\_stash\_backtrace} scans the older part of the stack, up to the next trap
        \item<6-> Controls jumps to the nested exception handler in \funcname{maybe\_raise}, which matches only on \typename{ExnB}
        \item<7-> \funcname{caml\_reraise\_exn} is called again, backtrace collection resumes with \funcname{caml\_stash\_backtrace}
      \end{itemize}
      \medskip
      \begin{itemize}
        \item<2-> Constructed backtrace:
        \begin{itemize}
          \item <2-> \funcname{definitely\_raise}
          \item <2-> \funcname{probably\_raise}
          \item <3-> \funcname{probably\_raise} (re-raise)
          \item <4-> \funcname{maybe\_raise}
          \item <5-> \funcname{main}
          \item <8-> \funcname{main} (re-raise)
        \end{itemize}
      \end{itemize}
      \vfill
      \onslide*<1-5>{\mintocaml[firstline=15,lastline=21]{../src/backtrace/backtrace.ml}}
      \onslide*<6>{\mintocaml[firstline=28,lastline=34,highlightlines=31]{../src/backtrace/backtrace.ml}}
      \onslide*<7->{\mintocaml[firstline=28,lastline=34,highlightlines=33]{../src/backtrace/backtrace.ml}}
      \endminipage
    \end{column}
    \begin{column}{0.45\textwidth}
      \raggedleft
      \onslide*<1>{\providecommand\step{6}\input{diagrams/backtrace/backtrace.tex}}%
      \onslide*<2>{\providecommand\step{6}\input{diagrams/backtrace/backtrace.tex}}%
      \onslide*<3>{\providecommand\step{7}\input{diagrams/backtrace/backtrace.tex}}%
      \onslide*<4>{\providecommand\step{8}\input{diagrams/backtrace/backtrace.tex}}%
      \onslide*<5>{\providecommand\step{9}\input{diagrams/backtrace/backtrace.tex}}%
      \onslide*<6>{\providecommand\step{10}\input{diagrams/backtrace/backtrace.tex}}%
      \onslide*<7>{\providecommand\step{11}\input{diagrams/backtrace/backtrace.tex}}%
      \onslide*<8>{\providecommand\step{12}\input{diagrams/backtrace/backtrace.tex}}%
      \onslide*<9>{\providecommand\step{13}\input{diagrams/backtrace/backtrace.tex}}%
    \end{column}
  \end{columns}
\end{frame}

\begin{frame}{Backtraces}{Printing the backtrace}
  \begin{columns}[c]
    \begin{column}{0.45\textwidth}
      \minipage[c][0.66\textheight][s]{\columnwidth}
      \begin{itemize}
        \item<1-> The exception backtrace can now be printed to \funcarg{stdout} with \funcname{Printexc.print_backtrace}
        \item<2-> First the exception backtrace is retrieved into an OCaml array with \funcname{get_raw_backtrace }
        \item<3-> Which is implemented as the \funcname{caml_get_exception_raw_backtrace} C function in the runtime
        \item<4-> And finally printed with \funcname{print_exception_backtrace}
      \end{itemize}
      \vfill
      \mintocaml[firstline=28,lastline=34,highlightlines=33]{../src/backtrace/backtrace.ml}
      \endminipage
    \end{column}
    \begin{column}{0.55\textwidth}
      \onslide*<1,2>{
        \mintocaml[firstline=100,lastline=101]{../ocaml/stdlib/printexc.ml}
      }
      \only<1>{
        \listocaml[firstline=170,lastline=172]{../ocaml/stdlib/printexc.ml}{stdlib/printexc.ml:170}
      }
      \only<2>{
        \listocaml[firstline=170,lastline=172,highlightlines=172]{../ocaml/stdlib/printexc.ml}{stdlib/printexc.ml:170}
      }
      \only<3>{
        \mintc[firstline=154,lastline=156]{../ocaml/runtime/backtrace.c}
        \listc[firstline=171,lastline=192,highlightlines={184-187}]{../ocaml/runtime/backtrace.c}{runtime/backtrace.c:154}
      }
      \only<4>{
        \listocaml[firstline=155,lastline=165]{../ocaml/stdlib/printexc.ml}{stdlib/printexc.ml:55}
      }
    \end{column}
  \end{columns}
\end{frame}


\subsection{The multiple kinds of raise}
\frameSubsection{}{}

\begin{frame}{Backtraces}{When backtraces are not needed}
  \begin{columns}[c]
    \begin{column}{0.45\textwidth}
      \minipage[c][0.66\textheight][s]{\columnwidth}
      \begin{itemize}
        \item<1-> What if the backtrace for an exception is never needed?
        \item<2-> It's possible to raise the exception explicitly with \funcname{raise\_notrace} to make raising as cheap as possible
        \item<3-> An example of use in the \funcname{dynlink} library of the OCaml compiler
      \end{itemize}
      \vfill
      \onslide<3->{
      \listocaml[firstline=205,lastline=209,highlightlines=207]{../ocaml/otherlibs/dynlink/dynlink\_compilerlibs/misc.ml}{otherlibs/dynlink/dynlink\_compilerlibs/misc.ml:205}
      }
      \endminipage
    \end{column}
    \begin{column}{0.55\textwidth}
    \only<4>{
      \mintcmm[highlightlines=3]{../src/notrace/camlNotrace\_\_fun_347.cmm}
      \listcmm{../src/notrace/camlNotrace\_\_all\_somes\_327.cmm}{all\_somes}
    }
    \onslide*<5->{
      \listobjdump[highlightlines={6-9}]{../src/notrace/camlNotrace\_\_fun_347.objdump}{anonymous function from \funcname{all\_somes}}
    }
    \end{column}
  \end{columns}
\end{frame}

\begin{frame}{Backtraces}{Summary table of the multiple kinds of raise}
  \begin{itemize}
    \item When debugging information is enabled and if the backtrace of an exception isn't needed, it's possible to raise with \funcname{raise\_notrace} for performance
    \item When debugging information is \emph{not} enabled, the compiler optimises \funcname{raise} calls to inline assembly (same effect as using \funcname{raise\_notrace})
  \end{itemize}
  \bigskip
  \centering
  \begin{tabular}{| l | c | c | c | c |}
    \hline
    \parbox{10em}{Debug information \\ (\funcarg{-g} flag)}  & \multicolumn{2}{|c|}{Disabled}                                     & \multicolumn{2}{|c|}{Enabled} \\
    \hline
    Raise kind                                               & \funcname{raise} & \funcname{raise\_notrace}                       & \funcname{raise} & \funcname{raise\_notrace} \\
    \hline
    Compilation                                              & \parbox{8em}{inline assembly\\ (optimization)} & inline assembly   & \funcname{caml\_raise\_exn} & inline assembly \\
    \hline
  \end{tabular}
\end{frame}


%
% Takeaway #5
%
\subsection*{Takeaway \#5}
\frameSubsectionTakeaway{}
\againframe<5>{takeaway}
}
    \end{column}
  \end{columns}
\end{frame}

\begin{frame}{Backtraces}{\funcname{caml\_probably\_raise}'s handler doesn't catch \typename{ExnC}}
  \begin{columns}[c]
    \begin{column}{0.45\textwidth}
      \minipage[c][0.75\textheight][s]{\columnwidth}
      \begin{itemize}
        \item<1-> \funcname{match \dots with}\footnotemark pattern matching can also match exceptions
        \item<1-> The CMM of \funcname{probably\_raise} shows that this \funcname{match \dots with} is implemented with a pattern matching and a \funcname{try \dots with}
        \item<1-> But this one only matches on \typename{ExnA} exception
        \item<2-> Since the pattern matching fails to catch the exception, \funcname{reraise} is called with our current \typename{ExnC} exception
        \item<3-> Disassembly shows that \funcname{reraise} is actually a call to \funcname{caml\_reraise\_exn}, which is also an assembly function provided by the runtime
      \end{itemize}
      \vfill
      \mintocaml[firstline=15,lastline=21,highlightlines={18-21}]{../src/backtrace/backtrace.ml}
      \endminipage
    \end{column}
    \begin{column}{0.55\textwidth}
      \centering
      \onslide*<1>{\mintcmm[highlightlines={10-18}]{../src/backtrace/camlBacktrace\_\_probably\_raise\_351.cmm}}
      \onslide*<2>{\listcmm[highlightlines=18]{../src/backtrace/camlBacktrace\_\_probably\_raise_351.cmm}{CMM of probably\_raise}}
      \onslide*<3>{\mintobjdump[firstline=5,lastline=35,highlightlines=31]{../src/backtrace/camlBacktrace\_\_probably\_raise_351.objdump}}
    \end{column}
  \end{columns}
  \only<1>{\footnotetext[1]{https://blog.janestreet.com/pattern-matching-and-exception-handling-unite/}}
\end{frame}

\newcommand\listCamlReraiseExn[1][]{
  \listasm[firstline=727,lastline=734,#1]{../ocaml/runtime/amd64.S}{runtime/amd64.S:727}}

\begin{frame}{Backtraces}{Introducing \funcname{caml\_reraise\_exn}}
  \begin{columns}[c]
    \begin{column}{0.5\textwidth}
      \minipage[c][0.66\textheight][s]{\columnwidth}
      \begin{itemize}
        \item<1-> The exception have to be reraised to the next handler
        \item<2-> First, checks if the backtrace should be collected with \funcarg{domain\_state->backtrace\_active}
        \only<3>{
        \begin{itemize}
            \item If not, behaves like a \funcname{raise\_notrace} with \funcname{RESTORE\_EXN\_HANDLER\_OCAML}
        \end{itemize}
        }
        \item<4-> Jumps into \funcname{caml\_raise\_exn}
        \item<5-> Calls \funcname{caml\_stash\_backtrace} again, to scan the next part of the stack: from the trap just pop-ed to the next trap
      \end{itemize}
      \vfill
      \mintocaml[firstline=15,lastline=21,highlightlines={18-21}]{../src/backtrace/backtrace.ml}
      \endminipage
    \end{column}
    \begin{column}{0.5\textwidth}
      \raggedleft
      \onslide*<1>{\listCamlReraiseExn{}}
      \onslide*<2>{\listCamlReraiseExn[highlightlines=729]{}}
      \onslide*<3>{
        \mintc[firstline=190,lastline=192,highlightlines={191-192}]{../ocaml/runtime/amd64.S}
        \mintc[firstline=234,lastline=237,highlightlines={235-237}]{../ocaml/runtime/amd64.S}
        \medskip
        \listCamlReraiseExn[highlightlines=731]{}
      }
      \onslide*<4-5>{
        % TODO refactor with \listCamlReraiseExn
        \mintasm[firstline=727,lastline=734,highlightlines=730]{../ocaml/runtime/amd64.S}
        \medskip
      }
      \onslide*<4>{\listCamlRaiseExn[highlightlines=711]{}}
      \onslide*<5>{\listCamlRaiseExn[highlightlines=720]{}}
    \end{column}
  \end{columns}
\end{frame}

\begin{frame}{Backtraces}{Backtrace collection continues}
  \begin{columns}[c]
    \begin{column}{0.55\textwidth}
      \minipage[c][0.75\textheight][s]{\columnwidth}
      \begin{itemize}
        \item<1-> \funcname{caml\_stash\_backtrace} scans the older part of the stack, up to the next trap
        \item<6-> Controls jumps to the nested exception handler in \funcname{maybe\_raise}, which matches only on \typename{ExnB}
        \item<7-> \funcname{caml\_reraise\_exn} is called again, backtrace collection resumes with \funcname{caml\_stash\_backtrace}
      \end{itemize}
      \medskip
      \begin{itemize}
        \item<2-> Constructed backtrace:
        \begin{itemize}
          \item <2-> \funcname{definitely\_raise}
          \item <2-> \funcname{probably\_raise}
          \item <3-> \funcname{probably\_raise} (re-raise)
          \item <4-> \funcname{maybe\_raise}
          \item <5-> \funcname{main}
          \item <8-> \funcname{main} (re-raise)
        \end{itemize}
      \end{itemize}
      \vfill
      \onslide*<1-5>{\mintocaml[firstline=15,lastline=21]{../src/backtrace/backtrace.ml}}
      \onslide*<6>{\mintocaml[firstline=28,lastline=34,highlightlines=31]{../src/backtrace/backtrace.ml}}
      \onslide*<7->{\mintocaml[firstline=28,lastline=34,highlightlines=33]{../src/backtrace/backtrace.ml}}
      \endminipage
    \end{column}
    \begin{column}{0.45\textwidth}
      \raggedleft
      \onslide*<1>{\providecommand\step{6}%
% Backtraces
%
\subsection{One advantage of raising with debugging information}
\frameSubsection{}{}

\begin{frame}{Backtraces}{Debugging information \& source code}
  \begin{columns}[c]
    \begin{column}{0.5\textwidth}
      \begin{itemize}
        \item Raising an exception is fun and games, but how do we know where the exception originated? $\rightarrow$ With the backtrace associated with the exception!
        \item In order to get a backtrace from the point where an exception was raised, it's necessary to compile with debugging information enabled (\funcarg{-g} flag for the compiler)
        \item Same example as in the `Nested handlers' section
      \end{itemize}
    \end{column}
    \begin{column}{0.5\textwidth}
      \listocaml[firstline=9,lastline=34]{../src/backtrace/backtrace.ml}{backtrace/backtrace.ml}
    \end{column}
  \end{columns}
\end{frame}

\begin{frame}{Backtraces}{CMM and disassembly of \funcname{definitely\_raise}}
  \begin{columns}[c]
    \begin{column}{0.5\textwidth}
      \minipage[c][0.66\textheight][s]{\columnwidth}
      \begin{itemize}
        \item<1-> An effect of compiling with debugging information (\funcarg{-g}): the CMM no longer uses \funcname{raise\_notrace} to raise the exception but \funcname{raise} instead
        \item<2-> Disassembly shows that CMM \funcname{raise} is actually a call to \funcname{caml\_raise\_exn}, a function in assembler provided by the runtime
      \end{itemize}
      \vfill
      \mintocaml[firstline=9,lastline=13,highlightlines=12]{../src/backtrace/backtrace.ml}
      \endminipage
    \end{column}
    \begin{column}{0.5\textwidth}
      \onslide*<1>{
        \mintcmm[highlightlines=5]{../src/backtrace/camlBacktrace\_\_definitely\_raise\_347.cmm}
      }
      \onslide*<2>{
        \mintobjdump[firstline=2,highlightlines=14]{../src/backtrace/camlBacktrace\_\_definitely\_raise\_347.objdump}
      }
    \end{column}
  \end{columns}
\end{frame}

\begin{frame}{Backtraces}{Introducing \funcname{caml\_raise\_exn}}
  \begin{columns}[c]
    \begin{column}{0.5\textwidth}
      \minipage[c][0.66\textheight][s]{\columnwidth}
      \onslide*<1>{
        \begin{itemize}
          \item Raising the exception is performed using \funcname{caml\_raise\_exn} which is
            \begin{itemize}
              \item An assembly function provided by the runtime
              \item Architecture specific
            \end{itemize}
          \item Let's see what it does differently from \funcname{raise\_notrace}\dots
          \item We are about to raise \typename{ExnC} with \funcname{caml\_raise\_exn} from \funcname{definitely\_raise}
        \end{itemize}
      }
      \onslide*<2>{
        \mintobjdump[firstline=2,highlightlines=14]{../src/backtrace/camlBacktrace\_\_definitely\_raise\_347.objdump}
      }
      \vfill
      \mintocaml[firstline=9,lastline=13,highlightlines=12]{../src/backtrace/backtrace.ml}
      \endminipage
    \end{column}
    \begin{column}{0.5\textwidth}
      \centering
      \providecommand\step{1}\input{diagrams/backtrace/backtrace.tex}
    \end{column}
  \end{columns}
\end{frame}

\begin{frame}{Backtraces}{But before that, we need more fields from \typename{caml\_domain\_state}}
  \begin{columns}
    \begin{column}{0.5\textwidth}
      \begin{itemize}
        \item Remember \typename{caml\_domain\_state} the per-domain runtime state?
        \item Some other members are of interest to us now\dots
        \item \localname{backtrace\_active} is used to know if the runtime should collect backtraces
        \item \localname{backtrace\_active} can be set by
          \begin{itemize}
            \item The \funcarg{b} flag in the \funcname{OCAMLRUNPARAM} environment variable
            \item \mintinline{ocaml}{Printexc.record_backtrace : bool -> unit} \footnotemark
          \end{itemize}
      \end{itemize}
    \end{column}
    \begin{column}{0.5\textwidth}
      \begin{listing}
        \mintc[highlightlines={9-11}]{src/ocaml/domain\_state\_backtrace.h}
        \caption{runtime/caml/domain\_state.h}
      \end{listing}
    \end{column}
  \end{columns}
  \footnotetext[1]{https://v2.ocaml.org/api/Printexc.html}
\end{frame}

\newcommand\listCamlRaiseExn[1][]{
  \listasm[firstline=702,lastline=725,#1]{../ocaml/runtime/amd64.S}{runtime/amd64.S:702}}

\begin{frame}{Backtraces}{Walk through \funcname{caml\_raise\_exn}}
  \begin{columns}[T]
    \begin{column}{0.5\textwidth}
      \begin{itemize}
        \item<1-> First checks whether exceptions backtrace should be recorded
        \item<1-> By checking the \funcarg{domain\_state->backtrace\_active} value
        \item<2-> If not, acts as \funcname{raise\_notrace} with \funcname{RESTORE\_EXN\_HANDLER\_OCAML}
          \begin{itemize}
            \item Set \regname{RSP} to \localname{domain\_state->exn\_handler} value
            \item Restore \localname{domain\_state->exn\_handler} from trap
            \item Jump with \funcname{ret} (same effect as using \funcname{pop} and \funcname{jmp})
          \end{itemize}
        \item<3-> Otherwise, switches to the C stack to be able to execute C code
        \item<4-> Calls \funcname{caml\_stash\_backtrace}, a C function provided by the runtime\ignorespaces
          \onslide<6->{, with
            \begin{itemize}
              \item \funcarg{exn}: exception value
              \item \funcarg{pc}: code address of the raising point
              \item \funcarg{sp}: stack address of the raising function
              \item \funcarg{trapsp}: stack address of the next handler
            \end{itemize}
          }
      \end{itemize}
      \bigskip
      \mintocaml[firstline=9,lastline=13,highlightlines=12]{../src/backtrace/backtrace.ml}
    \end{column}
    \begin{column}{0.5\textwidth}
      \raggedleft
      \onslide*<1,3-4>{
        \mintc[firstline=190,lastline=192]{../ocaml/runtime/amd64.S}
        \mintc[firstline=234,lastline=237]{../ocaml/runtime/amd64.S}
      }
      \onslide*<2>{
        \mintc[firstline=190,lastline=192,highlightlines={191-192}]{../ocaml/runtime/amd64.S}
        \mintc[firstline=234,lastline=237,highlightlines={235-237}]{../ocaml/runtime/amd64.S}
      }
      \onslide*<1>{\listCamlRaiseExn[highlightlines=705]{}}%
      \onslide*<2>{\listCamlRaiseExn[highlightlines={707-708}]{}}%
      \onslide*<3>{\listCamlRaiseExn[highlightlines={712-714}]{}}%
      \onslide*<4>{\listCamlRaiseExn[highlightlines={715-720}]{}}%
      \onslide*<5>{\providecommand\step{1}\input{diagrams/backtrace/backtrace.tex}}%
      \onslide*<6>{\providecommand\step{2}\input{diagrams/backtrace/backtrace.tex}}%
    \end{column}
  \end{columns}
\end{frame}

\newcommand\listStashBacktrace[1][]{
  \listc[firstline=91,lastline=120,#1]{../ocaml/runtime/backtrace\_nat.c}{runtime/backtrace\_nat.c:91}}

\begin{frame}{Backtraces}{Introducing \funcname{caml\_stash\_backtrace}}
  \begin{columns}[c]
    \begin{column}{0.5\textwidth}
      \minipage[c][0.66\textheight][s]{\columnwidth}
      \begin{itemize}
        \item<1-> A C function provided by the runtime (specific to native code)
        \item<2-> It scans the stack, between the stack pointer at the raising point up to the next trap, in order to collect the names of the detected function frames
          \onslide*<2>{
            \begin{itemize}
              \item And more information, like line number, column number...
            \end{itemize}
          }
        \item<3-> It uses the same technique and information as the Garbage Collector (GC) uses to scan the values live on the stack
          \onslide*<4>{
            \begin{itemize}
              \item \typename{frame\_descr} are at the core of stack unwinding
            \end{itemize}
          }
      \end{itemize}
      \vfill
      \mintocaml[firstline=9,lastline=13,highlightlines=12]{../src/backtrace/backtrace.ml}
      \endminipage
    \end{column}
    \begin{column}{0.5\textwidth}
      \onslide*<1>{\listStashBacktrace[highlightlines=91]{}}
      \onslide*<2>{\listStashBacktrace[highlightlines={91,117-118}]{}}
      \onslide*<3->{\listStashBacktrace[highlightlines={94,106,109-110}]{}}
    \end{column}
  \end{columns}
\end{frame}

\newcommand\listFrameDescr[1][]{
  \listocaml[firstline=111,lastline=124,#1]{../ocaml/asmcomp/emitaux.ml}{asmcomp/emitaux.ml:111}}
\newcommand\listDebugInfo[1][]{
  \listocaml[firstline=38,lastline=49,#1]{../ocaml/lambda/debuginfo.mli}{lambda/debuginfo.mli:38}}

\begin{frame}{Backtraces}{\typename{frame\_descr} and \typename{frame\_debuginfo} in a nutshell}
  \begin{columns}[c]
    \begin{column}{0.5\textwidth}
      \begin{itemize}
        \item<1-> For every return address in an OCaml program, there is a associated \typename{frame\_descr}
        \item<2-> A \typename{frame\_descr} specifies
        \begin{itemize}
          \item<2-> The size of the function stack frame
          \item<3-> Where live values are on the stack (so that the GC can mark them as live and prevent from being swept)
        \end{itemize}
        \item<4-> When compiled with debug information, \typename{frame\_descr} will be linked to a \typename{frame\_debuginfo}
        \begin{itemize}
          \item<5-> Contains information about the position in the source code (file name, line and column number)
        \end{itemize}
      \end{itemize}
    \end{column}
    \begin{column}{0.5\textwidth}
        \onslide*<1>{\listFrameDescr[highlightlines={118-122}]{}}
        \onslide*<2>{\listFrameDescr[highlightlines={120}]{}}
        \onslide*<3>{\listFrameDescr[highlightlines={121}]{}}
        \onslide*<4->{\listFrameDescr[highlightlines={122}]{}}
        \medskip
        \onslide*<1-3>{\listDebugInfo{}}
        \onslide*<4>{\listDebugInfo[highlightlines={38,49}]{}}
        \onslide*<5>{\listDebugInfo[highlightlines={39-46}]{}}
    \end{column}
  \end{columns}
\end{frame}

\begin{frame}{Backtraces}{Scanning the stack with \typename{frame\_descr}}
  \begin{columns}[c]
    \begin{column}{0.5\textwidth}
      \begin{itemize}
        \item Given the return address of the code that raises the exception by calling \funcname{caml\_raise\_exn} (\funcarg{pc} argument of \funcname{caml\_stash\_backtrace})
        \item And the position of the stack pointer (\funcarg{sp} argument of \funcname{caml\_stash\_backtrace})
        \item The frame size given by \typename{frame\_descr}.\funcarg{frame\_size}, allows to find the end of the previous frame
        \item And the return address of the caller of this function
        \bigskip
        \onslide<3->{
        \item Note that traps (here the reddish \funcname{ExnA}, \funcname{ExnB} and \funcname{ExnC} blocks) belong to the frame that installed them, and so the frame size changes when traps are removed
        }
      \end{itemize}
    \end{column}
    \begin{column}{0.5\textwidth}
      \raggedleft
      \onslide*<1>{\providecommand\step{2}\input{diagrams/backtrace/backtrace.tex}}%
      \onslide*<2>{\providecommand\step{1}\input{diagrams/backtrace/scanstack.tex}}%
      \onslide*<3>{\providecommand\step{2}\input{diagrams/backtrace/scanstack.tex}}%
      \onslide*<4>{\providecommand\step{3}\input{diagrams/backtrace/scanstack.tex}}%
      \onslide*<5>{\providecommand\step{4}\input{diagrams/backtrace/scanstack.tex}}%
      \onslide*<6>{\providecommand\step{5}\input{diagrams/backtrace/scanstack.tex}}%
    \end{column}
  \end{columns}
\end{frame}

% Duplicate of previous-previous slide
\begin{frame}{Backtraces}{Continuing on \funcname{caml\_stash\_backtrace}}
  \begin{columns}[c]
    \begin{column}{0.5\textwidth}
      \minipage[c][0.75\textheight][s]{\columnwidth}
      \begin{itemize}
          \color{gray}
        \item A C function provided by the runtime (specific to native code)
        \item It scans the stack, between the stack pointer at the raising point up to the next trap, in order to collect the names of the detected function frames
        \item It uses the same technique and information as the Garbage Collector (GC) uses to scan the values live on the stack
      \end{itemize}
      \begin{itemize}
        \item For each function frame detected, store a pointer to the \typename{frame\_descr} in the \localname{domain\_state->backtrace\_buffer} array, effectively constructing the backtrace
      \end{itemize}
      \onslide<2->{
        \begin{itemize}
            \smallskip
          \item Constructed backtrace:
            \funcname{definitely\_raise},
            \onslide<3->{\funcname{probably\_raise}}
        \end{itemize}
      }
      \vfill
      \mintocaml[firstline=9,lastline=13,highlightlines=12]{../src/backtrace/backtrace.ml}
      \endminipage
    \end{column}
    \begin{column}{0.5\textwidth}
      \raggedleft
      \onslide*<1>{\listStashBacktrace[highlightlines={114-115}]{}}
      \onslide*<2>{\providecommand\step{3}\input{diagrams/backtrace/backtrace.tex}}%
      \onslide*<3>{\providecommand\step{4}\input{diagrams/backtrace/backtrace.tex}}%
    \end{column}
  \end{columns}
\end{frame}

\begin{frame}{Backtraces}{Back to \funcname{caml\_raise\_exn}}
  \begin{columns}[c]
    \begin{column}{0.5\textwidth}
      \minipage[c][0.66\textheight][s]{\columnwidth}
      \begin{itemize}\color{gray}
        \item Checks wheteher exceptions backtrace should be recorded, by checking the \funcarg{domain\_state->backtrace\_active} value
        \item Switches to the C stack to be able to execute C code
        \item Calls \funcname{caml\_stash\_backtrace}, to collect the backtrace up to the next trap
      \end{itemize}
      \onslide*<2->{
        \begin{itemize}
          \item<2-> Executes the exception handler in \funcname{probably\_raise} with \funcname{RESTORE\_EXN\_HANDLER\_OCAML}
            \begin{itemize}
              \item Set \regname{RSP} to \localname{domain\_state->exn\_handler} value
              \item Restore \localname{domain\_state->exn\_handler} from trap
              \item Jump to the handler code with \funcname{ret}
            \end{itemize}
        \end{itemize}
      }
      \vfill
      \onslide*<1>{\mintocaml[firstline=9,lastline=13,highlightlines=12]{../src/backtrace/backtrace.ml}}
      \onslide*<2->{\mintocaml[firstline=15,lastline=21,highlightlines={19-21}]{../src/backtrace/backtrace.ml}}
      \endminipage
    \end{column}
    \begin{column}{0.5\textwidth}
      \centering
      \onslide*<1>{
        \mintc[firstline=190,lastline=192]{../ocaml/runtime/amd64.S}
        \mintc[firstline=234,lastline=237]{../ocaml/runtime/amd64.S}
      }
      \onslide*<2>{
        \mintc[firstline=190,lastline=192,highlightlines={191-192}]{../ocaml/runtime/amd64.S}
        \mintc[firstline=234,lastline=237,highlightlines={235-237}]{../ocaml/runtime/amd64.S}
      }
      \onslide*<1>{\listCamlRaiseExn[highlightlines=720]{}}
      \onslide*<2>{\listCamlRaiseExn[highlightlines=722]{}}
      \onslide*<3->{\providecommand\step{5}\input{diagrams/backtrace/backtrace.tex}}
    \end{column}
  \end{columns}
\end{frame}

\begin{frame}{Backtraces}{\funcname{caml\_probably\_raise}'s handler doesn't catch \typename{ExnC}}
  \begin{columns}[c]
    \begin{column}{0.45\textwidth}
      \minipage[c][0.75\textheight][s]{\columnwidth}
      \begin{itemize}
        \item<1-> \funcname{match \dots with}\footnotemark pattern matching can also match exceptions
        \item<1-> The CMM of \funcname{probably\_raise} shows that this \funcname{match \dots with} is implemented with a pattern matching and a \funcname{try \dots with}
        \item<1-> But this one only matches on \typename{ExnA} exception
        \item<2-> Since the pattern matching fails to catch the exception, \funcname{reraise} is called with our current \typename{ExnC} exception
        \item<3-> Disassembly shows that \funcname{reraise} is actually a call to \funcname{caml\_reraise\_exn}, which is also an assembly function provided by the runtime
      \end{itemize}
      \vfill
      \mintocaml[firstline=15,lastline=21,highlightlines={18-21}]{../src/backtrace/backtrace.ml}
      \endminipage
    \end{column}
    \begin{column}{0.55\textwidth}
      \centering
      \onslide*<1>{\mintcmm[highlightlines={10-18}]{../src/backtrace/camlBacktrace\_\_probably\_raise\_351.cmm}}
      \onslide*<2>{\listcmm[highlightlines=18]{../src/backtrace/camlBacktrace\_\_probably\_raise_351.cmm}{CMM of probably\_raise}}
      \onslide*<3>{\mintobjdump[firstline=5,lastline=35,highlightlines=31]{../src/backtrace/camlBacktrace\_\_probably\_raise_351.objdump}}
    \end{column}
  \end{columns}
  \only<1>{\footnotetext[1]{https://blog.janestreet.com/pattern-matching-and-exception-handling-unite/}}
\end{frame}

\newcommand\listCamlReraiseExn[1][]{
  \listasm[firstline=727,lastline=734,#1]{../ocaml/runtime/amd64.S}{runtime/amd64.S:727}}

\begin{frame}{Backtraces}{Introducing \funcname{caml\_reraise\_exn}}
  \begin{columns}[c]
    \begin{column}{0.5\textwidth}
      \minipage[c][0.66\textheight][s]{\columnwidth}
      \begin{itemize}
        \item<1-> The exception have to be reraised to the next handler
        \item<2-> First, checks if the backtrace should be collected with \funcarg{domain\_state->backtrace\_active}
        \only<3>{
        \begin{itemize}
            \item If not, behaves like a \funcname{raise\_notrace} with \funcname{RESTORE\_EXN\_HANDLER\_OCAML}
        \end{itemize}
        }
        \item<4-> Jumps into \funcname{caml\_raise\_exn}
        \item<5-> Calls \funcname{caml\_stash\_backtrace} again, to scan the next part of the stack: from the trap just pop-ed to the next trap
      \end{itemize}
      \vfill
      \mintocaml[firstline=15,lastline=21,highlightlines={18-21}]{../src/backtrace/backtrace.ml}
      \endminipage
    \end{column}
    \begin{column}{0.5\textwidth}
      \raggedleft
      \onslide*<1>{\listCamlReraiseExn{}}
      \onslide*<2>{\listCamlReraiseExn[highlightlines=729]{}}
      \onslide*<3>{
        \mintc[firstline=190,lastline=192,highlightlines={191-192}]{../ocaml/runtime/amd64.S}
        \mintc[firstline=234,lastline=237,highlightlines={235-237}]{../ocaml/runtime/amd64.S}
        \medskip
        \listCamlReraiseExn[highlightlines=731]{}
      }
      \onslide*<4-5>{
        % TODO refactor with \listCamlReraiseExn
        \mintasm[firstline=727,lastline=734,highlightlines=730]{../ocaml/runtime/amd64.S}
        \medskip
      }
      \onslide*<4>{\listCamlRaiseExn[highlightlines=711]{}}
      \onslide*<5>{\listCamlRaiseExn[highlightlines=720]{}}
    \end{column}
  \end{columns}
\end{frame}

\begin{frame}{Backtraces}{Backtrace collection continues}
  \begin{columns}[c]
    \begin{column}{0.55\textwidth}
      \minipage[c][0.75\textheight][s]{\columnwidth}
      \begin{itemize}
        \item<1-> \funcname{caml\_stash\_backtrace} scans the older part of the stack, up to the next trap
        \item<6-> Controls jumps to the nested exception handler in \funcname{maybe\_raise}, which matches only on \typename{ExnB}
        \item<7-> \funcname{caml\_reraise\_exn} is called again, backtrace collection resumes with \funcname{caml\_stash\_backtrace}
      \end{itemize}
      \medskip
      \begin{itemize}
        \item<2-> Constructed backtrace:
        \begin{itemize}
          \item <2-> \funcname{definitely\_raise}
          \item <2-> \funcname{probably\_raise}
          \item <3-> \funcname{probably\_raise} (re-raise)
          \item <4-> \funcname{maybe\_raise}
          \item <5-> \funcname{main}
          \item <8-> \funcname{main} (re-raise)
        \end{itemize}
      \end{itemize}
      \vfill
      \onslide*<1-5>{\mintocaml[firstline=15,lastline=21]{../src/backtrace/backtrace.ml}}
      \onslide*<6>{\mintocaml[firstline=28,lastline=34,highlightlines=31]{../src/backtrace/backtrace.ml}}
      \onslide*<7->{\mintocaml[firstline=28,lastline=34,highlightlines=33]{../src/backtrace/backtrace.ml}}
      \endminipage
    \end{column}
    \begin{column}{0.45\textwidth}
      \raggedleft
      \onslide*<1>{\providecommand\step{6}\input{diagrams/backtrace/backtrace.tex}}%
      \onslide*<2>{\providecommand\step{6}\input{diagrams/backtrace/backtrace.tex}}%
      \onslide*<3>{\providecommand\step{7}\input{diagrams/backtrace/backtrace.tex}}%
      \onslide*<4>{\providecommand\step{8}\input{diagrams/backtrace/backtrace.tex}}%
      \onslide*<5>{\providecommand\step{9}\input{diagrams/backtrace/backtrace.tex}}%
      \onslide*<6>{\providecommand\step{10}\input{diagrams/backtrace/backtrace.tex}}%
      \onslide*<7>{\providecommand\step{11}\input{diagrams/backtrace/backtrace.tex}}%
      \onslide*<8>{\providecommand\step{12}\input{diagrams/backtrace/backtrace.tex}}%
      \onslide*<9>{\providecommand\step{13}\input{diagrams/backtrace/backtrace.tex}}%
    \end{column}
  \end{columns}
\end{frame}

\begin{frame}{Backtraces}{Printing the backtrace}
  \begin{columns}[c]
    \begin{column}{0.45\textwidth}
      \minipage[c][0.66\textheight][s]{\columnwidth}
      \begin{itemize}
        \item<1-> The exception backtrace can now be printed to \funcarg{stdout} with \funcname{Printexc.print_backtrace}
        \item<2-> First the exception backtrace is retrieved into an OCaml array with \funcname{get_raw_backtrace }
        \item<3-> Which is implemented as the \funcname{caml_get_exception_raw_backtrace} C function in the runtime
        \item<4-> And finally printed with \funcname{print_exception_backtrace}
      \end{itemize}
      \vfill
      \mintocaml[firstline=28,lastline=34,highlightlines=33]{../src/backtrace/backtrace.ml}
      \endminipage
    \end{column}
    \begin{column}{0.55\textwidth}
      \onslide*<1,2>{
        \mintocaml[firstline=100,lastline=101]{../ocaml/stdlib/printexc.ml}
      }
      \only<1>{
        \listocaml[firstline=170,lastline=172]{../ocaml/stdlib/printexc.ml}{stdlib/printexc.ml:170}
      }
      \only<2>{
        \listocaml[firstline=170,lastline=172,highlightlines=172]{../ocaml/stdlib/printexc.ml}{stdlib/printexc.ml:170}
      }
      \only<3>{
        \mintc[firstline=154,lastline=156]{../ocaml/runtime/backtrace.c}
        \listc[firstline=171,lastline=192,highlightlines={184-187}]{../ocaml/runtime/backtrace.c}{runtime/backtrace.c:154}
      }
      \only<4>{
        \listocaml[firstline=155,lastline=165]{../ocaml/stdlib/printexc.ml}{stdlib/printexc.ml:55}
      }
    \end{column}
  \end{columns}
\end{frame}


\subsection{The multiple kinds of raise}
\frameSubsection{}{}

\begin{frame}{Backtraces}{When backtraces are not needed}
  \begin{columns}[c]
    \begin{column}{0.45\textwidth}
      \minipage[c][0.66\textheight][s]{\columnwidth}
      \begin{itemize}
        \item<1-> What if the backtrace for an exception is never needed?
        \item<2-> It's possible to raise the exception explicitly with \funcname{raise\_notrace} to make raising as cheap as possible
        \item<3-> An example of use in the \funcname{dynlink} library of the OCaml compiler
      \end{itemize}
      \vfill
      \onslide<3->{
      \listocaml[firstline=205,lastline=209,highlightlines=207]{../ocaml/otherlibs/dynlink/dynlink\_compilerlibs/misc.ml}{otherlibs/dynlink/dynlink\_compilerlibs/misc.ml:205}
      }
      \endminipage
    \end{column}
    \begin{column}{0.55\textwidth}
    \only<4>{
      \mintcmm[highlightlines=3]{../src/notrace/camlNotrace\_\_fun_347.cmm}
      \listcmm{../src/notrace/camlNotrace\_\_all\_somes\_327.cmm}{all\_somes}
    }
    \onslide*<5->{
      \listobjdump[highlightlines={6-9}]{../src/notrace/camlNotrace\_\_fun_347.objdump}{anonymous function from \funcname{all\_somes}}
    }
    \end{column}
  \end{columns}
\end{frame}

\begin{frame}{Backtraces}{Summary table of the multiple kinds of raise}
  \begin{itemize}
    \item When debugging information is enabled and if the backtrace of an exception isn't needed, it's possible to raise with \funcname{raise\_notrace} for performance
    \item When debugging information is \emph{not} enabled, the compiler optimises \funcname{raise} calls to inline assembly (same effect as using \funcname{raise\_notrace})
  \end{itemize}
  \bigskip
  \centering
  \begin{tabular}{| l | c | c | c | c |}
    \hline
    \parbox{10em}{Debug information \\ (\funcarg{-g} flag)}  & \multicolumn{2}{|c|}{Disabled}                                     & \multicolumn{2}{|c|}{Enabled} \\
    \hline
    Raise kind                                               & \funcname{raise} & \funcname{raise\_notrace}                       & \funcname{raise} & \funcname{raise\_notrace} \\
    \hline
    Compilation                                              & \parbox{8em}{inline assembly\\ (optimization)} & inline assembly   & \funcname{caml\_raise\_exn} & inline assembly \\
    \hline
  \end{tabular}
\end{frame}


%
% Takeaway #5
%
\subsection*{Takeaway \#5}
\frameSubsectionTakeaway{}
\againframe<5>{takeaway}
}%
      \onslide*<2>{\providecommand\step{6}%
% Backtraces
%
\subsection{One advantage of raising with debugging information}
\frameSubsection{}{}

\begin{frame}{Backtraces}{Debugging information \& source code}
  \begin{columns}[c]
    \begin{column}{0.5\textwidth}
      \begin{itemize}
        \item Raising an exception is fun and games, but how do we know where the exception originated? $\rightarrow$ With the backtrace associated with the exception!
        \item In order to get a backtrace from the point where an exception was raised, it's necessary to compile with debugging information enabled (\funcarg{-g} flag for the compiler)
        \item Same example as in the `Nested handlers' section
      \end{itemize}
    \end{column}
    \begin{column}{0.5\textwidth}
      \listocaml[firstline=9,lastline=34]{../src/backtrace/backtrace.ml}{backtrace/backtrace.ml}
    \end{column}
  \end{columns}
\end{frame}

\begin{frame}{Backtraces}{CMM and disassembly of \funcname{definitely\_raise}}
  \begin{columns}[c]
    \begin{column}{0.5\textwidth}
      \minipage[c][0.66\textheight][s]{\columnwidth}
      \begin{itemize}
        \item<1-> An effect of compiling with debugging information (\funcarg{-g}): the CMM no longer uses \funcname{raise\_notrace} to raise the exception but \funcname{raise} instead
        \item<2-> Disassembly shows that CMM \funcname{raise} is actually a call to \funcname{caml\_raise\_exn}, a function in assembler provided by the runtime
      \end{itemize}
      \vfill
      \mintocaml[firstline=9,lastline=13,highlightlines=12]{../src/backtrace/backtrace.ml}
      \endminipage
    \end{column}
    \begin{column}{0.5\textwidth}
      \onslide*<1>{
        \mintcmm[highlightlines=5]{../src/backtrace/camlBacktrace\_\_definitely\_raise\_347.cmm}
      }
      \onslide*<2>{
        \mintobjdump[firstline=2,highlightlines=14]{../src/backtrace/camlBacktrace\_\_definitely\_raise\_347.objdump}
      }
    \end{column}
  \end{columns}
\end{frame}

\begin{frame}{Backtraces}{Introducing \funcname{caml\_raise\_exn}}
  \begin{columns}[c]
    \begin{column}{0.5\textwidth}
      \minipage[c][0.66\textheight][s]{\columnwidth}
      \onslide*<1>{
        \begin{itemize}
          \item Raising the exception is performed using \funcname{caml\_raise\_exn} which is
            \begin{itemize}
              \item An assembly function provided by the runtime
              \item Architecture specific
            \end{itemize}
          \item Let's see what it does differently from \funcname{raise\_notrace}\dots
          \item We are about to raise \typename{ExnC} with \funcname{caml\_raise\_exn} from \funcname{definitely\_raise}
        \end{itemize}
      }
      \onslide*<2>{
        \mintobjdump[firstline=2,highlightlines=14]{../src/backtrace/camlBacktrace\_\_definitely\_raise\_347.objdump}
      }
      \vfill
      \mintocaml[firstline=9,lastline=13,highlightlines=12]{../src/backtrace/backtrace.ml}
      \endminipage
    \end{column}
    \begin{column}{0.5\textwidth}
      \centering
      \providecommand\step{1}\input{diagrams/backtrace/backtrace.tex}
    \end{column}
  \end{columns}
\end{frame}

\begin{frame}{Backtraces}{But before that, we need more fields from \typename{caml\_domain\_state}}
  \begin{columns}
    \begin{column}{0.5\textwidth}
      \begin{itemize}
        \item Remember \typename{caml\_domain\_state} the per-domain runtime state?
        \item Some other members are of interest to us now\dots
        \item \localname{backtrace\_active} is used to know if the runtime should collect backtraces
        \item \localname{backtrace\_active} can be set by
          \begin{itemize}
            \item The \funcarg{b} flag in the \funcname{OCAMLRUNPARAM} environment variable
            \item \mintinline{ocaml}{Printexc.record_backtrace : bool -> unit} \footnotemark
          \end{itemize}
      \end{itemize}
    \end{column}
    \begin{column}{0.5\textwidth}
      \begin{listing}
        \mintc[highlightlines={9-11}]{src/ocaml/domain\_state\_backtrace.h}
        \caption{runtime/caml/domain\_state.h}
      \end{listing}
    \end{column}
  \end{columns}
  \footnotetext[1]{https://v2.ocaml.org/api/Printexc.html}
\end{frame}

\newcommand\listCamlRaiseExn[1][]{
  \listasm[firstline=702,lastline=725,#1]{../ocaml/runtime/amd64.S}{runtime/amd64.S:702}}

\begin{frame}{Backtraces}{Walk through \funcname{caml\_raise\_exn}}
  \begin{columns}[T]
    \begin{column}{0.5\textwidth}
      \begin{itemize}
        \item<1-> First checks whether exceptions backtrace should be recorded
        \item<1-> By checking the \funcarg{domain\_state->backtrace\_active} value
        \item<2-> If not, acts as \funcname{raise\_notrace} with \funcname{RESTORE\_EXN\_HANDLER\_OCAML}
          \begin{itemize}
            \item Set \regname{RSP} to \localname{domain\_state->exn\_handler} value
            \item Restore \localname{domain\_state->exn\_handler} from trap
            \item Jump with \funcname{ret} (same effect as using \funcname{pop} and \funcname{jmp})
          \end{itemize}
        \item<3-> Otherwise, switches to the C stack to be able to execute C code
        \item<4-> Calls \funcname{caml\_stash\_backtrace}, a C function provided by the runtime\ignorespaces
          \onslide<6->{, with
            \begin{itemize}
              \item \funcarg{exn}: exception value
              \item \funcarg{pc}: code address of the raising point
              \item \funcarg{sp}: stack address of the raising function
              \item \funcarg{trapsp}: stack address of the next handler
            \end{itemize}
          }
      \end{itemize}
      \bigskip
      \mintocaml[firstline=9,lastline=13,highlightlines=12]{../src/backtrace/backtrace.ml}
    \end{column}
    \begin{column}{0.5\textwidth}
      \raggedleft
      \onslide*<1,3-4>{
        \mintc[firstline=190,lastline=192]{../ocaml/runtime/amd64.S}
        \mintc[firstline=234,lastline=237]{../ocaml/runtime/amd64.S}
      }
      \onslide*<2>{
        \mintc[firstline=190,lastline=192,highlightlines={191-192}]{../ocaml/runtime/amd64.S}
        \mintc[firstline=234,lastline=237,highlightlines={235-237}]{../ocaml/runtime/amd64.S}
      }
      \onslide*<1>{\listCamlRaiseExn[highlightlines=705]{}}%
      \onslide*<2>{\listCamlRaiseExn[highlightlines={707-708}]{}}%
      \onslide*<3>{\listCamlRaiseExn[highlightlines={712-714}]{}}%
      \onslide*<4>{\listCamlRaiseExn[highlightlines={715-720}]{}}%
      \onslide*<5>{\providecommand\step{1}\input{diagrams/backtrace/backtrace.tex}}%
      \onslide*<6>{\providecommand\step{2}\input{diagrams/backtrace/backtrace.tex}}%
    \end{column}
  \end{columns}
\end{frame}

\newcommand\listStashBacktrace[1][]{
  \listc[firstline=91,lastline=120,#1]{../ocaml/runtime/backtrace\_nat.c}{runtime/backtrace\_nat.c:91}}

\begin{frame}{Backtraces}{Introducing \funcname{caml\_stash\_backtrace}}
  \begin{columns}[c]
    \begin{column}{0.5\textwidth}
      \minipage[c][0.66\textheight][s]{\columnwidth}
      \begin{itemize}
        \item<1-> A C function provided by the runtime (specific to native code)
        \item<2-> It scans the stack, between the stack pointer at the raising point up to the next trap, in order to collect the names of the detected function frames
          \onslide*<2>{
            \begin{itemize}
              \item And more information, like line number, column number...
            \end{itemize}
          }
        \item<3-> It uses the same technique and information as the Garbage Collector (GC) uses to scan the values live on the stack
          \onslide*<4>{
            \begin{itemize}
              \item \typename{frame\_descr} are at the core of stack unwinding
            \end{itemize}
          }
      \end{itemize}
      \vfill
      \mintocaml[firstline=9,lastline=13,highlightlines=12]{../src/backtrace/backtrace.ml}
      \endminipage
    \end{column}
    \begin{column}{0.5\textwidth}
      \onslide*<1>{\listStashBacktrace[highlightlines=91]{}}
      \onslide*<2>{\listStashBacktrace[highlightlines={91,117-118}]{}}
      \onslide*<3->{\listStashBacktrace[highlightlines={94,106,109-110}]{}}
    \end{column}
  \end{columns}
\end{frame}

\newcommand\listFrameDescr[1][]{
  \listocaml[firstline=111,lastline=124,#1]{../ocaml/asmcomp/emitaux.ml}{asmcomp/emitaux.ml:111}}
\newcommand\listDebugInfo[1][]{
  \listocaml[firstline=38,lastline=49,#1]{../ocaml/lambda/debuginfo.mli}{lambda/debuginfo.mli:38}}

\begin{frame}{Backtraces}{\typename{frame\_descr} and \typename{frame\_debuginfo} in a nutshell}
  \begin{columns}[c]
    \begin{column}{0.5\textwidth}
      \begin{itemize}
        \item<1-> For every return address in an OCaml program, there is a associated \typename{frame\_descr}
        \item<2-> A \typename{frame\_descr} specifies
        \begin{itemize}
          \item<2-> The size of the function stack frame
          \item<3-> Where live values are on the stack (so that the GC can mark them as live and prevent from being swept)
        \end{itemize}
        \item<4-> When compiled with debug information, \typename{frame\_descr} will be linked to a \typename{frame\_debuginfo}
        \begin{itemize}
          \item<5-> Contains information about the position in the source code (file name, line and column number)
        \end{itemize}
      \end{itemize}
    \end{column}
    \begin{column}{0.5\textwidth}
        \onslide*<1>{\listFrameDescr[highlightlines={118-122}]{}}
        \onslide*<2>{\listFrameDescr[highlightlines={120}]{}}
        \onslide*<3>{\listFrameDescr[highlightlines={121}]{}}
        \onslide*<4->{\listFrameDescr[highlightlines={122}]{}}
        \medskip
        \onslide*<1-3>{\listDebugInfo{}}
        \onslide*<4>{\listDebugInfo[highlightlines={38,49}]{}}
        \onslide*<5>{\listDebugInfo[highlightlines={39-46}]{}}
    \end{column}
  \end{columns}
\end{frame}

\begin{frame}{Backtraces}{Scanning the stack with \typename{frame\_descr}}
  \begin{columns}[c]
    \begin{column}{0.5\textwidth}
      \begin{itemize}
        \item Given the return address of the code that raises the exception by calling \funcname{caml\_raise\_exn} (\funcarg{pc} argument of \funcname{caml\_stash\_backtrace})
        \item And the position of the stack pointer (\funcarg{sp} argument of \funcname{caml\_stash\_backtrace})
        \item The frame size given by \typename{frame\_descr}.\funcarg{frame\_size}, allows to find the end of the previous frame
        \item And the return address of the caller of this function
        \bigskip
        \onslide<3->{
        \item Note that traps (here the reddish \funcname{ExnA}, \funcname{ExnB} and \funcname{ExnC} blocks) belong to the frame that installed them, and so the frame size changes when traps are removed
        }
      \end{itemize}
    \end{column}
    \begin{column}{0.5\textwidth}
      \raggedleft
      \onslide*<1>{\providecommand\step{2}\input{diagrams/backtrace/backtrace.tex}}%
      \onslide*<2>{\providecommand\step{1}\input{diagrams/backtrace/scanstack.tex}}%
      \onslide*<3>{\providecommand\step{2}\input{diagrams/backtrace/scanstack.tex}}%
      \onslide*<4>{\providecommand\step{3}\input{diagrams/backtrace/scanstack.tex}}%
      \onslide*<5>{\providecommand\step{4}\input{diagrams/backtrace/scanstack.tex}}%
      \onslide*<6>{\providecommand\step{5}\input{diagrams/backtrace/scanstack.tex}}%
    \end{column}
  \end{columns}
\end{frame}

% Duplicate of previous-previous slide
\begin{frame}{Backtraces}{Continuing on \funcname{caml\_stash\_backtrace}}
  \begin{columns}[c]
    \begin{column}{0.5\textwidth}
      \minipage[c][0.75\textheight][s]{\columnwidth}
      \begin{itemize}
          \color{gray}
        \item A C function provided by the runtime (specific to native code)
        \item It scans the stack, between the stack pointer at the raising point up to the next trap, in order to collect the names of the detected function frames
        \item It uses the same technique and information as the Garbage Collector (GC) uses to scan the values live on the stack
      \end{itemize}
      \begin{itemize}
        \item For each function frame detected, store a pointer to the \typename{frame\_descr} in the \localname{domain\_state->backtrace\_buffer} array, effectively constructing the backtrace
      \end{itemize}
      \onslide<2->{
        \begin{itemize}
            \smallskip
          \item Constructed backtrace:
            \funcname{definitely\_raise},
            \onslide<3->{\funcname{probably\_raise}}
        \end{itemize}
      }
      \vfill
      \mintocaml[firstline=9,lastline=13,highlightlines=12]{../src/backtrace/backtrace.ml}
      \endminipage
    \end{column}
    \begin{column}{0.5\textwidth}
      \raggedleft
      \onslide*<1>{\listStashBacktrace[highlightlines={114-115}]{}}
      \onslide*<2>{\providecommand\step{3}\input{diagrams/backtrace/backtrace.tex}}%
      \onslide*<3>{\providecommand\step{4}\input{diagrams/backtrace/backtrace.tex}}%
    \end{column}
  \end{columns}
\end{frame}

\begin{frame}{Backtraces}{Back to \funcname{caml\_raise\_exn}}
  \begin{columns}[c]
    \begin{column}{0.5\textwidth}
      \minipage[c][0.66\textheight][s]{\columnwidth}
      \begin{itemize}\color{gray}
        \item Checks wheteher exceptions backtrace should be recorded, by checking the \funcarg{domain\_state->backtrace\_active} value
        \item Switches to the C stack to be able to execute C code
        \item Calls \funcname{caml\_stash\_backtrace}, to collect the backtrace up to the next trap
      \end{itemize}
      \onslide*<2->{
        \begin{itemize}
          \item<2-> Executes the exception handler in \funcname{probably\_raise} with \funcname{RESTORE\_EXN\_HANDLER\_OCAML}
            \begin{itemize}
              \item Set \regname{RSP} to \localname{domain\_state->exn\_handler} value
              \item Restore \localname{domain\_state->exn\_handler} from trap
              \item Jump to the handler code with \funcname{ret}
            \end{itemize}
        \end{itemize}
      }
      \vfill
      \onslide*<1>{\mintocaml[firstline=9,lastline=13,highlightlines=12]{../src/backtrace/backtrace.ml}}
      \onslide*<2->{\mintocaml[firstline=15,lastline=21,highlightlines={19-21}]{../src/backtrace/backtrace.ml}}
      \endminipage
    \end{column}
    \begin{column}{0.5\textwidth}
      \centering
      \onslide*<1>{
        \mintc[firstline=190,lastline=192]{../ocaml/runtime/amd64.S}
        \mintc[firstline=234,lastline=237]{../ocaml/runtime/amd64.S}
      }
      \onslide*<2>{
        \mintc[firstline=190,lastline=192,highlightlines={191-192}]{../ocaml/runtime/amd64.S}
        \mintc[firstline=234,lastline=237,highlightlines={235-237}]{../ocaml/runtime/amd64.S}
      }
      \onslide*<1>{\listCamlRaiseExn[highlightlines=720]{}}
      \onslide*<2>{\listCamlRaiseExn[highlightlines=722]{}}
      \onslide*<3->{\providecommand\step{5}\input{diagrams/backtrace/backtrace.tex}}
    \end{column}
  \end{columns}
\end{frame}

\begin{frame}{Backtraces}{\funcname{caml\_probably\_raise}'s handler doesn't catch \typename{ExnC}}
  \begin{columns}[c]
    \begin{column}{0.45\textwidth}
      \minipage[c][0.75\textheight][s]{\columnwidth}
      \begin{itemize}
        \item<1-> \funcname{match \dots with}\footnotemark pattern matching can also match exceptions
        \item<1-> The CMM of \funcname{probably\_raise} shows that this \funcname{match \dots with} is implemented with a pattern matching and a \funcname{try \dots with}
        \item<1-> But this one only matches on \typename{ExnA} exception
        \item<2-> Since the pattern matching fails to catch the exception, \funcname{reraise} is called with our current \typename{ExnC} exception
        \item<3-> Disassembly shows that \funcname{reraise} is actually a call to \funcname{caml\_reraise\_exn}, which is also an assembly function provided by the runtime
      \end{itemize}
      \vfill
      \mintocaml[firstline=15,lastline=21,highlightlines={18-21}]{../src/backtrace/backtrace.ml}
      \endminipage
    \end{column}
    \begin{column}{0.55\textwidth}
      \centering
      \onslide*<1>{\mintcmm[highlightlines={10-18}]{../src/backtrace/camlBacktrace\_\_probably\_raise\_351.cmm}}
      \onslide*<2>{\listcmm[highlightlines=18]{../src/backtrace/camlBacktrace\_\_probably\_raise_351.cmm}{CMM of probably\_raise}}
      \onslide*<3>{\mintobjdump[firstline=5,lastline=35,highlightlines=31]{../src/backtrace/camlBacktrace\_\_probably\_raise_351.objdump}}
    \end{column}
  \end{columns}
  \only<1>{\footnotetext[1]{https://blog.janestreet.com/pattern-matching-and-exception-handling-unite/}}
\end{frame}

\newcommand\listCamlReraiseExn[1][]{
  \listasm[firstline=727,lastline=734,#1]{../ocaml/runtime/amd64.S}{runtime/amd64.S:727}}

\begin{frame}{Backtraces}{Introducing \funcname{caml\_reraise\_exn}}
  \begin{columns}[c]
    \begin{column}{0.5\textwidth}
      \minipage[c][0.66\textheight][s]{\columnwidth}
      \begin{itemize}
        \item<1-> The exception have to be reraised to the next handler
        \item<2-> First, checks if the backtrace should be collected with \funcarg{domain\_state->backtrace\_active}
        \only<3>{
        \begin{itemize}
            \item If not, behaves like a \funcname{raise\_notrace} with \funcname{RESTORE\_EXN\_HANDLER\_OCAML}
        \end{itemize}
        }
        \item<4-> Jumps into \funcname{caml\_raise\_exn}
        \item<5-> Calls \funcname{caml\_stash\_backtrace} again, to scan the next part of the stack: from the trap just pop-ed to the next trap
      \end{itemize}
      \vfill
      \mintocaml[firstline=15,lastline=21,highlightlines={18-21}]{../src/backtrace/backtrace.ml}
      \endminipage
    \end{column}
    \begin{column}{0.5\textwidth}
      \raggedleft
      \onslide*<1>{\listCamlReraiseExn{}}
      \onslide*<2>{\listCamlReraiseExn[highlightlines=729]{}}
      \onslide*<3>{
        \mintc[firstline=190,lastline=192,highlightlines={191-192}]{../ocaml/runtime/amd64.S}
        \mintc[firstline=234,lastline=237,highlightlines={235-237}]{../ocaml/runtime/amd64.S}
        \medskip
        \listCamlReraiseExn[highlightlines=731]{}
      }
      \onslide*<4-5>{
        % TODO refactor with \listCamlReraiseExn
        \mintasm[firstline=727,lastline=734,highlightlines=730]{../ocaml/runtime/amd64.S}
        \medskip
      }
      \onslide*<4>{\listCamlRaiseExn[highlightlines=711]{}}
      \onslide*<5>{\listCamlRaiseExn[highlightlines=720]{}}
    \end{column}
  \end{columns}
\end{frame}

\begin{frame}{Backtraces}{Backtrace collection continues}
  \begin{columns}[c]
    \begin{column}{0.55\textwidth}
      \minipage[c][0.75\textheight][s]{\columnwidth}
      \begin{itemize}
        \item<1-> \funcname{caml\_stash\_backtrace} scans the older part of the stack, up to the next trap
        \item<6-> Controls jumps to the nested exception handler in \funcname{maybe\_raise}, which matches only on \typename{ExnB}
        \item<7-> \funcname{caml\_reraise\_exn} is called again, backtrace collection resumes with \funcname{caml\_stash\_backtrace}
      \end{itemize}
      \medskip
      \begin{itemize}
        \item<2-> Constructed backtrace:
        \begin{itemize}
          \item <2-> \funcname{definitely\_raise}
          \item <2-> \funcname{probably\_raise}
          \item <3-> \funcname{probably\_raise} (re-raise)
          \item <4-> \funcname{maybe\_raise}
          \item <5-> \funcname{main}
          \item <8-> \funcname{main} (re-raise)
        \end{itemize}
      \end{itemize}
      \vfill
      \onslide*<1-5>{\mintocaml[firstline=15,lastline=21]{../src/backtrace/backtrace.ml}}
      \onslide*<6>{\mintocaml[firstline=28,lastline=34,highlightlines=31]{../src/backtrace/backtrace.ml}}
      \onslide*<7->{\mintocaml[firstline=28,lastline=34,highlightlines=33]{../src/backtrace/backtrace.ml}}
      \endminipage
    \end{column}
    \begin{column}{0.45\textwidth}
      \raggedleft
      \onslide*<1>{\providecommand\step{6}\input{diagrams/backtrace/backtrace.tex}}%
      \onslide*<2>{\providecommand\step{6}\input{diagrams/backtrace/backtrace.tex}}%
      \onslide*<3>{\providecommand\step{7}\input{diagrams/backtrace/backtrace.tex}}%
      \onslide*<4>{\providecommand\step{8}\input{diagrams/backtrace/backtrace.tex}}%
      \onslide*<5>{\providecommand\step{9}\input{diagrams/backtrace/backtrace.tex}}%
      \onslide*<6>{\providecommand\step{10}\input{diagrams/backtrace/backtrace.tex}}%
      \onslide*<7>{\providecommand\step{11}\input{diagrams/backtrace/backtrace.tex}}%
      \onslide*<8>{\providecommand\step{12}\input{diagrams/backtrace/backtrace.tex}}%
      \onslide*<9>{\providecommand\step{13}\input{diagrams/backtrace/backtrace.tex}}%
    \end{column}
  \end{columns}
\end{frame}

\begin{frame}{Backtraces}{Printing the backtrace}
  \begin{columns}[c]
    \begin{column}{0.45\textwidth}
      \minipage[c][0.66\textheight][s]{\columnwidth}
      \begin{itemize}
        \item<1-> The exception backtrace can now be printed to \funcarg{stdout} with \funcname{Printexc.print_backtrace}
        \item<2-> First the exception backtrace is retrieved into an OCaml array with \funcname{get_raw_backtrace }
        \item<3-> Which is implemented as the \funcname{caml_get_exception_raw_backtrace} C function in the runtime
        \item<4-> And finally printed with \funcname{print_exception_backtrace}
      \end{itemize}
      \vfill
      \mintocaml[firstline=28,lastline=34,highlightlines=33]{../src/backtrace/backtrace.ml}
      \endminipage
    \end{column}
    \begin{column}{0.55\textwidth}
      \onslide*<1,2>{
        \mintocaml[firstline=100,lastline=101]{../ocaml/stdlib/printexc.ml}
      }
      \only<1>{
        \listocaml[firstline=170,lastline=172]{../ocaml/stdlib/printexc.ml}{stdlib/printexc.ml:170}
      }
      \only<2>{
        \listocaml[firstline=170,lastline=172,highlightlines=172]{../ocaml/stdlib/printexc.ml}{stdlib/printexc.ml:170}
      }
      \only<3>{
        \mintc[firstline=154,lastline=156]{../ocaml/runtime/backtrace.c}
        \listc[firstline=171,lastline=192,highlightlines={184-187}]{../ocaml/runtime/backtrace.c}{runtime/backtrace.c:154}
      }
      \only<4>{
        \listocaml[firstline=155,lastline=165]{../ocaml/stdlib/printexc.ml}{stdlib/printexc.ml:55}
      }
    \end{column}
  \end{columns}
\end{frame}


\subsection{The multiple kinds of raise}
\frameSubsection{}{}

\begin{frame}{Backtraces}{When backtraces are not needed}
  \begin{columns}[c]
    \begin{column}{0.45\textwidth}
      \minipage[c][0.66\textheight][s]{\columnwidth}
      \begin{itemize}
        \item<1-> What if the backtrace for an exception is never needed?
        \item<2-> It's possible to raise the exception explicitly with \funcname{raise\_notrace} to make raising as cheap as possible
        \item<3-> An example of use in the \funcname{dynlink} library of the OCaml compiler
      \end{itemize}
      \vfill
      \onslide<3->{
      \listocaml[firstline=205,lastline=209,highlightlines=207]{../ocaml/otherlibs/dynlink/dynlink\_compilerlibs/misc.ml}{otherlibs/dynlink/dynlink\_compilerlibs/misc.ml:205}
      }
      \endminipage
    \end{column}
    \begin{column}{0.55\textwidth}
    \only<4>{
      \mintcmm[highlightlines=3]{../src/notrace/camlNotrace\_\_fun_347.cmm}
      \listcmm{../src/notrace/camlNotrace\_\_all\_somes\_327.cmm}{all\_somes}
    }
    \onslide*<5->{
      \listobjdump[highlightlines={6-9}]{../src/notrace/camlNotrace\_\_fun_347.objdump}{anonymous function from \funcname{all\_somes}}
    }
    \end{column}
  \end{columns}
\end{frame}

\begin{frame}{Backtraces}{Summary table of the multiple kinds of raise}
  \begin{itemize}
    \item When debugging information is enabled and if the backtrace of an exception isn't needed, it's possible to raise with \funcname{raise\_notrace} for performance
    \item When debugging information is \emph{not} enabled, the compiler optimises \funcname{raise} calls to inline assembly (same effect as using \funcname{raise\_notrace})
  \end{itemize}
  \bigskip
  \centering
  \begin{tabular}{| l | c | c | c | c |}
    \hline
    \parbox{10em}{Debug information \\ (\funcarg{-g} flag)}  & \multicolumn{2}{|c|}{Disabled}                                     & \multicolumn{2}{|c|}{Enabled} \\
    \hline
    Raise kind                                               & \funcname{raise} & \funcname{raise\_notrace}                       & \funcname{raise} & \funcname{raise\_notrace} \\
    \hline
    Compilation                                              & \parbox{8em}{inline assembly\\ (optimization)} & inline assembly   & \funcname{caml\_raise\_exn} & inline assembly \\
    \hline
  \end{tabular}
\end{frame}


%
% Takeaway #5
%
\subsection*{Takeaway \#5}
\frameSubsectionTakeaway{}
\againframe<5>{takeaway}
}%
      \onslide*<3>{\providecommand\step{7}%
% Backtraces
%
\subsection{One advantage of raising with debugging information}
\frameSubsection{}{}

\begin{frame}{Backtraces}{Debugging information \& source code}
  \begin{columns}[c]
    \begin{column}{0.5\textwidth}
      \begin{itemize}
        \item Raising an exception is fun and games, but how do we know where the exception originated? $\rightarrow$ With the backtrace associated with the exception!
        \item In order to get a backtrace from the point where an exception was raised, it's necessary to compile with debugging information enabled (\funcarg{-g} flag for the compiler)
        \item Same example as in the `Nested handlers' section
      \end{itemize}
    \end{column}
    \begin{column}{0.5\textwidth}
      \listocaml[firstline=9,lastline=34]{../src/backtrace/backtrace.ml}{backtrace/backtrace.ml}
    \end{column}
  \end{columns}
\end{frame}

\begin{frame}{Backtraces}{CMM and disassembly of \funcname{definitely\_raise}}
  \begin{columns}[c]
    \begin{column}{0.5\textwidth}
      \minipage[c][0.66\textheight][s]{\columnwidth}
      \begin{itemize}
        \item<1-> An effect of compiling with debugging information (\funcarg{-g}): the CMM no longer uses \funcname{raise\_notrace} to raise the exception but \funcname{raise} instead
        \item<2-> Disassembly shows that CMM \funcname{raise} is actually a call to \funcname{caml\_raise\_exn}, a function in assembler provided by the runtime
      \end{itemize}
      \vfill
      \mintocaml[firstline=9,lastline=13,highlightlines=12]{../src/backtrace/backtrace.ml}
      \endminipage
    \end{column}
    \begin{column}{0.5\textwidth}
      \onslide*<1>{
        \mintcmm[highlightlines=5]{../src/backtrace/camlBacktrace\_\_definitely\_raise\_347.cmm}
      }
      \onslide*<2>{
        \mintobjdump[firstline=2,highlightlines=14]{../src/backtrace/camlBacktrace\_\_definitely\_raise\_347.objdump}
      }
    \end{column}
  \end{columns}
\end{frame}

\begin{frame}{Backtraces}{Introducing \funcname{caml\_raise\_exn}}
  \begin{columns}[c]
    \begin{column}{0.5\textwidth}
      \minipage[c][0.66\textheight][s]{\columnwidth}
      \onslide*<1>{
        \begin{itemize}
          \item Raising the exception is performed using \funcname{caml\_raise\_exn} which is
            \begin{itemize}
              \item An assembly function provided by the runtime
              \item Architecture specific
            \end{itemize}
          \item Let's see what it does differently from \funcname{raise\_notrace}\dots
          \item We are about to raise \typename{ExnC} with \funcname{caml\_raise\_exn} from \funcname{definitely\_raise}
        \end{itemize}
      }
      \onslide*<2>{
        \mintobjdump[firstline=2,highlightlines=14]{../src/backtrace/camlBacktrace\_\_definitely\_raise\_347.objdump}
      }
      \vfill
      \mintocaml[firstline=9,lastline=13,highlightlines=12]{../src/backtrace/backtrace.ml}
      \endminipage
    \end{column}
    \begin{column}{0.5\textwidth}
      \centering
      \providecommand\step{1}\input{diagrams/backtrace/backtrace.tex}
    \end{column}
  \end{columns}
\end{frame}

\begin{frame}{Backtraces}{But before that, we need more fields from \typename{caml\_domain\_state}}
  \begin{columns}
    \begin{column}{0.5\textwidth}
      \begin{itemize}
        \item Remember \typename{caml\_domain\_state} the per-domain runtime state?
        \item Some other members are of interest to us now\dots
        \item \localname{backtrace\_active} is used to know if the runtime should collect backtraces
        \item \localname{backtrace\_active} can be set by
          \begin{itemize}
            \item The \funcarg{b} flag in the \funcname{OCAMLRUNPARAM} environment variable
            \item \mintinline{ocaml}{Printexc.record_backtrace : bool -> unit} \footnotemark
          \end{itemize}
      \end{itemize}
    \end{column}
    \begin{column}{0.5\textwidth}
      \begin{listing}
        \mintc[highlightlines={9-11}]{src/ocaml/domain\_state\_backtrace.h}
        \caption{runtime/caml/domain\_state.h}
      \end{listing}
    \end{column}
  \end{columns}
  \footnotetext[1]{https://v2.ocaml.org/api/Printexc.html}
\end{frame}

\newcommand\listCamlRaiseExn[1][]{
  \listasm[firstline=702,lastline=725,#1]{../ocaml/runtime/amd64.S}{runtime/amd64.S:702}}

\begin{frame}{Backtraces}{Walk through \funcname{caml\_raise\_exn}}
  \begin{columns}[T]
    \begin{column}{0.5\textwidth}
      \begin{itemize}
        \item<1-> First checks whether exceptions backtrace should be recorded
        \item<1-> By checking the \funcarg{domain\_state->backtrace\_active} value
        \item<2-> If not, acts as \funcname{raise\_notrace} with \funcname{RESTORE\_EXN\_HANDLER\_OCAML}
          \begin{itemize}
            \item Set \regname{RSP} to \localname{domain\_state->exn\_handler} value
            \item Restore \localname{domain\_state->exn\_handler} from trap
            \item Jump with \funcname{ret} (same effect as using \funcname{pop} and \funcname{jmp})
          \end{itemize}
        \item<3-> Otherwise, switches to the C stack to be able to execute C code
        \item<4-> Calls \funcname{caml\_stash\_backtrace}, a C function provided by the runtime\ignorespaces
          \onslide<6->{, with
            \begin{itemize}
              \item \funcarg{exn}: exception value
              \item \funcarg{pc}: code address of the raising point
              \item \funcarg{sp}: stack address of the raising function
              \item \funcarg{trapsp}: stack address of the next handler
            \end{itemize}
          }
      \end{itemize}
      \bigskip
      \mintocaml[firstline=9,lastline=13,highlightlines=12]{../src/backtrace/backtrace.ml}
    \end{column}
    \begin{column}{0.5\textwidth}
      \raggedleft
      \onslide*<1,3-4>{
        \mintc[firstline=190,lastline=192]{../ocaml/runtime/amd64.S}
        \mintc[firstline=234,lastline=237]{../ocaml/runtime/amd64.S}
      }
      \onslide*<2>{
        \mintc[firstline=190,lastline=192,highlightlines={191-192}]{../ocaml/runtime/amd64.S}
        \mintc[firstline=234,lastline=237,highlightlines={235-237}]{../ocaml/runtime/amd64.S}
      }
      \onslide*<1>{\listCamlRaiseExn[highlightlines=705]{}}%
      \onslide*<2>{\listCamlRaiseExn[highlightlines={707-708}]{}}%
      \onslide*<3>{\listCamlRaiseExn[highlightlines={712-714}]{}}%
      \onslide*<4>{\listCamlRaiseExn[highlightlines={715-720}]{}}%
      \onslide*<5>{\providecommand\step{1}\input{diagrams/backtrace/backtrace.tex}}%
      \onslide*<6>{\providecommand\step{2}\input{diagrams/backtrace/backtrace.tex}}%
    \end{column}
  \end{columns}
\end{frame}

\newcommand\listStashBacktrace[1][]{
  \listc[firstline=91,lastline=120,#1]{../ocaml/runtime/backtrace\_nat.c}{runtime/backtrace\_nat.c:91}}

\begin{frame}{Backtraces}{Introducing \funcname{caml\_stash\_backtrace}}
  \begin{columns}[c]
    \begin{column}{0.5\textwidth}
      \minipage[c][0.66\textheight][s]{\columnwidth}
      \begin{itemize}
        \item<1-> A C function provided by the runtime (specific to native code)
        \item<2-> It scans the stack, between the stack pointer at the raising point up to the next trap, in order to collect the names of the detected function frames
          \onslide*<2>{
            \begin{itemize}
              \item And more information, like line number, column number...
            \end{itemize}
          }
        \item<3-> It uses the same technique and information as the Garbage Collector (GC) uses to scan the values live on the stack
          \onslide*<4>{
            \begin{itemize}
              \item \typename{frame\_descr} are at the core of stack unwinding
            \end{itemize}
          }
      \end{itemize}
      \vfill
      \mintocaml[firstline=9,lastline=13,highlightlines=12]{../src/backtrace/backtrace.ml}
      \endminipage
    \end{column}
    \begin{column}{0.5\textwidth}
      \onslide*<1>{\listStashBacktrace[highlightlines=91]{}}
      \onslide*<2>{\listStashBacktrace[highlightlines={91,117-118}]{}}
      \onslide*<3->{\listStashBacktrace[highlightlines={94,106,109-110}]{}}
    \end{column}
  \end{columns}
\end{frame}

\newcommand\listFrameDescr[1][]{
  \listocaml[firstline=111,lastline=124,#1]{../ocaml/asmcomp/emitaux.ml}{asmcomp/emitaux.ml:111}}
\newcommand\listDebugInfo[1][]{
  \listocaml[firstline=38,lastline=49,#1]{../ocaml/lambda/debuginfo.mli}{lambda/debuginfo.mli:38}}

\begin{frame}{Backtraces}{\typename{frame\_descr} and \typename{frame\_debuginfo} in a nutshell}
  \begin{columns}[c]
    \begin{column}{0.5\textwidth}
      \begin{itemize}
        \item<1-> For every return address in an OCaml program, there is a associated \typename{frame\_descr}
        \item<2-> A \typename{frame\_descr} specifies
        \begin{itemize}
          \item<2-> The size of the function stack frame
          \item<3-> Where live values are on the stack (so that the GC can mark them as live and prevent from being swept)
        \end{itemize}
        \item<4-> When compiled with debug information, \typename{frame\_descr} will be linked to a \typename{frame\_debuginfo}
        \begin{itemize}
          \item<5-> Contains information about the position in the source code (file name, line and column number)
        \end{itemize}
      \end{itemize}
    \end{column}
    \begin{column}{0.5\textwidth}
        \onslide*<1>{\listFrameDescr[highlightlines={118-122}]{}}
        \onslide*<2>{\listFrameDescr[highlightlines={120}]{}}
        \onslide*<3>{\listFrameDescr[highlightlines={121}]{}}
        \onslide*<4->{\listFrameDescr[highlightlines={122}]{}}
        \medskip
        \onslide*<1-3>{\listDebugInfo{}}
        \onslide*<4>{\listDebugInfo[highlightlines={38,49}]{}}
        \onslide*<5>{\listDebugInfo[highlightlines={39-46}]{}}
    \end{column}
  \end{columns}
\end{frame}

\begin{frame}{Backtraces}{Scanning the stack with \typename{frame\_descr}}
  \begin{columns}[c]
    \begin{column}{0.5\textwidth}
      \begin{itemize}
        \item Given the return address of the code that raises the exception by calling \funcname{caml\_raise\_exn} (\funcarg{pc} argument of \funcname{caml\_stash\_backtrace})
        \item And the position of the stack pointer (\funcarg{sp} argument of \funcname{caml\_stash\_backtrace})
        \item The frame size given by \typename{frame\_descr}.\funcarg{frame\_size}, allows to find the end of the previous frame
        \item And the return address of the caller of this function
        \bigskip
        \onslide<3->{
        \item Note that traps (here the reddish \funcname{ExnA}, \funcname{ExnB} and \funcname{ExnC} blocks) belong to the frame that installed them, and so the frame size changes when traps are removed
        }
      \end{itemize}
    \end{column}
    \begin{column}{0.5\textwidth}
      \raggedleft
      \onslide*<1>{\providecommand\step{2}\input{diagrams/backtrace/backtrace.tex}}%
      \onslide*<2>{\providecommand\step{1}\input{diagrams/backtrace/scanstack.tex}}%
      \onslide*<3>{\providecommand\step{2}\input{diagrams/backtrace/scanstack.tex}}%
      \onslide*<4>{\providecommand\step{3}\input{diagrams/backtrace/scanstack.tex}}%
      \onslide*<5>{\providecommand\step{4}\input{diagrams/backtrace/scanstack.tex}}%
      \onslide*<6>{\providecommand\step{5}\input{diagrams/backtrace/scanstack.tex}}%
    \end{column}
  \end{columns}
\end{frame}

% Duplicate of previous-previous slide
\begin{frame}{Backtraces}{Continuing on \funcname{caml\_stash\_backtrace}}
  \begin{columns}[c]
    \begin{column}{0.5\textwidth}
      \minipage[c][0.75\textheight][s]{\columnwidth}
      \begin{itemize}
          \color{gray}
        \item A C function provided by the runtime (specific to native code)
        \item It scans the stack, between the stack pointer at the raising point up to the next trap, in order to collect the names of the detected function frames
        \item It uses the same technique and information as the Garbage Collector (GC) uses to scan the values live on the stack
      \end{itemize}
      \begin{itemize}
        \item For each function frame detected, store a pointer to the \typename{frame\_descr} in the \localname{domain\_state->backtrace\_buffer} array, effectively constructing the backtrace
      \end{itemize}
      \onslide<2->{
        \begin{itemize}
            \smallskip
          \item Constructed backtrace:
            \funcname{definitely\_raise},
            \onslide<3->{\funcname{probably\_raise}}
        \end{itemize}
      }
      \vfill
      \mintocaml[firstline=9,lastline=13,highlightlines=12]{../src/backtrace/backtrace.ml}
      \endminipage
    \end{column}
    \begin{column}{0.5\textwidth}
      \raggedleft
      \onslide*<1>{\listStashBacktrace[highlightlines={114-115}]{}}
      \onslide*<2>{\providecommand\step{3}\input{diagrams/backtrace/backtrace.tex}}%
      \onslide*<3>{\providecommand\step{4}\input{diagrams/backtrace/backtrace.tex}}%
    \end{column}
  \end{columns}
\end{frame}

\begin{frame}{Backtraces}{Back to \funcname{caml\_raise\_exn}}
  \begin{columns}[c]
    \begin{column}{0.5\textwidth}
      \minipage[c][0.66\textheight][s]{\columnwidth}
      \begin{itemize}\color{gray}
        \item Checks wheteher exceptions backtrace should be recorded, by checking the \funcarg{domain\_state->backtrace\_active} value
        \item Switches to the C stack to be able to execute C code
        \item Calls \funcname{caml\_stash\_backtrace}, to collect the backtrace up to the next trap
      \end{itemize}
      \onslide*<2->{
        \begin{itemize}
          \item<2-> Executes the exception handler in \funcname{probably\_raise} with \funcname{RESTORE\_EXN\_HANDLER\_OCAML}
            \begin{itemize}
              \item Set \regname{RSP} to \localname{domain\_state->exn\_handler} value
              \item Restore \localname{domain\_state->exn\_handler} from trap
              \item Jump to the handler code with \funcname{ret}
            \end{itemize}
        \end{itemize}
      }
      \vfill
      \onslide*<1>{\mintocaml[firstline=9,lastline=13,highlightlines=12]{../src/backtrace/backtrace.ml}}
      \onslide*<2->{\mintocaml[firstline=15,lastline=21,highlightlines={19-21}]{../src/backtrace/backtrace.ml}}
      \endminipage
    \end{column}
    \begin{column}{0.5\textwidth}
      \centering
      \onslide*<1>{
        \mintc[firstline=190,lastline=192]{../ocaml/runtime/amd64.S}
        \mintc[firstline=234,lastline=237]{../ocaml/runtime/amd64.S}
      }
      \onslide*<2>{
        \mintc[firstline=190,lastline=192,highlightlines={191-192}]{../ocaml/runtime/amd64.S}
        \mintc[firstline=234,lastline=237,highlightlines={235-237}]{../ocaml/runtime/amd64.S}
      }
      \onslide*<1>{\listCamlRaiseExn[highlightlines=720]{}}
      \onslide*<2>{\listCamlRaiseExn[highlightlines=722]{}}
      \onslide*<3->{\providecommand\step{5}\input{diagrams/backtrace/backtrace.tex}}
    \end{column}
  \end{columns}
\end{frame}

\begin{frame}{Backtraces}{\funcname{caml\_probably\_raise}'s handler doesn't catch \typename{ExnC}}
  \begin{columns}[c]
    \begin{column}{0.45\textwidth}
      \minipage[c][0.75\textheight][s]{\columnwidth}
      \begin{itemize}
        \item<1-> \funcname{match \dots with}\footnotemark pattern matching can also match exceptions
        \item<1-> The CMM of \funcname{probably\_raise} shows that this \funcname{match \dots with} is implemented with a pattern matching and a \funcname{try \dots with}
        \item<1-> But this one only matches on \typename{ExnA} exception
        \item<2-> Since the pattern matching fails to catch the exception, \funcname{reraise} is called with our current \typename{ExnC} exception
        \item<3-> Disassembly shows that \funcname{reraise} is actually a call to \funcname{caml\_reraise\_exn}, which is also an assembly function provided by the runtime
      \end{itemize}
      \vfill
      \mintocaml[firstline=15,lastline=21,highlightlines={18-21}]{../src/backtrace/backtrace.ml}
      \endminipage
    \end{column}
    \begin{column}{0.55\textwidth}
      \centering
      \onslide*<1>{\mintcmm[highlightlines={10-18}]{../src/backtrace/camlBacktrace\_\_probably\_raise\_351.cmm}}
      \onslide*<2>{\listcmm[highlightlines=18]{../src/backtrace/camlBacktrace\_\_probably\_raise_351.cmm}{CMM of probably\_raise}}
      \onslide*<3>{\mintobjdump[firstline=5,lastline=35,highlightlines=31]{../src/backtrace/camlBacktrace\_\_probably\_raise_351.objdump}}
    \end{column}
  \end{columns}
  \only<1>{\footnotetext[1]{https://blog.janestreet.com/pattern-matching-and-exception-handling-unite/}}
\end{frame}

\newcommand\listCamlReraiseExn[1][]{
  \listasm[firstline=727,lastline=734,#1]{../ocaml/runtime/amd64.S}{runtime/amd64.S:727}}

\begin{frame}{Backtraces}{Introducing \funcname{caml\_reraise\_exn}}
  \begin{columns}[c]
    \begin{column}{0.5\textwidth}
      \minipage[c][0.66\textheight][s]{\columnwidth}
      \begin{itemize}
        \item<1-> The exception have to be reraised to the next handler
        \item<2-> First, checks if the backtrace should be collected with \funcarg{domain\_state->backtrace\_active}
        \only<3>{
        \begin{itemize}
            \item If not, behaves like a \funcname{raise\_notrace} with \funcname{RESTORE\_EXN\_HANDLER\_OCAML}
        \end{itemize}
        }
        \item<4-> Jumps into \funcname{caml\_raise\_exn}
        \item<5-> Calls \funcname{caml\_stash\_backtrace} again, to scan the next part of the stack: from the trap just pop-ed to the next trap
      \end{itemize}
      \vfill
      \mintocaml[firstline=15,lastline=21,highlightlines={18-21}]{../src/backtrace/backtrace.ml}
      \endminipage
    \end{column}
    \begin{column}{0.5\textwidth}
      \raggedleft
      \onslide*<1>{\listCamlReraiseExn{}}
      \onslide*<2>{\listCamlReraiseExn[highlightlines=729]{}}
      \onslide*<3>{
        \mintc[firstline=190,lastline=192,highlightlines={191-192}]{../ocaml/runtime/amd64.S}
        \mintc[firstline=234,lastline=237,highlightlines={235-237}]{../ocaml/runtime/amd64.S}
        \medskip
        \listCamlReraiseExn[highlightlines=731]{}
      }
      \onslide*<4-5>{
        % TODO refactor with \listCamlReraiseExn
        \mintasm[firstline=727,lastline=734,highlightlines=730]{../ocaml/runtime/amd64.S}
        \medskip
      }
      \onslide*<4>{\listCamlRaiseExn[highlightlines=711]{}}
      \onslide*<5>{\listCamlRaiseExn[highlightlines=720]{}}
    \end{column}
  \end{columns}
\end{frame}

\begin{frame}{Backtraces}{Backtrace collection continues}
  \begin{columns}[c]
    \begin{column}{0.55\textwidth}
      \minipage[c][0.75\textheight][s]{\columnwidth}
      \begin{itemize}
        \item<1-> \funcname{caml\_stash\_backtrace} scans the older part of the stack, up to the next trap
        \item<6-> Controls jumps to the nested exception handler in \funcname{maybe\_raise}, which matches only on \typename{ExnB}
        \item<7-> \funcname{caml\_reraise\_exn} is called again, backtrace collection resumes with \funcname{caml\_stash\_backtrace}
      \end{itemize}
      \medskip
      \begin{itemize}
        \item<2-> Constructed backtrace:
        \begin{itemize}
          \item <2-> \funcname{definitely\_raise}
          \item <2-> \funcname{probably\_raise}
          \item <3-> \funcname{probably\_raise} (re-raise)
          \item <4-> \funcname{maybe\_raise}
          \item <5-> \funcname{main}
          \item <8-> \funcname{main} (re-raise)
        \end{itemize}
      \end{itemize}
      \vfill
      \onslide*<1-5>{\mintocaml[firstline=15,lastline=21]{../src/backtrace/backtrace.ml}}
      \onslide*<6>{\mintocaml[firstline=28,lastline=34,highlightlines=31]{../src/backtrace/backtrace.ml}}
      \onslide*<7->{\mintocaml[firstline=28,lastline=34,highlightlines=33]{../src/backtrace/backtrace.ml}}
      \endminipage
    \end{column}
    \begin{column}{0.45\textwidth}
      \raggedleft
      \onslide*<1>{\providecommand\step{6}\input{diagrams/backtrace/backtrace.tex}}%
      \onslide*<2>{\providecommand\step{6}\input{diagrams/backtrace/backtrace.tex}}%
      \onslide*<3>{\providecommand\step{7}\input{diagrams/backtrace/backtrace.tex}}%
      \onslide*<4>{\providecommand\step{8}\input{diagrams/backtrace/backtrace.tex}}%
      \onslide*<5>{\providecommand\step{9}\input{diagrams/backtrace/backtrace.tex}}%
      \onslide*<6>{\providecommand\step{10}\input{diagrams/backtrace/backtrace.tex}}%
      \onslide*<7>{\providecommand\step{11}\input{diagrams/backtrace/backtrace.tex}}%
      \onslide*<8>{\providecommand\step{12}\input{diagrams/backtrace/backtrace.tex}}%
      \onslide*<9>{\providecommand\step{13}\input{diagrams/backtrace/backtrace.tex}}%
    \end{column}
  \end{columns}
\end{frame}

\begin{frame}{Backtraces}{Printing the backtrace}
  \begin{columns}[c]
    \begin{column}{0.45\textwidth}
      \minipage[c][0.66\textheight][s]{\columnwidth}
      \begin{itemize}
        \item<1-> The exception backtrace can now be printed to \funcarg{stdout} with \funcname{Printexc.print_backtrace}
        \item<2-> First the exception backtrace is retrieved into an OCaml array with \funcname{get_raw_backtrace }
        \item<3-> Which is implemented as the \funcname{caml_get_exception_raw_backtrace} C function in the runtime
        \item<4-> And finally printed with \funcname{print_exception_backtrace}
      \end{itemize}
      \vfill
      \mintocaml[firstline=28,lastline=34,highlightlines=33]{../src/backtrace/backtrace.ml}
      \endminipage
    \end{column}
    \begin{column}{0.55\textwidth}
      \onslide*<1,2>{
        \mintocaml[firstline=100,lastline=101]{../ocaml/stdlib/printexc.ml}
      }
      \only<1>{
        \listocaml[firstline=170,lastline=172]{../ocaml/stdlib/printexc.ml}{stdlib/printexc.ml:170}
      }
      \only<2>{
        \listocaml[firstline=170,lastline=172,highlightlines=172]{../ocaml/stdlib/printexc.ml}{stdlib/printexc.ml:170}
      }
      \only<3>{
        \mintc[firstline=154,lastline=156]{../ocaml/runtime/backtrace.c}
        \listc[firstline=171,lastline=192,highlightlines={184-187}]{../ocaml/runtime/backtrace.c}{runtime/backtrace.c:154}
      }
      \only<4>{
        \listocaml[firstline=155,lastline=165]{../ocaml/stdlib/printexc.ml}{stdlib/printexc.ml:55}
      }
    \end{column}
  \end{columns}
\end{frame}


\subsection{The multiple kinds of raise}
\frameSubsection{}{}

\begin{frame}{Backtraces}{When backtraces are not needed}
  \begin{columns}[c]
    \begin{column}{0.45\textwidth}
      \minipage[c][0.66\textheight][s]{\columnwidth}
      \begin{itemize}
        \item<1-> What if the backtrace for an exception is never needed?
        \item<2-> It's possible to raise the exception explicitly with \funcname{raise\_notrace} to make raising as cheap as possible
        \item<3-> An example of use in the \funcname{dynlink} library of the OCaml compiler
      \end{itemize}
      \vfill
      \onslide<3->{
      \listocaml[firstline=205,lastline=209,highlightlines=207]{../ocaml/otherlibs/dynlink/dynlink\_compilerlibs/misc.ml}{otherlibs/dynlink/dynlink\_compilerlibs/misc.ml:205}
      }
      \endminipage
    \end{column}
    \begin{column}{0.55\textwidth}
    \only<4>{
      \mintcmm[highlightlines=3]{../src/notrace/camlNotrace\_\_fun_347.cmm}
      \listcmm{../src/notrace/camlNotrace\_\_all\_somes\_327.cmm}{all\_somes}
    }
    \onslide*<5->{
      \listobjdump[highlightlines={6-9}]{../src/notrace/camlNotrace\_\_fun_347.objdump}{anonymous function from \funcname{all\_somes}}
    }
    \end{column}
  \end{columns}
\end{frame}

\begin{frame}{Backtraces}{Summary table of the multiple kinds of raise}
  \begin{itemize}
    \item When debugging information is enabled and if the backtrace of an exception isn't needed, it's possible to raise with \funcname{raise\_notrace} for performance
    \item When debugging information is \emph{not} enabled, the compiler optimises \funcname{raise} calls to inline assembly (same effect as using \funcname{raise\_notrace})
  \end{itemize}
  \bigskip
  \centering
  \begin{tabular}{| l | c | c | c | c |}
    \hline
    \parbox{10em}{Debug information \\ (\funcarg{-g} flag)}  & \multicolumn{2}{|c|}{Disabled}                                     & \multicolumn{2}{|c|}{Enabled} \\
    \hline
    Raise kind                                               & \funcname{raise} & \funcname{raise\_notrace}                       & \funcname{raise} & \funcname{raise\_notrace} \\
    \hline
    Compilation                                              & \parbox{8em}{inline assembly\\ (optimization)} & inline assembly   & \funcname{caml\_raise\_exn} & inline assembly \\
    \hline
  \end{tabular}
\end{frame}


%
% Takeaway #5
%
\subsection*{Takeaway \#5}
\frameSubsectionTakeaway{}
\againframe<5>{takeaway}
}%
      \onslide*<4>{\providecommand\step{8}%
% Backtraces
%
\subsection{One advantage of raising with debugging information}
\frameSubsection{}{}

\begin{frame}{Backtraces}{Debugging information \& source code}
  \begin{columns}[c]
    \begin{column}{0.5\textwidth}
      \begin{itemize}
        \item Raising an exception is fun and games, but how do we know where the exception originated? $\rightarrow$ With the backtrace associated with the exception!
        \item In order to get a backtrace from the point where an exception was raised, it's necessary to compile with debugging information enabled (\funcarg{-g} flag for the compiler)
        \item Same example as in the `Nested handlers' section
      \end{itemize}
    \end{column}
    \begin{column}{0.5\textwidth}
      \listocaml[firstline=9,lastline=34]{../src/backtrace/backtrace.ml}{backtrace/backtrace.ml}
    \end{column}
  \end{columns}
\end{frame}

\begin{frame}{Backtraces}{CMM and disassembly of \funcname{definitely\_raise}}
  \begin{columns}[c]
    \begin{column}{0.5\textwidth}
      \minipage[c][0.66\textheight][s]{\columnwidth}
      \begin{itemize}
        \item<1-> An effect of compiling with debugging information (\funcarg{-g}): the CMM no longer uses \funcname{raise\_notrace} to raise the exception but \funcname{raise} instead
        \item<2-> Disassembly shows that CMM \funcname{raise} is actually a call to \funcname{caml\_raise\_exn}, a function in assembler provided by the runtime
      \end{itemize}
      \vfill
      \mintocaml[firstline=9,lastline=13,highlightlines=12]{../src/backtrace/backtrace.ml}
      \endminipage
    \end{column}
    \begin{column}{0.5\textwidth}
      \onslide*<1>{
        \mintcmm[highlightlines=5]{../src/backtrace/camlBacktrace\_\_definitely\_raise\_347.cmm}
      }
      \onslide*<2>{
        \mintobjdump[firstline=2,highlightlines=14]{../src/backtrace/camlBacktrace\_\_definitely\_raise\_347.objdump}
      }
    \end{column}
  \end{columns}
\end{frame}

\begin{frame}{Backtraces}{Introducing \funcname{caml\_raise\_exn}}
  \begin{columns}[c]
    \begin{column}{0.5\textwidth}
      \minipage[c][0.66\textheight][s]{\columnwidth}
      \onslide*<1>{
        \begin{itemize}
          \item Raising the exception is performed using \funcname{caml\_raise\_exn} which is
            \begin{itemize}
              \item An assembly function provided by the runtime
              \item Architecture specific
            \end{itemize}
          \item Let's see what it does differently from \funcname{raise\_notrace}\dots
          \item We are about to raise \typename{ExnC} with \funcname{caml\_raise\_exn} from \funcname{definitely\_raise}
        \end{itemize}
      }
      \onslide*<2>{
        \mintobjdump[firstline=2,highlightlines=14]{../src/backtrace/camlBacktrace\_\_definitely\_raise\_347.objdump}
      }
      \vfill
      \mintocaml[firstline=9,lastline=13,highlightlines=12]{../src/backtrace/backtrace.ml}
      \endminipage
    \end{column}
    \begin{column}{0.5\textwidth}
      \centering
      \providecommand\step{1}\input{diagrams/backtrace/backtrace.tex}
    \end{column}
  \end{columns}
\end{frame}

\begin{frame}{Backtraces}{But before that, we need more fields from \typename{caml\_domain\_state}}
  \begin{columns}
    \begin{column}{0.5\textwidth}
      \begin{itemize}
        \item Remember \typename{caml\_domain\_state} the per-domain runtime state?
        \item Some other members are of interest to us now\dots
        \item \localname{backtrace\_active} is used to know if the runtime should collect backtraces
        \item \localname{backtrace\_active} can be set by
          \begin{itemize}
            \item The \funcarg{b} flag in the \funcname{OCAMLRUNPARAM} environment variable
            \item \mintinline{ocaml}{Printexc.record_backtrace : bool -> unit} \footnotemark
          \end{itemize}
      \end{itemize}
    \end{column}
    \begin{column}{0.5\textwidth}
      \begin{listing}
        \mintc[highlightlines={9-11}]{src/ocaml/domain\_state\_backtrace.h}
        \caption{runtime/caml/domain\_state.h}
      \end{listing}
    \end{column}
  \end{columns}
  \footnotetext[1]{https://v2.ocaml.org/api/Printexc.html}
\end{frame}

\newcommand\listCamlRaiseExn[1][]{
  \listasm[firstline=702,lastline=725,#1]{../ocaml/runtime/amd64.S}{runtime/amd64.S:702}}

\begin{frame}{Backtraces}{Walk through \funcname{caml\_raise\_exn}}
  \begin{columns}[T]
    \begin{column}{0.5\textwidth}
      \begin{itemize}
        \item<1-> First checks whether exceptions backtrace should be recorded
        \item<1-> By checking the \funcarg{domain\_state->backtrace\_active} value
        \item<2-> If not, acts as \funcname{raise\_notrace} with \funcname{RESTORE\_EXN\_HANDLER\_OCAML}
          \begin{itemize}
            \item Set \regname{RSP} to \localname{domain\_state->exn\_handler} value
            \item Restore \localname{domain\_state->exn\_handler} from trap
            \item Jump with \funcname{ret} (same effect as using \funcname{pop} and \funcname{jmp})
          \end{itemize}
        \item<3-> Otherwise, switches to the C stack to be able to execute C code
        \item<4-> Calls \funcname{caml\_stash\_backtrace}, a C function provided by the runtime\ignorespaces
          \onslide<6->{, with
            \begin{itemize}
              \item \funcarg{exn}: exception value
              \item \funcarg{pc}: code address of the raising point
              \item \funcarg{sp}: stack address of the raising function
              \item \funcarg{trapsp}: stack address of the next handler
            \end{itemize}
          }
      \end{itemize}
      \bigskip
      \mintocaml[firstline=9,lastline=13,highlightlines=12]{../src/backtrace/backtrace.ml}
    \end{column}
    \begin{column}{0.5\textwidth}
      \raggedleft
      \onslide*<1,3-4>{
        \mintc[firstline=190,lastline=192]{../ocaml/runtime/amd64.S}
        \mintc[firstline=234,lastline=237]{../ocaml/runtime/amd64.S}
      }
      \onslide*<2>{
        \mintc[firstline=190,lastline=192,highlightlines={191-192}]{../ocaml/runtime/amd64.S}
        \mintc[firstline=234,lastline=237,highlightlines={235-237}]{../ocaml/runtime/amd64.S}
      }
      \onslide*<1>{\listCamlRaiseExn[highlightlines=705]{}}%
      \onslide*<2>{\listCamlRaiseExn[highlightlines={707-708}]{}}%
      \onslide*<3>{\listCamlRaiseExn[highlightlines={712-714}]{}}%
      \onslide*<4>{\listCamlRaiseExn[highlightlines={715-720}]{}}%
      \onslide*<5>{\providecommand\step{1}\input{diagrams/backtrace/backtrace.tex}}%
      \onslide*<6>{\providecommand\step{2}\input{diagrams/backtrace/backtrace.tex}}%
    \end{column}
  \end{columns}
\end{frame}

\newcommand\listStashBacktrace[1][]{
  \listc[firstline=91,lastline=120,#1]{../ocaml/runtime/backtrace\_nat.c}{runtime/backtrace\_nat.c:91}}

\begin{frame}{Backtraces}{Introducing \funcname{caml\_stash\_backtrace}}
  \begin{columns}[c]
    \begin{column}{0.5\textwidth}
      \minipage[c][0.66\textheight][s]{\columnwidth}
      \begin{itemize}
        \item<1-> A C function provided by the runtime (specific to native code)
        \item<2-> It scans the stack, between the stack pointer at the raising point up to the next trap, in order to collect the names of the detected function frames
          \onslide*<2>{
            \begin{itemize}
              \item And more information, like line number, column number...
            \end{itemize}
          }
        \item<3-> It uses the same technique and information as the Garbage Collector (GC) uses to scan the values live on the stack
          \onslide*<4>{
            \begin{itemize}
              \item \typename{frame\_descr} are at the core of stack unwinding
            \end{itemize}
          }
      \end{itemize}
      \vfill
      \mintocaml[firstline=9,lastline=13,highlightlines=12]{../src/backtrace/backtrace.ml}
      \endminipage
    \end{column}
    \begin{column}{0.5\textwidth}
      \onslide*<1>{\listStashBacktrace[highlightlines=91]{}}
      \onslide*<2>{\listStashBacktrace[highlightlines={91,117-118}]{}}
      \onslide*<3->{\listStashBacktrace[highlightlines={94,106,109-110}]{}}
    \end{column}
  \end{columns}
\end{frame}

\newcommand\listFrameDescr[1][]{
  \listocaml[firstline=111,lastline=124,#1]{../ocaml/asmcomp/emitaux.ml}{asmcomp/emitaux.ml:111}}
\newcommand\listDebugInfo[1][]{
  \listocaml[firstline=38,lastline=49,#1]{../ocaml/lambda/debuginfo.mli}{lambda/debuginfo.mli:38}}

\begin{frame}{Backtraces}{\typename{frame\_descr} and \typename{frame\_debuginfo} in a nutshell}
  \begin{columns}[c]
    \begin{column}{0.5\textwidth}
      \begin{itemize}
        \item<1-> For every return address in an OCaml program, there is a associated \typename{frame\_descr}
        \item<2-> A \typename{frame\_descr} specifies
        \begin{itemize}
          \item<2-> The size of the function stack frame
          \item<3-> Where live values are on the stack (so that the GC can mark them as live and prevent from being swept)
        \end{itemize}
        \item<4-> When compiled with debug information, \typename{frame\_descr} will be linked to a \typename{frame\_debuginfo}
        \begin{itemize}
          \item<5-> Contains information about the position in the source code (file name, line and column number)
        \end{itemize}
      \end{itemize}
    \end{column}
    \begin{column}{0.5\textwidth}
        \onslide*<1>{\listFrameDescr[highlightlines={118-122}]{}}
        \onslide*<2>{\listFrameDescr[highlightlines={120}]{}}
        \onslide*<3>{\listFrameDescr[highlightlines={121}]{}}
        \onslide*<4->{\listFrameDescr[highlightlines={122}]{}}
        \medskip
        \onslide*<1-3>{\listDebugInfo{}}
        \onslide*<4>{\listDebugInfo[highlightlines={38,49}]{}}
        \onslide*<5>{\listDebugInfo[highlightlines={39-46}]{}}
    \end{column}
  \end{columns}
\end{frame}

\begin{frame}{Backtraces}{Scanning the stack with \typename{frame\_descr}}
  \begin{columns}[c]
    \begin{column}{0.5\textwidth}
      \begin{itemize}
        \item Given the return address of the code that raises the exception by calling \funcname{caml\_raise\_exn} (\funcarg{pc} argument of \funcname{caml\_stash\_backtrace})
        \item And the position of the stack pointer (\funcarg{sp} argument of \funcname{caml\_stash\_backtrace})
        \item The frame size given by \typename{frame\_descr}.\funcarg{frame\_size}, allows to find the end of the previous frame
        \item And the return address of the caller of this function
        \bigskip
        \onslide<3->{
        \item Note that traps (here the reddish \funcname{ExnA}, \funcname{ExnB} and \funcname{ExnC} blocks) belong to the frame that installed them, and so the frame size changes when traps are removed
        }
      \end{itemize}
    \end{column}
    \begin{column}{0.5\textwidth}
      \raggedleft
      \onslide*<1>{\providecommand\step{2}\input{diagrams/backtrace/backtrace.tex}}%
      \onslide*<2>{\providecommand\step{1}\input{diagrams/backtrace/scanstack.tex}}%
      \onslide*<3>{\providecommand\step{2}\input{diagrams/backtrace/scanstack.tex}}%
      \onslide*<4>{\providecommand\step{3}\input{diagrams/backtrace/scanstack.tex}}%
      \onslide*<5>{\providecommand\step{4}\input{diagrams/backtrace/scanstack.tex}}%
      \onslide*<6>{\providecommand\step{5}\input{diagrams/backtrace/scanstack.tex}}%
    \end{column}
  \end{columns}
\end{frame}

% Duplicate of previous-previous slide
\begin{frame}{Backtraces}{Continuing on \funcname{caml\_stash\_backtrace}}
  \begin{columns}[c]
    \begin{column}{0.5\textwidth}
      \minipage[c][0.75\textheight][s]{\columnwidth}
      \begin{itemize}
          \color{gray}
        \item A C function provided by the runtime (specific to native code)
        \item It scans the stack, between the stack pointer at the raising point up to the next trap, in order to collect the names of the detected function frames
        \item It uses the same technique and information as the Garbage Collector (GC) uses to scan the values live on the stack
      \end{itemize}
      \begin{itemize}
        \item For each function frame detected, store a pointer to the \typename{frame\_descr} in the \localname{domain\_state->backtrace\_buffer} array, effectively constructing the backtrace
      \end{itemize}
      \onslide<2->{
        \begin{itemize}
            \smallskip
          \item Constructed backtrace:
            \funcname{definitely\_raise},
            \onslide<3->{\funcname{probably\_raise}}
        \end{itemize}
      }
      \vfill
      \mintocaml[firstline=9,lastline=13,highlightlines=12]{../src/backtrace/backtrace.ml}
      \endminipage
    \end{column}
    \begin{column}{0.5\textwidth}
      \raggedleft
      \onslide*<1>{\listStashBacktrace[highlightlines={114-115}]{}}
      \onslide*<2>{\providecommand\step{3}\input{diagrams/backtrace/backtrace.tex}}%
      \onslide*<3>{\providecommand\step{4}\input{diagrams/backtrace/backtrace.tex}}%
    \end{column}
  \end{columns}
\end{frame}

\begin{frame}{Backtraces}{Back to \funcname{caml\_raise\_exn}}
  \begin{columns}[c]
    \begin{column}{0.5\textwidth}
      \minipage[c][0.66\textheight][s]{\columnwidth}
      \begin{itemize}\color{gray}
        \item Checks wheteher exceptions backtrace should be recorded, by checking the \funcarg{domain\_state->backtrace\_active} value
        \item Switches to the C stack to be able to execute C code
        \item Calls \funcname{caml\_stash\_backtrace}, to collect the backtrace up to the next trap
      \end{itemize}
      \onslide*<2->{
        \begin{itemize}
          \item<2-> Executes the exception handler in \funcname{probably\_raise} with \funcname{RESTORE\_EXN\_HANDLER\_OCAML}
            \begin{itemize}
              \item Set \regname{RSP} to \localname{domain\_state->exn\_handler} value
              \item Restore \localname{domain\_state->exn\_handler} from trap
              \item Jump to the handler code with \funcname{ret}
            \end{itemize}
        \end{itemize}
      }
      \vfill
      \onslide*<1>{\mintocaml[firstline=9,lastline=13,highlightlines=12]{../src/backtrace/backtrace.ml}}
      \onslide*<2->{\mintocaml[firstline=15,lastline=21,highlightlines={19-21}]{../src/backtrace/backtrace.ml}}
      \endminipage
    \end{column}
    \begin{column}{0.5\textwidth}
      \centering
      \onslide*<1>{
        \mintc[firstline=190,lastline=192]{../ocaml/runtime/amd64.S}
        \mintc[firstline=234,lastline=237]{../ocaml/runtime/amd64.S}
      }
      \onslide*<2>{
        \mintc[firstline=190,lastline=192,highlightlines={191-192}]{../ocaml/runtime/amd64.S}
        \mintc[firstline=234,lastline=237,highlightlines={235-237}]{../ocaml/runtime/amd64.S}
      }
      \onslide*<1>{\listCamlRaiseExn[highlightlines=720]{}}
      \onslide*<2>{\listCamlRaiseExn[highlightlines=722]{}}
      \onslide*<3->{\providecommand\step{5}\input{diagrams/backtrace/backtrace.tex}}
    \end{column}
  \end{columns}
\end{frame}

\begin{frame}{Backtraces}{\funcname{caml\_probably\_raise}'s handler doesn't catch \typename{ExnC}}
  \begin{columns}[c]
    \begin{column}{0.45\textwidth}
      \minipage[c][0.75\textheight][s]{\columnwidth}
      \begin{itemize}
        \item<1-> \funcname{match \dots with}\footnotemark pattern matching can also match exceptions
        \item<1-> The CMM of \funcname{probably\_raise} shows that this \funcname{match \dots with} is implemented with a pattern matching and a \funcname{try \dots with}
        \item<1-> But this one only matches on \typename{ExnA} exception
        \item<2-> Since the pattern matching fails to catch the exception, \funcname{reraise} is called with our current \typename{ExnC} exception
        \item<3-> Disassembly shows that \funcname{reraise} is actually a call to \funcname{caml\_reraise\_exn}, which is also an assembly function provided by the runtime
      \end{itemize}
      \vfill
      \mintocaml[firstline=15,lastline=21,highlightlines={18-21}]{../src/backtrace/backtrace.ml}
      \endminipage
    \end{column}
    \begin{column}{0.55\textwidth}
      \centering
      \onslide*<1>{\mintcmm[highlightlines={10-18}]{../src/backtrace/camlBacktrace\_\_probably\_raise\_351.cmm}}
      \onslide*<2>{\listcmm[highlightlines=18]{../src/backtrace/camlBacktrace\_\_probably\_raise_351.cmm}{CMM of probably\_raise}}
      \onslide*<3>{\mintobjdump[firstline=5,lastline=35,highlightlines=31]{../src/backtrace/camlBacktrace\_\_probably\_raise_351.objdump}}
    \end{column}
  \end{columns}
  \only<1>{\footnotetext[1]{https://blog.janestreet.com/pattern-matching-and-exception-handling-unite/}}
\end{frame}

\newcommand\listCamlReraiseExn[1][]{
  \listasm[firstline=727,lastline=734,#1]{../ocaml/runtime/amd64.S}{runtime/amd64.S:727}}

\begin{frame}{Backtraces}{Introducing \funcname{caml\_reraise\_exn}}
  \begin{columns}[c]
    \begin{column}{0.5\textwidth}
      \minipage[c][0.66\textheight][s]{\columnwidth}
      \begin{itemize}
        \item<1-> The exception have to be reraised to the next handler
        \item<2-> First, checks if the backtrace should be collected with \funcarg{domain\_state->backtrace\_active}
        \only<3>{
        \begin{itemize}
            \item If not, behaves like a \funcname{raise\_notrace} with \funcname{RESTORE\_EXN\_HANDLER\_OCAML}
        \end{itemize}
        }
        \item<4-> Jumps into \funcname{caml\_raise\_exn}
        \item<5-> Calls \funcname{caml\_stash\_backtrace} again, to scan the next part of the stack: from the trap just pop-ed to the next trap
      \end{itemize}
      \vfill
      \mintocaml[firstline=15,lastline=21,highlightlines={18-21}]{../src/backtrace/backtrace.ml}
      \endminipage
    \end{column}
    \begin{column}{0.5\textwidth}
      \raggedleft
      \onslide*<1>{\listCamlReraiseExn{}}
      \onslide*<2>{\listCamlReraiseExn[highlightlines=729]{}}
      \onslide*<3>{
        \mintc[firstline=190,lastline=192,highlightlines={191-192}]{../ocaml/runtime/amd64.S}
        \mintc[firstline=234,lastline=237,highlightlines={235-237}]{../ocaml/runtime/amd64.S}
        \medskip
        \listCamlReraiseExn[highlightlines=731]{}
      }
      \onslide*<4-5>{
        % TODO refactor with \listCamlReraiseExn
        \mintasm[firstline=727,lastline=734,highlightlines=730]{../ocaml/runtime/amd64.S}
        \medskip
      }
      \onslide*<4>{\listCamlRaiseExn[highlightlines=711]{}}
      \onslide*<5>{\listCamlRaiseExn[highlightlines=720]{}}
    \end{column}
  \end{columns}
\end{frame}

\begin{frame}{Backtraces}{Backtrace collection continues}
  \begin{columns}[c]
    \begin{column}{0.55\textwidth}
      \minipage[c][0.75\textheight][s]{\columnwidth}
      \begin{itemize}
        \item<1-> \funcname{caml\_stash\_backtrace} scans the older part of the stack, up to the next trap
        \item<6-> Controls jumps to the nested exception handler in \funcname{maybe\_raise}, which matches only on \typename{ExnB}
        \item<7-> \funcname{caml\_reraise\_exn} is called again, backtrace collection resumes with \funcname{caml\_stash\_backtrace}
      \end{itemize}
      \medskip
      \begin{itemize}
        \item<2-> Constructed backtrace:
        \begin{itemize}
          \item <2-> \funcname{definitely\_raise}
          \item <2-> \funcname{probably\_raise}
          \item <3-> \funcname{probably\_raise} (re-raise)
          \item <4-> \funcname{maybe\_raise}
          \item <5-> \funcname{main}
          \item <8-> \funcname{main} (re-raise)
        \end{itemize}
      \end{itemize}
      \vfill
      \onslide*<1-5>{\mintocaml[firstline=15,lastline=21]{../src/backtrace/backtrace.ml}}
      \onslide*<6>{\mintocaml[firstline=28,lastline=34,highlightlines=31]{../src/backtrace/backtrace.ml}}
      \onslide*<7->{\mintocaml[firstline=28,lastline=34,highlightlines=33]{../src/backtrace/backtrace.ml}}
      \endminipage
    \end{column}
    \begin{column}{0.45\textwidth}
      \raggedleft
      \onslide*<1>{\providecommand\step{6}\input{diagrams/backtrace/backtrace.tex}}%
      \onslide*<2>{\providecommand\step{6}\input{diagrams/backtrace/backtrace.tex}}%
      \onslide*<3>{\providecommand\step{7}\input{diagrams/backtrace/backtrace.tex}}%
      \onslide*<4>{\providecommand\step{8}\input{diagrams/backtrace/backtrace.tex}}%
      \onslide*<5>{\providecommand\step{9}\input{diagrams/backtrace/backtrace.tex}}%
      \onslide*<6>{\providecommand\step{10}\input{diagrams/backtrace/backtrace.tex}}%
      \onslide*<7>{\providecommand\step{11}\input{diagrams/backtrace/backtrace.tex}}%
      \onslide*<8>{\providecommand\step{12}\input{diagrams/backtrace/backtrace.tex}}%
      \onslide*<9>{\providecommand\step{13}\input{diagrams/backtrace/backtrace.tex}}%
    \end{column}
  \end{columns}
\end{frame}

\begin{frame}{Backtraces}{Printing the backtrace}
  \begin{columns}[c]
    \begin{column}{0.45\textwidth}
      \minipage[c][0.66\textheight][s]{\columnwidth}
      \begin{itemize}
        \item<1-> The exception backtrace can now be printed to \funcarg{stdout} with \funcname{Printexc.print_backtrace}
        \item<2-> First the exception backtrace is retrieved into an OCaml array with \funcname{get_raw_backtrace }
        \item<3-> Which is implemented as the \funcname{caml_get_exception_raw_backtrace} C function in the runtime
        \item<4-> And finally printed with \funcname{print_exception_backtrace}
      \end{itemize}
      \vfill
      \mintocaml[firstline=28,lastline=34,highlightlines=33]{../src/backtrace/backtrace.ml}
      \endminipage
    \end{column}
    \begin{column}{0.55\textwidth}
      \onslide*<1,2>{
        \mintocaml[firstline=100,lastline=101]{../ocaml/stdlib/printexc.ml}
      }
      \only<1>{
        \listocaml[firstline=170,lastline=172]{../ocaml/stdlib/printexc.ml}{stdlib/printexc.ml:170}
      }
      \only<2>{
        \listocaml[firstline=170,lastline=172,highlightlines=172]{../ocaml/stdlib/printexc.ml}{stdlib/printexc.ml:170}
      }
      \only<3>{
        \mintc[firstline=154,lastline=156]{../ocaml/runtime/backtrace.c}
        \listc[firstline=171,lastline=192,highlightlines={184-187}]{../ocaml/runtime/backtrace.c}{runtime/backtrace.c:154}
      }
      \only<4>{
        \listocaml[firstline=155,lastline=165]{../ocaml/stdlib/printexc.ml}{stdlib/printexc.ml:55}
      }
    \end{column}
  \end{columns}
\end{frame}


\subsection{The multiple kinds of raise}
\frameSubsection{}{}

\begin{frame}{Backtraces}{When backtraces are not needed}
  \begin{columns}[c]
    \begin{column}{0.45\textwidth}
      \minipage[c][0.66\textheight][s]{\columnwidth}
      \begin{itemize}
        \item<1-> What if the backtrace for an exception is never needed?
        \item<2-> It's possible to raise the exception explicitly with \funcname{raise\_notrace} to make raising as cheap as possible
        \item<3-> An example of use in the \funcname{dynlink} library of the OCaml compiler
      \end{itemize}
      \vfill
      \onslide<3->{
      \listocaml[firstline=205,lastline=209,highlightlines=207]{../ocaml/otherlibs/dynlink/dynlink\_compilerlibs/misc.ml}{otherlibs/dynlink/dynlink\_compilerlibs/misc.ml:205}
      }
      \endminipage
    \end{column}
    \begin{column}{0.55\textwidth}
    \only<4>{
      \mintcmm[highlightlines=3]{../src/notrace/camlNotrace\_\_fun_347.cmm}
      \listcmm{../src/notrace/camlNotrace\_\_all\_somes\_327.cmm}{all\_somes}
    }
    \onslide*<5->{
      \listobjdump[highlightlines={6-9}]{../src/notrace/camlNotrace\_\_fun_347.objdump}{anonymous function from \funcname{all\_somes}}
    }
    \end{column}
  \end{columns}
\end{frame}

\begin{frame}{Backtraces}{Summary table of the multiple kinds of raise}
  \begin{itemize}
    \item When debugging information is enabled and if the backtrace of an exception isn't needed, it's possible to raise with \funcname{raise\_notrace} for performance
    \item When debugging information is \emph{not} enabled, the compiler optimises \funcname{raise} calls to inline assembly (same effect as using \funcname{raise\_notrace})
  \end{itemize}
  \bigskip
  \centering
  \begin{tabular}{| l | c | c | c | c |}
    \hline
    \parbox{10em}{Debug information \\ (\funcarg{-g} flag)}  & \multicolumn{2}{|c|}{Disabled}                                     & \multicolumn{2}{|c|}{Enabled} \\
    \hline
    Raise kind                                               & \funcname{raise} & \funcname{raise\_notrace}                       & \funcname{raise} & \funcname{raise\_notrace} \\
    \hline
    Compilation                                              & \parbox{8em}{inline assembly\\ (optimization)} & inline assembly   & \funcname{caml\_raise\_exn} & inline assembly \\
    \hline
  \end{tabular}
\end{frame}


%
% Takeaway #5
%
\subsection*{Takeaway \#5}
\frameSubsectionTakeaway{}
\againframe<5>{takeaway}
}%
      \onslide*<5>{\providecommand\step{9}%
% Backtraces
%
\subsection{One advantage of raising with debugging information}
\frameSubsection{}{}

\begin{frame}{Backtraces}{Debugging information \& source code}
  \begin{columns}[c]
    \begin{column}{0.5\textwidth}
      \begin{itemize}
        \item Raising an exception is fun and games, but how do we know where the exception originated? $\rightarrow$ With the backtrace associated with the exception!
        \item In order to get a backtrace from the point where an exception was raised, it's necessary to compile with debugging information enabled (\funcarg{-g} flag for the compiler)
        \item Same example as in the `Nested handlers' section
      \end{itemize}
    \end{column}
    \begin{column}{0.5\textwidth}
      \listocaml[firstline=9,lastline=34]{../src/backtrace/backtrace.ml}{backtrace/backtrace.ml}
    \end{column}
  \end{columns}
\end{frame}

\begin{frame}{Backtraces}{CMM and disassembly of \funcname{definitely\_raise}}
  \begin{columns}[c]
    \begin{column}{0.5\textwidth}
      \minipage[c][0.66\textheight][s]{\columnwidth}
      \begin{itemize}
        \item<1-> An effect of compiling with debugging information (\funcarg{-g}): the CMM no longer uses \funcname{raise\_notrace} to raise the exception but \funcname{raise} instead
        \item<2-> Disassembly shows that CMM \funcname{raise} is actually a call to \funcname{caml\_raise\_exn}, a function in assembler provided by the runtime
      \end{itemize}
      \vfill
      \mintocaml[firstline=9,lastline=13,highlightlines=12]{../src/backtrace/backtrace.ml}
      \endminipage
    \end{column}
    \begin{column}{0.5\textwidth}
      \onslide*<1>{
        \mintcmm[highlightlines=5]{../src/backtrace/camlBacktrace\_\_definitely\_raise\_347.cmm}
      }
      \onslide*<2>{
        \mintobjdump[firstline=2,highlightlines=14]{../src/backtrace/camlBacktrace\_\_definitely\_raise\_347.objdump}
      }
    \end{column}
  \end{columns}
\end{frame}

\begin{frame}{Backtraces}{Introducing \funcname{caml\_raise\_exn}}
  \begin{columns}[c]
    \begin{column}{0.5\textwidth}
      \minipage[c][0.66\textheight][s]{\columnwidth}
      \onslide*<1>{
        \begin{itemize}
          \item Raising the exception is performed using \funcname{caml\_raise\_exn} which is
            \begin{itemize}
              \item An assembly function provided by the runtime
              \item Architecture specific
            \end{itemize}
          \item Let's see what it does differently from \funcname{raise\_notrace}\dots
          \item We are about to raise \typename{ExnC} with \funcname{caml\_raise\_exn} from \funcname{definitely\_raise}
        \end{itemize}
      }
      \onslide*<2>{
        \mintobjdump[firstline=2,highlightlines=14]{../src/backtrace/camlBacktrace\_\_definitely\_raise\_347.objdump}
      }
      \vfill
      \mintocaml[firstline=9,lastline=13,highlightlines=12]{../src/backtrace/backtrace.ml}
      \endminipage
    \end{column}
    \begin{column}{0.5\textwidth}
      \centering
      \providecommand\step{1}\input{diagrams/backtrace/backtrace.tex}
    \end{column}
  \end{columns}
\end{frame}

\begin{frame}{Backtraces}{But before that, we need more fields from \typename{caml\_domain\_state}}
  \begin{columns}
    \begin{column}{0.5\textwidth}
      \begin{itemize}
        \item Remember \typename{caml\_domain\_state} the per-domain runtime state?
        \item Some other members are of interest to us now\dots
        \item \localname{backtrace\_active} is used to know if the runtime should collect backtraces
        \item \localname{backtrace\_active} can be set by
          \begin{itemize}
            \item The \funcarg{b} flag in the \funcname{OCAMLRUNPARAM} environment variable
            \item \mintinline{ocaml}{Printexc.record_backtrace : bool -> unit} \footnotemark
          \end{itemize}
      \end{itemize}
    \end{column}
    \begin{column}{0.5\textwidth}
      \begin{listing}
        \mintc[highlightlines={9-11}]{src/ocaml/domain\_state\_backtrace.h}
        \caption{runtime/caml/domain\_state.h}
      \end{listing}
    \end{column}
  \end{columns}
  \footnotetext[1]{https://v2.ocaml.org/api/Printexc.html}
\end{frame}

\newcommand\listCamlRaiseExn[1][]{
  \listasm[firstline=702,lastline=725,#1]{../ocaml/runtime/amd64.S}{runtime/amd64.S:702}}

\begin{frame}{Backtraces}{Walk through \funcname{caml\_raise\_exn}}
  \begin{columns}[T]
    \begin{column}{0.5\textwidth}
      \begin{itemize}
        \item<1-> First checks whether exceptions backtrace should be recorded
        \item<1-> By checking the \funcarg{domain\_state->backtrace\_active} value
        \item<2-> If not, acts as \funcname{raise\_notrace} with \funcname{RESTORE\_EXN\_HANDLER\_OCAML}
          \begin{itemize}
            \item Set \regname{RSP} to \localname{domain\_state->exn\_handler} value
            \item Restore \localname{domain\_state->exn\_handler} from trap
            \item Jump with \funcname{ret} (same effect as using \funcname{pop} and \funcname{jmp})
          \end{itemize}
        \item<3-> Otherwise, switches to the C stack to be able to execute C code
        \item<4-> Calls \funcname{caml\_stash\_backtrace}, a C function provided by the runtime\ignorespaces
          \onslide<6->{, with
            \begin{itemize}
              \item \funcarg{exn}: exception value
              \item \funcarg{pc}: code address of the raising point
              \item \funcarg{sp}: stack address of the raising function
              \item \funcarg{trapsp}: stack address of the next handler
            \end{itemize}
          }
      \end{itemize}
      \bigskip
      \mintocaml[firstline=9,lastline=13,highlightlines=12]{../src/backtrace/backtrace.ml}
    \end{column}
    \begin{column}{0.5\textwidth}
      \raggedleft
      \onslide*<1,3-4>{
        \mintc[firstline=190,lastline=192]{../ocaml/runtime/amd64.S}
        \mintc[firstline=234,lastline=237]{../ocaml/runtime/amd64.S}
      }
      \onslide*<2>{
        \mintc[firstline=190,lastline=192,highlightlines={191-192}]{../ocaml/runtime/amd64.S}
        \mintc[firstline=234,lastline=237,highlightlines={235-237}]{../ocaml/runtime/amd64.S}
      }
      \onslide*<1>{\listCamlRaiseExn[highlightlines=705]{}}%
      \onslide*<2>{\listCamlRaiseExn[highlightlines={707-708}]{}}%
      \onslide*<3>{\listCamlRaiseExn[highlightlines={712-714}]{}}%
      \onslide*<4>{\listCamlRaiseExn[highlightlines={715-720}]{}}%
      \onslide*<5>{\providecommand\step{1}\input{diagrams/backtrace/backtrace.tex}}%
      \onslide*<6>{\providecommand\step{2}\input{diagrams/backtrace/backtrace.tex}}%
    \end{column}
  \end{columns}
\end{frame}

\newcommand\listStashBacktrace[1][]{
  \listc[firstline=91,lastline=120,#1]{../ocaml/runtime/backtrace\_nat.c}{runtime/backtrace\_nat.c:91}}

\begin{frame}{Backtraces}{Introducing \funcname{caml\_stash\_backtrace}}
  \begin{columns}[c]
    \begin{column}{0.5\textwidth}
      \minipage[c][0.66\textheight][s]{\columnwidth}
      \begin{itemize}
        \item<1-> A C function provided by the runtime (specific to native code)
        \item<2-> It scans the stack, between the stack pointer at the raising point up to the next trap, in order to collect the names of the detected function frames
          \onslide*<2>{
            \begin{itemize}
              \item And more information, like line number, column number...
            \end{itemize}
          }
        \item<3-> It uses the same technique and information as the Garbage Collector (GC) uses to scan the values live on the stack
          \onslide*<4>{
            \begin{itemize}
              \item \typename{frame\_descr} are at the core of stack unwinding
            \end{itemize}
          }
      \end{itemize}
      \vfill
      \mintocaml[firstline=9,lastline=13,highlightlines=12]{../src/backtrace/backtrace.ml}
      \endminipage
    \end{column}
    \begin{column}{0.5\textwidth}
      \onslide*<1>{\listStashBacktrace[highlightlines=91]{}}
      \onslide*<2>{\listStashBacktrace[highlightlines={91,117-118}]{}}
      \onslide*<3->{\listStashBacktrace[highlightlines={94,106,109-110}]{}}
    \end{column}
  \end{columns}
\end{frame}

\newcommand\listFrameDescr[1][]{
  \listocaml[firstline=111,lastline=124,#1]{../ocaml/asmcomp/emitaux.ml}{asmcomp/emitaux.ml:111}}
\newcommand\listDebugInfo[1][]{
  \listocaml[firstline=38,lastline=49,#1]{../ocaml/lambda/debuginfo.mli}{lambda/debuginfo.mli:38}}

\begin{frame}{Backtraces}{\typename{frame\_descr} and \typename{frame\_debuginfo} in a nutshell}
  \begin{columns}[c]
    \begin{column}{0.5\textwidth}
      \begin{itemize}
        \item<1-> For every return address in an OCaml program, there is a associated \typename{frame\_descr}
        \item<2-> A \typename{frame\_descr} specifies
        \begin{itemize}
          \item<2-> The size of the function stack frame
          \item<3-> Where live values are on the stack (so that the GC can mark them as live and prevent from being swept)
        \end{itemize}
        \item<4-> When compiled with debug information, \typename{frame\_descr} will be linked to a \typename{frame\_debuginfo}
        \begin{itemize}
          \item<5-> Contains information about the position in the source code (file name, line and column number)
        \end{itemize}
      \end{itemize}
    \end{column}
    \begin{column}{0.5\textwidth}
        \onslide*<1>{\listFrameDescr[highlightlines={118-122}]{}}
        \onslide*<2>{\listFrameDescr[highlightlines={120}]{}}
        \onslide*<3>{\listFrameDescr[highlightlines={121}]{}}
        \onslide*<4->{\listFrameDescr[highlightlines={122}]{}}
        \medskip
        \onslide*<1-3>{\listDebugInfo{}}
        \onslide*<4>{\listDebugInfo[highlightlines={38,49}]{}}
        \onslide*<5>{\listDebugInfo[highlightlines={39-46}]{}}
    \end{column}
  \end{columns}
\end{frame}

\begin{frame}{Backtraces}{Scanning the stack with \typename{frame\_descr}}
  \begin{columns}[c]
    \begin{column}{0.5\textwidth}
      \begin{itemize}
        \item Given the return address of the code that raises the exception by calling \funcname{caml\_raise\_exn} (\funcarg{pc} argument of \funcname{caml\_stash\_backtrace})
        \item And the position of the stack pointer (\funcarg{sp} argument of \funcname{caml\_stash\_backtrace})
        \item The frame size given by \typename{frame\_descr}.\funcarg{frame\_size}, allows to find the end of the previous frame
        \item And the return address of the caller of this function
        \bigskip
        \onslide<3->{
        \item Note that traps (here the reddish \funcname{ExnA}, \funcname{ExnB} and \funcname{ExnC} blocks) belong to the frame that installed them, and so the frame size changes when traps are removed
        }
      \end{itemize}
    \end{column}
    \begin{column}{0.5\textwidth}
      \raggedleft
      \onslide*<1>{\providecommand\step{2}\input{diagrams/backtrace/backtrace.tex}}%
      \onslide*<2>{\providecommand\step{1}\input{diagrams/backtrace/scanstack.tex}}%
      \onslide*<3>{\providecommand\step{2}\input{diagrams/backtrace/scanstack.tex}}%
      \onslide*<4>{\providecommand\step{3}\input{diagrams/backtrace/scanstack.tex}}%
      \onslide*<5>{\providecommand\step{4}\input{diagrams/backtrace/scanstack.tex}}%
      \onslide*<6>{\providecommand\step{5}\input{diagrams/backtrace/scanstack.tex}}%
    \end{column}
  \end{columns}
\end{frame}

% Duplicate of previous-previous slide
\begin{frame}{Backtraces}{Continuing on \funcname{caml\_stash\_backtrace}}
  \begin{columns}[c]
    \begin{column}{0.5\textwidth}
      \minipage[c][0.75\textheight][s]{\columnwidth}
      \begin{itemize}
          \color{gray}
        \item A C function provided by the runtime (specific to native code)
        \item It scans the stack, between the stack pointer at the raising point up to the next trap, in order to collect the names of the detected function frames
        \item It uses the same technique and information as the Garbage Collector (GC) uses to scan the values live on the stack
      \end{itemize}
      \begin{itemize}
        \item For each function frame detected, store a pointer to the \typename{frame\_descr} in the \localname{domain\_state->backtrace\_buffer} array, effectively constructing the backtrace
      \end{itemize}
      \onslide<2->{
        \begin{itemize}
            \smallskip
          \item Constructed backtrace:
            \funcname{definitely\_raise},
            \onslide<3->{\funcname{probably\_raise}}
        \end{itemize}
      }
      \vfill
      \mintocaml[firstline=9,lastline=13,highlightlines=12]{../src/backtrace/backtrace.ml}
      \endminipage
    \end{column}
    \begin{column}{0.5\textwidth}
      \raggedleft
      \onslide*<1>{\listStashBacktrace[highlightlines={114-115}]{}}
      \onslide*<2>{\providecommand\step{3}\input{diagrams/backtrace/backtrace.tex}}%
      \onslide*<3>{\providecommand\step{4}\input{diagrams/backtrace/backtrace.tex}}%
    \end{column}
  \end{columns}
\end{frame}

\begin{frame}{Backtraces}{Back to \funcname{caml\_raise\_exn}}
  \begin{columns}[c]
    \begin{column}{0.5\textwidth}
      \minipage[c][0.66\textheight][s]{\columnwidth}
      \begin{itemize}\color{gray}
        \item Checks wheteher exceptions backtrace should be recorded, by checking the \funcarg{domain\_state->backtrace\_active} value
        \item Switches to the C stack to be able to execute C code
        \item Calls \funcname{caml\_stash\_backtrace}, to collect the backtrace up to the next trap
      \end{itemize}
      \onslide*<2->{
        \begin{itemize}
          \item<2-> Executes the exception handler in \funcname{probably\_raise} with \funcname{RESTORE\_EXN\_HANDLER\_OCAML}
            \begin{itemize}
              \item Set \regname{RSP} to \localname{domain\_state->exn\_handler} value
              \item Restore \localname{domain\_state->exn\_handler} from trap
              \item Jump to the handler code with \funcname{ret}
            \end{itemize}
        \end{itemize}
      }
      \vfill
      \onslide*<1>{\mintocaml[firstline=9,lastline=13,highlightlines=12]{../src/backtrace/backtrace.ml}}
      \onslide*<2->{\mintocaml[firstline=15,lastline=21,highlightlines={19-21}]{../src/backtrace/backtrace.ml}}
      \endminipage
    \end{column}
    \begin{column}{0.5\textwidth}
      \centering
      \onslide*<1>{
        \mintc[firstline=190,lastline=192]{../ocaml/runtime/amd64.S}
        \mintc[firstline=234,lastline=237]{../ocaml/runtime/amd64.S}
      }
      \onslide*<2>{
        \mintc[firstline=190,lastline=192,highlightlines={191-192}]{../ocaml/runtime/amd64.S}
        \mintc[firstline=234,lastline=237,highlightlines={235-237}]{../ocaml/runtime/amd64.S}
      }
      \onslide*<1>{\listCamlRaiseExn[highlightlines=720]{}}
      \onslide*<2>{\listCamlRaiseExn[highlightlines=722]{}}
      \onslide*<3->{\providecommand\step{5}\input{diagrams/backtrace/backtrace.tex}}
    \end{column}
  \end{columns}
\end{frame}

\begin{frame}{Backtraces}{\funcname{caml\_probably\_raise}'s handler doesn't catch \typename{ExnC}}
  \begin{columns}[c]
    \begin{column}{0.45\textwidth}
      \minipage[c][0.75\textheight][s]{\columnwidth}
      \begin{itemize}
        \item<1-> \funcname{match \dots with}\footnotemark pattern matching can also match exceptions
        \item<1-> The CMM of \funcname{probably\_raise} shows that this \funcname{match \dots with} is implemented with a pattern matching and a \funcname{try \dots with}
        \item<1-> But this one only matches on \typename{ExnA} exception
        \item<2-> Since the pattern matching fails to catch the exception, \funcname{reraise} is called with our current \typename{ExnC} exception
        \item<3-> Disassembly shows that \funcname{reraise} is actually a call to \funcname{caml\_reraise\_exn}, which is also an assembly function provided by the runtime
      \end{itemize}
      \vfill
      \mintocaml[firstline=15,lastline=21,highlightlines={18-21}]{../src/backtrace/backtrace.ml}
      \endminipage
    \end{column}
    \begin{column}{0.55\textwidth}
      \centering
      \onslide*<1>{\mintcmm[highlightlines={10-18}]{../src/backtrace/camlBacktrace\_\_probably\_raise\_351.cmm}}
      \onslide*<2>{\listcmm[highlightlines=18]{../src/backtrace/camlBacktrace\_\_probably\_raise_351.cmm}{CMM of probably\_raise}}
      \onslide*<3>{\mintobjdump[firstline=5,lastline=35,highlightlines=31]{../src/backtrace/camlBacktrace\_\_probably\_raise_351.objdump}}
    \end{column}
  \end{columns}
  \only<1>{\footnotetext[1]{https://blog.janestreet.com/pattern-matching-and-exception-handling-unite/}}
\end{frame}

\newcommand\listCamlReraiseExn[1][]{
  \listasm[firstline=727,lastline=734,#1]{../ocaml/runtime/amd64.S}{runtime/amd64.S:727}}

\begin{frame}{Backtraces}{Introducing \funcname{caml\_reraise\_exn}}
  \begin{columns}[c]
    \begin{column}{0.5\textwidth}
      \minipage[c][0.66\textheight][s]{\columnwidth}
      \begin{itemize}
        \item<1-> The exception have to be reraised to the next handler
        \item<2-> First, checks if the backtrace should be collected with \funcarg{domain\_state->backtrace\_active}
        \only<3>{
        \begin{itemize}
            \item If not, behaves like a \funcname{raise\_notrace} with \funcname{RESTORE\_EXN\_HANDLER\_OCAML}
        \end{itemize}
        }
        \item<4-> Jumps into \funcname{caml\_raise\_exn}
        \item<5-> Calls \funcname{caml\_stash\_backtrace} again, to scan the next part of the stack: from the trap just pop-ed to the next trap
      \end{itemize}
      \vfill
      \mintocaml[firstline=15,lastline=21,highlightlines={18-21}]{../src/backtrace/backtrace.ml}
      \endminipage
    \end{column}
    \begin{column}{0.5\textwidth}
      \raggedleft
      \onslide*<1>{\listCamlReraiseExn{}}
      \onslide*<2>{\listCamlReraiseExn[highlightlines=729]{}}
      \onslide*<3>{
        \mintc[firstline=190,lastline=192,highlightlines={191-192}]{../ocaml/runtime/amd64.S}
        \mintc[firstline=234,lastline=237,highlightlines={235-237}]{../ocaml/runtime/amd64.S}
        \medskip
        \listCamlReraiseExn[highlightlines=731]{}
      }
      \onslide*<4-5>{
        % TODO refactor with \listCamlReraiseExn
        \mintasm[firstline=727,lastline=734,highlightlines=730]{../ocaml/runtime/amd64.S}
        \medskip
      }
      \onslide*<4>{\listCamlRaiseExn[highlightlines=711]{}}
      \onslide*<5>{\listCamlRaiseExn[highlightlines=720]{}}
    \end{column}
  \end{columns}
\end{frame}

\begin{frame}{Backtraces}{Backtrace collection continues}
  \begin{columns}[c]
    \begin{column}{0.55\textwidth}
      \minipage[c][0.75\textheight][s]{\columnwidth}
      \begin{itemize}
        \item<1-> \funcname{caml\_stash\_backtrace} scans the older part of the stack, up to the next trap
        \item<6-> Controls jumps to the nested exception handler in \funcname{maybe\_raise}, which matches only on \typename{ExnB}
        \item<7-> \funcname{caml\_reraise\_exn} is called again, backtrace collection resumes with \funcname{caml\_stash\_backtrace}
      \end{itemize}
      \medskip
      \begin{itemize}
        \item<2-> Constructed backtrace:
        \begin{itemize}
          \item <2-> \funcname{definitely\_raise}
          \item <2-> \funcname{probably\_raise}
          \item <3-> \funcname{probably\_raise} (re-raise)
          \item <4-> \funcname{maybe\_raise}
          \item <5-> \funcname{main}
          \item <8-> \funcname{main} (re-raise)
        \end{itemize}
      \end{itemize}
      \vfill
      \onslide*<1-5>{\mintocaml[firstline=15,lastline=21]{../src/backtrace/backtrace.ml}}
      \onslide*<6>{\mintocaml[firstline=28,lastline=34,highlightlines=31]{../src/backtrace/backtrace.ml}}
      \onslide*<7->{\mintocaml[firstline=28,lastline=34,highlightlines=33]{../src/backtrace/backtrace.ml}}
      \endminipage
    \end{column}
    \begin{column}{0.45\textwidth}
      \raggedleft
      \onslide*<1>{\providecommand\step{6}\input{diagrams/backtrace/backtrace.tex}}%
      \onslide*<2>{\providecommand\step{6}\input{diagrams/backtrace/backtrace.tex}}%
      \onslide*<3>{\providecommand\step{7}\input{diagrams/backtrace/backtrace.tex}}%
      \onslide*<4>{\providecommand\step{8}\input{diagrams/backtrace/backtrace.tex}}%
      \onslide*<5>{\providecommand\step{9}\input{diagrams/backtrace/backtrace.tex}}%
      \onslide*<6>{\providecommand\step{10}\input{diagrams/backtrace/backtrace.tex}}%
      \onslide*<7>{\providecommand\step{11}\input{diagrams/backtrace/backtrace.tex}}%
      \onslide*<8>{\providecommand\step{12}\input{diagrams/backtrace/backtrace.tex}}%
      \onslide*<9>{\providecommand\step{13}\input{diagrams/backtrace/backtrace.tex}}%
    \end{column}
  \end{columns}
\end{frame}

\begin{frame}{Backtraces}{Printing the backtrace}
  \begin{columns}[c]
    \begin{column}{0.45\textwidth}
      \minipage[c][0.66\textheight][s]{\columnwidth}
      \begin{itemize}
        \item<1-> The exception backtrace can now be printed to \funcarg{stdout} with \funcname{Printexc.print_backtrace}
        \item<2-> First the exception backtrace is retrieved into an OCaml array with \funcname{get_raw_backtrace }
        \item<3-> Which is implemented as the \funcname{caml_get_exception_raw_backtrace} C function in the runtime
        \item<4-> And finally printed with \funcname{print_exception_backtrace}
      \end{itemize}
      \vfill
      \mintocaml[firstline=28,lastline=34,highlightlines=33]{../src/backtrace/backtrace.ml}
      \endminipage
    \end{column}
    \begin{column}{0.55\textwidth}
      \onslide*<1,2>{
        \mintocaml[firstline=100,lastline=101]{../ocaml/stdlib/printexc.ml}
      }
      \only<1>{
        \listocaml[firstline=170,lastline=172]{../ocaml/stdlib/printexc.ml}{stdlib/printexc.ml:170}
      }
      \only<2>{
        \listocaml[firstline=170,lastline=172,highlightlines=172]{../ocaml/stdlib/printexc.ml}{stdlib/printexc.ml:170}
      }
      \only<3>{
        \mintc[firstline=154,lastline=156]{../ocaml/runtime/backtrace.c}
        \listc[firstline=171,lastline=192,highlightlines={184-187}]{../ocaml/runtime/backtrace.c}{runtime/backtrace.c:154}
      }
      \only<4>{
        \listocaml[firstline=155,lastline=165]{../ocaml/stdlib/printexc.ml}{stdlib/printexc.ml:55}
      }
    \end{column}
  \end{columns}
\end{frame}


\subsection{The multiple kinds of raise}
\frameSubsection{}{}

\begin{frame}{Backtraces}{When backtraces are not needed}
  \begin{columns}[c]
    \begin{column}{0.45\textwidth}
      \minipage[c][0.66\textheight][s]{\columnwidth}
      \begin{itemize}
        \item<1-> What if the backtrace for an exception is never needed?
        \item<2-> It's possible to raise the exception explicitly with \funcname{raise\_notrace} to make raising as cheap as possible
        \item<3-> An example of use in the \funcname{dynlink} library of the OCaml compiler
      \end{itemize}
      \vfill
      \onslide<3->{
      \listocaml[firstline=205,lastline=209,highlightlines=207]{../ocaml/otherlibs/dynlink/dynlink\_compilerlibs/misc.ml}{otherlibs/dynlink/dynlink\_compilerlibs/misc.ml:205}
      }
      \endminipage
    \end{column}
    \begin{column}{0.55\textwidth}
    \only<4>{
      \mintcmm[highlightlines=3]{../src/notrace/camlNotrace\_\_fun_347.cmm}
      \listcmm{../src/notrace/camlNotrace\_\_all\_somes\_327.cmm}{all\_somes}
    }
    \onslide*<5->{
      \listobjdump[highlightlines={6-9}]{../src/notrace/camlNotrace\_\_fun_347.objdump}{anonymous function from \funcname{all\_somes}}
    }
    \end{column}
  \end{columns}
\end{frame}

\begin{frame}{Backtraces}{Summary table of the multiple kinds of raise}
  \begin{itemize}
    \item When debugging information is enabled and if the backtrace of an exception isn't needed, it's possible to raise with \funcname{raise\_notrace} for performance
    \item When debugging information is \emph{not} enabled, the compiler optimises \funcname{raise} calls to inline assembly (same effect as using \funcname{raise\_notrace})
  \end{itemize}
  \bigskip
  \centering
  \begin{tabular}{| l | c | c | c | c |}
    \hline
    \parbox{10em}{Debug information \\ (\funcarg{-g} flag)}  & \multicolumn{2}{|c|}{Disabled}                                     & \multicolumn{2}{|c|}{Enabled} \\
    \hline
    Raise kind                                               & \funcname{raise} & \funcname{raise\_notrace}                       & \funcname{raise} & \funcname{raise\_notrace} \\
    \hline
    Compilation                                              & \parbox{8em}{inline assembly\\ (optimization)} & inline assembly   & \funcname{caml\_raise\_exn} & inline assembly \\
    \hline
  \end{tabular}
\end{frame}


%
% Takeaway #5
%
\subsection*{Takeaway \#5}
\frameSubsectionTakeaway{}
\againframe<5>{takeaway}
}%
      \onslide*<6>{\providecommand\step{10}%
% Backtraces
%
\subsection{One advantage of raising with debugging information}
\frameSubsection{}{}

\begin{frame}{Backtraces}{Debugging information \& source code}
  \begin{columns}[c]
    \begin{column}{0.5\textwidth}
      \begin{itemize}
        \item Raising an exception is fun and games, but how do we know where the exception originated? $\rightarrow$ With the backtrace associated with the exception!
        \item In order to get a backtrace from the point where an exception was raised, it's necessary to compile with debugging information enabled (\funcarg{-g} flag for the compiler)
        \item Same example as in the `Nested handlers' section
      \end{itemize}
    \end{column}
    \begin{column}{0.5\textwidth}
      \listocaml[firstline=9,lastline=34]{../src/backtrace/backtrace.ml}{backtrace/backtrace.ml}
    \end{column}
  \end{columns}
\end{frame}

\begin{frame}{Backtraces}{CMM and disassembly of \funcname{definitely\_raise}}
  \begin{columns}[c]
    \begin{column}{0.5\textwidth}
      \minipage[c][0.66\textheight][s]{\columnwidth}
      \begin{itemize}
        \item<1-> An effect of compiling with debugging information (\funcarg{-g}): the CMM no longer uses \funcname{raise\_notrace} to raise the exception but \funcname{raise} instead
        \item<2-> Disassembly shows that CMM \funcname{raise} is actually a call to \funcname{caml\_raise\_exn}, a function in assembler provided by the runtime
      \end{itemize}
      \vfill
      \mintocaml[firstline=9,lastline=13,highlightlines=12]{../src/backtrace/backtrace.ml}
      \endminipage
    \end{column}
    \begin{column}{0.5\textwidth}
      \onslide*<1>{
        \mintcmm[highlightlines=5]{../src/backtrace/camlBacktrace\_\_definitely\_raise\_347.cmm}
      }
      \onslide*<2>{
        \mintobjdump[firstline=2,highlightlines=14]{../src/backtrace/camlBacktrace\_\_definitely\_raise\_347.objdump}
      }
    \end{column}
  \end{columns}
\end{frame}

\begin{frame}{Backtraces}{Introducing \funcname{caml\_raise\_exn}}
  \begin{columns}[c]
    \begin{column}{0.5\textwidth}
      \minipage[c][0.66\textheight][s]{\columnwidth}
      \onslide*<1>{
        \begin{itemize}
          \item Raising the exception is performed using \funcname{caml\_raise\_exn} which is
            \begin{itemize}
              \item An assembly function provided by the runtime
              \item Architecture specific
            \end{itemize}
          \item Let's see what it does differently from \funcname{raise\_notrace}\dots
          \item We are about to raise \typename{ExnC} with \funcname{caml\_raise\_exn} from \funcname{definitely\_raise}
        \end{itemize}
      }
      \onslide*<2>{
        \mintobjdump[firstline=2,highlightlines=14]{../src/backtrace/camlBacktrace\_\_definitely\_raise\_347.objdump}
      }
      \vfill
      \mintocaml[firstline=9,lastline=13,highlightlines=12]{../src/backtrace/backtrace.ml}
      \endminipage
    \end{column}
    \begin{column}{0.5\textwidth}
      \centering
      \providecommand\step{1}\input{diagrams/backtrace/backtrace.tex}
    \end{column}
  \end{columns}
\end{frame}

\begin{frame}{Backtraces}{But before that, we need more fields from \typename{caml\_domain\_state}}
  \begin{columns}
    \begin{column}{0.5\textwidth}
      \begin{itemize}
        \item Remember \typename{caml\_domain\_state} the per-domain runtime state?
        \item Some other members are of interest to us now\dots
        \item \localname{backtrace\_active} is used to know if the runtime should collect backtraces
        \item \localname{backtrace\_active} can be set by
          \begin{itemize}
            \item The \funcarg{b} flag in the \funcname{OCAMLRUNPARAM} environment variable
            \item \mintinline{ocaml}{Printexc.record_backtrace : bool -> unit} \footnotemark
          \end{itemize}
      \end{itemize}
    \end{column}
    \begin{column}{0.5\textwidth}
      \begin{listing}
        \mintc[highlightlines={9-11}]{src/ocaml/domain\_state\_backtrace.h}
        \caption{runtime/caml/domain\_state.h}
      \end{listing}
    \end{column}
  \end{columns}
  \footnotetext[1]{https://v2.ocaml.org/api/Printexc.html}
\end{frame}

\newcommand\listCamlRaiseExn[1][]{
  \listasm[firstline=702,lastline=725,#1]{../ocaml/runtime/amd64.S}{runtime/amd64.S:702}}

\begin{frame}{Backtraces}{Walk through \funcname{caml\_raise\_exn}}
  \begin{columns}[T]
    \begin{column}{0.5\textwidth}
      \begin{itemize}
        \item<1-> First checks whether exceptions backtrace should be recorded
        \item<1-> By checking the \funcarg{domain\_state->backtrace\_active} value
        \item<2-> If not, acts as \funcname{raise\_notrace} with \funcname{RESTORE\_EXN\_HANDLER\_OCAML}
          \begin{itemize}
            \item Set \regname{RSP} to \localname{domain\_state->exn\_handler} value
            \item Restore \localname{domain\_state->exn\_handler} from trap
            \item Jump with \funcname{ret} (same effect as using \funcname{pop} and \funcname{jmp})
          \end{itemize}
        \item<3-> Otherwise, switches to the C stack to be able to execute C code
        \item<4-> Calls \funcname{caml\_stash\_backtrace}, a C function provided by the runtime\ignorespaces
          \onslide<6->{, with
            \begin{itemize}
              \item \funcarg{exn}: exception value
              \item \funcarg{pc}: code address of the raising point
              \item \funcarg{sp}: stack address of the raising function
              \item \funcarg{trapsp}: stack address of the next handler
            \end{itemize}
          }
      \end{itemize}
      \bigskip
      \mintocaml[firstline=9,lastline=13,highlightlines=12]{../src/backtrace/backtrace.ml}
    \end{column}
    \begin{column}{0.5\textwidth}
      \raggedleft
      \onslide*<1,3-4>{
        \mintc[firstline=190,lastline=192]{../ocaml/runtime/amd64.S}
        \mintc[firstline=234,lastline=237]{../ocaml/runtime/amd64.S}
      }
      \onslide*<2>{
        \mintc[firstline=190,lastline=192,highlightlines={191-192}]{../ocaml/runtime/amd64.S}
        \mintc[firstline=234,lastline=237,highlightlines={235-237}]{../ocaml/runtime/amd64.S}
      }
      \onslide*<1>{\listCamlRaiseExn[highlightlines=705]{}}%
      \onslide*<2>{\listCamlRaiseExn[highlightlines={707-708}]{}}%
      \onslide*<3>{\listCamlRaiseExn[highlightlines={712-714}]{}}%
      \onslide*<4>{\listCamlRaiseExn[highlightlines={715-720}]{}}%
      \onslide*<5>{\providecommand\step{1}\input{diagrams/backtrace/backtrace.tex}}%
      \onslide*<6>{\providecommand\step{2}\input{diagrams/backtrace/backtrace.tex}}%
    \end{column}
  \end{columns}
\end{frame}

\newcommand\listStashBacktrace[1][]{
  \listc[firstline=91,lastline=120,#1]{../ocaml/runtime/backtrace\_nat.c}{runtime/backtrace\_nat.c:91}}

\begin{frame}{Backtraces}{Introducing \funcname{caml\_stash\_backtrace}}
  \begin{columns}[c]
    \begin{column}{0.5\textwidth}
      \minipage[c][0.66\textheight][s]{\columnwidth}
      \begin{itemize}
        \item<1-> A C function provided by the runtime (specific to native code)
        \item<2-> It scans the stack, between the stack pointer at the raising point up to the next trap, in order to collect the names of the detected function frames
          \onslide*<2>{
            \begin{itemize}
              \item And more information, like line number, column number...
            \end{itemize}
          }
        \item<3-> It uses the same technique and information as the Garbage Collector (GC) uses to scan the values live on the stack
          \onslide*<4>{
            \begin{itemize}
              \item \typename{frame\_descr} are at the core of stack unwinding
            \end{itemize}
          }
      \end{itemize}
      \vfill
      \mintocaml[firstline=9,lastline=13,highlightlines=12]{../src/backtrace/backtrace.ml}
      \endminipage
    \end{column}
    \begin{column}{0.5\textwidth}
      \onslide*<1>{\listStashBacktrace[highlightlines=91]{}}
      \onslide*<2>{\listStashBacktrace[highlightlines={91,117-118}]{}}
      \onslide*<3->{\listStashBacktrace[highlightlines={94,106,109-110}]{}}
    \end{column}
  \end{columns}
\end{frame}

\newcommand\listFrameDescr[1][]{
  \listocaml[firstline=111,lastline=124,#1]{../ocaml/asmcomp/emitaux.ml}{asmcomp/emitaux.ml:111}}
\newcommand\listDebugInfo[1][]{
  \listocaml[firstline=38,lastline=49,#1]{../ocaml/lambda/debuginfo.mli}{lambda/debuginfo.mli:38}}

\begin{frame}{Backtraces}{\typename{frame\_descr} and \typename{frame\_debuginfo} in a nutshell}
  \begin{columns}[c]
    \begin{column}{0.5\textwidth}
      \begin{itemize}
        \item<1-> For every return address in an OCaml program, there is a associated \typename{frame\_descr}
        \item<2-> A \typename{frame\_descr} specifies
        \begin{itemize}
          \item<2-> The size of the function stack frame
          \item<3-> Where live values are on the stack (so that the GC can mark them as live and prevent from being swept)
        \end{itemize}
        \item<4-> When compiled with debug information, \typename{frame\_descr} will be linked to a \typename{frame\_debuginfo}
        \begin{itemize}
          \item<5-> Contains information about the position in the source code (file name, line and column number)
        \end{itemize}
      \end{itemize}
    \end{column}
    \begin{column}{0.5\textwidth}
        \onslide*<1>{\listFrameDescr[highlightlines={118-122}]{}}
        \onslide*<2>{\listFrameDescr[highlightlines={120}]{}}
        \onslide*<3>{\listFrameDescr[highlightlines={121}]{}}
        \onslide*<4->{\listFrameDescr[highlightlines={122}]{}}
        \medskip
        \onslide*<1-3>{\listDebugInfo{}}
        \onslide*<4>{\listDebugInfo[highlightlines={38,49}]{}}
        \onslide*<5>{\listDebugInfo[highlightlines={39-46}]{}}
    \end{column}
  \end{columns}
\end{frame}

\begin{frame}{Backtraces}{Scanning the stack with \typename{frame\_descr}}
  \begin{columns}[c]
    \begin{column}{0.5\textwidth}
      \begin{itemize}
        \item Given the return address of the code that raises the exception by calling \funcname{caml\_raise\_exn} (\funcarg{pc} argument of \funcname{caml\_stash\_backtrace})
        \item And the position of the stack pointer (\funcarg{sp} argument of \funcname{caml\_stash\_backtrace})
        \item The frame size given by \typename{frame\_descr}.\funcarg{frame\_size}, allows to find the end of the previous frame
        \item And the return address of the caller of this function
        \bigskip
        \onslide<3->{
        \item Note that traps (here the reddish \funcname{ExnA}, \funcname{ExnB} and \funcname{ExnC} blocks) belong to the frame that installed them, and so the frame size changes when traps are removed
        }
      \end{itemize}
    \end{column}
    \begin{column}{0.5\textwidth}
      \raggedleft
      \onslide*<1>{\providecommand\step{2}\input{diagrams/backtrace/backtrace.tex}}%
      \onslide*<2>{\providecommand\step{1}\input{diagrams/backtrace/scanstack.tex}}%
      \onslide*<3>{\providecommand\step{2}\input{diagrams/backtrace/scanstack.tex}}%
      \onslide*<4>{\providecommand\step{3}\input{diagrams/backtrace/scanstack.tex}}%
      \onslide*<5>{\providecommand\step{4}\input{diagrams/backtrace/scanstack.tex}}%
      \onslide*<6>{\providecommand\step{5}\input{diagrams/backtrace/scanstack.tex}}%
    \end{column}
  \end{columns}
\end{frame}

% Duplicate of previous-previous slide
\begin{frame}{Backtraces}{Continuing on \funcname{caml\_stash\_backtrace}}
  \begin{columns}[c]
    \begin{column}{0.5\textwidth}
      \minipage[c][0.75\textheight][s]{\columnwidth}
      \begin{itemize}
          \color{gray}
        \item A C function provided by the runtime (specific to native code)
        \item It scans the stack, between the stack pointer at the raising point up to the next trap, in order to collect the names of the detected function frames
        \item It uses the same technique and information as the Garbage Collector (GC) uses to scan the values live on the stack
      \end{itemize}
      \begin{itemize}
        \item For each function frame detected, store a pointer to the \typename{frame\_descr} in the \localname{domain\_state->backtrace\_buffer} array, effectively constructing the backtrace
      \end{itemize}
      \onslide<2->{
        \begin{itemize}
            \smallskip
          \item Constructed backtrace:
            \funcname{definitely\_raise},
            \onslide<3->{\funcname{probably\_raise}}
        \end{itemize}
      }
      \vfill
      \mintocaml[firstline=9,lastline=13,highlightlines=12]{../src/backtrace/backtrace.ml}
      \endminipage
    \end{column}
    \begin{column}{0.5\textwidth}
      \raggedleft
      \onslide*<1>{\listStashBacktrace[highlightlines={114-115}]{}}
      \onslide*<2>{\providecommand\step{3}\input{diagrams/backtrace/backtrace.tex}}%
      \onslide*<3>{\providecommand\step{4}\input{diagrams/backtrace/backtrace.tex}}%
    \end{column}
  \end{columns}
\end{frame}

\begin{frame}{Backtraces}{Back to \funcname{caml\_raise\_exn}}
  \begin{columns}[c]
    \begin{column}{0.5\textwidth}
      \minipage[c][0.66\textheight][s]{\columnwidth}
      \begin{itemize}\color{gray}
        \item Checks wheteher exceptions backtrace should be recorded, by checking the \funcarg{domain\_state->backtrace\_active} value
        \item Switches to the C stack to be able to execute C code
        \item Calls \funcname{caml\_stash\_backtrace}, to collect the backtrace up to the next trap
      \end{itemize}
      \onslide*<2->{
        \begin{itemize}
          \item<2-> Executes the exception handler in \funcname{probably\_raise} with \funcname{RESTORE\_EXN\_HANDLER\_OCAML}
            \begin{itemize}
              \item Set \regname{RSP} to \localname{domain\_state->exn\_handler} value
              \item Restore \localname{domain\_state->exn\_handler} from trap
              \item Jump to the handler code with \funcname{ret}
            \end{itemize}
        \end{itemize}
      }
      \vfill
      \onslide*<1>{\mintocaml[firstline=9,lastline=13,highlightlines=12]{../src/backtrace/backtrace.ml}}
      \onslide*<2->{\mintocaml[firstline=15,lastline=21,highlightlines={19-21}]{../src/backtrace/backtrace.ml}}
      \endminipage
    \end{column}
    \begin{column}{0.5\textwidth}
      \centering
      \onslide*<1>{
        \mintc[firstline=190,lastline=192]{../ocaml/runtime/amd64.S}
        \mintc[firstline=234,lastline=237]{../ocaml/runtime/amd64.S}
      }
      \onslide*<2>{
        \mintc[firstline=190,lastline=192,highlightlines={191-192}]{../ocaml/runtime/amd64.S}
        \mintc[firstline=234,lastline=237,highlightlines={235-237}]{../ocaml/runtime/amd64.S}
      }
      \onslide*<1>{\listCamlRaiseExn[highlightlines=720]{}}
      \onslide*<2>{\listCamlRaiseExn[highlightlines=722]{}}
      \onslide*<3->{\providecommand\step{5}\input{diagrams/backtrace/backtrace.tex}}
    \end{column}
  \end{columns}
\end{frame}

\begin{frame}{Backtraces}{\funcname{caml\_probably\_raise}'s handler doesn't catch \typename{ExnC}}
  \begin{columns}[c]
    \begin{column}{0.45\textwidth}
      \minipage[c][0.75\textheight][s]{\columnwidth}
      \begin{itemize}
        \item<1-> \funcname{match \dots with}\footnotemark pattern matching can also match exceptions
        \item<1-> The CMM of \funcname{probably\_raise} shows that this \funcname{match \dots with} is implemented with a pattern matching and a \funcname{try \dots with}
        \item<1-> But this one only matches on \typename{ExnA} exception
        \item<2-> Since the pattern matching fails to catch the exception, \funcname{reraise} is called with our current \typename{ExnC} exception
        \item<3-> Disassembly shows that \funcname{reraise} is actually a call to \funcname{caml\_reraise\_exn}, which is also an assembly function provided by the runtime
      \end{itemize}
      \vfill
      \mintocaml[firstline=15,lastline=21,highlightlines={18-21}]{../src/backtrace/backtrace.ml}
      \endminipage
    \end{column}
    \begin{column}{0.55\textwidth}
      \centering
      \onslide*<1>{\mintcmm[highlightlines={10-18}]{../src/backtrace/camlBacktrace\_\_probably\_raise\_351.cmm}}
      \onslide*<2>{\listcmm[highlightlines=18]{../src/backtrace/camlBacktrace\_\_probably\_raise_351.cmm}{CMM of probably\_raise}}
      \onslide*<3>{\mintobjdump[firstline=5,lastline=35,highlightlines=31]{../src/backtrace/camlBacktrace\_\_probably\_raise_351.objdump}}
    \end{column}
  \end{columns}
  \only<1>{\footnotetext[1]{https://blog.janestreet.com/pattern-matching-and-exception-handling-unite/}}
\end{frame}

\newcommand\listCamlReraiseExn[1][]{
  \listasm[firstline=727,lastline=734,#1]{../ocaml/runtime/amd64.S}{runtime/amd64.S:727}}

\begin{frame}{Backtraces}{Introducing \funcname{caml\_reraise\_exn}}
  \begin{columns}[c]
    \begin{column}{0.5\textwidth}
      \minipage[c][0.66\textheight][s]{\columnwidth}
      \begin{itemize}
        \item<1-> The exception have to be reraised to the next handler
        \item<2-> First, checks if the backtrace should be collected with \funcarg{domain\_state->backtrace\_active}
        \only<3>{
        \begin{itemize}
            \item If not, behaves like a \funcname{raise\_notrace} with \funcname{RESTORE\_EXN\_HANDLER\_OCAML}
        \end{itemize}
        }
        \item<4-> Jumps into \funcname{caml\_raise\_exn}
        \item<5-> Calls \funcname{caml\_stash\_backtrace} again, to scan the next part of the stack: from the trap just pop-ed to the next trap
      \end{itemize}
      \vfill
      \mintocaml[firstline=15,lastline=21,highlightlines={18-21}]{../src/backtrace/backtrace.ml}
      \endminipage
    \end{column}
    \begin{column}{0.5\textwidth}
      \raggedleft
      \onslide*<1>{\listCamlReraiseExn{}}
      \onslide*<2>{\listCamlReraiseExn[highlightlines=729]{}}
      \onslide*<3>{
        \mintc[firstline=190,lastline=192,highlightlines={191-192}]{../ocaml/runtime/amd64.S}
        \mintc[firstline=234,lastline=237,highlightlines={235-237}]{../ocaml/runtime/amd64.S}
        \medskip
        \listCamlReraiseExn[highlightlines=731]{}
      }
      \onslide*<4-5>{
        % TODO refactor with \listCamlReraiseExn
        \mintasm[firstline=727,lastline=734,highlightlines=730]{../ocaml/runtime/amd64.S}
        \medskip
      }
      \onslide*<4>{\listCamlRaiseExn[highlightlines=711]{}}
      \onslide*<5>{\listCamlRaiseExn[highlightlines=720]{}}
    \end{column}
  \end{columns}
\end{frame}

\begin{frame}{Backtraces}{Backtrace collection continues}
  \begin{columns}[c]
    \begin{column}{0.55\textwidth}
      \minipage[c][0.75\textheight][s]{\columnwidth}
      \begin{itemize}
        \item<1-> \funcname{caml\_stash\_backtrace} scans the older part of the stack, up to the next trap
        \item<6-> Controls jumps to the nested exception handler in \funcname{maybe\_raise}, which matches only on \typename{ExnB}
        \item<7-> \funcname{caml\_reraise\_exn} is called again, backtrace collection resumes with \funcname{caml\_stash\_backtrace}
      \end{itemize}
      \medskip
      \begin{itemize}
        \item<2-> Constructed backtrace:
        \begin{itemize}
          \item <2-> \funcname{definitely\_raise}
          \item <2-> \funcname{probably\_raise}
          \item <3-> \funcname{probably\_raise} (re-raise)
          \item <4-> \funcname{maybe\_raise}
          \item <5-> \funcname{main}
          \item <8-> \funcname{main} (re-raise)
        \end{itemize}
      \end{itemize}
      \vfill
      \onslide*<1-5>{\mintocaml[firstline=15,lastline=21]{../src/backtrace/backtrace.ml}}
      \onslide*<6>{\mintocaml[firstline=28,lastline=34,highlightlines=31]{../src/backtrace/backtrace.ml}}
      \onslide*<7->{\mintocaml[firstline=28,lastline=34,highlightlines=33]{../src/backtrace/backtrace.ml}}
      \endminipage
    \end{column}
    \begin{column}{0.45\textwidth}
      \raggedleft
      \onslide*<1>{\providecommand\step{6}\input{diagrams/backtrace/backtrace.tex}}%
      \onslide*<2>{\providecommand\step{6}\input{diagrams/backtrace/backtrace.tex}}%
      \onslide*<3>{\providecommand\step{7}\input{diagrams/backtrace/backtrace.tex}}%
      \onslide*<4>{\providecommand\step{8}\input{diagrams/backtrace/backtrace.tex}}%
      \onslide*<5>{\providecommand\step{9}\input{diagrams/backtrace/backtrace.tex}}%
      \onslide*<6>{\providecommand\step{10}\input{diagrams/backtrace/backtrace.tex}}%
      \onslide*<7>{\providecommand\step{11}\input{diagrams/backtrace/backtrace.tex}}%
      \onslide*<8>{\providecommand\step{12}\input{diagrams/backtrace/backtrace.tex}}%
      \onslide*<9>{\providecommand\step{13}\input{diagrams/backtrace/backtrace.tex}}%
    \end{column}
  \end{columns}
\end{frame}

\begin{frame}{Backtraces}{Printing the backtrace}
  \begin{columns}[c]
    \begin{column}{0.45\textwidth}
      \minipage[c][0.66\textheight][s]{\columnwidth}
      \begin{itemize}
        \item<1-> The exception backtrace can now be printed to \funcarg{stdout} with \funcname{Printexc.print_backtrace}
        \item<2-> First the exception backtrace is retrieved into an OCaml array with \funcname{get_raw_backtrace }
        \item<3-> Which is implemented as the \funcname{caml_get_exception_raw_backtrace} C function in the runtime
        \item<4-> And finally printed with \funcname{print_exception_backtrace}
      \end{itemize}
      \vfill
      \mintocaml[firstline=28,lastline=34,highlightlines=33]{../src/backtrace/backtrace.ml}
      \endminipage
    \end{column}
    \begin{column}{0.55\textwidth}
      \onslide*<1,2>{
        \mintocaml[firstline=100,lastline=101]{../ocaml/stdlib/printexc.ml}
      }
      \only<1>{
        \listocaml[firstline=170,lastline=172]{../ocaml/stdlib/printexc.ml}{stdlib/printexc.ml:170}
      }
      \only<2>{
        \listocaml[firstline=170,lastline=172,highlightlines=172]{../ocaml/stdlib/printexc.ml}{stdlib/printexc.ml:170}
      }
      \only<3>{
        \mintc[firstline=154,lastline=156]{../ocaml/runtime/backtrace.c}
        \listc[firstline=171,lastline=192,highlightlines={184-187}]{../ocaml/runtime/backtrace.c}{runtime/backtrace.c:154}
      }
      \only<4>{
        \listocaml[firstline=155,lastline=165]{../ocaml/stdlib/printexc.ml}{stdlib/printexc.ml:55}
      }
    \end{column}
  \end{columns}
\end{frame}


\subsection{The multiple kinds of raise}
\frameSubsection{}{}

\begin{frame}{Backtraces}{When backtraces are not needed}
  \begin{columns}[c]
    \begin{column}{0.45\textwidth}
      \minipage[c][0.66\textheight][s]{\columnwidth}
      \begin{itemize}
        \item<1-> What if the backtrace for an exception is never needed?
        \item<2-> It's possible to raise the exception explicitly with \funcname{raise\_notrace} to make raising as cheap as possible
        \item<3-> An example of use in the \funcname{dynlink} library of the OCaml compiler
      \end{itemize}
      \vfill
      \onslide<3->{
      \listocaml[firstline=205,lastline=209,highlightlines=207]{../ocaml/otherlibs/dynlink/dynlink\_compilerlibs/misc.ml}{otherlibs/dynlink/dynlink\_compilerlibs/misc.ml:205}
      }
      \endminipage
    \end{column}
    \begin{column}{0.55\textwidth}
    \only<4>{
      \mintcmm[highlightlines=3]{../src/notrace/camlNotrace\_\_fun_347.cmm}
      \listcmm{../src/notrace/camlNotrace\_\_all\_somes\_327.cmm}{all\_somes}
    }
    \onslide*<5->{
      \listobjdump[highlightlines={6-9}]{../src/notrace/camlNotrace\_\_fun_347.objdump}{anonymous function from \funcname{all\_somes}}
    }
    \end{column}
  \end{columns}
\end{frame}

\begin{frame}{Backtraces}{Summary table of the multiple kinds of raise}
  \begin{itemize}
    \item When debugging information is enabled and if the backtrace of an exception isn't needed, it's possible to raise with \funcname{raise\_notrace} for performance
    \item When debugging information is \emph{not} enabled, the compiler optimises \funcname{raise} calls to inline assembly (same effect as using \funcname{raise\_notrace})
  \end{itemize}
  \bigskip
  \centering
  \begin{tabular}{| l | c | c | c | c |}
    \hline
    \parbox{10em}{Debug information \\ (\funcarg{-g} flag)}  & \multicolumn{2}{|c|}{Disabled}                                     & \multicolumn{2}{|c|}{Enabled} \\
    \hline
    Raise kind                                               & \funcname{raise} & \funcname{raise\_notrace}                       & \funcname{raise} & \funcname{raise\_notrace} \\
    \hline
    Compilation                                              & \parbox{8em}{inline assembly\\ (optimization)} & inline assembly   & \funcname{caml\_raise\_exn} & inline assembly \\
    \hline
  \end{tabular}
\end{frame}


%
% Takeaway #5
%
\subsection*{Takeaway \#5}
\frameSubsectionTakeaway{}
\againframe<5>{takeaway}
}%
      \onslide*<7>{\providecommand\step{11}%
% Backtraces
%
\subsection{One advantage of raising with debugging information}
\frameSubsection{}{}

\begin{frame}{Backtraces}{Debugging information \& source code}
  \begin{columns}[c]
    \begin{column}{0.5\textwidth}
      \begin{itemize}
        \item Raising an exception is fun and games, but how do we know where the exception originated? $\rightarrow$ With the backtrace associated with the exception!
        \item In order to get a backtrace from the point where an exception was raised, it's necessary to compile with debugging information enabled (\funcarg{-g} flag for the compiler)
        \item Same example as in the `Nested handlers' section
      \end{itemize}
    \end{column}
    \begin{column}{0.5\textwidth}
      \listocaml[firstline=9,lastline=34]{../src/backtrace/backtrace.ml}{backtrace/backtrace.ml}
    \end{column}
  \end{columns}
\end{frame}

\begin{frame}{Backtraces}{CMM and disassembly of \funcname{definitely\_raise}}
  \begin{columns}[c]
    \begin{column}{0.5\textwidth}
      \minipage[c][0.66\textheight][s]{\columnwidth}
      \begin{itemize}
        \item<1-> An effect of compiling with debugging information (\funcarg{-g}): the CMM no longer uses \funcname{raise\_notrace} to raise the exception but \funcname{raise} instead
        \item<2-> Disassembly shows that CMM \funcname{raise} is actually a call to \funcname{caml\_raise\_exn}, a function in assembler provided by the runtime
      \end{itemize}
      \vfill
      \mintocaml[firstline=9,lastline=13,highlightlines=12]{../src/backtrace/backtrace.ml}
      \endminipage
    \end{column}
    \begin{column}{0.5\textwidth}
      \onslide*<1>{
        \mintcmm[highlightlines=5]{../src/backtrace/camlBacktrace\_\_definitely\_raise\_347.cmm}
      }
      \onslide*<2>{
        \mintobjdump[firstline=2,highlightlines=14]{../src/backtrace/camlBacktrace\_\_definitely\_raise\_347.objdump}
      }
    \end{column}
  \end{columns}
\end{frame}

\begin{frame}{Backtraces}{Introducing \funcname{caml\_raise\_exn}}
  \begin{columns}[c]
    \begin{column}{0.5\textwidth}
      \minipage[c][0.66\textheight][s]{\columnwidth}
      \onslide*<1>{
        \begin{itemize}
          \item Raising the exception is performed using \funcname{caml\_raise\_exn} which is
            \begin{itemize}
              \item An assembly function provided by the runtime
              \item Architecture specific
            \end{itemize}
          \item Let's see what it does differently from \funcname{raise\_notrace}\dots
          \item We are about to raise \typename{ExnC} with \funcname{caml\_raise\_exn} from \funcname{definitely\_raise}
        \end{itemize}
      }
      \onslide*<2>{
        \mintobjdump[firstline=2,highlightlines=14]{../src/backtrace/camlBacktrace\_\_definitely\_raise\_347.objdump}
      }
      \vfill
      \mintocaml[firstline=9,lastline=13,highlightlines=12]{../src/backtrace/backtrace.ml}
      \endminipage
    \end{column}
    \begin{column}{0.5\textwidth}
      \centering
      \providecommand\step{1}\input{diagrams/backtrace/backtrace.tex}
    \end{column}
  \end{columns}
\end{frame}

\begin{frame}{Backtraces}{But before that, we need more fields from \typename{caml\_domain\_state}}
  \begin{columns}
    \begin{column}{0.5\textwidth}
      \begin{itemize}
        \item Remember \typename{caml\_domain\_state} the per-domain runtime state?
        \item Some other members are of interest to us now\dots
        \item \localname{backtrace\_active} is used to know if the runtime should collect backtraces
        \item \localname{backtrace\_active} can be set by
          \begin{itemize}
            \item The \funcarg{b} flag in the \funcname{OCAMLRUNPARAM} environment variable
            \item \mintinline{ocaml}{Printexc.record_backtrace : bool -> unit} \footnotemark
          \end{itemize}
      \end{itemize}
    \end{column}
    \begin{column}{0.5\textwidth}
      \begin{listing}
        \mintc[highlightlines={9-11}]{src/ocaml/domain\_state\_backtrace.h}
        \caption{runtime/caml/domain\_state.h}
      \end{listing}
    \end{column}
  \end{columns}
  \footnotetext[1]{https://v2.ocaml.org/api/Printexc.html}
\end{frame}

\newcommand\listCamlRaiseExn[1][]{
  \listasm[firstline=702,lastline=725,#1]{../ocaml/runtime/amd64.S}{runtime/amd64.S:702}}

\begin{frame}{Backtraces}{Walk through \funcname{caml\_raise\_exn}}
  \begin{columns}[T]
    \begin{column}{0.5\textwidth}
      \begin{itemize}
        \item<1-> First checks whether exceptions backtrace should be recorded
        \item<1-> By checking the \funcarg{domain\_state->backtrace\_active} value
        \item<2-> If not, acts as \funcname{raise\_notrace} with \funcname{RESTORE\_EXN\_HANDLER\_OCAML}
          \begin{itemize}
            \item Set \regname{RSP} to \localname{domain\_state->exn\_handler} value
            \item Restore \localname{domain\_state->exn\_handler} from trap
            \item Jump with \funcname{ret} (same effect as using \funcname{pop} and \funcname{jmp})
          \end{itemize}
        \item<3-> Otherwise, switches to the C stack to be able to execute C code
        \item<4-> Calls \funcname{caml\_stash\_backtrace}, a C function provided by the runtime\ignorespaces
          \onslide<6->{, with
            \begin{itemize}
              \item \funcarg{exn}: exception value
              \item \funcarg{pc}: code address of the raising point
              \item \funcarg{sp}: stack address of the raising function
              \item \funcarg{trapsp}: stack address of the next handler
            \end{itemize}
          }
      \end{itemize}
      \bigskip
      \mintocaml[firstline=9,lastline=13,highlightlines=12]{../src/backtrace/backtrace.ml}
    \end{column}
    \begin{column}{0.5\textwidth}
      \raggedleft
      \onslide*<1,3-4>{
        \mintc[firstline=190,lastline=192]{../ocaml/runtime/amd64.S}
        \mintc[firstline=234,lastline=237]{../ocaml/runtime/amd64.S}
      }
      \onslide*<2>{
        \mintc[firstline=190,lastline=192,highlightlines={191-192}]{../ocaml/runtime/amd64.S}
        \mintc[firstline=234,lastline=237,highlightlines={235-237}]{../ocaml/runtime/amd64.S}
      }
      \onslide*<1>{\listCamlRaiseExn[highlightlines=705]{}}%
      \onslide*<2>{\listCamlRaiseExn[highlightlines={707-708}]{}}%
      \onslide*<3>{\listCamlRaiseExn[highlightlines={712-714}]{}}%
      \onslide*<4>{\listCamlRaiseExn[highlightlines={715-720}]{}}%
      \onslide*<5>{\providecommand\step{1}\input{diagrams/backtrace/backtrace.tex}}%
      \onslide*<6>{\providecommand\step{2}\input{diagrams/backtrace/backtrace.tex}}%
    \end{column}
  \end{columns}
\end{frame}

\newcommand\listStashBacktrace[1][]{
  \listc[firstline=91,lastline=120,#1]{../ocaml/runtime/backtrace\_nat.c}{runtime/backtrace\_nat.c:91}}

\begin{frame}{Backtraces}{Introducing \funcname{caml\_stash\_backtrace}}
  \begin{columns}[c]
    \begin{column}{0.5\textwidth}
      \minipage[c][0.66\textheight][s]{\columnwidth}
      \begin{itemize}
        \item<1-> A C function provided by the runtime (specific to native code)
        \item<2-> It scans the stack, between the stack pointer at the raising point up to the next trap, in order to collect the names of the detected function frames
          \onslide*<2>{
            \begin{itemize}
              \item And more information, like line number, column number...
            \end{itemize}
          }
        \item<3-> It uses the same technique and information as the Garbage Collector (GC) uses to scan the values live on the stack
          \onslide*<4>{
            \begin{itemize}
              \item \typename{frame\_descr} are at the core of stack unwinding
            \end{itemize}
          }
      \end{itemize}
      \vfill
      \mintocaml[firstline=9,lastline=13,highlightlines=12]{../src/backtrace/backtrace.ml}
      \endminipage
    \end{column}
    \begin{column}{0.5\textwidth}
      \onslide*<1>{\listStashBacktrace[highlightlines=91]{}}
      \onslide*<2>{\listStashBacktrace[highlightlines={91,117-118}]{}}
      \onslide*<3->{\listStashBacktrace[highlightlines={94,106,109-110}]{}}
    \end{column}
  \end{columns}
\end{frame}

\newcommand\listFrameDescr[1][]{
  \listocaml[firstline=111,lastline=124,#1]{../ocaml/asmcomp/emitaux.ml}{asmcomp/emitaux.ml:111}}
\newcommand\listDebugInfo[1][]{
  \listocaml[firstline=38,lastline=49,#1]{../ocaml/lambda/debuginfo.mli}{lambda/debuginfo.mli:38}}

\begin{frame}{Backtraces}{\typename{frame\_descr} and \typename{frame\_debuginfo} in a nutshell}
  \begin{columns}[c]
    \begin{column}{0.5\textwidth}
      \begin{itemize}
        \item<1-> For every return address in an OCaml program, there is a associated \typename{frame\_descr}
        \item<2-> A \typename{frame\_descr} specifies
        \begin{itemize}
          \item<2-> The size of the function stack frame
          \item<3-> Where live values are on the stack (so that the GC can mark them as live and prevent from being swept)
        \end{itemize}
        \item<4-> When compiled with debug information, \typename{frame\_descr} will be linked to a \typename{frame\_debuginfo}
        \begin{itemize}
          \item<5-> Contains information about the position in the source code (file name, line and column number)
        \end{itemize}
      \end{itemize}
    \end{column}
    \begin{column}{0.5\textwidth}
        \onslide*<1>{\listFrameDescr[highlightlines={118-122}]{}}
        \onslide*<2>{\listFrameDescr[highlightlines={120}]{}}
        \onslide*<3>{\listFrameDescr[highlightlines={121}]{}}
        \onslide*<4->{\listFrameDescr[highlightlines={122}]{}}
        \medskip
        \onslide*<1-3>{\listDebugInfo{}}
        \onslide*<4>{\listDebugInfo[highlightlines={38,49}]{}}
        \onslide*<5>{\listDebugInfo[highlightlines={39-46}]{}}
    \end{column}
  \end{columns}
\end{frame}

\begin{frame}{Backtraces}{Scanning the stack with \typename{frame\_descr}}
  \begin{columns}[c]
    \begin{column}{0.5\textwidth}
      \begin{itemize}
        \item Given the return address of the code that raises the exception by calling \funcname{caml\_raise\_exn} (\funcarg{pc} argument of \funcname{caml\_stash\_backtrace})
        \item And the position of the stack pointer (\funcarg{sp} argument of \funcname{caml\_stash\_backtrace})
        \item The frame size given by \typename{frame\_descr}.\funcarg{frame\_size}, allows to find the end of the previous frame
        \item And the return address of the caller of this function
        \bigskip
        \onslide<3->{
        \item Note that traps (here the reddish \funcname{ExnA}, \funcname{ExnB} and \funcname{ExnC} blocks) belong to the frame that installed them, and so the frame size changes when traps are removed
        }
      \end{itemize}
    \end{column}
    \begin{column}{0.5\textwidth}
      \raggedleft
      \onslide*<1>{\providecommand\step{2}\input{diagrams/backtrace/backtrace.tex}}%
      \onslide*<2>{\providecommand\step{1}\input{diagrams/backtrace/scanstack.tex}}%
      \onslide*<3>{\providecommand\step{2}\input{diagrams/backtrace/scanstack.tex}}%
      \onslide*<4>{\providecommand\step{3}\input{diagrams/backtrace/scanstack.tex}}%
      \onslide*<5>{\providecommand\step{4}\input{diagrams/backtrace/scanstack.tex}}%
      \onslide*<6>{\providecommand\step{5}\input{diagrams/backtrace/scanstack.tex}}%
    \end{column}
  \end{columns}
\end{frame}

% Duplicate of previous-previous slide
\begin{frame}{Backtraces}{Continuing on \funcname{caml\_stash\_backtrace}}
  \begin{columns}[c]
    \begin{column}{0.5\textwidth}
      \minipage[c][0.75\textheight][s]{\columnwidth}
      \begin{itemize}
          \color{gray}
        \item A C function provided by the runtime (specific to native code)
        \item It scans the stack, between the stack pointer at the raising point up to the next trap, in order to collect the names of the detected function frames
        \item It uses the same technique and information as the Garbage Collector (GC) uses to scan the values live on the stack
      \end{itemize}
      \begin{itemize}
        \item For each function frame detected, store a pointer to the \typename{frame\_descr} in the \localname{domain\_state->backtrace\_buffer} array, effectively constructing the backtrace
      \end{itemize}
      \onslide<2->{
        \begin{itemize}
            \smallskip
          \item Constructed backtrace:
            \funcname{definitely\_raise},
            \onslide<3->{\funcname{probably\_raise}}
        \end{itemize}
      }
      \vfill
      \mintocaml[firstline=9,lastline=13,highlightlines=12]{../src/backtrace/backtrace.ml}
      \endminipage
    \end{column}
    \begin{column}{0.5\textwidth}
      \raggedleft
      \onslide*<1>{\listStashBacktrace[highlightlines={114-115}]{}}
      \onslide*<2>{\providecommand\step{3}\input{diagrams/backtrace/backtrace.tex}}%
      \onslide*<3>{\providecommand\step{4}\input{diagrams/backtrace/backtrace.tex}}%
    \end{column}
  \end{columns}
\end{frame}

\begin{frame}{Backtraces}{Back to \funcname{caml\_raise\_exn}}
  \begin{columns}[c]
    \begin{column}{0.5\textwidth}
      \minipage[c][0.66\textheight][s]{\columnwidth}
      \begin{itemize}\color{gray}
        \item Checks wheteher exceptions backtrace should be recorded, by checking the \funcarg{domain\_state->backtrace\_active} value
        \item Switches to the C stack to be able to execute C code
        \item Calls \funcname{caml\_stash\_backtrace}, to collect the backtrace up to the next trap
      \end{itemize}
      \onslide*<2->{
        \begin{itemize}
          \item<2-> Executes the exception handler in \funcname{probably\_raise} with \funcname{RESTORE\_EXN\_HANDLER\_OCAML}
            \begin{itemize}
              \item Set \regname{RSP} to \localname{domain\_state->exn\_handler} value
              \item Restore \localname{domain\_state->exn\_handler} from trap
              \item Jump to the handler code with \funcname{ret}
            \end{itemize}
        \end{itemize}
      }
      \vfill
      \onslide*<1>{\mintocaml[firstline=9,lastline=13,highlightlines=12]{../src/backtrace/backtrace.ml}}
      \onslide*<2->{\mintocaml[firstline=15,lastline=21,highlightlines={19-21}]{../src/backtrace/backtrace.ml}}
      \endminipage
    \end{column}
    \begin{column}{0.5\textwidth}
      \centering
      \onslide*<1>{
        \mintc[firstline=190,lastline=192]{../ocaml/runtime/amd64.S}
        \mintc[firstline=234,lastline=237]{../ocaml/runtime/amd64.S}
      }
      \onslide*<2>{
        \mintc[firstline=190,lastline=192,highlightlines={191-192}]{../ocaml/runtime/amd64.S}
        \mintc[firstline=234,lastline=237,highlightlines={235-237}]{../ocaml/runtime/amd64.S}
      }
      \onslide*<1>{\listCamlRaiseExn[highlightlines=720]{}}
      \onslide*<2>{\listCamlRaiseExn[highlightlines=722]{}}
      \onslide*<3->{\providecommand\step{5}\input{diagrams/backtrace/backtrace.tex}}
    \end{column}
  \end{columns}
\end{frame}

\begin{frame}{Backtraces}{\funcname{caml\_probably\_raise}'s handler doesn't catch \typename{ExnC}}
  \begin{columns}[c]
    \begin{column}{0.45\textwidth}
      \minipage[c][0.75\textheight][s]{\columnwidth}
      \begin{itemize}
        \item<1-> \funcname{match \dots with}\footnotemark pattern matching can also match exceptions
        \item<1-> The CMM of \funcname{probably\_raise} shows that this \funcname{match \dots with} is implemented with a pattern matching and a \funcname{try \dots with}
        \item<1-> But this one only matches on \typename{ExnA} exception
        \item<2-> Since the pattern matching fails to catch the exception, \funcname{reraise} is called with our current \typename{ExnC} exception
        \item<3-> Disassembly shows that \funcname{reraise} is actually a call to \funcname{caml\_reraise\_exn}, which is also an assembly function provided by the runtime
      \end{itemize}
      \vfill
      \mintocaml[firstline=15,lastline=21,highlightlines={18-21}]{../src/backtrace/backtrace.ml}
      \endminipage
    \end{column}
    \begin{column}{0.55\textwidth}
      \centering
      \onslide*<1>{\mintcmm[highlightlines={10-18}]{../src/backtrace/camlBacktrace\_\_probably\_raise\_351.cmm}}
      \onslide*<2>{\listcmm[highlightlines=18]{../src/backtrace/camlBacktrace\_\_probably\_raise_351.cmm}{CMM of probably\_raise}}
      \onslide*<3>{\mintobjdump[firstline=5,lastline=35,highlightlines=31]{../src/backtrace/camlBacktrace\_\_probably\_raise_351.objdump}}
    \end{column}
  \end{columns}
  \only<1>{\footnotetext[1]{https://blog.janestreet.com/pattern-matching-and-exception-handling-unite/}}
\end{frame}

\newcommand\listCamlReraiseExn[1][]{
  \listasm[firstline=727,lastline=734,#1]{../ocaml/runtime/amd64.S}{runtime/amd64.S:727}}

\begin{frame}{Backtraces}{Introducing \funcname{caml\_reraise\_exn}}
  \begin{columns}[c]
    \begin{column}{0.5\textwidth}
      \minipage[c][0.66\textheight][s]{\columnwidth}
      \begin{itemize}
        \item<1-> The exception have to be reraised to the next handler
        \item<2-> First, checks if the backtrace should be collected with \funcarg{domain\_state->backtrace\_active}
        \only<3>{
        \begin{itemize}
            \item If not, behaves like a \funcname{raise\_notrace} with \funcname{RESTORE\_EXN\_HANDLER\_OCAML}
        \end{itemize}
        }
        \item<4-> Jumps into \funcname{caml\_raise\_exn}
        \item<5-> Calls \funcname{caml\_stash\_backtrace} again, to scan the next part of the stack: from the trap just pop-ed to the next trap
      \end{itemize}
      \vfill
      \mintocaml[firstline=15,lastline=21,highlightlines={18-21}]{../src/backtrace/backtrace.ml}
      \endminipage
    \end{column}
    \begin{column}{0.5\textwidth}
      \raggedleft
      \onslide*<1>{\listCamlReraiseExn{}}
      \onslide*<2>{\listCamlReraiseExn[highlightlines=729]{}}
      \onslide*<3>{
        \mintc[firstline=190,lastline=192,highlightlines={191-192}]{../ocaml/runtime/amd64.S}
        \mintc[firstline=234,lastline=237,highlightlines={235-237}]{../ocaml/runtime/amd64.S}
        \medskip
        \listCamlReraiseExn[highlightlines=731]{}
      }
      \onslide*<4-5>{
        % TODO refactor with \listCamlReraiseExn
        \mintasm[firstline=727,lastline=734,highlightlines=730]{../ocaml/runtime/amd64.S}
        \medskip
      }
      \onslide*<4>{\listCamlRaiseExn[highlightlines=711]{}}
      \onslide*<5>{\listCamlRaiseExn[highlightlines=720]{}}
    \end{column}
  \end{columns}
\end{frame}

\begin{frame}{Backtraces}{Backtrace collection continues}
  \begin{columns}[c]
    \begin{column}{0.55\textwidth}
      \minipage[c][0.75\textheight][s]{\columnwidth}
      \begin{itemize}
        \item<1-> \funcname{caml\_stash\_backtrace} scans the older part of the stack, up to the next trap
        \item<6-> Controls jumps to the nested exception handler in \funcname{maybe\_raise}, which matches only on \typename{ExnB}
        \item<7-> \funcname{caml\_reraise\_exn} is called again, backtrace collection resumes with \funcname{caml\_stash\_backtrace}
      \end{itemize}
      \medskip
      \begin{itemize}
        \item<2-> Constructed backtrace:
        \begin{itemize}
          \item <2-> \funcname{definitely\_raise}
          \item <2-> \funcname{probably\_raise}
          \item <3-> \funcname{probably\_raise} (re-raise)
          \item <4-> \funcname{maybe\_raise}
          \item <5-> \funcname{main}
          \item <8-> \funcname{main} (re-raise)
        \end{itemize}
      \end{itemize}
      \vfill
      \onslide*<1-5>{\mintocaml[firstline=15,lastline=21]{../src/backtrace/backtrace.ml}}
      \onslide*<6>{\mintocaml[firstline=28,lastline=34,highlightlines=31]{../src/backtrace/backtrace.ml}}
      \onslide*<7->{\mintocaml[firstline=28,lastline=34,highlightlines=33]{../src/backtrace/backtrace.ml}}
      \endminipage
    \end{column}
    \begin{column}{0.45\textwidth}
      \raggedleft
      \onslide*<1>{\providecommand\step{6}\input{diagrams/backtrace/backtrace.tex}}%
      \onslide*<2>{\providecommand\step{6}\input{diagrams/backtrace/backtrace.tex}}%
      \onslide*<3>{\providecommand\step{7}\input{diagrams/backtrace/backtrace.tex}}%
      \onslide*<4>{\providecommand\step{8}\input{diagrams/backtrace/backtrace.tex}}%
      \onslide*<5>{\providecommand\step{9}\input{diagrams/backtrace/backtrace.tex}}%
      \onslide*<6>{\providecommand\step{10}\input{diagrams/backtrace/backtrace.tex}}%
      \onslide*<7>{\providecommand\step{11}\input{diagrams/backtrace/backtrace.tex}}%
      \onslide*<8>{\providecommand\step{12}\input{diagrams/backtrace/backtrace.tex}}%
      \onslide*<9>{\providecommand\step{13}\input{diagrams/backtrace/backtrace.tex}}%
    \end{column}
  \end{columns}
\end{frame}

\begin{frame}{Backtraces}{Printing the backtrace}
  \begin{columns}[c]
    \begin{column}{0.45\textwidth}
      \minipage[c][0.66\textheight][s]{\columnwidth}
      \begin{itemize}
        \item<1-> The exception backtrace can now be printed to \funcarg{stdout} with \funcname{Printexc.print_backtrace}
        \item<2-> First the exception backtrace is retrieved into an OCaml array with \funcname{get_raw_backtrace }
        \item<3-> Which is implemented as the \funcname{caml_get_exception_raw_backtrace} C function in the runtime
        \item<4-> And finally printed with \funcname{print_exception_backtrace}
      \end{itemize}
      \vfill
      \mintocaml[firstline=28,lastline=34,highlightlines=33]{../src/backtrace/backtrace.ml}
      \endminipage
    \end{column}
    \begin{column}{0.55\textwidth}
      \onslide*<1,2>{
        \mintocaml[firstline=100,lastline=101]{../ocaml/stdlib/printexc.ml}
      }
      \only<1>{
        \listocaml[firstline=170,lastline=172]{../ocaml/stdlib/printexc.ml}{stdlib/printexc.ml:170}
      }
      \only<2>{
        \listocaml[firstline=170,lastline=172,highlightlines=172]{../ocaml/stdlib/printexc.ml}{stdlib/printexc.ml:170}
      }
      \only<3>{
        \mintc[firstline=154,lastline=156]{../ocaml/runtime/backtrace.c}
        \listc[firstline=171,lastline=192,highlightlines={184-187}]{../ocaml/runtime/backtrace.c}{runtime/backtrace.c:154}
      }
      \only<4>{
        \listocaml[firstline=155,lastline=165]{../ocaml/stdlib/printexc.ml}{stdlib/printexc.ml:55}
      }
    \end{column}
  \end{columns}
\end{frame}


\subsection{The multiple kinds of raise}
\frameSubsection{}{}

\begin{frame}{Backtraces}{When backtraces are not needed}
  \begin{columns}[c]
    \begin{column}{0.45\textwidth}
      \minipage[c][0.66\textheight][s]{\columnwidth}
      \begin{itemize}
        \item<1-> What if the backtrace for an exception is never needed?
        \item<2-> It's possible to raise the exception explicitly with \funcname{raise\_notrace} to make raising as cheap as possible
        \item<3-> An example of use in the \funcname{dynlink} library of the OCaml compiler
      \end{itemize}
      \vfill
      \onslide<3->{
      \listocaml[firstline=205,lastline=209,highlightlines=207]{../ocaml/otherlibs/dynlink/dynlink\_compilerlibs/misc.ml}{otherlibs/dynlink/dynlink\_compilerlibs/misc.ml:205}
      }
      \endminipage
    \end{column}
    \begin{column}{0.55\textwidth}
    \only<4>{
      \mintcmm[highlightlines=3]{../src/notrace/camlNotrace\_\_fun_347.cmm}
      \listcmm{../src/notrace/camlNotrace\_\_all\_somes\_327.cmm}{all\_somes}
    }
    \onslide*<5->{
      \listobjdump[highlightlines={6-9}]{../src/notrace/camlNotrace\_\_fun_347.objdump}{anonymous function from \funcname{all\_somes}}
    }
    \end{column}
  \end{columns}
\end{frame}

\begin{frame}{Backtraces}{Summary table of the multiple kinds of raise}
  \begin{itemize}
    \item When debugging information is enabled and if the backtrace of an exception isn't needed, it's possible to raise with \funcname{raise\_notrace} for performance
    \item When debugging information is \emph{not} enabled, the compiler optimises \funcname{raise} calls to inline assembly (same effect as using \funcname{raise\_notrace})
  \end{itemize}
  \bigskip
  \centering
  \begin{tabular}{| l | c | c | c | c |}
    \hline
    \parbox{10em}{Debug information \\ (\funcarg{-g} flag)}  & \multicolumn{2}{|c|}{Disabled}                                     & \multicolumn{2}{|c|}{Enabled} \\
    \hline
    Raise kind                                               & \funcname{raise} & \funcname{raise\_notrace}                       & \funcname{raise} & \funcname{raise\_notrace} \\
    \hline
    Compilation                                              & \parbox{8em}{inline assembly\\ (optimization)} & inline assembly   & \funcname{caml\_raise\_exn} & inline assembly \\
    \hline
  \end{tabular}
\end{frame}


%
% Takeaway #5
%
\subsection*{Takeaway \#5}
\frameSubsectionTakeaway{}
\againframe<5>{takeaway}
}%
      \onslide*<8>{\providecommand\step{12}%
% Backtraces
%
\subsection{One advantage of raising with debugging information}
\frameSubsection{}{}

\begin{frame}{Backtraces}{Debugging information \& source code}
  \begin{columns}[c]
    \begin{column}{0.5\textwidth}
      \begin{itemize}
        \item Raising an exception is fun and games, but how do we know where the exception originated? $\rightarrow$ With the backtrace associated with the exception!
        \item In order to get a backtrace from the point where an exception was raised, it's necessary to compile with debugging information enabled (\funcarg{-g} flag for the compiler)
        \item Same example as in the `Nested handlers' section
      \end{itemize}
    \end{column}
    \begin{column}{0.5\textwidth}
      \listocaml[firstline=9,lastline=34]{../src/backtrace/backtrace.ml}{backtrace/backtrace.ml}
    \end{column}
  \end{columns}
\end{frame}

\begin{frame}{Backtraces}{CMM and disassembly of \funcname{definitely\_raise}}
  \begin{columns}[c]
    \begin{column}{0.5\textwidth}
      \minipage[c][0.66\textheight][s]{\columnwidth}
      \begin{itemize}
        \item<1-> An effect of compiling with debugging information (\funcarg{-g}): the CMM no longer uses \funcname{raise\_notrace} to raise the exception but \funcname{raise} instead
        \item<2-> Disassembly shows that CMM \funcname{raise} is actually a call to \funcname{caml\_raise\_exn}, a function in assembler provided by the runtime
      \end{itemize}
      \vfill
      \mintocaml[firstline=9,lastline=13,highlightlines=12]{../src/backtrace/backtrace.ml}
      \endminipage
    \end{column}
    \begin{column}{0.5\textwidth}
      \onslide*<1>{
        \mintcmm[highlightlines=5]{../src/backtrace/camlBacktrace\_\_definitely\_raise\_347.cmm}
      }
      \onslide*<2>{
        \mintobjdump[firstline=2,highlightlines=14]{../src/backtrace/camlBacktrace\_\_definitely\_raise\_347.objdump}
      }
    \end{column}
  \end{columns}
\end{frame}

\begin{frame}{Backtraces}{Introducing \funcname{caml\_raise\_exn}}
  \begin{columns}[c]
    \begin{column}{0.5\textwidth}
      \minipage[c][0.66\textheight][s]{\columnwidth}
      \onslide*<1>{
        \begin{itemize}
          \item Raising the exception is performed using \funcname{caml\_raise\_exn} which is
            \begin{itemize}
              \item An assembly function provided by the runtime
              \item Architecture specific
            \end{itemize}
          \item Let's see what it does differently from \funcname{raise\_notrace}\dots
          \item We are about to raise \typename{ExnC} with \funcname{caml\_raise\_exn} from \funcname{definitely\_raise}
        \end{itemize}
      }
      \onslide*<2>{
        \mintobjdump[firstline=2,highlightlines=14]{../src/backtrace/camlBacktrace\_\_definitely\_raise\_347.objdump}
      }
      \vfill
      \mintocaml[firstline=9,lastline=13,highlightlines=12]{../src/backtrace/backtrace.ml}
      \endminipage
    \end{column}
    \begin{column}{0.5\textwidth}
      \centering
      \providecommand\step{1}\input{diagrams/backtrace/backtrace.tex}
    \end{column}
  \end{columns}
\end{frame}

\begin{frame}{Backtraces}{But before that, we need more fields from \typename{caml\_domain\_state}}
  \begin{columns}
    \begin{column}{0.5\textwidth}
      \begin{itemize}
        \item Remember \typename{caml\_domain\_state} the per-domain runtime state?
        \item Some other members are of interest to us now\dots
        \item \localname{backtrace\_active} is used to know if the runtime should collect backtraces
        \item \localname{backtrace\_active} can be set by
          \begin{itemize}
            \item The \funcarg{b} flag in the \funcname{OCAMLRUNPARAM} environment variable
            \item \mintinline{ocaml}{Printexc.record_backtrace : bool -> unit} \footnotemark
          \end{itemize}
      \end{itemize}
    \end{column}
    \begin{column}{0.5\textwidth}
      \begin{listing}
        \mintc[highlightlines={9-11}]{src/ocaml/domain\_state\_backtrace.h}
        \caption{runtime/caml/domain\_state.h}
      \end{listing}
    \end{column}
  \end{columns}
  \footnotetext[1]{https://v2.ocaml.org/api/Printexc.html}
\end{frame}

\newcommand\listCamlRaiseExn[1][]{
  \listasm[firstline=702,lastline=725,#1]{../ocaml/runtime/amd64.S}{runtime/amd64.S:702}}

\begin{frame}{Backtraces}{Walk through \funcname{caml\_raise\_exn}}
  \begin{columns}[T]
    \begin{column}{0.5\textwidth}
      \begin{itemize}
        \item<1-> First checks whether exceptions backtrace should be recorded
        \item<1-> By checking the \funcarg{domain\_state->backtrace\_active} value
        \item<2-> If not, acts as \funcname{raise\_notrace} with \funcname{RESTORE\_EXN\_HANDLER\_OCAML}
          \begin{itemize}
            \item Set \regname{RSP} to \localname{domain\_state->exn\_handler} value
            \item Restore \localname{domain\_state->exn\_handler} from trap
            \item Jump with \funcname{ret} (same effect as using \funcname{pop} and \funcname{jmp})
          \end{itemize}
        \item<3-> Otherwise, switches to the C stack to be able to execute C code
        \item<4-> Calls \funcname{caml\_stash\_backtrace}, a C function provided by the runtime\ignorespaces
          \onslide<6->{, with
            \begin{itemize}
              \item \funcarg{exn}: exception value
              \item \funcarg{pc}: code address of the raising point
              \item \funcarg{sp}: stack address of the raising function
              \item \funcarg{trapsp}: stack address of the next handler
            \end{itemize}
          }
      \end{itemize}
      \bigskip
      \mintocaml[firstline=9,lastline=13,highlightlines=12]{../src/backtrace/backtrace.ml}
    \end{column}
    \begin{column}{0.5\textwidth}
      \raggedleft
      \onslide*<1,3-4>{
        \mintc[firstline=190,lastline=192]{../ocaml/runtime/amd64.S}
        \mintc[firstline=234,lastline=237]{../ocaml/runtime/amd64.S}
      }
      \onslide*<2>{
        \mintc[firstline=190,lastline=192,highlightlines={191-192}]{../ocaml/runtime/amd64.S}
        \mintc[firstline=234,lastline=237,highlightlines={235-237}]{../ocaml/runtime/amd64.S}
      }
      \onslide*<1>{\listCamlRaiseExn[highlightlines=705]{}}%
      \onslide*<2>{\listCamlRaiseExn[highlightlines={707-708}]{}}%
      \onslide*<3>{\listCamlRaiseExn[highlightlines={712-714}]{}}%
      \onslide*<4>{\listCamlRaiseExn[highlightlines={715-720}]{}}%
      \onslide*<5>{\providecommand\step{1}\input{diagrams/backtrace/backtrace.tex}}%
      \onslide*<6>{\providecommand\step{2}\input{diagrams/backtrace/backtrace.tex}}%
    \end{column}
  \end{columns}
\end{frame}

\newcommand\listStashBacktrace[1][]{
  \listc[firstline=91,lastline=120,#1]{../ocaml/runtime/backtrace\_nat.c}{runtime/backtrace\_nat.c:91}}

\begin{frame}{Backtraces}{Introducing \funcname{caml\_stash\_backtrace}}
  \begin{columns}[c]
    \begin{column}{0.5\textwidth}
      \minipage[c][0.66\textheight][s]{\columnwidth}
      \begin{itemize}
        \item<1-> A C function provided by the runtime (specific to native code)
        \item<2-> It scans the stack, between the stack pointer at the raising point up to the next trap, in order to collect the names of the detected function frames
          \onslide*<2>{
            \begin{itemize}
              \item And more information, like line number, column number...
            \end{itemize}
          }
        \item<3-> It uses the same technique and information as the Garbage Collector (GC) uses to scan the values live on the stack
          \onslide*<4>{
            \begin{itemize}
              \item \typename{frame\_descr} are at the core of stack unwinding
            \end{itemize}
          }
      \end{itemize}
      \vfill
      \mintocaml[firstline=9,lastline=13,highlightlines=12]{../src/backtrace/backtrace.ml}
      \endminipage
    \end{column}
    \begin{column}{0.5\textwidth}
      \onslide*<1>{\listStashBacktrace[highlightlines=91]{}}
      \onslide*<2>{\listStashBacktrace[highlightlines={91,117-118}]{}}
      \onslide*<3->{\listStashBacktrace[highlightlines={94,106,109-110}]{}}
    \end{column}
  \end{columns}
\end{frame}

\newcommand\listFrameDescr[1][]{
  \listocaml[firstline=111,lastline=124,#1]{../ocaml/asmcomp/emitaux.ml}{asmcomp/emitaux.ml:111}}
\newcommand\listDebugInfo[1][]{
  \listocaml[firstline=38,lastline=49,#1]{../ocaml/lambda/debuginfo.mli}{lambda/debuginfo.mli:38}}

\begin{frame}{Backtraces}{\typename{frame\_descr} and \typename{frame\_debuginfo} in a nutshell}
  \begin{columns}[c]
    \begin{column}{0.5\textwidth}
      \begin{itemize}
        \item<1-> For every return address in an OCaml program, there is a associated \typename{frame\_descr}
        \item<2-> A \typename{frame\_descr} specifies
        \begin{itemize}
          \item<2-> The size of the function stack frame
          \item<3-> Where live values are on the stack (so that the GC can mark them as live and prevent from being swept)
        \end{itemize}
        \item<4-> When compiled with debug information, \typename{frame\_descr} will be linked to a \typename{frame\_debuginfo}
        \begin{itemize}
          \item<5-> Contains information about the position in the source code (file name, line and column number)
        \end{itemize}
      \end{itemize}
    \end{column}
    \begin{column}{0.5\textwidth}
        \onslide*<1>{\listFrameDescr[highlightlines={118-122}]{}}
        \onslide*<2>{\listFrameDescr[highlightlines={120}]{}}
        \onslide*<3>{\listFrameDescr[highlightlines={121}]{}}
        \onslide*<4->{\listFrameDescr[highlightlines={122}]{}}
        \medskip
        \onslide*<1-3>{\listDebugInfo{}}
        \onslide*<4>{\listDebugInfo[highlightlines={38,49}]{}}
        \onslide*<5>{\listDebugInfo[highlightlines={39-46}]{}}
    \end{column}
  \end{columns}
\end{frame}

\begin{frame}{Backtraces}{Scanning the stack with \typename{frame\_descr}}
  \begin{columns}[c]
    \begin{column}{0.5\textwidth}
      \begin{itemize}
        \item Given the return address of the code that raises the exception by calling \funcname{caml\_raise\_exn} (\funcarg{pc} argument of \funcname{caml\_stash\_backtrace})
        \item And the position of the stack pointer (\funcarg{sp} argument of \funcname{caml\_stash\_backtrace})
        \item The frame size given by \typename{frame\_descr}.\funcarg{frame\_size}, allows to find the end of the previous frame
        \item And the return address of the caller of this function
        \bigskip
        \onslide<3->{
        \item Note that traps (here the reddish \funcname{ExnA}, \funcname{ExnB} and \funcname{ExnC} blocks) belong to the frame that installed them, and so the frame size changes when traps are removed
        }
      \end{itemize}
    \end{column}
    \begin{column}{0.5\textwidth}
      \raggedleft
      \onslide*<1>{\providecommand\step{2}\input{diagrams/backtrace/backtrace.tex}}%
      \onslide*<2>{\providecommand\step{1}\input{diagrams/backtrace/scanstack.tex}}%
      \onslide*<3>{\providecommand\step{2}\input{diagrams/backtrace/scanstack.tex}}%
      \onslide*<4>{\providecommand\step{3}\input{diagrams/backtrace/scanstack.tex}}%
      \onslide*<5>{\providecommand\step{4}\input{diagrams/backtrace/scanstack.tex}}%
      \onslide*<6>{\providecommand\step{5}\input{diagrams/backtrace/scanstack.tex}}%
    \end{column}
  \end{columns}
\end{frame}

% Duplicate of previous-previous slide
\begin{frame}{Backtraces}{Continuing on \funcname{caml\_stash\_backtrace}}
  \begin{columns}[c]
    \begin{column}{0.5\textwidth}
      \minipage[c][0.75\textheight][s]{\columnwidth}
      \begin{itemize}
          \color{gray}
        \item A C function provided by the runtime (specific to native code)
        \item It scans the stack, between the stack pointer at the raising point up to the next trap, in order to collect the names of the detected function frames
        \item It uses the same technique and information as the Garbage Collector (GC) uses to scan the values live on the stack
      \end{itemize}
      \begin{itemize}
        \item For each function frame detected, store a pointer to the \typename{frame\_descr} in the \localname{domain\_state->backtrace\_buffer} array, effectively constructing the backtrace
      \end{itemize}
      \onslide<2->{
        \begin{itemize}
            \smallskip
          \item Constructed backtrace:
            \funcname{definitely\_raise},
            \onslide<3->{\funcname{probably\_raise}}
        \end{itemize}
      }
      \vfill
      \mintocaml[firstline=9,lastline=13,highlightlines=12]{../src/backtrace/backtrace.ml}
      \endminipage
    \end{column}
    \begin{column}{0.5\textwidth}
      \raggedleft
      \onslide*<1>{\listStashBacktrace[highlightlines={114-115}]{}}
      \onslide*<2>{\providecommand\step{3}\input{diagrams/backtrace/backtrace.tex}}%
      \onslide*<3>{\providecommand\step{4}\input{diagrams/backtrace/backtrace.tex}}%
    \end{column}
  \end{columns}
\end{frame}

\begin{frame}{Backtraces}{Back to \funcname{caml\_raise\_exn}}
  \begin{columns}[c]
    \begin{column}{0.5\textwidth}
      \minipage[c][0.66\textheight][s]{\columnwidth}
      \begin{itemize}\color{gray}
        \item Checks wheteher exceptions backtrace should be recorded, by checking the \funcarg{domain\_state->backtrace\_active} value
        \item Switches to the C stack to be able to execute C code
        \item Calls \funcname{caml\_stash\_backtrace}, to collect the backtrace up to the next trap
      \end{itemize}
      \onslide*<2->{
        \begin{itemize}
          \item<2-> Executes the exception handler in \funcname{probably\_raise} with \funcname{RESTORE\_EXN\_HANDLER\_OCAML}
            \begin{itemize}
              \item Set \regname{RSP} to \localname{domain\_state->exn\_handler} value
              \item Restore \localname{domain\_state->exn\_handler} from trap
              \item Jump to the handler code with \funcname{ret}
            \end{itemize}
        \end{itemize}
      }
      \vfill
      \onslide*<1>{\mintocaml[firstline=9,lastline=13,highlightlines=12]{../src/backtrace/backtrace.ml}}
      \onslide*<2->{\mintocaml[firstline=15,lastline=21,highlightlines={19-21}]{../src/backtrace/backtrace.ml}}
      \endminipage
    \end{column}
    \begin{column}{0.5\textwidth}
      \centering
      \onslide*<1>{
        \mintc[firstline=190,lastline=192]{../ocaml/runtime/amd64.S}
        \mintc[firstline=234,lastline=237]{../ocaml/runtime/amd64.S}
      }
      \onslide*<2>{
        \mintc[firstline=190,lastline=192,highlightlines={191-192}]{../ocaml/runtime/amd64.S}
        \mintc[firstline=234,lastline=237,highlightlines={235-237}]{../ocaml/runtime/amd64.S}
      }
      \onslide*<1>{\listCamlRaiseExn[highlightlines=720]{}}
      \onslide*<2>{\listCamlRaiseExn[highlightlines=722]{}}
      \onslide*<3->{\providecommand\step{5}\input{diagrams/backtrace/backtrace.tex}}
    \end{column}
  \end{columns}
\end{frame}

\begin{frame}{Backtraces}{\funcname{caml\_probably\_raise}'s handler doesn't catch \typename{ExnC}}
  \begin{columns}[c]
    \begin{column}{0.45\textwidth}
      \minipage[c][0.75\textheight][s]{\columnwidth}
      \begin{itemize}
        \item<1-> \funcname{match \dots with}\footnotemark pattern matching can also match exceptions
        \item<1-> The CMM of \funcname{probably\_raise} shows that this \funcname{match \dots with} is implemented with a pattern matching and a \funcname{try \dots with}
        \item<1-> But this one only matches on \typename{ExnA} exception
        \item<2-> Since the pattern matching fails to catch the exception, \funcname{reraise} is called with our current \typename{ExnC} exception
        \item<3-> Disassembly shows that \funcname{reraise} is actually a call to \funcname{caml\_reraise\_exn}, which is also an assembly function provided by the runtime
      \end{itemize}
      \vfill
      \mintocaml[firstline=15,lastline=21,highlightlines={18-21}]{../src/backtrace/backtrace.ml}
      \endminipage
    \end{column}
    \begin{column}{0.55\textwidth}
      \centering
      \onslide*<1>{\mintcmm[highlightlines={10-18}]{../src/backtrace/camlBacktrace\_\_probably\_raise\_351.cmm}}
      \onslide*<2>{\listcmm[highlightlines=18]{../src/backtrace/camlBacktrace\_\_probably\_raise_351.cmm}{CMM of probably\_raise}}
      \onslide*<3>{\mintobjdump[firstline=5,lastline=35,highlightlines=31]{../src/backtrace/camlBacktrace\_\_probably\_raise_351.objdump}}
    \end{column}
  \end{columns}
  \only<1>{\footnotetext[1]{https://blog.janestreet.com/pattern-matching-and-exception-handling-unite/}}
\end{frame}

\newcommand\listCamlReraiseExn[1][]{
  \listasm[firstline=727,lastline=734,#1]{../ocaml/runtime/amd64.S}{runtime/amd64.S:727}}

\begin{frame}{Backtraces}{Introducing \funcname{caml\_reraise\_exn}}
  \begin{columns}[c]
    \begin{column}{0.5\textwidth}
      \minipage[c][0.66\textheight][s]{\columnwidth}
      \begin{itemize}
        \item<1-> The exception have to be reraised to the next handler
        \item<2-> First, checks if the backtrace should be collected with \funcarg{domain\_state->backtrace\_active}
        \only<3>{
        \begin{itemize}
            \item If not, behaves like a \funcname{raise\_notrace} with \funcname{RESTORE\_EXN\_HANDLER\_OCAML}
        \end{itemize}
        }
        \item<4-> Jumps into \funcname{caml\_raise\_exn}
        \item<5-> Calls \funcname{caml\_stash\_backtrace} again, to scan the next part of the stack: from the trap just pop-ed to the next trap
      \end{itemize}
      \vfill
      \mintocaml[firstline=15,lastline=21,highlightlines={18-21}]{../src/backtrace/backtrace.ml}
      \endminipage
    \end{column}
    \begin{column}{0.5\textwidth}
      \raggedleft
      \onslide*<1>{\listCamlReraiseExn{}}
      \onslide*<2>{\listCamlReraiseExn[highlightlines=729]{}}
      \onslide*<3>{
        \mintc[firstline=190,lastline=192,highlightlines={191-192}]{../ocaml/runtime/amd64.S}
        \mintc[firstline=234,lastline=237,highlightlines={235-237}]{../ocaml/runtime/amd64.S}
        \medskip
        \listCamlReraiseExn[highlightlines=731]{}
      }
      \onslide*<4-5>{
        % TODO refactor with \listCamlReraiseExn
        \mintasm[firstline=727,lastline=734,highlightlines=730]{../ocaml/runtime/amd64.S}
        \medskip
      }
      \onslide*<4>{\listCamlRaiseExn[highlightlines=711]{}}
      \onslide*<5>{\listCamlRaiseExn[highlightlines=720]{}}
    \end{column}
  \end{columns}
\end{frame}

\begin{frame}{Backtraces}{Backtrace collection continues}
  \begin{columns}[c]
    \begin{column}{0.55\textwidth}
      \minipage[c][0.75\textheight][s]{\columnwidth}
      \begin{itemize}
        \item<1-> \funcname{caml\_stash\_backtrace} scans the older part of the stack, up to the next trap
        \item<6-> Controls jumps to the nested exception handler in \funcname{maybe\_raise}, which matches only on \typename{ExnB}
        \item<7-> \funcname{caml\_reraise\_exn} is called again, backtrace collection resumes with \funcname{caml\_stash\_backtrace}
      \end{itemize}
      \medskip
      \begin{itemize}
        \item<2-> Constructed backtrace:
        \begin{itemize}
          \item <2-> \funcname{definitely\_raise}
          \item <2-> \funcname{probably\_raise}
          \item <3-> \funcname{probably\_raise} (re-raise)
          \item <4-> \funcname{maybe\_raise}
          \item <5-> \funcname{main}
          \item <8-> \funcname{main} (re-raise)
        \end{itemize}
      \end{itemize}
      \vfill
      \onslide*<1-5>{\mintocaml[firstline=15,lastline=21]{../src/backtrace/backtrace.ml}}
      \onslide*<6>{\mintocaml[firstline=28,lastline=34,highlightlines=31]{../src/backtrace/backtrace.ml}}
      \onslide*<7->{\mintocaml[firstline=28,lastline=34,highlightlines=33]{../src/backtrace/backtrace.ml}}
      \endminipage
    \end{column}
    \begin{column}{0.45\textwidth}
      \raggedleft
      \onslide*<1>{\providecommand\step{6}\input{diagrams/backtrace/backtrace.tex}}%
      \onslide*<2>{\providecommand\step{6}\input{diagrams/backtrace/backtrace.tex}}%
      \onslide*<3>{\providecommand\step{7}\input{diagrams/backtrace/backtrace.tex}}%
      \onslide*<4>{\providecommand\step{8}\input{diagrams/backtrace/backtrace.tex}}%
      \onslide*<5>{\providecommand\step{9}\input{diagrams/backtrace/backtrace.tex}}%
      \onslide*<6>{\providecommand\step{10}\input{diagrams/backtrace/backtrace.tex}}%
      \onslide*<7>{\providecommand\step{11}\input{diagrams/backtrace/backtrace.tex}}%
      \onslide*<8>{\providecommand\step{12}\input{diagrams/backtrace/backtrace.tex}}%
      \onslide*<9>{\providecommand\step{13}\input{diagrams/backtrace/backtrace.tex}}%
    \end{column}
  \end{columns}
\end{frame}

\begin{frame}{Backtraces}{Printing the backtrace}
  \begin{columns}[c]
    \begin{column}{0.45\textwidth}
      \minipage[c][0.66\textheight][s]{\columnwidth}
      \begin{itemize}
        \item<1-> The exception backtrace can now be printed to \funcarg{stdout} with \funcname{Printexc.print_backtrace}
        \item<2-> First the exception backtrace is retrieved into an OCaml array with \funcname{get_raw_backtrace }
        \item<3-> Which is implemented as the \funcname{caml_get_exception_raw_backtrace} C function in the runtime
        \item<4-> And finally printed with \funcname{print_exception_backtrace}
      \end{itemize}
      \vfill
      \mintocaml[firstline=28,lastline=34,highlightlines=33]{../src/backtrace/backtrace.ml}
      \endminipage
    \end{column}
    \begin{column}{0.55\textwidth}
      \onslide*<1,2>{
        \mintocaml[firstline=100,lastline=101]{../ocaml/stdlib/printexc.ml}
      }
      \only<1>{
        \listocaml[firstline=170,lastline=172]{../ocaml/stdlib/printexc.ml}{stdlib/printexc.ml:170}
      }
      \only<2>{
        \listocaml[firstline=170,lastline=172,highlightlines=172]{../ocaml/stdlib/printexc.ml}{stdlib/printexc.ml:170}
      }
      \only<3>{
        \mintc[firstline=154,lastline=156]{../ocaml/runtime/backtrace.c}
        \listc[firstline=171,lastline=192,highlightlines={184-187}]{../ocaml/runtime/backtrace.c}{runtime/backtrace.c:154}
      }
      \only<4>{
        \listocaml[firstline=155,lastline=165]{../ocaml/stdlib/printexc.ml}{stdlib/printexc.ml:55}
      }
    \end{column}
  \end{columns}
\end{frame}


\subsection{The multiple kinds of raise}
\frameSubsection{}{}

\begin{frame}{Backtraces}{When backtraces are not needed}
  \begin{columns}[c]
    \begin{column}{0.45\textwidth}
      \minipage[c][0.66\textheight][s]{\columnwidth}
      \begin{itemize}
        \item<1-> What if the backtrace for an exception is never needed?
        \item<2-> It's possible to raise the exception explicitly with \funcname{raise\_notrace} to make raising as cheap as possible
        \item<3-> An example of use in the \funcname{dynlink} library of the OCaml compiler
      \end{itemize}
      \vfill
      \onslide<3->{
      \listocaml[firstline=205,lastline=209,highlightlines=207]{../ocaml/otherlibs/dynlink/dynlink\_compilerlibs/misc.ml}{otherlibs/dynlink/dynlink\_compilerlibs/misc.ml:205}
      }
      \endminipage
    \end{column}
    \begin{column}{0.55\textwidth}
    \only<4>{
      \mintcmm[highlightlines=3]{../src/notrace/camlNotrace\_\_fun_347.cmm}
      \listcmm{../src/notrace/camlNotrace\_\_all\_somes\_327.cmm}{all\_somes}
    }
    \onslide*<5->{
      \listobjdump[highlightlines={6-9}]{../src/notrace/camlNotrace\_\_fun_347.objdump}{anonymous function from \funcname{all\_somes}}
    }
    \end{column}
  \end{columns}
\end{frame}

\begin{frame}{Backtraces}{Summary table of the multiple kinds of raise}
  \begin{itemize}
    \item When debugging information is enabled and if the backtrace of an exception isn't needed, it's possible to raise with \funcname{raise\_notrace} for performance
    \item When debugging information is \emph{not} enabled, the compiler optimises \funcname{raise} calls to inline assembly (same effect as using \funcname{raise\_notrace})
  \end{itemize}
  \bigskip
  \centering
  \begin{tabular}{| l | c | c | c | c |}
    \hline
    \parbox{10em}{Debug information \\ (\funcarg{-g} flag)}  & \multicolumn{2}{|c|}{Disabled}                                     & \multicolumn{2}{|c|}{Enabled} \\
    \hline
    Raise kind                                               & \funcname{raise} & \funcname{raise\_notrace}                       & \funcname{raise} & \funcname{raise\_notrace} \\
    \hline
    Compilation                                              & \parbox{8em}{inline assembly\\ (optimization)} & inline assembly   & \funcname{caml\_raise\_exn} & inline assembly \\
    \hline
  \end{tabular}
\end{frame}


%
% Takeaway #5
%
\subsection*{Takeaway \#5}
\frameSubsectionTakeaway{}
\againframe<5>{takeaway}
}%
      \onslide*<9>{\providecommand\step{13}%
% Backtraces
%
\subsection{One advantage of raising with debugging information}
\frameSubsection{}{}

\begin{frame}{Backtraces}{Debugging information \& source code}
  \begin{columns}[c]
    \begin{column}{0.5\textwidth}
      \begin{itemize}
        \item Raising an exception is fun and games, but how do we know where the exception originated? $\rightarrow$ With the backtrace associated with the exception!
        \item In order to get a backtrace from the point where an exception was raised, it's necessary to compile with debugging information enabled (\funcarg{-g} flag for the compiler)
        \item Same example as in the `Nested handlers' section
      \end{itemize}
    \end{column}
    \begin{column}{0.5\textwidth}
      \listocaml[firstline=9,lastline=34]{../src/backtrace/backtrace.ml}{backtrace/backtrace.ml}
    \end{column}
  \end{columns}
\end{frame}

\begin{frame}{Backtraces}{CMM and disassembly of \funcname{definitely\_raise}}
  \begin{columns}[c]
    \begin{column}{0.5\textwidth}
      \minipage[c][0.66\textheight][s]{\columnwidth}
      \begin{itemize}
        \item<1-> An effect of compiling with debugging information (\funcarg{-g}): the CMM no longer uses \funcname{raise\_notrace} to raise the exception but \funcname{raise} instead
        \item<2-> Disassembly shows that CMM \funcname{raise} is actually a call to \funcname{caml\_raise\_exn}, a function in assembler provided by the runtime
      \end{itemize}
      \vfill
      \mintocaml[firstline=9,lastline=13,highlightlines=12]{../src/backtrace/backtrace.ml}
      \endminipage
    \end{column}
    \begin{column}{0.5\textwidth}
      \onslide*<1>{
        \mintcmm[highlightlines=5]{../src/backtrace/camlBacktrace\_\_definitely\_raise\_347.cmm}
      }
      \onslide*<2>{
        \mintobjdump[firstline=2,highlightlines=14]{../src/backtrace/camlBacktrace\_\_definitely\_raise\_347.objdump}
      }
    \end{column}
  \end{columns}
\end{frame}

\begin{frame}{Backtraces}{Introducing \funcname{caml\_raise\_exn}}
  \begin{columns}[c]
    \begin{column}{0.5\textwidth}
      \minipage[c][0.66\textheight][s]{\columnwidth}
      \onslide*<1>{
        \begin{itemize}
          \item Raising the exception is performed using \funcname{caml\_raise\_exn} which is
            \begin{itemize}
              \item An assembly function provided by the runtime
              \item Architecture specific
            \end{itemize}
          \item Let's see what it does differently from \funcname{raise\_notrace}\dots
          \item We are about to raise \typename{ExnC} with \funcname{caml\_raise\_exn} from \funcname{definitely\_raise}
        \end{itemize}
      }
      \onslide*<2>{
        \mintobjdump[firstline=2,highlightlines=14]{../src/backtrace/camlBacktrace\_\_definitely\_raise\_347.objdump}
      }
      \vfill
      \mintocaml[firstline=9,lastline=13,highlightlines=12]{../src/backtrace/backtrace.ml}
      \endminipage
    \end{column}
    \begin{column}{0.5\textwidth}
      \centering
      \providecommand\step{1}\input{diagrams/backtrace/backtrace.tex}
    \end{column}
  \end{columns}
\end{frame}

\begin{frame}{Backtraces}{But before that, we need more fields from \typename{caml\_domain\_state}}
  \begin{columns}
    \begin{column}{0.5\textwidth}
      \begin{itemize}
        \item Remember \typename{caml\_domain\_state} the per-domain runtime state?
        \item Some other members are of interest to us now\dots
        \item \localname{backtrace\_active} is used to know if the runtime should collect backtraces
        \item \localname{backtrace\_active} can be set by
          \begin{itemize}
            \item The \funcarg{b} flag in the \funcname{OCAMLRUNPARAM} environment variable
            \item \mintinline{ocaml}{Printexc.record_backtrace : bool -> unit} \footnotemark
          \end{itemize}
      \end{itemize}
    \end{column}
    \begin{column}{0.5\textwidth}
      \begin{listing}
        \mintc[highlightlines={9-11}]{src/ocaml/domain\_state\_backtrace.h}
        \caption{runtime/caml/domain\_state.h}
      \end{listing}
    \end{column}
  \end{columns}
  \footnotetext[1]{https://v2.ocaml.org/api/Printexc.html}
\end{frame}

\newcommand\listCamlRaiseExn[1][]{
  \listasm[firstline=702,lastline=725,#1]{../ocaml/runtime/amd64.S}{runtime/amd64.S:702}}

\begin{frame}{Backtraces}{Walk through \funcname{caml\_raise\_exn}}
  \begin{columns}[T]
    \begin{column}{0.5\textwidth}
      \begin{itemize}
        \item<1-> First checks whether exceptions backtrace should be recorded
        \item<1-> By checking the \funcarg{domain\_state->backtrace\_active} value
        \item<2-> If not, acts as \funcname{raise\_notrace} with \funcname{RESTORE\_EXN\_HANDLER\_OCAML}
          \begin{itemize}
            \item Set \regname{RSP} to \localname{domain\_state->exn\_handler} value
            \item Restore \localname{domain\_state->exn\_handler} from trap
            \item Jump with \funcname{ret} (same effect as using \funcname{pop} and \funcname{jmp})
          \end{itemize}
        \item<3-> Otherwise, switches to the C stack to be able to execute C code
        \item<4-> Calls \funcname{caml\_stash\_backtrace}, a C function provided by the runtime\ignorespaces
          \onslide<6->{, with
            \begin{itemize}
              \item \funcarg{exn}: exception value
              \item \funcarg{pc}: code address of the raising point
              \item \funcarg{sp}: stack address of the raising function
              \item \funcarg{trapsp}: stack address of the next handler
            \end{itemize}
          }
      \end{itemize}
      \bigskip
      \mintocaml[firstline=9,lastline=13,highlightlines=12]{../src/backtrace/backtrace.ml}
    \end{column}
    \begin{column}{0.5\textwidth}
      \raggedleft
      \onslide*<1,3-4>{
        \mintc[firstline=190,lastline=192]{../ocaml/runtime/amd64.S}
        \mintc[firstline=234,lastline=237]{../ocaml/runtime/amd64.S}
      }
      \onslide*<2>{
        \mintc[firstline=190,lastline=192,highlightlines={191-192}]{../ocaml/runtime/amd64.S}
        \mintc[firstline=234,lastline=237,highlightlines={235-237}]{../ocaml/runtime/amd64.S}
      }
      \onslide*<1>{\listCamlRaiseExn[highlightlines=705]{}}%
      \onslide*<2>{\listCamlRaiseExn[highlightlines={707-708}]{}}%
      \onslide*<3>{\listCamlRaiseExn[highlightlines={712-714}]{}}%
      \onslide*<4>{\listCamlRaiseExn[highlightlines={715-720}]{}}%
      \onslide*<5>{\providecommand\step{1}\input{diagrams/backtrace/backtrace.tex}}%
      \onslide*<6>{\providecommand\step{2}\input{diagrams/backtrace/backtrace.tex}}%
    \end{column}
  \end{columns}
\end{frame}

\newcommand\listStashBacktrace[1][]{
  \listc[firstline=91,lastline=120,#1]{../ocaml/runtime/backtrace\_nat.c}{runtime/backtrace\_nat.c:91}}

\begin{frame}{Backtraces}{Introducing \funcname{caml\_stash\_backtrace}}
  \begin{columns}[c]
    \begin{column}{0.5\textwidth}
      \minipage[c][0.66\textheight][s]{\columnwidth}
      \begin{itemize}
        \item<1-> A C function provided by the runtime (specific to native code)
        \item<2-> It scans the stack, between the stack pointer at the raising point up to the next trap, in order to collect the names of the detected function frames
          \onslide*<2>{
            \begin{itemize}
              \item And more information, like line number, column number...
            \end{itemize}
          }
        \item<3-> It uses the same technique and information as the Garbage Collector (GC) uses to scan the values live on the stack
          \onslide*<4>{
            \begin{itemize}
              \item \typename{frame\_descr} are at the core of stack unwinding
            \end{itemize}
          }
      \end{itemize}
      \vfill
      \mintocaml[firstline=9,lastline=13,highlightlines=12]{../src/backtrace/backtrace.ml}
      \endminipage
    \end{column}
    \begin{column}{0.5\textwidth}
      \onslide*<1>{\listStashBacktrace[highlightlines=91]{}}
      \onslide*<2>{\listStashBacktrace[highlightlines={91,117-118}]{}}
      \onslide*<3->{\listStashBacktrace[highlightlines={94,106,109-110}]{}}
    \end{column}
  \end{columns}
\end{frame}

\newcommand\listFrameDescr[1][]{
  \listocaml[firstline=111,lastline=124,#1]{../ocaml/asmcomp/emitaux.ml}{asmcomp/emitaux.ml:111}}
\newcommand\listDebugInfo[1][]{
  \listocaml[firstline=38,lastline=49,#1]{../ocaml/lambda/debuginfo.mli}{lambda/debuginfo.mli:38}}

\begin{frame}{Backtraces}{\typename{frame\_descr} and \typename{frame\_debuginfo} in a nutshell}
  \begin{columns}[c]
    \begin{column}{0.5\textwidth}
      \begin{itemize}
        \item<1-> For every return address in an OCaml program, there is a associated \typename{frame\_descr}
        \item<2-> A \typename{frame\_descr} specifies
        \begin{itemize}
          \item<2-> The size of the function stack frame
          \item<3-> Where live values are on the stack (so that the GC can mark them as live and prevent from being swept)
        \end{itemize}
        \item<4-> When compiled with debug information, \typename{frame\_descr} will be linked to a \typename{frame\_debuginfo}
        \begin{itemize}
          \item<5-> Contains information about the position in the source code (file name, line and column number)
        \end{itemize}
      \end{itemize}
    \end{column}
    \begin{column}{0.5\textwidth}
        \onslide*<1>{\listFrameDescr[highlightlines={118-122}]{}}
        \onslide*<2>{\listFrameDescr[highlightlines={120}]{}}
        \onslide*<3>{\listFrameDescr[highlightlines={121}]{}}
        \onslide*<4->{\listFrameDescr[highlightlines={122}]{}}
        \medskip
        \onslide*<1-3>{\listDebugInfo{}}
        \onslide*<4>{\listDebugInfo[highlightlines={38,49}]{}}
        \onslide*<5>{\listDebugInfo[highlightlines={39-46}]{}}
    \end{column}
  \end{columns}
\end{frame}

\begin{frame}{Backtraces}{Scanning the stack with \typename{frame\_descr}}
  \begin{columns}[c]
    \begin{column}{0.5\textwidth}
      \begin{itemize}
        \item Given the return address of the code that raises the exception by calling \funcname{caml\_raise\_exn} (\funcarg{pc} argument of \funcname{caml\_stash\_backtrace})
        \item And the position of the stack pointer (\funcarg{sp} argument of \funcname{caml\_stash\_backtrace})
        \item The frame size given by \typename{frame\_descr}.\funcarg{frame\_size}, allows to find the end of the previous frame
        \item And the return address of the caller of this function
        \bigskip
        \onslide<3->{
        \item Note that traps (here the reddish \funcname{ExnA}, \funcname{ExnB} and \funcname{ExnC} blocks) belong to the frame that installed them, and so the frame size changes when traps are removed
        }
      \end{itemize}
    \end{column}
    \begin{column}{0.5\textwidth}
      \raggedleft
      \onslide*<1>{\providecommand\step{2}\input{diagrams/backtrace/backtrace.tex}}%
      \onslide*<2>{\providecommand\step{1}\input{diagrams/backtrace/scanstack.tex}}%
      \onslide*<3>{\providecommand\step{2}\input{diagrams/backtrace/scanstack.tex}}%
      \onslide*<4>{\providecommand\step{3}\input{diagrams/backtrace/scanstack.tex}}%
      \onslide*<5>{\providecommand\step{4}\input{diagrams/backtrace/scanstack.tex}}%
      \onslide*<6>{\providecommand\step{5}\input{diagrams/backtrace/scanstack.tex}}%
    \end{column}
  \end{columns}
\end{frame}

% Duplicate of previous-previous slide
\begin{frame}{Backtraces}{Continuing on \funcname{caml\_stash\_backtrace}}
  \begin{columns}[c]
    \begin{column}{0.5\textwidth}
      \minipage[c][0.75\textheight][s]{\columnwidth}
      \begin{itemize}
          \color{gray}
        \item A C function provided by the runtime (specific to native code)
        \item It scans the stack, between the stack pointer at the raising point up to the next trap, in order to collect the names of the detected function frames
        \item It uses the same technique and information as the Garbage Collector (GC) uses to scan the values live on the stack
      \end{itemize}
      \begin{itemize}
        \item For each function frame detected, store a pointer to the \typename{frame\_descr} in the \localname{domain\_state->backtrace\_buffer} array, effectively constructing the backtrace
      \end{itemize}
      \onslide<2->{
        \begin{itemize}
            \smallskip
          \item Constructed backtrace:
            \funcname{definitely\_raise},
            \onslide<3->{\funcname{probably\_raise}}
        \end{itemize}
      }
      \vfill
      \mintocaml[firstline=9,lastline=13,highlightlines=12]{../src/backtrace/backtrace.ml}
      \endminipage
    \end{column}
    \begin{column}{0.5\textwidth}
      \raggedleft
      \onslide*<1>{\listStashBacktrace[highlightlines={114-115}]{}}
      \onslide*<2>{\providecommand\step{3}\input{diagrams/backtrace/backtrace.tex}}%
      \onslide*<3>{\providecommand\step{4}\input{diagrams/backtrace/backtrace.tex}}%
    \end{column}
  \end{columns}
\end{frame}

\begin{frame}{Backtraces}{Back to \funcname{caml\_raise\_exn}}
  \begin{columns}[c]
    \begin{column}{0.5\textwidth}
      \minipage[c][0.66\textheight][s]{\columnwidth}
      \begin{itemize}\color{gray}
        \item Checks wheteher exceptions backtrace should be recorded, by checking the \funcarg{domain\_state->backtrace\_active} value
        \item Switches to the C stack to be able to execute C code
        \item Calls \funcname{caml\_stash\_backtrace}, to collect the backtrace up to the next trap
      \end{itemize}
      \onslide*<2->{
        \begin{itemize}
          \item<2-> Executes the exception handler in \funcname{probably\_raise} with \funcname{RESTORE\_EXN\_HANDLER\_OCAML}
            \begin{itemize}
              \item Set \regname{RSP} to \localname{domain\_state->exn\_handler} value
              \item Restore \localname{domain\_state->exn\_handler} from trap
              \item Jump to the handler code with \funcname{ret}
            \end{itemize}
        \end{itemize}
      }
      \vfill
      \onslide*<1>{\mintocaml[firstline=9,lastline=13,highlightlines=12]{../src/backtrace/backtrace.ml}}
      \onslide*<2->{\mintocaml[firstline=15,lastline=21,highlightlines={19-21}]{../src/backtrace/backtrace.ml}}
      \endminipage
    \end{column}
    \begin{column}{0.5\textwidth}
      \centering
      \onslide*<1>{
        \mintc[firstline=190,lastline=192]{../ocaml/runtime/amd64.S}
        \mintc[firstline=234,lastline=237]{../ocaml/runtime/amd64.S}
      }
      \onslide*<2>{
        \mintc[firstline=190,lastline=192,highlightlines={191-192}]{../ocaml/runtime/amd64.S}
        \mintc[firstline=234,lastline=237,highlightlines={235-237}]{../ocaml/runtime/amd64.S}
      }
      \onslide*<1>{\listCamlRaiseExn[highlightlines=720]{}}
      \onslide*<2>{\listCamlRaiseExn[highlightlines=722]{}}
      \onslide*<3->{\providecommand\step{5}\input{diagrams/backtrace/backtrace.tex}}
    \end{column}
  \end{columns}
\end{frame}

\begin{frame}{Backtraces}{\funcname{caml\_probably\_raise}'s handler doesn't catch \typename{ExnC}}
  \begin{columns}[c]
    \begin{column}{0.45\textwidth}
      \minipage[c][0.75\textheight][s]{\columnwidth}
      \begin{itemize}
        \item<1-> \funcname{match \dots with}\footnotemark pattern matching can also match exceptions
        \item<1-> The CMM of \funcname{probably\_raise} shows that this \funcname{match \dots with} is implemented with a pattern matching and a \funcname{try \dots with}
        \item<1-> But this one only matches on \typename{ExnA} exception
        \item<2-> Since the pattern matching fails to catch the exception, \funcname{reraise} is called with our current \typename{ExnC} exception
        \item<3-> Disassembly shows that \funcname{reraise} is actually a call to \funcname{caml\_reraise\_exn}, which is also an assembly function provided by the runtime
      \end{itemize}
      \vfill
      \mintocaml[firstline=15,lastline=21,highlightlines={18-21}]{../src/backtrace/backtrace.ml}
      \endminipage
    \end{column}
    \begin{column}{0.55\textwidth}
      \centering
      \onslide*<1>{\mintcmm[highlightlines={10-18}]{../src/backtrace/camlBacktrace\_\_probably\_raise\_351.cmm}}
      \onslide*<2>{\listcmm[highlightlines=18]{../src/backtrace/camlBacktrace\_\_probably\_raise_351.cmm}{CMM of probably\_raise}}
      \onslide*<3>{\mintobjdump[firstline=5,lastline=35,highlightlines=31]{../src/backtrace/camlBacktrace\_\_probably\_raise_351.objdump}}
    \end{column}
  \end{columns}
  \only<1>{\footnotetext[1]{https://blog.janestreet.com/pattern-matching-and-exception-handling-unite/}}
\end{frame}

\newcommand\listCamlReraiseExn[1][]{
  \listasm[firstline=727,lastline=734,#1]{../ocaml/runtime/amd64.S}{runtime/amd64.S:727}}

\begin{frame}{Backtraces}{Introducing \funcname{caml\_reraise\_exn}}
  \begin{columns}[c]
    \begin{column}{0.5\textwidth}
      \minipage[c][0.66\textheight][s]{\columnwidth}
      \begin{itemize}
        \item<1-> The exception have to be reraised to the next handler
        \item<2-> First, checks if the backtrace should be collected with \funcarg{domain\_state->backtrace\_active}
        \only<3>{
        \begin{itemize}
            \item If not, behaves like a \funcname{raise\_notrace} with \funcname{RESTORE\_EXN\_HANDLER\_OCAML}
        \end{itemize}
        }
        \item<4-> Jumps into \funcname{caml\_raise\_exn}
        \item<5-> Calls \funcname{caml\_stash\_backtrace} again, to scan the next part of the stack: from the trap just pop-ed to the next trap
      \end{itemize}
      \vfill
      \mintocaml[firstline=15,lastline=21,highlightlines={18-21}]{../src/backtrace/backtrace.ml}
      \endminipage
    \end{column}
    \begin{column}{0.5\textwidth}
      \raggedleft
      \onslide*<1>{\listCamlReraiseExn{}}
      \onslide*<2>{\listCamlReraiseExn[highlightlines=729]{}}
      \onslide*<3>{
        \mintc[firstline=190,lastline=192,highlightlines={191-192}]{../ocaml/runtime/amd64.S}
        \mintc[firstline=234,lastline=237,highlightlines={235-237}]{../ocaml/runtime/amd64.S}
        \medskip
        \listCamlReraiseExn[highlightlines=731]{}
      }
      \onslide*<4-5>{
        % TODO refactor with \listCamlReraiseExn
        \mintasm[firstline=727,lastline=734,highlightlines=730]{../ocaml/runtime/amd64.S}
        \medskip
      }
      \onslide*<4>{\listCamlRaiseExn[highlightlines=711]{}}
      \onslide*<5>{\listCamlRaiseExn[highlightlines=720]{}}
    \end{column}
  \end{columns}
\end{frame}

\begin{frame}{Backtraces}{Backtrace collection continues}
  \begin{columns}[c]
    \begin{column}{0.55\textwidth}
      \minipage[c][0.75\textheight][s]{\columnwidth}
      \begin{itemize}
        \item<1-> \funcname{caml\_stash\_backtrace} scans the older part of the stack, up to the next trap
        \item<6-> Controls jumps to the nested exception handler in \funcname{maybe\_raise}, which matches only on \typename{ExnB}
        \item<7-> \funcname{caml\_reraise\_exn} is called again, backtrace collection resumes with \funcname{caml\_stash\_backtrace}
      \end{itemize}
      \medskip
      \begin{itemize}
        \item<2-> Constructed backtrace:
        \begin{itemize}
          \item <2-> \funcname{definitely\_raise}
          \item <2-> \funcname{probably\_raise}
          \item <3-> \funcname{probably\_raise} (re-raise)
          \item <4-> \funcname{maybe\_raise}
          \item <5-> \funcname{main}
          \item <8-> \funcname{main} (re-raise)
        \end{itemize}
      \end{itemize}
      \vfill
      \onslide*<1-5>{\mintocaml[firstline=15,lastline=21]{../src/backtrace/backtrace.ml}}
      \onslide*<6>{\mintocaml[firstline=28,lastline=34,highlightlines=31]{../src/backtrace/backtrace.ml}}
      \onslide*<7->{\mintocaml[firstline=28,lastline=34,highlightlines=33]{../src/backtrace/backtrace.ml}}
      \endminipage
    \end{column}
    \begin{column}{0.45\textwidth}
      \raggedleft
      \onslide*<1>{\providecommand\step{6}\input{diagrams/backtrace/backtrace.tex}}%
      \onslide*<2>{\providecommand\step{6}\input{diagrams/backtrace/backtrace.tex}}%
      \onslide*<3>{\providecommand\step{7}\input{diagrams/backtrace/backtrace.tex}}%
      \onslide*<4>{\providecommand\step{8}\input{diagrams/backtrace/backtrace.tex}}%
      \onslide*<5>{\providecommand\step{9}\input{diagrams/backtrace/backtrace.tex}}%
      \onslide*<6>{\providecommand\step{10}\input{diagrams/backtrace/backtrace.tex}}%
      \onslide*<7>{\providecommand\step{11}\input{diagrams/backtrace/backtrace.tex}}%
      \onslide*<8>{\providecommand\step{12}\input{diagrams/backtrace/backtrace.tex}}%
      \onslide*<9>{\providecommand\step{13}\input{diagrams/backtrace/backtrace.tex}}%
    \end{column}
  \end{columns}
\end{frame}

\begin{frame}{Backtraces}{Printing the backtrace}
  \begin{columns}[c]
    \begin{column}{0.45\textwidth}
      \minipage[c][0.66\textheight][s]{\columnwidth}
      \begin{itemize}
        \item<1-> The exception backtrace can now be printed to \funcarg{stdout} with \funcname{Printexc.print_backtrace}
        \item<2-> First the exception backtrace is retrieved into an OCaml array with \funcname{get_raw_backtrace }
        \item<3-> Which is implemented as the \funcname{caml_get_exception_raw_backtrace} C function in the runtime
        \item<4-> And finally printed with \funcname{print_exception_backtrace}
      \end{itemize}
      \vfill
      \mintocaml[firstline=28,lastline=34,highlightlines=33]{../src/backtrace/backtrace.ml}
      \endminipage
    \end{column}
    \begin{column}{0.55\textwidth}
      \onslide*<1,2>{
        \mintocaml[firstline=100,lastline=101]{../ocaml/stdlib/printexc.ml}
      }
      \only<1>{
        \listocaml[firstline=170,lastline=172]{../ocaml/stdlib/printexc.ml}{stdlib/printexc.ml:170}
      }
      \only<2>{
        \listocaml[firstline=170,lastline=172,highlightlines=172]{../ocaml/stdlib/printexc.ml}{stdlib/printexc.ml:170}
      }
      \only<3>{
        \mintc[firstline=154,lastline=156]{../ocaml/runtime/backtrace.c}
        \listc[firstline=171,lastline=192,highlightlines={184-187}]{../ocaml/runtime/backtrace.c}{runtime/backtrace.c:154}
      }
      \only<4>{
        \listocaml[firstline=155,lastline=165]{../ocaml/stdlib/printexc.ml}{stdlib/printexc.ml:55}
      }
    \end{column}
  \end{columns}
\end{frame}


\subsection{The multiple kinds of raise}
\frameSubsection{}{}

\begin{frame}{Backtraces}{When backtraces are not needed}
  \begin{columns}[c]
    \begin{column}{0.45\textwidth}
      \minipage[c][0.66\textheight][s]{\columnwidth}
      \begin{itemize}
        \item<1-> What if the backtrace for an exception is never needed?
        \item<2-> It's possible to raise the exception explicitly with \funcname{raise\_notrace} to make raising as cheap as possible
        \item<3-> An example of use in the \funcname{dynlink} library of the OCaml compiler
      \end{itemize}
      \vfill
      \onslide<3->{
      \listocaml[firstline=205,lastline=209,highlightlines=207]{../ocaml/otherlibs/dynlink/dynlink\_compilerlibs/misc.ml}{otherlibs/dynlink/dynlink\_compilerlibs/misc.ml:205}
      }
      \endminipage
    \end{column}
    \begin{column}{0.55\textwidth}
    \only<4>{
      \mintcmm[highlightlines=3]{../src/notrace/camlNotrace\_\_fun_347.cmm}
      \listcmm{../src/notrace/camlNotrace\_\_all\_somes\_327.cmm}{all\_somes}
    }
    \onslide*<5->{
      \listobjdump[highlightlines={6-9}]{../src/notrace/camlNotrace\_\_fun_347.objdump}{anonymous function from \funcname{all\_somes}}
    }
    \end{column}
  \end{columns}
\end{frame}

\begin{frame}{Backtraces}{Summary table of the multiple kinds of raise}
  \begin{itemize}
    \item When debugging information is enabled and if the backtrace of an exception isn't needed, it's possible to raise with \funcname{raise\_notrace} for performance
    \item When debugging information is \emph{not} enabled, the compiler optimises \funcname{raise} calls to inline assembly (same effect as using \funcname{raise\_notrace})
  \end{itemize}
  \bigskip
  \centering
  \begin{tabular}{| l | c | c | c | c |}
    \hline
    \parbox{10em}{Debug information \\ (\funcarg{-g} flag)}  & \multicolumn{2}{|c|}{Disabled}                                     & \multicolumn{2}{|c|}{Enabled} \\
    \hline
    Raise kind                                               & \funcname{raise} & \funcname{raise\_notrace}                       & \funcname{raise} & \funcname{raise\_notrace} \\
    \hline
    Compilation                                              & \parbox{8em}{inline assembly\\ (optimization)} & inline assembly   & \funcname{caml\_raise\_exn} & inline assembly \\
    \hline
  \end{tabular}
\end{frame}


%
% Takeaway #5
%
\subsection*{Takeaway \#5}
\frameSubsectionTakeaway{}
\againframe<5>{takeaway}
}%
    \end{column}
  \end{columns}
\end{frame}

\begin{frame}{Backtraces}{Printing the backtrace}
  \begin{columns}[c]
    \begin{column}{0.45\textwidth}
      \minipage[c][0.66\textheight][s]{\columnwidth}
      \begin{itemize}
        \item<1-> The exception backtrace can now be printed to \funcarg{stdout} with \funcname{Printexc.print_backtrace}
        \item<2-> First the exception backtrace is retrieved into an OCaml array with \funcname{get_raw_backtrace }
        \item<3-> Which is implemented as the \funcname{caml_get_exception_raw_backtrace} C function in the runtime
        \item<4-> And finally printed with \funcname{print_exception_backtrace}
      \end{itemize}
      \vfill
      \mintocaml[firstline=28,lastline=34,highlightlines=33]{../src/backtrace/backtrace.ml}
      \endminipage
    \end{column}
    \begin{column}{0.55\textwidth}
      \onslide*<1,2>{
        \mintocaml[firstline=100,lastline=101]{../ocaml/stdlib/printexc.ml}
      }
      \only<1>{
        \listocaml[firstline=170,lastline=172]{../ocaml/stdlib/printexc.ml}{stdlib/printexc.ml:170}
      }
      \only<2>{
        \listocaml[firstline=170,lastline=172,highlightlines=172]{../ocaml/stdlib/printexc.ml}{stdlib/printexc.ml:170}
      }
      \only<3>{
        \mintc[firstline=154,lastline=156]{../ocaml/runtime/backtrace.c}
        \listc[firstline=171,lastline=192,highlightlines={184-187}]{../ocaml/runtime/backtrace.c}{runtime/backtrace.c:154}
      }
      \only<4>{
        \listocaml[firstline=155,lastline=165]{../ocaml/stdlib/printexc.ml}{stdlib/printexc.ml:55}
      }
    \end{column}
  \end{columns}
\end{frame}


\subsection{The multiple kinds of raise}
\frameSubsection{}{}

\begin{frame}{Backtraces}{When backtraces are not needed}
  \begin{columns}[c]
    \begin{column}{0.45\textwidth}
      \minipage[c][0.66\textheight][s]{\columnwidth}
      \begin{itemize}
        \item<1-> What if the backtrace for an exception is never needed?
        \item<2-> It's possible to raise the exception explicitly with \funcname{raise\_notrace} to make raising as cheap as possible
        \item<3-> An example of use in the \funcname{dynlink} library of the OCaml compiler
      \end{itemize}
      \vfill
      \onslide<3->{
      \listocaml[firstline=205,lastline=209,highlightlines=207]{../ocaml/otherlibs/dynlink/dynlink\_compilerlibs/misc.ml}{otherlibs/dynlink/dynlink\_compilerlibs/misc.ml:205}
      }
      \endminipage
    \end{column}
    \begin{column}{0.55\textwidth}
    \only<4>{
      \mintcmm[highlightlines=3]{../src/notrace/camlNotrace\_\_fun_347.cmm}
      \listcmm{../src/notrace/camlNotrace\_\_all\_somes\_327.cmm}{all\_somes}
    }
    \onslide*<5->{
      \listobjdump[highlightlines={6-9}]{../src/notrace/camlNotrace\_\_fun_347.objdump}{anonymous function from \funcname{all\_somes}}
    }
    \end{column}
  \end{columns}
\end{frame}

\begin{frame}{Backtraces}{Summary table of the multiple kinds of raise}
  \begin{itemize}
    \item When debugging information is enabled and if the backtrace of an exception isn't needed, it's possible to raise with \funcname{raise\_notrace} for performance
    \item When debugging information is \emph{not} enabled, the compiler optimises \funcname{raise} calls to inline assembly (same effect as using \funcname{raise\_notrace})
  \end{itemize}
  \bigskip
  \centering
  \begin{tabular}{| l | c | c | c | c |}
    \hline
    \parbox{10em}{Debug information \\ (\funcarg{-g} flag)}  & \multicolumn{2}{|c|}{Disabled}                                     & \multicolumn{2}{|c|}{Enabled} \\
    \hline
    Raise kind                                               & \funcname{raise} & \funcname{raise\_notrace}                       & \funcname{raise} & \funcname{raise\_notrace} \\
    \hline
    Compilation                                              & \parbox{8em}{inline assembly\\ (optimization)} & inline assembly   & \funcname{caml\_raise\_exn} & inline assembly \\
    \hline
  \end{tabular}
\end{frame}


%
% Takeaway #5
%
\subsection*{Takeaway \#5}
\frameSubsectionTakeaway{}
\againframe<5>{takeaway}
}%
      \onslide*<9>{\providecommand\step{13}%
% Backtraces
%
\subsection{One advantage of raising with debugging information}
\frameSubsection{}{}

\begin{frame}{Backtraces}{Debugging information \& source code}
  \begin{columns}[c]
    \begin{column}{0.5\textwidth}
      \begin{itemize}
        \item Raising an exception is fun and games, but how do we know where the exception originated? $\rightarrow$ With the backtrace associated with the exception!
        \item In order to get a backtrace from the point where an exception was raised, it's necessary to compile with debugging information enabled (\funcarg{-g} flag for the compiler)
        \item Same example as in the `Nested handlers' section
      \end{itemize}
    \end{column}
    \begin{column}{0.5\textwidth}
      \listocaml[firstline=9,lastline=34]{../src/backtrace/backtrace.ml}{backtrace/backtrace.ml}
    \end{column}
  \end{columns}
\end{frame}

\begin{frame}{Backtraces}{CMM and disassembly of \funcname{definitely\_raise}}
  \begin{columns}[c]
    \begin{column}{0.5\textwidth}
      \minipage[c][0.66\textheight][s]{\columnwidth}
      \begin{itemize}
        \item<1-> An effect of compiling with debugging information (\funcarg{-g}): the CMM no longer uses \funcname{raise\_notrace} to raise the exception but \funcname{raise} instead
        \item<2-> Disassembly shows that CMM \funcname{raise} is actually a call to \funcname{caml\_raise\_exn}, a function in assembler provided by the runtime
      \end{itemize}
      \vfill
      \mintocaml[firstline=9,lastline=13,highlightlines=12]{../src/backtrace/backtrace.ml}
      \endminipage
    \end{column}
    \begin{column}{0.5\textwidth}
      \onslide*<1>{
        \mintcmm[highlightlines=5]{../src/backtrace/camlBacktrace\_\_definitely\_raise\_347.cmm}
      }
      \onslide*<2>{
        \mintobjdump[firstline=2,highlightlines=14]{../src/backtrace/camlBacktrace\_\_definitely\_raise\_347.objdump}
      }
    \end{column}
  \end{columns}
\end{frame}

\begin{frame}{Backtraces}{Introducing \funcname{caml\_raise\_exn}}
  \begin{columns}[c]
    \begin{column}{0.5\textwidth}
      \minipage[c][0.66\textheight][s]{\columnwidth}
      \onslide*<1>{
        \begin{itemize}
          \item Raising the exception is performed using \funcname{caml\_raise\_exn} which is
            \begin{itemize}
              \item An assembly function provided by the runtime
              \item Architecture specific
            \end{itemize}
          \item Let's see what it does differently from \funcname{raise\_notrace}\dots
          \item We are about to raise \typename{ExnC} with \funcname{caml\_raise\_exn} from \funcname{definitely\_raise}
        \end{itemize}
      }
      \onslide*<2>{
        \mintobjdump[firstline=2,highlightlines=14]{../src/backtrace/camlBacktrace\_\_definitely\_raise\_347.objdump}
      }
      \vfill
      \mintocaml[firstline=9,lastline=13,highlightlines=12]{../src/backtrace/backtrace.ml}
      \endminipage
    \end{column}
    \begin{column}{0.5\textwidth}
      \centering
      \providecommand\step{1}%
% Backtraces
%
\subsection{One advantage of raising with debugging information}
\frameSubsection{}{}

\begin{frame}{Backtraces}{Debugging information \& source code}
  \begin{columns}[c]
    \begin{column}{0.5\textwidth}
      \begin{itemize}
        \item Raising an exception is fun and games, but how do we know where the exception originated? $\rightarrow$ With the backtrace associated with the exception!
        \item In order to get a backtrace from the point where an exception was raised, it's necessary to compile with debugging information enabled (\funcarg{-g} flag for the compiler)
        \item Same example as in the `Nested handlers' section
      \end{itemize}
    \end{column}
    \begin{column}{0.5\textwidth}
      \listocaml[firstline=9,lastline=34]{../src/backtrace/backtrace.ml}{backtrace/backtrace.ml}
    \end{column}
  \end{columns}
\end{frame}

\begin{frame}{Backtraces}{CMM and disassembly of \funcname{definitely\_raise}}
  \begin{columns}[c]
    \begin{column}{0.5\textwidth}
      \minipage[c][0.66\textheight][s]{\columnwidth}
      \begin{itemize}
        \item<1-> An effect of compiling with debugging information (\funcarg{-g}): the CMM no longer uses \funcname{raise\_notrace} to raise the exception but \funcname{raise} instead
        \item<2-> Disassembly shows that CMM \funcname{raise} is actually a call to \funcname{caml\_raise\_exn}, a function in assembler provided by the runtime
      \end{itemize}
      \vfill
      \mintocaml[firstline=9,lastline=13,highlightlines=12]{../src/backtrace/backtrace.ml}
      \endminipage
    \end{column}
    \begin{column}{0.5\textwidth}
      \onslide*<1>{
        \mintcmm[highlightlines=5]{../src/backtrace/camlBacktrace\_\_definitely\_raise\_347.cmm}
      }
      \onslide*<2>{
        \mintobjdump[firstline=2,highlightlines=14]{../src/backtrace/camlBacktrace\_\_definitely\_raise\_347.objdump}
      }
    \end{column}
  \end{columns}
\end{frame}

\begin{frame}{Backtraces}{Introducing \funcname{caml\_raise\_exn}}
  \begin{columns}[c]
    \begin{column}{0.5\textwidth}
      \minipage[c][0.66\textheight][s]{\columnwidth}
      \onslide*<1>{
        \begin{itemize}
          \item Raising the exception is performed using \funcname{caml\_raise\_exn} which is
            \begin{itemize}
              \item An assembly function provided by the runtime
              \item Architecture specific
            \end{itemize}
          \item Let's see what it does differently from \funcname{raise\_notrace}\dots
          \item We are about to raise \typename{ExnC} with \funcname{caml\_raise\_exn} from \funcname{definitely\_raise}
        \end{itemize}
      }
      \onslide*<2>{
        \mintobjdump[firstline=2,highlightlines=14]{../src/backtrace/camlBacktrace\_\_definitely\_raise\_347.objdump}
      }
      \vfill
      \mintocaml[firstline=9,lastline=13,highlightlines=12]{../src/backtrace/backtrace.ml}
      \endminipage
    \end{column}
    \begin{column}{0.5\textwidth}
      \centering
      \providecommand\step{1}\input{diagrams/backtrace/backtrace.tex}
    \end{column}
  \end{columns}
\end{frame}

\begin{frame}{Backtraces}{But before that, we need more fields from \typename{caml\_domain\_state}}
  \begin{columns}
    \begin{column}{0.5\textwidth}
      \begin{itemize}
        \item Remember \typename{caml\_domain\_state} the per-domain runtime state?
        \item Some other members are of interest to us now\dots
        \item \localname{backtrace\_active} is used to know if the runtime should collect backtraces
        \item \localname{backtrace\_active} can be set by
          \begin{itemize}
            \item The \funcarg{b} flag in the \funcname{OCAMLRUNPARAM} environment variable
            \item \mintinline{ocaml}{Printexc.record_backtrace : bool -> unit} \footnotemark
          \end{itemize}
      \end{itemize}
    \end{column}
    \begin{column}{0.5\textwidth}
      \begin{listing}
        \mintc[highlightlines={9-11}]{src/ocaml/domain\_state\_backtrace.h}
        \caption{runtime/caml/domain\_state.h}
      \end{listing}
    \end{column}
  \end{columns}
  \footnotetext[1]{https://v2.ocaml.org/api/Printexc.html}
\end{frame}

\newcommand\listCamlRaiseExn[1][]{
  \listasm[firstline=702,lastline=725,#1]{../ocaml/runtime/amd64.S}{runtime/amd64.S:702}}

\begin{frame}{Backtraces}{Walk through \funcname{caml\_raise\_exn}}
  \begin{columns}[T]
    \begin{column}{0.5\textwidth}
      \begin{itemize}
        \item<1-> First checks whether exceptions backtrace should be recorded
        \item<1-> By checking the \funcarg{domain\_state->backtrace\_active} value
        \item<2-> If not, acts as \funcname{raise\_notrace} with \funcname{RESTORE\_EXN\_HANDLER\_OCAML}
          \begin{itemize}
            \item Set \regname{RSP} to \localname{domain\_state->exn\_handler} value
            \item Restore \localname{domain\_state->exn\_handler} from trap
            \item Jump with \funcname{ret} (same effect as using \funcname{pop} and \funcname{jmp})
          \end{itemize}
        \item<3-> Otherwise, switches to the C stack to be able to execute C code
        \item<4-> Calls \funcname{caml\_stash\_backtrace}, a C function provided by the runtime\ignorespaces
          \onslide<6->{, with
            \begin{itemize}
              \item \funcarg{exn}: exception value
              \item \funcarg{pc}: code address of the raising point
              \item \funcarg{sp}: stack address of the raising function
              \item \funcarg{trapsp}: stack address of the next handler
            \end{itemize}
          }
      \end{itemize}
      \bigskip
      \mintocaml[firstline=9,lastline=13,highlightlines=12]{../src/backtrace/backtrace.ml}
    \end{column}
    \begin{column}{0.5\textwidth}
      \raggedleft
      \onslide*<1,3-4>{
        \mintc[firstline=190,lastline=192]{../ocaml/runtime/amd64.S}
        \mintc[firstline=234,lastline=237]{../ocaml/runtime/amd64.S}
      }
      \onslide*<2>{
        \mintc[firstline=190,lastline=192,highlightlines={191-192}]{../ocaml/runtime/amd64.S}
        \mintc[firstline=234,lastline=237,highlightlines={235-237}]{../ocaml/runtime/amd64.S}
      }
      \onslide*<1>{\listCamlRaiseExn[highlightlines=705]{}}%
      \onslide*<2>{\listCamlRaiseExn[highlightlines={707-708}]{}}%
      \onslide*<3>{\listCamlRaiseExn[highlightlines={712-714}]{}}%
      \onslide*<4>{\listCamlRaiseExn[highlightlines={715-720}]{}}%
      \onslide*<5>{\providecommand\step{1}\input{diagrams/backtrace/backtrace.tex}}%
      \onslide*<6>{\providecommand\step{2}\input{diagrams/backtrace/backtrace.tex}}%
    \end{column}
  \end{columns}
\end{frame}

\newcommand\listStashBacktrace[1][]{
  \listc[firstline=91,lastline=120,#1]{../ocaml/runtime/backtrace\_nat.c}{runtime/backtrace\_nat.c:91}}

\begin{frame}{Backtraces}{Introducing \funcname{caml\_stash\_backtrace}}
  \begin{columns}[c]
    \begin{column}{0.5\textwidth}
      \minipage[c][0.66\textheight][s]{\columnwidth}
      \begin{itemize}
        \item<1-> A C function provided by the runtime (specific to native code)
        \item<2-> It scans the stack, between the stack pointer at the raising point up to the next trap, in order to collect the names of the detected function frames
          \onslide*<2>{
            \begin{itemize}
              \item And more information, like line number, column number...
            \end{itemize}
          }
        \item<3-> It uses the same technique and information as the Garbage Collector (GC) uses to scan the values live on the stack
          \onslide*<4>{
            \begin{itemize}
              \item \typename{frame\_descr} are at the core of stack unwinding
            \end{itemize}
          }
      \end{itemize}
      \vfill
      \mintocaml[firstline=9,lastline=13,highlightlines=12]{../src/backtrace/backtrace.ml}
      \endminipage
    \end{column}
    \begin{column}{0.5\textwidth}
      \onslide*<1>{\listStashBacktrace[highlightlines=91]{}}
      \onslide*<2>{\listStashBacktrace[highlightlines={91,117-118}]{}}
      \onslide*<3->{\listStashBacktrace[highlightlines={94,106,109-110}]{}}
    \end{column}
  \end{columns}
\end{frame}

\newcommand\listFrameDescr[1][]{
  \listocaml[firstline=111,lastline=124,#1]{../ocaml/asmcomp/emitaux.ml}{asmcomp/emitaux.ml:111}}
\newcommand\listDebugInfo[1][]{
  \listocaml[firstline=38,lastline=49,#1]{../ocaml/lambda/debuginfo.mli}{lambda/debuginfo.mli:38}}

\begin{frame}{Backtraces}{\typename{frame\_descr} and \typename{frame\_debuginfo} in a nutshell}
  \begin{columns}[c]
    \begin{column}{0.5\textwidth}
      \begin{itemize}
        \item<1-> For every return address in an OCaml program, there is a associated \typename{frame\_descr}
        \item<2-> A \typename{frame\_descr} specifies
        \begin{itemize}
          \item<2-> The size of the function stack frame
          \item<3-> Where live values are on the stack (so that the GC can mark them as live and prevent from being swept)
        \end{itemize}
        \item<4-> When compiled with debug information, \typename{frame\_descr} will be linked to a \typename{frame\_debuginfo}
        \begin{itemize}
          \item<5-> Contains information about the position in the source code (file name, line and column number)
        \end{itemize}
      \end{itemize}
    \end{column}
    \begin{column}{0.5\textwidth}
        \onslide*<1>{\listFrameDescr[highlightlines={118-122}]{}}
        \onslide*<2>{\listFrameDescr[highlightlines={120}]{}}
        \onslide*<3>{\listFrameDescr[highlightlines={121}]{}}
        \onslide*<4->{\listFrameDescr[highlightlines={122}]{}}
        \medskip
        \onslide*<1-3>{\listDebugInfo{}}
        \onslide*<4>{\listDebugInfo[highlightlines={38,49}]{}}
        \onslide*<5>{\listDebugInfo[highlightlines={39-46}]{}}
    \end{column}
  \end{columns}
\end{frame}

\begin{frame}{Backtraces}{Scanning the stack with \typename{frame\_descr}}
  \begin{columns}[c]
    \begin{column}{0.5\textwidth}
      \begin{itemize}
        \item Given the return address of the code that raises the exception by calling \funcname{caml\_raise\_exn} (\funcarg{pc} argument of \funcname{caml\_stash\_backtrace})
        \item And the position of the stack pointer (\funcarg{sp} argument of \funcname{caml\_stash\_backtrace})
        \item The frame size given by \typename{frame\_descr}.\funcarg{frame\_size}, allows to find the end of the previous frame
        \item And the return address of the caller of this function
        \bigskip
        \onslide<3->{
        \item Note that traps (here the reddish \funcname{ExnA}, \funcname{ExnB} and \funcname{ExnC} blocks) belong to the frame that installed them, and so the frame size changes when traps are removed
        }
      \end{itemize}
    \end{column}
    \begin{column}{0.5\textwidth}
      \raggedleft
      \onslide*<1>{\providecommand\step{2}\input{diagrams/backtrace/backtrace.tex}}%
      \onslide*<2>{\providecommand\step{1}\input{diagrams/backtrace/scanstack.tex}}%
      \onslide*<3>{\providecommand\step{2}\input{diagrams/backtrace/scanstack.tex}}%
      \onslide*<4>{\providecommand\step{3}\input{diagrams/backtrace/scanstack.tex}}%
      \onslide*<5>{\providecommand\step{4}\input{diagrams/backtrace/scanstack.tex}}%
      \onslide*<6>{\providecommand\step{5}\input{diagrams/backtrace/scanstack.tex}}%
    \end{column}
  \end{columns}
\end{frame}

% Duplicate of previous-previous slide
\begin{frame}{Backtraces}{Continuing on \funcname{caml\_stash\_backtrace}}
  \begin{columns}[c]
    \begin{column}{0.5\textwidth}
      \minipage[c][0.75\textheight][s]{\columnwidth}
      \begin{itemize}
          \color{gray}
        \item A C function provided by the runtime (specific to native code)
        \item It scans the stack, between the stack pointer at the raising point up to the next trap, in order to collect the names of the detected function frames
        \item It uses the same technique and information as the Garbage Collector (GC) uses to scan the values live on the stack
      \end{itemize}
      \begin{itemize}
        \item For each function frame detected, store a pointer to the \typename{frame\_descr} in the \localname{domain\_state->backtrace\_buffer} array, effectively constructing the backtrace
      \end{itemize}
      \onslide<2->{
        \begin{itemize}
            \smallskip
          \item Constructed backtrace:
            \funcname{definitely\_raise},
            \onslide<3->{\funcname{probably\_raise}}
        \end{itemize}
      }
      \vfill
      \mintocaml[firstline=9,lastline=13,highlightlines=12]{../src/backtrace/backtrace.ml}
      \endminipage
    \end{column}
    \begin{column}{0.5\textwidth}
      \raggedleft
      \onslide*<1>{\listStashBacktrace[highlightlines={114-115}]{}}
      \onslide*<2>{\providecommand\step{3}\input{diagrams/backtrace/backtrace.tex}}%
      \onslide*<3>{\providecommand\step{4}\input{diagrams/backtrace/backtrace.tex}}%
    \end{column}
  \end{columns}
\end{frame}

\begin{frame}{Backtraces}{Back to \funcname{caml\_raise\_exn}}
  \begin{columns}[c]
    \begin{column}{0.5\textwidth}
      \minipage[c][0.66\textheight][s]{\columnwidth}
      \begin{itemize}\color{gray}
        \item Checks wheteher exceptions backtrace should be recorded, by checking the \funcarg{domain\_state->backtrace\_active} value
        \item Switches to the C stack to be able to execute C code
        \item Calls \funcname{caml\_stash\_backtrace}, to collect the backtrace up to the next trap
      \end{itemize}
      \onslide*<2->{
        \begin{itemize}
          \item<2-> Executes the exception handler in \funcname{probably\_raise} with \funcname{RESTORE\_EXN\_HANDLER\_OCAML}
            \begin{itemize}
              \item Set \regname{RSP} to \localname{domain\_state->exn\_handler} value
              \item Restore \localname{domain\_state->exn\_handler} from trap
              \item Jump to the handler code with \funcname{ret}
            \end{itemize}
        \end{itemize}
      }
      \vfill
      \onslide*<1>{\mintocaml[firstline=9,lastline=13,highlightlines=12]{../src/backtrace/backtrace.ml}}
      \onslide*<2->{\mintocaml[firstline=15,lastline=21,highlightlines={19-21}]{../src/backtrace/backtrace.ml}}
      \endminipage
    \end{column}
    \begin{column}{0.5\textwidth}
      \centering
      \onslide*<1>{
        \mintc[firstline=190,lastline=192]{../ocaml/runtime/amd64.S}
        \mintc[firstline=234,lastline=237]{../ocaml/runtime/amd64.S}
      }
      \onslide*<2>{
        \mintc[firstline=190,lastline=192,highlightlines={191-192}]{../ocaml/runtime/amd64.S}
        \mintc[firstline=234,lastline=237,highlightlines={235-237}]{../ocaml/runtime/amd64.S}
      }
      \onslide*<1>{\listCamlRaiseExn[highlightlines=720]{}}
      \onslide*<2>{\listCamlRaiseExn[highlightlines=722]{}}
      \onslide*<3->{\providecommand\step{5}\input{diagrams/backtrace/backtrace.tex}}
    \end{column}
  \end{columns}
\end{frame}

\begin{frame}{Backtraces}{\funcname{caml\_probably\_raise}'s handler doesn't catch \typename{ExnC}}
  \begin{columns}[c]
    \begin{column}{0.45\textwidth}
      \minipage[c][0.75\textheight][s]{\columnwidth}
      \begin{itemize}
        \item<1-> \funcname{match \dots with}\footnotemark pattern matching can also match exceptions
        \item<1-> The CMM of \funcname{probably\_raise} shows that this \funcname{match \dots with} is implemented with a pattern matching and a \funcname{try \dots with}
        \item<1-> But this one only matches on \typename{ExnA} exception
        \item<2-> Since the pattern matching fails to catch the exception, \funcname{reraise} is called with our current \typename{ExnC} exception
        \item<3-> Disassembly shows that \funcname{reraise} is actually a call to \funcname{caml\_reraise\_exn}, which is also an assembly function provided by the runtime
      \end{itemize}
      \vfill
      \mintocaml[firstline=15,lastline=21,highlightlines={18-21}]{../src/backtrace/backtrace.ml}
      \endminipage
    \end{column}
    \begin{column}{0.55\textwidth}
      \centering
      \onslide*<1>{\mintcmm[highlightlines={10-18}]{../src/backtrace/camlBacktrace\_\_probably\_raise\_351.cmm}}
      \onslide*<2>{\listcmm[highlightlines=18]{../src/backtrace/camlBacktrace\_\_probably\_raise_351.cmm}{CMM of probably\_raise}}
      \onslide*<3>{\mintobjdump[firstline=5,lastline=35,highlightlines=31]{../src/backtrace/camlBacktrace\_\_probably\_raise_351.objdump}}
    \end{column}
  \end{columns}
  \only<1>{\footnotetext[1]{https://blog.janestreet.com/pattern-matching-and-exception-handling-unite/}}
\end{frame}

\newcommand\listCamlReraiseExn[1][]{
  \listasm[firstline=727,lastline=734,#1]{../ocaml/runtime/amd64.S}{runtime/amd64.S:727}}

\begin{frame}{Backtraces}{Introducing \funcname{caml\_reraise\_exn}}
  \begin{columns}[c]
    \begin{column}{0.5\textwidth}
      \minipage[c][0.66\textheight][s]{\columnwidth}
      \begin{itemize}
        \item<1-> The exception have to be reraised to the next handler
        \item<2-> First, checks if the backtrace should be collected with \funcarg{domain\_state->backtrace\_active}
        \only<3>{
        \begin{itemize}
            \item If not, behaves like a \funcname{raise\_notrace} with \funcname{RESTORE\_EXN\_HANDLER\_OCAML}
        \end{itemize}
        }
        \item<4-> Jumps into \funcname{caml\_raise\_exn}
        \item<5-> Calls \funcname{caml\_stash\_backtrace} again, to scan the next part of the stack: from the trap just pop-ed to the next trap
      \end{itemize}
      \vfill
      \mintocaml[firstline=15,lastline=21,highlightlines={18-21}]{../src/backtrace/backtrace.ml}
      \endminipage
    \end{column}
    \begin{column}{0.5\textwidth}
      \raggedleft
      \onslide*<1>{\listCamlReraiseExn{}}
      \onslide*<2>{\listCamlReraiseExn[highlightlines=729]{}}
      \onslide*<3>{
        \mintc[firstline=190,lastline=192,highlightlines={191-192}]{../ocaml/runtime/amd64.S}
        \mintc[firstline=234,lastline=237,highlightlines={235-237}]{../ocaml/runtime/amd64.S}
        \medskip
        \listCamlReraiseExn[highlightlines=731]{}
      }
      \onslide*<4-5>{
        % TODO refactor with \listCamlReraiseExn
        \mintasm[firstline=727,lastline=734,highlightlines=730]{../ocaml/runtime/amd64.S}
        \medskip
      }
      \onslide*<4>{\listCamlRaiseExn[highlightlines=711]{}}
      \onslide*<5>{\listCamlRaiseExn[highlightlines=720]{}}
    \end{column}
  \end{columns}
\end{frame}

\begin{frame}{Backtraces}{Backtrace collection continues}
  \begin{columns}[c]
    \begin{column}{0.55\textwidth}
      \minipage[c][0.75\textheight][s]{\columnwidth}
      \begin{itemize}
        \item<1-> \funcname{caml\_stash\_backtrace} scans the older part of the stack, up to the next trap
        \item<6-> Controls jumps to the nested exception handler in \funcname{maybe\_raise}, which matches only on \typename{ExnB}
        \item<7-> \funcname{caml\_reraise\_exn} is called again, backtrace collection resumes with \funcname{caml\_stash\_backtrace}
      \end{itemize}
      \medskip
      \begin{itemize}
        \item<2-> Constructed backtrace:
        \begin{itemize}
          \item <2-> \funcname{definitely\_raise}
          \item <2-> \funcname{probably\_raise}
          \item <3-> \funcname{probably\_raise} (re-raise)
          \item <4-> \funcname{maybe\_raise}
          \item <5-> \funcname{main}
          \item <8-> \funcname{main} (re-raise)
        \end{itemize}
      \end{itemize}
      \vfill
      \onslide*<1-5>{\mintocaml[firstline=15,lastline=21]{../src/backtrace/backtrace.ml}}
      \onslide*<6>{\mintocaml[firstline=28,lastline=34,highlightlines=31]{../src/backtrace/backtrace.ml}}
      \onslide*<7->{\mintocaml[firstline=28,lastline=34,highlightlines=33]{../src/backtrace/backtrace.ml}}
      \endminipage
    \end{column}
    \begin{column}{0.45\textwidth}
      \raggedleft
      \onslide*<1>{\providecommand\step{6}\input{diagrams/backtrace/backtrace.tex}}%
      \onslide*<2>{\providecommand\step{6}\input{diagrams/backtrace/backtrace.tex}}%
      \onslide*<3>{\providecommand\step{7}\input{diagrams/backtrace/backtrace.tex}}%
      \onslide*<4>{\providecommand\step{8}\input{diagrams/backtrace/backtrace.tex}}%
      \onslide*<5>{\providecommand\step{9}\input{diagrams/backtrace/backtrace.tex}}%
      \onslide*<6>{\providecommand\step{10}\input{diagrams/backtrace/backtrace.tex}}%
      \onslide*<7>{\providecommand\step{11}\input{diagrams/backtrace/backtrace.tex}}%
      \onslide*<8>{\providecommand\step{12}\input{diagrams/backtrace/backtrace.tex}}%
      \onslide*<9>{\providecommand\step{13}\input{diagrams/backtrace/backtrace.tex}}%
    \end{column}
  \end{columns}
\end{frame}

\begin{frame}{Backtraces}{Printing the backtrace}
  \begin{columns}[c]
    \begin{column}{0.45\textwidth}
      \minipage[c][0.66\textheight][s]{\columnwidth}
      \begin{itemize}
        \item<1-> The exception backtrace can now be printed to \funcarg{stdout} with \funcname{Printexc.print_backtrace}
        \item<2-> First the exception backtrace is retrieved into an OCaml array with \funcname{get_raw_backtrace }
        \item<3-> Which is implemented as the \funcname{caml_get_exception_raw_backtrace} C function in the runtime
        \item<4-> And finally printed with \funcname{print_exception_backtrace}
      \end{itemize}
      \vfill
      \mintocaml[firstline=28,lastline=34,highlightlines=33]{../src/backtrace/backtrace.ml}
      \endminipage
    \end{column}
    \begin{column}{0.55\textwidth}
      \onslide*<1,2>{
        \mintocaml[firstline=100,lastline=101]{../ocaml/stdlib/printexc.ml}
      }
      \only<1>{
        \listocaml[firstline=170,lastline=172]{../ocaml/stdlib/printexc.ml}{stdlib/printexc.ml:170}
      }
      \only<2>{
        \listocaml[firstline=170,lastline=172,highlightlines=172]{../ocaml/stdlib/printexc.ml}{stdlib/printexc.ml:170}
      }
      \only<3>{
        \mintc[firstline=154,lastline=156]{../ocaml/runtime/backtrace.c}
        \listc[firstline=171,lastline=192,highlightlines={184-187}]{../ocaml/runtime/backtrace.c}{runtime/backtrace.c:154}
      }
      \only<4>{
        \listocaml[firstline=155,lastline=165]{../ocaml/stdlib/printexc.ml}{stdlib/printexc.ml:55}
      }
    \end{column}
  \end{columns}
\end{frame}


\subsection{The multiple kinds of raise}
\frameSubsection{}{}

\begin{frame}{Backtraces}{When backtraces are not needed}
  \begin{columns}[c]
    \begin{column}{0.45\textwidth}
      \minipage[c][0.66\textheight][s]{\columnwidth}
      \begin{itemize}
        \item<1-> What if the backtrace for an exception is never needed?
        \item<2-> It's possible to raise the exception explicitly with \funcname{raise\_notrace} to make raising as cheap as possible
        \item<3-> An example of use in the \funcname{dynlink} library of the OCaml compiler
      \end{itemize}
      \vfill
      \onslide<3->{
      \listocaml[firstline=205,lastline=209,highlightlines=207]{../ocaml/otherlibs/dynlink/dynlink\_compilerlibs/misc.ml}{otherlibs/dynlink/dynlink\_compilerlibs/misc.ml:205}
      }
      \endminipage
    \end{column}
    \begin{column}{0.55\textwidth}
    \only<4>{
      \mintcmm[highlightlines=3]{../src/notrace/camlNotrace\_\_fun_347.cmm}
      \listcmm{../src/notrace/camlNotrace\_\_all\_somes\_327.cmm}{all\_somes}
    }
    \onslide*<5->{
      \listobjdump[highlightlines={6-9}]{../src/notrace/camlNotrace\_\_fun_347.objdump}{anonymous function from \funcname{all\_somes}}
    }
    \end{column}
  \end{columns}
\end{frame}

\begin{frame}{Backtraces}{Summary table of the multiple kinds of raise}
  \begin{itemize}
    \item When debugging information is enabled and if the backtrace of an exception isn't needed, it's possible to raise with \funcname{raise\_notrace} for performance
    \item When debugging information is \emph{not} enabled, the compiler optimises \funcname{raise} calls to inline assembly (same effect as using \funcname{raise\_notrace})
  \end{itemize}
  \bigskip
  \centering
  \begin{tabular}{| l | c | c | c | c |}
    \hline
    \parbox{10em}{Debug information \\ (\funcarg{-g} flag)}  & \multicolumn{2}{|c|}{Disabled}                                     & \multicolumn{2}{|c|}{Enabled} \\
    \hline
    Raise kind                                               & \funcname{raise} & \funcname{raise\_notrace}                       & \funcname{raise} & \funcname{raise\_notrace} \\
    \hline
    Compilation                                              & \parbox{8em}{inline assembly\\ (optimization)} & inline assembly   & \funcname{caml\_raise\_exn} & inline assembly \\
    \hline
  \end{tabular}
\end{frame}


%
% Takeaway #5
%
\subsection*{Takeaway \#5}
\frameSubsectionTakeaway{}
\againframe<5>{takeaway}

    \end{column}
  \end{columns}
\end{frame}

\begin{frame}{Backtraces}{But before that, we need more fields from \typename{caml\_domain\_state}}
  \begin{columns}
    \begin{column}{0.5\textwidth}
      \begin{itemize}
        \item Remember \typename{caml\_domain\_state} the per-domain runtime state?
        \item Some other members are of interest to us now\dots
        \item \localname{backtrace\_active} is used to know if the runtime should collect backtraces
        \item \localname{backtrace\_active} can be set by
          \begin{itemize}
            \item The \funcarg{b} flag in the \funcname{OCAMLRUNPARAM} environment variable
            \item \mintinline{ocaml}{Printexc.record_backtrace : bool -> unit} \footnotemark
          \end{itemize}
      \end{itemize}
    \end{column}
    \begin{column}{0.5\textwidth}
      \begin{listing}
        \mintc[highlightlines={9-11}]{src/ocaml/domain\_state\_backtrace.h}
        \caption{runtime/caml/domain\_state.h}
      \end{listing}
    \end{column}
  \end{columns}
  \footnotetext[1]{https://v2.ocaml.org/api/Printexc.html}
\end{frame}

\newcommand\listCamlRaiseExn[1][]{
  \listasm[firstline=702,lastline=725,#1]{../ocaml/runtime/amd64.S}{runtime/amd64.S:702}}

\begin{frame}{Backtraces}{Walk through \funcname{caml\_raise\_exn}}
  \begin{columns}[T]
    \begin{column}{0.5\textwidth}
      \begin{itemize}
        \item<1-> First checks whether exceptions backtrace should be recorded
        \item<1-> By checking the \funcarg{domain\_state->backtrace\_active} value
        \item<2-> If not, acts as \funcname{raise\_notrace} with \funcname{RESTORE\_EXN\_HANDLER\_OCAML}
          \begin{itemize}
            \item Set \regname{RSP} to \localname{domain\_state->exn\_handler} value
            \item Restore \localname{domain\_state->exn\_handler} from trap
            \item Jump with \funcname{ret} (same effect as using \funcname{pop} and \funcname{jmp})
          \end{itemize}
        \item<3-> Otherwise, switches to the C stack to be able to execute C code
        \item<4-> Calls \funcname{caml\_stash\_backtrace}, a C function provided by the runtime\ignorespaces
          \onslide<6->{, with
            \begin{itemize}
              \item \funcarg{exn}: exception value
              \item \funcarg{pc}: code address of the raising point
              \item \funcarg{sp}: stack address of the raising function
              \item \funcarg{trapsp}: stack address of the next handler
            \end{itemize}
          }
      \end{itemize}
      \bigskip
      \mintocaml[firstline=9,lastline=13,highlightlines=12]{../src/backtrace/backtrace.ml}
    \end{column}
    \begin{column}{0.5\textwidth}
      \raggedleft
      \onslide*<1,3-4>{
        \mintc[firstline=190,lastline=192]{../ocaml/runtime/amd64.S}
        \mintc[firstline=234,lastline=237]{../ocaml/runtime/amd64.S}
      }
      \onslide*<2>{
        \mintc[firstline=190,lastline=192,highlightlines={191-192}]{../ocaml/runtime/amd64.S}
        \mintc[firstline=234,lastline=237,highlightlines={235-237}]{../ocaml/runtime/amd64.S}
      }
      \onslide*<1>{\listCamlRaiseExn[highlightlines=705]{}}%
      \onslide*<2>{\listCamlRaiseExn[highlightlines={707-708}]{}}%
      \onslide*<3>{\listCamlRaiseExn[highlightlines={712-714}]{}}%
      \onslide*<4>{\listCamlRaiseExn[highlightlines={715-720}]{}}%
      \onslide*<5>{\providecommand\step{1}%
% Backtraces
%
\subsection{One advantage of raising with debugging information}
\frameSubsection{}{}

\begin{frame}{Backtraces}{Debugging information \& source code}
  \begin{columns}[c]
    \begin{column}{0.5\textwidth}
      \begin{itemize}
        \item Raising an exception is fun and games, but how do we know where the exception originated? $\rightarrow$ With the backtrace associated with the exception!
        \item In order to get a backtrace from the point where an exception was raised, it's necessary to compile with debugging information enabled (\funcarg{-g} flag for the compiler)
        \item Same example as in the `Nested handlers' section
      \end{itemize}
    \end{column}
    \begin{column}{0.5\textwidth}
      \listocaml[firstline=9,lastline=34]{../src/backtrace/backtrace.ml}{backtrace/backtrace.ml}
    \end{column}
  \end{columns}
\end{frame}

\begin{frame}{Backtraces}{CMM and disassembly of \funcname{definitely\_raise}}
  \begin{columns}[c]
    \begin{column}{0.5\textwidth}
      \minipage[c][0.66\textheight][s]{\columnwidth}
      \begin{itemize}
        \item<1-> An effect of compiling with debugging information (\funcarg{-g}): the CMM no longer uses \funcname{raise\_notrace} to raise the exception but \funcname{raise} instead
        \item<2-> Disassembly shows that CMM \funcname{raise} is actually a call to \funcname{caml\_raise\_exn}, a function in assembler provided by the runtime
      \end{itemize}
      \vfill
      \mintocaml[firstline=9,lastline=13,highlightlines=12]{../src/backtrace/backtrace.ml}
      \endminipage
    \end{column}
    \begin{column}{0.5\textwidth}
      \onslide*<1>{
        \mintcmm[highlightlines=5]{../src/backtrace/camlBacktrace\_\_definitely\_raise\_347.cmm}
      }
      \onslide*<2>{
        \mintobjdump[firstline=2,highlightlines=14]{../src/backtrace/camlBacktrace\_\_definitely\_raise\_347.objdump}
      }
    \end{column}
  \end{columns}
\end{frame}

\begin{frame}{Backtraces}{Introducing \funcname{caml\_raise\_exn}}
  \begin{columns}[c]
    \begin{column}{0.5\textwidth}
      \minipage[c][0.66\textheight][s]{\columnwidth}
      \onslide*<1>{
        \begin{itemize}
          \item Raising the exception is performed using \funcname{caml\_raise\_exn} which is
            \begin{itemize}
              \item An assembly function provided by the runtime
              \item Architecture specific
            \end{itemize}
          \item Let's see what it does differently from \funcname{raise\_notrace}\dots
          \item We are about to raise \typename{ExnC} with \funcname{caml\_raise\_exn} from \funcname{definitely\_raise}
        \end{itemize}
      }
      \onslide*<2>{
        \mintobjdump[firstline=2,highlightlines=14]{../src/backtrace/camlBacktrace\_\_definitely\_raise\_347.objdump}
      }
      \vfill
      \mintocaml[firstline=9,lastline=13,highlightlines=12]{../src/backtrace/backtrace.ml}
      \endminipage
    \end{column}
    \begin{column}{0.5\textwidth}
      \centering
      \providecommand\step{1}\input{diagrams/backtrace/backtrace.tex}
    \end{column}
  \end{columns}
\end{frame}

\begin{frame}{Backtraces}{But before that, we need more fields from \typename{caml\_domain\_state}}
  \begin{columns}
    \begin{column}{0.5\textwidth}
      \begin{itemize}
        \item Remember \typename{caml\_domain\_state} the per-domain runtime state?
        \item Some other members are of interest to us now\dots
        \item \localname{backtrace\_active} is used to know if the runtime should collect backtraces
        \item \localname{backtrace\_active} can be set by
          \begin{itemize}
            \item The \funcarg{b} flag in the \funcname{OCAMLRUNPARAM} environment variable
            \item \mintinline{ocaml}{Printexc.record_backtrace : bool -> unit} \footnotemark
          \end{itemize}
      \end{itemize}
    \end{column}
    \begin{column}{0.5\textwidth}
      \begin{listing}
        \mintc[highlightlines={9-11}]{src/ocaml/domain\_state\_backtrace.h}
        \caption{runtime/caml/domain\_state.h}
      \end{listing}
    \end{column}
  \end{columns}
  \footnotetext[1]{https://v2.ocaml.org/api/Printexc.html}
\end{frame}

\newcommand\listCamlRaiseExn[1][]{
  \listasm[firstline=702,lastline=725,#1]{../ocaml/runtime/amd64.S}{runtime/amd64.S:702}}

\begin{frame}{Backtraces}{Walk through \funcname{caml\_raise\_exn}}
  \begin{columns}[T]
    \begin{column}{0.5\textwidth}
      \begin{itemize}
        \item<1-> First checks whether exceptions backtrace should be recorded
        \item<1-> By checking the \funcarg{domain\_state->backtrace\_active} value
        \item<2-> If not, acts as \funcname{raise\_notrace} with \funcname{RESTORE\_EXN\_HANDLER\_OCAML}
          \begin{itemize}
            \item Set \regname{RSP} to \localname{domain\_state->exn\_handler} value
            \item Restore \localname{domain\_state->exn\_handler} from trap
            \item Jump with \funcname{ret} (same effect as using \funcname{pop} and \funcname{jmp})
          \end{itemize}
        \item<3-> Otherwise, switches to the C stack to be able to execute C code
        \item<4-> Calls \funcname{caml\_stash\_backtrace}, a C function provided by the runtime\ignorespaces
          \onslide<6->{, with
            \begin{itemize}
              \item \funcarg{exn}: exception value
              \item \funcarg{pc}: code address of the raising point
              \item \funcarg{sp}: stack address of the raising function
              \item \funcarg{trapsp}: stack address of the next handler
            \end{itemize}
          }
      \end{itemize}
      \bigskip
      \mintocaml[firstline=9,lastline=13,highlightlines=12]{../src/backtrace/backtrace.ml}
    \end{column}
    \begin{column}{0.5\textwidth}
      \raggedleft
      \onslide*<1,3-4>{
        \mintc[firstline=190,lastline=192]{../ocaml/runtime/amd64.S}
        \mintc[firstline=234,lastline=237]{../ocaml/runtime/amd64.S}
      }
      \onslide*<2>{
        \mintc[firstline=190,lastline=192,highlightlines={191-192}]{../ocaml/runtime/amd64.S}
        \mintc[firstline=234,lastline=237,highlightlines={235-237}]{../ocaml/runtime/amd64.S}
      }
      \onslide*<1>{\listCamlRaiseExn[highlightlines=705]{}}%
      \onslide*<2>{\listCamlRaiseExn[highlightlines={707-708}]{}}%
      \onslide*<3>{\listCamlRaiseExn[highlightlines={712-714}]{}}%
      \onslide*<4>{\listCamlRaiseExn[highlightlines={715-720}]{}}%
      \onslide*<5>{\providecommand\step{1}\input{diagrams/backtrace/backtrace.tex}}%
      \onslide*<6>{\providecommand\step{2}\input{diagrams/backtrace/backtrace.tex}}%
    \end{column}
  \end{columns}
\end{frame}

\newcommand\listStashBacktrace[1][]{
  \listc[firstline=91,lastline=120,#1]{../ocaml/runtime/backtrace\_nat.c}{runtime/backtrace\_nat.c:91}}

\begin{frame}{Backtraces}{Introducing \funcname{caml\_stash\_backtrace}}
  \begin{columns}[c]
    \begin{column}{0.5\textwidth}
      \minipage[c][0.66\textheight][s]{\columnwidth}
      \begin{itemize}
        \item<1-> A C function provided by the runtime (specific to native code)
        \item<2-> It scans the stack, between the stack pointer at the raising point up to the next trap, in order to collect the names of the detected function frames
          \onslide*<2>{
            \begin{itemize}
              \item And more information, like line number, column number...
            \end{itemize}
          }
        \item<3-> It uses the same technique and information as the Garbage Collector (GC) uses to scan the values live on the stack
          \onslide*<4>{
            \begin{itemize}
              \item \typename{frame\_descr} are at the core of stack unwinding
            \end{itemize}
          }
      \end{itemize}
      \vfill
      \mintocaml[firstline=9,lastline=13,highlightlines=12]{../src/backtrace/backtrace.ml}
      \endminipage
    \end{column}
    \begin{column}{0.5\textwidth}
      \onslide*<1>{\listStashBacktrace[highlightlines=91]{}}
      \onslide*<2>{\listStashBacktrace[highlightlines={91,117-118}]{}}
      \onslide*<3->{\listStashBacktrace[highlightlines={94,106,109-110}]{}}
    \end{column}
  \end{columns}
\end{frame}

\newcommand\listFrameDescr[1][]{
  \listocaml[firstline=111,lastline=124,#1]{../ocaml/asmcomp/emitaux.ml}{asmcomp/emitaux.ml:111}}
\newcommand\listDebugInfo[1][]{
  \listocaml[firstline=38,lastline=49,#1]{../ocaml/lambda/debuginfo.mli}{lambda/debuginfo.mli:38}}

\begin{frame}{Backtraces}{\typename{frame\_descr} and \typename{frame\_debuginfo} in a nutshell}
  \begin{columns}[c]
    \begin{column}{0.5\textwidth}
      \begin{itemize}
        \item<1-> For every return address in an OCaml program, there is a associated \typename{frame\_descr}
        \item<2-> A \typename{frame\_descr} specifies
        \begin{itemize}
          \item<2-> The size of the function stack frame
          \item<3-> Where live values are on the stack (so that the GC can mark them as live and prevent from being swept)
        \end{itemize}
        \item<4-> When compiled with debug information, \typename{frame\_descr} will be linked to a \typename{frame\_debuginfo}
        \begin{itemize}
          \item<5-> Contains information about the position in the source code (file name, line and column number)
        \end{itemize}
      \end{itemize}
    \end{column}
    \begin{column}{0.5\textwidth}
        \onslide*<1>{\listFrameDescr[highlightlines={118-122}]{}}
        \onslide*<2>{\listFrameDescr[highlightlines={120}]{}}
        \onslide*<3>{\listFrameDescr[highlightlines={121}]{}}
        \onslide*<4->{\listFrameDescr[highlightlines={122}]{}}
        \medskip
        \onslide*<1-3>{\listDebugInfo{}}
        \onslide*<4>{\listDebugInfo[highlightlines={38,49}]{}}
        \onslide*<5>{\listDebugInfo[highlightlines={39-46}]{}}
    \end{column}
  \end{columns}
\end{frame}

\begin{frame}{Backtraces}{Scanning the stack with \typename{frame\_descr}}
  \begin{columns}[c]
    \begin{column}{0.5\textwidth}
      \begin{itemize}
        \item Given the return address of the code that raises the exception by calling \funcname{caml\_raise\_exn} (\funcarg{pc} argument of \funcname{caml\_stash\_backtrace})
        \item And the position of the stack pointer (\funcarg{sp} argument of \funcname{caml\_stash\_backtrace})
        \item The frame size given by \typename{frame\_descr}.\funcarg{frame\_size}, allows to find the end of the previous frame
        \item And the return address of the caller of this function
        \bigskip
        \onslide<3->{
        \item Note that traps (here the reddish \funcname{ExnA}, \funcname{ExnB} and \funcname{ExnC} blocks) belong to the frame that installed them, and so the frame size changes when traps are removed
        }
      \end{itemize}
    \end{column}
    \begin{column}{0.5\textwidth}
      \raggedleft
      \onslide*<1>{\providecommand\step{2}\input{diagrams/backtrace/backtrace.tex}}%
      \onslide*<2>{\providecommand\step{1}\input{diagrams/backtrace/scanstack.tex}}%
      \onslide*<3>{\providecommand\step{2}\input{diagrams/backtrace/scanstack.tex}}%
      \onslide*<4>{\providecommand\step{3}\input{diagrams/backtrace/scanstack.tex}}%
      \onslide*<5>{\providecommand\step{4}\input{diagrams/backtrace/scanstack.tex}}%
      \onslide*<6>{\providecommand\step{5}\input{diagrams/backtrace/scanstack.tex}}%
    \end{column}
  \end{columns}
\end{frame}

% Duplicate of previous-previous slide
\begin{frame}{Backtraces}{Continuing on \funcname{caml\_stash\_backtrace}}
  \begin{columns}[c]
    \begin{column}{0.5\textwidth}
      \minipage[c][0.75\textheight][s]{\columnwidth}
      \begin{itemize}
          \color{gray}
        \item A C function provided by the runtime (specific to native code)
        \item It scans the stack, between the stack pointer at the raising point up to the next trap, in order to collect the names of the detected function frames
        \item It uses the same technique and information as the Garbage Collector (GC) uses to scan the values live on the stack
      \end{itemize}
      \begin{itemize}
        \item For each function frame detected, store a pointer to the \typename{frame\_descr} in the \localname{domain\_state->backtrace\_buffer} array, effectively constructing the backtrace
      \end{itemize}
      \onslide<2->{
        \begin{itemize}
            \smallskip
          \item Constructed backtrace:
            \funcname{definitely\_raise},
            \onslide<3->{\funcname{probably\_raise}}
        \end{itemize}
      }
      \vfill
      \mintocaml[firstline=9,lastline=13,highlightlines=12]{../src/backtrace/backtrace.ml}
      \endminipage
    \end{column}
    \begin{column}{0.5\textwidth}
      \raggedleft
      \onslide*<1>{\listStashBacktrace[highlightlines={114-115}]{}}
      \onslide*<2>{\providecommand\step{3}\input{diagrams/backtrace/backtrace.tex}}%
      \onslide*<3>{\providecommand\step{4}\input{diagrams/backtrace/backtrace.tex}}%
    \end{column}
  \end{columns}
\end{frame}

\begin{frame}{Backtraces}{Back to \funcname{caml\_raise\_exn}}
  \begin{columns}[c]
    \begin{column}{0.5\textwidth}
      \minipage[c][0.66\textheight][s]{\columnwidth}
      \begin{itemize}\color{gray}
        \item Checks wheteher exceptions backtrace should be recorded, by checking the \funcarg{domain\_state->backtrace\_active} value
        \item Switches to the C stack to be able to execute C code
        \item Calls \funcname{caml\_stash\_backtrace}, to collect the backtrace up to the next trap
      \end{itemize}
      \onslide*<2->{
        \begin{itemize}
          \item<2-> Executes the exception handler in \funcname{probably\_raise} with \funcname{RESTORE\_EXN\_HANDLER\_OCAML}
            \begin{itemize}
              \item Set \regname{RSP} to \localname{domain\_state->exn\_handler} value
              \item Restore \localname{domain\_state->exn\_handler} from trap
              \item Jump to the handler code with \funcname{ret}
            \end{itemize}
        \end{itemize}
      }
      \vfill
      \onslide*<1>{\mintocaml[firstline=9,lastline=13,highlightlines=12]{../src/backtrace/backtrace.ml}}
      \onslide*<2->{\mintocaml[firstline=15,lastline=21,highlightlines={19-21}]{../src/backtrace/backtrace.ml}}
      \endminipage
    \end{column}
    \begin{column}{0.5\textwidth}
      \centering
      \onslide*<1>{
        \mintc[firstline=190,lastline=192]{../ocaml/runtime/amd64.S}
        \mintc[firstline=234,lastline=237]{../ocaml/runtime/amd64.S}
      }
      \onslide*<2>{
        \mintc[firstline=190,lastline=192,highlightlines={191-192}]{../ocaml/runtime/amd64.S}
        \mintc[firstline=234,lastline=237,highlightlines={235-237}]{../ocaml/runtime/amd64.S}
      }
      \onslide*<1>{\listCamlRaiseExn[highlightlines=720]{}}
      \onslide*<2>{\listCamlRaiseExn[highlightlines=722]{}}
      \onslide*<3->{\providecommand\step{5}\input{diagrams/backtrace/backtrace.tex}}
    \end{column}
  \end{columns}
\end{frame}

\begin{frame}{Backtraces}{\funcname{caml\_probably\_raise}'s handler doesn't catch \typename{ExnC}}
  \begin{columns}[c]
    \begin{column}{0.45\textwidth}
      \minipage[c][0.75\textheight][s]{\columnwidth}
      \begin{itemize}
        \item<1-> \funcname{match \dots with}\footnotemark pattern matching can also match exceptions
        \item<1-> The CMM of \funcname{probably\_raise} shows that this \funcname{match \dots with} is implemented with a pattern matching and a \funcname{try \dots with}
        \item<1-> But this one only matches on \typename{ExnA} exception
        \item<2-> Since the pattern matching fails to catch the exception, \funcname{reraise} is called with our current \typename{ExnC} exception
        \item<3-> Disassembly shows that \funcname{reraise} is actually a call to \funcname{caml\_reraise\_exn}, which is also an assembly function provided by the runtime
      \end{itemize}
      \vfill
      \mintocaml[firstline=15,lastline=21,highlightlines={18-21}]{../src/backtrace/backtrace.ml}
      \endminipage
    \end{column}
    \begin{column}{0.55\textwidth}
      \centering
      \onslide*<1>{\mintcmm[highlightlines={10-18}]{../src/backtrace/camlBacktrace\_\_probably\_raise\_351.cmm}}
      \onslide*<2>{\listcmm[highlightlines=18]{../src/backtrace/camlBacktrace\_\_probably\_raise_351.cmm}{CMM of probably\_raise}}
      \onslide*<3>{\mintobjdump[firstline=5,lastline=35,highlightlines=31]{../src/backtrace/camlBacktrace\_\_probably\_raise_351.objdump}}
    \end{column}
  \end{columns}
  \only<1>{\footnotetext[1]{https://blog.janestreet.com/pattern-matching-and-exception-handling-unite/}}
\end{frame}

\newcommand\listCamlReraiseExn[1][]{
  \listasm[firstline=727,lastline=734,#1]{../ocaml/runtime/amd64.S}{runtime/amd64.S:727}}

\begin{frame}{Backtraces}{Introducing \funcname{caml\_reraise\_exn}}
  \begin{columns}[c]
    \begin{column}{0.5\textwidth}
      \minipage[c][0.66\textheight][s]{\columnwidth}
      \begin{itemize}
        \item<1-> The exception have to be reraised to the next handler
        \item<2-> First, checks if the backtrace should be collected with \funcarg{domain\_state->backtrace\_active}
        \only<3>{
        \begin{itemize}
            \item If not, behaves like a \funcname{raise\_notrace} with \funcname{RESTORE\_EXN\_HANDLER\_OCAML}
        \end{itemize}
        }
        \item<4-> Jumps into \funcname{caml\_raise\_exn}
        \item<5-> Calls \funcname{caml\_stash\_backtrace} again, to scan the next part of the stack: from the trap just pop-ed to the next trap
      \end{itemize}
      \vfill
      \mintocaml[firstline=15,lastline=21,highlightlines={18-21}]{../src/backtrace/backtrace.ml}
      \endminipage
    \end{column}
    \begin{column}{0.5\textwidth}
      \raggedleft
      \onslide*<1>{\listCamlReraiseExn{}}
      \onslide*<2>{\listCamlReraiseExn[highlightlines=729]{}}
      \onslide*<3>{
        \mintc[firstline=190,lastline=192,highlightlines={191-192}]{../ocaml/runtime/amd64.S}
        \mintc[firstline=234,lastline=237,highlightlines={235-237}]{../ocaml/runtime/amd64.S}
        \medskip
        \listCamlReraiseExn[highlightlines=731]{}
      }
      \onslide*<4-5>{
        % TODO refactor with \listCamlReraiseExn
        \mintasm[firstline=727,lastline=734,highlightlines=730]{../ocaml/runtime/amd64.S}
        \medskip
      }
      \onslide*<4>{\listCamlRaiseExn[highlightlines=711]{}}
      \onslide*<5>{\listCamlRaiseExn[highlightlines=720]{}}
    \end{column}
  \end{columns}
\end{frame}

\begin{frame}{Backtraces}{Backtrace collection continues}
  \begin{columns}[c]
    \begin{column}{0.55\textwidth}
      \minipage[c][0.75\textheight][s]{\columnwidth}
      \begin{itemize}
        \item<1-> \funcname{caml\_stash\_backtrace} scans the older part of the stack, up to the next trap
        \item<6-> Controls jumps to the nested exception handler in \funcname{maybe\_raise}, which matches only on \typename{ExnB}
        \item<7-> \funcname{caml\_reraise\_exn} is called again, backtrace collection resumes with \funcname{caml\_stash\_backtrace}
      \end{itemize}
      \medskip
      \begin{itemize}
        \item<2-> Constructed backtrace:
        \begin{itemize}
          \item <2-> \funcname{definitely\_raise}
          \item <2-> \funcname{probably\_raise}
          \item <3-> \funcname{probably\_raise} (re-raise)
          \item <4-> \funcname{maybe\_raise}
          \item <5-> \funcname{main}
          \item <8-> \funcname{main} (re-raise)
        \end{itemize}
      \end{itemize}
      \vfill
      \onslide*<1-5>{\mintocaml[firstline=15,lastline=21]{../src/backtrace/backtrace.ml}}
      \onslide*<6>{\mintocaml[firstline=28,lastline=34,highlightlines=31]{../src/backtrace/backtrace.ml}}
      \onslide*<7->{\mintocaml[firstline=28,lastline=34,highlightlines=33]{../src/backtrace/backtrace.ml}}
      \endminipage
    \end{column}
    \begin{column}{0.45\textwidth}
      \raggedleft
      \onslide*<1>{\providecommand\step{6}\input{diagrams/backtrace/backtrace.tex}}%
      \onslide*<2>{\providecommand\step{6}\input{diagrams/backtrace/backtrace.tex}}%
      \onslide*<3>{\providecommand\step{7}\input{diagrams/backtrace/backtrace.tex}}%
      \onslide*<4>{\providecommand\step{8}\input{diagrams/backtrace/backtrace.tex}}%
      \onslide*<5>{\providecommand\step{9}\input{diagrams/backtrace/backtrace.tex}}%
      \onslide*<6>{\providecommand\step{10}\input{diagrams/backtrace/backtrace.tex}}%
      \onslide*<7>{\providecommand\step{11}\input{diagrams/backtrace/backtrace.tex}}%
      \onslide*<8>{\providecommand\step{12}\input{diagrams/backtrace/backtrace.tex}}%
      \onslide*<9>{\providecommand\step{13}\input{diagrams/backtrace/backtrace.tex}}%
    \end{column}
  \end{columns}
\end{frame}

\begin{frame}{Backtraces}{Printing the backtrace}
  \begin{columns}[c]
    \begin{column}{0.45\textwidth}
      \minipage[c][0.66\textheight][s]{\columnwidth}
      \begin{itemize}
        \item<1-> The exception backtrace can now be printed to \funcarg{stdout} with \funcname{Printexc.print_backtrace}
        \item<2-> First the exception backtrace is retrieved into an OCaml array with \funcname{get_raw_backtrace }
        \item<3-> Which is implemented as the \funcname{caml_get_exception_raw_backtrace} C function in the runtime
        \item<4-> And finally printed with \funcname{print_exception_backtrace}
      \end{itemize}
      \vfill
      \mintocaml[firstline=28,lastline=34,highlightlines=33]{../src/backtrace/backtrace.ml}
      \endminipage
    \end{column}
    \begin{column}{0.55\textwidth}
      \onslide*<1,2>{
        \mintocaml[firstline=100,lastline=101]{../ocaml/stdlib/printexc.ml}
      }
      \only<1>{
        \listocaml[firstline=170,lastline=172]{../ocaml/stdlib/printexc.ml}{stdlib/printexc.ml:170}
      }
      \only<2>{
        \listocaml[firstline=170,lastline=172,highlightlines=172]{../ocaml/stdlib/printexc.ml}{stdlib/printexc.ml:170}
      }
      \only<3>{
        \mintc[firstline=154,lastline=156]{../ocaml/runtime/backtrace.c}
        \listc[firstline=171,lastline=192,highlightlines={184-187}]{../ocaml/runtime/backtrace.c}{runtime/backtrace.c:154}
      }
      \only<4>{
        \listocaml[firstline=155,lastline=165]{../ocaml/stdlib/printexc.ml}{stdlib/printexc.ml:55}
      }
    \end{column}
  \end{columns}
\end{frame}


\subsection{The multiple kinds of raise}
\frameSubsection{}{}

\begin{frame}{Backtraces}{When backtraces are not needed}
  \begin{columns}[c]
    \begin{column}{0.45\textwidth}
      \minipage[c][0.66\textheight][s]{\columnwidth}
      \begin{itemize}
        \item<1-> What if the backtrace for an exception is never needed?
        \item<2-> It's possible to raise the exception explicitly with \funcname{raise\_notrace} to make raising as cheap as possible
        \item<3-> An example of use in the \funcname{dynlink} library of the OCaml compiler
      \end{itemize}
      \vfill
      \onslide<3->{
      \listocaml[firstline=205,lastline=209,highlightlines=207]{../ocaml/otherlibs/dynlink/dynlink\_compilerlibs/misc.ml}{otherlibs/dynlink/dynlink\_compilerlibs/misc.ml:205}
      }
      \endminipage
    \end{column}
    \begin{column}{0.55\textwidth}
    \only<4>{
      \mintcmm[highlightlines=3]{../src/notrace/camlNotrace\_\_fun_347.cmm}
      \listcmm{../src/notrace/camlNotrace\_\_all\_somes\_327.cmm}{all\_somes}
    }
    \onslide*<5->{
      \listobjdump[highlightlines={6-9}]{../src/notrace/camlNotrace\_\_fun_347.objdump}{anonymous function from \funcname{all\_somes}}
    }
    \end{column}
  \end{columns}
\end{frame}

\begin{frame}{Backtraces}{Summary table of the multiple kinds of raise}
  \begin{itemize}
    \item When debugging information is enabled and if the backtrace of an exception isn't needed, it's possible to raise with \funcname{raise\_notrace} for performance
    \item When debugging information is \emph{not} enabled, the compiler optimises \funcname{raise} calls to inline assembly (same effect as using \funcname{raise\_notrace})
  \end{itemize}
  \bigskip
  \centering
  \begin{tabular}{| l | c | c | c | c |}
    \hline
    \parbox{10em}{Debug information \\ (\funcarg{-g} flag)}  & \multicolumn{2}{|c|}{Disabled}                                     & \multicolumn{2}{|c|}{Enabled} \\
    \hline
    Raise kind                                               & \funcname{raise} & \funcname{raise\_notrace}                       & \funcname{raise} & \funcname{raise\_notrace} \\
    \hline
    Compilation                                              & \parbox{8em}{inline assembly\\ (optimization)} & inline assembly   & \funcname{caml\_raise\_exn} & inline assembly \\
    \hline
  \end{tabular}
\end{frame}


%
% Takeaway #5
%
\subsection*{Takeaway \#5}
\frameSubsectionTakeaway{}
\againframe<5>{takeaway}
}%
      \onslide*<6>{\providecommand\step{2}%
% Backtraces
%
\subsection{One advantage of raising with debugging information}
\frameSubsection{}{}

\begin{frame}{Backtraces}{Debugging information \& source code}
  \begin{columns}[c]
    \begin{column}{0.5\textwidth}
      \begin{itemize}
        \item Raising an exception is fun and games, but how do we know where the exception originated? $\rightarrow$ With the backtrace associated with the exception!
        \item In order to get a backtrace from the point where an exception was raised, it's necessary to compile with debugging information enabled (\funcarg{-g} flag for the compiler)
        \item Same example as in the `Nested handlers' section
      \end{itemize}
    \end{column}
    \begin{column}{0.5\textwidth}
      \listocaml[firstline=9,lastline=34]{../src/backtrace/backtrace.ml}{backtrace/backtrace.ml}
    \end{column}
  \end{columns}
\end{frame}

\begin{frame}{Backtraces}{CMM and disassembly of \funcname{definitely\_raise}}
  \begin{columns}[c]
    \begin{column}{0.5\textwidth}
      \minipage[c][0.66\textheight][s]{\columnwidth}
      \begin{itemize}
        \item<1-> An effect of compiling with debugging information (\funcarg{-g}): the CMM no longer uses \funcname{raise\_notrace} to raise the exception but \funcname{raise} instead
        \item<2-> Disassembly shows that CMM \funcname{raise} is actually a call to \funcname{caml\_raise\_exn}, a function in assembler provided by the runtime
      \end{itemize}
      \vfill
      \mintocaml[firstline=9,lastline=13,highlightlines=12]{../src/backtrace/backtrace.ml}
      \endminipage
    \end{column}
    \begin{column}{0.5\textwidth}
      \onslide*<1>{
        \mintcmm[highlightlines=5]{../src/backtrace/camlBacktrace\_\_definitely\_raise\_347.cmm}
      }
      \onslide*<2>{
        \mintobjdump[firstline=2,highlightlines=14]{../src/backtrace/camlBacktrace\_\_definitely\_raise\_347.objdump}
      }
    \end{column}
  \end{columns}
\end{frame}

\begin{frame}{Backtraces}{Introducing \funcname{caml\_raise\_exn}}
  \begin{columns}[c]
    \begin{column}{0.5\textwidth}
      \minipage[c][0.66\textheight][s]{\columnwidth}
      \onslide*<1>{
        \begin{itemize}
          \item Raising the exception is performed using \funcname{caml\_raise\_exn} which is
            \begin{itemize}
              \item An assembly function provided by the runtime
              \item Architecture specific
            \end{itemize}
          \item Let's see what it does differently from \funcname{raise\_notrace}\dots
          \item We are about to raise \typename{ExnC} with \funcname{caml\_raise\_exn} from \funcname{definitely\_raise}
        \end{itemize}
      }
      \onslide*<2>{
        \mintobjdump[firstline=2,highlightlines=14]{../src/backtrace/camlBacktrace\_\_definitely\_raise\_347.objdump}
      }
      \vfill
      \mintocaml[firstline=9,lastline=13,highlightlines=12]{../src/backtrace/backtrace.ml}
      \endminipage
    \end{column}
    \begin{column}{0.5\textwidth}
      \centering
      \providecommand\step{1}\input{diagrams/backtrace/backtrace.tex}
    \end{column}
  \end{columns}
\end{frame}

\begin{frame}{Backtraces}{But before that, we need more fields from \typename{caml\_domain\_state}}
  \begin{columns}
    \begin{column}{0.5\textwidth}
      \begin{itemize}
        \item Remember \typename{caml\_domain\_state} the per-domain runtime state?
        \item Some other members are of interest to us now\dots
        \item \localname{backtrace\_active} is used to know if the runtime should collect backtraces
        \item \localname{backtrace\_active} can be set by
          \begin{itemize}
            \item The \funcarg{b} flag in the \funcname{OCAMLRUNPARAM} environment variable
            \item \mintinline{ocaml}{Printexc.record_backtrace : bool -> unit} \footnotemark
          \end{itemize}
      \end{itemize}
    \end{column}
    \begin{column}{0.5\textwidth}
      \begin{listing}
        \mintc[highlightlines={9-11}]{src/ocaml/domain\_state\_backtrace.h}
        \caption{runtime/caml/domain\_state.h}
      \end{listing}
    \end{column}
  \end{columns}
  \footnotetext[1]{https://v2.ocaml.org/api/Printexc.html}
\end{frame}

\newcommand\listCamlRaiseExn[1][]{
  \listasm[firstline=702,lastline=725,#1]{../ocaml/runtime/amd64.S}{runtime/amd64.S:702}}

\begin{frame}{Backtraces}{Walk through \funcname{caml\_raise\_exn}}
  \begin{columns}[T]
    \begin{column}{0.5\textwidth}
      \begin{itemize}
        \item<1-> First checks whether exceptions backtrace should be recorded
        \item<1-> By checking the \funcarg{domain\_state->backtrace\_active} value
        \item<2-> If not, acts as \funcname{raise\_notrace} with \funcname{RESTORE\_EXN\_HANDLER\_OCAML}
          \begin{itemize}
            \item Set \regname{RSP} to \localname{domain\_state->exn\_handler} value
            \item Restore \localname{domain\_state->exn\_handler} from trap
            \item Jump with \funcname{ret} (same effect as using \funcname{pop} and \funcname{jmp})
          \end{itemize}
        \item<3-> Otherwise, switches to the C stack to be able to execute C code
        \item<4-> Calls \funcname{caml\_stash\_backtrace}, a C function provided by the runtime\ignorespaces
          \onslide<6->{, with
            \begin{itemize}
              \item \funcarg{exn}: exception value
              \item \funcarg{pc}: code address of the raising point
              \item \funcarg{sp}: stack address of the raising function
              \item \funcarg{trapsp}: stack address of the next handler
            \end{itemize}
          }
      \end{itemize}
      \bigskip
      \mintocaml[firstline=9,lastline=13,highlightlines=12]{../src/backtrace/backtrace.ml}
    \end{column}
    \begin{column}{0.5\textwidth}
      \raggedleft
      \onslide*<1,3-4>{
        \mintc[firstline=190,lastline=192]{../ocaml/runtime/amd64.S}
        \mintc[firstline=234,lastline=237]{../ocaml/runtime/amd64.S}
      }
      \onslide*<2>{
        \mintc[firstline=190,lastline=192,highlightlines={191-192}]{../ocaml/runtime/amd64.S}
        \mintc[firstline=234,lastline=237,highlightlines={235-237}]{../ocaml/runtime/amd64.S}
      }
      \onslide*<1>{\listCamlRaiseExn[highlightlines=705]{}}%
      \onslide*<2>{\listCamlRaiseExn[highlightlines={707-708}]{}}%
      \onslide*<3>{\listCamlRaiseExn[highlightlines={712-714}]{}}%
      \onslide*<4>{\listCamlRaiseExn[highlightlines={715-720}]{}}%
      \onslide*<5>{\providecommand\step{1}\input{diagrams/backtrace/backtrace.tex}}%
      \onslide*<6>{\providecommand\step{2}\input{diagrams/backtrace/backtrace.tex}}%
    \end{column}
  \end{columns}
\end{frame}

\newcommand\listStashBacktrace[1][]{
  \listc[firstline=91,lastline=120,#1]{../ocaml/runtime/backtrace\_nat.c}{runtime/backtrace\_nat.c:91}}

\begin{frame}{Backtraces}{Introducing \funcname{caml\_stash\_backtrace}}
  \begin{columns}[c]
    \begin{column}{0.5\textwidth}
      \minipage[c][0.66\textheight][s]{\columnwidth}
      \begin{itemize}
        \item<1-> A C function provided by the runtime (specific to native code)
        \item<2-> It scans the stack, between the stack pointer at the raising point up to the next trap, in order to collect the names of the detected function frames
          \onslide*<2>{
            \begin{itemize}
              \item And more information, like line number, column number...
            \end{itemize}
          }
        \item<3-> It uses the same technique and information as the Garbage Collector (GC) uses to scan the values live on the stack
          \onslide*<4>{
            \begin{itemize}
              \item \typename{frame\_descr} are at the core of stack unwinding
            \end{itemize}
          }
      \end{itemize}
      \vfill
      \mintocaml[firstline=9,lastline=13,highlightlines=12]{../src/backtrace/backtrace.ml}
      \endminipage
    \end{column}
    \begin{column}{0.5\textwidth}
      \onslide*<1>{\listStashBacktrace[highlightlines=91]{}}
      \onslide*<2>{\listStashBacktrace[highlightlines={91,117-118}]{}}
      \onslide*<3->{\listStashBacktrace[highlightlines={94,106,109-110}]{}}
    \end{column}
  \end{columns}
\end{frame}

\newcommand\listFrameDescr[1][]{
  \listocaml[firstline=111,lastline=124,#1]{../ocaml/asmcomp/emitaux.ml}{asmcomp/emitaux.ml:111}}
\newcommand\listDebugInfo[1][]{
  \listocaml[firstline=38,lastline=49,#1]{../ocaml/lambda/debuginfo.mli}{lambda/debuginfo.mli:38}}

\begin{frame}{Backtraces}{\typename{frame\_descr} and \typename{frame\_debuginfo} in a nutshell}
  \begin{columns}[c]
    \begin{column}{0.5\textwidth}
      \begin{itemize}
        \item<1-> For every return address in an OCaml program, there is a associated \typename{frame\_descr}
        \item<2-> A \typename{frame\_descr} specifies
        \begin{itemize}
          \item<2-> The size of the function stack frame
          \item<3-> Where live values are on the stack (so that the GC can mark them as live and prevent from being swept)
        \end{itemize}
        \item<4-> When compiled with debug information, \typename{frame\_descr} will be linked to a \typename{frame\_debuginfo}
        \begin{itemize}
          \item<5-> Contains information about the position in the source code (file name, line and column number)
        \end{itemize}
      \end{itemize}
    \end{column}
    \begin{column}{0.5\textwidth}
        \onslide*<1>{\listFrameDescr[highlightlines={118-122}]{}}
        \onslide*<2>{\listFrameDescr[highlightlines={120}]{}}
        \onslide*<3>{\listFrameDescr[highlightlines={121}]{}}
        \onslide*<4->{\listFrameDescr[highlightlines={122}]{}}
        \medskip
        \onslide*<1-3>{\listDebugInfo{}}
        \onslide*<4>{\listDebugInfo[highlightlines={38,49}]{}}
        \onslide*<5>{\listDebugInfo[highlightlines={39-46}]{}}
    \end{column}
  \end{columns}
\end{frame}

\begin{frame}{Backtraces}{Scanning the stack with \typename{frame\_descr}}
  \begin{columns}[c]
    \begin{column}{0.5\textwidth}
      \begin{itemize}
        \item Given the return address of the code that raises the exception by calling \funcname{caml\_raise\_exn} (\funcarg{pc} argument of \funcname{caml\_stash\_backtrace})
        \item And the position of the stack pointer (\funcarg{sp} argument of \funcname{caml\_stash\_backtrace})
        \item The frame size given by \typename{frame\_descr}.\funcarg{frame\_size}, allows to find the end of the previous frame
        \item And the return address of the caller of this function
        \bigskip
        \onslide<3->{
        \item Note that traps (here the reddish \funcname{ExnA}, \funcname{ExnB} and \funcname{ExnC} blocks) belong to the frame that installed them, and so the frame size changes when traps are removed
        }
      \end{itemize}
    \end{column}
    \begin{column}{0.5\textwidth}
      \raggedleft
      \onslide*<1>{\providecommand\step{2}\input{diagrams/backtrace/backtrace.tex}}%
      \onslide*<2>{\providecommand\step{1}\input{diagrams/backtrace/scanstack.tex}}%
      \onslide*<3>{\providecommand\step{2}\input{diagrams/backtrace/scanstack.tex}}%
      \onslide*<4>{\providecommand\step{3}\input{diagrams/backtrace/scanstack.tex}}%
      \onslide*<5>{\providecommand\step{4}\input{diagrams/backtrace/scanstack.tex}}%
      \onslide*<6>{\providecommand\step{5}\input{diagrams/backtrace/scanstack.tex}}%
    \end{column}
  \end{columns}
\end{frame}

% Duplicate of previous-previous slide
\begin{frame}{Backtraces}{Continuing on \funcname{caml\_stash\_backtrace}}
  \begin{columns}[c]
    \begin{column}{0.5\textwidth}
      \minipage[c][0.75\textheight][s]{\columnwidth}
      \begin{itemize}
          \color{gray}
        \item A C function provided by the runtime (specific to native code)
        \item It scans the stack, between the stack pointer at the raising point up to the next trap, in order to collect the names of the detected function frames
        \item It uses the same technique and information as the Garbage Collector (GC) uses to scan the values live on the stack
      \end{itemize}
      \begin{itemize}
        \item For each function frame detected, store a pointer to the \typename{frame\_descr} in the \localname{domain\_state->backtrace\_buffer} array, effectively constructing the backtrace
      \end{itemize}
      \onslide<2->{
        \begin{itemize}
            \smallskip
          \item Constructed backtrace:
            \funcname{definitely\_raise},
            \onslide<3->{\funcname{probably\_raise}}
        \end{itemize}
      }
      \vfill
      \mintocaml[firstline=9,lastline=13,highlightlines=12]{../src/backtrace/backtrace.ml}
      \endminipage
    \end{column}
    \begin{column}{0.5\textwidth}
      \raggedleft
      \onslide*<1>{\listStashBacktrace[highlightlines={114-115}]{}}
      \onslide*<2>{\providecommand\step{3}\input{diagrams/backtrace/backtrace.tex}}%
      \onslide*<3>{\providecommand\step{4}\input{diagrams/backtrace/backtrace.tex}}%
    \end{column}
  \end{columns}
\end{frame}

\begin{frame}{Backtraces}{Back to \funcname{caml\_raise\_exn}}
  \begin{columns}[c]
    \begin{column}{0.5\textwidth}
      \minipage[c][0.66\textheight][s]{\columnwidth}
      \begin{itemize}\color{gray}
        \item Checks wheteher exceptions backtrace should be recorded, by checking the \funcarg{domain\_state->backtrace\_active} value
        \item Switches to the C stack to be able to execute C code
        \item Calls \funcname{caml\_stash\_backtrace}, to collect the backtrace up to the next trap
      \end{itemize}
      \onslide*<2->{
        \begin{itemize}
          \item<2-> Executes the exception handler in \funcname{probably\_raise} with \funcname{RESTORE\_EXN\_HANDLER\_OCAML}
            \begin{itemize}
              \item Set \regname{RSP} to \localname{domain\_state->exn\_handler} value
              \item Restore \localname{domain\_state->exn\_handler} from trap
              \item Jump to the handler code with \funcname{ret}
            \end{itemize}
        \end{itemize}
      }
      \vfill
      \onslide*<1>{\mintocaml[firstline=9,lastline=13,highlightlines=12]{../src/backtrace/backtrace.ml}}
      \onslide*<2->{\mintocaml[firstline=15,lastline=21,highlightlines={19-21}]{../src/backtrace/backtrace.ml}}
      \endminipage
    \end{column}
    \begin{column}{0.5\textwidth}
      \centering
      \onslide*<1>{
        \mintc[firstline=190,lastline=192]{../ocaml/runtime/amd64.S}
        \mintc[firstline=234,lastline=237]{../ocaml/runtime/amd64.S}
      }
      \onslide*<2>{
        \mintc[firstline=190,lastline=192,highlightlines={191-192}]{../ocaml/runtime/amd64.S}
        \mintc[firstline=234,lastline=237,highlightlines={235-237}]{../ocaml/runtime/amd64.S}
      }
      \onslide*<1>{\listCamlRaiseExn[highlightlines=720]{}}
      \onslide*<2>{\listCamlRaiseExn[highlightlines=722]{}}
      \onslide*<3->{\providecommand\step{5}\input{diagrams/backtrace/backtrace.tex}}
    \end{column}
  \end{columns}
\end{frame}

\begin{frame}{Backtraces}{\funcname{caml\_probably\_raise}'s handler doesn't catch \typename{ExnC}}
  \begin{columns}[c]
    \begin{column}{0.45\textwidth}
      \minipage[c][0.75\textheight][s]{\columnwidth}
      \begin{itemize}
        \item<1-> \funcname{match \dots with}\footnotemark pattern matching can also match exceptions
        \item<1-> The CMM of \funcname{probably\_raise} shows that this \funcname{match \dots with} is implemented with a pattern matching and a \funcname{try \dots with}
        \item<1-> But this one only matches on \typename{ExnA} exception
        \item<2-> Since the pattern matching fails to catch the exception, \funcname{reraise} is called with our current \typename{ExnC} exception
        \item<3-> Disassembly shows that \funcname{reraise} is actually a call to \funcname{caml\_reraise\_exn}, which is also an assembly function provided by the runtime
      \end{itemize}
      \vfill
      \mintocaml[firstline=15,lastline=21,highlightlines={18-21}]{../src/backtrace/backtrace.ml}
      \endminipage
    \end{column}
    \begin{column}{0.55\textwidth}
      \centering
      \onslide*<1>{\mintcmm[highlightlines={10-18}]{../src/backtrace/camlBacktrace\_\_probably\_raise\_351.cmm}}
      \onslide*<2>{\listcmm[highlightlines=18]{../src/backtrace/camlBacktrace\_\_probably\_raise_351.cmm}{CMM of probably\_raise}}
      \onslide*<3>{\mintobjdump[firstline=5,lastline=35,highlightlines=31]{../src/backtrace/camlBacktrace\_\_probably\_raise_351.objdump}}
    \end{column}
  \end{columns}
  \only<1>{\footnotetext[1]{https://blog.janestreet.com/pattern-matching-and-exception-handling-unite/}}
\end{frame}

\newcommand\listCamlReraiseExn[1][]{
  \listasm[firstline=727,lastline=734,#1]{../ocaml/runtime/amd64.S}{runtime/amd64.S:727}}

\begin{frame}{Backtraces}{Introducing \funcname{caml\_reraise\_exn}}
  \begin{columns}[c]
    \begin{column}{0.5\textwidth}
      \minipage[c][0.66\textheight][s]{\columnwidth}
      \begin{itemize}
        \item<1-> The exception have to be reraised to the next handler
        \item<2-> First, checks if the backtrace should be collected with \funcarg{domain\_state->backtrace\_active}
        \only<3>{
        \begin{itemize}
            \item If not, behaves like a \funcname{raise\_notrace} with \funcname{RESTORE\_EXN\_HANDLER\_OCAML}
        \end{itemize}
        }
        \item<4-> Jumps into \funcname{caml\_raise\_exn}
        \item<5-> Calls \funcname{caml\_stash\_backtrace} again, to scan the next part of the stack: from the trap just pop-ed to the next trap
      \end{itemize}
      \vfill
      \mintocaml[firstline=15,lastline=21,highlightlines={18-21}]{../src/backtrace/backtrace.ml}
      \endminipage
    \end{column}
    \begin{column}{0.5\textwidth}
      \raggedleft
      \onslide*<1>{\listCamlReraiseExn{}}
      \onslide*<2>{\listCamlReraiseExn[highlightlines=729]{}}
      \onslide*<3>{
        \mintc[firstline=190,lastline=192,highlightlines={191-192}]{../ocaml/runtime/amd64.S}
        \mintc[firstline=234,lastline=237,highlightlines={235-237}]{../ocaml/runtime/amd64.S}
        \medskip
        \listCamlReraiseExn[highlightlines=731]{}
      }
      \onslide*<4-5>{
        % TODO refactor with \listCamlReraiseExn
        \mintasm[firstline=727,lastline=734,highlightlines=730]{../ocaml/runtime/amd64.S}
        \medskip
      }
      \onslide*<4>{\listCamlRaiseExn[highlightlines=711]{}}
      \onslide*<5>{\listCamlRaiseExn[highlightlines=720]{}}
    \end{column}
  \end{columns}
\end{frame}

\begin{frame}{Backtraces}{Backtrace collection continues}
  \begin{columns}[c]
    \begin{column}{0.55\textwidth}
      \minipage[c][0.75\textheight][s]{\columnwidth}
      \begin{itemize}
        \item<1-> \funcname{caml\_stash\_backtrace} scans the older part of the stack, up to the next trap
        \item<6-> Controls jumps to the nested exception handler in \funcname{maybe\_raise}, which matches only on \typename{ExnB}
        \item<7-> \funcname{caml\_reraise\_exn} is called again, backtrace collection resumes with \funcname{caml\_stash\_backtrace}
      \end{itemize}
      \medskip
      \begin{itemize}
        \item<2-> Constructed backtrace:
        \begin{itemize}
          \item <2-> \funcname{definitely\_raise}
          \item <2-> \funcname{probably\_raise}
          \item <3-> \funcname{probably\_raise} (re-raise)
          \item <4-> \funcname{maybe\_raise}
          \item <5-> \funcname{main}
          \item <8-> \funcname{main} (re-raise)
        \end{itemize}
      \end{itemize}
      \vfill
      \onslide*<1-5>{\mintocaml[firstline=15,lastline=21]{../src/backtrace/backtrace.ml}}
      \onslide*<6>{\mintocaml[firstline=28,lastline=34,highlightlines=31]{../src/backtrace/backtrace.ml}}
      \onslide*<7->{\mintocaml[firstline=28,lastline=34,highlightlines=33]{../src/backtrace/backtrace.ml}}
      \endminipage
    \end{column}
    \begin{column}{0.45\textwidth}
      \raggedleft
      \onslide*<1>{\providecommand\step{6}\input{diagrams/backtrace/backtrace.tex}}%
      \onslide*<2>{\providecommand\step{6}\input{diagrams/backtrace/backtrace.tex}}%
      \onslide*<3>{\providecommand\step{7}\input{diagrams/backtrace/backtrace.tex}}%
      \onslide*<4>{\providecommand\step{8}\input{diagrams/backtrace/backtrace.tex}}%
      \onslide*<5>{\providecommand\step{9}\input{diagrams/backtrace/backtrace.tex}}%
      \onslide*<6>{\providecommand\step{10}\input{diagrams/backtrace/backtrace.tex}}%
      \onslide*<7>{\providecommand\step{11}\input{diagrams/backtrace/backtrace.tex}}%
      \onslide*<8>{\providecommand\step{12}\input{diagrams/backtrace/backtrace.tex}}%
      \onslide*<9>{\providecommand\step{13}\input{diagrams/backtrace/backtrace.tex}}%
    \end{column}
  \end{columns}
\end{frame}

\begin{frame}{Backtraces}{Printing the backtrace}
  \begin{columns}[c]
    \begin{column}{0.45\textwidth}
      \minipage[c][0.66\textheight][s]{\columnwidth}
      \begin{itemize}
        \item<1-> The exception backtrace can now be printed to \funcarg{stdout} with \funcname{Printexc.print_backtrace}
        \item<2-> First the exception backtrace is retrieved into an OCaml array with \funcname{get_raw_backtrace }
        \item<3-> Which is implemented as the \funcname{caml_get_exception_raw_backtrace} C function in the runtime
        \item<4-> And finally printed with \funcname{print_exception_backtrace}
      \end{itemize}
      \vfill
      \mintocaml[firstline=28,lastline=34,highlightlines=33]{../src/backtrace/backtrace.ml}
      \endminipage
    \end{column}
    \begin{column}{0.55\textwidth}
      \onslide*<1,2>{
        \mintocaml[firstline=100,lastline=101]{../ocaml/stdlib/printexc.ml}
      }
      \only<1>{
        \listocaml[firstline=170,lastline=172]{../ocaml/stdlib/printexc.ml}{stdlib/printexc.ml:170}
      }
      \only<2>{
        \listocaml[firstline=170,lastline=172,highlightlines=172]{../ocaml/stdlib/printexc.ml}{stdlib/printexc.ml:170}
      }
      \only<3>{
        \mintc[firstline=154,lastline=156]{../ocaml/runtime/backtrace.c}
        \listc[firstline=171,lastline=192,highlightlines={184-187}]{../ocaml/runtime/backtrace.c}{runtime/backtrace.c:154}
      }
      \only<4>{
        \listocaml[firstline=155,lastline=165]{../ocaml/stdlib/printexc.ml}{stdlib/printexc.ml:55}
      }
    \end{column}
  \end{columns}
\end{frame}


\subsection{The multiple kinds of raise}
\frameSubsection{}{}

\begin{frame}{Backtraces}{When backtraces are not needed}
  \begin{columns}[c]
    \begin{column}{0.45\textwidth}
      \minipage[c][0.66\textheight][s]{\columnwidth}
      \begin{itemize}
        \item<1-> What if the backtrace for an exception is never needed?
        \item<2-> It's possible to raise the exception explicitly with \funcname{raise\_notrace} to make raising as cheap as possible
        \item<3-> An example of use in the \funcname{dynlink} library of the OCaml compiler
      \end{itemize}
      \vfill
      \onslide<3->{
      \listocaml[firstline=205,lastline=209,highlightlines=207]{../ocaml/otherlibs/dynlink/dynlink\_compilerlibs/misc.ml}{otherlibs/dynlink/dynlink\_compilerlibs/misc.ml:205}
      }
      \endminipage
    \end{column}
    \begin{column}{0.55\textwidth}
    \only<4>{
      \mintcmm[highlightlines=3]{../src/notrace/camlNotrace\_\_fun_347.cmm}
      \listcmm{../src/notrace/camlNotrace\_\_all\_somes\_327.cmm}{all\_somes}
    }
    \onslide*<5->{
      \listobjdump[highlightlines={6-9}]{../src/notrace/camlNotrace\_\_fun_347.objdump}{anonymous function from \funcname{all\_somes}}
    }
    \end{column}
  \end{columns}
\end{frame}

\begin{frame}{Backtraces}{Summary table of the multiple kinds of raise}
  \begin{itemize}
    \item When debugging information is enabled and if the backtrace of an exception isn't needed, it's possible to raise with \funcname{raise\_notrace} for performance
    \item When debugging information is \emph{not} enabled, the compiler optimises \funcname{raise} calls to inline assembly (same effect as using \funcname{raise\_notrace})
  \end{itemize}
  \bigskip
  \centering
  \begin{tabular}{| l | c | c | c | c |}
    \hline
    \parbox{10em}{Debug information \\ (\funcarg{-g} flag)}  & \multicolumn{2}{|c|}{Disabled}                                     & \multicolumn{2}{|c|}{Enabled} \\
    \hline
    Raise kind                                               & \funcname{raise} & \funcname{raise\_notrace}                       & \funcname{raise} & \funcname{raise\_notrace} \\
    \hline
    Compilation                                              & \parbox{8em}{inline assembly\\ (optimization)} & inline assembly   & \funcname{caml\_raise\_exn} & inline assembly \\
    \hline
  \end{tabular}
\end{frame}


%
% Takeaway #5
%
\subsection*{Takeaway \#5}
\frameSubsectionTakeaway{}
\againframe<5>{takeaway}
}%
    \end{column}
  \end{columns}
\end{frame}

\newcommand\listStashBacktrace[1][]{
  \listc[firstline=91,lastline=120,#1]{../ocaml/runtime/backtrace\_nat.c}{runtime/backtrace\_nat.c:91}}

\begin{frame}{Backtraces}{Introducing \funcname{caml\_stash\_backtrace}}
  \begin{columns}[c]
    \begin{column}{0.5\textwidth}
      \minipage[c][0.66\textheight][s]{\columnwidth}
      \begin{itemize}
        \item<1-> A C function provided by the runtime (specific to native code)
        \item<2-> It scans the stack, between the stack pointer at the raising point up to the next trap, in order to collect the names of the detected function frames
          \onslide*<2>{
            \begin{itemize}
              \item And more information, like line number, column number...
            \end{itemize}
          }
        \item<3-> It uses the same technique and information as the Garbage Collector (GC) uses to scan the values live on the stack
          \onslide*<4>{
            \begin{itemize}
              \item \typename{frame\_descr} are at the core of stack unwinding
            \end{itemize}
          }
      \end{itemize}
      \vfill
      \mintocaml[firstline=9,lastline=13,highlightlines=12]{../src/backtrace/backtrace.ml}
      \endminipage
    \end{column}
    \begin{column}{0.5\textwidth}
      \onslide*<1>{\listStashBacktrace[highlightlines=91]{}}
      \onslide*<2>{\listStashBacktrace[highlightlines={91,117-118}]{}}
      \onslide*<3->{\listStashBacktrace[highlightlines={94,106,109-110}]{}}
    \end{column}
  \end{columns}
\end{frame}

\newcommand\listFrameDescr[1][]{
  \listocaml[firstline=111,lastline=124,#1]{../ocaml/asmcomp/emitaux.ml}{asmcomp/emitaux.ml:111}}
\newcommand\listDebugInfo[1][]{
  \listocaml[firstline=38,lastline=49,#1]{../ocaml/lambda/debuginfo.mli}{lambda/debuginfo.mli:38}}

\begin{frame}{Backtraces}{\typename{frame\_descr} and \typename{frame\_debuginfo} in a nutshell}
  \begin{columns}[c]
    \begin{column}{0.5\textwidth}
      \begin{itemize}
        \item<1-> For every return address in an OCaml program, there is a associated \typename{frame\_descr}
        \item<2-> A \typename{frame\_descr} specifies
        \begin{itemize}
          \item<2-> The size of the function stack frame
          \item<3-> Where live values are on the stack (so that the GC can mark them as live and prevent from being swept)
        \end{itemize}
        \item<4-> When compiled with debug information, \typename{frame\_descr} will be linked to a \typename{frame\_debuginfo}
        \begin{itemize}
          \item<5-> Contains information about the position in the source code (file name, line and column number)
        \end{itemize}
      \end{itemize}
    \end{column}
    \begin{column}{0.5\textwidth}
        \onslide*<1>{\listFrameDescr[highlightlines={118-122}]{}}
        \onslide*<2>{\listFrameDescr[highlightlines={120}]{}}
        \onslide*<3>{\listFrameDescr[highlightlines={121}]{}}
        \onslide*<4->{\listFrameDescr[highlightlines={122}]{}}
        \medskip
        \onslide*<1-3>{\listDebugInfo{}}
        \onslide*<4>{\listDebugInfo[highlightlines={38,49}]{}}
        \onslide*<5>{\listDebugInfo[highlightlines={39-46}]{}}
    \end{column}
  \end{columns}
\end{frame}

\begin{frame}{Backtraces}{Scanning the stack with \typename{frame\_descr}}
  \begin{columns}[c]
    \begin{column}{0.5\textwidth}
      \begin{itemize}
        \item Given the return address of the code that raises the exception by calling \funcname{caml\_raise\_exn} (\funcarg{pc} argument of \funcname{caml\_stash\_backtrace})
        \item And the position of the stack pointer (\funcarg{sp} argument of \funcname{caml\_stash\_backtrace})
        \item The frame size given by \typename{frame\_descr}.\funcarg{frame\_size}, allows to find the end of the previous frame
        \item And the return address of the caller of this function
        \bigskip
        \onslide<3->{
        \item Note that traps (here the reddish \funcname{ExnA}, \funcname{ExnB} and \funcname{ExnC} blocks) belong to the frame that installed them, and so the frame size changes when traps are removed
        }
      \end{itemize}
    \end{column}
    \begin{column}{0.5\textwidth}
      \raggedleft
      \onslide*<1>{\providecommand\step{2}%
% Backtraces
%
\subsection{One advantage of raising with debugging information}
\frameSubsection{}{}

\begin{frame}{Backtraces}{Debugging information \& source code}
  \begin{columns}[c]
    \begin{column}{0.5\textwidth}
      \begin{itemize}
        \item Raising an exception is fun and games, but how do we know where the exception originated? $\rightarrow$ With the backtrace associated with the exception!
        \item In order to get a backtrace from the point where an exception was raised, it's necessary to compile with debugging information enabled (\funcarg{-g} flag for the compiler)
        \item Same example as in the `Nested handlers' section
      \end{itemize}
    \end{column}
    \begin{column}{0.5\textwidth}
      \listocaml[firstline=9,lastline=34]{../src/backtrace/backtrace.ml}{backtrace/backtrace.ml}
    \end{column}
  \end{columns}
\end{frame}

\begin{frame}{Backtraces}{CMM and disassembly of \funcname{definitely\_raise}}
  \begin{columns}[c]
    \begin{column}{0.5\textwidth}
      \minipage[c][0.66\textheight][s]{\columnwidth}
      \begin{itemize}
        \item<1-> An effect of compiling with debugging information (\funcarg{-g}): the CMM no longer uses \funcname{raise\_notrace} to raise the exception but \funcname{raise} instead
        \item<2-> Disassembly shows that CMM \funcname{raise} is actually a call to \funcname{caml\_raise\_exn}, a function in assembler provided by the runtime
      \end{itemize}
      \vfill
      \mintocaml[firstline=9,lastline=13,highlightlines=12]{../src/backtrace/backtrace.ml}
      \endminipage
    \end{column}
    \begin{column}{0.5\textwidth}
      \onslide*<1>{
        \mintcmm[highlightlines=5]{../src/backtrace/camlBacktrace\_\_definitely\_raise\_347.cmm}
      }
      \onslide*<2>{
        \mintobjdump[firstline=2,highlightlines=14]{../src/backtrace/camlBacktrace\_\_definitely\_raise\_347.objdump}
      }
    \end{column}
  \end{columns}
\end{frame}

\begin{frame}{Backtraces}{Introducing \funcname{caml\_raise\_exn}}
  \begin{columns}[c]
    \begin{column}{0.5\textwidth}
      \minipage[c][0.66\textheight][s]{\columnwidth}
      \onslide*<1>{
        \begin{itemize}
          \item Raising the exception is performed using \funcname{caml\_raise\_exn} which is
            \begin{itemize}
              \item An assembly function provided by the runtime
              \item Architecture specific
            \end{itemize}
          \item Let's see what it does differently from \funcname{raise\_notrace}\dots
          \item We are about to raise \typename{ExnC} with \funcname{caml\_raise\_exn} from \funcname{definitely\_raise}
        \end{itemize}
      }
      \onslide*<2>{
        \mintobjdump[firstline=2,highlightlines=14]{../src/backtrace/camlBacktrace\_\_definitely\_raise\_347.objdump}
      }
      \vfill
      \mintocaml[firstline=9,lastline=13,highlightlines=12]{../src/backtrace/backtrace.ml}
      \endminipage
    \end{column}
    \begin{column}{0.5\textwidth}
      \centering
      \providecommand\step{1}\input{diagrams/backtrace/backtrace.tex}
    \end{column}
  \end{columns}
\end{frame}

\begin{frame}{Backtraces}{But before that, we need more fields from \typename{caml\_domain\_state}}
  \begin{columns}
    \begin{column}{0.5\textwidth}
      \begin{itemize}
        \item Remember \typename{caml\_domain\_state} the per-domain runtime state?
        \item Some other members are of interest to us now\dots
        \item \localname{backtrace\_active} is used to know if the runtime should collect backtraces
        \item \localname{backtrace\_active} can be set by
          \begin{itemize}
            \item The \funcarg{b} flag in the \funcname{OCAMLRUNPARAM} environment variable
            \item \mintinline{ocaml}{Printexc.record_backtrace : bool -> unit} \footnotemark
          \end{itemize}
      \end{itemize}
    \end{column}
    \begin{column}{0.5\textwidth}
      \begin{listing}
        \mintc[highlightlines={9-11}]{src/ocaml/domain\_state\_backtrace.h}
        \caption{runtime/caml/domain\_state.h}
      \end{listing}
    \end{column}
  \end{columns}
  \footnotetext[1]{https://v2.ocaml.org/api/Printexc.html}
\end{frame}

\newcommand\listCamlRaiseExn[1][]{
  \listasm[firstline=702,lastline=725,#1]{../ocaml/runtime/amd64.S}{runtime/amd64.S:702}}

\begin{frame}{Backtraces}{Walk through \funcname{caml\_raise\_exn}}
  \begin{columns}[T]
    \begin{column}{0.5\textwidth}
      \begin{itemize}
        \item<1-> First checks whether exceptions backtrace should be recorded
        \item<1-> By checking the \funcarg{domain\_state->backtrace\_active} value
        \item<2-> If not, acts as \funcname{raise\_notrace} with \funcname{RESTORE\_EXN\_HANDLER\_OCAML}
          \begin{itemize}
            \item Set \regname{RSP} to \localname{domain\_state->exn\_handler} value
            \item Restore \localname{domain\_state->exn\_handler} from trap
            \item Jump with \funcname{ret} (same effect as using \funcname{pop} and \funcname{jmp})
          \end{itemize}
        \item<3-> Otherwise, switches to the C stack to be able to execute C code
        \item<4-> Calls \funcname{caml\_stash\_backtrace}, a C function provided by the runtime\ignorespaces
          \onslide<6->{, with
            \begin{itemize}
              \item \funcarg{exn}: exception value
              \item \funcarg{pc}: code address of the raising point
              \item \funcarg{sp}: stack address of the raising function
              \item \funcarg{trapsp}: stack address of the next handler
            \end{itemize}
          }
      \end{itemize}
      \bigskip
      \mintocaml[firstline=9,lastline=13,highlightlines=12]{../src/backtrace/backtrace.ml}
    \end{column}
    \begin{column}{0.5\textwidth}
      \raggedleft
      \onslide*<1,3-4>{
        \mintc[firstline=190,lastline=192]{../ocaml/runtime/amd64.S}
        \mintc[firstline=234,lastline=237]{../ocaml/runtime/amd64.S}
      }
      \onslide*<2>{
        \mintc[firstline=190,lastline=192,highlightlines={191-192}]{../ocaml/runtime/amd64.S}
        \mintc[firstline=234,lastline=237,highlightlines={235-237}]{../ocaml/runtime/amd64.S}
      }
      \onslide*<1>{\listCamlRaiseExn[highlightlines=705]{}}%
      \onslide*<2>{\listCamlRaiseExn[highlightlines={707-708}]{}}%
      \onslide*<3>{\listCamlRaiseExn[highlightlines={712-714}]{}}%
      \onslide*<4>{\listCamlRaiseExn[highlightlines={715-720}]{}}%
      \onslide*<5>{\providecommand\step{1}\input{diagrams/backtrace/backtrace.tex}}%
      \onslide*<6>{\providecommand\step{2}\input{diagrams/backtrace/backtrace.tex}}%
    \end{column}
  \end{columns}
\end{frame}

\newcommand\listStashBacktrace[1][]{
  \listc[firstline=91,lastline=120,#1]{../ocaml/runtime/backtrace\_nat.c}{runtime/backtrace\_nat.c:91}}

\begin{frame}{Backtraces}{Introducing \funcname{caml\_stash\_backtrace}}
  \begin{columns}[c]
    \begin{column}{0.5\textwidth}
      \minipage[c][0.66\textheight][s]{\columnwidth}
      \begin{itemize}
        \item<1-> A C function provided by the runtime (specific to native code)
        \item<2-> It scans the stack, between the stack pointer at the raising point up to the next trap, in order to collect the names of the detected function frames
          \onslide*<2>{
            \begin{itemize}
              \item And more information, like line number, column number...
            \end{itemize}
          }
        \item<3-> It uses the same technique and information as the Garbage Collector (GC) uses to scan the values live on the stack
          \onslide*<4>{
            \begin{itemize}
              \item \typename{frame\_descr} are at the core of stack unwinding
            \end{itemize}
          }
      \end{itemize}
      \vfill
      \mintocaml[firstline=9,lastline=13,highlightlines=12]{../src/backtrace/backtrace.ml}
      \endminipage
    \end{column}
    \begin{column}{0.5\textwidth}
      \onslide*<1>{\listStashBacktrace[highlightlines=91]{}}
      \onslide*<2>{\listStashBacktrace[highlightlines={91,117-118}]{}}
      \onslide*<3->{\listStashBacktrace[highlightlines={94,106,109-110}]{}}
    \end{column}
  \end{columns}
\end{frame}

\newcommand\listFrameDescr[1][]{
  \listocaml[firstline=111,lastline=124,#1]{../ocaml/asmcomp/emitaux.ml}{asmcomp/emitaux.ml:111}}
\newcommand\listDebugInfo[1][]{
  \listocaml[firstline=38,lastline=49,#1]{../ocaml/lambda/debuginfo.mli}{lambda/debuginfo.mli:38}}

\begin{frame}{Backtraces}{\typename{frame\_descr} and \typename{frame\_debuginfo} in a nutshell}
  \begin{columns}[c]
    \begin{column}{0.5\textwidth}
      \begin{itemize}
        \item<1-> For every return address in an OCaml program, there is a associated \typename{frame\_descr}
        \item<2-> A \typename{frame\_descr} specifies
        \begin{itemize}
          \item<2-> The size of the function stack frame
          \item<3-> Where live values are on the stack (so that the GC can mark them as live and prevent from being swept)
        \end{itemize}
        \item<4-> When compiled with debug information, \typename{frame\_descr} will be linked to a \typename{frame\_debuginfo}
        \begin{itemize}
          \item<5-> Contains information about the position in the source code (file name, line and column number)
        \end{itemize}
      \end{itemize}
    \end{column}
    \begin{column}{0.5\textwidth}
        \onslide*<1>{\listFrameDescr[highlightlines={118-122}]{}}
        \onslide*<2>{\listFrameDescr[highlightlines={120}]{}}
        \onslide*<3>{\listFrameDescr[highlightlines={121}]{}}
        \onslide*<4->{\listFrameDescr[highlightlines={122}]{}}
        \medskip
        \onslide*<1-3>{\listDebugInfo{}}
        \onslide*<4>{\listDebugInfo[highlightlines={38,49}]{}}
        \onslide*<5>{\listDebugInfo[highlightlines={39-46}]{}}
    \end{column}
  \end{columns}
\end{frame}

\begin{frame}{Backtraces}{Scanning the stack with \typename{frame\_descr}}
  \begin{columns}[c]
    \begin{column}{0.5\textwidth}
      \begin{itemize}
        \item Given the return address of the code that raises the exception by calling \funcname{caml\_raise\_exn} (\funcarg{pc} argument of \funcname{caml\_stash\_backtrace})
        \item And the position of the stack pointer (\funcarg{sp} argument of \funcname{caml\_stash\_backtrace})
        \item The frame size given by \typename{frame\_descr}.\funcarg{frame\_size}, allows to find the end of the previous frame
        \item And the return address of the caller of this function
        \bigskip
        \onslide<3->{
        \item Note that traps (here the reddish \funcname{ExnA}, \funcname{ExnB} and \funcname{ExnC} blocks) belong to the frame that installed them, and so the frame size changes when traps are removed
        }
      \end{itemize}
    \end{column}
    \begin{column}{0.5\textwidth}
      \raggedleft
      \onslide*<1>{\providecommand\step{2}\input{diagrams/backtrace/backtrace.tex}}%
      \onslide*<2>{\providecommand\step{1}\input{diagrams/backtrace/scanstack.tex}}%
      \onslide*<3>{\providecommand\step{2}\input{diagrams/backtrace/scanstack.tex}}%
      \onslide*<4>{\providecommand\step{3}\input{diagrams/backtrace/scanstack.tex}}%
      \onslide*<5>{\providecommand\step{4}\input{diagrams/backtrace/scanstack.tex}}%
      \onslide*<6>{\providecommand\step{5}\input{diagrams/backtrace/scanstack.tex}}%
    \end{column}
  \end{columns}
\end{frame}

% Duplicate of previous-previous slide
\begin{frame}{Backtraces}{Continuing on \funcname{caml\_stash\_backtrace}}
  \begin{columns}[c]
    \begin{column}{0.5\textwidth}
      \minipage[c][0.75\textheight][s]{\columnwidth}
      \begin{itemize}
          \color{gray}
        \item A C function provided by the runtime (specific to native code)
        \item It scans the stack, between the stack pointer at the raising point up to the next trap, in order to collect the names of the detected function frames
        \item It uses the same technique and information as the Garbage Collector (GC) uses to scan the values live on the stack
      \end{itemize}
      \begin{itemize}
        \item For each function frame detected, store a pointer to the \typename{frame\_descr} in the \localname{domain\_state->backtrace\_buffer} array, effectively constructing the backtrace
      \end{itemize}
      \onslide<2->{
        \begin{itemize}
            \smallskip
          \item Constructed backtrace:
            \funcname{definitely\_raise},
            \onslide<3->{\funcname{probably\_raise}}
        \end{itemize}
      }
      \vfill
      \mintocaml[firstline=9,lastline=13,highlightlines=12]{../src/backtrace/backtrace.ml}
      \endminipage
    \end{column}
    \begin{column}{0.5\textwidth}
      \raggedleft
      \onslide*<1>{\listStashBacktrace[highlightlines={114-115}]{}}
      \onslide*<2>{\providecommand\step{3}\input{diagrams/backtrace/backtrace.tex}}%
      \onslide*<3>{\providecommand\step{4}\input{diagrams/backtrace/backtrace.tex}}%
    \end{column}
  \end{columns}
\end{frame}

\begin{frame}{Backtraces}{Back to \funcname{caml\_raise\_exn}}
  \begin{columns}[c]
    \begin{column}{0.5\textwidth}
      \minipage[c][0.66\textheight][s]{\columnwidth}
      \begin{itemize}\color{gray}
        \item Checks wheteher exceptions backtrace should be recorded, by checking the \funcarg{domain\_state->backtrace\_active} value
        \item Switches to the C stack to be able to execute C code
        \item Calls \funcname{caml\_stash\_backtrace}, to collect the backtrace up to the next trap
      \end{itemize}
      \onslide*<2->{
        \begin{itemize}
          \item<2-> Executes the exception handler in \funcname{probably\_raise} with \funcname{RESTORE\_EXN\_HANDLER\_OCAML}
            \begin{itemize}
              \item Set \regname{RSP} to \localname{domain\_state->exn\_handler} value
              \item Restore \localname{domain\_state->exn\_handler} from trap
              \item Jump to the handler code with \funcname{ret}
            \end{itemize}
        \end{itemize}
      }
      \vfill
      \onslide*<1>{\mintocaml[firstline=9,lastline=13,highlightlines=12]{../src/backtrace/backtrace.ml}}
      \onslide*<2->{\mintocaml[firstline=15,lastline=21,highlightlines={19-21}]{../src/backtrace/backtrace.ml}}
      \endminipage
    \end{column}
    \begin{column}{0.5\textwidth}
      \centering
      \onslide*<1>{
        \mintc[firstline=190,lastline=192]{../ocaml/runtime/amd64.S}
        \mintc[firstline=234,lastline=237]{../ocaml/runtime/amd64.S}
      }
      \onslide*<2>{
        \mintc[firstline=190,lastline=192,highlightlines={191-192}]{../ocaml/runtime/amd64.S}
        \mintc[firstline=234,lastline=237,highlightlines={235-237}]{../ocaml/runtime/amd64.S}
      }
      \onslide*<1>{\listCamlRaiseExn[highlightlines=720]{}}
      \onslide*<2>{\listCamlRaiseExn[highlightlines=722]{}}
      \onslide*<3->{\providecommand\step{5}\input{diagrams/backtrace/backtrace.tex}}
    \end{column}
  \end{columns}
\end{frame}

\begin{frame}{Backtraces}{\funcname{caml\_probably\_raise}'s handler doesn't catch \typename{ExnC}}
  \begin{columns}[c]
    \begin{column}{0.45\textwidth}
      \minipage[c][0.75\textheight][s]{\columnwidth}
      \begin{itemize}
        \item<1-> \funcname{match \dots with}\footnotemark pattern matching can also match exceptions
        \item<1-> The CMM of \funcname{probably\_raise} shows that this \funcname{match \dots with} is implemented with a pattern matching and a \funcname{try \dots with}
        \item<1-> But this one only matches on \typename{ExnA} exception
        \item<2-> Since the pattern matching fails to catch the exception, \funcname{reraise} is called with our current \typename{ExnC} exception
        \item<3-> Disassembly shows that \funcname{reraise} is actually a call to \funcname{caml\_reraise\_exn}, which is also an assembly function provided by the runtime
      \end{itemize}
      \vfill
      \mintocaml[firstline=15,lastline=21,highlightlines={18-21}]{../src/backtrace/backtrace.ml}
      \endminipage
    \end{column}
    \begin{column}{0.55\textwidth}
      \centering
      \onslide*<1>{\mintcmm[highlightlines={10-18}]{../src/backtrace/camlBacktrace\_\_probably\_raise\_351.cmm}}
      \onslide*<2>{\listcmm[highlightlines=18]{../src/backtrace/camlBacktrace\_\_probably\_raise_351.cmm}{CMM of probably\_raise}}
      \onslide*<3>{\mintobjdump[firstline=5,lastline=35,highlightlines=31]{../src/backtrace/camlBacktrace\_\_probably\_raise_351.objdump}}
    \end{column}
  \end{columns}
  \only<1>{\footnotetext[1]{https://blog.janestreet.com/pattern-matching-and-exception-handling-unite/}}
\end{frame}

\newcommand\listCamlReraiseExn[1][]{
  \listasm[firstline=727,lastline=734,#1]{../ocaml/runtime/amd64.S}{runtime/amd64.S:727}}

\begin{frame}{Backtraces}{Introducing \funcname{caml\_reraise\_exn}}
  \begin{columns}[c]
    \begin{column}{0.5\textwidth}
      \minipage[c][0.66\textheight][s]{\columnwidth}
      \begin{itemize}
        \item<1-> The exception have to be reraised to the next handler
        \item<2-> First, checks if the backtrace should be collected with \funcarg{domain\_state->backtrace\_active}
        \only<3>{
        \begin{itemize}
            \item If not, behaves like a \funcname{raise\_notrace} with \funcname{RESTORE\_EXN\_HANDLER\_OCAML}
        \end{itemize}
        }
        \item<4-> Jumps into \funcname{caml\_raise\_exn}
        \item<5-> Calls \funcname{caml\_stash\_backtrace} again, to scan the next part of the stack: from the trap just pop-ed to the next trap
      \end{itemize}
      \vfill
      \mintocaml[firstline=15,lastline=21,highlightlines={18-21}]{../src/backtrace/backtrace.ml}
      \endminipage
    \end{column}
    \begin{column}{0.5\textwidth}
      \raggedleft
      \onslide*<1>{\listCamlReraiseExn{}}
      \onslide*<2>{\listCamlReraiseExn[highlightlines=729]{}}
      \onslide*<3>{
        \mintc[firstline=190,lastline=192,highlightlines={191-192}]{../ocaml/runtime/amd64.S}
        \mintc[firstline=234,lastline=237,highlightlines={235-237}]{../ocaml/runtime/amd64.S}
        \medskip
        \listCamlReraiseExn[highlightlines=731]{}
      }
      \onslide*<4-5>{
        % TODO refactor with \listCamlReraiseExn
        \mintasm[firstline=727,lastline=734,highlightlines=730]{../ocaml/runtime/amd64.S}
        \medskip
      }
      \onslide*<4>{\listCamlRaiseExn[highlightlines=711]{}}
      \onslide*<5>{\listCamlRaiseExn[highlightlines=720]{}}
    \end{column}
  \end{columns}
\end{frame}

\begin{frame}{Backtraces}{Backtrace collection continues}
  \begin{columns}[c]
    \begin{column}{0.55\textwidth}
      \minipage[c][0.75\textheight][s]{\columnwidth}
      \begin{itemize}
        \item<1-> \funcname{caml\_stash\_backtrace} scans the older part of the stack, up to the next trap
        \item<6-> Controls jumps to the nested exception handler in \funcname{maybe\_raise}, which matches only on \typename{ExnB}
        \item<7-> \funcname{caml\_reraise\_exn} is called again, backtrace collection resumes with \funcname{caml\_stash\_backtrace}
      \end{itemize}
      \medskip
      \begin{itemize}
        \item<2-> Constructed backtrace:
        \begin{itemize}
          \item <2-> \funcname{definitely\_raise}
          \item <2-> \funcname{probably\_raise}
          \item <3-> \funcname{probably\_raise} (re-raise)
          \item <4-> \funcname{maybe\_raise}
          \item <5-> \funcname{main}
          \item <8-> \funcname{main} (re-raise)
        \end{itemize}
      \end{itemize}
      \vfill
      \onslide*<1-5>{\mintocaml[firstline=15,lastline=21]{../src/backtrace/backtrace.ml}}
      \onslide*<6>{\mintocaml[firstline=28,lastline=34,highlightlines=31]{../src/backtrace/backtrace.ml}}
      \onslide*<7->{\mintocaml[firstline=28,lastline=34,highlightlines=33]{../src/backtrace/backtrace.ml}}
      \endminipage
    \end{column}
    \begin{column}{0.45\textwidth}
      \raggedleft
      \onslide*<1>{\providecommand\step{6}\input{diagrams/backtrace/backtrace.tex}}%
      \onslide*<2>{\providecommand\step{6}\input{diagrams/backtrace/backtrace.tex}}%
      \onslide*<3>{\providecommand\step{7}\input{diagrams/backtrace/backtrace.tex}}%
      \onslide*<4>{\providecommand\step{8}\input{diagrams/backtrace/backtrace.tex}}%
      \onslide*<5>{\providecommand\step{9}\input{diagrams/backtrace/backtrace.tex}}%
      \onslide*<6>{\providecommand\step{10}\input{diagrams/backtrace/backtrace.tex}}%
      \onslide*<7>{\providecommand\step{11}\input{diagrams/backtrace/backtrace.tex}}%
      \onslide*<8>{\providecommand\step{12}\input{diagrams/backtrace/backtrace.tex}}%
      \onslide*<9>{\providecommand\step{13}\input{diagrams/backtrace/backtrace.tex}}%
    \end{column}
  \end{columns}
\end{frame}

\begin{frame}{Backtraces}{Printing the backtrace}
  \begin{columns}[c]
    \begin{column}{0.45\textwidth}
      \minipage[c][0.66\textheight][s]{\columnwidth}
      \begin{itemize}
        \item<1-> The exception backtrace can now be printed to \funcarg{stdout} with \funcname{Printexc.print_backtrace}
        \item<2-> First the exception backtrace is retrieved into an OCaml array with \funcname{get_raw_backtrace }
        \item<3-> Which is implemented as the \funcname{caml_get_exception_raw_backtrace} C function in the runtime
        \item<4-> And finally printed with \funcname{print_exception_backtrace}
      \end{itemize}
      \vfill
      \mintocaml[firstline=28,lastline=34,highlightlines=33]{../src/backtrace/backtrace.ml}
      \endminipage
    \end{column}
    \begin{column}{0.55\textwidth}
      \onslide*<1,2>{
        \mintocaml[firstline=100,lastline=101]{../ocaml/stdlib/printexc.ml}
      }
      \only<1>{
        \listocaml[firstline=170,lastline=172]{../ocaml/stdlib/printexc.ml}{stdlib/printexc.ml:170}
      }
      \only<2>{
        \listocaml[firstline=170,lastline=172,highlightlines=172]{../ocaml/stdlib/printexc.ml}{stdlib/printexc.ml:170}
      }
      \only<3>{
        \mintc[firstline=154,lastline=156]{../ocaml/runtime/backtrace.c}
        \listc[firstline=171,lastline=192,highlightlines={184-187}]{../ocaml/runtime/backtrace.c}{runtime/backtrace.c:154}
      }
      \only<4>{
        \listocaml[firstline=155,lastline=165]{../ocaml/stdlib/printexc.ml}{stdlib/printexc.ml:55}
      }
    \end{column}
  \end{columns}
\end{frame}


\subsection{The multiple kinds of raise}
\frameSubsection{}{}

\begin{frame}{Backtraces}{When backtraces are not needed}
  \begin{columns}[c]
    \begin{column}{0.45\textwidth}
      \minipage[c][0.66\textheight][s]{\columnwidth}
      \begin{itemize}
        \item<1-> What if the backtrace for an exception is never needed?
        \item<2-> It's possible to raise the exception explicitly with \funcname{raise\_notrace} to make raising as cheap as possible
        \item<3-> An example of use in the \funcname{dynlink} library of the OCaml compiler
      \end{itemize}
      \vfill
      \onslide<3->{
      \listocaml[firstline=205,lastline=209,highlightlines=207]{../ocaml/otherlibs/dynlink/dynlink\_compilerlibs/misc.ml}{otherlibs/dynlink/dynlink\_compilerlibs/misc.ml:205}
      }
      \endminipage
    \end{column}
    \begin{column}{0.55\textwidth}
    \only<4>{
      \mintcmm[highlightlines=3]{../src/notrace/camlNotrace\_\_fun_347.cmm}
      \listcmm{../src/notrace/camlNotrace\_\_all\_somes\_327.cmm}{all\_somes}
    }
    \onslide*<5->{
      \listobjdump[highlightlines={6-9}]{../src/notrace/camlNotrace\_\_fun_347.objdump}{anonymous function from \funcname{all\_somes}}
    }
    \end{column}
  \end{columns}
\end{frame}

\begin{frame}{Backtraces}{Summary table of the multiple kinds of raise}
  \begin{itemize}
    \item When debugging information is enabled and if the backtrace of an exception isn't needed, it's possible to raise with \funcname{raise\_notrace} for performance
    \item When debugging information is \emph{not} enabled, the compiler optimises \funcname{raise} calls to inline assembly (same effect as using \funcname{raise\_notrace})
  \end{itemize}
  \bigskip
  \centering
  \begin{tabular}{| l | c | c | c | c |}
    \hline
    \parbox{10em}{Debug information \\ (\funcarg{-g} flag)}  & \multicolumn{2}{|c|}{Disabled}                                     & \multicolumn{2}{|c|}{Enabled} \\
    \hline
    Raise kind                                               & \funcname{raise} & \funcname{raise\_notrace}                       & \funcname{raise} & \funcname{raise\_notrace} \\
    \hline
    Compilation                                              & \parbox{8em}{inline assembly\\ (optimization)} & inline assembly   & \funcname{caml\_raise\_exn} & inline assembly \\
    \hline
  \end{tabular}
\end{frame}


%
% Takeaway #5
%
\subsection*{Takeaway \#5}
\frameSubsectionTakeaway{}
\againframe<5>{takeaway}
}%
      \onslide*<2>{\providecommand\step{1}\input{diagrams/backtrace/scanstack.tex}}%
      \onslide*<3>{\providecommand\step{2}\input{diagrams/backtrace/scanstack.tex}}%
      \onslide*<4>{\providecommand\step{3}\input{diagrams/backtrace/scanstack.tex}}%
      \onslide*<5>{\providecommand\step{4}\input{diagrams/backtrace/scanstack.tex}}%
      \onslide*<6>{\providecommand\step{5}\input{diagrams/backtrace/scanstack.tex}}%
    \end{column}
  \end{columns}
\end{frame}

% Duplicate of previous-previous slide
\begin{frame}{Backtraces}{Continuing on \funcname{caml\_stash\_backtrace}}
  \begin{columns}[c]
    \begin{column}{0.5\textwidth}
      \minipage[c][0.75\textheight][s]{\columnwidth}
      \begin{itemize}
          \color{gray}
        \item A C function provided by the runtime (specific to native code)
        \item It scans the stack, between the stack pointer at the raising point up to the next trap, in order to collect the names of the detected function frames
        \item It uses the same technique and information as the Garbage Collector (GC) uses to scan the values live on the stack
      \end{itemize}
      \begin{itemize}
        \item For each function frame detected, store a pointer to the \typename{frame\_descr} in the \localname{domain\_state->backtrace\_buffer} array, effectively constructing the backtrace
      \end{itemize}
      \onslide<2->{
        \begin{itemize}
            \smallskip
          \item Constructed backtrace:
            \funcname{definitely\_raise},
            \onslide<3->{\funcname{probably\_raise}}
        \end{itemize}
      }
      \vfill
      \mintocaml[firstline=9,lastline=13,highlightlines=12]{../src/backtrace/backtrace.ml}
      \endminipage
    \end{column}
    \begin{column}{0.5\textwidth}
      \raggedleft
      \onslide*<1>{\listStashBacktrace[highlightlines={114-115}]{}}
      \onslide*<2>{\providecommand\step{3}%
% Backtraces
%
\subsection{One advantage of raising with debugging information}
\frameSubsection{}{}

\begin{frame}{Backtraces}{Debugging information \& source code}
  \begin{columns}[c]
    \begin{column}{0.5\textwidth}
      \begin{itemize}
        \item Raising an exception is fun and games, but how do we know where the exception originated? $\rightarrow$ With the backtrace associated with the exception!
        \item In order to get a backtrace from the point where an exception was raised, it's necessary to compile with debugging information enabled (\funcarg{-g} flag for the compiler)
        \item Same example as in the `Nested handlers' section
      \end{itemize}
    \end{column}
    \begin{column}{0.5\textwidth}
      \listocaml[firstline=9,lastline=34]{../src/backtrace/backtrace.ml}{backtrace/backtrace.ml}
    \end{column}
  \end{columns}
\end{frame}

\begin{frame}{Backtraces}{CMM and disassembly of \funcname{definitely\_raise}}
  \begin{columns}[c]
    \begin{column}{0.5\textwidth}
      \minipage[c][0.66\textheight][s]{\columnwidth}
      \begin{itemize}
        \item<1-> An effect of compiling with debugging information (\funcarg{-g}): the CMM no longer uses \funcname{raise\_notrace} to raise the exception but \funcname{raise} instead
        \item<2-> Disassembly shows that CMM \funcname{raise} is actually a call to \funcname{caml\_raise\_exn}, a function in assembler provided by the runtime
      \end{itemize}
      \vfill
      \mintocaml[firstline=9,lastline=13,highlightlines=12]{../src/backtrace/backtrace.ml}
      \endminipage
    \end{column}
    \begin{column}{0.5\textwidth}
      \onslide*<1>{
        \mintcmm[highlightlines=5]{../src/backtrace/camlBacktrace\_\_definitely\_raise\_347.cmm}
      }
      \onslide*<2>{
        \mintobjdump[firstline=2,highlightlines=14]{../src/backtrace/camlBacktrace\_\_definitely\_raise\_347.objdump}
      }
    \end{column}
  \end{columns}
\end{frame}

\begin{frame}{Backtraces}{Introducing \funcname{caml\_raise\_exn}}
  \begin{columns}[c]
    \begin{column}{0.5\textwidth}
      \minipage[c][0.66\textheight][s]{\columnwidth}
      \onslide*<1>{
        \begin{itemize}
          \item Raising the exception is performed using \funcname{caml\_raise\_exn} which is
            \begin{itemize}
              \item An assembly function provided by the runtime
              \item Architecture specific
            \end{itemize}
          \item Let's see what it does differently from \funcname{raise\_notrace}\dots
          \item We are about to raise \typename{ExnC} with \funcname{caml\_raise\_exn} from \funcname{definitely\_raise}
        \end{itemize}
      }
      \onslide*<2>{
        \mintobjdump[firstline=2,highlightlines=14]{../src/backtrace/camlBacktrace\_\_definitely\_raise\_347.objdump}
      }
      \vfill
      \mintocaml[firstline=9,lastline=13,highlightlines=12]{../src/backtrace/backtrace.ml}
      \endminipage
    \end{column}
    \begin{column}{0.5\textwidth}
      \centering
      \providecommand\step{1}\input{diagrams/backtrace/backtrace.tex}
    \end{column}
  \end{columns}
\end{frame}

\begin{frame}{Backtraces}{But before that, we need more fields from \typename{caml\_domain\_state}}
  \begin{columns}
    \begin{column}{0.5\textwidth}
      \begin{itemize}
        \item Remember \typename{caml\_domain\_state} the per-domain runtime state?
        \item Some other members are of interest to us now\dots
        \item \localname{backtrace\_active} is used to know if the runtime should collect backtraces
        \item \localname{backtrace\_active} can be set by
          \begin{itemize}
            \item The \funcarg{b} flag in the \funcname{OCAMLRUNPARAM} environment variable
            \item \mintinline{ocaml}{Printexc.record_backtrace : bool -> unit} \footnotemark
          \end{itemize}
      \end{itemize}
    \end{column}
    \begin{column}{0.5\textwidth}
      \begin{listing}
        \mintc[highlightlines={9-11}]{src/ocaml/domain\_state\_backtrace.h}
        \caption{runtime/caml/domain\_state.h}
      \end{listing}
    \end{column}
  \end{columns}
  \footnotetext[1]{https://v2.ocaml.org/api/Printexc.html}
\end{frame}

\newcommand\listCamlRaiseExn[1][]{
  \listasm[firstline=702,lastline=725,#1]{../ocaml/runtime/amd64.S}{runtime/amd64.S:702}}

\begin{frame}{Backtraces}{Walk through \funcname{caml\_raise\_exn}}
  \begin{columns}[T]
    \begin{column}{0.5\textwidth}
      \begin{itemize}
        \item<1-> First checks whether exceptions backtrace should be recorded
        \item<1-> By checking the \funcarg{domain\_state->backtrace\_active} value
        \item<2-> If not, acts as \funcname{raise\_notrace} with \funcname{RESTORE\_EXN\_HANDLER\_OCAML}
          \begin{itemize}
            \item Set \regname{RSP} to \localname{domain\_state->exn\_handler} value
            \item Restore \localname{domain\_state->exn\_handler} from trap
            \item Jump with \funcname{ret} (same effect as using \funcname{pop} and \funcname{jmp})
          \end{itemize}
        \item<3-> Otherwise, switches to the C stack to be able to execute C code
        \item<4-> Calls \funcname{caml\_stash\_backtrace}, a C function provided by the runtime\ignorespaces
          \onslide<6->{, with
            \begin{itemize}
              \item \funcarg{exn}: exception value
              \item \funcarg{pc}: code address of the raising point
              \item \funcarg{sp}: stack address of the raising function
              \item \funcarg{trapsp}: stack address of the next handler
            \end{itemize}
          }
      \end{itemize}
      \bigskip
      \mintocaml[firstline=9,lastline=13,highlightlines=12]{../src/backtrace/backtrace.ml}
    \end{column}
    \begin{column}{0.5\textwidth}
      \raggedleft
      \onslide*<1,3-4>{
        \mintc[firstline=190,lastline=192]{../ocaml/runtime/amd64.S}
        \mintc[firstline=234,lastline=237]{../ocaml/runtime/amd64.S}
      }
      \onslide*<2>{
        \mintc[firstline=190,lastline=192,highlightlines={191-192}]{../ocaml/runtime/amd64.S}
        \mintc[firstline=234,lastline=237,highlightlines={235-237}]{../ocaml/runtime/amd64.S}
      }
      \onslide*<1>{\listCamlRaiseExn[highlightlines=705]{}}%
      \onslide*<2>{\listCamlRaiseExn[highlightlines={707-708}]{}}%
      \onslide*<3>{\listCamlRaiseExn[highlightlines={712-714}]{}}%
      \onslide*<4>{\listCamlRaiseExn[highlightlines={715-720}]{}}%
      \onslide*<5>{\providecommand\step{1}\input{diagrams/backtrace/backtrace.tex}}%
      \onslide*<6>{\providecommand\step{2}\input{diagrams/backtrace/backtrace.tex}}%
    \end{column}
  \end{columns}
\end{frame}

\newcommand\listStashBacktrace[1][]{
  \listc[firstline=91,lastline=120,#1]{../ocaml/runtime/backtrace\_nat.c}{runtime/backtrace\_nat.c:91}}

\begin{frame}{Backtraces}{Introducing \funcname{caml\_stash\_backtrace}}
  \begin{columns}[c]
    \begin{column}{0.5\textwidth}
      \minipage[c][0.66\textheight][s]{\columnwidth}
      \begin{itemize}
        \item<1-> A C function provided by the runtime (specific to native code)
        \item<2-> It scans the stack, between the stack pointer at the raising point up to the next trap, in order to collect the names of the detected function frames
          \onslide*<2>{
            \begin{itemize}
              \item And more information, like line number, column number...
            \end{itemize}
          }
        \item<3-> It uses the same technique and information as the Garbage Collector (GC) uses to scan the values live on the stack
          \onslide*<4>{
            \begin{itemize}
              \item \typename{frame\_descr} are at the core of stack unwinding
            \end{itemize}
          }
      \end{itemize}
      \vfill
      \mintocaml[firstline=9,lastline=13,highlightlines=12]{../src/backtrace/backtrace.ml}
      \endminipage
    \end{column}
    \begin{column}{0.5\textwidth}
      \onslide*<1>{\listStashBacktrace[highlightlines=91]{}}
      \onslide*<2>{\listStashBacktrace[highlightlines={91,117-118}]{}}
      \onslide*<3->{\listStashBacktrace[highlightlines={94,106,109-110}]{}}
    \end{column}
  \end{columns}
\end{frame}

\newcommand\listFrameDescr[1][]{
  \listocaml[firstline=111,lastline=124,#1]{../ocaml/asmcomp/emitaux.ml}{asmcomp/emitaux.ml:111}}
\newcommand\listDebugInfo[1][]{
  \listocaml[firstline=38,lastline=49,#1]{../ocaml/lambda/debuginfo.mli}{lambda/debuginfo.mli:38}}

\begin{frame}{Backtraces}{\typename{frame\_descr} and \typename{frame\_debuginfo} in a nutshell}
  \begin{columns}[c]
    \begin{column}{0.5\textwidth}
      \begin{itemize}
        \item<1-> For every return address in an OCaml program, there is a associated \typename{frame\_descr}
        \item<2-> A \typename{frame\_descr} specifies
        \begin{itemize}
          \item<2-> The size of the function stack frame
          \item<3-> Where live values are on the stack (so that the GC can mark them as live and prevent from being swept)
        \end{itemize}
        \item<4-> When compiled with debug information, \typename{frame\_descr} will be linked to a \typename{frame\_debuginfo}
        \begin{itemize}
          \item<5-> Contains information about the position in the source code (file name, line and column number)
        \end{itemize}
      \end{itemize}
    \end{column}
    \begin{column}{0.5\textwidth}
        \onslide*<1>{\listFrameDescr[highlightlines={118-122}]{}}
        \onslide*<2>{\listFrameDescr[highlightlines={120}]{}}
        \onslide*<3>{\listFrameDescr[highlightlines={121}]{}}
        \onslide*<4->{\listFrameDescr[highlightlines={122}]{}}
        \medskip
        \onslide*<1-3>{\listDebugInfo{}}
        \onslide*<4>{\listDebugInfo[highlightlines={38,49}]{}}
        \onslide*<5>{\listDebugInfo[highlightlines={39-46}]{}}
    \end{column}
  \end{columns}
\end{frame}

\begin{frame}{Backtraces}{Scanning the stack with \typename{frame\_descr}}
  \begin{columns}[c]
    \begin{column}{0.5\textwidth}
      \begin{itemize}
        \item Given the return address of the code that raises the exception by calling \funcname{caml\_raise\_exn} (\funcarg{pc} argument of \funcname{caml\_stash\_backtrace})
        \item And the position of the stack pointer (\funcarg{sp} argument of \funcname{caml\_stash\_backtrace})
        \item The frame size given by \typename{frame\_descr}.\funcarg{frame\_size}, allows to find the end of the previous frame
        \item And the return address of the caller of this function
        \bigskip
        \onslide<3->{
        \item Note that traps (here the reddish \funcname{ExnA}, \funcname{ExnB} and \funcname{ExnC} blocks) belong to the frame that installed them, and so the frame size changes when traps are removed
        }
      \end{itemize}
    \end{column}
    \begin{column}{0.5\textwidth}
      \raggedleft
      \onslide*<1>{\providecommand\step{2}\input{diagrams/backtrace/backtrace.tex}}%
      \onslide*<2>{\providecommand\step{1}\input{diagrams/backtrace/scanstack.tex}}%
      \onslide*<3>{\providecommand\step{2}\input{diagrams/backtrace/scanstack.tex}}%
      \onslide*<4>{\providecommand\step{3}\input{diagrams/backtrace/scanstack.tex}}%
      \onslide*<5>{\providecommand\step{4}\input{diagrams/backtrace/scanstack.tex}}%
      \onslide*<6>{\providecommand\step{5}\input{diagrams/backtrace/scanstack.tex}}%
    \end{column}
  \end{columns}
\end{frame}

% Duplicate of previous-previous slide
\begin{frame}{Backtraces}{Continuing on \funcname{caml\_stash\_backtrace}}
  \begin{columns}[c]
    \begin{column}{0.5\textwidth}
      \minipage[c][0.75\textheight][s]{\columnwidth}
      \begin{itemize}
          \color{gray}
        \item A C function provided by the runtime (specific to native code)
        \item It scans the stack, between the stack pointer at the raising point up to the next trap, in order to collect the names of the detected function frames
        \item It uses the same technique and information as the Garbage Collector (GC) uses to scan the values live on the stack
      \end{itemize}
      \begin{itemize}
        \item For each function frame detected, store a pointer to the \typename{frame\_descr} in the \localname{domain\_state->backtrace\_buffer} array, effectively constructing the backtrace
      \end{itemize}
      \onslide<2->{
        \begin{itemize}
            \smallskip
          \item Constructed backtrace:
            \funcname{definitely\_raise},
            \onslide<3->{\funcname{probably\_raise}}
        \end{itemize}
      }
      \vfill
      \mintocaml[firstline=9,lastline=13,highlightlines=12]{../src/backtrace/backtrace.ml}
      \endminipage
    \end{column}
    \begin{column}{0.5\textwidth}
      \raggedleft
      \onslide*<1>{\listStashBacktrace[highlightlines={114-115}]{}}
      \onslide*<2>{\providecommand\step{3}\input{diagrams/backtrace/backtrace.tex}}%
      \onslide*<3>{\providecommand\step{4}\input{diagrams/backtrace/backtrace.tex}}%
    \end{column}
  \end{columns}
\end{frame}

\begin{frame}{Backtraces}{Back to \funcname{caml\_raise\_exn}}
  \begin{columns}[c]
    \begin{column}{0.5\textwidth}
      \minipage[c][0.66\textheight][s]{\columnwidth}
      \begin{itemize}\color{gray}
        \item Checks wheteher exceptions backtrace should be recorded, by checking the \funcarg{domain\_state->backtrace\_active} value
        \item Switches to the C stack to be able to execute C code
        \item Calls \funcname{caml\_stash\_backtrace}, to collect the backtrace up to the next trap
      \end{itemize}
      \onslide*<2->{
        \begin{itemize}
          \item<2-> Executes the exception handler in \funcname{probably\_raise} with \funcname{RESTORE\_EXN\_HANDLER\_OCAML}
            \begin{itemize}
              \item Set \regname{RSP} to \localname{domain\_state->exn\_handler} value
              \item Restore \localname{domain\_state->exn\_handler} from trap
              \item Jump to the handler code with \funcname{ret}
            \end{itemize}
        \end{itemize}
      }
      \vfill
      \onslide*<1>{\mintocaml[firstline=9,lastline=13,highlightlines=12]{../src/backtrace/backtrace.ml}}
      \onslide*<2->{\mintocaml[firstline=15,lastline=21,highlightlines={19-21}]{../src/backtrace/backtrace.ml}}
      \endminipage
    \end{column}
    \begin{column}{0.5\textwidth}
      \centering
      \onslide*<1>{
        \mintc[firstline=190,lastline=192]{../ocaml/runtime/amd64.S}
        \mintc[firstline=234,lastline=237]{../ocaml/runtime/amd64.S}
      }
      \onslide*<2>{
        \mintc[firstline=190,lastline=192,highlightlines={191-192}]{../ocaml/runtime/amd64.S}
        \mintc[firstline=234,lastline=237,highlightlines={235-237}]{../ocaml/runtime/amd64.S}
      }
      \onslide*<1>{\listCamlRaiseExn[highlightlines=720]{}}
      \onslide*<2>{\listCamlRaiseExn[highlightlines=722]{}}
      \onslide*<3->{\providecommand\step{5}\input{diagrams/backtrace/backtrace.tex}}
    \end{column}
  \end{columns}
\end{frame}

\begin{frame}{Backtraces}{\funcname{caml\_probably\_raise}'s handler doesn't catch \typename{ExnC}}
  \begin{columns}[c]
    \begin{column}{0.45\textwidth}
      \minipage[c][0.75\textheight][s]{\columnwidth}
      \begin{itemize}
        \item<1-> \funcname{match \dots with}\footnotemark pattern matching can also match exceptions
        \item<1-> The CMM of \funcname{probably\_raise} shows that this \funcname{match \dots with} is implemented with a pattern matching and a \funcname{try \dots with}
        \item<1-> But this one only matches on \typename{ExnA} exception
        \item<2-> Since the pattern matching fails to catch the exception, \funcname{reraise} is called with our current \typename{ExnC} exception
        \item<3-> Disassembly shows that \funcname{reraise} is actually a call to \funcname{caml\_reraise\_exn}, which is also an assembly function provided by the runtime
      \end{itemize}
      \vfill
      \mintocaml[firstline=15,lastline=21,highlightlines={18-21}]{../src/backtrace/backtrace.ml}
      \endminipage
    \end{column}
    \begin{column}{0.55\textwidth}
      \centering
      \onslide*<1>{\mintcmm[highlightlines={10-18}]{../src/backtrace/camlBacktrace\_\_probably\_raise\_351.cmm}}
      \onslide*<2>{\listcmm[highlightlines=18]{../src/backtrace/camlBacktrace\_\_probably\_raise_351.cmm}{CMM of probably\_raise}}
      \onslide*<3>{\mintobjdump[firstline=5,lastline=35,highlightlines=31]{../src/backtrace/camlBacktrace\_\_probably\_raise_351.objdump}}
    \end{column}
  \end{columns}
  \only<1>{\footnotetext[1]{https://blog.janestreet.com/pattern-matching-and-exception-handling-unite/}}
\end{frame}

\newcommand\listCamlReraiseExn[1][]{
  \listasm[firstline=727,lastline=734,#1]{../ocaml/runtime/amd64.S}{runtime/amd64.S:727}}

\begin{frame}{Backtraces}{Introducing \funcname{caml\_reraise\_exn}}
  \begin{columns}[c]
    \begin{column}{0.5\textwidth}
      \minipage[c][0.66\textheight][s]{\columnwidth}
      \begin{itemize}
        \item<1-> The exception have to be reraised to the next handler
        \item<2-> First, checks if the backtrace should be collected with \funcarg{domain\_state->backtrace\_active}
        \only<3>{
        \begin{itemize}
            \item If not, behaves like a \funcname{raise\_notrace} with \funcname{RESTORE\_EXN\_HANDLER\_OCAML}
        \end{itemize}
        }
        \item<4-> Jumps into \funcname{caml\_raise\_exn}
        \item<5-> Calls \funcname{caml\_stash\_backtrace} again, to scan the next part of the stack: from the trap just pop-ed to the next trap
      \end{itemize}
      \vfill
      \mintocaml[firstline=15,lastline=21,highlightlines={18-21}]{../src/backtrace/backtrace.ml}
      \endminipage
    \end{column}
    \begin{column}{0.5\textwidth}
      \raggedleft
      \onslide*<1>{\listCamlReraiseExn{}}
      \onslide*<2>{\listCamlReraiseExn[highlightlines=729]{}}
      \onslide*<3>{
        \mintc[firstline=190,lastline=192,highlightlines={191-192}]{../ocaml/runtime/amd64.S}
        \mintc[firstline=234,lastline=237,highlightlines={235-237}]{../ocaml/runtime/amd64.S}
        \medskip
        \listCamlReraiseExn[highlightlines=731]{}
      }
      \onslide*<4-5>{
        % TODO refactor with \listCamlReraiseExn
        \mintasm[firstline=727,lastline=734,highlightlines=730]{../ocaml/runtime/amd64.S}
        \medskip
      }
      \onslide*<4>{\listCamlRaiseExn[highlightlines=711]{}}
      \onslide*<5>{\listCamlRaiseExn[highlightlines=720]{}}
    \end{column}
  \end{columns}
\end{frame}

\begin{frame}{Backtraces}{Backtrace collection continues}
  \begin{columns}[c]
    \begin{column}{0.55\textwidth}
      \minipage[c][0.75\textheight][s]{\columnwidth}
      \begin{itemize}
        \item<1-> \funcname{caml\_stash\_backtrace} scans the older part of the stack, up to the next trap
        \item<6-> Controls jumps to the nested exception handler in \funcname{maybe\_raise}, which matches only on \typename{ExnB}
        \item<7-> \funcname{caml\_reraise\_exn} is called again, backtrace collection resumes with \funcname{caml\_stash\_backtrace}
      \end{itemize}
      \medskip
      \begin{itemize}
        \item<2-> Constructed backtrace:
        \begin{itemize}
          \item <2-> \funcname{definitely\_raise}
          \item <2-> \funcname{probably\_raise}
          \item <3-> \funcname{probably\_raise} (re-raise)
          \item <4-> \funcname{maybe\_raise}
          \item <5-> \funcname{main}
          \item <8-> \funcname{main} (re-raise)
        \end{itemize}
      \end{itemize}
      \vfill
      \onslide*<1-5>{\mintocaml[firstline=15,lastline=21]{../src/backtrace/backtrace.ml}}
      \onslide*<6>{\mintocaml[firstline=28,lastline=34,highlightlines=31]{../src/backtrace/backtrace.ml}}
      \onslide*<7->{\mintocaml[firstline=28,lastline=34,highlightlines=33]{../src/backtrace/backtrace.ml}}
      \endminipage
    \end{column}
    \begin{column}{0.45\textwidth}
      \raggedleft
      \onslide*<1>{\providecommand\step{6}\input{diagrams/backtrace/backtrace.tex}}%
      \onslide*<2>{\providecommand\step{6}\input{diagrams/backtrace/backtrace.tex}}%
      \onslide*<3>{\providecommand\step{7}\input{diagrams/backtrace/backtrace.tex}}%
      \onslide*<4>{\providecommand\step{8}\input{diagrams/backtrace/backtrace.tex}}%
      \onslide*<5>{\providecommand\step{9}\input{diagrams/backtrace/backtrace.tex}}%
      \onslide*<6>{\providecommand\step{10}\input{diagrams/backtrace/backtrace.tex}}%
      \onslide*<7>{\providecommand\step{11}\input{diagrams/backtrace/backtrace.tex}}%
      \onslide*<8>{\providecommand\step{12}\input{diagrams/backtrace/backtrace.tex}}%
      \onslide*<9>{\providecommand\step{13}\input{diagrams/backtrace/backtrace.tex}}%
    \end{column}
  \end{columns}
\end{frame}

\begin{frame}{Backtraces}{Printing the backtrace}
  \begin{columns}[c]
    \begin{column}{0.45\textwidth}
      \minipage[c][0.66\textheight][s]{\columnwidth}
      \begin{itemize}
        \item<1-> The exception backtrace can now be printed to \funcarg{stdout} with \funcname{Printexc.print_backtrace}
        \item<2-> First the exception backtrace is retrieved into an OCaml array with \funcname{get_raw_backtrace }
        \item<3-> Which is implemented as the \funcname{caml_get_exception_raw_backtrace} C function in the runtime
        \item<4-> And finally printed with \funcname{print_exception_backtrace}
      \end{itemize}
      \vfill
      \mintocaml[firstline=28,lastline=34,highlightlines=33]{../src/backtrace/backtrace.ml}
      \endminipage
    \end{column}
    \begin{column}{0.55\textwidth}
      \onslide*<1,2>{
        \mintocaml[firstline=100,lastline=101]{../ocaml/stdlib/printexc.ml}
      }
      \only<1>{
        \listocaml[firstline=170,lastline=172]{../ocaml/stdlib/printexc.ml}{stdlib/printexc.ml:170}
      }
      \only<2>{
        \listocaml[firstline=170,lastline=172,highlightlines=172]{../ocaml/stdlib/printexc.ml}{stdlib/printexc.ml:170}
      }
      \only<3>{
        \mintc[firstline=154,lastline=156]{../ocaml/runtime/backtrace.c}
        \listc[firstline=171,lastline=192,highlightlines={184-187}]{../ocaml/runtime/backtrace.c}{runtime/backtrace.c:154}
      }
      \only<4>{
        \listocaml[firstline=155,lastline=165]{../ocaml/stdlib/printexc.ml}{stdlib/printexc.ml:55}
      }
    \end{column}
  \end{columns}
\end{frame}


\subsection{The multiple kinds of raise}
\frameSubsection{}{}

\begin{frame}{Backtraces}{When backtraces are not needed}
  \begin{columns}[c]
    \begin{column}{0.45\textwidth}
      \minipage[c][0.66\textheight][s]{\columnwidth}
      \begin{itemize}
        \item<1-> What if the backtrace for an exception is never needed?
        \item<2-> It's possible to raise the exception explicitly with \funcname{raise\_notrace} to make raising as cheap as possible
        \item<3-> An example of use in the \funcname{dynlink} library of the OCaml compiler
      \end{itemize}
      \vfill
      \onslide<3->{
      \listocaml[firstline=205,lastline=209,highlightlines=207]{../ocaml/otherlibs/dynlink/dynlink\_compilerlibs/misc.ml}{otherlibs/dynlink/dynlink\_compilerlibs/misc.ml:205}
      }
      \endminipage
    \end{column}
    \begin{column}{0.55\textwidth}
    \only<4>{
      \mintcmm[highlightlines=3]{../src/notrace/camlNotrace\_\_fun_347.cmm}
      \listcmm{../src/notrace/camlNotrace\_\_all\_somes\_327.cmm}{all\_somes}
    }
    \onslide*<5->{
      \listobjdump[highlightlines={6-9}]{../src/notrace/camlNotrace\_\_fun_347.objdump}{anonymous function from \funcname{all\_somes}}
    }
    \end{column}
  \end{columns}
\end{frame}

\begin{frame}{Backtraces}{Summary table of the multiple kinds of raise}
  \begin{itemize}
    \item When debugging information is enabled and if the backtrace of an exception isn't needed, it's possible to raise with \funcname{raise\_notrace} for performance
    \item When debugging information is \emph{not} enabled, the compiler optimises \funcname{raise} calls to inline assembly (same effect as using \funcname{raise\_notrace})
  \end{itemize}
  \bigskip
  \centering
  \begin{tabular}{| l | c | c | c | c |}
    \hline
    \parbox{10em}{Debug information \\ (\funcarg{-g} flag)}  & \multicolumn{2}{|c|}{Disabled}                                     & \multicolumn{2}{|c|}{Enabled} \\
    \hline
    Raise kind                                               & \funcname{raise} & \funcname{raise\_notrace}                       & \funcname{raise} & \funcname{raise\_notrace} \\
    \hline
    Compilation                                              & \parbox{8em}{inline assembly\\ (optimization)} & inline assembly   & \funcname{caml\_raise\_exn} & inline assembly \\
    \hline
  \end{tabular}
\end{frame}


%
% Takeaway #5
%
\subsection*{Takeaway \#5}
\frameSubsectionTakeaway{}
\againframe<5>{takeaway}
}%
      \onslide*<3>{\providecommand\step{4}%
% Backtraces
%
\subsection{One advantage of raising with debugging information}
\frameSubsection{}{}

\begin{frame}{Backtraces}{Debugging information \& source code}
  \begin{columns}[c]
    \begin{column}{0.5\textwidth}
      \begin{itemize}
        \item Raising an exception is fun and games, but how do we know where the exception originated? $\rightarrow$ With the backtrace associated with the exception!
        \item In order to get a backtrace from the point where an exception was raised, it's necessary to compile with debugging information enabled (\funcarg{-g} flag for the compiler)
        \item Same example as in the `Nested handlers' section
      \end{itemize}
    \end{column}
    \begin{column}{0.5\textwidth}
      \listocaml[firstline=9,lastline=34]{../src/backtrace/backtrace.ml}{backtrace/backtrace.ml}
    \end{column}
  \end{columns}
\end{frame}

\begin{frame}{Backtraces}{CMM and disassembly of \funcname{definitely\_raise}}
  \begin{columns}[c]
    \begin{column}{0.5\textwidth}
      \minipage[c][0.66\textheight][s]{\columnwidth}
      \begin{itemize}
        \item<1-> An effect of compiling with debugging information (\funcarg{-g}): the CMM no longer uses \funcname{raise\_notrace} to raise the exception but \funcname{raise} instead
        \item<2-> Disassembly shows that CMM \funcname{raise} is actually a call to \funcname{caml\_raise\_exn}, a function in assembler provided by the runtime
      \end{itemize}
      \vfill
      \mintocaml[firstline=9,lastline=13,highlightlines=12]{../src/backtrace/backtrace.ml}
      \endminipage
    \end{column}
    \begin{column}{0.5\textwidth}
      \onslide*<1>{
        \mintcmm[highlightlines=5]{../src/backtrace/camlBacktrace\_\_definitely\_raise\_347.cmm}
      }
      \onslide*<2>{
        \mintobjdump[firstline=2,highlightlines=14]{../src/backtrace/camlBacktrace\_\_definitely\_raise\_347.objdump}
      }
    \end{column}
  \end{columns}
\end{frame}

\begin{frame}{Backtraces}{Introducing \funcname{caml\_raise\_exn}}
  \begin{columns}[c]
    \begin{column}{0.5\textwidth}
      \minipage[c][0.66\textheight][s]{\columnwidth}
      \onslide*<1>{
        \begin{itemize}
          \item Raising the exception is performed using \funcname{caml\_raise\_exn} which is
            \begin{itemize}
              \item An assembly function provided by the runtime
              \item Architecture specific
            \end{itemize}
          \item Let's see what it does differently from \funcname{raise\_notrace}\dots
          \item We are about to raise \typename{ExnC} with \funcname{caml\_raise\_exn} from \funcname{definitely\_raise}
        \end{itemize}
      }
      \onslide*<2>{
        \mintobjdump[firstline=2,highlightlines=14]{../src/backtrace/camlBacktrace\_\_definitely\_raise\_347.objdump}
      }
      \vfill
      \mintocaml[firstline=9,lastline=13,highlightlines=12]{../src/backtrace/backtrace.ml}
      \endminipage
    \end{column}
    \begin{column}{0.5\textwidth}
      \centering
      \providecommand\step{1}\input{diagrams/backtrace/backtrace.tex}
    \end{column}
  \end{columns}
\end{frame}

\begin{frame}{Backtraces}{But before that, we need more fields from \typename{caml\_domain\_state}}
  \begin{columns}
    \begin{column}{0.5\textwidth}
      \begin{itemize}
        \item Remember \typename{caml\_domain\_state} the per-domain runtime state?
        \item Some other members are of interest to us now\dots
        \item \localname{backtrace\_active} is used to know if the runtime should collect backtraces
        \item \localname{backtrace\_active} can be set by
          \begin{itemize}
            \item The \funcarg{b} flag in the \funcname{OCAMLRUNPARAM} environment variable
            \item \mintinline{ocaml}{Printexc.record_backtrace : bool -> unit} \footnotemark
          \end{itemize}
      \end{itemize}
    \end{column}
    \begin{column}{0.5\textwidth}
      \begin{listing}
        \mintc[highlightlines={9-11}]{src/ocaml/domain\_state\_backtrace.h}
        \caption{runtime/caml/domain\_state.h}
      \end{listing}
    \end{column}
  \end{columns}
  \footnotetext[1]{https://v2.ocaml.org/api/Printexc.html}
\end{frame}

\newcommand\listCamlRaiseExn[1][]{
  \listasm[firstline=702,lastline=725,#1]{../ocaml/runtime/amd64.S}{runtime/amd64.S:702}}

\begin{frame}{Backtraces}{Walk through \funcname{caml\_raise\_exn}}
  \begin{columns}[T]
    \begin{column}{0.5\textwidth}
      \begin{itemize}
        \item<1-> First checks whether exceptions backtrace should be recorded
        \item<1-> By checking the \funcarg{domain\_state->backtrace\_active} value
        \item<2-> If not, acts as \funcname{raise\_notrace} with \funcname{RESTORE\_EXN\_HANDLER\_OCAML}
          \begin{itemize}
            \item Set \regname{RSP} to \localname{domain\_state->exn\_handler} value
            \item Restore \localname{domain\_state->exn\_handler} from trap
            \item Jump with \funcname{ret} (same effect as using \funcname{pop} and \funcname{jmp})
          \end{itemize}
        \item<3-> Otherwise, switches to the C stack to be able to execute C code
        \item<4-> Calls \funcname{caml\_stash\_backtrace}, a C function provided by the runtime\ignorespaces
          \onslide<6->{, with
            \begin{itemize}
              \item \funcarg{exn}: exception value
              \item \funcarg{pc}: code address of the raising point
              \item \funcarg{sp}: stack address of the raising function
              \item \funcarg{trapsp}: stack address of the next handler
            \end{itemize}
          }
      \end{itemize}
      \bigskip
      \mintocaml[firstline=9,lastline=13,highlightlines=12]{../src/backtrace/backtrace.ml}
    \end{column}
    \begin{column}{0.5\textwidth}
      \raggedleft
      \onslide*<1,3-4>{
        \mintc[firstline=190,lastline=192]{../ocaml/runtime/amd64.S}
        \mintc[firstline=234,lastline=237]{../ocaml/runtime/amd64.S}
      }
      \onslide*<2>{
        \mintc[firstline=190,lastline=192,highlightlines={191-192}]{../ocaml/runtime/amd64.S}
        \mintc[firstline=234,lastline=237,highlightlines={235-237}]{../ocaml/runtime/amd64.S}
      }
      \onslide*<1>{\listCamlRaiseExn[highlightlines=705]{}}%
      \onslide*<2>{\listCamlRaiseExn[highlightlines={707-708}]{}}%
      \onslide*<3>{\listCamlRaiseExn[highlightlines={712-714}]{}}%
      \onslide*<4>{\listCamlRaiseExn[highlightlines={715-720}]{}}%
      \onslide*<5>{\providecommand\step{1}\input{diagrams/backtrace/backtrace.tex}}%
      \onslide*<6>{\providecommand\step{2}\input{diagrams/backtrace/backtrace.tex}}%
    \end{column}
  \end{columns}
\end{frame}

\newcommand\listStashBacktrace[1][]{
  \listc[firstline=91,lastline=120,#1]{../ocaml/runtime/backtrace\_nat.c}{runtime/backtrace\_nat.c:91}}

\begin{frame}{Backtraces}{Introducing \funcname{caml\_stash\_backtrace}}
  \begin{columns}[c]
    \begin{column}{0.5\textwidth}
      \minipage[c][0.66\textheight][s]{\columnwidth}
      \begin{itemize}
        \item<1-> A C function provided by the runtime (specific to native code)
        \item<2-> It scans the stack, between the stack pointer at the raising point up to the next trap, in order to collect the names of the detected function frames
          \onslide*<2>{
            \begin{itemize}
              \item And more information, like line number, column number...
            \end{itemize}
          }
        \item<3-> It uses the same technique and information as the Garbage Collector (GC) uses to scan the values live on the stack
          \onslide*<4>{
            \begin{itemize}
              \item \typename{frame\_descr} are at the core of stack unwinding
            \end{itemize}
          }
      \end{itemize}
      \vfill
      \mintocaml[firstline=9,lastline=13,highlightlines=12]{../src/backtrace/backtrace.ml}
      \endminipage
    \end{column}
    \begin{column}{0.5\textwidth}
      \onslide*<1>{\listStashBacktrace[highlightlines=91]{}}
      \onslide*<2>{\listStashBacktrace[highlightlines={91,117-118}]{}}
      \onslide*<3->{\listStashBacktrace[highlightlines={94,106,109-110}]{}}
    \end{column}
  \end{columns}
\end{frame}

\newcommand\listFrameDescr[1][]{
  \listocaml[firstline=111,lastline=124,#1]{../ocaml/asmcomp/emitaux.ml}{asmcomp/emitaux.ml:111}}
\newcommand\listDebugInfo[1][]{
  \listocaml[firstline=38,lastline=49,#1]{../ocaml/lambda/debuginfo.mli}{lambda/debuginfo.mli:38}}

\begin{frame}{Backtraces}{\typename{frame\_descr} and \typename{frame\_debuginfo} in a nutshell}
  \begin{columns}[c]
    \begin{column}{0.5\textwidth}
      \begin{itemize}
        \item<1-> For every return address in an OCaml program, there is a associated \typename{frame\_descr}
        \item<2-> A \typename{frame\_descr} specifies
        \begin{itemize}
          \item<2-> The size of the function stack frame
          \item<3-> Where live values are on the stack (so that the GC can mark them as live and prevent from being swept)
        \end{itemize}
        \item<4-> When compiled with debug information, \typename{frame\_descr} will be linked to a \typename{frame\_debuginfo}
        \begin{itemize}
          \item<5-> Contains information about the position in the source code (file name, line and column number)
        \end{itemize}
      \end{itemize}
    \end{column}
    \begin{column}{0.5\textwidth}
        \onslide*<1>{\listFrameDescr[highlightlines={118-122}]{}}
        \onslide*<2>{\listFrameDescr[highlightlines={120}]{}}
        \onslide*<3>{\listFrameDescr[highlightlines={121}]{}}
        \onslide*<4->{\listFrameDescr[highlightlines={122}]{}}
        \medskip
        \onslide*<1-3>{\listDebugInfo{}}
        \onslide*<4>{\listDebugInfo[highlightlines={38,49}]{}}
        \onslide*<5>{\listDebugInfo[highlightlines={39-46}]{}}
    \end{column}
  \end{columns}
\end{frame}

\begin{frame}{Backtraces}{Scanning the stack with \typename{frame\_descr}}
  \begin{columns}[c]
    \begin{column}{0.5\textwidth}
      \begin{itemize}
        \item Given the return address of the code that raises the exception by calling \funcname{caml\_raise\_exn} (\funcarg{pc} argument of \funcname{caml\_stash\_backtrace})
        \item And the position of the stack pointer (\funcarg{sp} argument of \funcname{caml\_stash\_backtrace})
        \item The frame size given by \typename{frame\_descr}.\funcarg{frame\_size}, allows to find the end of the previous frame
        \item And the return address of the caller of this function
        \bigskip
        \onslide<3->{
        \item Note that traps (here the reddish \funcname{ExnA}, \funcname{ExnB} and \funcname{ExnC} blocks) belong to the frame that installed them, and so the frame size changes when traps are removed
        }
      \end{itemize}
    \end{column}
    \begin{column}{0.5\textwidth}
      \raggedleft
      \onslide*<1>{\providecommand\step{2}\input{diagrams/backtrace/backtrace.tex}}%
      \onslide*<2>{\providecommand\step{1}\input{diagrams/backtrace/scanstack.tex}}%
      \onslide*<3>{\providecommand\step{2}\input{diagrams/backtrace/scanstack.tex}}%
      \onslide*<4>{\providecommand\step{3}\input{diagrams/backtrace/scanstack.tex}}%
      \onslide*<5>{\providecommand\step{4}\input{diagrams/backtrace/scanstack.tex}}%
      \onslide*<6>{\providecommand\step{5}\input{diagrams/backtrace/scanstack.tex}}%
    \end{column}
  \end{columns}
\end{frame}

% Duplicate of previous-previous slide
\begin{frame}{Backtraces}{Continuing on \funcname{caml\_stash\_backtrace}}
  \begin{columns}[c]
    \begin{column}{0.5\textwidth}
      \minipage[c][0.75\textheight][s]{\columnwidth}
      \begin{itemize}
          \color{gray}
        \item A C function provided by the runtime (specific to native code)
        \item It scans the stack, between the stack pointer at the raising point up to the next trap, in order to collect the names of the detected function frames
        \item It uses the same technique and information as the Garbage Collector (GC) uses to scan the values live on the stack
      \end{itemize}
      \begin{itemize}
        \item For each function frame detected, store a pointer to the \typename{frame\_descr} in the \localname{domain\_state->backtrace\_buffer} array, effectively constructing the backtrace
      \end{itemize}
      \onslide<2->{
        \begin{itemize}
            \smallskip
          \item Constructed backtrace:
            \funcname{definitely\_raise},
            \onslide<3->{\funcname{probably\_raise}}
        \end{itemize}
      }
      \vfill
      \mintocaml[firstline=9,lastline=13,highlightlines=12]{../src/backtrace/backtrace.ml}
      \endminipage
    \end{column}
    \begin{column}{0.5\textwidth}
      \raggedleft
      \onslide*<1>{\listStashBacktrace[highlightlines={114-115}]{}}
      \onslide*<2>{\providecommand\step{3}\input{diagrams/backtrace/backtrace.tex}}%
      \onslide*<3>{\providecommand\step{4}\input{diagrams/backtrace/backtrace.tex}}%
    \end{column}
  \end{columns}
\end{frame}

\begin{frame}{Backtraces}{Back to \funcname{caml\_raise\_exn}}
  \begin{columns}[c]
    \begin{column}{0.5\textwidth}
      \minipage[c][0.66\textheight][s]{\columnwidth}
      \begin{itemize}\color{gray}
        \item Checks wheteher exceptions backtrace should be recorded, by checking the \funcarg{domain\_state->backtrace\_active} value
        \item Switches to the C stack to be able to execute C code
        \item Calls \funcname{caml\_stash\_backtrace}, to collect the backtrace up to the next trap
      \end{itemize}
      \onslide*<2->{
        \begin{itemize}
          \item<2-> Executes the exception handler in \funcname{probably\_raise} with \funcname{RESTORE\_EXN\_HANDLER\_OCAML}
            \begin{itemize}
              \item Set \regname{RSP} to \localname{domain\_state->exn\_handler} value
              \item Restore \localname{domain\_state->exn\_handler} from trap
              \item Jump to the handler code with \funcname{ret}
            \end{itemize}
        \end{itemize}
      }
      \vfill
      \onslide*<1>{\mintocaml[firstline=9,lastline=13,highlightlines=12]{../src/backtrace/backtrace.ml}}
      \onslide*<2->{\mintocaml[firstline=15,lastline=21,highlightlines={19-21}]{../src/backtrace/backtrace.ml}}
      \endminipage
    \end{column}
    \begin{column}{0.5\textwidth}
      \centering
      \onslide*<1>{
        \mintc[firstline=190,lastline=192]{../ocaml/runtime/amd64.S}
        \mintc[firstline=234,lastline=237]{../ocaml/runtime/amd64.S}
      }
      \onslide*<2>{
        \mintc[firstline=190,lastline=192,highlightlines={191-192}]{../ocaml/runtime/amd64.S}
        \mintc[firstline=234,lastline=237,highlightlines={235-237}]{../ocaml/runtime/amd64.S}
      }
      \onslide*<1>{\listCamlRaiseExn[highlightlines=720]{}}
      \onslide*<2>{\listCamlRaiseExn[highlightlines=722]{}}
      \onslide*<3->{\providecommand\step{5}\input{diagrams/backtrace/backtrace.tex}}
    \end{column}
  \end{columns}
\end{frame}

\begin{frame}{Backtraces}{\funcname{caml\_probably\_raise}'s handler doesn't catch \typename{ExnC}}
  \begin{columns}[c]
    \begin{column}{0.45\textwidth}
      \minipage[c][0.75\textheight][s]{\columnwidth}
      \begin{itemize}
        \item<1-> \funcname{match \dots with}\footnotemark pattern matching can also match exceptions
        \item<1-> The CMM of \funcname{probably\_raise} shows that this \funcname{match \dots with} is implemented with a pattern matching and a \funcname{try \dots with}
        \item<1-> But this one only matches on \typename{ExnA} exception
        \item<2-> Since the pattern matching fails to catch the exception, \funcname{reraise} is called with our current \typename{ExnC} exception
        \item<3-> Disassembly shows that \funcname{reraise} is actually a call to \funcname{caml\_reraise\_exn}, which is also an assembly function provided by the runtime
      \end{itemize}
      \vfill
      \mintocaml[firstline=15,lastline=21,highlightlines={18-21}]{../src/backtrace/backtrace.ml}
      \endminipage
    \end{column}
    \begin{column}{0.55\textwidth}
      \centering
      \onslide*<1>{\mintcmm[highlightlines={10-18}]{../src/backtrace/camlBacktrace\_\_probably\_raise\_351.cmm}}
      \onslide*<2>{\listcmm[highlightlines=18]{../src/backtrace/camlBacktrace\_\_probably\_raise_351.cmm}{CMM of probably\_raise}}
      \onslide*<3>{\mintobjdump[firstline=5,lastline=35,highlightlines=31]{../src/backtrace/camlBacktrace\_\_probably\_raise_351.objdump}}
    \end{column}
  \end{columns}
  \only<1>{\footnotetext[1]{https://blog.janestreet.com/pattern-matching-and-exception-handling-unite/}}
\end{frame}

\newcommand\listCamlReraiseExn[1][]{
  \listasm[firstline=727,lastline=734,#1]{../ocaml/runtime/amd64.S}{runtime/amd64.S:727}}

\begin{frame}{Backtraces}{Introducing \funcname{caml\_reraise\_exn}}
  \begin{columns}[c]
    \begin{column}{0.5\textwidth}
      \minipage[c][0.66\textheight][s]{\columnwidth}
      \begin{itemize}
        \item<1-> The exception have to be reraised to the next handler
        \item<2-> First, checks if the backtrace should be collected with \funcarg{domain\_state->backtrace\_active}
        \only<3>{
        \begin{itemize}
            \item If not, behaves like a \funcname{raise\_notrace} with \funcname{RESTORE\_EXN\_HANDLER\_OCAML}
        \end{itemize}
        }
        \item<4-> Jumps into \funcname{caml\_raise\_exn}
        \item<5-> Calls \funcname{caml\_stash\_backtrace} again, to scan the next part of the stack: from the trap just pop-ed to the next trap
      \end{itemize}
      \vfill
      \mintocaml[firstline=15,lastline=21,highlightlines={18-21}]{../src/backtrace/backtrace.ml}
      \endminipage
    \end{column}
    \begin{column}{0.5\textwidth}
      \raggedleft
      \onslide*<1>{\listCamlReraiseExn{}}
      \onslide*<2>{\listCamlReraiseExn[highlightlines=729]{}}
      \onslide*<3>{
        \mintc[firstline=190,lastline=192,highlightlines={191-192}]{../ocaml/runtime/amd64.S}
        \mintc[firstline=234,lastline=237,highlightlines={235-237}]{../ocaml/runtime/amd64.S}
        \medskip
        \listCamlReraiseExn[highlightlines=731]{}
      }
      \onslide*<4-5>{
        % TODO refactor with \listCamlReraiseExn
        \mintasm[firstline=727,lastline=734,highlightlines=730]{../ocaml/runtime/amd64.S}
        \medskip
      }
      \onslide*<4>{\listCamlRaiseExn[highlightlines=711]{}}
      \onslide*<5>{\listCamlRaiseExn[highlightlines=720]{}}
    \end{column}
  \end{columns}
\end{frame}

\begin{frame}{Backtraces}{Backtrace collection continues}
  \begin{columns}[c]
    \begin{column}{0.55\textwidth}
      \minipage[c][0.75\textheight][s]{\columnwidth}
      \begin{itemize}
        \item<1-> \funcname{caml\_stash\_backtrace} scans the older part of the stack, up to the next trap
        \item<6-> Controls jumps to the nested exception handler in \funcname{maybe\_raise}, which matches only on \typename{ExnB}
        \item<7-> \funcname{caml\_reraise\_exn} is called again, backtrace collection resumes with \funcname{caml\_stash\_backtrace}
      \end{itemize}
      \medskip
      \begin{itemize}
        \item<2-> Constructed backtrace:
        \begin{itemize}
          \item <2-> \funcname{definitely\_raise}
          \item <2-> \funcname{probably\_raise}
          \item <3-> \funcname{probably\_raise} (re-raise)
          \item <4-> \funcname{maybe\_raise}
          \item <5-> \funcname{main}
          \item <8-> \funcname{main} (re-raise)
        \end{itemize}
      \end{itemize}
      \vfill
      \onslide*<1-5>{\mintocaml[firstline=15,lastline=21]{../src/backtrace/backtrace.ml}}
      \onslide*<6>{\mintocaml[firstline=28,lastline=34,highlightlines=31]{../src/backtrace/backtrace.ml}}
      \onslide*<7->{\mintocaml[firstline=28,lastline=34,highlightlines=33]{../src/backtrace/backtrace.ml}}
      \endminipage
    \end{column}
    \begin{column}{0.45\textwidth}
      \raggedleft
      \onslide*<1>{\providecommand\step{6}\input{diagrams/backtrace/backtrace.tex}}%
      \onslide*<2>{\providecommand\step{6}\input{diagrams/backtrace/backtrace.tex}}%
      \onslide*<3>{\providecommand\step{7}\input{diagrams/backtrace/backtrace.tex}}%
      \onslide*<4>{\providecommand\step{8}\input{diagrams/backtrace/backtrace.tex}}%
      \onslide*<5>{\providecommand\step{9}\input{diagrams/backtrace/backtrace.tex}}%
      \onslide*<6>{\providecommand\step{10}\input{diagrams/backtrace/backtrace.tex}}%
      \onslide*<7>{\providecommand\step{11}\input{diagrams/backtrace/backtrace.tex}}%
      \onslide*<8>{\providecommand\step{12}\input{diagrams/backtrace/backtrace.tex}}%
      \onslide*<9>{\providecommand\step{13}\input{diagrams/backtrace/backtrace.tex}}%
    \end{column}
  \end{columns}
\end{frame}

\begin{frame}{Backtraces}{Printing the backtrace}
  \begin{columns}[c]
    \begin{column}{0.45\textwidth}
      \minipage[c][0.66\textheight][s]{\columnwidth}
      \begin{itemize}
        \item<1-> The exception backtrace can now be printed to \funcarg{stdout} with \funcname{Printexc.print_backtrace}
        \item<2-> First the exception backtrace is retrieved into an OCaml array with \funcname{get_raw_backtrace }
        \item<3-> Which is implemented as the \funcname{caml_get_exception_raw_backtrace} C function in the runtime
        \item<4-> And finally printed with \funcname{print_exception_backtrace}
      \end{itemize}
      \vfill
      \mintocaml[firstline=28,lastline=34,highlightlines=33]{../src/backtrace/backtrace.ml}
      \endminipage
    \end{column}
    \begin{column}{0.55\textwidth}
      \onslide*<1,2>{
        \mintocaml[firstline=100,lastline=101]{../ocaml/stdlib/printexc.ml}
      }
      \only<1>{
        \listocaml[firstline=170,lastline=172]{../ocaml/stdlib/printexc.ml}{stdlib/printexc.ml:170}
      }
      \only<2>{
        \listocaml[firstline=170,lastline=172,highlightlines=172]{../ocaml/stdlib/printexc.ml}{stdlib/printexc.ml:170}
      }
      \only<3>{
        \mintc[firstline=154,lastline=156]{../ocaml/runtime/backtrace.c}
        \listc[firstline=171,lastline=192,highlightlines={184-187}]{../ocaml/runtime/backtrace.c}{runtime/backtrace.c:154}
      }
      \only<4>{
        \listocaml[firstline=155,lastline=165]{../ocaml/stdlib/printexc.ml}{stdlib/printexc.ml:55}
      }
    \end{column}
  \end{columns}
\end{frame}


\subsection{The multiple kinds of raise}
\frameSubsection{}{}

\begin{frame}{Backtraces}{When backtraces are not needed}
  \begin{columns}[c]
    \begin{column}{0.45\textwidth}
      \minipage[c][0.66\textheight][s]{\columnwidth}
      \begin{itemize}
        \item<1-> What if the backtrace for an exception is never needed?
        \item<2-> It's possible to raise the exception explicitly with \funcname{raise\_notrace} to make raising as cheap as possible
        \item<3-> An example of use in the \funcname{dynlink} library of the OCaml compiler
      \end{itemize}
      \vfill
      \onslide<3->{
      \listocaml[firstline=205,lastline=209,highlightlines=207]{../ocaml/otherlibs/dynlink/dynlink\_compilerlibs/misc.ml}{otherlibs/dynlink/dynlink\_compilerlibs/misc.ml:205}
      }
      \endminipage
    \end{column}
    \begin{column}{0.55\textwidth}
    \only<4>{
      \mintcmm[highlightlines=3]{../src/notrace/camlNotrace\_\_fun_347.cmm}
      \listcmm{../src/notrace/camlNotrace\_\_all\_somes\_327.cmm}{all\_somes}
    }
    \onslide*<5->{
      \listobjdump[highlightlines={6-9}]{../src/notrace/camlNotrace\_\_fun_347.objdump}{anonymous function from \funcname{all\_somes}}
    }
    \end{column}
  \end{columns}
\end{frame}

\begin{frame}{Backtraces}{Summary table of the multiple kinds of raise}
  \begin{itemize}
    \item When debugging information is enabled and if the backtrace of an exception isn't needed, it's possible to raise with \funcname{raise\_notrace} for performance
    \item When debugging information is \emph{not} enabled, the compiler optimises \funcname{raise} calls to inline assembly (same effect as using \funcname{raise\_notrace})
  \end{itemize}
  \bigskip
  \centering
  \begin{tabular}{| l | c | c | c | c |}
    \hline
    \parbox{10em}{Debug information \\ (\funcarg{-g} flag)}  & \multicolumn{2}{|c|}{Disabled}                                     & \multicolumn{2}{|c|}{Enabled} \\
    \hline
    Raise kind                                               & \funcname{raise} & \funcname{raise\_notrace}                       & \funcname{raise} & \funcname{raise\_notrace} \\
    \hline
    Compilation                                              & \parbox{8em}{inline assembly\\ (optimization)} & inline assembly   & \funcname{caml\_raise\_exn} & inline assembly \\
    \hline
  \end{tabular}
\end{frame}


%
% Takeaway #5
%
\subsection*{Takeaway \#5}
\frameSubsectionTakeaway{}
\againframe<5>{takeaway}
}%
    \end{column}
  \end{columns}
\end{frame}

\begin{frame}{Backtraces}{Back to \funcname{caml\_raise\_exn}}
  \begin{columns}[c]
    \begin{column}{0.5\textwidth}
      \minipage[c][0.66\textheight][s]{\columnwidth}
      \begin{itemize}\color{gray}
        \item Checks wheteher exceptions backtrace should be recorded, by checking the \funcarg{domain\_state->backtrace\_active} value
        \item Switches to the C stack to be able to execute C code
        \item Calls \funcname{caml\_stash\_backtrace}, to collect the backtrace up to the next trap
      \end{itemize}
      \onslide*<2->{
        \begin{itemize}
          \item<2-> Executes the exception handler in \funcname{probably\_raise} with \funcname{RESTORE\_EXN\_HANDLER\_OCAML}
            \begin{itemize}
              \item Set \regname{RSP} to \localname{domain\_state->exn\_handler} value
              \item Restore \localname{domain\_state->exn\_handler} from trap
              \item Jump to the handler code with \funcname{ret}
            \end{itemize}
        \end{itemize}
      }
      \vfill
      \onslide*<1>{\mintocaml[firstline=9,lastline=13,highlightlines=12]{../src/backtrace/backtrace.ml}}
      \onslide*<2->{\mintocaml[firstline=15,lastline=21,highlightlines={19-21}]{../src/backtrace/backtrace.ml}}
      \endminipage
    \end{column}
    \begin{column}{0.5\textwidth}
      \centering
      \onslide*<1>{
        \mintc[firstline=190,lastline=192]{../ocaml/runtime/amd64.S}
        \mintc[firstline=234,lastline=237]{../ocaml/runtime/amd64.S}
      }
      \onslide*<2>{
        \mintc[firstline=190,lastline=192,highlightlines={191-192}]{../ocaml/runtime/amd64.S}
        \mintc[firstline=234,lastline=237,highlightlines={235-237}]{../ocaml/runtime/amd64.S}
      }
      \onslide*<1>{\listCamlRaiseExn[highlightlines=720]{}}
      \onslide*<2>{\listCamlRaiseExn[highlightlines=722]{}}
      \onslide*<3->{\providecommand\step{5}%
% Backtraces
%
\subsection{One advantage of raising with debugging information}
\frameSubsection{}{}

\begin{frame}{Backtraces}{Debugging information \& source code}
  \begin{columns}[c]
    \begin{column}{0.5\textwidth}
      \begin{itemize}
        \item Raising an exception is fun and games, but how do we know where the exception originated? $\rightarrow$ With the backtrace associated with the exception!
        \item In order to get a backtrace from the point where an exception was raised, it's necessary to compile with debugging information enabled (\funcarg{-g} flag for the compiler)
        \item Same example as in the `Nested handlers' section
      \end{itemize}
    \end{column}
    \begin{column}{0.5\textwidth}
      \listocaml[firstline=9,lastline=34]{../src/backtrace/backtrace.ml}{backtrace/backtrace.ml}
    \end{column}
  \end{columns}
\end{frame}

\begin{frame}{Backtraces}{CMM and disassembly of \funcname{definitely\_raise}}
  \begin{columns}[c]
    \begin{column}{0.5\textwidth}
      \minipage[c][0.66\textheight][s]{\columnwidth}
      \begin{itemize}
        \item<1-> An effect of compiling with debugging information (\funcarg{-g}): the CMM no longer uses \funcname{raise\_notrace} to raise the exception but \funcname{raise} instead
        \item<2-> Disassembly shows that CMM \funcname{raise} is actually a call to \funcname{caml\_raise\_exn}, a function in assembler provided by the runtime
      \end{itemize}
      \vfill
      \mintocaml[firstline=9,lastline=13,highlightlines=12]{../src/backtrace/backtrace.ml}
      \endminipage
    \end{column}
    \begin{column}{0.5\textwidth}
      \onslide*<1>{
        \mintcmm[highlightlines=5]{../src/backtrace/camlBacktrace\_\_definitely\_raise\_347.cmm}
      }
      \onslide*<2>{
        \mintobjdump[firstline=2,highlightlines=14]{../src/backtrace/camlBacktrace\_\_definitely\_raise\_347.objdump}
      }
    \end{column}
  \end{columns}
\end{frame}

\begin{frame}{Backtraces}{Introducing \funcname{caml\_raise\_exn}}
  \begin{columns}[c]
    \begin{column}{0.5\textwidth}
      \minipage[c][0.66\textheight][s]{\columnwidth}
      \onslide*<1>{
        \begin{itemize}
          \item Raising the exception is performed using \funcname{caml\_raise\_exn} which is
            \begin{itemize}
              \item An assembly function provided by the runtime
              \item Architecture specific
            \end{itemize}
          \item Let's see what it does differently from \funcname{raise\_notrace}\dots
          \item We are about to raise \typename{ExnC} with \funcname{caml\_raise\_exn} from \funcname{definitely\_raise}
        \end{itemize}
      }
      \onslide*<2>{
        \mintobjdump[firstline=2,highlightlines=14]{../src/backtrace/camlBacktrace\_\_definitely\_raise\_347.objdump}
      }
      \vfill
      \mintocaml[firstline=9,lastline=13,highlightlines=12]{../src/backtrace/backtrace.ml}
      \endminipage
    \end{column}
    \begin{column}{0.5\textwidth}
      \centering
      \providecommand\step{1}\input{diagrams/backtrace/backtrace.tex}
    \end{column}
  \end{columns}
\end{frame}

\begin{frame}{Backtraces}{But before that, we need more fields from \typename{caml\_domain\_state}}
  \begin{columns}
    \begin{column}{0.5\textwidth}
      \begin{itemize}
        \item Remember \typename{caml\_domain\_state} the per-domain runtime state?
        \item Some other members are of interest to us now\dots
        \item \localname{backtrace\_active} is used to know if the runtime should collect backtraces
        \item \localname{backtrace\_active} can be set by
          \begin{itemize}
            \item The \funcarg{b} flag in the \funcname{OCAMLRUNPARAM} environment variable
            \item \mintinline{ocaml}{Printexc.record_backtrace : bool -> unit} \footnotemark
          \end{itemize}
      \end{itemize}
    \end{column}
    \begin{column}{0.5\textwidth}
      \begin{listing}
        \mintc[highlightlines={9-11}]{src/ocaml/domain\_state\_backtrace.h}
        \caption{runtime/caml/domain\_state.h}
      \end{listing}
    \end{column}
  \end{columns}
  \footnotetext[1]{https://v2.ocaml.org/api/Printexc.html}
\end{frame}

\newcommand\listCamlRaiseExn[1][]{
  \listasm[firstline=702,lastline=725,#1]{../ocaml/runtime/amd64.S}{runtime/amd64.S:702}}

\begin{frame}{Backtraces}{Walk through \funcname{caml\_raise\_exn}}
  \begin{columns}[T]
    \begin{column}{0.5\textwidth}
      \begin{itemize}
        \item<1-> First checks whether exceptions backtrace should be recorded
        \item<1-> By checking the \funcarg{domain\_state->backtrace\_active} value
        \item<2-> If not, acts as \funcname{raise\_notrace} with \funcname{RESTORE\_EXN\_HANDLER\_OCAML}
          \begin{itemize}
            \item Set \regname{RSP} to \localname{domain\_state->exn\_handler} value
            \item Restore \localname{domain\_state->exn\_handler} from trap
            \item Jump with \funcname{ret} (same effect as using \funcname{pop} and \funcname{jmp})
          \end{itemize}
        \item<3-> Otherwise, switches to the C stack to be able to execute C code
        \item<4-> Calls \funcname{caml\_stash\_backtrace}, a C function provided by the runtime\ignorespaces
          \onslide<6->{, with
            \begin{itemize}
              \item \funcarg{exn}: exception value
              \item \funcarg{pc}: code address of the raising point
              \item \funcarg{sp}: stack address of the raising function
              \item \funcarg{trapsp}: stack address of the next handler
            \end{itemize}
          }
      \end{itemize}
      \bigskip
      \mintocaml[firstline=9,lastline=13,highlightlines=12]{../src/backtrace/backtrace.ml}
    \end{column}
    \begin{column}{0.5\textwidth}
      \raggedleft
      \onslide*<1,3-4>{
        \mintc[firstline=190,lastline=192]{../ocaml/runtime/amd64.S}
        \mintc[firstline=234,lastline=237]{../ocaml/runtime/amd64.S}
      }
      \onslide*<2>{
        \mintc[firstline=190,lastline=192,highlightlines={191-192}]{../ocaml/runtime/amd64.S}
        \mintc[firstline=234,lastline=237,highlightlines={235-237}]{../ocaml/runtime/amd64.S}
      }
      \onslide*<1>{\listCamlRaiseExn[highlightlines=705]{}}%
      \onslide*<2>{\listCamlRaiseExn[highlightlines={707-708}]{}}%
      \onslide*<3>{\listCamlRaiseExn[highlightlines={712-714}]{}}%
      \onslide*<4>{\listCamlRaiseExn[highlightlines={715-720}]{}}%
      \onslide*<5>{\providecommand\step{1}\input{diagrams/backtrace/backtrace.tex}}%
      \onslide*<6>{\providecommand\step{2}\input{diagrams/backtrace/backtrace.tex}}%
    \end{column}
  \end{columns}
\end{frame}

\newcommand\listStashBacktrace[1][]{
  \listc[firstline=91,lastline=120,#1]{../ocaml/runtime/backtrace\_nat.c}{runtime/backtrace\_nat.c:91}}

\begin{frame}{Backtraces}{Introducing \funcname{caml\_stash\_backtrace}}
  \begin{columns}[c]
    \begin{column}{0.5\textwidth}
      \minipage[c][0.66\textheight][s]{\columnwidth}
      \begin{itemize}
        \item<1-> A C function provided by the runtime (specific to native code)
        \item<2-> It scans the stack, between the stack pointer at the raising point up to the next trap, in order to collect the names of the detected function frames
          \onslide*<2>{
            \begin{itemize}
              \item And more information, like line number, column number...
            \end{itemize}
          }
        \item<3-> It uses the same technique and information as the Garbage Collector (GC) uses to scan the values live on the stack
          \onslide*<4>{
            \begin{itemize}
              \item \typename{frame\_descr} are at the core of stack unwinding
            \end{itemize}
          }
      \end{itemize}
      \vfill
      \mintocaml[firstline=9,lastline=13,highlightlines=12]{../src/backtrace/backtrace.ml}
      \endminipage
    \end{column}
    \begin{column}{0.5\textwidth}
      \onslide*<1>{\listStashBacktrace[highlightlines=91]{}}
      \onslide*<2>{\listStashBacktrace[highlightlines={91,117-118}]{}}
      \onslide*<3->{\listStashBacktrace[highlightlines={94,106,109-110}]{}}
    \end{column}
  \end{columns}
\end{frame}

\newcommand\listFrameDescr[1][]{
  \listocaml[firstline=111,lastline=124,#1]{../ocaml/asmcomp/emitaux.ml}{asmcomp/emitaux.ml:111}}
\newcommand\listDebugInfo[1][]{
  \listocaml[firstline=38,lastline=49,#1]{../ocaml/lambda/debuginfo.mli}{lambda/debuginfo.mli:38}}

\begin{frame}{Backtraces}{\typename{frame\_descr} and \typename{frame\_debuginfo} in a nutshell}
  \begin{columns}[c]
    \begin{column}{0.5\textwidth}
      \begin{itemize}
        \item<1-> For every return address in an OCaml program, there is a associated \typename{frame\_descr}
        \item<2-> A \typename{frame\_descr} specifies
        \begin{itemize}
          \item<2-> The size of the function stack frame
          \item<3-> Where live values are on the stack (so that the GC can mark them as live and prevent from being swept)
        \end{itemize}
        \item<4-> When compiled with debug information, \typename{frame\_descr} will be linked to a \typename{frame\_debuginfo}
        \begin{itemize}
          \item<5-> Contains information about the position in the source code (file name, line and column number)
        \end{itemize}
      \end{itemize}
    \end{column}
    \begin{column}{0.5\textwidth}
        \onslide*<1>{\listFrameDescr[highlightlines={118-122}]{}}
        \onslide*<2>{\listFrameDescr[highlightlines={120}]{}}
        \onslide*<3>{\listFrameDescr[highlightlines={121}]{}}
        \onslide*<4->{\listFrameDescr[highlightlines={122}]{}}
        \medskip
        \onslide*<1-3>{\listDebugInfo{}}
        \onslide*<4>{\listDebugInfo[highlightlines={38,49}]{}}
        \onslide*<5>{\listDebugInfo[highlightlines={39-46}]{}}
    \end{column}
  \end{columns}
\end{frame}

\begin{frame}{Backtraces}{Scanning the stack with \typename{frame\_descr}}
  \begin{columns}[c]
    \begin{column}{0.5\textwidth}
      \begin{itemize}
        \item Given the return address of the code that raises the exception by calling \funcname{caml\_raise\_exn} (\funcarg{pc} argument of \funcname{caml\_stash\_backtrace})
        \item And the position of the stack pointer (\funcarg{sp} argument of \funcname{caml\_stash\_backtrace})
        \item The frame size given by \typename{frame\_descr}.\funcarg{frame\_size}, allows to find the end of the previous frame
        \item And the return address of the caller of this function
        \bigskip
        \onslide<3->{
        \item Note that traps (here the reddish \funcname{ExnA}, \funcname{ExnB} and \funcname{ExnC} blocks) belong to the frame that installed them, and so the frame size changes when traps are removed
        }
      \end{itemize}
    \end{column}
    \begin{column}{0.5\textwidth}
      \raggedleft
      \onslide*<1>{\providecommand\step{2}\input{diagrams/backtrace/backtrace.tex}}%
      \onslide*<2>{\providecommand\step{1}\input{diagrams/backtrace/scanstack.tex}}%
      \onslide*<3>{\providecommand\step{2}\input{diagrams/backtrace/scanstack.tex}}%
      \onslide*<4>{\providecommand\step{3}\input{diagrams/backtrace/scanstack.tex}}%
      \onslide*<5>{\providecommand\step{4}\input{diagrams/backtrace/scanstack.tex}}%
      \onslide*<6>{\providecommand\step{5}\input{diagrams/backtrace/scanstack.tex}}%
    \end{column}
  \end{columns}
\end{frame}

% Duplicate of previous-previous slide
\begin{frame}{Backtraces}{Continuing on \funcname{caml\_stash\_backtrace}}
  \begin{columns}[c]
    \begin{column}{0.5\textwidth}
      \minipage[c][0.75\textheight][s]{\columnwidth}
      \begin{itemize}
          \color{gray}
        \item A C function provided by the runtime (specific to native code)
        \item It scans the stack, between the stack pointer at the raising point up to the next trap, in order to collect the names of the detected function frames
        \item It uses the same technique and information as the Garbage Collector (GC) uses to scan the values live on the stack
      \end{itemize}
      \begin{itemize}
        \item For each function frame detected, store a pointer to the \typename{frame\_descr} in the \localname{domain\_state->backtrace\_buffer} array, effectively constructing the backtrace
      \end{itemize}
      \onslide<2->{
        \begin{itemize}
            \smallskip
          \item Constructed backtrace:
            \funcname{definitely\_raise},
            \onslide<3->{\funcname{probably\_raise}}
        \end{itemize}
      }
      \vfill
      \mintocaml[firstline=9,lastline=13,highlightlines=12]{../src/backtrace/backtrace.ml}
      \endminipage
    \end{column}
    \begin{column}{0.5\textwidth}
      \raggedleft
      \onslide*<1>{\listStashBacktrace[highlightlines={114-115}]{}}
      \onslide*<2>{\providecommand\step{3}\input{diagrams/backtrace/backtrace.tex}}%
      \onslide*<3>{\providecommand\step{4}\input{diagrams/backtrace/backtrace.tex}}%
    \end{column}
  \end{columns}
\end{frame}

\begin{frame}{Backtraces}{Back to \funcname{caml\_raise\_exn}}
  \begin{columns}[c]
    \begin{column}{0.5\textwidth}
      \minipage[c][0.66\textheight][s]{\columnwidth}
      \begin{itemize}\color{gray}
        \item Checks wheteher exceptions backtrace should be recorded, by checking the \funcarg{domain\_state->backtrace\_active} value
        \item Switches to the C stack to be able to execute C code
        \item Calls \funcname{caml\_stash\_backtrace}, to collect the backtrace up to the next trap
      \end{itemize}
      \onslide*<2->{
        \begin{itemize}
          \item<2-> Executes the exception handler in \funcname{probably\_raise} with \funcname{RESTORE\_EXN\_HANDLER\_OCAML}
            \begin{itemize}
              \item Set \regname{RSP} to \localname{domain\_state->exn\_handler} value
              \item Restore \localname{domain\_state->exn\_handler} from trap
              \item Jump to the handler code with \funcname{ret}
            \end{itemize}
        \end{itemize}
      }
      \vfill
      \onslide*<1>{\mintocaml[firstline=9,lastline=13,highlightlines=12]{../src/backtrace/backtrace.ml}}
      \onslide*<2->{\mintocaml[firstline=15,lastline=21,highlightlines={19-21}]{../src/backtrace/backtrace.ml}}
      \endminipage
    \end{column}
    \begin{column}{0.5\textwidth}
      \centering
      \onslide*<1>{
        \mintc[firstline=190,lastline=192]{../ocaml/runtime/amd64.S}
        \mintc[firstline=234,lastline=237]{../ocaml/runtime/amd64.S}
      }
      \onslide*<2>{
        \mintc[firstline=190,lastline=192,highlightlines={191-192}]{../ocaml/runtime/amd64.S}
        \mintc[firstline=234,lastline=237,highlightlines={235-237}]{../ocaml/runtime/amd64.S}
      }
      \onslide*<1>{\listCamlRaiseExn[highlightlines=720]{}}
      \onslide*<2>{\listCamlRaiseExn[highlightlines=722]{}}
      \onslide*<3->{\providecommand\step{5}\input{diagrams/backtrace/backtrace.tex}}
    \end{column}
  \end{columns}
\end{frame}

\begin{frame}{Backtraces}{\funcname{caml\_probably\_raise}'s handler doesn't catch \typename{ExnC}}
  \begin{columns}[c]
    \begin{column}{0.45\textwidth}
      \minipage[c][0.75\textheight][s]{\columnwidth}
      \begin{itemize}
        \item<1-> \funcname{match \dots with}\footnotemark pattern matching can also match exceptions
        \item<1-> The CMM of \funcname{probably\_raise} shows that this \funcname{match \dots with} is implemented with a pattern matching and a \funcname{try \dots with}
        \item<1-> But this one only matches on \typename{ExnA} exception
        \item<2-> Since the pattern matching fails to catch the exception, \funcname{reraise} is called with our current \typename{ExnC} exception
        \item<3-> Disassembly shows that \funcname{reraise} is actually a call to \funcname{caml\_reraise\_exn}, which is also an assembly function provided by the runtime
      \end{itemize}
      \vfill
      \mintocaml[firstline=15,lastline=21,highlightlines={18-21}]{../src/backtrace/backtrace.ml}
      \endminipage
    \end{column}
    \begin{column}{0.55\textwidth}
      \centering
      \onslide*<1>{\mintcmm[highlightlines={10-18}]{../src/backtrace/camlBacktrace\_\_probably\_raise\_351.cmm}}
      \onslide*<2>{\listcmm[highlightlines=18]{../src/backtrace/camlBacktrace\_\_probably\_raise_351.cmm}{CMM of probably\_raise}}
      \onslide*<3>{\mintobjdump[firstline=5,lastline=35,highlightlines=31]{../src/backtrace/camlBacktrace\_\_probably\_raise_351.objdump}}
    \end{column}
  \end{columns}
  \only<1>{\footnotetext[1]{https://blog.janestreet.com/pattern-matching-and-exception-handling-unite/}}
\end{frame}

\newcommand\listCamlReraiseExn[1][]{
  \listasm[firstline=727,lastline=734,#1]{../ocaml/runtime/amd64.S}{runtime/amd64.S:727}}

\begin{frame}{Backtraces}{Introducing \funcname{caml\_reraise\_exn}}
  \begin{columns}[c]
    \begin{column}{0.5\textwidth}
      \minipage[c][0.66\textheight][s]{\columnwidth}
      \begin{itemize}
        \item<1-> The exception have to be reraised to the next handler
        \item<2-> First, checks if the backtrace should be collected with \funcarg{domain\_state->backtrace\_active}
        \only<3>{
        \begin{itemize}
            \item If not, behaves like a \funcname{raise\_notrace} with \funcname{RESTORE\_EXN\_HANDLER\_OCAML}
        \end{itemize}
        }
        \item<4-> Jumps into \funcname{caml\_raise\_exn}
        \item<5-> Calls \funcname{caml\_stash\_backtrace} again, to scan the next part of the stack: from the trap just pop-ed to the next trap
      \end{itemize}
      \vfill
      \mintocaml[firstline=15,lastline=21,highlightlines={18-21}]{../src/backtrace/backtrace.ml}
      \endminipage
    \end{column}
    \begin{column}{0.5\textwidth}
      \raggedleft
      \onslide*<1>{\listCamlReraiseExn{}}
      \onslide*<2>{\listCamlReraiseExn[highlightlines=729]{}}
      \onslide*<3>{
        \mintc[firstline=190,lastline=192,highlightlines={191-192}]{../ocaml/runtime/amd64.S}
        \mintc[firstline=234,lastline=237,highlightlines={235-237}]{../ocaml/runtime/amd64.S}
        \medskip
        \listCamlReraiseExn[highlightlines=731]{}
      }
      \onslide*<4-5>{
        % TODO refactor with \listCamlReraiseExn
        \mintasm[firstline=727,lastline=734,highlightlines=730]{../ocaml/runtime/amd64.S}
        \medskip
      }
      \onslide*<4>{\listCamlRaiseExn[highlightlines=711]{}}
      \onslide*<5>{\listCamlRaiseExn[highlightlines=720]{}}
    \end{column}
  \end{columns}
\end{frame}

\begin{frame}{Backtraces}{Backtrace collection continues}
  \begin{columns}[c]
    \begin{column}{0.55\textwidth}
      \minipage[c][0.75\textheight][s]{\columnwidth}
      \begin{itemize}
        \item<1-> \funcname{caml\_stash\_backtrace} scans the older part of the stack, up to the next trap
        \item<6-> Controls jumps to the nested exception handler in \funcname{maybe\_raise}, which matches only on \typename{ExnB}
        \item<7-> \funcname{caml\_reraise\_exn} is called again, backtrace collection resumes with \funcname{caml\_stash\_backtrace}
      \end{itemize}
      \medskip
      \begin{itemize}
        \item<2-> Constructed backtrace:
        \begin{itemize}
          \item <2-> \funcname{definitely\_raise}
          \item <2-> \funcname{probably\_raise}
          \item <3-> \funcname{probably\_raise} (re-raise)
          \item <4-> \funcname{maybe\_raise}
          \item <5-> \funcname{main}
          \item <8-> \funcname{main} (re-raise)
        \end{itemize}
      \end{itemize}
      \vfill
      \onslide*<1-5>{\mintocaml[firstline=15,lastline=21]{../src/backtrace/backtrace.ml}}
      \onslide*<6>{\mintocaml[firstline=28,lastline=34,highlightlines=31]{../src/backtrace/backtrace.ml}}
      \onslide*<7->{\mintocaml[firstline=28,lastline=34,highlightlines=33]{../src/backtrace/backtrace.ml}}
      \endminipage
    \end{column}
    \begin{column}{0.45\textwidth}
      \raggedleft
      \onslide*<1>{\providecommand\step{6}\input{diagrams/backtrace/backtrace.tex}}%
      \onslide*<2>{\providecommand\step{6}\input{diagrams/backtrace/backtrace.tex}}%
      \onslide*<3>{\providecommand\step{7}\input{diagrams/backtrace/backtrace.tex}}%
      \onslide*<4>{\providecommand\step{8}\input{diagrams/backtrace/backtrace.tex}}%
      \onslide*<5>{\providecommand\step{9}\input{diagrams/backtrace/backtrace.tex}}%
      \onslide*<6>{\providecommand\step{10}\input{diagrams/backtrace/backtrace.tex}}%
      \onslide*<7>{\providecommand\step{11}\input{diagrams/backtrace/backtrace.tex}}%
      \onslide*<8>{\providecommand\step{12}\input{diagrams/backtrace/backtrace.tex}}%
      \onslide*<9>{\providecommand\step{13}\input{diagrams/backtrace/backtrace.tex}}%
    \end{column}
  \end{columns}
\end{frame}

\begin{frame}{Backtraces}{Printing the backtrace}
  \begin{columns}[c]
    \begin{column}{0.45\textwidth}
      \minipage[c][0.66\textheight][s]{\columnwidth}
      \begin{itemize}
        \item<1-> The exception backtrace can now be printed to \funcarg{stdout} with \funcname{Printexc.print_backtrace}
        \item<2-> First the exception backtrace is retrieved into an OCaml array with \funcname{get_raw_backtrace }
        \item<3-> Which is implemented as the \funcname{caml_get_exception_raw_backtrace} C function in the runtime
        \item<4-> And finally printed with \funcname{print_exception_backtrace}
      \end{itemize}
      \vfill
      \mintocaml[firstline=28,lastline=34,highlightlines=33]{../src/backtrace/backtrace.ml}
      \endminipage
    \end{column}
    \begin{column}{0.55\textwidth}
      \onslide*<1,2>{
        \mintocaml[firstline=100,lastline=101]{../ocaml/stdlib/printexc.ml}
      }
      \only<1>{
        \listocaml[firstline=170,lastline=172]{../ocaml/stdlib/printexc.ml}{stdlib/printexc.ml:170}
      }
      \only<2>{
        \listocaml[firstline=170,lastline=172,highlightlines=172]{../ocaml/stdlib/printexc.ml}{stdlib/printexc.ml:170}
      }
      \only<3>{
        \mintc[firstline=154,lastline=156]{../ocaml/runtime/backtrace.c}
        \listc[firstline=171,lastline=192,highlightlines={184-187}]{../ocaml/runtime/backtrace.c}{runtime/backtrace.c:154}
      }
      \only<4>{
        \listocaml[firstline=155,lastline=165]{../ocaml/stdlib/printexc.ml}{stdlib/printexc.ml:55}
      }
    \end{column}
  \end{columns}
\end{frame}


\subsection{The multiple kinds of raise}
\frameSubsection{}{}

\begin{frame}{Backtraces}{When backtraces are not needed}
  \begin{columns}[c]
    \begin{column}{0.45\textwidth}
      \minipage[c][0.66\textheight][s]{\columnwidth}
      \begin{itemize}
        \item<1-> What if the backtrace for an exception is never needed?
        \item<2-> It's possible to raise the exception explicitly with \funcname{raise\_notrace} to make raising as cheap as possible
        \item<3-> An example of use in the \funcname{dynlink} library of the OCaml compiler
      \end{itemize}
      \vfill
      \onslide<3->{
      \listocaml[firstline=205,lastline=209,highlightlines=207]{../ocaml/otherlibs/dynlink/dynlink\_compilerlibs/misc.ml}{otherlibs/dynlink/dynlink\_compilerlibs/misc.ml:205}
      }
      \endminipage
    \end{column}
    \begin{column}{0.55\textwidth}
    \only<4>{
      \mintcmm[highlightlines=3]{../src/notrace/camlNotrace\_\_fun_347.cmm}
      \listcmm{../src/notrace/camlNotrace\_\_all\_somes\_327.cmm}{all\_somes}
    }
    \onslide*<5->{
      \listobjdump[highlightlines={6-9}]{../src/notrace/camlNotrace\_\_fun_347.objdump}{anonymous function from \funcname{all\_somes}}
    }
    \end{column}
  \end{columns}
\end{frame}

\begin{frame}{Backtraces}{Summary table of the multiple kinds of raise}
  \begin{itemize}
    \item When debugging information is enabled and if the backtrace of an exception isn't needed, it's possible to raise with \funcname{raise\_notrace} for performance
    \item When debugging information is \emph{not} enabled, the compiler optimises \funcname{raise} calls to inline assembly (same effect as using \funcname{raise\_notrace})
  \end{itemize}
  \bigskip
  \centering
  \begin{tabular}{| l | c | c | c | c |}
    \hline
    \parbox{10em}{Debug information \\ (\funcarg{-g} flag)}  & \multicolumn{2}{|c|}{Disabled}                                     & \multicolumn{2}{|c|}{Enabled} \\
    \hline
    Raise kind                                               & \funcname{raise} & \funcname{raise\_notrace}                       & \funcname{raise} & \funcname{raise\_notrace} \\
    \hline
    Compilation                                              & \parbox{8em}{inline assembly\\ (optimization)} & inline assembly   & \funcname{caml\_raise\_exn} & inline assembly \\
    \hline
  \end{tabular}
\end{frame}


%
% Takeaway #5
%
\subsection*{Takeaway \#5}
\frameSubsectionTakeaway{}
\againframe<5>{takeaway}
}
    \end{column}
  \end{columns}
\end{frame}

\begin{frame}{Backtraces}{\funcname{caml\_probably\_raise}'s handler doesn't catch \typename{ExnC}}
  \begin{columns}[c]
    \begin{column}{0.45\textwidth}
      \minipage[c][0.75\textheight][s]{\columnwidth}
      \begin{itemize}
        \item<1-> \funcname{match \dots with}\footnotemark pattern matching can also match exceptions
        \item<1-> The CMM of \funcname{probably\_raise} shows that this \funcname{match \dots with} is implemented with a pattern matching and a \funcname{try \dots with}
        \item<1-> But this one only matches on \typename{ExnA} exception
        \item<2-> Since the pattern matching fails to catch the exception, \funcname{reraise} is called with our current \typename{ExnC} exception
        \item<3-> Disassembly shows that \funcname{reraise} is actually a call to \funcname{caml\_reraise\_exn}, which is also an assembly function provided by the runtime
      \end{itemize}
      \vfill
      \mintocaml[firstline=15,lastline=21,highlightlines={18-21}]{../src/backtrace/backtrace.ml}
      \endminipage
    \end{column}
    \begin{column}{0.55\textwidth}
      \centering
      \onslide*<1>{\mintcmm[highlightlines={10-18}]{../src/backtrace/camlBacktrace\_\_probably\_raise\_351.cmm}}
      \onslide*<2>{\listcmm[highlightlines=18]{../src/backtrace/camlBacktrace\_\_probably\_raise_351.cmm}{CMM of probably\_raise}}
      \onslide*<3>{\mintobjdump[firstline=5,lastline=35,highlightlines=31]{../src/backtrace/camlBacktrace\_\_probably\_raise_351.objdump}}
    \end{column}
  \end{columns}
  \only<1>{\footnotetext[1]{https://blog.janestreet.com/pattern-matching-and-exception-handling-unite/}}
\end{frame}

\newcommand\listCamlReraiseExn[1][]{
  \listasm[firstline=727,lastline=734,#1]{../ocaml/runtime/amd64.S}{runtime/amd64.S:727}}

\begin{frame}{Backtraces}{Introducing \funcname{caml\_reraise\_exn}}
  \begin{columns}[c]
    \begin{column}{0.5\textwidth}
      \minipage[c][0.66\textheight][s]{\columnwidth}
      \begin{itemize}
        \item<1-> The exception have to be reraised to the next handler
        \item<2-> First, checks if the backtrace should be collected with \funcarg{domain\_state->backtrace\_active}
        \only<3>{
        \begin{itemize}
            \item If not, behaves like a \funcname{raise\_notrace} with \funcname{RESTORE\_EXN\_HANDLER\_OCAML}
        \end{itemize}
        }
        \item<4-> Jumps into \funcname{caml\_raise\_exn}
        \item<5-> Calls \funcname{caml\_stash\_backtrace} again, to scan the next part of the stack: from the trap just pop-ed to the next trap
      \end{itemize}
      \vfill
      \mintocaml[firstline=15,lastline=21,highlightlines={18-21}]{../src/backtrace/backtrace.ml}
      \endminipage
    \end{column}
    \begin{column}{0.5\textwidth}
      \raggedleft
      \onslide*<1>{\listCamlReraiseExn{}}
      \onslide*<2>{\listCamlReraiseExn[highlightlines=729]{}}
      \onslide*<3>{
        \mintc[firstline=190,lastline=192,highlightlines={191-192}]{../ocaml/runtime/amd64.S}
        \mintc[firstline=234,lastline=237,highlightlines={235-237}]{../ocaml/runtime/amd64.S}
        \medskip
        \listCamlReraiseExn[highlightlines=731]{}
      }
      \onslide*<4-5>{
        % TODO refactor with \listCamlReraiseExn
        \mintasm[firstline=727,lastline=734,highlightlines=730]{../ocaml/runtime/amd64.S}
        \medskip
      }
      \onslide*<4>{\listCamlRaiseExn[highlightlines=711]{}}
      \onslide*<5>{\listCamlRaiseExn[highlightlines=720]{}}
    \end{column}
  \end{columns}
\end{frame}

\begin{frame}{Backtraces}{Backtrace collection continues}
  \begin{columns}[c]
    \begin{column}{0.55\textwidth}
      \minipage[c][0.75\textheight][s]{\columnwidth}
      \begin{itemize}
        \item<1-> \funcname{caml\_stash\_backtrace} scans the older part of the stack, up to the next trap
        \item<6-> Controls jumps to the nested exception handler in \funcname{maybe\_raise}, which matches only on \typename{ExnB}
        \item<7-> \funcname{caml\_reraise\_exn} is called again, backtrace collection resumes with \funcname{caml\_stash\_backtrace}
      \end{itemize}
      \medskip
      \begin{itemize}
        \item<2-> Constructed backtrace:
        \begin{itemize}
          \item <2-> \funcname{definitely\_raise}
          \item <2-> \funcname{probably\_raise}
          \item <3-> \funcname{probably\_raise} (re-raise)
          \item <4-> \funcname{maybe\_raise}
          \item <5-> \funcname{main}
          \item <8-> \funcname{main} (re-raise)
        \end{itemize}
      \end{itemize}
      \vfill
      \onslide*<1-5>{\mintocaml[firstline=15,lastline=21]{../src/backtrace/backtrace.ml}}
      \onslide*<6>{\mintocaml[firstline=28,lastline=34,highlightlines=31]{../src/backtrace/backtrace.ml}}
      \onslide*<7->{\mintocaml[firstline=28,lastline=34,highlightlines=33]{../src/backtrace/backtrace.ml}}
      \endminipage
    \end{column}
    \begin{column}{0.45\textwidth}
      \raggedleft
      \onslide*<1>{\providecommand\step{6}%
% Backtraces
%
\subsection{One advantage of raising with debugging information}
\frameSubsection{}{}

\begin{frame}{Backtraces}{Debugging information \& source code}
  \begin{columns}[c]
    \begin{column}{0.5\textwidth}
      \begin{itemize}
        \item Raising an exception is fun and games, but how do we know where the exception originated? $\rightarrow$ With the backtrace associated with the exception!
        \item In order to get a backtrace from the point where an exception was raised, it's necessary to compile with debugging information enabled (\funcarg{-g} flag for the compiler)
        \item Same example as in the `Nested handlers' section
      \end{itemize}
    \end{column}
    \begin{column}{0.5\textwidth}
      \listocaml[firstline=9,lastline=34]{../src/backtrace/backtrace.ml}{backtrace/backtrace.ml}
    \end{column}
  \end{columns}
\end{frame}

\begin{frame}{Backtraces}{CMM and disassembly of \funcname{definitely\_raise}}
  \begin{columns}[c]
    \begin{column}{0.5\textwidth}
      \minipage[c][0.66\textheight][s]{\columnwidth}
      \begin{itemize}
        \item<1-> An effect of compiling with debugging information (\funcarg{-g}): the CMM no longer uses \funcname{raise\_notrace} to raise the exception but \funcname{raise} instead
        \item<2-> Disassembly shows that CMM \funcname{raise} is actually a call to \funcname{caml\_raise\_exn}, a function in assembler provided by the runtime
      \end{itemize}
      \vfill
      \mintocaml[firstline=9,lastline=13,highlightlines=12]{../src/backtrace/backtrace.ml}
      \endminipage
    \end{column}
    \begin{column}{0.5\textwidth}
      \onslide*<1>{
        \mintcmm[highlightlines=5]{../src/backtrace/camlBacktrace\_\_definitely\_raise\_347.cmm}
      }
      \onslide*<2>{
        \mintobjdump[firstline=2,highlightlines=14]{../src/backtrace/camlBacktrace\_\_definitely\_raise\_347.objdump}
      }
    \end{column}
  \end{columns}
\end{frame}

\begin{frame}{Backtraces}{Introducing \funcname{caml\_raise\_exn}}
  \begin{columns}[c]
    \begin{column}{0.5\textwidth}
      \minipage[c][0.66\textheight][s]{\columnwidth}
      \onslide*<1>{
        \begin{itemize}
          \item Raising the exception is performed using \funcname{caml\_raise\_exn} which is
            \begin{itemize}
              \item An assembly function provided by the runtime
              \item Architecture specific
            \end{itemize}
          \item Let's see what it does differently from \funcname{raise\_notrace}\dots
          \item We are about to raise \typename{ExnC} with \funcname{caml\_raise\_exn} from \funcname{definitely\_raise}
        \end{itemize}
      }
      \onslide*<2>{
        \mintobjdump[firstline=2,highlightlines=14]{../src/backtrace/camlBacktrace\_\_definitely\_raise\_347.objdump}
      }
      \vfill
      \mintocaml[firstline=9,lastline=13,highlightlines=12]{../src/backtrace/backtrace.ml}
      \endminipage
    \end{column}
    \begin{column}{0.5\textwidth}
      \centering
      \providecommand\step{1}\input{diagrams/backtrace/backtrace.tex}
    \end{column}
  \end{columns}
\end{frame}

\begin{frame}{Backtraces}{But before that, we need more fields from \typename{caml\_domain\_state}}
  \begin{columns}
    \begin{column}{0.5\textwidth}
      \begin{itemize}
        \item Remember \typename{caml\_domain\_state} the per-domain runtime state?
        \item Some other members are of interest to us now\dots
        \item \localname{backtrace\_active} is used to know if the runtime should collect backtraces
        \item \localname{backtrace\_active} can be set by
          \begin{itemize}
            \item The \funcarg{b} flag in the \funcname{OCAMLRUNPARAM} environment variable
            \item \mintinline{ocaml}{Printexc.record_backtrace : bool -> unit} \footnotemark
          \end{itemize}
      \end{itemize}
    \end{column}
    \begin{column}{0.5\textwidth}
      \begin{listing}
        \mintc[highlightlines={9-11}]{src/ocaml/domain\_state\_backtrace.h}
        \caption{runtime/caml/domain\_state.h}
      \end{listing}
    \end{column}
  \end{columns}
  \footnotetext[1]{https://v2.ocaml.org/api/Printexc.html}
\end{frame}

\newcommand\listCamlRaiseExn[1][]{
  \listasm[firstline=702,lastline=725,#1]{../ocaml/runtime/amd64.S}{runtime/amd64.S:702}}

\begin{frame}{Backtraces}{Walk through \funcname{caml\_raise\_exn}}
  \begin{columns}[T]
    \begin{column}{0.5\textwidth}
      \begin{itemize}
        \item<1-> First checks whether exceptions backtrace should be recorded
        \item<1-> By checking the \funcarg{domain\_state->backtrace\_active} value
        \item<2-> If not, acts as \funcname{raise\_notrace} with \funcname{RESTORE\_EXN\_HANDLER\_OCAML}
          \begin{itemize}
            \item Set \regname{RSP} to \localname{domain\_state->exn\_handler} value
            \item Restore \localname{domain\_state->exn\_handler} from trap
            \item Jump with \funcname{ret} (same effect as using \funcname{pop} and \funcname{jmp})
          \end{itemize}
        \item<3-> Otherwise, switches to the C stack to be able to execute C code
        \item<4-> Calls \funcname{caml\_stash\_backtrace}, a C function provided by the runtime\ignorespaces
          \onslide<6->{, with
            \begin{itemize}
              \item \funcarg{exn}: exception value
              \item \funcarg{pc}: code address of the raising point
              \item \funcarg{sp}: stack address of the raising function
              \item \funcarg{trapsp}: stack address of the next handler
            \end{itemize}
          }
      \end{itemize}
      \bigskip
      \mintocaml[firstline=9,lastline=13,highlightlines=12]{../src/backtrace/backtrace.ml}
    \end{column}
    \begin{column}{0.5\textwidth}
      \raggedleft
      \onslide*<1,3-4>{
        \mintc[firstline=190,lastline=192]{../ocaml/runtime/amd64.S}
        \mintc[firstline=234,lastline=237]{../ocaml/runtime/amd64.S}
      }
      \onslide*<2>{
        \mintc[firstline=190,lastline=192,highlightlines={191-192}]{../ocaml/runtime/amd64.S}
        \mintc[firstline=234,lastline=237,highlightlines={235-237}]{../ocaml/runtime/amd64.S}
      }
      \onslide*<1>{\listCamlRaiseExn[highlightlines=705]{}}%
      \onslide*<2>{\listCamlRaiseExn[highlightlines={707-708}]{}}%
      \onslide*<3>{\listCamlRaiseExn[highlightlines={712-714}]{}}%
      \onslide*<4>{\listCamlRaiseExn[highlightlines={715-720}]{}}%
      \onslide*<5>{\providecommand\step{1}\input{diagrams/backtrace/backtrace.tex}}%
      \onslide*<6>{\providecommand\step{2}\input{diagrams/backtrace/backtrace.tex}}%
    \end{column}
  \end{columns}
\end{frame}

\newcommand\listStashBacktrace[1][]{
  \listc[firstline=91,lastline=120,#1]{../ocaml/runtime/backtrace\_nat.c}{runtime/backtrace\_nat.c:91}}

\begin{frame}{Backtraces}{Introducing \funcname{caml\_stash\_backtrace}}
  \begin{columns}[c]
    \begin{column}{0.5\textwidth}
      \minipage[c][0.66\textheight][s]{\columnwidth}
      \begin{itemize}
        \item<1-> A C function provided by the runtime (specific to native code)
        \item<2-> It scans the stack, between the stack pointer at the raising point up to the next trap, in order to collect the names of the detected function frames
          \onslide*<2>{
            \begin{itemize}
              \item And more information, like line number, column number...
            \end{itemize}
          }
        \item<3-> It uses the same technique and information as the Garbage Collector (GC) uses to scan the values live on the stack
          \onslide*<4>{
            \begin{itemize}
              \item \typename{frame\_descr} are at the core of stack unwinding
            \end{itemize}
          }
      \end{itemize}
      \vfill
      \mintocaml[firstline=9,lastline=13,highlightlines=12]{../src/backtrace/backtrace.ml}
      \endminipage
    \end{column}
    \begin{column}{0.5\textwidth}
      \onslide*<1>{\listStashBacktrace[highlightlines=91]{}}
      \onslide*<2>{\listStashBacktrace[highlightlines={91,117-118}]{}}
      \onslide*<3->{\listStashBacktrace[highlightlines={94,106,109-110}]{}}
    \end{column}
  \end{columns}
\end{frame}

\newcommand\listFrameDescr[1][]{
  \listocaml[firstline=111,lastline=124,#1]{../ocaml/asmcomp/emitaux.ml}{asmcomp/emitaux.ml:111}}
\newcommand\listDebugInfo[1][]{
  \listocaml[firstline=38,lastline=49,#1]{../ocaml/lambda/debuginfo.mli}{lambda/debuginfo.mli:38}}

\begin{frame}{Backtraces}{\typename{frame\_descr} and \typename{frame\_debuginfo} in a nutshell}
  \begin{columns}[c]
    \begin{column}{0.5\textwidth}
      \begin{itemize}
        \item<1-> For every return address in an OCaml program, there is a associated \typename{frame\_descr}
        \item<2-> A \typename{frame\_descr} specifies
        \begin{itemize}
          \item<2-> The size of the function stack frame
          \item<3-> Where live values are on the stack (so that the GC can mark them as live and prevent from being swept)
        \end{itemize}
        \item<4-> When compiled with debug information, \typename{frame\_descr} will be linked to a \typename{frame\_debuginfo}
        \begin{itemize}
          \item<5-> Contains information about the position in the source code (file name, line and column number)
        \end{itemize}
      \end{itemize}
    \end{column}
    \begin{column}{0.5\textwidth}
        \onslide*<1>{\listFrameDescr[highlightlines={118-122}]{}}
        \onslide*<2>{\listFrameDescr[highlightlines={120}]{}}
        \onslide*<3>{\listFrameDescr[highlightlines={121}]{}}
        \onslide*<4->{\listFrameDescr[highlightlines={122}]{}}
        \medskip
        \onslide*<1-3>{\listDebugInfo{}}
        \onslide*<4>{\listDebugInfo[highlightlines={38,49}]{}}
        \onslide*<5>{\listDebugInfo[highlightlines={39-46}]{}}
    \end{column}
  \end{columns}
\end{frame}

\begin{frame}{Backtraces}{Scanning the stack with \typename{frame\_descr}}
  \begin{columns}[c]
    \begin{column}{0.5\textwidth}
      \begin{itemize}
        \item Given the return address of the code that raises the exception by calling \funcname{caml\_raise\_exn} (\funcarg{pc} argument of \funcname{caml\_stash\_backtrace})
        \item And the position of the stack pointer (\funcarg{sp} argument of \funcname{caml\_stash\_backtrace})
        \item The frame size given by \typename{frame\_descr}.\funcarg{frame\_size}, allows to find the end of the previous frame
        \item And the return address of the caller of this function
        \bigskip
        \onslide<3->{
        \item Note that traps (here the reddish \funcname{ExnA}, \funcname{ExnB} and \funcname{ExnC} blocks) belong to the frame that installed them, and so the frame size changes when traps are removed
        }
      \end{itemize}
    \end{column}
    \begin{column}{0.5\textwidth}
      \raggedleft
      \onslide*<1>{\providecommand\step{2}\input{diagrams/backtrace/backtrace.tex}}%
      \onslide*<2>{\providecommand\step{1}\input{diagrams/backtrace/scanstack.tex}}%
      \onslide*<3>{\providecommand\step{2}\input{diagrams/backtrace/scanstack.tex}}%
      \onslide*<4>{\providecommand\step{3}\input{diagrams/backtrace/scanstack.tex}}%
      \onslide*<5>{\providecommand\step{4}\input{diagrams/backtrace/scanstack.tex}}%
      \onslide*<6>{\providecommand\step{5}\input{diagrams/backtrace/scanstack.tex}}%
    \end{column}
  \end{columns}
\end{frame}

% Duplicate of previous-previous slide
\begin{frame}{Backtraces}{Continuing on \funcname{caml\_stash\_backtrace}}
  \begin{columns}[c]
    \begin{column}{0.5\textwidth}
      \minipage[c][0.75\textheight][s]{\columnwidth}
      \begin{itemize}
          \color{gray}
        \item A C function provided by the runtime (specific to native code)
        \item It scans the stack, between the stack pointer at the raising point up to the next trap, in order to collect the names of the detected function frames
        \item It uses the same technique and information as the Garbage Collector (GC) uses to scan the values live on the stack
      \end{itemize}
      \begin{itemize}
        \item For each function frame detected, store a pointer to the \typename{frame\_descr} in the \localname{domain\_state->backtrace\_buffer} array, effectively constructing the backtrace
      \end{itemize}
      \onslide<2->{
        \begin{itemize}
            \smallskip
          \item Constructed backtrace:
            \funcname{definitely\_raise},
            \onslide<3->{\funcname{probably\_raise}}
        \end{itemize}
      }
      \vfill
      \mintocaml[firstline=9,lastline=13,highlightlines=12]{../src/backtrace/backtrace.ml}
      \endminipage
    \end{column}
    \begin{column}{0.5\textwidth}
      \raggedleft
      \onslide*<1>{\listStashBacktrace[highlightlines={114-115}]{}}
      \onslide*<2>{\providecommand\step{3}\input{diagrams/backtrace/backtrace.tex}}%
      \onslide*<3>{\providecommand\step{4}\input{diagrams/backtrace/backtrace.tex}}%
    \end{column}
  \end{columns}
\end{frame}

\begin{frame}{Backtraces}{Back to \funcname{caml\_raise\_exn}}
  \begin{columns}[c]
    \begin{column}{0.5\textwidth}
      \minipage[c][0.66\textheight][s]{\columnwidth}
      \begin{itemize}\color{gray}
        \item Checks wheteher exceptions backtrace should be recorded, by checking the \funcarg{domain\_state->backtrace\_active} value
        \item Switches to the C stack to be able to execute C code
        \item Calls \funcname{caml\_stash\_backtrace}, to collect the backtrace up to the next trap
      \end{itemize}
      \onslide*<2->{
        \begin{itemize}
          \item<2-> Executes the exception handler in \funcname{probably\_raise} with \funcname{RESTORE\_EXN\_HANDLER\_OCAML}
            \begin{itemize}
              \item Set \regname{RSP} to \localname{domain\_state->exn\_handler} value
              \item Restore \localname{domain\_state->exn\_handler} from trap
              \item Jump to the handler code with \funcname{ret}
            \end{itemize}
        \end{itemize}
      }
      \vfill
      \onslide*<1>{\mintocaml[firstline=9,lastline=13,highlightlines=12]{../src/backtrace/backtrace.ml}}
      \onslide*<2->{\mintocaml[firstline=15,lastline=21,highlightlines={19-21}]{../src/backtrace/backtrace.ml}}
      \endminipage
    \end{column}
    \begin{column}{0.5\textwidth}
      \centering
      \onslide*<1>{
        \mintc[firstline=190,lastline=192]{../ocaml/runtime/amd64.S}
        \mintc[firstline=234,lastline=237]{../ocaml/runtime/amd64.S}
      }
      \onslide*<2>{
        \mintc[firstline=190,lastline=192,highlightlines={191-192}]{../ocaml/runtime/amd64.S}
        \mintc[firstline=234,lastline=237,highlightlines={235-237}]{../ocaml/runtime/amd64.S}
      }
      \onslide*<1>{\listCamlRaiseExn[highlightlines=720]{}}
      \onslide*<2>{\listCamlRaiseExn[highlightlines=722]{}}
      \onslide*<3->{\providecommand\step{5}\input{diagrams/backtrace/backtrace.tex}}
    \end{column}
  \end{columns}
\end{frame}

\begin{frame}{Backtraces}{\funcname{caml\_probably\_raise}'s handler doesn't catch \typename{ExnC}}
  \begin{columns}[c]
    \begin{column}{0.45\textwidth}
      \minipage[c][0.75\textheight][s]{\columnwidth}
      \begin{itemize}
        \item<1-> \funcname{match \dots with}\footnotemark pattern matching can also match exceptions
        \item<1-> The CMM of \funcname{probably\_raise} shows that this \funcname{match \dots with} is implemented with a pattern matching and a \funcname{try \dots with}
        \item<1-> But this one only matches on \typename{ExnA} exception
        \item<2-> Since the pattern matching fails to catch the exception, \funcname{reraise} is called with our current \typename{ExnC} exception
        \item<3-> Disassembly shows that \funcname{reraise} is actually a call to \funcname{caml\_reraise\_exn}, which is also an assembly function provided by the runtime
      \end{itemize}
      \vfill
      \mintocaml[firstline=15,lastline=21,highlightlines={18-21}]{../src/backtrace/backtrace.ml}
      \endminipage
    \end{column}
    \begin{column}{0.55\textwidth}
      \centering
      \onslide*<1>{\mintcmm[highlightlines={10-18}]{../src/backtrace/camlBacktrace\_\_probably\_raise\_351.cmm}}
      \onslide*<2>{\listcmm[highlightlines=18]{../src/backtrace/camlBacktrace\_\_probably\_raise_351.cmm}{CMM of probably\_raise}}
      \onslide*<3>{\mintobjdump[firstline=5,lastline=35,highlightlines=31]{../src/backtrace/camlBacktrace\_\_probably\_raise_351.objdump}}
    \end{column}
  \end{columns}
  \only<1>{\footnotetext[1]{https://blog.janestreet.com/pattern-matching-and-exception-handling-unite/}}
\end{frame}

\newcommand\listCamlReraiseExn[1][]{
  \listasm[firstline=727,lastline=734,#1]{../ocaml/runtime/amd64.S}{runtime/amd64.S:727}}

\begin{frame}{Backtraces}{Introducing \funcname{caml\_reraise\_exn}}
  \begin{columns}[c]
    \begin{column}{0.5\textwidth}
      \minipage[c][0.66\textheight][s]{\columnwidth}
      \begin{itemize}
        \item<1-> The exception have to be reraised to the next handler
        \item<2-> First, checks if the backtrace should be collected with \funcarg{domain\_state->backtrace\_active}
        \only<3>{
        \begin{itemize}
            \item If not, behaves like a \funcname{raise\_notrace} with \funcname{RESTORE\_EXN\_HANDLER\_OCAML}
        \end{itemize}
        }
        \item<4-> Jumps into \funcname{caml\_raise\_exn}
        \item<5-> Calls \funcname{caml\_stash\_backtrace} again, to scan the next part of the stack: from the trap just pop-ed to the next trap
      \end{itemize}
      \vfill
      \mintocaml[firstline=15,lastline=21,highlightlines={18-21}]{../src/backtrace/backtrace.ml}
      \endminipage
    \end{column}
    \begin{column}{0.5\textwidth}
      \raggedleft
      \onslide*<1>{\listCamlReraiseExn{}}
      \onslide*<2>{\listCamlReraiseExn[highlightlines=729]{}}
      \onslide*<3>{
        \mintc[firstline=190,lastline=192,highlightlines={191-192}]{../ocaml/runtime/amd64.S}
        \mintc[firstline=234,lastline=237,highlightlines={235-237}]{../ocaml/runtime/amd64.S}
        \medskip
        \listCamlReraiseExn[highlightlines=731]{}
      }
      \onslide*<4-5>{
        % TODO refactor with \listCamlReraiseExn
        \mintasm[firstline=727,lastline=734,highlightlines=730]{../ocaml/runtime/amd64.S}
        \medskip
      }
      \onslide*<4>{\listCamlRaiseExn[highlightlines=711]{}}
      \onslide*<5>{\listCamlRaiseExn[highlightlines=720]{}}
    \end{column}
  \end{columns}
\end{frame}

\begin{frame}{Backtraces}{Backtrace collection continues}
  \begin{columns}[c]
    \begin{column}{0.55\textwidth}
      \minipage[c][0.75\textheight][s]{\columnwidth}
      \begin{itemize}
        \item<1-> \funcname{caml\_stash\_backtrace} scans the older part of the stack, up to the next trap
        \item<6-> Controls jumps to the nested exception handler in \funcname{maybe\_raise}, which matches only on \typename{ExnB}
        \item<7-> \funcname{caml\_reraise\_exn} is called again, backtrace collection resumes with \funcname{caml\_stash\_backtrace}
      \end{itemize}
      \medskip
      \begin{itemize}
        \item<2-> Constructed backtrace:
        \begin{itemize}
          \item <2-> \funcname{definitely\_raise}
          \item <2-> \funcname{probably\_raise}
          \item <3-> \funcname{probably\_raise} (re-raise)
          \item <4-> \funcname{maybe\_raise}
          \item <5-> \funcname{main}
          \item <8-> \funcname{main} (re-raise)
        \end{itemize}
      \end{itemize}
      \vfill
      \onslide*<1-5>{\mintocaml[firstline=15,lastline=21]{../src/backtrace/backtrace.ml}}
      \onslide*<6>{\mintocaml[firstline=28,lastline=34,highlightlines=31]{../src/backtrace/backtrace.ml}}
      \onslide*<7->{\mintocaml[firstline=28,lastline=34,highlightlines=33]{../src/backtrace/backtrace.ml}}
      \endminipage
    \end{column}
    \begin{column}{0.45\textwidth}
      \raggedleft
      \onslide*<1>{\providecommand\step{6}\input{diagrams/backtrace/backtrace.tex}}%
      \onslide*<2>{\providecommand\step{6}\input{diagrams/backtrace/backtrace.tex}}%
      \onslide*<3>{\providecommand\step{7}\input{diagrams/backtrace/backtrace.tex}}%
      \onslide*<4>{\providecommand\step{8}\input{diagrams/backtrace/backtrace.tex}}%
      \onslide*<5>{\providecommand\step{9}\input{diagrams/backtrace/backtrace.tex}}%
      \onslide*<6>{\providecommand\step{10}\input{diagrams/backtrace/backtrace.tex}}%
      \onslide*<7>{\providecommand\step{11}\input{diagrams/backtrace/backtrace.tex}}%
      \onslide*<8>{\providecommand\step{12}\input{diagrams/backtrace/backtrace.tex}}%
      \onslide*<9>{\providecommand\step{13}\input{diagrams/backtrace/backtrace.tex}}%
    \end{column}
  \end{columns}
\end{frame}

\begin{frame}{Backtraces}{Printing the backtrace}
  \begin{columns}[c]
    \begin{column}{0.45\textwidth}
      \minipage[c][0.66\textheight][s]{\columnwidth}
      \begin{itemize}
        \item<1-> The exception backtrace can now be printed to \funcarg{stdout} with \funcname{Printexc.print_backtrace}
        \item<2-> First the exception backtrace is retrieved into an OCaml array with \funcname{get_raw_backtrace }
        \item<3-> Which is implemented as the \funcname{caml_get_exception_raw_backtrace} C function in the runtime
        \item<4-> And finally printed with \funcname{print_exception_backtrace}
      \end{itemize}
      \vfill
      \mintocaml[firstline=28,lastline=34,highlightlines=33]{../src/backtrace/backtrace.ml}
      \endminipage
    \end{column}
    \begin{column}{0.55\textwidth}
      \onslide*<1,2>{
        \mintocaml[firstline=100,lastline=101]{../ocaml/stdlib/printexc.ml}
      }
      \only<1>{
        \listocaml[firstline=170,lastline=172]{../ocaml/stdlib/printexc.ml}{stdlib/printexc.ml:170}
      }
      \only<2>{
        \listocaml[firstline=170,lastline=172,highlightlines=172]{../ocaml/stdlib/printexc.ml}{stdlib/printexc.ml:170}
      }
      \only<3>{
        \mintc[firstline=154,lastline=156]{../ocaml/runtime/backtrace.c}
        \listc[firstline=171,lastline=192,highlightlines={184-187}]{../ocaml/runtime/backtrace.c}{runtime/backtrace.c:154}
      }
      \only<4>{
        \listocaml[firstline=155,lastline=165]{../ocaml/stdlib/printexc.ml}{stdlib/printexc.ml:55}
      }
    \end{column}
  \end{columns}
\end{frame}


\subsection{The multiple kinds of raise}
\frameSubsection{}{}

\begin{frame}{Backtraces}{When backtraces are not needed}
  \begin{columns}[c]
    \begin{column}{0.45\textwidth}
      \minipage[c][0.66\textheight][s]{\columnwidth}
      \begin{itemize}
        \item<1-> What if the backtrace for an exception is never needed?
        \item<2-> It's possible to raise the exception explicitly with \funcname{raise\_notrace} to make raising as cheap as possible
        \item<3-> An example of use in the \funcname{dynlink} library of the OCaml compiler
      \end{itemize}
      \vfill
      \onslide<3->{
      \listocaml[firstline=205,lastline=209,highlightlines=207]{../ocaml/otherlibs/dynlink/dynlink\_compilerlibs/misc.ml}{otherlibs/dynlink/dynlink\_compilerlibs/misc.ml:205}
      }
      \endminipage
    \end{column}
    \begin{column}{0.55\textwidth}
    \only<4>{
      \mintcmm[highlightlines=3]{../src/notrace/camlNotrace\_\_fun_347.cmm}
      \listcmm{../src/notrace/camlNotrace\_\_all\_somes\_327.cmm}{all\_somes}
    }
    \onslide*<5->{
      \listobjdump[highlightlines={6-9}]{../src/notrace/camlNotrace\_\_fun_347.objdump}{anonymous function from \funcname{all\_somes}}
    }
    \end{column}
  \end{columns}
\end{frame}

\begin{frame}{Backtraces}{Summary table of the multiple kinds of raise}
  \begin{itemize}
    \item When debugging information is enabled and if the backtrace of an exception isn't needed, it's possible to raise with \funcname{raise\_notrace} for performance
    \item When debugging information is \emph{not} enabled, the compiler optimises \funcname{raise} calls to inline assembly (same effect as using \funcname{raise\_notrace})
  \end{itemize}
  \bigskip
  \centering
  \begin{tabular}{| l | c | c | c | c |}
    \hline
    \parbox{10em}{Debug information \\ (\funcarg{-g} flag)}  & \multicolumn{2}{|c|}{Disabled}                                     & \multicolumn{2}{|c|}{Enabled} \\
    \hline
    Raise kind                                               & \funcname{raise} & \funcname{raise\_notrace}                       & \funcname{raise} & \funcname{raise\_notrace} \\
    \hline
    Compilation                                              & \parbox{8em}{inline assembly\\ (optimization)} & inline assembly   & \funcname{caml\_raise\_exn} & inline assembly \\
    \hline
  \end{tabular}
\end{frame}


%
% Takeaway #5
%
\subsection*{Takeaway \#5}
\frameSubsectionTakeaway{}
\againframe<5>{takeaway}
}%
      \onslide*<2>{\providecommand\step{6}%
% Backtraces
%
\subsection{One advantage of raising with debugging information}
\frameSubsection{}{}

\begin{frame}{Backtraces}{Debugging information \& source code}
  \begin{columns}[c]
    \begin{column}{0.5\textwidth}
      \begin{itemize}
        \item Raising an exception is fun and games, but how do we know where the exception originated? $\rightarrow$ With the backtrace associated with the exception!
        \item In order to get a backtrace from the point where an exception was raised, it's necessary to compile with debugging information enabled (\funcarg{-g} flag for the compiler)
        \item Same example as in the `Nested handlers' section
      \end{itemize}
    \end{column}
    \begin{column}{0.5\textwidth}
      \listocaml[firstline=9,lastline=34]{../src/backtrace/backtrace.ml}{backtrace/backtrace.ml}
    \end{column}
  \end{columns}
\end{frame}

\begin{frame}{Backtraces}{CMM and disassembly of \funcname{definitely\_raise}}
  \begin{columns}[c]
    \begin{column}{0.5\textwidth}
      \minipage[c][0.66\textheight][s]{\columnwidth}
      \begin{itemize}
        \item<1-> An effect of compiling with debugging information (\funcarg{-g}): the CMM no longer uses \funcname{raise\_notrace} to raise the exception but \funcname{raise} instead
        \item<2-> Disassembly shows that CMM \funcname{raise} is actually a call to \funcname{caml\_raise\_exn}, a function in assembler provided by the runtime
      \end{itemize}
      \vfill
      \mintocaml[firstline=9,lastline=13,highlightlines=12]{../src/backtrace/backtrace.ml}
      \endminipage
    \end{column}
    \begin{column}{0.5\textwidth}
      \onslide*<1>{
        \mintcmm[highlightlines=5]{../src/backtrace/camlBacktrace\_\_definitely\_raise\_347.cmm}
      }
      \onslide*<2>{
        \mintobjdump[firstline=2,highlightlines=14]{../src/backtrace/camlBacktrace\_\_definitely\_raise\_347.objdump}
      }
    \end{column}
  \end{columns}
\end{frame}

\begin{frame}{Backtraces}{Introducing \funcname{caml\_raise\_exn}}
  \begin{columns}[c]
    \begin{column}{0.5\textwidth}
      \minipage[c][0.66\textheight][s]{\columnwidth}
      \onslide*<1>{
        \begin{itemize}
          \item Raising the exception is performed using \funcname{caml\_raise\_exn} which is
            \begin{itemize}
              \item An assembly function provided by the runtime
              \item Architecture specific
            \end{itemize}
          \item Let's see what it does differently from \funcname{raise\_notrace}\dots
          \item We are about to raise \typename{ExnC} with \funcname{caml\_raise\_exn} from \funcname{definitely\_raise}
        \end{itemize}
      }
      \onslide*<2>{
        \mintobjdump[firstline=2,highlightlines=14]{../src/backtrace/camlBacktrace\_\_definitely\_raise\_347.objdump}
      }
      \vfill
      \mintocaml[firstline=9,lastline=13,highlightlines=12]{../src/backtrace/backtrace.ml}
      \endminipage
    \end{column}
    \begin{column}{0.5\textwidth}
      \centering
      \providecommand\step{1}\input{diagrams/backtrace/backtrace.tex}
    \end{column}
  \end{columns}
\end{frame}

\begin{frame}{Backtraces}{But before that, we need more fields from \typename{caml\_domain\_state}}
  \begin{columns}
    \begin{column}{0.5\textwidth}
      \begin{itemize}
        \item Remember \typename{caml\_domain\_state} the per-domain runtime state?
        \item Some other members are of interest to us now\dots
        \item \localname{backtrace\_active} is used to know if the runtime should collect backtraces
        \item \localname{backtrace\_active} can be set by
          \begin{itemize}
            \item The \funcarg{b} flag in the \funcname{OCAMLRUNPARAM} environment variable
            \item \mintinline{ocaml}{Printexc.record_backtrace : bool -> unit} \footnotemark
          \end{itemize}
      \end{itemize}
    \end{column}
    \begin{column}{0.5\textwidth}
      \begin{listing}
        \mintc[highlightlines={9-11}]{src/ocaml/domain\_state\_backtrace.h}
        \caption{runtime/caml/domain\_state.h}
      \end{listing}
    \end{column}
  \end{columns}
  \footnotetext[1]{https://v2.ocaml.org/api/Printexc.html}
\end{frame}

\newcommand\listCamlRaiseExn[1][]{
  \listasm[firstline=702,lastline=725,#1]{../ocaml/runtime/amd64.S}{runtime/amd64.S:702}}

\begin{frame}{Backtraces}{Walk through \funcname{caml\_raise\_exn}}
  \begin{columns}[T]
    \begin{column}{0.5\textwidth}
      \begin{itemize}
        \item<1-> First checks whether exceptions backtrace should be recorded
        \item<1-> By checking the \funcarg{domain\_state->backtrace\_active} value
        \item<2-> If not, acts as \funcname{raise\_notrace} with \funcname{RESTORE\_EXN\_HANDLER\_OCAML}
          \begin{itemize}
            \item Set \regname{RSP} to \localname{domain\_state->exn\_handler} value
            \item Restore \localname{domain\_state->exn\_handler} from trap
            \item Jump with \funcname{ret} (same effect as using \funcname{pop} and \funcname{jmp})
          \end{itemize}
        \item<3-> Otherwise, switches to the C stack to be able to execute C code
        \item<4-> Calls \funcname{caml\_stash\_backtrace}, a C function provided by the runtime\ignorespaces
          \onslide<6->{, with
            \begin{itemize}
              \item \funcarg{exn}: exception value
              \item \funcarg{pc}: code address of the raising point
              \item \funcarg{sp}: stack address of the raising function
              \item \funcarg{trapsp}: stack address of the next handler
            \end{itemize}
          }
      \end{itemize}
      \bigskip
      \mintocaml[firstline=9,lastline=13,highlightlines=12]{../src/backtrace/backtrace.ml}
    \end{column}
    \begin{column}{0.5\textwidth}
      \raggedleft
      \onslide*<1,3-4>{
        \mintc[firstline=190,lastline=192]{../ocaml/runtime/amd64.S}
        \mintc[firstline=234,lastline=237]{../ocaml/runtime/amd64.S}
      }
      \onslide*<2>{
        \mintc[firstline=190,lastline=192,highlightlines={191-192}]{../ocaml/runtime/amd64.S}
        \mintc[firstline=234,lastline=237,highlightlines={235-237}]{../ocaml/runtime/amd64.S}
      }
      \onslide*<1>{\listCamlRaiseExn[highlightlines=705]{}}%
      \onslide*<2>{\listCamlRaiseExn[highlightlines={707-708}]{}}%
      \onslide*<3>{\listCamlRaiseExn[highlightlines={712-714}]{}}%
      \onslide*<4>{\listCamlRaiseExn[highlightlines={715-720}]{}}%
      \onslide*<5>{\providecommand\step{1}\input{diagrams/backtrace/backtrace.tex}}%
      \onslide*<6>{\providecommand\step{2}\input{diagrams/backtrace/backtrace.tex}}%
    \end{column}
  \end{columns}
\end{frame}

\newcommand\listStashBacktrace[1][]{
  \listc[firstline=91,lastline=120,#1]{../ocaml/runtime/backtrace\_nat.c}{runtime/backtrace\_nat.c:91}}

\begin{frame}{Backtraces}{Introducing \funcname{caml\_stash\_backtrace}}
  \begin{columns}[c]
    \begin{column}{0.5\textwidth}
      \minipage[c][0.66\textheight][s]{\columnwidth}
      \begin{itemize}
        \item<1-> A C function provided by the runtime (specific to native code)
        \item<2-> It scans the stack, between the stack pointer at the raising point up to the next trap, in order to collect the names of the detected function frames
          \onslide*<2>{
            \begin{itemize}
              \item And more information, like line number, column number...
            \end{itemize}
          }
        \item<3-> It uses the same technique and information as the Garbage Collector (GC) uses to scan the values live on the stack
          \onslide*<4>{
            \begin{itemize}
              \item \typename{frame\_descr} are at the core of stack unwinding
            \end{itemize}
          }
      \end{itemize}
      \vfill
      \mintocaml[firstline=9,lastline=13,highlightlines=12]{../src/backtrace/backtrace.ml}
      \endminipage
    \end{column}
    \begin{column}{0.5\textwidth}
      \onslide*<1>{\listStashBacktrace[highlightlines=91]{}}
      \onslide*<2>{\listStashBacktrace[highlightlines={91,117-118}]{}}
      \onslide*<3->{\listStashBacktrace[highlightlines={94,106,109-110}]{}}
    \end{column}
  \end{columns}
\end{frame}

\newcommand\listFrameDescr[1][]{
  \listocaml[firstline=111,lastline=124,#1]{../ocaml/asmcomp/emitaux.ml}{asmcomp/emitaux.ml:111}}
\newcommand\listDebugInfo[1][]{
  \listocaml[firstline=38,lastline=49,#1]{../ocaml/lambda/debuginfo.mli}{lambda/debuginfo.mli:38}}

\begin{frame}{Backtraces}{\typename{frame\_descr} and \typename{frame\_debuginfo} in a nutshell}
  \begin{columns}[c]
    \begin{column}{0.5\textwidth}
      \begin{itemize}
        \item<1-> For every return address in an OCaml program, there is a associated \typename{frame\_descr}
        \item<2-> A \typename{frame\_descr} specifies
        \begin{itemize}
          \item<2-> The size of the function stack frame
          \item<3-> Where live values are on the stack (so that the GC can mark them as live and prevent from being swept)
        \end{itemize}
        \item<4-> When compiled with debug information, \typename{frame\_descr} will be linked to a \typename{frame\_debuginfo}
        \begin{itemize}
          \item<5-> Contains information about the position in the source code (file name, line and column number)
        \end{itemize}
      \end{itemize}
    \end{column}
    \begin{column}{0.5\textwidth}
        \onslide*<1>{\listFrameDescr[highlightlines={118-122}]{}}
        \onslide*<2>{\listFrameDescr[highlightlines={120}]{}}
        \onslide*<3>{\listFrameDescr[highlightlines={121}]{}}
        \onslide*<4->{\listFrameDescr[highlightlines={122}]{}}
        \medskip
        \onslide*<1-3>{\listDebugInfo{}}
        \onslide*<4>{\listDebugInfo[highlightlines={38,49}]{}}
        \onslide*<5>{\listDebugInfo[highlightlines={39-46}]{}}
    \end{column}
  \end{columns}
\end{frame}

\begin{frame}{Backtraces}{Scanning the stack with \typename{frame\_descr}}
  \begin{columns}[c]
    \begin{column}{0.5\textwidth}
      \begin{itemize}
        \item Given the return address of the code that raises the exception by calling \funcname{caml\_raise\_exn} (\funcarg{pc} argument of \funcname{caml\_stash\_backtrace})
        \item And the position of the stack pointer (\funcarg{sp} argument of \funcname{caml\_stash\_backtrace})
        \item The frame size given by \typename{frame\_descr}.\funcarg{frame\_size}, allows to find the end of the previous frame
        \item And the return address of the caller of this function
        \bigskip
        \onslide<3->{
        \item Note that traps (here the reddish \funcname{ExnA}, \funcname{ExnB} and \funcname{ExnC} blocks) belong to the frame that installed them, and so the frame size changes when traps are removed
        }
      \end{itemize}
    \end{column}
    \begin{column}{0.5\textwidth}
      \raggedleft
      \onslide*<1>{\providecommand\step{2}\input{diagrams/backtrace/backtrace.tex}}%
      \onslide*<2>{\providecommand\step{1}\input{diagrams/backtrace/scanstack.tex}}%
      \onslide*<3>{\providecommand\step{2}\input{diagrams/backtrace/scanstack.tex}}%
      \onslide*<4>{\providecommand\step{3}\input{diagrams/backtrace/scanstack.tex}}%
      \onslide*<5>{\providecommand\step{4}\input{diagrams/backtrace/scanstack.tex}}%
      \onslide*<6>{\providecommand\step{5}\input{diagrams/backtrace/scanstack.tex}}%
    \end{column}
  \end{columns}
\end{frame}

% Duplicate of previous-previous slide
\begin{frame}{Backtraces}{Continuing on \funcname{caml\_stash\_backtrace}}
  \begin{columns}[c]
    \begin{column}{0.5\textwidth}
      \minipage[c][0.75\textheight][s]{\columnwidth}
      \begin{itemize}
          \color{gray}
        \item A C function provided by the runtime (specific to native code)
        \item It scans the stack, between the stack pointer at the raising point up to the next trap, in order to collect the names of the detected function frames
        \item It uses the same technique and information as the Garbage Collector (GC) uses to scan the values live on the stack
      \end{itemize}
      \begin{itemize}
        \item For each function frame detected, store a pointer to the \typename{frame\_descr} in the \localname{domain\_state->backtrace\_buffer} array, effectively constructing the backtrace
      \end{itemize}
      \onslide<2->{
        \begin{itemize}
            \smallskip
          \item Constructed backtrace:
            \funcname{definitely\_raise},
            \onslide<3->{\funcname{probably\_raise}}
        \end{itemize}
      }
      \vfill
      \mintocaml[firstline=9,lastline=13,highlightlines=12]{../src/backtrace/backtrace.ml}
      \endminipage
    \end{column}
    \begin{column}{0.5\textwidth}
      \raggedleft
      \onslide*<1>{\listStashBacktrace[highlightlines={114-115}]{}}
      \onslide*<2>{\providecommand\step{3}\input{diagrams/backtrace/backtrace.tex}}%
      \onslide*<3>{\providecommand\step{4}\input{diagrams/backtrace/backtrace.tex}}%
    \end{column}
  \end{columns}
\end{frame}

\begin{frame}{Backtraces}{Back to \funcname{caml\_raise\_exn}}
  \begin{columns}[c]
    \begin{column}{0.5\textwidth}
      \minipage[c][0.66\textheight][s]{\columnwidth}
      \begin{itemize}\color{gray}
        \item Checks wheteher exceptions backtrace should be recorded, by checking the \funcarg{domain\_state->backtrace\_active} value
        \item Switches to the C stack to be able to execute C code
        \item Calls \funcname{caml\_stash\_backtrace}, to collect the backtrace up to the next trap
      \end{itemize}
      \onslide*<2->{
        \begin{itemize}
          \item<2-> Executes the exception handler in \funcname{probably\_raise} with \funcname{RESTORE\_EXN\_HANDLER\_OCAML}
            \begin{itemize}
              \item Set \regname{RSP} to \localname{domain\_state->exn\_handler} value
              \item Restore \localname{domain\_state->exn\_handler} from trap
              \item Jump to the handler code with \funcname{ret}
            \end{itemize}
        \end{itemize}
      }
      \vfill
      \onslide*<1>{\mintocaml[firstline=9,lastline=13,highlightlines=12]{../src/backtrace/backtrace.ml}}
      \onslide*<2->{\mintocaml[firstline=15,lastline=21,highlightlines={19-21}]{../src/backtrace/backtrace.ml}}
      \endminipage
    \end{column}
    \begin{column}{0.5\textwidth}
      \centering
      \onslide*<1>{
        \mintc[firstline=190,lastline=192]{../ocaml/runtime/amd64.S}
        \mintc[firstline=234,lastline=237]{../ocaml/runtime/amd64.S}
      }
      \onslide*<2>{
        \mintc[firstline=190,lastline=192,highlightlines={191-192}]{../ocaml/runtime/amd64.S}
        \mintc[firstline=234,lastline=237,highlightlines={235-237}]{../ocaml/runtime/amd64.S}
      }
      \onslide*<1>{\listCamlRaiseExn[highlightlines=720]{}}
      \onslide*<2>{\listCamlRaiseExn[highlightlines=722]{}}
      \onslide*<3->{\providecommand\step{5}\input{diagrams/backtrace/backtrace.tex}}
    \end{column}
  \end{columns}
\end{frame}

\begin{frame}{Backtraces}{\funcname{caml\_probably\_raise}'s handler doesn't catch \typename{ExnC}}
  \begin{columns}[c]
    \begin{column}{0.45\textwidth}
      \minipage[c][0.75\textheight][s]{\columnwidth}
      \begin{itemize}
        \item<1-> \funcname{match \dots with}\footnotemark pattern matching can also match exceptions
        \item<1-> The CMM of \funcname{probably\_raise} shows that this \funcname{match \dots with} is implemented with a pattern matching and a \funcname{try \dots with}
        \item<1-> But this one only matches on \typename{ExnA} exception
        \item<2-> Since the pattern matching fails to catch the exception, \funcname{reraise} is called with our current \typename{ExnC} exception
        \item<3-> Disassembly shows that \funcname{reraise} is actually a call to \funcname{caml\_reraise\_exn}, which is also an assembly function provided by the runtime
      \end{itemize}
      \vfill
      \mintocaml[firstline=15,lastline=21,highlightlines={18-21}]{../src/backtrace/backtrace.ml}
      \endminipage
    \end{column}
    \begin{column}{0.55\textwidth}
      \centering
      \onslide*<1>{\mintcmm[highlightlines={10-18}]{../src/backtrace/camlBacktrace\_\_probably\_raise\_351.cmm}}
      \onslide*<2>{\listcmm[highlightlines=18]{../src/backtrace/camlBacktrace\_\_probably\_raise_351.cmm}{CMM of probably\_raise}}
      \onslide*<3>{\mintobjdump[firstline=5,lastline=35,highlightlines=31]{../src/backtrace/camlBacktrace\_\_probably\_raise_351.objdump}}
    \end{column}
  \end{columns}
  \only<1>{\footnotetext[1]{https://blog.janestreet.com/pattern-matching-and-exception-handling-unite/}}
\end{frame}

\newcommand\listCamlReraiseExn[1][]{
  \listasm[firstline=727,lastline=734,#1]{../ocaml/runtime/amd64.S}{runtime/amd64.S:727}}

\begin{frame}{Backtraces}{Introducing \funcname{caml\_reraise\_exn}}
  \begin{columns}[c]
    \begin{column}{0.5\textwidth}
      \minipage[c][0.66\textheight][s]{\columnwidth}
      \begin{itemize}
        \item<1-> The exception have to be reraised to the next handler
        \item<2-> First, checks if the backtrace should be collected with \funcarg{domain\_state->backtrace\_active}
        \only<3>{
        \begin{itemize}
            \item If not, behaves like a \funcname{raise\_notrace} with \funcname{RESTORE\_EXN\_HANDLER\_OCAML}
        \end{itemize}
        }
        \item<4-> Jumps into \funcname{caml\_raise\_exn}
        \item<5-> Calls \funcname{caml\_stash\_backtrace} again, to scan the next part of the stack: from the trap just pop-ed to the next trap
      \end{itemize}
      \vfill
      \mintocaml[firstline=15,lastline=21,highlightlines={18-21}]{../src/backtrace/backtrace.ml}
      \endminipage
    \end{column}
    \begin{column}{0.5\textwidth}
      \raggedleft
      \onslide*<1>{\listCamlReraiseExn{}}
      \onslide*<2>{\listCamlReraiseExn[highlightlines=729]{}}
      \onslide*<3>{
        \mintc[firstline=190,lastline=192,highlightlines={191-192}]{../ocaml/runtime/amd64.S}
        \mintc[firstline=234,lastline=237,highlightlines={235-237}]{../ocaml/runtime/amd64.S}
        \medskip
        \listCamlReraiseExn[highlightlines=731]{}
      }
      \onslide*<4-5>{
        % TODO refactor with \listCamlReraiseExn
        \mintasm[firstline=727,lastline=734,highlightlines=730]{../ocaml/runtime/amd64.S}
        \medskip
      }
      \onslide*<4>{\listCamlRaiseExn[highlightlines=711]{}}
      \onslide*<5>{\listCamlRaiseExn[highlightlines=720]{}}
    \end{column}
  \end{columns}
\end{frame}

\begin{frame}{Backtraces}{Backtrace collection continues}
  \begin{columns}[c]
    \begin{column}{0.55\textwidth}
      \minipage[c][0.75\textheight][s]{\columnwidth}
      \begin{itemize}
        \item<1-> \funcname{caml\_stash\_backtrace} scans the older part of the stack, up to the next trap
        \item<6-> Controls jumps to the nested exception handler in \funcname{maybe\_raise}, which matches only on \typename{ExnB}
        \item<7-> \funcname{caml\_reraise\_exn} is called again, backtrace collection resumes with \funcname{caml\_stash\_backtrace}
      \end{itemize}
      \medskip
      \begin{itemize}
        \item<2-> Constructed backtrace:
        \begin{itemize}
          \item <2-> \funcname{definitely\_raise}
          \item <2-> \funcname{probably\_raise}
          \item <3-> \funcname{probably\_raise} (re-raise)
          \item <4-> \funcname{maybe\_raise}
          \item <5-> \funcname{main}
          \item <8-> \funcname{main} (re-raise)
        \end{itemize}
      \end{itemize}
      \vfill
      \onslide*<1-5>{\mintocaml[firstline=15,lastline=21]{../src/backtrace/backtrace.ml}}
      \onslide*<6>{\mintocaml[firstline=28,lastline=34,highlightlines=31]{../src/backtrace/backtrace.ml}}
      \onslide*<7->{\mintocaml[firstline=28,lastline=34,highlightlines=33]{../src/backtrace/backtrace.ml}}
      \endminipage
    \end{column}
    \begin{column}{0.45\textwidth}
      \raggedleft
      \onslide*<1>{\providecommand\step{6}\input{diagrams/backtrace/backtrace.tex}}%
      \onslide*<2>{\providecommand\step{6}\input{diagrams/backtrace/backtrace.tex}}%
      \onslide*<3>{\providecommand\step{7}\input{diagrams/backtrace/backtrace.tex}}%
      \onslide*<4>{\providecommand\step{8}\input{diagrams/backtrace/backtrace.tex}}%
      \onslide*<5>{\providecommand\step{9}\input{diagrams/backtrace/backtrace.tex}}%
      \onslide*<6>{\providecommand\step{10}\input{diagrams/backtrace/backtrace.tex}}%
      \onslide*<7>{\providecommand\step{11}\input{diagrams/backtrace/backtrace.tex}}%
      \onslide*<8>{\providecommand\step{12}\input{diagrams/backtrace/backtrace.tex}}%
      \onslide*<9>{\providecommand\step{13}\input{diagrams/backtrace/backtrace.tex}}%
    \end{column}
  \end{columns}
\end{frame}

\begin{frame}{Backtraces}{Printing the backtrace}
  \begin{columns}[c]
    \begin{column}{0.45\textwidth}
      \minipage[c][0.66\textheight][s]{\columnwidth}
      \begin{itemize}
        \item<1-> The exception backtrace can now be printed to \funcarg{stdout} with \funcname{Printexc.print_backtrace}
        \item<2-> First the exception backtrace is retrieved into an OCaml array with \funcname{get_raw_backtrace }
        \item<3-> Which is implemented as the \funcname{caml_get_exception_raw_backtrace} C function in the runtime
        \item<4-> And finally printed with \funcname{print_exception_backtrace}
      \end{itemize}
      \vfill
      \mintocaml[firstline=28,lastline=34,highlightlines=33]{../src/backtrace/backtrace.ml}
      \endminipage
    \end{column}
    \begin{column}{0.55\textwidth}
      \onslide*<1,2>{
        \mintocaml[firstline=100,lastline=101]{../ocaml/stdlib/printexc.ml}
      }
      \only<1>{
        \listocaml[firstline=170,lastline=172]{../ocaml/stdlib/printexc.ml}{stdlib/printexc.ml:170}
      }
      \only<2>{
        \listocaml[firstline=170,lastline=172,highlightlines=172]{../ocaml/stdlib/printexc.ml}{stdlib/printexc.ml:170}
      }
      \only<3>{
        \mintc[firstline=154,lastline=156]{../ocaml/runtime/backtrace.c}
        \listc[firstline=171,lastline=192,highlightlines={184-187}]{../ocaml/runtime/backtrace.c}{runtime/backtrace.c:154}
      }
      \only<4>{
        \listocaml[firstline=155,lastline=165]{../ocaml/stdlib/printexc.ml}{stdlib/printexc.ml:55}
      }
    \end{column}
  \end{columns}
\end{frame}


\subsection{The multiple kinds of raise}
\frameSubsection{}{}

\begin{frame}{Backtraces}{When backtraces are not needed}
  \begin{columns}[c]
    \begin{column}{0.45\textwidth}
      \minipage[c][0.66\textheight][s]{\columnwidth}
      \begin{itemize}
        \item<1-> What if the backtrace for an exception is never needed?
        \item<2-> It's possible to raise the exception explicitly with \funcname{raise\_notrace} to make raising as cheap as possible
        \item<3-> An example of use in the \funcname{dynlink} library of the OCaml compiler
      \end{itemize}
      \vfill
      \onslide<3->{
      \listocaml[firstline=205,lastline=209,highlightlines=207]{../ocaml/otherlibs/dynlink/dynlink\_compilerlibs/misc.ml}{otherlibs/dynlink/dynlink\_compilerlibs/misc.ml:205}
      }
      \endminipage
    \end{column}
    \begin{column}{0.55\textwidth}
    \only<4>{
      \mintcmm[highlightlines=3]{../src/notrace/camlNotrace\_\_fun_347.cmm}
      \listcmm{../src/notrace/camlNotrace\_\_all\_somes\_327.cmm}{all\_somes}
    }
    \onslide*<5->{
      \listobjdump[highlightlines={6-9}]{../src/notrace/camlNotrace\_\_fun_347.objdump}{anonymous function from \funcname{all\_somes}}
    }
    \end{column}
  \end{columns}
\end{frame}

\begin{frame}{Backtraces}{Summary table of the multiple kinds of raise}
  \begin{itemize}
    \item When debugging information is enabled and if the backtrace of an exception isn't needed, it's possible to raise with \funcname{raise\_notrace} for performance
    \item When debugging information is \emph{not} enabled, the compiler optimises \funcname{raise} calls to inline assembly (same effect as using \funcname{raise\_notrace})
  \end{itemize}
  \bigskip
  \centering
  \begin{tabular}{| l | c | c | c | c |}
    \hline
    \parbox{10em}{Debug information \\ (\funcarg{-g} flag)}  & \multicolumn{2}{|c|}{Disabled}                                     & \multicolumn{2}{|c|}{Enabled} \\
    \hline
    Raise kind                                               & \funcname{raise} & \funcname{raise\_notrace}                       & \funcname{raise} & \funcname{raise\_notrace} \\
    \hline
    Compilation                                              & \parbox{8em}{inline assembly\\ (optimization)} & inline assembly   & \funcname{caml\_raise\_exn} & inline assembly \\
    \hline
  \end{tabular}
\end{frame}


%
% Takeaway #5
%
\subsection*{Takeaway \#5}
\frameSubsectionTakeaway{}
\againframe<5>{takeaway}
}%
      \onslide*<3>{\providecommand\step{7}%
% Backtraces
%
\subsection{One advantage of raising with debugging information}
\frameSubsection{}{}

\begin{frame}{Backtraces}{Debugging information \& source code}
  \begin{columns}[c]
    \begin{column}{0.5\textwidth}
      \begin{itemize}
        \item Raising an exception is fun and games, but how do we know where the exception originated? $\rightarrow$ With the backtrace associated with the exception!
        \item In order to get a backtrace from the point where an exception was raised, it's necessary to compile with debugging information enabled (\funcarg{-g} flag for the compiler)
        \item Same example as in the `Nested handlers' section
      \end{itemize}
    \end{column}
    \begin{column}{0.5\textwidth}
      \listocaml[firstline=9,lastline=34]{../src/backtrace/backtrace.ml}{backtrace/backtrace.ml}
    \end{column}
  \end{columns}
\end{frame}

\begin{frame}{Backtraces}{CMM and disassembly of \funcname{definitely\_raise}}
  \begin{columns}[c]
    \begin{column}{0.5\textwidth}
      \minipage[c][0.66\textheight][s]{\columnwidth}
      \begin{itemize}
        \item<1-> An effect of compiling with debugging information (\funcarg{-g}): the CMM no longer uses \funcname{raise\_notrace} to raise the exception but \funcname{raise} instead
        \item<2-> Disassembly shows that CMM \funcname{raise} is actually a call to \funcname{caml\_raise\_exn}, a function in assembler provided by the runtime
      \end{itemize}
      \vfill
      \mintocaml[firstline=9,lastline=13,highlightlines=12]{../src/backtrace/backtrace.ml}
      \endminipage
    \end{column}
    \begin{column}{0.5\textwidth}
      \onslide*<1>{
        \mintcmm[highlightlines=5]{../src/backtrace/camlBacktrace\_\_definitely\_raise\_347.cmm}
      }
      \onslide*<2>{
        \mintobjdump[firstline=2,highlightlines=14]{../src/backtrace/camlBacktrace\_\_definitely\_raise\_347.objdump}
      }
    \end{column}
  \end{columns}
\end{frame}

\begin{frame}{Backtraces}{Introducing \funcname{caml\_raise\_exn}}
  \begin{columns}[c]
    \begin{column}{0.5\textwidth}
      \minipage[c][0.66\textheight][s]{\columnwidth}
      \onslide*<1>{
        \begin{itemize}
          \item Raising the exception is performed using \funcname{caml\_raise\_exn} which is
            \begin{itemize}
              \item An assembly function provided by the runtime
              \item Architecture specific
            \end{itemize}
          \item Let's see what it does differently from \funcname{raise\_notrace}\dots
          \item We are about to raise \typename{ExnC} with \funcname{caml\_raise\_exn} from \funcname{definitely\_raise}
        \end{itemize}
      }
      \onslide*<2>{
        \mintobjdump[firstline=2,highlightlines=14]{../src/backtrace/camlBacktrace\_\_definitely\_raise\_347.objdump}
      }
      \vfill
      \mintocaml[firstline=9,lastline=13,highlightlines=12]{../src/backtrace/backtrace.ml}
      \endminipage
    \end{column}
    \begin{column}{0.5\textwidth}
      \centering
      \providecommand\step{1}\input{diagrams/backtrace/backtrace.tex}
    \end{column}
  \end{columns}
\end{frame}

\begin{frame}{Backtraces}{But before that, we need more fields from \typename{caml\_domain\_state}}
  \begin{columns}
    \begin{column}{0.5\textwidth}
      \begin{itemize}
        \item Remember \typename{caml\_domain\_state} the per-domain runtime state?
        \item Some other members are of interest to us now\dots
        \item \localname{backtrace\_active} is used to know if the runtime should collect backtraces
        \item \localname{backtrace\_active} can be set by
          \begin{itemize}
            \item The \funcarg{b} flag in the \funcname{OCAMLRUNPARAM} environment variable
            \item \mintinline{ocaml}{Printexc.record_backtrace : bool -> unit} \footnotemark
          \end{itemize}
      \end{itemize}
    \end{column}
    \begin{column}{0.5\textwidth}
      \begin{listing}
        \mintc[highlightlines={9-11}]{src/ocaml/domain\_state\_backtrace.h}
        \caption{runtime/caml/domain\_state.h}
      \end{listing}
    \end{column}
  \end{columns}
  \footnotetext[1]{https://v2.ocaml.org/api/Printexc.html}
\end{frame}

\newcommand\listCamlRaiseExn[1][]{
  \listasm[firstline=702,lastline=725,#1]{../ocaml/runtime/amd64.S}{runtime/amd64.S:702}}

\begin{frame}{Backtraces}{Walk through \funcname{caml\_raise\_exn}}
  \begin{columns}[T]
    \begin{column}{0.5\textwidth}
      \begin{itemize}
        \item<1-> First checks whether exceptions backtrace should be recorded
        \item<1-> By checking the \funcarg{domain\_state->backtrace\_active} value
        \item<2-> If not, acts as \funcname{raise\_notrace} with \funcname{RESTORE\_EXN\_HANDLER\_OCAML}
          \begin{itemize}
            \item Set \regname{RSP} to \localname{domain\_state->exn\_handler} value
            \item Restore \localname{domain\_state->exn\_handler} from trap
            \item Jump with \funcname{ret} (same effect as using \funcname{pop} and \funcname{jmp})
          \end{itemize}
        \item<3-> Otherwise, switches to the C stack to be able to execute C code
        \item<4-> Calls \funcname{caml\_stash\_backtrace}, a C function provided by the runtime\ignorespaces
          \onslide<6->{, with
            \begin{itemize}
              \item \funcarg{exn}: exception value
              \item \funcarg{pc}: code address of the raising point
              \item \funcarg{sp}: stack address of the raising function
              \item \funcarg{trapsp}: stack address of the next handler
            \end{itemize}
          }
      \end{itemize}
      \bigskip
      \mintocaml[firstline=9,lastline=13,highlightlines=12]{../src/backtrace/backtrace.ml}
    \end{column}
    \begin{column}{0.5\textwidth}
      \raggedleft
      \onslide*<1,3-4>{
        \mintc[firstline=190,lastline=192]{../ocaml/runtime/amd64.S}
        \mintc[firstline=234,lastline=237]{../ocaml/runtime/amd64.S}
      }
      \onslide*<2>{
        \mintc[firstline=190,lastline=192,highlightlines={191-192}]{../ocaml/runtime/amd64.S}
        \mintc[firstline=234,lastline=237,highlightlines={235-237}]{../ocaml/runtime/amd64.S}
      }
      \onslide*<1>{\listCamlRaiseExn[highlightlines=705]{}}%
      \onslide*<2>{\listCamlRaiseExn[highlightlines={707-708}]{}}%
      \onslide*<3>{\listCamlRaiseExn[highlightlines={712-714}]{}}%
      \onslide*<4>{\listCamlRaiseExn[highlightlines={715-720}]{}}%
      \onslide*<5>{\providecommand\step{1}\input{diagrams/backtrace/backtrace.tex}}%
      \onslide*<6>{\providecommand\step{2}\input{diagrams/backtrace/backtrace.tex}}%
    \end{column}
  \end{columns}
\end{frame}

\newcommand\listStashBacktrace[1][]{
  \listc[firstline=91,lastline=120,#1]{../ocaml/runtime/backtrace\_nat.c}{runtime/backtrace\_nat.c:91}}

\begin{frame}{Backtraces}{Introducing \funcname{caml\_stash\_backtrace}}
  \begin{columns}[c]
    \begin{column}{0.5\textwidth}
      \minipage[c][0.66\textheight][s]{\columnwidth}
      \begin{itemize}
        \item<1-> A C function provided by the runtime (specific to native code)
        \item<2-> It scans the stack, between the stack pointer at the raising point up to the next trap, in order to collect the names of the detected function frames
          \onslide*<2>{
            \begin{itemize}
              \item And more information, like line number, column number...
            \end{itemize}
          }
        \item<3-> It uses the same technique and information as the Garbage Collector (GC) uses to scan the values live on the stack
          \onslide*<4>{
            \begin{itemize}
              \item \typename{frame\_descr} are at the core of stack unwinding
            \end{itemize}
          }
      \end{itemize}
      \vfill
      \mintocaml[firstline=9,lastline=13,highlightlines=12]{../src/backtrace/backtrace.ml}
      \endminipage
    \end{column}
    \begin{column}{0.5\textwidth}
      \onslide*<1>{\listStashBacktrace[highlightlines=91]{}}
      \onslide*<2>{\listStashBacktrace[highlightlines={91,117-118}]{}}
      \onslide*<3->{\listStashBacktrace[highlightlines={94,106,109-110}]{}}
    \end{column}
  \end{columns}
\end{frame}

\newcommand\listFrameDescr[1][]{
  \listocaml[firstline=111,lastline=124,#1]{../ocaml/asmcomp/emitaux.ml}{asmcomp/emitaux.ml:111}}
\newcommand\listDebugInfo[1][]{
  \listocaml[firstline=38,lastline=49,#1]{../ocaml/lambda/debuginfo.mli}{lambda/debuginfo.mli:38}}

\begin{frame}{Backtraces}{\typename{frame\_descr} and \typename{frame\_debuginfo} in a nutshell}
  \begin{columns}[c]
    \begin{column}{0.5\textwidth}
      \begin{itemize}
        \item<1-> For every return address in an OCaml program, there is a associated \typename{frame\_descr}
        \item<2-> A \typename{frame\_descr} specifies
        \begin{itemize}
          \item<2-> The size of the function stack frame
          \item<3-> Where live values are on the stack (so that the GC can mark them as live and prevent from being swept)
        \end{itemize}
        \item<4-> When compiled with debug information, \typename{frame\_descr} will be linked to a \typename{frame\_debuginfo}
        \begin{itemize}
          \item<5-> Contains information about the position in the source code (file name, line and column number)
        \end{itemize}
      \end{itemize}
    \end{column}
    \begin{column}{0.5\textwidth}
        \onslide*<1>{\listFrameDescr[highlightlines={118-122}]{}}
        \onslide*<2>{\listFrameDescr[highlightlines={120}]{}}
        \onslide*<3>{\listFrameDescr[highlightlines={121}]{}}
        \onslide*<4->{\listFrameDescr[highlightlines={122}]{}}
        \medskip
        \onslide*<1-3>{\listDebugInfo{}}
        \onslide*<4>{\listDebugInfo[highlightlines={38,49}]{}}
        \onslide*<5>{\listDebugInfo[highlightlines={39-46}]{}}
    \end{column}
  \end{columns}
\end{frame}

\begin{frame}{Backtraces}{Scanning the stack with \typename{frame\_descr}}
  \begin{columns}[c]
    \begin{column}{0.5\textwidth}
      \begin{itemize}
        \item Given the return address of the code that raises the exception by calling \funcname{caml\_raise\_exn} (\funcarg{pc} argument of \funcname{caml\_stash\_backtrace})
        \item And the position of the stack pointer (\funcarg{sp} argument of \funcname{caml\_stash\_backtrace})
        \item The frame size given by \typename{frame\_descr}.\funcarg{frame\_size}, allows to find the end of the previous frame
        \item And the return address of the caller of this function
        \bigskip
        \onslide<3->{
        \item Note that traps (here the reddish \funcname{ExnA}, \funcname{ExnB} and \funcname{ExnC} blocks) belong to the frame that installed them, and so the frame size changes when traps are removed
        }
      \end{itemize}
    \end{column}
    \begin{column}{0.5\textwidth}
      \raggedleft
      \onslide*<1>{\providecommand\step{2}\input{diagrams/backtrace/backtrace.tex}}%
      \onslide*<2>{\providecommand\step{1}\input{diagrams/backtrace/scanstack.tex}}%
      \onslide*<3>{\providecommand\step{2}\input{diagrams/backtrace/scanstack.tex}}%
      \onslide*<4>{\providecommand\step{3}\input{diagrams/backtrace/scanstack.tex}}%
      \onslide*<5>{\providecommand\step{4}\input{diagrams/backtrace/scanstack.tex}}%
      \onslide*<6>{\providecommand\step{5}\input{diagrams/backtrace/scanstack.tex}}%
    \end{column}
  \end{columns}
\end{frame}

% Duplicate of previous-previous slide
\begin{frame}{Backtraces}{Continuing on \funcname{caml\_stash\_backtrace}}
  \begin{columns}[c]
    \begin{column}{0.5\textwidth}
      \minipage[c][0.75\textheight][s]{\columnwidth}
      \begin{itemize}
          \color{gray}
        \item A C function provided by the runtime (specific to native code)
        \item It scans the stack, between the stack pointer at the raising point up to the next trap, in order to collect the names of the detected function frames
        \item It uses the same technique and information as the Garbage Collector (GC) uses to scan the values live on the stack
      \end{itemize}
      \begin{itemize}
        \item For each function frame detected, store a pointer to the \typename{frame\_descr} in the \localname{domain\_state->backtrace\_buffer} array, effectively constructing the backtrace
      \end{itemize}
      \onslide<2->{
        \begin{itemize}
            \smallskip
          \item Constructed backtrace:
            \funcname{definitely\_raise},
            \onslide<3->{\funcname{probably\_raise}}
        \end{itemize}
      }
      \vfill
      \mintocaml[firstline=9,lastline=13,highlightlines=12]{../src/backtrace/backtrace.ml}
      \endminipage
    \end{column}
    \begin{column}{0.5\textwidth}
      \raggedleft
      \onslide*<1>{\listStashBacktrace[highlightlines={114-115}]{}}
      \onslide*<2>{\providecommand\step{3}\input{diagrams/backtrace/backtrace.tex}}%
      \onslide*<3>{\providecommand\step{4}\input{diagrams/backtrace/backtrace.tex}}%
    \end{column}
  \end{columns}
\end{frame}

\begin{frame}{Backtraces}{Back to \funcname{caml\_raise\_exn}}
  \begin{columns}[c]
    \begin{column}{0.5\textwidth}
      \minipage[c][0.66\textheight][s]{\columnwidth}
      \begin{itemize}\color{gray}
        \item Checks wheteher exceptions backtrace should be recorded, by checking the \funcarg{domain\_state->backtrace\_active} value
        \item Switches to the C stack to be able to execute C code
        \item Calls \funcname{caml\_stash\_backtrace}, to collect the backtrace up to the next trap
      \end{itemize}
      \onslide*<2->{
        \begin{itemize}
          \item<2-> Executes the exception handler in \funcname{probably\_raise} with \funcname{RESTORE\_EXN\_HANDLER\_OCAML}
            \begin{itemize}
              \item Set \regname{RSP} to \localname{domain\_state->exn\_handler} value
              \item Restore \localname{domain\_state->exn\_handler} from trap
              \item Jump to the handler code with \funcname{ret}
            \end{itemize}
        \end{itemize}
      }
      \vfill
      \onslide*<1>{\mintocaml[firstline=9,lastline=13,highlightlines=12]{../src/backtrace/backtrace.ml}}
      \onslide*<2->{\mintocaml[firstline=15,lastline=21,highlightlines={19-21}]{../src/backtrace/backtrace.ml}}
      \endminipage
    \end{column}
    \begin{column}{0.5\textwidth}
      \centering
      \onslide*<1>{
        \mintc[firstline=190,lastline=192]{../ocaml/runtime/amd64.S}
        \mintc[firstline=234,lastline=237]{../ocaml/runtime/amd64.S}
      }
      \onslide*<2>{
        \mintc[firstline=190,lastline=192,highlightlines={191-192}]{../ocaml/runtime/amd64.S}
        \mintc[firstline=234,lastline=237,highlightlines={235-237}]{../ocaml/runtime/amd64.S}
      }
      \onslide*<1>{\listCamlRaiseExn[highlightlines=720]{}}
      \onslide*<2>{\listCamlRaiseExn[highlightlines=722]{}}
      \onslide*<3->{\providecommand\step{5}\input{diagrams/backtrace/backtrace.tex}}
    \end{column}
  \end{columns}
\end{frame}

\begin{frame}{Backtraces}{\funcname{caml\_probably\_raise}'s handler doesn't catch \typename{ExnC}}
  \begin{columns}[c]
    \begin{column}{0.45\textwidth}
      \minipage[c][0.75\textheight][s]{\columnwidth}
      \begin{itemize}
        \item<1-> \funcname{match \dots with}\footnotemark pattern matching can also match exceptions
        \item<1-> The CMM of \funcname{probably\_raise} shows that this \funcname{match \dots with} is implemented with a pattern matching and a \funcname{try \dots with}
        \item<1-> But this one only matches on \typename{ExnA} exception
        \item<2-> Since the pattern matching fails to catch the exception, \funcname{reraise} is called with our current \typename{ExnC} exception
        \item<3-> Disassembly shows that \funcname{reraise} is actually a call to \funcname{caml\_reraise\_exn}, which is also an assembly function provided by the runtime
      \end{itemize}
      \vfill
      \mintocaml[firstline=15,lastline=21,highlightlines={18-21}]{../src/backtrace/backtrace.ml}
      \endminipage
    \end{column}
    \begin{column}{0.55\textwidth}
      \centering
      \onslide*<1>{\mintcmm[highlightlines={10-18}]{../src/backtrace/camlBacktrace\_\_probably\_raise\_351.cmm}}
      \onslide*<2>{\listcmm[highlightlines=18]{../src/backtrace/camlBacktrace\_\_probably\_raise_351.cmm}{CMM of probably\_raise}}
      \onslide*<3>{\mintobjdump[firstline=5,lastline=35,highlightlines=31]{../src/backtrace/camlBacktrace\_\_probably\_raise_351.objdump}}
    \end{column}
  \end{columns}
  \only<1>{\footnotetext[1]{https://blog.janestreet.com/pattern-matching-and-exception-handling-unite/}}
\end{frame}

\newcommand\listCamlReraiseExn[1][]{
  \listasm[firstline=727,lastline=734,#1]{../ocaml/runtime/amd64.S}{runtime/amd64.S:727}}

\begin{frame}{Backtraces}{Introducing \funcname{caml\_reraise\_exn}}
  \begin{columns}[c]
    \begin{column}{0.5\textwidth}
      \minipage[c][0.66\textheight][s]{\columnwidth}
      \begin{itemize}
        \item<1-> The exception have to be reraised to the next handler
        \item<2-> First, checks if the backtrace should be collected with \funcarg{domain\_state->backtrace\_active}
        \only<3>{
        \begin{itemize}
            \item If not, behaves like a \funcname{raise\_notrace} with \funcname{RESTORE\_EXN\_HANDLER\_OCAML}
        \end{itemize}
        }
        \item<4-> Jumps into \funcname{caml\_raise\_exn}
        \item<5-> Calls \funcname{caml\_stash\_backtrace} again, to scan the next part of the stack: from the trap just pop-ed to the next trap
      \end{itemize}
      \vfill
      \mintocaml[firstline=15,lastline=21,highlightlines={18-21}]{../src/backtrace/backtrace.ml}
      \endminipage
    \end{column}
    \begin{column}{0.5\textwidth}
      \raggedleft
      \onslide*<1>{\listCamlReraiseExn{}}
      \onslide*<2>{\listCamlReraiseExn[highlightlines=729]{}}
      \onslide*<3>{
        \mintc[firstline=190,lastline=192,highlightlines={191-192}]{../ocaml/runtime/amd64.S}
        \mintc[firstline=234,lastline=237,highlightlines={235-237}]{../ocaml/runtime/amd64.S}
        \medskip
        \listCamlReraiseExn[highlightlines=731]{}
      }
      \onslide*<4-5>{
        % TODO refactor with \listCamlReraiseExn
        \mintasm[firstline=727,lastline=734,highlightlines=730]{../ocaml/runtime/amd64.S}
        \medskip
      }
      \onslide*<4>{\listCamlRaiseExn[highlightlines=711]{}}
      \onslide*<5>{\listCamlRaiseExn[highlightlines=720]{}}
    \end{column}
  \end{columns}
\end{frame}

\begin{frame}{Backtraces}{Backtrace collection continues}
  \begin{columns}[c]
    \begin{column}{0.55\textwidth}
      \minipage[c][0.75\textheight][s]{\columnwidth}
      \begin{itemize}
        \item<1-> \funcname{caml\_stash\_backtrace} scans the older part of the stack, up to the next trap
        \item<6-> Controls jumps to the nested exception handler in \funcname{maybe\_raise}, which matches only on \typename{ExnB}
        \item<7-> \funcname{caml\_reraise\_exn} is called again, backtrace collection resumes with \funcname{caml\_stash\_backtrace}
      \end{itemize}
      \medskip
      \begin{itemize}
        \item<2-> Constructed backtrace:
        \begin{itemize}
          \item <2-> \funcname{definitely\_raise}
          \item <2-> \funcname{probably\_raise}
          \item <3-> \funcname{probably\_raise} (re-raise)
          \item <4-> \funcname{maybe\_raise}
          \item <5-> \funcname{main}
          \item <8-> \funcname{main} (re-raise)
        \end{itemize}
      \end{itemize}
      \vfill
      \onslide*<1-5>{\mintocaml[firstline=15,lastline=21]{../src/backtrace/backtrace.ml}}
      \onslide*<6>{\mintocaml[firstline=28,lastline=34,highlightlines=31]{../src/backtrace/backtrace.ml}}
      \onslide*<7->{\mintocaml[firstline=28,lastline=34,highlightlines=33]{../src/backtrace/backtrace.ml}}
      \endminipage
    \end{column}
    \begin{column}{0.45\textwidth}
      \raggedleft
      \onslide*<1>{\providecommand\step{6}\input{diagrams/backtrace/backtrace.tex}}%
      \onslide*<2>{\providecommand\step{6}\input{diagrams/backtrace/backtrace.tex}}%
      \onslide*<3>{\providecommand\step{7}\input{diagrams/backtrace/backtrace.tex}}%
      \onslide*<4>{\providecommand\step{8}\input{diagrams/backtrace/backtrace.tex}}%
      \onslide*<5>{\providecommand\step{9}\input{diagrams/backtrace/backtrace.tex}}%
      \onslide*<6>{\providecommand\step{10}\input{diagrams/backtrace/backtrace.tex}}%
      \onslide*<7>{\providecommand\step{11}\input{diagrams/backtrace/backtrace.tex}}%
      \onslide*<8>{\providecommand\step{12}\input{diagrams/backtrace/backtrace.tex}}%
      \onslide*<9>{\providecommand\step{13}\input{diagrams/backtrace/backtrace.tex}}%
    \end{column}
  \end{columns}
\end{frame}

\begin{frame}{Backtraces}{Printing the backtrace}
  \begin{columns}[c]
    \begin{column}{0.45\textwidth}
      \minipage[c][0.66\textheight][s]{\columnwidth}
      \begin{itemize}
        \item<1-> The exception backtrace can now be printed to \funcarg{stdout} with \funcname{Printexc.print_backtrace}
        \item<2-> First the exception backtrace is retrieved into an OCaml array with \funcname{get_raw_backtrace }
        \item<3-> Which is implemented as the \funcname{caml_get_exception_raw_backtrace} C function in the runtime
        \item<4-> And finally printed with \funcname{print_exception_backtrace}
      \end{itemize}
      \vfill
      \mintocaml[firstline=28,lastline=34,highlightlines=33]{../src/backtrace/backtrace.ml}
      \endminipage
    \end{column}
    \begin{column}{0.55\textwidth}
      \onslide*<1,2>{
        \mintocaml[firstline=100,lastline=101]{../ocaml/stdlib/printexc.ml}
      }
      \only<1>{
        \listocaml[firstline=170,lastline=172]{../ocaml/stdlib/printexc.ml}{stdlib/printexc.ml:170}
      }
      \only<2>{
        \listocaml[firstline=170,lastline=172,highlightlines=172]{../ocaml/stdlib/printexc.ml}{stdlib/printexc.ml:170}
      }
      \only<3>{
        \mintc[firstline=154,lastline=156]{../ocaml/runtime/backtrace.c}
        \listc[firstline=171,lastline=192,highlightlines={184-187}]{../ocaml/runtime/backtrace.c}{runtime/backtrace.c:154}
      }
      \only<4>{
        \listocaml[firstline=155,lastline=165]{../ocaml/stdlib/printexc.ml}{stdlib/printexc.ml:55}
      }
    \end{column}
  \end{columns}
\end{frame}


\subsection{The multiple kinds of raise}
\frameSubsection{}{}

\begin{frame}{Backtraces}{When backtraces are not needed}
  \begin{columns}[c]
    \begin{column}{0.45\textwidth}
      \minipage[c][0.66\textheight][s]{\columnwidth}
      \begin{itemize}
        \item<1-> What if the backtrace for an exception is never needed?
        \item<2-> It's possible to raise the exception explicitly with \funcname{raise\_notrace} to make raising as cheap as possible
        \item<3-> An example of use in the \funcname{dynlink} library of the OCaml compiler
      \end{itemize}
      \vfill
      \onslide<3->{
      \listocaml[firstline=205,lastline=209,highlightlines=207]{../ocaml/otherlibs/dynlink/dynlink\_compilerlibs/misc.ml}{otherlibs/dynlink/dynlink\_compilerlibs/misc.ml:205}
      }
      \endminipage
    \end{column}
    \begin{column}{0.55\textwidth}
    \only<4>{
      \mintcmm[highlightlines=3]{../src/notrace/camlNotrace\_\_fun_347.cmm}
      \listcmm{../src/notrace/camlNotrace\_\_all\_somes\_327.cmm}{all\_somes}
    }
    \onslide*<5->{
      \listobjdump[highlightlines={6-9}]{../src/notrace/camlNotrace\_\_fun_347.objdump}{anonymous function from \funcname{all\_somes}}
    }
    \end{column}
  \end{columns}
\end{frame}

\begin{frame}{Backtraces}{Summary table of the multiple kinds of raise}
  \begin{itemize}
    \item When debugging information is enabled and if the backtrace of an exception isn't needed, it's possible to raise with \funcname{raise\_notrace} for performance
    \item When debugging information is \emph{not} enabled, the compiler optimises \funcname{raise} calls to inline assembly (same effect as using \funcname{raise\_notrace})
  \end{itemize}
  \bigskip
  \centering
  \begin{tabular}{| l | c | c | c | c |}
    \hline
    \parbox{10em}{Debug information \\ (\funcarg{-g} flag)}  & \multicolumn{2}{|c|}{Disabled}                                     & \multicolumn{2}{|c|}{Enabled} \\
    \hline
    Raise kind                                               & \funcname{raise} & \funcname{raise\_notrace}                       & \funcname{raise} & \funcname{raise\_notrace} \\
    \hline
    Compilation                                              & \parbox{8em}{inline assembly\\ (optimization)} & inline assembly   & \funcname{caml\_raise\_exn} & inline assembly \\
    \hline
  \end{tabular}
\end{frame}


%
% Takeaway #5
%
\subsection*{Takeaway \#5}
\frameSubsectionTakeaway{}
\againframe<5>{takeaway}
}%
      \onslide*<4>{\providecommand\step{8}%
% Backtraces
%
\subsection{One advantage of raising with debugging information}
\frameSubsection{}{}

\begin{frame}{Backtraces}{Debugging information \& source code}
  \begin{columns}[c]
    \begin{column}{0.5\textwidth}
      \begin{itemize}
        \item Raising an exception is fun and games, but how do we know where the exception originated? $\rightarrow$ With the backtrace associated with the exception!
        \item In order to get a backtrace from the point where an exception was raised, it's necessary to compile with debugging information enabled (\funcarg{-g} flag for the compiler)
        \item Same example as in the `Nested handlers' section
      \end{itemize}
    \end{column}
    \begin{column}{0.5\textwidth}
      \listocaml[firstline=9,lastline=34]{../src/backtrace/backtrace.ml}{backtrace/backtrace.ml}
    \end{column}
  \end{columns}
\end{frame}

\begin{frame}{Backtraces}{CMM and disassembly of \funcname{definitely\_raise}}
  \begin{columns}[c]
    \begin{column}{0.5\textwidth}
      \minipage[c][0.66\textheight][s]{\columnwidth}
      \begin{itemize}
        \item<1-> An effect of compiling with debugging information (\funcarg{-g}): the CMM no longer uses \funcname{raise\_notrace} to raise the exception but \funcname{raise} instead
        \item<2-> Disassembly shows that CMM \funcname{raise} is actually a call to \funcname{caml\_raise\_exn}, a function in assembler provided by the runtime
      \end{itemize}
      \vfill
      \mintocaml[firstline=9,lastline=13,highlightlines=12]{../src/backtrace/backtrace.ml}
      \endminipage
    \end{column}
    \begin{column}{0.5\textwidth}
      \onslide*<1>{
        \mintcmm[highlightlines=5]{../src/backtrace/camlBacktrace\_\_definitely\_raise\_347.cmm}
      }
      \onslide*<2>{
        \mintobjdump[firstline=2,highlightlines=14]{../src/backtrace/camlBacktrace\_\_definitely\_raise\_347.objdump}
      }
    \end{column}
  \end{columns}
\end{frame}

\begin{frame}{Backtraces}{Introducing \funcname{caml\_raise\_exn}}
  \begin{columns}[c]
    \begin{column}{0.5\textwidth}
      \minipage[c][0.66\textheight][s]{\columnwidth}
      \onslide*<1>{
        \begin{itemize}
          \item Raising the exception is performed using \funcname{caml\_raise\_exn} which is
            \begin{itemize}
              \item An assembly function provided by the runtime
              \item Architecture specific
            \end{itemize}
          \item Let's see what it does differently from \funcname{raise\_notrace}\dots
          \item We are about to raise \typename{ExnC} with \funcname{caml\_raise\_exn} from \funcname{definitely\_raise}
        \end{itemize}
      }
      \onslide*<2>{
        \mintobjdump[firstline=2,highlightlines=14]{../src/backtrace/camlBacktrace\_\_definitely\_raise\_347.objdump}
      }
      \vfill
      \mintocaml[firstline=9,lastline=13,highlightlines=12]{../src/backtrace/backtrace.ml}
      \endminipage
    \end{column}
    \begin{column}{0.5\textwidth}
      \centering
      \providecommand\step{1}\input{diagrams/backtrace/backtrace.tex}
    \end{column}
  \end{columns}
\end{frame}

\begin{frame}{Backtraces}{But before that, we need more fields from \typename{caml\_domain\_state}}
  \begin{columns}
    \begin{column}{0.5\textwidth}
      \begin{itemize}
        \item Remember \typename{caml\_domain\_state} the per-domain runtime state?
        \item Some other members are of interest to us now\dots
        \item \localname{backtrace\_active} is used to know if the runtime should collect backtraces
        \item \localname{backtrace\_active} can be set by
          \begin{itemize}
            \item The \funcarg{b} flag in the \funcname{OCAMLRUNPARAM} environment variable
            \item \mintinline{ocaml}{Printexc.record_backtrace : bool -> unit} \footnotemark
          \end{itemize}
      \end{itemize}
    \end{column}
    \begin{column}{0.5\textwidth}
      \begin{listing}
        \mintc[highlightlines={9-11}]{src/ocaml/domain\_state\_backtrace.h}
        \caption{runtime/caml/domain\_state.h}
      \end{listing}
    \end{column}
  \end{columns}
  \footnotetext[1]{https://v2.ocaml.org/api/Printexc.html}
\end{frame}

\newcommand\listCamlRaiseExn[1][]{
  \listasm[firstline=702,lastline=725,#1]{../ocaml/runtime/amd64.S}{runtime/amd64.S:702}}

\begin{frame}{Backtraces}{Walk through \funcname{caml\_raise\_exn}}
  \begin{columns}[T]
    \begin{column}{0.5\textwidth}
      \begin{itemize}
        \item<1-> First checks whether exceptions backtrace should be recorded
        \item<1-> By checking the \funcarg{domain\_state->backtrace\_active} value
        \item<2-> If not, acts as \funcname{raise\_notrace} with \funcname{RESTORE\_EXN\_HANDLER\_OCAML}
          \begin{itemize}
            \item Set \regname{RSP} to \localname{domain\_state->exn\_handler} value
            \item Restore \localname{domain\_state->exn\_handler} from trap
            \item Jump with \funcname{ret} (same effect as using \funcname{pop} and \funcname{jmp})
          \end{itemize}
        \item<3-> Otherwise, switches to the C stack to be able to execute C code
        \item<4-> Calls \funcname{caml\_stash\_backtrace}, a C function provided by the runtime\ignorespaces
          \onslide<6->{, with
            \begin{itemize}
              \item \funcarg{exn}: exception value
              \item \funcarg{pc}: code address of the raising point
              \item \funcarg{sp}: stack address of the raising function
              \item \funcarg{trapsp}: stack address of the next handler
            \end{itemize}
          }
      \end{itemize}
      \bigskip
      \mintocaml[firstline=9,lastline=13,highlightlines=12]{../src/backtrace/backtrace.ml}
    \end{column}
    \begin{column}{0.5\textwidth}
      \raggedleft
      \onslide*<1,3-4>{
        \mintc[firstline=190,lastline=192]{../ocaml/runtime/amd64.S}
        \mintc[firstline=234,lastline=237]{../ocaml/runtime/amd64.S}
      }
      \onslide*<2>{
        \mintc[firstline=190,lastline=192,highlightlines={191-192}]{../ocaml/runtime/amd64.S}
        \mintc[firstline=234,lastline=237,highlightlines={235-237}]{../ocaml/runtime/amd64.S}
      }
      \onslide*<1>{\listCamlRaiseExn[highlightlines=705]{}}%
      \onslide*<2>{\listCamlRaiseExn[highlightlines={707-708}]{}}%
      \onslide*<3>{\listCamlRaiseExn[highlightlines={712-714}]{}}%
      \onslide*<4>{\listCamlRaiseExn[highlightlines={715-720}]{}}%
      \onslide*<5>{\providecommand\step{1}\input{diagrams/backtrace/backtrace.tex}}%
      \onslide*<6>{\providecommand\step{2}\input{diagrams/backtrace/backtrace.tex}}%
    \end{column}
  \end{columns}
\end{frame}

\newcommand\listStashBacktrace[1][]{
  \listc[firstline=91,lastline=120,#1]{../ocaml/runtime/backtrace\_nat.c}{runtime/backtrace\_nat.c:91}}

\begin{frame}{Backtraces}{Introducing \funcname{caml\_stash\_backtrace}}
  \begin{columns}[c]
    \begin{column}{0.5\textwidth}
      \minipage[c][0.66\textheight][s]{\columnwidth}
      \begin{itemize}
        \item<1-> A C function provided by the runtime (specific to native code)
        \item<2-> It scans the stack, between the stack pointer at the raising point up to the next trap, in order to collect the names of the detected function frames
          \onslide*<2>{
            \begin{itemize}
              \item And more information, like line number, column number...
            \end{itemize}
          }
        \item<3-> It uses the same technique and information as the Garbage Collector (GC) uses to scan the values live on the stack
          \onslide*<4>{
            \begin{itemize}
              \item \typename{frame\_descr} are at the core of stack unwinding
            \end{itemize}
          }
      \end{itemize}
      \vfill
      \mintocaml[firstline=9,lastline=13,highlightlines=12]{../src/backtrace/backtrace.ml}
      \endminipage
    \end{column}
    \begin{column}{0.5\textwidth}
      \onslide*<1>{\listStashBacktrace[highlightlines=91]{}}
      \onslide*<2>{\listStashBacktrace[highlightlines={91,117-118}]{}}
      \onslide*<3->{\listStashBacktrace[highlightlines={94,106,109-110}]{}}
    \end{column}
  \end{columns}
\end{frame}

\newcommand\listFrameDescr[1][]{
  \listocaml[firstline=111,lastline=124,#1]{../ocaml/asmcomp/emitaux.ml}{asmcomp/emitaux.ml:111}}
\newcommand\listDebugInfo[1][]{
  \listocaml[firstline=38,lastline=49,#1]{../ocaml/lambda/debuginfo.mli}{lambda/debuginfo.mli:38}}

\begin{frame}{Backtraces}{\typename{frame\_descr} and \typename{frame\_debuginfo} in a nutshell}
  \begin{columns}[c]
    \begin{column}{0.5\textwidth}
      \begin{itemize}
        \item<1-> For every return address in an OCaml program, there is a associated \typename{frame\_descr}
        \item<2-> A \typename{frame\_descr} specifies
        \begin{itemize}
          \item<2-> The size of the function stack frame
          \item<3-> Where live values are on the stack (so that the GC can mark them as live and prevent from being swept)
        \end{itemize}
        \item<4-> When compiled with debug information, \typename{frame\_descr} will be linked to a \typename{frame\_debuginfo}
        \begin{itemize}
          \item<5-> Contains information about the position in the source code (file name, line and column number)
        \end{itemize}
      \end{itemize}
    \end{column}
    \begin{column}{0.5\textwidth}
        \onslide*<1>{\listFrameDescr[highlightlines={118-122}]{}}
        \onslide*<2>{\listFrameDescr[highlightlines={120}]{}}
        \onslide*<3>{\listFrameDescr[highlightlines={121}]{}}
        \onslide*<4->{\listFrameDescr[highlightlines={122}]{}}
        \medskip
        \onslide*<1-3>{\listDebugInfo{}}
        \onslide*<4>{\listDebugInfo[highlightlines={38,49}]{}}
        \onslide*<5>{\listDebugInfo[highlightlines={39-46}]{}}
    \end{column}
  \end{columns}
\end{frame}

\begin{frame}{Backtraces}{Scanning the stack with \typename{frame\_descr}}
  \begin{columns}[c]
    \begin{column}{0.5\textwidth}
      \begin{itemize}
        \item Given the return address of the code that raises the exception by calling \funcname{caml\_raise\_exn} (\funcarg{pc} argument of \funcname{caml\_stash\_backtrace})
        \item And the position of the stack pointer (\funcarg{sp} argument of \funcname{caml\_stash\_backtrace})
        \item The frame size given by \typename{frame\_descr}.\funcarg{frame\_size}, allows to find the end of the previous frame
        \item And the return address of the caller of this function
        \bigskip
        \onslide<3->{
        \item Note that traps (here the reddish \funcname{ExnA}, \funcname{ExnB} and \funcname{ExnC} blocks) belong to the frame that installed them, and so the frame size changes when traps are removed
        }
      \end{itemize}
    \end{column}
    \begin{column}{0.5\textwidth}
      \raggedleft
      \onslide*<1>{\providecommand\step{2}\input{diagrams/backtrace/backtrace.tex}}%
      \onslide*<2>{\providecommand\step{1}\input{diagrams/backtrace/scanstack.tex}}%
      \onslide*<3>{\providecommand\step{2}\input{diagrams/backtrace/scanstack.tex}}%
      \onslide*<4>{\providecommand\step{3}\input{diagrams/backtrace/scanstack.tex}}%
      \onslide*<5>{\providecommand\step{4}\input{diagrams/backtrace/scanstack.tex}}%
      \onslide*<6>{\providecommand\step{5}\input{diagrams/backtrace/scanstack.tex}}%
    \end{column}
  \end{columns}
\end{frame}

% Duplicate of previous-previous slide
\begin{frame}{Backtraces}{Continuing on \funcname{caml\_stash\_backtrace}}
  \begin{columns}[c]
    \begin{column}{0.5\textwidth}
      \minipage[c][0.75\textheight][s]{\columnwidth}
      \begin{itemize}
          \color{gray}
        \item A C function provided by the runtime (specific to native code)
        \item It scans the stack, between the stack pointer at the raising point up to the next trap, in order to collect the names of the detected function frames
        \item It uses the same technique and information as the Garbage Collector (GC) uses to scan the values live on the stack
      \end{itemize}
      \begin{itemize}
        \item For each function frame detected, store a pointer to the \typename{frame\_descr} in the \localname{domain\_state->backtrace\_buffer} array, effectively constructing the backtrace
      \end{itemize}
      \onslide<2->{
        \begin{itemize}
            \smallskip
          \item Constructed backtrace:
            \funcname{definitely\_raise},
            \onslide<3->{\funcname{probably\_raise}}
        \end{itemize}
      }
      \vfill
      \mintocaml[firstline=9,lastline=13,highlightlines=12]{../src/backtrace/backtrace.ml}
      \endminipage
    \end{column}
    \begin{column}{0.5\textwidth}
      \raggedleft
      \onslide*<1>{\listStashBacktrace[highlightlines={114-115}]{}}
      \onslide*<2>{\providecommand\step{3}\input{diagrams/backtrace/backtrace.tex}}%
      \onslide*<3>{\providecommand\step{4}\input{diagrams/backtrace/backtrace.tex}}%
    \end{column}
  \end{columns}
\end{frame}

\begin{frame}{Backtraces}{Back to \funcname{caml\_raise\_exn}}
  \begin{columns}[c]
    \begin{column}{0.5\textwidth}
      \minipage[c][0.66\textheight][s]{\columnwidth}
      \begin{itemize}\color{gray}
        \item Checks wheteher exceptions backtrace should be recorded, by checking the \funcarg{domain\_state->backtrace\_active} value
        \item Switches to the C stack to be able to execute C code
        \item Calls \funcname{caml\_stash\_backtrace}, to collect the backtrace up to the next trap
      \end{itemize}
      \onslide*<2->{
        \begin{itemize}
          \item<2-> Executes the exception handler in \funcname{probably\_raise} with \funcname{RESTORE\_EXN\_HANDLER\_OCAML}
            \begin{itemize}
              \item Set \regname{RSP} to \localname{domain\_state->exn\_handler} value
              \item Restore \localname{domain\_state->exn\_handler} from trap
              \item Jump to the handler code with \funcname{ret}
            \end{itemize}
        \end{itemize}
      }
      \vfill
      \onslide*<1>{\mintocaml[firstline=9,lastline=13,highlightlines=12]{../src/backtrace/backtrace.ml}}
      \onslide*<2->{\mintocaml[firstline=15,lastline=21,highlightlines={19-21}]{../src/backtrace/backtrace.ml}}
      \endminipage
    \end{column}
    \begin{column}{0.5\textwidth}
      \centering
      \onslide*<1>{
        \mintc[firstline=190,lastline=192]{../ocaml/runtime/amd64.S}
        \mintc[firstline=234,lastline=237]{../ocaml/runtime/amd64.S}
      }
      \onslide*<2>{
        \mintc[firstline=190,lastline=192,highlightlines={191-192}]{../ocaml/runtime/amd64.S}
        \mintc[firstline=234,lastline=237,highlightlines={235-237}]{../ocaml/runtime/amd64.S}
      }
      \onslide*<1>{\listCamlRaiseExn[highlightlines=720]{}}
      \onslide*<2>{\listCamlRaiseExn[highlightlines=722]{}}
      \onslide*<3->{\providecommand\step{5}\input{diagrams/backtrace/backtrace.tex}}
    \end{column}
  \end{columns}
\end{frame}

\begin{frame}{Backtraces}{\funcname{caml\_probably\_raise}'s handler doesn't catch \typename{ExnC}}
  \begin{columns}[c]
    \begin{column}{0.45\textwidth}
      \minipage[c][0.75\textheight][s]{\columnwidth}
      \begin{itemize}
        \item<1-> \funcname{match \dots with}\footnotemark pattern matching can also match exceptions
        \item<1-> The CMM of \funcname{probably\_raise} shows that this \funcname{match \dots with} is implemented with a pattern matching and a \funcname{try \dots with}
        \item<1-> But this one only matches on \typename{ExnA} exception
        \item<2-> Since the pattern matching fails to catch the exception, \funcname{reraise} is called with our current \typename{ExnC} exception
        \item<3-> Disassembly shows that \funcname{reraise} is actually a call to \funcname{caml\_reraise\_exn}, which is also an assembly function provided by the runtime
      \end{itemize}
      \vfill
      \mintocaml[firstline=15,lastline=21,highlightlines={18-21}]{../src/backtrace/backtrace.ml}
      \endminipage
    \end{column}
    \begin{column}{0.55\textwidth}
      \centering
      \onslide*<1>{\mintcmm[highlightlines={10-18}]{../src/backtrace/camlBacktrace\_\_probably\_raise\_351.cmm}}
      \onslide*<2>{\listcmm[highlightlines=18]{../src/backtrace/camlBacktrace\_\_probably\_raise_351.cmm}{CMM of probably\_raise}}
      \onslide*<3>{\mintobjdump[firstline=5,lastline=35,highlightlines=31]{../src/backtrace/camlBacktrace\_\_probably\_raise_351.objdump}}
    \end{column}
  \end{columns}
  \only<1>{\footnotetext[1]{https://blog.janestreet.com/pattern-matching-and-exception-handling-unite/}}
\end{frame}

\newcommand\listCamlReraiseExn[1][]{
  \listasm[firstline=727,lastline=734,#1]{../ocaml/runtime/amd64.S}{runtime/amd64.S:727}}

\begin{frame}{Backtraces}{Introducing \funcname{caml\_reraise\_exn}}
  \begin{columns}[c]
    \begin{column}{0.5\textwidth}
      \minipage[c][0.66\textheight][s]{\columnwidth}
      \begin{itemize}
        \item<1-> The exception have to be reraised to the next handler
        \item<2-> First, checks if the backtrace should be collected with \funcarg{domain\_state->backtrace\_active}
        \only<3>{
        \begin{itemize}
            \item If not, behaves like a \funcname{raise\_notrace} with \funcname{RESTORE\_EXN\_HANDLER\_OCAML}
        \end{itemize}
        }
        \item<4-> Jumps into \funcname{caml\_raise\_exn}
        \item<5-> Calls \funcname{caml\_stash\_backtrace} again, to scan the next part of the stack: from the trap just pop-ed to the next trap
      \end{itemize}
      \vfill
      \mintocaml[firstline=15,lastline=21,highlightlines={18-21}]{../src/backtrace/backtrace.ml}
      \endminipage
    \end{column}
    \begin{column}{0.5\textwidth}
      \raggedleft
      \onslide*<1>{\listCamlReraiseExn{}}
      \onslide*<2>{\listCamlReraiseExn[highlightlines=729]{}}
      \onslide*<3>{
        \mintc[firstline=190,lastline=192,highlightlines={191-192}]{../ocaml/runtime/amd64.S}
        \mintc[firstline=234,lastline=237,highlightlines={235-237}]{../ocaml/runtime/amd64.S}
        \medskip
        \listCamlReraiseExn[highlightlines=731]{}
      }
      \onslide*<4-5>{
        % TODO refactor with \listCamlReraiseExn
        \mintasm[firstline=727,lastline=734,highlightlines=730]{../ocaml/runtime/amd64.S}
        \medskip
      }
      \onslide*<4>{\listCamlRaiseExn[highlightlines=711]{}}
      \onslide*<5>{\listCamlRaiseExn[highlightlines=720]{}}
    \end{column}
  \end{columns}
\end{frame}

\begin{frame}{Backtraces}{Backtrace collection continues}
  \begin{columns}[c]
    \begin{column}{0.55\textwidth}
      \minipage[c][0.75\textheight][s]{\columnwidth}
      \begin{itemize}
        \item<1-> \funcname{caml\_stash\_backtrace} scans the older part of the stack, up to the next trap
        \item<6-> Controls jumps to the nested exception handler in \funcname{maybe\_raise}, which matches only on \typename{ExnB}
        \item<7-> \funcname{caml\_reraise\_exn} is called again, backtrace collection resumes with \funcname{caml\_stash\_backtrace}
      \end{itemize}
      \medskip
      \begin{itemize}
        \item<2-> Constructed backtrace:
        \begin{itemize}
          \item <2-> \funcname{definitely\_raise}
          \item <2-> \funcname{probably\_raise}
          \item <3-> \funcname{probably\_raise} (re-raise)
          \item <4-> \funcname{maybe\_raise}
          \item <5-> \funcname{main}
          \item <8-> \funcname{main} (re-raise)
        \end{itemize}
      \end{itemize}
      \vfill
      \onslide*<1-5>{\mintocaml[firstline=15,lastline=21]{../src/backtrace/backtrace.ml}}
      \onslide*<6>{\mintocaml[firstline=28,lastline=34,highlightlines=31]{../src/backtrace/backtrace.ml}}
      \onslide*<7->{\mintocaml[firstline=28,lastline=34,highlightlines=33]{../src/backtrace/backtrace.ml}}
      \endminipage
    \end{column}
    \begin{column}{0.45\textwidth}
      \raggedleft
      \onslide*<1>{\providecommand\step{6}\input{diagrams/backtrace/backtrace.tex}}%
      \onslide*<2>{\providecommand\step{6}\input{diagrams/backtrace/backtrace.tex}}%
      \onslide*<3>{\providecommand\step{7}\input{diagrams/backtrace/backtrace.tex}}%
      \onslide*<4>{\providecommand\step{8}\input{diagrams/backtrace/backtrace.tex}}%
      \onslide*<5>{\providecommand\step{9}\input{diagrams/backtrace/backtrace.tex}}%
      \onslide*<6>{\providecommand\step{10}\input{diagrams/backtrace/backtrace.tex}}%
      \onslide*<7>{\providecommand\step{11}\input{diagrams/backtrace/backtrace.tex}}%
      \onslide*<8>{\providecommand\step{12}\input{diagrams/backtrace/backtrace.tex}}%
      \onslide*<9>{\providecommand\step{13}\input{diagrams/backtrace/backtrace.tex}}%
    \end{column}
  \end{columns}
\end{frame}

\begin{frame}{Backtraces}{Printing the backtrace}
  \begin{columns}[c]
    \begin{column}{0.45\textwidth}
      \minipage[c][0.66\textheight][s]{\columnwidth}
      \begin{itemize}
        \item<1-> The exception backtrace can now be printed to \funcarg{stdout} with \funcname{Printexc.print_backtrace}
        \item<2-> First the exception backtrace is retrieved into an OCaml array with \funcname{get_raw_backtrace }
        \item<3-> Which is implemented as the \funcname{caml_get_exception_raw_backtrace} C function in the runtime
        \item<4-> And finally printed with \funcname{print_exception_backtrace}
      \end{itemize}
      \vfill
      \mintocaml[firstline=28,lastline=34,highlightlines=33]{../src/backtrace/backtrace.ml}
      \endminipage
    \end{column}
    \begin{column}{0.55\textwidth}
      \onslide*<1,2>{
        \mintocaml[firstline=100,lastline=101]{../ocaml/stdlib/printexc.ml}
      }
      \only<1>{
        \listocaml[firstline=170,lastline=172]{../ocaml/stdlib/printexc.ml}{stdlib/printexc.ml:170}
      }
      \only<2>{
        \listocaml[firstline=170,lastline=172,highlightlines=172]{../ocaml/stdlib/printexc.ml}{stdlib/printexc.ml:170}
      }
      \only<3>{
        \mintc[firstline=154,lastline=156]{../ocaml/runtime/backtrace.c}
        \listc[firstline=171,lastline=192,highlightlines={184-187}]{../ocaml/runtime/backtrace.c}{runtime/backtrace.c:154}
      }
      \only<4>{
        \listocaml[firstline=155,lastline=165]{../ocaml/stdlib/printexc.ml}{stdlib/printexc.ml:55}
      }
    \end{column}
  \end{columns}
\end{frame}


\subsection{The multiple kinds of raise}
\frameSubsection{}{}

\begin{frame}{Backtraces}{When backtraces are not needed}
  \begin{columns}[c]
    \begin{column}{0.45\textwidth}
      \minipage[c][0.66\textheight][s]{\columnwidth}
      \begin{itemize}
        \item<1-> What if the backtrace for an exception is never needed?
        \item<2-> It's possible to raise the exception explicitly with \funcname{raise\_notrace} to make raising as cheap as possible
        \item<3-> An example of use in the \funcname{dynlink} library of the OCaml compiler
      \end{itemize}
      \vfill
      \onslide<3->{
      \listocaml[firstline=205,lastline=209,highlightlines=207]{../ocaml/otherlibs/dynlink/dynlink\_compilerlibs/misc.ml}{otherlibs/dynlink/dynlink\_compilerlibs/misc.ml:205}
      }
      \endminipage
    \end{column}
    \begin{column}{0.55\textwidth}
    \only<4>{
      \mintcmm[highlightlines=3]{../src/notrace/camlNotrace\_\_fun_347.cmm}
      \listcmm{../src/notrace/camlNotrace\_\_all\_somes\_327.cmm}{all\_somes}
    }
    \onslide*<5->{
      \listobjdump[highlightlines={6-9}]{../src/notrace/camlNotrace\_\_fun_347.objdump}{anonymous function from \funcname{all\_somes}}
    }
    \end{column}
  \end{columns}
\end{frame}

\begin{frame}{Backtraces}{Summary table of the multiple kinds of raise}
  \begin{itemize}
    \item When debugging information is enabled and if the backtrace of an exception isn't needed, it's possible to raise with \funcname{raise\_notrace} for performance
    \item When debugging information is \emph{not} enabled, the compiler optimises \funcname{raise} calls to inline assembly (same effect as using \funcname{raise\_notrace})
  \end{itemize}
  \bigskip
  \centering
  \begin{tabular}{| l | c | c | c | c |}
    \hline
    \parbox{10em}{Debug information \\ (\funcarg{-g} flag)}  & \multicolumn{2}{|c|}{Disabled}                                     & \multicolumn{2}{|c|}{Enabled} \\
    \hline
    Raise kind                                               & \funcname{raise} & \funcname{raise\_notrace}                       & \funcname{raise} & \funcname{raise\_notrace} \\
    \hline
    Compilation                                              & \parbox{8em}{inline assembly\\ (optimization)} & inline assembly   & \funcname{caml\_raise\_exn} & inline assembly \\
    \hline
  \end{tabular}
\end{frame}


%
% Takeaway #5
%
\subsection*{Takeaway \#5}
\frameSubsectionTakeaway{}
\againframe<5>{takeaway}
}%
      \onslide*<5>{\providecommand\step{9}%
% Backtraces
%
\subsection{One advantage of raising with debugging information}
\frameSubsection{}{}

\begin{frame}{Backtraces}{Debugging information \& source code}
  \begin{columns}[c]
    \begin{column}{0.5\textwidth}
      \begin{itemize}
        \item Raising an exception is fun and games, but how do we know where the exception originated? $\rightarrow$ With the backtrace associated with the exception!
        \item In order to get a backtrace from the point where an exception was raised, it's necessary to compile with debugging information enabled (\funcarg{-g} flag for the compiler)
        \item Same example as in the `Nested handlers' section
      \end{itemize}
    \end{column}
    \begin{column}{0.5\textwidth}
      \listocaml[firstline=9,lastline=34]{../src/backtrace/backtrace.ml}{backtrace/backtrace.ml}
    \end{column}
  \end{columns}
\end{frame}

\begin{frame}{Backtraces}{CMM and disassembly of \funcname{definitely\_raise}}
  \begin{columns}[c]
    \begin{column}{0.5\textwidth}
      \minipage[c][0.66\textheight][s]{\columnwidth}
      \begin{itemize}
        \item<1-> An effect of compiling with debugging information (\funcarg{-g}): the CMM no longer uses \funcname{raise\_notrace} to raise the exception but \funcname{raise} instead
        \item<2-> Disassembly shows that CMM \funcname{raise} is actually a call to \funcname{caml\_raise\_exn}, a function in assembler provided by the runtime
      \end{itemize}
      \vfill
      \mintocaml[firstline=9,lastline=13,highlightlines=12]{../src/backtrace/backtrace.ml}
      \endminipage
    \end{column}
    \begin{column}{0.5\textwidth}
      \onslide*<1>{
        \mintcmm[highlightlines=5]{../src/backtrace/camlBacktrace\_\_definitely\_raise\_347.cmm}
      }
      \onslide*<2>{
        \mintobjdump[firstline=2,highlightlines=14]{../src/backtrace/camlBacktrace\_\_definitely\_raise\_347.objdump}
      }
    \end{column}
  \end{columns}
\end{frame}

\begin{frame}{Backtraces}{Introducing \funcname{caml\_raise\_exn}}
  \begin{columns}[c]
    \begin{column}{0.5\textwidth}
      \minipage[c][0.66\textheight][s]{\columnwidth}
      \onslide*<1>{
        \begin{itemize}
          \item Raising the exception is performed using \funcname{caml\_raise\_exn} which is
            \begin{itemize}
              \item An assembly function provided by the runtime
              \item Architecture specific
            \end{itemize}
          \item Let's see what it does differently from \funcname{raise\_notrace}\dots
          \item We are about to raise \typename{ExnC} with \funcname{caml\_raise\_exn} from \funcname{definitely\_raise}
        \end{itemize}
      }
      \onslide*<2>{
        \mintobjdump[firstline=2,highlightlines=14]{../src/backtrace/camlBacktrace\_\_definitely\_raise\_347.objdump}
      }
      \vfill
      \mintocaml[firstline=9,lastline=13,highlightlines=12]{../src/backtrace/backtrace.ml}
      \endminipage
    \end{column}
    \begin{column}{0.5\textwidth}
      \centering
      \providecommand\step{1}\input{diagrams/backtrace/backtrace.tex}
    \end{column}
  \end{columns}
\end{frame}

\begin{frame}{Backtraces}{But before that, we need more fields from \typename{caml\_domain\_state}}
  \begin{columns}
    \begin{column}{0.5\textwidth}
      \begin{itemize}
        \item Remember \typename{caml\_domain\_state} the per-domain runtime state?
        \item Some other members are of interest to us now\dots
        \item \localname{backtrace\_active} is used to know if the runtime should collect backtraces
        \item \localname{backtrace\_active} can be set by
          \begin{itemize}
            \item The \funcarg{b} flag in the \funcname{OCAMLRUNPARAM} environment variable
            \item \mintinline{ocaml}{Printexc.record_backtrace : bool -> unit} \footnotemark
          \end{itemize}
      \end{itemize}
    \end{column}
    \begin{column}{0.5\textwidth}
      \begin{listing}
        \mintc[highlightlines={9-11}]{src/ocaml/domain\_state\_backtrace.h}
        \caption{runtime/caml/domain\_state.h}
      \end{listing}
    \end{column}
  \end{columns}
  \footnotetext[1]{https://v2.ocaml.org/api/Printexc.html}
\end{frame}

\newcommand\listCamlRaiseExn[1][]{
  \listasm[firstline=702,lastline=725,#1]{../ocaml/runtime/amd64.S}{runtime/amd64.S:702}}

\begin{frame}{Backtraces}{Walk through \funcname{caml\_raise\_exn}}
  \begin{columns}[T]
    \begin{column}{0.5\textwidth}
      \begin{itemize}
        \item<1-> First checks whether exceptions backtrace should be recorded
        \item<1-> By checking the \funcarg{domain\_state->backtrace\_active} value
        \item<2-> If not, acts as \funcname{raise\_notrace} with \funcname{RESTORE\_EXN\_HANDLER\_OCAML}
          \begin{itemize}
            \item Set \regname{RSP} to \localname{domain\_state->exn\_handler} value
            \item Restore \localname{domain\_state->exn\_handler} from trap
            \item Jump with \funcname{ret} (same effect as using \funcname{pop} and \funcname{jmp})
          \end{itemize}
        \item<3-> Otherwise, switches to the C stack to be able to execute C code
        \item<4-> Calls \funcname{caml\_stash\_backtrace}, a C function provided by the runtime\ignorespaces
          \onslide<6->{, with
            \begin{itemize}
              \item \funcarg{exn}: exception value
              \item \funcarg{pc}: code address of the raising point
              \item \funcarg{sp}: stack address of the raising function
              \item \funcarg{trapsp}: stack address of the next handler
            \end{itemize}
          }
      \end{itemize}
      \bigskip
      \mintocaml[firstline=9,lastline=13,highlightlines=12]{../src/backtrace/backtrace.ml}
    \end{column}
    \begin{column}{0.5\textwidth}
      \raggedleft
      \onslide*<1,3-4>{
        \mintc[firstline=190,lastline=192]{../ocaml/runtime/amd64.S}
        \mintc[firstline=234,lastline=237]{../ocaml/runtime/amd64.S}
      }
      \onslide*<2>{
        \mintc[firstline=190,lastline=192,highlightlines={191-192}]{../ocaml/runtime/amd64.S}
        \mintc[firstline=234,lastline=237,highlightlines={235-237}]{../ocaml/runtime/amd64.S}
      }
      \onslide*<1>{\listCamlRaiseExn[highlightlines=705]{}}%
      \onslide*<2>{\listCamlRaiseExn[highlightlines={707-708}]{}}%
      \onslide*<3>{\listCamlRaiseExn[highlightlines={712-714}]{}}%
      \onslide*<4>{\listCamlRaiseExn[highlightlines={715-720}]{}}%
      \onslide*<5>{\providecommand\step{1}\input{diagrams/backtrace/backtrace.tex}}%
      \onslide*<6>{\providecommand\step{2}\input{diagrams/backtrace/backtrace.tex}}%
    \end{column}
  \end{columns}
\end{frame}

\newcommand\listStashBacktrace[1][]{
  \listc[firstline=91,lastline=120,#1]{../ocaml/runtime/backtrace\_nat.c}{runtime/backtrace\_nat.c:91}}

\begin{frame}{Backtraces}{Introducing \funcname{caml\_stash\_backtrace}}
  \begin{columns}[c]
    \begin{column}{0.5\textwidth}
      \minipage[c][0.66\textheight][s]{\columnwidth}
      \begin{itemize}
        \item<1-> A C function provided by the runtime (specific to native code)
        \item<2-> It scans the stack, between the stack pointer at the raising point up to the next trap, in order to collect the names of the detected function frames
          \onslide*<2>{
            \begin{itemize}
              \item And more information, like line number, column number...
            \end{itemize}
          }
        \item<3-> It uses the same technique and information as the Garbage Collector (GC) uses to scan the values live on the stack
          \onslide*<4>{
            \begin{itemize}
              \item \typename{frame\_descr} are at the core of stack unwinding
            \end{itemize}
          }
      \end{itemize}
      \vfill
      \mintocaml[firstline=9,lastline=13,highlightlines=12]{../src/backtrace/backtrace.ml}
      \endminipage
    \end{column}
    \begin{column}{0.5\textwidth}
      \onslide*<1>{\listStashBacktrace[highlightlines=91]{}}
      \onslide*<2>{\listStashBacktrace[highlightlines={91,117-118}]{}}
      \onslide*<3->{\listStashBacktrace[highlightlines={94,106,109-110}]{}}
    \end{column}
  \end{columns}
\end{frame}

\newcommand\listFrameDescr[1][]{
  \listocaml[firstline=111,lastline=124,#1]{../ocaml/asmcomp/emitaux.ml}{asmcomp/emitaux.ml:111}}
\newcommand\listDebugInfo[1][]{
  \listocaml[firstline=38,lastline=49,#1]{../ocaml/lambda/debuginfo.mli}{lambda/debuginfo.mli:38}}

\begin{frame}{Backtraces}{\typename{frame\_descr} and \typename{frame\_debuginfo} in a nutshell}
  \begin{columns}[c]
    \begin{column}{0.5\textwidth}
      \begin{itemize}
        \item<1-> For every return address in an OCaml program, there is a associated \typename{frame\_descr}
        \item<2-> A \typename{frame\_descr} specifies
        \begin{itemize}
          \item<2-> The size of the function stack frame
          \item<3-> Where live values are on the stack (so that the GC can mark them as live and prevent from being swept)
        \end{itemize}
        \item<4-> When compiled with debug information, \typename{frame\_descr} will be linked to a \typename{frame\_debuginfo}
        \begin{itemize}
          \item<5-> Contains information about the position in the source code (file name, line and column number)
        \end{itemize}
      \end{itemize}
    \end{column}
    \begin{column}{0.5\textwidth}
        \onslide*<1>{\listFrameDescr[highlightlines={118-122}]{}}
        \onslide*<2>{\listFrameDescr[highlightlines={120}]{}}
        \onslide*<3>{\listFrameDescr[highlightlines={121}]{}}
        \onslide*<4->{\listFrameDescr[highlightlines={122}]{}}
        \medskip
        \onslide*<1-3>{\listDebugInfo{}}
        \onslide*<4>{\listDebugInfo[highlightlines={38,49}]{}}
        \onslide*<5>{\listDebugInfo[highlightlines={39-46}]{}}
    \end{column}
  \end{columns}
\end{frame}

\begin{frame}{Backtraces}{Scanning the stack with \typename{frame\_descr}}
  \begin{columns}[c]
    \begin{column}{0.5\textwidth}
      \begin{itemize}
        \item Given the return address of the code that raises the exception by calling \funcname{caml\_raise\_exn} (\funcarg{pc} argument of \funcname{caml\_stash\_backtrace})
        \item And the position of the stack pointer (\funcarg{sp} argument of \funcname{caml\_stash\_backtrace})
        \item The frame size given by \typename{frame\_descr}.\funcarg{frame\_size}, allows to find the end of the previous frame
        \item And the return address of the caller of this function
        \bigskip
        \onslide<3->{
        \item Note that traps (here the reddish \funcname{ExnA}, \funcname{ExnB} and \funcname{ExnC} blocks) belong to the frame that installed them, and so the frame size changes when traps are removed
        }
      \end{itemize}
    \end{column}
    \begin{column}{0.5\textwidth}
      \raggedleft
      \onslide*<1>{\providecommand\step{2}\input{diagrams/backtrace/backtrace.tex}}%
      \onslide*<2>{\providecommand\step{1}\input{diagrams/backtrace/scanstack.tex}}%
      \onslide*<3>{\providecommand\step{2}\input{diagrams/backtrace/scanstack.tex}}%
      \onslide*<4>{\providecommand\step{3}\input{diagrams/backtrace/scanstack.tex}}%
      \onslide*<5>{\providecommand\step{4}\input{diagrams/backtrace/scanstack.tex}}%
      \onslide*<6>{\providecommand\step{5}\input{diagrams/backtrace/scanstack.tex}}%
    \end{column}
  \end{columns}
\end{frame}

% Duplicate of previous-previous slide
\begin{frame}{Backtraces}{Continuing on \funcname{caml\_stash\_backtrace}}
  \begin{columns}[c]
    \begin{column}{0.5\textwidth}
      \minipage[c][0.75\textheight][s]{\columnwidth}
      \begin{itemize}
          \color{gray}
        \item A C function provided by the runtime (specific to native code)
        \item It scans the stack, between the stack pointer at the raising point up to the next trap, in order to collect the names of the detected function frames
        \item It uses the same technique and information as the Garbage Collector (GC) uses to scan the values live on the stack
      \end{itemize}
      \begin{itemize}
        \item For each function frame detected, store a pointer to the \typename{frame\_descr} in the \localname{domain\_state->backtrace\_buffer} array, effectively constructing the backtrace
      \end{itemize}
      \onslide<2->{
        \begin{itemize}
            \smallskip
          \item Constructed backtrace:
            \funcname{definitely\_raise},
            \onslide<3->{\funcname{probably\_raise}}
        \end{itemize}
      }
      \vfill
      \mintocaml[firstline=9,lastline=13,highlightlines=12]{../src/backtrace/backtrace.ml}
      \endminipage
    \end{column}
    \begin{column}{0.5\textwidth}
      \raggedleft
      \onslide*<1>{\listStashBacktrace[highlightlines={114-115}]{}}
      \onslide*<2>{\providecommand\step{3}\input{diagrams/backtrace/backtrace.tex}}%
      \onslide*<3>{\providecommand\step{4}\input{diagrams/backtrace/backtrace.tex}}%
    \end{column}
  \end{columns}
\end{frame}

\begin{frame}{Backtraces}{Back to \funcname{caml\_raise\_exn}}
  \begin{columns}[c]
    \begin{column}{0.5\textwidth}
      \minipage[c][0.66\textheight][s]{\columnwidth}
      \begin{itemize}\color{gray}
        \item Checks wheteher exceptions backtrace should be recorded, by checking the \funcarg{domain\_state->backtrace\_active} value
        \item Switches to the C stack to be able to execute C code
        \item Calls \funcname{caml\_stash\_backtrace}, to collect the backtrace up to the next trap
      \end{itemize}
      \onslide*<2->{
        \begin{itemize}
          \item<2-> Executes the exception handler in \funcname{probably\_raise} with \funcname{RESTORE\_EXN\_HANDLER\_OCAML}
            \begin{itemize}
              \item Set \regname{RSP} to \localname{domain\_state->exn\_handler} value
              \item Restore \localname{domain\_state->exn\_handler} from trap
              \item Jump to the handler code with \funcname{ret}
            \end{itemize}
        \end{itemize}
      }
      \vfill
      \onslide*<1>{\mintocaml[firstline=9,lastline=13,highlightlines=12]{../src/backtrace/backtrace.ml}}
      \onslide*<2->{\mintocaml[firstline=15,lastline=21,highlightlines={19-21}]{../src/backtrace/backtrace.ml}}
      \endminipage
    \end{column}
    \begin{column}{0.5\textwidth}
      \centering
      \onslide*<1>{
        \mintc[firstline=190,lastline=192]{../ocaml/runtime/amd64.S}
        \mintc[firstline=234,lastline=237]{../ocaml/runtime/amd64.S}
      }
      \onslide*<2>{
        \mintc[firstline=190,lastline=192,highlightlines={191-192}]{../ocaml/runtime/amd64.S}
        \mintc[firstline=234,lastline=237,highlightlines={235-237}]{../ocaml/runtime/amd64.S}
      }
      \onslide*<1>{\listCamlRaiseExn[highlightlines=720]{}}
      \onslide*<2>{\listCamlRaiseExn[highlightlines=722]{}}
      \onslide*<3->{\providecommand\step{5}\input{diagrams/backtrace/backtrace.tex}}
    \end{column}
  \end{columns}
\end{frame}

\begin{frame}{Backtraces}{\funcname{caml\_probably\_raise}'s handler doesn't catch \typename{ExnC}}
  \begin{columns}[c]
    \begin{column}{0.45\textwidth}
      \minipage[c][0.75\textheight][s]{\columnwidth}
      \begin{itemize}
        \item<1-> \funcname{match \dots with}\footnotemark pattern matching can also match exceptions
        \item<1-> The CMM of \funcname{probably\_raise} shows that this \funcname{match \dots with} is implemented with a pattern matching and a \funcname{try \dots with}
        \item<1-> But this one only matches on \typename{ExnA} exception
        \item<2-> Since the pattern matching fails to catch the exception, \funcname{reraise} is called with our current \typename{ExnC} exception
        \item<3-> Disassembly shows that \funcname{reraise} is actually a call to \funcname{caml\_reraise\_exn}, which is also an assembly function provided by the runtime
      \end{itemize}
      \vfill
      \mintocaml[firstline=15,lastline=21,highlightlines={18-21}]{../src/backtrace/backtrace.ml}
      \endminipage
    \end{column}
    \begin{column}{0.55\textwidth}
      \centering
      \onslide*<1>{\mintcmm[highlightlines={10-18}]{../src/backtrace/camlBacktrace\_\_probably\_raise\_351.cmm}}
      \onslide*<2>{\listcmm[highlightlines=18]{../src/backtrace/camlBacktrace\_\_probably\_raise_351.cmm}{CMM of probably\_raise}}
      \onslide*<3>{\mintobjdump[firstline=5,lastline=35,highlightlines=31]{../src/backtrace/camlBacktrace\_\_probably\_raise_351.objdump}}
    \end{column}
  \end{columns}
  \only<1>{\footnotetext[1]{https://blog.janestreet.com/pattern-matching-and-exception-handling-unite/}}
\end{frame}

\newcommand\listCamlReraiseExn[1][]{
  \listasm[firstline=727,lastline=734,#1]{../ocaml/runtime/amd64.S}{runtime/amd64.S:727}}

\begin{frame}{Backtraces}{Introducing \funcname{caml\_reraise\_exn}}
  \begin{columns}[c]
    \begin{column}{0.5\textwidth}
      \minipage[c][0.66\textheight][s]{\columnwidth}
      \begin{itemize}
        \item<1-> The exception have to be reraised to the next handler
        \item<2-> First, checks if the backtrace should be collected with \funcarg{domain\_state->backtrace\_active}
        \only<3>{
        \begin{itemize}
            \item If not, behaves like a \funcname{raise\_notrace} with \funcname{RESTORE\_EXN\_HANDLER\_OCAML}
        \end{itemize}
        }
        \item<4-> Jumps into \funcname{caml\_raise\_exn}
        \item<5-> Calls \funcname{caml\_stash\_backtrace} again, to scan the next part of the stack: from the trap just pop-ed to the next trap
      \end{itemize}
      \vfill
      \mintocaml[firstline=15,lastline=21,highlightlines={18-21}]{../src/backtrace/backtrace.ml}
      \endminipage
    \end{column}
    \begin{column}{0.5\textwidth}
      \raggedleft
      \onslide*<1>{\listCamlReraiseExn{}}
      \onslide*<2>{\listCamlReraiseExn[highlightlines=729]{}}
      \onslide*<3>{
        \mintc[firstline=190,lastline=192,highlightlines={191-192}]{../ocaml/runtime/amd64.S}
        \mintc[firstline=234,lastline=237,highlightlines={235-237}]{../ocaml/runtime/amd64.S}
        \medskip
        \listCamlReraiseExn[highlightlines=731]{}
      }
      \onslide*<4-5>{
        % TODO refactor with \listCamlReraiseExn
        \mintasm[firstline=727,lastline=734,highlightlines=730]{../ocaml/runtime/amd64.S}
        \medskip
      }
      \onslide*<4>{\listCamlRaiseExn[highlightlines=711]{}}
      \onslide*<5>{\listCamlRaiseExn[highlightlines=720]{}}
    \end{column}
  \end{columns}
\end{frame}

\begin{frame}{Backtraces}{Backtrace collection continues}
  \begin{columns}[c]
    \begin{column}{0.55\textwidth}
      \minipage[c][0.75\textheight][s]{\columnwidth}
      \begin{itemize}
        \item<1-> \funcname{caml\_stash\_backtrace} scans the older part of the stack, up to the next trap
        \item<6-> Controls jumps to the nested exception handler in \funcname{maybe\_raise}, which matches only on \typename{ExnB}
        \item<7-> \funcname{caml\_reraise\_exn} is called again, backtrace collection resumes with \funcname{caml\_stash\_backtrace}
      \end{itemize}
      \medskip
      \begin{itemize}
        \item<2-> Constructed backtrace:
        \begin{itemize}
          \item <2-> \funcname{definitely\_raise}
          \item <2-> \funcname{probably\_raise}
          \item <3-> \funcname{probably\_raise} (re-raise)
          \item <4-> \funcname{maybe\_raise}
          \item <5-> \funcname{main}
          \item <8-> \funcname{main} (re-raise)
        \end{itemize}
      \end{itemize}
      \vfill
      \onslide*<1-5>{\mintocaml[firstline=15,lastline=21]{../src/backtrace/backtrace.ml}}
      \onslide*<6>{\mintocaml[firstline=28,lastline=34,highlightlines=31]{../src/backtrace/backtrace.ml}}
      \onslide*<7->{\mintocaml[firstline=28,lastline=34,highlightlines=33]{../src/backtrace/backtrace.ml}}
      \endminipage
    \end{column}
    \begin{column}{0.45\textwidth}
      \raggedleft
      \onslide*<1>{\providecommand\step{6}\input{diagrams/backtrace/backtrace.tex}}%
      \onslide*<2>{\providecommand\step{6}\input{diagrams/backtrace/backtrace.tex}}%
      \onslide*<3>{\providecommand\step{7}\input{diagrams/backtrace/backtrace.tex}}%
      \onslide*<4>{\providecommand\step{8}\input{diagrams/backtrace/backtrace.tex}}%
      \onslide*<5>{\providecommand\step{9}\input{diagrams/backtrace/backtrace.tex}}%
      \onslide*<6>{\providecommand\step{10}\input{diagrams/backtrace/backtrace.tex}}%
      \onslide*<7>{\providecommand\step{11}\input{diagrams/backtrace/backtrace.tex}}%
      \onslide*<8>{\providecommand\step{12}\input{diagrams/backtrace/backtrace.tex}}%
      \onslide*<9>{\providecommand\step{13}\input{diagrams/backtrace/backtrace.tex}}%
    \end{column}
  \end{columns}
\end{frame}

\begin{frame}{Backtraces}{Printing the backtrace}
  \begin{columns}[c]
    \begin{column}{0.45\textwidth}
      \minipage[c][0.66\textheight][s]{\columnwidth}
      \begin{itemize}
        \item<1-> The exception backtrace can now be printed to \funcarg{stdout} with \funcname{Printexc.print_backtrace}
        \item<2-> First the exception backtrace is retrieved into an OCaml array with \funcname{get_raw_backtrace }
        \item<3-> Which is implemented as the \funcname{caml_get_exception_raw_backtrace} C function in the runtime
        \item<4-> And finally printed with \funcname{print_exception_backtrace}
      \end{itemize}
      \vfill
      \mintocaml[firstline=28,lastline=34,highlightlines=33]{../src/backtrace/backtrace.ml}
      \endminipage
    \end{column}
    \begin{column}{0.55\textwidth}
      \onslide*<1,2>{
        \mintocaml[firstline=100,lastline=101]{../ocaml/stdlib/printexc.ml}
      }
      \only<1>{
        \listocaml[firstline=170,lastline=172]{../ocaml/stdlib/printexc.ml}{stdlib/printexc.ml:170}
      }
      \only<2>{
        \listocaml[firstline=170,lastline=172,highlightlines=172]{../ocaml/stdlib/printexc.ml}{stdlib/printexc.ml:170}
      }
      \only<3>{
        \mintc[firstline=154,lastline=156]{../ocaml/runtime/backtrace.c}
        \listc[firstline=171,lastline=192,highlightlines={184-187}]{../ocaml/runtime/backtrace.c}{runtime/backtrace.c:154}
      }
      \only<4>{
        \listocaml[firstline=155,lastline=165]{../ocaml/stdlib/printexc.ml}{stdlib/printexc.ml:55}
      }
    \end{column}
  \end{columns}
\end{frame}


\subsection{The multiple kinds of raise}
\frameSubsection{}{}

\begin{frame}{Backtraces}{When backtraces are not needed}
  \begin{columns}[c]
    \begin{column}{0.45\textwidth}
      \minipage[c][0.66\textheight][s]{\columnwidth}
      \begin{itemize}
        \item<1-> What if the backtrace for an exception is never needed?
        \item<2-> It's possible to raise the exception explicitly with \funcname{raise\_notrace} to make raising as cheap as possible
        \item<3-> An example of use in the \funcname{dynlink} library of the OCaml compiler
      \end{itemize}
      \vfill
      \onslide<3->{
      \listocaml[firstline=205,lastline=209,highlightlines=207]{../ocaml/otherlibs/dynlink/dynlink\_compilerlibs/misc.ml}{otherlibs/dynlink/dynlink\_compilerlibs/misc.ml:205}
      }
      \endminipage
    \end{column}
    \begin{column}{0.55\textwidth}
    \only<4>{
      \mintcmm[highlightlines=3]{../src/notrace/camlNotrace\_\_fun_347.cmm}
      \listcmm{../src/notrace/camlNotrace\_\_all\_somes\_327.cmm}{all\_somes}
    }
    \onslide*<5->{
      \listobjdump[highlightlines={6-9}]{../src/notrace/camlNotrace\_\_fun_347.objdump}{anonymous function from \funcname{all\_somes}}
    }
    \end{column}
  \end{columns}
\end{frame}

\begin{frame}{Backtraces}{Summary table of the multiple kinds of raise}
  \begin{itemize}
    \item When debugging information is enabled and if the backtrace of an exception isn't needed, it's possible to raise with \funcname{raise\_notrace} for performance
    \item When debugging information is \emph{not} enabled, the compiler optimises \funcname{raise} calls to inline assembly (same effect as using \funcname{raise\_notrace})
  \end{itemize}
  \bigskip
  \centering
  \begin{tabular}{| l | c | c | c | c |}
    \hline
    \parbox{10em}{Debug information \\ (\funcarg{-g} flag)}  & \multicolumn{2}{|c|}{Disabled}                                     & \multicolumn{2}{|c|}{Enabled} \\
    \hline
    Raise kind                                               & \funcname{raise} & \funcname{raise\_notrace}                       & \funcname{raise} & \funcname{raise\_notrace} \\
    \hline
    Compilation                                              & \parbox{8em}{inline assembly\\ (optimization)} & inline assembly   & \funcname{caml\_raise\_exn} & inline assembly \\
    \hline
  \end{tabular}
\end{frame}


%
% Takeaway #5
%
\subsection*{Takeaway \#5}
\frameSubsectionTakeaway{}
\againframe<5>{takeaway}
}%
      \onslide*<6>{\providecommand\step{10}%
% Backtraces
%
\subsection{One advantage of raising with debugging information}
\frameSubsection{}{}

\begin{frame}{Backtraces}{Debugging information \& source code}
  \begin{columns}[c]
    \begin{column}{0.5\textwidth}
      \begin{itemize}
        \item Raising an exception is fun and games, but how do we know where the exception originated? $\rightarrow$ With the backtrace associated with the exception!
        \item In order to get a backtrace from the point where an exception was raised, it's necessary to compile with debugging information enabled (\funcarg{-g} flag for the compiler)
        \item Same example as in the `Nested handlers' section
      \end{itemize}
    \end{column}
    \begin{column}{0.5\textwidth}
      \listocaml[firstline=9,lastline=34]{../src/backtrace/backtrace.ml}{backtrace/backtrace.ml}
    \end{column}
  \end{columns}
\end{frame}

\begin{frame}{Backtraces}{CMM and disassembly of \funcname{definitely\_raise}}
  \begin{columns}[c]
    \begin{column}{0.5\textwidth}
      \minipage[c][0.66\textheight][s]{\columnwidth}
      \begin{itemize}
        \item<1-> An effect of compiling with debugging information (\funcarg{-g}): the CMM no longer uses \funcname{raise\_notrace} to raise the exception but \funcname{raise} instead
        \item<2-> Disassembly shows that CMM \funcname{raise} is actually a call to \funcname{caml\_raise\_exn}, a function in assembler provided by the runtime
      \end{itemize}
      \vfill
      \mintocaml[firstline=9,lastline=13,highlightlines=12]{../src/backtrace/backtrace.ml}
      \endminipage
    \end{column}
    \begin{column}{0.5\textwidth}
      \onslide*<1>{
        \mintcmm[highlightlines=5]{../src/backtrace/camlBacktrace\_\_definitely\_raise\_347.cmm}
      }
      \onslide*<2>{
        \mintobjdump[firstline=2,highlightlines=14]{../src/backtrace/camlBacktrace\_\_definitely\_raise\_347.objdump}
      }
    \end{column}
  \end{columns}
\end{frame}

\begin{frame}{Backtraces}{Introducing \funcname{caml\_raise\_exn}}
  \begin{columns}[c]
    \begin{column}{0.5\textwidth}
      \minipage[c][0.66\textheight][s]{\columnwidth}
      \onslide*<1>{
        \begin{itemize}
          \item Raising the exception is performed using \funcname{caml\_raise\_exn} which is
            \begin{itemize}
              \item An assembly function provided by the runtime
              \item Architecture specific
            \end{itemize}
          \item Let's see what it does differently from \funcname{raise\_notrace}\dots
          \item We are about to raise \typename{ExnC} with \funcname{caml\_raise\_exn} from \funcname{definitely\_raise}
        \end{itemize}
      }
      \onslide*<2>{
        \mintobjdump[firstline=2,highlightlines=14]{../src/backtrace/camlBacktrace\_\_definitely\_raise\_347.objdump}
      }
      \vfill
      \mintocaml[firstline=9,lastline=13,highlightlines=12]{../src/backtrace/backtrace.ml}
      \endminipage
    \end{column}
    \begin{column}{0.5\textwidth}
      \centering
      \providecommand\step{1}\input{diagrams/backtrace/backtrace.tex}
    \end{column}
  \end{columns}
\end{frame}

\begin{frame}{Backtraces}{But before that, we need more fields from \typename{caml\_domain\_state}}
  \begin{columns}
    \begin{column}{0.5\textwidth}
      \begin{itemize}
        \item Remember \typename{caml\_domain\_state} the per-domain runtime state?
        \item Some other members are of interest to us now\dots
        \item \localname{backtrace\_active} is used to know if the runtime should collect backtraces
        \item \localname{backtrace\_active} can be set by
          \begin{itemize}
            \item The \funcarg{b} flag in the \funcname{OCAMLRUNPARAM} environment variable
            \item \mintinline{ocaml}{Printexc.record_backtrace : bool -> unit} \footnotemark
          \end{itemize}
      \end{itemize}
    \end{column}
    \begin{column}{0.5\textwidth}
      \begin{listing}
        \mintc[highlightlines={9-11}]{src/ocaml/domain\_state\_backtrace.h}
        \caption{runtime/caml/domain\_state.h}
      \end{listing}
    \end{column}
  \end{columns}
  \footnotetext[1]{https://v2.ocaml.org/api/Printexc.html}
\end{frame}

\newcommand\listCamlRaiseExn[1][]{
  \listasm[firstline=702,lastline=725,#1]{../ocaml/runtime/amd64.S}{runtime/amd64.S:702}}

\begin{frame}{Backtraces}{Walk through \funcname{caml\_raise\_exn}}
  \begin{columns}[T]
    \begin{column}{0.5\textwidth}
      \begin{itemize}
        \item<1-> First checks whether exceptions backtrace should be recorded
        \item<1-> By checking the \funcarg{domain\_state->backtrace\_active} value
        \item<2-> If not, acts as \funcname{raise\_notrace} with \funcname{RESTORE\_EXN\_HANDLER\_OCAML}
          \begin{itemize}
            \item Set \regname{RSP} to \localname{domain\_state->exn\_handler} value
            \item Restore \localname{domain\_state->exn\_handler} from trap
            \item Jump with \funcname{ret} (same effect as using \funcname{pop} and \funcname{jmp})
          \end{itemize}
        \item<3-> Otherwise, switches to the C stack to be able to execute C code
        \item<4-> Calls \funcname{caml\_stash\_backtrace}, a C function provided by the runtime\ignorespaces
          \onslide<6->{, with
            \begin{itemize}
              \item \funcarg{exn}: exception value
              \item \funcarg{pc}: code address of the raising point
              \item \funcarg{sp}: stack address of the raising function
              \item \funcarg{trapsp}: stack address of the next handler
            \end{itemize}
          }
      \end{itemize}
      \bigskip
      \mintocaml[firstline=9,lastline=13,highlightlines=12]{../src/backtrace/backtrace.ml}
    \end{column}
    \begin{column}{0.5\textwidth}
      \raggedleft
      \onslide*<1,3-4>{
        \mintc[firstline=190,lastline=192]{../ocaml/runtime/amd64.S}
        \mintc[firstline=234,lastline=237]{../ocaml/runtime/amd64.S}
      }
      \onslide*<2>{
        \mintc[firstline=190,lastline=192,highlightlines={191-192}]{../ocaml/runtime/amd64.S}
        \mintc[firstline=234,lastline=237,highlightlines={235-237}]{../ocaml/runtime/amd64.S}
      }
      \onslide*<1>{\listCamlRaiseExn[highlightlines=705]{}}%
      \onslide*<2>{\listCamlRaiseExn[highlightlines={707-708}]{}}%
      \onslide*<3>{\listCamlRaiseExn[highlightlines={712-714}]{}}%
      \onslide*<4>{\listCamlRaiseExn[highlightlines={715-720}]{}}%
      \onslide*<5>{\providecommand\step{1}\input{diagrams/backtrace/backtrace.tex}}%
      \onslide*<6>{\providecommand\step{2}\input{diagrams/backtrace/backtrace.tex}}%
    \end{column}
  \end{columns}
\end{frame}

\newcommand\listStashBacktrace[1][]{
  \listc[firstline=91,lastline=120,#1]{../ocaml/runtime/backtrace\_nat.c}{runtime/backtrace\_nat.c:91}}

\begin{frame}{Backtraces}{Introducing \funcname{caml\_stash\_backtrace}}
  \begin{columns}[c]
    \begin{column}{0.5\textwidth}
      \minipage[c][0.66\textheight][s]{\columnwidth}
      \begin{itemize}
        \item<1-> A C function provided by the runtime (specific to native code)
        \item<2-> It scans the stack, between the stack pointer at the raising point up to the next trap, in order to collect the names of the detected function frames
          \onslide*<2>{
            \begin{itemize}
              \item And more information, like line number, column number...
            \end{itemize}
          }
        \item<3-> It uses the same technique and information as the Garbage Collector (GC) uses to scan the values live on the stack
          \onslide*<4>{
            \begin{itemize}
              \item \typename{frame\_descr} are at the core of stack unwinding
            \end{itemize}
          }
      \end{itemize}
      \vfill
      \mintocaml[firstline=9,lastline=13,highlightlines=12]{../src/backtrace/backtrace.ml}
      \endminipage
    \end{column}
    \begin{column}{0.5\textwidth}
      \onslide*<1>{\listStashBacktrace[highlightlines=91]{}}
      \onslide*<2>{\listStashBacktrace[highlightlines={91,117-118}]{}}
      \onslide*<3->{\listStashBacktrace[highlightlines={94,106,109-110}]{}}
    \end{column}
  \end{columns}
\end{frame}

\newcommand\listFrameDescr[1][]{
  \listocaml[firstline=111,lastline=124,#1]{../ocaml/asmcomp/emitaux.ml}{asmcomp/emitaux.ml:111}}
\newcommand\listDebugInfo[1][]{
  \listocaml[firstline=38,lastline=49,#1]{../ocaml/lambda/debuginfo.mli}{lambda/debuginfo.mli:38}}

\begin{frame}{Backtraces}{\typename{frame\_descr} and \typename{frame\_debuginfo} in a nutshell}
  \begin{columns}[c]
    \begin{column}{0.5\textwidth}
      \begin{itemize}
        \item<1-> For every return address in an OCaml program, there is a associated \typename{frame\_descr}
        \item<2-> A \typename{frame\_descr} specifies
        \begin{itemize}
          \item<2-> The size of the function stack frame
          \item<3-> Where live values are on the stack (so that the GC can mark them as live and prevent from being swept)
        \end{itemize}
        \item<4-> When compiled with debug information, \typename{frame\_descr} will be linked to a \typename{frame\_debuginfo}
        \begin{itemize}
          \item<5-> Contains information about the position in the source code (file name, line and column number)
        \end{itemize}
      \end{itemize}
    \end{column}
    \begin{column}{0.5\textwidth}
        \onslide*<1>{\listFrameDescr[highlightlines={118-122}]{}}
        \onslide*<2>{\listFrameDescr[highlightlines={120}]{}}
        \onslide*<3>{\listFrameDescr[highlightlines={121}]{}}
        \onslide*<4->{\listFrameDescr[highlightlines={122}]{}}
        \medskip
        \onslide*<1-3>{\listDebugInfo{}}
        \onslide*<4>{\listDebugInfo[highlightlines={38,49}]{}}
        \onslide*<5>{\listDebugInfo[highlightlines={39-46}]{}}
    \end{column}
  \end{columns}
\end{frame}

\begin{frame}{Backtraces}{Scanning the stack with \typename{frame\_descr}}
  \begin{columns}[c]
    \begin{column}{0.5\textwidth}
      \begin{itemize}
        \item Given the return address of the code that raises the exception by calling \funcname{caml\_raise\_exn} (\funcarg{pc} argument of \funcname{caml\_stash\_backtrace})
        \item And the position of the stack pointer (\funcarg{sp} argument of \funcname{caml\_stash\_backtrace})
        \item The frame size given by \typename{frame\_descr}.\funcarg{frame\_size}, allows to find the end of the previous frame
        \item And the return address of the caller of this function
        \bigskip
        \onslide<3->{
        \item Note that traps (here the reddish \funcname{ExnA}, \funcname{ExnB} and \funcname{ExnC} blocks) belong to the frame that installed them, and so the frame size changes when traps are removed
        }
      \end{itemize}
    \end{column}
    \begin{column}{0.5\textwidth}
      \raggedleft
      \onslide*<1>{\providecommand\step{2}\input{diagrams/backtrace/backtrace.tex}}%
      \onslide*<2>{\providecommand\step{1}\input{diagrams/backtrace/scanstack.tex}}%
      \onslide*<3>{\providecommand\step{2}\input{diagrams/backtrace/scanstack.tex}}%
      \onslide*<4>{\providecommand\step{3}\input{diagrams/backtrace/scanstack.tex}}%
      \onslide*<5>{\providecommand\step{4}\input{diagrams/backtrace/scanstack.tex}}%
      \onslide*<6>{\providecommand\step{5}\input{diagrams/backtrace/scanstack.tex}}%
    \end{column}
  \end{columns}
\end{frame}

% Duplicate of previous-previous slide
\begin{frame}{Backtraces}{Continuing on \funcname{caml\_stash\_backtrace}}
  \begin{columns}[c]
    \begin{column}{0.5\textwidth}
      \minipage[c][0.75\textheight][s]{\columnwidth}
      \begin{itemize}
          \color{gray}
        \item A C function provided by the runtime (specific to native code)
        \item It scans the stack, between the stack pointer at the raising point up to the next trap, in order to collect the names of the detected function frames
        \item It uses the same technique and information as the Garbage Collector (GC) uses to scan the values live on the stack
      \end{itemize}
      \begin{itemize}
        \item For each function frame detected, store a pointer to the \typename{frame\_descr} in the \localname{domain\_state->backtrace\_buffer} array, effectively constructing the backtrace
      \end{itemize}
      \onslide<2->{
        \begin{itemize}
            \smallskip
          \item Constructed backtrace:
            \funcname{definitely\_raise},
            \onslide<3->{\funcname{probably\_raise}}
        \end{itemize}
      }
      \vfill
      \mintocaml[firstline=9,lastline=13,highlightlines=12]{../src/backtrace/backtrace.ml}
      \endminipage
    \end{column}
    \begin{column}{0.5\textwidth}
      \raggedleft
      \onslide*<1>{\listStashBacktrace[highlightlines={114-115}]{}}
      \onslide*<2>{\providecommand\step{3}\input{diagrams/backtrace/backtrace.tex}}%
      \onslide*<3>{\providecommand\step{4}\input{diagrams/backtrace/backtrace.tex}}%
    \end{column}
  \end{columns}
\end{frame}

\begin{frame}{Backtraces}{Back to \funcname{caml\_raise\_exn}}
  \begin{columns}[c]
    \begin{column}{0.5\textwidth}
      \minipage[c][0.66\textheight][s]{\columnwidth}
      \begin{itemize}\color{gray}
        \item Checks wheteher exceptions backtrace should be recorded, by checking the \funcarg{domain\_state->backtrace\_active} value
        \item Switches to the C stack to be able to execute C code
        \item Calls \funcname{caml\_stash\_backtrace}, to collect the backtrace up to the next trap
      \end{itemize}
      \onslide*<2->{
        \begin{itemize}
          \item<2-> Executes the exception handler in \funcname{probably\_raise} with \funcname{RESTORE\_EXN\_HANDLER\_OCAML}
            \begin{itemize}
              \item Set \regname{RSP} to \localname{domain\_state->exn\_handler} value
              \item Restore \localname{domain\_state->exn\_handler} from trap
              \item Jump to the handler code with \funcname{ret}
            \end{itemize}
        \end{itemize}
      }
      \vfill
      \onslide*<1>{\mintocaml[firstline=9,lastline=13,highlightlines=12]{../src/backtrace/backtrace.ml}}
      \onslide*<2->{\mintocaml[firstline=15,lastline=21,highlightlines={19-21}]{../src/backtrace/backtrace.ml}}
      \endminipage
    \end{column}
    \begin{column}{0.5\textwidth}
      \centering
      \onslide*<1>{
        \mintc[firstline=190,lastline=192]{../ocaml/runtime/amd64.S}
        \mintc[firstline=234,lastline=237]{../ocaml/runtime/amd64.S}
      }
      \onslide*<2>{
        \mintc[firstline=190,lastline=192,highlightlines={191-192}]{../ocaml/runtime/amd64.S}
        \mintc[firstline=234,lastline=237,highlightlines={235-237}]{../ocaml/runtime/amd64.S}
      }
      \onslide*<1>{\listCamlRaiseExn[highlightlines=720]{}}
      \onslide*<2>{\listCamlRaiseExn[highlightlines=722]{}}
      \onslide*<3->{\providecommand\step{5}\input{diagrams/backtrace/backtrace.tex}}
    \end{column}
  \end{columns}
\end{frame}

\begin{frame}{Backtraces}{\funcname{caml\_probably\_raise}'s handler doesn't catch \typename{ExnC}}
  \begin{columns}[c]
    \begin{column}{0.45\textwidth}
      \minipage[c][0.75\textheight][s]{\columnwidth}
      \begin{itemize}
        \item<1-> \funcname{match \dots with}\footnotemark pattern matching can also match exceptions
        \item<1-> The CMM of \funcname{probably\_raise} shows that this \funcname{match \dots with} is implemented with a pattern matching and a \funcname{try \dots with}
        \item<1-> But this one only matches on \typename{ExnA} exception
        \item<2-> Since the pattern matching fails to catch the exception, \funcname{reraise} is called with our current \typename{ExnC} exception
        \item<3-> Disassembly shows that \funcname{reraise} is actually a call to \funcname{caml\_reraise\_exn}, which is also an assembly function provided by the runtime
      \end{itemize}
      \vfill
      \mintocaml[firstline=15,lastline=21,highlightlines={18-21}]{../src/backtrace/backtrace.ml}
      \endminipage
    \end{column}
    \begin{column}{0.55\textwidth}
      \centering
      \onslide*<1>{\mintcmm[highlightlines={10-18}]{../src/backtrace/camlBacktrace\_\_probably\_raise\_351.cmm}}
      \onslide*<2>{\listcmm[highlightlines=18]{../src/backtrace/camlBacktrace\_\_probably\_raise_351.cmm}{CMM of probably\_raise}}
      \onslide*<3>{\mintobjdump[firstline=5,lastline=35,highlightlines=31]{../src/backtrace/camlBacktrace\_\_probably\_raise_351.objdump}}
    \end{column}
  \end{columns}
  \only<1>{\footnotetext[1]{https://blog.janestreet.com/pattern-matching-and-exception-handling-unite/}}
\end{frame}

\newcommand\listCamlReraiseExn[1][]{
  \listasm[firstline=727,lastline=734,#1]{../ocaml/runtime/amd64.S}{runtime/amd64.S:727}}

\begin{frame}{Backtraces}{Introducing \funcname{caml\_reraise\_exn}}
  \begin{columns}[c]
    \begin{column}{0.5\textwidth}
      \minipage[c][0.66\textheight][s]{\columnwidth}
      \begin{itemize}
        \item<1-> The exception have to be reraised to the next handler
        \item<2-> First, checks if the backtrace should be collected with \funcarg{domain\_state->backtrace\_active}
        \only<3>{
        \begin{itemize}
            \item If not, behaves like a \funcname{raise\_notrace} with \funcname{RESTORE\_EXN\_HANDLER\_OCAML}
        \end{itemize}
        }
        \item<4-> Jumps into \funcname{caml\_raise\_exn}
        \item<5-> Calls \funcname{caml\_stash\_backtrace} again, to scan the next part of the stack: from the trap just pop-ed to the next trap
      \end{itemize}
      \vfill
      \mintocaml[firstline=15,lastline=21,highlightlines={18-21}]{../src/backtrace/backtrace.ml}
      \endminipage
    \end{column}
    \begin{column}{0.5\textwidth}
      \raggedleft
      \onslide*<1>{\listCamlReraiseExn{}}
      \onslide*<2>{\listCamlReraiseExn[highlightlines=729]{}}
      \onslide*<3>{
        \mintc[firstline=190,lastline=192,highlightlines={191-192}]{../ocaml/runtime/amd64.S}
        \mintc[firstline=234,lastline=237,highlightlines={235-237}]{../ocaml/runtime/amd64.S}
        \medskip
        \listCamlReraiseExn[highlightlines=731]{}
      }
      \onslide*<4-5>{
        % TODO refactor with \listCamlReraiseExn
        \mintasm[firstline=727,lastline=734,highlightlines=730]{../ocaml/runtime/amd64.S}
        \medskip
      }
      \onslide*<4>{\listCamlRaiseExn[highlightlines=711]{}}
      \onslide*<5>{\listCamlRaiseExn[highlightlines=720]{}}
    \end{column}
  \end{columns}
\end{frame}

\begin{frame}{Backtraces}{Backtrace collection continues}
  \begin{columns}[c]
    \begin{column}{0.55\textwidth}
      \minipage[c][0.75\textheight][s]{\columnwidth}
      \begin{itemize}
        \item<1-> \funcname{caml\_stash\_backtrace} scans the older part of the stack, up to the next trap
        \item<6-> Controls jumps to the nested exception handler in \funcname{maybe\_raise}, which matches only on \typename{ExnB}
        \item<7-> \funcname{caml\_reraise\_exn} is called again, backtrace collection resumes with \funcname{caml\_stash\_backtrace}
      \end{itemize}
      \medskip
      \begin{itemize}
        \item<2-> Constructed backtrace:
        \begin{itemize}
          \item <2-> \funcname{definitely\_raise}
          \item <2-> \funcname{probably\_raise}
          \item <3-> \funcname{probably\_raise} (re-raise)
          \item <4-> \funcname{maybe\_raise}
          \item <5-> \funcname{main}
          \item <8-> \funcname{main} (re-raise)
        \end{itemize}
      \end{itemize}
      \vfill
      \onslide*<1-5>{\mintocaml[firstline=15,lastline=21]{../src/backtrace/backtrace.ml}}
      \onslide*<6>{\mintocaml[firstline=28,lastline=34,highlightlines=31]{../src/backtrace/backtrace.ml}}
      \onslide*<7->{\mintocaml[firstline=28,lastline=34,highlightlines=33]{../src/backtrace/backtrace.ml}}
      \endminipage
    \end{column}
    \begin{column}{0.45\textwidth}
      \raggedleft
      \onslide*<1>{\providecommand\step{6}\input{diagrams/backtrace/backtrace.tex}}%
      \onslide*<2>{\providecommand\step{6}\input{diagrams/backtrace/backtrace.tex}}%
      \onslide*<3>{\providecommand\step{7}\input{diagrams/backtrace/backtrace.tex}}%
      \onslide*<4>{\providecommand\step{8}\input{diagrams/backtrace/backtrace.tex}}%
      \onslide*<5>{\providecommand\step{9}\input{diagrams/backtrace/backtrace.tex}}%
      \onslide*<6>{\providecommand\step{10}\input{diagrams/backtrace/backtrace.tex}}%
      \onslide*<7>{\providecommand\step{11}\input{diagrams/backtrace/backtrace.tex}}%
      \onslide*<8>{\providecommand\step{12}\input{diagrams/backtrace/backtrace.tex}}%
      \onslide*<9>{\providecommand\step{13}\input{diagrams/backtrace/backtrace.tex}}%
    \end{column}
  \end{columns}
\end{frame}

\begin{frame}{Backtraces}{Printing the backtrace}
  \begin{columns}[c]
    \begin{column}{0.45\textwidth}
      \minipage[c][0.66\textheight][s]{\columnwidth}
      \begin{itemize}
        \item<1-> The exception backtrace can now be printed to \funcarg{stdout} with \funcname{Printexc.print_backtrace}
        \item<2-> First the exception backtrace is retrieved into an OCaml array with \funcname{get_raw_backtrace }
        \item<3-> Which is implemented as the \funcname{caml_get_exception_raw_backtrace} C function in the runtime
        \item<4-> And finally printed with \funcname{print_exception_backtrace}
      \end{itemize}
      \vfill
      \mintocaml[firstline=28,lastline=34,highlightlines=33]{../src/backtrace/backtrace.ml}
      \endminipage
    \end{column}
    \begin{column}{0.55\textwidth}
      \onslide*<1,2>{
        \mintocaml[firstline=100,lastline=101]{../ocaml/stdlib/printexc.ml}
      }
      \only<1>{
        \listocaml[firstline=170,lastline=172]{../ocaml/stdlib/printexc.ml}{stdlib/printexc.ml:170}
      }
      \only<2>{
        \listocaml[firstline=170,lastline=172,highlightlines=172]{../ocaml/stdlib/printexc.ml}{stdlib/printexc.ml:170}
      }
      \only<3>{
        \mintc[firstline=154,lastline=156]{../ocaml/runtime/backtrace.c}
        \listc[firstline=171,lastline=192,highlightlines={184-187}]{../ocaml/runtime/backtrace.c}{runtime/backtrace.c:154}
      }
      \only<4>{
        \listocaml[firstline=155,lastline=165]{../ocaml/stdlib/printexc.ml}{stdlib/printexc.ml:55}
      }
    \end{column}
  \end{columns}
\end{frame}


\subsection{The multiple kinds of raise}
\frameSubsection{}{}

\begin{frame}{Backtraces}{When backtraces are not needed}
  \begin{columns}[c]
    \begin{column}{0.45\textwidth}
      \minipage[c][0.66\textheight][s]{\columnwidth}
      \begin{itemize}
        \item<1-> What if the backtrace for an exception is never needed?
        \item<2-> It's possible to raise the exception explicitly with \funcname{raise\_notrace} to make raising as cheap as possible
        \item<3-> An example of use in the \funcname{dynlink} library of the OCaml compiler
      \end{itemize}
      \vfill
      \onslide<3->{
      \listocaml[firstline=205,lastline=209,highlightlines=207]{../ocaml/otherlibs/dynlink/dynlink\_compilerlibs/misc.ml}{otherlibs/dynlink/dynlink\_compilerlibs/misc.ml:205}
      }
      \endminipage
    \end{column}
    \begin{column}{0.55\textwidth}
    \only<4>{
      \mintcmm[highlightlines=3]{../src/notrace/camlNotrace\_\_fun_347.cmm}
      \listcmm{../src/notrace/camlNotrace\_\_all\_somes\_327.cmm}{all\_somes}
    }
    \onslide*<5->{
      \listobjdump[highlightlines={6-9}]{../src/notrace/camlNotrace\_\_fun_347.objdump}{anonymous function from \funcname{all\_somes}}
    }
    \end{column}
  \end{columns}
\end{frame}

\begin{frame}{Backtraces}{Summary table of the multiple kinds of raise}
  \begin{itemize}
    \item When debugging information is enabled and if the backtrace of an exception isn't needed, it's possible to raise with \funcname{raise\_notrace} for performance
    \item When debugging information is \emph{not} enabled, the compiler optimises \funcname{raise} calls to inline assembly (same effect as using \funcname{raise\_notrace})
  \end{itemize}
  \bigskip
  \centering
  \begin{tabular}{| l | c | c | c | c |}
    \hline
    \parbox{10em}{Debug information \\ (\funcarg{-g} flag)}  & \multicolumn{2}{|c|}{Disabled}                                     & \multicolumn{2}{|c|}{Enabled} \\
    \hline
    Raise kind                                               & \funcname{raise} & \funcname{raise\_notrace}                       & \funcname{raise} & \funcname{raise\_notrace} \\
    \hline
    Compilation                                              & \parbox{8em}{inline assembly\\ (optimization)} & inline assembly   & \funcname{caml\_raise\_exn} & inline assembly \\
    \hline
  \end{tabular}
\end{frame}


%
% Takeaway #5
%
\subsection*{Takeaway \#5}
\frameSubsectionTakeaway{}
\againframe<5>{takeaway}
}%
      \onslide*<7>{\providecommand\step{11}%
% Backtraces
%
\subsection{One advantage of raising with debugging information}
\frameSubsection{}{}

\begin{frame}{Backtraces}{Debugging information \& source code}
  \begin{columns}[c]
    \begin{column}{0.5\textwidth}
      \begin{itemize}
        \item Raising an exception is fun and games, but how do we know where the exception originated? $\rightarrow$ With the backtrace associated with the exception!
        \item In order to get a backtrace from the point where an exception was raised, it's necessary to compile with debugging information enabled (\funcarg{-g} flag for the compiler)
        \item Same example as in the `Nested handlers' section
      \end{itemize}
    \end{column}
    \begin{column}{0.5\textwidth}
      \listocaml[firstline=9,lastline=34]{../src/backtrace/backtrace.ml}{backtrace/backtrace.ml}
    \end{column}
  \end{columns}
\end{frame}

\begin{frame}{Backtraces}{CMM and disassembly of \funcname{definitely\_raise}}
  \begin{columns}[c]
    \begin{column}{0.5\textwidth}
      \minipage[c][0.66\textheight][s]{\columnwidth}
      \begin{itemize}
        \item<1-> An effect of compiling with debugging information (\funcarg{-g}): the CMM no longer uses \funcname{raise\_notrace} to raise the exception but \funcname{raise} instead
        \item<2-> Disassembly shows that CMM \funcname{raise} is actually a call to \funcname{caml\_raise\_exn}, a function in assembler provided by the runtime
      \end{itemize}
      \vfill
      \mintocaml[firstline=9,lastline=13,highlightlines=12]{../src/backtrace/backtrace.ml}
      \endminipage
    \end{column}
    \begin{column}{0.5\textwidth}
      \onslide*<1>{
        \mintcmm[highlightlines=5]{../src/backtrace/camlBacktrace\_\_definitely\_raise\_347.cmm}
      }
      \onslide*<2>{
        \mintobjdump[firstline=2,highlightlines=14]{../src/backtrace/camlBacktrace\_\_definitely\_raise\_347.objdump}
      }
    \end{column}
  \end{columns}
\end{frame}

\begin{frame}{Backtraces}{Introducing \funcname{caml\_raise\_exn}}
  \begin{columns}[c]
    \begin{column}{0.5\textwidth}
      \minipage[c][0.66\textheight][s]{\columnwidth}
      \onslide*<1>{
        \begin{itemize}
          \item Raising the exception is performed using \funcname{caml\_raise\_exn} which is
            \begin{itemize}
              \item An assembly function provided by the runtime
              \item Architecture specific
            \end{itemize}
          \item Let's see what it does differently from \funcname{raise\_notrace}\dots
          \item We are about to raise \typename{ExnC} with \funcname{caml\_raise\_exn} from \funcname{definitely\_raise}
        \end{itemize}
      }
      \onslide*<2>{
        \mintobjdump[firstline=2,highlightlines=14]{../src/backtrace/camlBacktrace\_\_definitely\_raise\_347.objdump}
      }
      \vfill
      \mintocaml[firstline=9,lastline=13,highlightlines=12]{../src/backtrace/backtrace.ml}
      \endminipage
    \end{column}
    \begin{column}{0.5\textwidth}
      \centering
      \providecommand\step{1}\input{diagrams/backtrace/backtrace.tex}
    \end{column}
  \end{columns}
\end{frame}

\begin{frame}{Backtraces}{But before that, we need more fields from \typename{caml\_domain\_state}}
  \begin{columns}
    \begin{column}{0.5\textwidth}
      \begin{itemize}
        \item Remember \typename{caml\_domain\_state} the per-domain runtime state?
        \item Some other members are of interest to us now\dots
        \item \localname{backtrace\_active} is used to know if the runtime should collect backtraces
        \item \localname{backtrace\_active} can be set by
          \begin{itemize}
            \item The \funcarg{b} flag in the \funcname{OCAMLRUNPARAM} environment variable
            \item \mintinline{ocaml}{Printexc.record_backtrace : bool -> unit} \footnotemark
          \end{itemize}
      \end{itemize}
    \end{column}
    \begin{column}{0.5\textwidth}
      \begin{listing}
        \mintc[highlightlines={9-11}]{src/ocaml/domain\_state\_backtrace.h}
        \caption{runtime/caml/domain\_state.h}
      \end{listing}
    \end{column}
  \end{columns}
  \footnotetext[1]{https://v2.ocaml.org/api/Printexc.html}
\end{frame}

\newcommand\listCamlRaiseExn[1][]{
  \listasm[firstline=702,lastline=725,#1]{../ocaml/runtime/amd64.S}{runtime/amd64.S:702}}

\begin{frame}{Backtraces}{Walk through \funcname{caml\_raise\_exn}}
  \begin{columns}[T]
    \begin{column}{0.5\textwidth}
      \begin{itemize}
        \item<1-> First checks whether exceptions backtrace should be recorded
        \item<1-> By checking the \funcarg{domain\_state->backtrace\_active} value
        \item<2-> If not, acts as \funcname{raise\_notrace} with \funcname{RESTORE\_EXN\_HANDLER\_OCAML}
          \begin{itemize}
            \item Set \regname{RSP} to \localname{domain\_state->exn\_handler} value
            \item Restore \localname{domain\_state->exn\_handler} from trap
            \item Jump with \funcname{ret} (same effect as using \funcname{pop} and \funcname{jmp})
          \end{itemize}
        \item<3-> Otherwise, switches to the C stack to be able to execute C code
        \item<4-> Calls \funcname{caml\_stash\_backtrace}, a C function provided by the runtime\ignorespaces
          \onslide<6->{, with
            \begin{itemize}
              \item \funcarg{exn}: exception value
              \item \funcarg{pc}: code address of the raising point
              \item \funcarg{sp}: stack address of the raising function
              \item \funcarg{trapsp}: stack address of the next handler
            \end{itemize}
          }
      \end{itemize}
      \bigskip
      \mintocaml[firstline=9,lastline=13,highlightlines=12]{../src/backtrace/backtrace.ml}
    \end{column}
    \begin{column}{0.5\textwidth}
      \raggedleft
      \onslide*<1,3-4>{
        \mintc[firstline=190,lastline=192]{../ocaml/runtime/amd64.S}
        \mintc[firstline=234,lastline=237]{../ocaml/runtime/amd64.S}
      }
      \onslide*<2>{
        \mintc[firstline=190,lastline=192,highlightlines={191-192}]{../ocaml/runtime/amd64.S}
        \mintc[firstline=234,lastline=237,highlightlines={235-237}]{../ocaml/runtime/amd64.S}
      }
      \onslide*<1>{\listCamlRaiseExn[highlightlines=705]{}}%
      \onslide*<2>{\listCamlRaiseExn[highlightlines={707-708}]{}}%
      \onslide*<3>{\listCamlRaiseExn[highlightlines={712-714}]{}}%
      \onslide*<4>{\listCamlRaiseExn[highlightlines={715-720}]{}}%
      \onslide*<5>{\providecommand\step{1}\input{diagrams/backtrace/backtrace.tex}}%
      \onslide*<6>{\providecommand\step{2}\input{diagrams/backtrace/backtrace.tex}}%
    \end{column}
  \end{columns}
\end{frame}

\newcommand\listStashBacktrace[1][]{
  \listc[firstline=91,lastline=120,#1]{../ocaml/runtime/backtrace\_nat.c}{runtime/backtrace\_nat.c:91}}

\begin{frame}{Backtraces}{Introducing \funcname{caml\_stash\_backtrace}}
  \begin{columns}[c]
    \begin{column}{0.5\textwidth}
      \minipage[c][0.66\textheight][s]{\columnwidth}
      \begin{itemize}
        \item<1-> A C function provided by the runtime (specific to native code)
        \item<2-> It scans the stack, between the stack pointer at the raising point up to the next trap, in order to collect the names of the detected function frames
          \onslide*<2>{
            \begin{itemize}
              \item And more information, like line number, column number...
            \end{itemize}
          }
        \item<3-> It uses the same technique and information as the Garbage Collector (GC) uses to scan the values live on the stack
          \onslide*<4>{
            \begin{itemize}
              \item \typename{frame\_descr} are at the core of stack unwinding
            \end{itemize}
          }
      \end{itemize}
      \vfill
      \mintocaml[firstline=9,lastline=13,highlightlines=12]{../src/backtrace/backtrace.ml}
      \endminipage
    \end{column}
    \begin{column}{0.5\textwidth}
      \onslide*<1>{\listStashBacktrace[highlightlines=91]{}}
      \onslide*<2>{\listStashBacktrace[highlightlines={91,117-118}]{}}
      \onslide*<3->{\listStashBacktrace[highlightlines={94,106,109-110}]{}}
    \end{column}
  \end{columns}
\end{frame}

\newcommand\listFrameDescr[1][]{
  \listocaml[firstline=111,lastline=124,#1]{../ocaml/asmcomp/emitaux.ml}{asmcomp/emitaux.ml:111}}
\newcommand\listDebugInfo[1][]{
  \listocaml[firstline=38,lastline=49,#1]{../ocaml/lambda/debuginfo.mli}{lambda/debuginfo.mli:38}}

\begin{frame}{Backtraces}{\typename{frame\_descr} and \typename{frame\_debuginfo} in a nutshell}
  \begin{columns}[c]
    \begin{column}{0.5\textwidth}
      \begin{itemize}
        \item<1-> For every return address in an OCaml program, there is a associated \typename{frame\_descr}
        \item<2-> A \typename{frame\_descr} specifies
        \begin{itemize}
          \item<2-> The size of the function stack frame
          \item<3-> Where live values are on the stack (so that the GC can mark them as live and prevent from being swept)
        \end{itemize}
        \item<4-> When compiled with debug information, \typename{frame\_descr} will be linked to a \typename{frame\_debuginfo}
        \begin{itemize}
          \item<5-> Contains information about the position in the source code (file name, line and column number)
        \end{itemize}
      \end{itemize}
    \end{column}
    \begin{column}{0.5\textwidth}
        \onslide*<1>{\listFrameDescr[highlightlines={118-122}]{}}
        \onslide*<2>{\listFrameDescr[highlightlines={120}]{}}
        \onslide*<3>{\listFrameDescr[highlightlines={121}]{}}
        \onslide*<4->{\listFrameDescr[highlightlines={122}]{}}
        \medskip
        \onslide*<1-3>{\listDebugInfo{}}
        \onslide*<4>{\listDebugInfo[highlightlines={38,49}]{}}
        \onslide*<5>{\listDebugInfo[highlightlines={39-46}]{}}
    \end{column}
  \end{columns}
\end{frame}

\begin{frame}{Backtraces}{Scanning the stack with \typename{frame\_descr}}
  \begin{columns}[c]
    \begin{column}{0.5\textwidth}
      \begin{itemize}
        \item Given the return address of the code that raises the exception by calling \funcname{caml\_raise\_exn} (\funcarg{pc} argument of \funcname{caml\_stash\_backtrace})
        \item And the position of the stack pointer (\funcarg{sp} argument of \funcname{caml\_stash\_backtrace})
        \item The frame size given by \typename{frame\_descr}.\funcarg{frame\_size}, allows to find the end of the previous frame
        \item And the return address of the caller of this function
        \bigskip
        \onslide<3->{
        \item Note that traps (here the reddish \funcname{ExnA}, \funcname{ExnB} and \funcname{ExnC} blocks) belong to the frame that installed them, and so the frame size changes when traps are removed
        }
      \end{itemize}
    \end{column}
    \begin{column}{0.5\textwidth}
      \raggedleft
      \onslide*<1>{\providecommand\step{2}\input{diagrams/backtrace/backtrace.tex}}%
      \onslide*<2>{\providecommand\step{1}\input{diagrams/backtrace/scanstack.tex}}%
      \onslide*<3>{\providecommand\step{2}\input{diagrams/backtrace/scanstack.tex}}%
      \onslide*<4>{\providecommand\step{3}\input{diagrams/backtrace/scanstack.tex}}%
      \onslide*<5>{\providecommand\step{4}\input{diagrams/backtrace/scanstack.tex}}%
      \onslide*<6>{\providecommand\step{5}\input{diagrams/backtrace/scanstack.tex}}%
    \end{column}
  \end{columns}
\end{frame}

% Duplicate of previous-previous slide
\begin{frame}{Backtraces}{Continuing on \funcname{caml\_stash\_backtrace}}
  \begin{columns}[c]
    \begin{column}{0.5\textwidth}
      \minipage[c][0.75\textheight][s]{\columnwidth}
      \begin{itemize}
          \color{gray}
        \item A C function provided by the runtime (specific to native code)
        \item It scans the stack, between the stack pointer at the raising point up to the next trap, in order to collect the names of the detected function frames
        \item It uses the same technique and information as the Garbage Collector (GC) uses to scan the values live on the stack
      \end{itemize}
      \begin{itemize}
        \item For each function frame detected, store a pointer to the \typename{frame\_descr} in the \localname{domain\_state->backtrace\_buffer} array, effectively constructing the backtrace
      \end{itemize}
      \onslide<2->{
        \begin{itemize}
            \smallskip
          \item Constructed backtrace:
            \funcname{definitely\_raise},
            \onslide<3->{\funcname{probably\_raise}}
        \end{itemize}
      }
      \vfill
      \mintocaml[firstline=9,lastline=13,highlightlines=12]{../src/backtrace/backtrace.ml}
      \endminipage
    \end{column}
    \begin{column}{0.5\textwidth}
      \raggedleft
      \onslide*<1>{\listStashBacktrace[highlightlines={114-115}]{}}
      \onslide*<2>{\providecommand\step{3}\input{diagrams/backtrace/backtrace.tex}}%
      \onslide*<3>{\providecommand\step{4}\input{diagrams/backtrace/backtrace.tex}}%
    \end{column}
  \end{columns}
\end{frame}

\begin{frame}{Backtraces}{Back to \funcname{caml\_raise\_exn}}
  \begin{columns}[c]
    \begin{column}{0.5\textwidth}
      \minipage[c][0.66\textheight][s]{\columnwidth}
      \begin{itemize}\color{gray}
        \item Checks wheteher exceptions backtrace should be recorded, by checking the \funcarg{domain\_state->backtrace\_active} value
        \item Switches to the C stack to be able to execute C code
        \item Calls \funcname{caml\_stash\_backtrace}, to collect the backtrace up to the next trap
      \end{itemize}
      \onslide*<2->{
        \begin{itemize}
          \item<2-> Executes the exception handler in \funcname{probably\_raise} with \funcname{RESTORE\_EXN\_HANDLER\_OCAML}
            \begin{itemize}
              \item Set \regname{RSP} to \localname{domain\_state->exn\_handler} value
              \item Restore \localname{domain\_state->exn\_handler} from trap
              \item Jump to the handler code with \funcname{ret}
            \end{itemize}
        \end{itemize}
      }
      \vfill
      \onslide*<1>{\mintocaml[firstline=9,lastline=13,highlightlines=12]{../src/backtrace/backtrace.ml}}
      \onslide*<2->{\mintocaml[firstline=15,lastline=21,highlightlines={19-21}]{../src/backtrace/backtrace.ml}}
      \endminipage
    \end{column}
    \begin{column}{0.5\textwidth}
      \centering
      \onslide*<1>{
        \mintc[firstline=190,lastline=192]{../ocaml/runtime/amd64.S}
        \mintc[firstline=234,lastline=237]{../ocaml/runtime/amd64.S}
      }
      \onslide*<2>{
        \mintc[firstline=190,lastline=192,highlightlines={191-192}]{../ocaml/runtime/amd64.S}
        \mintc[firstline=234,lastline=237,highlightlines={235-237}]{../ocaml/runtime/amd64.S}
      }
      \onslide*<1>{\listCamlRaiseExn[highlightlines=720]{}}
      \onslide*<2>{\listCamlRaiseExn[highlightlines=722]{}}
      \onslide*<3->{\providecommand\step{5}\input{diagrams/backtrace/backtrace.tex}}
    \end{column}
  \end{columns}
\end{frame}

\begin{frame}{Backtraces}{\funcname{caml\_probably\_raise}'s handler doesn't catch \typename{ExnC}}
  \begin{columns}[c]
    \begin{column}{0.45\textwidth}
      \minipage[c][0.75\textheight][s]{\columnwidth}
      \begin{itemize}
        \item<1-> \funcname{match \dots with}\footnotemark pattern matching can also match exceptions
        \item<1-> The CMM of \funcname{probably\_raise} shows that this \funcname{match \dots with} is implemented with a pattern matching and a \funcname{try \dots with}
        \item<1-> But this one only matches on \typename{ExnA} exception
        \item<2-> Since the pattern matching fails to catch the exception, \funcname{reraise} is called with our current \typename{ExnC} exception
        \item<3-> Disassembly shows that \funcname{reraise} is actually a call to \funcname{caml\_reraise\_exn}, which is also an assembly function provided by the runtime
      \end{itemize}
      \vfill
      \mintocaml[firstline=15,lastline=21,highlightlines={18-21}]{../src/backtrace/backtrace.ml}
      \endminipage
    \end{column}
    \begin{column}{0.55\textwidth}
      \centering
      \onslide*<1>{\mintcmm[highlightlines={10-18}]{../src/backtrace/camlBacktrace\_\_probably\_raise\_351.cmm}}
      \onslide*<2>{\listcmm[highlightlines=18]{../src/backtrace/camlBacktrace\_\_probably\_raise_351.cmm}{CMM of probably\_raise}}
      \onslide*<3>{\mintobjdump[firstline=5,lastline=35,highlightlines=31]{../src/backtrace/camlBacktrace\_\_probably\_raise_351.objdump}}
    \end{column}
  \end{columns}
  \only<1>{\footnotetext[1]{https://blog.janestreet.com/pattern-matching-and-exception-handling-unite/}}
\end{frame}

\newcommand\listCamlReraiseExn[1][]{
  \listasm[firstline=727,lastline=734,#1]{../ocaml/runtime/amd64.S}{runtime/amd64.S:727}}

\begin{frame}{Backtraces}{Introducing \funcname{caml\_reraise\_exn}}
  \begin{columns}[c]
    \begin{column}{0.5\textwidth}
      \minipage[c][0.66\textheight][s]{\columnwidth}
      \begin{itemize}
        \item<1-> The exception have to be reraised to the next handler
        \item<2-> First, checks if the backtrace should be collected with \funcarg{domain\_state->backtrace\_active}
        \only<3>{
        \begin{itemize}
            \item If not, behaves like a \funcname{raise\_notrace} with \funcname{RESTORE\_EXN\_HANDLER\_OCAML}
        \end{itemize}
        }
        \item<4-> Jumps into \funcname{caml\_raise\_exn}
        \item<5-> Calls \funcname{caml\_stash\_backtrace} again, to scan the next part of the stack: from the trap just pop-ed to the next trap
      \end{itemize}
      \vfill
      \mintocaml[firstline=15,lastline=21,highlightlines={18-21}]{../src/backtrace/backtrace.ml}
      \endminipage
    \end{column}
    \begin{column}{0.5\textwidth}
      \raggedleft
      \onslide*<1>{\listCamlReraiseExn{}}
      \onslide*<2>{\listCamlReraiseExn[highlightlines=729]{}}
      \onslide*<3>{
        \mintc[firstline=190,lastline=192,highlightlines={191-192}]{../ocaml/runtime/amd64.S}
        \mintc[firstline=234,lastline=237,highlightlines={235-237}]{../ocaml/runtime/amd64.S}
        \medskip
        \listCamlReraiseExn[highlightlines=731]{}
      }
      \onslide*<4-5>{
        % TODO refactor with \listCamlReraiseExn
        \mintasm[firstline=727,lastline=734,highlightlines=730]{../ocaml/runtime/amd64.S}
        \medskip
      }
      \onslide*<4>{\listCamlRaiseExn[highlightlines=711]{}}
      \onslide*<5>{\listCamlRaiseExn[highlightlines=720]{}}
    \end{column}
  \end{columns}
\end{frame}

\begin{frame}{Backtraces}{Backtrace collection continues}
  \begin{columns}[c]
    \begin{column}{0.55\textwidth}
      \minipage[c][0.75\textheight][s]{\columnwidth}
      \begin{itemize}
        \item<1-> \funcname{caml\_stash\_backtrace} scans the older part of the stack, up to the next trap
        \item<6-> Controls jumps to the nested exception handler in \funcname{maybe\_raise}, which matches only on \typename{ExnB}
        \item<7-> \funcname{caml\_reraise\_exn} is called again, backtrace collection resumes with \funcname{caml\_stash\_backtrace}
      \end{itemize}
      \medskip
      \begin{itemize}
        \item<2-> Constructed backtrace:
        \begin{itemize}
          \item <2-> \funcname{definitely\_raise}
          \item <2-> \funcname{probably\_raise}
          \item <3-> \funcname{probably\_raise} (re-raise)
          \item <4-> \funcname{maybe\_raise}
          \item <5-> \funcname{main}
          \item <8-> \funcname{main} (re-raise)
        \end{itemize}
      \end{itemize}
      \vfill
      \onslide*<1-5>{\mintocaml[firstline=15,lastline=21]{../src/backtrace/backtrace.ml}}
      \onslide*<6>{\mintocaml[firstline=28,lastline=34,highlightlines=31]{../src/backtrace/backtrace.ml}}
      \onslide*<7->{\mintocaml[firstline=28,lastline=34,highlightlines=33]{../src/backtrace/backtrace.ml}}
      \endminipage
    \end{column}
    \begin{column}{0.45\textwidth}
      \raggedleft
      \onslide*<1>{\providecommand\step{6}\input{diagrams/backtrace/backtrace.tex}}%
      \onslide*<2>{\providecommand\step{6}\input{diagrams/backtrace/backtrace.tex}}%
      \onslide*<3>{\providecommand\step{7}\input{diagrams/backtrace/backtrace.tex}}%
      \onslide*<4>{\providecommand\step{8}\input{diagrams/backtrace/backtrace.tex}}%
      \onslide*<5>{\providecommand\step{9}\input{diagrams/backtrace/backtrace.tex}}%
      \onslide*<6>{\providecommand\step{10}\input{diagrams/backtrace/backtrace.tex}}%
      \onslide*<7>{\providecommand\step{11}\input{diagrams/backtrace/backtrace.tex}}%
      \onslide*<8>{\providecommand\step{12}\input{diagrams/backtrace/backtrace.tex}}%
      \onslide*<9>{\providecommand\step{13}\input{diagrams/backtrace/backtrace.tex}}%
    \end{column}
  \end{columns}
\end{frame}

\begin{frame}{Backtraces}{Printing the backtrace}
  \begin{columns}[c]
    \begin{column}{0.45\textwidth}
      \minipage[c][0.66\textheight][s]{\columnwidth}
      \begin{itemize}
        \item<1-> The exception backtrace can now be printed to \funcarg{stdout} with \funcname{Printexc.print_backtrace}
        \item<2-> First the exception backtrace is retrieved into an OCaml array with \funcname{get_raw_backtrace }
        \item<3-> Which is implemented as the \funcname{caml_get_exception_raw_backtrace} C function in the runtime
        \item<4-> And finally printed with \funcname{print_exception_backtrace}
      \end{itemize}
      \vfill
      \mintocaml[firstline=28,lastline=34,highlightlines=33]{../src/backtrace/backtrace.ml}
      \endminipage
    \end{column}
    \begin{column}{0.55\textwidth}
      \onslide*<1,2>{
        \mintocaml[firstline=100,lastline=101]{../ocaml/stdlib/printexc.ml}
      }
      \only<1>{
        \listocaml[firstline=170,lastline=172]{../ocaml/stdlib/printexc.ml}{stdlib/printexc.ml:170}
      }
      \only<2>{
        \listocaml[firstline=170,lastline=172,highlightlines=172]{../ocaml/stdlib/printexc.ml}{stdlib/printexc.ml:170}
      }
      \only<3>{
        \mintc[firstline=154,lastline=156]{../ocaml/runtime/backtrace.c}
        \listc[firstline=171,lastline=192,highlightlines={184-187}]{../ocaml/runtime/backtrace.c}{runtime/backtrace.c:154}
      }
      \only<4>{
        \listocaml[firstline=155,lastline=165]{../ocaml/stdlib/printexc.ml}{stdlib/printexc.ml:55}
      }
    \end{column}
  \end{columns}
\end{frame}


\subsection{The multiple kinds of raise}
\frameSubsection{}{}

\begin{frame}{Backtraces}{When backtraces are not needed}
  \begin{columns}[c]
    \begin{column}{0.45\textwidth}
      \minipage[c][0.66\textheight][s]{\columnwidth}
      \begin{itemize}
        \item<1-> What if the backtrace for an exception is never needed?
        \item<2-> It's possible to raise the exception explicitly with \funcname{raise\_notrace} to make raising as cheap as possible
        \item<3-> An example of use in the \funcname{dynlink} library of the OCaml compiler
      \end{itemize}
      \vfill
      \onslide<3->{
      \listocaml[firstline=205,lastline=209,highlightlines=207]{../ocaml/otherlibs/dynlink/dynlink\_compilerlibs/misc.ml}{otherlibs/dynlink/dynlink\_compilerlibs/misc.ml:205}
      }
      \endminipage
    \end{column}
    \begin{column}{0.55\textwidth}
    \only<4>{
      \mintcmm[highlightlines=3]{../src/notrace/camlNotrace\_\_fun_347.cmm}
      \listcmm{../src/notrace/camlNotrace\_\_all\_somes\_327.cmm}{all\_somes}
    }
    \onslide*<5->{
      \listobjdump[highlightlines={6-9}]{../src/notrace/camlNotrace\_\_fun_347.objdump}{anonymous function from \funcname{all\_somes}}
    }
    \end{column}
  \end{columns}
\end{frame}

\begin{frame}{Backtraces}{Summary table of the multiple kinds of raise}
  \begin{itemize}
    \item When debugging information is enabled and if the backtrace of an exception isn't needed, it's possible to raise with \funcname{raise\_notrace} for performance
    \item When debugging information is \emph{not} enabled, the compiler optimises \funcname{raise} calls to inline assembly (same effect as using \funcname{raise\_notrace})
  \end{itemize}
  \bigskip
  \centering
  \begin{tabular}{| l | c | c | c | c |}
    \hline
    \parbox{10em}{Debug information \\ (\funcarg{-g} flag)}  & \multicolumn{2}{|c|}{Disabled}                                     & \multicolumn{2}{|c|}{Enabled} \\
    \hline
    Raise kind                                               & \funcname{raise} & \funcname{raise\_notrace}                       & \funcname{raise} & \funcname{raise\_notrace} \\
    \hline
    Compilation                                              & \parbox{8em}{inline assembly\\ (optimization)} & inline assembly   & \funcname{caml\_raise\_exn} & inline assembly \\
    \hline
  \end{tabular}
\end{frame}


%
% Takeaway #5
%
\subsection*{Takeaway \#5}
\frameSubsectionTakeaway{}
\againframe<5>{takeaway}
}%
      \onslide*<8>{\providecommand\step{12}%
% Backtraces
%
\subsection{One advantage of raising with debugging information}
\frameSubsection{}{}

\begin{frame}{Backtraces}{Debugging information \& source code}
  \begin{columns}[c]
    \begin{column}{0.5\textwidth}
      \begin{itemize}
        \item Raising an exception is fun and games, but how do we know where the exception originated? $\rightarrow$ With the backtrace associated with the exception!
        \item In order to get a backtrace from the point where an exception was raised, it's necessary to compile with debugging information enabled (\funcarg{-g} flag for the compiler)
        \item Same example as in the `Nested handlers' section
      \end{itemize}
    \end{column}
    \begin{column}{0.5\textwidth}
      \listocaml[firstline=9,lastline=34]{../src/backtrace/backtrace.ml}{backtrace/backtrace.ml}
    \end{column}
  \end{columns}
\end{frame}

\begin{frame}{Backtraces}{CMM and disassembly of \funcname{definitely\_raise}}
  \begin{columns}[c]
    \begin{column}{0.5\textwidth}
      \minipage[c][0.66\textheight][s]{\columnwidth}
      \begin{itemize}
        \item<1-> An effect of compiling with debugging information (\funcarg{-g}): the CMM no longer uses \funcname{raise\_notrace} to raise the exception but \funcname{raise} instead
        \item<2-> Disassembly shows that CMM \funcname{raise} is actually a call to \funcname{caml\_raise\_exn}, a function in assembler provided by the runtime
      \end{itemize}
      \vfill
      \mintocaml[firstline=9,lastline=13,highlightlines=12]{../src/backtrace/backtrace.ml}
      \endminipage
    \end{column}
    \begin{column}{0.5\textwidth}
      \onslide*<1>{
        \mintcmm[highlightlines=5]{../src/backtrace/camlBacktrace\_\_definitely\_raise\_347.cmm}
      }
      \onslide*<2>{
        \mintobjdump[firstline=2,highlightlines=14]{../src/backtrace/camlBacktrace\_\_definitely\_raise\_347.objdump}
      }
    \end{column}
  \end{columns}
\end{frame}

\begin{frame}{Backtraces}{Introducing \funcname{caml\_raise\_exn}}
  \begin{columns}[c]
    \begin{column}{0.5\textwidth}
      \minipage[c][0.66\textheight][s]{\columnwidth}
      \onslide*<1>{
        \begin{itemize}
          \item Raising the exception is performed using \funcname{caml\_raise\_exn} which is
            \begin{itemize}
              \item An assembly function provided by the runtime
              \item Architecture specific
            \end{itemize}
          \item Let's see what it does differently from \funcname{raise\_notrace}\dots
          \item We are about to raise \typename{ExnC} with \funcname{caml\_raise\_exn} from \funcname{definitely\_raise}
        \end{itemize}
      }
      \onslide*<2>{
        \mintobjdump[firstline=2,highlightlines=14]{../src/backtrace/camlBacktrace\_\_definitely\_raise\_347.objdump}
      }
      \vfill
      \mintocaml[firstline=9,lastline=13,highlightlines=12]{../src/backtrace/backtrace.ml}
      \endminipage
    \end{column}
    \begin{column}{0.5\textwidth}
      \centering
      \providecommand\step{1}\input{diagrams/backtrace/backtrace.tex}
    \end{column}
  \end{columns}
\end{frame}

\begin{frame}{Backtraces}{But before that, we need more fields from \typename{caml\_domain\_state}}
  \begin{columns}
    \begin{column}{0.5\textwidth}
      \begin{itemize}
        \item Remember \typename{caml\_domain\_state} the per-domain runtime state?
        \item Some other members are of interest to us now\dots
        \item \localname{backtrace\_active} is used to know if the runtime should collect backtraces
        \item \localname{backtrace\_active} can be set by
          \begin{itemize}
            \item The \funcarg{b} flag in the \funcname{OCAMLRUNPARAM} environment variable
            \item \mintinline{ocaml}{Printexc.record_backtrace : bool -> unit} \footnotemark
          \end{itemize}
      \end{itemize}
    \end{column}
    \begin{column}{0.5\textwidth}
      \begin{listing}
        \mintc[highlightlines={9-11}]{src/ocaml/domain\_state\_backtrace.h}
        \caption{runtime/caml/domain\_state.h}
      \end{listing}
    \end{column}
  \end{columns}
  \footnotetext[1]{https://v2.ocaml.org/api/Printexc.html}
\end{frame}

\newcommand\listCamlRaiseExn[1][]{
  \listasm[firstline=702,lastline=725,#1]{../ocaml/runtime/amd64.S}{runtime/amd64.S:702}}

\begin{frame}{Backtraces}{Walk through \funcname{caml\_raise\_exn}}
  \begin{columns}[T]
    \begin{column}{0.5\textwidth}
      \begin{itemize}
        \item<1-> First checks whether exceptions backtrace should be recorded
        \item<1-> By checking the \funcarg{domain\_state->backtrace\_active} value
        \item<2-> If not, acts as \funcname{raise\_notrace} with \funcname{RESTORE\_EXN\_HANDLER\_OCAML}
          \begin{itemize}
            \item Set \regname{RSP} to \localname{domain\_state->exn\_handler} value
            \item Restore \localname{domain\_state->exn\_handler} from trap
            \item Jump with \funcname{ret} (same effect as using \funcname{pop} and \funcname{jmp})
          \end{itemize}
        \item<3-> Otherwise, switches to the C stack to be able to execute C code
        \item<4-> Calls \funcname{caml\_stash\_backtrace}, a C function provided by the runtime\ignorespaces
          \onslide<6->{, with
            \begin{itemize}
              \item \funcarg{exn}: exception value
              \item \funcarg{pc}: code address of the raising point
              \item \funcarg{sp}: stack address of the raising function
              \item \funcarg{trapsp}: stack address of the next handler
            \end{itemize}
          }
      \end{itemize}
      \bigskip
      \mintocaml[firstline=9,lastline=13,highlightlines=12]{../src/backtrace/backtrace.ml}
    \end{column}
    \begin{column}{0.5\textwidth}
      \raggedleft
      \onslide*<1,3-4>{
        \mintc[firstline=190,lastline=192]{../ocaml/runtime/amd64.S}
        \mintc[firstline=234,lastline=237]{../ocaml/runtime/amd64.S}
      }
      \onslide*<2>{
        \mintc[firstline=190,lastline=192,highlightlines={191-192}]{../ocaml/runtime/amd64.S}
        \mintc[firstline=234,lastline=237,highlightlines={235-237}]{../ocaml/runtime/amd64.S}
      }
      \onslide*<1>{\listCamlRaiseExn[highlightlines=705]{}}%
      \onslide*<2>{\listCamlRaiseExn[highlightlines={707-708}]{}}%
      \onslide*<3>{\listCamlRaiseExn[highlightlines={712-714}]{}}%
      \onslide*<4>{\listCamlRaiseExn[highlightlines={715-720}]{}}%
      \onslide*<5>{\providecommand\step{1}\input{diagrams/backtrace/backtrace.tex}}%
      \onslide*<6>{\providecommand\step{2}\input{diagrams/backtrace/backtrace.tex}}%
    \end{column}
  \end{columns}
\end{frame}

\newcommand\listStashBacktrace[1][]{
  \listc[firstline=91,lastline=120,#1]{../ocaml/runtime/backtrace\_nat.c}{runtime/backtrace\_nat.c:91}}

\begin{frame}{Backtraces}{Introducing \funcname{caml\_stash\_backtrace}}
  \begin{columns}[c]
    \begin{column}{0.5\textwidth}
      \minipage[c][0.66\textheight][s]{\columnwidth}
      \begin{itemize}
        \item<1-> A C function provided by the runtime (specific to native code)
        \item<2-> It scans the stack, between the stack pointer at the raising point up to the next trap, in order to collect the names of the detected function frames
          \onslide*<2>{
            \begin{itemize}
              \item And more information, like line number, column number...
            \end{itemize}
          }
        \item<3-> It uses the same technique and information as the Garbage Collector (GC) uses to scan the values live on the stack
          \onslide*<4>{
            \begin{itemize}
              \item \typename{frame\_descr} are at the core of stack unwinding
            \end{itemize}
          }
      \end{itemize}
      \vfill
      \mintocaml[firstline=9,lastline=13,highlightlines=12]{../src/backtrace/backtrace.ml}
      \endminipage
    \end{column}
    \begin{column}{0.5\textwidth}
      \onslide*<1>{\listStashBacktrace[highlightlines=91]{}}
      \onslide*<2>{\listStashBacktrace[highlightlines={91,117-118}]{}}
      \onslide*<3->{\listStashBacktrace[highlightlines={94,106,109-110}]{}}
    \end{column}
  \end{columns}
\end{frame}

\newcommand\listFrameDescr[1][]{
  \listocaml[firstline=111,lastline=124,#1]{../ocaml/asmcomp/emitaux.ml}{asmcomp/emitaux.ml:111}}
\newcommand\listDebugInfo[1][]{
  \listocaml[firstline=38,lastline=49,#1]{../ocaml/lambda/debuginfo.mli}{lambda/debuginfo.mli:38}}

\begin{frame}{Backtraces}{\typename{frame\_descr} and \typename{frame\_debuginfo} in a nutshell}
  \begin{columns}[c]
    \begin{column}{0.5\textwidth}
      \begin{itemize}
        \item<1-> For every return address in an OCaml program, there is a associated \typename{frame\_descr}
        \item<2-> A \typename{frame\_descr} specifies
        \begin{itemize}
          \item<2-> The size of the function stack frame
          \item<3-> Where live values are on the stack (so that the GC can mark them as live and prevent from being swept)
        \end{itemize}
        \item<4-> When compiled with debug information, \typename{frame\_descr} will be linked to a \typename{frame\_debuginfo}
        \begin{itemize}
          \item<5-> Contains information about the position in the source code (file name, line and column number)
        \end{itemize}
      \end{itemize}
    \end{column}
    \begin{column}{0.5\textwidth}
        \onslide*<1>{\listFrameDescr[highlightlines={118-122}]{}}
        \onslide*<2>{\listFrameDescr[highlightlines={120}]{}}
        \onslide*<3>{\listFrameDescr[highlightlines={121}]{}}
        \onslide*<4->{\listFrameDescr[highlightlines={122}]{}}
        \medskip
        \onslide*<1-3>{\listDebugInfo{}}
        \onslide*<4>{\listDebugInfo[highlightlines={38,49}]{}}
        \onslide*<5>{\listDebugInfo[highlightlines={39-46}]{}}
    \end{column}
  \end{columns}
\end{frame}

\begin{frame}{Backtraces}{Scanning the stack with \typename{frame\_descr}}
  \begin{columns}[c]
    \begin{column}{0.5\textwidth}
      \begin{itemize}
        \item Given the return address of the code that raises the exception by calling \funcname{caml\_raise\_exn} (\funcarg{pc} argument of \funcname{caml\_stash\_backtrace})
        \item And the position of the stack pointer (\funcarg{sp} argument of \funcname{caml\_stash\_backtrace})
        \item The frame size given by \typename{frame\_descr}.\funcarg{frame\_size}, allows to find the end of the previous frame
        \item And the return address of the caller of this function
        \bigskip
        \onslide<3->{
        \item Note that traps (here the reddish \funcname{ExnA}, \funcname{ExnB} and \funcname{ExnC} blocks) belong to the frame that installed them, and so the frame size changes when traps are removed
        }
      \end{itemize}
    \end{column}
    \begin{column}{0.5\textwidth}
      \raggedleft
      \onslide*<1>{\providecommand\step{2}\input{diagrams/backtrace/backtrace.tex}}%
      \onslide*<2>{\providecommand\step{1}\input{diagrams/backtrace/scanstack.tex}}%
      \onslide*<3>{\providecommand\step{2}\input{diagrams/backtrace/scanstack.tex}}%
      \onslide*<4>{\providecommand\step{3}\input{diagrams/backtrace/scanstack.tex}}%
      \onslide*<5>{\providecommand\step{4}\input{diagrams/backtrace/scanstack.tex}}%
      \onslide*<6>{\providecommand\step{5}\input{diagrams/backtrace/scanstack.tex}}%
    \end{column}
  \end{columns}
\end{frame}

% Duplicate of previous-previous slide
\begin{frame}{Backtraces}{Continuing on \funcname{caml\_stash\_backtrace}}
  \begin{columns}[c]
    \begin{column}{0.5\textwidth}
      \minipage[c][0.75\textheight][s]{\columnwidth}
      \begin{itemize}
          \color{gray}
        \item A C function provided by the runtime (specific to native code)
        \item It scans the stack, between the stack pointer at the raising point up to the next trap, in order to collect the names of the detected function frames
        \item It uses the same technique and information as the Garbage Collector (GC) uses to scan the values live on the stack
      \end{itemize}
      \begin{itemize}
        \item For each function frame detected, store a pointer to the \typename{frame\_descr} in the \localname{domain\_state->backtrace\_buffer} array, effectively constructing the backtrace
      \end{itemize}
      \onslide<2->{
        \begin{itemize}
            \smallskip
          \item Constructed backtrace:
            \funcname{definitely\_raise},
            \onslide<3->{\funcname{probably\_raise}}
        \end{itemize}
      }
      \vfill
      \mintocaml[firstline=9,lastline=13,highlightlines=12]{../src/backtrace/backtrace.ml}
      \endminipage
    \end{column}
    \begin{column}{0.5\textwidth}
      \raggedleft
      \onslide*<1>{\listStashBacktrace[highlightlines={114-115}]{}}
      \onslide*<2>{\providecommand\step{3}\input{diagrams/backtrace/backtrace.tex}}%
      \onslide*<3>{\providecommand\step{4}\input{diagrams/backtrace/backtrace.tex}}%
    \end{column}
  \end{columns}
\end{frame}

\begin{frame}{Backtraces}{Back to \funcname{caml\_raise\_exn}}
  \begin{columns}[c]
    \begin{column}{0.5\textwidth}
      \minipage[c][0.66\textheight][s]{\columnwidth}
      \begin{itemize}\color{gray}
        \item Checks wheteher exceptions backtrace should be recorded, by checking the \funcarg{domain\_state->backtrace\_active} value
        \item Switches to the C stack to be able to execute C code
        \item Calls \funcname{caml\_stash\_backtrace}, to collect the backtrace up to the next trap
      \end{itemize}
      \onslide*<2->{
        \begin{itemize}
          \item<2-> Executes the exception handler in \funcname{probably\_raise} with \funcname{RESTORE\_EXN\_HANDLER\_OCAML}
            \begin{itemize}
              \item Set \regname{RSP} to \localname{domain\_state->exn\_handler} value
              \item Restore \localname{domain\_state->exn\_handler} from trap
              \item Jump to the handler code with \funcname{ret}
            \end{itemize}
        \end{itemize}
      }
      \vfill
      \onslide*<1>{\mintocaml[firstline=9,lastline=13,highlightlines=12]{../src/backtrace/backtrace.ml}}
      \onslide*<2->{\mintocaml[firstline=15,lastline=21,highlightlines={19-21}]{../src/backtrace/backtrace.ml}}
      \endminipage
    \end{column}
    \begin{column}{0.5\textwidth}
      \centering
      \onslide*<1>{
        \mintc[firstline=190,lastline=192]{../ocaml/runtime/amd64.S}
        \mintc[firstline=234,lastline=237]{../ocaml/runtime/amd64.S}
      }
      \onslide*<2>{
        \mintc[firstline=190,lastline=192,highlightlines={191-192}]{../ocaml/runtime/amd64.S}
        \mintc[firstline=234,lastline=237,highlightlines={235-237}]{../ocaml/runtime/amd64.S}
      }
      \onslide*<1>{\listCamlRaiseExn[highlightlines=720]{}}
      \onslide*<2>{\listCamlRaiseExn[highlightlines=722]{}}
      \onslide*<3->{\providecommand\step{5}\input{diagrams/backtrace/backtrace.tex}}
    \end{column}
  \end{columns}
\end{frame}

\begin{frame}{Backtraces}{\funcname{caml\_probably\_raise}'s handler doesn't catch \typename{ExnC}}
  \begin{columns}[c]
    \begin{column}{0.45\textwidth}
      \minipage[c][0.75\textheight][s]{\columnwidth}
      \begin{itemize}
        \item<1-> \funcname{match \dots with}\footnotemark pattern matching can also match exceptions
        \item<1-> The CMM of \funcname{probably\_raise} shows that this \funcname{match \dots with} is implemented with a pattern matching and a \funcname{try \dots with}
        \item<1-> But this one only matches on \typename{ExnA} exception
        \item<2-> Since the pattern matching fails to catch the exception, \funcname{reraise} is called with our current \typename{ExnC} exception
        \item<3-> Disassembly shows that \funcname{reraise} is actually a call to \funcname{caml\_reraise\_exn}, which is also an assembly function provided by the runtime
      \end{itemize}
      \vfill
      \mintocaml[firstline=15,lastline=21,highlightlines={18-21}]{../src/backtrace/backtrace.ml}
      \endminipage
    \end{column}
    \begin{column}{0.55\textwidth}
      \centering
      \onslide*<1>{\mintcmm[highlightlines={10-18}]{../src/backtrace/camlBacktrace\_\_probably\_raise\_351.cmm}}
      \onslide*<2>{\listcmm[highlightlines=18]{../src/backtrace/camlBacktrace\_\_probably\_raise_351.cmm}{CMM of probably\_raise}}
      \onslide*<3>{\mintobjdump[firstline=5,lastline=35,highlightlines=31]{../src/backtrace/camlBacktrace\_\_probably\_raise_351.objdump}}
    \end{column}
  \end{columns}
  \only<1>{\footnotetext[1]{https://blog.janestreet.com/pattern-matching-and-exception-handling-unite/}}
\end{frame}

\newcommand\listCamlReraiseExn[1][]{
  \listasm[firstline=727,lastline=734,#1]{../ocaml/runtime/amd64.S}{runtime/amd64.S:727}}

\begin{frame}{Backtraces}{Introducing \funcname{caml\_reraise\_exn}}
  \begin{columns}[c]
    \begin{column}{0.5\textwidth}
      \minipage[c][0.66\textheight][s]{\columnwidth}
      \begin{itemize}
        \item<1-> The exception have to be reraised to the next handler
        \item<2-> First, checks if the backtrace should be collected with \funcarg{domain\_state->backtrace\_active}
        \only<3>{
        \begin{itemize}
            \item If not, behaves like a \funcname{raise\_notrace} with \funcname{RESTORE\_EXN\_HANDLER\_OCAML}
        \end{itemize}
        }
        \item<4-> Jumps into \funcname{caml\_raise\_exn}
        \item<5-> Calls \funcname{caml\_stash\_backtrace} again, to scan the next part of the stack: from the trap just pop-ed to the next trap
      \end{itemize}
      \vfill
      \mintocaml[firstline=15,lastline=21,highlightlines={18-21}]{../src/backtrace/backtrace.ml}
      \endminipage
    \end{column}
    \begin{column}{0.5\textwidth}
      \raggedleft
      \onslide*<1>{\listCamlReraiseExn{}}
      \onslide*<2>{\listCamlReraiseExn[highlightlines=729]{}}
      \onslide*<3>{
        \mintc[firstline=190,lastline=192,highlightlines={191-192}]{../ocaml/runtime/amd64.S}
        \mintc[firstline=234,lastline=237,highlightlines={235-237}]{../ocaml/runtime/amd64.S}
        \medskip
        \listCamlReraiseExn[highlightlines=731]{}
      }
      \onslide*<4-5>{
        % TODO refactor with \listCamlReraiseExn
        \mintasm[firstline=727,lastline=734,highlightlines=730]{../ocaml/runtime/amd64.S}
        \medskip
      }
      \onslide*<4>{\listCamlRaiseExn[highlightlines=711]{}}
      \onslide*<5>{\listCamlRaiseExn[highlightlines=720]{}}
    \end{column}
  \end{columns}
\end{frame}

\begin{frame}{Backtraces}{Backtrace collection continues}
  \begin{columns}[c]
    \begin{column}{0.55\textwidth}
      \minipage[c][0.75\textheight][s]{\columnwidth}
      \begin{itemize}
        \item<1-> \funcname{caml\_stash\_backtrace} scans the older part of the stack, up to the next trap
        \item<6-> Controls jumps to the nested exception handler in \funcname{maybe\_raise}, which matches only on \typename{ExnB}
        \item<7-> \funcname{caml\_reraise\_exn} is called again, backtrace collection resumes with \funcname{caml\_stash\_backtrace}
      \end{itemize}
      \medskip
      \begin{itemize}
        \item<2-> Constructed backtrace:
        \begin{itemize}
          \item <2-> \funcname{definitely\_raise}
          \item <2-> \funcname{probably\_raise}
          \item <3-> \funcname{probably\_raise} (re-raise)
          \item <4-> \funcname{maybe\_raise}
          \item <5-> \funcname{main}
          \item <8-> \funcname{main} (re-raise)
        \end{itemize}
      \end{itemize}
      \vfill
      \onslide*<1-5>{\mintocaml[firstline=15,lastline=21]{../src/backtrace/backtrace.ml}}
      \onslide*<6>{\mintocaml[firstline=28,lastline=34,highlightlines=31]{../src/backtrace/backtrace.ml}}
      \onslide*<7->{\mintocaml[firstline=28,lastline=34,highlightlines=33]{../src/backtrace/backtrace.ml}}
      \endminipage
    \end{column}
    \begin{column}{0.45\textwidth}
      \raggedleft
      \onslide*<1>{\providecommand\step{6}\input{diagrams/backtrace/backtrace.tex}}%
      \onslide*<2>{\providecommand\step{6}\input{diagrams/backtrace/backtrace.tex}}%
      \onslide*<3>{\providecommand\step{7}\input{diagrams/backtrace/backtrace.tex}}%
      \onslide*<4>{\providecommand\step{8}\input{diagrams/backtrace/backtrace.tex}}%
      \onslide*<5>{\providecommand\step{9}\input{diagrams/backtrace/backtrace.tex}}%
      \onslide*<6>{\providecommand\step{10}\input{diagrams/backtrace/backtrace.tex}}%
      \onslide*<7>{\providecommand\step{11}\input{diagrams/backtrace/backtrace.tex}}%
      \onslide*<8>{\providecommand\step{12}\input{diagrams/backtrace/backtrace.tex}}%
      \onslide*<9>{\providecommand\step{13}\input{diagrams/backtrace/backtrace.tex}}%
    \end{column}
  \end{columns}
\end{frame}

\begin{frame}{Backtraces}{Printing the backtrace}
  \begin{columns}[c]
    \begin{column}{0.45\textwidth}
      \minipage[c][0.66\textheight][s]{\columnwidth}
      \begin{itemize}
        \item<1-> The exception backtrace can now be printed to \funcarg{stdout} with \funcname{Printexc.print_backtrace}
        \item<2-> First the exception backtrace is retrieved into an OCaml array with \funcname{get_raw_backtrace }
        \item<3-> Which is implemented as the \funcname{caml_get_exception_raw_backtrace} C function in the runtime
        \item<4-> And finally printed with \funcname{print_exception_backtrace}
      \end{itemize}
      \vfill
      \mintocaml[firstline=28,lastline=34,highlightlines=33]{../src/backtrace/backtrace.ml}
      \endminipage
    \end{column}
    \begin{column}{0.55\textwidth}
      \onslide*<1,2>{
        \mintocaml[firstline=100,lastline=101]{../ocaml/stdlib/printexc.ml}
      }
      \only<1>{
        \listocaml[firstline=170,lastline=172]{../ocaml/stdlib/printexc.ml}{stdlib/printexc.ml:170}
      }
      \only<2>{
        \listocaml[firstline=170,lastline=172,highlightlines=172]{../ocaml/stdlib/printexc.ml}{stdlib/printexc.ml:170}
      }
      \only<3>{
        \mintc[firstline=154,lastline=156]{../ocaml/runtime/backtrace.c}
        \listc[firstline=171,lastline=192,highlightlines={184-187}]{../ocaml/runtime/backtrace.c}{runtime/backtrace.c:154}
      }
      \only<4>{
        \listocaml[firstline=155,lastline=165]{../ocaml/stdlib/printexc.ml}{stdlib/printexc.ml:55}
      }
    \end{column}
  \end{columns}
\end{frame}


\subsection{The multiple kinds of raise}
\frameSubsection{}{}

\begin{frame}{Backtraces}{When backtraces are not needed}
  \begin{columns}[c]
    \begin{column}{0.45\textwidth}
      \minipage[c][0.66\textheight][s]{\columnwidth}
      \begin{itemize}
        \item<1-> What if the backtrace for an exception is never needed?
        \item<2-> It's possible to raise the exception explicitly with \funcname{raise\_notrace} to make raising as cheap as possible
        \item<3-> An example of use in the \funcname{dynlink} library of the OCaml compiler
      \end{itemize}
      \vfill
      \onslide<3->{
      \listocaml[firstline=205,lastline=209,highlightlines=207]{../ocaml/otherlibs/dynlink/dynlink\_compilerlibs/misc.ml}{otherlibs/dynlink/dynlink\_compilerlibs/misc.ml:205}
      }
      \endminipage
    \end{column}
    \begin{column}{0.55\textwidth}
    \only<4>{
      \mintcmm[highlightlines=3]{../src/notrace/camlNotrace\_\_fun_347.cmm}
      \listcmm{../src/notrace/camlNotrace\_\_all\_somes\_327.cmm}{all\_somes}
    }
    \onslide*<5->{
      \listobjdump[highlightlines={6-9}]{../src/notrace/camlNotrace\_\_fun_347.objdump}{anonymous function from \funcname{all\_somes}}
    }
    \end{column}
  \end{columns}
\end{frame}

\begin{frame}{Backtraces}{Summary table of the multiple kinds of raise}
  \begin{itemize}
    \item When debugging information is enabled and if the backtrace of an exception isn't needed, it's possible to raise with \funcname{raise\_notrace} for performance
    \item When debugging information is \emph{not} enabled, the compiler optimises \funcname{raise} calls to inline assembly (same effect as using \funcname{raise\_notrace})
  \end{itemize}
  \bigskip
  \centering
  \begin{tabular}{| l | c | c | c | c |}
    \hline
    \parbox{10em}{Debug information \\ (\funcarg{-g} flag)}  & \multicolumn{2}{|c|}{Disabled}                                     & \multicolumn{2}{|c|}{Enabled} \\
    \hline
    Raise kind                                               & \funcname{raise} & \funcname{raise\_notrace}                       & \funcname{raise} & \funcname{raise\_notrace} \\
    \hline
    Compilation                                              & \parbox{8em}{inline assembly\\ (optimization)} & inline assembly   & \funcname{caml\_raise\_exn} & inline assembly \\
    \hline
  \end{tabular}
\end{frame}


%
% Takeaway #5
%
\subsection*{Takeaway \#5}
\frameSubsectionTakeaway{}
\againframe<5>{takeaway}
}%
      \onslide*<9>{\providecommand\step{13}%
% Backtraces
%
\subsection{One advantage of raising with debugging information}
\frameSubsection{}{}

\begin{frame}{Backtraces}{Debugging information \& source code}
  \begin{columns}[c]
    \begin{column}{0.5\textwidth}
      \begin{itemize}
        \item Raising an exception is fun and games, but how do we know where the exception originated? $\rightarrow$ With the backtrace associated with the exception!
        \item In order to get a backtrace from the point where an exception was raised, it's necessary to compile with debugging information enabled (\funcarg{-g} flag for the compiler)
        \item Same example as in the `Nested handlers' section
      \end{itemize}
    \end{column}
    \begin{column}{0.5\textwidth}
      \listocaml[firstline=9,lastline=34]{../src/backtrace/backtrace.ml}{backtrace/backtrace.ml}
    \end{column}
  \end{columns}
\end{frame}

\begin{frame}{Backtraces}{CMM and disassembly of \funcname{definitely\_raise}}
  \begin{columns}[c]
    \begin{column}{0.5\textwidth}
      \minipage[c][0.66\textheight][s]{\columnwidth}
      \begin{itemize}
        \item<1-> An effect of compiling with debugging information (\funcarg{-g}): the CMM no longer uses \funcname{raise\_notrace} to raise the exception but \funcname{raise} instead
        \item<2-> Disassembly shows that CMM \funcname{raise} is actually a call to \funcname{caml\_raise\_exn}, a function in assembler provided by the runtime
      \end{itemize}
      \vfill
      \mintocaml[firstline=9,lastline=13,highlightlines=12]{../src/backtrace/backtrace.ml}
      \endminipage
    \end{column}
    \begin{column}{0.5\textwidth}
      \onslide*<1>{
        \mintcmm[highlightlines=5]{../src/backtrace/camlBacktrace\_\_definitely\_raise\_347.cmm}
      }
      \onslide*<2>{
        \mintobjdump[firstline=2,highlightlines=14]{../src/backtrace/camlBacktrace\_\_definitely\_raise\_347.objdump}
      }
    \end{column}
  \end{columns}
\end{frame}

\begin{frame}{Backtraces}{Introducing \funcname{caml\_raise\_exn}}
  \begin{columns}[c]
    \begin{column}{0.5\textwidth}
      \minipage[c][0.66\textheight][s]{\columnwidth}
      \onslide*<1>{
        \begin{itemize}
          \item Raising the exception is performed using \funcname{caml\_raise\_exn} which is
            \begin{itemize}
              \item An assembly function provided by the runtime
              \item Architecture specific
            \end{itemize}
          \item Let's see what it does differently from \funcname{raise\_notrace}\dots
          \item We are about to raise \typename{ExnC} with \funcname{caml\_raise\_exn} from \funcname{definitely\_raise}
        \end{itemize}
      }
      \onslide*<2>{
        \mintobjdump[firstline=2,highlightlines=14]{../src/backtrace/camlBacktrace\_\_definitely\_raise\_347.objdump}
      }
      \vfill
      \mintocaml[firstline=9,lastline=13,highlightlines=12]{../src/backtrace/backtrace.ml}
      \endminipage
    \end{column}
    \begin{column}{0.5\textwidth}
      \centering
      \providecommand\step{1}\input{diagrams/backtrace/backtrace.tex}
    \end{column}
  \end{columns}
\end{frame}

\begin{frame}{Backtraces}{But before that, we need more fields from \typename{caml\_domain\_state}}
  \begin{columns}
    \begin{column}{0.5\textwidth}
      \begin{itemize}
        \item Remember \typename{caml\_domain\_state} the per-domain runtime state?
        \item Some other members are of interest to us now\dots
        \item \localname{backtrace\_active} is used to know if the runtime should collect backtraces
        \item \localname{backtrace\_active} can be set by
          \begin{itemize}
            \item The \funcarg{b} flag in the \funcname{OCAMLRUNPARAM} environment variable
            \item \mintinline{ocaml}{Printexc.record_backtrace : bool -> unit} \footnotemark
          \end{itemize}
      \end{itemize}
    \end{column}
    \begin{column}{0.5\textwidth}
      \begin{listing}
        \mintc[highlightlines={9-11}]{src/ocaml/domain\_state\_backtrace.h}
        \caption{runtime/caml/domain\_state.h}
      \end{listing}
    \end{column}
  \end{columns}
  \footnotetext[1]{https://v2.ocaml.org/api/Printexc.html}
\end{frame}

\newcommand\listCamlRaiseExn[1][]{
  \listasm[firstline=702,lastline=725,#1]{../ocaml/runtime/amd64.S}{runtime/amd64.S:702}}

\begin{frame}{Backtraces}{Walk through \funcname{caml\_raise\_exn}}
  \begin{columns}[T]
    \begin{column}{0.5\textwidth}
      \begin{itemize}
        \item<1-> First checks whether exceptions backtrace should be recorded
        \item<1-> By checking the \funcarg{domain\_state->backtrace\_active} value
        \item<2-> If not, acts as \funcname{raise\_notrace} with \funcname{RESTORE\_EXN\_HANDLER\_OCAML}
          \begin{itemize}
            \item Set \regname{RSP} to \localname{domain\_state->exn\_handler} value
            \item Restore \localname{domain\_state->exn\_handler} from trap
            \item Jump with \funcname{ret} (same effect as using \funcname{pop} and \funcname{jmp})
          \end{itemize}
        \item<3-> Otherwise, switches to the C stack to be able to execute C code
        \item<4-> Calls \funcname{caml\_stash\_backtrace}, a C function provided by the runtime\ignorespaces
          \onslide<6->{, with
            \begin{itemize}
              \item \funcarg{exn}: exception value
              \item \funcarg{pc}: code address of the raising point
              \item \funcarg{sp}: stack address of the raising function
              \item \funcarg{trapsp}: stack address of the next handler
            \end{itemize}
          }
      \end{itemize}
      \bigskip
      \mintocaml[firstline=9,lastline=13,highlightlines=12]{../src/backtrace/backtrace.ml}
    \end{column}
    \begin{column}{0.5\textwidth}
      \raggedleft
      \onslide*<1,3-4>{
        \mintc[firstline=190,lastline=192]{../ocaml/runtime/amd64.S}
        \mintc[firstline=234,lastline=237]{../ocaml/runtime/amd64.S}
      }
      \onslide*<2>{
        \mintc[firstline=190,lastline=192,highlightlines={191-192}]{../ocaml/runtime/amd64.S}
        \mintc[firstline=234,lastline=237,highlightlines={235-237}]{../ocaml/runtime/amd64.S}
      }
      \onslide*<1>{\listCamlRaiseExn[highlightlines=705]{}}%
      \onslide*<2>{\listCamlRaiseExn[highlightlines={707-708}]{}}%
      \onslide*<3>{\listCamlRaiseExn[highlightlines={712-714}]{}}%
      \onslide*<4>{\listCamlRaiseExn[highlightlines={715-720}]{}}%
      \onslide*<5>{\providecommand\step{1}\input{diagrams/backtrace/backtrace.tex}}%
      \onslide*<6>{\providecommand\step{2}\input{diagrams/backtrace/backtrace.tex}}%
    \end{column}
  \end{columns}
\end{frame}

\newcommand\listStashBacktrace[1][]{
  \listc[firstline=91,lastline=120,#1]{../ocaml/runtime/backtrace\_nat.c}{runtime/backtrace\_nat.c:91}}

\begin{frame}{Backtraces}{Introducing \funcname{caml\_stash\_backtrace}}
  \begin{columns}[c]
    \begin{column}{0.5\textwidth}
      \minipage[c][0.66\textheight][s]{\columnwidth}
      \begin{itemize}
        \item<1-> A C function provided by the runtime (specific to native code)
        \item<2-> It scans the stack, between the stack pointer at the raising point up to the next trap, in order to collect the names of the detected function frames
          \onslide*<2>{
            \begin{itemize}
              \item And more information, like line number, column number...
            \end{itemize}
          }
        \item<3-> It uses the same technique and information as the Garbage Collector (GC) uses to scan the values live on the stack
          \onslide*<4>{
            \begin{itemize}
              \item \typename{frame\_descr} are at the core of stack unwinding
            \end{itemize}
          }
      \end{itemize}
      \vfill
      \mintocaml[firstline=9,lastline=13,highlightlines=12]{../src/backtrace/backtrace.ml}
      \endminipage
    \end{column}
    \begin{column}{0.5\textwidth}
      \onslide*<1>{\listStashBacktrace[highlightlines=91]{}}
      \onslide*<2>{\listStashBacktrace[highlightlines={91,117-118}]{}}
      \onslide*<3->{\listStashBacktrace[highlightlines={94,106,109-110}]{}}
    \end{column}
  \end{columns}
\end{frame}

\newcommand\listFrameDescr[1][]{
  \listocaml[firstline=111,lastline=124,#1]{../ocaml/asmcomp/emitaux.ml}{asmcomp/emitaux.ml:111}}
\newcommand\listDebugInfo[1][]{
  \listocaml[firstline=38,lastline=49,#1]{../ocaml/lambda/debuginfo.mli}{lambda/debuginfo.mli:38}}

\begin{frame}{Backtraces}{\typename{frame\_descr} and \typename{frame\_debuginfo} in a nutshell}
  \begin{columns}[c]
    \begin{column}{0.5\textwidth}
      \begin{itemize}
        \item<1-> For every return address in an OCaml program, there is a associated \typename{frame\_descr}
        \item<2-> A \typename{frame\_descr} specifies
        \begin{itemize}
          \item<2-> The size of the function stack frame
          \item<3-> Where live values are on the stack (so that the GC can mark them as live and prevent from being swept)
        \end{itemize}
        \item<4-> When compiled with debug information, \typename{frame\_descr} will be linked to a \typename{frame\_debuginfo}
        \begin{itemize}
          \item<5-> Contains information about the position in the source code (file name, line and column number)
        \end{itemize}
      \end{itemize}
    \end{column}
    \begin{column}{0.5\textwidth}
        \onslide*<1>{\listFrameDescr[highlightlines={118-122}]{}}
        \onslide*<2>{\listFrameDescr[highlightlines={120}]{}}
        \onslide*<3>{\listFrameDescr[highlightlines={121}]{}}
        \onslide*<4->{\listFrameDescr[highlightlines={122}]{}}
        \medskip
        \onslide*<1-3>{\listDebugInfo{}}
        \onslide*<4>{\listDebugInfo[highlightlines={38,49}]{}}
        \onslide*<5>{\listDebugInfo[highlightlines={39-46}]{}}
    \end{column}
  \end{columns}
\end{frame}

\begin{frame}{Backtraces}{Scanning the stack with \typename{frame\_descr}}
  \begin{columns}[c]
    \begin{column}{0.5\textwidth}
      \begin{itemize}
        \item Given the return address of the code that raises the exception by calling \funcname{caml\_raise\_exn} (\funcarg{pc} argument of \funcname{caml\_stash\_backtrace})
        \item And the position of the stack pointer (\funcarg{sp} argument of \funcname{caml\_stash\_backtrace})
        \item The frame size given by \typename{frame\_descr}.\funcarg{frame\_size}, allows to find the end of the previous frame
        \item And the return address of the caller of this function
        \bigskip
        \onslide<3->{
        \item Note that traps (here the reddish \funcname{ExnA}, \funcname{ExnB} and \funcname{ExnC} blocks) belong to the frame that installed them, and so the frame size changes when traps are removed
        }
      \end{itemize}
    \end{column}
    \begin{column}{0.5\textwidth}
      \raggedleft
      \onslide*<1>{\providecommand\step{2}\input{diagrams/backtrace/backtrace.tex}}%
      \onslide*<2>{\providecommand\step{1}\input{diagrams/backtrace/scanstack.tex}}%
      \onslide*<3>{\providecommand\step{2}\input{diagrams/backtrace/scanstack.tex}}%
      \onslide*<4>{\providecommand\step{3}\input{diagrams/backtrace/scanstack.tex}}%
      \onslide*<5>{\providecommand\step{4}\input{diagrams/backtrace/scanstack.tex}}%
      \onslide*<6>{\providecommand\step{5}\input{diagrams/backtrace/scanstack.tex}}%
    \end{column}
  \end{columns}
\end{frame}

% Duplicate of previous-previous slide
\begin{frame}{Backtraces}{Continuing on \funcname{caml\_stash\_backtrace}}
  \begin{columns}[c]
    \begin{column}{0.5\textwidth}
      \minipage[c][0.75\textheight][s]{\columnwidth}
      \begin{itemize}
          \color{gray}
        \item A C function provided by the runtime (specific to native code)
        \item It scans the stack, between the stack pointer at the raising point up to the next trap, in order to collect the names of the detected function frames
        \item It uses the same technique and information as the Garbage Collector (GC) uses to scan the values live on the stack
      \end{itemize}
      \begin{itemize}
        \item For each function frame detected, store a pointer to the \typename{frame\_descr} in the \localname{domain\_state->backtrace\_buffer} array, effectively constructing the backtrace
      \end{itemize}
      \onslide<2->{
        \begin{itemize}
            \smallskip
          \item Constructed backtrace:
            \funcname{definitely\_raise},
            \onslide<3->{\funcname{probably\_raise}}
        \end{itemize}
      }
      \vfill
      \mintocaml[firstline=9,lastline=13,highlightlines=12]{../src/backtrace/backtrace.ml}
      \endminipage
    \end{column}
    \begin{column}{0.5\textwidth}
      \raggedleft
      \onslide*<1>{\listStashBacktrace[highlightlines={114-115}]{}}
      \onslide*<2>{\providecommand\step{3}\input{diagrams/backtrace/backtrace.tex}}%
      \onslide*<3>{\providecommand\step{4}\input{diagrams/backtrace/backtrace.tex}}%
    \end{column}
  \end{columns}
\end{frame}

\begin{frame}{Backtraces}{Back to \funcname{caml\_raise\_exn}}
  \begin{columns}[c]
    \begin{column}{0.5\textwidth}
      \minipage[c][0.66\textheight][s]{\columnwidth}
      \begin{itemize}\color{gray}
        \item Checks wheteher exceptions backtrace should be recorded, by checking the \funcarg{domain\_state->backtrace\_active} value
        \item Switches to the C stack to be able to execute C code
        \item Calls \funcname{caml\_stash\_backtrace}, to collect the backtrace up to the next trap
      \end{itemize}
      \onslide*<2->{
        \begin{itemize}
          \item<2-> Executes the exception handler in \funcname{probably\_raise} with \funcname{RESTORE\_EXN\_HANDLER\_OCAML}
            \begin{itemize}
              \item Set \regname{RSP} to \localname{domain\_state->exn\_handler} value
              \item Restore \localname{domain\_state->exn\_handler} from trap
              \item Jump to the handler code with \funcname{ret}
            \end{itemize}
        \end{itemize}
      }
      \vfill
      \onslide*<1>{\mintocaml[firstline=9,lastline=13,highlightlines=12]{../src/backtrace/backtrace.ml}}
      \onslide*<2->{\mintocaml[firstline=15,lastline=21,highlightlines={19-21}]{../src/backtrace/backtrace.ml}}
      \endminipage
    \end{column}
    \begin{column}{0.5\textwidth}
      \centering
      \onslide*<1>{
        \mintc[firstline=190,lastline=192]{../ocaml/runtime/amd64.S}
        \mintc[firstline=234,lastline=237]{../ocaml/runtime/amd64.S}
      }
      \onslide*<2>{
        \mintc[firstline=190,lastline=192,highlightlines={191-192}]{../ocaml/runtime/amd64.S}
        \mintc[firstline=234,lastline=237,highlightlines={235-237}]{../ocaml/runtime/amd64.S}
      }
      \onslide*<1>{\listCamlRaiseExn[highlightlines=720]{}}
      \onslide*<2>{\listCamlRaiseExn[highlightlines=722]{}}
      \onslide*<3->{\providecommand\step{5}\input{diagrams/backtrace/backtrace.tex}}
    \end{column}
  \end{columns}
\end{frame}

\begin{frame}{Backtraces}{\funcname{caml\_probably\_raise}'s handler doesn't catch \typename{ExnC}}
  \begin{columns}[c]
    \begin{column}{0.45\textwidth}
      \minipage[c][0.75\textheight][s]{\columnwidth}
      \begin{itemize}
        \item<1-> \funcname{match \dots with}\footnotemark pattern matching can also match exceptions
        \item<1-> The CMM of \funcname{probably\_raise} shows that this \funcname{match \dots with} is implemented with a pattern matching and a \funcname{try \dots with}
        \item<1-> But this one only matches on \typename{ExnA} exception
        \item<2-> Since the pattern matching fails to catch the exception, \funcname{reraise} is called with our current \typename{ExnC} exception
        \item<3-> Disassembly shows that \funcname{reraise} is actually a call to \funcname{caml\_reraise\_exn}, which is also an assembly function provided by the runtime
      \end{itemize}
      \vfill
      \mintocaml[firstline=15,lastline=21,highlightlines={18-21}]{../src/backtrace/backtrace.ml}
      \endminipage
    \end{column}
    \begin{column}{0.55\textwidth}
      \centering
      \onslide*<1>{\mintcmm[highlightlines={10-18}]{../src/backtrace/camlBacktrace\_\_probably\_raise\_351.cmm}}
      \onslide*<2>{\listcmm[highlightlines=18]{../src/backtrace/camlBacktrace\_\_probably\_raise_351.cmm}{CMM of probably\_raise}}
      \onslide*<3>{\mintobjdump[firstline=5,lastline=35,highlightlines=31]{../src/backtrace/camlBacktrace\_\_probably\_raise_351.objdump}}
    \end{column}
  \end{columns}
  \only<1>{\footnotetext[1]{https://blog.janestreet.com/pattern-matching-and-exception-handling-unite/}}
\end{frame}

\newcommand\listCamlReraiseExn[1][]{
  \listasm[firstline=727,lastline=734,#1]{../ocaml/runtime/amd64.S}{runtime/amd64.S:727}}

\begin{frame}{Backtraces}{Introducing \funcname{caml\_reraise\_exn}}
  \begin{columns}[c]
    \begin{column}{0.5\textwidth}
      \minipage[c][0.66\textheight][s]{\columnwidth}
      \begin{itemize}
        \item<1-> The exception have to be reraised to the next handler
        \item<2-> First, checks if the backtrace should be collected with \funcarg{domain\_state->backtrace\_active}
        \only<3>{
        \begin{itemize}
            \item If not, behaves like a \funcname{raise\_notrace} with \funcname{RESTORE\_EXN\_HANDLER\_OCAML}
        \end{itemize}
        }
        \item<4-> Jumps into \funcname{caml\_raise\_exn}
        \item<5-> Calls \funcname{caml\_stash\_backtrace} again, to scan the next part of the stack: from the trap just pop-ed to the next trap
      \end{itemize}
      \vfill
      \mintocaml[firstline=15,lastline=21,highlightlines={18-21}]{../src/backtrace/backtrace.ml}
      \endminipage
    \end{column}
    \begin{column}{0.5\textwidth}
      \raggedleft
      \onslide*<1>{\listCamlReraiseExn{}}
      \onslide*<2>{\listCamlReraiseExn[highlightlines=729]{}}
      \onslide*<3>{
        \mintc[firstline=190,lastline=192,highlightlines={191-192}]{../ocaml/runtime/amd64.S}
        \mintc[firstline=234,lastline=237,highlightlines={235-237}]{../ocaml/runtime/amd64.S}
        \medskip
        \listCamlReraiseExn[highlightlines=731]{}
      }
      \onslide*<4-5>{
        % TODO refactor with \listCamlReraiseExn
        \mintasm[firstline=727,lastline=734,highlightlines=730]{../ocaml/runtime/amd64.S}
        \medskip
      }
      \onslide*<4>{\listCamlRaiseExn[highlightlines=711]{}}
      \onslide*<5>{\listCamlRaiseExn[highlightlines=720]{}}
    \end{column}
  \end{columns}
\end{frame}

\begin{frame}{Backtraces}{Backtrace collection continues}
  \begin{columns}[c]
    \begin{column}{0.55\textwidth}
      \minipage[c][0.75\textheight][s]{\columnwidth}
      \begin{itemize}
        \item<1-> \funcname{caml\_stash\_backtrace} scans the older part of the stack, up to the next trap
        \item<6-> Controls jumps to the nested exception handler in \funcname{maybe\_raise}, which matches only on \typename{ExnB}
        \item<7-> \funcname{caml\_reraise\_exn} is called again, backtrace collection resumes with \funcname{caml\_stash\_backtrace}
      \end{itemize}
      \medskip
      \begin{itemize}
        \item<2-> Constructed backtrace:
        \begin{itemize}
          \item <2-> \funcname{definitely\_raise}
          \item <2-> \funcname{probably\_raise}
          \item <3-> \funcname{probably\_raise} (re-raise)
          \item <4-> \funcname{maybe\_raise}
          \item <5-> \funcname{main}
          \item <8-> \funcname{main} (re-raise)
        \end{itemize}
      \end{itemize}
      \vfill
      \onslide*<1-5>{\mintocaml[firstline=15,lastline=21]{../src/backtrace/backtrace.ml}}
      \onslide*<6>{\mintocaml[firstline=28,lastline=34,highlightlines=31]{../src/backtrace/backtrace.ml}}
      \onslide*<7->{\mintocaml[firstline=28,lastline=34,highlightlines=33]{../src/backtrace/backtrace.ml}}
      \endminipage
    \end{column}
    \begin{column}{0.45\textwidth}
      \raggedleft
      \onslide*<1>{\providecommand\step{6}\input{diagrams/backtrace/backtrace.tex}}%
      \onslide*<2>{\providecommand\step{6}\input{diagrams/backtrace/backtrace.tex}}%
      \onslide*<3>{\providecommand\step{7}\input{diagrams/backtrace/backtrace.tex}}%
      \onslide*<4>{\providecommand\step{8}\input{diagrams/backtrace/backtrace.tex}}%
      \onslide*<5>{\providecommand\step{9}\input{diagrams/backtrace/backtrace.tex}}%
      \onslide*<6>{\providecommand\step{10}\input{diagrams/backtrace/backtrace.tex}}%
      \onslide*<7>{\providecommand\step{11}\input{diagrams/backtrace/backtrace.tex}}%
      \onslide*<8>{\providecommand\step{12}\input{diagrams/backtrace/backtrace.tex}}%
      \onslide*<9>{\providecommand\step{13}\input{diagrams/backtrace/backtrace.tex}}%
    \end{column}
  \end{columns}
\end{frame}

\begin{frame}{Backtraces}{Printing the backtrace}
  \begin{columns}[c]
    \begin{column}{0.45\textwidth}
      \minipage[c][0.66\textheight][s]{\columnwidth}
      \begin{itemize}
        \item<1-> The exception backtrace can now be printed to \funcarg{stdout} with \funcname{Printexc.print_backtrace}
        \item<2-> First the exception backtrace is retrieved into an OCaml array with \funcname{get_raw_backtrace }
        \item<3-> Which is implemented as the \funcname{caml_get_exception_raw_backtrace} C function in the runtime
        \item<4-> And finally printed with \funcname{print_exception_backtrace}
      \end{itemize}
      \vfill
      \mintocaml[firstline=28,lastline=34,highlightlines=33]{../src/backtrace/backtrace.ml}
      \endminipage
    \end{column}
    \begin{column}{0.55\textwidth}
      \onslide*<1,2>{
        \mintocaml[firstline=100,lastline=101]{../ocaml/stdlib/printexc.ml}
      }
      \only<1>{
        \listocaml[firstline=170,lastline=172]{../ocaml/stdlib/printexc.ml}{stdlib/printexc.ml:170}
      }
      \only<2>{
        \listocaml[firstline=170,lastline=172,highlightlines=172]{../ocaml/stdlib/printexc.ml}{stdlib/printexc.ml:170}
      }
      \only<3>{
        \mintc[firstline=154,lastline=156]{../ocaml/runtime/backtrace.c}
        \listc[firstline=171,lastline=192,highlightlines={184-187}]{../ocaml/runtime/backtrace.c}{runtime/backtrace.c:154}
      }
      \only<4>{
        \listocaml[firstline=155,lastline=165]{../ocaml/stdlib/printexc.ml}{stdlib/printexc.ml:55}
      }
    \end{column}
  \end{columns}
\end{frame}


\subsection{The multiple kinds of raise}
\frameSubsection{}{}

\begin{frame}{Backtraces}{When backtraces are not needed}
  \begin{columns}[c]
    \begin{column}{0.45\textwidth}
      \minipage[c][0.66\textheight][s]{\columnwidth}
      \begin{itemize}
        \item<1-> What if the backtrace for an exception is never needed?
        \item<2-> It's possible to raise the exception explicitly with \funcname{raise\_notrace} to make raising as cheap as possible
        \item<3-> An example of use in the \funcname{dynlink} library of the OCaml compiler
      \end{itemize}
      \vfill
      \onslide<3->{
      \listocaml[firstline=205,lastline=209,highlightlines=207]{../ocaml/otherlibs/dynlink/dynlink\_compilerlibs/misc.ml}{otherlibs/dynlink/dynlink\_compilerlibs/misc.ml:205}
      }
      \endminipage
    \end{column}
    \begin{column}{0.55\textwidth}
    \only<4>{
      \mintcmm[highlightlines=3]{../src/notrace/camlNotrace\_\_fun_347.cmm}
      \listcmm{../src/notrace/camlNotrace\_\_all\_somes\_327.cmm}{all\_somes}
    }
    \onslide*<5->{
      \listobjdump[highlightlines={6-9}]{../src/notrace/camlNotrace\_\_fun_347.objdump}{anonymous function from \funcname{all\_somes}}
    }
    \end{column}
  \end{columns}
\end{frame}

\begin{frame}{Backtraces}{Summary table of the multiple kinds of raise}
  \begin{itemize}
    \item When debugging information is enabled and if the backtrace of an exception isn't needed, it's possible to raise with \funcname{raise\_notrace} for performance
    \item When debugging information is \emph{not} enabled, the compiler optimises \funcname{raise} calls to inline assembly (same effect as using \funcname{raise\_notrace})
  \end{itemize}
  \bigskip
  \centering
  \begin{tabular}{| l | c | c | c | c |}
    \hline
    \parbox{10em}{Debug information \\ (\funcarg{-g} flag)}  & \multicolumn{2}{|c|}{Disabled}                                     & \multicolumn{2}{|c|}{Enabled} \\
    \hline
    Raise kind                                               & \funcname{raise} & \funcname{raise\_notrace}                       & \funcname{raise} & \funcname{raise\_notrace} \\
    \hline
    Compilation                                              & \parbox{8em}{inline assembly\\ (optimization)} & inline assembly   & \funcname{caml\_raise\_exn} & inline assembly \\
    \hline
  \end{tabular}
\end{frame}


%
% Takeaway #5
%
\subsection*{Takeaway \#5}
\frameSubsectionTakeaway{}
\againframe<5>{takeaway}
}%
    \end{column}
  \end{columns}
\end{frame}

\begin{frame}{Backtraces}{Printing the backtrace}
  \begin{columns}[c]
    \begin{column}{0.45\textwidth}
      \minipage[c][0.66\textheight][s]{\columnwidth}
      \begin{itemize}
        \item<1-> The exception backtrace can now be printed to \funcarg{stdout} with \funcname{Printexc.print_backtrace}
        \item<2-> First the exception backtrace is retrieved into an OCaml array with \funcname{get_raw_backtrace }
        \item<3-> Which is implemented as the \funcname{caml_get_exception_raw_backtrace} C function in the runtime
        \item<4-> And finally printed with \funcname{print_exception_backtrace}
      \end{itemize}
      \vfill
      \mintocaml[firstline=28,lastline=34,highlightlines=33]{../src/backtrace/backtrace.ml}
      \endminipage
    \end{column}
    \begin{column}{0.55\textwidth}
      \onslide*<1,2>{
        \mintocaml[firstline=100,lastline=101]{../ocaml/stdlib/printexc.ml}
      }
      \only<1>{
        \listocaml[firstline=170,lastline=172]{../ocaml/stdlib/printexc.ml}{stdlib/printexc.ml:170}
      }
      \only<2>{
        \listocaml[firstline=170,lastline=172,highlightlines=172]{../ocaml/stdlib/printexc.ml}{stdlib/printexc.ml:170}
      }
      \only<3>{
        \mintc[firstline=154,lastline=156]{../ocaml/runtime/backtrace.c}
        \listc[firstline=171,lastline=192,highlightlines={184-187}]{../ocaml/runtime/backtrace.c}{runtime/backtrace.c:154}
      }
      \only<4>{
        \listocaml[firstline=155,lastline=165]{../ocaml/stdlib/printexc.ml}{stdlib/printexc.ml:55}
      }
    \end{column}
  \end{columns}
\end{frame}


\subsection{The multiple kinds of raise}
\frameSubsection{}{}

\begin{frame}{Backtraces}{When backtraces are not needed}
  \begin{columns}[c]
    \begin{column}{0.45\textwidth}
      \minipage[c][0.66\textheight][s]{\columnwidth}
      \begin{itemize}
        \item<1-> What if the backtrace for an exception is never needed?
        \item<2-> It's possible to raise the exception explicitly with \funcname{raise\_notrace} to make raising as cheap as possible
        \item<3-> An example of use in the \funcname{dynlink} library of the OCaml compiler
      \end{itemize}
      \vfill
      \onslide<3->{
      \listocaml[firstline=205,lastline=209,highlightlines=207]{../ocaml/otherlibs/dynlink/dynlink\_compilerlibs/misc.ml}{otherlibs/dynlink/dynlink\_compilerlibs/misc.ml:205}
      }
      \endminipage
    \end{column}
    \begin{column}{0.55\textwidth}
    \only<4>{
      \mintcmm[highlightlines=3]{../src/notrace/camlNotrace\_\_fun_347.cmm}
      \listcmm{../src/notrace/camlNotrace\_\_all\_somes\_327.cmm}{all\_somes}
    }
    \onslide*<5->{
      \listobjdump[highlightlines={6-9}]{../src/notrace/camlNotrace\_\_fun_347.objdump}{anonymous function from \funcname{all\_somes}}
    }
    \end{column}
  \end{columns}
\end{frame}

\begin{frame}{Backtraces}{Summary table of the multiple kinds of raise}
  \begin{itemize}
    \item When debugging information is enabled and if the backtrace of an exception isn't needed, it's possible to raise with \funcname{raise\_notrace} for performance
    \item When debugging information is \emph{not} enabled, the compiler optimises \funcname{raise} calls to inline assembly (same effect as using \funcname{raise\_notrace})
  \end{itemize}
  \bigskip
  \centering
  \begin{tabular}{| l | c | c | c | c |}
    \hline
    \parbox{10em}{Debug information \\ (\funcarg{-g} flag)}  & \multicolumn{2}{|c|}{Disabled}                                     & \multicolumn{2}{|c|}{Enabled} \\
    \hline
    Raise kind                                               & \funcname{raise} & \funcname{raise\_notrace}                       & \funcname{raise} & \funcname{raise\_notrace} \\
    \hline
    Compilation                                              & \parbox{8em}{inline assembly\\ (optimization)} & inline assembly   & \funcname{caml\_raise\_exn} & inline assembly \\
    \hline
  \end{tabular}
\end{frame}


%
% Takeaway #5
%
\subsection*{Takeaway \#5}
\frameSubsectionTakeaway{}
\againframe<5>{takeaway}
}%
    \end{column}
  \end{columns}
\end{frame}

\begin{frame}{Backtraces}{Printing the backtrace}
  \begin{columns}[c]
    \begin{column}{0.45\textwidth}
      \minipage[c][0.66\textheight][s]{\columnwidth}
      \begin{itemize}
        \item<1-> The exception backtrace can now be printed to \funcarg{stdout} with \funcname{Printexc.print_backtrace}
        \item<2-> First the exception backtrace is retrieved into an OCaml array with \funcname{get_raw_backtrace }
        \item<3-> Which is implemented as the \funcname{caml_get_exception_raw_backtrace} C function in the runtime
        \item<4-> And finally printed with \funcname{print_exception_backtrace}
      \end{itemize}
      \vfill
      \mintocaml[firstline=28,lastline=34,highlightlines=33]{../src/backtrace/backtrace.ml}
      \endminipage
    \end{column}
    \begin{column}{0.55\textwidth}
      \onslide*<1,2>{
        \mintocaml[firstline=100,lastline=101]{../ocaml/stdlib/printexc.ml}
      }
      \only<1>{
        \listocaml[firstline=170,lastline=172]{../ocaml/stdlib/printexc.ml}{stdlib/printexc.ml:170}
      }
      \only<2>{
        \listocaml[firstline=170,lastline=172,highlightlines=172]{../ocaml/stdlib/printexc.ml}{stdlib/printexc.ml:170}
      }
      \only<3>{
        \mintc[firstline=154,lastline=156]{../ocaml/runtime/backtrace.c}
        \listc[firstline=171,lastline=192,highlightlines={184-187}]{../ocaml/runtime/backtrace.c}{runtime/backtrace.c:154}
      }
      \only<4>{
        \listocaml[firstline=155,lastline=165]{../ocaml/stdlib/printexc.ml}{stdlib/printexc.ml:55}
      }
    \end{column}
  \end{columns}
\end{frame}


\subsection{The multiple kinds of raise}
\frameSubsection{}{}

\begin{frame}{Backtraces}{When backtraces are not needed}
  \begin{columns}[c]
    \begin{column}{0.45\textwidth}
      \minipage[c][0.66\textheight][s]{\columnwidth}
      \begin{itemize}
        \item<1-> What if the backtrace for an exception is never needed?
        \item<2-> It's possible to raise the exception explicitly with \funcname{raise\_notrace} to make raising as cheap as possible
        \item<3-> An example of use in the \funcname{dynlink} library of the OCaml compiler
      \end{itemize}
      \vfill
      \onslide<3->{
      \listocaml[firstline=205,lastline=209,highlightlines=207]{../ocaml/otherlibs/dynlink/dynlink\_compilerlibs/misc.ml}{otherlibs/dynlink/dynlink\_compilerlibs/misc.ml:205}
      }
      \endminipage
    \end{column}
    \begin{column}{0.55\textwidth}
    \only<4>{
      \mintcmm[highlightlines=3]{../src/notrace/camlNotrace\_\_fun_347.cmm}
      \listcmm{../src/notrace/camlNotrace\_\_all\_somes\_327.cmm}{all\_somes}
    }
    \onslide*<5->{
      \listobjdump[highlightlines={6-9}]{../src/notrace/camlNotrace\_\_fun_347.objdump}{anonymous function from \funcname{all\_somes}}
    }
    \end{column}
  \end{columns}
\end{frame}

\begin{frame}{Backtraces}{Summary table of the multiple kinds of raise}
  \begin{itemize}
    \item When debugging information is enabled and if the backtrace of an exception isn't needed, it's possible to raise with \funcname{raise\_notrace} for performance
    \item When debugging information is \emph{not} enabled, the compiler optimises \funcname{raise} calls to inline assembly (same effect as using \funcname{raise\_notrace})
  \end{itemize}
  \bigskip
  \centering
  \begin{tabular}{| l | c | c | c | c |}
    \hline
    \parbox{10em}{Debug information \\ (\funcarg{-g} flag)}  & \multicolumn{2}{|c|}{Disabled}                                     & \multicolumn{2}{|c|}{Enabled} \\
    \hline
    Raise kind                                               & \funcname{raise} & \funcname{raise\_notrace}                       & \funcname{raise} & \funcname{raise\_notrace} \\
    \hline
    Compilation                                              & \parbox{8em}{inline assembly\\ (optimization)} & inline assembly   & \funcname{caml\_raise\_exn} & inline assembly \\
    \hline
  \end{tabular}
\end{frame}


%
% Takeaway #5
%
\subsection*{Takeaway \#5}
\frameSubsectionTakeaway{}
\againframe<5>{takeaway}
}%
    \end{column}
  \end{columns}
\end{frame}

\begin{frame}{Backtraces}{Printing the backtrace}
  \begin{columns}[c]
    \begin{column}{0.45\textwidth}
      \minipage[c][0.66\textheight][s]{\columnwidth}
      \begin{itemize}
        \item<1-> The exception backtrace can now be printed to \funcarg{stdout} with \funcname{Printexc.print_backtrace}
        \item<2-> First the exception backtrace is retrieved into an OCaml array with \funcname{get_raw_backtrace }
        \item<3-> Which is implemented as the \funcname{caml_get_exception_raw_backtrace} C function in the runtime
        \item<4-> And finally printed with \funcname{print_exception_backtrace}
      \end{itemize}
      \vfill
      \mintocaml[firstline=28,lastline=34,highlightlines=33]{../src/backtrace/backtrace.ml}
      \endminipage
    \end{column}
    \begin{column}{0.55\textwidth}
      \onslide*<1,2>{
        \mintocaml[firstline=100,lastline=101]{../ocaml/stdlib/printexc.ml}
      }
      \only<1>{
        \listocaml[firstline=170,lastline=172]{../ocaml/stdlib/printexc.ml}{stdlib/printexc.ml:170}
      }
      \only<2>{
        \listocaml[firstline=170,lastline=172,highlightlines=172]{../ocaml/stdlib/printexc.ml}{stdlib/printexc.ml:170}
      }
      \only<3>{
        \mintc[firstline=154,lastline=156]{../ocaml/runtime/backtrace.c}
        \listc[firstline=171,lastline=192,highlightlines={184-187}]{../ocaml/runtime/backtrace.c}{runtime/backtrace.c:154}
      }
      \only<4>{
        \listocaml[firstline=155,lastline=165]{../ocaml/stdlib/printexc.ml}{stdlib/printexc.ml:55}
      }
    \end{column}
  \end{columns}
\end{frame}


\subsection{The multiple kinds of raise}
\frameSubsection{}{}

\begin{frame}{Backtraces}{When backtraces are not needed}
  \begin{columns}[c]
    \begin{column}{0.45\textwidth}
      \minipage[c][0.66\textheight][s]{\columnwidth}
      \begin{itemize}
        \item<1-> What if the backtrace for an exception is never needed?
        \item<2-> It's possible to raise the exception explicitly with \funcname{raise\_notrace} to make raising as cheap as possible
        \item<3-> An example of use in the \funcname{dynlink} library of the OCaml compiler
      \end{itemize}
      \vfill
      \onslide<3->{
      \listocaml[firstline=205,lastline=209,highlightlines=207]{../ocaml/otherlibs/dynlink/dynlink\_compilerlibs/misc.ml}{otherlibs/dynlink/dynlink\_compilerlibs/misc.ml:205}
      }
      \endminipage
    \end{column}
    \begin{column}{0.55\textwidth}
    \only<4>{
      \mintcmm[highlightlines=3]{../src/notrace/camlNotrace\_\_fun_347.cmm}
      \listcmm{../src/notrace/camlNotrace\_\_all\_somes\_327.cmm}{all\_somes}
    }
    \onslide*<5->{
      \listobjdump[highlightlines={6-9}]{../src/notrace/camlNotrace\_\_fun_347.objdump}{anonymous function from \funcname{all\_somes}}
    }
    \end{column}
  \end{columns}
\end{frame}

\begin{frame}{Backtraces}{Summary table of the multiple kinds of raise}
  \begin{itemize}
    \item When debugging information is enabled and if the backtrace of an exception isn't needed, it's possible to raise with \funcname{raise\_notrace} for performance
    \item When debugging information is \emph{not} enabled, the compiler optimises \funcname{raise} calls to inline assembly (same effect as using \funcname{raise\_notrace})
  \end{itemize}
  \bigskip
  \centering
  \begin{tabular}{| l | c | c | c | c |}
    \hline
    \parbox{10em}{Debug information \\ (\funcarg{-g} flag)}  & \multicolumn{2}{|c|}{Disabled}                                     & \multicolumn{2}{|c|}{Enabled} \\
    \hline
    Raise kind                                               & \funcname{raise} & \funcname{raise\_notrace}                       & \funcname{raise} & \funcname{raise\_notrace} \\
    \hline
    Compilation                                              & \parbox{8em}{inline assembly\\ (optimization)} & inline assembly   & \funcname{caml\_raise\_exn} & inline assembly \\
    \hline
  \end{tabular}
\end{frame}


%
% Takeaway #5
%
\subsection*{Takeaway \#5}
\frameSubsectionTakeaway{}
\againframe<5>{takeaway}
